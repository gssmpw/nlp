% This must be in the first 5 lines to tell arXiv to use pdfLaTeX, which is strongly recommended.
\pdfoutput=1
% In particular, the hyperref package requires pdfLaTeX in order to break URLs across lines.

\documentclass[11pt]{article}

% Change "review" to "final" to generate the final (sometimes called camera-ready) version.
% Change to "preprint" to generate a non-anonymous version with page numbers.
%\usepackage[review]{acl}
\usepackage[preprint]{acl}

% Standard package includes
\usepackage{times}
\usepackage{latexsym}

% For proper rendering and hyphenation of words containing Latin characters (including in bib files)
\usepackage[T1]{fontenc}
% For Vietnamese characters
% \usepackage[T5]{fontenc}
% See https://www.latex-project.org/help/documentation/encguide.pdf for other character sets

% This assumes your files are encoded as UTF8
\usepackage[utf8]{inputenc}

% This is not strictly necessary, and may be commented out,
% but it will improve the layout of the manuscript,
% and will typically save some space.
\usepackage{microtype}

% This is also not strictly necessary, and may be commented out.
% However, it will improve the aesthetics of text in
% the typewriter font.
\usepackage{inconsolata}

%Including images in your LaTeX document requires adding
%additional package(s)
\usepackage{graphicx}
\usepackage{subcaption}
\usepackage{longtable}
\usepackage{booktabs}
\usepackage{multirow}
\usepackage{xcolor,colortbl}
\usepackage{adjustbox}


\definecolor{g1}{HTML}{DBF9CC}
\definecolor{g2}{HTML}{B0F893}
\definecolor{g3}{HTML}{4BE930}
\definecolor{g4}{HTML}{37D32A}
\definecolor{g5}{HTML}{048A21}

\definecolor{r1}{HTML}{FCE2D5}
\definecolor{r2}{HTML}{F8C3AF}
\definecolor{r3}{HTML}{ED7A65}
\definecolor{r4}{HTML}{E65241}
\definecolor{r5}{HTML}{DE271F}

\definecolor{infoseek}{HTML}{78206E}
\definecolor{ours}{HTML}{215F9A}


% If the title and author information does not fit in the area allocated, uncomment the following
%
%\setlength\titlebox{<dim>}
%
% and set <dim> to something 5cm or larger.

\newcommand{\ds}{\textsc{Visual-RAG }}
\newcommand{\dsns}{\textsc{Visual-RAG}}
\newcommand{\cc}[1]{\cellcolor{#1}}
\newcommand{\quanyu}[1]{\textcolor{blue}{[#1 ---\textsc{quanyu}]}}
\newcommand{\wenya}[1]{\textcolor{brown}{[#1 ---\textsc{wy}]}}

\title{Visual-RAG: Benchmarking Text-to-Image Retrieval Augmented Generation for Visual Knowledge Intensive Queries}

% Author information can be set in various styles:
% For several authors from the same institution:
% \author{Author 1 \and ... \and Author n \\
%         Address line \\ ... \\ Address line}
% if the names do not fit well on one line use
%         Author 1 \\ {\bf Author 2} \\ ... \\ {\bf Author n} \\
% For authors from different institutions:
% \author{Author 1 \\ Address line \\  ... \\ Address line
%         \And  ... \And
%         Author n \\ Address line \\ ... \\ Address line}
% To start a separate ``row'' of authors use \AND, as in
% \author{Author 1 \\ Address line \\  ... \\ Address line
%         \AND
%         Author 2 \\ Address line \\ ... \\ Address line \And
%         Author 3 \\ Address line \\ ... \\ Address line}

\author{Yin Wu\textsuperscript{1}, Quanyu Long\textsuperscript{1}, Jing Li\textsuperscript{2}, Jianfei Yu\textsuperscript{3}, Wenya Wang\textsuperscript{1}\\
  \textsuperscript{1} Nanyang Technological University, \\
  \textsuperscript{2} Harbin Institute of Technology (Shenzhen), \textsuperscript{3} Nanjing University of Science and Technology\\
  \texttt{wuyi0023@e.ntu.edu.sg}}

%\author{
%  \textbf{First Author\textsuperscript{1}},
%  \textbf{Second Author\textsuperscript{1,2}},
%  \textbf{Third T. Author\textsuperscript{1}},
%  \textbf{Fourth Author\textsuperscript{1}},
%\\
%  \textbf{Fifth Author\textsuperscript{1,2}},
%  \textbf{Sixth Author\textsuperscript{1}},
%  \textbf{Seventh Author\textsuperscript{1}},
%  \textbf{Eighth Author \textsuperscript{1,2,3,4}},
%\\
%  \textbf{Ninth Author\textsuperscript{1}},
%  \textbf{Tenth Author\textsuperscript{1}},
%  \textbf{Eleventh E. Author\textsuperscript{1,2,3,4,5}},
%  \textbf{Twelfth Author\textsuperscript{1}},
%\\
%  \textbf{Thirteenth Author\textsuperscript{3}},
%  \textbf{Fourteenth F. Author\textsuperscript{2,4}},
%  \textbf{Fifteenth Author\textsuperscript{1}},
%  \textbf{Sixteenth Author\textsuperscript{1}},
%\\
%  \textbf{Seventeenth S. Author\textsuperscript{4,5}},
%  \textbf{Eighteenth Author\textsuperscript{3,4}},
%  \textbf{Nineteenth N. Author\textsuperscript{2,5}},
%  \textbf{Twentieth Author\textsuperscript{1}}
%\\
%\\
%  \textsuperscript{1}Affiliation 1,
%  \textsuperscript{2}Affiliation 2,
%  \textsuperscript{3}Affiliation 3,
%  \textsuperscript{4}Affiliation 4,
%  \textsuperscript{5}Affiliation 5
%\\
%  \small{
%    \textbf{Correspondence:} \href{mailto:email@domain}{email@domain}
%  }
%}

\begin{document}
\maketitle
\begin{abstract}
Retrieval-Augmented Generation (RAG) is a popular approach for enhancing Large Language Models (LLMs) by addressing their limitations in verifying facts and answering knowledge-intensive questions. As the research in LLM extends their capability to handle input modality other than text, e.g. image, several multimodal RAG benchmarks are proposed. Nonetheless, they mainly use textual knowledge bases as the primary source of evidences for augmentation. There still lack benchmarks designed to evaluate images as augmentation in RAG systems and how they leverage visual knowledge.
We propose \dsns, a novel Question Answering benchmark that emphasizes visual knowledge intensive questions. Unlike prior works relying on text-based evidence, \ds necessitates text-to-image retrieval and integration of relevant clue images to extract visual knowledge as evidence.
With \dsns, we evaluate 5 open-sourced and 3 proprietary Multimodal LLMs (MLLMs), revealing that images can serve as good evidence in RAG; however, even the SoTA models struggle with effectively extracting and utilizing visual knowledge\footnote{Benchmark and evaluation codes will be available at \url{https://github.com/LuciusLan/Visual-RAG}.}.
\end{abstract}

\section{Introduction}
\label{sec:intro}

\begin{figure*}[t]
    \centering
    % \setlength{\abovecaptionskip}{0.2cm}
    %\setlength{\belowcaptionskip}{-0.7em}
    \includegraphics[trim={3em, 1em, 0.5em, 1em},width=0.95\textwidth, clip]{image/question_sample.pdf}
    \caption{Comparison on query format and QA pipeline between InfoSeek-style and our \dsns. \textcolor{infoseek}{InfoSeek-style}: The entity in the query image must first be recognized to extract textual evidence, which is then used independently to answer the question. \textcolor{ours}{\dsns}: The query is text-only, but direct answers are rarely available in textual form. The model must retrieve relevant images and interpret them to generate an answer.}
    \label{fig:question_sample}
\end{figure*}

\newcommand{\iconimg}{\includegraphics[trim={-10px, 25px, 0px, 20px},width=1.2em, clip]{image/image.png}}
\newcommand{\icontext}{\includegraphics[trim={-15px, -10px, -15px, 5px},width=1.3em, height=1.02em,clip]{image/text.png}}
\newcommand{\iconarrow}{\includegraphics[width=1.1em]{image/rightarrow.png}}
\newcommand{\iconlink}{\includegraphics[trim={0em, 0em, 0.3em, 0em},width=1em, clip]{image/link.png}}

\begin{table*}[ht]
    \centering
    \begin{adjustbox}{max width=0.85\textwidth}
    \begin{tabular}{l|ccc}
    \toprule
         Benchmark  &Query Modality   &Knowledge Modality  &Retrieval Modality\\
        \midrule
         InfoSeek~\citep{chen-etal-2023-pre-trained}   &\raisebox{-2pt}{\iconimg\icontext} &\raisebox{-2pt}{\icontext}  &\raisebox{-2pt}{\iconimg  \iconarrow \iconimg \iconlink \hspace{-5pt} \icontext \iconarrow \icontext}\\
         E-VQA~\citep{Mensink_2023_ICCV}  &\raisebox{-2pt}{\iconimg\icontext} &\raisebox{-2pt}{\icontext}  &\raisebox{-2pt}{\iconimg  \iconarrow \iconimg \iconlink\hspace{-2pt}\icontext \iconarrow \icontext}\\
         WebQA~\citep{Chang_2022_CVPR}  &\raisebox{-2pt}{\icontext}  &\raisebox{-2pt}{\iconimg\icontext}  &\raisebox{-2pt}{\icontext \iconarrow \icontext\hspace{-2pt}\iconlink \hspace{-3pt} \iconimg}\\
         MRAG-Bench~\citep{hu2024mragbenchvisioncentricevaluationretrievalaugmented} &\raisebox{-2pt}{\iconimg\icontext}  &\raisebox{-2pt}{\iconimg}   &\raisebox{-2pt}{\iconimg\icontext \iconarrow \iconimg}\\
         \ds (Ours) &\raisebox{-3pt}{\icontext}  &\raisebox{-4pt}{\iconimg}   &\raisebox{-4pt}{\icontext \iconarrow \iconimg}\\
    \bottomrule
    \end{tabular}
    \end{adjustbox}
    %\setlength{\belowcaptionskip}{-1.5em}
    \caption{Comparing modalities of knowledge intensive VQA benchmarks. The link icon \raisebox{-2pt}{\iconlink} denotes that the two objects are paired, retrieving one will automatically link to the other. For InfoSeek and E-VQA, \citet{yan-xie-2024-echosight} demonstrated that by image-to-image retrieval on Wikipedia images, the Wikipedia articles containing those images are also retrieved at a significantly higher recall comparing to image-to-text retrieval. Similarly for WebQA, a question-to-caption (text-to-text) retrieval achieves better performance than text-to-image retrieval, as shown in Table \ref{tab:ret_webqa} in Appendix. Our \ds enforces text-to-image retrieval for visual knowledge.}
    \label{tab:comapre_bench}
\end{table*}


We have observed rapid advancement of Large Language Models (LLMs) in recent years. However, they are prone to generating hallucinated responses~\cite{10.1145/3571730}. Retrieval-Augmented Generation (RAG) has emerged as a widely adopted approach to mitigate this challenge~\citep{NEURIPS2020_6b493230, karpukhin-etal-2020-dense, izacard-grave-2021-leveraging}. By retrieving and integrating relevant textual documents from external knowledge sources, it enables LLMs to answer knowledge-intensive questions more accurately. Yet, many real-world scenarios demand multimodal knowledge (e.g., visual cues or geographical maps), suggesting that text-only retrieval may be insufficient.

%Advanced LLM service providers are expanding their models to incorporate modalities beyond text, and particularly, images~\citep{openai_chatgpt_2024, claude, gemini}. The open-source research community has also introduced numerous Multimodal LLM (MLLM) systems~\citep{abdin2024phi3technicalreporthighly, Qwen2VL, chen2024expandingperformanceboundariesopensource,grattafiori2024llama3herdmodels,pixtral}. %This naturally raises the question: \textit{can images be used as augmentation material in RAG, and how effectively could multimodal RAG enhance MLLMs performance in multimodal-knowledge-intensive QA?}

Building on the success of text-only RAG, various evaluation benchmarks have been proposed to assess the capability of Multimodal LLMs (MLLMs, a.k.a. Large Vision Language Models LVLM) in using multimodal RAG to address complex, multimodal knowledge intensive questions, including InfoSeek~\citep{chen-etal-2023-pre-trained}, Encyclopedic-VQA~\citep{Mensink_2023_ICCV}, etc. They feature mixed-modal queries where the question is asked based on a paired image.

%\wenya{This paragraph is not pointwise clear. Need to separate them into points with clear summaries.} 
However, existing datasets reveal several limitations. Firstly, although images are included in the query, \textbf{MLLMs do not utilize the visual knowledge from these images when generating answers}. Secondly, \textbf{questions in these datasets typically do not require reasoning over knowledge extracted from images}. Figure \ref{fig:question_sample} gives an example on InfoSeek-styled query. To answer the query, a model needs to first recognize the entity in query image, then locate the Wikipedia article and augment relevant passage to LLMs to generate an answer. Query image information is only required in the entity recognition step. %Also, the nature of queries makes the relevant knowledge very unlikely to be recorded in images: it is usually regarding an ``encyclopedic'' property of an entity, such as the habitat in this case.
These existing multimodal benchmarks are \textbf{textual} knowledge intensive, highlighting the need for a new benchmark for \textbf{visual} knowledge intensive QA.
%They predominantly focus on text-based knowledge as evidence, and the query image only serves as the anchor of the entity. LLM is \textbf{not utilizing visual knowledge in query image} when answering question.  Also, \textbf{the questions are not regarding visual features}, consequently under such problem setting, \textbf{images cannot be used as augmentation}, and such benchmarks cannot evaluate how effectively retrieval-augmented MLLMs leverage visual knowledge. 

%\quanyu{Motivation: first, previous works' queries contain image, which is not practical in real-world, since text-only query is more common; second, those queries is not visual knowledge-intensive, the visual kownledge-augmented generation ability of VLM cannot be evaluated; third, previous works are retrieving texts and LLM can answer the quries, in this work, we are retrieving the images as visual augmentations, and aim to benchmarking the visual augmented generation. (rethink about the motivations, and reorganize them in this paragraph.)}

To this end, We present \dsns, a visual knowledge intensive QA benchmark designed to evaluate text-to-image retrieval and answer generation augmented by retrieved images. Key features of \ds are:
\vspace{-0.5em}
\begin{enumerate}
    %\item The queries and answers are text-only. The queries are regarding visual features, enabling image to serve as evidence for answering them.
    \item The entities in queries are obscure, with their descriptions being brief, rarely documented in standard references (e.g., Wikipedia)
    \vspace{-0.5em}
    \item The visual features queried are fine-grained, with only 5.35\% of the images in the corpus containing the specific visual evidence needed to answer the questions, posing a challenging text-to-image retrieval task.
    \item The majority of non-clue images are hard-negatives which depict the same entities and are visually similar. This subtle distinction further increases the difficulty of cross-modal retrieval.
\end{enumerate}
%\wenya{The above summary is not well organized. The first feature should mention why text-only queries are necessary. The second and third points both talk about ``long-tail'' and could be combined. You can add an additional point emphasizing the challenge of retrieval here, as we have many hard negatives.}

%\wenya{This paragraph is illogical. You should first refer to Figure 1 as a comparison illustration, following the above three features. Then you need to briefly summarize how you create the dataset instead of only referring to Section numbers. Then following introduction to dataset creation, you can use the next paragraph to briefly introduce your evaluation.}
\vspace{-0.5em}
%The right part of Figure\ref{fig:question_sample} demonstrates the QA workflow in our \ds that differs largely with InfoSeek. Given a textual query, a cross-model retriever needs to first retrieve relevant images within the image knowledge base. There are typically larger amounts of hard negative images than the desired clue images within the retrieved image set. The MLLM, while augmented with the retrieved possibly relevant images, need to distinguish the clue image and extract visual knowledge to generate answers.

To construct \dsns, we begin with LLM-generated candidate queries, which are then refined through human filtering and rewriting. Next, we employ open-sourced MLLM and human annotators to perform another filtering step to ensure an appropriate level of difficulty for the queries. Detailed procedures for the entire benchmark construction can be found in Section \ref{sec:const}. 

%Current version of \ds contains 98 questions with an image knowledge base of 24083 images. The images are photos of wildlife and plants collected from the iNaturalist 2021 (iNat21) dataset~\citep{Van_Horn_2021_CVPR}, a fine-grained organism image classification dataset. 


%It is worth noting that, queries and answers of \ds are \textbf{text-only}, with respect to visual features that are rarely displayed on photos, and such queries are hard to be answered without viewing relevant clue image. Current version of \ds contains 98 questions with an image knowledge base of 24083 images. The images are photos of wildlife and plants collected from the iNaturalist 2021 (iNat21) dataset~\citep{Van_Horn_2021_CVPR}, a fine-grained organism image classification dataset. With \dsns, we are able to evaluate the ability of MLLMs to utilize visual knowledge embodied in retrieved images as evidence for augmented generation. Figure \ref{fig:question_sample} compares the differences in query format and QA workflow between InfoSeek-style benchmark and ours. \quanyu{this paragraph should mention the exclusive key features of the constructed benchmark.}

With \dsns, we comprehensively evaluate 8 mainstream MLLMs, including 5 open-sourced models and 3 proprietary ones. Firstly, we demonstrate that images can serve as powerful evidence for augmented generation through baseline experiments; secondly, we experiment with various RAG settings, testing the ability of MLLMs digesting images as augmentation, and analyse model behaviour on identifying clue from irrelevant images. Our benchmark sheds light on real-world scenarios for evaluating MLLMs, when users raise natural language queries for knowledge-intensive questions where textual evidence is scarce but visual evidence possibly exists, posing the challenging task of retrieving and utilizing relevant images in the wild.

%Other than image as augmentation for multimodal RAG, \ds could also potentially benefit future MLLM research in fields including but not limited to: multi-image reasoning, long-multimodal-context processing 
%\wenya{No need to mention this as the connection is not so clear. It feels more important to point out here that our benchmark significantly benefit real-world scenarios where users raise language queries for knowledge-intensive questions which require challenging retrieval from similar images in the wild}. %\quanyu{more details about how you conduct diverse experiments and how those experiments connect with your motivation. And with such benchmark, researchers can do what kind of research, like long-context, evaluating VLM ability of fetching knowledge in multiple retrieved images.}

The key findings from our experiments are summarized as follows: %\wenya{You are missing an important finding regarding the contrastive behaviors of open-sourced and proprietary models} %\wenya{As shown in my comments following all the points below, a clearer categorization of findings should be included.}
\begin{itemize}
    \item \textbf{Cross-modal retrieval is challenging}. The commonly used small cross-modal retriever (CLIP) faces significant challenges addressing difficult queries requiring the identification of fine-grained visual features. %\wenya{this is about retrieval only}
    \item \textbf{The amount of evidence affects retrieval-augmented generation.} The evaluated MLLMs demonstrate the ability to extract visual knowledge to answer questions. However, a single ground-truth clue image is insufficient for the models to achieve optimal performance.
    \item \textbf{Negatives can be helpful.} Providing multiple images as input, including both clue images and non-clue images, could enhance performance for proprietary models. This improvement likely stems from the models' ability to contrast clue and non-clue images, enabling them to better process the relationship between the question and the images.
    \item \textbf{Inverted dynamics of open-sourced and proprietary models.} Open-source models excel with a single clue image but fail to differentiate the clue when irrelevant images are also introduced; while proprietary models are initially conservative with a single clue but ultimately achieve better performance through multi-image augmentation. This reveals open-source models are weaker at handling multiple images, compared to proprietary models. %Also, when provided with highly irrelevant distractors (easy-negatives), open-source models surprisingly appear to be more confused while the clue becomes more prominent, while proprietary model remain powerful in filtering the distractors.
    %\item \textbf{Highly irrelevant distractors reduce model confidence.} By mixing irrelevant images from other irrelevant entities with clue images, the clue images might appear more distinct -- for instance, if the question is about a bird but the distractors depict fish. Whereas experiments shows that presence of irrelevant images creates additional confusion. It becomes more difficult for models to isolate and interpret the visual evidence.%\wenya{this is about augmented generation using irrelevant distractors.}
    %\item Open sourced models tested exhibit the ``lost-in-the-middle'' phenomenon, they fail to extract visual knowledge when the clue image is not positioned at the beginning of image inputs. On the contrary, proprietary models manage to extract the visual knowledge regardless of position of clue image. \wenya{still on augmented generation, but on the position of the clue image.}
\end{itemize}



\section{Related Work}
While numerous benchmarks evaluate the capabilities of MLLMs, we focus here on those designed for knowledge-intensive visual question answering (VQA), where RAG is the primary solution paradigm.  Table \ref{tab:comapre_bench} compares the modalities covered by commonly used benchmarks. A more detailed discussion of how \ds differs from prior work is provided in Appendix~\ref{sec:app_relation}.
%\vspace{-0.8em}

\paragraph{Textual Knowledge Centric VQA Benchmarks}
OK-VQA~\citep{Marino_2019_CVPR} and its augmented version A-OKVQA~\citep{10.1007/978-3-031-20074-8_9} focus on commonsense or world knowledge, though they do not strongly depend on external knowledge retrieval. ViQuAE~\citep{10.1145/3477495.3531753}, InfoSeek~\citep{chen-etal-2023-pre-trained} and Encyclopedic-VQA  (E-VQA, \citealp{Mensink_2023_ICCV}) address ``encyclopedic'' questions, paired with query images of specific entities, utilizing information retrieved from external textual sources (e.g., Wikipedia) to answer.

%There exists a series of works focusing on knowledge-intensive VQA (KI-VQA), e.g. the OK-VQA~\citep{Marino_2019_CVPR} and its augmented version A-OKVQA~\citep{10.1007/978-3-031-20074-8_9}. These benchmarks are featuring commonsense knowledge or world knowledge. Each question is paired with one image displaying a certain scenario, and the question usually requires some extent of reasoning within the scenario. Some questions are hard to be answered solely without searching knowledge from external knowledge base, though, these benchmarks' main focus is on reasoning with visual information rather than knowledge retrieval from external KB.

%KB-centric benchmarks were then proposed, including but not limited to: ViQuAE~\citep{10.1145/3477495.3531753}, InfoSeek~\citep{chen-etal-2023-pre-trained} and Encyclopedic-VQA~\citep{Mensink_2023_ICCV}. These benchmarks contain questions paired with images of certain entities, and knowledge for such entities can be found in an external textual knowledge base (e.g., Wikipedia). The questions are deliberately constructed to omit the entity names (e.g., ``How many feet tall does the plant grow to?'' instead of ``How many feet tall does Acacia paradoxa grow to?'').

%Typically, given a question and its associated image, a model is expected to: 1) identify the entity depicted in the image, 2) retrieve relevant documents or passages from a knowledge base that pertain to both the question and the entity, and 3) generate an answer based on the question, image, and retrieved documents. Step 1) and 2) by leveraging vision-language pretrained encoders, such as CLIP~\citep{pmlr-v139-radford21a}, to facilitate cross-modal (image-to-text) retrieval.
%\vspace{-0.8em}
\paragraph{Visual Knowledge Centric QA Benchmarks}
%\wenya{In summary, I do not see a very convincing justification of the difference and exciting contribution compared to the following two related works based on the descriptions. More distinct and interesting features should be emphasized.} 
While the above benchmarks all position themselves as knowledge-intensive VQA tasks, it is important to note that the required knowledge remains primarily textual, with images functioning mostly as entity anchors. In contrast, to our knowledge, only two related works truly require image retrieval as evidence for answering questions.

WebQA~\citep{Chang_2022_CVPR} is an earlier effort that requires retrieving images for visual knowledge. However, it heavily relies on captions because the images themselves usually lack indicators of which are clue images. 
%\textcolor{red}{Can delete the following example if overlength} For example, consider the query: ``\textit{Are the land dinosaurs guarded by rail in both the Display Museum of Natural History in University of Michigan and the Museo Jurassic de Asturias?}'' Without captions, the model cannot discern which fossil image belongs to which museum.
A concurrent work, MRAG-Bench~\citep{hu2024mragbenchvisioncentricevaluationretrievalaugmented} shares a similar motivation in using image retrieval for question answering. However, it primarily aims to recognize an entity when the instance in query image is incomplete or partially obscured, relying on retrieved images that present a complete view of the same entity. This entity-recognition objective differs from our focus on answering visual-feature-oriented factual queries. 
%\textcolor{red}{Delete the detailed discussion if overlength} Furthermore, non-clue images in MRAG-Bench are from different entities, which are visually distinct, in contrast to the numerous hard negatives in our \ds. 
As queries in MRAG-Bench include image, the model can use the “shortcut” of matching a visually similar retrieved candidate, bypassing the need for deeper text-to-image reasoning.
%\wenya{This comparison is not convincing to me. From Table 1, it seems they use both textual and visual features to retrieve? Can we mention this distinction? How about long-tail? And again, shall we emphasize the challenge of retrieval due to hard negatives and augmented generation due to the necessity of differentiating sophisticated features in similar images in a separate paragraph as a clear distinction from existing methods (not sure if this is indeed a distinction)?}


%While the above mentioned benchmarks are all claimed to be Knowledge-Intensive VQA benchmarks, it is noteworthy that the \textbf{knowledge} to be retrieved is textual knowledge. The image is only served as an anchor of the entity. While cross-modal retrieval was commonly used, the knowledge retrieval process can actually be achieved in uni-modal manner. A recent work~\citep{yan-xie-2024-echosight} achieved ``SoTA'' performance on InfoSeek and E-VQA employing image-to-image retrieval within a knowledge corpus comprising Wikipedia pages and their associated images. Entity recognition is performed by identifying the most visually similar Wikipedia image to the query image, which automatically enables retrieval of the  related knowledge document (i.e., the Wikipedia page containing that image). Answers are then generated using a text-only LLM, augmented with the relevant paragraph from the Wikipedia page.

%An earlier work, WebQA dataset~\citep{Chang_2022_CVPR}, also included questions that require retrieving images and utilizing visual knowledge to generate answers. However, its retrieval process is heavily reliant on image captions, as the images themselves typically lack indicators identifying them as clue images for answering a question. 

%A concurrent work, MRAG-Bench~\citep{hu2024mragbenchvisioncentricevaluationretrievalaugmented} shares a similar objective to ours by proposing a benchmark that requires retrieving visual knowledge to solve questions. However, despite this similarity, the focus remains on entity recognition. Most tasks are structured such that an image displaying an incomplete or partially visible entity is provided, and the retrieval system is tasked with identifying images that depicts the entire entity. This approach primarily enhances the MLLM's ability to recognize entities rather than addressing broader knowledge-intensive tasks.

%\subsection{OCR fashioned VQA}



\section{Benchmark Overview}
\label{sec:overview}
%\wenya{Need to reorganize this section. There are two pieces of information to deliver here; one is the characteristics of the dataset and the other is the distribution plot. For characteristics, we should make a point-wise description instead of enumerating everything without clear separation. Let's break it down. The first paragraph gives a brief summary which is ok. The second paragraph mentions two aspects, namely queries (i.e., textual, factual knowledge) and retrieval knowledge base. Under retrieval knowledge base, you talk about (1) multiple clue images and (2) hard negatives, both reflecting real-world scenarios. Then in the next paragraph, you discuss query again. This should be included in the first point when talking about queries. So for query aspect, you have two points to mention, one is text-only and the other is factual knowledge. Please use clear indicators, either a small title for each point or just numbers to separate each characteristics.}
\vspace{-0.5em}
We propose \dsns, a \textbf{text-only} Question Answering benchmark for challenging, \textbf{visual} knowledge intensive factual questions. It is an evaluation benchmark for visual knowledge centric text-to-image retrieval, as well as a benchmark for evaluating MLLMs' abilities on extracting visual knowledge as augmentation under retrieval-augmented generation settings. 

\vspace{-0.3em}
\paragraph{Visual Knowledge Intensive Queries} 
Our \ds focuses on text-only queries that probe factual knowledge of visual feature which generally hold true of an entity (e.g. ``\textit{What colour are the stamens of Chamaecrista nictitans}''), rather than instance-specific attributes commonly addressed in standard VQA tasks (e.g. ``\textit{What colour is the cup on table}''). Hence, we specifically select the organism domain which enables specialized queries. Unlike existing works, we deliberately exclude images from the queries due to the reason that including an image often reduces the problem to mere entity recognition or unimodal-similarity match, instead of our main objective of cross-modal knowledge extraction. Numerous established benchmarks already cover entity recognition, including the OVEN~\citep{Hu_2023_ICCV} dataset which serves as the foundation for the the InfoSeek dataset, and the iNaturalist 2021 (\citealp{Van_Horn_2021_CVPR}) dataset, which we employ in this work. Moreover, text-only queries better align with real-world scenarios where users tend to raise questions about an entity without an image.
%due to the task nature, if we include an image in the query, the query image shall not depict queried visual feature; however, in real-world scenarios,  chatbot users are unlikely to upload an image and ask about visual features \textbf{not} found in that image.%; rather, they tend to inquire about elements that are visible. For features that are absent, a more natural user question would be, “What is the name of this butterfly?” rather than, “What does the caterpillar of this butterfly look like?” 

\vspace{-0.3em}
\paragraph{Naturally Co-occurring Hard Negative Images} For each question, multiple clue images might be available, reflecting real-world scenarios. Additionally, the dataset incorporates a substantial number of naturally co-occurring hard negative images, i.e., images of the same species that do not exhibit the visual feature in query. Notably, all clue images for a given question can be retrieved purely based on visual information, without reliance on captions.



% Existing works~\citep{chen-etal-2023-pre-trained, Mensink_2023_ICCV} primarily focus on using images to identify entities, with the required knowledge typically existing in textual descriptions of those entities.
 %Previous work~\citep{Chang_2022_CVPR} focuses on questions requiring a single image (or two images for two-hop reasoning question) as a knowledge source, with hard negatives curated from data mining.
 %(see section \ref{sec:eval_ret} for detailed explanation). %Previous work~\citep{Chang_2022_CVPR} relies heavily on captions to identify names or locations of entities.

\vspace{-0.3em}
\paragraph{Dataset Statistics}
The current version of \ds contains 400 queries, forming an image knowledge base of total 103824 images. The average clue image count for each query is 13.89, and the clue images occupy 5.35\% of the image knowledge base. The distribution of organism categories and query types is shown in Figure \ref{fig:q_distribution}.

\begin{figure}[ht]
    \begin{subfigure}[]{\linewidth}
    \centering
        \includegraphics[trim={220px, 15px, -50px, 0px}, scale=0.3, clip]{image/test_pie.pdf}
        \caption{Distribution of organism categories in \ds}
    \end{subfigure}
    \vspace{1em}
    \setlength{\belowcaptionskip}{-0.5cm}
    \begin{subfigure}[]{\linewidth}
    \centering
        \includegraphics[scale=0.3,trim={65px, 65px, 65px, 45px},clip]{image/test_pie_qtype.pdf}
        \caption{Distribution of question categories}
    \end{subfigure}
    \setlength{\belowcaptionskip}{-0.2cm}
    \caption{Distribution of organisms and question categories.} %\wenya{Can you better align the colors? For example, in (a), you can use blue-like colors for animals and yellow-like colors for plants. Then all the subcategories should follow this color code. For (b), use less saturated colors because the texts are hard to visualize now. In addition, the box in (a) looks ugly (missing the right boundary and is too close to the figure).}
    \label{fig:q_distribution}
\end{figure}

\section{Benchmark Construction}
\label{sec:const}

\subsection{Data source}
Following Encyclopedic VQA~\cite{Mensink_2023_ICCV}, we adopt iNaturalist 2021 (iNat21, \citealp{Van_Horn_2021_CVPR}) as the cornerstone of our benchmark. iNat21 contains 2.7M images of organism (animals, plants, fungi, etc.), spanning across 10000 taxonomy species. Each species in iNat21 has 200–300 images, originally collected for species-level image classification. 

For organism species selection in query construction, we aim to ensure that the chosen species are relatively underrepresented in textual resources. To achieve that, we filter them based on the length of their Wikipedia summary and description sections. Well-known species typically have multiple sections (e.g., habitats, behaviour, etc.), indicating extensive documentation. We therefore include only those species whose summary and description account for no more than 50\% length of the entire Wikipedia article. This also helps minimize reliance on prior knowledge: widely recognized species may have detailed descriptions on visual features across LLM pre-training corpora, even if these details are not explicitly mentioned in Wikipedia.

\subsection{Query Collection}
\paragraph{Question generation.}
Our objective is to create queries about distinct visual features of organisms, ensuring that only a small fraction of images within each species display the queried feature. Each question must yield a unique answer and generally hold true for every instance of that species, preserving both specificity and accuracy. Typically, composing such specialized queries would require domain expertise and a labour-intensive review of the 200–300 images per species.

Consequently, we employ OpenAI-o1~\citep{o1} for query generation to replace human experts. As creating meaningful questions requires expert knowledge, we provide the Wikipedia summary and description sections for the model to understand the species. We explicitly instruct the model NOT to generate questions about properties already covered in these text passages, prompting it to focus on more specialized visual attributes unlikely to appear in standard references. The reasoning and self-validating ability of o1 enables it to follow the complex instructions, leading to high-quality and diverse prototype queries. The model is tasked with generating both a set of candidate queries and their corresponding visual features. A refined human annotator filtering and rewriting process is then applied based on these visual features, to ensure query validity\footnote{More details about the query generation prompt and human filtering process can be found in Appendix \ref{sec:app-anno}, \ref{sec:app_prompt}.}.


%For the 1577 species sampled, we generated $N_{question\_before\_filter}$ questions. For each species, the model generates 2-4 questions.

\paragraph{Question Filtering and Image Annotation.}
The question generation process cannot guarantee compliance with the previously mentioned requirement, i.e., clue images containing the answer should remain a minority in the set; also, query should not be so difficult that no clue image exists in the corpus. We designate a threshold clue image rate, that at most 20\% of images in a species level sub-corpus can be clue.

Firstly, we employ an open-sourced MLLM\footnote{Llama-3.2-11B-Vision-Instruct was used. Prompt template can be found in Appendix \ref{sec:app_prompt}} as a coarse-level filter. For each query, the model iterates over the sub-corpus one image at a time, binary labelling the images by whether the referenced feature is present. These labels are used to roughly filter out queries that exceed the 20\% clue-image threshold. Our rationale is that if an unusually large fraction of images exhibit a visual feature, it is likely prominent for the MLLM to detect and generate labels that are reliable enough.

Secondly, we recruit volunteer university students as annotators to label the images. These annotators identify whether each image provides evidence for the question and thereby confirm whether (a) at least one valid clue image exists, excluding overly difficult queries having no clue image and (b) fewer than 20\% of the images are clue, excluding remaining “easy” queries not recognized by the coarse-filter. At the same time, the annotators provide the ground-truth answers for each query. This two-stage process ensures our final dataset meets both the clue-image minority requirement and the fundamental criterion that each query should be indeed answerable by evidence clue image within the corpus.

In the current version of the benchmark, we initially select around 1000 questions that passed the coarse filter. After human annotation, 400 of these questions satisfy the clue-image rate threshold and are confirmed answerable with at least one valid clue image.


\section{Experimental Setup}
\label{sec:exp}

\subsection{Retrieval Evaluation}
\label{sec:eval_ret}
For retrieval, we evaluate the performance with Normalized Discounted Cumulative Gain (NDCG) and hit rate -- the probability that at least one clue image is found among the top-$k$ retrieved samples (NDCG@k and hit@k). %We omit the recall score as the number of clue images fluctuates greatly across the queries (from 1 to 60). \textcolor{red}{Or don't mention recall at all?}  %We evaluate retrieval performance under three levels of granularity:


%\textcolor{red}{As lack of labels for new queries, delete the other two settings of retrieval, only left within corpus}
%Due to the challenging domain specific queries, the performance of retrievers at whole corpus (2.7M images) or restricted corpus (sub-corpus of the 100k images that are associated with queries) is not satisfactory. Hence, we limit the retrieval corpus to the image set of 200-300 images for each query. 
Due to the challenge of complex domain-specific queries, it is significantly difficult for the retrievers to search through the entire iNat21 corpus (2.7M images) while the query is about a sophisticated feature for a single species. To enhance retrieval effectiveness and emphasize the more fine-grained and distinguishing features of a species, we narrow the search space by limiting the retrieval corpus to a set of 200–300 images corresponding to the species for each query.
In all following experiments, we adopt the commonly used retriever \verb|clip-vit-large-patch14-336|~\citep{pmlr-v139-radford21a} for image retrieval. The search indexes are constructed with the FAISS library~\citep{douze2024faisslibrary} using flat inner product index.

%\paragraph{Within Species Retrieval}
%For each question, text-to-image retrieval is performed within the image set of the corresponding species. Each species has 200–300 images, with 1–60 clue images displaying the visual features referenced in the question.

%\paragraph{Restricted Corpus Retrieval}
%For each question, text-to-image retrieval is conducted within a restricted subset of the full corpus, comprising 24,083 images that are explicitly paired with questions.

%\paragraph{Full Corpus Retrieval}
%For each question, text-to-mixed retrieval is performed over the entire iNat21 corpus of 2.7M images. Given the challenge of distinguishing between visually similar species, particularly for smaller retriever models, we incorporate the species name as a caption and transform to imagetext-to-image retrieval. %Retrieval is then based on the sum of the image embedding and the text embedding of the species name to enhance similarity search performance.



\subsection{Generation Evaluation}
\label{sec:eval_gen}

%\paragraph{AI-assisted evaluation}
Previous work has relied predominantly on exact-match or recall-based metrics to assess answer correctness. However, such methods are prone to overlooking partial hallucinations blended with ground-truth answers. For example, if the true answer is “black and white” but the model predicts “black and white with yellow dots,” the spurious mention of “yellow dots” remains undetected under exact-match criteria.
To address this limitation, we employ GPT-4o as a more nuanced evaluator. Detailed evaluation prompt template can be found in Appendix~\ref{sec:app_prompt}. A series of works, including but not limited to \citet{kamalloo-etal-2023-evaluating,zheng2023judging,huang2024empiricalstudyllmasajudgellm}, have demonstrated that general-purpose LLMs can reliably judge open-ended QA responses. We also manually validate the automatic evaluation results on a sample of 100 predicted answers, achieving a 94\% accuracy rate. %\wenya{Can you cite a relevant work showing LLM is good at open-ended answer evaluation? Provide a percentage showing the alignment between humans and GPT-4o using a small portion of the dataset.} %We provide GPT-4o with the text-only question, ground-truth answer, and predicted answer, along with several in-context examples in the prompt to leverage its in-context learning capability for more accurate evaluation.

In addition, we report the ROUGE score~\citep{lin-2004-rouge}, a metric commonly used in summarization evaluation, as a relaxed version of exact match score. The ROUGE scores are omitted in result analysis section, and can be found in Appendix~\ref{sec:app_results}. %Given that our ground-truth answers often consist of single keywords or short phrases, the ROUGE recall score accommodates variations such as word order changes or phrase segmentation, making it better suited for our dataset compared to strict exact-match criteria.

\subsection{Multimodal LLMs}
We evaluate 3 advanced proprietary models: GPT-4o~\citep{openai_chatgpt_2024}, Gemini-1.5-Pro~\citep{gemini}, Claude-3.5-Sonnet~\citep{claude}; and 5 recent open-sourced models that can process multiple images, of sizes ranging from 4B to 12B: Phi-3.5-vision-instruct (4B, \citealp{abdin2024phi3technicalreporthighly}), Qwen2-VL-7B-Instruct~\citep{Qwen2VL}, InternVL-2.5-8B~\citep{chen2024expandingperformanceboundariesopensource}, Llama-3.2-11B-Vision-Instruct~\citep{grattafiori2024llama3herdmodels}, and Pixtral-12B~\citep{pixtral}. We use the default setting given by each model to perform question answering.


\section{Results}
\subsection{Text-to-Image Retrieval}
\label{sec:ret_result}
\vspace{-0.5em}

\begin{table}[ht]
    \centering
    \resizebox{\linewidth}{!}{
    \begin{tabular}{lrrrrr}
        \toprule
        Metric &@1 &@5 &@10 &@20 &@30 \\
        \midrule
        NDCG  &26.06	&26.46	&30.67	&38.50	&45.80\\
        Hit Rate &26.06	&52.73	&64.55	&72.73	&77.87\\
        Hit Count  &0.26   &1.12   &2.08   &3.60   &4.85\\
        % \midrule
        % \multirow{3}{*}{Restricted}&Recall  &0.34	&1.91	&3.48	&6.50	&9.12\\
        % &Hit Rate &10.23	&31.82	&42.05	&57.95	&63.64\\
        % &Hit Count  &0.10	&0.63	&1.15	&2.15	&3.05\\

        % \midrule
        % \multirow{3}{*}{Full}&Recall  &0.34	&1.14	&2.34	&4.33	&5.55\\
        % &Hit Rate  &10.23	&22.73	&31.82	&42.05	&43.18\\
        % &Hit Count  &0.10	&0.35	&0.75	&1.40	&1.80\\
        \bottomrule
    \end{tabular}}
    %\setlength{\belowcaptionskip}{-0.8em}
    %\setlength{\abovecaptionskip}{0.1em}
    \caption{Retrieval Results using the sub-corpus of 200-300 images for each query. Even within the small species level corpus, the model struggles with our challenging text-to-image retrieval task.}
    \label{tab:ret_result}
\end{table}

Successfully retrieving clue images that contain the required visual knowledge is critical to the overall performance of RAG systems in answering the questions.
The hit rate highlights the difficulty of text-to-image retrieval in our {\dsns} benchmark. Even within the small sub-corpus of 200 - 300 images, there is still 22\% chance that no clue image can be retrieved within the top-30 samples. The NDCG score also indicates the challenge of this task, i.e., the retrieved clue images are not assigned top similarity scores comparing to the hard negative images. 

%We observe a notable drop in retrieval performance for the restricted corpus and full corpus setting compared to the within-species setting. Since the questions focus on the visual features of specific species, distinguishing between visually similar species poses a significant challenge, particularly for smaller retriever models.

\subsection{Retrieval-Augmented Generation}
\label{sec:rag_result}

\begin{table*}[ht]
    \centering
    \resizebox{0.85\textwidth}{!}{
    \begin{tabular}{c|ccccc|ccc}
         \toprule
         Model&Phi3.5-V&Qwen2VL&InternVL2.5&Pixtral&Llama3.2-V&GPT-4o&Gemini&Claude\\[-0.2em]
         \midrule
         \multicolumn{9}{c}{\textit{Baselines}}\\[-0.3em]
         \midrule
    Zero-shot    &27.57	&32.04	&26.17	&30.21	&35.80  &47.21	&42.78	&42.43\\
    Oracle  &32.91	&43.19	&38.58	&37.79	&41.51  &37.65	&18.56	&32.19\\
    Non-clue &27.77	&33.27	&31.53	&34.94	&31.63 &16.61	&6.25	&8.92\\[-0.3em]
         \midrule
         \multicolumn{9}{c}{\textit{Retrieved Top-K}}\\[-0.3em]
         \midrule
    K=1 &\cc{g1}33.47	&\cc{r1}39.24	&\cc{r1}34.13	&\cc{r1}37.45	&\cc{r1}34.59  &\cc{r1}28.09	&\cc{g1}19.42	& \cc{r1}18.53\\
    3   &\cc{r1}31.14	&\cc{r1}38.05	&\cc{r1}32.93	&\cc{g1}\textbf{39.44}	&\cc{r1}36.72  &\cc{r1}35.26	&\cc{g1}45.48	& \cc{r1}28.49\\
    5   &\cc{g1}\textbf{35.66}	&\cc{r1}40.24	&\cc{r1}35.52	&\cc{r1}37.25	&\cc{r1}\textbf{39.51}  &\cc{g1}40.84	&\cc{g1}45.35	& \cc{r1}30.88\\
    7   &\cc{g1}33.67	&\cc{r1}\textbf{40.44}	&\cc{r1}35.33	&\cc{r1}37.25	&\cc{r1}36.65  &\cc{g1}43.82	&\cc{g1}46.81	& \cc{g1}35.86\\
    10  &\cc{g1}33.47	&\cc{r1}39.04	&\cc{r1}\textbf{37.18}	&\cc{r1}36.85	&\cc{r1}37.18  &\cc{g1}44.02	&\cc{g1}48.40	& \cc{g1}38.84\\
    15  &\cc{g1}33.86	&\cc{r1}39.64	&\cc{r1}34.39	&\cc{r1}37.25	&\cc{r1}35.99  &\cc{g1}44.42	&\cc{g1}51.20	& \cc{g1}\textbf{45.62}\\
    20  &\cc{r1}32.27	&\cc{r1}40.84	&\cc{r1}36.59	&-	&-  &\cc{g1}\textbf{47.61}	&\cc{g1}\textbf{53.19}	& \cc{g1}42.23\\
        \hline
        Oracle  &32.91	&43.19	&38.58	&37.79	&41.51    &37.65	&18.56	&32.79\\[-0.3em]
        \midrule
        \multicolumn{9}{c}{\textit{One-in-K}}\\[-0.3em]
        \midrule
    K=3 &\cc{g1}33.03	&\cc{r1}\textbf{40.72}	&\cc{r1}\textbf{36.72}	&\cc{r1}37.21	&\cc{r1}\textbf{37.00}   &\cc{g1}43.90	&\cc{g1}45.64	& \cc{g1}32.87\\
    5   &\cc{g1}\textbf{33.97}	&\cc{r1}39.68	&\cc{r1}35.22	&\cc{g1}\textbf{39.70}	&\cc{r1}35.92 &\cc{g1}45.90	&\cc{g1}46.35	& \cc{g1}35.98\\
    7   &\cc{g1}33.35	&\cc{r1}39.36	&\cc{r1}34.80	&\cc{r1}35.66	&\cc{r1}36.53 &\cc{g1}45.31	&\cc{g1}45.46	& \cc{g1}36.60\\
    10  &\cc{r1}32.27	&\cc{r1}39.16	&\cc{r1}35.61	&\cc{r1}36.57	&\cc{r1}34.51    &\cc{g1}46.14	&\cc{g1}46.90	& \cc{g1}37.63\\
    15  &\cc{r1}32.23	&\cc{r1}39.40	&\cc{r1}34.99	&\cc{r1}37.32	&\cc{r1}34.59    &\cc{g1}\textbf{46.54}	&\cc{g1}46.58	& \cc{g1}\textbf{37.86}\\
    20  &\cc{r1}32.39	&\cc{r1}37.73	&\cc{r1}34.08	&-	&-    &\cc{g1}45.62	&\cc{g1}\textbf{49.57}	& \cc{g1}35.43\\
        \hline
        Oracle  &32.91	&43.19	&38.58	&37.79	&41.51    &37.65	&18.56	&32.79\\[-0.3em]
        \bottomrule
    \end{tabular}}
    %\setlength{\belowcaptionskip}{-1.5em}
    %\setlength{\abovecaptionskip}{0.1em}
    \caption{Main experiment results. The coloured cells shows the difference with oracle score, best performing $k$ values for each model is highlighted with \textbf{boldface}. }
    \label{tab:main_results}
\end{table*}


% \begin{table*}[ht]
%     \centering
%     \resizebox{\textwidth}{!}{
%     \begin{tabular}{c|ccccc|ccc}
%          \toprule
%          Model&Phi3.5-V&Qwen2VL&InternVL2.5&Pixtral&Llama3.2-V&GPT-4o&Gemini&Claude\\
%          \midrule
%          \multicolumn{9}{c}{\textit{Baselines}}\\
%          \midrule
%          Zeroshot   &33.52	&34.66	&26.70	&27.84	&42.05	&52.27	&48.30	&51.14\\
%          Oracle  &40.68	&43.86	&40.12	&39.32	&42.27	&47.95	&46.48	&46.48\\
%          Non-clue    &33.52	&32.61	&29.88	&32.05	&33.86	&17.84	&22.73	&12.96\\
%          \midrule
%          \multicolumn{9}{c}{\textit{Retrieved Top-K}}\\
%          \midrule
% K=1 &\cc{r4}36.93	&\cc{r3}40.91	&\cc{r4}35.80	&\cc{r3}35.80	&\cc{r3}39.20	&\cc{r5}38.07	&\cc{r5}29.55	&\cc{r5}25.00\\
% 3   &\cc{r2}39.20	&\cc{r5}38.07	&\cc{r4}35.80	&\cc{g3}42.05	&\cc{r1}41.48	&\cc{r4}43.18	&\cc{r1}44.89	&\cc{r4}36.36\\
% 5   &\cc{r1}40.34	&\cc{r2}42.61	&\cc{r1}39.77	&\cc{g2}40.91	&\cc{g2}44.32	&\cc{g2}50.57	&\cc{r1}44.89	&\cc{r3}37.50\\
% 7   &\cc{r1}40.34	&\cc{r3}40.91	&\cc{r1}39.20	&\cc{r1}38.64	&\cc{r4}38.64	&\cc{r1}47.73	&\cc{r1}44.89	&\cc{r3}38.07\\
% 10  &\cc{g4}43.75	&\cc{r4}39.77	&\cc{r5}33.52	&\cc{g2}41.48	&\cc{r4}38.07	&\cc{g4}55.11	&\cc{g1}48.30	&\cc{r1}44.32\\
% 15  &\cc{g3}42.61	&\cc{r5}38.64	&\cc{r3}37.50	&\cc{g5}44.32	&\cc{g1}43.18	&\cc{g4}53.98	&\cc{g4}51.70	&\cc{g1}47.73\\
% 20  &\cc{g1}40.91	&\cc{r3}40.91	&\cc{g2}42.61	&-	           &-      	   &\cc{g5}59.09	&\cc{g5}56.25	&\cc{r2}42.61\\
%         \hline
%         Oracle&40.68	&43.86	&40.12	&39.32	&42.27	&47.95	&46.48	&46.48\\
%         \midrule
%         \multicolumn{9}{c}{\textit{One-in-K}}\\
%         \midrule
% K=3 &\cc{g2}42.73	&\cc{r4}39.55	&\cc{r3}36.82	&\cc{g1}39.43	&\cc{r5}35.91	&\cc{g5}55.57	&\cc{g5}51.70	&\cc{r4}42.50\\
% 5   &\cc{r1}40.34	&\cc{r5}37.73	&\cc{r5}35.00	&\cc{r1}38.64	&\cc{r4}37.61	&\cc{g4}54.89	&\cc{g4}50.91	&\cc{r3}43.86\\
% 7   &\cc{r1}40.57	&\cc{r4}39.89	&\cc{r2}38.64	&\cc{r1}38.07	&\cc{r3}40.11	&\cc{g5}56.36	&\cc{g4}50.79	&\cc{r4}42.27\\
% 10  &\cc{g3}43.75	&\cc{r3}41.02	&\cc{r1}39.43	&\cc{r2}36.70	&\cc{r3}38.75	&\cc{g5}56.48	&\cc{g3}50.45	&\cc{r4}42.73\\
% 15  &\cc{r2}38.30	&\cc{r5}37.27	&\cc{r3}36.82	&\cc{r2}36.70	&\cc{r2}40.57	&\cc{g5}55.68	&\cc{g5}51.70	&\cc{r3}43.52\\
% 20  &\cc{r2}37.96	&\cc{r3}39.77	&\cc{r1}39.43	&-	&-	&\cc{g3}53.52	&\cc{g5}52.05	&\cc{r5}40.91\\
%         \hline
%         Oracle&40.68	&43.86	&40.12	&39.32	&42.27	&47.95	&46.48	&46.48\\
%         \bottomrule
%     \end{tabular}}
%     \caption{Temp caption for main results}
%     \label{tab:main_results}
% \end{table*}

\paragraph{Does image as augmentation benefit MLLMs?} 

\noindent To assess whether image can be leveraged as evidence in the RAG system, we conduct baseline experiments and compare performance under three conditions: 1) \textbf{Zero-shot}: directly prompting MLLMs with questions without retrieval; 2) \textbf{Oracle}: augmenting the question with a gold clue image; 3) \textbf{Non-clue}: augmenting each question with a non-clue image within the same species, which does not contain visual knowledge relevant to the question\footnote{For 2) and 3), the results reported are the average of 5 runs with different clue / non-clue images selected as augmentation.}, as shown in the first section of Table \ref{tab:main_results}.

All five open-sourced models exhibit performance gains of up to 15 points when using oracle image as an augmentation. This indicates the necessity of including relevant visual knowledge to answer the questions. We conclude that for open-sourced models, a single relevant image can serve as good evidence to enhance RAG performance.

Whereas interestingly, oracle performance for proprietary models drops below the zero-shot setting. We hypothesize that the single-image clue provides insufficient evidence to establish the correct visual relationship, causing the model to respond conservatively with ``I cannot decide''. These conservative responses count as failing to produce an answer, thus reducing the overall scores. Specifically, such conservative response occurs in 15.8\% of samples for GPT-4o, 7.4\% for Gemini, and 16.4\% for Claude. Meanwhile, results for the Non-clue setting indicate that the models effectively identify these non-clue images as not containing visual knowledge. Specifically, all proprietary models frequently respond with ``I don't know'' with non-clue images, leading to a substantial drop in scores comparing to the zero-shot and oracle setting.

%Most open-sourced models also experienced some performance decline compared to the oracle setting. However, for InternVL and Pixtral, performance slightly exceeded the zero-shot baseline. We attribute this phenomenon to the possibility that, even when the provided image lacks direct visual knowledge, it may assist the model in identifying the species referenced in the question. This, in turn, allows the model to leverage its prior knowledge, enabling it to make a more educated guess.

\paragraph{Can MLLMs distinguish the clue from retrieved hard negatives?}

We have shown that a relevant clue image can serve as good evidence in augmentation. However, retrieval performance for these challenging queries is far from perfect, as shown in the previous section. For RAG system in real implementation where a number of possibly irrelevant images are retrieved, does retrieval still help the final QA performance?

To explore this realistic RAG scenario, we provide the top-$k$ retrieved images, derived from Section \ref{sec:ret_result}, as augmentation to the MLLMs\footnote{For pixtral and Llama3.2-V, $k=20$ was not evaluated due to hardware limitation.}. %Image retrieval is conducted under the within-species setting, which represents the easiest retrieval scenario. 
The retrieved images are sorted in descending order based on their similarity scores. 
As shown in the second section of Table \ref{tab:main_results}, the open-source models generally lag behind their single-oracle scores, indicating limited multi-image processing capabilities. Most open-source models struggle to isolate the clue image and effectively leverage its visual information when it is mingled with hard negatives.

%However, most open-sourced models except phi3.5 show renewed improvement as $k$ further increases, reaching an optimal range at $k=20$.

In contrast, all three proprietary models exhibit a generally monotonic increase in performance as $k$ increases, except Claude which decreases at $k=20$. Notably, the performance at $k=5$ begins to surpass oracle baselines, continuing to improve along with higher $k$ values.
We attribute this improvement to the comparative evaluation enabled by providing both clue and non-clue images. Compared to a single clue image, the top-$k$ RAG setting allows the models to contrast relevant and irrelevant visual features, leading to greater confidence in extracting knowledge from the clue images, whereas a single clue image may be insufficient to fully establish its relationship to the question. However, the proprietary models fail to exceed their zero-shot scores until fairly large $k$ values, suggesting that interpreting complex visual cues remains challenging, and they may even struggle with reconciling wrongly interpreted visual knowledge with their prior knowledge gained through pre-training.

As indicated by the retrieval results, even at $k=20$, there is 27\% chance that no clue image is included in the retrieved set. This likely explains the low QA performance at smaller $k$ values, where failing to retrieve the clue image entirely negates any benefit of visual augmentation.
To investigate further, we conduct an additional experiment, constructing image sets in which exactly one clue image and $k-1$ non-clue images are provided\footnote{Both clue and non-clue images were randomly selected across 5 different combinations, results reported are the average of 5 runs for each $k$.}. The clue image is positioned at the beginning of the $k$ images to simulate the perfect retrieval scenario.

Demonstrated in the last section of Table \ref{tab:main_results}, the open-sourced models now exhibit decreasing performance as $k$ increases, further strengthen the conclusion of their limited capacity for handling multiple images simultaneously. By contrast, all three proprietary models surpass their single-image oracle baseline from $k=3$ onward and maintain near-constant performance at higher $k$ values. This suggests that proprietary models are capable of isolating the clue image from irrelevant images, regardless of the number of input images\footnote{Additional experiments on randomly positioning the clue image within the set is also carried out. The proprietary models keeps the flat performance curve against $k$, indicating there is no positional bias.}.

%While the open-sourced models largely retained their ``N''-shaped performance curves, the proprietary models exhibited almost flat performance curves. This suggests that proprietary models are capable of isolating the clue image from irrelevant images, regardless of the number of input images for $k$ values up to 20. This behavior also explains the monotonic increasing trend observed in the basic RAG setting, where a larger $k$ improves performance by increasing the chance of retrieving clue images.

\begin{table}[h]
    \centering
    \small
    \begin{tabular}{l|p{4em}p{5em}}
    \toprule
         Model& Zeroshot Score & Augmenting Text \\
         \midrule
         Phi3.5-V   &27.57  & 26.02\\
         Qwen2VL    &32.04  & 25.61\\
         InternVL2.5   &26.17  & 19.70\\
         Pixtral    &30.21  & 19.74\\
         Llama3.2-V &35.80  & 29.73\\
    \bottomrule
    \end{tabular}
    \setlength{\belowcaptionskip}{-0.3em}
    \caption{Results when augmenting retrieved Wikipedia text as evidence.}
    \label{tab:aug_text}
\end{table}


\subsection{Ablation}
\paragraph{Can the answers for \ds be found in Wikipedia?}
%It is a natural question that, descriptions of appearance of organisms can also be made in textual form, and maybe augmenting such knowledge can also let LLMs answer these questions regarding visual features?

%We collect a subset of Wikipedia articles corresponding to the 9614 species that have a matched Wikipedia page in the iNat21 dataset. %We then applied basic text chunking to these articles, splitting them into chunks of approximately 150 tokens. Each chunk was carefully bounded within a single section of the original article to preserve contextual integrity. Sections such a ``Reference'', ``See also'', ``External links'', etc., which lack informative content, were excluded. 
We construct a small textual knowledge base from Wikipedia articles for the 10k species in iNat21 dataset. This minimum knowledge base comprising approximately 97,000 chunks, each consisting of $\sim$150 tokens. The detailed collection process is described in Appendix \ref{sec:text_kb}. For text retrieval, we use a pre-trained dense text retriever, DRAGON Plus~\citep{lin-etal-2023-train}. Given that the ground-truth labels for relevant text chunks are not available, we only assess whether the retrieved text chunks correspond to the correct Wikipedia page, and the retriever achieves HitRate@10 of 97\%. Next, we present the QA performance across open-sourced models augmented with top-10 retrieved textual knowledge chunks. As shown in Table \ref{tab:aug_text}, augmenting the question with textual knowledge from Wikipedia does not improve answer quality for most models. Instead, in numerous cases, models fail to locate relevant information within the augmented text and output ``I don't know''. This behaviour in turn leads to scores lower than the zero-shot setting.

These experiments provide strong evidence that the knowledge required to answer the questions related to visual features in our benchmark is unlikely to be contained in textual knowledge bases.

% \begin{table}[h]
%     \centering
%     \resizebox{\linewidth}{!}{
%     \begin{tabular}{l|p{4em}p{4.5em}p{5em}p{6em}}
%     \toprule
%          Model& Oracle Score & Best TopK Score & Ensemble w/ ret similarity &Ensemble w/o ret similarity \\
%          \midrule
%          Phi3.5-V   &40.68  &43.75  &35.80&38.64\\
%          Qwen2VL    &43.86  &42.61  &40.91&42.05\\
%          InternVL2.5   &40.12  &42.61  &42.61&40.91\\
%          Pixtral    &39.32  &44.32  &43.18&39.77\\
%          Llama3.2-V &42.27  &44.32  &39.77&47.73\\
%     \bottomrule
%     \end{tabular}}
%     \caption{Ensemble top-30 performance}
%     \label{tab:ensemble}
% \end{table}

\paragraph{Is More Clues Always Better?}
We observed that for proprietary models, top-$k$ scores at higher $k$ values excel their 1-in-$k$ score. Similarly, the open-source models also achieve their best performance at larger $k$ value in top-$k$ setting than in 1-in-$k$.  A likely explanation is that the expected number of clue images retrieved in top-$k$ is substantially higher than when only a single clue image is guaranteed. We in turn ablate with effect of multiple clue images. Fixing $k=10$, we experiment with varying numbers of clue images with $n\in\{1,2,3,4,5\}$ ($n$-in-10). We select queries that contain at least 5 clues to enable this experiment. As shown in Figure \ref{fig:nin10}, all models exhibit increasing performance as $n$ grows, aligning with the intuition that additional clue images provide more robust visual evidence.
%From the task nature, the queries require some extent of ``induction'', that it need to see more than one instance to induce the general fact (e.g. in Figure 1 in intro, ``What color is stamen of sensitive cassia?'' -- see one image having purple color, but not sure about others.) In this sense, it naturally requires viewing multiple images, preferably multiple clue images.

\begin{figure}
    \centering
    \includegraphics[width=0.9\linewidth, trim={12px, 5px, 15px, 10px},clip]{image/nin10.pdf}
    %\setlength{\abovecaptionskip}{-1.2em}
    %\setlength{\belowcaptionskip}{-1em}
    \caption{Impact from number of clue images}
    \label{fig:nin10}
\end{figure}

\paragraph{Hard-negatives can be helpful; how about easy-negatives?}
As an ablation to the previous experiment on whether multiple images benefit the overall QA performance, we now mix one clue image with $k-1$ highly irrelevant images from subsets that are not the query species. The results indicate that both open-source and proprietary models benefit more from hard negatives coming from the same species compared with easy negatives coming from different species. This further strengthen our conclusion that hard negatives are helpful in contrasting the queried features to inform the correct answer.
%have instead been confused by the highly irrelevant images, underlining our previous observation that models need to reason across multiple images of the queried species to confirm which visual feature is related to the question. 
Detailed results can be found in Table \ref{tab:diff_distractor} in Appendix.


%\paragraph{\wenya{Same here, change to questions} Ensemble of single retrieved image input does not fit problem setting in \ds} \wenya{This is still interesting. You can just summarize your findings here and move all the details to Appendix as this is not high priority.}

%In earlier text-based RAG models, a popular approach involves having the generator LLM process one piece of the top-$k$ retrieved documents at a time as augmentation, followed by generating predictions through an ensemble of $k$ rounds. The ensemble can be implemented in two ways: Token-by-token aggregation—calculating the log probabilities of each token at each position across all $k$ branches; or Sequence-level voting—generating $k$ output sequences. These methods were primarily developed to overcome the input sequence length limitations of earlier LLMs, which typically could not handle sequences exceeding 4096 or 8192 tokens. Also, introducing the retrieval similarity may benefit model in assessing relevance of knowledge documents to the query. Here, we apply the sequence level voting strategy.

%However, as demonstrated in our baseline experiments, the performance of the one-oracle clue image setting is generally outperformed by the standard retrieval setting. Consequently, we do not expect the ensemble setting to achieve substantially better performance than the oracle setting, and we do observe QA scores in ensemble setting not surpassing normal top-$k$ setting as shown in Table \ref{tab:ensemble}.

%\textcolor{violet}{When multiplying softmax normalized retrieval probability, there hardly exists candidate after top-3, and the last most candidate ever selected is the 7th, for top-30 ensemble.}


\section{Threats to Vadility}
\label{sec: vadility}
\begin{itemize}
    % \item \textbf{Construct Validity}: We use BLEU \modify{and ROUGE?} to assess the quality of the generated log messages. Although text similarity metrics may not fully capture the quality of the generated text~\cite{Gros2020CodeTC, Roy2021ReassessingAE}, we follow previous works~\cite{xie2024fastlog, Xu2024UniLogAL, He2018CharacterizingTN, Ding2023LoGenTextPlusIN} to use BLEU as widely-accepted quality measures for the generated log messages. We also adapt Cosine Similarity based on TF-IDF to evaluate the quality of log content. Since log content, excluding the log header, is primarily composed of natural language, cosine similarity effectively captures the semantic meaning~\cite{Salton1975AVS}.
    \item \textbf{Construct Validity}: The first threat comes from the use of two widely-accepted syntactic metrics in previous logging research (BLEU and ROUGE)~\cite{xie2024fastlog, Xu2024UniLogAL, He2018CharacterizingTN, Ding2023LoGenTextPlusIN}, as they may not fully capture the semantics of the log messages~\cite{Gros2020CodeTC, Roy2021ReassessingAE}. To mitigate this threat, we also adapt Cosine Similarity to evaluate the semantic quality of log messages.  Since all log headers are excluded, leaving log messages primarily in natural language, Cosine Similarity can effectively capture the log's semantic meaning~\cite{Salton1975AVS}.
    \item \textbf{Internal Validity}: The second threat to validity concerns reproducing the baseline. To minimize inconsistencies, we adapted the released models from previous work~\cite{xie2024fastlog, Mastropaolo2022UsingDL, Mastropaolo2023LogSG} and sought guidance from the authors of the closed-source model UniLog~\cite{Xu2024UniLogAL} to reproduce the tool under supervision.
    \item \textbf{External Validity}: 
    % In this work, we evaluate the tools with 10 widely recognized Java projects in different domains. The results of each tool are examined in different requirements of logging practices. The novel evaluation method we propose is not limited to Java projects, we will explore its effects in other programming languages in the future. 
    {The third threat comes from the potential data contamination risks in our dataset. 
    % we adopt the methodology proposed by UniLog~\cite{Xu2024UniLogAL}.
    To mitigate it, we follow the methodology used by GPT-3~\cite{Brown2020LanguageMA} to assess data contamination severity, i.e., considering test samples with more than a 13-gram overlap with the original code samples as contaminated.
    % Following the established GPT-3 contamination assessment protocol 
    Specifically, we perform a 13-gram overlap analysis between test and training samples of all evaluation methods. Results indicate a peak contamination rate of 1.24\% across all training datasets, empirically demonstrating no evidence of data leakage in our framework.} 
\end{itemize}


\section{Conclusion}
\label{sec: conclusion}
% This paper introduces \methodname, designed specifically for automatic logging tools. It includes a high-quality dataset and a novel dynamic evaluation method focused on runtime logs, addressing key limitations of prior studies and bridging the gap between real-world requirements and existing evaluation frameworks. This dynamic evaluation method assesses both the compilability of predicted log statements and their effectiveness in generating runtime logs. Using \methodname, we evaluate popular end-to-end automatic logging tools and find that generated log statements fail to compile in 20.1\% to 83.6\% of cases. Even the best predictions achieve only 0.213 cosine similarity between generated and ground-truth runtime logs. These results show that automatic logging still has a long way to go.

In this paper, we introduced \methodname, a comprehensive benchmark for automatic logging tools evaluation, featuring a diverse dataset from 10 established projects and a novel dynamic evaluation methodology. Our benchmark revealed significant limitations in existing tools through both static and dynamic perspectives: substantial accuracy drops in log position, level, and message prediction compared to reported results, high compilation failure rates (20.1\%-83.6\%), and low runtime log similarity (maximum 21.32\%). These findings highlight critical gaps between current tool capabilities and practical requirements. We believe \methodname establishes a solid foundation for advancing automatic logging research by providing standardized evaluation metrics and exposing areas requiring improvement.

\section{DATA AVAILABILITY}
All the code and data used in our study are publicly available on \url{https://github.com/shuaijiumei/logging-benchmark-scripts}.

\bibliography{acl_latex}

\appendix

\section{Relation with previous works}
\label{sec:app_relation}

Knowledge intensive VQA benchmarks focusing textual knowledge, including but not limited to: ViQuAE~\citep{10.1145/3477495.3531753}, InfoSeek~\citep{chen-etal-2023-pre-trained} and Encyclopedic-VQA~\citep{Mensink_2023_ICCV}, are discussed in related works. These benchmarks contain questions paired with images of certain entities, and knowledge for such entities can be found in an external textual knowledge base (e.g., Wikipedia). The questions are deliberately constructed to omit the entity names (e.g., ``How many feet tall does the plant grow to?'' instead of ``How many feet tall does Acacia paradoxa grow to?'').

Typically, given a question and its associated image, a model is expected to: 1) identify the entity depicted in the image, 2) retrieve relevant documents or passages from a knowledge base that pertain to both the question and the entity, and 3) generate an answer based on the question, image, and retrieved documents. Step 1) and 2) is usually carried out by leveraging vision-language pretrained encoders, such as CLIP~\citep{pmlr-v139-radford21a}, to facilitate cross-modal (image-to-text) retrieval.

While the above mentioned benchmarks are all claimed to be Knowledge-Intensive VQA benchmarks, it is noteworthy that the \textbf{knowledge} to be retrieved is textual knowledge. The image is only served as an anchor of the entity. While cross-modal retrieval was commonly used, the knowledge retrieval process can actually be achieved in uni-modal manner. A recent work~\citep{yan-xie-2024-echosight} achieved ``SoTA'' performance on InfoSeek and E-VQA employing image-to-image retrieval within a knowledge corpus comprising Wikipedia pages and their associated images. Entity recognition is performed by identifying the most visually similar Wikipedia image to the query image, which automatically enables retrieval of the related knowledge document (i.e., the Wikipedia page containing that image). Answers are then generated using a text-only LLM, augmented with the relevant paragraph from the Wikipedia page, which is extracted through fine-grained reranking.

WebQA dataset~\citep{Chang_2022_CVPR} included questions that require retrieving images and utilizing visual knowledge to generate answers. However, its retrieval process is heavily reliant on image captions, as the images themselves typically lack indicators identifying them as clue images for answering a question. For instance, considering the question: ``\textit{Are the land dinosaurs guarded by rail in both the Display Museum of Natural History in University of Michigan and the Museo Jurassic de Asturias?}'' Without captions, the model would be unable to discern which dinosaur fossil image corresponds to which museum.

Table \ref{tab:ret_webqa} shows the retrieval scores in our implementation of retrieval experiments on WebQA benchmark for queries having images as knowledge source. While the queries are text-only, aligning with our setting, the text-to-image retrieval scores significantly lag behind text-to-caption retrieval, especially for lower $k$.

\begin{table}[h]
    \centering
    \resizebox{\linewidth}{!}{
    \begin{tabular}{l|ccccc}
    \toprule
        Recall  & @1  &@5 &@10    &@20    &@50 \\
        \midrule
        Text-to-Image &15.17	&37.08	&47.63	&58.43	&72.32\\
        Text-to-Caption &43.25	&68.74	&76.86	&83.79	&89.92\\
    \bottomrule
    \end{tabular}}
    \caption{Retrieval scores for WebQA dataset.}
    \label{tab:ret_webqa}
\end{table}

MRAG-Bench~\citep{hu2024mragbenchvisioncentricevaluationretrievalaugmented} shares a similar objective to ours by proposing a benchmark that requires retrieving visual knowledge to solve questions. However, despite this similarity, the focus remains on entity recognition. Most tasks are structured such that an image displaying an incomplete or partially visible entity is provided, and the retrieval system is tasked with identifying images that depicts the entire entity. This approach primarily enhances the MLLM's ability to recognize entities rather than addressing broader knowledge-intensive tasks. Furthermore, non-clue images in MRAG-Bench are from different entities, which are visually distinct, in contrast to the numerous hard negatives in our \ds. 

Concurrent to our work, the iNaturalist team released 2024 version of their natural image dataset, iNat24~\citep{vendrow2024inquire}, coming with a text-to-image retrieval benchmark named INQUIRE, containing 250 queries. Though, the queries in INQUIRE are designated for image retrieval only, and are not QA style questions that have explicit answers. As iNat24 and iNat21 share the same taxonomy classes of organisms, we do consider adding the new images in iNat24 to our benchmark.

We also note a related line of research~\citep{10.1609/aaai.v37i11.26598,Van_Landeghem_2023_ICCV} on extracting knowledge from document images (e.g., PDF files, Powerpoint slides), conventionally referred to as OCR-based VQA. Although these tasks also require models to retrieve relevant page and interpret images to answer textual queries, the underlying knowledge is primarily textual—obtained via OCR. This focus diverges considerably from our goal of visual feature extraction, where visual cues are inherently sparser in the wild.
\section{Construction of Minimum Textual Knowledge Base}
\label{sec:text_kb}

Among the 10000 taxonomy species in iNat21 dataset, we matched 9614 Wikipedia pages. iNat21 provides the scientific name as well as one common name of the species. We conduct multiple rounds of matching: initially, Wikipedia articles are matched based on species scientific names with Wikipedia titles; for unmatched species, match their common names with article titles; further unmatched species will go through matching within article content. All articles containing the scientific name are collected, sorting according to the position of the scientific name's first occurrence within the article. The article with earliest occurrence is selected as matched page. Due to the fact that taxonomy study frequently change the attribution of a species, the scientific name in iNat21 dataset can be deprecated and no matched page can be found for them.

We then applied basic text chunking to these articles, splitting them into chunks of approximately 150 tokens. Each chunk is bounded within a single section of the original article to preserve contextual integrity. Sections such as ``Reference'', ``See also'', ``External links'', etc., which lack informative content, were excluded. The chunks are cut upon period mark, so that they won't end or start with incomplete sentence. Two adjacent sections, having total length less than 150 tokens will be merged, with the section titles kept.

\section{Annotation details}
\label{sec:app-anno}

For the query generation, we mentioned that we have gone through a human annotator originated refined filtering and rewriting process. The prompt template is shown in Table \ref{tab:prompt_qgen}, and following shows the example prototype queries generated with the sample prompt:

\begin{enumerate}
    \item Does rescuegrass (scientific name: Bromus catharticus) have a visible ligule when the leaf sheath is gently parted, and if so, what is its shape? (Explanation: The ligule is a thin membrane at the junction of leaf blade and sheath, typically hidden in standard photos) (Feature: shape of ligule) \textcolor{red}{Dropped. Too detailed and hard to distinguish}

    \item During the flowering stage of rescuegrass (scientific name: Bromus catharticus), what color do the exposed anthers display? (Explanation: The anthers are only briefly visible and require a closer or angled view) (Feature: color of anthers) \textcolor{teal}{Passed. Annotator found a image of flower anther (Figure \ref{fig:sample_qgen})  and confirm it is rare image.}

    \item Is there any distinct pattern or texture on the backside of the spikelets of rescuegrass (scientific name: Bromus catharticus), observable only when the spikelets are lifted or spread apart? (Feature: pattern on backside of spikelets) \textcolor{red}{Dropped. Viewing the spikelets image, there is no ``backside'' on spikelets as they are symmetric.}

    \item When the plant is fully mature, how does the collar region (the area where the leaf blade meets the sheath) of rescuegrass (scientific name: Bromus catharticus) appear from the underside view? (Explanation: This area may be covered by overlapping leaves in typical photos) (Feature: shape/appearance of collar region) \textcolor{red}{Dropped. Query too obscure and likely no clue image can be found.}
\end{enumerate}

\begin{figure}[h]
    \centering
    \includegraphics[width=\linewidth]{image/sample_qgen.jpg}
    \caption{Image of Bromus catharticus flower anther. Owned by original uploader at www.inaturalist.org, shared under CC-BY-NC license.}
    \label{fig:sample_qgen}
\end{figure}

Human annotator carefully check the candidate prototype queries by browsing all available material online with the scientific name as search keyword, as well as briefly scan through the image corpus by viewing the image thumbnails.

After query generation and rewriting, and coarse-level filtering described in section \ref{sec:const}, we dispatch the validated queries to the next round of human annotation of image labels.
Figure \ref{fig:anno_interface} shows the annotation interface for image labelling.\footnote{Similar interface is used for query filtering and rewriting annotator to browse, except the image matrix is more dense and displaying only thumbnail sized images.} Annotators are required to only label ``Y'', to indicate that this image contain the queried feature and is a clue to the query. Leaving the checkbox blank will be considered as negative label. This design aims to save annotators' effort, as clue images are assumed to be the minority. The rough clue image rate predicted by MLLM in the previous step is also provided (``Y\_Rate: 0.003'' after the question text) as reference for annotators to estimate the query difficulty. In the meanwhile, annotators provide query answers upon finding a clue image, as shown in Figure \ref{fig:anno_interface_ans}. Annotators are allowed to skip for a query when finding more than 9 clues after the first 27 images, indicating this query likly fail the clue image threshold. The order of images are randomly shuffled.


\begin{figure*}
    \centering
    \includegraphics[width=\linewidth]{image/anno_interface.png}
    \caption{Annotation interface for image labelling}
    \label{fig:anno_interface}
\end{figure*}

\begin{figure*}
    \centering
    \includegraphics[width=\linewidth]{image/anno_interface_answer.png}
    \caption{Annotation interface for image label and query answer}
    \label{fig:anno_interface_ans}
\end{figure*}


\section{Prompt Template}
\label{sec:app_prompt}

This section list the prompt templates used through out our experiments.

\onecolumn
\begin{longtable}{p{0.95\linewidth}}
\label{tab:prompt_qgen}\\

\toprule[2pt]
\textbf{Prompt Template for Query Generation}\\ \midrule
\endfirsthead

\midrule
\endhead

\hline \multicolumn{1}{c}{{Continued on next page}} \\ \hline
\endfoot

\bottomrule\\[-0.8em]
\caption{Prompt Template for Query Generation. Texts in \textit{\textcolor{violet}{italic and purple}} are not part of prompt.}
\endlastfoot


Given an organism's scientific taxonomy name and common name (if available), please generate some questions regarding its visual feature, like color, shape, pattern, texture, etc. Please limit the question asked to be regarding features that may not be usually found in photos. This can be regarding the following but not limited to: uncommon-to-see parts of animals or plants / parts that can only be viewed at specific posture or require photo taken at specific angle / less commonly seen stage of life cycle of incect, etc. (e.g. ``What is the color of paw pad of giant panda'', photos showing the paw pad of giant panda would be only a small portion; ``What pattern is on the back of brown peacock's larva?'', asking for the larva form of this butterfly.) Questions regarding non-visual feature should not be asked. (e.g., ``how long can giant panda live'', ``what does puma eat'' are not suitable questions, as such information cannot be extracted from a photo of that animal). Please avoid questions regarding exact length/area (of tails, ears, leaf, etc.), as such attribute is hard to be estimated from photo. Please avoid questions regarding eggs, nests, nets, etc., and focus on the organism's body. Please avoid generating multiple paraphrased questions regarding a same feature or body part. Please avoid questions comparing another species / form, but you can compare within the same individual (e.g. Does the upper beak or lower beak of XXX bird have darker color?).\\Except these constraints, try to generate ``interesting'' questions: for example, photos showing underside of leaf may be rare, but asking what color is underside of leaf is not interesting -- it's green anyway. Hence after generating each question, try to check if the questions is ``interesting'' or not. Please avoid asking for features that ``technically'' appear on a photo but hard to see (e.g., ``What color is the tiny hair on the surface of XYZ plant's stem?'' Any photo displays the stem will show the tiny hair, but it needs to be zoomed in / photo taken at very close, to see. The desired questions is regarding features that is hidden in some photos, not features that appear in every photo but simply too small to see).\\A likely related paragraph from Wikipedia is provided for you to better understand the organism. Please avoid question regarding visual features explicitly mentioned in the Wikipedia paragraphs. (e.g., paragraph: ``The seeds are colored in red'' -> Don't ask ``What color is the seed''; paragraph: ``The plant have symbolic seed pods'' (but doesn't mention its color) -> You may ask ``What shape/color is the seed pod'')\\Following are format instructions. You don't need to give answer to the questions.Please generate questions by asking the common name and add scientific name in brackets, and indicate the questioned visual feature after the question sentence, e.g.: ``What color is *** of \{common\_name\} (scientific name: \{scientific\_name\})? (Feature: color of ***)''; if there is no common name, simply ask with the scientific name. If explanation to the question is needed (such as explain a technical term), do not start a new line, append the explanation after the question enclosed by small brackets ().\\\\
Following are a few examples:
\\\\Scientific name: Diastictis fracturalis\\Common name: fractured western snout moth\\Question:\\1. What pattern, if any, can be observed on fractured western snout moth's (scientific name: Diastictis fracturalis) underside of wing during flying? (Explanation: the moth's wing usually cover its whole abdomen when resting, and underside will be less commonly appear in photos) (Feature: pattern on underside of wing during flying)\\2. What is the common color of larva fractured western snout moth (scientific name: Diastictis fracturalis)? (Feature: color of larva)?
\\\\Scientific name: Tortula muralis\\Common name: wall-screw moss\\Question:\\1. What is the common color of tip of leaf of wall-screw moss (scientific name: Tortula muralis)? (Explanation: The moss can be distinguished from similar mosses by its \"hair-pointed\" leaves.) (Feature: color of leaf tip)
\\\\Scientific name: Ptiliogonys cinereus\\Common name: Gray Silky-flycatcher\\Question:\\1. What color are the undertail covert (short plumages that cover the long tail feather) of Gray Silky-flycatcher (scientific name: Ptiliogonys cinereus) when viewing from front? (Feature: color of undertail covert viewing from front)\\2. What pattern can be observed on the tail feather of Gray Silky-flycatcher (scientific name: Ptiliogonys cinereus) from dorsal (back) view? (Feature: pattern on tail feather from dorsal view)

\\
\textcolor{violet}{... \textit{More ICL Examples omitted} ...}
\\\\

%\\\\Scientific name: Puma concolor\\Common name: Cougar\\Question:\\1. Are cougars (scientific name: Puma concolor) fangs in upper and lower jaw having roughly similar length, or one side's obviously longer than another? (Explanation: only photos showing fully opened mouth, displaying both upper and lower fang can answer this question)\\\\Scientific name: Adiantum aleuticum\\Common name: western maidenhair fern\\Question:\\1. Are the sori (spore-producing structures) of the western maidenhair fern (scientific name: Adiantum aleuticum) round or elongated?\\\\Scientific name: Oxalis pes-caprae\\Common name: Bermuda-buttercup\\Question:\\1. Does Bermuda-buttercup (scientific name: *Oxalis pes-caprae*) have distinct nectar guides in its flower? (Explanation: the nectar guides are patterns usually near the center of petal to guide pollinators like bees to the pollen.)\\\\Scientific name: Arothron hispidus\\Common name: Stripebelly Puffer\\Question: 1. Are the dot pattern appearing on the back also present on the ventral side of Stripebelly Puffer (scientific name: Arothron hispidus)?\\\\
End of examples.
\\\\\\Now, please generate questions for the given scientific name.\\
\textcolor{violet}{\textit{Example new query}}\\
Scientific name: Bromus catharticus\\Common Name: Rescuegrass\\Wiki Paragraph:\\Bromus catharticus is a species of brome grass known by the common names rescuegrass, grazing brome, prairie grass, and Schrader's bromegrass. The specific epithet catharticus is Latin, meaning cathartic. The common name rescuegrass refers to the ability of the grass to provide forage after harsh droughts or severe winters. The grass has a diploid number of 42 ...... \textcolor{violet}{\textit{Omitted due to space. Actual prompt includes whole description passage.}}\\ \\Question:\\
\end{longtable}

%%%%%%%%%%%%%%%%%%%%%%%%%%%%%%%%%%%%%%%%%%%%%%%%%%%%

\begin{longtable}{p{0.95\linewidth}}
\label{tab:prompt_qa}\\

\toprule[2pt]
\textbf{Prompt Template for Main Question Answering Experiments of Different Settings}\\ \midrule[1.2pt]
\endfirsthead

\midrule
\endhead

\hline \multicolumn{1}{c}{{Continued on next page}} \\ \hline
\endfoot

\bottomrule[1.2pt]\\[-0.8em]
\caption{Prompt Template for Main Question Answering Experiments of Different Settings.}
\endlastfoot


\textbf{Zero-shot}\\
\midrule
Please answer the question regarding a visual feature of an organism (animal, plant, etc.). Please follow the answer format: ``Answer: \{answer\_text\}''\\\\Question: \\
\midrule[1.2pt]

\textbf{Oracle (one clue)}\\
Please answer the question regarding a visual feature of an organism (animal, plant, etc.). You will be provided with a image regarding that organism, it is likely to contain the key information for answering the question. Please follow the answer format: ``Answer: \{answer\_text\}''\\\\Question:\\<image>\\
\midrule[1.2pt]

\textbf{Non-Clue}\\
Please answer the question regarding a visual feature of an organism (animal, plant, etc.). You will be provided with a image regarding that organism. Please follow the answer format: ``Answer: \{answer\_text\}''\\\\Question:\\<image>\\
\midrule[1.2pt]

\textbf{Multiple Images}\\
Please answer the question regarding a visual feature of an organism (animal, plant, etc.). You will be provided with several images, all of them are regarding that organism, but not all images contain the key information for answering the question. Only one to few images, or even none of them can be used to answer. Please follow the answer format: ``Answer: \{answer\_text\}''\\\\Question:\\<image> <image> <image>...\\


\end{longtable}


%%%%%%%%%%%%%%%%%%%%%%%%%%%%%%%%%%%%%%%%%%%%%%%%%%%%

\begin{longtable}{p{0.95\linewidth}}
\label{tab:prompt_eval}\\

\toprule[2pt]
\textbf{Prompt Template For Automatic Evaluation}\\ \midrule
\endfirsthead

\midrule
\endhead

\hline \multicolumn{1}{c}{{Continued on next page}} \\ \hline
\endfoot

\bottomrule\\[-0.8em]
\caption{Prompt Template For Automatic Evaluation. Note that for evaluation, no images is provided, and it becomes pure-text open-ended QA evaluation. Texts in \textit{\textcolor{violet}{italic and purple}} are not part of prompt.}
\endlastfoot

Please evaluate the answer to a question, score from 0 to 1. The reference answer is provided, and the reference is usually short phrases or a single keyword. If the student answer is containing the keywords or similar expressions (including similar color), without any additional guessed information, it is full correct. If the student answer have missed some important part in the reference answer, please assign partial score. Usually, when there are 2 key features and only 1 is being answered, assign 0.5 score; if there are more than 2 key features, adjust partial score by ratio of correctly answered key feature. The reference answer can be in the form of a Python list, in this case, any one of the list item is correct. \\If student answer contain irrelevant information not related to question, mark it with ``Redundant'', but it does not affect score if related part are correct. (e.g. Question: what shape is leave of Sanguinaria canadensis, Student Answer: shape is xxx, color is yyy, this is Redundant answer)\\If student answer contain features not listed in reference answer, mark it with ``Likely Hallucination'' and deduct 0.5 score. (e.g., Reference Answer: black and white. Student Answer: black white, with yellow dots, ``yellow dots'' is not mentioned in reference)\\Separate the remarks with score using ``|'', that is, use the syntax of: ``Score: {score} | Likely Hallucination'', ``Score: \{score\}'', ``Score: \{score\} | Likely Hallucination | Redundant'', etc. If any explanation on why giving the score is needed, do not start a new line and append after remark with brackets, e.g. ``Score: \{score\} | Redundant | (Explanation: abc)''.\\\\Following are few examples:\\\\
Question: Is there any specific color marking around the eyes of a semipalmated plover (scientific name: Charadrius semipalmatus)?\\Reference Answer: black eye-round feather, white stripe above eyes, sometimes connected to the white forehead\\\\Student Answer: Yes, the bird has a distinctive black line that runs through the eye, which is a key identifying feature.\\Score: 0 | Likely Hallucination\\\\Student Answer: They have a black vertical band in front of the eye, a white band above the eye, and a single black band that wraps partially around the eye, creating a partial ``mask'' appearance.\\Score: 1\\\\Student Answer: Yes, the semipalmated plover has a distinctive black/dark ring around its eye, surrounded by a bright white ring or patch\\Score: 0.5 | Likely Hallucination (Explanation: not white ring, but only a line above the eye)\\\\
Question: ...\\
\textcolor{violet}{... \textit{More ICL Examples omitted} ...}\\
Now, Score the following question:\\\\

\end{longtable}


\twocolumn
\section{Detailed Results}
\label{sec:app_results}

This section contains additional experimental results. We also include ROUGE-1 recall scores\footnote{Implemented with https://pypi.org/project/rouge/, version 1.0.1} for readers who are interested. However, because ROUGE relies on strict word matching, the results for open-ended QA can vary significantly due to the different generation styles across models.

\begin{table}[h]
    \centering
    \resizebox{\linewidth}{!}{
    \begin{tabular}{l|cc|cc|cc}
    \toprule
         Model& \multicolumn{2}{c}{Qwen2VL} &\multicolumn{2}{c}{Llama3.2-V} &\multicolumn{2}{c}{GPT-4o} \\
         & Same & Irrel &Same &Irrel &Same &Irrel\\
         \midrule
         k=3& 40.72 &36.27 &37.00 &36.75 &43.90 & 39.79\\
         5& 39.68   &37.53  &35.92  &32.16  &45.90  & 45.03\\
         7& 39.36   &35.66  &36.53  &29.55  &45.31  &40.36\\
         10& 39.16  &33.30  &34.51  &27.80  &46.14  &45.29\\
         15& 39.40  &35.56  &34.59  &28.19  &46.54  &45.64\\
         20& 37.73  &32.12  &-  &-  &45.62  &44.50\\
    \bottomrule
    \end{tabular}}
    \caption{Impact from using highly irrelevant distractor in 1-in-k setting. ``Same'' denotes hard-negatives from same species, ``Irrel'' denotes irrelevant images from other species.} %Irrelevant distractor experiment only conducted for one run, and appear some fluctuations.}
    \label{tab:diff_distractor}
\end{table}

\begin{table*}[t]
    \centering
    \resizebox{\textwidth}{!}{
    \begin{tabular}{l|ccccc|ccc}
    \toprule
         Setting&Phi3.5-V&	Qwen2VL&	InternVL2.5&	Pixtral&	Llama3.2-V&	GPT-4o&	Gemini&	Claude  \\
         \midrule
        Zeroshot   &27.57&	32.04&	26.17&	30.21&	35.80&	47.21&	42.78&	42.43\\
         
         \midrule
\multirow{5}{*}{Oracle}&
33.67&	44.82&	38.38&	35.96&	43.89&	38.45&	19.6&	33.67\\
&34.26&	43.63&	37.05&	37.25&	42.82&	39.24&	18.49&	32.47\\
&30.88&	43.82&	39.37&	39.44&	37.12&	37.25&	18.22&	27.69\\
&32.67&	40.64&	39.97&	39.04&	45.08&	36.25&	19.55&	33.47\\
&33.07&	43.03&	38.11&	37.25&	38.65&	37.05&	16.94&	33.67\vspace{0.3em}\\
\midrule
AVG&32.91&	43.19&	38.58&	37.79&	41.51&	37.65&	18.56&	32.19\\
STD&1.285&	1.563&	1.136&	1.433&	3.449&	1.188&	1.096&	2.566\\

\midrule
\multirow{5}{*}{Non-clue}&
26.69&	33.27&	31.21&	33.27&	30.28&	14.74&	4.46&	8.17\\
&27.49&	34.26&	34.06&	35.86&	34.66&	17.73&	7.41&	8.76\\
&30.28&	33.86&	32.07&	37.05&	29.88&	17.93&	6.71&	10.16\\
&27.09&	33.47&	29.68&	34.66&	32.53&	16.53&	7.07&	8.96\\
&27.29&	31.47&	30.61&	33.86&	30.81&	16.14&	5.59&	8.57\vspace{0.3em}\\
\midrule
AVG&27.77&	33.27&	31.53&	34.94&	31.63&	16.61&	6.25&	8.92\\
STD&1.435&	1.073&	1.663&	1.527&	1.972&	1.296&	1.211&	0.750\\
\bottomrule
    \end{tabular}}
    \caption{Baselines detailed results}
    \label{tab:baseline_detail}
\end{table*}

\begin{table*}[t]
    \centering
    \resizebox{\textwidth}{!}{
    \begin{tabular}{l|ccccc|ccc}
\toprule
1 in K&	Phi3.5-V&	Qwen2VL&	InternVL2.5&	Pixtral&	Llama3.2-V&	GPT-4o&	Gemini&	Claude\\
\midrule
\multirow{5}{*}{k=3}&	32.27&	40.44&	37.58&	37.45&	39.04&	41.63&	42.94&	32.27\\
&32.27&	39.64&	36.52&	35.26&	34.79&	44.22&	49.09&	32.67\\
&34.06&	40.24&	37.71&	37.85&	34.53&	44.02&	47.43&	35.26\\
&33.86&	41.63&	36.12&	39.64&	38.84&	43.03&	41.99&	30.68\\
&32.67&	41.63&	35.65&	35.86&	37.78&	46.61&	46.75&	33.47\\
\midrule
AVG&33.03&	40.72&	36.72&	37.21&	37.00&	43.90&	45.64&	32.87\\
STD&0.871&	0.885&	0.903&	1.732&	2.187&	1.828&	3.039&	1.678\\
\midrule
\multirow{5}{*}{k=5}&33.86&	39.84&	34.39&	40.93&	37.78&	46.81&	47.8&	37.25\\
&33.20&	39.64&	34.46&	38.05&	33.20&	46.02&	46.54&	34.86\\
&34.46&	40.24&	35.45&	44.2&	35.06&	47.01&	46.61&	36.06\\
&33.67&	39.64&	35.99&	36.85&	37.12&	44.42&	43.61&	35.26\\
&34.66&	39.04&	35.79&	38.45&	36.45&	45.22&	47.18&	36.45\\
\midrule
AVG&33.97&	39.68&	35.22&	39.70&	35.92&	45.90&	46.35&	35.98\\
STD&0.594&	0.434&	0.748&	2.923&	1.825&	1.087&	1.613&	0.950\\
\midrule
\multirow{5}{*}{k=7}&31.08&	38.65&	35.45&	34.66&	37.91&	43.03&	46.25&	37.4\\
&32.27&	39.44&	34.13&	37.05&	35.39&	44.02&	44.97&	35.88\\
&32.27&	38.84&	36.32&	33.47&	34.59&	45.29&	44.71&	36.83\\
&36.85&	39.44&	35.06&	36.65&	35.85&	46.22&	45.66&	35.88\\
&34.26&	40.44&	33.06&	36.46&	38.91&	48.01&	45.71&	37.02\\
\midrule
AVG&33.35&	39.36&	34.80&	35.66&	36.53&	45.31&	45.46&	36.60\\
STD&2.267&	0.699&	1.253&	1.530&	1.809&	1.935&	0.618&	0.690\\
\midrule
\multirow{5}{*}{k=10}&32.27&	39.04&	35.72&	37.00&	33.53&	45.62&	46.73&	34.15\\
&33.07&	39.64&	37.05&	37.10&	34.45&	44.82&	48.79&	37.65\\
&32.87&	39.84&	35.12&	37.70&	36.39&	46.41&	49.66&	38.25\\
&31.47&	38.45&	35.78&	36.23&	33.20&	47.61&	43.54&	39.24\\
&31.67&	38.84&	34.39&	34.82&	34.99&	46.22&	45.76&	38.84\\
\midrule
AVG&32.27&	39.16&	35.61&	36.57&	34.51&	46.14&	46.90&	37.63\\
STD&0.707&	0.573&	0.980&	1.109&	1.269&	1.031&	2.440&	2.034\\
\midrule
\multirow{5}{*}{k=15}&31.27&	40.44&	33.86&	42.01&	31.94&	45.42&	47.6&	34.15\\
&33.86&	39.24&	35.99&	36.55&	34.92&	48.01&	49.63&	37.79\\
&30.08&	40.24&	34.66&	36.38&	35.52&	47.81&	42.42&	38.74\\
&32.47&	38.84&	36.65&	36.38&	36.58&	46.02&	47.6&	39.5\\
&33.47&	38.25&	33.79&	35.27&	34.00&	45.42&	45.66&	39.12\\
\midrule
AVG&32.23&	39.40&	34.99&	37.32&	34.59&	46.54&	46.58&	37.86\\
STD&1.565&	0.929&	1.283&	2.672&	1.754&	1.280&	2.717&	2.169\\
\midrule
\multirow{5}{*}{k=20}&33.47&	38.05&	33.93&	-&	-&	44.82&	51.15&	35.67\\
&29.48&	39.44&	35.32&	-&	-&	44.62&	47.99&	34.76\\
&33.47&	36.25&	31.94&	-&	-&	45.22&	47.13&	36.39\\
&32.47&	37.65&	34.00&	-&	-&	47.21&	48.57&	33.47\\
&33.07&	37.25&	35.19&	-&	-&	46.22&	53.01&	36.85\\
\midrule
AVG&32.39&	37.73&	34.08&	-&	-&	45.62&	49.57&	35.43\\
STD&1.679&	1.168&	1.358&	-&	-&	1.083&	2.438&	1.350\\
\bottomrule
    \end{tabular}}
    \caption{1-in-K detailed results}
    \label{tab:1ink_detail}
\end{table*}


\begin{table*}[t]
    \centering
    \resizebox{\textwidth}{!}{
    \begin{tabular}{l|ccccc|ccc}
    \toprule
         Setting&Phi3.5-V&	Qwen2VL&	InternVL2.5&	Pixtral&	Llama3.2-V&	GPT-4o&	Gemini&	Claude  \\
         \midrule
        Zeroshot&14.66&	14.87&	17.03&	20.44&	14.05&	11.96&	12.58&	9.65\\
         
         \midrule
\multirow{5}{*}{Oracle}&
15.26&	26.75&	17.16&	25.83&	6.72&	16.48&	16.35&	24.59\\
&15.85&	26.63&	19.01&	25.96&	8.1&	15.9&	14.1&	23.26\\
&15.22&	25.86&	18.72&	29.25&	7.19&	18.44&	14.77&	23.63\\
&15.79&	25.08&	19.46&	28.42&	10.39&	20.07&	14.58&	23.83\\
&15.27&	26.44&	17.06&	25.86&	7.88&	20.36&	11.47&	26.57\vspace{0.3em}\\
\midrule
AVG&15.48&	26.15&	18.28&	27.06&	8.06&	18.25&	14.25&	24.38\\
STD&0.313&	0.690&	1.102&	1.644&	1.416&	2.028&	1.771&	1.319\\

\midrule
\multirow{5}{*}{Non-clue}&
9.49&	21.59&	8.07&	23.25&	3.95&	9.23&	9.04&	17.11\\
&9.27&	21.13&	9.24&	23.94&	5.4&	7.86&	10.03&	17.18\\
&8.89&	20.96&	10.78&	24.4&	2.87&	10.06&	9.21&	17.42\\
&9.85&	21.23&	8.69&	24.63&	5&	6.62&	11.4&	15.26\\
&9.27&	20.86&	9.05&	24.18&	4.05&	9.9&	10.14&	16.09\vspace{0.3em}\\
\midrule
AVG&9.35&	21.15&	9.17&	24.08&	4.25&	8.73&	9.96&	16.61\\
STD&0.351&	0.283&	1.006&	0.530&	0.990&	1.466&	0.938&	0.912\\
\bottomrule
    \end{tabular}}
    \caption{Baselines rouge scores for reference. Please note that rouge score fluctuates largely across models, as the generated text style largely affect the matching with ground truth.}
    \label{tab:baseline_rouge}
\end{table*}

\begin{table*}[t]
    \centering
    \resizebox{\textwidth}{!}{
    \begin{tabular}{l|ccccc|ccc}
\toprule
1 in K&	Phi3.5-V&	Qwen2VL&	InternVL2.5&	Pixtral&	Llama3.2-V&	GPT-4o&	Gemini&	Claude\\
\midrule
\multirow{5}{*}{k=3}&10.49&	23.80&	8.79&	24.31&	6.92&	26.23&	16.14&	23.27\\
&10.30&	25.47&	9.65&	24.12&	6.64&	20.81&	18.71&	22.65\\
&10.57&	23.89&	8.22&	25.83&	5.55&	24.87&	19.38&	22.69\\
&9.89&	24.72&	8.78&	25.27&	7.97&	23.78&	17.91&	22.74\\
&10.29&	25.80&	9.34&	25.79&	6.76&	24.18&	19.19&	21.31\\
\midrule
AVG&10.31&	24.74&	8.96&	25.06&	6.77&	23.97&	18.27&	22.53\\
STD&0.263&	0.903&	0.554&	0.809&	0.861&	1.999&	1.317&	0.728\\
\midrule
\multirow{5}{*}{k=5}&12.18&	25.82&	7.77&	25.82&	8.71&	22.44&	16.94&	22.37\\
&9.87&	23.39&	8.22&	25.45&	4.79&	21.04&	16.36&	23.2\\
&10.50&	23.54&	8.65&	26.76&	5.55&	25.28&	16.95&	23.62\\
&11.76&	22.36&	7.72&	22.11&	7.65&	17.79&	16.17&	23.17\\
&10.30&	24.67&	8.62&	24.56&	5.12&	28.04&	17.84&	24.35\\
\midrule
AVG&10.92&	23.72&	8.20&	24.94&	6.36&	22.92&	16.85&	23.34\\
STD&0.995&	0.967&	0.446&	1.768&	1.721&	3.933&	0.652&	0.722\\
\midrule
\multirow{5}{*}{k=7}&9.28& 23.26&	8.63&	21.65&	7.57&	22.62&	15.67&	24.38\\
&10.72&	24.28&	8.17&	24.61&	5.66&	25.55&	20.08&	21.73\\
&11.45&	24.74&	8.38&	21.51&	7.14&	20.67&	16.35&	24\\
&12.10&	24.81&	8.22&	24.40&	7.63&	25.97&	17.62&	22.79\\
&12.48&	24.32&	7.67&	23.57&	7.01&	24.4&	18.82&	23.07\\
\midrule
AVG&11.21&	24.28&	8.21&	23.15&	7.00&	23.84&	17.71&	23.19\\
STD&1.268&	0.619&	0.353&	1.484&	0.796&	2.197&	1.794&	1.046\\
\midrule
\multirow{5}{*}{k=10}&11.85&	23.90&	7.65&	25.71&	5.81&	23.86&	18.92&	23.03\\
&12.15&	24.00&	8.81&	25.50&	6.93&	24.44&	17.5&	21.05\\
&11.83&	23.69&	8.19&	24.96&	6.15&	23.38&	18.64&	21.52\\
&11.79&	24.45&	8.77&	25.28&	6.62&	24.34&	17.57&	23.31\\
&11.06&	24.63&	8.06&	26.36&	6.78&	26.39&	19.17&	21.71\\
\midrule
AVG&11.74&	24.13&	8.30&	25.56&	6.46&	24.48&	18.36&	22.12\\
STD&0.404&	0.392&	0.493&	0.525&	0.466&	1.147&	0.776&	0.990\\
\midrule
\multirow{5}{*}{k=15}&11.42&	23.21&	8.31&	24.68&	7.44&	21.5&	19.12&	20.55\\
&11.52&	24.76&	8.93&	25.77&	8.41&	25.05&	19.79&	21.93\\
&10.14&	24.49&	7.84&	26.46&	7.07&	23&	18.47&	24.1\\
&11.77&	23.49&	8.11&	23.52&	7.89&	28.41&	20.25&	24.11\\
&11.77&	23.34&	8.52&	26.43&	6.96&	21.25&	20.24&	23\\
\midrule
AVG&11.32&	23.86&	8.34&	25.37&	7.55&	23.84&	19.57&	22.74\\
STD&0.680&	0.714&	0.414&	1.262&	0.601&	2.968&	0.770&	1.520\\
\midrule
\multirow{5}{*}{k=20}&13.09&	22.95&	8.02&	-&	-&	24.21&	18.57&	20.79\\
&11.51&	23.58&	9.40&	-&	-&	21.31&	16.66&	20.56\\
&11.76&	23.24&	8.30&	-&	-&	22.27&	17.95&	23.95\\
&11.85&	23.94&	8.20&	-&	-&	21.99&	19.25&	20.61\\
&11.95&	22.80&	9.00&	-&	-&	25.61&	20.19&	23.31\\
\midrule
AVG&12.03&	23.30&	8.58&	-&	-&	23.08&	18.52&	21.84\\
STD&0.614&	0.465&	0.588&	-&	-&	1.779&	1.333&	1.648\\
\bottomrule
    \end{tabular}}
    \caption{1-in-K rouge scores for reference. Please note that rouge score fluctuates largely across models, as the generated text style largely affect the matching with ground truth.}
    \label{tab:1ink_rouge}
\end{table*}

\end{document}
