\section{Construction of Minimum Textual Knowledge Base}
\label{sec:text_kb}

Among the 10000 taxonomy species in iNat21 dataset, we matched 9614 Wikipedia pages. iNat21 provides the scientific name as well as one common name of the species. We conduct multiple rounds of matching: initially, Wikipedia articles are matched based on species scientific names with Wikipedia titles; for unmatched species, match their common names with article titles; further unmatched species will go through matching within article content. All articles containing the scientific name are collected, sorting according to the position of the scientific name's first occurrence within the article. The article with earliest occurrence is selected as matched page. Due to the fact that taxonomy study frequently change the attribution of a species, the scientific name in iNat21 dataset can be deprecated and no matched page can be found for them.

We then applied basic text chunking to these articles, splitting them into chunks of approximately 150 tokens. Each chunk is bounded within a single section of the original article to preserve contextual integrity. Sections such as ``Reference'', ``See also'', ``External links'', etc., which lack informative content, were excluded. The chunks are cut upon period mark, so that they won't end or start with incomplete sentence. Two adjacent sections, having total length less than 150 tokens will be merged, with the section titles kept.
