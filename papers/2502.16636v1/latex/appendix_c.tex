\section{Annotation details}
\label{sec:app-anno}

For the query generation, we mentioned that we have gone through a human annotator originated refined filtering and rewriting process. The prompt template is shown in Table \ref{tab:prompt_qgen}, and following shows the example prototype queries generated with the sample prompt:

\begin{enumerate}
    \item Does rescuegrass (scientific name: Bromus catharticus) have a visible ligule when the leaf sheath is gently parted, and if so, what is its shape? (Explanation: The ligule is a thin membrane at the junction of leaf blade and sheath, typically hidden in standard photos) (Feature: shape of ligule) \textcolor{red}{Dropped. Too detailed and hard to distinguish}

    \item During the flowering stage of rescuegrass (scientific name: Bromus catharticus), what color do the exposed anthers display? (Explanation: The anthers are only briefly visible and require a closer or angled view) (Feature: color of anthers) \textcolor{teal}{Passed. Annotator found a image of flower anther (Figure \ref{fig:sample_qgen})  and confirm it is rare image.}

    \item Is there any distinct pattern or texture on the backside of the spikelets of rescuegrass (scientific name: Bromus catharticus), observable only when the spikelets are lifted or spread apart? (Feature: pattern on backside of spikelets) \textcolor{red}{Dropped. Viewing the spikelets image, there is no ``backside'' on spikelets as they are symmetric.}

    \item When the plant is fully mature, how does the collar region (the area where the leaf blade meets the sheath) of rescuegrass (scientific name: Bromus catharticus) appear from the underside view? (Explanation: This area may be covered by overlapping leaves in typical photos) (Feature: shape/appearance of collar region) \textcolor{red}{Dropped. Query too obscure and likely no clue image can be found.}
\end{enumerate}

\begin{figure}[h]
    \centering
    \includegraphics[width=\linewidth]{image/sample_qgen.jpg}
    \caption{Image of Bromus catharticus flower anther. Owned by original uploader at www.inaturalist.org, shared under CC-BY-NC license.}
    \label{fig:sample_qgen}
\end{figure}

Human annotator carefully check the candidate prototype queries by browsing all available material online with the scientific name as search keyword, as well as briefly scan through the image corpus by viewing the image thumbnails.

After query generation and rewriting, and coarse-level filtering described in section \ref{sec:const}, we dispatch the validated queries to the next round of human annotation of image labels.
Figure \ref{fig:anno_interface} shows the annotation interface for image labelling.\footnote{Similar interface is used for query filtering and rewriting annotator to browse, except the image matrix is more dense and displaying only thumbnail sized images.} Annotators are required to only label ``Y'', to indicate that this image contain the queried feature and is a clue to the query. Leaving the checkbox blank will be considered as negative label. This design aims to save annotators' effort, as clue images are assumed to be the minority. The rough clue image rate predicted by MLLM in the previous step is also provided (``Y\_Rate: 0.003'' after the question text) as reference for annotators to estimate the query difficulty. In the meanwhile, annotators provide query answers upon finding a clue image, as shown in Figure \ref{fig:anno_interface_ans}. Annotators are allowed to skip for a query when finding more than 9 clues after the first 27 images, indicating this query likly fail the clue image threshold. The order of images are randomly shuffled.


\begin{figure*}
    \centering
    \includegraphics[width=\linewidth]{image/anno_interface.png}
    \caption{Annotation interface for image labelling}
    \label{fig:anno_interface}
\end{figure*}

\begin{figure*}
    \centering
    \includegraphics[width=\linewidth]{image/anno_interface_answer.png}
    \caption{Annotation interface for image label and query answer}
    \label{fig:anno_interface_ans}
\end{figure*}


\section{Prompt Template}
\label{sec:app_prompt}

This section list the prompt templates used through out our experiments.

\onecolumn
\begin{longtable}{p{0.95\linewidth}}
\label{tab:prompt_qgen}\\

\toprule[2pt]
\textbf{Prompt Template for Query Generation}\\ \midrule
\endfirsthead

\midrule
\endhead

\hline \multicolumn{1}{c}{{Continued on next page}} \\ \hline
\endfoot

\bottomrule\\[-0.8em]
\caption{Prompt Template for Query Generation. Texts in \textit{\textcolor{violet}{italic and purple}} are not part of prompt.}
\endlastfoot


Given an organism's scientific taxonomy name and common name (if available), please generate some questions regarding its visual feature, like color, shape, pattern, texture, etc. Please limit the question asked to be regarding features that may not be usually found in photos. This can be regarding the following but not limited to: uncommon-to-see parts of animals or plants / parts that can only be viewed at specific posture or require photo taken at specific angle / less commonly seen stage of life cycle of incect, etc. (e.g. ``What is the color of paw pad of giant panda'', photos showing the paw pad of giant panda would be only a small portion; ``What pattern is on the back of brown peacock's larva?'', asking for the larva form of this butterfly.) Questions regarding non-visual feature should not be asked. (e.g., ``how long can giant panda live'', ``what does puma eat'' are not suitable questions, as such information cannot be extracted from a photo of that animal). Please avoid questions regarding exact length/area (of tails, ears, leaf, etc.), as such attribute is hard to be estimated from photo. Please avoid questions regarding eggs, nests, nets, etc., and focus on the organism's body. Please avoid generating multiple paraphrased questions regarding a same feature or body part. Please avoid questions comparing another species / form, but you can compare within the same individual (e.g. Does the upper beak or lower beak of XXX bird have darker color?).\\Except these constraints, try to generate ``interesting'' questions: for example, photos showing underside of leaf may be rare, but asking what color is underside of leaf is not interesting -- it's green anyway. Hence after generating each question, try to check if the questions is ``interesting'' or not. Please avoid asking for features that ``technically'' appear on a photo but hard to see (e.g., ``What color is the tiny hair on the surface of XYZ plant's stem?'' Any photo displays the stem will show the tiny hair, but it needs to be zoomed in / photo taken at very close, to see. The desired questions is regarding features that is hidden in some photos, not features that appear in every photo but simply too small to see).\\A likely related paragraph from Wikipedia is provided for you to better understand the organism. Please avoid question regarding visual features explicitly mentioned in the Wikipedia paragraphs. (e.g., paragraph: ``The seeds are colored in red'' -> Don't ask ``What color is the seed''; paragraph: ``The plant have symbolic seed pods'' (but doesn't mention its color) -> You may ask ``What shape/color is the seed pod'')\\Following are format instructions. You don't need to give answer to the questions.Please generate questions by asking the common name and add scientific name in brackets, and indicate the questioned visual feature after the question sentence, e.g.: ``What color is *** of \{common\_name\} (scientific name: \{scientific\_name\})? (Feature: color of ***)''; if there is no common name, simply ask with the scientific name. If explanation to the question is needed (such as explain a technical term), do not start a new line, append the explanation after the question enclosed by small brackets ().\\\\
Following are a few examples:
\\\\Scientific name: Diastictis fracturalis\\Common name: fractured western snout moth\\Question:\\1. What pattern, if any, can be observed on fractured western snout moth's (scientific name: Diastictis fracturalis) underside of wing during flying? (Explanation: the moth's wing usually cover its whole abdomen when resting, and underside will be less commonly appear in photos) (Feature: pattern on underside of wing during flying)\\2. What is the common color of larva fractured western snout moth (scientific name: Diastictis fracturalis)? (Feature: color of larva)?
\\\\Scientific name: Tortula muralis\\Common name: wall-screw moss\\Question:\\1. What is the common color of tip of leaf of wall-screw moss (scientific name: Tortula muralis)? (Explanation: The moss can be distinguished from similar mosses by its \"hair-pointed\" leaves.) (Feature: color of leaf tip)
\\\\Scientific name: Ptiliogonys cinereus\\Common name: Gray Silky-flycatcher\\Question:\\1. What color are the undertail covert (short plumages that cover the long tail feather) of Gray Silky-flycatcher (scientific name: Ptiliogonys cinereus) when viewing from front? (Feature: color of undertail covert viewing from front)\\2. What pattern can be observed on the tail feather of Gray Silky-flycatcher (scientific name: Ptiliogonys cinereus) from dorsal (back) view? (Feature: pattern on tail feather from dorsal view)

\\
\textcolor{violet}{... \textit{More ICL Examples omitted} ...}
\\\\

%\\\\Scientific name: Puma concolor\\Common name: Cougar\\Question:\\1. Are cougars (scientific name: Puma concolor) fangs in upper and lower jaw having roughly similar length, or one side's obviously longer than another? (Explanation: only photos showing fully opened mouth, displaying both upper and lower fang can answer this question)\\\\Scientific name: Adiantum aleuticum\\Common name: western maidenhair fern\\Question:\\1. Are the sori (spore-producing structures) of the western maidenhair fern (scientific name: Adiantum aleuticum) round or elongated?\\\\Scientific name: Oxalis pes-caprae\\Common name: Bermuda-buttercup\\Question:\\1. Does Bermuda-buttercup (scientific name: *Oxalis pes-caprae*) have distinct nectar guides in its flower? (Explanation: the nectar guides are patterns usually near the center of petal to guide pollinators like bees to the pollen.)\\\\Scientific name: Arothron hispidus\\Common name: Stripebelly Puffer\\Question: 1. Are the dot pattern appearing on the back also present on the ventral side of Stripebelly Puffer (scientific name: Arothron hispidus)?\\\\
End of examples.
\\\\\\Now, please generate questions for the given scientific name.\\
\textcolor{violet}{\textit{Example new query}}\\
Scientific name: Bromus catharticus\\Common Name: Rescuegrass\\Wiki Paragraph:\\Bromus catharticus is a species of brome grass known by the common names rescuegrass, grazing brome, prairie grass, and Schrader's bromegrass. The specific epithet catharticus is Latin, meaning cathartic. The common name rescuegrass refers to the ability of the grass to provide forage after harsh droughts or severe winters. The grass has a diploid number of 42 ...... \textcolor{violet}{\textit{Omitted due to space. Actual prompt includes whole description passage.}}\\ \\Question:\\
\end{longtable}

%%%%%%%%%%%%%%%%%%%%%%%%%%%%%%%%%%%%%%%%%%%%%%%%%%%%

\begin{longtable}{p{0.95\linewidth}}
\label{tab:prompt_qa}\\

\toprule[2pt]
\textbf{Prompt Template for Main Question Answering Experiments of Different Settings}\\ \midrule[1.2pt]
\endfirsthead

\midrule
\endhead

\hline \multicolumn{1}{c}{{Continued on next page}} \\ \hline
\endfoot

\bottomrule[1.2pt]\\[-0.8em]
\caption{Prompt Template for Main Question Answering Experiments of Different Settings.}
\endlastfoot


\textbf{Zero-shot}\\
\midrule
Please answer the question regarding a visual feature of an organism (animal, plant, etc.). Please follow the answer format: ``Answer: \{answer\_text\}''\\\\Question: \\
\midrule[1.2pt]

\textbf{Oracle (one clue)}\\
Please answer the question regarding a visual feature of an organism (animal, plant, etc.). You will be provided with a image regarding that organism, it is likely to contain the key information for answering the question. Please follow the answer format: ``Answer: \{answer\_text\}''\\\\Question:\\<image>\\
\midrule[1.2pt]

\textbf{Non-Clue}\\
Please answer the question regarding a visual feature of an organism (animal, plant, etc.). You will be provided with a image regarding that organism. Please follow the answer format: ``Answer: \{answer\_text\}''\\\\Question:\\<image>\\
\midrule[1.2pt]

\textbf{Multiple Images}\\
Please answer the question regarding a visual feature of an organism (animal, plant, etc.). You will be provided with several images, all of them are regarding that organism, but not all images contain the key information for answering the question. Only one to few images, or even none of them can be used to answer. Please follow the answer format: ``Answer: \{answer\_text\}''\\\\Question:\\<image> <image> <image>...\\


\end{longtable}


%%%%%%%%%%%%%%%%%%%%%%%%%%%%%%%%%%%%%%%%%%%%%%%%%%%%

\begin{longtable}{p{0.95\linewidth}}
\label{tab:prompt_eval}\\

\toprule[2pt]
\textbf{Prompt Template For Automatic Evaluation}\\ \midrule
\endfirsthead

\midrule
\endhead

\hline \multicolumn{1}{c}{{Continued on next page}} \\ \hline
\endfoot

\bottomrule\\[-0.8em]
\caption{Prompt Template For Automatic Evaluation. Note that for evaluation, no images is provided, and it becomes pure-text open-ended QA evaluation. Texts in \textit{\textcolor{violet}{italic and purple}} are not part of prompt.}
\endlastfoot

Please evaluate the answer to a question, score from 0 to 1. The reference answer is provided, and the reference is usually short phrases or a single keyword. If the student answer is containing the keywords or similar expressions (including similar color), without any additional guessed information, it is full correct. If the student answer have missed some important part in the reference answer, please assign partial score. Usually, when there are 2 key features and only 1 is being answered, assign 0.5 score; if there are more than 2 key features, adjust partial score by ratio of correctly answered key feature. The reference answer can be in the form of a Python list, in this case, any one of the list item is correct. \\If student answer contain irrelevant information not related to question, mark it with ``Redundant'', but it does not affect score if related part are correct. (e.g. Question: what shape is leave of Sanguinaria canadensis, Student Answer: shape is xxx, color is yyy, this is Redundant answer)\\If student answer contain features not listed in reference answer, mark it with ``Likely Hallucination'' and deduct 0.5 score. (e.g., Reference Answer: black and white. Student Answer: black white, with yellow dots, ``yellow dots'' is not mentioned in reference)\\Separate the remarks with score using ``|'', that is, use the syntax of: ``Score: {score} | Likely Hallucination'', ``Score: \{score\}'', ``Score: \{score\} | Likely Hallucination | Redundant'', etc. If any explanation on why giving the score is needed, do not start a new line and append after remark with brackets, e.g. ``Score: \{score\} | Redundant | (Explanation: abc)''.\\\\Following are few examples:\\\\
Question: Is there any specific color marking around the eyes of a semipalmated plover (scientific name: Charadrius semipalmatus)?\\Reference Answer: black eye-round feather, white stripe above eyes, sometimes connected to the white forehead\\\\Student Answer: Yes, the bird has a distinctive black line that runs through the eye, which is a key identifying feature.\\Score: 0 | Likely Hallucination\\\\Student Answer: They have a black vertical band in front of the eye, a white band above the eye, and a single black band that wraps partially around the eye, creating a partial ``mask'' appearance.\\Score: 1\\\\Student Answer: Yes, the semipalmated plover has a distinctive black/dark ring around its eye, surrounded by a bright white ring or patch\\Score: 0.5 | Likely Hallucination (Explanation: not white ring, but only a line above the eye)\\\\
Question: ...\\
\textcolor{violet}{... \textit{More ICL Examples omitted} ...}\\
Now, Score the following question:\\\\

\end{longtable}


\twocolumn