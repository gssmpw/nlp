\section{Introduction}
\label{sec:introduction}
When a large language model (LLM) performs remarkably well on a benchmark, can you confidently attribute its success to true generalization---or is it simply a reflection of what the model has already seen during pre-training? The reality is, we often don’t know. Beneath the surface of those impressive performance scores lies a critical vulnerability: \emph{dataset contamination}, a phenomenon where evaluation datasets overlap with the pretraining data of the model~\cite{golchintime}. This overlap artificially inflates reported performance metrics, obscures true generalization capabilities, and raises critical concerns about the reliability of benchmark evaluations. This brings us to a pressing and underexplored question: \emph{\textbf{How can we quantify the degree of dataset contamination}?} 

Addressing this question is crucial to ensuring that performance evaluations genuinely reflect a model's ability to generalize to unseen data, rather than benefiting from overlap with pretraining data. To formalize the problem, we aim to develop a scoring function $S: (\mathcal{D}, \mathcal{M}) \rightarrow \mathbb{R},$ that takes a benchmark dataset $\mathcal{D}$ as input and produces a score indicative of its relative contamination level with respect to the given model $\mathcal{M}$. A higher score corresponds to a greater contamination level. Such a score is valuable because researchers can use it to rank multiple benchmarks and prioritize the less contaminated ones, enabling more informed comparisons and reliable evaluation. For the score to be reliable, the scoring function must satisfy two essential properties:  \emph{monotonicity}, which ensures that the score exhibits a positive correlation with the contamination level, and \emph{consistency}, which means that the score remains stable across independently sampled subsets with the same contamination rate.

\section{Introduction}


\begin{figure}[t]
\centering
\includegraphics[width=0.6\columnwidth]{figures/evaluation_desiderata_V5.pdf}
\vspace{-0.5cm}
\caption{\systemName is a platform for conducting realistic evaluations of code LLMs, collecting human preferences of coding models with real users, real tasks, and in realistic environments, aimed at addressing the limitations of existing evaluations.
}
\label{fig:motivation}
\end{figure}

\begin{figure*}[t]
\centering
\includegraphics[width=\textwidth]{figures/system_design_v2.png}
\caption{We introduce \systemName, a VSCode extension to collect human preferences of code directly in a developer's IDE. \systemName enables developers to use code completions from various models. The system comprises a) the interface in the user's IDE which presents paired completions to users (left), b) a sampling strategy that picks model pairs to reduce latency (right, top), and c) a prompting scheme that allows diverse LLMs to perform code completions with high fidelity.
Users can select between the top completion (green box) using \texttt{tab} or the bottom completion (blue box) using \texttt{shift+tab}.}
\label{fig:overview}
\end{figure*}

As model capabilities improve, large language models (LLMs) are increasingly integrated into user environments and workflows.
For example, software developers code with AI in integrated developer environments (IDEs)~\citep{peng2023impact}, doctors rely on notes generated through ambient listening~\citep{oberst2024science}, and lawyers consider case evidence identified by electronic discovery systems~\citep{yang2024beyond}.
Increasing deployment of models in productivity tools demands evaluation that more closely reflects real-world circumstances~\citep{hutchinson2022evaluation, saxon2024benchmarks, kapoor2024ai}.
While newer benchmarks and live platforms incorporate human feedback to capture real-world usage, they almost exclusively focus on evaluating LLMs in chat conversations~\citep{zheng2023judging,dubois2023alpacafarm,chiang2024chatbot, kirk2024the}.
Model evaluation must move beyond chat-based interactions and into specialized user environments.



 

In this work, we focus on evaluating LLM-based coding assistants. 
Despite the popularity of these tools---millions of developers use Github Copilot~\citep{Copilot}---existing
evaluations of the coding capabilities of new models exhibit multiple limitations (Figure~\ref{fig:motivation}, bottom).
Traditional ML benchmarks evaluate LLM capabilities by measuring how well a model can complete static, interview-style coding tasks~\citep{chen2021evaluating,austin2021program,jain2024livecodebench, white2024livebench} and lack \emph{real users}. 
User studies recruit real users to evaluate the effectiveness of LLMs as coding assistants, but are often limited to simple programming tasks as opposed to \emph{real tasks}~\citep{vaithilingam2022expectation,ross2023programmer, mozannar2024realhumaneval}.
Recent efforts to collect human feedback such as Chatbot Arena~\citep{chiang2024chatbot} are still removed from a \emph{realistic environment}, resulting in users and data that deviate from typical software development processes.
We introduce \systemName to address these limitations (Figure~\ref{fig:motivation}, top), and we describe our three main contributions below.


\textbf{We deploy \systemName in-the-wild to collect human preferences on code.} 
\systemName is a Visual Studio Code extension, collecting preferences directly in a developer's IDE within their actual workflow (Figure~\ref{fig:overview}).
\systemName provides developers with code completions, akin to the type of support provided by Github Copilot~\citep{Copilot}. 
Over the past 3 months, \systemName has served over~\completions suggestions from 10 state-of-the-art LLMs, 
gathering \sampleCount~votes from \userCount~users.
To collect user preferences,
\systemName presents a novel interface that shows users paired code completions from two different LLMs, which are determined based on a sampling strategy that aims to 
mitigate latency while preserving coverage across model comparisons.
Additionally, we devise a prompting scheme that allows a diverse set of models to perform code completions with high fidelity.
See Section~\ref{sec:system} and Section~\ref{sec:deployment} for details about system design and deployment respectively.



\textbf{We construct a leaderboard of user preferences and find notable differences from existing static benchmarks and human preference leaderboards.}
In general, we observe that smaller models seem to overperform in static benchmarks compared to our leaderboard, while performance among larger models is mixed (Section~\ref{sec:leaderboard_calculation}).
We attribute these differences to the fact that \systemName is exposed to users and tasks that differ drastically from code evaluations in the past. 
Our data spans 103 programming languages and 24 natural languages as well as a variety of real-world applications and code structures, while static benchmarks tend to focus on a specific programming and natural language and task (e.g. coding competition problems).
Additionally, while all of \systemName interactions contain code contexts and the majority involve infilling tasks, a much smaller fraction of Chatbot Arena's coding tasks contain code context, with infilling tasks appearing even more rarely. 
We analyze our data in depth in Section~\ref{subsec:comparison}.



\textbf{We derive new insights into user preferences of code by analyzing \systemName's diverse and distinct data distribution.}
We compare user preferences across different stratifications of input data (e.g., common versus rare languages) and observe which affect observed preferences most (Section~\ref{sec:analysis}).
For example, while user preferences stay relatively consistent across various programming languages, they differ drastically between different task categories (e.g. frontend/backend versus algorithm design).
We also observe variations in user preference due to different features related to code structure 
(e.g., context length and completion patterns).
We open-source \systemName and release a curated subset of code contexts.
Altogether, our results highlight the necessity of model evaluation in realistic and domain-specific settings.






To quantify dataset contamination, we introduce the \textbf{Kernel Divergence Score} (KDS), which computes the divergence of the kernel similarity matrix of sample embeddings before and after fine-tuning on the benchmark dataset. By analyzing changes in the kernel similarity matrix, KDS captures how fine-tuning reshapes the embeddings for seen and unseen data, providing a more holistic and nuanced perspective on dataset contamination.
%Unlike traditional methods that rely on instance-level properties, KDS leverages the relational information embedded within the dataset, capturing the impact of fine-tuning on kernel entries that encode inter-sample relationships. 
This approach is motivated by the fact that fine-tuning has a more significant effect on the embeddings of unseen samples, which the model must adapt to, while seen samples exhibit minimal changes due to prior exposure during pre-training. Furthermore, as the proportion of unseen samples increases, their cumulative effect on the kernel divergence score becomes more pronounced. By quantifying these changes, KDS can provide a reliable and interpretable measure of dataset contamination, with scores that proportionally reflect the level of contamination. 


To evaluate KDS, we perform extensive experiments, systematically controlling contamination ratios across multiple datasets.  
Our results demonstrate that KDS achieves near-perfect correlation with contamination levels, generally outperforming existing baselines across multiple datasets.
Additionally, we show that KDS is robust to design choices, including kernel functions, kernel bandwidth, and the extraction location of embeddings.
Overall, KDS provides stable scores across diverse scenarios, enabling researchers to reliably identify benchmarks based on contamination levels. We summarize our {contributions} as follows: 


\vspace{-2mm}
\begin{itemize}[leftmargin=*]
    \item We propose \textit{Kernel Divergence Score}, a reliable dataset-level scoring function for quantifying benchmark contamination. To the best of our knowledge, we are the first to leverage the fine-grained information of kernels for scoring contamination levels.
    \item We validate Kernel Divergence Score through extensive experiments on controlled contamination scenarios, showing strong performance over existing baselines. 
    \item We perform comprehensive ablations to analyze the impact of various design choices. Further practical discussions are presented, providing deeper insights into our kernel-based approach.
\end{itemize}
