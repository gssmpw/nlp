\section{Introduction}
\label{sec:introduction}
When a large language model (LLM) performs remarkably well on a benchmark, can you confidently attribute its success to true generalization---or is it simply a reflection of what the model has already seen during pre-training? The reality is, we often don’t know. Beneath the surface of those impressive performance scores lies a critical vulnerability: \emph{dataset contamination}, a phenomenon where evaluation datasets overlap with the pretraining data of the model~\cite{golchintime}. This overlap artificially inflates reported performance metrics, obscures true generalization capabilities, and raises critical concerns about the reliability of benchmark evaluations. This brings us to a pressing and underexplored question: \emph{\textbf{How can we quantify the degree of dataset contamination}?} 

Addressing this question is crucial to ensuring that performance evaluations genuinely reflect a model's ability to generalize to unseen data, rather than benefiting from overlap with pretraining data. To formalize the problem, we aim to develop a scoring function $S: (\mathcal{D}, \mathcal{M}) \rightarrow \mathbb{R},$ that takes a benchmark dataset $\mathcal{D}$ as input and produces a score indicative of its relative contamination level with respect to the given model $\mathcal{M}$. A higher score corresponds to a greater contamination level. Such a score is valuable because researchers can use it to rank multiple benchmarks and prioritize the less contaminated ones, enabling more informed comparisons and reliable evaluation. For the score to be reliable, the scoring function must satisfy two essential properties:  \emph{monotonicity}, which ensures that the score exhibits a positive correlation with the contamination level, and \emph{consistency}, which means that the score remains stable across independently sampled subsets with the same contamination rate.

\section{Introduction}
\label{sec:intro}

\begin{figure*}[tb]
    \centering
    \includegraphics[width=0.848\linewidth]{figs/circuitnn.pdf} 
    \caption{Illustration of differentiable CircuitNN. CircuitNN is designed based on differentiable NAND gates. After DAS is guided by PI and PO pairs of the truth table, CircuitNN can get the precise circuit architecture logic equivalent to the truth table.}
    \label{fig:circuitnn}
\end{figure*}

% 1. Describe the importance of logic synthesis
% 2. Existing Problems
% (a) Neural Architecture Search: Unstable, Predefined Setting, etc.
% (b) Circuit Generation: Probabilistic Model, Logic Equivalence

With the rapid advancement of technology, the scale of integrated circuits (ICs) has expanded exponentially. 
This expansion has introduced significant challenges in chip manufacturing, particularly concerning power and area metrics.
A primary objective in IC design is achieving the same circuit function with fewer transistors, thereby reducing power usage and area occupancy.

Logic synthesis~\cite{hachtel2005logicsynth}, a critical step in electronic design automation (EDA), transforms behavioral-level circuit designs into optimized gate-level circuits, ultimately yielding the final IC layout. 
The primary goal of logic synthesis is to identify the physical implementation with the fewest gates for a given circuit function. 
This task constitutes a challenging NP-hard combinatorial optimization problem. 
Current logic synthesis tools~\cite{brayton2010abc, wolf2013yosys} rely on human-designed heuristics, often leading to sub-optimal outcomes.

Differentiable architecture search (DAS) techniques~\cite{liu2018darts, chu2020darts} offer novel perspectives on addressing challenges in this problem.
Circuit functions can be represented through truth tables, which map binary inputs to their corresponding outputs. 
Truth tables provide a precise representation of input-output relationships, ensuring the design of functionally equivalent circuits.
Inspired by this, researchers~\cite{deepmind2024ai4sys, wang2024tnet} have begun exploring the application of DAS to synthesize circuits directly from truth tables.
Specifically, \citet{deepmind2024ai4sys} proposed CircuitNN, a framework that learns differentiable connection structures with logic gates, enabling the automatic generation of logic circuits from truth tables.
This approach significantly reduces the complexity of traditional circuit generation. 
Building on this, \citet{wang2024tnet} introduced T-Net, a triangle-shaped variant of CircuitNN, incorporating regularization techniques to enhance the efficiency of DAS.

Despite these advancements, several challenges remain. 
The computational complexity of DAS grows quadratically with the number of gates, posing scalability issues.
Although triangle-shaped architecture~\cite{wang2024tnet} partially mitigates this problem, redundancy persists. 
%Additionally, DAS is susceptible to converging to local optima, limiting the ability to search architectures that satisfy the given truth tables~\cite{liu2018darts}. 
%Furthermore, hyperparameters (network depth and layer width) require extensive searches, introducing complexity and prolonging the synthesis process. 
Additionally, DAS is susceptible to converging to local optima~\cite{liu2018darts} and hyperparameters (network depth and layer width) require extensive searches. 
The challenges arise from the vast search space in DAS. 
% Even with predefined settings for CircuitNN, finding a configuration that meets the truth table requires extensive trial and error during the DAS process. 
Intuitively, limiting the search space through predefined parameters (network depth, gates per layer, and connection probabilities) can significantly reduce the complexity.

Recent advances~\cite{openai2023gpt4, abramson2024alphafold3, esser2024sd3, li2024mar} in conditional generative models have demonstrated remarkable performance across language, vision, and graph generation tasks. 
Motivated by these developments, we propose a novel approach to circuit generation that generates preliminary circuit structures to guide DAS in generating refined circuits matching specified truth tables. 
Firstly, we introduce CircuitVQ, a tokenizer with a discrete codebook for circuit tokenization. 
Built upon our Circuit AutoEncoder framework~\cite{hou2022graphmae,li2023maskgae,wu2025mgvga}, CircuitVQ is trained through a circuit reconstruction task. 
Specifically, the CircuitVQ encoder encodes input circuits into discrete tokens using a learnable codebook, while the decoder reconstructs the circuit adjacency matrix based on these tokens.
Subsequently, the CircuitVQ encoder serves as a circuit tokenizer for CircuitAR pretraining, which employs a masked autoregressive modeling paradigm~\cite{chang2022maskgit, li2023mage}. 
In this process, the discrete codes function as supervision signals. 
After training, CircuitAR can generate discrete tokens progressively, which can be decoded into initial circuit structures by the decoder of the CircuitVQ. 
These prior insights can guide DAS in producing refined circuits that match the target truth tables precisely.

Our key contributions can be summarized as follows:
\begin{itemize}
\item We introduce CircuitVQ, a circuit tokenizer that facilitates graph autoregressive modeling for circuit generation, based on our Circuit AutoEncoder framework;
\item Develop CircuitAR, a model trained using masked autoregressive modeling, which generates initial circuit structures conditioned on given truth tables;
\item Propose a refinement framework that integrates differentiable architecture search to produce functionally equivalent circuits guided by target truth tables;
\item Comprehensive experiments demonstrating the scalability and capability emergence of our CircuitAR and the superior performance of the proposed circuit generation approach.
\end{itemize}

% Motivation
% (a) Diffusion (Vision, Graph), Autoregressive (Language, Vision)
% (b) Circuit Generation for Predefined Setting
% (c) Neural Architecture Search for Strict Logic Equivalence

% Contribution
% (a) Circuit Tokenizer (new transformer arch, training strategy)
% (b) CircuitAR (train and gen strategies, post-ar strategy)
% (c) Extensive Evaluation including BitD (Bit Distance) for Scalability


To quantify dataset contamination, we introduce the \textbf{Kernel Divergence Score} (KDS), which computes the divergence of the kernel similarity matrix of sample embeddings before and after fine-tuning on the benchmark dataset. By analyzing changes in the kernel similarity matrix, KDS captures how fine-tuning reshapes the embeddings for seen and unseen data, providing a more holistic and nuanced perspective on dataset contamination.
%Unlike traditional methods that rely on instance-level properties, KDS leverages the relational information embedded within the dataset, capturing the impact of fine-tuning on kernel entries that encode inter-sample relationships. 
This approach is motivated by the fact that fine-tuning has a more significant effect on the embeddings of unseen samples, which the model must adapt to, while seen samples exhibit minimal changes due to prior exposure during pre-training. Furthermore, as the proportion of unseen samples increases, their cumulative effect on the kernel divergence score becomes more pronounced. By quantifying these changes, KDS can provide a reliable and interpretable measure of dataset contamination, with scores that proportionally reflect the level of contamination. 


To evaluate KDS, we perform extensive experiments, systematically controlling contamination ratios across multiple datasets.  
Our results demonstrate that KDS achieves near-perfect correlation with contamination levels, generally outperforming existing baselines across multiple datasets.
Additionally, we show that KDS is robust to design choices, including kernel functions, kernel bandwidth, and the extraction location of embeddings.
Overall, KDS provides stable scores across diverse scenarios, enabling researchers to reliably identify benchmarks based on contamination levels. We summarize our {contributions} as follows: 


\vspace{-2mm}
\begin{itemize}[leftmargin=*]
    \item We propose \textit{Kernel Divergence Score}, a reliable dataset-level scoring function for quantifying benchmark contamination. To the best of our knowledge, we are the first to leverage the fine-grained information of kernels for scoring contamination levels.
    \item We validate Kernel Divergence Score through extensive experiments on controlled contamination scenarios, showing strong performance over existing baselines. 
    \item We perform comprehensive ablations to analyze the impact of various design choices. Further practical discussions are presented, providing deeper insights into our kernel-based approach.
\end{itemize}
