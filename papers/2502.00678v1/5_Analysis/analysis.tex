



\section{Ablation Studies}
\label{sec:analysis}



In this section, we conduct an in-depth ablation to understand various design choices of our kernel divergence score. 
% \begin{table}[!t]
% \centering
% \scalebox{0.68}{
%     \begin{tabular}{ll cccc}
%       \toprule
%       & \multicolumn{4}{c}{\textbf{Intellipro Dataset}}\\
%       & \multicolumn{2}{c}{Rank Resume} & \multicolumn{2}{c}{Rank Job} \\
%       \cmidrule(lr){2-3} \cmidrule(lr){4-5} 
%       \textbf{Method}
%       &  Recall@100 & nDCG@100 & Recall@10 & nDCG@10 \\
%       \midrule
%       \confitold{}
%       & 71.28 &34.79 &76.50 &52.57 
%       \\
%       \cmidrule{2-5}
%       \confitsimple{}
%     & 82.53 &48.17
%        & 85.58 &64.91
     
%        \\
%        +\RunnerUpMiningShort{}
%     &85.43 &50.99 &91.38 &71.34 
%       \\
%       +\HyReShort
%         &- & -
%        &-&-\\
       
%       \bottomrule

%     \end{tabular}
%   }
% \caption{Ablation studies using Jina-v2-base as the encoder. ``\confitsimple{}'' refers using a simplified encoder architecture. \framework{} trains \confitsimple{} with \RunnerUpMiningShort{} and \HyReShort{}.}
% \label{tbl:ablation}
% \end{table}
\begin{table*}[!t]
\centering
\scalebox{0.75}{
    \begin{tabular}{l cccc cccc}
      \toprule
      & \multicolumn{4}{c}{\textbf{Recruiting Dataset}}
      & \multicolumn{4}{c}{\textbf{AliYun Dataset}}\\
      & \multicolumn{2}{c}{Rank Resume} & \multicolumn{2}{c}{Rank Job} 
      & \multicolumn{2}{c}{Rank Resume} & \multicolumn{2}{c}{Rank Job}\\
      \cmidrule(lr){2-3} \cmidrule(lr){4-5} 
      \cmidrule(lr){6-7} \cmidrule(lr){8-9} 
      \textbf{Method}
      & Recall@100 & nDCG@100 & Recall@10 & nDCG@10
      & Recall@100 & nDCG@100 & Recall@10 & nDCG@10\\
      \midrule
      \confitold{}
      & 71.28 & 34.79 & 76.50 & 52.57 
      & 87.81 & 65.06 & 72.39 & 56.12
      \\
      \cmidrule{2-9}
      \confitsimple{}
      & 82.53 & 48.17 & 85.58 & 64.91
      & 94.90&78.40 & 78.70& 65.45
       \\
      +\HyReShort{}
       &85.28 & 49.50
       &90.25 & 70.22
       & 96.62&81.99 & \textbf{81.16}& 67.63
       \\
      +\RunnerUpMiningShort{}
       % & 85.14& 49.82
       % &90.75&72.51
       & \textbf{86.13}&\textbf{51.90} & \textbf{94.25}&\textbf{73.32}
       & \textbf{97.07}&\textbf{83.11} & 80.49& \textbf{68.02}
       \\
   %     +\RunnerUpMiningShort{}
   %    & 85.43 & 50.99 & 91.38 & 71.34 
   %    & 96.24 & 82.95 & 80.12 & 66.96
   %    \\
   %    +\HyReShort{} old
   %     &85.28 & 49.50
   %     &90.25 & 70.22
   %     & 96.62&81.99 & 81.16& 67.63
   %     \\
   % +\HyReShort{} 
   %     % & 85.14& 49.82
   %     % &90.75&72.51
   %     & 86.83&51.77 &92.00 &72.04
   %     & 97.07&83.11 & 80.49& 68.02
   %     \\
      \bottomrule

    \end{tabular}
  }
\caption{\framework{} ablation studies. ``\confitsimple{}'' refers using a simplified encoder architecture. \framework{} trains \confitsimple{} with \RunnerUpMiningShort{} and \HyReShort{}. We use Jina-v2-base as the encoder due to its better performance.
}
\label{tbl:ablation}
\end{table*}
% \subsection{Design Choices}
% \label{sec:design_choices}



% We analyze the impact of design choices: (1) embedding extraction location, and (2) SFT configurations.
% % In this section, we demonstrate what effects these design choices have.





% All configurations exhibit near-perfect Spearman and Pearson correlations, indicating a robust monotonic relationship between the contamination scores and the contamination ratios $\lambda$. 
% This highlights the reliability of the Kernel Divergence Score across varying training setups. The highest correlations are observed with stochastic gradient descent. 
% This suggests that SGD introduces finer-grained updates that enhance the sensitivity of the Kernel Divergence Score to dataset contamination. 
% Overall, these results underscore the effectiveness of our method across different training configurations for reliable quantification of contamination.

%Our Kernel Divergence Score utilizes the difference between kernels before and after supervised fine-tuning. 
%To eliminate the influence of the data ordering presented during a single epoch of stochastic gradient descent (SGD), we accumulated gradients for all samples and updated the model once per epoch. 
% This approach ensures fair comparisons across different baselines and multiple runs.
% In Table~\ref{tab:training_config}, we demonstrate the impact of various training configurations.
% Increasing the number of parameter updates via SGD may lead to improved compliance with the Coherence requirement.
% Interestingly, however, multiple epochs of batch gradient descent may degrade performance.
% One possible explanation is that multiple epochs make the boundary between seen and unseen samples less distinct.
% \textcolor{blue}{it's better to repeat the experiments multiple times and report the mean and std, for the results to be reliable. }


% \textcolor{blue}{can go into appendix if we run out of space}



% \paragraph{What's the role of the normalization factor?}
% Recall that in our Kernel Divergence Score, we define the normalizer $E$ as the square root of the sum of entries in the kernel matrix.
% We employ this square-root normalizer because, despite the second-order nature of Eq.~\eqref{eq:score_decomp}, the sum of kernel entries reveals a linear relationship with varying data subset sizes, as shown in Figure~\ref{fig:normalizer} (Appendix).
% A linear fit to the data yields an $R^2$ value of 0.9766, confirming the linearity of this relationship. 
% Therefore, to mitigate the influence of dataset size and prevent over-penalizing the score scale, we utilize the square-root normalizer.
% \textcolor{blue}{This doesn't seem to be as critical as other ablations. Consider moving to the appendix if we run out of space.}
% \FC{Will be important if we discuss dataset size, as the normalizer is the key to consistent score magnitude across datasets. But if we're not discussing dataset size, this is not very important.}




\section{Discussions}
\label{sec:discussion}
Our findings challenge the conjecture that code-comment coherence, as measured by SIDE \cite{mastropaolo2024evaluating}, is a critical quality attribute for filtering instances of code summarization datasets. By selecting $\langle code, summary \rangle$ pairs with high-coherence for training allow to achieve the same results that would be achieved by randomly selecting such a number of instances. At the same time, we observed that reducing the datasets size up to 50\% of the training instances does not significantly affect the effectiveness of the models, even when the instances are randomly selected. These results have several implications.

First, code-comment consistency might not be a problem in state-of-the-art datasets in the first place, as also suggested in the results of RQ$_0$. Also, the DL models we adopted (and, probably, bigger models as well) are not affected by inconsistent code-comment pairs, even when these inconsistencies are present in the training set.
Despite the theoretical benefits of filtering by SIDE \cite{mastropaolo2024evaluating}, that is the state-of-the-art metric for measuring code-comment alignment, our results indicate its limitations in improving the \textit{overall} quality of the training sets for code summarization task.
Nevertheless, other quality aspects of code and comments that have not been explored yet (such as readability) may be important for smartly selecting the training instances.
Future work should explore such quality aspects further.

Our results clearly indicate that state-of-the-art datasets contain instances that do not contribute to improving the models' effectiveness. This finding is related to a general phenomenon observed in Machine Learning and Deep Learning. Models reach convergence when they are trained for a certain amount of time (epochs). Additional training provides smaller improvements and increases the risk of overfitting. We show that the same is true for data. In terms of effectiveness, model convergence is achieved with fewer training instances than previously assumed. Limiting the number of epochs may make it possible to reach model convergence with a subset of training data, maintaining model effectiveness, reducing resource demands and minimizing the risk of overfitting.
Future work could explore different criteria for data selection that identify the most informative subsets for training.
Conversely, this insight suggests that currently available datasets suffer from poor diversity (thus causing the previously discussed phenomenon).
This latter insight constitutes a clear warning for researchers interested in building code summarization datasets, which should include instances that add relevant information instead of adding more data, which might turn out to be useless.

Finally, it is worth pointing out that another benefit of the reduction we performed is the environmental impact. Reducing the number of training instances implies a reduced training time, which, in turn, lowers the resources necessary to perform training and, thus, energy consumption and CO$_2$ emissions.
We performed a rough estimation of the training time across different selections of \textit{TL-CodeSum} and \textit{Funcom} datasets and estimated a proxy of the CO$_2$ emissions for each model training phase by relying on the ML CO$_2$ impact calculator\footnote{\url{https://mlco2.github.io/impact/\#compute}} \cite{lacoste2019quantifying}. Such a calculator considers factors such as the total training time, the infrastructure used, the carbon efficiency, and the amount of carbon offset purchased. The estimation of CO$_{2}$ emissions needed to train the model with the \textit{Full} selection of \textit{Funcom} ($\sim$ 200 hours) is equal to 26.05 Kg, while with the optimized training set, \ie $SIDE_{0.9}$ ($\sim$ 90 hours), the estimation is 11.69 Kg of $CO_2$ (-55\% emissions).
While we recognize that this method provides an estimation rather than a precise measurement, it offers a glimpse into the environmental impact of applying data reduction.


\paragraph{Temporal shift problems of MIA benchmarks.}
Recent studies have expressed concerns regarding the temporal shift issues in existing Membership Inference Attack~(MIA) benchmarks~\cite{duan2024membership,das2024blind,maini2024llm}.
Notably, datasets such as WikiMIA, BookMIA, and ArxivTection have been identified as susceptible to temporal cues, which can inadvertently simplify the membership inference task.
This simplification arises because models can exploit temporal information to distinguish between seen versus unseen data, leading to a potential overestimation of detection performance.

To ensure the robustness of our approach and mitigate potential biases introduced by temporal shifts, we conducted evaluations using 500 samples from six subsets of the Pile dataset~\cite{gao2020pile}. 
The subsets include text data from various sources, including expository proses~(Wikipedia), academic papers~(PhilPapers), emails~(Enron), news articles~(HackerNews),  web-scraped data~(Pile-CC), and user-contributed questions and answers~(StackExchange).
For each subset, the `train' set is regarded as seen data, while the `val' set serves as unseen data.
We mix these two sets according to varying contamination rates to assess our model's performance under different conditions. 
This methodology provides a rigorous assessment, ensuring that our model does not exploit temporal cues. 
As presented in Table~\ref{tab:pile}, the Spearman and Pearson correlation coefficients are both near 1.0, averaging at 0.944 and 0.948, respectively. 
These findings demonstrate that our method reliably scores contamination levels without relying on temporal shifts.



\begin{table}[t!]
    \centering
    \setlength{\tabcolsep}{8pt}
    \caption{\textbf{Evaluation on benchmarks with IID setup.} We evaluate the monotonicity of our kernel divergence score on six subsets from the Pile dataset~\cite{gao2020pile}.}
\vspace{3mm}
\resizebox{0.75\linewidth}{!}{
\begin{tabular}{l | cc }
\toprule
 \textbf{Data Subset} & $\textbf{Spearman} \uparrow$ & $\textbf{Pearson} \uparrow$ \\
 \midrule
{Wikipedia} & 0.891 & 0.922 \\
{PhilPapers} & 0.982 & 0.974 \\
{Enron} & 1.000 & 0.965 \\
{HackerNews} & 0.897 & 0.920 \\
{Pile-CC} & 0.895 & 0.908 \\
{StackExchange} & 1.000 & 0.998 \\
\midrule
\textbf{Average} & \textbf{0.944} & \textbf{0.948} \\
\bottomrule
\end{tabular}
}
\vspace{-2mm}
\label{tab:pile}
\end{table}


\begin{table}[t!]
    \centering
    \setlength{\tabcolsep}{3pt}
    \caption{\textbf{Evaluation using various models.} We evaluate the Monotonicity on the WikiMIA dataset.}
\vspace{3mm}
\resizebox{\linewidth}{!}{
\begin{tabular}{l | cc }
\toprule
 \textbf{Model} & $\textbf{Spearman} \uparrow$ & $\textbf{Pearson} \uparrow$ \\
 \midrule
Mistral-7B-Instruct-v0.2~\cite{jiang2023mistral} & 0.999 & 0.996 \\
Llama-3.1-8B-Instruct~\cite{dubey2024llama} & 0.982 & 0.952 \\
Phi-3-small-128k-instruct~\cite{abdin2024phi} & 0.892 & 0.890 \\
\bottomrule
\end{tabular}
}
\vspace{-4mm}
\label{tab:other_models}
\end{table}




\paragraph{Extension to various model families.}
We extend our evaluation to diverse models to demonstrate the versatility of our approach. 
As presented in Table~\ref{tab:other_models}, our approach consistently exhibits near-perfect correlation values across all models tested. 
These findings underscore the robustness and applicability of our method across diverse model families.


\paragraph{Computational cost.}

Our kernel divergence score  involves a fine-tuning step to obtain two kernel matrices, followed by the computation of our scoring function.
Given a dataset with $N$ samples, the fine-tuning step operates with a time complexity of $O(c_1\cdot N)$, where $c_1$ is a constant influenced by factors such as average sample length, batch size, and model dimension.
The computation of the KDS score has complexity $O(c_2\cdot N^2)$, due to the quadratic nature of kernel matrix operations.
In practice, the latency overhead caused by scoring is minimal, as these operations are highly optimized through vectorized computations.
As demonstrated in Table~\ref{tab:time}, the latency measured in seconds for each dataset confirms the efficiency and scalability of our approach. In Appendix~\ref{apdx:inthewild}, we confirm that our method can be applied to real-world benchmark datasets, where we employ our method across 11 diverse and widely used benchmarks, most of which have sizes around a few hundred to thousand samples. 

% \usepackage{booktabs}


\begin{table}
	\centering
	\caption{Analysis of models' inference speed.}
	\begin{tabular}{c|ccc} 
		\toprule
		& Parms (MB) & Speed & Moderate  \\ 
		\hline
		IA-SSD     & 2.7        & 84    & 79.12     \\
		PDM-SSD(A) & 3.3        & 84    & 79.37     \\
		PDM-SSD(J) & 3.3        & 68    & 79.75     \\
		\bottomrule
	\end{tabular}
\label{tabel8}
\end{table}