



\section{Ablation Studies}
\label{sec:analysis}



In this section, we conduct an in-depth ablation to understand various design choices of our kernel divergence score. 
\begin{table*}
  [t]
  \centering
  \resizebox{\textwidth}{!}{%
  \begin{tabular}{cccccccccccc}
    \toprule \multicolumn{2}{c}{Components}                                                             & \multicolumn{5}{c}{Re-executability Rate (\%)} & \multicolumn{5}{c}{Readability (\#)} \\
    \cmidrule(lr){1-2} \cmidrule(lr){3-7} \cmidrule(lr){8-12}        \hspace{8pt}\labelemoji\hspace{8pt}                                                                & \hspace{8pt}\toolemoji\hspace{8pt}                                      & O0                                 & O1             & O2             & O3             & AVG            & O0             & O1             & O2             & O3             & AVG            \\
    \hline
    \rowcolor[rgb]{0.93,0.93,0.93}\multicolumn{12}{c}{\textbf{Initialize with LLM4Decompile-End-6.7B~\citep{llm4decompile}}}   \\
    \xmark                                                                                              & \xmark                                    & 69.51                              & 46.95          & 50.61          & 46.34          & 53.35          & 3.98 & 3.41 & 3.44 & 3.38 & 3.55 \\
    \cmark                                                                                              & \xmark                                    & 75.61                              & 50.61          & 50.00          & 50.00          & 56.55          & 4.01 & 3.44 & 3.39 & \textbf{3.49} & 3.58 \\
    \xmark                                                                                              & \cmark                                    & 83.54                     & \textbf{56.10}          & 51.22          & 50.61 & 60.37 & 4.05 & 3.51 & 3.51 & 3.42 & 3.62 \\
    \cmark                                                                                              & \cmark                                    & \textbf{85.37}                            & \textbf{56.10}                     & \textbf{51.83} & \textbf{52.43}          & \textbf{61.43} & \textbf{4.13} & \textbf{3.60} & \textbf{3.54} & \textbf{3.49} & \textbf{3.69} \\

    \rowcolor[rgb]{0.93,0.93,0.93}\multicolumn{12}{c}{\textbf{Initialize with Deepseek-Coder-6.7B-base~\citep{deepseekcoder}}} \\
    \xmark                                                                                              & \xmark                                    & 59.15                              & 35.98          & 39.02          & 37.80          & 42.99          & 3.71 & 3.05 & 3.16 & 3.05 & 3.24 \\
    \cmark                                                                                              & \xmark                                    & 66.46                              & 41.46          & 38.41          & 36.59          & 45.73          & 3.76 & 3.17 & \textbf{3.21} & 3.08 & 3.31 \\
    \xmark                                                                                              & \cmark                                    & 70.73                              & 39.63          & 39.02          & 40.24          & 47.41          & 3.90 & 3.17 & 3.08 & 3.11 & 3.31 \\
    \cmark                                                                                              & \cmark                                    & \textbf{79.88}                     & \textbf{45.73} & \textbf{43.90} & \textbf{42.68} & \textbf{53.05} & \textbf{3.96} & \textbf{3.21} & 3.18 & \textbf{3.19} & \textbf{3.38} \\
    \bottomrule
  \end{tabular}%
  }
  \caption{The ablation study of different methods across four optimization levels
  (O0, O1, O2, O3), as well as their average scores (AVG). The results in bold represent the optimal performance. The ~\labelemoji~ and ~\toolemoji~ means Relabedling and Function Call. \textbf{Bold} denotes the best performance.}
  \label{tab:ablation}
\end{table*}
% \subsection{Design Choices}
% \label{sec:design_choices}



% We analyze the impact of design choices: (1) embedding extraction location, and (2) SFT configurations.
% % In this section, we demonstrate what effects these design choices have.





% All configurations exhibit near-perfect Spearman and Pearson correlations, indicating a robust monotonic relationship between the contamination scores and the contamination ratios $\lambda$. 
% This highlights the reliability of the Kernel Divergence Score across varying training setups. The highest correlations are observed with stochastic gradient descent. 
% This suggests that SGD introduces finer-grained updates that enhance the sensitivity of the Kernel Divergence Score to dataset contamination. 
% Overall, these results underscore the effectiveness of our method across different training configurations for reliable quantification of contamination.

%Our Kernel Divergence Score utilizes the difference between kernels before and after supervised fine-tuning. 
%To eliminate the influence of the data ordering presented during a single epoch of stochastic gradient descent (SGD), we accumulated gradients for all samples and updated the model once per epoch. 
% This approach ensures fair comparisons across different baselines and multiple runs.
% In Table~\ref{tab:training_config}, we demonstrate the impact of various training configurations.
% Increasing the number of parameter updates via SGD may lead to improved compliance with the Coherence requirement.
% Interestingly, however, multiple epochs of batch gradient descent may degrade performance.
% One possible explanation is that multiple epochs make the boundary between seen and unseen samples less distinct.
% \textcolor{blue}{it's better to repeat the experiments multiple times and report the mean and std, for the results to be reliable. }


% \textcolor{blue}{can go into appendix if we run out of space}



% \paragraph{What's the role of the normalization factor?}
% Recall that in our Kernel Divergence Score, we define the normalizer $E$ as the square root of the sum of entries in the kernel matrix.
% We employ this square-root normalizer because, despite the second-order nature of Eq.~\eqref{eq:score_decomp}, the sum of kernel entries reveals a linear relationship with varying data subset sizes, as shown in Figure~\ref{fig:normalizer} (Appendix).
% A linear fit to the data yields an $R^2$ value of 0.9766, confirming the linearity of this relationship. 
% Therefore, to mitigate the influence of dataset size and prevent over-penalizing the score scale, we utilize the square-root normalizer.
% \textcolor{blue}{This doesn't seem to be as critical as other ablations. Consider moving to the appendix if we run out of space.}
% \FC{Will be important if we discuss dataset size, as the normalizer is the key to consistent score magnitude across datasets. But if we're not discussing dataset size, this is not very important.}




\section{Discussions}
\label{sec:discussion}
\section{Discussions}

\subsection{The Asymmetry of the Point Spread Function (PSF) in Microscopy}

In the ideal imaging model, the Point Spread Function (PSF) of a microscope is symmetric with respect to the focal plane. 
This symmetry allows algorithms to estimate the absolute defocus distance but prevents them from determining whether the defocus is above or below the focal plane, thereby rendering one-shot autofocusing seemingly impractical.
However, in real optical microscopy systems, the PSF often exhibits significant asymmetry due to refractive index mismatches among the different media in the imaging path, such as the slide, sample, cover slip, and the surrounding medium like air or immersion oil. 
These mismatches introduce aberrations, including spherical aberration, coma, astigmatism, field curvature, and distortion, which disrupt the ideal symmetric distribution of the PSF. 
Figures~\ref{fig:psf}(a) and \ref{fig:psf}(b) present the 3D PSF and 2D PSF of our developed Whole Slide Imaging (WSI) device.
These visualizations are generated using the Gibson \& Lanni PSF model~\cite{Gibson:89} within the open-source software Fiji~\cite{Schindelin2012-jh} and the PSF Generator plugin\footnote{http://bigwww.epfl.ch/publications/kirshner1103.html}.
Figure~\ref{fig:psf}(c) shows images at symmetric defocus distances on both sides of the focal plane.
Figure~\ref{fig:psf}(d) illustrates the differences in pixel grayscale values at the same position for images at 5\si{\micro\meter} and -5\si{\micro\meter}.
These visualizations also demonstrate that, in real optical microscopy systems, the PSF is asymmetric.
Additional theoretical analysis on the PSF is provided in the supplementary materials.

\begin{figure}[H]
	\centering
	\includegraphics[width=\linewidth]{figs/PSF_fire_sq.pdf}
	\caption{\textbf{The asymmetry of PSF.} (a) and (b) illustrate the PSF of a microscopy imaging system, highlighting its asymmetry with respect to the focal plane. (c) demonstrates the defocused imaging relative to the focal plane, and (d) presents the comparison of the gray values of pixels at the same positions corresponding to 5\si{\micro\meter} and -5\si{\micro\meter}. Both of them provide corroborative evidence for the disparities in images at the corresponding locations.}
	\label{fig:psf}
\end{figure}

The asymmetry of the PSF, though potentially detrimental to image quality, presents a unique opportunity for one-shot autofocusing. This phenomenon results in images with positive or negative defocus on either side of the focal plane exhibiting distinct characteristics. Although these differences are subtle, the sophisticated feature extraction capabilities of deep learning can effectively discern them. By capitalizing on this physical principle, we propose a one-shot learning-based network designed to estimate both the defocus distance and direction from a single image.

\subsection{Autofocus for Thick Specimens}

Autofocus is generally designed for a specific focal plane, assuming that most samples exhibit little variation in elevation over a field of view.
However, for very thick samples, such as those resulting from the slicing of pathological sections, different regions within the same field of view may lie on different focal planes (see supplementary material). This can lead to a scenario where focusing on one region causes others to appear blurry, complicating autofocus efforts.
To address such challenges, strategies may contain: 1) Designate a specific region of interest for the autofocus algorithm to target exclusively; 2) Employ z-stack image fusion strategy, capturing and fusing images at various z-axis positions to achieve a uniformly sharp image across the entire field of view.




\paragraph{Temporal shift problems of MIA benchmarks.}
Recent studies have expressed concerns regarding the temporal shift issues in existing Membership Inference Attack~(MIA) benchmarks~\cite{duan2024membership,das2024blind,maini2024llm}.
Notably, datasets such as WikiMIA, BookMIA, and ArxivTection have been identified as susceptible to temporal cues, which can inadvertently simplify the membership inference task.
This simplification arises because models can exploit temporal information to distinguish between seen versus unseen data, leading to a potential overestimation of detection performance.

To ensure the robustness of our approach and mitigate potential biases introduced by temporal shifts, we conducted evaluations using 500 samples from six subsets of the Pile dataset~\cite{gao2020pile}. 
The subsets include text data from various sources, including expository proses~(Wikipedia), academic papers~(PhilPapers), emails~(Enron), news articles~(HackerNews),  web-scraped data~(Pile-CC), and user-contributed questions and answers~(StackExchange).
For each subset, the `train' set is regarded as seen data, while the `val' set serves as unseen data.
We mix these two sets according to varying contamination rates to assess our model's performance under different conditions. 
This methodology provides a rigorous assessment, ensuring that our model does not exploit temporal cues. 
As presented in Table~\ref{tab:pile}, the Spearman and Pearson correlation coefficients are both near 1.0, averaging at 0.944 and 0.948, respectively. 
These findings demonstrate that our method reliably scores contamination levels without relying on temporal shifts.



\begin{table}[t!]
    \centering
    \setlength{\tabcolsep}{8pt}
    \caption{\textbf{Evaluation on benchmarks with IID setup.} We evaluate the monotonicity of our kernel divergence score on six subsets from the Pile dataset~\cite{gao2020pile}.}
\vspace{3mm}
\resizebox{0.75\linewidth}{!}{
\begin{tabular}{l | cc }
\toprule
 \textbf{Data Subset} & $\textbf{Spearman} \uparrow$ & $\textbf{Pearson} \uparrow$ \\
 \midrule
{Wikipedia} & 0.891 & 0.922 \\
{PhilPapers} & 0.982 & 0.974 \\
{Enron} & 1.000 & 0.965 \\
{HackerNews} & 0.897 & 0.920 \\
{Pile-CC} & 0.895 & 0.908 \\
{StackExchange} & 1.000 & 0.998 \\
\midrule
\textbf{Average} & \textbf{0.944} & \textbf{0.948} \\
\bottomrule
\end{tabular}
}
\vspace{-2mm}
\label{tab:pile}
\end{table}


\begin{table}[t!]
    \centering
    \setlength{\tabcolsep}{3pt}
    \caption{\textbf{Evaluation using various models.} We evaluate the Monotonicity on the WikiMIA dataset.}
\vspace{3mm}
\resizebox{\linewidth}{!}{
\begin{tabular}{l | cc }
\toprule
 \textbf{Model} & $\textbf{Spearman} \uparrow$ & $\textbf{Pearson} \uparrow$ \\
 \midrule
Mistral-7B-Instruct-v0.2~\cite{jiang2023mistral} & 0.999 & 0.996 \\
Llama-3.1-8B-Instruct~\cite{dubey2024llama} & 0.982 & 0.952 \\
Phi-3-small-128k-instruct~\cite{abdin2024phi} & 0.892 & 0.890 \\
\bottomrule
\end{tabular}
}
\vspace{-4mm}
\label{tab:other_models}
\end{table}




\paragraph{Extension to various model families.}
We extend our evaluation to diverse models to demonstrate the versatility of our approach. 
As presented in Table~\ref{tab:other_models}, our approach consistently exhibits near-perfect correlation values across all models tested. 
These findings underscore the robustness and applicability of our method across diverse model families.


\paragraph{Computational cost.}

Our kernel divergence score  involves a fine-tuning step to obtain two kernel matrices, followed by the computation of our scoring function.
Given a dataset with $N$ samples, the fine-tuning step operates with a time complexity of $O(c_1\cdot N)$, where $c_1$ is a constant influenced by factors such as average sample length, batch size, and model dimension.
The computation of the KDS score has complexity $O(c_2\cdot N^2)$, due to the quadratic nature of kernel matrix operations.
In practice, the latency overhead caused by scoring is minimal, as these operations are highly optimized through vectorized computations.
As demonstrated in Table~\ref{tab:time}, the latency measured in seconds for each dataset confirms the efficiency and scalability of our approach. In Appendix~\ref{apdx:inthewild}, we confirm that our method can be applied to real-world benchmark datasets, where we employ our method across 11 diverse and widely used benchmarks, most of which have sizes around a few hundred to thousand samples. 

\begin{table}[htb]
\centering
\caption{Average time consumption of LLMs across different processes.}
\label{tab:time_consumption}
\begin{tabular}{lcc}
\toprule[1.5pt]
\textbf{Model} & \textbf{Identify Sensitive Function} & \textbf{Thought \& Action} \\ 
\midrule[0.8pt]
GPT-4o & 3.1s & 8.3s \\ 
Qwen2.5:32b & 7.7s & 18.2s \\ 
Deepseek-v3 & 2.7s & 7.3s \\ 
\bottomrule[1.5pt]
\end{tabular}
\end{table}
