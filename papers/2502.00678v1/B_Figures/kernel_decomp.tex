
\begin{figure*}[t!]
    \begin{center}
    % \vspace{2mm}
    \includegraphics[width=0.8\linewidth]{B_Figures/figs/kernel_decomposition.pdf}
    \end{center}
    \vspace{-2mm}
    \caption{\textbf{Decomposition of the Kernel Divergence Score.} Each component of the Kernel Divergence Score function is shown. $\Phi(\cdot)$ denotes the kernel similarity matrix, $Z$ and $Z'$ represent normalized sample embeddings before and after fine-tuning, and $\odot$ is the Hadamard product. 
    %We use the perturbed TruthfulQA~\cite{li2023estimating} dataset on
    Score and embeddings are based on Llama-3.1-8B-Instruct~\cite{dubey2024llama}. 
    (\textbf{Left}) shows that the original kernel similarity matrix before fine-tuning. Note, that diagonal values are zeroed for better visualization, because all diagonal values are 1 in RBF kernels. (\textbf{Middle}) reveals that fine-tuning alters relationships among unseen samples more than those among seen samples. (\textbf{Right}) Combining the two panels enhances the distinction between seen and unseen samples, thereby enabling a more reliable measurement of contamination levels.}
\label{fig:k_decomp}
\end{figure*}
