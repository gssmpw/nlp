\begin{abstract}
Dataset contamination, where evaluation datasets overlap with pre-training corpora, inflates performance metrics and undermines the reliability of model evaluations. Quantifying dataset contamination thus becomes essential to ensure that performance evaluations genuinely reflect a model's ability to generalize to unseen data, rather than relying on memorized examples. 
To address this problem, we propose 
Kernel Divergence Score (KDS), a novel method that quantifies dataset contamination by computing the divergence between the kernel similarity matrix of sample embeddings, before and after fine-tuning on the benchmark dataset. Leveraging the insight that fine-tuning affects unseen samples more significantly than seen ones, KDS provides a reliable measure of contamination. Through extensive experiments on controlled contamination scenarios, KDS demonstrates a near-perfect correlation with contamination levels and outperforms existing baselines. Additionally, we perform comprehensive ablation studies to analyze the impact of key design choices, providing deeper insights into the components and effectiveness of KDS. These ablations highlight the importance of leveraging fine-grained kernel-based information and confirm the reliability of the proposed framework across diverse datasets and settings.

\end{abstract}