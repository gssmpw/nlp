\section{Conclusion}

In this paper, we explore young job seekers' perceptions and strategy use in the face of the increasing prevalence of automated employment decision tools (AEDTs) in workplace hiring practices. To do so, we survey several hundred computer science students from three universities in Philadelphia, asking about their perceptions of procedural fairness and willingness to be evaluated by different levels of automation (ranging from human-only to AI-only review) for different evaluation types (coding assessments, resume reviews, and interviews). We find that students perceive AEDTs as much less fair than human review and have a lower willingness to be evaluated by AI decision-makers for less-technical evaluation types (resume review and interviews). While they do rate automated methods as somewhat fair and are somewhat more willing for more technical evaluation (coding assessments), across the board they always prefer some amount of human involvement to AI-only decision-making. In particular and contrary to the actual implementation of AEDTs, which favor employer priorities, participants showed a preference for the use of automation as long as a human reviewed any rejections. Also, when asked whether they would like to be evaluated using these methods participants' responses were lower than their fairness ratings. On the whole, these results speak to young job seekers' distrust of and distaste for the use of automation in hiring. 

Finally, we also examined participants' strategies for navigating these systems, as well as demographic attributes, and the effect of both on reported job outcomes (whether they were successful in receiving any job offers). Strikingly, we found that most strategies did not significantly impact perceptions or job success. The only predictors of job success were the percentage of jobs applied to with a referral and family income, reflecting the continued role of socioeconomic privilege in the work and in life.
Far from being ``bias-free'' or ``bias-mitigated,'' as many AEDTs self-advertise, our work suggests they may instead exacerbate an existing pattern of inequity, since those more financially or socially privileged may be able to circumvent them altogether. Given these findings, we implore AEDT developers, employers, and policymakers to reconsider the use of entirely automated hiring pipelines (especially for less technical tasks), to proactively consider and combat existing social inequalities, and to create meaningful requirements for transparency and fairness in the use of these tools. 