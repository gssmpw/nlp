\documentclass[manuscript, nonacm]{acmart}

%% \BibTeX command to typeset BibTeX logo in the docs
\AtBeginDocument{%
  \providecommand\BibTeX{{%
    Bib\TeX}}}

\begin{document}

\title[Navigating Automated Hiring]{Navigating Automated Hiring: Perceptions, Strategy Use, and Outcomes Among Young Job Seekers}

\author{Lena Armstrong}
\email{larmstrong@g.harvard.edu}
\orcid{0000-0001-8306-9270}
\affiliation{%
  \institution{Harvard University}
  \city{Cambridge}
  \state{MA}
  \country{USA}
}

\author{Danaé Metaxa}
\email{metaxa@seas.upenn.edu }
\orcid{0000-0001-9359-6090}
\affiliation{%
  \institution{University of Pennsylvania}
  \city{Philadelphia}
  \state{PA}
  \country{USA}
}

\renewcommand{\shortauthors}{Lena Armstrong and Danaé Metaxa}

\begin{abstract}
  As the use of automated employment decision tools (AEDTs) has rapidly increased in hiring contexts, especially for computing jobs, there is still limited work on applicants' perceptions of these emerging tools and their experiences navigating them. To investigate, we conducted a survey with 448 computer science students (young, current technology job-seekers) about perceptions of the procedural fairness of AEDTs, their willingness to be evaluated by different AEDTs, the strategies they use relating to automation in the hiring process, and their job seeking success. We find that young job seekers' procedural fairness perceptions of and willingness to be evaluated by AEDTs varied with the level of automation involved in the AEDT, the technical nature of the task being evaluated, and their own use of strategies, such as job referrals. Examining the relationship of their strategies with job outcomes, notably, we find that referrals and family household income have significant and positive impacts on hiring success, while more egalitarian strategies (using free online coding assessment practice or adding keywords to resumes) did not. Overall, our work speaks to young job seekers' distrust of automation in hiring contexts, as well as the continued role of social and socioeconomic privilege in job seeking, despite the use of AEDTs that promise to make hiring ``unbiased.'' 
\end{abstract}

%% The code below is generated by the tool at http://dl.acm.org/ccs.cfm.
%% Please copy and paste the code instead of the example below.

\begin{CCSXML}
<ccs2012>
<concept>
<concept_id>10003456.10003457.10003458</concept_id>
<concept_desc>Social and professional topics~Computing industry</concept_desc>
<concept_significance>500</concept_significance>
</concept>
<concept>
<concept_id>10003456.10003457.10003580.10003568</concept_id>
<concept_desc>Social and professional topics~Employment issues</concept_desc>
<concept_significance>500</concept_significance>
</concept>
<concept>
<concept_id>10003456.10003457.10003567.10003569</concept_id>
<concept_desc>Social and professional topics~Automation</concept_desc>
<concept_significance>500</concept_significance>
</concept>
</ccs2012>
\end{CCSXML}

\ccsdesc[500]{Social and professional topics~Computing industry}
\ccsdesc[500]{Social and professional topics~Employment issues}
\ccsdesc[500]{Social and professional topics~Automation}

\keywords{automated hiring, AI fairness, job outcomes, automated employment decision tools}

\maketitle

\section{Introduction}
\label{sec:intro}

Foundational models (FMs)~\cite{zhang2024data, zhou2023comprehensive} have shown remarkable progress in the healthcare domain, enabling professional-like assessment of disease diagnosis, treatment decision-making, and monitoring~\cite{zhang2023text, wang2022medclip, lu2023mi-zero}. 
Examples include LLaVA-Med~\cite{li2023llava}, Med-PaLM Multimodal~\cite{tu2024towards}, and Med-Flamingo~\cite{moor2023med}, have demonstrated their capacity on question answering, medical image analysis, and report generation.
These studies follow a predominant top-down model development strategy that requires upstream developers to collect data and train models for downstream tasks. 
Consequently, the developed model capabilities are heavily dependent on the training data, limiting their generalization performance in diverse clinical scenarios. 
For instance, Med-Gemini~\cite{yang2024advancing} reveals promising general capabilities in report generation while it lags behind state-of-the-art (SoTA) models on classification tasks, especially for out-of-domain applications. 
This indicates that while the generalizability of the foundation model is promising, more solutions are expected to meet the various specialized clinical needs.

To address these challenges, multi-center data centralization becomes essential to enhance model capacity and robustness across varied clinical scenarios~\cite{rajpurkar2022ai}. 
Centralizing distributed data can significantly improve model training and inference performance.
However, the process of medical data storage, transfer, and aggregation among centers requires extra efforts to ensure data security and system interoperability~\cite{bradford2020international}.
Moreover, a growing concern for patient privacy makes large-scale multi-center data sharing particularly challenging. 
While efforts like federated learning~\cite{wen2023survey, li2020review} can achieve good model performance on local data, the need for synchronized system coordination presents significant challenges, as clients are unable to update asynchronously. This limitation greatly restricts the practical capability of such approaches.
As a result, without a flexible collaboration, medical community still struggles to fully utilize the isolated data and local computation resources for comprehensive medical AI model development. 
To address this dilemma, open-source platforms encourage public data sharing and knowledge integration~\cite{markiewicz2021openneuro, zenodo}.
However, these platforms focus solely on raw data sharing while seldom providing collaborative model training or cooperation between different institutions.
Recently, collaborative learning has emerged as a viable approach for enhancing multi-model robustness~\cite{boulemtafes2020review}. 
For instance, software-like model development~\cite{raffel2023building} mimics software engineering practices by introducing structured workflows, enabling merging, version control, and continuous model integration.
Under this design, model ability can be strengthened with incremental knowledge updates similar to the version updating in software development. 

Although collaborative learning provides a multi-model collaboration, two key challenges remain in the leakage of raw data during collaboration~\cite{huang2023lorahub} and the synchronization of multiple collaborators~\cite{mcmahan2017communication} in the medical AI community. It is still challenging to integrate decentralized, privacy-sensitive data across institutions, leading to under-utilized insights and fragmented knowledge sharing~\cite{kaissis2020secure, rajpurkar2022ai, abdullah2021ethics}.
 To address these challenges, inspired by the collaborative software development, we propose \textbf{Med}ical \textbf{Fo}undation Models Me\textbf{rg}ing (\textbf{MedForge}), a cooperative workflow enabling continuously community-driven foundation model (FM) development.
MedForge enables a lightweight manner for individual centers to share their knowledge among multiple centers, minimizing the burden of data transmission and integration while enhancing model robustness.
Meanwhile, MedForge facilitates asynchronous and flexible collaboration, allowing individual centers to continuously update and improve medical FMs without the need for real-time synchronization.
Similar to open-source software development, MedForge incrementally updates medical knowledge and follows a sustainable model development scheme. 
This key design emphasizes a bottom-up construction of a multi-task medical FM, allowing downstream users to collaboratively build, refine, and update the upstream model according to their local resources. Our major contributions of MedForge are as below: 
\begin{enumerate}
    \item[$\bullet$] We introduce a collaborative workflow to promote the merging scheme of open-source software development. Our proposed MedForge allows distributed clinical centers to asynchronously contribute to comprehensive medical model construction while reducing transmitting costs among centers and avoiding the leakage of raw data, thus enhancing the utilization of private resources in the healthcare system. 
    \item[$\bullet$] We propose two effective knowledge-merging strategies for the asynchronous branch contribution. The MedForge-Fusion strategy updates the plugin module parameters of the main model during the merging phase, whereas the MedForge-Mixture strategy integrates the output of the plugin module by memorizing each contributor's coefficient. These strategies make MedForge more flexible and versatile. MedForge-Fusion is friendly to implement, while the MedForge-Mixture offers better performance and robustness.
    \item[$\bullet$]  We comprehensively evaluate model merging strategies to accumulate medical knowledge among multiple branch plugin modules. MedForge yields superior performance on medical classification tasks compared to other collaborative baselines across multiple datasets. We demonstrate the robustness of MedForge by shuffling the task order and evaluating various configurations of plugin modules and dataset distillation methods.
\end{enumerate}




\section{Background} \label{sec:background}

% \subsection{Capture the Flag (CTF) Challenges}

% CTF challenges simulate real-world cyber-attack scenarios and have emerged as a popular medium for practical cybersecurity training, evaluation, and research. These challenges can simulate real-world attack and defense scenarios and thus assist competitors in developing practical skills in areas such as cryptography, binary exploitation, and reverse engineering. 
% Evaluation of autonomous LLM agents works best with jeopardy-style CTF challenges that focus on standalone software that must be compromised \cite{shao2024nyu,pieterse2024friend}.
% The standalone software may be a binary that can be reverse engineered or exploited, encrypted data that can be decrypted, or a web server whose authentication can be bypassed. After successfully compromising the software, a unique ``flag'' string is either found or revealed by the software server.
% The unique flag string is a concrete indicator of the success of a CTF challenge.
% Recent studies use benchmarks of CTF challenges to evaluate LLM agents on their ability to solve complex tasks and demonstrate practical skills in cybersecurity \cite{shao2024nyu,shao2024empirical,abramovich2024enigma, muzsai2024hacksynth, zhang2024cybenchframeworkevaluatingcybersecurity,yang2023language,turtayev2024hacking}
% Platforms like PicoCTF~\cite{picoctf}, TryHackMe~\cite{tryhackme}, CTFTime~\cite{ctftime} and HackTheBox~\cite{hackthebox} have popularized these formats by providing structured challenges for learners at various skill levels.

% Research indicates that CTF challenges can foster cybersecurity expertise and serve as tools for evaluating facility with cybersecurity skills~\cite{chicone2018using}. They are widely used in academia to enhance learning outcomes in cybersecurity education, with studies demonstrating their effectiveness in promoting analytical thinking and teamwork~\cite{hanafi2021ctf,leune2017using,vykopal2020benefits}. Furthermore, the integration of CTF challenges into research environments enables benchmarking of advanced AI systems like LLMs. .

% Yet, challenges in CTF design persist. These include achieving significant performance, preserving context across tasks, and handling complex, dynamic CTFs that rely on multidisciplinary approaches. Implementing strategies to address these issues enhances problem-solving efficiency, enabling more accurate, adaptive, and effective responses to evolving challenges within CTF environments.


% \subsection{Prompt Engineering}
% \subsection{Prompt Engineering for CTF}
% \subsection{LLM Agents}

% As the use of LLMs to solve CFT challenges expands, prompt engineering is becoming a critical technique for enhancing performance. Various methods have been explored to craft prompts that effectively guide LLMs to the solution of complex cybersecurity problems. Each of these solutions have their own unique strengths and limitations.
%\meet{add more references for LLM agents in other domains, like SWE-Agent, also talk about use of function calling}
Text-based LLMs take a text prompt as input from the user, and produce a text output that follows the user prompt.
LLMs have a finite length of text tokens that they can process called the context.
An alternating sequence of user prompts and LLM outputs makes a conversation and is the basis of chat-based LLM interfaces like ChatGPT.
To remove the user from the loop and create autonomous agents, a feedback mechanism is added based on the LLM outputs, so that the LLM can autonomously continue the conversation.
\citet{yang2023intercode} introduce iterative feedback prompting where the LLM is tasked with writing a piece of code, and the code's compilation and execution logs are provided as feedback, which the LLM uses to iteratively refine it's output.
Recent LLMs support function calling, a way to provide a set of actions to the LLM that it may choose to ``call'' as a function.
In this manner, LLM agents can be provided with many ``tools'' such as a command line, web search, file editing, and code execution \cite{wang2024surveyllmagents}, so that they can autonomously perform various tasks like software development \cite{yang2024sweagent}, web browsing \cite{yoran2024assistantbench}, or solve CTF challenges~\cite{shao2024nyu, abramovich2024enigma}.

With access to the command line and file editing tools, LLM agents can autonomously solve many tasks, but they still struggle on complex long-horizon tasks such as CTF challenges that require multiple steps.
Plan-and-solve prompting \cite{wang2023planandsolve} enhances long-term focus of the agent by incorporating a planning phase before iterative execution. This helps agents tackle ambiguous or complex tasks by structured strategies \cite{turtayev2024hacking}.
ReAct (reasoning + action) \cite{yao2022react} combines step-by-step reasoning with action, allowing the agent to adjust dynamically through iterative cycles. ReWOO (Reasoning without Observation) \cite{xu2023rewoo} separates the reasoning process from tool outputs and observations, allowing it to handle multi-step reasoning tasks efficiently while maintaining focus.
The prompting methods in these agents involve static hard-coded templates where environment and task information is filled in.
While static prompts provide straightforward guidance, they often fail to adapt to different problems and complex tasks, limiting their effectiveness.
Auto-prompting~\cite{shin-etal-2020-autoprompt, zhou-etal-2023-revisiting, zhang2023automatic} is a technique to allow the LLM itself to generate a highly-relevant prompt. Auto-prompting invokes more factual responses and reduces hallucinations in LLMs.
D-CIPHER incorporates auto-prompting as a separate agent that can explore the environment and generate a better prompt.
%Based on the given prompt, LLM agents make a decision and proceed further to find flags.  To address this gap, we propose \textbf{dynamic prompting}, where the LLM agent autonomously generates prompts based on the CTF challenge's context and stage.
%include a static template which needs to be given to LLM to solve the CTF challenges. For instance, the NYU CTF framework provides a static prompt as \emph{``Please proceed to the next step using your best judgment"} for each decision making point. 

% To address this gap, we introduce a novel approach where the LLM agent generates the next prompt autonomously based on the current context and stage of the CTF challenge, a technique we call \textbf{dynamic prompting}.


Expanding on single LLM agents, multi-agent LLM systems are a powerful approach to enhance problem-solving by simulating team-based collaboration. Specialized agents, each with distinct objectives, work together to tackle different aspects of complex tasks \cite{guo2024largelanguagemodelbased}
Multi-agent systems are effective in cybersecurity applications. For instance, Audit-LLM~\cite{song2024audit} deploys a  multi-agent system for insider threat detection by employing agents to decompose tasks, build tools, and use collaborative reasoning to enhance detection accuracy. Liu~\cite{liu2024multi} explores multi-agent systems to enhance incident response in cybersecurity by examining centralized, decentralized, and hybrid team structures to assess how LLM agents can improve decision-making, adaptability, and coordination during cyber-attack scenarios. AutoSafeCoder~\cite{nunez2024autosafecoder} enhances the security of code generated by LLMs by incorporating a coding agent for code generation, a static analyzer agent that identifies vulnerabilities, and a fuzz testing agent for dynamic testing to detect runtime errors. Division of responsibilities among different agents allows AutoSafeCoder to produce secure, functionally correct code. 

% With the growing use of LLMs in CTF challenges, prompt engineering is key to enhancing performance. Various methods guide LLMs in solving complex cybersecurity tasks, each with distinct strengths and limitations.

% \textbf{Single Turn (Zero-Shot Prompting)} involves providing the model with a one-time task description that outputs  an immediate solution. This is efficient for straightforward tasks~\cite{yang2023intercode}. In contrast, \textbf{Try Again (Iterative Feedback Prompting)} uses iterative feedback to refine responses over multiple attempts, mimicking real-world problem-solving~\cite{yang2023intercode}. The \textbf{Plan \& Solve} enhances adaptability by incorporating a planning phase before iterative execution. This helps models tackle ambiguous or complex tasks by  structured strategies~\cite{turtayev2024hacking}. Additionally, \textbf{ReAct (Reasoning + Action)} combines step-by-step reasoning with action, allowing the model to adjust dynamically through iterative cycles. This makes it particularly effective for evolving and complex challenges like CTFs~\cite{yao2023react}. 
% These prompting techniques highlight diverse approaches to optimizing LLM performance in cybersecurity tasks. 

% Multi-agents!


%\meet{Add references for auto-prompting, and shorten this para}
%\nanda{Maybe we can add this to previous paragraphs which discusses other prompting methods such as plan-and-solve and ReAct method}
% All of these prompting methods include a static template which needs to be given to LLM to solve the CTF challenges. For instance, the NYU CTF framework provides a static prompt as \emph{``Please proceed to the next step using your best judgment"} for each decision making point. 
% Based on the given prompt, LLM agents make a decision and proceed further to find flags. While static prompts provide straightforward guidance, they often fail to account for the evolving nature of complex tasks, limiting their effectiveness in multi-step or ambiguous CTF challenges. To address this gap, we propose \textbf{dynamic prompting}, where the LLM agent autonomously generates prompts based on the CTF challenge's context and stage.
% % To address this gap, we introduce a novel approach where the LLM agent generates the next prompt autonomously based on the current context and stage of the CTF challenge, a technique we call \textbf{dynamic prompting}.
% Dynamic prompting adapts instructions to task progress, ensuring instructions are contextually relevant and reflective of the specific obstacles encountered. By iterating based on feedback and intermediate outputs, it continuously refines the LLM’s approach, enhancing problem-solving for dynamic tasks like CTFs.
% This adaptive process not only mirrors how humans tackle complex problems but also improves the model’s ability to handle unpredictable scenarios, making it particularly advantageous for cybersecurity tasks like CTFs where conditions change dynamically.


% The very first prompt type used in several applications is \textbf{Single Turn (Zero-Shot Prompting)}~\cite{yang2023intercode}. In single-turn prompting, the model receives a one-time, straightforward task description and is expected to generate a complete response without further interaction. The initial output is directly assessed, making this approach efficient for tasks where minimal feedback or iteration is required. This method tests the model’s ability to understand and respond to tasks immediately, relying heavily on the model's pre-trained knowledge and generalization capabilities.

% Along with this, The prompting method named \textbf{Try Again (Iterative Feedback Prompting)}~\cite{yang2023intercode} has been also used in several appreciations specially to solve CTF challenges. It is an iterative prompting method involves continuous interaction, where the model is provided with feedback after each attempt. The model can refine its responses over multiple turns based on the observations or execution results from previous outputs. This iterative process continues until the task is successfully completed or a maximum number of interactions is reached. This approach closely mirrors real-world problem-solving, where adjustments are made iteratively based on evolving circumstances or feedback.

% Some application are also using \textbf{Plan \& Solve}~\cite{turtayev2024hacking} prompting method which enhances problem-solving by dividing the process into a planning phase followed by execution. Initially, the model formulates a strategy based on the task description and available information, allowing for a structured approach to ambiguous or complex problems. This plan guides the subsequent execution phase, where the model carries out actions iteratively, refining its approach based on feedback. In more challenging scenarios, re-planning mid-task further improves adaptability and performance. This method proves effective in tasks like CTF challenges, where vague instructions require careful analysis and step-by-step resolution.

% Further some application are also using \textbf{ReAct (Reasoning + Action)}~\cite{yao2023react} prompting method blends reasoning with action by guiding the model to think through tasks step-by-step before executing actions. At each step, the model generates a thought based on the task and observations, which informs the next action. The action is executed, and the resulting feedback refines the model’s understanding for the next cycle. This continuous process helps the model adapt dynamically to complex tasks, making it effective for CTF challenges where logical reasoning and step-by-step execution are essential.

\section{Related Works} \label{sec:related_work}


\begin{table}[htpb]
    \centering
    \caption{Feature comparison of LLM agents for solving CTFs.}
    \label{tab:related_work_comparison}
    \begin{tabular}{lcccccc}
    \toprule
         \textbf{Study} & \rotatebox{90}{\textbf{\# CTFs}} & \rotatebox{90}{\textbf{Open bench}} & \rotatebox{90}{\textbf{Tool use}}  & \rotatebox{90}{\textbf{Autonomous}} & \rotatebox{90}{\textbf{Multi-agent}} &\rotatebox{90}{\textbf{Auto-prompt}} \\
    \cmidrule{2-7}
     % \textbf{Study} & \textbf{Dynamic} & \textbf{Used} & \textbf{Multi-} & \textbf{Automatic} & \textbf{Tool} & \textbf{\# of} \\
         Tann et al. \cite{tann2023using} &  $7$ & \purplecross & \purplecross & \purplecross & \purplecross & \purplecross  \\
         Shao et al. \cite{shao2024empirical} & $26$ & \purplecross & \tealcheck & \tealcheck & \purplecross & \purplecross  \\
         InterCode-CTF\cite{yang2023language} & $100$ & \tealcheck & \tealcheck & \tealcheck & \purplecross & \purplecross   \\
         NYU CTF Bench \cite{shao2024nyu} & $200$ & \tealcheck & \tealcheck & \tealcheck & \purplecross & \purplecross \\
         Turtayev et al. \cite{turtayev2024hacking} & $100$ & \tealcheck & \tealcheck & \tealcheck & \purplecross & \purplecross\\
         Cybench \cite{zhang2024cybenchframeworkevaluatingcybersecurity} & $40$ & \tealcheck & \tealcheck & \tealcheck & \purplecross & \purplecross \\
         EnIGMA \cite{abramovich2024enigma} & $350$ & \tealcheck & \tealcheck & \tealcheck & \purplecross & \purplecross\\
         HackSynth \cite{muzsai2024hacksynth} & $200$ & \tealcheck & \tealcheck & \tealcheck & \tealcheck & \purplecross \\
         \textbf{D-CIPHER (ours)} & $290$ & \tealcheck & \tealcheck & \tealcheck & \tealcheck & \tealcheck \\
    \bottomrule
    \end{tabular}
\end{table}



% \subsection{LLMs on Cybersecurity}
% \subsection{LLM Agents for CTF}

%LLMs have a vast knowledge base that can be tapped for cybersecurity use.
Tann et al.~\cite{tann2023using} evaluate early LLMs such as ChatGPT and Google Bard in solving CTF challenges and answering professional certification questions, showing that LLM responses contain key task information.
%Many works extend the LLM capabilities by providing them access to programming and command execution tools, to form autonomous agents. 
The InterCode-CTF agent~\cite{yang2023intercode} reveals that LLM agents demonstrate basic cybersecurity skills, however they struggle with more complex tasks.
The NYU CTF baseline agent~\cite{shao2024empirical} integrates external tools into the LLM's function-calling features and demonstrate improved potential of tool-assisted LLMs to solve CTFs, however it exhausts the LLM context length when command output history becomes very long. InterCode-CTF manages this issue by truncating the history to only show the LLM the last few iterations. Even so, LLM agents face issues with longer tasks.
%NYU CTF Bench~\cite{shao2024nyu}, a benchmark of 200 CTF challenges, presents a baseline agent with specialized reverse engineering tools and category-specific prompts, demonstrating their importance to solve CTFs.
% The NYU CTF baseline agent faces issues of LLM context length when complex tasks run for several iterations and the entire command and output history becomes longer than the LLM's context window size. The InterCode agent manages this issue by truncating the history to only show the LLM the last few iterations.


Excessive tool availability and verbose interfaces can overwhelm agents, leading to inefficiencies. Agents perform better with a focused set of tools with well-defined interfaces~\cite{yang2024sweagent}.
EnIGMA~\cite{abramovich2024enigma} agent incorporates interactive tools and in-context learning techniques to achieve state-of-the-art results. % on the NYU CTF Bench, HackTheBox, and Cybench benchmarks.
For better context management, EnIGMA also uses an LLM summarizer that summarizes the command outputs for the main agent.

HackSynth~\cite{muzsai2024hacksynth}, an LLM agent for autonomous penetration testing, shows that iterative planning and feedback summarization stages help the agent finish multiple tasks and improves overall problem solving.
Similarly, Cybench~\cite{zhang2024cybenchframeworkevaluatingcybersecurity} introduces a benchmark of 40 CTF challenges augmented with step-by-step tasks, demonstrating better focus of LLM agents on smaller tasks, leading to improved success and alleviating the context length issue.
\citet{turtayev2024hacking} expand on InterCode-CTF by implementing plan-and-solve prompting, achieve significant improvement on the InterCode-CTF benchmark. They show that prompting techniques can improve performance even with simple toolsets.
% . Their baseline agent is evaluated in unguided mode (i.e. fully autonomous), and guided mode where the agent is given one task at a time. Their results indicate that providing smaller tasks to the LLM agents improve their focus yielding improved success on complex challenges while .

These works highlight that LLM agents excel at implementing code and executing commands to accomplish small concrete tasks when provided with dynamic feedback and task-specific toolsets. While these works  involved using multiple LLMs with different tasks such as planning and summarizing along-side a main agent, D-CIPHER is the first work to formulate a multi-agent system where there is a bifurcation of responsibilities between agents and meaningful well-defined interactions for dynamic feedback.
Table~\ref{tab:related_work_comparison} shows a feature comparison of D-CIPHER with related works on LLM agents for autonomous CTF solving.
%\meet{some description of the feature comparison?}
% Recent research has focused on enable autonomous solving of CTF challenges~\cite{shao2024empirical,shao2024nyu,abramovich2024enigma}. These agents typically operate in containerized environments to ensure reproducibility and modularity. 

% As an early effort, Tann et al.~\cite{tann2023using} evaluated the effectiveness of LLMs, such as OpenAI's ChatGPT, Google Bard, and Microsoft Bing, in solving cybersecurity CTF challenges and answering professional certification questions. 
% % Their study results show that LLMs performed well on $7$ CTF test cases, with ChatGPT solving $6$, Bard $2$, and Bing $1$. 
% The study shows that LLM responses often contain key information essential for solving tasks.

% The InterCode framework~\cite{yang2023intercode} approaches coding as an interactive process and uses execution feedback to improve code generation. As described in Yang et al.~\cite{yang2023intercode}, InterCode-CTF integrates CTF benchmarks into a reinforcement learning environment that can evaluate the cybersecurity capabilities of language agents. It features $100$ tasks that tapskills such as reverse engineering, forensics, and binary exploitation. While existing language agents demonstrate basic cybersecurity skills, evaluations indicate they struggle with more complicated complex tasks unless the system is fine-tuned or given external support. 
% cite Intercode: Standardizing and benchmarking interactive coding with execution feedback

% Another notable example is an LM agent developed by Shao et al. specifically to automate CTF tasks. 
% Shao et al.~\cite{shao2024empirical} developed a LM agent to automate CTF tasks.
% % They report an accuracy rate of  $46\%$ on $26$ CTF challenges sourced from CSAW'23 Qualifying round competition using GPT-4.
% By effectively combining LLM capabilities with external tools, the researchers demonstrated the potential of tool-assisted LLMs to solve complex problems. Building on this, the team incorporated a broader range of cybersecurity tools and interfaces that enhance both accuracy and versatility. 
% Empirical results show their system outperforms baselines on both the InterCode CTF benchmark and the NYU CTF benchmark.

% Shao et al.~\cite{shao2024nyu} presented a diverse, open-source database of CTF challenges that can be used to benchmark an LLM's ability to solve cybersecurity problems.
% It provides a scalable platform for developing and testing AI-driven approaches for vulnerability detection and resolution, facilitating advancements in automated cybersecurity tasks. The benchmark database and automated framework were successfully applied to the performance of five LLMs. 

% The Cybench benchmark~\cite{zhang2024cybenchframeworkevaluatingcybersecurity} provides another significant contribution by creating a framework tailored to solving CTF challenges. % Cybench: A framework for evaluating cybersecurity capabilities and risk
% % Their benchmark environment achieves an accuracy of $17.5\%$ using Claude 3.5 Sonnet. 
% Such frameworks operate in Linux-based containerized environments, such as Kali Linux, which includes pre-installed cybersecurity tools. However, excessive tool availability can overwhelm agents, leading to inefficiencies. Research indicates that agents perform better with a focused set of tools that have well-defined interfaces~\cite{yang2024sweagent}. % Swe-agent: Agent-computer interfaces enable automated software engineering



% Muzsai et al. introduced HackSynth~\cite{muzsai2024hacksynth}, an LLM-based agent for autonomous penetration testing. It uses a dual-module architecture that consists of a Planner and a Summarizer, allowing for iterative command generation and feedback processing. The framework is evaluated using two benchmark sets from platforms like PicoCTF~\cite{picoctf} and OverTheWire~\cite{overthewire}. These benchmarks address $200$ challenges drawn from various domains and difficulty levels. Results of their study show that HackSynth, especially with the GPT-4o model, achieves the best performance. This highlights the potential of LLM-based agents in advancing autonomous penetration testing.
 % Using basic prompting techniques and expanding tool availability, the study highlights how straightforward approaches can unlock the latent potential of LLMs for cybersecurity tasks. Their work emphasizes that simple LLM designs can effectively solve CTF challenges, and thus broaden the number of cybersecurity applications without the need for advanced engineering.

% \begin{table*}[]
%     \centering
%     \begin{tabular}{|c|c|>{\centering\arraybackslash}p{4.5cm}|c|c|c|c|c|c|}
%     \hline
%          \textbf{Study} & \textbf{Dynamic} & \textbf{Used} & \textbf{Multi-} & \textbf{Open} & \textbf{Automatic} & \textbf{Tool} & \textbf{\# of} & \textbf{\# of} \\
%          & \textbf{Prompt} & \textbf{Benchmarks} & \textbf{Agents} & \textbf{Dataset} & \textbf{Framework} & \textbf{Use} & \textbf{LLMs} & \textbf{CTFs}\\
%          \hline
%          Tann et al.~\cite{tann2023using} & \purplecross & Manual collected & \purplecross & \purplecross & \purplecross & \purplecross & $3$ & $7$ \\
%          \hline
%          InterCode-CTF~\cite{yang2023language} & \purplecross &  PicoCTF~\cite{picoctf} & \purplecross & \purplecross& \purplecross & \purplecross & $1$ & $100$  \\
%          \hline
%          Shao et al.~\cite{shao2024empirical} & \purplecross & CSAW 2023 & \purplecross & \purplecross & \tealcheck & \tealcheck & $4$ & $26$ \\
%          \hline
%          Shao et al.~\cite{shao2024nyu} & \purplecross & NYU CTF~\cite{shao2024nyu} & \purplecross & \tealcheck & \tealcheck & \tealcheck & $5$ & $200$ \\
%          \hline
%          Cybench~\cite{zhang2024cybenchframeworkevaluatingcybersecurity} & \purplecross & Cybench~\cite{zhang2024cybenchframeworkevaluatingcybersecurity}  & \purplecross & \tealcheck & \tealcheck & & $8$ & $40$ \\
%          \hline
%          EnIGMA~\cite{abramovich2024enigma} & \purplecross & NYU CTF~\cite{shao2024nyu}, InterCode-CTF~\cite{yang2023language},  HackTheBox~\cite{hackthebox} & \purplecross & \purplecross & \tealcheck & \tealcheck & $3$ & $350$ \\
%          \hline
%          HackSynth~\cite{muzsai2024hacksynth} & \purplecross & PicoCTF~\cite{picoctf}, OverTheWire~\cite{overthewire} & \tealcheck & \tealcheck & \tealcheck & \tealcheck & $8$ & $200$ \\
%          \hline
%          Turtayev et al.~\cite{turtayev2024hacking} & \purplecross & InterCode-CTF~\cite{yang2023language} & \purplecross & \purplecross & \purplecross & \purplecross & $4$ & $100$ \\
%          \hline
%          \textbf{D-CIPHER (Proposed)} & \tealcheck & NYU CTF~\cite{shao2024nyu}, Cybench \cite{zhang2024cybenchframeworkevaluatingcybersecurity}, HackTheBox \cite{hackthebox} & \tealcheck & \tealcheck & \tealcheck & \tealcheck & 5 & 290 \\
%          \hline
%     \end{tabular}
%     \caption{Comparison with LLM-based CTF solving Literature}
%     \label{tab:related_work_comparison}
% \end{table*}




% \subsection{Multi-agent framework}

% The use of multi-agent LLM systems in Capture the Flag (CTF) challenges is emerging as a powerful approach to enhance cybersecurity problem-solving. Multi-agent frameworks mimic team-based collaboration, where multiple LLM agents, each with specialized expertise, work together to tackle complex tasks. This approach reflects real-world cybersecurity operations, where success often depends on coordinated efforts from teams with diverse skills, each addressing different components of a security challenge.
% Multi-agent LLM systems are emerging as a powerful approach to enhance cybersecurity problem-solving by simulating team-based collaboration. Specialized agents, each with distinct objectives, work together to tackle different aspects of complex security tasks. This mirrors real-world cybersecurity operations, where coordinated efforts and diverse skills are essential for addressing evolving threats and vulnerabilities.

% CTF challenges cover a wide range of domains, including cryptography, reverse engineering, forensics, and web exploitation. Multi-agent systems can distribute the workload by assigning agents to handle specific tasks. This enables parallel problem-solving and emulates the collaborative nature of human teams. For example, one agent may specialize in guiding the fellow agents to what needs to be done, while another executes the instructions, ensuring that tasks are addressed without losing the context, and implementing reasoning from multiple LLMs. This division of labor boosts efficiency and enables problem-solving from multiple perspectives.
% This division of labor enhances efficiency and allows the system to approach problems from multiple perspectives, reflecting the interdisciplinary approach often used in cybersecurity teams.

% Guo et al.~\cite{guo2024largelanguagemodelbased} highlight the strengths of multi-agent LLMs in complex, multi-step tasks where different agents handle specific roles The framework HackSynth~\cite{muzsai2024hacksynth} is a multi-agent penetration testing framework in which agents operate collaboratively to address vulnerabilities in staged environments. Their work emphasizes that when agents work as a cohesive team, they outperform single-agent approaches. This is particularly true when facing layered, iterative challenges. 
% This team-based model of problem-solving aligns closely with how cybersecurity professionals approach real-world security incidents and penetration testing exercises.

% Multi-agent LLM systems have shown effectiveness in various other applications. For instance,  Audit-LLM~\cite{song2024audit} presents a multi-agent framework for insider threat detection using log analysis. It employs agents to decompose tasks, build tools, and use collaborative reasoning to enhance detection accuracy. Liu~\cite{liu2024multi} explores the application of LLM-based multi-agent systems to enhance incident response (IR) in cybersecurity. Utilizing the ``Backdoors \& Breaches" tabletop game as a simulation environment, the study examines centralized, decentralized, and hybrid team structures to assess how LLM agents can improve decision-making, adaptability, and coordination during cyberattack scenarios. AutoSafeCoder~\cite{nunez2024autosafecoder} is a multi-agent system designed to enhance the security of code generated by LLMs. The framework comprises three agents: a Coding Agent responsible for code generation, a Static Analyzer Agent that identifies vulnerabilities through static analysis, and a Fuzzing Agent that performs dynamic testing using mutation-based fuzzing to detect runtime errors. By integrating both static and dynamic testing in an iterative process, AutoSafeCoder aims to produce secure, functionally correct code. 

% To enhance CTF-solving by promoting team-based specialization, we employ a multi-agent CTF solving agent. Within this framework, agents tackle tasks aligned with their strengths. Tasks are executed in parallel, improving efficiency and accelerating progress. Agents share insights, adapt refining strategies based on feedback, and overcome obstacles collectively. This collaborative approach boosts scalability, adaptability, and and resilience, and improves performance in complex challenges.

% This paper presents a comprehensive comparison of D-CIPHER with existing LLM-based CTF-solving literature, as shown in Table~\ref{tab:related_work_comparison}.
% This paper documents the results of  our comprehensive comparison of D-CIPHER with existing LLM-based CTF-solving literature. These results are presented in Table~\ref{tab:related_work_comparison}.

\vspace{-5pt}
\section{Method}
\label{sec:method}
\begin{figure*}[t]
\begin{center}
\includegraphics[width=.85\linewidth]{fig_overview_v3.pdf}
\end{center}
\caption{
FastAtlas Overview: In each frame, we compute charts spanning fully or partially visible triangles (a), determine texture space bounding boxes for the visible portions of the view-space projections of each chart, and tightly pack these boxes into atlases (b, here $2K \times 2K$). We simultaneously bijectively parameterize and shade the charts into their atlas boxes, obtaining high quality texture space shading (c), and use this shading to render the shaded frames (d).}
\label{fig:overview}
\label{fig:alg_overview}
\end{figure*}

\section{Overview}
\label{sec:overview}
Our work has two core contributions: a real-time, GPU-based algorithm for tight packing of general parameterized charts into compact atlases; and a real-time TSS method that
utilizes this packing.  

\paragraph*{FastAtlas Packing.}
FastAtlas runs entirely on the GPU as a series of compute shaders. It takes the bounding boxes of parameterized charts as input, and packs them into an atlas (Fig~\ref{fig:overview}b, Sec.~\ref{sec:pack}). As such, the only input it requires are the dimensions of the bounding boxes.
Its outputs are deterministic; identical input charts are packed into identical atlases. This is critical for TSS and similar applications, as it ensures that consecutive frames taken from the same camera view have the same shading. Even minute shading differences across such frames can cause sampling jitter, leading to undesirable flicker \cite{baker2012rock}. 
While prior methods such as \cite{mueller2018shading,hladky2019tessellated,hladky2021snakebinning,Neff2022MSA} cap the dimensions of the charts that can be packed as-is for a given atlas size, and scale down all charts that exceed these dimensions, we scale all charts by the same factor, and do so only when strictly necessary to achieve packing success (Figs~\ref{fig:atlas},~\ref{fig:sas_issues}). 

\paragraph*{TSS using FastAtlas.}
Our end-to-end TSS atlas generation method combines the packing method above with a novel approach for computing seamless per-frame charts. 
We define our charts as the connected components of the visible surfaces in each frame (Fig.~\ref{fig:overview}a), and efficiently compute them using a parallel union-find algorithm (Sec.~\ref{sec:visible}). Since the boundaries of these charts coincide with the contours of the rendered surface, they are {\em invisible} to the viewer. This approach 
eliminates the artifacts caused by shading discontinuities along visible seams (Fig.~\ref{fig:seams}). 

\begin{parWithWrapFigure}
\begin{wrapfigure}{l}{.27\columnwidth}%
\includegraphics[width=\linewidth]{fig_inset_view_plane.pdf}%
\end{wrapfigure}
We bijectively parametrize the {\em visible portions} of our charts by projecting them to view space (inset). This maps a constant number of texels to each pixel in the final rendered output, evenly distributing residual undersampling error across all image pixels. While conceptually straightforward, efficiently parameterizing charts containing partially visible triangles using viewspace projection is non-trivial, as the visible portions may no longer be triangular (e.g. green triangle in the inset); applying naive projection to triangles with vertices behind the camera may produce ill-posed results. Clipping triangles before projection is both computationally expensive and significantly complicates downstream operations. We avoid explicit clipping by observing that all that is required for atlas packing is the dimensions of, potentially conservative, bounding boxes of these projected visible portions. We compute such bounding boxes without explicit chart clipping by adapting a conservative screen coverage estimator \shortcite{Blinn:CalculatingScreenCoverage} (Sec.~\ref{sec:box}). We then pack the computed boxes using FastAtlas. 
\end{parWithWrapFigure}

Finally, we shade the visible portion of each chart into its corresponding atlas bounding box (Fig~\ref{fig:overview}c). 
The resulting texture is then used during rasterization as a standard texture map (Fig. ~\ref{fig:overview}d). 
Our framework is compatible with all existing approaches for texture space shading, including forward shading, raytraced illumination, or deferred shading in texture space \cite{baker:2016}. In the examples shown, we use the standard forward shading based rendering pipeline included in the G3D Innovation Engine \cite{G3D17}, a commercial grade renderer.


Our goal is to increase the robustness of T2I models, particularly with rare or unseen concepts, which they struggle to generate. To do so, we investigate a retrieval-augmented generation approach, through which we dynamically select images that can provide the model with missing visual cues. Importantly, we focus on models that were not trained for RAG, and show that existing image conditioning tools can be leveraged to support RAG post-hoc.
As depicted in \cref{fig:overview}, given a text prompt and a T2I generative model, we start by generating an image with the given prompt. Then, we query a VLM with the image, and ask it to decide if the image matches the prompt. If it does not, we aim to retrieve images representing the concepts that are missing from the image, and provide them as additional context to the model to guide it toward better alignment with the prompt.
In the following sections, we describe our method by answering key questions:
(1) How do we know which images to retrieve? 
(2) How can we retrieve the required images? 
and (3) How can we use the retrieved images for unknown concept generation?
By answering these questions, we achieve our goal of generating new concepts that the model struggles to generate on its own.

\vspace{-3pt}
\subsection{Which images to retrieve?}
The amount of images we can pass to a model is limited, hence we need to decide which images to pass as references to guide the generation of a base model. As T2I models are already capable of generating many concepts successfully, an efficient strategy would be passing only concepts they struggle to generate as references, and not all the concepts in a prompt.
To find the challenging concepts,
we utilize a VLM and apply a step-by-step method, as depicted in the bottom part of \cref{fig:overview}. First, we generate an initial image with a T2I model. Then, we provide the VLM with the initial prompt and image, and ask it if they match. If not, we ask the VLM to identify missing concepts and
focus on content and style, since these are easy to convey through visual cues.
As demonstrated in \cref{tab:ablations}, empirical experiments show that image retrieval from detailed image captions yields better results than retrieval from brief, generic concept descriptions.
Therefore, after identifying the missing concepts, we ask the VLM to suggest detailed image captions for images that describe each of the concepts. 

\vspace{-4pt}
\subsubsection{Error Handling}
\label{subsec:err_hand}

The VLM may sometimes fail to identify the missing concepts in an image, and will respond that it is ``unable to respond''. In these rare cases, we allow up to 3 query repetitions, while increasing the query temperature in each repetition. Increasing the temperature allows for more diverse responses by encouraging the model to sample less probable words.
In most cases, using our suggested step-by-step method yields better results than retrieving images directly from the given prompt (see 
\cref{subsec:ablations}).
However, if the VLM still fails to identify the missing concepts after multiple attempts, we fall back to retrieving images directly from the prompt, as it usually means the VLM does not know what is the meaning of the prompt.

The used prompts can be found in \cref{app:prompts}.
Next, we turn to retrieve images based on the acquired image captions.

\vspace{-3pt}
\subsection{How to retrieve the required images?}

Given $n$ image captions, our goal is to retrieve the images that are most similar to these captions from a dataset. 
To retrieve images matching a given image caption, we compare the caption to all the images in the dataset using a text-image similarity metric and retrieve the top $k$ most similar images.
Text-to-image retrieval is an active research field~\cite{radford2021learning, zhai2023sigmoid, ray2024cola, vendrowinquire}, where no single method is perfect.
Retrieval is especially hard when the dataset does not contain an exact match to the query \cite{biswas2024efficient} or when the task is fine-grained retrieval, that depends on subtle details~\cite{wei2022fine}.
Hence, a common retrieval workflow is to first retrieve image candidates using pre-computed embeddings, and then re-rank the retrieved candidates using a different, often more expensive but accurate, method \cite{vendrowinquire}.
Following this workflow, we experimented with cosine similarity over different embeddings, and with multiple re-ranking methods of reference candidates.
Although re-ranking sometimes yields better results compared to simply using cosine similarity between CLIP~\cite{radford2021learning} embeddings, the difference was not significant in most of our experiments. Therefore, for simplicity, we use cosine similarity between CLIP embeddings as our similarity metric (see \cref{tab:sim_metrics}, \cref{subsec:ablations} for more details about our experiments with different similarity metrics).

\vspace{-3pt}
\subsection{How to use the retrieved images?}
Putting it all together, after retrieving relevant images, all that is left to do is to use them as context so they are beneficial for the model.
We experimented with two types of models; models that are trained to receive images as input in addition to text and have ICL capabilities (e.g., OmniGen~\cite{xiao2024omnigen}), and T2I models augmented with an image encoder in post-training (e.g., SDXL~\cite{podellsdxl} with IP-adapter~\cite{ye2023ip}).
As the first model type has ICL capabilities, we can supply the retrieved images as examples that it can learn from, by adjusting the original prompt.
Although the second model type lacks true ICL capabilities, it offers image-based control functionalities, which we can leverage for applying RAG over it with our method.
Hence, for both model types, we augment the input prompt to contain a reference of the retrieved images as examples.
Formally, given a prompt $p$, $n$ concepts, and $k$ compatible images for each concept, we use the following template to create a new prompt:
``According to these examples of 
$\mathord{<}c_1\mathord{>:<}img_{1,1}\mathord{>}, ... , \mathord{<}img_{1,k}\mathord{>}, ... , \mathord{<}c_n\mathord{>:<}img_{n,1}\mathord{>}, ... , $
$\mathord{<}img_{n,k}\mathord{>}$,
generate $\mathord{<}p\mathord{>}$'', 
where $c_i$ for $i\in{[1,n]}$ is a compatible image caption of the image $\mathord{<}img_{i,j}\mathord{>},  j\in{[1,k]}$. 

This prompt allows models to learn missing concepts from the images, guiding them to generate the required result. 

\textbf{Personalized Generation}: 
For models that support multiple input images, we can apply our method for personalized generation as well, to generate rare concept combinations with personal concepts. In this case, we use one image for personal content, and 1+ other reference images for missing concepts. For example, given an image of a specific cat, we can generate diverse images of it, ranging from a mug featuring the cat to a lego of it or atypical situations like the cat writing code or teaching a classroom of dogs (\cref{fig:personalization}).
\vspace{-2pt}
\begin{figure}[htp]
  \centering
   \includegraphics[width=\linewidth]{Assets/personalization.pdf}
   \caption{\textbf{Personalized generation example.}
   \emph{ImageRAG} can work in parallel with personalization methods and enhance their capabilities. For example, although OmniGen can generate images of a subject based on an image, it struggles to generate some concepts. Using references retrieved by our method, it can generate the required result.
}
   \label{fig:personalization}\vspace{-10pt}
\end{figure}

\vspace{-0.2cm}
\section{Results}\label{sec:results}




\subsection{Benchmark quality after watermarking}\label{subsec:results_rephrasing}


\paragraph{\textbf{Set-up.}}
For the watermark embedding, we rephrase with Llama-3.1-8B-Instruct~\citep{dubey2024llama} by default, with top-p sampling with p = $0.7$ and temperature = $0.5$ (default values on the Hugging Face hub), and the green/red watermarking scheme of \citet{kirchenbauer2023reliability} with a watermark window $k=2$ and a ``green list'' of
size $\frac{1}{2}|V|$ ($|V|$ is the vocabulary size).
We compare different values of $\delta$ when rephrasing: 0 (no watermarking), 1, 2, and 4.
We choose to watermark ARC-Challenge, ARC-Easy, and MMLU due to their widespread use in model evaluation.
In practice, one would need to watermark their own benchmark before release.
For MMLU, we select a subset of 5000 questions, randomly chosen across all disciplines, to accelerate experimentation and maintain a comparable size to the other benchmarks.
We refer to this subset as MMLU$^*$.
ARC-Easy contains 1172 questions, and ARC-Challenge contains 2372 questions.
In~\autoref{fig:example_answers_big} of \autoref{app:appendix}, we show the exact instructions given to the rephrasing model (identical for all benchmarks) and the results for different watermarking strengths on one example from ARC-Easy.
\emph{We use a different watermarking key $\sk$ for each benchmark.}

% Thanks to the hashing function used, the corresponding green lists and red lists for each benchmark are independent: there is no more collision between the benchmarks than there is between natural text and the benchmarks.

\paragraph{\textbf{Even strong watermarking keeps benchmark utility.}} 
We evaluate the performance of Llama-3.3-1B, Llama-3.3-3B and Llama-3.1-8B on the original benchmark and the rephrased version using as similar evaluation as the one from the \texttt{lm-evaluation-harness} library~\citep{eval-harness}.
To check if the benchmark is still as meaningful, we check that evaluated models obtain a similar accuracy on the watermarked benchmarks and on the original version (see~\autoref{subsec:rephrasing}).
\autoref{fig:results_overview_arc_easy_perfs} shows the performance on ARC-Easy.
All models perform very similarly on all the rephrased versions of the benchmark, even when pushing the watermark to $80\%$ of green tokens.
Importantly, they rank the same.
Similar results are shown for MMLU$^*$ and ARC-Challenge in \autoref{fig:results_overview_arc_easy_perfs} of \autoref{app:appendix}, although for MMLU$^*$, we observe some discrepancies. 
For instance, when using a watermarking window size of 2 (subfig i), the performance of Llama-3.2-1B increases from 38$\%$ to $42\%$ between the original and the other versions. 
However we observe the same issue when rephrasing without watermarking in that case.
As detailed in \autoref{subsec:rephrasing}, designing better instructions that are more specific to each benchmark could help.
We have tried increasing $\delta$ even further, but it broke the decoding process. 
The choice of $\delta$ depends on the benchmark and the model used for rephrasing, and needs to be empirically tested.



\begin{figure}[b!] % 't' places the figure at the top of the page
    \centering
    \begin{minipage}{0.49\textwidth}
        \centering
        \includegraphics[width=1.0\textwidth, clip, trim=0 0cm 0 0]{figs/main/k2/arc-easy_delta_barplot.pdf}
        \subcaption{Watermarking questions does not degrade utility.}
        \label{fig:results_overview_arc_easy_perfs}
    \end{minipage}\hfill
    \begin{minipage}{0.49\textwidth}
        \centering
        \includegraphics[width=1.0\textwidth, clip, trim=0 0cm 0 0]{figs/main/k2/contamination_35317.pdf}
        \subcaption{More contaminations \& stronger wm $\uparrow$ detection.}
        \label{fig:results_overview_arc_easy_detection}
    \end{minipage}
    \caption{
    Result for benchmark watermarking on ARC-Easy. %Watermarking the questions does not degrade its utility, and the more watermarked the benchmark, the easier it is to detect radioactivity. 
    (Left) We rephrase the questions from ARC-Easy using Llama-3.1-8B-Instruct while adding watermarks of varying strength. 
    The performance of multiple Llama-3 models on rephrased ARC-Easy is comparable to the original, preserving the benchmark's usefulness for ranking models and assessing accuracy (Sec.~\ref{subsec:rephrasing}, Sec.~\ref{subsec:results_rephrasing}). (Right) We train 1B models from scratch on 10B tokens while intentionally contaminating its training set with the watermarked benchmark dataset. 
    Increasing the number of contaminations and watermark strength enhances detection confidence (Sec.~\ref{subsec:detection}, Sec.~\ref{subsec:result_detection})}
    \vspace{-0.3cm}\label{fig:results_overview_arc_easy}
\end{figure}

\subsection{Contamination detection through radioactivity}\label{subsec:result_detection}

We now propose an experimental design to control benchmark contamination, and evaluate both the impact on model performance and on contamination detection.

\paragraph{\textbf{Training set-up.}}
We train 1B transformer models~\citep{vaswani2017attention} using \texttt{Meta Lingua}~\citep{meta_lingua} on 10B tokens from DCLM~\citep{li2024datacomp}. 
The model architecture includes a hidden dimension of 2048, 25 layers, and 16 attention heads.
The training process consists of 10,000 steps, using a batch size of 4 and a sequence length of 4096. 
Each training is distributed across 64 A-100 GPUs, and takes approximately three hours to finish.
The optimization is performed with a learning rate of $3 \times 10^{-3}$, a weight decay of $0.033$, and a warmup period of 5,000 steps. 
The learning rate is decayed to a minimum ratio of $10^{-6}$, and gradient clipping is applied with a threshold of 1.0.

\paragraph{\textbf{Contamination set-up.}}
Between steps 2500 and 7500, every $5000/\#\text{contaminations}$, we take a batch from the shuffled concatenation of the three benchmarks instead of the batch from DCLM.
Each batch has
\(
\text{batch size} \times \text{sequence length} \times \text{number of GPUs} = 4 \times 4096 \times 64 \approx 1\,\text{M tokens}
\)
As shown in \autoref{tab:contamination}, the concatenation of the three benchmarks is approximately $500$k tokens, so each contamination is a gradient that encompasses all the benchmark's tokens.
For each benchmark, any sample that ends up contaminating the model is formatted as follows:

\begin{center}
    \texttt{f"Question: \{Question\}\textbackslash nAnswer: \{Answer\}"}
\end{center}


% \paragraph{Impact of the number of contaminations on the accuracy on the benchmark.} 
\paragraph{\textbf{Evaluation.}}
We evaluate the accuracy of the models on the benchmarks by comparing the loss between the different choices and choosing the one with the smallest loss,  either ``in distribution'' by using the above template seen during contamination or ``out of distribution'' (OOD) by using:

\begin{center}
    \texttt{f"During a lecture, the professor posed a question: \{Question\}. \\ After discussion, it was revealed that the answer is: \{Answer\}"}
\end{center}

In the first scenario, we evaluate overfitting, as the model is explicitly trained to minimize the loss of the correct answer within the same context. 
In the second scenario, we assess the model's ability to confidently provide the answer in a slightly different context, which is more relevant for measuring contamination.
Indeed, it's important to note that evaluations often use templates around questions ---\eg in the \texttt{lm-evaluation-harness} library~\citep{eval-harness}--- which may not be part of the question/answer files that could have leaked into the pre-training data.
% Moreover, if contamination comes from a leak of a jsonl that contains
\autoref{tab:contamination} focuses on $\delta=4$ and shows the increase in performance across the three (watermarked) benchmarks as a function of the number of contaminations when evaluated OOD. 
Results for in-distribution evaluation are provided in \autoref{tab:contamination_indist} of \autoref{app:appendix} (without contamination, the model performs similarly across the two templates).


\paragraph{\textbf{Contamination detection.}}
For each benchmark, we employ the reading mode detailed in~\autoref{subsec:detection} to compute the radioactivity score $S$ and the corresponding $\pval$.
% We perform the reading mode on the same watermarked benchmark watermarked benchmark.
Results are illustrated in~\autoref{fig:results_overview_arc_easy_detection} for ARC-Easy, and in~\autoref{fig:appendix_watermark_contamination} of \autoref{app:appendix} for the other two benchmarks, across different numbers of contaminations and varying watermark strengths $\delta$.
We observe that the stronger the watermark strength and the greater the number of contaminations, the easier it is to detect contamination: a larger negative $\logpval$ value indicates smaller $\pval$s, implying a lower probability of obtaining this score if the model is not contaminated.
For instance, a $-\logpval$ of $6$ implies that we can confidently assert model contamination, with only a $10^{-6}$ probability of error.
% , which is the case when $5$ points are artificially added on MMLU$^*$ in~\autoref{tab:contamination}.
Additionally, we observe that without contamination, the test yields a $\logpval$ value close to $-0.3 = \log_{10}(0.5) $, as expected under $\mathcal{H}_0$.
Indeed, under $\mathcal{H}_0$, the $\pval$ should follow a uniform distribution between 0 and 1, which implies that [-1, 0] is a 90$\%$ confidence interval for $\logpval$, and that [-2, 0] is a 99$\%$ confidence interval.

\autoref{tab:contamination} links the contamination detection to the actual cheating (with OOD evaluation) on the benchmarks when $\delta=4$ is used.
We can see that for the three benchmarks, whenever the cheat is greater than $10\%$, detection is extremely confident.
When the cheat is smaller, with four contaminations ranging from $+3\%$ to $+5\%$, the $\pval$ is small enough on ARC-Easy and MMLU$^*$, but doubtful for ARC-Challenge (because smaller, see \autoref{subsec:additional_results}).
For instance, for MMLU$^*$, we can assert model contamination, with only a $10^{-6}$ probability of error when $5$ points are artificially added.




% \begin{table}[t!]
%     \centering
%     \vspace{-0.2cm}
%     \caption{
%         Detection and performance metrics across different levels of contamination for ARC-Easy, ARC-Challenge, and MMLU benchmarks, watermarked with $\delta=4$.
%         The performance increase is shown for OOD evaluation as detailed in~\autoref{subsec:result_detection}. 
%         Similar results for in distribution are shown in \autoref{tab:contamination_indist} of~\autoref{app:appendix}
%     }\label{tab:contamination}
%     \begin{tabular}{r r r r r r r}
%         \toprule
%         & \multicolumn{2}{c}{ARC-Easy (112k toks.)} & \multicolumn{2}{c}{ARC-Challenge (64k toks.)} & \multicolumn{2}{c}{MMLU$^*$ (325k toks.)} \\
%         \cmidrule(lr){2-3} \cmidrule(lr){4-5} \cmidrule(lr){6-7}
%         Cont & \multicolumn{1}{r}{log10 p-val} & \multicolumn{1}{r}{Perf (\% $\Delta$)} & \multicolumn{1}{r}{log10 p-val} & \multicolumn{1}{r}{Perf (\% $\Delta$)} & \multicolumn{1}{r}{log10 p-val} & \multicolumn{1}{r}{Perf (\% $\Delta$)} \\
%         \midrule
%         0  & -0.3 & 53.5 (+0) & -0.3 & 29.4 (+0) & -0.9 & 30.6 (+0) \\
%         4  & -3.0 & 57.9 (+4.3) & -1.2 & 32.4 (+3.1) & -5.7 & 35.7 (+5.1) \\
%         8  & -5.5 & 63.0 (+9.5) & -4.5 & 39.3 (+9.9) & \textless{-12} & 40.8 (+10.2) \\
%         16 & \textless{-12} & 71.7 (+18.2) & \textless{-12} & 54.3 (+24.9) & \textless{-12} & 54.0 (+23.5) \\
%         \bottomrule
%     \end{tabular}
%     \vspace{-0.3cm}
% \end{table}

% \newcommand{\graydelta}[1]{\textcolor{gray}{\footnotesize (#1)}}
\begin{table}[t!]
    \centering
    \vspace{-0.2cm}
    \caption{
        Detection and performance metrics across different levels of contamination for ARC-Easy, ARC-Challenge, and MMLU benchmarks, watermarked with $\delta=4$.
        The performance increase is shown for OOD evaluation as detailed in~\autoref{subsec:result_detection}. 
        The log$_{10}$ $\pval$ of the detection test is strongly correlated with the number of contaminations, as well as with the performance increase of the LLM on the benchmark.
        % Similar results for in distribution are shown in \autoref{tab:contamination_indist} of~\autoref{app:appendix} \pierre{not necessary in the fig.}
    }\label{tab:contamination}
    \resizebox{\textwidth}{!}{
    \begin{tabular}{r rr@{\hspace{0.5em}}l rr@{\hspace{0.5em}}l rr@{\hspace{0.5em}}l}
        \toprule
        & \multicolumn{3}{c}{ARC-Easy (112k toks.)} & \multicolumn{3}{c}{ARC-Challenge (64k toks.)} & \multicolumn{3}{c}{MMLU$^*$ (325k toks.)} \\
        \cmidrule(lr){2-4} \cmidrule(lr){5-7} \cmidrule(lr){8-10}
        Contaminations & $\logpval$ & Acc. & \graydelta{\% $\Delta$} & $\logpval$ & Acc. & \graydelta{\% $\Delta$} & $\logpval$ & Acc.& \graydelta{\% $\Delta$} \\
        \midrule
        0  & -0.3 & 53.5 & \graydelta{+0.0} & -0.3 & 29.4 & \graydelta{+0.0} & -0.9 & 30.6 & \graydelta{+0.0} \\
        4  & -3.0 & 57.9 & \graydelta{+4.3} & -1.2 & 32.4 & \graydelta{+3.1} & -5.7 & 35.7 & \graydelta{+5.1} \\
        8  & -5.5 & 63.0 & \graydelta{+9.5} & -4.5 & 39.3 & \graydelta{+9.9} & \textless{-12} & 40.8 & \graydelta{+10.2} \\
        16 & \textless{-12} & 71.7 & \graydelta{+18.2} & \textless{-12} & 54.3 & \graydelta{+24.9} & \textless{-12} & 54.0 & \graydelta{+23.5} \\
        \bottomrule
    \end{tabular}
    }
    \vspace{-0.3cm}
\end{table}

\vspace{-0.2cm}
\subsection{Additional Results}\label{subsec:additional_results}


\paragraph{\textbf{Impact of window size.}}
\begin{wraptable}{r}{0.4\textwidth}
    \centering
    \vspace{-0.4cm}
    \caption{\small Proportion of green tokens in the predictions (see~\autoref{eq:def_S_N}), number of tokens scored after dedup and log$_{10}$ $\pval$s for different watermark window sizes, with 16 contaminations and $\delta=4$ on ARC-Easy.}
    \small % Reduce font size for the table
    \begin{tabular}{r r r r}
        \toprule
        $k$ & \multicolumn{1}{c}{$\rho$} & \multicolumn{1}{r}{Tokens} & \multicolumn{1}{r}{$\logpval$} \\
        \midrule
        0 & 0.53 & 5k & -6.07 \\
        1 & 0.53 & 28k & -25.89 \\
        2 & 0.53 & 47k & -38.69 \\
        \bottomrule
    \end{tabular}
    \vspace{-0.2cm}
    \label{tab:window_size}
\end{wraptable}
Watermark insertion through rephrasing (\autoref{subsec:rephrasing}) depends on the watermark window size $k$. 
Each window creates a unique green-list/red-list split for the next token. 
Larger windows reduce repeated biases but are less robust.
Because of repetitions, \citet{sander2024watermarking} show that smaller windows can lead to bigger overfitting on token-level watermark biases, aiding radioactivity detection.
In our case, benchmark sizes are relatively small and deduplication limits the number of tokens tested, because each $\{$window + predicted token$\}$ is scored only once. 
Thus, smaller windows mean fewer tokens to score.
Moreoever, as shown in~\autoref{tab:window_size}, the proportion of predicted green tokens is not even larger for smaller windows: there is not enough repetitions for increased over-fitting on smaller windows.
The two factors combined result in lower confidence. 
A comparison of contamination detection across benchmarks and window sizes is shown in \autoref{fig:appendix_watermark_performance}, and for the utility of the benchmarks in~\autoref{fig:appendix_watermark_contamination}.
Overall, larger window size ($k=2$) yields better results.

\vspace{-0.1cm}
\paragraph{\textbf{Impact of benchmark size.}} The benchmark size can significantly affect the method's effectiveness.
With a fixed proportion of predicted green tokens, more evidence (\ie more scored tokens) increases test confidence. 
As shown in~\autoref{tab:contamination}, at a fixed level of cheating (\eg $+10\%$ on all benchmarks after $8$ contaminations), contamination detection confidence is proportional to benchmark size.
This is similar to our observations on watermark window sizes in~\autoref{tab:window_size}.
% So at fixed cheating level, it will be easier to detect contamination of bigger benchmarks.




% In classical watermarking, however, a larger watermark window means smaller robustness as changing one every $k$ tokens on average can break all the watermark.

% But in our case, we are going the do the radioactivity detection test on the dataset without any changes, but we may want more robustness if the suspect model tries to break the watermark before training on it.


% \paragraph{Impact of rephrasing model.}
% The difficulty of the questions, and their entropy, can have an important impact on the method.
% For instance, some math questions are hard to rephrase, and adding a watermark can further mess-up the meaning. 
% The method may thus require a stronger model for highly technical benchmarks (\eg Llama3-70B instead of Llama3-8B).
% Moreover, typically for math or code, the rephrasing inherently does not let a lot of entropy, as many invariants need to be respected.
% Possibilities would be to add watermarked verbose text around the math instead of rephrasing, and use as entropy-aware LLM watermarking~\citep{lee2023wrote}.
% We have tested rephrasing the benchmarks using Llama3-70B-Instruct instead of the 8B version. 
% We observe that we need to increase $\delta$ to $8$ in in order to obtain the same proportion of green tokens as with $\delta=2$ with the 8B model, while using the exact same decoding parameters.
% This can be because there is less entropy in the generation of the 70B or that the logits are for some reasons bigger, as the bias towards the greenlist is added before the softmax (see~\autoref{subsec:rephrasing}).
% However, we observe that some failure cases with the 8B (specifically for questions with important numbers) are correct with the 70B, but this is hard to quantify. 
% We give one example bellow in~\autoref{fig:example_answers_70B}.
\vspace{-0.1cm}
\paragraph{\textbf{Impact of rephrasing model.}}
The difficulty and entropy of questions can significantly affect the method's performance. 
Indeed, math questions for instance can be challenging to rephrase, even more with watermarks. 
Thus, better models may be needed for technical benchmarks.
We tested rephrasing with Llama3-70B-Instruct instead of the 8B version, and  observed that some 8B model failures, especially on math questions, are resolved with the 70B model, though quantifying this is challenging. 
An example is provided in~\autoref{fig:example_answers_70B}.
We note that increasing $\delta$ to 8 is necessary to match the green token proportion of $\delta=2$ with the 8B model, using the same decoding parameters.
This may result from lower entropy in generation or bigger logits, as the greenlist bias is applied before the softmax (see~\autoref{subsec:rephrasing}).
Moreover, in math or code, rephrasing can offer limited entropy, and even better models will not be enough.
An alternative would be to add watermarked verbose text \emph{around} the questions, or using entropy-aware LLM watermarking~\citep{lee2023wrote}.

\begin{figure}[b!]
    \vspace{-0.3cm}
    \centering
    \begin{tcolorbox}[colframe=metablue, colback=white]
        \footnotesize
        \textbf{Original question:} 
        An object accelerates at 3 meters per second$^2$ when a 10-newton (N) force is applied to it. Which force would cause this object to accelerate at 6 meters per second$^2$?
        \begin{minipage}{0.42\textwidth}
            \vspace{0.1cm}
            \textbf{Llama-3-8B-Instruct, $\delta=2$:} What additional force, applied in conjunction with the existing 10-N force, would cause the object to experience an acceleration of 6 meters per second$^2$? (70$\%$)
        \end{minipage}\hspace{0.04\textwidth}%
        \begin{minipage}{0.54\textwidth}
            \vspace{0.1cm}
            \textbf{Llama-3-70B-Instruct, $\delta=8$:} What force would be necessary to apply to the object in order to increase its acceleration to 6 meters per second$^2$, given that an acceleration of 3 meters per second$^2$is achieved with a 10-newton force? (65$\%$)
        \end{minipage}
    \end{tcolorbox}
    \vspace{-0.2cm}
    \caption{
    Watermarking failure on an ARC-Challenge question with an $8$B model, while the $70$B model succeeds.
    }
    
\label{fig:example_answers_70B}
\end{figure}



\begin{wrapfigure}{r}{0.5\textwidth}
  \centering
  \vspace{-0.5cm}
\includegraphics[width=0.48\textwidth]{figs/main/detection_vs_performance.pdf} % Replace with your image file
  \vspace{-0.25cm}
  \caption{Detection confidence as a function of performance increase on MMLU$^*$ for different model sizes and \#contaminations, for $\delta=4$ and OOD evaluation.}
  \vspace{-0.35cm}
\end{wrapfigure}\label{fig:model_size}
\paragraph{\textbf{Impact of model size.}}
We also test radioactivity detection on 135M and 360M transformer models using the architectures of~\href{https://github.com/huggingface/smollm}{\texttt{SmolLM}} and the same training pipeline as described in \autoref{subsec:result_detection}, training each model on 10B tokens as well. 
\autoref{fig:model_size} shows the detection confidence as a function of the cheat on MMLU$^*$.
We find that, for a fixed number of contaminations, smaller models show less performance increase -- expected as they memorize less -- and we obtain lower confidence in the contamination detection test. 
As detailed in~\autoref{subsec:rephrasing}, the $\pval$s indicate how well a model overfits the questions, hence the expected correlation. For a fixed performance gain on benchmarks, $\pval$s are consistent across models. For example, after $4$, $8$, and $16$ contaminations on the $1$B, $360$M, and $135$M parameter models respectively, all models show around $+6$\% gain, with detection tests yielding $\pval$s around $10^{-5}$.
Thus, while larger models require fewer contaminated batches to achieve the same gain on the benchmark, radioactivity effectively measures ``cheating''.





% \begin{wrapfigure}{r}{0.45\textwidth}
%   \centering
%   \vspace{-0.4cm}
%   \includegraphics[width=0.43\textwidth]{figs/main/arc-easy.pdf}
%   \vspace{-0.3cm}
%   \captionsetup{font=small}
%   \caption{Performance of Llama-3 models on different versions of the arc-easy benchmark.}
%   \vspace{-1cm}
%   \label{fig:impact-wm-arc-easy}
% \end{wrapfigure}



\section{Discussions}

\subsection{Transparency in Ride-Sharing Platform Algorithms}
The publicly available Chicago Transportation Network Provider dataset helped us answer many research questions, but ride-sharing platforms still make many of their mechanisms opaque. The lack of transparency in key platform mechanisms---such as pricing models, driver--rider matching algorithms, and driver ranking systems---makes it difficult to pinpoint the exact causes of these disparities. Without greater visibility into these proprietary algorithms, drivers also remain at an information disadvantage, unable to anticipate fare fluctuations or optimize their work schedules effectively.

Pricing models remain opaque, with our analysis revealing that fare adjustments over time have failed to keep pace with inflation, effectively reducing real driver earnings (\cref{sec:results-pricing-stablization}). While platforms advertise dynamic pricing mechanisms that respond to demand surges, drivers have limited insight into how much of the fare they actually receive after platform fees~\cite{santos2020dynamic}. Previous research has shown that drivers tend to work more during peaks for higher compensation~\cite{chen2016dynamic}. A real-time, large-scale understanding of the surge pricing model can help drivers become more informed in planning and organizing their workday, beyond anecdotal observations. Furthermore, researchers can provide prediction models of price surges, helping both drivers and riders adjust plans accordingly. Another key limitation of using the Chicago dataset is the lack of driver earning information. As a result, our analysis can only use the trip fare as a proxy for driver earning. Making such information available can significantly increase transparency into platform operations.

Similarly, the driver-rider matching algorithm remains a black box. Our inferred driver profiles suggest that trip assignments may systematically disadvantage certain groups, particularly those operating in lower-income areas. If the matching algorithm disproportionately favors drivers in high-demand or high-fare regions, it could reinforce existing geographic disparities in earnings. However, such analysis is hard to conduct without access to driver-level information. As discussed in \cref{sec:methods-driver-simulation}, releasing such data may lead to privacy concerns. Our approach is an effort to approximate driver working conditions without needing detailed driver data. However, researchers should still work with ride-sharing platforms to come up with privacy-preserving ways to analyze such data for insights. Also, driver ranking algorithms---which determine access to high-value trips---are equally opaque. While platforms often cite factors such as acceptance rate, customer ratings, and trip history, the lack of public accountability raises concerns regarding potential biases. Accessing such information can support researchers in identifying potential biases, also help drivers provide more desired services to riders.

In all, we call for increased regulatory oversight and platform-level efforts to improve algorithmic transparency. Without clear disclosures on how these systems operate, ride-sharing drivers remain vulnerable to unfair decision-making and fluctuating incomes that they cannot predict or control.

\subsection{Data Analysis Methodology Improvements}
Our study demonstrates the feasibility of simulating reasonable driver profiles from trip-level data, even in the absence of driver-related information. By leveraging a simulation-based approach, we were able to approximate driver earnings, work patterns, and geographic activity. However, there are still areas for improvement for our methodology.

First, a robust evaluation benchmark is needed to validate the accuracy of inferred driver profiles. While our approach provides valuable insights and matches previous empirical findings, the lack of direct ground truth data means we rely on approximations. We need alternative data sources to cross-verify our inferred driver activities. Tools for driver task management, such as Driver's Seat~\cite{calacci2023access}, asks drivers to upload their work tasks and can serve as a potential data source. More autonomous approaches that uses UI understanding techniques and directly collects data from drivers' phones can also scale up this effort~\cite{lu2024crepe}. 

Moreover, expanding the scope of inferred information would provide deeper insights into platform operations. Currently, we infer earnings and work patterns for drivers. Newer algorithms can be developed to analyze additional opaque platform mechanisms as discussed above. Future studies could aim to reconstruct other aspects of opaque platform algorithms, as discussed above, directly from publicly available, large-scale datasets.

Given a large-scale dataset that misses key information aspects, a potential future approach is to self-collect a smaller dataset that contains the necessary details and conduct a joint analysis of both datasets. For example, a smaller dataset that we collect directly from drivers, containing both driver and trip information, can serve both as a benchmark and a basis for use to train machine learning models that predict driver profiles from existing large-scale datasets. Future research can investigate effective measures to combine these different data sources~\cite{harris2018federal} for joint analysis. These methodological advancements can help us to use large-scale ridesharing datasets more effectively and accurately while maintaining driver and rider privacy.


\subsection{Societal Implications: Ride-Sharing as a Reflection of Broader Inequalities}

Our findings revealed regional ride-sharing disparities in the city of Chicago, which largely reflect the broader existing societal inequalities. Drivers working in lower-income neighborhoods---in our case, drivers that service the southern regions of Chicago---consistently earn less, even despite longer work hours. Structural disadvantages, such as lower infrastructure quality, longer wait times, and increased safety concerns---compound the challenges faced by gig workers. Chicago South Side, as a community suffering from violence and poverty, has been an example of social segregation and studied by numerous researchers~\cite{moore2016south, bachin2004building, bell1993community}. As an aspect of a deep-rooted societal issue, ride-sharing inequality in lower-income neighborhoods calls for holistic policymaking efforts from multiple stakeholders.

Our findings provide practical implications for labor activists and policy makers. By providing a more transparent view of drivers’ potential workday experiences, policymakers can better evaluate the labor conditions these platforms create, ensuring that emerging mobility systems align with equity goals. Urban planners and regulators can use these insights to inform policy interventions---such as driver support programs, driver caps, or incentive structures---that promote fairness and mitigate algorithmic biases. Similarly, platform operators themselves might harness these findings to improve their matching algorithms, advancing a more equitable ecosystem that benefits both drivers and passengers.

Research has shown that transportation access can have a positive impact on regional economic growth and productivity~\cite{targa2005economic, banerjee2020road, alstadt2012relationship}. Ride-sharing, as an increasingly critical way of transportation, especially where public transportation is scarce, can support individual and community access to growth opportunities. The persistence of regional earning gaps raises important questions about equity in urban transportation. If ride-sharing platforms are designed primarily to maximize efficiency and revenue, they may inadvertently exacerbate existing economic inequalities by steering high-value rides away from underserved areas~\cite{durand2022access, bocarejo2012transport}.

To address these issues, we call for policy interventions aimed at ensuring fair compensation and equitable access to earning opportunities. Regulators should consider implementing transparency mandates, income stability measures, and algorithmic accountability frameworks to prevent platforms from disproportionately disadvantaging certain driver groups. Moreover, these efforts should be in orchestration with existing efforts to promote infrastructural improvements and public safety in underserved regions. Collaborative initiatives between policymakers, ride-sharing companies, and community organizations can help create a more inclusive transportation ecosystem that benefits both drivers and passengers alike~\cite{baber2022new}.

\section{Conclusion}
We introduced \methodname, an effective training framework defending against MIAs for LLMs. The extensive experiments demonstrate its robustness in protecting privacy while maintaining strong language modeling performance across various datasets and architectures. Although our study focuses on fine-tuning due to computational constraints, \methodname can be seamlessly applied to large-scale pretraining, as done in prior selective pretraining work~\cite{lin2024not}. By categorizing tokens and treating them appropriately, \methodname opens a novel pathway for MIA defense. Future work can explore improved token selection strategies and multi-objective training approaches.

\bibliographystyle{ACM-Reference-Format}
\bibliography{ref}

\appendix
\section{Appendix}
\label{appendix}

\subsection{Survey Questions}
\label{app:survey}

\subsubsection{Scenarios}

Participants were asked about three classes of hiring scenarios: technical coding assessments, resume review, and behavioral interviews (the scenarios are listed by class below). For each scenario, they answered two questions, both on 5-point Likert scales:
\begin{itemize}
    \item How fair does this hiring process seem to you? (``This hiring process seems fair'', 1: Strongly disagree to 5: Strongly agree)
    \item If you were applying for a technology job, would you want to be evaluated this way? (``I want to be evaluated this way'', 1: Strongly disagree to 5: Strongly agree)
\end{itemize}

[Technical Coding Assessments]
\begin{enumerate}
\item An applicant submits a sample of code, which is reviewed by a recruitment team, who determines whether the applicant advances to the next phase.
\item An applicant is given an online coding assessment, which is evaluated by an algorithm. If the applicant reaches a certain score on the autograder, the applicant advances to the next phase. All algorithmic decisions are reviewed by a recruitment team.
\item An applicant is given an online coding assessment, which is evaluated by an algorithm. If the algorithm rejects the applicant, the decision is reviewed by a recruitment team. 
\item An applicant is given an online coding assessment, which is evaluated by an algorithm. If the algorithm advances the applicant to the next phase, the decision is reviewed by a recruitment team. 
\item An applicant is given an online coding assessment, which is evaluated by an algorithm that determines whether an applicant advances to the next phase. 
% \item Why did you select the answers above for the different scenarios related to coding assessments?
\end{enumerate}

[Resume Review]
\begin{enumerate}
\item An applicant submits a resume, which is reviewed by a recruitment team, who determines whether the applicant advances to the next phase.
\item An applicant submits a resume, which is evaluated by an algorithm. The algorithm determines whether the applicant advances to the next phase. All algorithmic decisions are reviewed by a recruitment team. 
\item An applicant submits a resume, which is evaluated by an algorithm. If the algorithm rejects your application, the decision is reviewed by a recruitment team. 
\item An applicant submits a resume, which is evaluated by an algorithm. If the algorithm advances the applicant to the next phase, the decision is reviewed by a recruitment team. 
\item An applicant submits a resume, which is evaluated by an algorithm that determines whether an applicant advances to the next phase. 
% \item Why did you select the answers above for the different scenarios related to resumes?
\end{enumerate}

[Behavioral Interviews]
\begin{enumerate}
\item An applicant has an interview with a member of the recruitment team. The recruitment team determines whether the applicant advances to the next phase.
\item An applicant participates in an automated video interview, where the applicant receives interview questions and records video responses. The video, including the applicant’s speech (fluency, prosody, pronunciation, language usage) and nonverbal behaviors (facial expressions, posture, and eye movements), is evaluated by an algorithm. Whether you advance to the next phase is determined by the algorithm. All algorithmic decisions are reviewed by a recruitment team.
\item An applicant participates in an automated video interview, where the applicant receives interview questions and records video responses. The video, including the applicant’s speech (fluency, prosody, pronunciation, language usage) and nonverbal behaviors (facial expressions, posture, and eye movements), is evaluated by an algorithm. If the algorithm rejects the applicant,  the decision is reviewed by a recruitment team. 
\item An applicant participates in an automated video interview, where the applicant receives interview questions and records video responses. The video, including the applicant’s speech (fluency, prosody, pronunciation, language usage) and nonverbal behaviors (facial expressions, posture, and eye movements), is evaluated by an algorithm. If the algorithm advances the applicant to the next phase, the decision is reviewed by a recruitment team. 
\item An applicant participates in an automated video interview, where the applicant receives interview questions and records video responses. The video, including the applicant’s speech (fluency, prosody, pronunciation, language usage) and nonverbal behaviors (facial expressions, posture, and eye movements), is evaluated by an algorithm that determines whether an applicant advances to the next phase.
% \item Why did you select the answers above for the different scenarios related to interviews?
\end{enumerate}

At the end of each set of Likert questions, participants were also asked an open response question (``Why did you select the answers above for the different scenarios related to [coding assessments/resumes/interviews]?'').

\subsubsection{Awareness of AEDTs}

In this section, participants were asked for each hiring process (online coding assessment, automated resume readers, and automated interviews) to check the box to indicate whether they have experience or knowledge of it:
\begin{itemize}
    \item[$\square$] Yes, I have experienced it
    \item[$\square$] No, but I have heard of it
    \item[$\square$] I'm not sure, but have heard of it
    \item[$\square$] No, I have not heard of or experienced it
\end{itemize}

Participants also responded to ``I know how my data was used in the hiring process'' and ``I received feedback from automated hiring algorithms'' from 1: Strongly disagree to 5: Strongly agree.

\subsubsection{Strategy Use}

Participants were asked the following questions about strategy use:
\begin{itemize}
\item Have you modified your resume specifically for automated resume readers? (Yes/No)
\item Have you added keywords from your job description? (Yes/No)
\item Have you changed the layout? (Yes/No)
\item Have you put it through a resume scanner? (Yes/No)
\item Have you modified your resume in some other way for automated hiring? (please specify)
\item Did you use a tool (LeetCode, HackerRank, etc.) to practice for coding assessments? (Yes/No)
\item Have you used anything else to prepare for automated assessments? (please specify)
\item Have you ever received a job referral? (Yes/No)
\item What proportion of your job applications did you have a referral for? (approximate percentage)
\item Approximately how many companies did you apply to? 
\item How did you learn about the application process? (check all that apply)
    \begin{itemize}
        \item[$\square$] Application materials and descriptions
        \item[$\square$] Online resources
        \item[$\square$] Career services through university 
        \item[$\square$] People who had gone through the application process
        \item[$\square$] Recruiter outside of company
        \item[$\square$] Recruiter through company
        \item[$\square$] Family members who worked at companies 
        \item[$\square$] Friends who worked at companies 
        \item[$\square$] Other people who worked at companies
    \end{itemize}
There was also an option to include additional strategies and an attention check in this stage.
\end{itemize}

\subsubsection{Hiring Outcome}
Participants were also asked about their hiring process and its outcome.
\begin{itemize}
\item Have you completed your hiring process? (Yes/No/Not applying to jobs)
\item I am satisfied with my hiring process so far. (1: Strongly disagree to 5: Strongly agree)
\item What is the outcome of your hiring process so far? 
    \begin{itemize}
        \item[$\square$] Multiple job offers
        \item[$\square$] One job offer
        \item[$\square$] No job offers
        \item[$\square$] Not applying to jobs
    \end{itemize}
\end{itemize}

\subsubsection{Demographic Information}
All questions in this section were optional and asked participants to disclose demographic information.

\begin{itemize}
    \item How would you describe your gender identity? (Select all that apply)
        \begin{itemize}
            \item[$\square$] Woman
            \item[$\square$] Man
            \item[$\square$] Non-binary
            \item[$\square$] Genderqueer
            \item[$\square$] Agender
            \item[$\square$] A gender not listed
        \end{itemize}
    \item What best describes you? (Select all that apply)
        \begin{itemize}
            \item[$\square$] Black or African-American
            \item[$\square$] American Indian or Alaskan Native
            \item[$\square$] Asian American or Asian
            \item[$\square$] Hispanic or Latinx
            \item[$\square$] Middle Eastern or North African
            \item[$\square$] Pacific Islander
            \item[$\square$] White or Caucasian
            \item[$\square$] Some other race, ethnicity, or origin 
        \end{itemize}
    \item What is your family’s approximate household income? 
\end{itemize}

\clearpage 

\subsection{Complete Statistical Results}
\label{app:stats}

\begin{table}[ht]
\begin{tabular}{lrrrrl}
\hline
\textbf{}                                            & \textbf{Estimate} & \textbf{Std. Error} & \textbf{t value} & \textbf{Pr(\textgreater{}|t|)} & \textbf{} \\ \hline
(Intercept)                                       & 2.786  & 0.266 & 10.493 & \textless{}0.01 &   \\
Added job description keywords to resume & 0.139  & -1.468    & 0.144 & 0.121            &   \\
Modified resume layout for resume readers & 0.150         & 0.133           & 1.119            & 0.265                         &           \\
Put resume through a resume scanner               & 0.001  & 0.136 & 0.007  & 0.995           &   \\
Practiced for online coding assessment            & 0.249  & 0.140 & 1.787  & 0.075           &   \\
Used referrals                                    & -0.336 & 0.136 & -2.478 & 0.014           & * \\
Percent of companies applied to with referral   & 0.002         & 0.003           & 0.817            & 0.415                         &           \\
Number of companies applied to                    & 0.001  & -0.516    & 0.606 & 0.405            &   \\
Awareness of online coding assessments            & -0.551 & 0.235 & -2.349 & 0.020           & * \\
Awareness of resume scanners                      & 0.014  & 0.183 & 0.076  & 0.940           &   \\
Awareness of automated video interviews           & 0.354  & 0.170 & 2.113  & 0.036           & * \\
Knowledge of data use                             & 0.055  & 0.047 & 1.162  & 0.247           &   \\
Received feedback in the hiring process           & 0.058  & 0.046 & 1.257  & 0.210           &   \\
Used application materials and descriptions       & -0.176 & 0.114 & -1.539 & 0.125           &   \\
Used online resources                             & 0.288  & 0.133 & 2.160  & 0.032           & * \\
Used career services through university           & 0.063  & 0.108 & 0.588  & 0.557           &   \\
Talked with people who had recently applied       & 0.129  & 0.127 & 1.012  & 0.313           &   \\
Connected with recruiter outside of company       & 0.053  & 0.159 & 0.336  & 0.737           &   \\
Connected with recruiter through company          & 0.124  & -1.346    & 0.180 & 0.191            &   \\
Had family who worked at companies        & 0.044  & 0.144 & 0.306  & 0.760           &   \\
Had friends who worked at companies               & 0.140  & 0.112 & 1.247  & 0.214           &   \\
Connected with other company contacts             & -0.022 & 0.126 & -0.179 & 0.858           &   \\
Race                                              & 0.005  & 0.109 & 0.425  & 0.671           &   \\
Gender                                            & -0.003 & 0.142 & -0.024 & 0.981           &   \\
Income                                            & 0.0000002  & 0.0000003 & 0.569  & 0.570           &   \\ \hline
\end{tabular}
\caption{\label{tab:fairStats} Linear regression model of procedural fairness perceptions for automated processes based on strategy use, awareness of AEDTs, gender, race, and income.}
\end{table}

\begin{table}[ht]
\begin{tabular}{lrrrrl}
\hline
\textbf{}                                            & \textbf{Estimate} & \textbf{Std. Error} & \textbf{t value} & \textbf{Pr(\textgreater{}|t|)} & \textbf{} \\ \hline
(Intercept)                                 & 2.479  & 0.268 & 9.267  & \textless{}0.01 &    \\
Added job description keywords to resume    & 0.140         & -1.374              & 0.171           & 0.210                         &           \\
Modified resume layout for resume readers & 0.169         & 0.135          & 1.257            & 0.210                         &           \\
Put resume through a resume scanner         & 0.038  & 0.137 & 0.273  & 0.785           &    \\
Practiced for online coding assessment      & 0.201  & 0.141 & 1.427  & 0.155           &    \\
Used referrals                              & -0.316 & 0.137 & -2.312 & 0.022           & *  \\
Percent of companies applied to with referral   & 0.002         & 0.003           & 0.670            & 0.504                         &           \\
Number of companies applied to              & 0.0004  & 0.001 & 0.544  & 0.589           &    \\
Awareness of online coding assessments      & -0.557 & 0.237 & -2.356 & 0.019           & *  \\
Awareness of resume scanners                & -0.046 & 0.184 & -0.248 & 0.805           &    \\
Awareness of automated video interviews     & 0.440  & 0.169 & 2.608  & {0.010}           & * \\
Knowledge of data use                       & 0.106  & 0.047 & 2.240  & 0.026           & *  \\
Received feedback in the hiring process     & 0.027  & 0.046 & 0.588  & 0.558           &    \\
Used application materials and descriptions & -0.220 & 0.012 & -1.911 & 0.057           &    \\
Used online resources                       & 0.261  & 0.134 & 1.942  & 0.054           &    \\
Used career services through university     & 0.152  & 0.108 & 1.399  & 0.163           &    \\
Talked with people who had recently applied & 0.172  & 0.128 & 1.344  & 0.181           &    \\
Connected with recruiter outside of company & 0.160  & -0.005    & 0.996 & 0.180          &    \\
Connected with recruiter through company    & 0.125  & -1.392    & 0.165 & 0.968           &    \\
Had family who worked at companies  & -0.006 & 0.145 & -0.040 & 0.968           &    \\
Had friends who worked at companies         & 0.134  & 0.113 & 1.188  & 0.236           &    \\
Connected with other company contacts       & 0.049  & 0.127 & 0.385  & 0.700           &    \\
Race                                        & 0.013  & 0.110 & 0.122  & 0.903           &    \\
Gender                                      & -0.116 & 0.143 & -0.815 & 0.416           &    \\
Income                                      & 0.0000002  & 0.0000003 & 0.623  & 0.534           &    \\ \hline
\end{tabular}
\caption{\label{tab:evalStats} Linear regression model of willingness to be evaluated by automated processes based on strategy use, awareness of AEDTs, gender, race, and income.}
\end{table}

\clearpage

\begin{table}[ht]
\begin{tabular}{lrrrrrl}
\toprule
& \textbf{Estimate}  & \textbf{Std. Error} & \textbf{t value} & \textbf{Pr(\textgreater{}|t|)} &   \\
\hline
(Intercept)                                          & 0.329     & 0.237      & 1.386   & 0.168                &   \\
Added job description keywords to resume    & 0.168     & 0.107      & 1.563   & 0.121                 &   \\
Modified resume layout for resume readers & 0.103     & -0.724     & 0.471   & 0.515                 &   \\
Put resume through a resume scanner                  & 0.020     & 0.101      & 0.201   & 0.841                 &   \\
Practiced for online coding assessment               & -0.201    & 0.133      & -1.513  & 0.133                 &   \\
Used referrals                                       & 0.122     & 0.100      & 1.213   & 0.227                 &   \\
Percent of companies applied to with referral   & 0.004     & 0.002      & 2.063   & 0.041                 & * \\
Number of companies applied to                       & 0.0004    & 0.001     & 0.835   & 0.405                 &   \\
Awareness of online coding assessments               & 0.050     & 0.199      & 0.252   & 0.801                 &   \\
Awareness of resume scanners                         & -0.019    & 0.173      & -0.109  & 0.913                 &   \\
Awareness of automated video interviews              & -0.036    & 0.157      & -0.228  & 0.820                 &   \\
Knowledge of data use                                & 0.039     & 0.004      & 0.984   & 0.327                 &   \\
Received feedback in the hiring process              & 0.011     & 0.004      & 0.302   & 0.763                 &   \\
Used application materials and descriptions          & 0.025     & 0.009      & 0.279   & 0.781                 &   \\
Used online resources                                & -0.174    & 0.115      & -1.518  & 0.132                 &   \\
Used career services through university              & 0.055     & 0.085      & 0.644   & 0.521                 &   \\
Talked with people who had recently applied          & 0.024     & 0.107      & 0.225   & 0.823                 &   \\
Connected with recruiter outside of company          & 0.009     & 0.112      & 0.080   & 0.937                 &   \\
Connected with recruiter through company             & 0.115     & 0.088      & 1.314   & 0.191                 &   \\
Had family who worked at companies           & -0.140    & 0.109      & -1.287  & 0.200                 &   \\
Had friends who worked at companies                  & 0.160     & 0.087      & 1.841   & 0.068                 &   \\
Connected with other company contacts                & -0.101    & 0.093     & -1.089  & 0.278                &   \\
Race                                                 & -0.008    & 0.119      & -0.070  & 0.945                 &   \\
Gender                                               & 0.081     & 0.081     & 0.991   & 0.324                 &   \\
Income                                               & 0.000001 & 0.0000002  & 2.530   & 0.013                 & * \\
\bottomrule
\end{tabular}
\caption{\label{tab:jobStats} Linear regression model of job success based on strategy use, awareness of AEDTs, gender, race, and income.}
\end{table}



\end{document}