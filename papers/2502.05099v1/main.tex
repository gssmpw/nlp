\documentclass[manuscript, nonacm]{acmart}

%% \BibTeX command to typeset BibTeX logo in the docs
\AtBeginDocument{%
  \providecommand\BibTeX{{%
    Bib\TeX}}}

\begin{document}

\title[Navigating Automated Hiring]{Navigating Automated Hiring: Perceptions, Strategy Use, and Outcomes Among Young Job Seekers}

\author{Lena Armstrong}
\email{larmstrong@g.harvard.edu}
\orcid{0000-0001-8306-9270}
\affiliation{%
  \institution{Harvard University}
  \city{Cambridge}
  \state{MA}
  \country{USA}
}

\author{Danaé Metaxa}
\email{metaxa@seas.upenn.edu }
\orcid{0000-0001-9359-6090}
\affiliation{%
  \institution{University of Pennsylvania}
  \city{Philadelphia}
  \state{PA}
  \country{USA}
}

\renewcommand{\shortauthors}{Lena Armstrong and Danaé Metaxa}

\begin{abstract}
  As the use of automated employment decision tools (AEDTs) has rapidly increased in hiring contexts, especially for computing jobs, there is still limited work on applicants' perceptions of these emerging tools and their experiences navigating them. To investigate, we conducted a survey with 448 computer science students (young, current technology job-seekers) about perceptions of the procedural fairness of AEDTs, their willingness to be evaluated by different AEDTs, the strategies they use relating to automation in the hiring process, and their job seeking success. We find that young job seekers' procedural fairness perceptions of and willingness to be evaluated by AEDTs varied with the level of automation involved in the AEDT, the technical nature of the task being evaluated, and their own use of strategies, such as job referrals. Examining the relationship of their strategies with job outcomes, notably, we find that referrals and family household income have significant and positive impacts on hiring success, while more egalitarian strategies (using free online coding assessment practice or adding keywords to resumes) did not. Overall, our work speaks to young job seekers' distrust of automation in hiring contexts, as well as the continued role of social and socioeconomic privilege in job seeking, despite the use of AEDTs that promise to make hiring ``unbiased.'' 
\end{abstract}

%% The code below is generated by the tool at http://dl.acm.org/ccs.cfm.
%% Please copy and paste the code instead of the example below.

\begin{CCSXML}
<ccs2012>
<concept>
<concept_id>10003456.10003457.10003458</concept_id>
<concept_desc>Social and professional topics~Computing industry</concept_desc>
<concept_significance>500</concept_significance>
</concept>
<concept>
<concept_id>10003456.10003457.10003580.10003568</concept_id>
<concept_desc>Social and professional topics~Employment issues</concept_desc>
<concept_significance>500</concept_significance>
</concept>
<concept>
<concept_id>10003456.10003457.10003567.10003569</concept_id>
<concept_desc>Social and professional topics~Automation</concept_desc>
<concept_significance>500</concept_significance>
</concept>
</ccs2012>
\end{CCSXML}

\ccsdesc[500]{Social and professional topics~Computing industry}
\ccsdesc[500]{Social and professional topics~Employment issues}
\ccsdesc[500]{Social and professional topics~Automation}

\keywords{automated hiring, AI fairness, job outcomes, automated employment decision tools}

\maketitle

%!TEX root = gcn.tex
\section{Introduction}
Graphs, representing structural data and topology, are widely used across various domains, such as social networks and merchandising transactions.
Graph convolutional networks (GCN)~\cite{iclr/KipfW17} have significantly enhanced model training on these interconnected nodes.
However, these graphs often contain sensitive information that should not be leaked to untrusted parties.
For example, companies may analyze sensitive demographic and behavioral data about users for applications ranging from targeted advertising to personalized medicine.
Given the data-centric nature and analytical power of GCN training, addressing these privacy concerns is imperative.

Secure multi-party computation (MPC)~\cite{crypto/ChaumDG87,crypto/ChenC06,eurocrypt/CiampiRSW22} is a critical tool for privacy-preserving machine learning, enabling mutually distrustful parties to collaboratively train models with privacy protection over inputs and (intermediate) computations.
While research advances (\eg,~\cite{ccs/RatheeRKCGRS20,uss/NgC21,sp21/TanKTW,uss/WatsonWP22,icml/Keller022,ccs/ABY318,folkerts2023redsec}) support secure training on convolutional neural networks (CNNs) efficiently, private GCN training with MPC over graphs remains challenging.

Graph convolutional layers in GCNs involve multiplications with a (normalized) adjacency matrix containing $\numedge$ non-zero values in a $\numnode \times \numnode$ matrix for a graph with $\numnode$ nodes and $\numedge$ edges.
The graphs are typically sparse but large.
One could use the standard Beaver-triple-based protocol to securely perform these sparse matrix multiplications by treating graph convolution as ordinary dense matrix multiplication.
However, this approach incurs $O(\numnode^2)$ communication and memory costs due to computations on irrelevant nodes.
%
Integrating existing cryptographic advances, the initial effort of SecGNN~\cite{tsc/WangZJ23,nips/RanXLWQW23} requires heavy communication or computational overhead.
Recently, CoGNN~\cite{ccs/ZouLSLXX24} optimizes the overhead in terms of  horizontal data partitioning, proposing a semi-honest secure framework.
Research for secure GCN over vertical data  remains nascent.

Current MPC studies, for GCN or not, have primarily targeted settings where participants own different data samples, \ie, horizontally partitioned data~\cite{ccs/ZouLSLXX24}.
MPC specialized for scenarios where parties hold different types of features~\cite{tkde/LiuKZPHYOZY24,icml/CastigliaZ0KBP23,nips/Wang0ZLWL23} is rare.
This paper studies $2$-party secure GCN training for these vertical partition cases, where one party holds private graph topology (\eg, edges) while the other owns private node features.
For instance, LinkedIn holds private social relationships between users, while banks own users' private bank statements.
Such real-world graph structures underpin the relevance of our focus.
To our knowledge, no prior work tackles secure GCN training in this context, which is crucial for cross-silo collaboration.


To realize secure GCN over vertically split data, we tailor MPC protocols for sparse graph convolution, which fundamentally involves sparse (adjacency) matrix multiplication.
Recent studies have begun exploring MPC protocols for sparse matrix multiplication (SMM).
ROOM~\cite{ccs/SchoppmannG0P19}, a seminal work on SMM, requires foreknowledge of sparsity types: whether the input matrices are row-sparse or column-sparse.
Unfortunately, GCN typically trains on graphs with arbitrary sparsity, where nodes have varying degrees and no specific sparsity constraints.
Moreover, the adjacency matrix in GCN often contains a self-loop operation represented by adding the identity matrix, which is neither row- nor column-sparse.
Araki~\etal~\cite{ccs/Araki0OPRT21} avoid this limitation in their scalable, secure graph analysis work, yet it does not cover vertical partition.

% and related primitives
To bridge this gap, we propose a secure sparse matrix multiplication protocol, \osmm, achieving \emph{accurate, efficient, and secure GCN training over vertical data} for the first time.

\subsection{New Techniques for Sparse Matrices}
The cost of evaluating a GCN layer is dominated by SMM in the form of $\adjmat\feamat$, where $\adjmat$ is a sparse adjacency matrix of a (directed) graph $\graph$ and $\feamat$ is a dense matrix of node features.
For unrelated nodes, which often constitute a substantial portion, the element-wise products $0\cdot x$ are always zero.
Our efficient MPC design 
avoids unnecessary secure computation over unrelated nodes by focusing on computing non-zero results while concealing the sparse topology.
We achieve this~by:
1) decomposing the sparse matrix $\adjmat$ into a product of matrices (\S\ref{sec::sgc}), including permutation and binary diagonal matrices, that can \emph{faithfully} represent the original graph topology;
2) devising specialized protocols (\S\ref{sec::smm_protocol}) for efficiently multiplying the structured matrices while hiding sparsity topology.


 
\subsubsection{Sparse Matrix Decomposition}
We decompose adjacency matrix $\adjmat$ of $\graph$ into two bipartite graphs: one represented by sparse matrix $\adjout$, linking the out-degree nodes to edges, the other 
by sparse matrix $\adjin$,
linking edges to in-degree nodes.

%\ie, we decompose $\adjmat$ into $\adjout \adjin$, where $\adjout$ and $\adjin$ are sparse matrices representing these connections.
%linking out-degree nodes to edges and edges to in-degree nodes of $\graph$, respectively.

We then permute the columns of $\adjout$ and the rows of $\adjin$ so that the permuted matrices $\adjout'$ and $\adjin'$ have non-zero positions with \emph{monotonically non-decreasing} row and column indices.
A permutation $\sigma$ is used to preserve the edge topology, leading to an initial decomposition of $\adjmat = \adjout'\sigma \adjin'$.
This is further refined into a sequence of \emph{linear transformations}, 
which can be efficiently computed by our MPC protocols for 
\emph{oblivious permutation}
%($\Pi_{\ssp}$) 
and \emph{oblivious selection-multiplication}.
% ($\Pi_\SM$)
\iffalse
Our approach leverages bipartite graph representation and the monotonicity of non-zero positions to decompose a general sparse matrix into linear transformations, enhancing the efficiency of our MPC protocols.
\fi
Our decomposition approach is not limited to GCNs but also general~SMM 
by 
%simply 
treating them 
as adjacency matrices.
%of a graph.
%Since any sparse matrix can be viewed 

%allowing the same technique to be applied.

 
\subsubsection{New Protocols for Linear Transformations}
\emph{Oblivious permutation} (OP) is a two-party protocol taking a private permutation $\sigma$ and a private vector $\xvec$ from the two parties, respectively, and generating a secret share $\l\sigma \xvec\r$ between them.
Our OP protocol employs correlated randomnesses generated in an input-independent offline phase to mask $\sigma$ and $\xvec$ for secure computations on intermediate results, requiring only $1$ round in the online phase (\cf, $\ge 2$ in previous works~\cite{ccs/AsharovHIKNPTT22, ccs/Araki0OPRT21}).

Another crucial two-party protocol in our work is \emph{oblivious selection-multiplication} (OSM).
It takes a private bit~$s$ from a party and secret share $\l x\r$ of an arithmetic number~$x$ owned by the two parties as input and generates secret share $\l sx\r$.
%between them.
%Like our OP protocol, o
Our $1$-round OSM protocol also uses pre-computed randomnesses to mask $s$ and $x$.
%for secure computations.
Compared to the Beaver-triple-based~\cite{crypto/Beaver91a} and oblivious-transfer (OT)-based approaches~\cite{pkc/Tzeng02}, our protocol saves ${\sim}50\%$ of online communication while having the same offline communication and round complexities.

By decomposing the sparse matrix into linear transformations and applying our specialized protocols, our \osmm protocol
%($\prosmm$) 
reduces the complexity of evaluating $\numnode \times \numnode$ sparse matrices with $\numedge$ non-zero values from $O(\numnode^2)$ to $O(\numedge)$.

%(\S\ref{sec::secgcn})
\subsection{\cgnn: Secure GCN made Efficient}
Supported by our new sparsity techniques, we build \cgnn, 
a two-party computation (2PC) framework for GCN inference and training over vertical
%ly split
data.
Our contributions include:

1) We are the first to explore sparsity over vertically split, secret-shared data in MPC, enabling decompositions of sparse matrices with arbitrary sparsity and isolating computations that can be performed in plaintext without sacrificing privacy.

2) We propose two efficient $2$PC primitives for OP and OSM, both optimally single-round.
Combined with our sparse matrix decomposition approach, our \osmm protocol ($\prosmm$) achieves constant-round communication costs of $O(\numedge)$, reducing memory requirements and avoiding out-of-memory errors for large matrices.
In practice, it saves $99\%+$ communication
%(Table~\ref{table:comm_smm}) 
and reduces ${\sim}72\%$ memory usage over large $(5000\times5000)$ matrices compared with using Beaver triples.
%(Table~\ref{table:mem_smm_sparse}) ${\sim}16\%$-

3) We build an end-to-end secure GCN framework for inference and training over vertically split data, maintaining accuracy on par with plaintext computations.
We will open-source our evaluation code for research and deployment.

To evaluate the performance of $\cgnn$, we conducted extensive experiments over three standard graph datasets (Cora~\cite{aim/SenNBGGE08}, Citeseer~\cite{dl/GilesBL98}, and Pubmed~\cite{ijcnlp/DernoncourtL17}),
reporting communication, memory usage, accuracy, and running time under varying network conditions, along with an ablation study with or without \osmm.
Below, we highlight our key achievements.

\textit{Communication (\S\ref{sec::comm_compare_gcn}).}
$\cgnn$ saves communication by $50$-$80\%$.
(\cf,~CoGNN~\cite{ccs/KotiKPG24}, OblivGNN~\cite{uss/XuL0AYY24}).

\textit{Memory usage (\S\ref{sec::smmmemory}).}
\cgnn alleviates out-of-memory problems of using %the standard 
Beaver-triples~\cite{crypto/Beaver91a} for large datasets.

\textit{Accuracy (\S\ref{sec::acc_compare_gcn}).}
$\cgnn$ achieves inference and training accuracy comparable to plaintext counterparts.
%training accuracy $\{76\%$, $65.1\%$, $75.2\%\}$ comparable to $\{75.7\%$, $65.4\%$, $74.5\%\}$ in plaintext.

{\textit{Computational efficiency (\S\ref{sec::time_net}).}} 
%If the network is worse in bandwidth and better in latency, $\cgnn$ shows more benefits.
$\cgnn$ is faster by $6$-$45\%$ in inference and $28$-$95\%$ in training across various networks and excels in narrow-bandwidth and low-latency~ones.

{\textit{Impact of \osmm (\S\ref{sec:ablation}).}}
Our \osmm protocol shows a $10$-$42\times$ speed-up for $5000\times 5000$ matrices and saves $10$-2$1\%$ memory for ``small'' datasets and up to $90\%$+ for larger ones.


\section{Background}
\label{sec:background}


\subsection{Code Review Automation}
Code review is a widely adopted practice among software developers where a reviewer examines changes submitted in a pull request \cite{hong2022commentfinder, ben2024improving, siow2020core}. If the pull request is not approved, the reviewer must describe the issues or improvements required, providing constructive feedback and identifying potential issues. This step involves review commment generation, which play a key role in the review process by generating review comments for a given code difference. These comments can be descriptive, offering detailed explanations of the issues, or actionable, suggesting specific solutions to address the problems identified \cite{ben2024improving}.


Various approaches have been explored to automate the code review comments process  \cite{tufano2023automating, tufano2024code, yang2024survey}. 
Early efforts centered on knowledge-based systems, which are designed to detect common issues in code. Although these traditional tools provide some support to programmers, they often fall short in addressing complex scenarios encountered during code reviews \cite{dehaerne2022code}. More recently, with advancements in deep learning, researchers have shifted their focus toward using large-language models to enhance the effectiveness of code issue detection and code review comment generation.

\subsection{Knowledge-based Code Review Comments Automation}

Knowledge-based systems (KBS) are software applications designed to emulate human expertise in specific domains by using a collection of rules, logic, and expert knowledge. KBS often consist of facts, rules, an explanation facility, and knowledge acquisition. In the context of software development, these systems are used to analyze the source code, identifying issues such as coding standard violations, bugs, and inefficiencies~\cite{singh2017evaluating, delaitre2015evaluating, ayewah2008using, habchi2018adopting}. By applying a vast set of predefined rules and best practices, they provide automated feedback and recommendations to developers. Tools such as FindBugs \cite{findBugs}, PMD \cite{pmd}, Checkstyle \cite{checkstyle}, and SonarQube \cite{sonarqube} are prominent examples of knowledge-based systems in code analysis, often referred to as static analyzers. These tools have been utilized since the early 1960s, initially to optimize compiler operations, and have since expanded to include debugging tools and software development frameworks \cite{stefanovic2020static, beller2016analyzing}.



\subsection{LLMs-based Code Review Comments Automation}
As the field of machine learning in software engineering evolves, researchers are increasingly leveraging machine learning (ML) and deep learning (DL) techniques to automate the generation of review comments \cite{li2022automating, tufano2022using, balachandran2013reducing, siow2020core, li2022auger, hong2022commentfinder}. Large language models (LLMs) are large-scale Transformer models, which are distinguished by their large number of parameters and extensive pre-training on diverse datasets.  Recently, LLMs have made substantial progress and have been applied across a broad spectrum of domains. Within the software engineering field, LLMs can be categorized into two main types: unified language models and code-specific models, each serving distinct purposes \cite{lu2023llama}.

Code-specific LLMs, such as CodeGen \cite{nijkamp2022codegen}, StarCoder \cite{li2023starcoder} and CodeLlama \cite{roziere2023code} are optimized to excel in code comprehension, code generation, and other programming-related tasks. These specialized models are increasingly utilized in code review activities to detect potential issues, suggest improvements, and automate review comments \cite{yang2024survey, lu2023llama}. 




\subsection{Retrieval-Augmented Generation}
Retrieval-Augmented Generation (RAG) is a general paradigm that enhances LLMs outputs by including relevant information retrieved from external databases into the input prompt \cite{gao2023retrieval}. Traditional LLMs generate responses based solely on the extensive data used in pre-training, which can result in limitations, especially when it comes to domain-specific, time-sensitive, or highly specialized information. RAG addresses these limitations by dynamically retrieving pertinent external knowledge, expanding the model's informational scope and allowing it to generate responses that are more accurate, up-to-date, and contextually relevant \cite{arslan2024business}. 

To build an effective end-to-end RAG pipeline, the system must first establish a comprehensive knowledge base. It requires a retrieval model that captures the semantic meaning of presented data, ensuring relevant information is retrieved. Finally, a capable LLM integrates this retrieved knowledge to generate accurate and coherent results \cite{ibtasham2024towards}.




\subsection{LLM as a Judge Mechanism}

LLM evaluators, often referred to as LLM-as-a-Judge, have gained significant attention due to their ability to align closely with human evaluators' judgments \cite{zhu2023judgelm, shi2024judging}. Their adaptability and scalability make them highly suitable for handling an increasing volume of evaluative tasks. 

Recent studies have shown that certain LLMs, such as Llama-3 70B and GPT-4 Turbo, exhibit strong alignment with human evaluators, making them promising candidates for automated judgment tasks \cite{thakur2024judging}

To enable such evaluations, a proper benchmarking system should be set up with specific components: \emph{prompt design}, which clearly instructs the LLM to evaluate based on a given metric, such as accuracy, relevance, or coherence; \emph{response presentation}, guiding the LLM to present its verdicts in a structured format; and \emph{scoring}, enabling the LLM to assign a score according to a predefined scale \cite{ibtasham2024towards}. Additionally, this evaluation system can be enriched with the ability to explain reasoning behind verdicts, which is a significant advantage of LLM-based evaluation \cite{zheng2023judging}. The LLM can outline the criteria it used to reach its judgment, offering deeper insights into its decision-making process.






\vspace{-5pt}
\section{Method}
\label{sec:method}
\section{Overview}

\revision{In this section, we first explain the foundational concept of Hausdorff distance-based penetration depth algorithms, which are essential for understanding our method (Sec.~\ref{sec:preliminary}).
We then provide a brief overview of our proposed RT-based penetration depth algorithm (Sec.~\ref{subsec:algo_overview}).}



\section{Preliminaries }
\label{sec:Preliminaries}

% Before we introduce our method, we first overview the important basics of 3D dynamic human modeling with Gaussian splatting. Then, we discuss the diffusion-based 3d generation techniques, and how they can be applied to human modeling.
% \ZY{I stopp here. TBC.}
% \subsection{Dynamic human modeling with Gaussian splatting}
\subsection{3D Gaussian Splatting}
3D Gaussian splatting~\cite{kerbl3Dgaussians} is an explicit scene representation that allows high-quality real-time rendering. The given scene is represented by a set of static 3D Gaussians, which are parameterized as follows: Gaussian center $x\in {\mathbb{R}^3}$, color $c\in {\mathbb{R}^3}$, opacity $\alpha\in {\mathbb{R}}$, spatial rotation in the form of quaternion $q\in {\mathbb{R}^4}$, and scaling factor $s\in {\mathbb{R}^3}$. Given these properties, the rendering process is represented as:
\begin{equation}
  I = Splatting(x, c, s, \alpha, q, r),
  \label{eq:splattingGA}
\end{equation}
where $I$ is the rendered image, $r$ is a set of query rays crossing the scene, and $Splatting(\cdot)$ is a differentiable rendering process. We refer readers to Kerbl et al.'s paper~\cite{kerbl3Dgaussians} for the details of Gaussian splatting. 



% \ZY{I would suggest move this part to the method part.}
% GaissianAvatar is a dynamic human generation model based on Gaussian splitting. Given a sequence of RGB images, this method utilizes fitted SMPLs and sampled points on its surface to obtain a pose-dependent feature map by a pose encoder. The pose-dependent features and a geometry feature are fed in a Gaussian decoder, which is employed to establish a functional mapping from the underlying geometry of the human form to diverse attributes of 3D Gaussians on the canonical surfaces. The parameter prediction process is articulated as follows:
% \begin{equation}
%   (\Delta x,c,s)=G_{\theta}(S+P),
%   \label{eq:gaussiandecoder}
% \end{equation}
%  where $G_{\theta}$ represents the Gaussian decoder, and $(S+P)$ is the multiplication of geometry feature S and pose feature P. Instead of optimizing all attributes of Gaussian, this decoder predicts 3D positional offset $\Delta{x} \in {\mathbb{R}^3}$, color $c\in\mathbb{R}^3$, and 3D scaling factor $ s\in\mathbb{R}^3$. To enhance geometry reconstruction accuracy, the opacity $\alpha$ and 3D rotation $q$ are set to fixed values of $1$ and $(1,0,0,0)$ respectively.
 
%  To render the canonical avatar in observation space, we seamlessly combine the Linear Blend Skinning function with the Gaussian Splatting~\cite{kerbl3Dgaussians} rendering process: 
% \begin{equation}
%   I_{\theta}=Splatting(x_o,Q,d),
%   \label{eq:splatting}
% \end{equation}
% \begin{equation}
%   x_o = T_{lbs}(x_c,p,w),
%   \label{eq:LBS}
% \end{equation}
% where $I_{\theta}$ represents the final rendered image, and the canonical Gaussian position $x_c$ is the sum of the initial position $x$ and the predicted offset $\Delta x$. The LBS function $T_{lbs}$ applies the SMPL skeleton pose $p$ and blending weights $w$ to deform $x_c$ into observation space as $x_o$. $Q$ denotes the remaining attributes of the Gaussians. With the rendering process, they can now reposition these canonical 3D Gaussians into the observation space.



\subsection{Score Distillation Sampling}
Score Distillation Sampling (SDS)~\cite{poole2022dreamfusion} builds a bridge between diffusion models and 3D representations. In SDS, the noised input is denoised in one time-step, and the difference between added noise and predicted noise is considered SDS loss, expressed as:

% \begin{equation}
%   \mathcal{L}_{SDS}(I_{\Phi}) \triangleq E_{t,\epsilon}[w(t)(\epsilon_{\phi}(z_t,y,t)-\epsilon)\frac{\partial I_{\Phi}}{\partial\Phi}],
%   \label{eq:SDSObserv}
% \end{equation}
\begin{equation}
    \mathcal{L}_{\text{SDS}}(I_{\Phi}) \triangleq \mathbb{E}_{t,\epsilon} \left[ w(t) \left( \epsilon_{\phi}(z_t, y, t) - \epsilon \right) \frac{\partial I_{\Phi}}{\partial \Phi} \right],
  \label{eq:SDSObservGA}
\end{equation}
where the input $I_{\Phi}$ represents a rendered image from a 3D representation, such as 3D Gaussians, with optimizable parameters $\Phi$. $\epsilon_{\phi}$ corresponds to the predicted noise of diffusion networks, which is produced by incorporating the noise image $z_t$ as input and conditioning it with a text or image $y$ at timestep $t$. The noise image $z_t$ is derived by introducing noise $\epsilon$ into $I_{\Phi}$ at timestep $t$. The loss is weighted by the diffusion scheduler $w(t)$. 
% \vspace{-3mm}

\subsection{Overview of the RTPD Algorithm}\label{subsec:algo_overview}
Fig.~\ref{fig:Overview} presents an overview of our RTPD algorithm.
It is grounded in the Hausdorff distance-based penetration depth calculation method (Sec.~\ref{sec:preliminary}).
%, similar to that of Tang et al.~\shortcite{SIG09HIST}.
The process consists of two primary phases: penetration surface extraction and Hausdorff distance calculation.
We leverage the RTX platform's capabilities to accelerate both of these steps.

\begin{figure*}[t]
    \centering
    \includegraphics[width=0.8\textwidth]{Image/overview.pdf}
    \caption{The overview of RT-based penetration depth calculation algorithm overview}
    \label{fig:Overview}
\end{figure*}

The penetration surface extraction phase focuses on identifying the overlapped region between two objects.
\revision{The penetration surface is defined as a set of polygons from one object, where at least one of its vertices lies within the other object. 
Note that in our work, we focus on triangles rather than general polygons, as they are processed most efficiently on the RTX platform.}
To facilitate this extraction, we introduce a ray-tracing-based \revision{Point-in-Polyhedron} test (RT-PIP), significantly accelerated through the use of RT cores (Sec.~\ref{sec:RT-PIP}).
This test capitalizes on the ray-surface intersection capabilities of the RTX platform.
%
Initially, a Geometry Acceleration Structure (GAS) is generated for each object, as required by the RTX platform.
The RT-PIP module takes the GAS of one object (e.g., $GAS_{A}$) and the point set of the other object (e.g., $P_{B}$).
It outputs a set of points (e.g., $P_{\partial B}$) representing the penetration region, indicating their location inside the opposing object.
Subsequently, a penetration surface (e.g., $\partial B$) is constructed using this point set (e.g., $P_{\partial B}$) (Sec.~\ref{subsec:surfaceGen}).
%
The generated penetration surfaces (e.g., $\partial A$ and $\partial B$) are then forwarded to the next step. 

The Hausdorff distance calculation phase utilizes the ray-surface intersection test of the RTX platform (Sec.~\ref{sec:RT-Hausdorff}) to compute the Hausdorff distance between two objects.
We introduce a novel Ray-Tracing-based Hausdorff DISTance algorithm, RT-HDIST.
It begins by generating GAS for the two penetration surfaces, $P_{\partial A}$ and $P_{\partial B}$, derived from the preceding step.
RT-HDIST processes the GAS of a penetration surface (e.g., $GAS_{\partial A}$) alongside the point set of the other penetration surface (e.g., $P_{\partial B}$) to compute the penetration depth between them.
The algorithm operates bidirectionally, considering both directions ($\partial A \to \partial B$ and $\partial B \to \partial A$).
The final penetration depth between the two objects, A and B, is determined by selecting the larger value from these two directional computations.

%In the Hausdorff distance calculation step, we compute the Hausdorff distance between given two objects using a ray-surface-intersection test. (Sec.~\ref{sec:RT-Hausdorff}) Initially, we construct the GAS for both $\partial A$ and $\partial B$ to utilize the RT-core effectively. The RT-based Hausdorff distance algorithms then determine the Hausdorff distance by processing the GAS of one object (e.g. $GAS_{\partial A}$) and set of the vertices of the other (e.g. $P_{\partial B}$). Following the Hausdorff distance definition (Eq.~\ref{equation:hausdorff_definition}), we compute the Hausdorff distance to both directions ($\partial A \to \partial B$) and ($\partial B \to \partial A$). As a result, the bigger one is the final Hausdorff distance, and also it is the penetration depth between input object $A$ and $B$.


%the proposed RT-based penetration depth calculation pipeline.
%Our proposed methods adopt Tang's Hausdorff-based penetration depth methods~\cite{SIG09HIST}. The pipeline is divided into the penetration surface extraction step and the Hausdorff distance calculation between the penetration surface steps. However, since Tang's approach is not suitable for the RT platform in detail, we modified and applied it with appropriate methods.

%The penetration surface extraction step is extracting overlapped surfaces on other objects. To utilize the RT core, we use the ray-intersection-based PIP(Point-In-Polygon) algorithms instead of collision detection between two objects which Tang et al.~\cite{SIG09HIST} used. (Sec.~\ref{sec:RT-PIP})
%RT core-based PIP test uses a ray-surface intersection test. For purpose this, we generate the GAS(Geometry Acceleration Structure) for each object. RT core-based PIP test takes the GAS of one object (e.g. $GAS_{A}$) and a set of vertex of another one (e.g. $P_{B}$). Then this computes the penetrated vertex set of another one (e.g. $P_{\partial B}$). To calculate the Hausdorff distance, these vertex sets change to objects constructed by penetrated surface (e.g. $\partial B$). Finally, the two generated overlapped surface objects $\partial A$ and $\partial B$ are used in the Hausdorff distance calculation step.

Our goal is to increase the robustness of T2I models, particularly with rare or unseen concepts, which they struggle to generate. To do so, we investigate a retrieval-augmented generation approach, through which we dynamically select images that can provide the model with missing visual cues. Importantly, we focus on models that were not trained for RAG, and show that existing image conditioning tools can be leveraged to support RAG post-hoc.
As depicted in \cref{fig:overview}, given a text prompt and a T2I generative model, we start by generating an image with the given prompt. Then, we query a VLM with the image, and ask it to decide if the image matches the prompt. If it does not, we aim to retrieve images representing the concepts that are missing from the image, and provide them as additional context to the model to guide it toward better alignment with the prompt.
In the following sections, we describe our method by answering key questions:
(1) How do we know which images to retrieve? 
(2) How can we retrieve the required images? 
and (3) How can we use the retrieved images for unknown concept generation?
By answering these questions, we achieve our goal of generating new concepts that the model struggles to generate on its own.

\vspace{-3pt}
\subsection{Which images to retrieve?}
The amount of images we can pass to a model is limited, hence we need to decide which images to pass as references to guide the generation of a base model. As T2I models are already capable of generating many concepts successfully, an efficient strategy would be passing only concepts they struggle to generate as references, and not all the concepts in a prompt.
To find the challenging concepts,
we utilize a VLM and apply a step-by-step method, as depicted in the bottom part of \cref{fig:overview}. First, we generate an initial image with a T2I model. Then, we provide the VLM with the initial prompt and image, and ask it if they match. If not, we ask the VLM to identify missing concepts and
focus on content and style, since these are easy to convey through visual cues.
As demonstrated in \cref{tab:ablations}, empirical experiments show that image retrieval from detailed image captions yields better results than retrieval from brief, generic concept descriptions.
Therefore, after identifying the missing concepts, we ask the VLM to suggest detailed image captions for images that describe each of the concepts. 

\vspace{-4pt}
\subsubsection{Error Handling}
\label{subsec:err_hand}

The VLM may sometimes fail to identify the missing concepts in an image, and will respond that it is ``unable to respond''. In these rare cases, we allow up to 3 query repetitions, while increasing the query temperature in each repetition. Increasing the temperature allows for more diverse responses by encouraging the model to sample less probable words.
In most cases, using our suggested step-by-step method yields better results than retrieving images directly from the given prompt (see 
\cref{subsec:ablations}).
However, if the VLM still fails to identify the missing concepts after multiple attempts, we fall back to retrieving images directly from the prompt, as it usually means the VLM does not know what is the meaning of the prompt.

The used prompts can be found in \cref{app:prompts}.
Next, we turn to retrieve images based on the acquired image captions.

\vspace{-3pt}
\subsection{How to retrieve the required images?}

Given $n$ image captions, our goal is to retrieve the images that are most similar to these captions from a dataset. 
To retrieve images matching a given image caption, we compare the caption to all the images in the dataset using a text-image similarity metric and retrieve the top $k$ most similar images.
Text-to-image retrieval is an active research field~\cite{radford2021learning, zhai2023sigmoid, ray2024cola, vendrowinquire}, where no single method is perfect.
Retrieval is especially hard when the dataset does not contain an exact match to the query \cite{biswas2024efficient} or when the task is fine-grained retrieval, that depends on subtle details~\cite{wei2022fine}.
Hence, a common retrieval workflow is to first retrieve image candidates using pre-computed embeddings, and then re-rank the retrieved candidates using a different, often more expensive but accurate, method \cite{vendrowinquire}.
Following this workflow, we experimented with cosine similarity over different embeddings, and with multiple re-ranking methods of reference candidates.
Although re-ranking sometimes yields better results compared to simply using cosine similarity between CLIP~\cite{radford2021learning} embeddings, the difference was not significant in most of our experiments. Therefore, for simplicity, we use cosine similarity between CLIP embeddings as our similarity metric (see \cref{tab:sim_metrics}, \cref{subsec:ablations} for more details about our experiments with different similarity metrics).

\vspace{-3pt}
\subsection{How to use the retrieved images?}
Putting it all together, after retrieving relevant images, all that is left to do is to use them as context so they are beneficial for the model.
We experimented with two types of models; models that are trained to receive images as input in addition to text and have ICL capabilities (e.g., OmniGen~\cite{xiao2024omnigen}), and T2I models augmented with an image encoder in post-training (e.g., SDXL~\cite{podellsdxl} with IP-adapter~\cite{ye2023ip}).
As the first model type has ICL capabilities, we can supply the retrieved images as examples that it can learn from, by adjusting the original prompt.
Although the second model type lacks true ICL capabilities, it offers image-based control functionalities, which we can leverage for applying RAG over it with our method.
Hence, for both model types, we augment the input prompt to contain a reference of the retrieved images as examples.
Formally, given a prompt $p$, $n$ concepts, and $k$ compatible images for each concept, we use the following template to create a new prompt:
``According to these examples of 
$\mathord{<}c_1\mathord{>:<}img_{1,1}\mathord{>}, ... , \mathord{<}img_{1,k}\mathord{>}, ... , \mathord{<}c_n\mathord{>:<}img_{n,1}\mathord{>}, ... , $
$\mathord{<}img_{n,k}\mathord{>}$,
generate $\mathord{<}p\mathord{>}$'', 
where $c_i$ for $i\in{[1,n]}$ is a compatible image caption of the image $\mathord{<}img_{i,j}\mathord{>},  j\in{[1,k]}$. 

This prompt allows models to learn missing concepts from the images, guiding them to generate the required result. 

\textbf{Personalized Generation}: 
For models that support multiple input images, we can apply our method for personalized generation as well, to generate rare concept combinations with personal concepts. In this case, we use one image for personal content, and 1+ other reference images for missing concepts. For example, given an image of a specific cat, we can generate diverse images of it, ranging from a mug featuring the cat to a lego of it or atypical situations like the cat writing code or teaching a classroom of dogs (\cref{fig:personalization}).
\vspace{-2pt}
\begin{figure}[htp]
  \centering
   \includegraphics[width=\linewidth]{Assets/personalization.pdf}
   \caption{\textbf{Personalized generation example.}
   \emph{ImageRAG} can work in parallel with personalization methods and enhance their capabilities. For example, although OmniGen can generate images of a subject based on an image, it struggles to generate some concepts. Using references retrieved by our method, it can generate the required result.
}
   \label{fig:personalization}\vspace{-10pt}
\end{figure}

\section{Results \& analysis}
Our fine-tuned models set a new state of the art for <10B models on long-context reasoning benchmarks while also improving baseline performance on related narrative understanding tasks. Below, we discuss our results in detail.
% Further analysis reveals room for improvement, as the models benefit more from chapter-level claims rather than book-level claims.


\begin{table*}[htbp] 
    \centering 
    \scriptsize 
    \resizebox{\textwidth}{!}{
        \begin{tabular}{p{\textwidth}} 
        \toprule 
        \cellcolor{gray!10}\textbf{Claim:} "Jacques consistently uses pastoral themes to express his love for Belle-bouche, first through metaphorical discussions of shepherds and shepherdesses, and later while walking through an orchard and discussing Arcadian lovers." \\
        \midrule 
        \end{tabular}
    }
    \resizebox{\textwidth}{!}{
        \begin{tabular}[t]{p{0.4\textwidth} p{0.6\textwidth}} 
        \parbox[t]{0.4\textwidth}{%
        \textbf{\llamainst's CoT:} ``The statement claims that Jacques consistently uses pastoral themes to express his love for Belle-bouche. To evaluate this statement, we need to examine the text and identify instances where Jacques uses pastoral themes to express his love for Belle-bouche. {\color{purple}The text mentions} that Jacques discusses crooks, flowers, and shepherdesses with Belle-bouche, and later, while walking through an orchard, he talks about Arcadian lovers, specifically mentioning Strephon, Chloe, Corydon, Daphnis, and Narcissus. These instances suggest that Jacques does use pastoral themes to express his love for Belle-bouche. However, we need to consider whether this is a consistent pattern throughout the text." \newline \textit{Answer: TRUE}%
        } & 
        \parbox[t]{0.6\textwidth}{%
        \textbf{\llamaftbalanced's CoT:} ``Here are the relevant details from the text: \\ \vspace{-0.12in}
        \begin{itemize}[leftmargin=*]
            \item {\color{teal}Chapter 2}: Jacques engages in a metaphorical discussion about shepherds and shepherdesses, expressing his love for Belle-bouche.
            \item {\color{teal}Chapter 17}: Jacques and Belle-bouche walk through the orchard, discussing the twentieth of May as a day for lovers.
            \item Relationship: Recurring theme of pastoral love
        \end{itemize}
        In {\color{teal}Chapter 2}, Jacques engages in a metaphorical discussion about shepherds and shepherdesses to express his love for Belle-bouche. This theme continues in {\color{teal}Chapter 17}, where Jacques and Belle-bouche walk through an orchard, discussing the twentieth of May as a day for lovers. This connection shows Jacques' consistent use of pastoral imagery to convey his romantic feelings for Belle-bouche throughout the story." \newline \textit{Answer: TRUE}%
        \vspace{2pt}
        } \\
        \bottomrule
        \end{tabular}
    }
    \caption{Comparison of CoT responses produced by \llamainst\ and \llamaftbalanced\ for a claim from John Esten Cooke's \textit{The Youth of Jefferson}. \llamaftbalanced's CoT response contains more explicit chapter references (highlighed in {\color{teal}green}) and details compared to that of \llamainst\ (highlighted in {\color{purple}red}).}
    \label{tab:cot-examples} 
    \vspace{-0.2in} 
\end{table*}


\subsection{\pipeline\ models outperform baselines on narrative claim verification} \label{subsec:main_results}
% \mi{you may want to split this into one para on your test set and one on nocha, each with headers}

% \yapei{todo: address the prolong base issue}
% \mi{more descriptive header!}
% \paragraph{Fine-tuning on our data improves performance on \pipeline-test:} 
On \pipeline-test, our fine-tuned models significantly outperform the instruct models they are initialized from (referred to as baselines),\footnote{\prolongftbalanced\ is initialized from \prolongbase\ instead of \prolonginst. However, since performing evaluation intended for instruct models on a continually pretrained model may not be ideal, we exclude \prolongbase's results from Table \ref{tab:main-result}. As shown in Table \ref{tab:prolong-base-acc}, \prolongbase\ performs significantly worse than \prolonginst\ on \pipeline-test.} as shown in Table \ref{tab:main-result}. 
% This improvement, while expected, is notable in its magnitude.
For example, \qwenftbalanced\ achieves over a 20\% performance gain compared to \qweninst, while \llamaftbalanced\ sees nearly triple the performance of \llamainst. These substantial improvements demonstrate the effectiveness of \pipeline-generated data.

% \mi{same here!}
\paragraph{Fine-tuning on our data improves performance on NoCha:} A similar trend is observed on NoCha. The performance improvements range from an 8\% gain for strong baselines like \qweninst\ to a dramatic twofold increase for weaker baselines such as \llamainst\ and \prolonginst. It is worth noting that all three baseline models initially perform below the random chance baseline of 25\%, but our fine-tuned models consistently surpass this threshold. 

\paragraph{Performance gap between \pipeline-test and NoCha:} We note that the performance gap between NoCha and \pipeline-test\ is likely due to the nature of the events involved in the claims. While \pipeline-test\ consists of synthetic claims derived from events in model-generated outlines, NoCha’s human-written claims may involve reasoning about low-level details that may not typically appear in such generated outlines. Future work could incorporate more low-level events into chapter outlines to create a more diverse set of claims.
%We hope future work will explore synthetic data generation strategies that can help models improve more on complex reasoning tasks like NoCha.
% \mi{add sentence on implication for future work!}

%\yapei{but aren't chapter level claims also about details?},
% On both NoCha and our test set, our models significantly outperform their respective baseline models (Table \ref{tab:main-result}).\mi{i dont think this terminology is easy to understand. maybe write instead that our fine-tuned models outperform the instruct models that they are initialized to? this is also not surprising so you may want to state that.} The performance gains on our test set vary: \qwenftbalanced\ improves by over 20\%, while other fine-tuned models nearly triple their baseline performance by more than 40\%. We observe a similar trend on NoCha, with improvements as small as 8\% for already strong baselines like \qweninst, and as large as a twofold increase for weaker baselines such as \llamainst\ and \prolonginst. Notably, all baseline models initially perform below the random chance baseline of 25\%, but after fine-tuning, they consistently surpass this threshold. We note that the performance gap between NoCha and our test set is likely due to the nature of the events involved in the claims. \pipeline\ contains synthetic claims constructed with major events in the outline, which might make verification more straightforward. In contrast, NoCha's human-written claims contain lower-level plot details, which might be more challenging for LLMs.
%\mi{but we argue that clipper can generate claims about low-level events... maybe say nocha includes reasoning over things that wouldnt make it into an outline in the first place?}

% \mi{rewrite header, very confusing}\chau{is this better?\mi{how about something like Finetuning on our dataset also improves other narrative-related tasks}}
\subsection{Fine-tuning on \pipeline\ improves on other narrative reasoning tasks}  Beyond long-context reasoning, our models also show improvements in narrative understanding and short-context reasoning tasks. On NarrativeQA, which requires comprehension of movie scripts or full books, our best-performing models, \llamaftbalanced\ and \prolongftbalanced, achieve a 2\% and 5\% absolute improvement over their respective baselines. Similarly, on MuSR, a short-form reasoning benchmark, our strongest model, \qwenft, achieves 45.2\% accuracy, surpassing the 41.2\% baseline. However, these improvements are not consistent across all tasks. On $\infty$Bench QA, only \qwenftbalanced\ outperforms the baseline by approximately 7\%. In contrast, \llamaftbalanced\ and \prolongftbalanced\ show slight performance declines of up to 4\%. Thus, while fine-tuning on \pipeline\ data improves performance on reasoning and some aspects of narrative understanding, its transferability is not universal across domains.


% \mi{emphasize that it doesnt improve it THAT much compared to our data}
\subsection{Short-context claim data is less helpful}
% \yapei{can refer back to 3.1 and mention that for our task, training on long data is more effective than training on short data, which contradicts prev findings. then highlight importance of good long data.}
Contrary to prior studies suggesting short-form data benefits long-context tasks \cite{dubey2024llama, gao2024trainlongcontextlanguagemodels} more than long data, our results show otherwise. While \prolongwp, trained on short data, outperforms baselines, it underperforms across all four long-context benchmarks compared to models fine-tuned on our data. This underscores the need for high-quality long-context data generation pipelines like \pipeline.
% outperforms our three baseline models on \dataname-test (60.4\%), NoCha (24.1\%), and MuSR (45.2\%). However, when comparing to our fine-tuned models, 
% While strong performance on MuSR is expected given the benchmark's focus on short-form reasoning, the fact that short-form reasoning also improves performance on other long-context tasks is particularly interesting. This suggests that our training data format, which features detailed reasoning chains on relevant events and their relationships, contributes meaningfully to model improvement.

\subsection{Finetuning on CoTs results in more informative explanations}
We evaluate the groundedness of CoT reasoning generated by our fine-tuned models using DeepSeek-R1-Distill-Llama-70B (\S\ref{data:cot_validation}). Here, a reasoning chain is counted as grounded when every plot event in the chain can be found in the chapter outline that it cites. Table \ref{tab:cot-groundedness} shows that fine-tuning significantly improves groundedness across all models, with \prolongftbalanced\ achieving the highest rate (80.6\%), followed closely by \llamaftbalanced\ (75.9\%). Looking closer at the content of the explanations (Table \ref{tab:cot-examples}), the baseline model (\llamainst) often gives a generic response without citing any evidence, whereas \llamaftbalanced\ explicitly references Chapter 9 and specifies the cause-and-effect relationship.





\begin{table*}[htbp]
\centering
\footnotesize
\scalebox{0.87}{
\begin{tabular}{p{0.1\textwidth}p{0.06\textwidth}p{0.42\textwidth}p{0.42\textwidth}}
\toprule
\multicolumn{1}{c}{\textsc{Category}} & \multicolumn{1}{c}{\textsc{Freq (\%)}} & \multicolumn{1}{c}{\textsc{True Claim}} & \multicolumn{1}{c}{\textsc{False Claim}} \\
\midrule
Event & 43.2 & The Polaris unit, initially assigned to test a new audio transmitter on Tara, explores the planet's surface {\color{teal}using a jet boat without landing}. & The Polaris unit, initially assigned to test a new audio transmitter on Tara, explores the planet's surface by {\color{purple}landing their spaceship}. \\
\midrule
Person & 31.6 & The cattle herd stolen from Yeager by masked rustlers is later found in {\color{teal}General Pasquale}'s possession at Noche Buena. & The cattle herd stolen from Yeager by masked rustlers is later found in {\color{purple}Harrison}'s possession at Noche Buena. \\
\midrule
Object & 15.8 & The alien structure Ross enters contains both a chamber with {\color{teal}a jelly-like bed} and {\color{teal}a control panel capable of communicating with other alien vessels}. & The alien structure Ross enters contains both a chamber with {\color{purple}a metal bed} and {\color{purple}a control panel capable of time travel}. \\
\midrule
Location & 13.7 & Costigan rescues Clio twice: first from Roger on his planetoid, and later from a {\color{teal}Nevian city} using a stolen space-speedster. & Costigan rescues Clio twice: first from Roger on his planetoid, and later from a {\color{purple}Triplanetary city} using a stolen space-speedster. \\
\midrule
Time & 6.3 & Jean Briggerland's meeting with ex-convicts Mr. Hoggins and Mr. Talmot, where she suggests a burglary target, {\color{teal}follows} a failed attempt on Lydia's life involving a speeding car on the sidewalk. & Jean Briggerland's meeting with ex-convicts Mr. Hoggins and Mr. Talmot, where she suggests a burglary target, {\color{purple}precedes} a failed attempt on Lydia's life involving a speeding car on the sidewalk. \\
\midrule
Affect & 4.2 & David Mullins, who initially expresses {\color{teal}skepticism} about Chester's hiring, later fires Chester on false pretenses and immediately replaces him with Felix. & David Mullins, who initially expresses {\color{purple}enthusiasm} about Chester's hiring, later fires Chester on false pretenses and immediately replaces him with Felix. \\
\bottomrule
\end{tabular}}
\caption{A taxonomy of perturbations that cause a false claim to be misclassified as true. Corresponding details in true and false claims are respectively highlighed in {\color{teal}green} and {\color{purple}red}. Frequencies do not sum to 100\%, as some claims receive multiple labels. Category definitions and more detailed analysis in \S\ref{appendix:error-analysis}.}
\label{tab:error-analysis-dist}
\vspace{-0.2in}
\end{table*}

% \begin{table*}[htbp] 
%     \centering 
%     \scriptsize 
%     \resizebox{\textwidth}{!}{
%         \begin{tabular}{p{0.95\textwidth}} 
%         \toprule 
%         \cellcolor{gray!5}\textbf{Claim:} "Jacques consistently uses pastoral themes to express his love for Belle-bouche, first through metaphorical discussions of shepherds and shepherdesses, and later while walking through an orchard and discussing Arcadian lovers." \\
%         \midrule 
%         \textbf{\llamainst's CoT:} \\ 
%         ``The statement claims that Jacques consistently uses pastoral themes to express his love for Belle-bouche. To evaluate this statement, we need to examine the text and identify instances where Jacques uses pastoral themes to express his love for Belle-bouche. The text mentions that Jacques discusses crooks, flowers, and shepherdesses with Belle-bouche, and later, while walking through an orchard, he talks about Arcadian lovers, specifically mentioning Strephon, Chloe, Corydon, Daphnis, and Narcissus. These instances suggest that Jacques does use pastoral themes to express his love for Belle-bouche. However, we need to consider whether this is a consistent pattern throughout the text." \\
%         \textit{Answer: TRUE} \\ 
%         \midrule 
%         \textbf{\llamaftbalanced's CoT:} \\
%         ``Here are the relevant details from the text: \\ \vspace{-2pt} \begin{itemize} \item Chapter 2: Jacques engages in a metaphorical discussion about shepherds and shepherdesses, expressing his love for Belle-bouche. \item Chapter 17: Jacques and Belle-bouche walk through the orchard, discussing the twentieth of May as a day for lovers. \item Relationship: Recurring theme of pastoral love \end{itemize} In Chapter 2, Jacques engages in a metaphorical discussion about shepherds and shepherdesses to express his love for Belle-bouche. This theme continues in Chapter 17, where Jacques and Belle-bouche walk through an orchard, discussing the twentieth of May as a day for lovers. This connection shows Jacques' consistent use of pastoral imagery to convey his romantic feelings for Belle-bouche throughout the story." \\ 
%         \textit{Answer: TRUE} \\ 
%     \bottomrule 
%     \end{tabular} 
%     } 
%     \caption{Comparison of CoT responses produced by \llamainst\ and \llamaftbalanced\ for a claim from John Esten Cooke's \textit{The Youth of Jefferson}.} 
%     \label{tab:cot-examples} 
%     \vspace{-0.2in} 
% \end{table*}



% \mi{header is too informal}
\subsection{Small models struggle with book-level reasoning} 
\label{subsection:chap-book-ft}
Trained only on 8K chapter-level claims, \prolongftchapter\ outperforms \prolongftbook\ on both chapter- and book-level test subsets (Table \ref{tab:chapter_vs_book}). This likely reflects the limitations of smaller models (7B/8B) in handling the complex reasoning required for book-level claims, aligning with prior findings \cite{qi2024quantifyinggeneralizationcomplexitylarge}. The performance gap between the models is modest (4.2\%), and we leave exploration of larger models (>70B) to future work due to compute constraints.
% Although larger models (>70B) might be able to effectively learn the complex reasoning patterns in these multi-chapter claims, we leave this for future work due to limited compute resources. 


\subsection{Fine-tuned models have a difficult time verifying False claims} \label{sec:error-analysis}
% \mi{this can def be heavily shortened / go to appendix, the table itself is sufficient along with a couple sentences}
To study cases where fine-tuned models struggle, we analyze \qwenftbalanced\ outputs. Among 1,000 book-level claim pairs in \pipeline-test, the model fails to verify 37 true claims and 97 false claims, aligning with NoCha findings \cite{karpinska_one_2024} that models struggle more with false claims. We investigate perturbations that make false claims appear true and present a taxonomy with examples in Table \ref{tab:error-analysis-dist}, with further details in \S\ref{appendix:error-analysis}.
% Notably, in 95 cases, the model successfully validates the true claim but fails to validate the corresponding false claim. This raises an important question: \textit{What specific perturbations make a false claim appear true to the model?} Through careful manual analysis, we derive a taxonomy of such perturbations and present them in Table \ref{tab:error-analysis-dist}. The most frequent perturbations are changes to events (43.2\%) and people (31.6\%), such as altering actions or misattributing roles. Less frequent but notable are modifications to objects (15.8\%), locations (13.7\%), time (6.3\%), and affect (4.2\%). All these perturbations introduce plausible-sounding variations that the model may struggle to detect without fully understanding the narrative.\footnote{We provide definitions for each category in Appendix \ref{appendix:error-analysis}} 
%A closer examination of the chain-of-thoughts generated for these 95 claims reveals some recurring patterns: the model often fabricates evidence, applies incorrect reasoning, or completely ignores the perturbed details. Specific examples can be found in Appendix X. \yapei{do we need this part on CoT?}


\section{Discussion}

% Shift from findings to discussion
This study on robotic art explores human-machine relationships in creative processes.
It first contributes as an empirical description of artistic creativity in robotic art practice---an unconventional use of robots---examined through the artists' perspectives on their creative experiences. Our analysis reveals three facets of creativity in robotic art practices: the \textit{social}, \textit{material}, and \textit{temporal}. Creativity emerges from the co-constitution between artists, robots, audience, and environment in spatial-temporal dimensions, as illustrated in \autoref{PracticeDiagram}. Acknowledging the audience as an important actor reflects the social dimension in that creativity does not stem from the artists but from their interactions with the audience. Robots are the major material and technological actants characterizing creative practices, shaping the conditions for creativity to emerge. The axis of the temporal process signifies that the practice is situated within a time continuum, and the actors/actants and their relations shift over time. In this way, temporality constitutes another dimension of creativity in robotic art.

Accordingly, as the second contribution, this study outlines implications for \textit{socially informed}, \textit{material-attentive}, and \textit{process-oriented} creation with computing systems\footnote{For the sake of clarity, we mean ``creation with computing systems'' by three types of scenarios: human creator(s) create computing system(s) as the final artifact(s) (e.g., robots are artworks themselves); human creator(s) use computing system(s) to create the artifact(s) (e.g., robots create artworks as human planned); and human creator(s) and system(s) work in tandem to produce the artifact(s) (e.g., human-robot co-creation).} to facilitate creation practices. These insights can inform related HCI research on media/art creation, crafting, digital fabrication, and tangible computing.
In each following subsection, we present each implication with a grounding in corresponding findings from our study and relevant literature in HCI and adjacent fields on art, creativity, and creation.

\begin{figure*}[htbp]
    \centering
    \includegraphics[width=0.88\textwidth]{Writings/figure/PracticeDiagram.pdf}
    \caption{Actors/actants in robotic art practice and their interactive relations. Robotic art practice unfolds primarily in two spaces: the creation space where interactions happen mainly between artists and robots, and the exhibition space where interactions mostly involve audiences and robots. The two spaces constitute the ENVIRONMENT plane. Within the plane, directed arrows between the actors indicate the types of interaction. For example, the \textit{Design} arrow indicates that the artist designs the robot(s), and the \textit{Revise} arrow indicates that the robot(s) make the artist revise artistic ideas. All the actors/actants may also intra-act with the ENVIRONMENT. The actors/actants and their interactive relations may differ at different times along the axis of TEMPORAL PROCESS that is orthogonal to the plane.}
    \Description{This figure shows the actors/actants in robotic art practice and their interactive relations.}
    \label{PracticeDiagram}
\end{figure*}

\subsection{Socially Informed Creation}

% Introduce social aspect of distributed creativity
The sociality of creativity means that creativity is distributed among different human actors, commonly within the creators or between the creators and the audience. Glăveanu’s ethnographic study on Easter egg decoration in northern Romania~\cite{glaveanu_distributed_2014} showed that artisans anticipate how others might appreciate their work and adjust their creative decisions accordingly. Even in the absence of direct interaction, the audience’s potential responses become part of the creative process, as artisans imagine feedback and predict reactions. In this sense, the sociologist Katherine Giuffre argues that ``\textit{creative individuals are embedded within specific network contexts so that creativity itself, rather than being an individual personality characteristic is, instead, a collective phenomenon}''~\cite[p. 1]{giuffre2012collective}.

% Recall findings about audience feedback
We found that the practice of robotic art manifests this sociality as it involves, particularly artists and audiences. 
Our artists take audiences' reactions to their artwork as feedback and then revise the robots' functions and aesthetics accordingly. 
For example, as shown earlier, Robert added a protective fuse onto his robot because he expected that children would squeeze the springs together and cause a short circuit; Alex's enthusiasm and attention to the audience's imagination about his robots led him to new aesthetic designs of both the robots and the scene layouts. The artists may directly ask about the audience's judgment of quality but they often receive feedback just by observing the audience's reactions or sometimes by learning from the audience's imagination about the robots.
% Recall findings about audience's sociocultural expectations and codes
Meanwhile, our findings reveal that audience reception is not an individual matter but is often associated with their sociocultural codes, including shared cultural norms, beliefs, expectations, and aesthetic values. The audience can be seen as representatives of these broader cultural codes. For example, Mark and Robert observed that the animist tendency in some East Asian societies is associated with higher acceptance of and interest among the audience in intelligence and agency of robots and non-human entities. Such sociocultural contexts influence not only how audiences interpret the work but also how artists anticipate and respond to these perspectives in their creative process.

% Situate in HCI literature
A creative process, including creation and reception, is essentially a social activity. The second wave of creativity research in psychology has argued for creativity's dependency on sociocultural settings and group dynamics~\cite{sawyer2024explaining}. Recent discussions from creativity-support and social computing researchers also called for more attention to the social aspect of creativity~\cite{kato2023special, fischer2005beyond, fischer2009creativity}. There is a clear need to consider the audience when producing creative content. For instance, researchers studying video-creation support have examined audience preferences to inform system designs that align with these preferences~\cite{wang2024podreels}. Such work highlights how creative activities extend beyond individual creators to co-creators and heterogeneous audiences. Some HCI researchers conceptualize creativity as by large a socially constructed concept, perceived and determined by social groups~\cite{fischer2009creativity}. 
Prior HCI work examined the social aspects between art creators. For example, creators and performers in music and dance form social relationships through artifacts, making the final work a collaborative outcome~\cite{hsueh2019deconstructing}. There is also a system designed to support collaborative creation between artists~\cite{striner2022co}. However, the social creative process between creators and audience is less articulated in HCI. Jeon et al.'s work~\cite{jeon2019rituals} stands as an exception, suggesting that professional interactive art can involve evaluation with the audience in the creation stage. 
Another relevant approach in HCI involves enabling the general public to participate in co-creation alongside professional creators. ~\citet{matarasso2019restless}, for instance, promoted ``participatory art'' as ``\textit{the creation of an artwork by professional artists and non-professional artists working together}'' with non-professional artists referring to the general public engaged in the art-making process. Similarly, socially inclusive community-based art also considers target communities' perception of the artwork during creation~\cite{clark2016situated, clarke2014socially}. But like participatory design~\cite{schuler1993participatory}, these art projects aim for social justice more than creativity in the work~\cite{murray2024designing}, let alone that direct participation in art creation is not always feasible. Our findings suggest that feedback from the audience can lead to creative ideas, as well as that the feedback can be generative and remain low-effort for the audience.

Unlike conventional design feedback---which is typically expected to be specific, justified, and actionable~\cite{yen2024give, krishna2021ready}---the feedback that resonates with our artists is often implicit, creative, and generative. Such feedback may include audiences' imaginations stimulated by the work, personal and societal reflections, and even emotions, facial expressions, micro-actions, and observable behaviors following the art experience. Our artists gathered this implicit feedback not by posing evaluative questions, as commonly done in typical design processes (e.g., usability testing, think-aloud protocols), which seek to elicit clear, relatively structured responses. Instead, they closely observe the audience's reactions and interpret their subjective perceptions. This form of implicit feedback, while indirect, can lead to more creative ideas by embracing open, multifaceted interpretations of the work~\cite{sengers2006staying}. Computing systems for creation should better incorporate implicit feedback in addition to explicit ones from the audience into the creation process. Implicit feedback can be indirect, creative, inspirational, and heuristic about functions and aesthetics. A hypothetical instance of such design can be a system that helps creators perceive audiences' implicit reactions and perceptions and variously interpret them, for further iteration.

% Recall findings about audience interacting with robots as a performative art
Moreover, as seen in Robert and Daniel's experiences, the audience may participate in robotic live performances by interacting with the robots, who may change actions accordingly, triggering a loop of simultaneous mutual influence that makes the work performative and improvisational.
% Situate in HCI
HCI researchers explored performative and improvisational creation with machines, focusing on developing and evaluating systems with performative capabilities, including music improvisation with robots~\cite{hoffman2010shimon}, dance with virtual agents~\cite{jacob2015viewpoints, triebus2023precious}, and narrative theatre~\cite{magerko2011employing, piplica2012full}. \citet{kang2018intermodulation} discussed the improvisational nature of interactions between humans and computers and argued that an HCI researcher-designers' improvisation with the environment facilitates the emergence of creativity and knowledge. Designs of computing systems for creation can leverage performativity in service of creative experience. One possible direction could be to allow the audience to embed themselves in and interact with elements of static artwork in a virtual space, turning the exhibition into an improvisational on-site creation~\cite{zhou2023painterly}.
% Our new implication different from current discussion on perf and impr
While interactions with machines during performance are mostly physical or embodied, we posit that they can also be a \textit{symbolic engagement}. Alex's audience projected themselves and their personalities onto his robots, which established a symbolic relevance, generating creative imaginations. During exhibitions, East Asian audiences carried the animist views shaped by their sociocultural backgrounds, and robots, through the performance, were successful in symbolically matching the views, stimulating aesthetic satisfaction. Symbolic engagement resonates with what ~\citet{nam2014interactive} called the ``reference'' of the interactive installation performance to participants' sociocultural conditions.
As such, we propose that designers of computing systems for creation may consider establishing symbolic engagement between the produced artifacts and the audience as a way to enhance perceived creativity or enrich the creative experience. One example is an interactive installation, \textit{Boundary Functions}~\cite{snibbe1998}, which encourages viewers to reflect on their personal spaces while interacting with the installation and others. Another example is \textit{Blendie}, a voice-controlled blender that requires a user to ``speak'' the machine's language to use it. This interaction builds a symbolic connection between the user and the device, transforming the act of blending into a novel experience~\cite{dobson2004blendie}.


\subsection{Material-Attentive Creation}

% Intro paragraph to the importance of materiality for creative activities with machines and the end goal of this discussion--- design suggestions
The theory of distributed creativity by Glaveanu claims that creativity distributes across humans and materials, so the creation practice itself is inevitably shaped by objects~\cite{glaveanu_distributed_2014}. In his case of Easter egg decoration, materials are not passive objects but active participants in artistic creation; e.g., the egg decorators face challenges from color pigments not matching the shell, wax not melted at the desired temperature, to eggs that break at the last step of decoration; hence, materials often go against the decorators' intentions and influence future creative pathways~\cite{glaveanu_distributed_2014}.
Materials manifest specific properties, which afford certain uses of the materials while constraining others~\cite{leonardi2012materiality}. Our findings highlight the critical role of materiality in artistic practice, showing that artists intentionally arrange materials to enhance the creative values of their work.

% Materiality aspect One: physicality and embodiment
% Embodiment or physicality fascilitates creative interaction with machines
Robotic art relies on the material properties of robots and other objects. An apparent property of most materials is their physicality~\cite{leonardi2012materiality}, meaning they possess a tangible presence that enables interaction with other physical entities. Here, we consider physicality and embodiment interchangeable as computational creativity researchers have conceptualized~\cite{guckelsberger2021embodiment}.
% Recall findings on embodiment's value in making art
Our findings support both the conceptual and operational contributions of embodiment for creative activities. For the conceptual aspect, the embodied presence of robotic systems supports creative thinking for our artists, exemplary in Linda's case where she found new art ideas around the difference between human and robot bodies through bodily engagement with robots. 
For the operational aspect, the embodied nature of robotic artworks and their creation processes exhibit original aesthetics that are based on physics much different from disembodied works, e.g., embodied drawings by David's non-industrial robotic arms are dynamic due to physical movements and thus artistically pleasant, which is hard to replicate in simulated programs.

% References: embodied interaction, embodied cognition theories, tangible computing
These findings on embodiment of robotic art (Section \ref{f:emb}) closely relate to HCI's attention on embodied interaction as a way to leverage human bodies and environmental objects to expand disembodied user experiences. 
For example, as~\citet{hollan2000distributed} explained, a blind person's cane and a cell biologist's microscope as embodied materials are part of the distributed system of cognitive control, showing that cognition is distributed and embodied. 
Similarly, theories of embodied interaction in HCI explicate how bodily interactions shape perception, experience, and cognition~\cite{marshall2013introduction, antle2011workshop, antle2009body}, backed up by the framework of 4E cognition (embodied, embedded, enactive, and extended)~\cite{wheeler2005reconstructing, newen20184E}. 
Prior works suggest that creative activities with interactive machines rely on similar embodied cognitive mechanisms ~\cite{guckelsberger2021embodiment, malinin2019radical}, which are operationalized by tangible computing~\cite{hornecker2011role}. 
% References: embodiment's consequence in creation
As related to robots in creation, HCI researchers show that physicality or embodiment of robots in creation may lead to some beneficial outcomes, such as curiosity from the audience, feelings of co-presence, body engagement, and mutuality, which are hard to simulate through computer programs~\cite{dell2022ah, hoggenmueller2020woodie}. Embodied robotic motions convey emotional expressions and social cues that potentially enrich and facilitate creation activities like drawings~\cite{ariccia2022make, grinberg2023implicit, dietz2017human, santos2021motions}. Guckelsberger et al.~\cite{guckelsberger2021embodiment} showed in their review that embodiment-related constraints (e.g., the physical limitations of a moving robotic arm) can also stimulate creativity. These constraints push creators to develop new and useful movements, echoing the broader principle that encountering obstacles in forms or materials can lead to generative processes. This phenomenon is similarly observed in activities such as art and digital fabrication~\cite{devendorf2015being, hirsch2023nothing}. In co-drawing with robots, physical touch and textures of drawing materials made the artists prefer tangible mediums (e.g., pencils) than digital tools (e.g., tablets) that fall short in these respects~\cite{jansen2021exploring}.

% Transit to materiality aspect two
% Materiality aspect Two: malfunction as manifestation of unique materiality of robots
% Intro to materials of robots
Materiality plays a crucial role in the embodiment of robots, as the choice of materials fundamentally shapes the physical forms and properties. This focus on materials extends to art practices, where robots made with soft materials introduce new aesthetics and sensory experiences~\cite{jorgensen2019constructing, belling2021rhythm}, and the use of plants and soil in robotic printing creates unique visual effects~\cite{harmon2022living}. Following Leonardi's ~\cite{leonardi2012materiality} conceptualization of materiality, we refer to the materials of robots as encompassing physical and digital components---including the shell, hardware, mechanical parts, software, programs, data, and controllers---each significant to the artist's intent. ~\citet{nam2023dreams} found that the material constraints of robots can limit creative expression but simultaneously stimulate creativity when artists push the boundaries.

%-----maybe here the real "malfuction" start ------------------
% Move to introduce malfunctions as unique materiality

Even carefully designed, digital and mechanical components in robots are prone to errors or bugs in everyday runs, causing malfunctions or unexpected consequences. This reflects the unique materiality of robots as complex computing systems. From an engineering perspective, errors signal unreliability and must be eliminated, driving advancements in robotics---where error detection and recovery are central~\cite{gini1987monitoring}---as well as in digital fabrication, which prioritizes precision over creative exploration~\cite{yildirim2020digital}. % Recall findings on embracing malfunctions
However, material failures and accidents are inevitable, exemplifying what has been called the `craftsmanship of risk'~\cite{glaveanu_distributed_2014} in material art. For our artists, these risks are often creatively utilized and incorporated into their work: these moments of breakdown---whether physical or digital---become resources for new creative expression. Errors are anticipated and intentionally designed into the process and work of our artists. In some cases, such as for Alex, the entire concept of one of his works is machine errors.

% Situate in literature
Reports on how artists view errors within engineering and creation processes are dispersed throughout HCI literature. ~\citet{nam2023dreams} showed that the accumulation of ``contingency'' and ``accidents''---unexpected, serendipitous, and emergent events during art creation like errors---meaningfully constituted the final presentation of the artwork. Song and Paulos's concept of ``unmaking'' highlighted the values of material failures in enabling new aesthetics and creativity~\cite{song2021unmaking}. Kang et al.~\cite{kang2022electronicists, kang2023lady} introduced the notion of an ``error-engaged studio'' for design research in which errors in creative processes are identified, accommodated, and leveraged for their creative potential. Collectively, these works advocate for reframing errors from something to avoid to something to embrace and recognize. We want to push this further by arguing that errors can be intended and be part or sometimes entire of the design. Several artists, including participants from our study, have been deliberately seeking errors to formulate their designs. Roboticist Damith Herath recounted when he mistakenly programmed a motion sequence of a robotic arm, his collaborator, robotic artist Stelac responded with ``[W]e need to make more mistakes;'' as many mistakes were made, the initial pointless movements became beautiful, rendering the robot ``alive'' and ``seductive'' \cite{herath2016robots}. Similarly, AI artists sometimes look for program glitches to generate unusual styles and content~\cite{chang2023prompt}. Therefore, creators may not only passively accept errors but can actively seek and utilize them. Errors can be integral to the design itself---errors can \textit{be designed into} an artifact, and the design/idea of the artifact can be all about errors.

Thus, to focus on material-attentive creation---considering the creative arrangement of materials---we suggest exploring the embodiment and materiality of creation materials, objects, and environments to recognize their creative potential. %This perspective aligns with insights from professional digital fabrication practitioners, who advocate for systems that integrate support for machine settings and material properties~\cite{hirsch2023nothing}.
Specifically, we propose using a design method/probe that enables creators to realize both the conceptual and operational contributions of materiality. This approach may build on the material probe developed by~\citet{jung2010material}, which calls for exploring the materiality of digital artifacts. A material-attentive probe would enable creators to engage with diverse materials, objects, and environments through embodied interaction, encouraging them to speculate on material preferences and limitations, and to compare and contrast material qualities---insights that can inform creative decisions.
To accommodate, seek, and actively harness the creative potential of errors, we propose embracing failures, glitches, randomness, and malfunctions in computing systems as critical design materials---elements that creators can intentionally control and manipulate. By doing so, we can begin to systematically approach errors. For instance, as part of the design process, we may document how to replicate these errors and changes, allowing creators to explore them further at their discretion. This could include intentionally inducing errors or random changes to influence the creative process or outcomes.

\subsection{Process-Oriented Creation}

% Introduce the key idea: process itself embeds creative value and can be pursued as the goal of creation
As shown in our findings, the creation process itself embeds creative values and meanings, and experiencing the process can be pursued as the goal of creation with computing systems.
% Recall findings
For the robotic artists in our study, artistic values were often placed on the creation process rather than the outcome.  For example, in Alex's robotic live drawing performance, the drawing process is more important than the drawn pattern on canvas. Techniques used, decisions made, or stimuli received by robots during creation or exhibition reflect artistic ideas and nuanced thinking, as seen in Sophie's exploration of interactive decision-making in robotic drawing.

% Situate in HCI lit
Previous HCI work has touched on the value of the process of creation. ~\citet{bremers2024designing} shared a vignette where a robotic pen plotter simultaneously imitates the creator's drawing, serving as a material presence rather than a pragmatic co-creator; here the focus of the work is no longer the outcome but the process of drawing itself. ~\citet{devendorf2015reimagining} concluded that performative actions of digital fabrication systems, rather than the fabricated products themselves, convey artistic meanings tied to histories, public spaces, time, environments, audiences, and gestures. This emphasis on process is particularly significant for media such as improvisational theatre, where the creation itself is an integral part of the final work~\cite{o2011knowledge}. ~\citet{davis2016empirically} named their improvisational co-drawing robotic agents as ``casual creators,'' who are meant to creatively engage users and provide enjoyable creative experiences rather than necessarily helping users make a higher quality product. Shifting the focus from product to process and experiences \textit{in} creation may generate alternative creative meanings.

% Findings about process extends beyond creation
Our artists pointed out that even a ``finished'' artwork in an exhibition is not truly finished. A crack in Daniel's robotic artwork introduced a new artistic meaning, ultimately subverting the entire work. As the properties of the work change over time---whether due to the artist's intent, material characteristics, or environmental factors---the artwork evolves, revealing new aesthetics and meanings. % Situate in HCI lit
Based on these observations, we argue that creation processes should not be regarded as one-shot transactions, as creative artifacts, particularly physical ones, continue to change and generate artistic values. For instance, material wear and destruction bring unique aesthetics, often contrasting with the original form ~\cite{zoran2013hybrid}, and are seen as signs of mature use~\cite{giaccardi2014growing}.
Changes such as material failure, destruction, decay, and deformation---what~\citet{song2021unmaking} referred to as ``unmaking,'' a process that occurs after making---meaningfully transforms the original objects. Similarly, through Broken Probes, a process of assembling fractured objects, ~\citet{ikemiya2014broken} demonstrated that personal connections, reminiscence, and reflections related to material wear and breakage add new values to the objects. Drawing from Japanese philosophy Wabi-Sabi, ~\citet{tsaknaki2016expanding} reflected on the creeds of `Nothing lasts,' `Nothing is finished,' and `Nothing is perfect' and pointed to the impermanence, incompleteness, and imperfection of artifacts as a resource that designers, producers, and users can utilize to achieve long-term, improving, and richer interactive experience~\cite{tsaknaki2016things}. Insights from this study contribute to this line of thought by showing how robotic artists appreciate the aesthetics and meanings of temporal changes after the creation phase.

The findings underscore the need to reconceptualize creation as encompassing more than just the process aimed at producing a final product; it also includes what we term \textit{post-creation}. Distinct from repair, maintenance, or recycle, \textit{post-creation} entails anticipating and managing how an artifact evolves after its ``completion'' in the conventional sense. Specifically, we encourage creators to anticipate and strategically engage with the post-creation phase, considering potential changes to the artifact and their consequences for interactions with human users. For instance, during the creation process, creators may focus on possible material changes the artifact might undergo post-creation, allowing them to either mitigate or creatively exploit these potential changes. This expanded view of creation invites us to trace post-creation developments and to plan how our creative intentions can be embedded in its potential degradation, transformation, or evolution over time.

% A conclusion paragraph
We categorize the design implications into three aspects, but we do not suggest that a computing system must implement all simultaneously, nor that each aspect should be considered in isolation. Social interactions, such as those between artists and audiences, already presume the presence of material actants like robots, and these interactions inform future arrangements of materials. Thus the social and material aspects can be entangled and mutually constitutive as seen in sociomaterial practices~\cite{orlikowski2007sociomaterial, cheatle2015digital, rosner2012material}. The temporal aspect is orthogonal to the other aspects because both social interactions and material manifestations unfold and shift in a temporal continuum.


\section{Conclusion }
This paper introduces the Latent Radiance Field (LRF), which to our knowledge, is the first work to construct radiance field representations directly in the 2D latent space for 3D reconstruction. We present a novel framework for incorporating 3D awareness into 2D representation learning, featuring a correspondence-aware autoencoding method and a VAE-Radiance Field (VAE-RF) alignment strategy to bridge the domain gap between the 2D latent space and the natural 3D space, thereby significantly enhancing the visual quality of our LRF.
Future work will focus on incorporating our method with more compact 3D representations, efficient NVS, few-shot NVS in latent space, as well as exploring its application with potential 3D latent diffusion models.


\bibliographystyle{ACM-Reference-Format}
\bibliography{ref}

\appendix
\section{Additional Related Works}
\label{sec:app-add-rel-works}
\subsection{Training Data Selection}

\begin{figure*}[!ht]
    \centering
    \includegraphics[width=\textwidth]{figs/per-token-loss-diff.pdf}
    \caption{Histograms of MIA signal of tokens. Each figure depicts a sample. Blue means the member samples while orange represents the non-member samples. We limited the y-axis range to -3 to 3 for better visibility, so it can result in missing several non-significant outliers.}
    \label{fig:add-per-token-loss}
\end{figure*}

Training data selection are methods that filter high-quality data from noisy big data \textit{before training} to improve the model utility and training efficiency. There are several works leveraging reference models~\cite{Coleman2020Selection, xie2023doremi}, prompting LLMs~\cite{li-etal-2024-one}, deduplication~\cite{lee2022deduplicating, kandpal2022deduplicating}, and distribution matching~\cite{kang2024get}. However, we do not aim to cover this data selection approach, as it is orthogonal and can be combined with ours.


\subsection{Selective Training}
Selective training refers to methods that \textit{dynamically choose} specific samples or tokens \textit{during training}. Selective training methods are the most relevant to our work. Generally, sample selection has been widely studied in the context of traditional classification models via online batch selection~\cite{loshchilov2016o, Angelosonl, pmlr-v108-kawaguchi20a}. These batch selection methods replace the naive random mini-batch sampling with mechanisms that consider the importance of each sample mainly via their loss values. ~\citet{2022PrioritizedTraining} indeed choose highly important samples from regular random batches by utilizing a reference model. However, due to the sequential nature of LLMs, which makes the training significantly different from the traditional classification ML, sample-level selection is not effective for language modeling~\cite{kaddour2023no}. \citet{lin2024not} extend the reference model-based framework to select meaningful tokens within batches. All of the previous methods for selective training aim to improve the training performance and compute efficiency. Our work is the first looking at this aspect for defending against MIAs.

\section{Token-level membership inference risk analysis}
Figures~\ref{fig:add-per-token-loss} and~\ref{fig:add-per-token-dynamics} present the analysis for additional samples. Generally, the trends are consistent with the one presented in Section~\ref{sec:analysis}.

\begin{figure*}[!ht]
    \centering
    \includegraphics[width=0.28\textwidth]{figs/mia-ranking_1.png}
    \includegraphics[width=0.28\textwidth]{figs/mia-ranking_2.png}
    \includegraphics[width=0.3\textwidth]{figs/mia-ranking_3.png}    
    \caption{MIA signal ranking of tokens during training. Each figure illustrates a sample.}
    \label{fig:add-per-token-dynamics}
\end{figure*}

\label{sec:app-analysis}

\section{Experiment settings}
\subsection{Implementation details}
\label{sec:app-implementation}
$\bullet$ \textbf{FT}. We implement the conventional fine tuning using Huggingface Trainer. We manually tune the learning rate to make sure no significant underfitting or overfitting. The batch size is selected appropriately to fit the physical memory and comparable with the other methods'.

\noindent $\bullet$ \textbf{Goldfish}. Goldfish is also implemented with Huggingface Trainer, where we custom the \texttt{compute\_loss} function. We implement the deterministic masking version rather than the random masking to make sure the same tokens are masked over epochs, potentially leading to better preventing memorization. The learning rate is also manually tuned, we noticed that the optimal Goldfish learning rate is usually slightly greater than FT's. This can be the gradients of two methods are almost similar, Goldfish just removes some tokens' contribution to the loss calculation. The batch size of FT can set as the same as FT, as Goldfish does not have significant overhead on memory.

\noindent $\bullet$ \textbf{DPSGD}. DPSGD is implemented by FastDP~\cite{bu2023zero}. We implement DPSGD with fastDP~\cite{bu2023zero} which offers state-of-the-art efficiency in terms of memory and training speed. We also use automatic clipping~\cite{bu2023automatic} and a mixed optimization strategy~\cite{mixclipping} between per-layer and per-sample clipping for robust performance and stability.

\noindent $\bullet$ \textbf{\methodname}. We implement \methodname using Huggingface Trainer, same as FT and Goldfish. The learning is reused from FT. The batch size of \methodname is usually smaller than FT and Goldfish when the model becomes large such as Pythia and Llama 2 due to the reference model, which consumes some memory.

For a fair comparison, we aim to implement the same batch size for all methods if feasible. In case of OOM (out of memory), we perform gradient accumulation, so all the methods can have comparable batch sizes. We provide the hyper-parameters of method for GPT2 in Table~\ref{tab:hyperparameter}. For Pythia and Llama 2, the learning rate, batch size, and number of epochs are tuned again, but the hyper-parameters regarding the privacy mechanisms remain the same. To make sure there is no naive overfitting, we evaluate the methods by selecting the best models on a validation set. Moreover, the testing and attack datasets remains identical for evaluating all methods. Additionally, we balance the number of member and non-member samples for MIA evaluation. It is worth noting that for the ablation study and analysis, if not state, the default model architecture and dataset are GPT2 and CC-news.

\begin{table*}[!ht]
    \centering
    \begin{tabular}{c|clc}
    \textbf{LLM} & \textbf{Method} & \textbf{Hyper-parameter} & \textbf{Value}  \\ \hline
     \multirow{22}{*}{\textbf{GPT2}}  &  \multirow{4}{*}{FT} &  Learning rate & 1.75e-5 \\ 
     & & Batch size & 96 \\
     & & Gradient accumulation steps & 1 \\
     & & Number of epochs & 20 \\ \cline{2-4}
       &  \multirow{5}{*}{Goldfish} &  Learning rate & 2e-5 \\ 
     & & Batch size & 96 \\
     & & Grad accumulation steps & 1 \\
     & & Number of epochs & 20 \\
     & & Masking Rate & 25\% \\ \cline{2-4}
           &  \multirow{6}{*}{DPSGD} &  Learning rate & 1.5e-3 \\ 
     & & Batch size & 96 \\
     & & Grad accumulation steps & 1 \\
     & & Number of epochs & 10 \\
     & & Clipping & automatic clipping \\ 
     & & Privacy budget & (8, 1e-5)-DP \\ \cline{2-4}
           &  \multirow{6}{*}{DuoLearn} &  Learning rate & 1.75e-3 \\ 
     & & Batch size & 96 \\
     & & Grad accumulation steps & 1 \\
     & & Number of epochs & 20 \\
     & & $K_h$ & 60\% \\ 
     & & $K_m$ & 20\% \\
     & & $\tau$ & 0 \\
     & & $\alpha$ & 0.8 \\ \hline
    \end{tabular}
    \caption{Hyper-parameters of the methods for GPT2.}
    \label{tab:hyperparameter}
\end{table*}


\section{Additional Results}
\label{sec:app-add-res}

\begin{figure}[!ht]
    \centering
    \includegraphics[width=0.8\linewidth]{figs/add_loss_vs_steps_ft_duolearn.pdf}
    \caption{Breakdown to the cross entropy loss values of FT on the testing set and \methodname on the training set during training.}
    \label{fig:add-overlap-breakdown}
\end{figure}

\subsection{Overall Evaluation}
% \begin{table*}[htp]
%     \centering
%     \begin{tabular}{cl|ccccc|ccccc}
%      \multirow{3}{*}{\textbf{LLM}}  & \multirow{3}{*}{\textbf{Method}} &  \multicolumn{5}{c|}{\textbf{CCNews}} & \multicolumn{5}{c}{\textbf{Wikipedia}} \\ \cmidrule(lr){3-7}  \cmidrule(lr){8-12}
%       &  & PPL & Loss & Ref & min-k & \multicolumn{1}{c|}{zlib} & PPL & Loss & Ref & min-k & zlib \\ \midrule
%       \multirow{4}{*}{GPT2} & \textit{Base} & \textit{29.442} & \textit{0.018} & \textit{0.002} & \textit{0.022} & \textit{0.006} & \textit{34.429} & \textit{0.002} & \textit{0.014} & \textit{0.010} & \textit{0.002} \\ 
%       \multirow{4}{*}{124M} & FT & \textbf{21.861} & 0.030 & 0.026 & 0.016 & 0.016 & \textbf{12.729} & 0.018 & 0.574 & 0.016 & 0.014 \\
%       & Goldfish & 21.902 & 0.030 & 0.024 & 0.028 & 0.016 & 12.853 & 0.018 & 0.632 & 0.016 & 0.010 \\
%       & DPSGD & 26.022 & \textbf{0.018} & \textbf{0.004} & \textbf{0.018} & 0.008 & 18.523 & \textbf{0.004} & 0.036 & 0.018 & 0.006 \\
%       & \methodname & 23.733 & 0.030 & 0.022 & 0.026 & \textbf{0.006} & 13.628 & 0.014 & \textbf{0.010} & \textbf{0.014} & \textbf{0.004} \\ \midrule
      
%       \multirow{4}{*}{Pythia} & \textit{Base} & \textit{13.973} & \textit{0.002} & \textit{0.008} & \textit{0.020} & \textit{0.014} & \textit{10.287} & \textit{0.002} & \textit{0.014} & \textit{0.006} & \textit{0.008} \\ 
%       \multirow{4}{*}{1.4B} & FT & 11.922 & 0.014 & 0.008 & 0.022 & 0.020 & \textbf{6.439} & 0.020 & 0.440 & 0.010 & 0.020 \\
%       & Goldfish & \textbf{11.903} & 0.014 & 0.008 & 0.024 & 0.018 & 6.465 & 0.016 & 0.412 & 0.010 & 0.020 \\
%       & DPSGD & 13.286 & \textbf{0.002} & \textbf{0.004} & \textbf{0.018} & \textbf{0.014} & 7.751 & \textbf{0.004} & \textbf{0.016} & {0.010} & \textbf{0.004} \\
%       & \methodname & 12.670 & 0.004 & 0.020 & \textbf{0.018} & 0.016 & 6.553 & 0.008 & 0.030 & \textbf{0.006} & 0.006 \\ \midrule
      
%       \multirow{4}{*}{Llama-2} & \textit{Base} & \textit{9.364} & \textit{0.006} & \textit{0.006} & \textit{0.024} & \textit{0.006} & \textit{7.014} & \textit{0.006} & \textit{0.016} & \textit{0.016} & \textit{0.010} \\ 
%       \multirow{4}{*}{7B} & FT & \textbf{6.261} & 0.002 & 0.018 & 0.002 & 0.002 & \textbf{3.830} & 0.028 & 0.170 & 0.030 & 0.028 \\
%       & Goldfish & 6.280 & 0.002 & 0.018 & 0.002 & 0.006 & 3.839 & 0.028 & 0.198 & 0.028 & 0.028 \\
%       & DPSGD & 6.777 & 0.008 & 0.026 & 0.016 & 0.010 & 4.490 & \textbf{0.006} & 0.014 & \textbf{0.020} & \textbf{0.010} \\
%       & \methodname & 6.395 & \textbf{0.002} & \textbf{0.020} & \textbf{0.004} & \textbf{0.002} & 4.006 & 0.010 & \textbf{0.002} & 0.028 & 0.012 \\ 
%     \end{tabular}
%     \caption{TPR at FPR of 1\% \textcolor{red}{TODO: check consistency with the main table of MIA AUC scores}}
%     \label{tab:tpr}
% \end{table*}


\begin{table*}[!ht]
  \centering
  \resizebox{\textwidth}{!}{\begin{tabular}{cl|ccccc|ccccc}
   \multirow{3}{*}{\textbf{LLM}}  & \multirow{3}{*}{\textbf{Method}} &  \multicolumn{5}{c|}{\textbf{Wikipedia}} & \multicolumn{5}{c}{\textbf{CC-news}} \\ \cmidrule(lr){3-7}  \cmidrule(lr){8-12}
    &  & PPL & Loss & Ref & min-k & \multicolumn{1}{c|}{zlib} & PPL & Loss & Ref & min-k & zlib \\ \midrule
    \multirow{4}{*}{GPT2} & \textit{Base} & \textit{34.429} & \textit{0.002} & \textit{0.014} & \textit{0.010} & \textit{0.002} & \textit{29.442} & \textit{0.018} & \textit{0.002} & \textit{0.022} & \textit{0.006} \\ 
    \multirow{4}{*}{124M} & FT & \textbf{12.729} & 0.018 & 0.574 & 0.016 & 0.014 & \textbf{21.861} & 0.030 & 0.026 & 0.016 & 0.016 \\
    & Goldfish & 12.853 & 0.018 & 0.632 & 0.016 & 0.010 & 21.902 & 0.030 & 0.024 & 0.028 & 0.016 \\
    & DPSGD & 18.523 & \textbf{0.004} & 0.036 & 0.018 & 0.006 & 26.022 & \textbf{0.018} & \textbf{0.004} & \textbf{0.018} & 0.008 \\
    & \methodname & 13.628 & 0.014 & \textbf{0.010} & \textbf{0.014} & \textbf{0.004} & 23.733 & 0.030 & 0.022 & 0.026 & \textbf{0.006} \\ \midrule
    
    \multirow{4}{*}{Pythia} & \textit{Base} & \textit{10.287} & \textit{0.002} & \textit{0.014} & \textit{0.006} & \textit{0.008} & \textit{13.973} & \textit{0.002} & \textit{0.008} & \textit{0.020} & \textit{0.014} \\ 
    \multirow{4}{*}{1.4B} & FT & \textbf{6.439} & 0.020 & 0.440 & 0.010 & 0.020 & 11.922 & 0.014 & 0.008 & 0.022 & 0.020 \\
    & Goldfish & 6.465 & 0.016 & 0.412 & 0.010 & 0.020 & \textbf{11.903} & 0.014 & 0.008 & 0.024 & 0.018 \\
    & DPSGD & 7.751 & \textbf{0.004} & \textbf{0.016} & {0.010} & \textbf{0.004} & 13.286 & \textbf{0.002} & \textbf{0.004} & \textbf{0.018} & \textbf{0.014} \\
    & \methodname & 6.553 & 0.008 & 0.030 & \textbf{0.006} & 0.006 & 12.670 & 0.004 & 0.020 & \textbf{0.018} & 0.016 \\ \midrule
    
    \multirow{4}{*}{Llama-2} & \textit{Base} & \textit{7.014} & \textit{0.006} & \textit{0.016} & \textit{0.016} & \textit{0.010} & \textit{9.364} & \textit{0.006} & \textit{0.006} & \textit{0.024} & \textit{0.006} \\ 
    \multirow{4}{*}{7B} & FT & \textbf{3.830} & 0.028 & 0.170 & 0.030 & 0.028 & \textbf{6.261} & 0.002 & 0.018 & 0.002 & 0.002 \\
    & Goldfish & 3.839 & 0.028 & 0.198 & 0.028 & 0.028 & 6.280 & 0.002 & 0.018 & 0.002 & 0.006 \\
    & DPSGD & 4.490 & \textbf{0.006} & 0.014 & \textbf{0.020} & \textbf{0.010} & 6.777 & 0.008 & 0.026 & 0.016 & 0.010 \\
    & \methodname & 4.006 & 0.010 & \textbf{0.002} & 0.028 & 0.012 & 6.395 & \textbf{0.002} & \textbf{0.020} & \textbf{0.004} & \textbf{0.002} \\ 
  \end{tabular}}
  \caption{Overall Evaluation: Perplexity (PPL) and TPR at FPR of 1\% scores of the MIAs with different signals (Loss/Ref/Min-k/Zlib). For all metrics, the lower the value, the better the result.}
  \label{tab:tpr}
\end{table*}
Table~\ref{tab:tpr} provides the True Positive Rate (TPR) at low False Positive Rate (FPR) of the overall evaluation. Generally, compared to CC-news, Wikipedia poses a significant higher risk at low FPR. For example, the reference-based attack can achieve a score of 0.57~ on GPT2 if no protection. In general, Goldfish fails to mitigate the risk in this scenario, while both DPSGD and \methodname offer robust protection.

\subsection{Auxiliary dataset}
We investigate the size of the auxiliary dataset which is disjoint with the training data of the target model and the attack model. In this experiment, the other methods are trained with 3K samples. Figure~\ref{fig:aux_size} presents the language modeling performance while varying the auxiliary dataset's size. The result demonstrates that the better reference model, the better language modeling performance. It is worth noting that even with a very small number of samples, \methodname can still outperform DPSGD. Additionally, there is only a little benefit when increasing from 1000 to 3000, this indicates that the reference model is not needed to be perfect, as it just serves as a calibration factor. This phenomena is consistent with previous selective training works~\cite{lin2024not, 2022PrioritizedTraining}.
\begin{figure}
    \centering
    \includegraphics[width=0.8\linewidth]{figs/auxiliary_size.pdf}
    \caption{Language modeling performance while varying the auxiliary dataset's size. Note that the results of FT and Goldfish are significantly overlapping.}
    \label{fig:aux_size}
\end{figure}

\subsection{Training time}
We report the training time for full fine-tuning Pythia 1.4B. We manually increase the batch size that could fit into the GPU's physical memory. As a results, FT and Goldfish can run with a batch size of 48, while DPSGD and \methodname can reach the batch size of 32. We also implement gradient accumulation, so all the methods can have the same virtual batch size.

\begin{table}[!ht]
    \centering
    \begin{tabular}{c|c}
        \textbf{Training Time} & \textbf{\textbf{1 epoch}} (in minutes) \\ \hline
        {FT} & 2.10 \\ 
        {Goldfish} & 2.10 \\
        % {RelaxLoss} & 2.10 \\        
        {DPSGD} & 3.19 \\ 
        {DuoLearn} & 2.85 
    \end{tabular}
    \caption{Training time for one epoch of (full) Pythia 1.4B on a single H100 GPU}
    \label{tab:training-time}
\end{table}

Table~\ref{tab:training-time} presents the training time for one epoch. Goldfish has little to zero overhead compared to FT. DPSGD and \methodname have a slightly higher training time due to the additional computation of the privacy mechanism. In particular, DPSGD has the highest overhead due to the clipping and noise addition mechanisms. Meanwhile, \methodname requires an additional forward pass on the reference model to select the learning and unlearning tokens. \methodname is also feasible to work at scale that has been demonstrated in the pretraining settings of the previous work~\cite{lin2024not}.

\section{Limitations}
The main limitation of our work is the small-scale experiment setting due to the limited computing resources. However, we believe \methodname can be directly applied to large-scale pretraining without requiring any modifications, as done in previous selective pretraining work~\cite{lin2024not}. Another limitation is the reference model, which may be restrictive in highly sensitive or domain-limited settings~\cite{tramr2024position}. From a technical perspective, while we show that \methodname performs well across different datasets and architectures, there is room for improvement. The current approach selects a fixed number of tokens, which may not be optimal since selected tokens contribute unequally. Future work could explore adaptive selection or weighted tokens' contribution. At a high-level, compared to DPSGD, \methodname has not been supported by theoretical guarantees. Future work can investigate the convergence and overfitting analysis.

\end{document}