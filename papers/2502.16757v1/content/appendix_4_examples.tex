\begin{tabular}{l l l l l l}
\toprule
\multirow{2}{*}{Task} & \multirow{2}{*}{Pool} & \multicolumn{4}{c}{Sequence} \\ \cmidrule(lr){3-6}
                      &                       & Prompt & Task & Completed & Combined \\
\midrule
Sorting & Int                     & 5 3 6      & S R A E    & 3 5 6         & Q 5 3 6 S R A E 3 5 6        \\
Adding  & Int                     & 5 3 6      & A E R S    & 6 4 7         & Q 5 3 6 A E R S 6 4 7        \\
Reverse Sorting  & Int                     & 5 3 6      & R E A S    & 6 5 3         & Q 5 3 6 R E A S 6 5 3        \\
Even-Odd  & Int                     & 5 3 6      &  E R A S    & 6 3 5         & Q 5 3 6 E R A S 6 3 5        \\
Sorting & Int + Char                     & 13 5 c a      & S E R A    & 13 5 a c & Q 13 5 c a S E R A        \\
Adding  & Int + Char                     & 13 5 c a      & A S R E    & 14 6 d b         & Q 13 5 c a A S R E        \\
Reverse Sorting  & Int + Char                     & 13 5 c a      & R E A S    & c a 5 13         & Q 13 5 c a R E A S c a 5 13        \\
Even-Odd  & Int + Char                     & 13 5 c a      & E S A R    & a c 13 5 & Q 13 5 c a E S A R 13 5 c a        \\

\bottomrule
\end{tabular}% 


\section{Examples}
\label{sec:appendix-examples}

Table \ref{tab:examples} shows the examples sampled from three datasets (EntailmentBank, eQASC, and e-SNLI), and shows how the FOL representations changed from \texttt{T5-Iter0} to \texttt{T5-Iter5}.

First, it is noticeable that the entailment-preserving FOL representations have more \textit{atomic} predicate names (\textit{e.g.} \texttt{GlowingBand}), compared to complex predicate names generated by \texttt{T5-Iter0} (\texttt{Appears...NightSky}). This corresponds to the analyses in Section \ref{sec:arbitrariness}, where the bootstrapping reduces the lexical diversity of the predicate names.

Furthermore, example \textbf{e-SNLI\_6529} shows that a new predicate that does not lexically match the original NL premise, $\text{Has}(x, y)$, was introduced by the model. This shows the potential of our method to overcome the \textit{brittleness} of FOL parses, where FOL parses omit underlying commonsense or pragmatic assumptions required for RTE tasks \citep{bos-nli}, by data-driven approaches in the EPF task.