\section{Prompts for GPT-4o(-mini)}
\label{sec:appendix-prompt}

Figure \ref{fig:prompts} describes the prompt used for \texttt{GPT-4o} and \texttt{GPT-4o-mini} baseline experiments. For both models, Five few-shot examples were used to perform in-context learning.

\newtcolorbox{fullwidthbox}[1][]{
    width=\textwidth, % Full page width
    left=0pt,         % No padding on the left
    right=0pt,        % No padding on the right
    boxrule=1pt,      % Border thickness
    sharp corners,    % Square corners
    #1                % Additional options
}

\begin{figure*}[ht]
\begin{fullwidthbox}[title=System prompt]
\small
You will see a natural language sentence. Translate it into first-order logic (FOL).

You MUST use a common set of predicates to represent the meaning in FOL format.

Below are instructions for the format of FOL logical formulas:

1. \textbf{Variables}: Use lowercase (\texttt{x}, \texttt{y}, etc.) for generic objects.

2. \textbf{Constants}: Use lowercase names (\texttt{john}, \texttt{sun}) for specific entities.

3. \textbf{Predicates}: Represent properties/relations as \texttt{Predicate(arg1, arg2)}, e.g., \texttt{Rises(sun)}, \texttt{Loves(john, mary)}.

4. \textbf{Connectives}:

   - \textbf{Negation (\texttt{-})}: Not, e.g., \texttt{-Rains(x)}
   
   - \textbf{Conjunction (\texttt{\&})}: And, e.g., \texttt{Walks(john) \& Talks(john)}
   
   - \textbf{Disjunction (\texttt{|})}: Or, e.g., \texttt{Walks(john) | Talks(john)}
   
   - \textbf{Implication (\texttt{->})}: If...then, e.g., \texttt{Rains(x) -> Wet(x)}
   
   - \textbf{Biconditional (\texttt{<->})}: If and only if, e.g., \texttt{Rains(x) <-> Wet(x)}
   
5. \textbf{Quantifiers}:

   - \textbf{Universal (\texttt{all})}: For all, e.g., \texttt{all x. (Human(x) >> Mortal(x))}
   
   - \textbf{Existential (\texttt{exists})}: There exists, e.g., `exists x. (Human(x) \& Smart(x))
\end{fullwidthbox}

\begin{fullwidthbox}[title=User prompt (5-shot)]
\small
Few-shot Example 1:

Sentence: All planets orbit a star.

FOL Translation:
\texttt{all x. (Planet(x) -> exists y. (Star(y) \& Orbits(x, y)))}

Few-shot Example 2:

Sentence: Mars is a planet.

FOL Translation:
\texttt{Planet(mars)}

Few-shot Example 3:

Sentence: If a person is a scientist and has access to a laboratory, they can conduct experiments.

FOL Translation:
\texttt{all x. ((Scientist(x) \& HasAccessToLab(x)) -> CanConductExperiments(x))}

Few-shot Example 4:

Sentence: Butterflies are insects.

FOL Translation:
\texttt{all y. (Butterfly(y) -> Insect(y))}

Few-shot Example 5:

Sentence: All insects that have wings can fly.

FOL Translation:
\texttt{all x. (Insect(x) \& HasWings(x) -> CanFly(x))}
\end{fullwidthbox}
\caption{System and user prompts used for GPT-4o and GPT-4o-mini.}
\label{fig:prompts}

\end{figure*}