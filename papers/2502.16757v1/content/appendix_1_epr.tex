\section{Details on Entailment-Preserving Rate (EPR)}
\label{sec:appendix-epr}

\subsection{Spurious entailment detection}

In first-order logic, a contradictory statement such as $P(a) \land \lnot P(a)$ can derive any FOL formula. Therefore, the notion of entailment-preserving, $FOL(p_1), ..., FOL(p_N) \vdash FOL(h)$, is susceptible to \textit{spurious entailments} where the hypothesis is derived only because there was a contradiction in the premises.

EPR evaluation imposes two verification steps for the external prover's results to filter out these spurious entailments and align the semantics of entailment-preserving to human instincts. First, the hypothesis must not introduce predicates and constants that were not present in the premises. Second, the automatically generated proof of the hypothesis must include all premises. These checks not only prevent trivial contradictions falsely entailing the hypothesis but also more complex cases, such as $P(c), \forall x(\lnot Q(x) \rightarrow (R(x) \land \lnot R(x))) \vdash Q(c)$ where $P(c)$ is not used to prove the conclusion because of an embedded contradiction in $\forall x(\lnot Q(x) \rightarrow (R(x) \land \lnot R(x)))$.

\subsection{EPR@K-Oracle}
\label{sec:appendix-oracle}

EPR@K-Oracle score is calculated by selecting one from the $K$ parses of each sentence that maximizes the overall EPR score. If all sentences are used only in a single premises-hypothesis pair, this score will be identical to EPR@K. However, as some sentences are included in multiple premises-hypothesis pairs, the EPR@K-Oracle score is generally lower than EPR@K.

This problem of finding the EPR@K-Oracle score is an instance of the constraint satisfaction problem that can be reduced to the maximum satisfiability problem, which is NP-complete. The problem can be formulated in three conditions:
\begin{itemize}
    \item For all boolean predicates $\texttt{fol}(i, j)$ that represent $S(p_i)_j$, only one $j$ from each $i$ can be selected. This condition can be expressed as $\forall i\ \exists! j\ \texttt{select}(i, j)$.
    \item If a $(j_1, ..., j_N, j)$ tuple satisfies the $b$-th premises-hypothesis pair in the dataset, \textit{i.e.} $S(p_1)_{j_1}, ..., S(p_N)_{j_N} \vdash S(h)_j$, $\texttt{select}(1, j_1) \land ... \land \texttt{select}(N, j_N) \land \texttt{select}(h, j)$ implies $\texttt{success}(b)$.
    \item The goal is to maximize the number of different \texttt{success(b)} that are true.
\end{itemize}

To solve this constraint optimization problem, we use Answer Set Programming \citep{asp} to formulate the problem and run an ASP solver, Clingo  \citep{clingo}. We impose a 600-second time limit\footnote{Clingo was executed on a single core of AMD EPYC-Milan. Search strategies and heuristics were set as default.} for each constraint satisfaction problem instance, and use the maximum number of $\texttt{success}(b)$ returned from the solver to calculate EPR@K-Oracle.