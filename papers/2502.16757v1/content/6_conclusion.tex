\section{Conclusion}

% \heng{future work - take more complex docs as input? where the logical reasoning needs to be done across sentences? like those in regulation guidelines?}
FOL representations provide an intuitive way to express logical entailment in NL. However, using FOL for checking natural language entailment is a highly complex task, where classic meaning representation-based \nltofol\ parsers and end-to-end generative models both suffer.

In this study, we formalize the Entailment-Preserving FOL representations (EPF) task and reference-free metrics for EPF. Furthermore, we provide an effective method for training an end-to-end generative \nltofol\ translator, \textit{iterative learning-to-rank}, which significantly outperforms baselines. These positive results shed light on a new data-driven approach for understanding natural language entailment with logic, addressing a longstanding challenge in the field.


\section{Limitations}

While our proposed method achieves the state-of-the-art Entailment Preservation Rate (EPR) in Entailment-Preserving FOL representation (EPF) task across multiple datasets and against a broad range of baselines, the gap between the EPR of the proposed approach and the \textit{observed upper bound} performance in EPF (EPR@K-Oracle) still remains. This highlights the significant potential for further advancements in data-driven approaches for \nltofol\ semantic parsing, including the usage of more powerful models and larger training data with broader linguistic coverage.

Moreover, the datasets employed in our study (EntailmentBank, eQASC, e-SNLI) lack linguistically controlled minimal pairs, which are instances designed to highlight subtle but crucial distinctions between sentences. This may cause the \nltofol\ translator to overlook fine-grained differences that are critical for accurately expressing natural language entailment. While previous efforts on linguistically grounded FOL representations \citep{bos-nli, amr2fol} successfully encode human intuition into formal logic, our experimental results reveal that these methods struggle to capture natural entailment. Balancing the empirical strengths of data-driven approaches with the precision of linguistically grounded methods remains an open challenge for future work on the EPF task.