\section{Introduction}
\label{sec:intro}

The increasing deployment of robots in real-world environments for infrastructure inspection~\cite{lattanzi2017review}, search-and-rescue~\cite{chung2023into, kruijff2012rescue} and agriculture~\cite{fountas2020agricultural} necessitates the development of systems capable of natural language interaction with humans, providing both textual and visual feedback. 
This, in turn, requires robots to autonomously perceive their surroundings, gather relevant information, and make safe and efficient decisions -- capabilities crucial for a variety of \textbf{\emph{open-world tasking approaches over kilometer-scale environments with sparse semantics}}.
To enable these capabilities on-board robots with privacy \& compute constraints, we develop a framework to efficiently store and plan on hierarchical metric-semantic maps with visual and inertial sensors only.
An overview of our method is shown in Fig.~\ref{fig:system-diagram}.

A cornerstone of autonomous navigation is the creation of actionable maps that effectively represent the environment and support diverse navigation and task-specific operations.  
Such maps must possess several key desiderata: (1) a consistent association between semantic and geometric information derived from observations, enabling a holistic understanding of the environment; (2) an efficient storage mechanism for navigation, such as a hierarchical organization of semantic information into submaps coupled with the representation of geometric information using Gaussian distributions, providing scalability and precise spatial modeling; and (3) task-relevant identification and adaptability, achieved through scoring Gaussian components to facilitate generalization across various tasks and adaptation to new objectives.  
%

These properties collectively ensure that the proposed map is not only manageable but also capable of supporting large-scale autonomous navigation to complete tasks provided in natural language. 
%
To achieve these goals, we propose an agglomerative data structure that is consistent across both geometric and semantic scales built upon 3D Gaussian Splatting~\cite{kerbl20233dgs} (3DGS).
We extract semantics using a backbone vision-language model~\cite{wysoczanska2025clip} and store a compressed feature vector alongside the gaussian points (see Sec.~\ref{sec:mapping}).
This data structure integrates semantic and geometric information seamlessly, allowing for efficient navigation, and task-driven retrieval. 
The design addresses the key challenges of map storage, scalability, and task adaptability, making it suitable for real-world applications.
%
A critical consideration in autonomous navigation is the ability to efficiently access and process relevant map information for motion planning.  
Therefore, the proposed map structure is designed to enable rapid, in-silico retrieval of the map region immediately surrounding the robot (required for efficient planning, see Sec.~\ref{sec:planning}), allowing for timely and efficient local path planning. 
Simultaneously, the map must retain and manage long-range environmental information (see Sec.~\ref{sec:mapping}), providing a comprehensive understanding of the overall environment for tasks such as global path planning and loop closure. 
%

Beyond these map storage requirements, we recognize the need for this actionable map to complement typical hierarchical motion-planning strategies, which often employ a discrete planner for high-level guidance and a low-level planner for dynamically-feasible and safe navigation.  
This allows for a \textbf{\emph{unified framework for mapping and planning from the same set of  sensors}} which allows our method to be deployed on low SWaP (size, weight \& power) robots. 
In addition to the reduced computational and power burden of operation, this formulation allows us to mitigate any discrepancies in the maps used between mapping and planning.
We formulate the planning problem within this two-stage framework (see Sec.~\ref{sec:planning}) and leverage task-relevant scoring for utility calculation in the discrete planning problem, coupled with a sampling-based motion planner for dynamically-feasible trajectory generation.

The key contributions of this work are summarized as follows:
\begin{itemize}
    \item \textbf{Memory-efficient online language-embedded Gaussian splatting:}  
    We introduce a method for Gaussian splatting that incorporates language embeddings efficiently to optimize memory usage. This approach combines dimensionality reduction techniques, such as principal component analysis (PCA), with efficient online updates, ensuring that the map maintains a compact and scalable representation while preserving the necessary geometric and semantic fidelity.

    \item \textbf{Submapping with sparse semantic hierarchical clusters and dense geometry:}  
    The proposed map is structured into sparse semantic clusters that represent regions or objects and dense geometric representations for navigation. This design leverages a tectonic structure, which updates submap anchor poses after loop closure. As a result, the map is compatible with external odometry sources and pre-built maps, enabling consistent updates and large-scale navigation.

    \item \textbf{Large-scale ($>$km) autonomous task-driven navigation with reusable maps:}  
    Our approach supports task-driven navigation by dynamically identifying and retrieving map regions relevant to specific objectives. The system is also compatible with real-time, interactive task re-specification, allowing for adaptive exploration and navigation in large-scale environments.
    
\end{itemize}

%
We demonstrate these contributions through experiments that show the efficiency and semantic retrieval capability of our mapping framework on datasets (Sec.~\ref{sec:seg-experiments}), and real-world demonstrations of the full framework on a mobile robot (Sec.~\ref{sec:robot-exp}).
