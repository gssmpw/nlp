\section{Problem Formulation} 

We consider the problem of optimizing the policy to minimize the time $T$ it takes a robot to complete a given task. 
% $K$ given tasks. 
The robot dynamics are given by 
\begin{equation}
x_{t+1} = f(x_{t}, u_{t}) + w_{t},
\end{equation}0
where $x_{t} \in \mathcal{X}$ and $u_{t} \in \mathcal{U}$ are the state and control input of the robot at time $t$, $w_{t}$ is dynamics noise distributed according to $\mathcal{N}(0, \Sigma_{dyn})$, and the function $f : \mathcal{X} \times \mathcal{U} \rightarrow \mathcal{X}$ represents the disturbance-free equation of motion. 
The state space $\mathcal{X}$ is a subset of $\mathbb{R}^{d_{x}}$ and the feasible control set $\mathcal{U}$ is a bounded region in $\mathbb{R}^{d_{u}}$.

The true, initially unknown, map of the environment $m^{*}$ belongs to the space of maps $\mathcal{M}$. 
In this work, $\mathcal{M}$ is a vector of 3D Gaussian point parameters.
The measurement received from the camera on board the robot at time $t$ is given by 
\begin{equation}
y_{t} = h(x_{t}, m^{*}) + v_{t},
\end{equation}
where $h : \mathcal{X} \times \mathcal{M} \rightarrow \mathcal{Y}$ is the observation function, $\mathcal{Y}$ is the vector space of images captured by the camera, and $v_{t}$ is sensing noise distributed according to $\mathcal{N}(0, \Sigma_{obs})$. 
We assume that $(v_{t})_{t \geq 1}$ and $(w_{t})_{t \geq 1}$ are independent random variables. 
A policy is a sequence of functions $\pi = (\pi_{1}, \pi_{2}, \cdots)$ such that for each $t \geq 1$, $\pi_{t}$ is a function mapping the history of observations received up to time $t$ to a control input $u$: 
\begin{equation}
\pi_{t} : \underbrace{\mathcal{Y} \times \cdots \times \mathcal{Y}}_{=: \mathcal{Y}^{t}}  \rightarrow \mathcal{U}  \quad \text{with} \quad 
u_{t} := \pi_{t}(y_{1:t}).
\end{equation}
We assume that we are provided with an oracle function $\Psi$ that determines whether the task has been completed or not. 
Each task is encoded by a feature vector $z$ that belongs to the space of all tasks $\mathcal{Z}$, which lies in the same space as the natural language embedding vector.
Since a task can be completed after a different number of observations, we define the set of all finite sequences of observations as $\mathcal{Y}^{*} = \cup_{n \in \mathbb{N}} \mathcal{Y}^{n}$. 
In this way, the oracle takes as input the task embedding and the sequence of observations, and outputs the confidence regarding whether the task has been completed or not:
\begin{equation}
\Psi : \mathcal{Z} \times \mathcal{Y}^{*} \rightarrow [0,1].
\end{equation}
Ultimately, we are interested in solving the following problem:
\begin{equation}
\begin{aligned}
& \min_{\pi, T} \ \mathbb{E}[T] \\
& \text{subject to} \\
& \quad \Psi(z, y_{0:T}) = 1, \\
& \text{and } \forall 1 \leq t \leq T: \\
& \quad x_{t+1} = f(x_{t}, u_{t}) + w_{t} \\
& \quad y_{t} = h(x_{t}, m^{*}) + v_{t} \quad \\
& \quad \mathbb{P}(x_{t} \in \mathcal{X}_{free}) \geq 1-\eta.
\end{aligned}
\end{equation}
The last constraint requires the robot to remain within the set of collision-free states $\mathcal{X}_{free} \subseteq \mathcal{X}$ above a certain probability.
In this way $\eta \in (0,1)$ represents the tolerance on the collision probability at each point in time. 
The expectation is taken over the randomness generated by the dynamics and sensing noise. 
