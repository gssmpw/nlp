% CVPR 2025 Paper Template; see https://github.com/cvpr-org/author-kit

\documentclass[10pt,twocolumn,letterpaper]{article}

%%%%%%%%% PAPER TYPE  - PLEASE UPDATE FOR FINAL VERSION
%\usepackage{cvpr}              % To produce the CAMERA-READY version
%\usepackage[review]{cvpr}      % To produce the REVIEW version
\usepackage[pagenumbers]{cvpr} % To force page numbers, e.g. for an arXiv version
\usepackage{makecell}
\usepackage{multirow}
\usepackage{graphicx}

% Import additional packages in the preamble file, before hyperref
\newcommand{\CG}{\mathcal{G}\xspace}
\newcommand{\CV}{\mathcal{V}\xspace}
\newcommand{\CE}{\mathcal{E}\xspace}
\newcommand{\CA}{\mathcal{A}\xspace}
\newcommand{\CF}{\mathcal{F}\xspace}
\newcommand{\CR}{\mathcal{R}\xspace}
\newcommand{\CB}{\mathcal{B}\xspace}
\newcommand{\CX}{\mathcal{X}\xspace}
\newcommand{\CK}{\mathcal{K}\xspace}
\newcommand{\CM}{\mathcal{M}\xspace}
\newcommand{\CC}{\mathcal{C}\xspace}
\newcommand{\CL}{\mathcal{L}\xspace}
\newcommand{\CI}{\mathcal{I}\xspace}
\newcommand{\CQ}{\mathcal{Q}\xspace}
\newcommand{\CO}{\mathcal{O}\xspace}
\newcommand{\CP}{\mathcal{P}\xspace}
\newcommand{\CS}{\mathcal{S}\xspace}
\newcommand{\CT}{\mathcal{T}\xspace}
\newcommand{\CJ}{\mathcal{J}\xspace}
\usepackage[para]{footmisc}
\usepackage{subfig}
% \usepackage{subcaption}
% \usepackage{array}
% \usepackage{colortbl}



% It is strongly recommended to use hyperref, especially for the review version.
% hyperref with option pagebackref eases the reviewers' job.
% Please disable hyperref *only* if you encounter grave issues, 
% e.g. with the file validation for the camera-ready version.
%
% If you comment hyperref and then uncomment it, you should delete *.aux before re-running LaTeX.
% (Or just hit 'q' on the first LaTeX run, let it finish, and you should be clear).
\definecolor{cvprblue}{rgb}{0.21,0.49,0.74}
\DeclareMathOperator*{\argmax}{arg\,max}
\DeclareMathOperator*{\argmin}{arg\,min}
\newcommand{\mycomment}[1]{}
\usepackage[pagebackref,breaklinks,colorlinks,allcolors=cvprblue]{hyperref}

%%%%%%%%% PAPER ID  - PLEASE UPDATE
\def\paperID{14699} % *** Enter the Paper ID here
\def\confName{CVPR}
\def\confYear{2025}

%%%%%%%%% TITLE - PLEASE UPDATE
\title{Realistic Image-to-Image Machine Unlearning via \\ Decoupling and Knowledge Retention}

%%%%%%%%% AUTHORS - PLEASE UPDATE
\author{Ayush K. Varshney \\
Umeå University\\
Umeå, Sweden\\
{\tt\small ayushkv@cs.umu.se}
% For a paper whose authors are all at the same institution,
% omit the following lines up until the closing ``}''.
% Additional authors and addresses can be added with ``\and'',
% just like the second author.
% To save space, use either the email address or home page, not both
\and
Vicen\c{c} Torra\\
Umeå University\\
Umeå, Sweden\\
{\tt\small vtorra@cs.umu.se}
}

\begin{document}
\maketitle
\footnotetext[1]{Under Review.}
\begin{abstract}

% Recent works to jointly reconstruct 3D human and object from a single RGB image, are mostly model-based, that fail to capture the fine details of the clothed human body and object surface. In this paper, we introduce ReCHOR, a novel, model-free, first-method to produce realistic clothed human-object reconstructions from a monocular view. This is extremely challenging due to human-object occlusions, diverse interactions and depth ambiguity, as it needs to infer both 3D spatial awareness and high resolution details. Our core idea is based on estimating neural implicit representations for human and object respectively by an attention-based neural implicit model that attends to pixel-aligned features from both the global human-object image for spatial awareness and  the local separate view of human and object images for high quality details. Additionally, the network is conditioned on semantic features from an initial estimated human-object pose prior and a generative diffusion model that inpaints occluded regions, thus enabling the retrieval of details from them.
% We also propose a synthetic dataset with rendered scenes of diverse, inter-occluded 3D human and object scans, to train our network. We evaluate our method on the synthetic and real world BEHAVE dataset. Our experiments show that our method outperforms the SOTA in achieving realistic clothed human-object reconstructions.
Recent approaches to jointly reconstruct 3D humans and objects from a single RGB image represent 3D shapes with template-based or coarse models, which fail to capture details of loose clothing on human bodies. In this paper, we introduce a novel implicit approach for jointly reconstructing realistic 3D clothed humans and objects from a monocular view. For the first time, we model both the human and the object with an implicit representation, allowing to capture more realistic details such as clothing. This task is extremely challenging due to human-object occlusions and the lack of 3D information in 2D images, often leading to poor detail reconstruction and depth ambiguity. To address these problems, we propose a novel attention-based neural implicit model that leverages image pixel alignment from both the input human-object image for a global understanding of the human-object scene and from local separate views of the human and object images to improve realism with, for example, clothing details. Additionally, the network is conditioned on semantic features derived from an estimated human-object pose prior, which provides 3D spatial information about the shared space of humans and objects. To handle human occlusion caused by objects, we use a generative diffusion model that inpaints the occluded regions, recovering otherwise lost details. For training and evaluation, we introduce a synthetic dataset featuring rendered scenes of inter-occluded 3D human scans and diverse objects. Extensive evaluation on both synthetic and real-world datasets demonstrates the superior quality of the proposed human-object reconstructions over competitive methods.
\end{abstract}    
\section{Introduction}
\label{sec:intro}
% Image editing methods in diffusion models depend on user-defined control directions - users can unlock their creativity using these methods by specifying the desired manipulation through prompts~\cite{gandikota2023concept}, reference images~\cite{ruiz2022dreambooth, kumari2022customdiffusion, gal2022image, chen2024trainingfreeregionalpromptingdiffusion}, or attribute vectors~\cite{parmar2023zero,hertz2022prompt}. In this work, we ask a fundamentally different question: \emph{Can we automatically discover the underlying visual structure of a concept within diffusion model's knowledge?} %Rather than requiring user-specified controls, we aim to decompose the model's internal knowledge into meaningful directions.

% This question touches on a fundamental limitation in how we interact with diffusion models. Current control methods ~\cite{zhang2023addingconditionalcontroltexttoimage, gandikota2023concept, ye2023ipadaptertextcompatibleimage,ye2023ipadaptertextcompatibleimage, hertz2024stylealignedimagegeneration, li2023photomaker, shi2024instantbooth, chen2024trainingfreeregionalpromptingdiffusion} require users to specify their desired manipulations in advance, limiting interactive creativity. This contrasts with natural human artistic workflows, where creators dynamically explore creative ideas while jointly refining them toward meaningful artistic outcomes~\cite{hoffmann2016modeling}. This synergy between specification and exploration is not new to generative models. Early GAN architectures naturally developed disentangled latent spaces that enabled continuous\cite{harkonen2020ganspace,radford2015unsupervised, wu2021stylespace, shen2020interfacegan}, compositional control over generated images. Users could explore these spaces to discover interesting variations that would be difficult to describe in words~\cite{wu2021stylespace}, then combine them to achieve their creative goals~\cite{grabe2022towards}. 


% While diffusion models have largely superseded GANs in conditional image synthesis~\cite{dhariwal2021diffusion},  their underlying structure remains less understood. Diffusion models achieve remarkable diversity through high-dimensional latents, unlike GANs' compact latent spaces.  With a single prompt, diffusion models can generate radically different variations through different random initializations of input noise. We ask - Is it possible to discover interpretable structure within this vast space of variations?

Text-to-image diffusion models are capable of generating remarkable visual variations from a single prompt through different random initializations. However, this vast creative potential remains largely opaque to users---while we can generate diverse images, we lack understanding of the underlying structure of these variations. This presents a fundamental challenge: how can we discover and expose the latent visual capabilities encoded within these models?

\let\thefootnote\relax \footnote{$^{*}$Correspondence to \texttt{gandikota.ro@northeastern.edu}}

The challenge touches on a key limitation in how we interact with diffusion models today. Current control methods require users to explicitly specify their desired edits in advance through prompts~\cite{gandikota2023concept}, reference images~\cite{zhang2023addingconditionalcontroltexttoimage, chen2024trainingfreeregionalpromptingdiffusion, ruiz2022dreambooth,kumari2022customdiffusion, Ryu_lora, hu2021lora}, or attribute vectors~\cite{ye2023ipadaptertextcompatibleimage, hertz2024stylealignedimagegeneration, li2023photomaker, shi2024instantbooth,parmar2023zero,hertz2022prompt}. That contrasts sharply with natural human creative workflows, where artists dynamically explore creative ideas and jointly refine them toward meaningful artistic outcomes~\cite{hoffmann2016modeling}. The need for pre-specified controls creates a barrier between users and the full creative potential of these models.

Interestingly, earlier generative models like GANs~\cite{gans,karras2019style,brock2018large} naturally developed more interpretable internal structures. Their compact latent spaces often exhibited emergent disentanglement~\cite{harkonen2020ganspace,radford2015unsupervised, wu2021stylespace, shen2020interfacegan}, enabling continuous and compositional control over generated images. Users could explore these spaces to discover interesting variations that would be difficult to describe in words~\cite{wu2021stylespace}, then combine them to achieve their creative goals~\cite{grabe2022towards}.

Diffusion models have largely superseded GANs in conditional image synthesis~\cite{dhariwal2021diffusion}, achieving greater diversity through much higher-dimensional latents. And yet an understanding of the underlying structure of these larger latent spaces has remained elusive. In this work, we ask a fundamental question: \emph{Can we automatically discover the visual structure within a diffusion model's knowledge of a concept?} Rather than requiring user-specified controls, we aim to decompose the model's internal representations into expressive directions that users can explore and combine.

To address these needs, we present \textbf{SliderSpace}, a framework that brings systematic explorability to diffusion models. Given just a text prompt, SliderSpace discovers a canonical set of meaningful, diverse, and controllable directions within the model's knowledge of that concept. Each direction is implemented as a low-rank adapter~\cite{hu2021lora} that can be scaled and composed with others, allowing users to explore and smoothly combine different aspects of variation, as shown in Figure~\ref{fig:intro}.

We ground SliderSpace discovery in three key requirements for meaningful decomposition of a diffusion model's visual manifold: 
\begin{enumerate}
    \item \textbf{Unsupervised Discovery:} The decomposition process should emerge from the intrinsic structure of the model's learned representation, rather than being guided by predefined attributes. This ensures we capture the true topology of the model's knowledge space rather than projecting our assumptions onto it.
    
    \item \textbf{Semantic Orthogonality:} Each discovered control must represent a distinct semantic direction. This is enforced in a semantic feature space, like CLIP, where every slider has an orthogonal effect in embeddings. This prevents discovering multiple controls that create similar semantic effects, making the system more efficient and easier.
    
    \item \textbf{Distribution Consistency:} Directions must induce consistent transformations across both random seeds and prompt variations. 
\end{enumerate}

These requirements naturally lead to our proposed framework, which we formalize in Section~\ref{sec:method}. As we show in our experiments, SliderSpace is architecture-agnostic, working with both conventional U-Net based models like Stable Diffusion~\cite{rombach2022high, rombach2022sd20, podell2023sdxl, turbo, dmd} and recent transformer-based architectures like Flux~\cite{flux}.

We demonstrate the expressiveness of SliderSpace through three applications: First, we show how SliderSpace can decompose high-level concepts into diverse and expressive components, revealing the natural axes of variation in the model's understanding. Second, we explore artistic style variation, where SliderSpace discovers directions that match or exceed the diversity of manually curated artist lists while being judged more useful by human evaluators. Finally, we show how SliderSpace can help reverse the mode collapse commonly observed in distilled diffusion models, restoring diversity while maintaining generation speed.

Beyond providing practical creative control, SliderSpace opens new avenues for understanding and utilizing the latent capabilities of diffusion models. By mapping these models' visual potential into intuitive, composable directions, we take a step toward making their creative possibilities more accessible and interpretable to users.

% Image editing methods in diffusion models unlock the creativity of users. In this work we ask an alternate question: \emph{Can we organize and expose what of the diffusion model is already capable of?}.
% Existing methods for controlling image generation typically require users to manually specify edit directions for desired changes. This process is time-consuming, requires technical expertise, and limits the spontaneity of the creative process. For instance, if a user wants to adjust the smile of a generated person, they must explicitly request this edit, often through imprecise prompt engineering or model fine-tuning. This approach of predefined controls or manual specifications restricts users from fully exploring the latent capabilities of the model. There may be interesting stylistic variations or attributes that the model can generate, but users have no easy way to discover or utilize these.

% Natural visual disentanglement was an emergent property in the latent space of Generative Adversarial Models (GANs) \cite{harkonen2020ganspace,radford2015unsupervised, wu2021stylespace, shen2020interfacegan}. In particular, it has been observed that StyleGAN~\cite{karras2019style} stylespace neurons offer detailed control over many meaningful aspects of images that would be difficult to describe in words~\cite{wu2021stylespace}. However, diffusion models do not share such a compact latent space~\cite{park2023unsupervised}; and efforts to uncover such a space in the semantic embeddings of the text conditioning have met with limited success \nik{Nick - is there a specific citation you were thinking about?}.

% In this work we introduce \textbf{SliderSpace}, which takes a step towards uncovering an analogous low dimensional representation of diffusion models' visual breadth; in essence treating the diffusion model as many generators sharing parameters, where a particular generator is defined by a specific prompt. For a given prompt we sample many random seeds (and optionally prompt expansions using an LLM), generate the corresponding images, and apply an off the shelf feature extractor (in this work CLIP, but our method can be applied to any differentiable feature extractor). We use PCA to analyze these features, and for each of the leading $k$ principal components we train a LoRA \cite{} which causes the diffusion model to produces images which increase the feature magnitude along that component when passed back through the same feature extractor. This leads to a 'Slider' for each principal component, because each LoRA can be scaled and applied to the original diffusion model, continuously varying those visual features in the generated results (as measured, in our case, by CLIP).

% There are many other works that enhance the controllability of diffusion models. One common approach is enabling users to add spatial constraints to a generation either manually, or via a reference image \cite{zhang2023addingconditionalcontroltexttoimage, chen2024trainingfreeregionalpromptingdiffusion}, a second is leveraging more abstract embeddings (e.g. identity, style) extracted from a reference image \cite{ye2023ipadaptertextcompatibleimage, hertz2024stylealignedimagegeneration, li2023photomaker, shi2024instantbooth}, a third is finetuning a foundation model to better generate a concept important to the user \cite{ruiz2022dreambooth, kumari2022customdiffusion, Ryu_lora, hu2021lora}, and a fourth (most relevant to this work) is finding low-rank adaptors of the model based on a prompt or small training set which can be scaled to provide continous control over one aspect of generated image (e.g. night vs day, basic vs luxury, etc.) \cite{gandikota2023concept}. SliderSpace is complementary to all of these methods and offers something distinct. All of the other methods we are aware require the user (and / or model designer) to know in advance what type of control they want. In contrast SliderSpace assists users in discovering and controlling hidden capabilities present in the diffusion model's distribution of possible generations.

%We propose that truly intuitive creative control in a text-to-image model should meet three key criteria: \emph{discoverability}, \emph{intuitiveness}, and \emph{specificity}. The model should reveal controllable attributes that may not be immediately obvious, offer controls that are easy to understand and manipulate, and ensure each control affects a distinct attribute of the generated image.

% We demonstrate the utility and power of SliderSpace using three applications built on top of SDXL-DMD \cite{dmd}, because its fast generation speed lends itself well to the continuous control offered by SliderSpace.

% First, we study concept decomposition (Section \ref{sec:concept_exp}), where we learn sliders for a specific concept (e.g. 'monster', 'waterfall', 'car'). Through quantitative metrics of diversity and text alignment we demonstrate that the learned sliders dramatically boost the diversity of generations when randomly applied without harming text alignment; we also ask humans to qualitatively judge these results in a user study where they find the SliderSpace results to be more 'Diverse', 'Useful', and 'Creative' than our baselines.

% Second, we attempt to compare the automatic discoveries of SliderSpace to a large scale manual study of artistic styles (Section \ref{sec:art_exp}), open-sourced by ParrotZone \cite{parrotzone}. In this study SDXL was prompted with over 4300 artist names,  and based on visual inspection the cases of successful stylistic mimicry recorded. Quantitatively SliderSpace more closely matches the distribution of artistic variation discovered by ParrotZone than other baselines, and in our user studies was judged to be significantly more 'Diverse' and 'Useful' than the baselines. To our surprise humans even judged SliderSpace results to be slightly more 'Diverse' than the results generated by the manually discovered artist names of \cite{parrotzone}.

% Third, we attempt to use SliderSpace to reverse the mode collapse commonly observed in distilled few-step diffusion models relative to the original teacher model (Section \ref{sec:diverse_exp}). We quantitatively demonstrate that applying SliderSpace to SDXL-DMD leads to more closely matching the distribution of images by the original teacher, SDXL.

%Through extensive experiments on various state-of-the-art text-to-image models, we demonstrate that SliderSpace significantly enhances user control and creative expression in AI-assisted image generation tasks. Our method enables a range of applications, including concept decomposition and control, diversity improvement in generated images, customization dissection and edits, and the exploration of artistic styles inherent in the model.

% SliderSpace goes beyond providing a practical tool for enhanced creative control. By mapping the visual potential of diffusion models it can open new avenues for generative creativity and deepens our understanding of each model's hidden potential.
\section{Related Work}
\label{sec:related_work}

The original investigation \cite{gibson1979ecological} on the relationship between visual perception and human action defines \emph{affordance} as the opportunities for interaction with the surrounding environment. Behavioral studies on regular and cognitively impaired persons have shown evidence that perception results in both visual and motor signals in the human brain. An extended study \cite{anderson2002attentional} shows that visual attention to the spatial characteristics of the perceived objects initiates automatic motor signals for different actions. In computer vision, human affordance learning involves novel pose prediction such that the estimated pose represents a valid human action within the scene context. The task is fundamental to many problems requiring robust semantic reasoning about the environment, such as human motion synthesis \cite{wang2021scene} and scene-aware human pose generation \cite{wang2017binge, roy2016multi, zhang2022inpaint, yao2023scene}.

Earlier methods of affordance learning have explored knowledge mining \cite{zhu2014reasoning} and multimodal feature cues \cite{roy2016multi} to address the problem. In \cite{zhu2014reasoning}, the authors use a Markov Logic Network for constructing a knowledge base by extracting several object attributes from different image and metadata sources, which can perform various downstream visual inference tasks without any additional classifier, including zero-shot affordance prediction. In \cite{roy2016multi}, the authors use depth map, surface normals, and segmentation map as multimodal cues to train a multi-scale convolutional neural network (CNN) for scene-level semantic label assignment associated with specific human actions. In \cite{do2018affordancenet}, the authors design a multi-branch end-to-end CNN with two separate pathways for object detection and affordance label assignment to achieve high real-time inference throughput. Researchers \cite{chuang2018learning} have also explored socially imposed constraints for affordance learning. In \cite{chuang2018learning}, the authors propose a graph neural network (GNN) to propagate contextual scene information from egocentric views for action-object affordance reasoning.

Probabilistic modeling of scene-aware human motion generation also involves semantic reasoning of human interaction with the environment. Initial works on human motion synthesis have taken different architectural approaches, such as sequence-to-sequence models \cite{barsoum2018hp}, generative adversarial networks (GAN) \cite{barsoum2018hp, cai2018deep, yang2018pose}, graph convolutional networks (GCN) \cite{yan2019convolutional}, and variational autoencoders (VAE) \cite{guo2020action2motion}. However, these methods have mostly ignored the role of environmental semantics. Due to potential uncertainty in human motion, in a recent approach \cite{wang2021scene}, the authors address such motion synthesis with a GAN conditioned on scene attributes and motion trajectory to predict probable body pose dynamics.

One key challenge of human affordance generation in 2D scenes is the lack of large-scale datasets with rich pose annotations. In \cite{wang2017binge}, the authors compile the only public dataset of annotated human body poses in complex 2D indoor scenes by extracting frames from sitcom videos. Aiming to generate a contextually valid human affordance at a user-defined location, the authors propose sampling the scale and deformation parameters for an existing human pose template using a VAE conditioned on the localized image patches as scene context. In \cite{zhang2022inpaint}, the authors introduce a two-stage GAN architecture for achieving a similar goal by estimating the affine bounding box parameters to localize a probable human in the scene and then generating a potential body pose at that location. The method uses the input scene, corresponding depth, and segmentation maps as semantic guidance. In \cite{yao2023scene}, the authors propose a transformer-based approach with knowledge distillation for generating human affordances in 2D indoor scenes.


\section{Preliminaries}
\label{sec:prelims}

%\subsection{Gradient Ascent}

%Gradient ascent (GA) is an optimization technique aimed at maximizing a given loss function for a specified set of examples. This approach is particularly useful for unlearning, as it allows for the approximate removal of specific samples by adjusting the model weights in the direction that maximizes the loss on the target samples \cite{tarun2023fast}. Let $\theta$ represent the model weight, $\eta$ be the learning rate and $L$ be the loss function, then GA iteratively updates the model weights in the following manner.
%\begin{equation}
%    \theta_{t+1} = \theta_t + \eta\frac{\delta L}{\delta \theta_t}
%\end{equation}

\subsection{Out-Of-Distribution data}

Out-of-distribution (OOD) data refers to the inputs that fall outside the range or characteristics of the data used to train a machine learning model. Such inputs can differ significantly from the training dataset, often causing the model to produce inaccurate or unreliable predictions. Detecting OOD data can be framed as a binary classification task, as discussed by Sun et al. (2022). In this context, a sample is classified as OOD if it lies at least a distance of $\lambda$ away from the in-distribution data, where $\lambda$ represents a predefined threshold for OOD detection.

\subsection{Machine Unlearning in I2I models}

The typical I2I models i.e., AutoEncoders (AEs), Variational AutoEncoders (VAEs), and Diffusion models, employ an encoder-decoder architecture. The encoder part $E_\gamma$ of the model transforms the input image to a latent vector. The decoder $D_\phi$ takes the latent vector as input and decodes it into an output image i.e., for an input image $x$, the output of an I2I model can be written as follows.
\begin{equation}
    \theta_{\gamma, \phi} = D_\phi \circ E_\gamma 
\end{equation}
\text{Therefore, for any input image } $x, \; \theta_{\gamma, \phi} (\tau(x)) = D_\phi (E(\tau(x)))$, where $\tau(x)$ is some operation which results in cropped or masked image $x$; $\circ$ is the composition operator; $\gamma$ and $\phi$ are the trainable model parameters of encoder and decoder respectively.

For a given dataset $\mathcal{D}$, the goal of machine unlearning in image-to-image (I2I) generative models is to effectively remove the influence of a specific subset, known as the forget set $\mathbb{D}_f$, from the model parameters with an unlearning algorithm $A_f$, such that the updated parameters $\gamma, \phi = A_f(\gamma^0, \phi^0)$ no longer retain any information about $\mathbb{D}_f$. Crucially, the model must preserve its performance on the retain set $\mathbb{D}_r$, ($\mathbb{D}_r = \mathcal{D} \setminus \mathbb{D}_f$). In the literature of machine unlearning \cite{li2024machine, feng2024controllable}, the following complementary objectives have been considered.
\begin{enumerate}
    \item On the retain set $\mathbb{D}_r$, the generated images from $\theta_{\gamma, \phi}$ should have the same distribution as in $\theta_{\gamma^0, \phi^0}$ (after unlearning).
    \item On the forget set $\mathbb{D}_f$, the distribution of the generated images from $\theta_{\gamma, \phi}$ should be \textit{as far as possible} from the distribution of images from $\theta_{\gamma^0, \phi^0}$ (after unlearning).
\end{enumerate}

From a probabilistic distribution perspective, the unlearning methodology in \cite{li2024machine} considers the following combined objectives:
\begin{equation}
    \begin{split}
        \argmin_{\gamma, \phi} D \left( P_{\theta_{\gamma_0, \phi_0} (\tau(\mathbb{D}_r))} || P_{\theta_{\gamma, \phi} (\tau(\mathbb{D}_r))} \right ) \text{, and } & \\
        \argmax_{\gamma, \phi} D \left ( P_{\theta_{\gamma_0, \phi_0} (\tau(\mathbb{D}_f))} || P_{\theta_{\gamma, \phi} (\tau(\mathbb{D}_f))} \right )
    \end{split}
\end{equation}

An approach to solve the second objective function is to push the model weights to generate Gaussian noise and, thus to forget the set $\mathbb{D}_f$. That is to solve $\argmin_{\gamma, \phi} D \left ( N(0, \Sigma) || P_{\theta_{\gamma, \phi} (\tau(\mathbb{D}_f))} \right )$. The unlearning methodology in \cite{feng2024controllable} introduces a $\varepsilon$-constrained optimization for this objective, where $\varepsilon$ controls the degree of unlearning. 

\section{Proposed Approach} \label{sec:proposed approach}

\begin{figure}[t]
  \centering
  %\fbox{\rule{0pt}{2in} \rule{0.9\linewidth}{0pt}}
   \includegraphics[width=\linewidth]{fig/I2I_ours_flowchart.png}
   \caption{Overview of our proposed approach. We maximize the loss on the forget samples ($D_f$) to get unlearned model. We get the updated I2I model by further fine-tuning the unlearned model on the retain samples to maintain the performance.}
   \label{fig:our framework}
\end{figure}

In this work, we aim to realistically address the problem of machine unlearning in Image-to-Image (I2I) generative models which reconstruct the image from its partial or noisy counterpart. Recall that the objective of unlearning is to modify the model parameters in such a way that the model's weights, after unlearning specific data, closely match the weights the model would have if it were retrained from scratch without that data. Because of that, we consider that minimizing the distance between gaussian noise and the output of decoder is \textit{not} equivalent to the model which we will get from a retrained model. As an alternative, in our approach we consider that the forget set should be an out-of-distribution (OOD) after unlearning. Mathematically, we define the unlearning objective using the following expression:
\begin{equation} \label{unlearn_obj}
    \arg_{\gamma, \phi} \| \theta_{\gamma_0, \phi_0} (\tau(\mathbb{D}_f)) - \theta_{\gamma, \phi} (\tau(\mathbb{D}_f)) \| \geq \lambda 
\end{equation}
where $\| .\|$ is a distance measure, and $\lambda$ is a threshold which is used to determine out-of-distribution data. In summary, we consider the following combined objective.
\begin{enumerate}
    \item On the retain set $\mathbb{D}_r$, the generated images from $\theta_{\gamma, \phi}$ should have the same distribution as the images generated from $\theta_{\gamma^0, \phi^0}$ after unlearning.
    \item On the forget set $\mathbb{D}_f$, the distribution of the generated images from $\theta_{\gamma, \phi}$ should be atleast some $\lambda$ distance apart from the distribution of images from $\theta_{\gamma^0, \phi^0}$ after unlearning.
\end{enumerate}

Let $\mathcal{L}$ be the loss function for I2I model $f(x, \gamma, \phi): \mathbb{R}^d \rightarrow \mathbb{R}^d$ $\forall x \in $ $\mathcal{D} \; (\mathcal{D} = \mathbb{D}_r \cup \mathbb{D}_f)$, where $\gamma, \phi$ are the model parameters. The typical objective of a machine learning model is to minimize $\mathcal{L}$ i.e., $\argmin_{\gamma, \phi} \mathcal{L}(\gamma, \phi, \mathcal{D})$. The typical update rule for machine learning model can be written as:
\begin{equation}
    \theta_{\gamma^{t+1}, \phi^{t+1}} = \theta_{\gamma^t, \phi^t} - \eta\nabla f(\gamma^t, \phi^t)
\end{equation}
where $\eta$ is the learning rate. Let $\gamma^*, \phi^*$ be the optimal parameters which have perfect generation i.e.,
\begin{equation}
    \theta_{\gamma^*, \phi^*}(\tau(x)) = x
\end{equation}
Under the assumption that SGD converges for the loss function $\mathcal{L}$, we can say that the expected value of $\theta_{\gamma^*, \phi^*} (\tau(x)) - \theta_{\gamma, \phi}(\tau(x))$ decreases over time across training epochs, although it may not be strictly monotonically decreasing in every epoch. 

\subsection{Decoupling via Gradient Ascent}

Gradient ascent is an approach whose objective is to maximize the model loss on a given set. It is a reverse of gradient descent where the model update is written as:
\begin{equation}
    \theta_{\gamma^{t+1}, \phi^{t+1}} = \theta_{\gamma^t, \phi^t} + \eta \nabla f(\gamma^t, \phi^t)
\end{equation}
In our work, we use gradient ascent to forget the influence of $\mathbb{D}_f$ on the model $\theta_{\gamma_0, \phi_0}$. In gradient ascent, we expect that the loss function $\mathcal{L}(\theta_{\gamma, \phi}) $ is an increasing function in $t$ (the loss increases with training) in contrast to the decreasing function in gradient descent. In the next steps, we prove that the output on $\mathbb{D}_f$ with updated model parameters $\theta_{\gamma, \phi}$ after $T$ iterations of gradient ascent is out-of-distribution from the output with $\theta_{\gamma^0, \phi^0}$. We have the following assumption in our work.

%(x, \gamma, \phi)
\textbf{Assumption 1.} $f: \mathbb{R}^d \rightarrow \mathbb{R}^d$ is convex and differentiable. 
\begin{equation} \label{ass1}
    \forall a,b \in \mathbb{R}^d, f(a) \geq f(b) + \langle \nabla f(a), a-b \rangle
\end{equation}

\textbf{Assumption 2.} During all the training epochs, the expected norm of gradient is lower and upper bounded i.e., $g\leq\mathbb{E}\|\nabla f\| \leq G$, where $g, G>0$.

%\textbf{Assumption 3.} Observe that the stochastic gradient is the sum of $S$ independent, uniformly sampled contributions. With central limit theorem, we assume that the gradient noise is Gaussian with covariance $\frac{1}{S}\Sigma(\theta)$.

The convexity assumption is typically invoked in machine unlearning literature to ensure that the model is well-trained and to quantify the influence of data removal on model parameters \cite{guo2019certified}. Although I2I models are usually formulated as non-convex optimization problems, several studies have adopted the convexity assumption to derive theoretical guarantees \cite{sahiner2021hidden, de2022convergence, zhang2024analyzing}. The lower bound in Assumption 2 is needed to make sure the model before unlearning is not at the optimum (i.e., $\theta_{\gamma^0\phi^0} \neq \theta_{\gamma^*\phi^*}$), while the upper bound is considered to prove $(\epsilon, \delta)$-unlearning. We know that for gradient ascent we have,

\begin{equation} \label{GA}
    \theta_{\gamma^{t+1}, \phi^{t+1}} = \theta_{\gamma^t, \phi^t} + \eta \nabla f(\gamma^t, \phi^t)
\end{equation}

Let us consider Assumption 1 for $\gamma^{t+1}, \phi^{t+1}$ and $\gamma^{t}, \phi^{t}$.
\begin{equation}
    \begin{split}
        f(\theta_{\gamma^{t+1}, \phi^{t+1}}) &\geq f(\theta_{\gamma^{t}, \phi^{t}}) \\
        &\;\;\;\; + \langle \nabla f(\theta_{\gamma^{t+1}, \phi^{t+1}}), \theta_{\gamma^{t+1}, \phi^{t+1}} - \theta_{\gamma^{t}, \phi^{t}} \rangle \\
        &\!\!\!\! \geq f(\theta_{\gamma^{t}, \phi^{t}}) + \eta \| \nabla f(\theta_{\gamma^{t+1}, \phi^{t+1}})\| \|\nabla f(\theta_{\gamma^{t}, \phi^{t}}) \| \\
        %&\geq f(\theta_{\gamma^{t-1}, \phi^{t-1}}) + \eta \| \nabla f(\theta_{\gamma^{t}, \phi^{t}})\| \|\nabla f(\theta_{\gamma^{t-1}, \phi^{t-1}}) \| + \eta \| \nabla f(\theta_{\gamma^{t+1}, \phi^{t+1}})\| \|\nabla f(\theta_{\gamma^{t}, \phi^{t}}) \| \\
        &\!\!\!\! \geq f(\theta_{\gamma^{0}, \phi^{0}}) \\
        &\;\;\;\;\; + \eta \sum_{t=0}^{T-1} \|\nabla f(\theta_{\gamma^{t+1}, \phi^{t+1}}) \| \| \nabla f(\theta_{\gamma^{t}, \phi^{t}})\| 
    \end{split}
\end{equation}

Using Assumption 2, we have $\|\nabla f(\theta_{\gamma^{t+1}, \phi^{t+1}}) \|, \| \nabla f(\theta_{\gamma^{t}, \phi^{t}})\| \geq g$. Then we can rewrite Eq. (9) as:
\begin{equation} \label{lower bound}
    \|f(\theta_{\gamma^{T}, \phi^{T}}) - f(\theta_{\gamma^{0}, \phi^{0}}) \| \geq \eta Tg^2 
\end{equation}

Let us consider Assumption 1 again to get an upper bound for $\gamma^{t+1}, \phi^{t+1}$ and $\gamma^{t}, \phi^{t}$ in the opposite order.
\begin{equation}
    \begin{split}
        f(\theta_{\gamma^{t}, \phi^{t}}) &\geq f(\theta_{\gamma^{t+1}, \phi^{t+1}}) \\
        &\;\;\;\; + \langle \nabla f(\theta_{\gamma^{t}, \phi^{t}}), \theta_{\gamma^{t}, \phi^{t}} - \theta_{\gamma^{t+1}, \phi^{t+1}} \rangle \\
        f(\theta_{\gamma^{t+1}, \phi^{t+1}}) &\leq f(\theta_{\gamma^{t}, \phi^{t}}) \\
        &\;\;\;\; + \langle \nabla f(\theta_{\gamma^{t}, \phi^{t}}), \theta_{\gamma^{t+1}, \phi^{t+1}} - \theta_{\gamma^{t}, \phi^{t}} \rangle \\
        %&\leq f(\theta_{\gamma^{t-1}, \phi^{t-1}}) + \langle \nabla f(\theta_{\gamma^{t-1}, \phi^{t-1}}), \theta_{\gamma^{t}, \phi^{t}} - \theta_{\gamma^{t-1}, \phi^{t-1}} \rangle \\ 
        %&+ f(\theta_{\gamma^{t}, \phi^{t}}) + \| \nabla f(\theta_{\gamma^{t}, \phi^{t}}), \theta_{\gamma^{t+1}, \phi^{t+1}} - \theta_{\gamma^{t}, \phi^{t}} \| \\
        &\leq f(\theta_{\gamma^{0}, \phi^{0}}) + \eta \sum_{t=0}^{T} \| \nabla f(\theta_{\gamma^{t}, \phi^{t}}) \|^2
    \end{split}
\end{equation}
Again using Assumption 2, i.e., $\| \nabla f(\theta_{\gamma^{t}, \phi^{t}}) \| \leq G$. Then we get,
\begin{equation} \label{upper bound}
    f(\theta_{\gamma^{T}, \phi^{T}}) \leq  f(\theta_{\gamma^{0}, \phi^{0}}) + \eta (T+1) G^2 
\end{equation}

\mycomment{
\begin{equation} \label{increasing step_GA}
    \| \theta_{\gamma^{t+1}, \phi^{t+1}} - \theta_{\gamma^t, \phi^t} \| ^2 = \eta^2\|\nabla f(\gamma^t, \phi^t)\|^2
\end{equation}
i.e., $\| \theta_{\gamma^{t+1}, \phi^{t+1}} - \theta_{\gamma^0, \phi^0} \| ^2$ is an increasing sequence in $t$. Thus, for $T$ iteration we have,
\begin{equation}
    \| \theta_{\gamma^{T}, \phi^{T}} - \theta_{\gamma^0, \phi^0} \| ^2 = \eta^2 \sum_{t=1}^T \|\nabla f(\gamma^t, \phi^t)\|^2
\end{equation}
Taking expectation both sides and using Assumption 2, we get,
\begin{equation}\label{bounding model_Weights}
    T\eta^2g^2 \leq \mathbb{E}\| \theta_{\gamma^{T}, \phi^{T}} - \theta_{\gamma^0, \phi^0} \| ^2 \leq T \eta^2 G^2
\end{equation}


Considering the left hand side of the inequality and by using the Assumption 1, we get:
\begin{gather*}
    f(\theta_{\gamma^{T}, \phi^{T}}) \geq f(\theta_{\gamma^{0}, \phi^{0}}) + \langle \nabla f(\theta_{\gamma^{0}, \phi^{0}}), \theta_{\gamma^{T}, \phi^{T}} - \theta_{\gamma^{0}, \phi^{0}}\rangle
\end{gather*}
We can rewrite it as:
\begin{gather*}
    f(\theta_{\gamma^{T}, \phi^{T}}) - f(\theta_{\gamma^{0}, \phi^{0}}) \geq \| \nabla f(\theta_{\gamma^{0}, \phi^{0}}) \| \|\theta_{\gamma^{T}, \phi^{T}} - \theta_{\gamma^{0}, \phi^{0}}\|
\end{gather*}
Taking expectation on both sides, and then using Eq. (\ref{bounding model_Weights}) we get:
\begin{equation} \label{Th1}
    \begin{split}
        \mathbb{E}\|f(\theta_{\gamma^{T}, \phi^{T}}) - f(\theta_{\gamma^{0}, \phi^{0}})\| & \geq \mathbb{E}\| \nabla f(\theta_{\gamma^{0}, \phi^{0}}) \| \mathbb{E}\|\theta_{\gamma^{T}, \phi^{T}} - \theta_{\gamma^{0}, \phi^{0}}\| \\
        & \geq T\eta^2g^3
    \end{split}
\end{equation}
On the similar lines, with Assumption 1 we have:
\begin{equation}
    f(\theta_{\gamma^{T}, \phi^{T}}) \leq f(\theta_{\gamma^{0}, \phi^{0}}) + \nabla f(\theta_{\gamma^{T}, \phi^{T}}) \langle \theta_{\gamma^{T}, \phi^{T}} - \theta_{\gamma^{0}, \phi^{0}}\rangle 
\end{equation}
Taking expectation both sides and using right hand side of the inequality from Eq. (\ref{bounding model_Weights}), we get:
\begin{equation} \label{upper bound}
    \begin{split}
        \mathbb{E}\|f(\theta_{\gamma^{T}, \phi^{T}})\| & \leq \mathbb{E} \|f(\theta_{\gamma^{0}, \phi^{0}}) \| + \mathbb{E}\| \nabla f(\theta_{\gamma^{T}, \phi^{T}}) \| \mathbb{E}\|\theta_{\gamma^{T}, \phi^{T}} - \theta_{\gamma^{0}, \phi^{0}}\| \\
        & \leq \mathbb{E} \|f(\theta_{\gamma^{0}, \phi^{0}}) \| + T\eta^2G^3
    \end{split}
\end{equation}
}

Based on this result, we establish the following theorems.

\textbf{Theorem 1.} Under the Assumption 1, 2, the model weights $(\theta_{\gamma, \phi})$ trained with gradient ascent for $T$ iterations are out-of-distribution for the initial trained model $\theta_{\gamma^0, \phi^0}$ on forget set iff
\begin{equation*}
    \lambda \leq \eta Tg^2
\end{equation*}
where $\lambda$ is the predefined out-of-distribution threshold.

\textbf{Theorem 2.} Under Assumption 1, 2, the gradient ascent provides $(\epsilon = 0, \delta = \eta (T+1) G^2)$-unlearning guarantee for model weight $(\theta_{\gamma, \phi})$ after $T$ iterations.

\subsection{Knowledge Retention}

As shown in Fig. \ref{fig:our framework}, the unlearned model is then fine-tuned with retain samples ($\mathbb{D}_r$) with the following objective,
\begin{equation}
    \argmin_{\gamma, \phi} D \left( P_{\theta_{\gamma_0, \phi_0} (\tau(\mathbb{D}_r))} || P_{\theta_{\gamma, \phi} (\tau(\mathbb{D}_r))} \right )
\end{equation}
in order to preserve its performance.

\subsection{Auditing Unlearning} \label{data poisoning attack}

Auditing effective unlearning is crucial to make sure that the model has truly forgotten the forget samples. In this paper, we introduce a novel auditing mechanism based on a data poisoning attack to verify whether unlearning has occurred. Specifically, we fine-tune the model on poisoned versions of the forget samples, embedding a distinct pattern that is recognizable during inference. The aim of this approach is to introduce a recognizable trace into the model's responses that can later serve as an indicator. After the unlearning process, an effective unlearning mechanism should prevent the model from replicating this pattern. If the model, after unlearning, no longer produces outputs associated with the poisoned pattern, it provides strong evidence that the forget samples have been thoroughly erased. This method ensures a more robust validation of the unlearning process by focusing not only on performance metrics but also on detecting residual data influence, thus confirming the removal/unlearning of the impact of the poisoned data.

\begin{table}[h]
    \centering
    \begin{tabular}{ccccc}
      \toprule
      \multirow{2}{*}{Approach} & \multicolumn{2}{c}{FID} & \multicolumn{2}{c}{IS} \\ \cline{2-5}
      & $\mathbb{D}_f\downarrow$ & $\mathbb{D}_r\downarrow$ & $\mathbb{D}_f$ & $\mathbb{D}_r$ \\
      \hline
      GA model & 237.7 & 210.1 & 1.08 & 1.05 \\
      Fine-tuned model & 46.85 & \textbf{3.79} & 1.09 & \textbf{1.13} \\
      SOTA I2I model & 123.6 & 74.2 & 1.09 & 1.12\\
      Merged obj model & 202.1 & 127.7 & 1.08 & 1.08 \\
      Realistic-I2I model & \textbf{16.22} & \textbf{6.10} & 1.10 & \textbf{1.13} \\
      \bottomrule
    \end{tabular}
    \caption{Results of cropping $8 \times 8$ patch at the center of the image where forget samples were poisoned with the '$+$' sign in the CIFAR-10 dataset. $\mathbb{D}_f$ and $\mathbb{D}_r$ account for the forget samples and retain samples respectively. FID scores are compute with respect to retrained model, hence $\downarrow$ is better. Overall, the results highlight that our approach effectively unlearns forget samples and is closer to the retrained model.}
    \label{tab:CIFAR10_res}
\end{table}
\section{Experimental Results}

\subsection{Experimental Setup}

\begin{table*}[h]
    \centering
    \begin{tabular}{ccccccc||cccccc}
      \toprule
      \multirow{3}{*}{Approach} & \multicolumn{6}{c||}{$4\times4$} & \multicolumn{6}{c}{$8\times8$} \\
      & \multicolumn{2}{c}{FID} & \multicolumn{2}{c}{IS} & \multicolumn{2}{c||}{CLIP} & \multicolumn{2}{c}{FID} & \multicolumn{2}{c}{IS} & \multicolumn{2}{c}{CLIP} \\ \cline{2-13}
      & $\mathbb{D}_f\uparrow$ & $\mathbb{D}_r\downarrow$ & $\mathbb{D}_f$ & $\mathbb{D}_r$ & $\mathbb{D}_f$ & $\mathbb{D}_r$ & $\mathbb{D}_f\uparrow$ & $\mathbb{D}_r\downarrow$ & $\mathbb{D}_f$ & $\mathbb{D}_r$ & $\mathbb{D}_f$ & $\mathbb{D}_r$ \\
      \hline
      Max loss & \textbf{56.75} & 9.12 & \textbf{12.07} & 15.06 & 0.80 & \textbf{0.834} & \textbf{109.9} & 16.07 & \textbf{6.33} & 17.03 & 0.64 & 0.735 \\
      Random label & 22.4 & \textbf{8.88} & 13.82 & 14.9 & 0.80 & \textbf{0.834} & 48.84 & \textbf{14.77} & 11.29 & 17.27 & 0.64 & \textbf{0.741} \\
      Random encoder & 23.39 & 9.15 & 13.77 & 15.05 & 0.83 & 0.831 & 25.86 & 15.84 & 16.96 & 17.42 & 0.72 & 0.736 \\
      I2I SOTA & 22.99 & 9.08 & 13.86 & 15.19 & \textbf{0.79} & 0.831 & 53.58 & 15.79 & 12.00 & 17.64 & \textbf{0.61} & 0.736 \\
      Ours& \textbf{24.68} & \textbf{8.93} & \textbf{14.03} & \textbf{15.13} & 0.83 & \textbf{0.834} & 27.43 & \textbf{14.78} & \textbf{18.98} & \textbf{18.77} & 0.731 & \textbf{0.741} \\
      \bottomrule
    \end{tabular}
    \caption{Comparison of various unlearning approaches with different cropped patches ($4\times4 \text{ and } 8\times8$) for VQ-GAN where forget samples were poisoned with the '$+$' sign in the ImageNet-1K dataset. $\mathbb{D}_f$ and $\mathbb{D}_r$ account for the forget samples and retain samples, respectively. FID scores are computed with respect to attack model, hence $\uparrow$ is better for $\mathbb{D}_f$ and $\downarrow$ for $\mathbb{D}_r$. IS score highlight that our approach create good quality images even when the FID distance is significantly far from the attack model. Similarly, we find high CLIP values for our approach indicating that generated image still captures the semantics with an image (not just random noise).}
    %Overall, the results highlight that our approach effectively unlearns forget samples and is closer to the retrained model.
    \label{tab:VQ-GAN results}
\end{table*}

\begin{table*}[h]
    \centering
    \begin{tabular}{ccccccc||cccccc}
      \toprule
      \multirow{3}{*}{Approach} & \multicolumn{6}{c||}{$4\times4$} & \multicolumn{6}{c}{$8\times8$} \\
      & \multicolumn{2}{c}{FID} & \multicolumn{2}{c}{IS} & \multicolumn{2}{c||}{CLIP} & \multicolumn{2}{c}{FID} & \multicolumn{2}{c}{IS} & \multicolumn{2}{c}{CLIP} \\ \cline{2-13}
      & $\mathbb{D}_f\downarrow$ & $\mathbb{D}_r\downarrow$ & $\mathbb{D}_f$ & $\mathbb{D}_r$ & $\mathbb{D}_f$ & $\mathbb{D}_r$ & $\mathbb{D}_f\downarrow$ & $\mathbb{D}_r\downarrow$ & $\mathbb{D}_f$ & $\mathbb{D}_r$ & $\mathbb{D}_f$ & $\mathbb{D}_r$ \\
      \hline
      Max loss & 32.79 & 55.86 & 48.04 & 32.33 & 0.86 & 0.733 & 89.4 & 114.2 & 17.4 & 12.97 & 0.686 & 0.65 \\
      Random label & 19.16 & 19.29 & 56.89 & \textbf{36.22} & 0.92 & 0.867 & 54.60 & 12.55 & 33.24 & 26.47 & 0.759 & 0.87 \\
      Random encoder & 12.95 & 21.25 & 52.79 & 33.47 & 0.93 & 0.85 & 44.32 & 18.77 & 42.01 & \textbf{27.85} & 0.755 & 0.83 \\
      I2I SOTA & 17.16 & \textbf{12.95} & 26.59 & 34.26 & \textbf{0.59} & 0.895 & 101.8 & 13.79 & 9.37 & 21.74 & 0.498 & \textbf{0.88} \\
      Ours & \textbf{9.65} & \textbf{15.14} & \textbf{58.38} & \textbf{35.05} & 0.88 & \textbf{0.904} & \textbf{13.98} & \textbf{13.27} & \textbf{52.6} & 21.02 & \textbf{0.945} & \textbf{0.88} \\
      \bottomrule
    \end{tabular}
    \caption{Comparison of various unlearning approaches for diffusion model with the output of \textit{the original model} for different cropped patches ($4 \times 4 \text{ and } 8\times8$) where forget samples were poisoned with the '$+$' sign in the ImageNet-1K dataset. $\mathbb{D}_f$ and $\mathbb{D}_r$ account for the forget samples and retain samples, respectively. FID scores are computed with respect to original model (to show that our approach mitigate '$+$' sign), hence $\downarrow$ is better for $\mathbb{D}_f$ and $\mathbb{D}_r$. IS score highlight that our approach create good quality images even when the FID distance is significantly far from the attack model. Similarly, we find high CLIP values for our approach indicating that generated image still captures the semantics with an image (not just random noise).}
    %Overall, the results highlight that our approach effectively unlearns forget samples and is closer to the retrained model.
    \label{tab:diff_model results}
\end{table*}

\begin{figure}
  \centering
  \includegraphics[width=\columnwidth]{fig/attack_res_CIFAR_updated.png}
  %\caption{Comparison of the various unlearning approaches along with the retrained model on forget samples poisoned with $'+'$ sign on AutoEncoder. It clearly evident that our method has effectively unlearned the $'+'$ sign and has the output generated closest to the retrained model.}
  \caption{Comparison of various unlearning approaches, along with the retrained model, on forget samples that were poisoned with a '$+$' sign using an AutoEncoder. The results clearly demonstrate that our method effectively unlearns the '$+$' sign, producing outputs that are most similar to those of the retrained model.}
  \label{fig:attack_CIFAR10}
\end{figure}
% First Table

We evaluate our approach using three mainstream I2I architectures: $(i)$ AutoEncoder, $(ii)$ VQ\_VAE \cite{li2023mage}, and $(iii)$ diffusion model \cite{saharia2022palette}. We validate our framework on two widely-used large-scale datasets, namely Places-365 and ImageNet-1K. Additionally, to compare the effectiveness of our framework against a retrained model, we conduct AutoEncoder experiments on the CIFAR-10 dataset.

For the ImageNet-1K dataset, we randomly sampled 100 classes as $\mathbb{D}_f$ and 100 classes as $\mathbb{D}_r$. Similarly, for the Places-365 dataset, we selected 50 classes each for $\mathbb{D}_f$ and $\mathbb{D}_r$. Due to the limited number of classes in the CIFAR-10 dataset, we randomly designated one class as $\mathbb{D}_f$ and the remaining nine classes as $\mathbb{D}_r$.

\begin{figure}[ht]
    \centering
    \begin{subfigure}[b]{0.49\linewidth}
        \centering
        \includegraphics[width=\linewidth]{fig/dm_mask_ratio_diff_tsne_ours.pdf}
        \label{fig:diff_tsne}
    \end{subfigure}
    \hfill
    \begin{subfigure}[b]{0.49\linewidth}
        \centering
        \includegraphics[width=\linewidth]{fig/vqgan_mask_ratio_tsne_I2I_attack1_orig_temp.pdf}
        \label{fig:vqgan_tsne}
    \end{subfigure}
    \vspace{-2em}
    \caption{T-SNE analysis of the generated images with our unlearning framework. After unlearning in both the cases, the generated image from retain samples closely overlaps the ground truth, while the image generated from forget samples diverge from the ground truth.}
    \label{fig:T-SNE analysis}
\end{figure}

%\textbf{Baselines.} Recall from \cref{data poisoning attack}, that in order to audit effectively unlearning, we introduce a data poisoning attack. We have introduced a '$+$' at the center of the forget images to train an attack model. In the main paper, we compare the results of several baselines and benchmark and in the Appendix you find the results for the normal datasets (CIFAR10, Places-365, and ImageNet-1K). We have compared the unlearning results of AutoEncoder, VQ-GAN, and the diffusion model with $(i)$ Max loss baseline, which maximizes the model loss on forget samples \cite{halimi2022federated, warneckemachine}; $(ii)$ Noisy Label, which minimizes training loss with Gaussian noise as ground truth for forget samples \cite{gandikota2023erasing}; $(iii)$ Random Encoder, which minimizes the distance between the output of the encoder on the forget set and Gaussian Noise \cite{tarun2023deep}; and $(iv)$ state of the art I2I unlearning model in \cite{li2024machine}, which minimizes the distance between the Encoder output and gaussian noise while fine-tuning the encoder parameters on retain samples. 

\begin{figure*}
  \centering
    \includegraphics[width = \linewidth]{fig/VQ-GAN_comparison_approaches.png}
    %\caption{Our approach works well on all major I2I architectures, i.e., VQ-GAN, Diffusion model and autoEncoders (see Section 5). This figure also shows comparison with the state of the art (SOTA) I2I unlearning algorithm. For retain samples, our approach generates similar images before and after unlearning, however, SOTA approach struggles in many cases. On forget samples, our approach generates inaccurate/unreliable predictions, matching the expectation of realistic unlearning.}
    \caption{Results of cropping $8 \times 8$ patch at the center of the image on VQ\_GAN models.  The results demonstrate that our model effectively removes the embedded '$+$' pattern, not just replacing it with Gaussian noise. For retain samples, our method shows no signs of the embedded '$+$' sign from the forget samples, in contrast to other baseline and benchmark methods, which often retain subtle remnants of the pattern.}
    \label{fig:VQGAN_comp}
\end{figure*}

\begin{table*}[h]
    \centering
    \begin{tabular}{ccccccc||cccccc}
      \toprule
      \multirow{3}{*}{Approach} & \multicolumn{6}{c||}{VQ-GAN} & \multicolumn{6}{c}{Diffusion model} \\
      & \multicolumn{2}{c}{FID$\downarrow$} & \multicolumn{2}{c}{IS$\uparrow$} & \multicolumn{2}{c||}{CLIP $\uparrow$ } & \multicolumn{2}{c}{FID $\downarrow$ } & \multicolumn{2}{c}{IS $\uparrow$} & \multicolumn{2}{c}{CLIP $\uparrow$} \\ \cline{2-13}
      & $4 \times 4$ & $8 \times 8$ & $4 \times 4$ & $8 \times 8$ & $4 \times 4$ & $8 \times 8$ & $4 \times 4$ & $8 \times 8$ & $4 \times 4$ & $8 \times 8$ & $4 \times 4$ & $8 \times 8$ \\
      \hline
      Max loss & 14.5 & 36.7 & 48.2 & 32.1 & 0.884 & \textbf{0.88} & 15.1 & 49.4 & 38.1 & 29.8 & 0.91 & 0.74 \\
      Random label & 11.7 & 27.9 & 51.2 & 36.7 & 0.880 & 0.87 & 15.15 & 56.5 & 35.1 & 33.23 & 0.92 & 0.82 \\
      Random encoder & 8.1 & 12.6 & 55.4 & 53.8 & 0.885 & 0.87 & 11.1 & 26.17 & 60.8 & 51.62 & 0.91 & 0.81 \\
      I2I SOTA & 11.0 & 26.3 & 51.6 & 38.4 & 0.883 & 0.87 & 33.3 & 79.1 & 57.1 & 28.01 & 0.87 & 0.74 \\
      Ours & \textbf{7.98} & \textbf{12.4} & \textbf{56.2} & \textbf{54.4} & \textbf{0.886} & \textbf{0.88} & \textbf{4.5} & \textbf{12.70} & \textbf{67.1} & \textbf{58.1} & \textbf{0.97} & \textbf{0.89} \\
      \bottomrule
    \end{tabular}
    \caption{Comparison of unlearning approaches on VQ-GAN and Diffusion models for generating unseen data. We evaluate performance on randomly selected 50 classes from the ImageNet-1K dataset for VQ-GAN and the Places365 dataset for the Diffusion model, using different cropped patch sizes ($4 \times 4$ and $8 \times 8$).  FID scores are computed with respect to the original model, where lower values ($\downarrow$) indicate better alignment with the target distribution. Similarly, $\uparrow$ in IS and CLIP score is better to demonstrate the models ability to generate good quality images on unseen data as well.}
    
    %with the output of \textit{the original model}  where forget samples were poisoned with the '$+$' sign in the ImageNet-1K dataset. $\mathbb{D}_f$ and $\mathbb{D}_r$ account for the forget samples and retain samples, respectively. FID scores are computed with respect to original model (to show that our approach mitigate '$+$' sign), hence $\downarrow$ is better for $\mathbb{D}_f$ and $\mathbb{D}_r$. IS score highlight that our approach create good quality images even when the FID distance is significantly far from the attack model. Similarly, we find high CLIP values for our approach indicating that generated image still captures the semantics with an image (not just random noise).}
    %Overall, the results highlight that our approach effectively unlearns forget samples and is closer to the retrained model.
    \label{tab:OOD_model results}
\end{table*}

\textbf{Baselines.} As discussed in Section \ref{data poisoning attack}, we introduce a data poisoning attack as a mechanism to effectively audit unlearning. Specifically, we embed a pattern '$+$' at the center of the forget images to train an attack model. In the main paper, we compare the results of several baselines and benchmark on the poisoned data while in the Appendix you find the results for the non-perturbed datasets (CIFAR10, Places-365, and ImageNet-1K) and image-outpainting. We have compared the unlearning results of AutoEncoder, VQ-GAN, and the diffusion model with $(i)$ Max loss baseline, it maximizes the model loss on forget samples \cite{halimi2022federated, warneckemachine}; $(ii)$ Noisy Label, it minimizes training loss with Gaussian noise as ground truth for forget samples \cite{gandikota2023erasing}; $(iii)$ Random Encoder, it minimizes the distance between the output of the encoder on the forget set and Gaussian Noise \cite{tarun2023deep}; and $(iv)$ state of the art I2I unlearning model \cite{li2024machine} (we call I2I SOTA), it minimizes the distance between the Encoder output and Gaussian noise while fine-tuning the encoder parameters on retain samples. 

\textbf{Evaluation Metrics.} To comprehensively evaluate the effectiveness of our unlearning approach, we utilize three key metrics: the Inception Score (IS) \cite{salimans2016improved}, which assesses the quality and diversity of the generated images by measuring how confidently they can be classified; $(ii)$ Fr\`echet inception distance (FID) \cite{heusel2017gans}, which quantifies the similarity between the distribution of generated images and real ground-truth images; and $(iii)$ CLIP embedding distance \cite{radford2021learning}, measures whether the generated outputs still captures similar semantics.

\subsection{Results and Discussions}

\cref{fig:attack_CIFAR10} and \cref{tab:CIFAR10_res} present the comparison of various unlearning algorithms on CIFAR-10 dataset, where the forget samples are embedded with the '$+$' at the center of the image. As shown in \cref{fig:attack_CIFAR10}, our model, like the retrained model, successfully avoids generating the '$+$' symbol in its outputs, indicating effective unlearning of the forget class. Notably, both our model and the retrained model retain their ability to generalize lines, colors, and patterns from the retain samples, preserving essential generative capabilities. Furthermore, \cref{tab:CIFAR10_res} confirms that our approach produces outputs closely aligned with those of the retrained model. At the same time, it achieves high-quality image generation, as indicated by a higher Inception Score (IS), reflecting crisp and detailed outputs. 
%\cref{tab:CIFAR10_res} also confirms that our approach generates outputs which are similar to the outputs from retrained model. At the same time, generate crisp outputs which is evident from higher IS score.   

In \cref{tab:VQ-GAN results} we extend the comparison of various unlearning algorithms on the attack model with $4 \times 4$, and $8 \times 8$ cropped patches on ImageNet-1K dataset. Our model consistently achieves superior performance on retained samples. For forget samples, it successfully generates outputs that are significantly distinct from those produced by the attack model, while maintaining high image quality (not necessarily accurate or reliable) and capturing similar semantics. \cref{fig:VQGAN_comp} shows some of the results from VQ-GAN model, as shown in the figure our approach preserves the performance on retain set while other approaches have traces of poisoned forget samples. For forget samples, our approach effectively unlearns the embedded '+' sign while introducing inaccurate patterns from the retain samples. 

Furthermore, to assess whether our approach effectively prevents the generation of the embedded '$+$' symbol, we benchmark various baselines and state-of-the-art (SOTA) algorithms on diffusion models using the Places365 dataset. \cref{tab:diff_model results} compares outputs with those of the original model that was not trained on forget samples containing the '$+$' marker. Our model performs well on retained samples in most cases. For forget samples, a low FID score relative to the original model indicates that our approach effectively removes the '$+$' sign, as further evidenced by high IS and CLIP scores. This demonstrates that our model not only maintains high-quality output, but also ensures effective unlearning of sensitive data. 

%Since, the output generated by our model is of high quality. In order to show whether our approach successfully does not generate the embedded '$+$' sign, we compare ours and various baseline and SOTA algorithms for diffusion models on Places365 dataset with the output generated by original model which is not trained on forget samples with '$+$' in \cref{tab:diff_model results}. Here as well, our algorithm has good results for the retain samples in almost all the cases. In case of forget samples, a low FID score against original model would suggest the images generated from our model does not generate '$+$' sign in the image, which is further verified by the high IS score and high CLIP score on forget set.

We compare the performance of all the approaches for unseen data in \cref{tab:OOD_model results}. We randomly sample the next 50 classes from ImageNet-1K and Places365 dataset for VQ-GAN and diffusion model respectively. It is clear from the results that our model has the best generative results on unseen data, particularly with diffusion models. We also perform T-SNE analysis to further validate the effectiveness of our approach, we randomly choose 50 outputs from retain and forget samples. We then compute the CLIP embedding vector for the generated out and the attack model. \cref{fig:T-SNE analysis} shows that the embedding vector from retain samples closely matches the ground truth, while the embedding vector from forget samples diverges. 

\section{Limitations and Future Work}
The proposed OpenFly platform incorporates various rendering engines/techniques to provide high-quality scenes. Specifically, this is the first attempt to use 3D GS reconstructed scenes to support real-to-sim training and testing, while in the reconstruction of large-scale areas, a few visual artifacts are inevitably present. Future work will focus on exploring more effective reconstruction methods to enhance realism in large-scale scenes. Besides, the proposed OpenFly-Agent is built upon the large VLN model architecture, which is not practical for real-time deployment on UAVs. To address this, future research should focus on developing more efficient architectures and effective quantization techniques. 


\section{Conclusion}
In this work, we present OpenFly, a platform designed for large-scale data collection in aerial Vision-and-Language Navigation (VLN). OpenFly integrates multiple rendering engines and advanced real-to-sim techniques for data generation, enabling efficient collection of diverse, high-quality aerial VLN data. The resulting large-scale dataset comprises 100k trajectories across 18 distinct scenes, spanning a wide range of altitudes and difficulty levels, which is significantly superior than existing ones. Furthermore, we propose OpenFly-Agent, a keyframe-aware aerial navigation model capable of directly predicting flight actions based on observations and language instructions. Extensive experiments validate the effectiveness of the proposed method, and establishing a comprehensive benchmark for future advancements in aerial navigation. 
%The toolchain, dataset, and code will be publicly released, providing a valuable resource for future research in this field.
%\section{Related work}
\label{sec:formatting}

\subsection{Text-to-Video Generation}

T2V generation has made notable progress, evolving from early GAN-based models \cite{saito2017temporal,tulyakov2018mocogan,fu2023tell,li2018video,wu2022nuwa,yu2022generating} to newer transformer \cite{yan2021videogpt,arnab2021vivit,esser2021taming,ramesh2021zero,yu2022scaling} and diffusion models \cite{kirkpatrick2017overcoming,sohl2015deep,song2020denoising,zhang2022gddim}. Early efforts like MoCoGAN~\cite{tulyakov2018mocogan} focused on short video clips but faced issues with stability and coherence. The introduction of transformers improved sequential data handling, enhancing video generation, while diffusion models further improved video quality by progressively denoising the input. 
Despite these advances, T2V models still struggle to reflect human preferences, with the generated videos generally lacking aesthetic quality. Additionally, the scarcity of paired video preference data hinders effective model training and may lead to insufficient flexibility and poor quality in the generated videos.


\subsection{RLHF}

\iffalse
Aligning LLMs \cite{dai1901transformer,radford2019language,zhang2023opt} typically involves two steps: supervised fine-tuning followed by Reinforcement Learning with Human Feedback (RLHF) \cite{gao2023scaling,stiennon2020learning,rafailov2024direct}. Although effective, RLHF is computationally expensive and can lead to issues like reward hacking. Methods like DPO have streamlined alignment by leveraging feedback data directly, improving efficiency.

In contrast, diffusion model alignment is still evolving, focusing mainly on enhancing visual quality through curated datasets. Techniques like DOODL \cite{wallace2023end} and AlignProp \cite{prabhudesai2023aligning} target aesthetic improvements but face challenges with complex tasks such as text-image alignment. Reinforcement learning methods like DPOK \cite{fan2024reinforcement} and DDPO \cite{black2023training} improve reward optimization but struggle with scalability. DPO-SDXL integrates DPO into T2I generation, boosting both alignment and aesthetics.

However, aligning video generation remains a largely unaddressed challenge, especially when dealing with motion consistency and semantic coherence across frames.
\fi

RLHF \cite{gao2023scaling,stiennon2020learning,rafailov2024direct} is a method that utilizes human feedback to guide machine learning models. Early RLHF algorithms, such as DDPG~\cite{lillicrap2015continuous} and PPO~\cite{schulman2017proximal}, typically relied on complex reward models to quantify human feedback. These reward models require a large amount of annotated data and face challenges during tuning. As research has progressed, more efficient preference learning methods have emerged, among which DPO has become a new framework. DPO does not depend on a separate reward model; instead, it obtains human preferences through pairwise comparisons and directly optimizes these preferences. This shift not only simplifies the application of RLHF but also enhances the alignment of models with human values. Furthermore, DPO has been successfully introduced into T2I tasks~\cite{wallace2024diffusion,yang2024using}, providing new insights for generative models in addressing the alignment of human preferences and showcasing DPO's potential in the field of AIGC~\cite{shi2024instantbooth,
qing2024hierarchical,menapace2024snap,koley2024s}. However, there remains a gap in current research regarding the application of DPO strategies to T2V tasks. Effectively integrating DPO into T2V tasks presents a challenging endeavor.


\small
\bibliographystyle{unsrt}
\bibliography{main}


\clearpage
% \setcounter{page}{1}
% \maketitlesupplementary
\begin{center}
Supplementary Material
\end{center}

% {
%     \onecolumn
%     \centering
%     \Large
%     \textbf{\thetitle}\\
%     \vspace{0.5em}Supplementary Material \\
%     \vspace{1.0em}
% }

\section{Proof of \cref{theorem:dr}}
We require some additional regularity assumptions:
\begin{assumption} 1) The number of classes $C$ is bounded w.r.t the number of samples $N$, 2) the missingness mechanism $P(A=1|Y,\theta)$, as well as its estimated counterpart $P(A=1|Y,\theta)$, are bounded below by some constant $\epsilon > 0$, 3) the quantities $P(Y|X,\theta)$ and $P(A|Y,\theta)$ are estimated using auxiliary samples independent of samples used for the sample averaging.
\label{assumption:extra}
\end{assumption}
Assumptions 1 and 2 are natural. For the missingness mechanism, the ground truth being bounded means that there is a non-vanishing proportion of samples for every class. The boundedness of the estimate can be enforced by clipping the estimate. Assumption 3 is called sample splitting in \cite{kennedy-dr}.

For convenience we use operator $\E_N$ to denote the average of $N$ samples i.e. $\frac{1}{N}\sum_{i=1}^N$. Note that this is by itself a random variable, in contrast to $\E$ which is a fixed number.

\begin{proof}[Proof of \cref{theorem:dr}] Because $C$ is bounded (assumption \ref{assumption:extra}), we can fix a class $c$ and prove the theorem.
Let us define the influence function $\phi$, parameterized by $\theta$, as
\begin{equation}
\phi(O | \theta)(c) = P(Y=c|X,\theta) + \frac{\one(A=1)}{P(A=1|Y,\theta)} (\one(Y=c) - P(Y=c|X,\theta)) - P(Y=c)
\end{equation}
As we have done in the main text, we use $\phi(O)$ to denote the same function but all estimated quantities are replaced with their truths. In other words, we use $\phi(O)$ for $\phi(O|\theta_0)$ where $\theta_0$ is the truth, given that our model contains $\theta_0$ e.g. when the model is consistent.

Recall that:
\begin{equation}
\begin{aligned}
\Psi_{dr}(\theta)(c) &= \frac{1}{N}\sum_{i=1}^N \left\{P(Y=c|X,\theta) + \frac{\one(A=1)}{P(A=1|Y,\theta)} (\one(Y=c) - P(Y=c|X,\theta))\right\}\\
&= \E_N [\phi(O|\theta)(c)] + P(Y=c)
\end{aligned}
\end{equation}

We will show that:
\begin{equation}
\Psi_{dr}(\theta)(c) - P(Y=c) = (\E_N - \E)[\phi(O)(c)] + o_P(N^{-1/2})
\label{eq:proof-linearity}
\end{equation}
To do that, we use the following decomposition
\begin{equation}
\begin{aligned}
\Psi_{dr}(\theta)(c) - P(Y=c) &= \E_N [\phi(O|\theta)(c)] \\
&= (\E_N - \E)[\phi(O)(c)] + (\E_N - \E)[\phi(O|\theta)(c) - \phi(O)(c)] + \E[\phi(O|\theta)(c)]
% &+ (\E_n - \E)[\phi(O;\theta) - \phi(O)]\\
% &+ \E[P(Y=c|X,\theta)] - \E[P(Y=c|X)] + \E[\phi(O,\theta)]
\end{aligned}
\end{equation}
and analyze the second and third term. The third term is:
\begin{equation}
\begin{aligned}
\E[\phi(O|\theta)(c)] &= \E[P(Y=c|X,\theta)] + \E\left[\frac{\one(A=1)}{P(A=1|Y,\theta)}(\one(Y=c) - P(Y=c|X,\theta))\right]- P(Y=c) \\
&= \E\left[P(Y=c|X,\theta) + \frac{P(A=1|Y)}{P(A=1|Y,\theta)}(P(Y=c|X) - P(Y=c|X,\theta))\right] - \E[P(Y=c|X)]\\
&= \E\left[(P(Y=c|X,\theta) - P(Y=c|X)) (P(A=1|Y,\theta) -P(A=1|Y)) \frac{1}{P(A=1|Y,\theta)}\right]\\
\end{aligned}
\end{equation}
by Cauchy-Schwarz inequality:
\begin{equation}
\begin{aligned}
\E[\phi(O|\theta)(c)] &\le \frac{1}{\epsilon} \|P(A=1|Y,\theta) - P(A=1|Y)\|_2 \|P(Y=c|X,\theta) - P(Y=c|X)\|_{L_2(P)}\\
&= \frac{1}{\epsilon} o_P(N^{-1/4} N^{-1/4}) = o_P(N^{-1/2})
\end{aligned}
\end{equation}
by assumption \ref{assumption:4th-root-n} and that $P(A=1|Y,\theta) > \epsilon$ (assumption \ref{assumption:extra}). The second term can be bounded by Chebyshev inequality
% \begin{equation}
% \begin{aligned}
% \E[\E_N[\phi(O|\theta)(c) - \phi(O)(c)]] &= \E[\phi(O|\theta)(c) - \phi(O)(c)]\\
% \var[\E_N[\phi(O|\theta)(c) - \phi(O)(c)]] &= \frac{1}{N}\var[\phi(O|\theta)(c) - \phi(O)(c)] \le 
% \end{aligned}
% \end{equation}
\begin{equation}
P(|(\E_N - \E)[\phi(O|\theta)(c) - \phi(O)(c)]| \ge t) \le \frac{\var[\E_N[\phi(O|\theta)(c) - \phi(O)(c)]]}{t^2} = \frac{\var[\phi(O|\theta)(c) - \phi(O)(c)]}{Nt^2}
\end{equation}
note here that $\theta$ is independent of the samples used for $\E_N$ by assumption \ref{assumption:extra}. For any $\varepsilon > 0$, by picking $t = \frac{1}{\sqrt{N\varepsilon}}$ we get
\begin{equation}
P\left(\left|\frac{(\E_N - \E)[\phi(O|\theta)(c) - \phi(O)(c)]}{N^{-1/2}}\right| \ge \frac{1}{\sqrt{\varepsilon}}\right) \le \varepsilon \var[\phi(O|\theta)(c) - \phi(O)(c)]
\end{equation}
by the definition of $O_P$, we then get
\begin{equation}
(\E_N - \E)[\phi(O|\theta)(c) - \phi(O)(c)] = O_P(N^{-1/2}\var[\phi(O|\theta)(c) - \phi(O)(c)])
\end{equation}
Because $\phi$ is a continuous function of $P(Y|X,\theta)$ and $P(A|Y,\theta)$ (given $P(A|Y,\theta) > \epsilon$, assumption \ref{assumption:extra}), by the continuous mapping theorem and the fact that $P(Y|X,\theta)$ and $P(A|Y,\theta)$ are convergent in probability (assumption \ref{assumption:4th-root-n}), we get $\var[\phi(O|\theta)(c) - \phi(O)(c)] = o_P(1)$. This gives
\begin{equation}
(\E_N - \E)[\phi(O|\theta)(c) - \phi(O)(c)] = o_P(N^{-1/2})
\end{equation}
Therefore, we have shown that the second and third term are both $o_P(N^{-1/2})$, proving \cref{eq:proof-linearity}. As the final step, multiply both sides of this equation by $\sqrt{N}$ we get:
\begin{equation}
\sqrt{N}(\Psi_{dr}(\theta)(c) - P(Y=c)) = \sqrt{N} (\E_N - \E)[\phi(O)(c)] + o_P(1) \rightsquigarrow \mathcal{N}(0, \var[\phi(O)(c)])
\end{equation}
by the central limit theorem, and $\var[\phi(O)(c)] = \E[\phi(O)(c)^2]$ because $\E[\phi(O)(c)] = 0$.
\end{proof}

While we started with the definition of $\phi$, \cref{eq:proof-linearity} shows that $\phi$ is indeed an influence function. Now we show that $\phi$ is also the efficient influence function, by using the characterization of the model's tangent space \cite{tsiatis-missingdata}. Note that the joint probability factorizes as $P(X,A,Y) = P(X)P(Y|X)P(A|Y)$, therefore the tangent space $\mathcal{T}$ factorizes as $\mathcal{T} = \mathcal{T}_{X} \oplus \mathcal{T}_{Y|X} \oplus \mathcal{T}_{A|Y}$ where $\mathcal{T}_X = \{h(X): \E[h] = 0\}$, $\mathcal{T}_{Y|X} = \{h(X,Y): \E[h|X] = 0\}$, $\mathcal{T}_{A|Y} = \{h(A,Y): \E[h|Y] = 0\}$, and the 3 subspaces are pairwise orthogonal. All influence functions are orthogonal to the tangent space, but the influence function that is also in the tangent space has the smallest variance and is called the efficient influence function. As $\phi$ is already an influence function, we need only show that $\phi$ is in $\mathcal{T}$. We write $\phi$ as
\begin{equation}
\phi(O)(c) = (P(Y=c|X) - P(Y=c)) + \left[\frac{\one(A=1)}{P(A=1|Y)} - 1\right](\one(Y=c) - P(Y=c|X)) + (\one(Y=c) - P(Y=c|X))
\end{equation}
and note that the first, second and third term are in $\mathcal{T}_X$, $\mathcal{T}_{A|Y}$ and $\mathcal{T}_{Y|X}$ respectively. Therefore, $\phi$ is indeed in $\mathcal{T}$. The efficient influence function has the smallest variance of all influence function, and therefore our estimator being asymptotically linear in $\phi$ (\cref{eq:proof-linearity}) has the smallest mean squared error in a local asymptotic minimax sense \cite{kennedy-dr, asymptoticstatistics}

\section{Further background and related work}
\paragraph{Discussion on semi-supervised EM.}
It appears that semi-supervised EM was first used for parameter estimation when the missingness mechanism is non-ignorable in \cite{ibrahim1996parameter}, but has not been used for label shift estimation.
Perhaps this is because the semi-supervised situation where additional unlabeled data is available during training is rarer than the test-time adaptation case. EM is well suited to take advantage of the extra unlabeled data to improve the classifier under very scarce and long-tailed labeled data. While the connection between pseudo-labeling and EM has been explored before \cite{entropyminimization}, the situation with label shift has not until recently \cite{simpro}. Here the application of EM is much more interesting, because other than simply giving pseudo-labeling a rigorous formulation, EM also estimates the missingness mechanism (equivalently the label distribution shift), which is important for shift correction and thus high-quality pseudo-labels \cite{acr}. The application of confidence thresholding can be seen as a sparse variant of EM \cite{neal1998view}.

\paragraph{The doubly-robust risk.} 
\label{subsec:dr-risk}
A technique that also derives from the theory of semi-parametric efficiency is orthogonal statistical learning \citep{foster2023orthogonal}. The idea is to minimize the doubly-robust risk:
\label{subsec:method-dr-risk}
\begin{equation}
\label{eq:dr-risk}
\mathcal{R}(\theta_2) = \frac{1}{N} \sum_{i=1}^N \Bigg[ l(x_i, \hat y_i|\theta_2) + \frac{\one(a_i=1)}{P(A=a_i|Y=y_i, \theta_1)} (l(x_i, y_i | \theta_2) - l(x_i, \hat y_i | \theta_2))\Bigg]
\end{equation}
where $l(x,y|\theta) = -\sum_{c=1}^C [y]_c \log P(Y=c|X=x,\theta)$ is the negative cross-entropy. 
The notation $[y]_c$ means that we are using the $c$-entry in a C-dimension probability vector $y$. 
Thus, $y_i$ denotes the one-hot label of observation $i$, while $\hat y_i$ denotes the pseudo-label, which can be one-hot or all-zero. 
Finally, we use $\theta_1$ to denote that $P(a|y,\theta_1)$ is an estimation from a previous stage, but it can be estimated with $\theta_2$ as well. 
The risk $\mathcal{R}(\theta_2)$ can be used as a training loss in a straightforward fashion. 
Similar to the doubly robust estimation of $P(Y)$, the doubly robust risk provides approximately unbiased estimation of the risk. 
This property has been used in \citep{arelabelsinformative, onnonrandommissinglabels, drst} also in the semi-supervised learning setting.
More broadly, it is at the heart of one of the core techniques in heterogenous treatment effect estimation in causal estimation \cite{kennedy2023towards, foster2023orthogonal, wager2018estimation}. 
The focus here is not the estimation of $\mathcal{R}(\theta_2)$ per se, but the quality of the learned model \cite{foster2023orthogonal}.
By using the doubly-robust risk, we can achieve an optimality result similar in spirit to our theorem \cref{theorem:dr}, but for the generalization error.
While this is appealing, in practice there are 2 problems with this approach. First, the inverse probability weight $P(A=a_i|Y=y_i,\theta_1)$ can be very large if the class ratio is highly unlabeled, making training unstable \cite{kallus2020deepmatch, pham2023stable}. 
This problem exists for our estimation as well. However, it is much easier to control for estimation than for training because of the iterative nature of model update. Secondly, we can further write $\mathcal{R}$ as:
\begin{equation}
\mathcal{R}(\theta_2) = \frac{1}{N}\sum_{i=1}^N l\left(x_i, \hat y_i + \frac{\one(a_i=1)}{P(A=a_i|Y=y_i,\theta_1)} (y_i - \hat y_i)\Bigg\vert\theta_2\right)
\end{equation}
which is a cross-entropy loss with new meta-pseudo-labels. However, these labels are not meant to be learned exactly, and furthermore they can be negative. Thus, theoretical works have to put stringent assumptions on the models. In \cref{subsec:ablation-1}, we show that experimentally that the instability problem makes doubly-robust risk performance worse than our 2-stage approach.

\section{Training and hyperparameter settings.}
\label{subsec:training-setting}
For neural network training, we follow the implementation and hyperparameter settings of \cite{simpro}. In particular, we adapt the core code of SimPro for Supervised, MLE and EM. For MLE, we update $P(A|Y)$ using the Adam optimizer with learning rate 1e-3, while for EM we use a momentum update similar to SimPro's update of $P(Y|A)$ because it has a a closed-form solution at each mini-batch. We use Wide ResNet-28-2 on all methods and all datasets in this section, including Imagenet-127, because we are motivated by the fact that stage-1's goal is not classification accuracy but the estimation of a finite-dimensional parameter. When using Wide ResNet-28-2 for Imagenet-127, we use the hyperparameters of CIFAR-100, except we lower the batch size of unlabeled data to 2 times that of labeled data instead of 8 for memory reason. We do not perform additional hyperparameter tuning. All experiments can be performed on 1 A6000 RTX GPU, and are run 3 times. We report the total variation distance between the estimated and the ground truth unlabeled class distribution, similar to its usage in Theorem 3.1 of \cite{lsc}, and the top-1 classification accuracy.

In the second stage of our algorithm, we freeze our estimation and plug it in SimPro and BOAT.
We keep exactly the same hyperparameter settings that SimPro and BOAT use. In particular, for Imagenet-127, we now use ResNet-50 and run each experiment once.
In SimPro, we set the unlabeled class distribution $P(Y|A=0)$ at the E-step;  however, we still keep a running estimate of the class distribution $P(Y)$ in the logit adjustment loss \cref{eq:simpro-la-loss}. While it is possible to use the first stage estimate in the logit adjustment loss, we observe that doing so results in lower accuracy than using the the running average. This is conceptually consistent with the role of the running average - serving not as an accurate estimate of $P(Y)$ but to make the classifier's class distribution uniform through the logit adjustment loss, which is good for the test set. Similarly, in BOAT, we only replace $\Delta_c = \log P(Y|A=1) - \log P(Y|A=0)$ in equation (4) of \cite{boat}, which is adjusting a classifier's predictions from the labeled to the unlabeled class distribution, with our SimPro + DR estimate instead of their on-the-fly estimate. 


% \section{Additional experiments}
% % \begin{table*}[t]
\centering
\caption{Total Variation Distance on CIFAR-10-LT ($N_l = 500$, $M_l = 4000$) with different class imbalance ratios $\gamma_l$ and $\gamma_u$ under five different unlabeled class distributions.}
\label{tab:cifar10-tv}
\resizebox{\textwidth}{!}{
\begin{tabular}{lccccccccccc}
\toprule
& & \multicolumn{2}{c}{consistent} & \multicolumn{2}{c}{uniform} & \multicolumn{2}{c}{reversed} & \multicolumn{2}{c}{middle} & \multicolumn{2}{c}{head-tail} \\
\cmidrule(lr){3-4} \cmidrule(lr){5-6} \cmidrule(lr){7-8} \cmidrule(lr){9-10} \cmidrule(lr){11-12}
& & $\gamma_l = 150$ & $\gamma_l = 100$ & $\gamma_l = 150$ & $\gamma_l = 100$ & $\gamma_l = 150$ & $\gamma_l = 100$ & $\gamma_l = 150$ & $\gamma_l = 100$ & $\gamma_l = 150$ & $\gamma_l = 100$ \\
Model & Estimator & $\gamma_u = 150$ & $\gamma_u = 100$ & $\gamma_u = 1$ & $\gamma_u = 1$ & $\gamma_u = 1/150$ & $\gamma_u = 1/100$ & $\gamma_u = 150$ & $\gamma_u = 100$ & $\gamma_u = 150$ & $\gamma_u = 100$ \\
\midrule
Supervised & MLLS & 0.269 ± 0.252 & 0.038 ± 0.006 & 0.251 ± 0.046 & 0.255 ± 0.060 & 0.429 ± 0.028 & 0.493 ± 0.050 & 0.333 ± 0.042 & 0.320 ± 0.009 & 0.457 ± 0.034 & 0.444 ± 0.043 \\
Supervised & RLLS & 0.043 ± 0.001 & 0.044 ± 0.010 & 0.348 ± 0.034 & 0.305 ± 0.068 & 0.769 ± 0.016 & 0.678 ± 0.028 & 0.430 ± 0.008 & 0.368 ± 0.013 & 0.539 ± 0.018 & 0.503 ± 0.020 \\
\midrule
MLE & IPW & 0.027 ± 0.001 & 0.027 ± 0.000 & 0.319 ± 0.072 & 0.243 ± 0.010 & 0.674 ± 0.020 & 0.646 ± 0.041 & 0.438 ± 0.020 & 0.454 ± 0.026 & 0.547 ± 0.049 & 0.491 ± 0.059 \\
MLE & OR & 0.045 ± 0.004 & 0.042 ± 0.000 & 0.215 ± 0.026 & 0.203 ± 0.032 & 0.433 ± 0.017 & 0.395 ± 0.033 & 0.193 ± 0.006 & 0.209 ± 0.037 & 0.307 ± 0.147 & 0.249 ± 0.130 \\
MLE & DR & 0.090 ± 0.002 & 0.079 ± 0.000 & 0.407 ± 0.027 & 0.360 ± 0.007 & 0.425 ± 0.007 & 0.421 ± 0.029 & 0.256 ± 0.001 & 0.286 ± 0.031 & 0.435 ± 0.136 & 0.362 ± 0.122 \\
\midrule
EM & IPW & 0.035 ± 0.002 & 0.040 ± 0.001 & 0.021 ± 0.001 & 0.029 ± 0.015 & 0.303 ± 0.187 & 0.091 ± 0.010 & 0.119 ± 0.011 & 0.105 ± 0.022 & 0.104 ± 0.026 & 0.104 ± 0.051 \\
EM & OR & 0.037 ± 0.003 & 0.042 ± 0.002 & 0.016 ± 0.001 & 0.024 ± 0.012 & 0.269 ± 0.183 & 0.090 ± 0.008 & 0.122 ± 0.012 & 0.103 ± 0.022 & 0.072 ± 0.012 & 0.073 ± 0.024 \\
EM & DR & 0.034 ± 0.004 & 0.037 ± 0.001 & 0.014 ± 0.001 & 0.027 ± 0.020 & 0.264 ± 0.191 & 0.092 ± 0.005 & 0.111 ± 0.019 & 0.097 ± 0.026 & 0.077 ± 0.016 & 0.073 ± 0.028 \\
\midrule
SimPro & IPW & 0.070 ± 0.011 & 0.058 ± 0.000 & 0.046 ± 0.001 & 0.049 ± 0.005 & 0.254 ± 0.074 & 0.223 ± 0.098 & 0.097 ± 0.025 & 0.067 ± 0.002 & 0.105 ± 0.066 & 0.110 ± 0.079 \\
SimPro & OR & 0.071 ± 0.012 & 0.058 ± 0.000 & 0.045 ± 0.001 & 0.049 ± 0.006 & 0.040 ± 0.003 & 0.059 ± 0.017 & 0.074 ± 0.006 & 0.075 ± 0.002 & 0.033 ± 0.003 & 0.033 ± 0.003 \\
SimPro & DR & 0.017 ± 0.004 & 0.026 ± 0.001 & 0.019 ± 0.002 & 0.018 ± 0.003 & 0.039 ± 0.003 & 0.058 ± 0.025 & 0.091 ± 0.007 & 0.031 ± 0.001 & 0.015 ± 0.003 & 0.019 ± 0.007 \\
\bottomrule
\end{tabular}
}
\end{table*}
% 

\begin{table*}[t]
\centering
\caption{Total Variation Distance on CIFAR-100-LT ($N_l = 50$, $M_l = 400$) with different class imbalance ratios $\gamma_l$ and $\gamma_u$ under five different unlabeled class distributions.}
\label{tab:cifar100-tv}
\resizebox{\textwidth}{!}{
\begin{tabular}{lccccccccccc}
\toprule
& & \multicolumn{2}{c}{consistent} & \multicolumn{2}{c}{uniform} & \multicolumn{2}{c}{reversed} & \multicolumn{2}{c}{middle} & \multicolumn{2}{c}{head-tail} \\
\cmidrule(lr){3-4} \cmidrule(lr){5-6} \cmidrule(lr){7-8} \cmidrule(lr){9-10} \cmidrule(lr){11-12}
& & $\gamma_l = 20$ & $\gamma_l = 10$ & $\gamma_l = 20$ & $\gamma_l = 10$ & $\gamma_l = 20$ & $\gamma_l = 10$ & $\gamma_l = 20$ & $\gamma_l = 10$ & $\gamma_l = 20$ & $\gamma_l = 10$ \\
Model & Estimator & $\gamma_u = 20$ & $\gamma_u = 10$ & $\gamma_u = 1$ & $\gamma_u = 1$ & $\gamma_u = 1/20$ & $\gamma_u = 1/10$ & $\gamma_u = 20$ & $\gamma_u = 10$ & $\gamma_u = 20$ & $\gamma_u = 10$ \\
\midrule
Supervised & MLLS & 0.707 ± 0.016 & 0.313 ± 0.100 & 0.445 ± 0.172 & 0.309 ± 0.119 & 0.383 ± 0.075 & 0.397 ± 0.006 & 0.570 ± 0.001 & 0.373 ± 0.107 & 0.543 ± 0.009 & 0.231 ± 0.057 \\
Supervised & RLLS & 0.520 ± 0.007 & 0.133 ± 0.003 & 0.337 ± 0.125 & 0.253 ± 0.082 & 0.424 ± 0.060 & 0.463 ± 0.003 & 0.454 ± 0.021 & 0.306 ± 0.074 & 0.460 ± 0.028 & 0.241 ± 0.040 \\
\midrule
MLE & IPW & 0.075 ± 0.000 & 0.071 ± 0.001 & 0.229 ± 0.001 & 0.167 ± 0.002 & 0.565 ± 0.005 & 0.443 ± 0.007 & 0.415 ± 0.000 & 0.311 ± 0.005 & 0.343 ± 0.000 & 0.280 ± 0.001 \\
MLE & OR & 0.065 ± 0.002 & 0.061 ± 0.001 & 0.200 ± 0.007 & 0.143 ± 0.001 & 0.526 ± 0.011 & 0.399 ± 0.023 & 0.360 ± 0.003 & 0.256 ± 0.012 & 0.328 ± 0.003 & 0.266 ± 0.005 \\
MLE & DR & 0.149 ± 0.019 & 0.145 ± 0.010 & 0.243 ± 0.004 & 0.214 ± 0.019 & 0.568 ± 0.005 & 0.464 ± 0.014 & 0.403 ± 0.014 & 0.309 ± 0.012 & 0.365 ± 0.007 & 0.320 ± 0.004 \\
\midrule
EM & IPW & 0.097 ± 0.008 & 0.092 ± 0.004 & 0.239 ± 0.007 & 0.179 ± 0.003 & 0.478 ± 0.012 & 0.329 ± 0.020 & 0.262 ± 0.016 & 0.202 ± 0.003 & 0.312 ± 0.002 & 0.227 ± 0.001 \\
EM & OR & 0.121 ± 0.007 & 0.108 ± 0.005 & 0.261 ± 0.007 & 0.189 ± 0.004 & 0.489 ± 0.013 & 0.335 ± 0.020 & 0.274 ± 0.016 & 0.211 ± 0.004 & 0.336 ± 0.003 & 0.235 ± 0.001 \\
EM & DR & 0.125 ± 0.005 & 0.111 ± 0.004 & 0.269 ± 0.007 & 0.194 ± 0.005 & 0.497 ± 0.010 & 0.336 ± 0.024 & 0.281 ± 0.019 & 0.219 ± 0.008 & 0.336 ± 0.007 & 0.233 ± 0.004 \\
\midrule
SimPro & IPW & 0.125 ± 0.001 & 0.100 ± 0.005 & 0.166 ± 0.007 & 0.141 ± 0.009 & 0.353 ± 0.023 & 0.261 ± 0.008 & 0.202 ± 0.003 & 0.158 ± 0.005 & 0.277 ± 0.009 & 0.197 ± 0.003 \\
SimPro & OR & 0.133 ± 0.005 & 0.100 ± 0.004 & 0.160 ± 0.007 & 0.138 ± 0.010 & 0.322 ± 0.014 & 0.253 ± 0.008 & 0.202 ± 0.003 & 0.156 ± 0.005 & 0.269 ± 0.006 & 0.191 ± 0.004 \\
SimPro & DR & 0.122 ± 0.003 & 0.106 ± 0.006 & 0.188 ± 0.009 & 0.149 ± 0.006 & 0.343 ± 0.023 & 0.257 ± 0.007 & 0.219 ± 0.010 & 0.172 ± 0.002 & 0.279 ± 0.007 & 0.198 ± 0.004 \\
\bottomrule
\end{tabular}
}
\end{table*}

\end{document}