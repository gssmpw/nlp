% CVPR 2025 Paper Template; see https://github.com/cvpr-org/author-kit

\documentclass[10pt,twocolumn,letterpaper]{article}

%%%%%%%%% PAPER TYPE  - PLEASE UPDATE FOR FINAL VERSION
%\usepackage{cvpr}              % To produce the CAMERA-READY version
%\usepackage[review]{cvpr}      % To produce the REVIEW version
\usepackage[pagenumbers]{cvpr} % To force page numbers, e.g. for an arXiv version
\usepackage{makecell}
\usepackage{multirow}
\usepackage{graphicx}

% Import additional packages in the preamble file, before hyperref
%
% --- inline annotations
%
\newcommand{\red}[1]{{\color{red}#1}}
\newcommand{\todo}[1]{{\color{red}#1}}
\newcommand{\TODO}[1]{\textbf{\color{red}[TODO: #1]}}
% --- disable by uncommenting  
% \renewcommand{\TODO}[1]{}
% \renewcommand{\todo}[1]{#1}



\newcommand{\VLM}{LVLM\xspace} 
\newcommand{\ours}{PeKit\xspace}
\newcommand{\yollava}{Yo’LLaVA\xspace}

\newcommand{\thisismy}{This-Is-My-Img\xspace}
\newcommand{\myparagraph}[1]{\noindent\textbf{#1}}
\newcommand{\vdoro}[1]{{\color[rgb]{0.4, 0.18, 0.78} {[V] #1}}}
% --- disable by uncommenting  
% \renewcommand{\TODO}[1]{}
% \renewcommand{\todo}[1]{#1}
\usepackage{slashbox}
% Vectors
\newcommand{\bB}{\mathcal{B}}
\newcommand{\bw}{\mathbf{w}}
\newcommand{\bs}{\mathbf{s}}
\newcommand{\bo}{\mathbf{o}}
\newcommand{\bn}{\mathbf{n}}
\newcommand{\bc}{\mathbf{c}}
\newcommand{\bp}{\mathbf{p}}
\newcommand{\bS}{\mathbf{S}}
\newcommand{\bk}{\mathbf{k}}
\newcommand{\bmu}{\boldsymbol{\mu}}
\newcommand{\bx}{\mathbf{x}}
\newcommand{\bg}{\mathbf{g}}
\newcommand{\be}{\mathbf{e}}
\newcommand{\bX}{\mathbf{X}}
\newcommand{\by}{\mathbf{y}}
\newcommand{\bv}{\mathbf{v}}
\newcommand{\bz}{\mathbf{z}}
\newcommand{\bq}{\mathbf{q}}
\newcommand{\bff}{\mathbf{f}}
\newcommand{\bu}{\mathbf{u}}
\newcommand{\bh}{\mathbf{h}}
\newcommand{\bb}{\mathbf{b}}

\newcommand{\rone}{\textcolor{green}{R1}}
\newcommand{\rtwo}{\textcolor{orange}{R2}}
\newcommand{\rthree}{\textcolor{red}{R3}}
\usepackage{amsmath}
%\usepackage{arydshln}
\DeclareMathOperator{\similarity}{sim}
\DeclareMathOperator{\AvgPool}{AvgPool}

\newcommand{\argmax}{\mathop{\mathrm{argmax}}}     



% It is strongly recommended to use hyperref, especially for the review version.
% hyperref with option pagebackref eases the reviewers' job.
% Please disable hyperref *only* if you encounter grave issues, 
% e.g. with the file validation for the camera-ready version.
%
% If you comment hyperref and then uncomment it, you should delete *.aux before re-running LaTeX.
% (Or just hit 'q' on the first LaTeX run, let it finish, and you should be clear).
\definecolor{cvprblue}{rgb}{0.21,0.49,0.74}
\DeclareMathOperator*{\argmax}{arg\,max}
\DeclareMathOperator*{\argmin}{arg\,min}
\newcommand{\mycomment}[1]{}
\usepackage[pagebackref,breaklinks,colorlinks,allcolors=cvprblue]{hyperref}

%%%%%%%%% PAPER ID  - PLEASE UPDATE
\def\paperID{14699} % *** Enter the Paper ID here
\def\confName{CVPR}
\def\confYear{2025}

%%%%%%%%% TITLE - PLEASE UPDATE
\title{Realistic Image-to-Image Machine Unlearning via \\ Decoupling and Knowledge Retention}

%%%%%%%%% AUTHORS - PLEASE UPDATE
\author{Ayush K. Varshney \\
Umeå University\\
Umeå, Sweden\\
{\tt\small ayushkv@cs.umu.se}
% For a paper whose authors are all at the same institution,
% omit the following lines up until the closing ``}''.
% Additional authors and addresses can be added with ``\and'',
% just like the second author.
% To save space, use either the email address or home page, not both
\and
Vicen\c{c} Torra\\
Umeå University\\
Umeå, Sweden\\
{\tt\small vtorra@cs.umu.se}
}

\begin{document}
\maketitle
\footnotetext[1]{Under Review.}
\begin{abstract}


The choice of representation for geographic location significantly impacts the accuracy of models for a broad range of geospatial tasks, including fine-grained species classification, population density estimation, and biome classification. Recent works like SatCLIP and GeoCLIP learn such representations by contrastively aligning geolocation with co-located images. While these methods work exceptionally well, in this paper, we posit that the current training strategies fail to fully capture the important visual features. We provide an information theoretic perspective on why the resulting embeddings from these methods discard crucial visual information that is important for many downstream tasks. To solve this problem, we propose a novel retrieval-augmented strategy called RANGE. We build our method on the intuition that the visual features of a location can be estimated by combining the visual features from multiple similar-looking locations. We evaluate our method across a wide variety of tasks. Our results show that RANGE outperforms the existing state-of-the-art models with significant margins in most tasks. We show gains of up to 13.1\% on classification tasks and 0.145 $R^2$ on regression tasks. All our code and models will be made available at: \href{https://github.com/mvrl/RANGE}{https://github.com/mvrl/RANGE}.

\end{abstract}

    
\section{Introduction}
Backdoor attacks pose a concealed yet profound security risk to machine learning (ML) models, for which the adversaries can inject a stealth backdoor into the model during training, enabling them to illicitly control the model's output upon encountering predefined inputs. These attacks can even occur without the knowledge of developers or end-users, thereby undermining the trust in ML systems. As ML becomes more deeply embedded in critical sectors like finance, healthcare, and autonomous driving \citep{he2016deep, liu2020computing, tournier2019mrtrix3, adjabi2020past}, the potential damage from backdoor attacks grows, underscoring the emergency for developing robust defense mechanisms against backdoor attacks.

To address the threat of backdoor attacks, researchers have developed a variety of strategies \cite{liu2018fine,wu2021adversarial,wang2019neural,zeng2022adversarial,zhu2023neural,Zhu_2023_ICCV, wei2024shared,wei2024d3}, aimed at purifying backdoors within victim models. These methods are designed to integrate with current deployment workflows seamlessly and have demonstrated significant success in mitigating the effects of backdoor triggers \cite{wubackdoorbench, wu2023defenses, wu2024backdoorbench,dunnett2024countering}.  However, most state-of-the-art (SOTA) backdoor purification methods operate under the assumption that a small clean dataset, often referred to as \textbf{auxiliary dataset}, is available for purification. Such an assumption poses practical challenges, especially in scenarios where data is scarce. To tackle this challenge, efforts have been made to reduce the size of the required auxiliary dataset~\cite{chai2022oneshot,li2023reconstructive, Zhu_2023_ICCV} and even explore dataset-free purification techniques~\cite{zheng2022data,hong2023revisiting,lin2024fusing}. Although these approaches offer some improvements, recent evaluations \cite{dunnett2024countering, wu2024backdoorbench} continue to highlight the importance of sufficient auxiliary data for achieving robust defenses against backdoor attacks.

While significant progress has been made in reducing the size of auxiliary datasets, an equally critical yet underexplored question remains: \emph{how does the nature of the auxiliary dataset affect purification effectiveness?} In  real-world  applications, auxiliary datasets can vary widely, encompassing in-distribution data, synthetic data, or external data from different sources. Understanding how each type of auxiliary dataset influences the purification effectiveness is vital for selecting or constructing the most suitable auxiliary dataset and the corresponding technique. For instance, when multiple datasets are available, understanding how different datasets contribute to purification can guide defenders in selecting or crafting the most appropriate dataset. Conversely, when only limited auxiliary data is accessible, knowing which purification technique works best under those constraints is critical. Therefore, there is an urgent need for a thorough investigation into the impact of auxiliary datasets on purification effectiveness to guide defenders in  enhancing the security of ML systems. 

In this paper, we systematically investigate the critical role of auxiliary datasets in backdoor purification, aiming to bridge the gap between idealized and practical purification scenarios.  Specifically, we first construct a diverse set of auxiliary datasets to emulate real-world conditions, as summarized in Table~\ref{overall}. These datasets include in-distribution data, synthetic data, and external data from other sources. Through an evaluation of SOTA backdoor purification methods across these datasets, we uncover several critical insights: \textbf{1)} In-distribution datasets, particularly those carefully filtered from the original training data of the victim model, effectively preserve the model’s utility for its intended tasks but may fall short in eliminating backdoors. \textbf{2)} Incorporating OOD datasets can help the model forget backdoors but also bring the risk of forgetting critical learned knowledge, significantly degrading its overall performance. Building on these findings, we propose Guided Input Calibration (GIC), a novel technique that enhances backdoor purification by adaptively transforming auxiliary data to better align with the victim model’s learned representations. By leveraging the victim model itself to guide this transformation, GIC optimizes the purification process, striking a balance between preserving model utility and mitigating backdoor threats. Extensive experiments demonstrate that GIC significantly improves the effectiveness of backdoor purification across diverse auxiliary datasets, providing a practical and robust defense solution.

Our main contributions are threefold:
\textbf{1) Impact analysis of auxiliary datasets:} We take the \textbf{first step}  in systematically investigating how different types of auxiliary datasets influence backdoor purification effectiveness. Our findings provide novel insights and serve as a foundation for future research on optimizing dataset selection and construction for enhanced backdoor defense.
%
\textbf{2) Compilation and evaluation of diverse auxiliary datasets:}  We have compiled and rigorously evaluated a diverse set of auxiliary datasets using SOTA purification methods, making our datasets and code publicly available to facilitate and support future research on practical backdoor defense strategies.
%
\textbf{3) Introduction of GIC:} We introduce GIC, the \textbf{first} dedicated solution designed to align auxiliary datasets with the model’s learned representations, significantly enhancing backdoor mitigation across various dataset types. Our approach sets a new benchmark for practical and effective backdoor defense.



\section{Related Work}

\subsection{Large 3D Reconstruction Models}
Recently, generalized feed-forward models for 3D reconstruction from sparse input views have garnered considerable attention due to their applicability in heavily under-constrained scenarios. The Large Reconstruction Model (LRM)~\cite{hong2023lrm} uses a transformer-based encoder-decoder pipeline to infer a NeRF reconstruction from just a single image. Newer iterations have shifted the focus towards generating 3D Gaussian representations from four input images~\cite{tang2025lgm, xu2024grm, zhang2025gslrm, charatan2024pixelsplat, chen2025mvsplat, liu2025mvsgaussian}, showing remarkable novel view synthesis results. The paradigm of transformer-based sparse 3D reconstruction has also successfully been applied to lifting monocular videos to 4D~\cite{ren2024l4gm}. \\
Yet, none of the existing works in the domain have studied the use-case of inferring \textit{animatable} 3D representations from sparse input images, which is the focus of our work. To this end, we build on top of the Large Gaussian Reconstruction Model (GRM)~\cite{xu2024grm}.

\subsection{3D-aware Portrait Animation}
A different line of work focuses on animating portraits in a 3D-aware manner.
MegaPortraits~\cite{drobyshev2022megaportraits} builds a 3D Volume given a source and driving image, and renders the animated source actor via orthographic projection with subsequent 2D neural rendering.
3D morphable models (3DMMs)~\cite{blanz19993dmm} are extensively used to obtain more interpretable control over the portrait animation. For example, StyleRig~\cite{tewari2020stylerig} demonstrates how a 3DMM can be used to control the data generated from a pre-trained StyleGAN~\cite{karras2019stylegan} network. ROME~\cite{khakhulin2022rome} predicts vertex offsets and texture of a FLAME~\cite{li2017flame} mesh from the input image.
A TriPlane representation is inferred and animated via FLAME~\cite{li2017flame} in multiple methods like Portrait4D~\cite{deng2024portrait4d}, Portrait4D-v2~\cite{deng2024portrait4dv2}, and GPAvatar~\cite{chu2024gpavatar}.
Others, such as VOODOO 3D~\cite{tran2024voodoo3d} and VOODOO XP~\cite{tran2024voodooxp}, learn their own expression encoder to drive the source person in a more detailed manner. \\
All of the aforementioned methods require nothing more than a single image of a person to animate it. This allows them to train on large monocular video datasets to infer a very generic motion prior that even translates to paintings or cartoon characters. However, due to their task formulation, these methods mostly focus on image synthesis from a frontal camera, often trading 3D consistency for better image quality by using 2D screen-space neural renderers. In contrast, our work aims to produce a truthful and complete 3D avatar representation from the input images that can be viewed from any angle.  

\subsection{Photo-realistic 3D Face Models}
The increasing availability of large-scale multi-view face datasets~\cite{kirschstein2023nersemble, ava256, pan2024renderme360, yang2020facescape} has enabled building photo-realistic 3D face models that learn a detailed prior over both geometry and appearance of human faces. HeadNeRF~\cite{hong2022headnerf} conditions a Neural Radiance Field (NeRF)~\cite{mildenhall2021nerf} on identity, expression, albedo, and illumination codes. VRMM~\cite{yang2024vrmm} builds a high-quality and relightable 3D face model using volumetric primitives~\cite{lombardi2021mvp}. One2Avatar~\cite{yu2024one2avatar} extends a 3DMM by anchoring a radiance field to its surface. More recently, GPHM~\cite{xu2025gphm} and HeadGAP~\cite{zheng2024headgap} have adopted 3D Gaussians to build a photo-realistic 3D face model. \\
Photo-realistic 3D face models learn a powerful prior over human facial appearance and geometry, which can be fitted to a single or multiple images of a person, effectively inferring a 3D head avatar. However, the fitting procedure itself is non-trivial and often requires expensive test-time optimization, impeding casual use-cases on consumer-grade devices. While this limitation may be circumvented by learning a generalized encoder that maps images into the 3D face model's latent space, another fundamental limitation remains. Even with more multi-view face datasets being published, the number of available training subjects rarely exceeds the thousands, making it hard to truly learn the full distibution of human facial appearance. Instead, our approach avoids generalizing over the identity axis by conditioning on some images of a person, and only generalizes over the expression axis for which plenty of data is available. 

A similar motivation has inspired recent work on codec avatars where a generalized network infers an animatable 3D representation given a registered mesh of a person~\cite{cao2022authentic, li2024uravatar}.
The resulting avatars exhibit excellent quality at the cost of several minutes of video capture per subject and expensive test-time optimization.
For example, URAvatar~\cite{li2024uravatar} finetunes their network on the given video recording for 3 hours on 8 A100 GPUs, making inference on consumer-grade devices impossible. In contrast, our approach directly regresses the final 3D head avatar from just four input images without the need for expensive test-time fine-tuning.


\section{Preliminaries}
\label{sec:prelims}

%\subsection{Gradient Ascent}

%Gradient ascent (GA) is an optimization technique aimed at maximizing a given loss function for a specified set of examples. This approach is particularly useful for unlearning, as it allows for the approximate removal of specific samples by adjusting the model weights in the direction that maximizes the loss on the target samples \cite{tarun2023fast}. Let $\theta$ represent the model weight, $\eta$ be the learning rate and $L$ be the loss function, then GA iteratively updates the model weights in the following manner.
%\begin{equation}
%    \theta_{t+1} = \theta_t + \eta\frac{\delta L}{\delta \theta_t}
%\end{equation}

\subsection{Out-Of-Distribution data}

Out-of-distribution (OOD) data refers to the inputs that fall outside the range or characteristics of the data used to train a machine learning model. Such inputs can differ significantly from the training dataset, often causing the model to produce inaccurate or unreliable predictions. Detecting OOD data can be framed as a binary classification task, as discussed by Sun et al. (2022). In this context, a sample is classified as OOD if it lies at least a distance of $\lambda$ away from the in-distribution data, where $\lambda$ represents a predefined threshold for OOD detection.

\subsection{Machine Unlearning in I2I models}

The typical I2I models i.e., AutoEncoders (AEs), Variational AutoEncoders (VAEs), and Diffusion models, employ an encoder-decoder architecture. The encoder part $E_\gamma$ of the model transforms the input image to a latent vector. The decoder $D_\phi$ takes the latent vector as input and decodes it into an output image i.e., for an input image $x$, the output of an I2I model can be written as follows.
\begin{equation}
    \theta_{\gamma, \phi} = D_\phi \circ E_\gamma 
\end{equation}
\text{Therefore, for any input image } $x, \; \theta_{\gamma, \phi} (\tau(x)) = D_\phi (E(\tau(x)))$, where $\tau(x)$ is some operation which results in cropped or masked image $x$; $\circ$ is the composition operator; $\gamma$ and $\phi$ are the trainable model parameters of encoder and decoder respectively.

For a given dataset $\mathcal{D}$, the goal of machine unlearning in image-to-image (I2I) generative models is to effectively remove the influence of a specific subset, known as the forget set $\mathbb{D}_f$, from the model parameters with an unlearning algorithm $A_f$, such that the updated parameters $\gamma, \phi = A_f(\gamma^0, \phi^0)$ no longer retain any information about $\mathbb{D}_f$. Crucially, the model must preserve its performance on the retain set $\mathbb{D}_r$, ($\mathbb{D}_r = \mathcal{D} \setminus \mathbb{D}_f$). In the literature of machine unlearning \cite{li2024machine, feng2024controllable}, the following complementary objectives have been considered.
\begin{enumerate}
    \item On the retain set $\mathbb{D}_r$, the generated images from $\theta_{\gamma, \phi}$ should have the same distribution as in $\theta_{\gamma^0, \phi^0}$ (after unlearning).
    \item On the forget set $\mathbb{D}_f$, the distribution of the generated images from $\theta_{\gamma, \phi}$ should be \textit{as far as possible} from the distribution of images from $\theta_{\gamma^0, \phi^0}$ (after unlearning).
\end{enumerate}

From a probabilistic distribution perspective, the unlearning methodology in \cite{li2024machine} considers the following combined objectives:
\begin{equation}
    \begin{split}
        \argmin_{\gamma, \phi} D \left( P_{\theta_{\gamma_0, \phi_0} (\tau(\mathbb{D}_r))} || P_{\theta_{\gamma, \phi} (\tau(\mathbb{D}_r))} \right ) \text{, and } & \\
        \argmax_{\gamma, \phi} D \left ( P_{\theta_{\gamma_0, \phi_0} (\tau(\mathbb{D}_f))} || P_{\theta_{\gamma, \phi} (\tau(\mathbb{D}_f))} \right )
    \end{split}
\end{equation}

An approach to solve the second objective function is to push the model weights to generate Gaussian noise and, thus to forget the set $\mathbb{D}_f$. That is to solve $\argmin_{\gamma, \phi} D \left ( N(0, \Sigma) || P_{\theta_{\gamma, \phi} (\tau(\mathbb{D}_f))} \right )$. The unlearning methodology in \cite{feng2024controllable} introduces a $\varepsilon$-constrained optimization for this objective, where $\varepsilon$ controls the degree of unlearning. 

\section{Proposed Approach} \label{sec:proposed approach}

\begin{figure}[t]
  \centering
  %\fbox{\rule{0pt}{2in} \rule{0.9\linewidth}{0pt}}
   \includegraphics[width=\linewidth]{fig/I2I_ours_flowchart.png}
   \caption{Overview of our proposed approach. We maximize the loss on the forget samples ($D_f$) to get unlearned model. We get the updated I2I model by further fine-tuning the unlearned model on the retain samples to maintain the performance.}
   \label{fig:our framework}
\end{figure}

In this work, we aim to realistically address the problem of machine unlearning in Image-to-Image (I2I) generative models which reconstruct the image from its partial or noisy counterpart. Recall that the objective of unlearning is to modify the model parameters in such a way that the model's weights, after unlearning specific data, closely match the weights the model would have if it were retrained from scratch without that data. Because of that, we consider that minimizing the distance between gaussian noise and the output of decoder is \textit{not} equivalent to the model which we will get from a retrained model. As an alternative, in our approach we consider that the forget set should be an out-of-distribution (OOD) after unlearning. Mathematically, we define the unlearning objective using the following expression:
\begin{equation} \label{unlearn_obj}
    \arg_{\gamma, \phi} \| \theta_{\gamma_0, \phi_0} (\tau(\mathbb{D}_f)) - \theta_{\gamma, \phi} (\tau(\mathbb{D}_f)) \| \geq \lambda 
\end{equation}
where $\| .\|$ is a distance measure, and $\lambda$ is a threshold which is used to determine out-of-distribution data. In summary, we consider the following combined objective.
\begin{enumerate}
    \item On the retain set $\mathbb{D}_r$, the generated images from $\theta_{\gamma, \phi}$ should have the same distribution as the images generated from $\theta_{\gamma^0, \phi^0}$ after unlearning.
    \item On the forget set $\mathbb{D}_f$, the distribution of the generated images from $\theta_{\gamma, \phi}$ should be atleast some $\lambda$ distance apart from the distribution of images from $\theta_{\gamma^0, \phi^0}$ after unlearning.
\end{enumerate}

Let $\mathcal{L}$ be the loss function for I2I model $f(x, \gamma, \phi): \mathbb{R}^d \rightarrow \mathbb{R}^d$ $\forall x \in $ $\mathcal{D} \; (\mathcal{D} = \mathbb{D}_r \cup \mathbb{D}_f)$, where $\gamma, \phi$ are the model parameters. The typical objective of a machine learning model is to minimize $\mathcal{L}$ i.e., $\argmin_{\gamma, \phi} \mathcal{L}(\gamma, \phi, \mathcal{D})$. The typical update rule for machine learning model can be written as:
\begin{equation}
    \theta_{\gamma^{t+1}, \phi^{t+1}} = \theta_{\gamma^t, \phi^t} - \eta\nabla f(\gamma^t, \phi^t)
\end{equation}
where $\eta$ is the learning rate. Let $\gamma^*, \phi^*$ be the optimal parameters which have perfect generation i.e.,
\begin{equation}
    \theta_{\gamma^*, \phi^*}(\tau(x)) = x
\end{equation}
Under the assumption that SGD converges for the loss function $\mathcal{L}$, we can say that the expected value of $\theta_{\gamma^*, \phi^*} (\tau(x)) - \theta_{\gamma, \phi}(\tau(x))$ decreases over time across training epochs, although it may not be strictly monotonically decreasing in every epoch. 

\subsection{Decoupling via Gradient Ascent}

Gradient ascent is an approach whose objective is to maximize the model loss on a given set. It is a reverse of gradient descent where the model update is written as:
\begin{equation}
    \theta_{\gamma^{t+1}, \phi^{t+1}} = \theta_{\gamma^t, \phi^t} + \eta \nabla f(\gamma^t, \phi^t)
\end{equation}
In our work, we use gradient ascent to forget the influence of $\mathbb{D}_f$ on the model $\theta_{\gamma_0, \phi_0}$. In gradient ascent, we expect that the loss function $\mathcal{L}(\theta_{\gamma, \phi}) $ is an increasing function in $t$ (the loss increases with training) in contrast to the decreasing function in gradient descent. In the next steps, we prove that the output on $\mathbb{D}_f$ with updated model parameters $\theta_{\gamma, \phi}$ after $T$ iterations of gradient ascent is out-of-distribution from the output with $\theta_{\gamma^0, \phi^0}$. We have the following assumption in our work.

%(x, \gamma, \phi)
\textbf{Assumption 1.} $f: \mathbb{R}^d \rightarrow \mathbb{R}^d$ is convex and differentiable. 
\begin{equation} \label{ass1}
    \forall a,b \in \mathbb{R}^d, f(a) \geq f(b) + \langle \nabla f(a), a-b \rangle
\end{equation}

\textbf{Assumption 2.} During all the training epochs, the expected norm of gradient is lower and upper bounded i.e., $g\leq\mathbb{E}\|\nabla f\| \leq G$, where $g, G>0$.

%\textbf{Assumption 3.} Observe that the stochastic gradient is the sum of $S$ independent, uniformly sampled contributions. With central limit theorem, we assume that the gradient noise is Gaussian with covariance $\frac{1}{S}\Sigma(\theta)$.

The convexity assumption is typically invoked in machine unlearning literature to ensure that the model is well-trained and to quantify the influence of data removal on model parameters \cite{guo2019certified}. Although I2I models are usually formulated as non-convex optimization problems, several studies have adopted the convexity assumption to derive theoretical guarantees \cite{sahiner2021hidden, de2022convergence, zhang2024analyzing}. The lower bound in Assumption 2 is needed to make sure the model before unlearning is not at the optimum (i.e., $\theta_{\gamma^0\phi^0} \neq \theta_{\gamma^*\phi^*}$), while the upper bound is considered to prove $(\epsilon, \delta)$-unlearning. We know that for gradient ascent we have,

\begin{equation} \label{GA}
    \theta_{\gamma^{t+1}, \phi^{t+1}} = \theta_{\gamma^t, \phi^t} + \eta \nabla f(\gamma^t, \phi^t)
\end{equation}

Let us consider Assumption 1 for $\gamma^{t+1}, \phi^{t+1}$ and $\gamma^{t}, \phi^{t}$.
\begin{equation}
    \begin{split}
        f(\theta_{\gamma^{t+1}, \phi^{t+1}}) &\geq f(\theta_{\gamma^{t}, \phi^{t}}) \\
        &\;\;\;\; + \langle \nabla f(\theta_{\gamma^{t+1}, \phi^{t+1}}), \theta_{\gamma^{t+1}, \phi^{t+1}} - \theta_{\gamma^{t}, \phi^{t}} \rangle \\
        &\!\!\!\! \geq f(\theta_{\gamma^{t}, \phi^{t}}) + \eta \| \nabla f(\theta_{\gamma^{t+1}, \phi^{t+1}})\| \|\nabla f(\theta_{\gamma^{t}, \phi^{t}}) \| \\
        %&\geq f(\theta_{\gamma^{t-1}, \phi^{t-1}}) + \eta \| \nabla f(\theta_{\gamma^{t}, \phi^{t}})\| \|\nabla f(\theta_{\gamma^{t-1}, \phi^{t-1}}) \| + \eta \| \nabla f(\theta_{\gamma^{t+1}, \phi^{t+1}})\| \|\nabla f(\theta_{\gamma^{t}, \phi^{t}}) \| \\
        &\!\!\!\! \geq f(\theta_{\gamma^{0}, \phi^{0}}) \\
        &\;\;\;\;\; + \eta \sum_{t=0}^{T-1} \|\nabla f(\theta_{\gamma^{t+1}, \phi^{t+1}}) \| \| \nabla f(\theta_{\gamma^{t}, \phi^{t}})\| 
    \end{split}
\end{equation}

Using Assumption 2, we have $\|\nabla f(\theta_{\gamma^{t+1}, \phi^{t+1}}) \|, \| \nabla f(\theta_{\gamma^{t}, \phi^{t}})\| \geq g$. Then we can rewrite Eq. (9) as:
\begin{equation} \label{lower bound}
    \|f(\theta_{\gamma^{T}, \phi^{T}}) - f(\theta_{\gamma^{0}, \phi^{0}}) \| \geq \eta Tg^2 
\end{equation}

Let us consider Assumption 1 again to get an upper bound for $\gamma^{t+1}, \phi^{t+1}$ and $\gamma^{t}, \phi^{t}$ in the opposite order.
\begin{equation}
    \begin{split}
        f(\theta_{\gamma^{t}, \phi^{t}}) &\geq f(\theta_{\gamma^{t+1}, \phi^{t+1}}) \\
        &\;\;\;\; + \langle \nabla f(\theta_{\gamma^{t}, \phi^{t}}), \theta_{\gamma^{t}, \phi^{t}} - \theta_{\gamma^{t+1}, \phi^{t+1}} \rangle \\
        f(\theta_{\gamma^{t+1}, \phi^{t+1}}) &\leq f(\theta_{\gamma^{t}, \phi^{t}}) \\
        &\;\;\;\; + \langle \nabla f(\theta_{\gamma^{t}, \phi^{t}}), \theta_{\gamma^{t+1}, \phi^{t+1}} - \theta_{\gamma^{t}, \phi^{t}} \rangle \\
        %&\leq f(\theta_{\gamma^{t-1}, \phi^{t-1}}) + \langle \nabla f(\theta_{\gamma^{t-1}, \phi^{t-1}}), \theta_{\gamma^{t}, \phi^{t}} - \theta_{\gamma^{t-1}, \phi^{t-1}} \rangle \\ 
        %&+ f(\theta_{\gamma^{t}, \phi^{t}}) + \| \nabla f(\theta_{\gamma^{t}, \phi^{t}}), \theta_{\gamma^{t+1}, \phi^{t+1}} - \theta_{\gamma^{t}, \phi^{t}} \| \\
        &\leq f(\theta_{\gamma^{0}, \phi^{0}}) + \eta \sum_{t=0}^{T} \| \nabla f(\theta_{\gamma^{t}, \phi^{t}}) \|^2
    \end{split}
\end{equation}
Again using Assumption 2, i.e., $\| \nabla f(\theta_{\gamma^{t}, \phi^{t}}) \| \leq G$. Then we get,
\begin{equation} \label{upper bound}
    f(\theta_{\gamma^{T}, \phi^{T}}) \leq  f(\theta_{\gamma^{0}, \phi^{0}}) + \eta (T+1) G^2 
\end{equation}

\mycomment{
\begin{equation} \label{increasing step_GA}
    \| \theta_{\gamma^{t+1}, \phi^{t+1}} - \theta_{\gamma^t, \phi^t} \| ^2 = \eta^2\|\nabla f(\gamma^t, \phi^t)\|^2
\end{equation}
i.e., $\| \theta_{\gamma^{t+1}, \phi^{t+1}} - \theta_{\gamma^0, \phi^0} \| ^2$ is an increasing sequence in $t$. Thus, for $T$ iteration we have,
\begin{equation}
    \| \theta_{\gamma^{T}, \phi^{T}} - \theta_{\gamma^0, \phi^0} \| ^2 = \eta^2 \sum_{t=1}^T \|\nabla f(\gamma^t, \phi^t)\|^2
\end{equation}
Taking expectation both sides and using Assumption 2, we get,
\begin{equation}\label{bounding model_Weights}
    T\eta^2g^2 \leq \mathbb{E}\| \theta_{\gamma^{T}, \phi^{T}} - \theta_{\gamma^0, \phi^0} \| ^2 \leq T \eta^2 G^2
\end{equation}


Considering the left hand side of the inequality and by using the Assumption 1, we get:
\begin{gather*}
    f(\theta_{\gamma^{T}, \phi^{T}}) \geq f(\theta_{\gamma^{0}, \phi^{0}}) + \langle \nabla f(\theta_{\gamma^{0}, \phi^{0}}), \theta_{\gamma^{T}, \phi^{T}} - \theta_{\gamma^{0}, \phi^{0}}\rangle
\end{gather*}
We can rewrite it as:
\begin{gather*}
    f(\theta_{\gamma^{T}, \phi^{T}}) - f(\theta_{\gamma^{0}, \phi^{0}}) \geq \| \nabla f(\theta_{\gamma^{0}, \phi^{0}}) \| \|\theta_{\gamma^{T}, \phi^{T}} - \theta_{\gamma^{0}, \phi^{0}}\|
\end{gather*}
Taking expectation on both sides, and then using Eq. (\ref{bounding model_Weights}) we get:
\begin{equation} \label{Th1}
    \begin{split}
        \mathbb{E}\|f(\theta_{\gamma^{T}, \phi^{T}}) - f(\theta_{\gamma^{0}, \phi^{0}})\| & \geq \mathbb{E}\| \nabla f(\theta_{\gamma^{0}, \phi^{0}}) \| \mathbb{E}\|\theta_{\gamma^{T}, \phi^{T}} - \theta_{\gamma^{0}, \phi^{0}}\| \\
        & \geq T\eta^2g^3
    \end{split}
\end{equation}
On the similar lines, with Assumption 1 we have:
\begin{equation}
    f(\theta_{\gamma^{T}, \phi^{T}}) \leq f(\theta_{\gamma^{0}, \phi^{0}}) + \nabla f(\theta_{\gamma^{T}, \phi^{T}}) \langle \theta_{\gamma^{T}, \phi^{T}} - \theta_{\gamma^{0}, \phi^{0}}\rangle 
\end{equation}
Taking expectation both sides and using right hand side of the inequality from Eq. (\ref{bounding model_Weights}), we get:
\begin{equation} \label{upper bound}
    \begin{split}
        \mathbb{E}\|f(\theta_{\gamma^{T}, \phi^{T}})\| & \leq \mathbb{E} \|f(\theta_{\gamma^{0}, \phi^{0}}) \| + \mathbb{E}\| \nabla f(\theta_{\gamma^{T}, \phi^{T}}) \| \mathbb{E}\|\theta_{\gamma^{T}, \phi^{T}} - \theta_{\gamma^{0}, \phi^{0}}\| \\
        & \leq \mathbb{E} \|f(\theta_{\gamma^{0}, \phi^{0}}) \| + T\eta^2G^3
    \end{split}
\end{equation}
}

Based on this result, we establish the following theorems.

\textbf{Theorem 1.} Under the Assumption 1, 2, the model weights $(\theta_{\gamma, \phi})$ trained with gradient ascent for $T$ iterations are out-of-distribution for the initial trained model $\theta_{\gamma^0, \phi^0}$ on forget set iff
\begin{equation*}
    \lambda \leq \eta Tg^2
\end{equation*}
where $\lambda$ is the predefined out-of-distribution threshold.

\textbf{Theorem 2.} Under Assumption 1, 2, the gradient ascent provides $(\epsilon = 0, \delta = \eta (T+1) G^2)$-unlearning guarantee for model weight $(\theta_{\gamma, \phi})$ after $T$ iterations.

\subsection{Knowledge Retention}

As shown in Fig. \ref{fig:our framework}, the unlearned model is then fine-tuned with retain samples ($\mathbb{D}_r$) with the following objective,
\begin{equation}
    \argmin_{\gamma, \phi} D \left( P_{\theta_{\gamma_0, \phi_0} (\tau(\mathbb{D}_r))} || P_{\theta_{\gamma, \phi} (\tau(\mathbb{D}_r))} \right )
\end{equation}
in order to preserve its performance.

\subsection{Auditing Unlearning} \label{data poisoning attack}

Auditing effective unlearning is crucial to make sure that the model has truly forgotten the forget samples. In this paper, we introduce a novel auditing mechanism based on a data poisoning attack to verify whether unlearning has occurred. Specifically, we fine-tune the model on poisoned versions of the forget samples, embedding a distinct pattern that is recognizable during inference. The aim of this approach is to introduce a recognizable trace into the model's responses that can later serve as an indicator. After the unlearning process, an effective unlearning mechanism should prevent the model from replicating this pattern. If the model, after unlearning, no longer produces outputs associated with the poisoned pattern, it provides strong evidence that the forget samples have been thoroughly erased. This method ensures a more robust validation of the unlearning process by focusing not only on performance metrics but also on detecting residual data influence, thus confirming the removal/unlearning of the impact of the poisoned data.

\begin{table}[h]
    \centering
    \begin{tabular}{ccccc}
      \toprule
      \multirow{2}{*}{Approach} & \multicolumn{2}{c}{FID} & \multicolumn{2}{c}{IS} \\ \cline{2-5}
      & $\mathbb{D}_f\downarrow$ & $\mathbb{D}_r\downarrow$ & $\mathbb{D}_f$ & $\mathbb{D}_r$ \\
      \hline
      GA model & 237.7 & 210.1 & 1.08 & 1.05 \\
      Fine-tuned model & 46.85 & \textbf{3.79} & 1.09 & \textbf{1.13} \\
      SOTA I2I model & 123.6 & 74.2 & 1.09 & 1.12\\
      Merged obj model & 202.1 & 127.7 & 1.08 & 1.08 \\
      Realistic-I2I model & \textbf{16.22} & \textbf{6.10} & 1.10 & \textbf{1.13} \\
      \bottomrule
    \end{tabular}
    \caption{Results of cropping $8 \times 8$ patch at the center of the image where forget samples were poisoned with the '$+$' sign in the CIFAR-10 dataset. $\mathbb{D}_f$ and $\mathbb{D}_r$ account for the forget samples and retain samples respectively. FID scores are compute with respect to retrained model, hence $\downarrow$ is better. Overall, the results highlight that our approach effectively unlearns forget samples and is closer to the retrained model.}
    \label{tab:CIFAR10_res}
\end{table}
\section{Experimental Results}

\subsection{Experimental Setup}

\begin{table*}[h]
    \centering
    \begin{tabular}{ccccccc||cccccc}
      \toprule
      \multirow{3}{*}{Approach} & \multicolumn{6}{c||}{$4\times4$} & \multicolumn{6}{c}{$8\times8$} \\
      & \multicolumn{2}{c}{FID} & \multicolumn{2}{c}{IS} & \multicolumn{2}{c||}{CLIP} & \multicolumn{2}{c}{FID} & \multicolumn{2}{c}{IS} & \multicolumn{2}{c}{CLIP} \\ \cline{2-13}
      & $\mathbb{D}_f\uparrow$ & $\mathbb{D}_r\downarrow$ & $\mathbb{D}_f$ & $\mathbb{D}_r$ & $\mathbb{D}_f$ & $\mathbb{D}_r$ & $\mathbb{D}_f\uparrow$ & $\mathbb{D}_r\downarrow$ & $\mathbb{D}_f$ & $\mathbb{D}_r$ & $\mathbb{D}_f$ & $\mathbb{D}_r$ \\
      \hline
      Max loss & \textbf{56.75} & 9.12 & \textbf{12.07} & 15.06 & 0.80 & \textbf{0.834} & \textbf{109.9} & 16.07 & \textbf{6.33} & 17.03 & 0.64 & 0.735 \\
      Random label & 22.4 & \textbf{8.88} & 13.82 & 14.9 & 0.80 & \textbf{0.834} & 48.84 & \textbf{14.77} & 11.29 & 17.27 & 0.64 & \textbf{0.741} \\
      Random encoder & 23.39 & 9.15 & 13.77 & 15.05 & 0.83 & 0.831 & 25.86 & 15.84 & 16.96 & 17.42 & 0.72 & 0.736 \\
      I2I SOTA & 22.99 & 9.08 & 13.86 & 15.19 & \textbf{0.79} & 0.831 & 53.58 & 15.79 & 12.00 & 17.64 & \textbf{0.61} & 0.736 \\
      Ours& \textbf{24.68} & \textbf{8.93} & \textbf{14.03} & \textbf{15.13} & 0.83 & \textbf{0.834} & 27.43 & \textbf{14.78} & \textbf{18.98} & \textbf{18.77} & 0.731 & \textbf{0.741} \\
      \bottomrule
    \end{tabular}
    \caption{Comparison of various unlearning approaches with different cropped patches ($4\times4 \text{ and } 8\times8$) for VQ-GAN where forget samples were poisoned with the '$+$' sign in the ImageNet-1K dataset. $\mathbb{D}_f$ and $\mathbb{D}_r$ account for the forget samples and retain samples, respectively. FID scores are computed with respect to attack model, hence $\uparrow$ is better for $\mathbb{D}_f$ and $\downarrow$ for $\mathbb{D}_r$. IS score highlight that our approach create good quality images even when the FID distance is significantly far from the attack model. Similarly, we find high CLIP values for our approach indicating that generated image still captures the semantics with an image (not just random noise).}
    %Overall, the results highlight that our approach effectively unlearns forget samples and is closer to the retrained model.
    \label{tab:VQ-GAN results}
\end{table*}

\begin{table*}[h]
    \centering
    \begin{tabular}{ccccccc||cccccc}
      \toprule
      \multirow{3}{*}{Approach} & \multicolumn{6}{c||}{$4\times4$} & \multicolumn{6}{c}{$8\times8$} \\
      & \multicolumn{2}{c}{FID} & \multicolumn{2}{c}{IS} & \multicolumn{2}{c||}{CLIP} & \multicolumn{2}{c}{FID} & \multicolumn{2}{c}{IS} & \multicolumn{2}{c}{CLIP} \\ \cline{2-13}
      & $\mathbb{D}_f\downarrow$ & $\mathbb{D}_r\downarrow$ & $\mathbb{D}_f$ & $\mathbb{D}_r$ & $\mathbb{D}_f$ & $\mathbb{D}_r$ & $\mathbb{D}_f\downarrow$ & $\mathbb{D}_r\downarrow$ & $\mathbb{D}_f$ & $\mathbb{D}_r$ & $\mathbb{D}_f$ & $\mathbb{D}_r$ \\
      \hline
      Max loss & 32.79 & 55.86 & 48.04 & 32.33 & 0.86 & 0.733 & 89.4 & 114.2 & 17.4 & 12.97 & 0.686 & 0.65 \\
      Random label & 19.16 & 19.29 & 56.89 & \textbf{36.22} & 0.92 & 0.867 & 54.60 & 12.55 & 33.24 & 26.47 & 0.759 & 0.87 \\
      Random encoder & 12.95 & 21.25 & 52.79 & 33.47 & 0.93 & 0.85 & 44.32 & 18.77 & 42.01 & \textbf{27.85} & 0.755 & 0.83 \\
      I2I SOTA & 17.16 & \textbf{12.95} & 26.59 & 34.26 & \textbf{0.59} & 0.895 & 101.8 & 13.79 & 9.37 & 21.74 & 0.498 & \textbf{0.88} \\
      Ours & \textbf{9.65} & \textbf{15.14} & \textbf{58.38} & \textbf{35.05} & 0.88 & \textbf{0.904} & \textbf{13.98} & \textbf{13.27} & \textbf{52.6} & 21.02 & \textbf{0.945} & \textbf{0.88} \\
      \bottomrule
    \end{tabular}
    \caption{Comparison of various unlearning approaches for diffusion model with the output of \textit{the original model} for different cropped patches ($4 \times 4 \text{ and } 8\times8$) where forget samples were poisoned with the '$+$' sign in the ImageNet-1K dataset. $\mathbb{D}_f$ and $\mathbb{D}_r$ account for the forget samples and retain samples, respectively. FID scores are computed with respect to original model (to show that our approach mitigate '$+$' sign), hence $\downarrow$ is better for $\mathbb{D}_f$ and $\mathbb{D}_r$. IS score highlight that our approach create good quality images even when the FID distance is significantly far from the attack model. Similarly, we find high CLIP values for our approach indicating that generated image still captures the semantics with an image (not just random noise).}
    %Overall, the results highlight that our approach effectively unlearns forget samples and is closer to the retrained model.
    \label{tab:diff_model results}
\end{table*}

\begin{figure}
  \centering
  \includegraphics[width=\columnwidth]{fig/attack_res_CIFAR_updated.png}
  %\caption{Comparison of the various unlearning approaches along with the retrained model on forget samples poisoned with $'+'$ sign on AutoEncoder. It clearly evident that our method has effectively unlearned the $'+'$ sign and has the output generated closest to the retrained model.}
  \caption{Comparison of various unlearning approaches, along with the retrained model, on forget samples that were poisoned with a '$+$' sign using an AutoEncoder. The results clearly demonstrate that our method effectively unlearns the '$+$' sign, producing outputs that are most similar to those of the retrained model.}
  \label{fig:attack_CIFAR10}
\end{figure}
% First Table

We evaluate our approach using three mainstream I2I architectures: $(i)$ AutoEncoder, $(ii)$ VQ\_VAE \cite{li2023mage}, and $(iii)$ diffusion model \cite{saharia2022palette}. We validate our framework on two widely-used large-scale datasets, namely Places-365 and ImageNet-1K. Additionally, to compare the effectiveness of our framework against a retrained model, we conduct AutoEncoder experiments on the CIFAR-10 dataset.

For the ImageNet-1K dataset, we randomly sampled 100 classes as $\mathbb{D}_f$ and 100 classes as $\mathbb{D}_r$. Similarly, for the Places-365 dataset, we selected 50 classes each for $\mathbb{D}_f$ and $\mathbb{D}_r$. Due to the limited number of classes in the CIFAR-10 dataset, we randomly designated one class as $\mathbb{D}_f$ and the remaining nine classes as $\mathbb{D}_r$.

\begin{figure}[ht]
    \centering
    \begin{subfigure}[b]{0.49\linewidth}
        \centering
        \includegraphics[width=\linewidth]{fig/dm_mask_ratio_diff_tsne_ours.pdf}
        \label{fig:diff_tsne}
    \end{subfigure}
    \hfill
    \begin{subfigure}[b]{0.49\linewidth}
        \centering
        \includegraphics[width=\linewidth]{fig/vqgan_mask_ratio_tsne_I2I_attack1_orig_temp.pdf}
        \label{fig:vqgan_tsne}
    \end{subfigure}
    \vspace{-2em}
    \caption{T-SNE analysis of the generated images with our unlearning framework. After unlearning in both the cases, the generated image from retain samples closely overlaps the ground truth, while the image generated from forget samples diverge from the ground truth.}
    \label{fig:T-SNE analysis}
\end{figure}

%\textbf{Baselines.} Recall from \cref{data poisoning attack}, that in order to audit effectively unlearning, we introduce a data poisoning attack. We have introduced a '$+$' at the center of the forget images to train an attack model. In the main paper, we compare the results of several baselines and benchmark and in the Appendix you find the results for the normal datasets (CIFAR10, Places-365, and ImageNet-1K). We have compared the unlearning results of AutoEncoder, VQ-GAN, and the diffusion model with $(i)$ Max loss baseline, which maximizes the model loss on forget samples \cite{halimi2022federated, warneckemachine}; $(ii)$ Noisy Label, which minimizes training loss with Gaussian noise as ground truth for forget samples \cite{gandikota2023erasing}; $(iii)$ Random Encoder, which minimizes the distance between the output of the encoder on the forget set and Gaussian Noise \cite{tarun2023deep}; and $(iv)$ state of the art I2I unlearning model in \cite{li2024machine}, which minimizes the distance between the Encoder output and gaussian noise while fine-tuning the encoder parameters on retain samples. 

\begin{figure*}
  \centering
    \includegraphics[width = \linewidth]{fig/VQ-GAN_comparison_approaches.png}
    %\caption{Our approach works well on all major I2I architectures, i.e., VQ-GAN, Diffusion model and autoEncoders (see Section 5). This figure also shows comparison with the state of the art (SOTA) I2I unlearning algorithm. For retain samples, our approach generates similar images before and after unlearning, however, SOTA approach struggles in many cases. On forget samples, our approach generates inaccurate/unreliable predictions, matching the expectation of realistic unlearning.}
    \caption{Results of cropping $8 \times 8$ patch at the center of the image on VQ\_GAN models.  The results demonstrate that our model effectively removes the embedded '$+$' pattern, not just replacing it with Gaussian noise. For retain samples, our method shows no signs of the embedded '$+$' sign from the forget samples, in contrast to other baseline and benchmark methods, which often retain subtle remnants of the pattern.}
    \label{fig:VQGAN_comp}
\end{figure*}

\begin{table*}[h]
    \centering
    \begin{tabular}{ccccccc||cccccc}
      \toprule
      \multirow{3}{*}{Approach} & \multicolumn{6}{c||}{VQ-GAN} & \multicolumn{6}{c}{Diffusion model} \\
      & \multicolumn{2}{c}{FID$\downarrow$} & \multicolumn{2}{c}{IS$\uparrow$} & \multicolumn{2}{c||}{CLIP $\uparrow$ } & \multicolumn{2}{c}{FID $\downarrow$ } & \multicolumn{2}{c}{IS $\uparrow$} & \multicolumn{2}{c}{CLIP $\uparrow$} \\ \cline{2-13}
      & $4 \times 4$ & $8 \times 8$ & $4 \times 4$ & $8 \times 8$ & $4 \times 4$ & $8 \times 8$ & $4 \times 4$ & $8 \times 8$ & $4 \times 4$ & $8 \times 8$ & $4 \times 4$ & $8 \times 8$ \\
      \hline
      Max loss & 14.5 & 36.7 & 48.2 & 32.1 & 0.884 & \textbf{0.88} & 15.1 & 49.4 & 38.1 & 29.8 & 0.91 & 0.74 \\
      Random label & 11.7 & 27.9 & 51.2 & 36.7 & 0.880 & 0.87 & 15.15 & 56.5 & 35.1 & 33.23 & 0.92 & 0.82 \\
      Random encoder & 8.1 & 12.6 & 55.4 & 53.8 & 0.885 & 0.87 & 11.1 & 26.17 & 60.8 & 51.62 & 0.91 & 0.81 \\
      I2I SOTA & 11.0 & 26.3 & 51.6 & 38.4 & 0.883 & 0.87 & 33.3 & 79.1 & 57.1 & 28.01 & 0.87 & 0.74 \\
      Ours & \textbf{7.98} & \textbf{12.4} & \textbf{56.2} & \textbf{54.4} & \textbf{0.886} & \textbf{0.88} & \textbf{4.5} & \textbf{12.70} & \textbf{67.1} & \textbf{58.1} & \textbf{0.97} & \textbf{0.89} \\
      \bottomrule
    \end{tabular}
    \caption{Comparison of unlearning approaches on VQ-GAN and Diffusion models for generating unseen data. We evaluate performance on randomly selected 50 classes from the ImageNet-1K dataset for VQ-GAN and the Places365 dataset for the Diffusion model, using different cropped patch sizes ($4 \times 4$ and $8 \times 8$).  FID scores are computed with respect to the original model, where lower values ($\downarrow$) indicate better alignment with the target distribution. Similarly, $\uparrow$ in IS and CLIP score is better to demonstrate the models ability to generate good quality images on unseen data as well.}
    
    %with the output of \textit{the original model}  where forget samples were poisoned with the '$+$' sign in the ImageNet-1K dataset. $\mathbb{D}_f$ and $\mathbb{D}_r$ account for the forget samples and retain samples, respectively. FID scores are computed with respect to original model (to show that our approach mitigate '$+$' sign), hence $\downarrow$ is better for $\mathbb{D}_f$ and $\mathbb{D}_r$. IS score highlight that our approach create good quality images even when the FID distance is significantly far from the attack model. Similarly, we find high CLIP values for our approach indicating that generated image still captures the semantics with an image (not just random noise).}
    %Overall, the results highlight that our approach effectively unlearns forget samples and is closer to the retrained model.
    \label{tab:OOD_model results}
\end{table*}

\textbf{Baselines.} As discussed in Section \ref{data poisoning attack}, we introduce a data poisoning attack as a mechanism to effectively audit unlearning. Specifically, we embed a pattern '$+$' at the center of the forget images to train an attack model. In the main paper, we compare the results of several baselines and benchmark on the poisoned data while in the Appendix you find the results for the non-perturbed datasets (CIFAR10, Places-365, and ImageNet-1K) and image-outpainting. We have compared the unlearning results of AutoEncoder, VQ-GAN, and the diffusion model with $(i)$ Max loss baseline, it maximizes the model loss on forget samples \cite{halimi2022federated, warneckemachine}; $(ii)$ Noisy Label, it minimizes training loss with Gaussian noise as ground truth for forget samples \cite{gandikota2023erasing}; $(iii)$ Random Encoder, it minimizes the distance between the output of the encoder on the forget set and Gaussian Noise \cite{tarun2023deep}; and $(iv)$ state of the art I2I unlearning model \cite{li2024machine} (we call I2I SOTA), it minimizes the distance between the Encoder output and Gaussian noise while fine-tuning the encoder parameters on retain samples. 

\textbf{Evaluation Metrics.} To comprehensively evaluate the effectiveness of our unlearning approach, we utilize three key metrics: the Inception Score (IS) \cite{salimans2016improved}, which assesses the quality and diversity of the generated images by measuring how confidently they can be classified; $(ii)$ Fr\`echet inception distance (FID) \cite{heusel2017gans}, which quantifies the similarity between the distribution of generated images and real ground-truth images; and $(iii)$ CLIP embedding distance \cite{radford2021learning}, measures whether the generated outputs still captures similar semantics.

\subsection{Results and Discussions}

\cref{fig:attack_CIFAR10} and \cref{tab:CIFAR10_res} present the comparison of various unlearning algorithms on CIFAR-10 dataset, where the forget samples are embedded with the '$+$' at the center of the image. As shown in \cref{fig:attack_CIFAR10}, our model, like the retrained model, successfully avoids generating the '$+$' symbol in its outputs, indicating effective unlearning of the forget class. Notably, both our model and the retrained model retain their ability to generalize lines, colors, and patterns from the retain samples, preserving essential generative capabilities. Furthermore, \cref{tab:CIFAR10_res} confirms that our approach produces outputs closely aligned with those of the retrained model. At the same time, it achieves high-quality image generation, as indicated by a higher Inception Score (IS), reflecting crisp and detailed outputs. 
%\cref{tab:CIFAR10_res} also confirms that our approach generates outputs which are similar to the outputs from retrained model. At the same time, generate crisp outputs which is evident from higher IS score.   

In \cref{tab:VQ-GAN results} we extend the comparison of various unlearning algorithms on the attack model with $4 \times 4$, and $8 \times 8$ cropped patches on ImageNet-1K dataset. Our model consistently achieves superior performance on retained samples. For forget samples, it successfully generates outputs that are significantly distinct from those produced by the attack model, while maintaining high image quality (not necessarily accurate or reliable) and capturing similar semantics. \cref{fig:VQGAN_comp} shows some of the results from VQ-GAN model, as shown in the figure our approach preserves the performance on retain set while other approaches have traces of poisoned forget samples. For forget samples, our approach effectively unlearns the embedded '+' sign while introducing inaccurate patterns from the retain samples. 

Furthermore, to assess whether our approach effectively prevents the generation of the embedded '$+$' symbol, we benchmark various baselines and state-of-the-art (SOTA) algorithms on diffusion models using the Places365 dataset. \cref{tab:diff_model results} compares outputs with those of the original model that was not trained on forget samples containing the '$+$' marker. Our model performs well on retained samples in most cases. For forget samples, a low FID score relative to the original model indicates that our approach effectively removes the '$+$' sign, as further evidenced by high IS and CLIP scores. This demonstrates that our model not only maintains high-quality output, but also ensures effective unlearning of sensitive data. 

%Since, the output generated by our model is of high quality. In order to show whether our approach successfully does not generate the embedded '$+$' sign, we compare ours and various baseline and SOTA algorithms for diffusion models on Places365 dataset with the output generated by original model which is not trained on forget samples with '$+$' in \cref{tab:diff_model results}. Here as well, our algorithm has good results for the retain samples in almost all the cases. In case of forget samples, a low FID score against original model would suggest the images generated from our model does not generate '$+$' sign in the image, which is further verified by the high IS score and high CLIP score on forget set.

We compare the performance of all the approaches for unseen data in \cref{tab:OOD_model results}. We randomly sample the next 50 classes from ImageNet-1K and Places365 dataset for VQ-GAN and diffusion model respectively. It is clear from the results that our model has the best generative results on unseen data, particularly with diffusion models. We also perform T-SNE analysis to further validate the effectiveness of our approach, we randomly choose 50 outputs from retain and forget samples. We then compute the CLIP embedding vector for the generated out and the attack model. \cref{fig:T-SNE analysis} shows that the embedding vector from retain samples closely matches the ground truth, while the embedding vector from forget samples diverges. 

In this paper, we systematically investigate the position bias problem in the multi-constraint instruction following. To quantitatively measure the disparity of constraint order, we propose a novel Difficulty Distribution Index (CDDI). Based on the CDDI, we design a probing task. First, we construct a large number of instructions consisting of different constraint orders. Then, we conduct experiments in two distinct scenarios. Extensive results reveal a clear preference of LLMs for ``hard-to-easy'' constraint orders. To further explore this, we conduct an explanation study. We visualize the importance of different constraints located in different positions and demonstrate the strong correlation between the model's attention distribution and its performance.
%\section{Related work}
\label{sec:formatting}

\subsection{Text-to-Video Generation}

T2V generation has made notable progress, evolving from early GAN-based models \cite{saito2017temporal,tulyakov2018mocogan,fu2023tell,li2018video,wu2022nuwa,yu2022generating} to newer transformer \cite{yan2021videogpt,arnab2021vivit,esser2021taming,ramesh2021zero,yu2022scaling} and diffusion models \cite{kirkpatrick2017overcoming,sohl2015deep,song2020denoising,zhang2022gddim}. Early efforts like MoCoGAN~\cite{tulyakov2018mocogan} focused on short video clips but faced issues with stability and coherence. The introduction of transformers improved sequential data handling, enhancing video generation, while diffusion models further improved video quality by progressively denoising the input. 
Despite these advances, T2V models still struggle to reflect human preferences, with the generated videos generally lacking aesthetic quality. Additionally, the scarcity of paired video preference data hinders effective model training and may lead to insufficient flexibility and poor quality in the generated videos.


\subsection{RLHF}

\iffalse
Aligning LLMs \cite{dai1901transformer,radford2019language,zhang2023opt} typically involves two steps: supervised fine-tuning followed by Reinforcement Learning with Human Feedback (RLHF) \cite{gao2023scaling,stiennon2020learning,rafailov2024direct}. Although effective, RLHF is computationally expensive and can lead to issues like reward hacking. Methods like DPO have streamlined alignment by leveraging feedback data directly, improving efficiency.

In contrast, diffusion model alignment is still evolving, focusing mainly on enhancing visual quality through curated datasets. Techniques like DOODL \cite{wallace2023end} and AlignProp \cite{prabhudesai2023aligning} target aesthetic improvements but face challenges with complex tasks such as text-image alignment. Reinforcement learning methods like DPOK \cite{fan2024reinforcement} and DDPO \cite{black2023training} improve reward optimization but struggle with scalability. DPO-SDXL integrates DPO into T2I generation, boosting both alignment and aesthetics.

However, aligning video generation remains a largely unaddressed challenge, especially when dealing with motion consistency and semantic coherence across frames.
\fi

RLHF \cite{gao2023scaling,stiennon2020learning,rafailov2024direct} is a method that utilizes human feedback to guide machine learning models. Early RLHF algorithms, such as DDPG~\cite{lillicrap2015continuous} and PPO~\cite{schulman2017proximal}, typically relied on complex reward models to quantify human feedback. These reward models require a large amount of annotated data and face challenges during tuning. As research has progressed, more efficient preference learning methods have emerged, among which DPO has become a new framework. DPO does not depend on a separate reward model; instead, it obtains human preferences through pairwise comparisons and directly optimizes these preferences. This shift not only simplifies the application of RLHF but also enhances the alignment of models with human values. Furthermore, DPO has been successfully introduced into T2I tasks~\cite{wallace2024diffusion,yang2024using}, providing new insights for generative models in addressing the alignment of human preferences and showcasing DPO's potential in the field of AIGC~\cite{shi2024instantbooth,
qing2024hierarchical,menapace2024snap,koley2024s}. However, there remains a gap in current research regarding the application of DPO strategies to T2V tasks. Effectively integrating DPO into T2V tasks presents a challenging endeavor.


\small
\bibliographystyle{unsrt}
\bibliography{main}

\clearpage
\pagenumbering{gobble}
\maketitlesupplementary

\section{Additional Results on Embodied Tasks}

To evaluate the broader applicability of our EgoAgent's learned representation beyond video-conditioned 3D human motion prediction, we test its ability to improve visual policy learning for embodiments other than the human skeleton.
Following the methodology in~\cite{majumdar2023we}, we conduct experiments on the TriFinger benchmark~\cite{wuthrich2020trifinger}, which involves a three-finger robot performing two tasks: reach cube and move cube. 
We freeze the pretrained representations and use a 3-layer MLP as the policy network, training each task with 100 demonstrations.

\begin{table}[h]
\centering
\caption{Success rate (\%) on the TriFinger benchmark, where each model's pretrained representation is fixed, and additional linear layers are trained as the policy network.}
\label{tab:trifinger}
\resizebox{\linewidth}{!}{%
\begin{tabular}{llcc}
\toprule
Methods       & Training Dataset & Reach Cube & Move Cube \\
\midrule
DINO~\cite{caron2021emerging}         & WT Venice        & 78.03     & 47.42     \\
DoRA~\cite{venkataramanan2023imagenet}          & WT Venice        & 81.62     & 53.76     \\
DoRA~\cite{venkataramanan2023imagenet}          & WT All           & 82.40     & 48.13     \\
\midrule
EgoAgent-300M & WT+Ego-Exo4D      & 82.61    & 54.21      \\
EgoAgent-1B   & WT+Ego-Exo4D      & \textbf{85.72}      & \textbf{57.66}   \\
\bottomrule
\end{tabular}%
}
\end{table}

As shown in Table~\ref{tab:trifinger}, EgoAgent achieves the highest success rates on both tasks, outperforming the best models from DoRA~\cite{venkataramanan2023imagenet} with increases of +3.32\% and +3.9\% respectively.
This result shows that by incorporating human action prediction into the learning process, EgoAgent demonstrates the ability to learn more effective representations that benefit both image classification and embodied manipulation tasks.
This highlights the potential of leveraging human-centric motion data to bridge the gap between visual understanding and actionable policy learning.



\section{Additional Results on Egocentric Future State Prediction}

In this section, we provide additional qualitative results on the egocentric future state prediction task. Additionally, we describe our approach to finetune video diffusion model on the Ego-Exo4D dataset~\cite{grauman2024ego} and generate future video frames conditioned on initial frames as shown in Figure~\ref{fig:opensora_finetune}.

\begin{figure}[b]
    \centering
    \includegraphics[width=\linewidth]{figures/opensora_finetune.pdf}
    \caption{Comparison of OpenSora V1.1 first-frame-conditioned video generation results before and after finetuning on Ego-Exo4D. Fine-tuning enhances temporal consistency, but the predicted pixel-space future states still exhibit errors, such as inaccuracies in the basketball's trajectory.}
    \label{fig:opensora_finetune}
\end{figure}

\subsection{Visualizations and Comparisons}

More visualizations of our method, DoRA, and OpenSora in different scenes (as shown in Figure~\ref{fig:supp pred}). For OpenSora, when predicting the states of $t_k$, we use all the ground truth frames from $t_{0}$ to $t_{k-1}$ as conditions. As OpenSora takes only past observations as input and neglects human motion, it performs well only when the human has relatively small motions (see top cases in Figure~\ref{fig:supp pred}), but can not adjust to large movements of the human body or quick viewpoint changes (see bottom cases in Figure~\ref{fig:supp pred}).

\begin{figure*}
    \centering
    \includegraphics[width=\linewidth]{figures/supp_pred.pdf}
    \caption{Retrieval and generation results for egocentric future state prediction. Correct and wrong retrieval images are marked with green and red boundaries, respectively.}
    \label{fig:supp pred}
\end{figure*}

\begin{figure*}[t]
    \centering
    \includegraphics[width=0.9\linewidth]{figures/motion_prediction.pdf}
    \vspace{-0.5mm}
    \caption{Motion prediction results in scenes with minor changes in observation.}
    \vspace{-1.5mm}
    \label{fig:motion_prediction}
\end{figure*}

\subsection{Finetuning OpenSora on Ego-Exo4D}

OpenSora V1.1~\cite{opensora}, initially trained on internet videos and images, produces severely inconsistent results when directly applied to infer future videos on the Ego-Exo4D dataset, as illustrated in Figure~\ref{fig:opensora_finetune}.
To address the gap between general internet content and egocentric video data, we fine-tune the official checkpoint on the Ego-Exo4D training set for 50 epochs.
OpenSora V1.1 proposed a random mask strategy during training to enable video generation by image and video conditioning. We adopted the default masking rate, which applies: 75\% with no masking, 2.5\% with random masking of 1 frame to 1/4 of the total frames, 2.5\% with masking at either the beginning or the end for 1 frame to 1/4 of the total frames, and 5\% with random masking spanning 1 frame to 1/4 of the total frames at both the beginning and the end.

As shown in Fig.~\ref{fig:opensora_finetune}, despite being trained on a large dataset, OpenSora struggles to generalize to the Ego-Exo4D dataset, producing future video frames with minimal consistency relative to the conditioning frame. While fine-tuning improves temporal consistency, the moving trajectories of objects like the basketball and soccer ball still deviate from realistic physical laws. Compared with our feature space prediction results, this suggests that training world models in a reconstructive latent space is more challenging than training them in a feature space.


\section{Additional Results on 3D Human Motion Prediction}

We present additional qualitative results for the 3D human motion prediction task, highlighting a particularly challenging scenario where egocentric observations exhibit minimal variation. This scenario poses significant difficulties for video-conditioned motion prediction, as the model must effectively capture and interpret subtle changes. As demonstrated in Fig.~\ref{fig:motion_prediction}, EgoAgent successfully generates accurate predictions that closely align with the ground truth motion, showcasing its ability to handle fine-grained temporal dynamics and nuanced contextual cues.

\section{OpenSora for Image Classification}

In this section, we detail the process of extracting features from OpenSora V1.1~\cite{opensora} (without fine-tuning) for an image classification task. Following the approach of~\cite{xiang2023denoising}, we leverage the insight that diffusion models can be interpreted as multi-level denoising autoencoders. These models inherently learn linearly separable representations within their intermediate layers, without relying on auxiliary encoders. The quality of the extracted features depends on both the layer depth and the noise level applied during extraction.


\begin{table}[h]
\centering
\caption{$k$-NN evaluation results of OpenSora V1.1 features from different layer depths and noising scales on ImageNet-100. Top1 and Top5 accuracy (\%) are reported.}
\label{tab:opensora-knn}
\resizebox{0.95\linewidth}{!}{%
\begin{tabular}{lcccccc}
\toprule
\multirow{2}{*}{Timesteps} & \multicolumn{2}{c}{First Layer} & \multicolumn{2}{c}{Middle Layer} & \multicolumn{2}{c}{Last Layer} \\
\cmidrule(r){2-3}   \cmidrule(r){4-5}  \cmidrule(r){6-7}  & Top1           & Top5           & Top1            & Top5           & Top1           & Top5          \\
\midrule
32        &  6.10           & 18.20             & 34.04               & 59.50             & 30.40             & 55.74             \\
64        & 6.12              & 18.48              & 36.04               & 61.84              & 31.80         & 57.06         \\
128       & 5.84             & 18.14             & 38.08               & 64.16              & 33.44       & 58.42 \\
256       & 5.60             & 16.58              & 30.34               & 56.38              &28.14          & 52.32        \\
512       & 3.66              & 11.70            & 6.24              & 17.62              & 7.24              & 19.44  \\ 
\bottomrule
\end{tabular}%
}
\end{table}

As shown in Table~\ref{tab:opensora-knn}, we first evaluate $k$-NN classification performance on the ImageNet-100 dataset using three intermediate layers and five different noise scales. We find that a noise timestep of 128 yields the best results, with the middle and last layers performing significantly better than the first layer.
We then test this optimal configuration on ImageNet-1K and find that the last layer with 128 noising timesteps achieves the best classification accuracy.

\section{Data Preprocess}
For egocentric video sequences, we utilize videos from the Ego-Exo4D~\cite{grauman2024ego} and WT~\cite{venkataramanan2023imagenet} datasets.
The original resolution of Ego-Exo4D videos is 1408×1408, captured at 30 fps. We sample one frame every five frames and use the original resolution to crop local views (224×224) for computing the self-supervised representation loss. For computing the prediction and action loss, the videos are downsampled to 224×224 resolution.
WT primarily consists of 4K videos (3840×2160) recorded at 60 or 30 fps. Similar to Ego-Exo4D, we use the original resolution and downsample the frame rate to 6 fps for representation loss computation.
As Ego-Exo4D employs fisheye cameras, we undistort the images to a pinhole camera model using the official Project Aria Tools to align them with the WT videos.

For motion sequences, the Ego-Exo4D dataset provides synchronized 3D motion annotations and camera extrinsic parameters for various tasks and scenes. While some annotations are manually labeled, others are automatically generated using 3D motion estimation algorithms from multiple exocentric views. To maximize data utility and maintain high-quality annotations, manual labels are prioritized wherever available, and automated annotations are used only when manual labels are absent.
Each pose is converted into the egocentric camera's coordinate system using transformation matrices derived from the camera extrinsics. These matrices also enable the computation of trajectory vectors for each frame in a sequence. Beyond the x, y, z coordinates, a visibility dimension is appended to account for keypoints invisible to all exocentric views. Finally, a sliding window approach segments sequences into fixed-size windows to serve as input for the model. Note that we do not downsample the frame rate of 3D motions.

\section{Training Details}
\subsection{Architecture Configurations}
In Table~\ref{tab:arch}, we provide detailed architecture configurations for EgoAgent following the scaling-up strategy of InternLM~\cite{team2023internlm}. To ensure the generalization, we do not modify the internal modules in InternML, \emph{i.e.}, we adopt the RMSNorm and 1D RoPE. We show that, without specific modules designed for vision tasks, EgoAgent can perform well on vision and action tasks.

\begin{table}[ht]
  \centering
  \caption{Architecture configurations of EgoAgent.}
  \resizebox{0.8\linewidth}{!}{%
    \begin{tabular}{lcc}
    \toprule
          & EgoAgent-300M & EgoAgent-1B \\
          \midrule
    Depth & 22    & 22 \\
    Embedding dim & 1024  & 2048 \\
    Number of heads & 8     & 16 \\
    MLP ratio &    8/3   & 8/3 \\
    $\#$param.  & 284M & 1.13B \\
    \bottomrule
    \end{tabular}%
    }
  \label{tab:arch}%
\end{table}%

Table~\ref{tab:io_structure} presents the detailed configuration of the embedding and prediction modules in EgoAgent, including the image projector ($\text{Proj}_i$), representation head/state prediction head ($\text{MLP}_i$), action projector ($\text{Proj}_a$) and action prediction head ($\text{MLP}_a$).
Note that the representation head and the state prediction head share the same architecture but have distinct weights.

\begin{table}[t]
\centering
\caption{Architecture of the embedding ($\text{Proj}_i$, $\text{Proj}_a$) and prediction ($\text{MLP}_i$, $\text{MLP}_a$) modules in EgoAgent. For details on module connections and functions, please refer to Fig.~2 in the main paper.}
\label{tab:io_structure}
\resizebox{\linewidth}{!}{%
\begin{tabular}{lcl}
\toprule
       & \multicolumn{1}{c}{Norm \& Activation} & \multicolumn{1}{c}{Output Shape}  \\
\midrule
\multicolumn{3}{l}{$\text{Proj}_i$ (\textit{Image projector})} \\
\midrule
Input image  & -          & 3$\times$224$\times$224 \\
Conv 2D (16$\times$16) & -       & Embedding dim$\times$14$\times$14    \\
\midrule
\multicolumn{3}{l}{$\text{MLP}_i$ (\textit{State prediction head} \& \textit{Representation head)}} \\
\midrule
Input embedding  & -          & Embedding dim \\
Linear & GELU       & 2048          \\
Linear & GELU       & 2048          \\
Linear & -          & 256           \\
Linear & -          & 65536     \\
\midrule
\multicolumn{3}{l}{$\text{Proj}_a$ (\textit{Action projector})} \\
\midrule
Input pose sequence  & -          & 4$\times$5$\times$17 \\
Conv 2D (5$\times$17) & LN, GELU   & Embedding dim$\times$1$\times$1    \\
\midrule
\multicolumn{3}{l}{$\text{MLP}_a$ (\textit{Action prediction head})} \\
\midrule
Input embedding  & -          & Embedding dim$\times$1$\times$1 \\
Linear & -          & 4$\times$5$\times$17     \\
\bottomrule
\end{tabular}%
}
\end{table}


\subsection{Training Configurations}
In Table~\ref{tab:training hyper}, we provide the detailed training hyper-parameters for experiments in the main manuscripts.

\begin{table}[ht]
  \centering
  \caption{Hyper-parameters for training EgoAgent.}
  \resizebox{0.86\linewidth}{!}{%
    \begin{tabular}{lc}
    \toprule
    Training Configuration & EgoAgent-300M/1B \\
    \midrule
    Training recipe: &  \\
    optimizer & AdamW~\cite{loshchilov2017decoupled} \\
    optimizer momentum & $\beta_1=0.9, \beta_2=0.999$ \\
    \midrule
    Learning hyper-parameters: &  \\
    base learning rate & 6.0E-04 \\
    learning rate schedule & cosine \\
    base weight decay & 0.04 \\
    end weight decay & 0.4 \\
    batch size & 1920 \\
    training iters & 72,000 \\
    lr warmup iters & 1,800 \\
    warmup schedule & linear \\
    gradient clip & 1.0 \\
    data type & float16 \\
    norm epsilon & 1.0E-06 \\
    \midrule
    EMA hyper-parameters: &  \\
    momentum & 0.996 \\
    \bottomrule
    \end{tabular}%
    }
  \label{tab:training hyper}%
\end{table}%

\clearpage


\end{document}