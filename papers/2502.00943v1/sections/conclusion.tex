\section{Conclusion}
In this paper, we introduced \ours, a zero-shot framework that utilizes \acp{LLM} to automate medical abstraction across multiple attributes from unstructured clinical notes in a fast and effective manner. Through a flexible universal prompt template, \ours achieves universal abstraction in the way that it can easily generalize to all types of attributes (including both simple short-context and complex long-context oncology attributes) and can cope with long input, complex reasoning and involving guidelines. Compared with the conventional approaches that build attribute-specific models, \ours is both more scalable with much lower adaptation costs for onboarding new attributes and achieving better overall performance, providing a promising direction for medical abstraction in the future
with great potential to enhance the efficiency, scalability and usability in the clinical workflow.

\section{Human subjects/IRB, data security, and patient privacy}
This work was performed under an institutional review board (IRB)-approved research protocol (Providence protocol ID 2019000204) and was conducted in compliance with human subjects research and clinical data management procedures—as well as cloud information security policies and controls—administered within Providence Health. All study data were integrated, managed, and analyzed exclusively and solely on Providence-managed cloud infrastructure. All study personnel completed and were credentialed in training modules covering human subjects research, use of clinical data in research, and appropriate use of IT resources and IRB-approved data assets.
