\section{Conclusion}
 
Our findings unveil a novel measure of data diversity through the learned monosemanticity in sparse autoencoders. Based on the activated features in the SAEs, we proposed a new data selection algorithm for instruction tuning corset selection. The models fine-tuned on our selected data consistently outperform other selection methods in instruction-following abilities across different models and datasets. Also, when we change the ratio of selected data, our approach consistently achieves better results. Besides, we can use our data diversity measure to explain why longer instruction-response data usually leads to better model performance. In the future, we hope to extend this pipeline to other directions such as preference data selection, or how to make model safer through data selection.
 
\section*{Impact Statement}
Our research has the potential to impact the fields of finetuning LLMs by introducing a novel measure of data diversity through learned monosemanticity in sparse autoencoders. The proposed data selection algorithm for instruction tuning corset selection can lead to improved model performance, increased efficiency, and enhanced reliability.

\textbf{Improved Model Performance}: Our approach consistently outperforms other selection methods in instruction-following abilities across different models and datasets, resulting in better overall model performance.

\textbf{Increased Efficiency}: By selecting the most relevant data, our method reduces the amount of data required for fine-tuning, leading to faster training times and lower computational costs.

\textbf{Enhanced Reliability}: Our data diversity measure provides insights into why longer instruction-response data typically leads to better model performance, enabling more informed decisions about data selection and model development.

\textbf{Broader Applications}: Our pipeline has the potential to be extended to other areas, such as preference data selection and enhancing model safety through data selection, further expanding its impact on the field.



