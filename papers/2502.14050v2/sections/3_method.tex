\section{Sparse Autoencoder}\label{sec:method}

\textbf{Formulation.}
Let $x^j$ denotes the $d$-dimension residual stream of the $j$th-layer of the transformer layer before the final layer of logits (we discard $j$ hereafter and only use $x$ for simplicity). A typical SAE can be decomposed into two parts: the encoder and decoder, where the encoder is used for representation decomposition and the decoder is only used during training for loss reconstruction and discarded during inference. A ReLU-SAE can be written as 
\begin{equation}
\begin{aligned}
   \textit{Encoder: } z &= \text{ReLU}(W_{\text{encoder}}(x - b_{\text{norm}}) + b_{\text{encoder}}) \\
   \textit{Decoder: }  \hat{x} &= W_{\text{decoder}}z + b_{\text{norm}}
\end{aligned}
\end{equation}
, where $b_{\text{norm}}$ means all inputs are normalized to the unit norm before passing to the autoencoder and computing the reconstruction errors. 
Notice that $W_{\text{encoder}}$ $\in$ $\mathbb{R}^{n \times d}$, $b_{\text{encoder}}$ $\in$ $\mathbb{R}^n$, where $n$ is number of pre-defined latents in SAE.

Similarly, a k-sparse autoencoder \citep{makhzani2013k} regulates the count of active latent units by employing the TopK activation function, which retains only the k largest latent values, setting all others to zero. Thus,
\begin{equation}
\begin{aligned}
   \textit{Encoder: } z &= \text{TopK}(W_{\text{encoder}}(x - b_{\text{norm}}) )
\end{aligned}
\end{equation}
and the decoder is the same. The model is trained by gradient descent through training loss $\mathcal{L} = \|x - \hat{x}\|_2^2$. The TopK-SAE has already been verified by OpenAI \citep{gao2024scaling} for its superior performance in explaining GPT-4, thus we also follow their training tricks, such as forcing dead neuron activation if it has not activated in 10 million tokens (with an auxiliary loss coefficient of $1/32$), normalizing the decoder weights to have unit norm, and setting $W_{\text{decoder}}$ to be the transpose of $W_{\text{encoder}}$ for improving the training process.

\textbf{SAE Training.}
We pick the last residual stream from the final layer of Llama-3.1-8b-instruct \footnote{For better visualization, we summarize all the used open-source links in App. \ref{app:links} hereafter.} for training our TopK-SAE. And we set the number of latents $n$ to $131,072$,  and $K$ to \{16, 32, 64, 128, 256\}. We use Top-$128$-SAE for our main experiments for its moderate size. 
We use the 10B tokens from RedPajama-Data-V2 \citep{together2023redpajama} for SAE training considering its high quality and diverse source. We tried various batch sizes from \{$4,096$, $8,192$, $12,288$\} and found $4,096$ to be the optimal, and more can be found in Appendix \ref{app: sae-bs}. 
For a total batch size of $4,096$, the batch size is 32 per device, and grad\_acc\_steps and micro\_acc\_steps are 4 and 2, respectively.
For all training, we use 4 nodes with 8 Nvidia A100 80G GPUs per node through model parallel. The lr\_warmup\_ratio is $0.5$ and the learning rate is $7e-5$. 
We set epoch to 4 and we do not find additional benefits with more epochs. 
Since the Llama-3 tokenizer contains a $BOS$ token which is useless and even detrimental to SAE training, we discard all $BOS$ tokens for SAE training. We preprocess all the data to the same length through concatenating and chunking passages.

Figure \ref{fig: loss_sae31} shows the training loss over different TopKs under the same configurations. It is generally observed that the training loss almost saturates after $4,000$ steps, and a larger k leads to a better final loss. However, a smaller loss does not necessarily correlate with better-decomposed features since more activated features will be more difficult to interpret. 

\textbf{SAE Inference}
After training the SAEs, inspired by the significant performance boost achieved with JumpReLU-SAE \citep{rajamanoharan2024jumping}, we further use JumpReLU during inference to rectify the activations and only treat the activation value larger than the threshold as true activation. 

\begin{equation}
\text{JumpReLU}(x) =
\begin{cases} 
x, & \text{if } x > \theta, \\
0, & \text{otherwise.}
\end{cases}
\end{equation}
, where $\theta$ is the jump threshold.

\begin{figure}[ht]
\centering
\includegraphics[width=0.95\linewidth]{figures/loss_vs_step.pdf}
\caption{The training loss of TopK-SAE on the layer 31 of Llama-3.1-8b-instruct.}\label{fig: loss_sae31}
\end{figure} 

\begin{table*}[t]
\centering
\small
\begin{tabular}{l|c|ccccc|cc} 
\hline
\multirow{3}{*}{\textbf{Method}} & \multirow{3}{*}{\textbf{Parameters}} & \multicolumn{5}{c|}{\textbf{Question Answering}} & \multicolumn{2}{c}{\textbf{Fact Verification}} \\ \cline{3-9}
 & &  \textbf{TABMWP} & \textbf{WTQ} & \textbf{HiTab} & \textbf{TAT-QA} & \textbf{FeTaQA} & \textbf{TabFact} & \textbf{InfoTabs} \\ 
 & & (Acc.) & (Acc.) & (Acc.) & (Acc.) & (BLEU) & (Acc.) & (Acc.) \\ 
\hline
\multicolumn{9}{l}{{\cellcolor[rgb]{0.957,0.957,0.957}}\textit{LLM (Text)}} \\
Llama2 & 7B & 22.82 & 16.39 & 10.72 & 13.73 & 10.93 & 9.20 & 38.92 \\
TableLlama & 7B & 10.10 & 24.97 & 46.57 & 19.04 & 38.38 & 79.37 & 46.57 \\
Llama3-Instruct & 8B & 42.01 & 21.24 & 6.97 & 13.08 & 12.66 & 73.89 & 54.00 \\
\multicolumn{9}{l}{{\cellcolor[rgb]{0.957,0.957,0.957}}\textit{MLLM (Image)}} \\
MiniGPT-4 & 7B & 0.22 & 0.90 & 0.20 & 0.13 & 0.39 & 0 & 0.10 \\
Qwen-VL & 7B & 3.30 & 0.09 & 0.06 & 0.13 & 0.45 & 1.12 & 0.65 \\
InternLM-XComposer& 7B & 0.06 & 0.05 & 0.12 & 0.26 & 2.62 & 1.19 & 1.11 \\
mPLUG-Owl & 7B & 1.76 & 0.62 & 0.25 & 0.13 & 7.42 & 7.46 & 5.53 \\
mPLUG-Owl2 & 7B & 6.83 & 0.67 & 0.13 & 0.39 & 11.91 & 8.21 & 26.19 \\
LLaVA v1.5 & 7B & 6.05 & 1.24 & 2.03 & 2.97 & 8.24 & 18.9 & 28.31 \\
Vary-toy & 1.8B & 4.42 & 7.96 & 3.42 & 8.81 & 2.44 & 6.33 & 6.98 \\
Monkey & 7B & 13.26 & 19.07 & 6.41 & 12.31 & 3.41 & 22.56 & 22.11 \\
Table-LLaVA  & 7B & 57.78 & 18.43 & 10.09 & 12.82 & 25.60 & 59.85 & 65.26 \\
Table-LLaVA & 13B & 59.77 & 20.41 & 10.85 & 15.67 & 28.03 & 65.00 & 66.91 \\
MiniCPM-V-2.6 & 8B & 83.68  & 47.97 & 56.53 & 51.55 & 32.68 & 78.48 & 73.03 \\

\multicolumn{9}{l}{{\cellcolor[rgb]{0.957,0.957,0.957}}\textit{MLLM (Image \& Text)}} \\
Table-LLaVA & 13B & 84.58 & 39.89 & 46.00 & 29.27 & \textbf{33.50} & 69.93 & 74.88\\
MiniCPM-V-2.6 & 8B & \uline{86.06} & 52.30 & 58.56 & 52.46 & 32.96 & 79.31 & 73.18 \\
w/ Vanilla SFT & 8B & 76.69 & \uline{55.54} & \uline{62.88} & \uline{58.91} & 16.92 & \textbf{82.54} & \textbf{76.22} \\
w/ \method{}  & 8B & \textbf{87.50} & \textbf{55.77} & \textbf{63.00} & \textbf{60.75} & \uline{33.18} & \uline{82.27} & \uline{75.74} \\
\hline
\end{tabular}
\caption{Overall Performance on TQA and TFV Tasks. The \textbf{best} results are marked in bold, while the \uline{second-best} results are underlined. We establish baselines using LLM (Text) and MLLM (Image) by feeding unimodal table representations to language models. Next, we use image-based and text-based table representations as inputs to train various MLLM (Image \& Text) models, demonstrating the effectiveness of our \method{}.} \label{tab:overall}
\end{table*}

\begin{algorithm}[h]
\caption{Two Diversity-driven Data Selection Methods}\label{algorithm:merged}
\begin{algorithmic}[1]
\REQUIRE The Whole Dataset $\mathcal{D}$, Sub-Dataset Size $N$, Sampling Mode (\textit{Greedy} or \textit{Similarity-based}), Threshold $\theta$ (if \textit{Similarity-based})
\ENSURE The Sampled Sub-Dataset $\mathcal{D}_s$
\STATE Initialize Empty $\mathcal{D}_s$
\STATE Sort Queries in $\mathcal{D}$ by instruction length in descending order
\WHILE{$|\mathcal{D}_s| < N$}
    \STATE Set $\mathcal{T}_s^B \leftarrow \emptyset$  \#$\mathcal{T}$ is the set of activated features
    \FOR{each Query $q \in \mathcal{D}$}
        \IF{\textcolor{blue}{$|\mathcal{T}_s^B \cup \mathcal{T}_q| > |\mathcal{T}_s^B|$ \textbf{(if \textit{Greedy: \one})} } \textbf{or} 
        \textcolor{red}{ $|\mathcal{T}_s^B \cap \mathcal{T}_q| / |\mathcal{T}_s^B| < \theta$ \textbf{(if \textit{Similarity-based: \two})}}}
            \STATE $\mathcal{D}_s \leftarrow \mathcal{D}_s \cup \{q\}$
            \STATE $\mathcal{T}_s^B \leftarrow \mathcal{T}_s^B \cup \mathcal{T}_q$
            \STATE $\mathcal{D} \leftarrow \mathcal{D} \setminus \{q\}$
            \IF{$|\mathcal{D}_s| = N$}
                \STATE break
            \ENDIF
        \ENDIF
    \ENDFOR
\ENDWHILE
\STATE \textbf{return} $\mathcal{D}_s$
\end{algorithmic}
\end{algorithm}


 
