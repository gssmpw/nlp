This work addressed the curse of dimensionality in the filtering problem for partially observable \acp{SRN}. Based on Theorem~\ref{th:FMP}, we developed the \ac{FMP} method to reduce the dimensionality (i.e., the number of species in the underlying \ac{SRN}). This approach is a modification of the standard \ac{MP} technique for the filtering problems. The \ac{FMP} is structurally identical to the standard \ac{MP}; the only difference is the additional conditioning on the observed trajectory in the expectation for the projected propensities. 

The proposed approach is to construct an \ac{SRN} with fewer species and solve the filtering problem for this network instead of the original one. This approach significantly reduces the dimensionality of the state space if the \ac{QOI} depends only on a small subset of hidden species. However, some propensities of this projected \ac{SRN} have no analytical expression and require numerical approximations. This work applies \ac{PF} to the original model to estimate the projected propensities. Using standard \ac{MP} propensities estimated with the \ac{MC} methods is also possible but introduces additional errors. For the projected \ac{SRN}, we employed the \ac{FFSP} method to solve the filtering problem, demonstrating that the dimensionality reduction significantly increases its efficiency.

This work showed that applying \ac{FMP} and \ac{MP} significantly reduces the computational complexity of the \ac{FFSP} method by reducing the dimensionality. In addition, the \ac{FMP} method can be considered as a variance reduction for the \ac{PF}.

The theoretical analysis demonstrated the consistency of the \ac{FMP} method. The algorithm converges as $O(M^{-1/2})$, where $M$ is the number of particles to estimate the projected propensity functions. The numerical results confirmed the superiority of \ac{FMP} over the commonly used \ac{PF}. 

A possible direction for future work is applying \ac{FMP} (and standard \ac{MP}) to the \ac{PF}. Similarly to the \ac{FFSP} method, the \ac{PF} suffers from the curse of dimensionality due to weight degeneracy \cite{Snyder2008HighDimPF, Djuric2013HighDimPF}. The \ac{FMP} can significantly increase the efficiency of the \ac{PF}, but also requires an additional step for the propensity estimation, resulting in a two-step algorithm. The first step uses the \ac{PF} for the full model to estimate the projected propensities, and the second step employs \ac{PF} for the projected model to estimate the \ac{QOI}. The particles from the first step can also be applied for a rough estimation of the \ac{QOI}, which can be employed as a control variate. 

Another possibility for future work is to adapt \ac{FMP} for the filtering problem with noisy or discrete-time observations. In this case, the filtering equations have a different form but should also admit a linear equation for the unnormalized conditional \ac{PMF}. This linear equation allows applying the same techniques as that in the proof of Theorem~\ref{th:FMP}. Furthermore, it is also possible to incorporate parameter estimation into the filtering problem by including these parameters in the state vector, further increasing the dimensionality of the state space and making the \ac{FMP} approach even more relevant. 

Finally, one could extend the \ac{FMP} to the filtering problem for It\^o processes by deriving an equation for the marginalized conditional density by integrating both sides of the Zakai equation \cite{Zakai1969Optimal}. The idea is similar to the proof of Theorem~\ref{th:FMP} but may cause difficulties related to the continuity of the state space.
