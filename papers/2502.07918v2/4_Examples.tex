This section presents two numerical examples of solving the marginal filtering problem \eqref{eq:filtering_problem_margin} for biochemical systems. The source code is available at \href{https://github.com/maksimchup/Markovian-Projection-in-filtering-for-SRNs}{github.com/maksimchup/Markovian-Projection-in-filtering-for-SRNs}.

\subsection{Bistable Gene Expression Network}
\label{subsec:bistable_gene}
Consider an intracellular system with two genes \cite{Duso2018SelectedNodeSSA} sketched in Figure~\ref{fig:bistable_gene_diagram}. In an activated state, each gene can produce mRNA, which produces protein molecules. The amount of protein of each type affects the deactivation of the opposite gene. The model reactions are written as follows:
\begin{align*}
    \text{for } &i = 1,2 \quad i \neq j: \\[1.5pt]
    \text{mRNA}_i &\longleftrightarrow \emptyset \\
    \text{mRNA}_i &\longrightarrow \text{mRNA}_i + \text{protein}_i \\
    \text{protein}_i &\longrightarrow \emptyset \\
    \text{G}^\ast_j + \text{protein}_i &\longrightarrow \text{G}_j + \text{protein}_i  \\
    \text{G}^\ast_i &\longrightarrow \text{G}^\ast_i + \text{mRNA}_i \\
    \text{G}^\ast_i &\longleftrightarrow \text{G}_i,
\end{align*}
where $\text{G}^{\ast}$ and $\text{G}$ denote the activated and deactivated gene states.
For further numerical simulations, we use propensities according to the mass action kinetics \eqref{eq:mass_action} with the following reaction rates \cite{Duso2018SelectedNodeSSA}: $\theta_1 = \theta_2 = 0.1$, $\theta_3 = \theta_4 = 0.05$, $\theta_5 = \theta_6 = 5$, $\theta_7 = \theta_8 = 0.2$, $\theta_9 = \theta_{10} = 0.1$, $\theta_{11} = \theta_{12} = 1$, $\theta_{13} = \theta_{14} = 0.03$ and $\theta_{15} = \theta_{16} = 10^{-6}$.

\begin{figure}
    \centering
    \includegraphics[width=0.6\linewidth]{pics/bistable_network_diagram.png}
    \caption{Reaction diagram of the bistable gene expression network (Section~\ref{subsec:bistable_gene}).}
    \label{fig:bistable_gene_diagram}
\end{figure}

Assume that the copy number of each protein is observed (i.e., the observed process $\boldsymbol{Y}$ is two-dimensional), and the goal is to estimate the conditional distribution of the amount of $\text{mRNA}_2$. We generate the observed part of the process using the \ac{SSA} (Figure~\ref{fig:bistable_gene_sol}(a)). Because we use synthetic data, the true trajectory of the hidden part is available for comparison with the corresponding conditional expectation from the solution of the filtering problem. However, the discrepancy in this case is explained not only by numerical error, but also by the stochastic nature of the problem itself. Determining the exact trajectory of the hidden part based on the information on the observed part is impossible.

\begin{figure}
    \centering
    \includegraphics[width=1\linewidth]{pics/bistable_network_sol.png}
    \caption{
    Numerical results for the bistable gene expression network (Section~\ref{subsec:bistable_gene}). The projection methods reduce the dimensionality of the hidden process from $6$ to $1$.
     (a): Observed trajectory of $\text{protein}_1$ and $\text{protein}_2$. 
     (b): Hidden trajectory of $\text{mRNA}_2$ and the corresponding estimates of its conditional expectation obtain with standard \ac{MP} and \ac{FMP} methods and the \ac{FFSP} method for reference. 
     (c)--(d): Conditional distribution of $\text{mRNA}_2$ at time $t = 2, 4.5$ obtained with the \ac{MP} and \ac{FMP} methods and the \ac{FFSP} method for reference. 
     The standard \ac{MP} has a larger error compared to the \ac{FMP}, which agrees with our theoretical results. 
     }
    \label{fig:bistable_gene_sol}
\end{figure}

The \ac{FFSP} solution for the full-dimensional system is utilized as a reference solution. In this model, gene states can only be 0 or 1, and the amounts of mRNA molecules are not bounded, so only mRNA$_1$ and mRNA$_2$ must be truncated. The upper bounds are set as $\text{mRNA}_1^{\text{max}} = \text{mRNA}_2^{\text{max}} = 30$, yielding $N = 2^4 \cdot 31^2 = 15 \, 376$ possible states for the full \ac{SRN}. 

For the proposed \ac{MP} and \ac{FMP} methods, the projected process is three-dimensional (two observed  and one hidden species). The hidden space is still one-dimensional; thus, a significant efficiency improvement is expected compared to the full-dimensional process. With the same upper bound for mRNA, there are only $N' = 31$ hidden states. Thus, instead of $15 \, 376$ equations for the original \ac{SRN}, only $31$ should be solved for the projected \ac{SRN} (in both proposed projection approaches).
 
 For the \ac{MP} and \ac{FMP} algorithms, we use sample size $M=10^3$ to estimate the projected propensities.

The computational complexity of solving the filtering problem using the \ac{FMP} and \ac{MP} methods compared to \ac{FFSP} is summarized in Table~\ref{tab:bistable_network_results}. The projection methods reduce the dimensionality of the hidden state space from $\dim{\boldsymbol{X}} = 6$ to $\dim{\boldsymbol{X}'} = 1$, resulting in a reduction in computational time from 841 seconds (s) for the full-dimensional \ac{SRN} to an average of 5 s for the \ac{MP} and \ac{FMP} algorithms (i.e., an acceleration of about 160 times). Figure~\ref{fig:bistable_gene_sol} reveals that both \ac{MP} and \ac{FMP} surrogates provide a reasonable estimate for the conditional expectation. Despite the inconsistency of the standard \ac{MP} algorithm, we obtain a result close to the reference solution. The accuracy improves when using \ac{FMP} instead of \ac{MP}, which verifies the inconsistency of the \ac{MP} approach.

\begin{table}%[ht]
    \centering 
    \def\arraystretch{1.25}
    \begin{tabular}{|l|c|c|c|} \cline{2-4}
        \multicolumn{1}{c|}{} & \textbf{Full model} & \textbf{\ac{MP} surrogate} & \textbf{\ac{FMP} surrogate} \\ \hline
        {Dimensionality of hidden space} & 6 & 1 & 1 \\ \hline
        {Number of hidden states} & $15 \, 376$ & 31 & 31 \\ \hline
        {CPU time in seconds} & 841 & 5 & 5 \\ \hline
    \end{tabular}
    \caption{The computational complexity of solving the filtering equation for the bistable gene network (Section~\ref{subsec:bistable_gene}) via \ac{FFSP} method for the full model and projection methods (\ac{MP} and \ac{FMP}).}
    \label{tab:bistable_network_results}
\end{table}














\subsection{Linear Cascade}
\label{subsec:linear_cascade}
Consider a linear cascade model \cite{Gupta2021DeepCME} consisting of $d$ species $S_1, \dots, S_d$. The reactions are given by
\begin{align*}
    \text{for } &i = 1,\dots, d: \\
    S_{i-1} &\longrightarrow S_{i}, \\
    S_i &\longrightarrow \emptyset,
\end{align*}
where $S_0 = \emptyset$. A sketch of this model is presented in Figure~\ref{fig:linear_cascade_diagram}. Further numerical simulations employ propensities according to the mass action kinetics \eqref{eq:mass_action} with the following reaction rates: $\theta_1 = 10$, $\theta_i = 5$ for $i=2,\dots, d$ and $\theta_{i}=1$ for $i=d,\dots, 2d$.

\begin{figure}
    \centering
    \includegraphics[width=0.6\linewidth]{pics/linear_cascade_diagram.png}
    \caption{Reaction diagram of the bistable gene expression network (Section~\ref{subsec:linear_cascade}).}
    \label{fig:linear_cascade_diagram}
\end{figure}


Let us denote the copy number of $S_i$ at time $t$ by $Z_i(t)$ for $i = 1,\dots, d$ and consider an \ac{SRN} $\boldsymbol{Z}(t) = \left(Z_1(t), \dots, Z_d(t) \right)$. Assume that the $Z_d$ (copy number of $S_d$) is observed, and the goal is to estimate the conditional distribution of $Z_1$. As prior, the observed trajectory was simulated using the \ac{SSA}.

To obtain a reference solution, we used The \ac{FFSP} method with $(d-1)$-dimensional truncated state space $\mathsf{X}_N = \{0, \dots, 10 \}^{(d-1)}$, resulting in a system of $N = 11^{(d-1)}$ equations. For the \ac{MP} and \ac{FMP} methods, the projected hidden space is one-dimensional: $\mathsf{X}'_N = \{0, \dots, 10 \}$, which yields only  $N' = 11$ equations. For the \ac{MP} and \ac{FMP} algorithms, we use sample size $M=500$ to estimate the projected propensities. The computational complexity of solving the filtering problem for $d=8$ using the \ac{FMP} and \ac{MP} methods compared to \ac{FFSP} is summarized in Table~\ref{tab:linear_cascade_results}. 

\begin{figure}
    \centering
    \includegraphics[width=\linewidth]{pics/linear_cascade_sol.png}
    \caption{
    Numerical results for the linear cascade model (Section~\ref{subsec:linear_cascade}) for $d=5$. 
     (a): Observed trajectory of $Z_d$. 
     (b): Hidden trajectory of $Z_1$ and the corresponding estimates of its conditional expectation obtained with the standard \ac{MP} and \ac{FMP} methods and the \ac{FFSP} method for reference.
    }
    \label{fig:linear_cascade_means}
\end{figure}

The simulation results for $d=5$ are presented in Figure~\ref{fig:linear_cascade_means}. The reference solution shows that the estimated expectation is almost independent of the observations and rapidly reaches a nearly stationary state. It can be explained as follows: $S_1$ and $S_5$ are linked by reactions through three other species and therefore are almost independent. Due to the same reason, the additional conditioning on the observed trajectory $\{ \boldsymbol{Y}(s) = \boldsymbol{y}(s), s \leq t \} = \{ Z_d (s) = y(s), s \leq t \}$ should not significantly change the \ac{FMP} propensities \eqref{eq:FMP_a_tilde_def} from the \ac{MP} propensities \eqref{eq:MP_a_bar_def}. At the beginning, \ac{MP} even outperforms \ac{FMP}, but then it deviates more from the reference solution. Because there is no condition on $Z_d$ in the \ac{MP} propensities, we have to extrapolate them in the two-dimensional state space $(Z_1, Z_d)$, which can introduce larger errors compared to \ac{FMP} propensities for which $Z_d(s)$ can only be in state $y(s)$.

\begin{figure}
    \centering
    \includegraphics[width=0.7\linewidth]{pics/linear_cascade_log_distr.png}
    \caption{
    Numerical results for the $d=5$ dimensional linear cascade model (Section~\ref{subsec:linear_cascade}). Conditional \ac{PMF} of $Z_1(T)$ (in log scale), estimated with the \ac{PF}, \ac{MP}, and \ac{FMP} with sample size $M = 500$. The reference solution is obtained with the \ac{FFSP} method for the full model. For the \ac{PF}, no particles hit the region $\{ Z_1 \geq 9 \}$, however, the \ac{FMP} method based on the same set of particles led to a reasonable estimate of the tail probability.
    }
    \label{fig:linear_cascade_log_distr}
\end{figure}

Figure~\ref{fig:linear_cascade_log_distr} shows the difference between \ac{MP} and \ac{FMP} methods in estimating the tails of the conditional distribution at the final time $T=5$. We also provide a \ac{PF} estimate, based on the same particles used to estimate the projected propensities in \ac{FMP}. Clearly, the sample size $M=500$ is insufficient for the \ac{PF} to accurately estimate the probabilities in the tail, but applying \ac{FMP} significantly improves the estimate.

\begin{figure}
    \centering
    \includegraphics[width=\linewidth]{pics/linear_cascade_errors.png}
    \caption{
    Numerical results for the linear cascade model (Section~\ref{subsec:linear_cascade}). 
    Relative errors in estimating $\Probcondmu{Z_1(T) \geq 8}{Z_d(s) = y(s), s \leq T}$ for $d=3$ (a) and $d=5$ (b) with the \ac{PF}, \ac{MP}, and \ac{FMP} methods, depending on the sample size (log-log scale). 
    Simulations were performed for fixed observed trajectories $Z_d$ for each plot, and errors were averaged over 30 runs. The vertical bars show $95 \%$ confidence intervals (for the MP with $M < 10^4$, confidence intervals are not shown because they are larger than the estimate itself).
    The results verify our convergence estimate for \ac{FMP} (Corollary~\ref{corollary:FMP_error_PF}) and show the advantage of the \ac{FMP} over the standard \ac{MP} approach.
    }
    \label{fig:linear_cascade_convergence}
\end{figure}

For further comparison with the \ac{PF}, we consider three- and five-dimensional systems and estimate the following \ac{QOI}: 
$$ 
    Q_d := \Probcondmu{Z_1(T) \geq 8}{Z_d(s) = y(s), s \leq T}.
$$ 
The reference solutions obtained with \ac{FFSP} are $Q^{\text{ref}}_{3} =  3.35 \times 10^{-4} $ for $d=3$ and $Q^{\text{ref}}_{5} =  3.34 \times 10^{-4}$ for $d=5$. Figure~\ref{fig:linear_cascade_convergence} presents the relative error of the \ac{PF}, \ac{MP}, and \ac{FMP} methods depending on the sample size $M$. The error of the \ac{FMP} is smaller than the error of \ac{PF}, confirming that the \ac{FMP} can be employed as an additional refining step for the \ac{PF} (see Remark~\ref{remark:FMP_PF_refining}). Moreover, the convergence rate of the \ac{FMP} estimate is $O(M^{-1/2})$, as derived in Section~\ref{subsec:FMP_error_analysis}. For the three-dimensional model, the \ac{MP} error decreases slower for large $M$, which confirms the inconsistency of this method. However, this behavior is not seen for the five-dimensional model, since the condition on the observed trajectory has less effect on the propensities in this model.

Figure~\ref{fig:linear_cascade_cpu_time} illustrates that the execution time of the \ac{FFSP} algorithm for the full-dimensional system increases exponentially as the dimensionality increases. In contrast, the time of the projection-based method does not change as the dimensionality increases. The execution time for the \ac{FMP} algorithm is less than that for the \ac{MP} because the \ac{PF} sampling involves only the hidden reactions from $\mathcal{U}$, whereas the \ac{MC} sampling involves all reactions.


\begin{table}%[ht]
    \centering 
    \def\arraystretch{1.25}
    \begin{tabular}{|l|c|c|c|} \cline{2-4}
        \multicolumn{1}{c|}{} & \textbf{Full model} & \textbf{\ac{MP} surrogate} & \textbf{\ac{FMP} surrogate} \\ \hline
        {Dimensionality of hidden space} & 7 & 1 & 1 \\ \hline
        {Number of hidden states} & $> 1.9\times 10^7$ & 11 & 11 \\ \hline
        {CPU time in seconds} & $38 \, 166$ & 4.0 & 1.9 \\ \hline
    \end{tabular}
    \caption{The computational complexity of solving the filtering problem for the linear cascade model (Section~\ref{subsec:linear_cascade}) with $d=8$ species via \ac{FFSP} method for the full model and projection methods (\ac{MP} and \ac{FMP}).}
    \label{tab:linear_cascade_results}
\end{table}



\begin{figure}
    \centering
    \includegraphics[width=0.7\linewidth]{pics/linear_cascade_cpu_times.png}
    \caption{
     CPU times (log scale) of the standard \ac{MP} and \ac{FMP} methods and the \ac{FFSP} for reference, depending on the number of species. 
     }
    \label{fig:linear_cascade_cpu_time}
\end{figure}