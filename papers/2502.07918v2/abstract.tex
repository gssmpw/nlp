Stochastic reaction networks (SRNs) model stochastic effects for various applications, including intracellular chemical or biological processes and epidemiology. A typical challenge in practical problems modeled by SRNs is that only a few state variables can be dynamically observed. Given the measurement trajectories, one can estimate the conditional probability distribution of unobserved (hidden) state variables by solving a stochastic filtering problem. In this setting, the conditional distribution evolves over time according to an extensive or potentially infinite-dimensional system of coupled ordinary differential equations with jumps, known as the filtering equation. The current numerical filtering techniques, such as the Filtered Finite State Projection \cite{DAmbrosio2022FFSP}, are hindered by the curse of dimensionality, significantly affecting their computational performance. To address these limitations, we propose to use a dimensionality reduction technique based on the Markovian projection (MP), initially introduced for forward problems \cite{Hammouda2023MP}. In this work, we explore how to adapt the existing MP approach to the filtering problem and introduce a novel version of the MP, the Filtered MP, that guarantees the consistency of the resulting estimator. The novel method employs a reduced-variance particle filter for estimating the jump intensities of the projected model and solves the filtering equations in a low-dimensional space. The analysis and empirical results highlight the superior computational efficiency of projection methods compared to the existing filtered finite state projection in the large dimensional setting.

\sloppy
\textbf{Keywords:} Stochastic reaction network, stochastic filtering, dimensionality reduction, Markovian projection, marginalized filter

\textbf{Mathematics Subject Classification (2020):} 60J22, 60J74, 60J27, 60G35, 92C40
\fussy