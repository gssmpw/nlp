\section{Allophony of S3M features}\label{sec:motiv}
\Cref{sec:exp} empirically demonstrates that leveraging S3M features with MixGoP helps enhance downstream performance compared to other features, such as MFCCs and Mel spectrograms.
This section aims to further verify the suitability of S3Ms for representing individual phonemes with allophonic variations.
First, we examine S3M features at the phoneme-level, using the dimensionality reduction technique in \Cref{subsec:motiv_qual}.
Next, we design a metric to quantify the ability of capturing allophonic variations, comparing S3M features to MFCCs and Mel spectrogram in \Cref{subsec:motiv_quant}.
For our analyses, we used the healthy speech recordings from the TORGO dataset \citep{rudzicz2012torgo}, which includes gold-standard phonemic transcriptions and alignments.


\subsection{Motivating Observation}\label{subsec:motiv_qual}
S3Ms are trained to reconstruct masked signals using surrounding information.
Hence, we hypothesize that this will allow the S3Ms to capture local acoustic characteristics, including allophones from various phonetic environments.
To verify such phenomena, we observed the final layer features of WavLM for each phoneme, which have generally shown the best performance across datasets (See \Cref{fig:layerwise}).
Specifically, we use UMAP dimensionality reduction \citep{mcinnes2018umap} with the cosine distance metric to visualize the features, similar to \citet{choi2022opening}.
We also extract the four utterances closest to each of the ten centroids to observe the phonetic environments of each cluster. 


\Cref{fig:motiv} demonstrates one example, with the distribution of \textipa{/2/} (\texttt{/AH/} in ARPABET) and its environments of the healthy subset of TORGO.
We observed multiple clusters for each phoneme, each with phonetically similar environments, which motivates the metric for quantifying allophony ability.


\subsection{Quantifying Allophony}\label{subsec:motiv_quant}
Previous studies have found that S3M features model the phoneme distributions with multiple clusters \citep{wells22_interspeech,martin23_interspeech}.
However, there has been limited analysis on directly quantifying the relationship between the S3M feature subclusters and allophony.
To this end, we design a setting that measures the mutual information between the S3M feature subcluster indices and the surrounding phonetic environment of each phoneme, which is an indicator of allophones.

\begin{figure}[t!]
    \centering
    \includegraphics[width=0.44\textwidth]{figures/AH_phone.pdf}
    \caption{
        Visualization of WavLM-Large features of the \texttt{/AH/} phoneme in the TORGO healthy subset.
        Phonemes are indicated using ARPABET.
        We observe that \texttt{/AH/} consists of subclusters, each reflecting allophones from different surrounding phonetic environments.
    }
    \label{fig:motiv}
\end{figure}


        % We observe that single phoneme is not represented as a single cluster.


First, to obtain the subclusters within MFCCs, Mel spectrograms, and S3M features, we apply the k-means algorithm with $k=32$ clusters to the features of each phoneme $V \in \mathcal{V}$.
Then, each utterance has the designated k-means cluster index $I$.
For a dataset with a total of $|\mathcal{V}|$ phonemes, we train $|\mathcal{V}|$ different k-means models.


Note that our MixGoP uses k-means clusters as the initializer.
Also, we observed few to no EM optimization steps due to high dimensionality \citep{wang2015high}.
As a result, the initialized cluster centroids will likely be similar to the final centroids $\pmb{\mu}_p^c$ in \cref{eq:GM} for calculating phoneme likelihood $P_\theta(\mathbf{s}|p)$.


We then compare the utterance-wise cluster indices with their allophony.
Since the TORGO dataset does not provide phonetic transcriptions, we utilize the surrounding phonetic environment, which is closely linked to allophonic variation.
For simplicity, we define the environment $E$ as the natural class of the preceding and following phonemes, similar to phoneme environment clustering \citep{sagayama1989phoneme}.
We use the height, backness, and roundness for the vowels and the place and manner of the consonants for the natural class.
For example, each \textipa{/i/} and \textipa{/k/} is represented as \texttt{close-front-unrounded} and \texttt{velar-plosive}, respectively.
If the phoneme is word-initial or word-final, we also include them as the environment information.
Therefore, with $|\mathcal{C}|$ number of natural classes, the number of all the possible environments is $(|\mathcal{C}| + 1)^2$.

\begin{figure}[t!]
    \centering
    \includegraphics[width=0.3\textwidth]{figures/layerwise_pnmi.pdf}
    \caption{
        Normalized Mutual Information $\text{MI}(I; E)/H(E)$ between the k-means cluster indices $I$ and the phonetic environment $E$ on the TORGO healthy subset.
        We show the layerwise NMI for S3Ms and absolute value for MFCC and Mel spectrogram.
        % We observe that S3Ms contain more information than traditional features, albeit having layerwise differences.
    }
    \label{fig:pnmi}
\end{figure}


To quantify allophonic information within each subclusters, we measure the mutual information $\text{MI}(I; E)$ between the cluster indices $I$ and the environment $E$.
We normalize the value by the environment entropy $H(E)$ so that the resulting value is between 0 and 1.
We call the metric \textit{Allophone environment-Normalized Mutual Information} (ANMI) $\text{MI}(I; E) / H(E)$.
% After obtaining the ANMI for each phoneme, we average the ANMI values to obtain the final value in \Cref{fig:pnmi}.
The actual calculation is nearly identical to Phoneme Normalized Mutual Information (PNMI) \citep{hsu2021hubert}, except we replaced phoneme $V$ to environment $E$.


\paragraph{Results.} 
\Cref{fig:pnmi} shows the phonetic environment information inside cluster indices for MFCCs, Mel spectrograms, XLS-R, and WavLM.
For S3M models, we plot across different layers.
We can observe that S3Ms contain more information on the phonetic environment compared to traditional features, implying that S3Ms successfully capture allophony.
This finding introduces an interesting implication regarding the effect of varying cluster sizes when applying k-means to S3M features for discrete units \citep{chang2024exploring}. 
It is known that different cluster sizes lead to varying levels of granularity, with smaller cluster size capturing phoneme information while larger cluster size capturing speaker information \citep{sicherman2023analysing}.
Our results suggest that cluster size in between may capture allophonic variations.



\section{Analysis}\label{sec:discuss}
\subsection{Does capturing phonetic environment leads to better downstream performance?} \label{subsec:pnmi_vs_task}
\begin{figure}[t!]
    \centering
    \includegraphics[width=0.3\textwidth]{figures/nmi_vs_downstream.pdf}
    \caption{
        Comparing phonetic environment information with the downstream task performance.
        % It correlates until around NMI $<0.72$.
    }
    \label{fig:nmi_vs_downstream}
\end{figure}

It is crucial to examine whether capturing phonetic environments (or allophones) actually improves downstream performance. 
To assess this, we compare the amount of phonetic environment information inside S3Ms and the actual downstream performance on the pronunciation assessment.
In \Cref{fig:nmi_vs_downstream}, we observe that the downstream performance positively correlates until around NMI $< 0.72$, where the downstream performance saturates even if the amount of phonetic environment information increases.
We suspect this behavior is due to S3Ms capturing more surrounding information, which may not be useful for the pronunciation assessment task.
Our hypothesis aligns with the previous empirical observation of \citet{pasad2023comparative} and \citet{choi2024self} that S3Ms have non-negligible word-level modeling abilities, which requires a larger temporal receptive field.
Moreover, the layerwise trends of \Cref{fig:pnmi}, \textit{i.e.}, WavLM persistently increasing and XLS-R peaking in the middle, are also similar to previous empirical observations on word-level layerwise information \citep{pasad2023comparative}.


\subsection{Sample efficiency of MixGoP}\label{subsec:ablation}
We randomly subsampled training set samples to check the influence of training data size.
To train the GMM for each phoneme, we can either use all the occurrences in the dataset or limit the maximum number of samples.
For optimal performance, we searched for the maximum number of samples between 64, 128, 256, 512, or using the full dataset.
For example, if we set the maximum as 64, and \textipa{/a/} and \textipa{/i/} each have a total of 100 and 50 samples, to train the GMM of \textipa{/a/}, we randomly subsample 64 samples.
On the other hand, since there are only 50 samples for \textipa{/i/}, all 50 samples are used.

\begin{table}[]
\caption{Ablation on random subsampling.}
\label{tab:numberGM}

\vspace{0.5em}
\resizebox{1.0\linewidth}{!}{%
\begin{tabular}{l|ccccc}
\toprule
Dataset & 64 & 128 & 256 & 512 & Full \\
\midrule
UASpeech & 0.615 & 0.620 & \textbf{ 0.624 } & 0.623 & 0.620 \\
TORGO & 0.704 & 0.709 & 0.712 & \textbf{ 0.713 } & \textbf{ 0.713 } \\
SSNCE & 0.539 & 0.545 & 0.549 & \textbf{ 0.553 } & 0.548 \\
speechocean762 & 0.502 & 0.516 & 0.528 & \textbf{ 0.539 } & 0.536 \\
L2-ARCTIC & 0.182 & 0.178 & 0.181 & 0.182 & \textbf{ 0.197 } \\
\bottomrule
\end{tabular}
}
\end{table}



% \begin{table}[]
% \caption{Ablation on random subsampling.}
% \label{tab:numberGM}

% \vspace{0.5em}
% \resizebox{1.0\linewidth}{!}{%
% \begin{tabular}{l|ccccc}
% \toprule
% Dataset        & 64    & 128   & 256   & 512   & Full  \\
% \midrule
% UASpeech       & 0.6145 & 0.6202 & \textbf{ 0.6235 } & 0.6233 & 0.6200 \\
% TORGO          & 0.7039 & 0.7093 & 0.7124 & \textbf{ 0.7125 } & \textbf{ 0.7125 } \\
% SSNCE          & 0.5392 & 0.5448 & 0.5487 & \textbf{ 0.5533 } & 0.5475 \\
% speechocean762 & 0.5021 & 0.5157 & 0.5283 & \textbf{ 0.5387 } & 0.5364 \\
% L2-ARCTIC      & 0.1816 & 0.1775 & 0.1809 & 0.1821 & \textbf{ 0.1965 } \\
% \bottomrule
% \end{tabular}
% }
% \end{table}

\Cref{tab:numberGM} shows that increasing the number of training samples generally improves performance. 
However, performance plateaus as the sample size increases, with 512 samples yielding the best results across various datasets, except for L2-ARCTIC. 
This suggests MixGoP performs well even with a relatively small number of samples (fewer than 100), which is advantageous for dysarthric and nonnative speech, where data is often limited.
However, this also indicates that adding more data does not necessarily lead to further improvements, indicating that the model's performance may be constrained by data scalability.
