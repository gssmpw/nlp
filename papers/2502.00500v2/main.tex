\def\isarxiv{1} %%% for icml submission version, we comment this line

\ifdefined\isarxiv
\documentclass[11pt]{article}

\usepackage[numbers]{natbib}

\else
\documentclass{article}

% Recommended, but optional, packages for figures and better typesetting:




% Attempt to make hyperref and algorithmic work together better:
\newcommand{\theHalgorithm}{\arabic{algorithm}}

% Use the following line for the initial blind version submitted for review:
% \usepackage[accepted]{icml2025}
\usepackage{icml2025}

% Todonotes is useful during development; simply uncomment the next line
%    and comment out the line below the next line to turn off comments
%\usepackage[disable,textsize=tiny]{todonotes}
\usepackage[textsize=tiny]{todonotes}

\icmltitlerunning{Video Latent Flow Matching: Optimal Polynomial Projections for Video Interpolation and Extrapolation}

\fi


\usepackage{amsmath}
\usepackage{amsthm}
\usepackage{amssymb}
\usepackage{algorithm}
% \usepackage{subfig}
\usepackage{algpseudocode}
\usepackage{graphicx}
\usepackage{grffile}
\usepackage{wrapfig,epsfig}
\usepackage{url}
\usepackage{xcolor}
\usepackage{epstopdf}

\usepackage{microtype}
\usepackage{graphicx}
\usepackage{subfig}
% \usepackage{subfigure}
\usepackage{booktabs} % for professional tables

% hyperref makes hyperlinks in the resulting PDF.
% If your build breaks (sometimes temporarily if a hyperlink spans a page)
% please comment out the following usepackage line and replace
% \usepackage{icml2025} with \usepackage[nohyperref]{icml2025} above.
\usepackage{hyperref}

\usepackage{bbm}
\usepackage{dsfont}

 
\allowdisplaybreaks
 

\ifdefined\isarxiv

\let\C\relax
\usepackage{tikz}
\usepackage{hyperref}  %%% arxiv don't allow this.
\hypersetup{colorlinks=true,citecolor=blue,linkcolor=blue} %%% Zhao : maybe we should comment this in submission.
\usetikzlibrary{arrows}
\usepackage[margin=1in]{geometry}
 

\fi
 
\graphicspath{{./figs/}}

\theoremstyle{plain}
\newtheorem{theorem}{Theorem}[section]
\newtheorem{lemma}[theorem]{Lemma}
\newtheorem{definition}[theorem]{Definition}
\newtheorem{notation}[theorem]{Notation}
%\newtheorem{proof}[theorem]{Proof}
\newtheorem{proposition}[theorem]{Proposition}
\newtheorem{corollary}[theorem]{Corollary}
\newtheorem{conjecture}[theorem]{Conjecture}
\newtheorem{assumption}[theorem]{Assumption}
\newtheorem{observation}[theorem]{Observation}
\newtheorem{fact}[theorem]{Fact}
\newtheorem{remark}[theorem]{Remark}
\newtheorem{claim}[theorem]{Claim}
\newtheorem{example}[theorem]{Example}
\newtheorem{problem}[theorem]{Problem}
\newtheorem{open}[theorem]{Open Problem}
\newtheorem{property}[theorem]{Property}
\newtheorem{hypothesis}[theorem]{Hypothesis}

\newcommand{\wh}{\widehat}
\newcommand{\wt}{\widetilde}
\newcommand{\ov}{\overline}
\newcommand{\N}{\mathcal{N}}
\newcommand{\R}{\mathbb{R}}
\newcommand{\RHS}{\mathrm{RHS}}
\newcommand{\LHS}{\mathrm{LHS}}
\renewcommand{\d}{\mathrm{d}}
\renewcommand{\i}{\mathbf{i}}
\renewcommand{\tilde}{\wt}
\renewcommand{\hat}{\wh}
\newcommand{\Tmat}{{\cal T}_{\mathrm{mat}}}

\DeclareMathOperator*{\E}{{\mathbb{E}}}
\DeclareMathOperator*{\var}{\mathrm{Var}}
\DeclareMathOperator*{\Z}{\mathbb{Z}}
\DeclareMathOperator*{\C}{\mathbb{C}}
\DeclareMathOperator*{\D}{\mathcal{D}}
\DeclareMathOperator*{\median}{median}
\DeclareMathOperator*{\mean}{mean}
\DeclareMathOperator{\OPT}{OPT}
\DeclareMathOperator{\supp}{supp}
\DeclareMathOperator{\poly}{poly}

\DeclareMathOperator{\nnz}{nnz}
\DeclareMathOperator{\sparsity}{sparsity}
\DeclareMathOperator{\rank}{rank}
\DeclareMathOperator{\diag}{diag}
\DeclareMathOperator{\dist}{dist}
\DeclareMathOperator{\cost}{cost}
\DeclareMathOperator{\vect}{vec}
\DeclareMathOperator{\tr}{tr}
\DeclareMathOperator{\dis}{dis}
\DeclareMathOperator{\cts}{cts}



\makeatletter
\newcommand*{\RN}[1]{\expandafter\@slowromancap\romannumeral #1@}
\makeatother
% \newcommand{\Zhao}[1]{{\color{red}[Zhao: #1]}}
% \newcommand{\Yang}[1]{{\color{purple}[Yang: #1]}}
% \newcommand{\Chiwun}[1]{{\color{blue}[Chiwun: #1]}}
% \newcommand{\InernNameB}[1]{{\color{blue}[InternNameB: #1]}} %%%Change to intern name


\usepackage{lineno}
% \def\linenumberfont{\normalfont\small}


\begin{document}

\ifdefined\isarxiv

\date{}


\title{Video Latent Flow Matching: Optimal Polynomial Projections for Video Interpolation and Extrapolation}

\author{
Yang Cao\thanks{\texttt{ycao4@wyomingseminary.org}. Wyoming Seminary.}
\and
Zhao Song\thanks{\texttt{magic.linuxkde@gmail.com}. Simons Institute for the Theory of Computing, University of California, Berkeley.}
\and
Chiwun Yang\thanks{\texttt{christiannyang37@gmail.com}. Sun Yat-sen University.}
}

\else

\twocolumn[
\icmltitle{Video Latent Flow Matching:\\Optimal Polynomial Projections for Video Interpolation and Extrapolation}
% \icmlsetsymbol{equal}{*}

\begin{icmlauthorlist}
\icmlauthor{Yang Cao}{sem}
\icmlauthor{Zhao Song}{simons}
\icmlauthor{Chiwun Yang}{sun}
\end{icmlauthorlist}

\icmlaffiliation{sem}{Wyoming Seminary. Kingston, PA 18704, USA}
\icmlaffiliation{simons}{Simons Institute for the Theory of Computing, University of California, Berkeley. Berkeley, CA 94720, USA}
\icmlaffiliation{sun}{Sun Yat-sen University. Guangzhou, Guangdong, China}

\icmlcorrespondingauthor{Yang Cao}{ycao4@wyomingseminary.org}
\icmlcorrespondingauthor{Zhao Song}{magic.linuxkde@gmail.com}
\icmlcorrespondingauthor{Chiwun Yang}{christiannyang37@gmail.com}

% You may provide any keywords that you
% find helpful for describing your paper; these are used to populate
% the "keywords" metadata in the PDF but will not be shown in the document
\icmlkeywords{Deep Learning, Video Generation, Diffusion Models, Flow Matching}

\vskip 0.3in
]
% \icmlEqualContribution
\printAffiliationsAndNotice{}

\fi


\ifdefined\isarxiv
\begin{titlepage}
  \maketitle
\begin{abstract}
\begin{abstract}

% Recent works to jointly reconstruct 3D human and object from a single RGB image, are mostly model-based, that fail to capture the fine details of the clothed human body and object surface. In this paper, we introduce ReCHOR, a novel, model-free, first-method to produce realistic clothed human-object reconstructions from a monocular view. This is extremely challenging due to human-object occlusions, diverse interactions and depth ambiguity, as it needs to infer both 3D spatial awareness and high resolution details. Our core idea is based on estimating neural implicit representations for human and object respectively by an attention-based neural implicit model that attends to pixel-aligned features from both the global human-object image for spatial awareness and  the local separate view of human and object images for high quality details. Additionally, the network is conditioned on semantic features from an initial estimated human-object pose prior and a generative diffusion model that inpaints occluded regions, thus enabling the retrieval of details from them.
% We also propose a synthetic dataset with rendered scenes of diverse, inter-occluded 3D human and object scans, to train our network. We evaluate our method on the synthetic and real world BEHAVE dataset. Our experiments show that our method outperforms the SOTA in achieving realistic clothed human-object reconstructions.
Recent approaches to jointly reconstruct 3D humans and objects from a single RGB image represent 3D shapes with template-based or coarse models, which fail to capture details of loose clothing on human bodies. In this paper, we introduce a novel implicit approach for jointly reconstructing realistic 3D clothed humans and objects from a monocular view. For the first time, we model both the human and the object with an implicit representation, allowing to capture more realistic details such as clothing. This task is extremely challenging due to human-object occlusions and the lack of 3D information in 2D images, often leading to poor detail reconstruction and depth ambiguity. To address these problems, we propose a novel attention-based neural implicit model that leverages image pixel alignment from both the input human-object image for a global understanding of the human-object scene and from local separate views of the human and object images to improve realism with, for example, clothing details. Additionally, the network is conditioned on semantic features derived from an estimated human-object pose prior, which provides 3D spatial information about the shared space of humans and objects. To handle human occlusion caused by objects, we use a generative diffusion model that inpaints the occluded regions, recovering otherwise lost details. For training and evaluation, we introduce a synthetic dataset featuring rendered scenes of inter-occluded 3D human scans and diverse objects. Extensive evaluation on both synthetic and real-world datasets demonstrates the superior quality of the proposed human-object reconstructions over competitive methods.
\end{abstract}

\end{abstract}
\thispagestyle{empty}
\end{titlepage}

{\hypersetup{linkcolor=black}
\tableofcontents
}
\newpage

\else

\begin{abstract}
\begin{abstract}

% Recent works to jointly reconstruct 3D human and object from a single RGB image, are mostly model-based, that fail to capture the fine details of the clothed human body and object surface. In this paper, we introduce ReCHOR, a novel, model-free, first-method to produce realistic clothed human-object reconstructions from a monocular view. This is extremely challenging due to human-object occlusions, diverse interactions and depth ambiguity, as it needs to infer both 3D spatial awareness and high resolution details. Our core idea is based on estimating neural implicit representations for human and object respectively by an attention-based neural implicit model that attends to pixel-aligned features from both the global human-object image for spatial awareness and  the local separate view of human and object images for high quality details. Additionally, the network is conditioned on semantic features from an initial estimated human-object pose prior and a generative diffusion model that inpaints occluded regions, thus enabling the retrieval of details from them.
% We also propose a synthetic dataset with rendered scenes of diverse, inter-occluded 3D human and object scans, to train our network. We evaluate our method on the synthetic and real world BEHAVE dataset. Our experiments show that our method outperforms the SOTA in achieving realistic clothed human-object reconstructions.
Recent approaches to jointly reconstruct 3D humans and objects from a single RGB image represent 3D shapes with template-based or coarse models, which fail to capture details of loose clothing on human bodies. In this paper, we introduce a novel implicit approach for jointly reconstructing realistic 3D clothed humans and objects from a monocular view. For the first time, we model both the human and the object with an implicit representation, allowing to capture more realistic details such as clothing. This task is extremely challenging due to human-object occlusions and the lack of 3D information in 2D images, often leading to poor detail reconstruction and depth ambiguity. To address these problems, we propose a novel attention-based neural implicit model that leverages image pixel alignment from both the input human-object image for a global understanding of the human-object scene and from local separate views of the human and object images to improve realism with, for example, clothing details. Additionally, the network is conditioned on semantic features derived from an estimated human-object pose prior, which provides 3D spatial information about the shared space of humans and objects. To handle human occlusion caused by objects, we use a generative diffusion model that inpaints the occluded regions, recovering otherwise lost details. For training and evaluation, we introduce a synthetic dataset featuring rendered scenes of inter-occluded 3D human scans and diverse objects. Extensive evaluation on both synthetic and real-world datasets demonstrates the superior quality of the proposed human-object reconstructions over competitive methods.
\end{abstract}
\end{abstract}

\fi


\section{Introduction}
\label{sec:intro}
% Image editing methods in diffusion models depend on user-defined control directions - users can unlock their creativity using these methods by specifying the desired manipulation through prompts~\cite{gandikota2023concept}, reference images~\cite{ruiz2022dreambooth, kumari2022customdiffusion, gal2022image, chen2024trainingfreeregionalpromptingdiffusion}, or attribute vectors~\cite{parmar2023zero,hertz2022prompt}. In this work, we ask a fundamentally different question: \emph{Can we automatically discover the underlying visual structure of a concept within diffusion model's knowledge?} %Rather than requiring user-specified controls, we aim to decompose the model's internal knowledge into meaningful directions.

% This question touches on a fundamental limitation in how we interact with diffusion models. Current control methods ~\cite{zhang2023addingconditionalcontroltexttoimage, gandikota2023concept, ye2023ipadaptertextcompatibleimage,ye2023ipadaptertextcompatibleimage, hertz2024stylealignedimagegeneration, li2023photomaker, shi2024instantbooth, chen2024trainingfreeregionalpromptingdiffusion} require users to specify their desired manipulations in advance, limiting interactive creativity. This contrasts with natural human artistic workflows, where creators dynamically explore creative ideas while jointly refining them toward meaningful artistic outcomes~\cite{hoffmann2016modeling}. This synergy between specification and exploration is not new to generative models. Early GAN architectures naturally developed disentangled latent spaces that enabled continuous\cite{harkonen2020ganspace,radford2015unsupervised, wu2021stylespace, shen2020interfacegan}, compositional control over generated images. Users could explore these spaces to discover interesting variations that would be difficult to describe in words~\cite{wu2021stylespace}, then combine them to achieve their creative goals~\cite{grabe2022towards}. 


% While diffusion models have largely superseded GANs in conditional image synthesis~\cite{dhariwal2021diffusion},  their underlying structure remains less understood. Diffusion models achieve remarkable diversity through high-dimensional latents, unlike GANs' compact latent spaces.  With a single prompt, diffusion models can generate radically different variations through different random initializations of input noise. We ask - Is it possible to discover interpretable structure within this vast space of variations?

Text-to-image diffusion models are capable of generating remarkable visual variations from a single prompt through different random initializations. However, this vast creative potential remains largely opaque to users---while we can generate diverse images, we lack understanding of the underlying structure of these variations. This presents a fundamental challenge: how can we discover and expose the latent visual capabilities encoded within these models?

\let\thefootnote\relax \footnote{$^{*}$Correspondence to \texttt{gandikota.ro@northeastern.edu}}

The challenge touches on a key limitation in how we interact with diffusion models today. Current control methods require users to explicitly specify their desired edits in advance through prompts~\cite{gandikota2023concept}, reference images~\cite{zhang2023addingconditionalcontroltexttoimage, chen2024trainingfreeregionalpromptingdiffusion, ruiz2022dreambooth,kumari2022customdiffusion, Ryu_lora, hu2021lora}, or attribute vectors~\cite{ye2023ipadaptertextcompatibleimage, hertz2024stylealignedimagegeneration, li2023photomaker, shi2024instantbooth,parmar2023zero,hertz2022prompt}. That contrasts sharply with natural human creative workflows, where artists dynamically explore creative ideas and jointly refine them toward meaningful artistic outcomes~\cite{hoffmann2016modeling}. The need for pre-specified controls creates a barrier between users and the full creative potential of these models.

Interestingly, earlier generative models like GANs~\cite{gans,karras2019style,brock2018large} naturally developed more interpretable internal structures. Their compact latent spaces often exhibited emergent disentanglement~\cite{harkonen2020ganspace,radford2015unsupervised, wu2021stylespace, shen2020interfacegan}, enabling continuous and compositional control over generated images. Users could explore these spaces to discover interesting variations that would be difficult to describe in words~\cite{wu2021stylespace}, then combine them to achieve their creative goals~\cite{grabe2022towards}.

Diffusion models have largely superseded GANs in conditional image synthesis~\cite{dhariwal2021diffusion}, achieving greater diversity through much higher-dimensional latents. And yet an understanding of the underlying structure of these larger latent spaces has remained elusive. In this work, we ask a fundamental question: \emph{Can we automatically discover the visual structure within a diffusion model's knowledge of a concept?} Rather than requiring user-specified controls, we aim to decompose the model's internal representations into expressive directions that users can explore and combine.

To address these needs, we present \textbf{SliderSpace}, a framework that brings systematic explorability to diffusion models. Given just a text prompt, SliderSpace discovers a canonical set of meaningful, diverse, and controllable directions within the model's knowledge of that concept. Each direction is implemented as a low-rank adapter~\cite{hu2021lora} that can be scaled and composed with others, allowing users to explore and smoothly combine different aspects of variation, as shown in Figure~\ref{fig:intro}.

We ground SliderSpace discovery in three key requirements for meaningful decomposition of a diffusion model's visual manifold: 
\begin{enumerate}
    \item \textbf{Unsupervised Discovery:} The decomposition process should emerge from the intrinsic structure of the model's learned representation, rather than being guided by predefined attributes. This ensures we capture the true topology of the model's knowledge space rather than projecting our assumptions onto it.
    
    \item \textbf{Semantic Orthogonality:} Each discovered control must represent a distinct semantic direction. This is enforced in a semantic feature space, like CLIP, where every slider has an orthogonal effect in embeddings. This prevents discovering multiple controls that create similar semantic effects, making the system more efficient and easier.
    
    \item \textbf{Distribution Consistency:} Directions must induce consistent transformations across both random seeds and prompt variations. 
\end{enumerate}

These requirements naturally lead to our proposed framework, which we formalize in Section~\ref{sec:method}. As we show in our experiments, SliderSpace is architecture-agnostic, working with both conventional U-Net based models like Stable Diffusion~\cite{rombach2022high, rombach2022sd20, podell2023sdxl, turbo, dmd} and recent transformer-based architectures like Flux~\cite{flux}.

We demonstrate the expressiveness of SliderSpace through three applications: First, we show how SliderSpace can decompose high-level concepts into diverse and expressive components, revealing the natural axes of variation in the model's understanding. Second, we explore artistic style variation, where SliderSpace discovers directions that match or exceed the diversity of manually curated artist lists while being judged more useful by human evaluators. Finally, we show how SliderSpace can help reverse the mode collapse commonly observed in distilled diffusion models, restoring diversity while maintaining generation speed.

Beyond providing practical creative control, SliderSpace opens new avenues for understanding and utilizing the latent capabilities of diffusion models. By mapping these models' visual potential into intuitive, composable directions, we take a step toward making their creative possibilities more accessible and interpretable to users.

% Image editing methods in diffusion models unlock the creativity of users. In this work we ask an alternate question: \emph{Can we organize and expose what of the diffusion model is already capable of?}.
% Existing methods for controlling image generation typically require users to manually specify edit directions for desired changes. This process is time-consuming, requires technical expertise, and limits the spontaneity of the creative process. For instance, if a user wants to adjust the smile of a generated person, they must explicitly request this edit, often through imprecise prompt engineering or model fine-tuning. This approach of predefined controls or manual specifications restricts users from fully exploring the latent capabilities of the model. There may be interesting stylistic variations or attributes that the model can generate, but users have no easy way to discover or utilize these.

% Natural visual disentanglement was an emergent property in the latent space of Generative Adversarial Models (GANs) \cite{harkonen2020ganspace,radford2015unsupervised, wu2021stylespace, shen2020interfacegan}. In particular, it has been observed that StyleGAN~\cite{karras2019style} stylespace neurons offer detailed control over many meaningful aspects of images that would be difficult to describe in words~\cite{wu2021stylespace}. However, diffusion models do not share such a compact latent space~\cite{park2023unsupervised}; and efforts to uncover such a space in the semantic embeddings of the text conditioning have met with limited success \nik{Nick - is there a specific citation you were thinking about?}.

% In this work we introduce \textbf{SliderSpace}, which takes a step towards uncovering an analogous low dimensional representation of diffusion models' visual breadth; in essence treating the diffusion model as many generators sharing parameters, where a particular generator is defined by a specific prompt. For a given prompt we sample many random seeds (and optionally prompt expansions using an LLM), generate the corresponding images, and apply an off the shelf feature extractor (in this work CLIP, but our method can be applied to any differentiable feature extractor). We use PCA to analyze these features, and for each of the leading $k$ principal components we train a LoRA \cite{} which causes the diffusion model to produces images which increase the feature magnitude along that component when passed back through the same feature extractor. This leads to a 'Slider' for each principal component, because each LoRA can be scaled and applied to the original diffusion model, continuously varying those visual features in the generated results (as measured, in our case, by CLIP).

% There are many other works that enhance the controllability of diffusion models. One common approach is enabling users to add spatial constraints to a generation either manually, or via a reference image \cite{zhang2023addingconditionalcontroltexttoimage, chen2024trainingfreeregionalpromptingdiffusion}, a second is leveraging more abstract embeddings (e.g. identity, style) extracted from a reference image \cite{ye2023ipadaptertextcompatibleimage, hertz2024stylealignedimagegeneration, li2023photomaker, shi2024instantbooth}, a third is finetuning a foundation model to better generate a concept important to the user \cite{ruiz2022dreambooth, kumari2022customdiffusion, Ryu_lora, hu2021lora}, and a fourth (most relevant to this work) is finding low-rank adaptors of the model based on a prompt or small training set which can be scaled to provide continous control over one aspect of generated image (e.g. night vs day, basic vs luxury, etc.) \cite{gandikota2023concept}. SliderSpace is complementary to all of these methods and offers something distinct. All of the other methods we are aware require the user (and / or model designer) to know in advance what type of control they want. In contrast SliderSpace assists users in discovering and controlling hidden capabilities present in the diffusion model's distribution of possible generations.

%We propose that truly intuitive creative control in a text-to-image model should meet three key criteria: \emph{discoverability}, \emph{intuitiveness}, and \emph{specificity}. The model should reveal controllable attributes that may not be immediately obvious, offer controls that are easy to understand and manipulate, and ensure each control affects a distinct attribute of the generated image.

% We demonstrate the utility and power of SliderSpace using three applications built on top of SDXL-DMD \cite{dmd}, because its fast generation speed lends itself well to the continuous control offered by SliderSpace.

% First, we study concept decomposition (Section \ref{sec:concept_exp}), where we learn sliders for a specific concept (e.g. 'monster', 'waterfall', 'car'). Through quantitative metrics of diversity and text alignment we demonstrate that the learned sliders dramatically boost the diversity of generations when randomly applied without harming text alignment; we also ask humans to qualitatively judge these results in a user study where they find the SliderSpace results to be more 'Diverse', 'Useful', and 'Creative' than our baselines.

% Second, we attempt to compare the automatic discoveries of SliderSpace to a large scale manual study of artistic styles (Section \ref{sec:art_exp}), open-sourced by ParrotZone \cite{parrotzone}. In this study SDXL was prompted with over 4300 artist names,  and based on visual inspection the cases of successful stylistic mimicry recorded. Quantitatively SliderSpace more closely matches the distribution of artistic variation discovered by ParrotZone than other baselines, and in our user studies was judged to be significantly more 'Diverse' and 'Useful' than the baselines. To our surprise humans even judged SliderSpace results to be slightly more 'Diverse' than the results generated by the manually discovered artist names of \cite{parrotzone}.

% Third, we attempt to use SliderSpace to reverse the mode collapse commonly observed in distilled few-step diffusion models relative to the original teacher model (Section \ref{sec:diverse_exp}). We quantitatively demonstrate that applying SliderSpace to SDXL-DMD leads to more closely matching the distribution of images by the original teacher, SDXL.

%Through extensive experiments on various state-of-the-art text-to-image models, we demonstrate that SliderSpace significantly enhances user control and creative expression in AI-assisted image generation tasks. Our method enables a range of applications, including concept decomposition and control, diversity improvement in generated images, customization dissection and edits, and the exploration of artistic styles inherent in the model.

% SliderSpace goes beyond providing a practical tool for enhanced creative control. By mapping the visual potential of diffusion models it can open new avenues for generative creativity and deepens our understanding of each model's hidden potential.

\section{Related Work}
\label{sec:related_work}

The original investigation \cite{gibson1979ecological} on the relationship between visual perception and human action defines \emph{affordance} as the opportunities for interaction with the surrounding environment. Behavioral studies on regular and cognitively impaired persons have shown evidence that perception results in both visual and motor signals in the human brain. An extended study \cite{anderson2002attentional} shows that visual attention to the spatial characteristics of the perceived objects initiates automatic motor signals for different actions. In computer vision, human affordance learning involves novel pose prediction such that the estimated pose represents a valid human action within the scene context. The task is fundamental to many problems requiring robust semantic reasoning about the environment, such as human motion synthesis \cite{wang2021scene} and scene-aware human pose generation \cite{wang2017binge, roy2016multi, zhang2022inpaint, yao2023scene}.

Earlier methods of affordance learning have explored knowledge mining \cite{zhu2014reasoning} and multimodal feature cues \cite{roy2016multi} to address the problem. In \cite{zhu2014reasoning}, the authors use a Markov Logic Network for constructing a knowledge base by extracting several object attributes from different image and metadata sources, which can perform various downstream visual inference tasks without any additional classifier, including zero-shot affordance prediction. In \cite{roy2016multi}, the authors use depth map, surface normals, and segmentation map as multimodal cues to train a multi-scale convolutional neural network (CNN) for scene-level semantic label assignment associated with specific human actions. In \cite{do2018affordancenet}, the authors design a multi-branch end-to-end CNN with two separate pathways for object detection and affordance label assignment to achieve high real-time inference throughput. Researchers \cite{chuang2018learning} have also explored socially imposed constraints for affordance learning. In \cite{chuang2018learning}, the authors propose a graph neural network (GNN) to propagate contextual scene information from egocentric views for action-object affordance reasoning.

Probabilistic modeling of scene-aware human motion generation also involves semantic reasoning of human interaction with the environment. Initial works on human motion synthesis have taken different architectural approaches, such as sequence-to-sequence models \cite{barsoum2018hp}, generative adversarial networks (GAN) \cite{barsoum2018hp, cai2018deep, yang2018pose}, graph convolutional networks (GCN) \cite{yan2019convolutional}, and variational autoencoders (VAE) \cite{guo2020action2motion}. However, these methods have mostly ignored the role of environmental semantics. Due to potential uncertainty in human motion, in a recent approach \cite{wang2021scene}, the authors address such motion synthesis with a GAN conditioned on scene attributes and motion trajectory to predict probable body pose dynamics.

One key challenge of human affordance generation in 2D scenes is the lack of large-scale datasets with rich pose annotations. In \cite{wang2017binge}, the authors compile the only public dataset of annotated human body poses in complex 2D indoor scenes by extracting frames from sitcom videos. Aiming to generate a contextually valid human affordance at a user-defined location, the authors propose sampling the scale and deformation parameters for an existing human pose template using a VAE conditioned on the localized image patches as scene context. In \cite{zhang2022inpaint}, the authors introduce a two-stage GAN architecture for achieving a similar goal by estimating the affine bounding box parameters to localize a probable human in the scene and then generating a potential body pose at that location. The method uses the input scene, corresponding depth, and segmentation maps as semantic guidance. In \cite{yao2023scene}, the authors propose a transformer-based approach with knowledge distillation for generating human affordances in 2D indoor scenes.



\section{Preliminary} \label{sec:preli}

In Section~\ref{sub:notation}, we introduce all the notations we used in our paper. Then, in Section~\ref{sub:flow_matching}, we show the basic facts about flow matching. In Section~\ref{sub:special_relativity}, we present the basic background of special relativity and define the relativistic force.

\subsection{Notations} \label{sub:notation}

For any positive integer $n$, we use $[n]$ to denote set $\{1,2,\cdots, n\}$. 
For two vectors $x \in \R^n$ and $y \in \R^n$, we use $\langle x, y \rangle$ to denote the inner product between $x,y$.
For a vector $v \in \R^n$, we use $\|v\|_2$ to denote the $\ell_2$-norm of $v$.
We use ${\bf 1}_n$ to denote a length-$n$ vector where all the entries are ones.
We use the symbol $ \perp $ to represent a component that is perpendicular to the direction of velocity, as exemplified by $ a_{\perp t} $, which denotes the perpendicular acceleration. Similarly, the symbol $ \parallel $ is employed to indicate a component parallel to the direction of velocity, such as $ f_{\parallel t} $, which represents the parallel force. We use $\dot{x}_t$ to denote $\frac{\d x_t}{\d t}$, and $\ddot{x}_t$ to denote $\frac{\d^2 x_t}{\d t^2}$.


\subsection{Flow Matching} \label{sub:flow_matching}

Flow Matching (FM) \cite{lcb+22,lgl22} is a generative modeling technique that constructs a smooth, invertible (i.e., diffeomorphic) mapping from a simple prior distribution to a complex target distribution. In FM, a time-dependent mapping $Z_t$ is defined to evolve according to an ordinary differential equation (ODE) driven by a vector field:
\begin{align*}
    \frac{\d x_t}{\d t} = V_t(x_t), \quad t \in [0, T].
\end{align*}
The goal is to ensure that, at the terminal time $T$, the ODE transforms a sample $x_0$ from a simple distribution (e.g., a Gaussian) into a sample $x_T$ from the target data distribution $\mathcal{D}$.

To achieve this, Flow Matching (FM) constructs a stochastic interpolation between a sample $x_1 \sim \mathcal{D}$ and a sample $x_0$ drawn from a known prior distribution, typically $\N(0,I)$. The interpolation is defined as
\begin{align*}
    x_t := \alpha_t x_1 + \sigma_t x_0, \quad t\in [0,T],
\end{align*}
where the time-dependent coefficients $\alpha_t$ and $\sigma_t$ are chosen so that
\begin{align*}
    \alpha_0 = 0,\quad \sigma_0 = 1,\quad \alpha_T = 1,\quad \sigma_T = 0.
\end{align*}
Thus, at $t=0$ the interpolated sample is purely the prior ($x_0$), and at $t=T$ it becomes a data sample ($x_1$).

The instantaneous change of $x$ is obtained by differentiating the interpolation:
\begin{align*}
    \frac{\d x_t}{\d t} = \frac{\d \alpha_t}{\d t} x_1 + \frac{\d \sigma_t}{\d t} x_0.
\end{align*}

The vector field is approximated by a neural network $V_t(x_t)$ with learnable parameters $\theta$. The FM training objective is then given by
\begin{align*}
    \mathcal{L}_\mathrm{FM}(\theta) := \E_{t\sim {\sf Uniform}[0,T], x_1 \sim \mathcal{D}} [\| V_t(x_t) - v_t(x_t) \|_2^2 ].
\end{align*}
This loss ensures that the learned velocity field $V_t(x_t)$ closely tracks the conditional dynamics $v_t(x_t)$ along the interpolation path.

After training, samples are generated by solving the ODE
\begin{align*}
    \frac{\d x_t}{\d t} = V_t(x_t),
\end{align*}
starting from an initial sample $x_0 \sim \N(0,I)$. Integrating this ODE from $t=0$ to $t=T$ yields a sample $x_T$ that approximates a draw from the target distribution. This ODE-based formulation offers a flexible and powerful framework for modeling complex data distributions while naturally incorporating conditional sampling.

\subsection{Background on Special Relativity} \label{sub:special_relativity}

We first introduce several essential ideas of special relativity \cite{e+05}.

\begin{definition}[Lorentz Factor]
\label{def:LorentzFactor}
According to special relativity~\cite{e+05}, the Lorentz factor at lab time $t$ is given by
\begin{align*}
\gamma_t := \frac{1}{\sqrt{1 - {\|v_t^{\rm lab}\|_2^2}/{c^2}}},
\end{align*}
where $v_t^{\rm lab}$ is the velocity at lab frame of reference, $c = 3 \times 10^8$ is the speed of light in vacuum.
\end{definition}

Then, we introduce the proper time of special relativity.

\begin{definition}[Proper Time]
\label{def:ProperTime}
The proper time is defined as the time interval measured in the rest frame of a moving object according to special relativity~\cite{e+05}. The differential form of the proper time is given by
\begin{align*}
    \d \tau = \frac{\d t}{\gamma_t},
\end{align*}
where $\d t$ is the time interval in the laboratory frame of reference, and $\gamma_t$ is the Lorentz factor at time lab time $t$ as defined in Definition~\ref{def:LorentzFactor}.
\end{definition}

Next, we define the force under special relativity here.

\begin{definition}[Relativistic Force]
\label{def:RelativisticForce}
In the framework of special relativity, the \emph{local force} (i.e., the force measured in the instantaneous rest frame of the particle) denoted as $f^{\rm local}$ has
\begin{align}
    f^{\rm local} := \frac{\d p^{\rm lab}}{\d \tau}, \label{eq:f_local}
\end{align}
where $p^{\rm lab}$ is the momentum at lab frame of reference, $\tau$ denotes the proper time defined in Definition~\ref{def:ProperTime}.

The momentum in the lab frame is defined as
\begin{align}
    p^{\rm lab} := m^{\rm lab} v_t^{\rm lab}, \label{eq:p}
\end{align}
where $m^{\rm lab}$ is the mass at lab frame of reference, and $v_t^{\rm lab}$ is the velocity at lab frame of reference.
\end{definition}

We state an equivalence lemma. Due to the space limitation, we delayed the proofs into the appendix.
\begin{lemma}[Equivalent Form of Relativistic Force, informal version of Lemma~\ref{lem:equiv_relativistic_force:formal}]\label{lem:equiv_relativistic_force:informal}
Let $p^{\rm lab}$ be the momentum defined in Eq.~\eqref{eq:p}, $\gamma_t$ be the Lorentz factor at lab time $t$ defined in Definition~\ref{def:LorentzFactor}, $\tau$ denotes the proper time, $v_t^{\rm lab} = \dot{x}_t$ denotes the velocity, 
$a_t^{\rm lab} = \ddot{x}_t$ denotes the acceleration.
The relativistic force, defined as the time derivative of the momentum in the lab frame, can be written as
\begin{align*}
f^{\rm local} =  m^{\rm lab}  (\gamma_t a_t^{\rm lab} + \gamma_t^3 \frac{ \langle v_t^{\rm lab}, a_t^{\rm lab} \rangle}{c^2} v_t^{\rm lab}).
\end{align*}

\end{lemma} 

\section{Video Latent Flow Matching} \label{sec:vlfm}

In this section, we propose Video Latent Flow Matching (VLFM) in response to the main problem in Section~\ref{sub:problem_def}. Especially, we briefly review the HiPPO (high-order polynomial projection operators) framework \cite{gde+20} in Section~\ref{sub:hippo}. We state the derivation of our VLFM based on the popular flow matching approach \cite{lcb+22} in Section~\ref{sub:vlfm}. Finally, we define the training objective of the VLFM for efficient video modeling in Section~\ref{sub:training_objective}.


\subsection{HiPPO Framework and LegS State Space Model}\label{sub:hippo}

Given an input function $f(t) \in \R$ for $t \ge 0$, we use $f_{\leq t}$ to denote the cumulative history of $f(t)$ for every time $t \ge 0$. We choose integer $s \ge 1$ as the order of approximation. Then, any $s$-dimensional subspace ${\cal G}$ of this function space is a suitable candidate for the approximation. Given a time-varying measure family $p(t)$ supported on $(-\infty, t)$, a sequence of basis functions ${\cal G} = {\rm span}\{ g_{i}(t) \}_{i=1}^s$. HiPPO \cite{gde+20} defines an operator that maps $f$ to the optimal projection coefficients $c: \R_{\ge 0} \rightarrow \R^s$, such that:
\begin{align*}
    g(t) := & ~ \arg \min_{g \in {\cal G}} \| f_{\leq t} -  g \|_{p(t)}, \\
    g(t) = & ~ \sum_{i=1}^s c_i(t) \cdot g_i(t).
\end{align*}
We focus on the case where the coefficients $c(t)$ have the form of a linear ODE satisfying $\nabla c(t) = A(t) c(t) + B(t) f(t)$ for some $A(t) \in \R^{s \times s}$ and $B(t) \in \R^{s \times 1}$. This equation is now also known as the state space model (SSM) in many works \cite{kds+15,aia+22,gd23,dg24,zlz+24,xyy+24,mlw24,rx24,sld+24}.

{\bf Discrete HiPPO-LegS.} The setting of HiPPO-LegS defines the update rule of SSM and the discrete version of $A$ and $B$ matrices, which are $c_{\tau + 1} = (I_s - \frac{A}{\tau}) c_\tau + \frac{1}{\tau} B f_\tau$ and:
\begin{align*}
    A_{i_1, i_2} & ~ = \begin{cases}
        \sqrt{(2i_1 + 1)(2i_2 + 1)}, & \text{if $i_1 > i_2$} \\
        i_1 + 1, & \text{if $i_1 = i_2$} \\
        0, & \text{if $i_1 < i_2$}
    \end{cases}, \\
    B_{i_1} & ~ = \sqrt{2i_1 + 1}, \forall i_1, i_2 \in [s].
\end{align*}

\subsection{Conditional Video Latent Flow}\label{sub:vlfm}


Here we emphasize the core idea of VLFM is to approximate a continuous video distribution from limited discrete video frames data utilizing the optimal high-order polynomial approximation. 

Given a video-caption distribution ${\cal V}_c$, then for any video-caption data pair $(V, c) \sim {\cal V}_c$, we obtain the data $\wt{u}_\tau \in \R^d, \forall \tau \in [N]$ via Eq.~\eqref{eq:u_tau:informal}.
We aim to define a time-dependent flow $\psi_t(\wt{u})$ that takes inputs $\wt{u}$ and time $t$, and could match $\widehat{u}_\tau$ for all time $\tau \in [N]$. Since $\widehat{u}$ is discrete, HiPPO-LegS will be the best solution to approximate the continuous data. We define the \emph{Video Latent Flow} as:
\begin{align}\label{eq:psi}
    \psi_t(\wt{u}) := \sigma_t( \wt{u} ) \cdot z + \mu_t(\wt{u}) \in \R^d,
\end{align}
where $t \in [0,T]$ and $z \sim \mathcal{N}(0, I_d)$, $\sigma: [0, T] \times \R^{N \times d} \rightarrow \R_{> 0}$ denotes the time-dependent standard deviation, where $\sigma_0 ( \wt{u} ) = 1$, and $\sigma_{\frac{T}{N} \cdot \tau}( \wt{u} ) = \sigma_{\min}$, for all $\tau \in [N]$ ; $\mu: [0, T] \times \R^{N \times d} \rightarrow \R^d$ denotes the time-dependent mean of Gaussian distribution, where $\mu_0(\wt{u}) = {\bf 0}_d$, $\mu_{\frac{T}{N} \cdot \tau}(\wt{u}) = \wt{u}_\tau,$ for all $\tau \in [N]$.

Especially, we define:
\begin{align*}
    \mu_t(\wt{u} ) := & ~  H_N g(t), \\
    H_{\tau + 1} := & ~ H_{\tau} (I_s - \frac{1}{\tau} A)^\top + \frac{1}{\tau} \wt{u}_\tau B^\top,
\end{align*}
where $g(t) := [\sqrt{\frac{1}{2}} P_0(t), \sqrt{\frac{3}{2}} P_1(t), \cdots, \sqrt{\frac{2s-1}{2}} P_{s-1}(t)]^\top $ $\in \R^{s}$, $P_i(t), \forall i \in [s]$ is Legendre polynomials. We initialize $H_0 := {\bf 0}_{d \times s}$.

Besides, having a large scalar $\alpha > 0$, we give:
\begin{align*}
    \sigma_t(\wt{u}) := (1 - \sigma_{\min}) \cdot [\sin^2( \pi \frac{N}{T} t ) +  \exp(-\alpha t) ] + \sigma_{\min}.
\end{align*}

\subsection{Training Objective}\label{sub:training_objective}

Here we define a model function $F_\theta: \R^d \times \R^\ell \times [0, T] \rightarrow \R^d$ with parameters $\theta$ to learn the conditional video latent flow $\psi_t(\wt{u})$ defined in Eq.~\eqref{eq:psi}. This function takes inputs of flow and time to predict the vector field. The training objective is based on the Flow Matching framework \cite{lcb+22}, which aims to minimize the distance between the model's prediction and the true derivative of the flow.

The training objective of VLFM is defined as the expectation of the square $\ell_2$ norm of the difference, which is:
\begin{align*}
    {\cal L}(\theta) := \E_{z, t, (V, c)}[\| F_\theta( \psi_t(\wt{u}), c, t ) - \frac{\d }{\d t} \psi_t(\wt{u}) \|_2^2],
\end{align*}
where $z \sim \mathcal{N}(0, I_d)$,  $t \sim {\sf Uniform}[0, T]$ and $(V, c) \sim {\cal V}_c$. By minimizing this objective, the model learns to approximate the vector field that transports the initial noise distribution to the distribution of video latent patches. Formally, we solve: $\min_{\theta} {\cal L}(\theta)$. 

{\bf Close-form solution.} Furthermore, the close-form solution could be easily obtained as follows:
\begin{theorem}
    The minimum solution for function $F_\theta$ that takes $z \sim N(0, I_d)$ and $t \sim {\sf Uniform}[0, T]$ is:
    \begin{align*}
        F_\theta(z, c, t) = \frac{\sigma_t'(\wt{u})}{\sigma_t(\wt{u})} (z - \mu_t(\wt{u})) + \mu_t'(\wt{u}).
    \end{align*}
\end{theorem}

\begin{proof}
    This proof follows from Theorem 3 in \cite{lcb+22}.
\end{proof}



\section{Theory}\label{sec:theory}

This section provides several theoretical advantages of our VLFM. The approximation theory in this approach builds up based on using the Diffusion Transformer (DiT) \cite{px23}, which is a popular choice in previous empirical and theoretical part generative model works \cite{chzw23, hwsl24}, we briefly state its definitions in Section~\ref{sub:DiT}.

In addition, we provide the optimal polynomial projection guarantee and universal approximation theorem (with DiT) of VLFM in Section~\ref{sub:approx} to confirm its approximating ability. Besides, Section~\ref{sub:inter-extra_polation_theory} gives error bound of interpolation and extrapolation, and Section~\ref{sub:timescale_robustness} gives the supplementary property that VLFM's timescale robustness, which indicates its theoretical advantages.

\subsection{Diffusion Transformer (DiT)}\label{sub:DiT}

Diffusion Transformer \cite{px23} is a framework that utilizes Transformers \cite{vnn+17} as the backbone for Diffusion Models \cite{hja20,sme20}. Specifically, a Transformer block consists of a multi-head self-attention layer and a feed-forward layer, with both layers having a skip connection. 
We use ${\sf TF}^{h, m, r}: \R^{n \times d_0}\rightarrow \R^{n \times d_0}$ to denote a Transformer block.
Here $h$ and $m$ are the number of heads and head size in self-attention layer, and $r$ is the hidden dimension in feed-forward layer.
Let $X \in \R^{n \times d_0}$ be the model input. Then, we have the model output:
\ifdefined\isarxiv
\begin{align*}
    {\sf Attn}(X) := \sum_{i=1}^h {\sf Softmax}( X W_Q^i {W_K^i}^\top X^\top ) \cdot X W_V^i {W_O^i}^\top + X,
\end{align*}
\else
\begin{align*}
    & ~ {\sf Attn}(X) \\
    & ~ := \sum_{i=1}^h {\sf Softmax}( X W_Q^i {W_K^i}^\top X^\top ) \cdot X W_V^i {W_O^i}^\top + X,
\end{align*}
\fi
where the projection weights $W_K^i, W_Q^i, W_V^i, W_O^i \in \R^{d_0 \times m}$. Moreover,
\begin{align*}
    {\sf FF}(X) := \phi(X W_1 + {\bf 1}_n b_1^\top) \cdot W_2^\top + {\bf 1}_n b_2^\top + X.
\end{align*}
where  the projection weights $W_1, W_2 \in \R^{d_0 \times r}$, bias $b_1 \in \R^{r}, b_2 \in \R^{d_0}$, and $\phi$ is usually considered as the ReLU activated function.

In our work, we use Transformer networks with positional encoding $E\in\R^{n \times d_0}$. The transformer networks are then defined as the composition of Transformer blocks:
\begin{align*}
    {\cal T}_{P}^{h,m,r} = & ~ \{f_{{\cal T}}:\R^{ n \times d_0 }\rightarrow {\R^{n \times d_0}} \\
    & ~ \mid f_{{\cal T}}\text{ is a composition of blocks }{\sf TF}^{h,m,r}\text{'s}\}.
\end{align*}
For example, the following is a Transformer network consisting $K$ blocks and positional encoding
\begin{align*}
f_{{\cal T}}(X)= {\sf FF}^{(K)} \circ {\sf Attn}^{(K)} \circ  \cdots {\sf FF}^{(1)} \circ  {\sf Attn}^{(1)} (X+E).
\end{align*}

\subsection{Approximation via DiT}\label{sub:approx}



Before we state the approximation theorem, we define a reshaped layer that transforms concatenated input in flow matching into a length-fixed sequence of vectors. It is denoted as $R: \R^{d+\ell+1} \rightarrow \R^{n \times d_0}$. Therefore, in the following, we give the theorem utilizing DiT to minimize training objective ${\cal L}(\theta)$ to arbitrary error.

\begin{theorem}[Informal version of Theorem~\ref{thm:uat}]\label{thm:uat:informal}
    There exists a transformer network $f_{\cal T} \in {\cal T}_{P}^{2, 1, 4}$ defining function $F_\theta(z, c, t) := f_{\cal T}( R([z^\top, c^\top, t]^\top) )$ with parameters $\theta$ that satisfies ${\cal L}(\theta) \leq \epsilon$ for any error $\epsilon > 0$. 
\end{theorem}

\begin{proof}[Proof sketch of Theorem~\ref{thm:uat:informal}]
    Please refer to the proof of Theorem~\ref{thm:uat} for the detailed analysis.
\end{proof}


\subsection{Interpolation and Extrapolation}\label{sub:inter-extra_polation_theory}

Now, we theoretically discuss the approximating error of our VLFM in processing interpolation and extrapolation. It is considered a recovery of the original idea data from limited sub-sampled observations. This analysis is achieved by splitting the error into three parts, which are: 1) approximating error $\epsilon_1$ for HiPPO-LegS approximating the original data; 2) Gaussian error $\epsilon_2$ for the boundary of Gaussian vector $z$; 3) interpolation and extrapolation error $\epsilon_3$ that represents the training and predicting the difference between using original idea data $V$ and limited sub-sampled observations $\Phi \wt{V}$. We state the results as follows:
\begin{lemma}[Informal version of Lemma~\ref{lem:hippo_error}]\label{lem:hippo_error:informal}
    Denote failure probability $\delta \in (0, 0.1)$. Let the flow $\psi_t( \wt{u} )$ defined in Eq.~\eqref{eq:psi}. Denote $G := [g(\Delta t), g(2 \Delta t), \cdots, g(T)]^\top \in \R^{\frac{T}{\Delta t} \times s}$ and $\lambda^* := \lambda_{\min}(G) > 0$ as the minimum eigenvalue of $G$. Choosing $s = O(\frac{\Delta t}{T}\log((\frac{\Delta t}{T})^{1.5}\lambda^*))$. Denote $u_t = {\cal D}( V_{t} )$ for any $t \in [0, T]$. Especially, we define:
    \begin{itemize}
        \item Approximating error $\epsilon_1 := O(T^{k} s^{-k+1/2})$.
        \item Gaussian error $\epsilon_2 := O(\sqrt{d\log(d/\delta)})$.
        \item Interpolation and extrapolation error $\epsilon_3 := U d^{0.5} \sqrt{\frac{T}{\Delta t} - N} \cdot \exp(O(\frac{T}{\Delta t}s)) / \lambda^*$.
    \end{itemize}
    Then with a probability at least $1 - \delta$, we have:
    \begin{align*}
        \| \psi_t( \wt{u} ) - u_t \|_2 \leq \epsilon_1 + \epsilon_2 + \epsilon_3.
    \end{align*}
\end{lemma}

\begin{proof}{Proof sketch of Lemma~\ref{lem:hippo_error:informal}}
    This proof follows from its formal version in Lemma~\ref{lem:hippo_error}
\end{proof}

Having Lemma~\ref{lem:hippo_error:informal}, the concise bound for solving Eq.~\eqref{eq:main} could be given below:
\begin{theorem}[Informal version of Theorem~\ref{thm:inter_extra_polation}]\label{thm:inter_extra_polation:informal}
    Following Theorem~\ref{thm:uat:informal}, denote failure probability $\delta \in (0, 0.1)$ and arbitrary error $\epsilon_0 > 0$. Then with a probability at least $1 - \delta$, the network in Theorem~\ref{thm:uat:informal} satisfies Eq.~\eqref{eq:main} with $p = 2$ and
    \begin{align*}
        \epsilon = \epsilon_0 + L_0(\epsilon_1 + \epsilon_2 + \epsilon_3).
    \end{align*}
\end{theorem}

\begin{proof}[Proof sketch of Theorem~\ref{thm:inter_extra_polation:informal}]
    Please refer to Theorem~\ref{thm:inter_extra_polation} for complete proofs.
\end{proof}

{\bf Discussions.} Following the results of Lemma~\ref{lem:hippo_error:informal} and Theorem~\ref{thm:inter_extra_polation:informal}, we thus derive few insights as follows:
\begin{itemize}
    \item {\bf Optimal choice of $s$: A trade-off between $\epsilon_1$ and $\epsilon_3$. } As shown in the conditions of Lemma~\ref{lem:hippo_error:informal}, the larger value of the order of polynomials $s$ helps to decrease approximating error in the training dataset while also ruining the generalization ability.
    \ifdefined\isarxiv
    \else
    \vspace{-2mm}
    \fi
    \item {\bf Stable visual decoder. } Theorem~\ref{thm:inter_extra_polation:informal} shows a small value of $L_0$ (the stability and smoothness of visual decoder), which is important for the error of interpolation and extrapolation with an arbitrary frame rate.
    \ifdefined\isarxiv
    \else
    \vspace{-3mm}
    \fi
    \item {\bf Information. } Besides, a sub-linear factor $\sqrt{\frac{T}{\Delta t} - N}$, which stands for the obtained information about the continuous video, is vital as well for interpolation and extrapolation on data in distribution.
\end{itemize}
\ifdefined\isarxiv
\else
\vspace{-6mm}
\fi

\subsection{Timescale Robustness}\label{sub:timescale_robustness}

Following \cite{gde+20}, we demonstrate that projection onto latent patches $u_t$ is robust to timescales. Formally, the HiPPO-LegS operator is {\it timescale-equivariant}: dilating the input $u$ does not change the approximation coefficients $H_N$. At the same time, this property is working in the case of the discretized form $\wt{u}$. We emphasize that it is crucial to use flow matching to model the latent patches, where whatever the sampling method and frame rate are, it will not greatly harm VLFM's performance. We give its formal statement below.

\begin{lemma}[Proposition 3 of \cite{gde+20}, informal version of Lemma~\ref{lem:timescale_robustness}]\label{lem:timescale_robustness:informal}
    For any integer scale factor $\beta > 0$, the frames of video $\wt{V}_\tau$ is scaled to $\wt{V}_{\beta \tau}$ for each $\tau \in [\frac{T}{\Delta t}]$, it doesn’t affect the result of $H_N$.
\end{lemma}
\ifdefined\isarxiv
\else
\vspace{-3mm}
\fi
\begin{proof}
    This lemma follows from Proposition 3 in \cite{gde+20}.
\end{proof}
\ifdefined\isarxiv
\else
\vspace{-6mm}
\fi


\section{Evaluation}
\label{sec:exps}

\subsection{Experimental Setup}


\paragraphbe{Models.} We evaluate our attack on three closed-source (o1, o1-mini and o3-mini) and open-source (DeepSeek-R1) reasoning models. These models leverage advanced reasoning methods such as CoT, and are well-known for excelling on a range of complex tasks and benchmarks~\cite{guo2025deepseek, sun2023survey}.

\paragraphbe{Datasets.} We evaluate our attack using FreshQA~\cite{vu2023freshllms} and SQuAD~\cite{rajpurkar2018know}. FreshQA is a dynamic question-answering (QA) benchmark designed to assess the factual accuracy of LLMs by incorporating both stable and evolving real-world knowledge. The benchmark includes 600 natural questions categorized into four types: never-changing, slow-changing, fast changing, and false-premise. These questions vary in complexity, requiring both single-hop and multi-hop reasoning, and are linked to regularly updated Wikipedia entries. The original query consists of an average of 11.6$\pm$1.85 tokens. However, due to the randomness and the length of the context extracted from Wikipedia, the total input token count increases to an average of 11278.2$\pm$6011.49 tokens when the context is appended. This leads to a noticeable variation in input length.

SQuAD contains more than 100k questions based on more than 500 articles retrieved from Wikipedia. While the average length of a query in the dataset is similar to FreshQA with $11.5\pm3.4$ tokens, the context is significantly shorter and shows less variance in length. An average context in the dataset contains $117.5\pm37.3$ tokens.  Utilizing these two datasets allows us to study our attack and impact of factors like context length and complexity.

We select a subset of the dataset containing samples with ground-truth that changes infrequently and has lower likelihood of unintentional errors. To minimize costs and adhere to ethical considerations, we restrict our evaluations of different attack types, attack transferability and reasoning effort to five data samples from FreshQA. This ensures minimal impact on existing infrastructure while allowing us to test our attack methodologies. Subsequently, we study the impact of context-agnostic attacks on 100 samples from the FreshQA and SQuAD datasets across four models (o1, o1-mini, DeepSeek-R1, and o3-mini) and present a comprehensive analysis of the attack performance on a larger scale.







\paragraphbe{Evaluation Metrics.} 
Since we evaluated our attack using QA datasets, we measure \textbf{claim accuracy}~\citep{ min2023factscore}. This is done by using LLM-as-a-judge, where the model verifies claims in its output against a list of ground truths. A score of 1 is assigned if the claims align and 0 if they do not. For longer outputs, more sophisticated claim verification metrics could be used~\cite{song2024veriscore, wei2024long}. Additionally, since our attack introduces a decoy problem, we assess output stealthiness by measuring the presence of decoy-related information in the final output, which we refer to as \textbf{contextual correctness}. This metric evaluates how much of the output belongs to the context surround the user query versus the decoy task. We assign a score of 1 if the output contains only claims relevant to the user query's context, 0.5 if it includes claims for both contexts, and 0 if it consists entirely of decoy-related information. All the results were also manually reviewed for errors. Fig.~\ref{fig:cc evaluation prompt} and Fig.~\ref{fig:cc_score_example} in Appendix show the contextual correctness evaluation prompt and output examples respectively.




\begin{table*}[t]

\centering
\vskip 0.15in
\begin{center}
\begin{small}
\begin{sc}
\begin{tabular}{lrrrrrr}
  \toprule
  \multirow{2}{*}{Attack Type} & \multirow{2}{*}{Input} & \multirow{2}{*}{Output} & \multirow{2}{*}{Reasoning} & Reasoning & \multirow{2}{*}{Accuracy} & Contextual \\
  & & & & Increase & & Correctness \\
  \midrule
  No Attack     & 7899$\pm$5797 & 102$\pm$53 & 751$\pm$410& 1 & 100\% & 100\% \\
  \midrule
  Context-Aware       & 11282$\pm$6660 & 37$\pm$11 & 1711$\pm$693 & 2.3$\times$ & 100\% & 100\% \\  
  Context-Agnostic   & 8237$\pm$5796 & 86$\pm$30 & 7313$\pm$347 & 9.7$\times$ & 100\% & 100\% \\ 
  ICL-Genetic (Aware) & 11320$\pm$6669 & 86$\pm$96 & 5850$\pm$978 & 7.8$\times$ & 100\% & 90\% \\  
  ICL-Genetic (Agnostic)  & 11191$\pm$6657 & 98$\pm$61 & \textbf{13555$\pm$3219} & \textbf{18.1$\times$} & 100\% & 100\% \\
  \bottomrule
\end{tabular}
\end{sc}
\end{small}
\end{center}
\vskip -0.1in

\caption{Average number of reasoning tokens for different attacks for o1 (\textbf{Dataset}: FreshQA, \textbf{Decoy}: MDP).}
\label{tab:o1_main_tab}
\end{table*}


\begin{table*}[t]

\centering
\vskip 0.15in
\begin{center}
\begin{small}
\begin{sc}
\begin{tabular}{lrrrrrr}
  \toprule
  \multirow{2}{*}{Attack Type} & \multirow{2}{*}{Input} & \multirow{2}{*}{Output} & \multirow{2}{*}{Reasoning} & Reasoning & \multirow{2}{*}{Accuracy} & Contextual \\
  & & & & Increase & & Correctness \\
  \midrule
  No Attack             & 10897$\pm$6011 & 245$\pm$191 & 711$\pm$635\;\; & 1 & 100\% & 100\% \\
  \midrule
  Context-Aware               & 11338$\pm$6014 & 177$\pm$151 & 1868$\pm$2020 & 4.8$\times$ & 80\% & 100\% \\  
  Context-Agnostic       & 11236$\pm$6011 & 77$\pm$26\;\;   & 2872$\pm$2820 & 4.0$\times$ & 80\% & 100\% \\ 
  ICL-Genetic (Aware) & 11393$ \pm$5964 & 93$\pm$63\;\;  & 6980$ \pm$5693  & 5.9$\times$ & 100\% & 80\% \\  
  ICL-Genetic (Agnostic)  & 11261$\pm$6011 & 68$\pm$16\;\;   & \textbf{7489$\pm$1305} & \textbf{10.5$\times$} & 80\% & 100\% \\
  \bottomrule
\end{tabular}
\end{sc}
\end{small}
\end{center}
\vskip -0.1in
\centering
\caption{Average number of reasoning tokens for different attacks for DeepSeek-R1 (\textbf{Dataset}: FreshQA, \textbf{Decoy}: MDP).}
\label{tab:deepseek_main_tab}
\end{table*}





\begin{table}[t]
\vspace{-13pt}
\caption{Ablation results with response number under fine-tuning setting. See Reward Margins in~\Cref{tab:rm}. \vspace{-0.5em}}
\label{tab:ablation}
\vskip 0.1in
\begin{center}
\scalebox{0.75}{
\begin{tabular}{clcccc}
\toprule
\multirow{1}{*}{\textbf{Number}} & \multirow{1}{*}{\textbf{Method}} & \textbf{BLEU}$\uparrow$ & \textbf{Reward} & $\textbf{RM}_{\text{DPO}}$$\uparrow$ & $\textbf{RM}_{\text{R-DPO}}$$\uparrow$ \\
\midrule
\multirow{2}{*}{\textbf{5}} & \textbf{DPO-BT} & \textbf{0.229} & \textbf{{0.432}} & 0.166 & -0.516 \\

& \textbf{DPO-HPS} & \textbf{0.229} & 0.431 & \textbf{0.600} & \textbf{-0.273} \\
\midrule
\multirow{2}{*}{\textbf{20}} & \textbf{DPO-BT} & \textbf{0.231} & 0.430 & 0.227 & -0.490 \\

& \textbf{DPO-HPS} & 0.224 & \textbf{{0.432}} & \textbf{0.822} & \textbf{-0.181} \\
\midrule
\multirow{2}{*}{\textbf{50}} & \textbf{DPO-BT} & \textbf{0.230} & \textbf{0.431} & 0.279 & -0.507 \\

& \textbf{DPO-HPS} & \textbf{0.230} & \textbf{0.431} & \textbf{1.645} & \textbf{1.037} \\
\midrule
\multirow{2}{*}{\textbf{100}} & \textbf{DPO-BT} & 0.230 & \textbf{0.431} & 0.349 & -0.455 \\

& \textbf{DPO-HPS} & \textbf{{0.232}} & 0.430 & \textbf{{2.723}} & \textbf{{2.040}} \\
\bottomrule
\end{tabular}}
\end{center}
\vspace{-1em}
\vspace{-9pt}
\end{table}


\begin{table}[t]


\vskip 0.15in
\begin{center}
\begin{small}
\begin{sc}
\begin{tabular}{lrrr}
  \toprule
  Source / Target & o1 & o1-mini & DeepSeek-R1 \\
  \midrule
  o1        & 18.1$\times$ & $6.2\times$ & $12.0\times$  \\
  o1-mini   & $2.9\times$  & $16.8\times$ & $7.5\times$ \\
  DeepSeek-R1  & $11.4\times$  & $4.4\times$  & $10.5\times$ \\
  \bottomrule
\end{tabular}
\end{sc}
\end{small}
\end{center}
\vskip -0.1in
\caption{Transferability matrix between models: o1, o1-mini, and DeepSeek-R1 (Attack Type: Manual Injection).}
\label{tab:transferability}
\end{table}



\begin{table}[t]


\vskip 0.15in
\begin{center}
\begin{small}
\begin{sc}
\begin{tabular}{lrrr}
  \toprule
  Effort & No Attack & Attack & Increase \\
  \midrule
  low     & 345$\pm$248 & 4940$\pm$686 & 14.3$\times$ \\
  medium$^*$ & 751$\pm$410 & 7313$\pm$347 & 9.7$\times$ \\
  high & 806$\pm$354 & 10176$\pm$1003 & 12.6$\times$ \\
  \bottomrule
\end{tabular}
\end{sc}
\end{small}
\end{center}
\vskip -0.1in
\caption{Average number of reasoning tokens for different reasoning effort levels in the o1 model (\textbf{Attack Type}: Context-Agnostic). $^*$Medium is a default effort for all other experiments.}
\label{tab:effort_comparison}
\end{table}



\begin{table*}[tpb]
\begin{center}
\begin{small}
\begin{sc}
\begin{tabular}{llrrrr}
\toprule
& \multirow{3}{*}{Metrics} & \multicolumn{2}{c}{o1} & \multicolumn{2}{c}{DeepSeek-R1} \\ 
\cmidrule(lr){3-4} \cmidrule(lr){5-6}
& & No Attack & Attack & No Attack & Attack \\ \midrule
\multirow{5}{*}{SQuAD} & Input Tokens& $155\pm37$ & $493\pm37$ & 149$\pm$37 & 489$\pm$39 \\
& Output Tokens& $32\pm8.4$ & $44\pm15$ & 63.41$\pm$21.1 & 41$\pm$11 \\
& Reasoning Tokens & $162\pm95$ & \textbf{7435}$\pm$\textbf{847(46}$\times$\textbf{)} & $222\pm116$ & \textbf{4452}$\pm$\textbf{1487(20}$\times$\textbf{)} \\ 
& Accuracy & 100\% & 100\% & 100\% & 100\% \\ 
&Contextual Correctness  & 100\% & 100\% & 100\% & 98\% \\ \cmidrule{1-6}
\multirow{5}{*}{FreshQA}& Input & $7265\pm6724$ & $7603\pm6725$ & 7344$\pm$6774 & 7684$\pm$6774 \\
& Output & $73\pm10$ & $68\pm26$ & 7684$\pm$6774 & 61$\pm$22 \\ 
& Reasoning Tokens & $565\pm558$ & \textbf{7146}$\pm$\textbf{984(13}$\times$\textbf{)} & $546\pm664$ & \textbf{3187}$\pm$\textbf{2011(6}$\times$\textbf{)} \\ 
& Accuracy & 91\% & 95\% & 89\% & 87\% \\ 
& Contextual Correctness & 100\% & 100\% & 100\% & 98.5\% \\ 
\bottomrule
\end{tabular}
\end{sc}
\end{small}
\end{center}
\caption{Performance of Context-Agnostic attack on o1 and DeepSeek-R1 models on 100 samples from SQuAD and FreshQA.}
\label{tab:Context_Agnostic_100samples_big_models}
\end{table*}

\begin{table*}[!ht]
\begin{center}
\begin{small}

\begin{sc}
\begin{tabular}{llrrrr}
\toprule
& \multirow{3}{*}{Metrics} & \multicolumn{2}{c}{o1-mini} & \multicolumn{2}{c}{o3-mini} \\ 
\cmidrule(lr){3-4} \cmidrule(lr){5-6}
& & No Attack & Attack & No Attack & Attack \\ \midrule
\multirow{5}{*}{SQuAD} & Input Tokens &$156\pm37$ &$397\pm37$ & 155$\pm$37 & 493$\pm$36 \\ 
& Output Tokens & $53\pm31$&$29\pm10$ & $31.12\pm11.3$ & $39.45\pm17.0$ \\ 
& Reasoning Tokens & $392\pm180$ & \textbf{3306}$\pm$\textbf{2791(8}$\times$\textbf{)} & $139\pm100$ & \textbf{4902}$\pm$\textbf{745(35}$\times$\textbf{)} \\ 
& Accuracy & 99\% & 98\% & 100\% & 100\% \\ 
&Contextual Correctness  & 100\% & 100\% & 100\% & 100\% \\ \cmidrule{1-6}
\multirow{5}{*}{FreshQA} & Input Tokens & $7399\pm6826$& $7639\pm6826$& 7270$\pm$6695 & 7608$\pm$6695 \\ 
& Output Tokens &$191\pm135$ &$49\pm31$ & 68$\pm$42 & 50$\pm$29 \\ 
& Reasoning Tokens & $456\pm269$ & \textbf{3136}$\pm$\textbf{3077(7}$\times$\textbf{)} & $559\pm446$ & \textbf{2182}$\pm$\textbf{776(4}$\times$\textbf{)} \\ 
& Accuracy & 91\% & 87\% & 88\% & 84\% \\ 
& Contextual Correctness  & 100\% & 100\% & 100\% & 100\% \\ 
\bottomrule
\end{tabular}
\end{sc}
\end{small}
\end{center}
\caption{Performance of Context-Agnostic attack on o1-mini and o3-mini on 100 samples from SQuAD and FreshQA.}
\label{tab:Context_Agnostic_100samples_mini_models}
\end{table*}







\subsection{Attack Setup}
To orchestrate the attack, we first retrieve context related to the question either directly from the dataset or using the links present in the dataset. We then inject manually written attack templates (discussed in sections \ref{subsec:manual_injection} and \ref{subsec:weaving_injection}) in the retrieved context and compare the model's responses to both the original and compromised contexts for evaluation. 
We select the best performing decoy problems from Table~\ref{tab:dataset_comparison} i.e Sudoku and MDP. For example of injection templates, refer to Figure~\ref{fig:context_agnostic_prompt_sudoku} and Figure~\ref{fig:context_agnostic_prompt} in Appendix~\ref{appendix: used_prompts}.
Finally, we utilize decoy-optimized context generated using Algorithm \ref{alg:ICL-genetic} to produce output and evaluate ICL-Genetic based attacks.

\subsection{Experimental Results}
We demonstrate the main experimental results of our \sys attack against o1 and DeepSeek-R1 models, demonstrating that all attack types significantly amplify the models' reasoning complexity. For the o1 model, Table~\ref{tab:o1_main_tab} shows that the baseline processing involves $751\pm410$ reasoning tokens. The ICL-Genetic (Agnostic) attack causes the largest increase -- an $18\times$ rise. Context-Agnostic and Context-Aware attacks also increase the token count significantly, by $9.7\times$ and over $2\times$, respectively. 

Similarly, Table~\ref{tab:deepseek_main_tab} shows that all attack types severely raise the number of reasoning tokens in the DeepSeek-R1 model. The baseline of $711\pm635$ tokens increase more than $10\times$ under the ICL-Genetic (Agnostic) attack. Other attacks, such as Context-Agnostic, Context-Aware, and ICL-Genetic (Aware), also lead to substantial increases in reasoning complexity. Overall, our results demonstrate that ICL-based attacks, especially those involving Context-Agnostic, severely disrupt reasoning efficiency for both models by drastically increasing reasoning token counts. This trend persists across all attack types. Similarly Tables~\ref{tab:Context_Agnostic_100samples_big_models} and~\ref{tab:Context_Agnostic_100samples_mini_models} show an increase in reasoning tokens across all four models tested on both the SQuAD and FreshQA datasets using the context-agnostic attack. We observe a $46\times$ increase in reasoning tokens for the SQuAD dataset on the o1 model. This highlights the effectiveness of our attack methodology across a diverse set of contexts and model families. Figure~\ref{fig:reasoning_content_example} in the Appendix gives an insight of how the decoy task causes increase in  reasoning steps for R1 model.

\pbe{ICL Ablation.}
Table~\ref{tab:o1_main_tab} and~\ref{tab:deepseek_main_tab} show that the results based on ICL outperform both context-agnostic and context-aware settings. In Table~\ref{tab:ablation_table}, we present an ablation study on ICL-Genetic with context-agnostic attack framework to evaluate the efficacy of each individual components and its contributions on crafting the final attack. It shows that while both ICL-Genetic and context-agnostic attacks independently have higher reasoning token count than baseline, both of them are lower than the attack  conducted by combining both techniques. We hypothesize that this occurs because the attack-agnostic samples used to generate the initial population allow the algorithm to narrow down the search space, thereby enabling it to take a more exploitative route in finding an effective injection.


\begin{table*}[ht!]
\centering  
\footnotesize
\begin{tcolorbox}[enhanced,breakable,
    colframe=gray!50!white,
    colbacktitle=white,
    coltitle=black,
    colback=white,
    borderline={0.5mm}{0mm}{gray!15!white},
    borderline={0.5mm}{0mm}{gray!50!white,dashed},
    attach boxed title to top center={yshift=-2mm},
    boxed title style={boxrule=0.8pt},
    title=\normalsize\textbf{Prompt for Filtering Correct Responses}]
    \renewcommand{\arraystretch}{1.3}
    \begin{tabular}{p{.95\linewidth}}
        \textbf{System}\\
        You are an assistant specializing in chemistry and biology. You are provided with a molecule's IUPAC name and its \{level\} description.\\
        Your task is to evaluate the factual accuracy of the given description based on the provided IUPAC name. \\
        Assign a score from 1 to 4 based on the following criteria: \\
        1: All contents are factually incorrect \\
        2: Some contents are factually correct, but most are factually incorrect \\
        3: Most contents are factually correct, but some are factually incorrect \\
        4: All contents are factually correct \\
        Indicate your score in the format: ``Score: ...''.\\
        \midrule
        \textbf{User}\\
        Input molecule (IUPAC name): \{IUPAC name\} \\
        Description: \{Description\}
    \end{tabular}
\end{tcolorbox}
\caption{Prompts for filtering correct samples. \{level\} is one of the following: `structural', `chemical', and `biological'.}
\label{app:tab:prompts_filtering}
\end{table*}




\begin{figure*}[ht!]
    \centering
    \begin{tcolorbox}[
        enhanced,                  %
        colframe=blue!70!black,    %
        colback=blue!5,            %
        coltitle=white,            %
        colbacktitle=blue!70!black, %
        width=\textwidth,          %
        arc=4mm,                   %
        boxrule=1mm,               %
        drop shadow,               %
        title=Copy-Editing Task Description, %
        fonttitle=\bfseries\large  %
    ]

    \textbf{Task Description:}\\%[1em]

    As an expert copy-editor, please rewrite the following text in your own voice while ensuring that the final output contains the same information as the original text and has roughly the same length. Please paraphrase all sentences and do not omit any crucial details.\\[1em]

    \textbf{Input Text:}\\[1em]
    \texttt{<Input Text Placeholder>}

    \end{tcolorbox}
    \caption{Prompt used by the RAG system to rewrite the input query.}
    \label{fig:copy_editing_task}
\end{figure*}


\pbe{Attack Transferability.}
We evaluate the transferability of \sys across o1, o1-mini, and DeepSeek-R1 models under the Context-Agnostic attack. Contexts optimized using the ICL-Genetic attack on a source model are applied to target models to assess transferability. The o1 model demonstrates strong transferability, achieving a 12$\times$ increase on DeepSeek-R1, exceeding the 10.5$\times$ increase from context optimized directly on DeepSeek-R1. Similarly, o1's transfer to o1-mini results in a 6.2$\times$ increase. DeepSeek-R1 also transfers effectively to o1 with an 11.4$\times$ increase but less so to o1-mini (4.4$\times$). In contrast, o1-mini shows moderate transferability with a 7.5$\times$ increase on DeepSeek-R1 and only 2.9$\times$ on o1. These findings demonstrate that context optimized from various source models can significantly increase reasoning tokens across different target models. 




\pbe{Reasoning Effort Tuning.} 
The o1 model API provides \textit{reasoning effort} hyperparameter that controls the size of thought in generating responses, with low effort yielding quick, simple answers and high effort producing more detailed explanations~\cite{openai_reasoning_effort, openai_reasoning_guide}. We use this parameter to evaluate our attack across different effort levels. Table~\ref{tab:effort_comparison} shows that the Context-Agnostic attack significantly increases reasoning tokens at all effort levels. For high effort, the token count rises over 12$\times$. Medium and low effort also show large increases, reaching up to 14$\times$. These results demonstrate that the attack disrupts the model's reasoning efficiency across tasks of varying complexity, with even low-effort tasks experiencing significant reasoning overhead.














 

    





\section{Conclusion and future directions} \label{sec:conclusion}

In this paper we proposed a nested MLMC framework that offers important computational savings by performing most calculations in low precision and exploiting approximate random normal variables for the low precision path calculations. The low precision calculations could be performed in fixed precision on an FPGA for greater efficiency, and we suggested a procedure to optimise the bit-widths of every variable at each Monte Carlo level. This is an important improvement over previous mixed precision MLMC frameworks which held the lower precision fixed \cite{Rounding_error_oliver} or defined uniform bit-width at every level heuristically \cite{brugger2014mixed}. Our numerical results suggest that for the first levels our procedure reduces the cost at these levels by a factor 5 or 7. Hence the overall savings are significant since most paths are calculated on the first levels. Our approach would be even more efficient for the Milstein scheme because its higher order strong convergence leads to a greater proportion of the computational costs being on the coarsest levels.

The next stage of the research project will be to implement the RNG methods and the nested framework on FPGAs to determine the hardware requirements and confirm the extent of the computational savings. It would also be good to compare the performance benefits to using half-precision floating point arithmetic on GPUs or CPUs for the low-accuracy computations.




\ifdefined \isarxiv
\else
\input{9999_impact}
\section*{Acknowledgments}
\bibliography{ref}
%\bibliographystyle{icml2022}
\bibliographystyle{icml2025}
\fi



\newpage
\onecolumn
\appendix


\begin{center}
	\textbf{\LARGE Appendix }
\end{center}


In the appendix, we present more experimental text-to-video generation results in Appendix~\ref{sec:app:more_1} and more interpolation and extrapolation results in Appendix~\ref{sec:app:more_2}. Then we introduce the preliminary in Appendix~\ref{sec:app:preli}. Next, we illustrate Video Latent Flow Matching formally in Appendix~\ref{sec:app:vlfm}. In Appendix~\ref{sec:app:dit}, we demonstrate the Diffusion Transformer, and finally, in Appendix~\ref{sec:app:inter_extra}, we present the interpolation and extrapolation of VLFM.

\section{More Text-to-Video Generation Results} \label{sec:app:more_1}

We give more text-to-video generation results with different frame rates to demonstrate the generative ability of our VLFM in Figure~\ref{fig:gen_addtional1} and Figure~\ref{fig:gen_addtional2}.

\begin{figure*}[!ht]
\begin{center}
\centering
    \subfloat[{\it Video caption: Venus spinning in the space.}]{
    \includegraphics[width=0.95\textwidth]{gen/venus}} \\
    \subfloat[{\it Video caption: Steam is coming out of a pot.}]{
    \includegraphics[width=0.95\textwidth]{gen/pot}} 
\end{center}
\caption{Generated videos with different frame rates $\{8, 12, 16\}$. }
\label{fig:gen_addtional1}
\end{figure*}

\begin{figure*}[!ht]
\begin{center}
\centering
    \subfloat[{\it Video caption: Flame flickers on the candles.}]{
    \includegraphics[width=0.95\textwidth]{gen/candle}} \\
    \subfloat[{\it Video caption: A train is running through the rail road near the coast.}]{
    \includegraphics[width=0.95\textwidth]{gen/train}}
\end{center}
\caption{Generated videos with different frame rates $\{8, 12, 16\}$. }
\label{fig:gen_addtional2}
\end{figure*}

\section{More Interpolation and Extrapolation Results} \label{sec:app:more_2}

We give more results of interpolation and extrapolation of VLFM in Figure~\ref{fig:more_inter_extra}.

\begin{figure*}[!ht]
\begin{center}
\centering
    \subfloat{
    \includegraphics[width=0.95\textwidth]{inter/turbine}} \\
    \subfloat{
    \includegraphics[width=0.95\textwidth]{extra/wave}} \\
    \subfloat{
    \includegraphics[width=0.95\textwidth]{extra/starry}}
\end{center}
\caption{Interpolation and Extrapolation of VLFM.}
\label{fig:more_inter_extra}
\end{figure*}

\section{Preliminary} \label{sec:app:preli}

In the preliminary section, we first introduce our notation in the appendix in Appendix~\ref{sub:app:notations}.  Then, in Appendix~\ref{sub:app:video}, we formally define the video-caption data and visual decoder. In Appendix~\ref{sub:app:latent_patches}, we define the latent patches. Appendix~\ref{sub:app:assumption} makes some assumptions which we will use later. Finally, in Appendix~\ref{sec:app:facts}, we list some basic useful facts.

\subsection{Notations} \label{sub:app:notations}

\paragraph{Notations.} We use $D$ to denote the flattened dimension of real-world images. We use $d$ to represent the dimension of latent patches. We introduce $d_0$ as the dimension of Diffusion Transformers. We utilize $V: [0, T] \rightarrow \R^D$ to denote a video with $T$ duration, where $T$ is the longest time for each video. We omit $\nabla_t a(t)$ and $a'(t)$ to denote taking differentiation to some function $a(t)$ w.r.t. time $t$. We use integer $s$ to denote the order of polynomials. The dimensional number of the text embedding vector is given by integer $\ell$.

\subsection{Video-Caption Data} \label{sub:app:video}

\begin{definition}[Video-caption data pairs and their distribution]\label{def:V_c}
    We define a video caption distribution $(V, c) \sim {\cal V}_c$. Here, $V: [0, T] \rightarrow \R^D$ is considered as a function and $c \in \R^\ell$ is the corresponding text embedding vector.
\end{definition}

\begin{definition}\label{def:wt_V}
    Given a video caption distribution ${\cal V}_c$ as Definition~\ref{def:V_c}. We denote $\Delta t$ as the minimal time unit of measurement in the real world (Planck time). For any $(V, c) \sim {\cal V}_c$, we define the discretized form of $V: [0, T] \rightarrow \R^D$, which is $\wt{V} \in \R^{\frac{T}{\Delta t} \times D}$, and its $\tau$-th row $ \forall \tau \in [\frac{T}{\Delta t}]$ is given by:
    \begin{align*}
        \wt{V}_\tau := V_{\Delta t \cdot \tau} \in \R^D.
    \end{align*}
\end{definition}

\begin{definition}[Obtained data in real-world cases]\label{def:Phi}
    If the following conditions hold:
    \begin{itemize}
        \item Given a video caption distribution ${\cal V}_c$ as Definition~\ref{def:V_c}.
        \item For any $(V, c) \sim {\cal V}_c$, we define the discretized form of video $\wt{V}$ as Definition~\ref{def:wt_V}.
    \end{itemize}
    We define an observation matrix $\Phi: \{0, 1\}^{N \times \frac{T}{\Delta t}}$. The obtained data in real-world cases then is denoted as $\Phi \wt{V} \in \R^{N \times D}$.
\end{definition}

\begin{definition}[Bijective Visual Decoder]\label{def:visual_decoder}
    We define the visual decoder ${\cal D}: \R^d \rightarrow \R^D$ satisfies that:
    \begin{itemize}
        \item For any flattened image $V \in \R^D$, there is a unique $u \in \R^d$ such that ${\cal D}(u) = V$.
    \end{itemize}
    Then we say ${\cal D}$ is bijective. Denote the reverse function of ${\cal D}$ as ${\cal D}^{-1}: \R^D \rightarrow \R^d$.
\end{definition}

\subsection{Latent Patches Data} \label{sub:app:latent_patches}

\begin{definition}\label{def:u}
    If the following conditions hold:
    \begin{itemize}
        \item Given a video caption distribution ${\cal V}_c$ as Definition~\ref{def:V_c}.
        \item For any $(V, c) \sim {\cal V}_c$, we define the discretized form of video $\wt{V}$ as Definition~\ref{def:wt_V}.
        \item Let the observation matrix $\Phi: \{0, 1\}^{N \times \frac{T}{\Delta t}}$ be defined as Definition~\ref{def:Phi}.
        \item Let the visual decoder function $D: \R^d \rightarrow \R^D$ be defined as Definition~\ref{def:visual_decoder}.
    \end{itemize}
    We define the ideal version (without observation matrix) of the sequence of latent patches $u \in \R^{\frac{T}{\Delta t} \times d}$, and its $\tau$-th $ \forall \tau \in [\frac{T}{\Delta t}]$ row is defined as follows:
    \begin{align*}
        u_\tau := {\cal D}^{-1}( \wt{V}_\tau ).
    \end{align*}
\end{definition}

\begin{definition}\label{def:wt_u}
    If the following conditions hold:
    \begin{itemize}
        \item Given a video caption distribution ${\cal V}_c$ as Definition~\ref{def:V_c}.
        \item For any $(V, c) \sim {\cal V}_c$, we define the discretized form of video as Definition~\ref{def:wt_V}.
        \item Let the observation matrix $\Phi: \{0, 1\}^{N \times \frac{T}{\Delta t}}$ be defined as Definition~\ref{def:Phi}.
        \item Let the visual decoder function $D: \R^d \rightarrow \R^D$ be defined as Definition~\ref{def:visual_decoder}.
    \end{itemize}
    We define the real-world version (with observation matrix) of the sequence of latent patches $\wt{u} \in \R^{\frac{T}{\Delta t} \times d}$, and its $\tau$-th $ \forall \tau \in [N]$ row is defined as follows:
    \begin{align*}
        \wt{u}_\tau := {\cal D}^{-1}\Big( (\Phi V)_\tau \Big).
    \end{align*}
\end{definition}


\subsection{Assumptions} \label{sub:app:assumption}

\begin{assumption}\label{ass:k}
    If the following conditions hold:
    \begin{itemize}
        \item Given a video caption distribution ${\cal V}_c$ as Definition~\ref{def:V_c}.
        \item For any $(V, c) \sim {\cal V}_c$, we define the discretized form of video as Definition~\ref{def:wt_V}.
        \item Let the observation matrix $\Phi: \{0, 1\}^{N \times \frac{T}{\Delta t}}$ be defined as Definition~\ref{def:Phi}.
        \item Let the visual decoder function $D: \R^d \rightarrow \R^D$ be defined as Definition~\ref{def:visual_decoder}.
        \item Let the ideal version of the sequence of latent patches $u \in \R^{\frac{T}{\Delta t} \times d}$ be defined as Definition~\ref{def:u}.
    \end{itemize}
    We assume $u_\tau$ is $k$-differentiable, there exists:
    \begin{align*}
        u_{\tau}^{(i)} = \lim_{\Delta t \rightarrow 0} \frac{u_{\tau+1}^{(i-1)} - u_{\tau}^{(i-1)}}{ \Delta t }, \forall i \in [k], \tau \in [\frac{T}{\Delta t}],
    \end{align*}
    where, we use $u_\tau^{(i)}$ to denote the $i$-th derivation of $u$.
\end{assumption}

\begin{assumption}\label{ass:L_0}
    If the following conditions hold:
    \begin{itemize}
        \item Let the visual decoder function $D: \R^d \rightarrow \R^D$ be defined as Definition~\ref{def:visual_decoder}.
    \end{itemize}
    We assume the visual decoder function ${\cal D}$ is $L_0$-smooth for constant $L_0 > 0$, such that:
    \begin{align*}
        \| {\cal D}(x) - {\cal D}(y) \|_2 \leq L_0 \| x - y \|_2, \forall x, y \in \R^d.
    \end{align*}
\end{assumption}

\begin{assumption}\label{ass:U}
    If the following conditions hold:
    \begin{itemize}
        \item Given a video caption distribution ${\cal V}_c$ as Definition~\ref{def:V_c}.
        \item For any $(V, c) \sim {\cal V}_c$, we define the discretized form of video as Definition~\ref{def:wt_V}.
        \item Let the observation matrix $\Phi: \{0, 1\}^{N \times \frac{T}{\Delta t}}$ be defined as Definition~\ref{def:Phi}.
        \item Let the visual decoder function $D: \R^d \rightarrow \R^D$ be defined as Definition~\ref{def:visual_decoder}.
        \item Let the ideal version of the sequence of latent patches $u \in \R^{\frac{T}{\Delta t} \times d}$ be defined as Definition~\ref{def:u}.
    \end{itemize}
    We assume each entry in latent patches $u$ is bounded by a constant $U > 0$.
\end{assumption}

\begin{assumption}\label{ass:M}
    If the following conditions hold:
    \begin{itemize}
        \item Given a video caption distribution ${\cal V}_c$ as Definition~\ref{def:V_c}.
        \item For any $(V, c) \sim {\cal V}_c$
    \end{itemize}
    For any $(V, c) \sim {\cal V}_c$, we assume there exists a function ${\cal M}: [0, T] \times \R^\ell \rightarrow \R^D$ satisfies $V_t = {\cal M}_t(c)$. 
\end{assumption}

\subsection{Basic Facts} \label{sec:app:facts}

\begin{fact}\label{fac:gaussian_tail}
    For a variable $x \sim \mathcal{N}(0, \sigma^2)$, then with probability at least $1 - \delta$, we have:
    \begin{align*}
        |x| \leq C \sigma \sqrt{\log(1/\delta)}
    \end{align*}
\end{fact}

\begin{fact}\label{fac:infity_norm_pesdueo_inverse}
    For a PD matrix $A \in \R^{d_1 \times d_2}$ with a positive minimum eigenvalue $\lambda_{\min}(A) > 0$, the infinite norm of its pesdueo-inverse matrix $A^\dag$ is given by:
    \begin{align*}
        \| A^\dagger \|_\infty \leq \frac{1}{\lambda_{\min}(A)}.
    \end{align*}
\end{fact}

\begin{fact}\label{fac:pesdueo_inverse_diff}
    For two matrices $A , B \in\R^{d_1 \times d_2}$, we have:
    \begin{align*}
        \| A^\dagger - B^\dagger \|_2 \leq \frac{\| A^\dagger \|_2^2 \| A - B\|_2 }{1 - \| A^\dagger \|_2 \cdot 
        \| A - B \|_2}
    \end{align*}
\end{fact}

\section{Video Latent Flow Matching}
\label{sec:app:vlfm}

This section, we first introduce the HiPPO Framework and LegS in Appendix~\ref{sub:app:hippo}. Then, we formally define the video latent flow in Appendix~\ref{sub:app:vlf}. Last, we introduce the training objective of VLFM in Appendix~\ref{sub:app:train_obj}.

\subsection{HiPPO Framework and LegS} \label{sub:app:hippo}

\begin{definition}\label{def:A}
    We define matrix $A \in \R^{s \times s}$ where its $(i_1, i_2)$-th entry $\forall i_1, i_2 \in [s]$ is given by:
    \begin{align*}
        A_{i_1, i_2} & ~ = \begin{cases}
        \sqrt{(2i_1 + 1)(2i_2 + 1)}, & \text{if $i_1 > i_2$} \\
        i_1 + 1, & \text{if $i_1 = i_2$} \\
        0, & \text{if $i_1 < i_2$}
    \end{cases}.
    \end{align*}
\end{definition}

\begin{definition}\label{def:B}
    We define matrix $B \in \R^{s \times 1}$ where its $i_1$-th entry $\forall i_1 \in [s]$ is given by:
    \begin{align*}
        B_{i_1} & ~ = \sqrt{2i_1 + 1}.
    \end{align*}
\end{definition}

\begin{definition}\label{def:H}
    If the following conditions hold:
    \begin{itemize}
        \item Let matrix $A \in \R^{s \times s}$ be defined as Definition~\ref{def:A}.
        \item Let matrix $B \in \R^{s \times 1}$ be defined as Definition~\ref{def:B}.
    \end{itemize}
    We initialize a matrix $H_0 = {\bf 0}_{d \times s}$. Then we define:
    \begin{align*}
        H_{\tau} := H_{\tau-1}( I_s - \frac{1}{\tau} A )^\top + \frac{1}{\tau} \wt{u}_\tau B^\top, \forall \tau \in [N].
    \end{align*}
\end{definition}

\begin{definition}\label{def:g_t}
    We define $g(t) := [\sqrt{\frac{1}{2}} P_0(t), \sqrt{\frac{3}{2}} P_1(t), \cdots, \sqrt{\frac{2s-1}{2}} P_{s-1}(t)]^\top $ $\in \R^{s} $, where$ P_i(t), \forall i \in [s]$ is some polynomials. Especially, $g(t)$ satisfies:
    \begin{itemize}
        \item Define $G := \begin{bmatrix}
            g(\Delta t)^\top \\
            g(2 \Delta t)^\top \\
            \vdots \\
            g(T)^\top
        \end{bmatrix}$, $\lambda_{\min} (G) > 0$. Here, $\lambda_{\min}$ is the function that outputs the minimal eigenvalue of the input matrix.
        \item $|G_{\tau, i}| \leq \exp(O(\frac{T}{\Delta t}s))$ for any $\tau \in [\frac{T}{ \Delta t}], i \in [s]$.
    \end{itemize}
\end{definition}

\subsection{Video Latent Flow} \label{sub:app:vlf}

\begin{definition}\label{def:mu}
    If the following conditions hold:
    \begin{itemize}
        \item Given a video caption distribution ${\cal V}_c$ as Definition~\ref{def:V_c}.
        \item For any $(V, c) \sim {\cal V}_c$, we define the discretized form of video as Definition~\ref{def:wt_V}.
        \item Let the observation matrix $\Phi: \{0, 1\}^{N \times \frac{T}{\Delta t}}$ be defined as Definition~\ref{def:Phi}.
        \item Let the visual decoder function $D: \R^d \rightarrow \R^D$ be defined as Definition~\ref{def:visual_decoder}.
        \item Let the ideal version of the sequence of latent patches $u \in \R^{\frac{T}{\Delta t} \times d}$ be defined as Definition~\ref{def:u}.
        \item Let the real-world version of the sequence of latent patches $\wt{u} \in \R^{N \times d}$ be defined as Definition~\ref{def:wt_u}.
        \item Let $H_N \in \R^{d \times s}$ be defined as Definition~\ref{def:H}.
        \item Let the function of polynomials $g(t)$ be defined as Definition~\ref{def:g_t}.
    \end{itemize}
    We define the time-dependent mean of Gaussian distribution as follows:
    \begin{align*}
        \mu_t(\wt{u}) := H_N g(t) \in \R^d
    \end{align*}
\end{definition}

\begin{definition}\label{def:sigma}
    If the following conditions hold:
    \begin{itemize}
        \item Given a video caption distribution ${\cal V}_c$ as Definition~\ref{def:V_c}.
        \item For any $(V, c) \sim {\cal V}_c$, we define the discretized form of video as Definition~\ref{def:wt_V}.
        \item Let the observation matrix $\Phi: \{0, 1\}^{N \times \frac{T}{\Delta t}}$ be defined as Definition~\ref{def:Phi}.
        \item Let the visual decoder function $D: \R^d \rightarrow \R^D$ be defined as Definition~\ref{def:visual_decoder}.
        \item Let the ideal version of the sequence of latent patches $u \in \R^{\frac{T}{\Delta t} \times d}$ be defined as Definition~\ref{def:u}.
        \item Let the real-world version of the sequence of latent patches $\wt{u} \in \R^{N \times d}$ be defined as Definition~\ref{def:wt_u}.
        \item Let $H_N \in \R^{d \times s}$ be defined as Definition~\ref{def:H}.
        \item Let the function of polynomials $g(t)$ be defined as Definition~\ref{def:g_t}.
        \item Denote $\sigma_{\min} > 0$.
        \item Given a hyper-parameter $\alpha > 0$.
    \end{itemize}
    We define  the time-dependent standard deviation as follows:
    \begin{align*}
        \sigma_t(\wt{u}) := (1 - \sigma_{\min}) \cdot [\sin^2( \pi \frac{N}{T} t ) +  \exp(-\alpha t) ] + \sigma_{\min} \in \R_{\ge 0}.
    \end{align*}
\end{definition}

\begin{lemma}\label{lem:bound_diff_sigma}
    If the following conditions hold:
    \begin{itemize}
        \item Given a video caption distribution ${\cal V}_c$ as Definition~\ref{def:V_c}.
        \item For any $(V, c) \sim {\cal V}_c$, we define the discretized form of video as Definition~\ref{def:wt_V}.
        \item Let the observation matrix $\Phi: \{0, 1\}^{N \times \frac{T}{\Delta t}}$ be defined as Definition~\ref{def:Phi}.
        \item Let the visual decoder function $D: \R^d \rightarrow \R^D$ be defined as Definition~\ref{def:visual_decoder}.
        \item Let the ideal version of the sequence of latent patches $u \in \R^{\frac{T}{\Delta t} \times d}$ be defined as Definition~\ref{def:u}.
        \item Let the real-world version of the sequence of latent patches $\wt{u} \in \R^{N \times d}$ be defined as Definition~\ref{def:wt_u}.
        \item Let $H_N \in \R^{d \times s}$ be defined as Definition~\ref{def:H}.
        \item Let the function of polynomials $g(t)$ be defined as Definition~\ref{def:g_t}.
        \item Let the time-dependent mean of Gaussian distribution $\mu_t(\wt{u})$ be defined as Definition~\ref{def:mu}.
        \item Let the time-dependent standard deviation $\sigma_t(\wt{u})$ be defined as Definition~\ref{def:sigma}.
        \item Denote $\sigma_{\min} > 0$.
        \item Given a hyper-parameter $\alpha > 0$.
    \end{itemize}
    Then for any $\alpha >0$, we have:
    \begin{align*}
        | \frac{\sigma_t'(\wt{u})}{\sigma_t(\wt{u})} | \leq \frac{1 - \sigma_{\min}}{\sigma_{\min}}.
    \end{align*}
\end{lemma}

\begin{proof}
    This result can be obtained following very simple algebras.
\end{proof}

\begin{definition}\label{def:psi}
    If the following conditions hold:
    \begin{itemize}
        \item Given a video caption distribution ${\cal V}_c$ as Definition~\ref{def:V_c}.
        \item For any $(V, c) \sim {\cal V}_c$, we define the discretized form of video as Definition~\ref{def:wt_V}.
        \item Let the observation matrix $\Phi: \{0, 1\}^{N \times \frac{T}{\Delta t}}$ be defined as Definition~\ref{def:Phi}.
        \item Let the visual decoder function $D: \R^d \rightarrow \R^D$ be defined as Definition~\ref{def:visual_decoder}.
        \item Let the ideal version of the sequence of latent patches $u \in \R^{\frac{T}{\Delta t} \times d}$ be defined as Definition~\ref{def:u}.
        \item Let the real-world version of the sequence of latent patches $\wt{u} \in \R^{N \times d}$ be defined as Definition~\ref{def:wt_u}.
        \item Let $H_N \in \R^{d \times s}$ be defined as Definition~\ref{def:H}.
        \item Let the function of polynomials $g(t)$ be defined as Definition~\ref{def:g_t}.
        \item Let the time-dependent mean of Gaussian distribution $\mu_t(\wt{u})$ be defined as Definition~\ref{def:mu}.
        \item Let the time-dependent standard deviation $\sigma_t(\wt{u})$ be defined as Definition~\ref{def:sigma}.
        \item Denote $\sigma_{\min} > 0$.
        \item Sample $z \sim \mathcal{N}(0, I_d)$.
    \end{itemize}
    We define the Video Latent Flow:
    \begin{align*}
        \psi_t(\wt{u}) := \sigma_t(\wt{u}) \cdot z + \mu_t(\wt{u}) \in \R^d.
    \end{align*}
\end{definition}

\subsection{Training Objective} \label{sub:app:train_obj}

\begin{definition}\label{def:L}
    If the following conditions hold:
    \begin{itemize}
        \item Given a video caption distribution ${\cal V}_c$ as Definition~\ref{def:V_c}.
        \item For any $(V, c) \sim {\cal V}_c$, we define the discretized form of video as Definition~\ref{def:wt_V}.
        \item Let the observation matrix $\Phi: \{0, 1\}^{N \times \frac{T}{\Delta t}}$ be defined as Definition~\ref{def:Phi}.
        \item Let the visual decoder function $D: \R^d \rightarrow \R^D$ be defined as Definition~\ref{def:visual_decoder}.
        \item Let the ideal version of the sequence of latent patches $u \in \R^{\frac{T}{\Delta t} \times d}$ be defined as Definition~\ref{def:u}.
        \item Let the real-world version of the sequence of latent patches $\wt{u} \in \R^{N \times d}$ be defined as Definition~\ref{def:wt_u}.
        \item Let $H_N \in \R^{d \times s}$ be defined as Definition~\ref{def:H}.
        \item Let the function of polynomials $g(t)$ be defined as Definition~\ref{def:g_t}.
        \item Let the time-dependent mean of Gaussian distribution $\mu_t(\wt{u})$ be defined as Definition~\ref{def:mu}.
        \item Let the time-dependent standard deviation $\sigma_t(\wt{u})$ be defined as Definition~\ref{def:sigma}.
        \item Denote $\sigma_{\min} > 0$.
        \item Sample $z \sim \mathcal{N}(0, I_d)$.
        \item Define a model function $F_\theta: \R^d \times \R^\ell \times [0, T] \rightarrow \R^d$ with parameters $\theta$.
    \end{itemize}
    We define the training objective of Video Latent Flow Matching as follows:
    \begin{align*}
        {\cal L}(\theta) := \E_{z \sim \mathcal{N}(0, I_d), t \sim {\sf Uniform}[0, T], (V, c) \sim {\cal V}_c}[\| F_\theta( \psi_t(\wt{u}), c, t ) - \frac{\d }{\d t} \psi_t(\wt{u}) \|_2^2].
    \end{align*}
\end{definition}

\begin{theorem}\label{thm:close_form}
    If the following conditions hold:
    \begin{itemize}
        \item Given a video caption distribution ${\cal V}_c$ as Definition~\ref{def:V_c}.
        \item For any $(V, c) \sim {\cal V}_c$, we define the discretized form of video as Definition~\ref{def:wt_V}.
        \item Let the observation matrix $\Phi: \{0, 1\}^{N \times \frac{T}{\Delta t}}$ be defined as Definition~\ref{def:Phi}.
        \item Let the visual decoder function $D: \R^d \rightarrow \R^D$ be defined as Definition~\ref{def:visual_decoder}.
        \item Let the ideal version of the sequence of latent patches $u \in \R^{\frac{T}{\Delta t} \times d}$ be defined as Definition~\ref{def:u}.
        \item Let the real-world version of the sequence of latent patches $\wt{u} \in \R^{N \times d}$ be defined as Definition~\ref{def:wt_u}.
        \item Let $H_N \in \R^{d \times s}$ be defined as Definition~\ref{def:H}.
        \item Let the function of polynomials $g(t)$ be defined as Definition~\ref{def:g_t}.
        \item Let the time-dependent mean of Gaussian distribution $\mu_t(\wt{u})$ be defined as Definition~\ref{def:mu}.
        \item Let the time-dependent standard deviation $\sigma_t(\wt{u})$ be defined as Definition~\ref{def:sigma}.
        \item Denote $\sigma_{\min} > 0$.
        \item Sample $z \sim \mathcal{N}(0, I_d)$.
        \item Define a model function $F_\theta: \R^d \times \R^\ell \times [0, T] \rightarrow \R^d$ with parameters $\theta$.
        \item Let the training objective ${\cal L}(\theta)$ be defined as Definition~\ref{def:L}.
    \end{itemize}
    Then the minimum solution for function $F_\theta$ that takes $z \sim N(0, I_d)$ and $t \sim {\sf Uniform}[0, T]$ is:
    \begin{align*}
        F_\theta(z, c, t) = \frac{\sigma_t'(\wt{u})}{\sigma_t(\wt{u})} (z - \mu_t(\wt{u})) + \mu_t'(\wt{u}).
    \end{align*}
\end{theorem}

\begin{proof}
    This proof follows from Theorem 3 in \cite{lcb+22}.
\end{proof}

\section{Diffusion Transformer} \label{sec:app:dit}

In this section, we first define the Diffusion Transformer in Appendix~\ref{sub:app:def}. Moreover, we introduce the Approximation via DiT in Appendix~\ref{sub:app:approx_dit}.

\subsection{Definitions} \label{sub:app:def}

\begin{definition}[Multi-head self-attention]\label{def:attn}
    Given $h$-heads query, key, value and output projection weights $\{(W_Q^i, W_K^i, W_V^i, W_O^i)\}_{i=1}^h \subset \R^{d_0 \times 4m}$ with each weight is a $d_0 \times m$ shape matrix, for an input matrix $X \in \R^{n \times d_0}$, we define a multi-head self-attention computation as follows:
    \begin{align*}
        {\sf Attn}(X) := \sum_{i=1}^h {\sf Softmax}( X W_Q^i {W_K^i}^\top X^\top ) \cdot X W_V^i {W_O^i}^\top + X \in \R^{n \times d_0}.
    \end{align*}
\end{definition}

\begin{definition}[Feed-forward]\label{def:feed_forward}
    Given two projection weights $W_1, W_2 \in \R^{d_0 \times r}$ and two bias vectors $b_1 \in \R^r$ and $b_2 \in \R^{d_0}$, for an input matrix $X \in \R^{n \times d_0}$, we define a feed-forward computation as follows:
    \begin{align*}
        {\sf FF}(X) := \phi(X W_1 + {\bf 1}_n b_1^\top) \cdot W_2^\top + {\bf 1}_n b_2^\top + X \in \R^{n \times d_0}.
    \end{align*}
    Here, $\phi$ is an activation function and usually be considered as ReLU.
\end{definition}

\begin{definition}[Transformer block]\label{def:transformer_tf}
    Given a set of model weights $\theta^{h, m, r} = \{ \{(W_Q^i, W_K^i, W_V^i, W_O^i)\}_{i=1}^h,$ $ W_1, W_2, b_1, b_2 \}$, the computation of a transformer block is given by the combination of multi-head self-attention computation (Definition~\ref{def:attn}) and feed-forward computation (Definition~\ref{def:feed_forward}). Formally, for an input matrix $X \in \R^{n \times d_0}$, we define:
    \begin{align*}
        {\sf TF}_{\theta^{h, m, r}}(X) := {\sf FF} \circ {\sf Attn}(X) \in \R^{n \times d_0}
    \end{align*}
\end{definition}

\begin{definition}[Reshape Layer]\label{def:R}
    We define the reshape network $R: \R^d \rightarrow \R^{n \times d_0}$.
\end{definition}

\begin{definition}[Complete transformer network]\label{def:model}
    We consider a transformer network as a composition of a transformer block (Definition~\ref{def:transformer_tf}) with model weight $\theta^{h, m, r}$, which is:
    \begin{align*}
        & ~ {\cal T}^{h, m, r} \\
        := & ~ \{ {\cal F}: \R^{n \times d_0} \rightarrow \R^{n \times d_0}~\\
        & ~ |~\text{${\cal F}$ is a composition of Transformer blocks ${\sf TF}_{\theta^{h, m, r}}$’s with positional embedding $E \in \R^{n \times d_0}$}\}
    \end{align*}
    We especially say $\theta^{h, m, r}$ is the model weight that contains $h$ heads, $m$ hidden size for attention and $r$ hidden size for feed-forward. See Example~\ref{exp:cal_F} for further explanation of the sequence-to-sequence mapping ${\cal F}$.
\end{definition}

\begin{example}\label{exp:cal_F}
    We here give an example for the sequence-to-sequence mapping ${\cal F}$ in Definition~\ref{def:model}: Denote $L$ as the number of layers in some transformer network. For an input matrix $X \in \R^{n \times d}$, we use $E \in \R^{n \times d}$ to denote the positional encoding, we then define:
    \begin{align*}
        {\cal F}(X) := {\sf TF}^L \circ {\sf TF}^{L-1} \circ \cdots \circ {\sf TF}^2 \circ {\sf TF}^1(X + E)
    \end{align*}
\end{example}

\subsection{Approximation via DiT} \label{sub:app:approx_dit}

\begin{theorem}\label{thm:uat}
    If the following conditions hold:
    \begin{itemize}
        \item Given a video caption distribution ${\cal V}_c$ as Definition~\ref{def:V_c}.
        \item For any $(V, c) \sim {\cal V}_c$, we define the discretized form of video as Definition~\ref{def:wt_V}.
        \item Let the observation matrix $\Phi: \{0, 1\}^{N \times \frac{T}{\Delta t}}$ be defined as Definition~\ref{def:Phi}.
        \item Let the visual decoder function $D: \R^d \rightarrow \R^D$ be defined as Definition~\ref{def:visual_decoder}.
        \item Let the ideal version of the sequence of latent patches $u \in \R^{\frac{T}{\Delta t} \times d}$ be defined as Definition~\ref{def:u}.
        \item Let the real-world version of the sequence of latent patches $\wt{u} \in \R^{N \times d}$ be defined as Definition~\ref{def:wt_u}.
        \item Let $H_N \in \R^{d \times s}$ be defined as Definition~\ref{def:H}.
        \item Let the function of polynomials $g(t)$ be defined as Definition~\ref{def:g_t}.
        \item Let the time-dependent mean of Gaussian distribution $\mu_t(\wt{u})$ be defined as Definition~\ref{def:mu}.
        \item Let the time-dependent standard deviation $\sigma_t(\wt{u})$ be defined as Definition~\ref{def:sigma}.
        \item Denote $\sigma_{\min} > 0$.
        \item Sample $z \sim \mathcal{N}(0, I_d)$.
        \item Define a model function $F_\theta: \R^d \times \R^\ell \times [0, T] \rightarrow \R^d$ with parameters $\theta$.
        \item Let the training objective ${\cal L}(\theta)$ be defined as Definition~\ref{def:L}.
    \end{itemize}
    Then there exists a transformer network $f_{\cal T} \in {\cal T}_{P}^{2, 1, 4}$ defining function $F_\theta(z, c, t) := f_{\cal T}( R([z^\top, c^\top, t]^\top) )$ with parameters $\theta$ that satisfies ${\cal L}(\theta) \leq \epsilon$ for any error $\epsilon > 0$.
\end{theorem}

\begin{proof}
    Following Assumption~\ref{ass:M}, we first denote $\wt{V}_{\tau} = \wt{{\cal M}}_\tau(c)$ for any $\tau \in [\frac{T}{\Delta t}]$ to discretize function ${\cal M}$. Then we have:
    \begin{align}\label{eq:wt_u_func}
        \wt{u}_{\tau} = {\cal D}^{-1}\Big( (\Phi \wt{{\cal M}}(c))_\tau \Big).
    \end{align}
    where this step follows from Definition~\ref{def:Phi} and Definition~\ref{def:visual_decoder}.

    Besides, we also have:
    \begin{align}\label{eq:mu_func}
        \mu_t(\wt{u}) = & ~ H_N g(t) \notag \\
        = & ~ \Big( H_{N-1} ( I_s - \frac{1}{N} A )^\top + \frac{1}{N} \wt{u}_{N} B^\top \Big) g(t) \notag\\
        = & ~ \Bigg( H_{N-2} \Big ( (I_s - \frac{1}{N-1} A )^\top + \frac{1}{N-1} \wt{u}_{N} B^\top \Big) ( I_s - \frac{1}{N} A )^\top + \frac{1}{N} \wt{u}_{N} B^\top \Bigg) g(t) \notag\\
        = & ~ \Bigg( H_0 \prod_{\tau=1}^N (I_s - \frac{1}{\tau}A)^\top + \sum_{\tau=1}^N \Big( \prod_{\tau'=1}^{\tau - 1} (I_s - \frac{1}{\tau'} A )^\top\Big) \cdot \frac{1}{N+1-\tau} \wt{u}_{N+1-\tau} B^\top \Bigg) g(t)
    \end{align}
    where these steps follow from Definition~\ref{def:mu} and simple algebras.

    Recall $F_\theta(z, c, t) := f_{\cal T}( R([z^\top, c^\top, t]^\top) )$, we choose $n=1$, then there is a target function given by:
    \begin{align*}
        &  ~ f_{\cal T}([z^\top, c^\top, t]) \\
        = & ~ \frac{\sigma_t'(\wt{u})}{\sigma_t(\wt{u})} ( z - \Bigg( H_0 \prod_{\tau=1}^N (I_s - \frac{1}{\tau}A)^\top + \sum_{\tau=1}^N \Big( \prod_{\tau'=1}^{\tau - 1} (I_s - \frac{1}{\tau'} A )^\top\Big) \cdot \frac{1}{N+1-\tau} \wt{u}_{N+1-\tau} B^\top \Bigg) g(t)  ) \\
        & ~ + \Bigg( H_0 \prod_{\tau=1}^N (I_s - \frac{1}{\tau}A)^\top + \sum_{\tau=1}^N \Big( \prod_{\tau'=1}^{\tau - 1} (I_s - \frac{1}{\tau'} A )^\top\Big) \cdot \frac{1}{N+1-\tau} \wt{u}_{N+1-\tau}' B^\top \Bigg) g(t) \\
        & ~ + \Bigg( H_0 \prod_{\tau=1}^N (I_s - \frac{1}{\tau}A)^\top + \sum_{\tau=1}^N \Big( \prod_{\tau'=1}^{\tau - 1} (I_s - \frac{1}{\tau'} A )^\top\Big) \cdot \frac{1}{N+1-\tau} \wt{u}_{N+1-\tau} B^\top \Bigg) g'(t) \\
        = & ~ \frac{\sigma_t'(\wt{u})}{\sigma_t(\wt{u})} ( z \\ 
        & ~ - \Bigg( H_0 \prod_{\tau=1}^N (I_s - \frac{1}{\tau}A)^\top + \sum_{\tau=1}^N \Big( \prod_{\tau'=1}^{\tau - 1} (I_s - \frac{1}{\tau'} A )^\top\Big) \cdot \frac{1}{N+1-\tau} {\cal D}^{-1}\Big( (\Phi \wt{{\cal M}}(c))_{N+1-\tau} \Big) B^\top \Bigg) g(t)  ) \\
        & ~ + \Bigg( H_0 \prod_{\tau=1}^N (I_s - \frac{1}{\tau}A)^\top + \sum_{\tau=1}^N \Big( \prod_{\tau'=1}^{\tau - 1} (I_s - \frac{1}{\tau'} A )^\top\Big) \cdot \frac{1}{N+1-\tau} \Big({\cal D}^{-1}\Big( (\Phi \wt{{\cal M}}(c))_{N+1-\tau} \Big) \Big)' B^\top \Bigg) g(t) \\
        & ~ + \Bigg( H_0 \prod_{\tau=1}^N (I_s - \frac{1}{\tau}A)^\top + \sum_{\tau=1}^N \Big( \prod_{\tau'=1}^{\tau - 1} (I_s - \frac{1}{\tau'} A )^\top\Big) \cdot \frac{1}{N+1-\tau} {\cal D}^{-1}\Big( (\Phi \wt{{\cal M}}(c))_{N + 1 - \tau} \Big) B^\top \Bigg) g'(t)
    \end{align*}
    where the first step follows the combination of Theorem~\ref{thm:close_form} and Eq.~\eqref{eq:mu_func}, and the differentiablity of $\wt{u}_\tau$ is ensure by Assumption~\ref{ass:k}, the second step follows from Eq.~\eqref{eq:wt_u_func}.

    Following Theorem 2 and Theorem 3 in \cite{ybr+19}, we thus complete the proof by obtaining the theorem result.
\end{proof}

\section{Interpolation and Extrapolation}
\label{sec:app:inter_extra}

This section first introduce properties of HiPPO-LegS in Appendix~\ref{sub:app:hippo_property}. Also, we bound the error of VLFM in Appendix~\ref{sub:app:error}.

\subsection{HiPPO-LegS Properties} \label{sub:app:hippo_property}

\begin{lemma}[Proposition 6 in \cite{gde+20}]\label{lem:optimal_projs}
    If the following conditions hold:
    \begin{itemize}
        \item Given a video caption distribution ${\cal V}_c$ as Definition~\ref{def:V_c}.
        \item For any $(V, c) \sim {\cal V}_c$, we define the discretized form of video as Definition~\ref{def:wt_V}.
        \item Let the observation matrix $\Phi: \{0, 1\}^{N \times \frac{T}{\Delta t}}$ be defined as Definition~\ref{def:Phi}.
        \item Let the visual decoder function $D: \R^d \rightarrow \R^D$ be defined as Definition~\ref{def:visual_decoder}.
        \item Let the ideal version of the sequence of latent patches $u \in \R^{\frac{T}{\Delta t} \times d}$ be defined as Definition~\ref{def:u}.
        \item Let the real-world version of the sequence of latent patches $\wt{u} \in \R^{N \times d}$ be defined as Definition~\ref{def:wt_u}.
        \item Let $H_N \in \R^{d \times s}$ be defined as Definition~\ref{def:H}.
        \item Let the function of polynomials $g(t)$ be defined as Definition~\ref{def:g_t}.
        \item Let the time-dependent mean of Gaussian distribution $\mu_t(\wt{u})$ be defined as Definition~\ref{def:mu}.
        \item Let the time-dependent standard deviation $\sigma_t(\wt{u})$ be defined as Definition~\ref{def:sigma}.
        \item Denote $\sigma_{\min} > 0$.
        \item Sample $z \sim \mathcal{N}(0, I_d)$.
        \item Define a model function $F_\theta: \R^d \times \R^\ell \times [0, T] \rightarrow \R^d$ with parameters $\theta$.
        \item Let the training objective ${\cal L}(\theta)$ be defined as Definition~\ref{def:L}.
        \item Let Assumptions~\ref{ass:k}, Assumption~\ref{ass:L_0}, Assumption~\ref{ass:M} and Assumption~\ref{ass:U} hold.
    \end{itemize}
    Then we have:
    \begin{align*}
        \| \mu_{\tau \cdot \Delta t}(\wt{u}) - \wt{u}_\tau \|_2 = O(t^{k}s^{-k+1/2})
    \end{align*}
\end{lemma}

\begin{proof}
    This lemma is a re-statement of Proposition 6 in \cite{gde+20}.
\end{proof}

\begin{lemma}[Proposition 3 in \cite{gde+20}]\label{lem:timescale_robustness}
    If the following conditions hold:
    \begin{itemize}
        \item Given a video caption distribution ${\cal V}_c$ as Definition~\ref{def:V_c}.
        \item For any $(V, c) \sim {\cal V}_c$, we define the discretized form of video as Definition~\ref{def:wt_V}.
        \item Let the observation matrix $\Phi: \{0, 1\}^{N \times \frac{T}{\Delta t}}$ be defined as Definition~\ref{def:Phi}.
        \item Let the visual decoder function $D: \R^d \rightarrow \R^D$ be defined as Definition~\ref{def:visual_decoder}.
        \item Let the ideal version of the sequence of latent patches $u \in \R^{\frac{T}{\Delta t} \times d}$ be defined as Definition~\ref{def:u}.
        \item Let the real-world version of the sequence of latent patches $\wt{u} \in \R^{N \times d}$ be defined as Definition~\ref{def:wt_u}.
        \item Let $H_N \in \R^{d \times s}$ be defined as Definition~\ref{def:H}.
        \item Let the function of polynomials $g(t)$ be defined as Definition~\ref{def:g_t}.
        \item Let the time-dependent mean of Gaussian distribution $\mu_t(\wt{u})$ be defined as Definition~\ref{def:mu}.
        \item Let the time-dependent standard deviation $\sigma_t(\wt{u})$ be defined as Definition~\ref{def:sigma}.
        \item Denote $\sigma_{\min} > 0$.
        \item Sample $z \sim \mathcal{N}(0, I_d)$.
        \item Define a model function $F_\theta: \R^d \times \R^\ell \times [0, T] \rightarrow \R^d$ with parameters $\theta$.
        \item Let the training objective ${\cal L}(\theta)$ be defined as Definition~\ref{def:L}.
        \item Let Assumptions~\ref{ass:k}, Assumption~\ref{ass:L_0}, Assumption~\ref{ass:M} and Assumption~\ref{ass:U} hold.
    \end{itemize}
    For any integer scale factor $\beta > 0$, the frames of video $\wt{V}_{\tau}$ is scaled to $\wt{V}_{\beta \tau}$, it doesn't affect the result of $H_N$ (Definition~\ref{def:H}).
\end{lemma}

\begin{proof}
    This lemma is a re-statement of Proposition 3 in \cite{gde+20}.
\end{proof}

\subsection{Error Bounds} \label{sub:app:error}

\begin{lemma}\label{lem:hippo_error}
    If the following conditions hold:
    \begin{itemize}
        \item Given a video caption distribution ${\cal V}_c$ as Definition~\ref{def:V_c}.
        \item For any $(V, c) \sim {\cal V}_c$, we define the discretized form of video as Definition~\ref{def:wt_V}.
        \item Let the observation matrix $\Phi: \{0, 1\}^{N \times \frac{T}{\Delta t}}$ be defined as Definition~\ref{def:Phi}.
        \item Let the visual decoder function $D: \R^d \rightarrow \R^D$ be defined as Definition~\ref{def:visual_decoder}.
        \item Let the ideal version of the sequence of latent patches $u \in \R^{\frac{T}{\Delta t} \times d}$ be defined as Definition~\ref{def:u}.
        \item Let the real-world version of the sequence of latent patches $\wt{u} \in \R^{N \times d}$ be defined as Definition~\ref{def:wt_u}.
        \item Let $H_N \in \R^{d \times s}$ be defined as Definition~\ref{def:H}.
        \item Let the function of polynomials $g(t)$ and matrix $G$ be defined as Definition~\ref{def:g_t}.
        \item Denote $1/\lambda^* := \lambda_{\min}(G) > 0$.
        \item Let the time-dependent mean of Gaussian distribution $\mu_t(\wt{u})$ be defined as Definition~\ref{def:mu}.
        \item Let the time-dependent standard deviation $\sigma_t(\wt{u})$ be defined as Definition~\ref{def:sigma}.
        \item Denote $\sigma_{\min} > 0$.
        \item Sample $z \sim \mathcal{N}(0, I_d)$.
        \item Define a model function $F_\theta: \R^d \times \R^\ell \times [0, T] \rightarrow \R^d$ with parameters $\theta$.
        \item Let the training objective ${\cal L}(\theta)$ be defined as Definition~\ref{def:L}.
        \item Let Assumptions~\ref{ass:k}, Assumption~\ref{ass:L_0}, Assumption~\ref{ass:M} and Assumption~\ref{ass:U} hold.
        \item $\delta \in (0, 1)$.
        \item Choosing $s = O(\frac{\Delta t}{T}\log((\frac{\Delta t}{T})^{1.5}/1/\lambda^*))$.
    \end{itemize}
    Particularly, we define:
    \begin{itemize}
        \item $\epsilon_1 := O(T^k s^{-k+1/2})$.
        \item $\epsilon_2 := O(\sqrt{d\log(d/\delta)})$.
        \item $\epsilon_3 := 1/\lambda^* U d^{0.5} \sqrt{\frac{T}{\Delta t} - N} \cdot \exp(O(\frac{T}{\Delta t}s))$.
    \end{itemize}
    Then with a probability at least $1 - \delta$, we have:
    \begin{align*}
        \| \psi_t(\wt{u}) - u_t \|_2 \leq \epsilon_1 + \epsilon_2 + \epsilon_3.
    \end{align*}
\end{lemma}

\begin{proof}
    We have:
    \begin{align*}
        \| \psi_t(\wt{u}) - u_t \|_2
        = & ~ \| \sigma_t(\wt{u}) \cdot z + \mu_t(\wt{u}) - u_t \|_2 \\
        \leq & ~ \| \sigma_t(\wt{u}) \cdot z \|_2 + \|  \mu_t(\wt{u}) - u_t \|_2 \\
        \leq & ~ \| z \|_2 + \|  \mu_t(\wt{u}) - u_t \|_2 \\
        \leq & ~ O( \sqrt{d \log(d/\delta)} ) +  \|  \mu_t(\wt{u}) - u_t \|_2 \\
        = & ~ \epsilon_2 + \|  \mu_t(\wt{u}) - u_t \|_2
    \end{align*}
    where the first step follows from Definition~\ref{def:psi}, the second step follows from triangle inequality, the third step follows from $\sigma_t(\wt{u}) \leq 1, \forall t \in [0, T]$ by some simple algebras and Definition~\ref{def:sigma}, the fourth step follows from the union bound of Gaussian tail bound (Fact~\ref{fac:gaussian_tail}), the last step follows from the definition of $\epsilon_2$.

    Then we get:
    \begin{align*}
        \|  \mu_t(\wt{u}) - u_t \|_2
        = & ~ \| H_N g(t) - u_t \|_2 \\
        = & ~ \| (M \cdot G)^\dagger (M \cdot u) \cdot g(t) - u_t \|_2 \\
        \leq & ~ \| (M \cdot G)^\dagger (M \cdot u) \cdot g(t) - G^\dagger u \cdot g(t) \|_2 + O((\frac{T}{\Delta t})^k s^{-k+1/2}) \\
        \leq & ~ \| ((M \cdot G)^\dagger (M \cdot u) - G^\dagger u \|_2 \cdot \| g(t) \|_2 + O((\frac{T}{\Delta t})^k s^{-k+1/2}) \\
        = & ~ \| ((M \cdot G)^\dagger (M \cdot u) - G^\dagger u \|_2 \cdot \| g(t) \|_2 + \epsilon_1
    \end{align*}
    where the first step follows from Definition~\ref{def:mu}, the second step follows from optimal error of solving $\| M G H - M u \|_2^2$, pesdueo-inverse matrix $(M \cdot G)^\dag \in \R^{d \times \frac{T}{\Delta t}}$ and defining a mask $M = \diag(m)$ where $m := \{0, 1\}^{\frac{T}{\Delta t}}$ and $\langle m, {\bf 1}_{\frac{T}{\Delta t}} \rangle = N$, the third step follows from the optimal error of solving $\| G H - u \|_2^2$, pesdueo-inverse matrix $G^\dag \in \R^{d \times \frac{T}{\Delta}}$ and Lemma~\ref{lem:optimal_projs}, the fourth step follows from Cauchy–Schwarz inequality and the last step follows from the definition of $\epsilon_2$.

    Next, we can show that:
    \begin{align*}
        \| (M \cdot G)^\dagger (M \cdot u) - G^\dagger u \|_2
        = & ~ \| (M \cdot G)^\dagger (M \cdot u) - G^\dagger (M \cdot u) + G^\dagger (M \cdot u) - G^\dagger u \|_2 \\
        \leq &~ \| (M \cdot G)^\dagger (M \cdot u)  - G^\dagger (M \cdot u) \|_2 + \| G^\dagger (M \cdot u) - G^\dagger u \|_2 \\
        \leq & ~ \| (M \cdot G)^\dagger - G^\dagger \|_2 \| (M \cdot u) \|_2 + \| G^\dagger \|_2 \| (M \cdot u) -  u \|_2
    \end{align*}
    where the first step follows from simple algebras, the second step follows from triangle inequality, the last step follows from Cauchy–Schwarz inequality.

    We first give:
    \begin{align}\label{eq:bound_G_dag}
        \| G^\dagger\|_2 \leq &~ 1/\lambda^* \sqrt{\frac{T}{\Delta t} \cdot s}
    \end{align}
    where this step follows from Definition~\ref{def:g_t}, Fact~\ref{fac:infity_norm_pesdueo_inverse} and the definition of $\ell_2$ norm.

    And:
    \begin{align*}
        \| u\|_2 \leq & ~ U \sqrt{\frac{T}{\Delta t} \cdot d}
    \end{align*}
    where this step follows from Assumption~\ref{ass:U} and the definition of $\ell_2$ norm.

    Also:
    \begin{align}\label{eq:bound_G}
        \| G\|_2 \leq & ~ \sqrt{\frac{T}{\Delta t} \cdot s} \exp( O(\frac{T}{\Delta t} \cdot s) ) 
    \end{align}
   where this step follows from Definition~\ref{def:g_t} and the definition of $\ell_2$ norm.

    Besides, we have:
    \begin{align*}
        \| (M \cdot G)^\dagger - G^\dagger \|_2
        \leq & ~ \frac{\| G^\dagger\|_2^2 \| I_{\frac{T}{\Delta t}} - M\|_2 \cdot \| G\|_2}{1 - \| G^\dagger\|_2 \cdot \| I_{\frac{T}{\Delta t}} - M\|_2 \cdot \| G\|_2} \\
        \leq & ~ \frac{{1/\lambda^*}^2 (\frac{T}{\Delta t} s)^{1.5} \sqrt{\frac{T}{\Delta t} - N} \cdot \exp(O(\frac{T}{\Delta t}s))}{1 - {1/\lambda^*} \frac{T}{\Delta t} s \sqrt{\frac{T}{\Delta t} - N}\cdot \exp(O(\frac{T}{\Delta t}s)) }
    \end{align*}
    where the first step follows from Fact~\ref{fac:pesdueo_inverse_diff}, simple algebras, and Cauchy–Schwarz inequality, the second step follows from Eq.~\eqref{eq:bound_G_dag}, Eq.~\eqref{eq:bound_G}, Definition~\ref{def:g_t} and simeple algebras.

    Combining all results, we get:
    \begin{align*}
        & ~ \| ((M \cdot G)^\dagger (M \cdot u) - G^\dagger u \|_2 \\
        \leq & ~ \frac{{1/\lambda^*}^2 (\frac{T}{\Delta t} s)^{1.5} \sqrt{\frac{T}{\Delta t} - N} \cdot \exp(O(\frac{T}{\Delta t}s))}{1 - {1/\lambda^*} \frac{T}{\Delta t} s \sqrt{\frac{T}{\Delta t} - N}\cdot \exp(O(\frac{T}{\Delta t}s)) } \cdot U \sqrt{\frac{T}{\Delta t} N d} + 1/\lambda^* \sqrt{\frac{T}{\Delta t} - N} \cdot U \sqrt{\frac{T}{\Delta t} \cdot d} \\
        \leq & ~ 1/\lambda^* U d^{0.5}  \sqrt{\frac{T}{\Delta t} (\frac{T}{\Delta t} - N)}  \cdot \Big( \frac{ {1/\lambda^*} (\frac{T}{\Delta t})^{1.5} N^{0.5} s^{1.5}  \cdot \exp(O(\frac{T}{\Delta t}s))}{1 - {1/\lambda^*} (\frac{T}{\Delta t})^{1.5} s \cdot \exp(O(\frac{T}{\Delta t}s))} + 1 \Big) \\
        \leq & ~ 1/\lambda^* U d^{0.5}  \sqrt{\frac{T}{\Delta t} (\frac{T}{\Delta t} - N)}  \cdot  \frac{ 1}{1 - {1/\lambda^*} (\frac{T}{\Delta t})^{1.5} s \cdot \exp(O(\frac{T}{\Delta t}s))}  \\
        \leq & ~ O\Big( 1/\lambda^* U d^{0.5}  \sqrt{\frac{T}{\Delta t} (\frac{T}{\Delta t} - N)}\Big)
    \end{align*}
    where the second and third steps follow from simple algebras, the last step follows from plugging the choice of $s$.

    Finally, we have:
    \begin{align*}
        \| ((M \cdot G)^\dagger (M \cdot u) - G^\dagger u \|_2 \cdot \| g(t) \|_2 
        \leq & ~ O\Big( 1/\lambda^* U d^{0.5}  \sqrt{\frac{T}{\Delta t} (\frac{T}{\Delta t} - N)}\Big) \cdot \sqrt{s} \exp(O(\frac{T}{\Delta t}s)) \\
        \leq & ~ 1/\lambda^* U d^{0.5} \sqrt{\frac{T}{\Delta t} - N} \cdot \exp(O(\frac{T}{\Delta t}s))  \\
        = & ~ \epsilon_3
    \end{align*}
    these steps follow from simple algebras, Definition~\ref{def:g_t} and the definition of $\epsilon_3$.
\end{proof}

\begin{theorem}\label{thm:inter_extra_polation}
    If the following conditions hold:
    \begin{itemize}
        \item Given a video caption distribution ${\cal V}_c$ as Definition~\ref{def:V_c}.
        \item For any $(V, c) \sim {\cal V}_c$, we define the discretized form of video as Definition~\ref{def:wt_V}.
        \item Let the observation matrix $\Phi: \{0, 1\}^{N \times \frac{T}{\Delta t}}$ be defined as Definition~\ref{def:Phi}.
        \item Let the visual decoder function $D: \R^d \rightarrow \R^D$ be defined as Definition~\ref{def:visual_decoder}.
        \item Let the ideal version of the sequence of latent patches $u \in \R^{\frac{T}{\Delta t} \times d}$ be defined as Definition~\ref{def:u}.
        \item Let the real-world version of the sequence of latent patches $\wt{u} \in \R^{N \times d}$ be defined as Definition~\ref{def:wt_u}.
        \item Let $H_N \in \R^{d \times s}$ be defined as Definition~\ref{def:H}.
        \item Let the function of polynomials $g(t)$ and matrix $G$ be defined as Definition~\ref{def:g_t}.
        \item Denote $1/\lambda^* := \lambda_{\min}(G) > 0$.
        \item Let the time-dependent mean of Gaussian distribution $\mu_t(\wt{u})$ be defined as Definition~\ref{def:mu}.
        \item Let the time-dependent standard deviation $\sigma_t(\wt{u})$ be defined as Definition~\ref{def:sigma}.
        \item Denote $\sigma_{\min} > 0$.
        \item Sample $z \sim \mathcal{N}(0, I_d)$.
        \item Define a model function $F_\theta: \R^d \times \R^\ell \times [0, T] \rightarrow \R^d$ with parameters $\theta$.
        \item Let the training objective ${\cal L}(\theta)$ be defined as Definition~\ref{def:L}.
        \item Let Assumptions~\ref{ass:k}, Assumption~\ref{ass:L_0}, Assumption~\ref{ass:M} and Assumption~\ref{ass:U} hold.
        \item $\delta \in (0, 1)$.
    \end{itemize}
    Particularly, we define:
    \begin{itemize}
        \item $\epsilon_1 := O(T^k s^{-k+1/2})$.
        \item $\epsilon_2 := O(\sqrt{d\log(d/\delta)})$.
        \item $\epsilon_3 := 1/\lambda^* U d^{0.5} \sqrt{\frac{T}{\Delta t} - N} \cdot \exp(O(\frac{T}{\Delta t}s))$.
    \end{itemize}
    Then with a probability at least $1 - \delta$, we have:
    \begin{align*}
        \| {\cal D}(z + \int_0^{t} F_\theta(z, c, t') \d t') - u_t\|_2 \leq \epsilon_0 + L_0 (\epsilon_1 + \epsilon_2 + \epsilon_3).
    \end{align*}
\end{theorem}


\begin{proof}
    This proof follows from the combination of Assumption~\ref{ass:L_0}, Theorem~\ref{thm:uat} and Lemma~\ref{lem:hippo_error}.
\end{proof}

\ifdefined \isarxiv
\bibliography{ref}
\bibliographystyle{alpha}
\fi


%%% some writing rules

%% Writing rule for creating tags.
%% Tags :
%% Theorem    \ref{thm:bla_bla}
%% Lemma      \ref{lem:bla_bla}
%% Claim      \ref{cla:bla_bla}
%% Corollary  \ref{cor:bla_bla}
%% Fact       \ref{fac:bla_bla}
%% Definition \ref{def:bla_bla}
%% Section    \ref{sec:bla_bla}
%% Subsection \ref{sub:bla_bla}
%% Equation   \ref{eq:bla_bla}



\end{document}



%%%%%%%%%%%%%%%%%%%%%%%%%%%%%%%%%%%%%%%%%%%%%%%%%%%%%%%%%%%%%%%%%%%%%%%%%%%%%%%%%%%%%%%%%%%%%%%%%%%%%%%%%%%%%%%%%%%%%%%%%%%%%%%%%%%%%%%%%%%%%%%%%%%%%%%%%%%%%%%%%%%%%%%%%%%%%%%%%%%%%%%%%%%%%%%%%%%%%%%%%%%%%%%%%%%%%%%%%%%%%%%%%%%%%%%%%%%%%%%%%%%%%%%%%%%%%%%%%%%%%%%%%%%%%%%%%%%%%%%%%%%%%%%%%%%%%%%%%%%%%%%%%%%%%%%%%%%%%%%%%%%%%%%%%%%%%%%%%%%%%%%%%%%%%%%%%%%%%%%%%%%%%%%%%%%%%%%%%%%%%%%%%%%%%%%%%%%%%%%%%%%%%%%%%%%%%%%%%%%%%%%%%%%%%%%%%%%%%%%%%%%%%%%%%%%%%%%%%%%%%%
