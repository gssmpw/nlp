\def\isarxiv{1} %%% for icml submission version, we comment this line

\ifdefined\isarxiv
\documentclass[11pt]{article}

\usepackage[numbers]{natbib}

\else
\documentclass{article}

% Recommended, but optional, packages for figures and better typesetting:




% Attempt to make hyperref and algorithmic work together better:
\newcommand{\theHalgorithm}{\arabic{algorithm}}

% Use the following line for the initial blind version submitted for review:
% \usepackage[accepted]{icml2025}
\usepackage{icml2025}

% Todonotes is useful during development; simply uncomment the next line
%    and comment out the line below the next line to turn off comments
%\usepackage[disable,textsize=tiny]{todonotes}
\usepackage[textsize=tiny]{todonotes}

\icmltitlerunning{Video Latent Flow Matching: Optimal Polynomial Projections for Video Interpolation and Extrapolation}

\fi


\usepackage{amsmath}
\usepackage{amsthm}
\usepackage{amssymb}
\usepackage{algorithm}
% \usepackage{subfig}
\usepackage{algpseudocode}
\usepackage{graphicx}
\usepackage{grffile}
\usepackage{wrapfig,epsfig}
\usepackage{url}
\usepackage{xcolor}
\usepackage{epstopdf}

\usepackage{microtype}
\usepackage{graphicx}
\usepackage{subfig}
% \usepackage{subfigure}
\usepackage{booktabs} % for professional tables

% hyperref makes hyperlinks in the resulting PDF.
% If your build breaks (sometimes temporarily if a hyperlink spans a page)
% please comment out the following usepackage line and replace
% \usepackage{icml2025} with \usepackage[nohyperref]{icml2025} above.
\usepackage{hyperref}

\usepackage{bbm}
\usepackage{dsfont}

 
\allowdisplaybreaks
 

\ifdefined\isarxiv

\let\C\relax
\usepackage{tikz}
\usepackage{hyperref}  %%% arxiv don't allow this.
\hypersetup{colorlinks=true,citecolor=blue,linkcolor=blue} %%% Zhao : maybe we should comment this in submission.
\usetikzlibrary{arrows}
\usepackage[margin=1in]{geometry}
 

\fi
 
\graphicspath{{./figs/}}

\theoremstyle{plain}
\newtheorem{theorem}{Theorem}[section]
\newtheorem{lemma}[theorem]{Lemma}
\newtheorem{definition}[theorem]{Definition}
\newtheorem{notation}[theorem]{Notation}
%\newtheorem{proof}[theorem]{Proof}
\newtheorem{proposition}[theorem]{Proposition}
\newtheorem{corollary}[theorem]{Corollary}
\newtheorem{conjecture}[theorem]{Conjecture}
\newtheorem{assumption}[theorem]{Assumption}
\newtheorem{observation}[theorem]{Observation}
\newtheorem{fact}[theorem]{Fact}
\newtheorem{remark}[theorem]{Remark}
\newtheorem{claim}[theorem]{Claim}
\newtheorem{example}[theorem]{Example}
\newtheorem{problem}[theorem]{Problem}
\newtheorem{open}[theorem]{Open Problem}
\newtheorem{property}[theorem]{Property}
\newtheorem{hypothesis}[theorem]{Hypothesis}

\newcommand{\wh}{\widehat}
\newcommand{\wt}{\widetilde}
\newcommand{\ov}{\overline}
\newcommand{\N}{\mathcal{N}}
\newcommand{\R}{\mathbb{R}}
\newcommand{\RHS}{\mathrm{RHS}}
\newcommand{\LHS}{\mathrm{LHS}}
\renewcommand{\d}{\mathrm{d}}
\renewcommand{\i}{\mathbf{i}}
\renewcommand{\tilde}{\wt}
\renewcommand{\hat}{\wh}
\newcommand{\Tmat}{{\cal T}_{\mathrm{mat}}}

\DeclareMathOperator*{\E}{{\mathbb{E}}}
\DeclareMathOperator*{\var}{\mathrm{Var}}
\DeclareMathOperator*{\Z}{\mathbb{Z}}
\DeclareMathOperator*{\C}{\mathbb{C}}
\DeclareMathOperator*{\D}{\mathcal{D}}
\DeclareMathOperator*{\median}{median}
\DeclareMathOperator*{\mean}{mean}
\DeclareMathOperator{\OPT}{OPT}
\DeclareMathOperator{\supp}{supp}
\DeclareMathOperator{\poly}{poly}

\DeclareMathOperator{\nnz}{nnz}
\DeclareMathOperator{\sparsity}{sparsity}
\DeclareMathOperator{\rank}{rank}
\DeclareMathOperator{\diag}{diag}
\DeclareMathOperator{\dist}{dist}
\DeclareMathOperator{\cost}{cost}
\DeclareMathOperator{\vect}{vec}
\DeclareMathOperator{\tr}{tr}
\DeclareMathOperator{\dis}{dis}
\DeclareMathOperator{\cts}{cts}



\makeatletter
\newcommand*{\RN}[1]{\expandafter\@slowromancap\romannumeral #1@}
\makeatother
% \newcommand{\Zhao}[1]{{\color{red}[Zhao: #1]}}
% \newcommand{\Yang}[1]{{\color{purple}[Yang: #1]}}
% \newcommand{\Chiwun}[1]{{\color{blue}[Chiwun: #1]}}
% \newcommand{\InernNameB}[1]{{\color{blue}[InternNameB: #1]}} %%%Change to intern name


\usepackage{lineno}
% \def\linenumberfont{\normalfont\small}


\begin{document}

\ifdefined\isarxiv

\date{}


\title{Video Latent Flow Matching: Optimal Polynomial Projections for Video Interpolation and Extrapolation}

\author{
Yang Cao\thanks{\texttt{ycao4@wyomingseminary.org}. Wyoming Seminary.}
\and
Zhao Song\thanks{\texttt{magic.linuxkde@gmail.com}. Simons Institute for the Theory of Computing, University of California, Berkeley.}
\and
Chiwun Yang\thanks{\texttt{christiannyang37@gmail.com}. Sun Yat-sen University.}
}

\else

\twocolumn[
\icmltitle{Video Latent Flow Matching:\\Optimal Polynomial Projections for Video Interpolation and Extrapolation}
% \icmlsetsymbol{equal}{*}

\begin{icmlauthorlist}
\icmlauthor{Yang Cao}{sem}
\icmlauthor{Zhao Song}{simons}
\icmlauthor{Chiwun Yang}{sun}
\end{icmlauthorlist}

\icmlaffiliation{sem}{Wyoming Seminary. Kingston, PA 18704, USA}
\icmlaffiliation{simons}{Simons Institute for the Theory of Computing, University of California, Berkeley. Berkeley, CA 94720, USA}
\icmlaffiliation{sun}{Sun Yat-sen University. Guangzhou, Guangdong, China}

\icmlcorrespondingauthor{Yang Cao}{ycao4@wyomingseminary.org}
\icmlcorrespondingauthor{Zhao Song}{magic.linuxkde@gmail.com}
\icmlcorrespondingauthor{Chiwun Yang}{christiannyang37@gmail.com}

% You may provide any keywords that you
% find helpful for describing your paper; these are used to populate
% the "keywords" metadata in the PDF but will not be shown in the document
\icmlkeywords{Deep Learning, Video Generation, Diffusion Models, Flow Matching}

\vskip 0.3in
]
% \icmlEqualContribution
\printAffiliationsAndNotice{}

\fi


\ifdefined\isarxiv
\begin{titlepage}
  \maketitle
\begin{abstract}
\begin{abstract}


The choice of representation for geographic location significantly impacts the accuracy of models for a broad range of geospatial tasks, including fine-grained species classification, population density estimation, and biome classification. Recent works like SatCLIP and GeoCLIP learn such representations by contrastively aligning geolocation with co-located images. While these methods work exceptionally well, in this paper, we posit that the current training strategies fail to fully capture the important visual features. We provide an information theoretic perspective on why the resulting embeddings from these methods discard crucial visual information that is important for many downstream tasks. To solve this problem, we propose a novel retrieval-augmented strategy called RANGE. We build our method on the intuition that the visual features of a location can be estimated by combining the visual features from multiple similar-looking locations. We evaluate our method across a wide variety of tasks. Our results show that RANGE outperforms the existing state-of-the-art models with significant margins in most tasks. We show gains of up to 13.1\% on classification tasks and 0.145 $R^2$ on regression tasks. All our code and models will be made available at: \href{https://github.com/mvrl/RANGE}{https://github.com/mvrl/RANGE}.

\end{abstract}



\end{abstract}
\thispagestyle{empty}
\end{titlepage}

{\hypersetup{linkcolor=black}
\tableofcontents
}
\newpage

\else

\begin{abstract}
\begin{abstract}


The choice of representation for geographic location significantly impacts the accuracy of models for a broad range of geospatial tasks, including fine-grained species classification, population density estimation, and biome classification. Recent works like SatCLIP and GeoCLIP learn such representations by contrastively aligning geolocation with co-located images. While these methods work exceptionally well, in this paper, we posit that the current training strategies fail to fully capture the important visual features. We provide an information theoretic perspective on why the resulting embeddings from these methods discard crucial visual information that is important for many downstream tasks. To solve this problem, we propose a novel retrieval-augmented strategy called RANGE. We build our method on the intuition that the visual features of a location can be estimated by combining the visual features from multiple similar-looking locations. We evaluate our method across a wide variety of tasks. Our results show that RANGE outperforms the existing state-of-the-art models with significant margins in most tasks. We show gains of up to 13.1\% on classification tasks and 0.145 $R^2$ on regression tasks. All our code and models will be made available at: \href{https://github.com/mvrl/RANGE}{https://github.com/mvrl/RANGE}.

\end{abstract}


\end{abstract}

\fi


\section{Introduction}
Backdoor attacks pose a concealed yet profound security risk to machine learning (ML) models, for which the adversaries can inject a stealth backdoor into the model during training, enabling them to illicitly control the model's output upon encountering predefined inputs. These attacks can even occur without the knowledge of developers or end-users, thereby undermining the trust in ML systems. As ML becomes more deeply embedded in critical sectors like finance, healthcare, and autonomous driving \citep{he2016deep, liu2020computing, tournier2019mrtrix3, adjabi2020past}, the potential damage from backdoor attacks grows, underscoring the emergency for developing robust defense mechanisms against backdoor attacks.

To address the threat of backdoor attacks, researchers have developed a variety of strategies \cite{liu2018fine,wu2021adversarial,wang2019neural,zeng2022adversarial,zhu2023neural,Zhu_2023_ICCV, wei2024shared,wei2024d3}, aimed at purifying backdoors within victim models. These methods are designed to integrate with current deployment workflows seamlessly and have demonstrated significant success in mitigating the effects of backdoor triggers \cite{wubackdoorbench, wu2023defenses, wu2024backdoorbench,dunnett2024countering}.  However, most state-of-the-art (SOTA) backdoor purification methods operate under the assumption that a small clean dataset, often referred to as \textbf{auxiliary dataset}, is available for purification. Such an assumption poses practical challenges, especially in scenarios where data is scarce. To tackle this challenge, efforts have been made to reduce the size of the required auxiliary dataset~\cite{chai2022oneshot,li2023reconstructive, Zhu_2023_ICCV} and even explore dataset-free purification techniques~\cite{zheng2022data,hong2023revisiting,lin2024fusing}. Although these approaches offer some improvements, recent evaluations \cite{dunnett2024countering, wu2024backdoorbench} continue to highlight the importance of sufficient auxiliary data for achieving robust defenses against backdoor attacks.

While significant progress has been made in reducing the size of auxiliary datasets, an equally critical yet underexplored question remains: \emph{how does the nature of the auxiliary dataset affect purification effectiveness?} In  real-world  applications, auxiliary datasets can vary widely, encompassing in-distribution data, synthetic data, or external data from different sources. Understanding how each type of auxiliary dataset influences the purification effectiveness is vital for selecting or constructing the most suitable auxiliary dataset and the corresponding technique. For instance, when multiple datasets are available, understanding how different datasets contribute to purification can guide defenders in selecting or crafting the most appropriate dataset. Conversely, when only limited auxiliary data is accessible, knowing which purification technique works best under those constraints is critical. Therefore, there is an urgent need for a thorough investigation into the impact of auxiliary datasets on purification effectiveness to guide defenders in  enhancing the security of ML systems. 

In this paper, we systematically investigate the critical role of auxiliary datasets in backdoor purification, aiming to bridge the gap between idealized and practical purification scenarios.  Specifically, we first construct a diverse set of auxiliary datasets to emulate real-world conditions, as summarized in Table~\ref{overall}. These datasets include in-distribution data, synthetic data, and external data from other sources. Through an evaluation of SOTA backdoor purification methods across these datasets, we uncover several critical insights: \textbf{1)} In-distribution datasets, particularly those carefully filtered from the original training data of the victim model, effectively preserve the model’s utility for its intended tasks but may fall short in eliminating backdoors. \textbf{2)} Incorporating OOD datasets can help the model forget backdoors but also bring the risk of forgetting critical learned knowledge, significantly degrading its overall performance. Building on these findings, we propose Guided Input Calibration (GIC), a novel technique that enhances backdoor purification by adaptively transforming auxiliary data to better align with the victim model’s learned representations. By leveraging the victim model itself to guide this transformation, GIC optimizes the purification process, striking a balance between preserving model utility and mitigating backdoor threats. Extensive experiments demonstrate that GIC significantly improves the effectiveness of backdoor purification across diverse auxiliary datasets, providing a practical and robust defense solution.

Our main contributions are threefold:
\textbf{1) Impact analysis of auxiliary datasets:} We take the \textbf{first step}  in systematically investigating how different types of auxiliary datasets influence backdoor purification effectiveness. Our findings provide novel insights and serve as a foundation for future research on optimizing dataset selection and construction for enhanced backdoor defense.
%
\textbf{2) Compilation and evaluation of diverse auxiliary datasets:}  We have compiled and rigorously evaluated a diverse set of auxiliary datasets using SOTA purification methods, making our datasets and code publicly available to facilitate and support future research on practical backdoor defense strategies.
%
\textbf{3) Introduction of GIC:} We introduce GIC, the \textbf{first} dedicated solution designed to align auxiliary datasets with the model’s learned representations, significantly enhancing backdoor mitigation across various dataset types. Our approach sets a new benchmark for practical and effective backdoor defense.




\section{Related Work}

\subsection{Large 3D Reconstruction Models}
Recently, generalized feed-forward models for 3D reconstruction from sparse input views have garnered considerable attention due to their applicability in heavily under-constrained scenarios. The Large Reconstruction Model (LRM)~\cite{hong2023lrm} uses a transformer-based encoder-decoder pipeline to infer a NeRF reconstruction from just a single image. Newer iterations have shifted the focus towards generating 3D Gaussian representations from four input images~\cite{tang2025lgm, xu2024grm, zhang2025gslrm, charatan2024pixelsplat, chen2025mvsplat, liu2025mvsgaussian}, showing remarkable novel view synthesis results. The paradigm of transformer-based sparse 3D reconstruction has also successfully been applied to lifting monocular videos to 4D~\cite{ren2024l4gm}. \\
Yet, none of the existing works in the domain have studied the use-case of inferring \textit{animatable} 3D representations from sparse input images, which is the focus of our work. To this end, we build on top of the Large Gaussian Reconstruction Model (GRM)~\cite{xu2024grm}.

\subsection{3D-aware Portrait Animation}
A different line of work focuses on animating portraits in a 3D-aware manner.
MegaPortraits~\cite{drobyshev2022megaportraits} builds a 3D Volume given a source and driving image, and renders the animated source actor via orthographic projection with subsequent 2D neural rendering.
3D morphable models (3DMMs)~\cite{blanz19993dmm} are extensively used to obtain more interpretable control over the portrait animation. For example, StyleRig~\cite{tewari2020stylerig} demonstrates how a 3DMM can be used to control the data generated from a pre-trained StyleGAN~\cite{karras2019stylegan} network. ROME~\cite{khakhulin2022rome} predicts vertex offsets and texture of a FLAME~\cite{li2017flame} mesh from the input image.
A TriPlane representation is inferred and animated via FLAME~\cite{li2017flame} in multiple methods like Portrait4D~\cite{deng2024portrait4d}, Portrait4D-v2~\cite{deng2024portrait4dv2}, and GPAvatar~\cite{chu2024gpavatar}.
Others, such as VOODOO 3D~\cite{tran2024voodoo3d} and VOODOO XP~\cite{tran2024voodooxp}, learn their own expression encoder to drive the source person in a more detailed manner. \\
All of the aforementioned methods require nothing more than a single image of a person to animate it. This allows them to train on large monocular video datasets to infer a very generic motion prior that even translates to paintings or cartoon characters. However, due to their task formulation, these methods mostly focus on image synthesis from a frontal camera, often trading 3D consistency for better image quality by using 2D screen-space neural renderers. In contrast, our work aims to produce a truthful and complete 3D avatar representation from the input images that can be viewed from any angle.  

\subsection{Photo-realistic 3D Face Models}
The increasing availability of large-scale multi-view face datasets~\cite{kirschstein2023nersemble, ava256, pan2024renderme360, yang2020facescape} has enabled building photo-realistic 3D face models that learn a detailed prior over both geometry and appearance of human faces. HeadNeRF~\cite{hong2022headnerf} conditions a Neural Radiance Field (NeRF)~\cite{mildenhall2021nerf} on identity, expression, albedo, and illumination codes. VRMM~\cite{yang2024vrmm} builds a high-quality and relightable 3D face model using volumetric primitives~\cite{lombardi2021mvp}. One2Avatar~\cite{yu2024one2avatar} extends a 3DMM by anchoring a radiance field to its surface. More recently, GPHM~\cite{xu2025gphm} and HeadGAP~\cite{zheng2024headgap} have adopted 3D Gaussians to build a photo-realistic 3D face model. \\
Photo-realistic 3D face models learn a powerful prior over human facial appearance and geometry, which can be fitted to a single or multiple images of a person, effectively inferring a 3D head avatar. However, the fitting procedure itself is non-trivial and often requires expensive test-time optimization, impeding casual use-cases on consumer-grade devices. While this limitation may be circumvented by learning a generalized encoder that maps images into the 3D face model's latent space, another fundamental limitation remains. Even with more multi-view face datasets being published, the number of available training subjects rarely exceeds the thousands, making it hard to truly learn the full distibution of human facial appearance. Instead, our approach avoids generalizing over the identity axis by conditioning on some images of a person, and only generalizes over the expression axis for which plenty of data is available. 

A similar motivation has inspired recent work on codec avatars where a generalized network infers an animatable 3D representation given a registered mesh of a person~\cite{cao2022authentic, li2024uravatar}.
The resulting avatars exhibit excellent quality at the cost of several minutes of video capture per subject and expensive test-time optimization.
For example, URAvatar~\cite{li2024uravatar} finetunes their network on the given video recording for 3 hours on 8 A100 GPUs, making inference on consumer-grade devices impossible. In contrast, our approach directly regresses the final 3D head avatar from just four input images without the need for expensive test-time fine-tuning.





\section{Preliminary} \label{sec:preliminary}
In this section, we first introduce the notations in Section~\ref{sec:notations}. Then, we present the general problem formulation in Section~\ref{sec:problem_formulation}.

\subsection{Notations}\label{sec:notations}
For any positive integer $n$, we use $[n]$ to denote the set $\{1, 2, \ldots, n\}$. We use $\mathbb{N}_+$ to represent the set of all positive integers. For two sets $\mathcal{B}$ and $\mathcal{C}$, we denote the set difference as $\mathcal{B} \setminus \mathcal{C}:=\{x\in \mathcal{B}:x\notin\mathcal{C}\}$. For a vector $x \in \mathbb{R}^d$, $\Diag(d)$ denotes a diagonal matrix $X \in \mathbb{R}^{d \times d}$, where the diagonal entries satisfy $X_{i,i} = x_i$ for all $i \in [d]$, and all off-diagonal entries are zero. We use $\mathbf{1}_n$ to denote an $n$-dimensional column vector with all entries equal to one.


\begin{figure}[!ht]
    \centering
    \includegraphics[width=0.95\linewidth]{our_research.pdf}
    \caption{
    Our research objective. This figure presents the goal of our study: creating a more equitable desk-rejection system. Consider Professor A, who has carelessly submitted numerous papers exceeding the submission limit, collaborating with another senior researcher (Professor B) with many submissions, and a young student with only one paper. Our proposed system prioritizes desk-rejecting papers from authors with a large number of submissions first, thereby increasing the student’s chances of having their paper accepted. This approach aims to mitigate the disparity in the impact of desk rejections and promote fairness.
    }
    \label{fig:our_research}
\end{figure}

\subsection{Problem Formulation} \label{sec:problem_formulation}

In this section, we further introduce the actual problem we will investigate in this paper, where we begin with introducing the definition for three kinds of authors that will appear later in our discussion. 


\begin{definition}[Submission Limit Problem]\label{def:submit_limit_problem}
    Let $\mathcal{A} = \{a_1, a_2, \dots, a_n\}$ denote the set of $n$ authors, and let $\mathcal{P} = \{p_1, p_2, \dots, p_m\}$ denote the set of $m$ papers. Each author $a_i \in \mathcal{A}$ has a subset of papers $P_i \subseteq \mathcal{P}$, and each paper $p_j \in \mathcal{P}$ is authored by a subset of authors $A_j \subseteq \mathcal{A}$. For each author, $a_i \in \mathcal{A}$, let $C_i$ denote the set of all coauthors of $a_i$ and let $x \in \mathbb{N}_+$ denote the maximum number of papers each author can submit. 

    The goal is to find a subset $S \subseteq \mathcal{P}$ of papers (to keep) such that for every $a_i\in\mathcal{A},$
    \begin{align*}
        \underbrace{|\{p_j \in  S : a_i \in A_j\}|}_{\#\mathrm{remained~papers~of~author}~a_i} \leq x.  
    \end{align*}
    or equivalently find a subset $\ov{S} \subseteq \mathcal{P}$ of papers (to reject) such that for every $a_i \in \mathcal{A}$,
        \begin{align*}
        |P_i| - \underbrace{|\{j \in  \ov{S} : i \in A_j\}|}_{\#{\mathrm{rejected~papers~of~author}~}a_i} \leq x.  
    \end{align*}
\end{definition}

We now present several fundamental facts related to Definition~\ref{def:submit_limit_problem}, which can be easily verified through basic set theory. 
\begin{fact}
    For any author $a_i \in \mathcal{A}$ and paper $p_j \in \mathcal{P}$, $a_i \in A_j$ if and only if $p_j \in P_i$.
\end{fact}

\begin{fact}
    For each author $a_i \in \mathcal{A}$, the number of papers submitted by the author can be formulated as:
    \begin{align*}
        |P_i| = |\{p_j \in \mathcal{P} : a_i \in A_j\}|.
    \end{align*}
\end{fact}

\begin{fact}
    For each paper $j \in [m]$, the number of authors of this paper can be formulated as:
    \begin{align*}
        |A_j| = |\{a_i \in \mathcal{A} : p_j \in P_i\}|.
    \end{align*}
\end{fact}

\begin{fact}
    For each author $a_i \in \mathcal{A}$, the set of coauthors for author $a_i$ can be formulated as:
    \begin{align*}
        C_i = (\bigcup_{p_j \in P_i} A_j) \setminus \{ a_i \}.
    \end{align*}
\end{fact}



\section{Video Latent Flow Matching} \label{sec:vlfm}

In this section, we propose Video Latent Flow Matching (VLFM) in response to the main problem in Section~\ref{sub:problem_def}. Especially, we briefly review the HiPPO (high-order polynomial projection operators) framework \cite{gde+20} in Section~\ref{sub:hippo}. We state the derivation of our VLFM based on the popular flow matching approach \cite{lcb+22} in Section~\ref{sub:vlfm}. Finally, we define the training objective of the VLFM for efficient video modeling in Section~\ref{sub:training_objective}.


\subsection{HiPPO Framework and LegS State Space Model}\label{sub:hippo}

Given an input function $f(t) \in \R$ for $t \ge 0$, we use $f_{\leq t}$ to denote the cumulative history of $f(t)$ for every time $t \ge 0$. We choose integer $s \ge 1$ as the order of approximation. Then, any $s$-dimensional subspace ${\cal G}$ of this function space is a suitable candidate for the approximation. Given a time-varying measure family $p(t)$ supported on $(-\infty, t)$, a sequence of basis functions ${\cal G} = {\rm span}\{ g_{i}(t) \}_{i=1}^s$. HiPPO \cite{gde+20} defines an operator that maps $f$ to the optimal projection coefficients $c: \R_{\ge 0} \rightarrow \R^s$, such that:
\begin{align*}
    g(t) := & ~ \arg \min_{g \in {\cal G}} \| f_{\leq t} -  g \|_{p(t)}, \\
    g(t) = & ~ \sum_{i=1}^s c_i(t) \cdot g_i(t).
\end{align*}
We focus on the case where the coefficients $c(t)$ have the form of a linear ODE satisfying $\nabla c(t) = A(t) c(t) + B(t) f(t)$ for some $A(t) \in \R^{s \times s}$ and $B(t) \in \R^{s \times 1}$. This equation is now also known as the state space model (SSM) in many works \cite{kds+15,aia+22,gd23,dg24,zlz+24,xyy+24,mlw24,rx24,sld+24}.

{\bf Discrete HiPPO-LegS.} The setting of HiPPO-LegS defines the update rule of SSM and the discrete version of $A$ and $B$ matrices, which are $c_{\tau + 1} = (I_s - \frac{A}{\tau}) c_\tau + \frac{1}{\tau} B f_\tau$ and:
\begin{align*}
    A_{i_1, i_2} & ~ = \begin{cases}
        \sqrt{(2i_1 + 1)(2i_2 + 1)}, & \text{if $i_1 > i_2$} \\
        i_1 + 1, & \text{if $i_1 = i_2$} \\
        0, & \text{if $i_1 < i_2$}
    \end{cases}, \\
    B_{i_1} & ~ = \sqrt{2i_1 + 1}, \forall i_1, i_2 \in [s].
\end{align*}

\subsection{Conditional Video Latent Flow}\label{sub:vlfm}


Here we emphasize the core idea of VLFM is to approximate a continuous video distribution from limited discrete video frames data utilizing the optimal high-order polynomial approximation. 

Given a video-caption distribution ${\cal V}_c$, then for any video-caption data pair $(V, c) \sim {\cal V}_c$, we obtain the data $\wt{u}_\tau \in \R^d, \forall \tau \in [N]$ via Eq.~\eqref{eq:u_tau:informal}.
We aim to define a time-dependent flow $\psi_t(\wt{u})$ that takes inputs $\wt{u}$ and time $t$, and could match $\widehat{u}_\tau$ for all time $\tau \in [N]$. Since $\widehat{u}$ is discrete, HiPPO-LegS will be the best solution to approximate the continuous data. We define the \emph{Video Latent Flow} as:
\begin{align}\label{eq:psi}
    \psi_t(\wt{u}) := \sigma_t( \wt{u} ) \cdot z + \mu_t(\wt{u}) \in \R^d,
\end{align}
where $t \in [0,T]$ and $z \sim \mathcal{N}(0, I_d)$, $\sigma: [0, T] \times \R^{N \times d} \rightarrow \R_{> 0}$ denotes the time-dependent standard deviation, where $\sigma_0 ( \wt{u} ) = 1$, and $\sigma_{\frac{T}{N} \cdot \tau}( \wt{u} ) = \sigma_{\min}$, for all $\tau \in [N]$ ; $\mu: [0, T] \times \R^{N \times d} \rightarrow \R^d$ denotes the time-dependent mean of Gaussian distribution, where $\mu_0(\wt{u}) = {\bf 0}_d$, $\mu_{\frac{T}{N} \cdot \tau}(\wt{u}) = \wt{u}_\tau,$ for all $\tau \in [N]$.

Especially, we define:
\begin{align*}
    \mu_t(\wt{u} ) := & ~  H_N g(t), \\
    H_{\tau + 1} := & ~ H_{\tau} (I_s - \frac{1}{\tau} A)^\top + \frac{1}{\tau} \wt{u}_\tau B^\top,
\end{align*}
where $g(t) := [\sqrt{\frac{1}{2}} P_0(t), \sqrt{\frac{3}{2}} P_1(t), \cdots, \sqrt{\frac{2s-1}{2}} P_{s-1}(t)]^\top $ $\in \R^{s}$, $P_i(t), \forall i \in [s]$ is Legendre polynomials. We initialize $H_0 := {\bf 0}_{d \times s}$.

Besides, having a large scalar $\alpha > 0$, we give:
\begin{align*}
    \sigma_t(\wt{u}) := (1 - \sigma_{\min}) \cdot [\sin^2( \pi \frac{N}{T} t ) +  \exp(-\alpha t) ] + \sigma_{\min}.
\end{align*}

\subsection{Training Objective}\label{sub:training_objective}

Here we define a model function $F_\theta: \R^d \times \R^\ell \times [0, T] \rightarrow \R^d$ with parameters $\theta$ to learn the conditional video latent flow $\psi_t(\wt{u})$ defined in Eq.~\eqref{eq:psi}. This function takes inputs of flow and time to predict the vector field. The training objective is based on the Flow Matching framework \cite{lcb+22}, which aims to minimize the distance between the model's prediction and the true derivative of the flow.

The training objective of VLFM is defined as the expectation of the square $\ell_2$ norm of the difference, which is:
\begin{align*}
    {\cal L}(\theta) := \E_{z, t, (V, c)}[\| F_\theta( \psi_t(\wt{u}), c, t ) - \frac{\d }{\d t} \psi_t(\wt{u}) \|_2^2],
\end{align*}
where $z \sim \mathcal{N}(0, I_d)$,  $t \sim {\sf Uniform}[0, T]$ and $(V, c) \sim {\cal V}_c$. By minimizing this objective, the model learns to approximate the vector field that transports the initial noise distribution to the distribution of video latent patches. Formally, we solve: $\min_{\theta} {\cal L}(\theta)$. 

{\bf Close-form solution.} Furthermore, the close-form solution could be easily obtained as follows:
\begin{theorem}
    The minimum solution for function $F_\theta$ that takes $z \sim N(0, I_d)$ and $t \sim {\sf Uniform}[0, T]$ is:
    \begin{align*}
        F_\theta(z, c, t) = \frac{\sigma_t'(\wt{u})}{\sigma_t(\wt{u})} (z - \mu_t(\wt{u})) + \mu_t'(\wt{u}).
    \end{align*}
\end{theorem}

\begin{proof}
    This proof follows from Theorem 3 in \cite{lcb+22}.
\end{proof}



\section{Theory}\label{sec:theory}

This section provides several theoretical advantages of our VLFM. The approximation theory in this approach builds up based on using the Diffusion Transformer (DiT) \cite{px23}, which is a popular choice in previous empirical and theoretical part generative model works \cite{chzw23, hwsl24}, we briefly state its definitions in Section~\ref{sub:DiT}.

In addition, we provide the optimal polynomial projection guarantee and universal approximation theorem (with DiT) of VLFM in Section~\ref{sub:approx} to confirm its approximating ability. Besides, Section~\ref{sub:inter-extra_polation_theory} gives error bound of interpolation and extrapolation, and Section~\ref{sub:timescale_robustness} gives the supplementary property that VLFM's timescale robustness, which indicates its theoretical advantages.

\subsection{Diffusion Transformer (DiT)}\label{sub:DiT}

Diffusion Transformer \cite{px23} is a framework that utilizes Transformers \cite{vnn+17} as the backbone for Diffusion Models \cite{hja20,sme20}. Specifically, a Transformer block consists of a multi-head self-attention layer and a feed-forward layer, with both layers having a skip connection. 
We use ${\sf TF}^{h, m, r}: \R^{n \times d_0}\rightarrow \R^{n \times d_0}$ to denote a Transformer block.
Here $h$ and $m$ are the number of heads and head size in self-attention layer, and $r$ is the hidden dimension in feed-forward layer.
Let $X \in \R^{n \times d_0}$ be the model input. Then, we have the model output:
\ifdefined\isarxiv
\begin{align*}
    {\sf Attn}(X) := \sum_{i=1}^h {\sf Softmax}( X W_Q^i {W_K^i}^\top X^\top ) \cdot X W_V^i {W_O^i}^\top + X,
\end{align*}
\else
\begin{align*}
    & ~ {\sf Attn}(X) \\
    & ~ := \sum_{i=1}^h {\sf Softmax}( X W_Q^i {W_K^i}^\top X^\top ) \cdot X W_V^i {W_O^i}^\top + X,
\end{align*}
\fi
where the projection weights $W_K^i, W_Q^i, W_V^i, W_O^i \in \R^{d_0 \times m}$. Moreover,
\begin{align*}
    {\sf FF}(X) := \phi(X W_1 + {\bf 1}_n b_1^\top) \cdot W_2^\top + {\bf 1}_n b_2^\top + X.
\end{align*}
where  the projection weights $W_1, W_2 \in \R^{d_0 \times r}$, bias $b_1 \in \R^{r}, b_2 \in \R^{d_0}$, and $\phi$ is usually considered as the ReLU activated function.

In our work, we use Transformer networks with positional encoding $E\in\R^{n \times d_0}$. The transformer networks are then defined as the composition of Transformer blocks:
\begin{align*}
    {\cal T}_{P}^{h,m,r} = & ~ \{f_{{\cal T}}:\R^{ n \times d_0 }\rightarrow {\R^{n \times d_0}} \\
    & ~ \mid f_{{\cal T}}\text{ is a composition of blocks }{\sf TF}^{h,m,r}\text{'s}\}.
\end{align*}
For example, the following is a Transformer network consisting $K$ blocks and positional encoding
\begin{align*}
f_{{\cal T}}(X)= {\sf FF}^{(K)} \circ {\sf Attn}^{(K)} \circ  \cdots {\sf FF}^{(1)} \circ  {\sf Attn}^{(1)} (X+E).
\end{align*}

\subsection{Approximation via DiT}\label{sub:approx}



Before we state the approximation theorem, we define a reshaped layer that transforms concatenated input in flow matching into a length-fixed sequence of vectors. It is denoted as $R: \R^{d+\ell+1} \rightarrow \R^{n \times d_0}$. Therefore, in the following, we give the theorem utilizing DiT to minimize training objective ${\cal L}(\theta)$ to arbitrary error.

\begin{theorem}[Informal version of Theorem~\ref{thm:uat}]\label{thm:uat:informal}
    There exists a transformer network $f_{\cal T} \in {\cal T}_{P}^{2, 1, 4}$ defining function $F_\theta(z, c, t) := f_{\cal T}( R([z^\top, c^\top, t]^\top) )$ with parameters $\theta$ that satisfies ${\cal L}(\theta) \leq \epsilon$ for any error $\epsilon > 0$. 
\end{theorem}

\begin{proof}[Proof sketch of Theorem~\ref{thm:uat:informal}]
    Please refer to the proof of Theorem~\ref{thm:uat} for the detailed analysis.
\end{proof}


\subsection{Interpolation and Extrapolation}\label{sub:inter-extra_polation_theory}

Now, we theoretically discuss the approximating error of our VLFM in processing interpolation and extrapolation. It is considered a recovery of the original idea data from limited sub-sampled observations. This analysis is achieved by splitting the error into three parts, which are: 1) approximating error $\epsilon_1$ for HiPPO-LegS approximating the original data; 2) Gaussian error $\epsilon_2$ for the boundary of Gaussian vector $z$; 3) interpolation and extrapolation error $\epsilon_3$ that represents the training and predicting the difference between using original idea data $V$ and limited sub-sampled observations $\Phi \wt{V}$. We state the results as follows:
\begin{lemma}[Informal version of Lemma~\ref{lem:hippo_error}]\label{lem:hippo_error:informal}
    Denote failure probability $\delta \in (0, 0.1)$. Let the flow $\psi_t( \wt{u} )$ defined in Eq.~\eqref{eq:psi}. Denote $G := [g(\Delta t), g(2 \Delta t), \cdots, g(T)]^\top \in \R^{\frac{T}{\Delta t} \times s}$ and $\lambda^* := \lambda_{\min}(G) > 0$ as the minimum eigenvalue of $G$. Choosing $s = O(\frac{\Delta t}{T}\log((\frac{\Delta t}{T})^{1.5}\lambda^*))$. Denote $u_t = {\cal D}( V_{t} )$ for any $t \in [0, T]$. Especially, we define:
    \begin{itemize}
        \item Approximating error $\epsilon_1 := O(T^{k} s^{-k+1/2})$.
        \item Gaussian error $\epsilon_2 := O(\sqrt{d\log(d/\delta)})$.
        \item Interpolation and extrapolation error $\epsilon_3 := U d^{0.5} \sqrt{\frac{T}{\Delta t} - N} \cdot \exp(O(\frac{T}{\Delta t}s)) / \lambda^*$.
    \end{itemize}
    Then with a probability at least $1 - \delta$, we have:
    \begin{align*}
        \| \psi_t( \wt{u} ) - u_t \|_2 \leq \epsilon_1 + \epsilon_2 + \epsilon_3.
    \end{align*}
\end{lemma}

\begin{proof}{Proof sketch of Lemma~\ref{lem:hippo_error:informal}}
    This proof follows from its formal version in Lemma~\ref{lem:hippo_error}
\end{proof}

Having Lemma~\ref{lem:hippo_error:informal}, the concise bound for solving Eq.~\eqref{eq:main} could be given below:
\begin{theorem}[Informal version of Theorem~\ref{thm:inter_extra_polation}]\label{thm:inter_extra_polation:informal}
    Following Theorem~\ref{thm:uat:informal}, denote failure probability $\delta \in (0, 0.1)$ and arbitrary error $\epsilon_0 > 0$. Then with a probability at least $1 - \delta$, the network in Theorem~\ref{thm:uat:informal} satisfies Eq.~\eqref{eq:main} with $p = 2$ and
    \begin{align*}
        \epsilon = \epsilon_0 + L_0(\epsilon_1 + \epsilon_2 + \epsilon_3).
    \end{align*}
\end{theorem}

\begin{proof}[Proof sketch of Theorem~\ref{thm:inter_extra_polation:informal}]
    Please refer to Theorem~\ref{thm:inter_extra_polation} for complete proofs.
\end{proof}

{\bf Discussions.} Following the results of Lemma~\ref{lem:hippo_error:informal} and Theorem~\ref{thm:inter_extra_polation:informal}, we thus derive few insights as follows:
\begin{itemize}
    \item {\bf Optimal choice of $s$: A trade-off between $\epsilon_1$ and $\epsilon_3$. } As shown in the conditions of Lemma~\ref{lem:hippo_error:informal}, the larger value of the order of polynomials $s$ helps to decrease approximating error in the training dataset while also ruining the generalization ability.
    \ifdefined\isarxiv
    \else
    \vspace{-2mm}
    \fi
    \item {\bf Stable visual decoder. } Theorem~\ref{thm:inter_extra_polation:informal} shows a small value of $L_0$ (the stability and smoothness of visual decoder), which is important for the error of interpolation and extrapolation with an arbitrary frame rate.
    \ifdefined\isarxiv
    \else
    \vspace{-3mm}
    \fi
    \item {\bf Information. } Besides, a sub-linear factor $\sqrt{\frac{T}{\Delta t} - N}$, which stands for the obtained information about the continuous video, is vital as well for interpolation and extrapolation on data in distribution.
\end{itemize}
\ifdefined\isarxiv
\else
\vspace{-6mm}
\fi

\subsection{Timescale Robustness}\label{sub:timescale_robustness}

Following \cite{gde+20}, we demonstrate that projection onto latent patches $u_t$ is robust to timescales. Formally, the HiPPO-LegS operator is {\it timescale-equivariant}: dilating the input $u$ does not change the approximation coefficients $H_N$. At the same time, this property is working in the case of the discretized form $\wt{u}$. We emphasize that it is crucial to use flow matching to model the latent patches, where whatever the sampling method and frame rate are, it will not greatly harm VLFM's performance. We give its formal statement below.

\begin{lemma}[Proposition 3 of \cite{gde+20}, informal version of Lemma~\ref{lem:timescale_robustness}]\label{lem:timescale_robustness:informal}
    For any integer scale factor $\beta > 0$, the frames of video $\wt{V}_\tau$ is scaled to $\wt{V}_{\beta \tau}$ for each $\tau \in [\frac{T}{\Delta t}]$, it doesn’t affect the result of $H_N$.
\end{lemma}
\ifdefined\isarxiv
\else
\vspace{-3mm}
\fi
\begin{proof}
    This lemma follows from Proposition 3 in \cite{gde+20}.
\end{proof}
\ifdefined\isarxiv
\else
\vspace{-6mm}
\fi

\begin{figure*}[!ht]
\begin{center}
\centering
    \subfloat[{\it Video caption: A green turtle swimming under the sea.}]{
    \includegraphics[width=0.95\textwidth]{gen/turtle}} \\
    \subfloat[{\it Video caption: Viewing countless sunflowers in a field from top.}]{
    \includegraphics[width=0.95\textwidth]{gen/sunflower}}
\end{center}
\caption{Generated videos with different frame rates $\{8, 12, 16\}$. }
\label{fig:gen}
\ifdefined\isarxiv
\else
\vspace{-3mm}
\fi
\end{figure*}

\begin{figure*}[!ht]
\begin{center}
\centering
    \subfloat{
    \includegraphics[width=0.95\textwidth]{inter/lion}} \\
    \subfloat{
    \includegraphics[width=0.95\textwidth]{inter/aurora}} \\
    \subfloat{
    \includegraphics[width=0.95\textwidth]{extra/cloud}} \\
\end{center}
\caption{Interpolation and Extrapolation of VLFM.}
\label{fig:inter_extra}
\ifdefined\isarxiv
\else
\vspace{-2mm}
\fi
\end{figure*}

\section{Experiments}\label{sec:exp}

In this section, we conduct experiments to evaluate the effectiveness of our approach. We first introduce our experimental setups in Section~\ref{sub:exp_setup}. Then, we demonstrate text-to-video generation using VLFM and VLFM's capability of generating videos in arbitrary frame rate in Section~\ref{sub:exp_gen}. Furthermore, we showcase the strong performance of interpolation and extrapolation of VLFM in Section~\ref{sub:exp_inter_extra}. We also perform an ablation study to discuss the importance of the flow matching algorithm in Section~\ref{sub:exp_ablation}.

\subsection{Setup} \label{sub:exp_setup}

In our experiments, we apply Stable Diffusion v1.5 \cite{rbl+22} with DDIM scheduler \cite{sme20} as the visual decoder. Then, we use a DiT-XL-2 \cite{px23} as the backbone for the Flow Matching algorithm \cite{lcb+22,lgl22}, and the choice of hyper-parameters of $\sigma_t(\wt{u})$ is given by $\sigma_{\rm min} = 0.01$ and $\alpha = 10$. We optimize the DiT using Grams optimizer \cite{cls24}. We sample and combine 7 data resources for comprehensive training and validation of our method. They are:
OpenVid-1M \cite{nxz+24},
UCF-101 \cite{szs12},
Kinetics-400 \cite{kcs+17},
YouTube-8M \cite{akl+16},
InternVid \cite{whl+23},
MiraData \cite{jgz+24}, and
Pixabay \cite{pixabay}. 

\ifdefined\isarxiv
\else
\vspace{-4mm}
\fi

\subsection{Text-to-Video Generation with Arbitrary Frame Rate} \label{sub:exp_gen}

In this section, we recover several videos with different frame rates using VLFM with given video captions in the training dataset. We extract $T= 0.5$ for demonstrations as Figure~\ref{fig:gen}. In detail, we choose three frame rates for generation $\{8, 12, 16\}$. As shown, our VLFM performs fairly on text-to-video generation while it requires very small resource that is equivalent to training a new flow matching text-to-image video, which ensures its efficiency. Moreover, we give more results that are generated by VLFM in Appendix~\ref{sec:app:more_1} and \ref{sec:app:more_2}.
\ifdefined\isarxiv
\else
\vspace{-3mm}
\fi

\subsection{Interpolation and Extrapolation} \label{sub:exp_inter_extra}

In this section, we test the interpolation and extrapolation of VLFM. For the interpolation experiment, the model is trained with 24 FPS and evaluated to generate video with 48 FPS. For the extrapolation, the model is trained with the first video with $T = 2$ and evaluated to generate the whole video with $T = 8$. Referring the results in Figure~\ref{fig:inter_extra}, this demonstrates the strong performance of our VLFM under our mathematical guarantee of the error bound and its effectiveness.

\subsection{Ablation Study} \label{sub:exp_ablation}

In this section, we compared training VLFM with the Flow Matching algorithm and directly used DiT to predict the latent patches to showcase the importance of utilizing flow matching in our VLFM. We compare VLFM with and without flow matching by training the model with 1000 steps and compare the PSNR (peak signal-to-noise ratio) before and after training for video recovery with given captions in the training dataset. We state the results in Table~\ref{tab:ablation}. Denote ${\rm MSE}(x,y)$ as the mean squared error function, the computation of the metric PSNR is given by ($x,y \in \R^{r\times r}$):
\ifdefined\isarxiv
\else
\vspace{-3mm}
\fi
\begin{align*}
    {\rm PSNR}(x,y) := 10 \log_{10}(\frac{r^2}{{\rm MSE}(x,y)}), 
\end{align*}
\ifdefined\isarxiv
\else
\vspace{-3mm}
\fi

\begin{table}[!ht]
\ifdefined\isarxiv
\else
\vspace{-2mm}
\fi
\begin{center}
\begin{small}
\begin{sc}
\begin{tabular}{r | c c}
    \toprule
    Algorithm & Initial PSNR$\uparrow$ & Final PSNR$\uparrow$ \\
    \midrule
    Flow Matching & {\bf 57.20} & {\bf 61.18} \\
    Direct Predicting & 9.81 & 53.77 \\
    \bottomrule
\end{tabular}
\end{sc}
\end{small}
\end{center}
\caption{PSNR comparison (the greater, the better) of Flow Matching and direct generation from DiT. We boldface the better scores.}
\label{tab:ablation}
\ifdefined\isarxiv
\else
\vspace{-4mm}
\fi

\end{table}

\section{Discussion}\label{sec:discussion}



\subsection{From Interactive Prompting to Interactive Multi-modal Prompting}
The rapid advancements of large pre-trained generative models including large language models and text-to-image generation models, have inspired many HCI researchers to develop interactive tools to support users in crafting appropriate prompts.
% Studies on this topic in last two years' HCI conferences are predominantly focused on helping users refine single-modality textual prompts.
Many previous studies are focused on helping users refine single-modality textual prompts.
However, for many real-world applications concerning data beyond text modality, such as multi-modal AI and embodied intelligence, information from other modalities is essential in constructing sophisticated multi-modal prompts that fully convey users' instruction.
This demand inspires some researchers to develop multimodal prompting interactions to facilitate generation tasks ranging from visual modality image generation~\cite{wang2024promptcharm, promptpaint} to textual modality story generation~\cite{chung2022tale}.
% Some previous studies contributed relevant findings on this topic. 
Specifically, for the image generation task, recent studies have contributed some relevant findings on multi-modal prompting.
For example, PromptCharm~\cite{wang2024promptcharm} discovers the importance of multimodal feedback in refining initial text-based prompting in diffusion models.
However, the multi-modal interactions in PromptCharm are mainly focused on the feedback empowered the inpainting function, instead of supporting initial multimodal sketch-prompt control. 

\begin{figure*}[t]
    \centering
    \includegraphics[width=0.9\textwidth]{src/img/novice_expert.pdf}
    \vspace{-2mm}
    \caption{The comparison between novice and expert participants in painting reveals that experts produce more accurate and fine-grained sketches, resulting in closer alignment with reference images in close-ended tasks. Conversely, in open-ended tasks, expert fine-grained strokes fail to generate precise results due to \tool's lack of control at the thin stroke level.}
    \Description{The comparison between novice and expert participants in painting reveals that experts produce more accurate and fine-grained sketches, resulting in closer alignment with reference images in close-ended tasks. Novice users create rougher sketches with less accuracy in shape. Conversely, in open-ended tasks, expert fine-grained strokes fail to generate precise results due to \tool's lack of control at the thin stroke level, while novice users' broader strokes yield results more aligned with their sketches.}
    \label{fig:novice_expert}
    % \vspace{-3mm}
\end{figure*}


% In particular, in the initial control input, users are unable to explicitly specify multi-modal generation intents.
In another example, PromptPaint~\cite{promptpaint} stresses the importance of paint-medium-like interactions and introduces Prompt stencil functions that allow users to perform fine-grained controls with localized image generation. 
However, insufficient spatial control (\eg, PromptPaint only allows for single-object prompt stencil at a time) and unstable models can still leave some users feeling the uncertainty of AI and a varying degree of ownership of the generated artwork~\cite{promptpaint}.
% As a result, the gap between intuitive multi-modal or paint-medium-like control and the current prompting interface still exists, which requires further research on multi-modal prompting interactions.
From this perspective, our work seeks to further enhance multi-object spatial-semantic prompting control by users' natural sketching.
However, there are still some challenges to be resolved, such as consistent multi-object generation in multiple rounds to increase stability and improved understanding of user sketches.   


% \new{
% From this perspective, our work is a step forward in this direction by allowing multi-object spatial-semantic prompting control by users' natural sketching, which considers the interplay between multiple sketch regions.
% % To further advance the multi-modal prompting experience, there are some aspects we identify to be important.
% % One of the important aspects is enhancing the consistency and stability of multiple rounds of generation to reduce the uncertainty and loss of control on users' part.
% % For this purpose, we need to develop techniques to incorporate consistent generation~\cite{tewel2024training} into multi-modal prompting framework.}
% % Another important aspect is improving generative models' understanding of the implicit user intents \new{implied by the paint-medium-like or sketch-based input (\eg, sketch of two people with their hands slightly overlapping indicates holding hand without needing explicit prompt).
% % This can facilitate more natural control and alleviate users' effort in tuning the textual prompt.
% % In addition, it can increase users' sense of ownership as the generated results can be more aligned with their sketching intents.
% }
% For example, when users draw sketches of two people with their hands slightly overlapping, current region-based models cannot automatically infer users' implicit intention that the two people are holding hands.
% Instead, they still require users to explicitly specify in the prompt such relationship.
% \tool addresses this through sketch-aware prompt recommendation to fill in the necessary semantic information, alleviating users' workload.
% However, some users want the generative AI in the future to be able to directly infer this natural implicit intentions from the sketches without additional prompting since prompt recommendation can still be unstable sometimes.


% \new{
% Besides visual generation, 
% }
% For example, one of the important aspect is referring~\cite{he2024multi}, linking specific text semantics with specific spatial object, which is partly what we do in our sketch-aware prompt recommendation.
% Analogously, in natural communication between humans, text or audio alone often cannot suffice in expressing the speakers' intentions, and speakers often need to refer to an existing spatial object or draw out an illustration of her ideas for better explanation.
% Philosophically, we HCI researchers are mostly concerned about the human-end experience in human-AI communications.
% However, studies on prompting is unique in that we should not just care about the human-end interaction, but also make sure that AI can really get what the human means and produce intention-aligned output.
% Such consideration can drastically impact the design of prompting interactions in human-AI collaboration applications.
% On this note, although studies on multi-modal interactions is a well-established topic in HCI community, it remains a challenging problem what kind of multi-modal information is really effective in helping humans convey their ideas to current and next generation large AI models.




\subsection{Novice Performance vs. Expert Performance}\label{sec:nVe}
In this section we discuss the performance difference between novice and expert regarding experience in painting and prompting.
First, regarding painting skills, some participants with experience (4/12) preferred to draw accurate and fine-grained shapes at the beginning. 
All novice users (5/12) draw rough and less accurate shapes, while some participants with basic painting skills (3/12) also favored sketching rough areas of objects, as exemplified in Figure~\ref{fig:novice_expert}.
The experienced participants using fine-grained strokes (4/12, none of whom were experienced in prompting) achieved higher IoU scores (0.557) in the close-ended task (0.535) when using \tool. 
This is because their sketches were closer in shape and location to the reference, making the single object decomposition result more accurate.
Also, experienced participants are better at arranging spatial location and size of objects than novice participants.
However, some experienced participants (3/12) have mentioned that the fine-grained stroke sometimes makes them frustrated.
As P1's comment for his result in open-ended task: "\emph{It seems it cannot understand thin strokes; even if the shape is accurate, it can only generate content roughly around the area, especially when there is overlapping.}" 
This suggests that while \tool\ provides rough control to produce reasonably fine results from less accurate sketches for novice users, it may disappoint experienced users seeking more precise control through finer strokes. 
As shown in the last column in Figure~\ref{fig:novice_expert}, the dragon hovering in the sky was wrongly turned into a standing large dragon by \tool.

Second, regarding prompting skills, 3 out of 12 participants had one or more years of experience in T2I prompting. These participants used more modifiers than others during both T2I and R2I tasks.
Their performance in the T2I (0.335) and R2I (0.469) tasks showed higher scores than the average T2I (0.314) and R2I (0.418), but there was no performance improvement with \tool\ between their results (0.508) and the overall average score (0.528). 
This indicates that \tool\ can assist novice users in prompting, enabling them to produce satisfactory images similar to those created by users with prompting expertise.



\subsection{Applicability of \tool}
The feedback from user study highlighted several potential applications for our system. 
Three participants (P2, P6, P8) mentioned its possible use in commercial advertising design, emphasizing the importance of controllability for such work. 
They noted that the system's flexibility allows designers to quickly experiment with different settings.
Some participants (N = 3) also mentioned its potential for digital asset creation, particularly for game asset design. 
P7, a game mod developer, found the system highly useful for mod development. 
He explained: "\emph{Mods often require a series of images with a consistent theme and specific spatial requirements. 
For example, in a sacrifice scene, how the objects are arranged is closely tied to the mod's background. It would be difficult for a developer without professional skills, but with this system, it is possible to quickly construct such images}."
A few participants expressed similar thoughts regarding its use in scene construction, such as in film production. 
An interesting suggestion came from participant P4, who proposed its application in crime scene description. 
She pointed out that witnesses are often not skilled artists, and typically describe crime scenes verbally while someone else illustrates their account. 
With this system, witnesses could more easily express what they saw themselves, potentially producing depictions closer to the real events. "\emph{Details like object locations and distances from buildings can be easily conveyed using the system}," she added.

% \subsection{Model Understanding of Users' Implicit Intents}
% In region-sketch-based control of generative models, a significant gap between interaction design and actual implementation is the model's failure in understanding users' naturally expressed intentions.
% For example, when users draw sketches of two people with their hands slightly overlapping, current region-based models cannot automatically infer users' implicit intention that the two people are holding hands.
% Instead, they still require users to explicitly specify in the prompt such relationship.
% \tool addresses this through sketch-aware prompt recommendation to fill in the necessary semantic information, alleviating users' workload.
% However, some users want the generative AI in the future to be able to directly infer this natural implicit intentions from the sketches without additional prompting since prompt recommendation can still be unstable sometimes.
% This problem reflects a more general dilemma, which ubiquitously exists in all forms of conditioned control for generative models such as canny or scribble control.
% This is because all the control models are trained on pairs of explicit control signal and target image, which is lacking further interpretation or customization of the user intentions behind the seemingly straightforward input.
% For another example, the generative models cannot understand what abstraction level the user has in mind for her personal scribbles.
% Such problems leave more challenges to be addressed by future human-AI co-creation research.
% One possible direction is fine-tuning the conditioned models on individual user's conditioned control data to provide more customized interpretation. 

% \subsection{Balance between recommendation and autonomy}
% AIGC tools are a typical example of 
\subsection{Progressive Sketching}
Currently \tool is mainly aimed at novice users who are only capable of creating very rough sketches by themselves.
However, more accomplished painters or even professional artists typically have a coarse-to-fine creative process. 
Such a process is most evident in painting styles like traditional oil painting or digital impasto painting, where artists first quickly lay down large color patches to outline the most primitive proportion and structure of visual elements.
After that, the artists will progressively add layers of finer color strokes to the canvas to gradually refine the painting to an exquisite piece of artwork.
One participant in our user study (P1) , as a professional painter, has mentioned a similar point "\emph{
I think it is useful for laying out the big picture, give some inspirations for the initial drawing stage}."
Therefore, rough sketch also plays a part in the professional artists' creation process, yet it is more challenging to integrate AI into this more complex coarse-to-fine procedure.
Particularly, artists would like to preserve some of their finer strokes in later progression, not just the shape of the initial sketch.
In addition, instead of requiring the tool to generate a finished piece of artwork, some artists may prefer a model that can generate another more accurate sketch based on the initial one, and leave the final coloring and refining to the artists themselves.
To accommodate these diverse progressive sketching requirements, a more advanced sketch-based AI-assisted creation tool should be developed that can seamlessly enable artist intervention at any stage of the sketch and maximally preserve their creative intents to the finest level. 

\subsection{Ethical Issues}
Intellectual property and unethical misuse are two potential ethical concerns of AI-assisted creative tools, particularly those targeting novice users.
In terms of intellectual property, \tool hands over to novice users more control, giving them a higher sense of ownership of the creation.
However, the question still remains: how much contribution from the user's part constitutes full authorship of the artwork?
As \tool still relies on backbone generative models which may be trained on uncopyrighted data largely responsible for turning the sketch into finished artwork, we should design some mechanisms to circumvent this risk.
For example, we can allow artists to upload backbone models trained on their own artworks to integrate with our sketch control.
Regarding unethical misuse, \tool makes fine-grained spatial control more accessible to novice users, who may maliciously generate inappropriate content such as more realistic deepfake with specific postures they want or other explicit content.
To address this issue, we plan to incorporate a more sophisticated filtering mechanism that can detect and screen unethical content with more complex spatial-semantic conditions. 
% In the future, we plan to enable artists to upload their own style model

% \subsection{From interactive prompting to interactive spatial prompting}


\subsection{Limitations and Future work}

    \textbf{User Study Design}. Our open-ended task assesses the usability of \tool's system features in general use cases. To further examine aspects such as creativity and controllability across different methods, the open-ended task could be improved by incorporating baselines to provide more insightful comparative analysis. 
    Besides, in close-ended tasks, while the fixing order of tool usage prevents prior knowledge leakage, it might introduce learning effects. In our study, we include practice sessions for the three systems before the formal task to mitigate these effects. In the future, utilizing parallel tests (\textit{e.g.} different content with the same difficulty) or adding a control group could further reduce the learning effects.

    \textbf{Failure Cases}. There are certain failure cases with \tool that can limit its usability. 
    Firstly, when there are three or more objects with similar semantics, objects may still be missing despite prompt recommendations. 
    Secondly, if an object's stroke is thin, \tool may incorrectly interpret it as a full area, as demonstrated in the expert results of the open-ended task in Figure~\ref{fig:novice_expert}. 
    Finally, sometimes inclusion relationships (\textit{e.g.} inside) between objects cannot be generated correctly, partially due to biases in the base model that lack training samples with such relationship. 

    \textbf{More support for single object adjustment}.
    Participants (N=4) suggested that additional control features should be introduced, beyond just adjusting size and location. They noted that when objects overlap, they cannot freely control which object appears on top or which should be covered, and overlapping areas are currently not allowed.
    They proposed adding features such as layer control and depth control within the single-object mask manipulation. Currently, the system assigns layers based on color order, but future versions should allow users to adjust the layer of each object freely, while considering weighted prompts for overlapping areas.

    \textbf{More customized generation ability}.
    Our current system is built around a single model $ColorfulXL-Lightning$, which limits its ability to fully support the diverse creative needs of users. Feedback from participants has indicated a strong desire for more flexibility in style and personalization, such as integrating fine-tuned models that cater to specific artistic styles or individual preferences. 
    This limitation restricts the ability to adapt to varied creative intents across different users and contexts.
    In future iterations, we plan to address this by embedding a model selection feature, allowing users to choose from a variety of pre-trained or custom fine-tuned models that better align with their stylistic preferences. 
    
    \textbf{Integrate other model functions}.
    Our current system is compatible with many existing tools, such as Promptist~\cite{hao2024optimizing} and Magic Prompt, allowing users to iteratively generate prompts for single objects. However, the integration of these functions is somewhat limited in scope, and users may benefit from a broader range of interactive options, especially for more complex generation tasks. Additionally, for multimodal large models, users can currently explore using affordable or open-source models like Qwen2-VL~\cite{qwen} and InternVL2-Llama3~\cite{llama}, which have demonstrated solid inference performance in our tests. While GPT-4o remains a leading choice, alternative models also offer competitive results.
    Moving forward, we aim to integrate more multimodal large models into the system, giving users the flexibility to choose the models that best fit their needs. 
    


\section{Conclusion}\label{sec:conclusion}
In this paper, we present \tool, an interactive system designed to help novice users create high-quality, fine-grained images that align with their intentions based on rough sketches. 
The system first refines the user's initial prompt into a complete and coherent one that matches the rough sketch, ensuring the generated results are both stable, coherent and high quality.
To further support users in achieving fine-grained alignment between the generated image and their creative intent without requiring professional skills, we introduce a decompose-and-recompose strategy. 
This allows users to select desired, refined object shapes for individual decomposed objects and then recombine them, providing flexible mask manipulation for precise spatial control.
The framework operates through a coarse-to-fine process, enabling iterative and fine-grained control that is not possible with traditional end-to-end generation methods. 
Our user study demonstrates that \tool offers novice users enhanced flexibility in control and fine-grained alignment between their intentions and the generated images.


\ifdefined \isarxiv
\else
\section*{Impact Statement}
This paper presents work whose goal is to advance the field of Machine Learning. There are many potential societal consequences of our work, none of which we feel must be specifically highlighted here.
\section*{Acknowledgments}
\bibliography{ref}
%\bibliographystyle{icml2022}
\bibliographystyle{icml2025}
\fi



\newpage
\onecolumn
\appendix


\begin{center}
	\textbf{\LARGE Appendix }
\end{center}


In the appendix, we present more experimental text-to-video generation results in Appendix~\ref{sec:app:more_1} and more interpolation and extrapolation results in Appendix~\ref{sec:app:more_2}. Then we introduce the preliminary in Appendix~\ref{sec:app:preli}. Next, we illustrate Video Latent Flow Matching formally in Appendix~\ref{sec:app:vlfm}. In Appendix~\ref{sec:app:dit}, we demonstrate the Diffusion Transformer, and finally, in Appendix~\ref{sec:app:inter_extra}, we present the interpolation and extrapolation of VLFM.

\section{More Text-to-Video Generation Results} \label{sec:app:more_1}

We give more text-to-video generation results with different frame rates to demonstrate the generative ability of our VLFM in Figure~\ref{fig:gen_addtional1} and Figure~\ref{fig:gen_addtional2}.

\begin{figure*}[!ht]
\begin{center}
\centering
    \subfloat[{\it Video caption: Venus spinning in the space.}]{
    \includegraphics[width=0.95\textwidth]{gen/venus}} \\
    \subfloat[{\it Video caption: Steam is coming out of a pot.}]{
    \includegraphics[width=0.95\textwidth]{gen/pot}} 
\end{center}
\caption{Generated videos with different frame rates $\{8, 12, 16\}$. }
\label{fig:gen_addtional1}
\end{figure*}

\begin{figure*}[!ht]
\begin{center}
\centering
    \subfloat[{\it Video caption: Flame flickers on the candles.}]{
    \includegraphics[width=0.95\textwidth]{gen/candle}} \\
    \subfloat[{\it Video caption: A train is running through the rail road near the coast.}]{
    \includegraphics[width=0.95\textwidth]{gen/train}}
\end{center}
\caption{Generated videos with different frame rates $\{8, 12, 16\}$. }
\label{fig:gen_addtional2}
\end{figure*}

\section{More Interpolation and Extrapolation Results} \label{sec:app:more_2}

We give more results of interpolation and extrapolation of VLFM in Figure~\ref{fig:more_inter_extra}.

\begin{figure*}[!ht]
\begin{center}
\centering
    \subfloat{
    \includegraphics[width=0.95\textwidth]{inter/turbine}} \\
    \subfloat{
    \includegraphics[width=0.95\textwidth]{extra/wave}} \\
    \subfloat{
    \includegraphics[width=0.95\textwidth]{extra/starry}}
\end{center}
\caption{Interpolation and Extrapolation of VLFM.}
\label{fig:more_inter_extra}
\end{figure*}

\section{Preliminary} \label{sec:app:preli}

In the preliminary section, we first introduce our notation in the appendix in Appendix~\ref{sub:app:notations}.  Then, in Appendix~\ref{sub:app:video}, we formally define the video-caption data and visual decoder. In Appendix~\ref{sub:app:latent_patches}, we define the latent patches. Appendix~\ref{sub:app:assumption} makes some assumptions which we will use later. Finally, in Appendix~\ref{sec:app:facts}, we list some basic useful facts.

\subsection{Notations} \label{sub:app:notations}

\paragraph{Notations.} We use $D$ to denote the flattened dimension of real-world images. We use $d$ to represent the dimension of latent patches. We introduce $d_0$ as the dimension of Diffusion Transformers. We utilize $V: [0, T] \rightarrow \R^D$ to denote a video with $T$ duration, where $T$ is the longest time for each video. We omit $\nabla_t a(t)$ and $a'(t)$ to denote taking differentiation to some function $a(t)$ w.r.t. time $t$. We use integer $s$ to denote the order of polynomials. The dimensional number of the text embedding vector is given by integer $\ell$.

\subsection{Video-Caption Data} \label{sub:app:video}

\begin{definition}[Video-caption data pairs and their distribution]\label{def:V_c}
    We define a video caption distribution $(V, c) \sim {\cal V}_c$. Here, $V: [0, T] \rightarrow \R^D$ is considered as a function and $c \in \R^\ell$ is the corresponding text embedding vector.
\end{definition}

\begin{definition}\label{def:wt_V}
    Given a video caption distribution ${\cal V}_c$ as Definition~\ref{def:V_c}. We denote $\Delta t$ as the minimal time unit of measurement in the real world (Planck time). For any $(V, c) \sim {\cal V}_c$, we define the discretized form of $V: [0, T] \rightarrow \R^D$, which is $\wt{V} \in \R^{\frac{T}{\Delta t} \times D}$, and its $\tau$-th row $ \forall \tau \in [\frac{T}{\Delta t}]$ is given by:
    \begin{align*}
        \wt{V}_\tau := V_{\Delta t \cdot \tau} \in \R^D.
    \end{align*}
\end{definition}

\begin{definition}[Obtained data in real-world cases]\label{def:Phi}
    If the following conditions hold:
    \begin{itemize}
        \item Given a video caption distribution ${\cal V}_c$ as Definition~\ref{def:V_c}.
        \item For any $(V, c) \sim {\cal V}_c$, we define the discretized form of video $\wt{V}$ as Definition~\ref{def:wt_V}.
    \end{itemize}
    We define an observation matrix $\Phi: \{0, 1\}^{N \times \frac{T}{\Delta t}}$. The obtained data in real-world cases then is denoted as $\Phi \wt{V} \in \R^{N \times D}$.
\end{definition}

\begin{definition}[Bijective Visual Decoder]\label{def:visual_decoder}
    We define the visual decoder ${\cal D}: \R^d \rightarrow \R^D$ satisfies that:
    \begin{itemize}
        \item For any flattened image $V \in \R^D$, there is a unique $u \in \R^d$ such that ${\cal D}(u) = V$.
    \end{itemize}
    Then we say ${\cal D}$ is bijective. Denote the reverse function of ${\cal D}$ as ${\cal D}^{-1}: \R^D \rightarrow \R^d$.
\end{definition}

\subsection{Latent Patches Data} \label{sub:app:latent_patches}

\begin{definition}\label{def:u}
    If the following conditions hold:
    \begin{itemize}
        \item Given a video caption distribution ${\cal V}_c$ as Definition~\ref{def:V_c}.
        \item For any $(V, c) \sim {\cal V}_c$, we define the discretized form of video $\wt{V}$ as Definition~\ref{def:wt_V}.
        \item Let the observation matrix $\Phi: \{0, 1\}^{N \times \frac{T}{\Delta t}}$ be defined as Definition~\ref{def:Phi}.
        \item Let the visual decoder function $D: \R^d \rightarrow \R^D$ be defined as Definition~\ref{def:visual_decoder}.
    \end{itemize}
    We define the ideal version (without observation matrix) of the sequence of latent patches $u \in \R^{\frac{T}{\Delta t} \times d}$, and its $\tau$-th $ \forall \tau \in [\frac{T}{\Delta t}]$ row is defined as follows:
    \begin{align*}
        u_\tau := {\cal D}^{-1}( \wt{V}_\tau ).
    \end{align*}
\end{definition}

\begin{definition}\label{def:wt_u}
    If the following conditions hold:
    \begin{itemize}
        \item Given a video caption distribution ${\cal V}_c$ as Definition~\ref{def:V_c}.
        \item For any $(V, c) \sim {\cal V}_c$, we define the discretized form of video as Definition~\ref{def:wt_V}.
        \item Let the observation matrix $\Phi: \{0, 1\}^{N \times \frac{T}{\Delta t}}$ be defined as Definition~\ref{def:Phi}.
        \item Let the visual decoder function $D: \R^d \rightarrow \R^D$ be defined as Definition~\ref{def:visual_decoder}.
    \end{itemize}
    We define the real-world version (with observation matrix) of the sequence of latent patches $\wt{u} \in \R^{\frac{T}{\Delta t} \times d}$, and its $\tau$-th $ \forall \tau \in [N]$ row is defined as follows:
    \begin{align*}
        \wt{u}_\tau := {\cal D}^{-1}\Big( (\Phi V)_\tau \Big).
    \end{align*}
\end{definition}


\subsection{Assumptions} \label{sub:app:assumption}

\begin{assumption}\label{ass:k}
    If the following conditions hold:
    \begin{itemize}
        \item Given a video caption distribution ${\cal V}_c$ as Definition~\ref{def:V_c}.
        \item For any $(V, c) \sim {\cal V}_c$, we define the discretized form of video as Definition~\ref{def:wt_V}.
        \item Let the observation matrix $\Phi: \{0, 1\}^{N \times \frac{T}{\Delta t}}$ be defined as Definition~\ref{def:Phi}.
        \item Let the visual decoder function $D: \R^d \rightarrow \R^D$ be defined as Definition~\ref{def:visual_decoder}.
        \item Let the ideal version of the sequence of latent patches $u \in \R^{\frac{T}{\Delta t} \times d}$ be defined as Definition~\ref{def:u}.
    \end{itemize}
    We assume $u_\tau$ is $k$-differentiable, there exists:
    \begin{align*}
        u_{\tau}^{(i)} = \lim_{\Delta t \rightarrow 0} \frac{u_{\tau+1}^{(i-1)} - u_{\tau}^{(i-1)}}{ \Delta t }, \forall i \in [k], \tau \in [\frac{T}{\Delta t}],
    \end{align*}
    where, we use $u_\tau^{(i)}$ to denote the $i$-th derivation of $u$.
\end{assumption}

\begin{assumption}\label{ass:L_0}
    If the following conditions hold:
    \begin{itemize}
        \item Let the visual decoder function $D: \R^d \rightarrow \R^D$ be defined as Definition~\ref{def:visual_decoder}.
    \end{itemize}
    We assume the visual decoder function ${\cal D}$ is $L_0$-smooth for constant $L_0 > 0$, such that:
    \begin{align*}
        \| {\cal D}(x) - {\cal D}(y) \|_2 \leq L_0 \| x - y \|_2, \forall x, y \in \R^d.
    \end{align*}
\end{assumption}

\begin{assumption}\label{ass:U}
    If the following conditions hold:
    \begin{itemize}
        \item Given a video caption distribution ${\cal V}_c$ as Definition~\ref{def:V_c}.
        \item For any $(V, c) \sim {\cal V}_c$, we define the discretized form of video as Definition~\ref{def:wt_V}.
        \item Let the observation matrix $\Phi: \{0, 1\}^{N \times \frac{T}{\Delta t}}$ be defined as Definition~\ref{def:Phi}.
        \item Let the visual decoder function $D: \R^d \rightarrow \R^D$ be defined as Definition~\ref{def:visual_decoder}.
        \item Let the ideal version of the sequence of latent patches $u \in \R^{\frac{T}{\Delta t} \times d}$ be defined as Definition~\ref{def:u}.
    \end{itemize}
    We assume each entry in latent patches $u$ is bounded by a constant $U > 0$.
\end{assumption}

\begin{assumption}\label{ass:M}
    If the following conditions hold:
    \begin{itemize}
        \item Given a video caption distribution ${\cal V}_c$ as Definition~\ref{def:V_c}.
        \item For any $(V, c) \sim {\cal V}_c$
    \end{itemize}
    For any $(V, c) \sim {\cal V}_c$, we assume there exists a function ${\cal M}: [0, T] \times \R^\ell \rightarrow \R^D$ satisfies $V_t = {\cal M}_t(c)$. 
\end{assumption}

\subsection{Basic Facts} \label{sec:app:facts}

\begin{fact}\label{fac:gaussian_tail}
    For a variable $x \sim \mathcal{N}(0, \sigma^2)$, then with probability at least $1 - \delta$, we have:
    \begin{align*}
        |x| \leq C \sigma \sqrt{\log(1/\delta)}
    \end{align*}
\end{fact}

\begin{fact}\label{fac:infity_norm_pesdueo_inverse}
    For a PD matrix $A \in \R^{d_1 \times d_2}$ with a positive minimum eigenvalue $\lambda_{\min}(A) > 0$, the infinite norm of its pesdueo-inverse matrix $A^\dag$ is given by:
    \begin{align*}
        \| A^\dagger \|_\infty \leq \frac{1}{\lambda_{\min}(A)}.
    \end{align*}
\end{fact}

\begin{fact}\label{fac:pesdueo_inverse_diff}
    For two matrices $A , B \in\R^{d_1 \times d_2}$, we have:
    \begin{align*}
        \| A^\dagger - B^\dagger \|_2 \leq \frac{\| A^\dagger \|_2^2 \| A - B\|_2 }{1 - \| A^\dagger \|_2 \cdot 
        \| A - B \|_2}
    \end{align*}
\end{fact}

\section{Video Latent Flow Matching}
\label{sec:app:vlfm}

This section, we first introduce the HiPPO Framework and LegS in Appendix~\ref{sub:app:hippo}. Then, we formally define the video latent flow in Appendix~\ref{sub:app:vlf}. Last, we introduce the training objective of VLFM in Appendix~\ref{sub:app:train_obj}.

\subsection{HiPPO Framework and LegS} \label{sub:app:hippo}

\begin{definition}\label{def:A}
    We define matrix $A \in \R^{s \times s}$ where its $(i_1, i_2)$-th entry $\forall i_1, i_2 \in [s]$ is given by:
    \begin{align*}
        A_{i_1, i_2} & ~ = \begin{cases}
        \sqrt{(2i_1 + 1)(2i_2 + 1)}, & \text{if $i_1 > i_2$} \\
        i_1 + 1, & \text{if $i_1 = i_2$} \\
        0, & \text{if $i_1 < i_2$}
    \end{cases}.
    \end{align*}
\end{definition}

\begin{definition}\label{def:B}
    We define matrix $B \in \R^{s \times 1}$ where its $i_1$-th entry $\forall i_1 \in [s]$ is given by:
    \begin{align*}
        B_{i_1} & ~ = \sqrt{2i_1 + 1}.
    \end{align*}
\end{definition}

\begin{definition}\label{def:H}
    If the following conditions hold:
    \begin{itemize}
        \item Let matrix $A \in \R^{s \times s}$ be defined as Definition~\ref{def:A}.
        \item Let matrix $B \in \R^{s \times 1}$ be defined as Definition~\ref{def:B}.
    \end{itemize}
    We initialize a matrix $H_0 = {\bf 0}_{d \times s}$. Then we define:
    \begin{align*}
        H_{\tau} := H_{\tau-1}( I_s - \frac{1}{\tau} A )^\top + \frac{1}{\tau} \wt{u}_\tau B^\top, \forall \tau \in [N].
    \end{align*}
\end{definition}

\begin{definition}\label{def:g_t}
    We define $g(t) := [\sqrt{\frac{1}{2}} P_0(t), \sqrt{\frac{3}{2}} P_1(t), \cdots, \sqrt{\frac{2s-1}{2}} P_{s-1}(t)]^\top $ $\in \R^{s} $, where$ P_i(t), \forall i \in [s]$ is some polynomials. Especially, $g(t)$ satisfies:
    \begin{itemize}
        \item Define $G := \begin{bmatrix}
            g(\Delta t)^\top \\
            g(2 \Delta t)^\top \\
            \vdots \\
            g(T)^\top
        \end{bmatrix}$, $\lambda_{\min} (G) > 0$. Here, $\lambda_{\min}$ is the function that outputs the minimal eigenvalue of the input matrix.
        \item $|G_{\tau, i}| \leq \exp(O(\frac{T}{\Delta t}s))$ for any $\tau \in [\frac{T}{ \Delta t}], i \in [s]$.
    \end{itemize}
\end{definition}

\subsection{Video Latent Flow} \label{sub:app:vlf}

\begin{definition}\label{def:mu}
    If the following conditions hold:
    \begin{itemize}
        \item Given a video caption distribution ${\cal V}_c$ as Definition~\ref{def:V_c}.
        \item For any $(V, c) \sim {\cal V}_c$, we define the discretized form of video as Definition~\ref{def:wt_V}.
        \item Let the observation matrix $\Phi: \{0, 1\}^{N \times \frac{T}{\Delta t}}$ be defined as Definition~\ref{def:Phi}.
        \item Let the visual decoder function $D: \R^d \rightarrow \R^D$ be defined as Definition~\ref{def:visual_decoder}.
        \item Let the ideal version of the sequence of latent patches $u \in \R^{\frac{T}{\Delta t} \times d}$ be defined as Definition~\ref{def:u}.
        \item Let the real-world version of the sequence of latent patches $\wt{u} \in \R^{N \times d}$ be defined as Definition~\ref{def:wt_u}.
        \item Let $H_N \in \R^{d \times s}$ be defined as Definition~\ref{def:H}.
        \item Let the function of polynomials $g(t)$ be defined as Definition~\ref{def:g_t}.
    \end{itemize}
    We define the time-dependent mean of Gaussian distribution as follows:
    \begin{align*}
        \mu_t(\wt{u}) := H_N g(t) \in \R^d
    \end{align*}
\end{definition}

\begin{definition}\label{def:sigma}
    If the following conditions hold:
    \begin{itemize}
        \item Given a video caption distribution ${\cal V}_c$ as Definition~\ref{def:V_c}.
        \item For any $(V, c) \sim {\cal V}_c$, we define the discretized form of video as Definition~\ref{def:wt_V}.
        \item Let the observation matrix $\Phi: \{0, 1\}^{N \times \frac{T}{\Delta t}}$ be defined as Definition~\ref{def:Phi}.
        \item Let the visual decoder function $D: \R^d \rightarrow \R^D$ be defined as Definition~\ref{def:visual_decoder}.
        \item Let the ideal version of the sequence of latent patches $u \in \R^{\frac{T}{\Delta t} \times d}$ be defined as Definition~\ref{def:u}.
        \item Let the real-world version of the sequence of latent patches $\wt{u} \in \R^{N \times d}$ be defined as Definition~\ref{def:wt_u}.
        \item Let $H_N \in \R^{d \times s}$ be defined as Definition~\ref{def:H}.
        \item Let the function of polynomials $g(t)$ be defined as Definition~\ref{def:g_t}.
        \item Denote $\sigma_{\min} > 0$.
        \item Given a hyper-parameter $\alpha > 0$.
    \end{itemize}
    We define  the time-dependent standard deviation as follows:
    \begin{align*}
        \sigma_t(\wt{u}) := (1 - \sigma_{\min}) \cdot [\sin^2( \pi \frac{N}{T} t ) +  \exp(-\alpha t) ] + \sigma_{\min} \in \R_{\ge 0}.
    \end{align*}
\end{definition}

\begin{lemma}\label{lem:bound_diff_sigma}
    If the following conditions hold:
    \begin{itemize}
        \item Given a video caption distribution ${\cal V}_c$ as Definition~\ref{def:V_c}.
        \item For any $(V, c) \sim {\cal V}_c$, we define the discretized form of video as Definition~\ref{def:wt_V}.
        \item Let the observation matrix $\Phi: \{0, 1\}^{N \times \frac{T}{\Delta t}}$ be defined as Definition~\ref{def:Phi}.
        \item Let the visual decoder function $D: \R^d \rightarrow \R^D$ be defined as Definition~\ref{def:visual_decoder}.
        \item Let the ideal version of the sequence of latent patches $u \in \R^{\frac{T}{\Delta t} \times d}$ be defined as Definition~\ref{def:u}.
        \item Let the real-world version of the sequence of latent patches $\wt{u} \in \R^{N \times d}$ be defined as Definition~\ref{def:wt_u}.
        \item Let $H_N \in \R^{d \times s}$ be defined as Definition~\ref{def:H}.
        \item Let the function of polynomials $g(t)$ be defined as Definition~\ref{def:g_t}.
        \item Let the time-dependent mean of Gaussian distribution $\mu_t(\wt{u})$ be defined as Definition~\ref{def:mu}.
        \item Let the time-dependent standard deviation $\sigma_t(\wt{u})$ be defined as Definition~\ref{def:sigma}.
        \item Denote $\sigma_{\min} > 0$.
        \item Given a hyper-parameter $\alpha > 0$.
    \end{itemize}
    Then for any $\alpha >0$, we have:
    \begin{align*}
        | \frac{\sigma_t'(\wt{u})}{\sigma_t(\wt{u})} | \leq \frac{1 - \sigma_{\min}}{\sigma_{\min}}.
    \end{align*}
\end{lemma}

\begin{proof}
    This result can be obtained following very simple algebras.
\end{proof}

\begin{definition}\label{def:psi}
    If the following conditions hold:
    \begin{itemize}
        \item Given a video caption distribution ${\cal V}_c$ as Definition~\ref{def:V_c}.
        \item For any $(V, c) \sim {\cal V}_c$, we define the discretized form of video as Definition~\ref{def:wt_V}.
        \item Let the observation matrix $\Phi: \{0, 1\}^{N \times \frac{T}{\Delta t}}$ be defined as Definition~\ref{def:Phi}.
        \item Let the visual decoder function $D: \R^d \rightarrow \R^D$ be defined as Definition~\ref{def:visual_decoder}.
        \item Let the ideal version of the sequence of latent patches $u \in \R^{\frac{T}{\Delta t} \times d}$ be defined as Definition~\ref{def:u}.
        \item Let the real-world version of the sequence of latent patches $\wt{u} \in \R^{N \times d}$ be defined as Definition~\ref{def:wt_u}.
        \item Let $H_N \in \R^{d \times s}$ be defined as Definition~\ref{def:H}.
        \item Let the function of polynomials $g(t)$ be defined as Definition~\ref{def:g_t}.
        \item Let the time-dependent mean of Gaussian distribution $\mu_t(\wt{u})$ be defined as Definition~\ref{def:mu}.
        \item Let the time-dependent standard deviation $\sigma_t(\wt{u})$ be defined as Definition~\ref{def:sigma}.
        \item Denote $\sigma_{\min} > 0$.
        \item Sample $z \sim \mathcal{N}(0, I_d)$.
    \end{itemize}
    We define the Video Latent Flow:
    \begin{align*}
        \psi_t(\wt{u}) := \sigma_t(\wt{u}) \cdot z + \mu_t(\wt{u}) \in \R^d.
    \end{align*}
\end{definition}

\subsection{Training Objective} \label{sub:app:train_obj}

\begin{definition}\label{def:L}
    If the following conditions hold:
    \begin{itemize}
        \item Given a video caption distribution ${\cal V}_c$ as Definition~\ref{def:V_c}.
        \item For any $(V, c) \sim {\cal V}_c$, we define the discretized form of video as Definition~\ref{def:wt_V}.
        \item Let the observation matrix $\Phi: \{0, 1\}^{N \times \frac{T}{\Delta t}}$ be defined as Definition~\ref{def:Phi}.
        \item Let the visual decoder function $D: \R^d \rightarrow \R^D$ be defined as Definition~\ref{def:visual_decoder}.
        \item Let the ideal version of the sequence of latent patches $u \in \R^{\frac{T}{\Delta t} \times d}$ be defined as Definition~\ref{def:u}.
        \item Let the real-world version of the sequence of latent patches $\wt{u} \in \R^{N \times d}$ be defined as Definition~\ref{def:wt_u}.
        \item Let $H_N \in \R^{d \times s}$ be defined as Definition~\ref{def:H}.
        \item Let the function of polynomials $g(t)$ be defined as Definition~\ref{def:g_t}.
        \item Let the time-dependent mean of Gaussian distribution $\mu_t(\wt{u})$ be defined as Definition~\ref{def:mu}.
        \item Let the time-dependent standard deviation $\sigma_t(\wt{u})$ be defined as Definition~\ref{def:sigma}.
        \item Denote $\sigma_{\min} > 0$.
        \item Sample $z \sim \mathcal{N}(0, I_d)$.
        \item Define a model function $F_\theta: \R^d \times \R^\ell \times [0, T] \rightarrow \R^d$ with parameters $\theta$.
    \end{itemize}
    We define the training objective of Video Latent Flow Matching as follows:
    \begin{align*}
        {\cal L}(\theta) := \E_{z \sim \mathcal{N}(0, I_d), t \sim {\sf Uniform}[0, T], (V, c) \sim {\cal V}_c}[\| F_\theta( \psi_t(\wt{u}), c, t ) - \frac{\d }{\d t} \psi_t(\wt{u}) \|_2^2].
    \end{align*}
\end{definition}

\begin{theorem}\label{thm:close_form}
    If the following conditions hold:
    \begin{itemize}
        \item Given a video caption distribution ${\cal V}_c$ as Definition~\ref{def:V_c}.
        \item For any $(V, c) \sim {\cal V}_c$, we define the discretized form of video as Definition~\ref{def:wt_V}.
        \item Let the observation matrix $\Phi: \{0, 1\}^{N \times \frac{T}{\Delta t}}$ be defined as Definition~\ref{def:Phi}.
        \item Let the visual decoder function $D: \R^d \rightarrow \R^D$ be defined as Definition~\ref{def:visual_decoder}.
        \item Let the ideal version of the sequence of latent patches $u \in \R^{\frac{T}{\Delta t} \times d}$ be defined as Definition~\ref{def:u}.
        \item Let the real-world version of the sequence of latent patches $\wt{u} \in \R^{N \times d}$ be defined as Definition~\ref{def:wt_u}.
        \item Let $H_N \in \R^{d \times s}$ be defined as Definition~\ref{def:H}.
        \item Let the function of polynomials $g(t)$ be defined as Definition~\ref{def:g_t}.
        \item Let the time-dependent mean of Gaussian distribution $\mu_t(\wt{u})$ be defined as Definition~\ref{def:mu}.
        \item Let the time-dependent standard deviation $\sigma_t(\wt{u})$ be defined as Definition~\ref{def:sigma}.
        \item Denote $\sigma_{\min} > 0$.
        \item Sample $z \sim \mathcal{N}(0, I_d)$.
        \item Define a model function $F_\theta: \R^d \times \R^\ell \times [0, T] \rightarrow \R^d$ with parameters $\theta$.
        \item Let the training objective ${\cal L}(\theta)$ be defined as Definition~\ref{def:L}.
    \end{itemize}
    Then the minimum solution for function $F_\theta$ that takes $z \sim N(0, I_d)$ and $t \sim {\sf Uniform}[0, T]$ is:
    \begin{align*}
        F_\theta(z, c, t) = \frac{\sigma_t'(\wt{u})}{\sigma_t(\wt{u})} (z - \mu_t(\wt{u})) + \mu_t'(\wt{u}).
    \end{align*}
\end{theorem}

\begin{proof}
    This proof follows from Theorem 3 in \cite{lcb+22}.
\end{proof}

\section{Diffusion Transformer} \label{sec:app:dit}

In this section, we first define the Diffusion Transformer in Appendix~\ref{sub:app:def}. Moreover, we introduce the Approximation via DiT in Appendix~\ref{sub:app:approx_dit}.

\subsection{Definitions} \label{sub:app:def}

\begin{definition}[Multi-head self-attention]\label{def:attn}
    Given $h$-heads query, key, value and output projection weights $\{(W_Q^i, W_K^i, W_V^i, W_O^i)\}_{i=1}^h \subset \R^{d_0 \times 4m}$ with each weight is a $d_0 \times m$ shape matrix, for an input matrix $X \in \R^{n \times d_0}$, we define a multi-head self-attention computation as follows:
    \begin{align*}
        {\sf Attn}(X) := \sum_{i=1}^h {\sf Softmax}( X W_Q^i {W_K^i}^\top X^\top ) \cdot X W_V^i {W_O^i}^\top + X \in \R^{n \times d_0}.
    \end{align*}
\end{definition}

\begin{definition}[Feed-forward]\label{def:feed_forward}
    Given two projection weights $W_1, W_2 \in \R^{d_0 \times r}$ and two bias vectors $b_1 \in \R^r$ and $b_2 \in \R^{d_0}$, for an input matrix $X \in \R^{n \times d_0}$, we define a feed-forward computation as follows:
    \begin{align*}
        {\sf FF}(X) := \phi(X W_1 + {\bf 1}_n b_1^\top) \cdot W_2^\top + {\bf 1}_n b_2^\top + X \in \R^{n \times d_0}.
    \end{align*}
    Here, $\phi$ is an activation function and usually be considered as ReLU.
\end{definition}

\begin{definition}[Transformer block]\label{def:transformer_tf}
    Given a set of model weights $\theta^{h, m, r} = \{ \{(W_Q^i, W_K^i, W_V^i, W_O^i)\}_{i=1}^h,$ $ W_1, W_2, b_1, b_2 \}$, the computation of a transformer block is given by the combination of multi-head self-attention computation (Definition~\ref{def:attn}) and feed-forward computation (Definition~\ref{def:feed_forward}). Formally, for an input matrix $X \in \R^{n \times d_0}$, we define:
    \begin{align*}
        {\sf TF}_{\theta^{h, m, r}}(X) := {\sf FF} \circ {\sf Attn}(X) \in \R^{n \times d_0}
    \end{align*}
\end{definition}

\begin{definition}[Reshape Layer]\label{def:R}
    We define the reshape network $R: \R^d \rightarrow \R^{n \times d_0}$.
\end{definition}

\begin{definition}[Complete transformer network]\label{def:model}
    We consider a transformer network as a composition of a transformer block (Definition~\ref{def:transformer_tf}) with model weight $\theta^{h, m, r}$, which is:
    \begin{align*}
        & ~ {\cal T}^{h, m, r} \\
        := & ~ \{ {\cal F}: \R^{n \times d_0} \rightarrow \R^{n \times d_0}~\\
        & ~ |~\text{${\cal F}$ is a composition of Transformer blocks ${\sf TF}_{\theta^{h, m, r}}$’s with positional embedding $E \in \R^{n \times d_0}$}\}
    \end{align*}
    We especially say $\theta^{h, m, r}$ is the model weight that contains $h$ heads, $m$ hidden size for attention and $r$ hidden size for feed-forward. See Example~\ref{exp:cal_F} for further explanation of the sequence-to-sequence mapping ${\cal F}$.
\end{definition}

\begin{example}\label{exp:cal_F}
    We here give an example for the sequence-to-sequence mapping ${\cal F}$ in Definition~\ref{def:model}: Denote $L$ as the number of layers in some transformer network. For an input matrix $X \in \R^{n \times d}$, we use $E \in \R^{n \times d}$ to denote the positional encoding, we then define:
    \begin{align*}
        {\cal F}(X) := {\sf TF}^L \circ {\sf TF}^{L-1} \circ \cdots \circ {\sf TF}^2 \circ {\sf TF}^1(X + E)
    \end{align*}
\end{example}

\subsection{Approximation via DiT} \label{sub:app:approx_dit}

\begin{theorem}\label{thm:uat}
    If the following conditions hold:
    \begin{itemize}
        \item Given a video caption distribution ${\cal V}_c$ as Definition~\ref{def:V_c}.
        \item For any $(V, c) \sim {\cal V}_c$, we define the discretized form of video as Definition~\ref{def:wt_V}.
        \item Let the observation matrix $\Phi: \{0, 1\}^{N \times \frac{T}{\Delta t}}$ be defined as Definition~\ref{def:Phi}.
        \item Let the visual decoder function $D: \R^d \rightarrow \R^D$ be defined as Definition~\ref{def:visual_decoder}.
        \item Let the ideal version of the sequence of latent patches $u \in \R^{\frac{T}{\Delta t} \times d}$ be defined as Definition~\ref{def:u}.
        \item Let the real-world version of the sequence of latent patches $\wt{u} \in \R^{N \times d}$ be defined as Definition~\ref{def:wt_u}.
        \item Let $H_N \in \R^{d \times s}$ be defined as Definition~\ref{def:H}.
        \item Let the function of polynomials $g(t)$ be defined as Definition~\ref{def:g_t}.
        \item Let the time-dependent mean of Gaussian distribution $\mu_t(\wt{u})$ be defined as Definition~\ref{def:mu}.
        \item Let the time-dependent standard deviation $\sigma_t(\wt{u})$ be defined as Definition~\ref{def:sigma}.
        \item Denote $\sigma_{\min} > 0$.
        \item Sample $z \sim \mathcal{N}(0, I_d)$.
        \item Define a model function $F_\theta: \R^d \times \R^\ell \times [0, T] \rightarrow \R^d$ with parameters $\theta$.
        \item Let the training objective ${\cal L}(\theta)$ be defined as Definition~\ref{def:L}.
    \end{itemize}
    Then there exists a transformer network $f_{\cal T} \in {\cal T}_{P}^{2, 1, 4}$ defining function $F_\theta(z, c, t) := f_{\cal T}( R([z^\top, c^\top, t]^\top) )$ with parameters $\theta$ that satisfies ${\cal L}(\theta) \leq \epsilon$ for any error $\epsilon > 0$.
\end{theorem}

\begin{proof}
    Following Assumption~\ref{ass:M}, we first denote $\wt{V}_{\tau} = \wt{{\cal M}}_\tau(c)$ for any $\tau \in [\frac{T}{\Delta t}]$ to discretize function ${\cal M}$. Then we have:
    \begin{align}\label{eq:wt_u_func}
        \wt{u}_{\tau} = {\cal D}^{-1}\Big( (\Phi \wt{{\cal M}}(c))_\tau \Big).
    \end{align}
    where this step follows from Definition~\ref{def:Phi} and Definition~\ref{def:visual_decoder}.

    Besides, we also have:
    \begin{align}\label{eq:mu_func}
        \mu_t(\wt{u}) = & ~ H_N g(t) \notag \\
        = & ~ \Big( H_{N-1} ( I_s - \frac{1}{N} A )^\top + \frac{1}{N} \wt{u}_{N} B^\top \Big) g(t) \notag\\
        = & ~ \Bigg( H_{N-2} \Big ( (I_s - \frac{1}{N-1} A )^\top + \frac{1}{N-1} \wt{u}_{N} B^\top \Big) ( I_s - \frac{1}{N} A )^\top + \frac{1}{N} \wt{u}_{N} B^\top \Bigg) g(t) \notag\\
        = & ~ \Bigg( H_0 \prod_{\tau=1}^N (I_s - \frac{1}{\tau}A)^\top + \sum_{\tau=1}^N \Big( \prod_{\tau'=1}^{\tau - 1} (I_s - \frac{1}{\tau'} A )^\top\Big) \cdot \frac{1}{N+1-\tau} \wt{u}_{N+1-\tau} B^\top \Bigg) g(t)
    \end{align}
    where these steps follow from Definition~\ref{def:mu} and simple algebras.

    Recall $F_\theta(z, c, t) := f_{\cal T}( R([z^\top, c^\top, t]^\top) )$, we choose $n=1$, then there is a target function given by:
    \begin{align*}
        &  ~ f_{\cal T}([z^\top, c^\top, t]) \\
        = & ~ \frac{\sigma_t'(\wt{u})}{\sigma_t(\wt{u})} ( z - \Bigg( H_0 \prod_{\tau=1}^N (I_s - \frac{1}{\tau}A)^\top + \sum_{\tau=1}^N \Big( \prod_{\tau'=1}^{\tau - 1} (I_s - \frac{1}{\tau'} A )^\top\Big) \cdot \frac{1}{N+1-\tau} \wt{u}_{N+1-\tau} B^\top \Bigg) g(t)  ) \\
        & ~ + \Bigg( H_0 \prod_{\tau=1}^N (I_s - \frac{1}{\tau}A)^\top + \sum_{\tau=1}^N \Big( \prod_{\tau'=1}^{\tau - 1} (I_s - \frac{1}{\tau'} A )^\top\Big) \cdot \frac{1}{N+1-\tau} \wt{u}_{N+1-\tau}' B^\top \Bigg) g(t) \\
        & ~ + \Bigg( H_0 \prod_{\tau=1}^N (I_s - \frac{1}{\tau}A)^\top + \sum_{\tau=1}^N \Big( \prod_{\tau'=1}^{\tau - 1} (I_s - \frac{1}{\tau'} A )^\top\Big) \cdot \frac{1}{N+1-\tau} \wt{u}_{N+1-\tau} B^\top \Bigg) g'(t) \\
        = & ~ \frac{\sigma_t'(\wt{u})}{\sigma_t(\wt{u})} ( z \\ 
        & ~ - \Bigg( H_0 \prod_{\tau=1}^N (I_s - \frac{1}{\tau}A)^\top + \sum_{\tau=1}^N \Big( \prod_{\tau'=1}^{\tau - 1} (I_s - \frac{1}{\tau'} A )^\top\Big) \cdot \frac{1}{N+1-\tau} {\cal D}^{-1}\Big( (\Phi \wt{{\cal M}}(c))_{N+1-\tau} \Big) B^\top \Bigg) g(t)  ) \\
        & ~ + \Bigg( H_0 \prod_{\tau=1}^N (I_s - \frac{1}{\tau}A)^\top + \sum_{\tau=1}^N \Big( \prod_{\tau'=1}^{\tau - 1} (I_s - \frac{1}{\tau'} A )^\top\Big) \cdot \frac{1}{N+1-\tau} \Big({\cal D}^{-1}\Big( (\Phi \wt{{\cal M}}(c))_{N+1-\tau} \Big) \Big)' B^\top \Bigg) g(t) \\
        & ~ + \Bigg( H_0 \prod_{\tau=1}^N (I_s - \frac{1}{\tau}A)^\top + \sum_{\tau=1}^N \Big( \prod_{\tau'=1}^{\tau - 1} (I_s - \frac{1}{\tau'} A )^\top\Big) \cdot \frac{1}{N+1-\tau} {\cal D}^{-1}\Big( (\Phi \wt{{\cal M}}(c))_{N + 1 - \tau} \Big) B^\top \Bigg) g'(t)
    \end{align*}
    where the first step follows the combination of Theorem~\ref{thm:close_form} and Eq.~\eqref{eq:mu_func}, and the differentiablity of $\wt{u}_\tau$ is ensure by Assumption~\ref{ass:k}, the second step follows from Eq.~\eqref{eq:wt_u_func}.

    Following Theorem 2 and Theorem 3 in \cite{ybr+19}, we thus complete the proof by obtaining the theorem result.
\end{proof}

\section{Interpolation and Extrapolation}
\label{sec:app:inter_extra}

This section first introduce properties of HiPPO-LegS in Appendix~\ref{sub:app:hippo_property}. Also, we bound the error of VLFM in Appendix~\ref{sub:app:error}.

\subsection{HiPPO-LegS Properties} \label{sub:app:hippo_property}

\begin{lemma}[Proposition 6 in \cite{gde+20}]\label{lem:optimal_projs}
    If the following conditions hold:
    \begin{itemize}
        \item Given a video caption distribution ${\cal V}_c$ as Definition~\ref{def:V_c}.
        \item For any $(V, c) \sim {\cal V}_c$, we define the discretized form of video as Definition~\ref{def:wt_V}.
        \item Let the observation matrix $\Phi: \{0, 1\}^{N \times \frac{T}{\Delta t}}$ be defined as Definition~\ref{def:Phi}.
        \item Let the visual decoder function $D: \R^d \rightarrow \R^D$ be defined as Definition~\ref{def:visual_decoder}.
        \item Let the ideal version of the sequence of latent patches $u \in \R^{\frac{T}{\Delta t} \times d}$ be defined as Definition~\ref{def:u}.
        \item Let the real-world version of the sequence of latent patches $\wt{u} \in \R^{N \times d}$ be defined as Definition~\ref{def:wt_u}.
        \item Let $H_N \in \R^{d \times s}$ be defined as Definition~\ref{def:H}.
        \item Let the function of polynomials $g(t)$ be defined as Definition~\ref{def:g_t}.
        \item Let the time-dependent mean of Gaussian distribution $\mu_t(\wt{u})$ be defined as Definition~\ref{def:mu}.
        \item Let the time-dependent standard deviation $\sigma_t(\wt{u})$ be defined as Definition~\ref{def:sigma}.
        \item Denote $\sigma_{\min} > 0$.
        \item Sample $z \sim \mathcal{N}(0, I_d)$.
        \item Define a model function $F_\theta: \R^d \times \R^\ell \times [0, T] \rightarrow \R^d$ with parameters $\theta$.
        \item Let the training objective ${\cal L}(\theta)$ be defined as Definition~\ref{def:L}.
        \item Let Assumptions~\ref{ass:k}, Assumption~\ref{ass:L_0}, Assumption~\ref{ass:M} and Assumption~\ref{ass:U} hold.
    \end{itemize}
    Then we have:
    \begin{align*}
        \| \mu_{\tau \cdot \Delta t}(\wt{u}) - \wt{u}_\tau \|_2 = O(t^{k}s^{-k+1/2})
    \end{align*}
\end{lemma}

\begin{proof}
    This lemma is a re-statement of Proposition 6 in \cite{gde+20}.
\end{proof}

\begin{lemma}[Proposition 3 in \cite{gde+20}]\label{lem:timescale_robustness}
    If the following conditions hold:
    \begin{itemize}
        \item Given a video caption distribution ${\cal V}_c$ as Definition~\ref{def:V_c}.
        \item For any $(V, c) \sim {\cal V}_c$, we define the discretized form of video as Definition~\ref{def:wt_V}.
        \item Let the observation matrix $\Phi: \{0, 1\}^{N \times \frac{T}{\Delta t}}$ be defined as Definition~\ref{def:Phi}.
        \item Let the visual decoder function $D: \R^d \rightarrow \R^D$ be defined as Definition~\ref{def:visual_decoder}.
        \item Let the ideal version of the sequence of latent patches $u \in \R^{\frac{T}{\Delta t} \times d}$ be defined as Definition~\ref{def:u}.
        \item Let the real-world version of the sequence of latent patches $\wt{u} \in \R^{N \times d}$ be defined as Definition~\ref{def:wt_u}.
        \item Let $H_N \in \R^{d \times s}$ be defined as Definition~\ref{def:H}.
        \item Let the function of polynomials $g(t)$ be defined as Definition~\ref{def:g_t}.
        \item Let the time-dependent mean of Gaussian distribution $\mu_t(\wt{u})$ be defined as Definition~\ref{def:mu}.
        \item Let the time-dependent standard deviation $\sigma_t(\wt{u})$ be defined as Definition~\ref{def:sigma}.
        \item Denote $\sigma_{\min} > 0$.
        \item Sample $z \sim \mathcal{N}(0, I_d)$.
        \item Define a model function $F_\theta: \R^d \times \R^\ell \times [0, T] \rightarrow \R^d$ with parameters $\theta$.
        \item Let the training objective ${\cal L}(\theta)$ be defined as Definition~\ref{def:L}.
        \item Let Assumptions~\ref{ass:k}, Assumption~\ref{ass:L_0}, Assumption~\ref{ass:M} and Assumption~\ref{ass:U} hold.
    \end{itemize}
    For any integer scale factor $\beta > 0$, the frames of video $\wt{V}_{\tau}$ is scaled to $\wt{V}_{\beta \tau}$, it doesn't affect the result of $H_N$ (Definition~\ref{def:H}).
\end{lemma}

\begin{proof}
    This lemma is a re-statement of Proposition 3 in \cite{gde+20}.
\end{proof}

\subsection{Error Bounds} \label{sub:app:error}

\begin{lemma}\label{lem:hippo_error}
    If the following conditions hold:
    \begin{itemize}
        \item Given a video caption distribution ${\cal V}_c$ as Definition~\ref{def:V_c}.
        \item For any $(V, c) \sim {\cal V}_c$, we define the discretized form of video as Definition~\ref{def:wt_V}.
        \item Let the observation matrix $\Phi: \{0, 1\}^{N \times \frac{T}{\Delta t}}$ be defined as Definition~\ref{def:Phi}.
        \item Let the visual decoder function $D: \R^d \rightarrow \R^D$ be defined as Definition~\ref{def:visual_decoder}.
        \item Let the ideal version of the sequence of latent patches $u \in \R^{\frac{T}{\Delta t} \times d}$ be defined as Definition~\ref{def:u}.
        \item Let the real-world version of the sequence of latent patches $\wt{u} \in \R^{N \times d}$ be defined as Definition~\ref{def:wt_u}.
        \item Let $H_N \in \R^{d \times s}$ be defined as Definition~\ref{def:H}.
        \item Let the function of polynomials $g(t)$ and matrix $G$ be defined as Definition~\ref{def:g_t}.
        \item Denote $1/\lambda^* := \lambda_{\min}(G) > 0$.
        \item Let the time-dependent mean of Gaussian distribution $\mu_t(\wt{u})$ be defined as Definition~\ref{def:mu}.
        \item Let the time-dependent standard deviation $\sigma_t(\wt{u})$ be defined as Definition~\ref{def:sigma}.
        \item Denote $\sigma_{\min} > 0$.
        \item Sample $z \sim \mathcal{N}(0, I_d)$.
        \item Define a model function $F_\theta: \R^d \times \R^\ell \times [0, T] \rightarrow \R^d$ with parameters $\theta$.
        \item Let the training objective ${\cal L}(\theta)$ be defined as Definition~\ref{def:L}.
        \item Let Assumptions~\ref{ass:k}, Assumption~\ref{ass:L_0}, Assumption~\ref{ass:M} and Assumption~\ref{ass:U} hold.
        \item $\delta \in (0, 1)$.
        \item Choosing $s = O(\frac{\Delta t}{T}\log((\frac{\Delta t}{T})^{1.5}/1/\lambda^*))$.
    \end{itemize}
    Particularly, we define:
    \begin{itemize}
        \item $\epsilon_1 := O(T^k s^{-k+1/2})$.
        \item $\epsilon_2 := O(\sqrt{d\log(d/\delta)})$.
        \item $\epsilon_3 := 1/\lambda^* U d^{0.5} \sqrt{\frac{T}{\Delta t} - N} \cdot \exp(O(\frac{T}{\Delta t}s))$.
    \end{itemize}
    Then with a probability at least $1 - \delta$, we have:
    \begin{align*}
        \| \psi_t(\wt{u}) - u_t \|_2 \leq \epsilon_1 + \epsilon_2 + \epsilon_3.
    \end{align*}
\end{lemma}

\begin{proof}
    We have:
    \begin{align*}
        \| \psi_t(\wt{u}) - u_t \|_2
        = & ~ \| \sigma_t(\wt{u}) \cdot z + \mu_t(\wt{u}) - u_t \|_2 \\
        \leq & ~ \| \sigma_t(\wt{u}) \cdot z \|_2 + \|  \mu_t(\wt{u}) - u_t \|_2 \\
        \leq & ~ \| z \|_2 + \|  \mu_t(\wt{u}) - u_t \|_2 \\
        \leq & ~ O( \sqrt{d \log(d/\delta)} ) +  \|  \mu_t(\wt{u}) - u_t \|_2 \\
        = & ~ \epsilon_2 + \|  \mu_t(\wt{u}) - u_t \|_2
    \end{align*}
    where the first step follows from Definition~\ref{def:psi}, the second step follows from triangle inequality, the third step follows from $\sigma_t(\wt{u}) \leq 1, \forall t \in [0, T]$ by some simple algebras and Definition~\ref{def:sigma}, the fourth step follows from the union bound of Gaussian tail bound (Fact~\ref{fac:gaussian_tail}), the last step follows from the definition of $\epsilon_2$.

    Then we get:
    \begin{align*}
        \|  \mu_t(\wt{u}) - u_t \|_2
        = & ~ \| H_N g(t) - u_t \|_2 \\
        = & ~ \| (M \cdot G)^\dagger (M \cdot u) \cdot g(t) - u_t \|_2 \\
        \leq & ~ \| (M \cdot G)^\dagger (M \cdot u) \cdot g(t) - G^\dagger u \cdot g(t) \|_2 + O((\frac{T}{\Delta t})^k s^{-k+1/2}) \\
        \leq & ~ \| ((M \cdot G)^\dagger (M \cdot u) - G^\dagger u \|_2 \cdot \| g(t) \|_2 + O((\frac{T}{\Delta t})^k s^{-k+1/2}) \\
        = & ~ \| ((M \cdot G)^\dagger (M \cdot u) - G^\dagger u \|_2 \cdot \| g(t) \|_2 + \epsilon_1
    \end{align*}
    where the first step follows from Definition~\ref{def:mu}, the second step follows from optimal error of solving $\| M G H - M u \|_2^2$, pesdueo-inverse matrix $(M \cdot G)^\dag \in \R^{d \times \frac{T}{\Delta t}}$ and defining a mask $M = \diag(m)$ where $m := \{0, 1\}^{\frac{T}{\Delta t}}$ and $\langle m, {\bf 1}_{\frac{T}{\Delta t}} \rangle = N$, the third step follows from the optimal error of solving $\| G H - u \|_2^2$, pesdueo-inverse matrix $G^\dag \in \R^{d \times \frac{T}{\Delta}}$ and Lemma~\ref{lem:optimal_projs}, the fourth step follows from Cauchy–Schwarz inequality and the last step follows from the definition of $\epsilon_2$.

    Next, we can show that:
    \begin{align*}
        \| (M \cdot G)^\dagger (M \cdot u) - G^\dagger u \|_2
        = & ~ \| (M \cdot G)^\dagger (M \cdot u) - G^\dagger (M \cdot u) + G^\dagger (M \cdot u) - G^\dagger u \|_2 \\
        \leq &~ \| (M \cdot G)^\dagger (M \cdot u)  - G^\dagger (M \cdot u) \|_2 + \| G^\dagger (M \cdot u) - G^\dagger u \|_2 \\
        \leq & ~ \| (M \cdot G)^\dagger - G^\dagger \|_2 \| (M \cdot u) \|_2 + \| G^\dagger \|_2 \| (M \cdot u) -  u \|_2
    \end{align*}
    where the first step follows from simple algebras, the second step follows from triangle inequality, the last step follows from Cauchy–Schwarz inequality.

    We first give:
    \begin{align}\label{eq:bound_G_dag}
        \| G^\dagger\|_2 \leq &~ 1/\lambda^* \sqrt{\frac{T}{\Delta t} \cdot s}
    \end{align}
    where this step follows from Definition~\ref{def:g_t}, Fact~\ref{fac:infity_norm_pesdueo_inverse} and the definition of $\ell_2$ norm.

    And:
    \begin{align*}
        \| u\|_2 \leq & ~ U \sqrt{\frac{T}{\Delta t} \cdot d}
    \end{align*}
    where this step follows from Assumption~\ref{ass:U} and the definition of $\ell_2$ norm.

    Also:
    \begin{align}\label{eq:bound_G}
        \| G\|_2 \leq & ~ \sqrt{\frac{T}{\Delta t} \cdot s} \exp( O(\frac{T}{\Delta t} \cdot s) ) 
    \end{align}
   where this step follows from Definition~\ref{def:g_t} and the definition of $\ell_2$ norm.

    Besides, we have:
    \begin{align*}
        \| (M \cdot G)^\dagger - G^\dagger \|_2
        \leq & ~ \frac{\| G^\dagger\|_2^2 \| I_{\frac{T}{\Delta t}} - M\|_2 \cdot \| G\|_2}{1 - \| G^\dagger\|_2 \cdot \| I_{\frac{T}{\Delta t}} - M\|_2 \cdot \| G\|_2} \\
        \leq & ~ \frac{{1/\lambda^*}^2 (\frac{T}{\Delta t} s)^{1.5} \sqrt{\frac{T}{\Delta t} - N} \cdot \exp(O(\frac{T}{\Delta t}s))}{1 - {1/\lambda^*} \frac{T}{\Delta t} s \sqrt{\frac{T}{\Delta t} - N}\cdot \exp(O(\frac{T}{\Delta t}s)) }
    \end{align*}
    where the first step follows from Fact~\ref{fac:pesdueo_inverse_diff}, simple algebras, and Cauchy–Schwarz inequality, the second step follows from Eq.~\eqref{eq:bound_G_dag}, Eq.~\eqref{eq:bound_G}, Definition~\ref{def:g_t} and simeple algebras.

    Combining all results, we get:
    \begin{align*}
        & ~ \| ((M \cdot G)^\dagger (M \cdot u) - G^\dagger u \|_2 \\
        \leq & ~ \frac{{1/\lambda^*}^2 (\frac{T}{\Delta t} s)^{1.5} \sqrt{\frac{T}{\Delta t} - N} \cdot \exp(O(\frac{T}{\Delta t}s))}{1 - {1/\lambda^*} \frac{T}{\Delta t} s \sqrt{\frac{T}{\Delta t} - N}\cdot \exp(O(\frac{T}{\Delta t}s)) } \cdot U \sqrt{\frac{T}{\Delta t} N d} + 1/\lambda^* \sqrt{\frac{T}{\Delta t} - N} \cdot U \sqrt{\frac{T}{\Delta t} \cdot d} \\
        \leq & ~ 1/\lambda^* U d^{0.5}  \sqrt{\frac{T}{\Delta t} (\frac{T}{\Delta t} - N)}  \cdot \Big( \frac{ {1/\lambda^*} (\frac{T}{\Delta t})^{1.5} N^{0.5} s^{1.5}  \cdot \exp(O(\frac{T}{\Delta t}s))}{1 - {1/\lambda^*} (\frac{T}{\Delta t})^{1.5} s \cdot \exp(O(\frac{T}{\Delta t}s))} + 1 \Big) \\
        \leq & ~ 1/\lambda^* U d^{0.5}  \sqrt{\frac{T}{\Delta t} (\frac{T}{\Delta t} - N)}  \cdot  \frac{ 1}{1 - {1/\lambda^*} (\frac{T}{\Delta t})^{1.5} s \cdot \exp(O(\frac{T}{\Delta t}s))}  \\
        \leq & ~ O\Big( 1/\lambda^* U d^{0.5}  \sqrt{\frac{T}{\Delta t} (\frac{T}{\Delta t} - N)}\Big)
    \end{align*}
    where the second and third steps follow from simple algebras, the last step follows from plugging the choice of $s$.

    Finally, we have:
    \begin{align*}
        \| ((M \cdot G)^\dagger (M \cdot u) - G^\dagger u \|_2 \cdot \| g(t) \|_2 
        \leq & ~ O\Big( 1/\lambda^* U d^{0.5}  \sqrt{\frac{T}{\Delta t} (\frac{T}{\Delta t} - N)}\Big) \cdot \sqrt{s} \exp(O(\frac{T}{\Delta t}s)) \\
        \leq & ~ 1/\lambda^* U d^{0.5} \sqrt{\frac{T}{\Delta t} - N} \cdot \exp(O(\frac{T}{\Delta t}s))  \\
        = & ~ \epsilon_3
    \end{align*}
    these steps follow from simple algebras, Definition~\ref{def:g_t} and the definition of $\epsilon_3$.
\end{proof}

\begin{theorem}\label{thm:inter_extra_polation}
    If the following conditions hold:
    \begin{itemize}
        \item Given a video caption distribution ${\cal V}_c$ as Definition~\ref{def:V_c}.
        \item For any $(V, c) \sim {\cal V}_c$, we define the discretized form of video as Definition~\ref{def:wt_V}.
        \item Let the observation matrix $\Phi: \{0, 1\}^{N \times \frac{T}{\Delta t}}$ be defined as Definition~\ref{def:Phi}.
        \item Let the visual decoder function $D: \R^d \rightarrow \R^D$ be defined as Definition~\ref{def:visual_decoder}.
        \item Let the ideal version of the sequence of latent patches $u \in \R^{\frac{T}{\Delta t} \times d}$ be defined as Definition~\ref{def:u}.
        \item Let the real-world version of the sequence of latent patches $\wt{u} \in \R^{N \times d}$ be defined as Definition~\ref{def:wt_u}.
        \item Let $H_N \in \R^{d \times s}$ be defined as Definition~\ref{def:H}.
        \item Let the function of polynomials $g(t)$ and matrix $G$ be defined as Definition~\ref{def:g_t}.
        \item Denote $1/\lambda^* := \lambda_{\min}(G) > 0$.
        \item Let the time-dependent mean of Gaussian distribution $\mu_t(\wt{u})$ be defined as Definition~\ref{def:mu}.
        \item Let the time-dependent standard deviation $\sigma_t(\wt{u})$ be defined as Definition~\ref{def:sigma}.
        \item Denote $\sigma_{\min} > 0$.
        \item Sample $z \sim \mathcal{N}(0, I_d)$.
        \item Define a model function $F_\theta: \R^d \times \R^\ell \times [0, T] \rightarrow \R^d$ with parameters $\theta$.
        \item Let the training objective ${\cal L}(\theta)$ be defined as Definition~\ref{def:L}.
        \item Let Assumptions~\ref{ass:k}, Assumption~\ref{ass:L_0}, Assumption~\ref{ass:M} and Assumption~\ref{ass:U} hold.
        \item $\delta \in (0, 1)$.
    \end{itemize}
    Particularly, we define:
    \begin{itemize}
        \item $\epsilon_1 := O(T^k s^{-k+1/2})$.
        \item $\epsilon_2 := O(\sqrt{d\log(d/\delta)})$.
        \item $\epsilon_3 := 1/\lambda^* U d^{0.5} \sqrt{\frac{T}{\Delta t} - N} \cdot \exp(O(\frac{T}{\Delta t}s))$.
    \end{itemize}
    Then with a probability at least $1 - \delta$, we have:
    \begin{align*}
        \| {\cal D}(z + \int_0^{t} F_\theta(z, c, t') \d t') - u_t\|_2 \leq \epsilon_0 + L_0 (\epsilon_1 + \epsilon_2 + \epsilon_3).
    \end{align*}
\end{theorem}


\begin{proof}
    This proof follows from the combination of Assumption~\ref{ass:L_0}, Theorem~\ref{thm:uat} and Lemma~\ref{lem:hippo_error}.
\end{proof}

\ifdefined \isarxiv
\bibliography{ref}
\bibliographystyle{alpha}
\fi


%%% some writing rules

%% Writing rule for creating tags.
%% Tags :
%% Theorem    \ref{thm:bla_bla}
%% Lemma      \ref{lem:bla_bla}
%% Claim      \ref{cla:bla_bla}
%% Corollary  \ref{cor:bla_bla}
%% Fact       \ref{fac:bla_bla}
%% Definition \ref{def:bla_bla}
%% Section    \ref{sec:bla_bla}
%% Subsection \ref{sub:bla_bla}
%% Equation   \ref{eq:bla_bla}



\end{document}



%%%%%%%%%%%%%%%%%%%%%%%%%%%%%%%%%%%%%%%%%%%%%%%%%%%%%%%%%%%%%%%%%%%%%%%%%%%%%%%%%%%%%%%%%%%%%%%%%%%%%%%%%%%%%%%%%%%%%%%%%%%%%%%%%%%%%%%%%%%%%%%%%%%%%%%%%%%%%%%%%%%%%%%%%%%%%%%%%%%%%%%%%%%%%%%%%%%%%%%%%%%%%%%%%%%%%%%%%%%%%%%%%%%%%%%%%%%%%%%%%%%%%%%%%%%%%%%%%%%%%%%%%%%%%%%%%%%%%%%%%%%%%%%%%%%%%%%%%%%%%%%%%%%%%%%%%%%%%%%%%%%%%%%%%%%%%%%%%%%%%%%%%%%%%%%%%%%%%%%%%%%%%%%%%%%%%%%%%%%%%%%%%%%%%%%%%%%%%%%%%%%%%%%%%%%%%%%%%%%%%%%%%%%%%%%%%%%%%%%%%%%%%%%%%%%%%%%%%%%%%%
