\section{Preliminary} \label{sec:app:preli}

In the preliminary section, we first introduce our notation in the appendix in Appendix~\ref{sub:app:notations}.  Then, in Appendix~\ref{sub:app:video}, we formally define the video-caption data and visual decoder. In Appendix~\ref{sub:app:latent_patches}, we define the latent patches. Appendix~\ref{sub:app:assumption} makes some assumptions which we will use later. Finally, in Appendix~\ref{sec:app:facts}, we list some basic useful facts.

\subsection{Notations} \label{sub:app:notations}

\paragraph{Notations.} We use $D$ to denote the flattened dimension of real-world images. We use $d$ to represent the dimension of latent patches. We introduce $d_0$ as the dimension of Diffusion Transformers. We utilize $V: [0, T] \rightarrow \R^D$ to denote a video with $T$ duration, where $T$ is the longest time for each video. We omit $\nabla_t a(t)$ and $a'(t)$ to denote taking differentiation to some function $a(t)$ w.r.t. time $t$. We use integer $s$ to denote the order of polynomials. The dimensional number of the text embedding vector is given by integer $\ell$.

\subsection{Video-Caption Data} \label{sub:app:video}

\begin{definition}[Video-caption data pairs and their distribution]\label{def:V_c}
    We define a video caption distribution $(V, c) \sim {\cal V}_c$. Here, $V: [0, T] \rightarrow \R^D$ is considered as a function and $c \in \R^\ell$ is the corresponding text embedding vector.
\end{definition}

\begin{definition}\label{def:wt_V}
    Given a video caption distribution ${\cal V}_c$ as Definition~\ref{def:V_c}. We denote $\Delta t$ as the minimal time unit of measurement in the real world (Planck time). For any $(V, c) \sim {\cal V}_c$, we define the discretized form of $V: [0, T] \rightarrow \R^D$, which is $\wt{V} \in \R^{\frac{T}{\Delta t} \times D}$, and its $\tau$-th row $ \forall \tau \in [\frac{T}{\Delta t}]$ is given by:
    \begin{align*}
        \wt{V}_\tau := V_{\Delta t \cdot \tau} \in \R^D.
    \end{align*}
\end{definition}

\begin{definition}[Obtained data in real-world cases]\label{def:Phi}
    If the following conditions hold:
    \begin{itemize}
        \item Given a video caption distribution ${\cal V}_c$ as Definition~\ref{def:V_c}.
        \item For any $(V, c) \sim {\cal V}_c$, we define the discretized form of video $\wt{V}$ as Definition~\ref{def:wt_V}.
    \end{itemize}
    We define an observation matrix $\Phi: \{0, 1\}^{N \times \frac{T}{\Delta t}}$. The obtained data in real-world cases then is denoted as $\Phi \wt{V} \in \R^{N \times D}$.
\end{definition}

\begin{definition}[Bijective Visual Decoder]\label{def:visual_decoder}
    We define the visual decoder ${\cal D}: \R^d \rightarrow \R^D$ satisfies that:
    \begin{itemize}
        \item For any flattened image $V \in \R^D$, there is a unique $u \in \R^d$ such that ${\cal D}(u) = V$.
    \end{itemize}
    Then we say ${\cal D}$ is bijective. Denote the reverse function of ${\cal D}$ as ${\cal D}^{-1}: \R^D \rightarrow \R^d$.
\end{definition}

\subsection{Latent Patches Data} \label{sub:app:latent_patches}

\begin{definition}\label{def:u}
    If the following conditions hold:
    \begin{itemize}
        \item Given a video caption distribution ${\cal V}_c$ as Definition~\ref{def:V_c}.
        \item For any $(V, c) \sim {\cal V}_c$, we define the discretized form of video $\wt{V}$ as Definition~\ref{def:wt_V}.
        \item Let the observation matrix $\Phi: \{0, 1\}^{N \times \frac{T}{\Delta t}}$ be defined as Definition~\ref{def:Phi}.
        \item Let the visual decoder function $D: \R^d \rightarrow \R^D$ be defined as Definition~\ref{def:visual_decoder}.
    \end{itemize}
    We define the ideal version (without observation matrix) of the sequence of latent patches $u \in \R^{\frac{T}{\Delta t} \times d}$, and its $\tau$-th $ \forall \tau \in [\frac{T}{\Delta t}]$ row is defined as follows:
    \begin{align*}
        u_\tau := {\cal D}^{-1}( \wt{V}_\tau ).
    \end{align*}
\end{definition}

\begin{definition}\label{def:wt_u}
    If the following conditions hold:
    \begin{itemize}
        \item Given a video caption distribution ${\cal V}_c$ as Definition~\ref{def:V_c}.
        \item For any $(V, c) \sim {\cal V}_c$, we define the discretized form of video as Definition~\ref{def:wt_V}.
        \item Let the observation matrix $\Phi: \{0, 1\}^{N \times \frac{T}{\Delta t}}$ be defined as Definition~\ref{def:Phi}.
        \item Let the visual decoder function $D: \R^d \rightarrow \R^D$ be defined as Definition~\ref{def:visual_decoder}.
    \end{itemize}
    We define the real-world version (with observation matrix) of the sequence of latent patches $\wt{u} \in \R^{\frac{T}{\Delta t} \times d}$, and its $\tau$-th $ \forall \tau \in [N]$ row is defined as follows:
    \begin{align*}
        \wt{u}_\tau := {\cal D}^{-1}\Big( (\Phi V)_\tau \Big).
    \end{align*}
\end{definition}


\subsection{Assumptions} \label{sub:app:assumption}

\begin{assumption}\label{ass:k}
    If the following conditions hold:
    \begin{itemize}
        \item Given a video caption distribution ${\cal V}_c$ as Definition~\ref{def:V_c}.
        \item For any $(V, c) \sim {\cal V}_c$, we define the discretized form of video as Definition~\ref{def:wt_V}.
        \item Let the observation matrix $\Phi: \{0, 1\}^{N \times \frac{T}{\Delta t}}$ be defined as Definition~\ref{def:Phi}.
        \item Let the visual decoder function $D: \R^d \rightarrow \R^D$ be defined as Definition~\ref{def:visual_decoder}.
        \item Let the ideal version of the sequence of latent patches $u \in \R^{\frac{T}{\Delta t} \times d}$ be defined as Definition~\ref{def:u}.
    \end{itemize}
    We assume $u_\tau$ is $k$-differentiable, there exists:
    \begin{align*}
        u_{\tau}^{(i)} = \lim_{\Delta t \rightarrow 0} \frac{u_{\tau+1}^{(i-1)} - u_{\tau}^{(i-1)}}{ \Delta t }, \forall i \in [k], \tau \in [\frac{T}{\Delta t}],
    \end{align*}
    where, we use $u_\tau^{(i)}$ to denote the $i$-th derivation of $u$.
\end{assumption}

\begin{assumption}\label{ass:L_0}
    If the following conditions hold:
    \begin{itemize}
        \item Let the visual decoder function $D: \R^d \rightarrow \R^D$ be defined as Definition~\ref{def:visual_decoder}.
    \end{itemize}
    We assume the visual decoder function ${\cal D}$ is $L_0$-smooth for constant $L_0 > 0$, such that:
    \begin{align*}
        \| {\cal D}(x) - {\cal D}(y) \|_2 \leq L_0 \| x - y \|_2, \forall x, y \in \R^d.
    \end{align*}
\end{assumption}

\begin{assumption}\label{ass:U}
    If the following conditions hold:
    \begin{itemize}
        \item Given a video caption distribution ${\cal V}_c$ as Definition~\ref{def:V_c}.
        \item For any $(V, c) \sim {\cal V}_c$, we define the discretized form of video as Definition~\ref{def:wt_V}.
        \item Let the observation matrix $\Phi: \{0, 1\}^{N \times \frac{T}{\Delta t}}$ be defined as Definition~\ref{def:Phi}.
        \item Let the visual decoder function $D: \R^d \rightarrow \R^D$ be defined as Definition~\ref{def:visual_decoder}.
        \item Let the ideal version of the sequence of latent patches $u \in \R^{\frac{T}{\Delta t} \times d}$ be defined as Definition~\ref{def:u}.
    \end{itemize}
    We assume each entry in latent patches $u$ is bounded by a constant $U > 0$.
\end{assumption}

\begin{assumption}\label{ass:M}
    If the following conditions hold:
    \begin{itemize}
        \item Given a video caption distribution ${\cal V}_c$ as Definition~\ref{def:V_c}.
        \item For any $(V, c) \sim {\cal V}_c$
    \end{itemize}
    For any $(V, c) \sim {\cal V}_c$, we assume there exists a function ${\cal M}: [0, T] \times \R^\ell \rightarrow \R^D$ satisfies $V_t = {\cal M}_t(c)$. 
\end{assumption}

\subsection{Basic Facts} \label{sec:app:facts}

\begin{fact}\label{fac:gaussian_tail}
    For a variable $x \sim \mathcal{N}(0, \sigma^2)$, then with probability at least $1 - \delta$, we have:
    \begin{align*}
        |x| \leq C \sigma \sqrt{\log(1/\delta)}
    \end{align*}
\end{fact}

\begin{fact}\label{fac:infity_norm_pesdueo_inverse}
    For a PD matrix $A \in \R^{d_1 \times d_2}$ with a positive minimum eigenvalue $\lambda_{\min}(A) > 0$, the infinite norm of its pesdueo-inverse matrix $A^\dag$ is given by:
    \begin{align*}
        \| A^\dagger \|_\infty \leq \frac{1}{\lambda_{\min}(A)}.
    \end{align*}
\end{fact}

\begin{fact}\label{fac:pesdueo_inverse_diff}
    For two matrices $A , B \in\R^{d_1 \times d_2}$, we have:
    \begin{align*}
        \| A^\dagger - B^\dagger \|_2 \leq \frac{\| A^\dagger \|_2^2 \| A - B\|_2 }{1 - \| A^\dagger \|_2 \cdot 
        \| A - B \|_2}
    \end{align*}
\end{fact}