\section{Prior Work}
\subsection{AI literacy Frameworks and Reviews}
Long and Magerko created the most influential framework by describing a set of seventeen competencies related to AI \citep{long_what_2020}. Their definition of AI literacy as “a set of competencies that enables individuals to critically evaluate AI technologies; communicate and collaborate effectively with AI; and use AI as a tool online, at home, and in the workplace” is widely cited in empirical work on AI literacy interventions \cite{long_what_2020}. A group of University of Hong Kong researchers also conducted a series of highly cited AI literacy reviews. Their work included both high-level overviews of the conceptualization of AI literacy \citep{ng_ai_2021, ng_conceptualizing_2021} and reviews of empirical studies within various contexts, including K-12 education in the Asia-Pacific region \citep{su_meta-review_2022}, early childhood \citep{su_artificial_2023-1}, and secondary education \citep{ng_fostering_2024-1}. They proposed four aspects of AI literacy, “know and understand, use and apply, evaluate and create, and ethical issues”, and define AI literacy as a series of competencies similar to those of Long and Magerko \citep{long_what_2020, ng_conceptualizing_2021}. Since these studies were published prior to the mainstream introduction of generative AI, they may need an update. 
%This review seeks to augment their work by expanding our understanding of AI literacy beyond technical competencies and include interventions that emphasize empowerment and critical awareness surrounding AI. 

%There are several other review studies published since 2020 that examined AI literacy. 
Several reviews explicitly acknowledged the nascent and changing nature of AI literacy and the resulting difficulty in defining it \citep{laupichler_artificial_2022, benvenuti_artificial_2023, tenorio_artificial_2023, lee_systematic_2024}. Other reviews have proposed frameworks to capture the multitudes of AI literacy, including a twelve-faceted taxonomy synthesized from prominent AI concept classifications \citep{shiri_artificial_2024}, a competency-based framework similar to the work by Ng et al. \citep{almatrafi_systematic_2024}, and two learner-oriented frameworks that categorize AI literacy interventions based on learners’ roles in relation to AI \citep{faruqe_competency_2022, schuller_better_2023}. While these taxonomies are comprehensive, they do not address educators' radically different interpretations of AI literacy and the resulting implementations. This review's proposed framework focuses on the conceptualizations of AI literacy, thus yielding insights into the motivations behind different AI literacy approaches.

%Lastly, two papers proposed learner-oriented frameworks that categorize AI literacy interventions based on learners’ roles in relation to AI. They categorized learner roles as “the informed prosumer, the skilled user, and the expert creator” (prosumer means an individual who both consumes and produces) \citep{schuller_better_2023} and “consumers, coworkers, collaborators, and creators” \citep{faruqe_competency_2022}. %These frameworks provide insights on how conceptualizing learners in different roles surrounding AI leads to different AI literacy approaches. 
%This work seeks to complement these frameworks by exploring how these learner roles are shaped by interactions between perspectives on AI and perspectives on literacy.

\subsection{Definitions of AI and Literacy}
Complicating the matter of defining AI literacy, both constituent concepts of AI and literacy are complex and lack consensus definitions. To start with AI, the term has shifted to encompass different technologies and paradigms since its conception in the 1950s \citep{haenlein_brief_2019}. In its technical sense, AI usually refers to machine learning and generative AI technologies in the reviewed studies. Older paradigms in AI, such as knowledge representation and reasoning, are almost never mentioned among the studies included in this review. 

However, it is insufficient to understand AI only as a technical field. AI's has been integrated into society to form sociotechnical systems, with social, cultural, and political implications \citep{winner_artifacts_1985}. To capture the full breadth of AI conceptualizations, this review draws from the Dagstuhl Triangle, which was originally developed in Germany to describe the objectives of teaching learners to interact with digital systems in general \citep{brinda_dagstuhl-erklarung_2016}. It involves three perspectives: the technological perspective of understanding how the digital system functions, the user-oriented perspective of how to use the digital system, and the sociocultural perspective of how the digital system interacts with society \citep{michaeli_data_2023}. Similar theoretical frameworks have been applied to AI literacy. For instance, Kong et al. expressed their AI literacy learning objectives in terms of “cognitive”, “affective”, and “sociocultural” \citep{kong_evaluating_2023}. The Dagstuhl Triangle scaffolds the conceptual framework of this review, as the perspectives effectively capture the prominent ways AI literacy scholars discuss AI. 

The concept of literacy is similarly difficult to define. “Literacy” is borrowed from its original context of reading and writing, which adds some confusion. Even in its original meaning, literacy is complex, entailing different objectives and approaches \citep{scribner_literacy_1984}. In her 1984 work, Scribner described literacy in three metaphors, as adaptation---the functional knowledge to perform reading and writing tasks as society requires, as power---being empowered to communicate and organize for community advancement, and as state of grace---self-improvement that signals greater social status, virtue, or intelligence \citep{scribner_literacy_1984}. Scribner’s underlying perspectives on literacy---as functional knowledge, as empowerment and critical thinking, and as indirectly beneficial virtues---are valuable when making sense of the divergent motivations behind promoting AI literacy.

More closely related to AI literacy, there is an extensive body of literature on digital literacy and other technological literacies, such as data literacy \citep{gummer_building_2015} and computational literacy \citep{disessa_computational_2018}. Echoing Scribner’s three metaphors of literacy, Selber described an influential framework of digital multiliteracies, which includes functional, critical, and rhetorical literacies \citep{selber_multiliteracies_2004}. Selber's functional literacy and critical literacy share significant similarities with Scribner’s first two metaphors, meaning functional skills of using digital technologies and critical evaluation of digital artifacts respectively \citep{selber_multiliteracies_2004}. Selber’s rhetorical literacy is a combination of functional and critical capabilities that enable individuals to create new digital media to express themselves \citep{selber_multiliteracies_2004}. Other digital literacy frameworks are more focused on listing fine-grained competencies \citep{eshet_digital_2004}, which reflects some of the aforementioned AI literacy reviews.