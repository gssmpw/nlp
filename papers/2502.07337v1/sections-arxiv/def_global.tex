\usepackage{bm}
\usepackage{float}
\usepackage{amsthm}
\usepackage{amsmath}
\usepackage{amssymb}
\usepackage{booktabs}
\usepackage{graphicx}
\usepackage{graphbox}
\usepackage{notes2bib}
% \usepackage{subfigure}
\usepackage{subcaption}
\usepackage{multirow}
\usepackage{algorithm, algpseudocode}

\usepackage[colorlinks,linkcolor=blue]{hyperref}
% \usepackage[dvipsnames]{xcolor}
\definecolor{cite_color}{HTML}{114083}
\definecolor{link_color}{RGB}{0,102,102}%  green-style
\definecolor{link_color}{RGB}{153, 0,0}  %  red
\definecolor{url_color}{RGB}{153, 102,  0}
\definecolor{emp_color}{RGB}{0,0,255}
% {51, 102, 51}
\hypersetup{
 colorlinks,
 citecolor=cite_color,
 linkcolor=link_color,
 urlcolor=url_color}
 % graph
\graphicspath{{./figs/}}

\DeclarePairedDelimiterX{\infdivx}[2]{(}{)}{%
  #1\;\delimsize\|\;#2%
  }


\newcommand{\tx}{\tilde{x}}
\newcommand{\tp}{\tilde{p}}
\newcommand{\tq}{\tilde{q}}
\newcommand{\SFD}{\widetilde{\mathrm{FD}}}
\newcommand{\balpha}{\bar{\alpha}}
\newcommand{\bbeta}{\bar{\beta}}
\newcommand*\dif{\mathop{}\!\mathrm{d}}

\newcommand{\infdiv}{\operatorname{KL}\infdivx}
\def\E{{\mathbb E}}
\def \w{\mathbf{w}}
\def \x{\mathbf{x}}
\def \y{\mathbf{y}}
\def \a{\mathbf{a}}
\def \b{\mathbf{b}}
\def \h{\mathbf{h}}
\def \z{\mathbf{z}}
\def \xib{\boldsymbol{\xi}}

\def \ED{\mathrm{ED}}
\def \KL{\mathrm{KL}}
\def \KLC{\mathrm{KLC}}
\def \data{\mathrm{data}}
\def \ebm{\mathrm{ebm}}
\def \e{\mathbf{e}}

% bold large letters
\def \W{\mathbf{W}}
\def \J{\mathbf{J}}

\def \xdisc{\hat{\x}}

\renewcommand{\mid}{|}

\newtheorem{innercustomgeneric}{\customgenericname}
\providecommand{\customgenericname}{}
\newcommand{\newcustomtheorem}[2]{%
  \newenvironment{#1}[1]
  {%
   \renewcommand\customgenericname{#2}%
   \renewcommand\theinnercustomgeneric{##1}%
   \innercustomgeneric
  }
  {\endinnercustomgeneric}
}

\newcustomtheorem{customthm}{Theorem}


\DeclareMathOperator*{\argmax}{argmax}
\DeclareMathOperator*{\argmin}{argmin}

% table setting
\newcommand\ica[1][0.25]{\hspace{#1cm}}

\newcommand{\norm}[1]{\left\lVert#1\right\rVert}

\newtheorem{theorem}{Theorem}
\newtheorem{proposition}{Proposition}
\newtheorem{lemma}{Lemma}
\newtheorem{corollary}{Corollary}
\newtheorem{definition}{Definition}
\newtheorem{assumption}{Assumption}
\newtheorem*{remark}{Remark}

\usepackage{thmtools}
\usepackage{thm-restate}

\usepackage{tikz}
\usetikzlibrary{decorations.pathreplacing,calc}
\newcommand{\tikzmark}[1]{\tikz[overlay,remember picture] \node (#1) {};}

\newcommand*{\AddNote}[5]{%
    \begin{tikzpicture}[overlay, remember picture]
        \draw [decoration={brace,amplitude=0.5em},decorate,ultra thick,ForestGreen]
            ($(#4)!(#2.north)!($(#4)-(0,1)$)$) --  
            ($(#4)!(#3.south)!($(#4)-(0,1)$)$)
                node [align=center, text width={#1}, pos=0.5, anchor=west] {#5};
    \end{tikzpicture}
}%

% colorful text
\newcommand{\red}[1]{{\color{red} #1}}
\newcommand{\blue}[1]{{\color{blue} #1}}

\newcommand\mpwid{0.146}
\newcommand\hinterval{0.23cm}
\newcommand\samplempwid{0.146}
\newcommand\samplehinterval{0.23cm}
\newcommand\samplefigwid{\textwidth}

\usepackage{xspace}
\newcommand\EDB{\text{ED-Bern}\xspace}
\newcommand\EDP{\text{ED-Pool}\xspace}
\newcommand\EDG{\text{ED-Grid}\xspace}
\newcommand\EDBG{\text{ED-}$\nabla$\text{Bern}\xspace}
\newcommand\EDGG{\text{ED-}$\nabla$\text{Grid}\xspace}
