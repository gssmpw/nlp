\section{Background: Continuity Equation}
In this section, we introduce the key preliminary: continuity equation. Let $p_t, t \in [0,1]$ be the probability path on $\mathbb{R}^d$, we say the velocity $v_t: \mathbb{R}^d \rightarrow \mathbb{R}^d$ generates the path $p_t$ if the continuity equation holds: $\partial_t p_t (x) = -\nabla_x \cdot [p_t(x) v_t(x)], \forall x \in \mathbb{R}^d$, where $\nabla_x \cdot$ denotes the divergence operator. Thus sampling from the path $p_t$ can be done by solving the integral $x_t = x_0 + \int_{s=0}^t v_s (x_s) \dif s$, with $x_0 \sim p_0$. Dividing both sides of the continuity equation by $p_t$ leads to 
\begin{align} \label{eq:partial_t_log_pt}
    \partial_t \log p_t (x) = - \nabla_x \cdot v_t (x) - v_t (x) \cdot \nabla_x \log p_t (x), \forall x \in \mathbb{R}^d.
\end{align}
This equation further leads to the instantaneous change of variable
\begin{align} \label{eq:change-of-variable}
    \partial_t [\log p_t(x_t)] = \partial_t \log p_t (x_t) + \nabla_{x_t} \log p_t (x_t) \cdot v_t (x_t) = - \nabla_{x_t} \cdot v_t (x_t),
\end{align}
where $\partial_t [\log p_t(x_t)]$ denotes the total derivative w.r.t. $t$ and we apply the fact that $\partial_t x_t = v_t (x_t)$. Thus, \cref{eq:change-of-variable} can be used to evaluate the log-likelihood of the sample $x_1 \sim p_1$. Next, we introduce how to employ \cref{eq:partial_t_log_pt} to learn a model-based velocity for sampling from the target density $\pi$, followed by the further improvement of using shortcut models.