\section{Related Work}
Mathematical theorem proving in machine learning has garnered significant attention, particularly for developing artificial intelligence capable of logical reasoning. As a result, various studies have been conducted, with those most closely related to our work outlined below.

\subsubsection{Formal and informal languages}

Languages used for the integration of machine learning with mathematical reasoning can be broadly categorized as either formal or informal.
Informal language has a format that is familiar to the human eye and is thus easily understood and used. 
The authors in \cite{math_machine_intution} propose a machine learning-based framework and leverage informal language to offer mathematical insight, and the models in \cite{tora, NaturalLanguage} are fine-tuned to prove mathematical theorems in informal language. 
However, informal language proofs have the disadvantage of being difficult to verify. To address this issue, many studies have utilized verifiable programming languages, spanning across formal languages such as Lean \cite{ProofNet,pact,curriculum,leandojo}, Metamath \cite{GPTf}, HOL4 \cite{TacticZero}, Isabelle \cite{thor,DraftSketchProve,lego}, Coq \cite{CoqGym}, Python \cite{codex}, or a mix of these environments \cite{HyperTree,copra}. 
These formal languages can represent mathematical ideas precisely, making them easier to utilize in a computing system. However, formal languages differ vastly from the mathematical arguments written by humans, making it difficult to interpret without ample understanding of the formal setting used. Given that we consider the verifiability of proofs to be essential, we choose to work with formal language to construct the models in our study, despite its lower accessibility.

\subsubsection{Large language models with theorem proving}
Large language models (LLMs) have become increasingly useful in natural language processing \cite{gpt4,gpt3,codex,palm,llama}, inspiring many studies on these models.
The researchers from \cite{MIT} use Codex to find solutions for informal language problems by converting the problems into programming tasks, while the authors of \cite{copra} improve their model's performance by composing effective prompts for ChatGPT.
Many also create new models based on LLMs, such as Llemma \cite{llemma}, which achieves high performance in mathematical proof generation by fine-tuning LLaMA \cite{llama} on newly proposed data.
Others, as those in \cite{DraftSketchProve,lego}, link together LLMs and theorem proving by writing formal proofs informed by the informal proofs generated for a mathematical problem.
In this paper, we aim to construct a simple and easily reproducible model that can be tried by anyone.
To this end, we utilize ChatGPT and enhance its performance by modifying existing proof search algorithms specifically tailored for this LLM.

\subsubsection{Development of proof search algorithms}
There have been various efforts towards developing formal language models. Among these, there are many models that have modified the proof search algorithm in innovative ways, often displaying strong performances. A pioneer in this endeavor is GPT-f \cite{GPTf}, which introduced the proof search process and evaluation criteria for the field. 
Some models rely on extra training, such as expert iteration in \cite{curriculum} to tackle problems of increasing difficulty and control the lengths of formal proofs, or reinforcement learning in \cite{TacticZero}.
Other models focus on premise selection to call relevant theorems or other mathematical statements. For example, Thor \cite{thor} utilizes Hammer to aid the proof search process by choosing appropriate premises in interactive theorem provers, and LeanDojo \cite{leandojo} and DS-Prover \cite{ds-prover} integrate models dedicated to this task.

Some of the leading models have displayed impressive prowess in proof generation, such as the HyperTree proof search \cite{HyperTree}, a new method related to the Monte Carlo tree search which uses online training to achieve the highest performance in the field.
Also, AlphaGeometry \cite{alphageometry} has recently demonstrated exceptional performance in the field of geometry proof search by combining a symbolic deduction engine, a type of computer program, with language models fine-tuned on more than 100 million geometry problems. 
We aim to preserve the overall structure of the existing proof search process, but make adaptations to the proof search algorithm to explore a wider range of proofs. Additionally, we make our adjustments to enhance performance based on ChatGPT's specific parameters, bypassing the challenges that existing algorithms face when used with ChatGPT.