\section{Quantitative example from \S\ref{sec:cisc}}
\label{appendix:example}

Consider a simplified \emph{binary} setting in which there are two possible answers: correct and incorrect.  Given a number of samples $n$ and a probability $p=0.6$ of generating the correct answer, the number of samples with the correct answer follows the Binomial distribution  $X \sim \text{Binomial}(n, p)$. For such distribution, the majority vote is accurate whenever $X > \frac{n}{2}$ and it has $50\%$ chance to be accurate when $X = \frac{n}{2}$ (i.e., a random choice). 

Now, to illustrate how the self-assessment score of LLMs can be helpful, consider that we have an oracle that assigns twice the weight for answers that are correct. In this case, a weighted majority vote would be accurate whenever $X > \frac{n}{3}$ and it has $50\%$ chance to be accurate when $X = \frac{n}{3}$. 

In Figure \ref{fig:thought_experiment} we plot the relationship between, (x-axis) the number of samples, and (y-axis) the accuracy of the \emph{weighted} majority vote over these samples. The graph features two lines: (blue) each sample gets an equal weight, and (orange) correct answers are assigned twice the weight of incorrect ones.

While this intuition about cost-saving also applies to the general case of an \emph{arbitrary} set of answers, this setting is trickier to analyze in closed-form  because the specific distribution of incorrect answers impacts the majority vote. E.g., an answer that appears only 20\% of the time can still be correct under majority vote if all the other 80\% incorrect answers are different from one another. This could be obtained by placing additional distributional assumptions on the sampled answers. The analysis of the binary case can be thought of as a worst-case analysis of the general case, since in the worst case, all the incorrect answers are identical and the majority will be accurate if and only if more than half the sampled answers are correct.

\begin{figure}[h]
\setlength{\belowcaptionskip}{-10pt}
    \centering
    \includegraphics[width=1\linewidth]{latex/figures/thought_experiment.pdf}
    \caption{The relationship between the number of samples (x-axis) and the accuracy of majority vote over these samples (y-axis), for two different hypothetical cases: 
     (blue) Each sample receives an equal weight in majority voting, and (orange) Correct answers are assigned double the weight of incorrect ones. Adding this additional weighting information translated into $4X$ reduction in the number of samples required for the majority vote to reach 90\% accuracy.
    }
    \label{fig:thought_experiment}
    % \vspace{-0.2cm}
\end{figure}