\section{Introduction}
%
Internet-of-Things (IoT), namely smart devices are becoming increasingly widespread. They are encountered in many settings,
including home, office, factory, public venues, and transportation of all sorts. Such devices are specialized sensors,
actuators, or hybrids thereof; and their computational and communication resources are usually meager due to size, cost,
and power constraints. Unsurprisingly, they represent an attractive set of targets for malware and other exploits.
For example, the most infamous IoT malware incident was the Mirai botnet attack that infected hundreds of thousands of 
DVRs, IP cameras, routers, and printers in 2016~\cite{antonakakis2017understanding}. There are many other types
of IoT-focused malware exploiting diverse and often new attack vectors. Malware's job is made easier in terms
of infection scalability by the general consumer tendency towards monocultures, as witnessed by the immense popularity 
of certain ``smart'' voice assistants, doorbells, cameras, and appliances. This motivates the need for new malware mitigation
techniques geared for the IoT ecosystem, distinct from those used to combat more traditional malware~\cite{alrawi2021circle}. 

\parheading{Limitations of Prior Work.}
The research community has been keenly aware of the IoT security issues and many prior works proposed various mitigation
techniques. A large body of such work is focused on Remote Attestation (\ra{}) -- a means of measuring the internal 
software state of a remote device to determine whether it is infected~\cite{sancus,grisafi2022pistis,ammar2020simple,vrasedp}. 
Most of prior work proposed and evaluated \textit{new} \ra{} techniques but often side-stepped an important practical
aspect: how to schedule the \ra? Doing it infrequently makes instantaneous detection of malware impossible. In the best case, it increases the detection latency and, in the worst case, allows the malware to come and go between successive measurements and thus escape undetected. Meanwhile, doing it too often is impractical and wasteful since each \ra instance consumes resources by 
taking the attesting device away from its main task(s), as well as expending its power due to computation and communication.

Another major direction of prior work is detection of malicious activity via Traffic Analysis (\ta)~\cite{tekiner2022lightweight,marin2019deep,meidan2018n,alrashdi2019ad,wozniak2020recurrent}.
Most \ta{} approaches rely on Machine Learning (ML) techniques, which are typically not lightweight 
and use many traffic features. 
Packets sent and received by \iot{} devices are captured and their relevant features are extracted and used to build data samples for training ML models. Then, during the monitoring phase, classifiers label each new data 
sample as either benign or suspicious. 
Nevertheless, \ta, by itself, only flags suspicious
activity and does nothing to confirm whether a device is indeed infected. Note that some \ta techniques blacklist
suspected devices without confirming infection and discard all of their traffic, which is overkill.

\parheading{\system{} Approach.}
To address the above issues, this paper constructs \system{}, a system that blends \ra and \ta, resulting in a more comprehensive IoT malware detection than either of the two alone. To the best of our knowledge, \system is the first to leverage the benefits of both \ta and \ra. We provide a comparison of \system with prominent existing approaches of \ta \cite{nguyen2019diot,meidan2018n} and \ra \cite{vrasedp} in Table \ref{tab:novelty table}.
\begin{table}[t!]
	\centering
        \caption{Comparison of \system with prior approaches.}
	\begin{tabularx}{\linewidth}
{p{15mm} p{20mm}p{18mm} p{18mm}}
  \toprule
  \textbf{Approach} & \textbf{Search Vector} & \textbf{Instantaneous} & \textbf{Malware} \\
  & & \textbf{Detection} & \textbf{Confirmation} \\
  \midrule
  \system{} & Network Packet \& Device Memory & \cellcolor{green!25}\cmark & \cellcolor{green!25}\cmark\\
  \midrule
  N-BaIoT  & Network Packet & \cellcolor{green!25}\cmark & \cellcolor{red!25}\xmark \\
  \midrule
  D\"IoT  & Network Packet & \cellcolor{green!25}\cmark & \cellcolor{red!25}\xmark \\
  \midrule
  VRASED  & Device Memory & \cellcolor{red!25}\xmark & \cellcolor{green!25}\cmark \\
  \bottomrule
\end{tabularx}
\vspace{-1em}
\label{tab:novelty table}
\end{table}

\system{} is aimed at a typical smart home/office 
IoT device deployment. Conveniently, 
traffic patterns of such devices tend to be limited and predictable. They usually communicate with only a handful of external endpoints and, as shown in prior work~\cite{trimananda2020packet,oconnor2019homesnitch}, variations in packet sizes are minor. 
Infected devices show diverging traffic patterns by communicating with different IP addresses and using different packet sizes, typically talking to their Command-and-Control Center (CCC) or other infected devices.

\system{} consists of three main components: \textit{Profiler},  \textit{Monitor}, and \textit{Attester}. 
Profiler trains one \ta{} profile per device by collecting all packets generated by the device during a 
typical (benign) operation, and extracting and storing three features per packet: endpoint addresses, 
packet length, and direction.
Next, Monitor detects potentially malicious activity in real time through \textit{whitelisting}: 
a packet is considered as suspicious if it does not match the features of any packets stored in the corresponding device profile.
Whenever that occurs, \system{} invokes Attester on the suspected device to determine whether it is infected. Infection can be detected by changes in the program memory of the device.
If an infection is confirmed, \system{} reports this incident; otherwise, it includes the triggered traffic pattern in the benign patterns database. Our experiments demonstrate that \system{} achieves 100\% True Positive Rate (TPR) with at most 1.2\%
false positive rate (FPR), on average.

\parheading{Contributions.} 
First, \system{} performs \textit{lightweight \ta} of IoT devices via {\em per-packet} detection, using only three features, to match each packet in real-time against device profiles while maintaining high detection accuracy. Notably, \system outperforms machine learning based state of the art approach, \cite{nguyen2019diot} with a 1.05$\times$ better TPR and 160$\times$ faster detection time. 

Second, \system{} benefits from triggering \ra only when a suspicious traffic pattern is detected. 
This also allows periodic/scheduled \ra{} to be done infrequently, which avoids unnecessary \ra{} 
instances that would otherwise waste device resources and take devices away from their actual primary 
tasks that might be safety-critical. In our evaluation, without the \ra{}-\ta{} combination, a camera with 1.84 W power consumption would need to consume \textit{an additional $\sim$26 Wh of power in a year for periodic attestation without any guarantee of detecting the malware.}

Furthermore, the entire \system{} system is made available as an open-source implementation at \cite{madea-repo}. Its proof-of-concept includes two components: 
(1) a prototype IoT device based on
Raspberry Pi 4, in which the Attester module is implemented
by extending the Raspbian OS to measure the currently
running processes and (2) an emulated router, 
also based on Raspberry Pi 4, which profiles and monitors device traffic.

\parheading{Organization.} 
Section~\ref{sec:background} provides background and related work in \ra{} and \ta{} for IoT malware.
Next, Section~\ref{sec:systemmodel} describes \system{}'s system and adversary models and
Section~\ref{sec:design} lays out the high-level design and implementation details.
Then, Section~\ref{sec: Evaluation} presents the evaluation results, followed by
Section~\ref{sec:discussion}, which discusses various aspects and limitations of \system.
Section~\ref{sec:conclusion} concludes the paper.