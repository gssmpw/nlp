\section{System \& Adversary Models} \label{sec:systemmodel}
This section describes \system\ system model and adversarial assumptions. 

\subsection{System Model} 
\system considers four entities: an IoT device, a gateway, a user device, and a remote server.
\begin{figure}
    \centering
    \includegraphics[width=\linewidth]{figures/System_Model.pdf}
    \caption{\system system model.}
    \label{fig:system_model}
\end{figure}

IoT device is a typical smart device (as in Section \ref{subsec:targeted_devices}) that is the main target of the adversary. It is assumed to be installed in a private space (such as a home/office setting) or a public place (such as a grocery store or a hotel), connected to a gateway. A gateway can be a local router, an IoT controller hub, or any other gateway that connects a set of IoT devices directly to (or towards) the Internet. We assume that an IoT device is equipped with a Trusted Computing Base (TCB), that performs device attestation, as in Section \ref{sec:background}. A remote server is a remote back-end server (generally hosted in the cloud) maintained by the IoT service provider. The server is responsible for registering, updating, and assisting the IoT device. A user device is a personal item (such as a smartphone, smartwatch, laptop, or tablet) that controls the IoT device via a dedicated app or a web interface. 

\system{} works in an IoT deployment with at least one IoT device, one or more user devices, and one gateway, as shown in Figure \ref{fig:system_model}. User devices communicate with IoT devices either directly or via the server. All packets from or to IoT devices pass through the gateway: we assume that peer-to-peer communication is not allowed. The gateway captures packets and makes decisions based on their features, without keeping the state of on-going traffic flows/connections. In the context of \ra, IoT devices and gateway play the roles of \prv and \vrf, respectively.

\subsection{Adversary Model} \label{subsec:advesarymodel}
We make the following assumptions about adversarial capabilities: 
\begin{itemize}
    \item \textit{Remote adversary:} infects IoT devices with malware over the network without having physical access to them, for example, 
    by exploiting a software vulnerability in an application running on IoT devices or similar to Mirai, through telnet by exploiting default/weak telnet credentials.
    \item \textit{TCB}: IoT device's TCB is trusted and not subject to compromise. Each TCB has a distinct pair of public and secret keys for computing signatures as part of \ra. Each public key is known to the gateway.
    \item \textit{Malware-generated traffic:} We assume that malware must generate some amount of traffic, in order to 
    communicate with its Command-and-Control Center (CCC). In most cases, malware activities are coordinated by CCC -- a remote server maintained by the adversary which controls a network of infected devices. This network is usually called a botnet and infected devices are called zombies/bots. CCC sends various commands to be executed by IoT devices, \eg{} to download code, infect other devices, or launch attacks. 
    \item \textit{Trusted entities:} We assume that the gateway, server, and user devices are sufficiently powerful computing platforms that can protect themselves from malware and remain trusted. Moreover, since the service provider is trusted, each newly deployed IoT device is assumed to be initially healthy. DNS resolver and other routing protocols are considered to be uncompromised: an adversary cannot spoof IP address of CCC to match the IP address of the remote server.
\end{itemize}

\subsubsection{DoS and Other Attacks}
An adversary can launch a DoS attack
on an IoT device by flooding it with a high volume of traffic from
the outside, i.e., from beyond the gateway. Alternatively, the adversary can use
infected IoT devices to launch a DoS attack on other peer (healthy) devices. \system detects both DoS attacks. Since we assume
that the gateway runs on a trusted high-end platform, DoS attacks on it are considered out-of-scope. DoS attacks on IoT devices are further discussed in Section VI.

We consider all physical attacks (i.e., those requiring adversary's
physical presence) to be out of the scope of MADEA. The
same holds for malware that does not involve any of its own
incoming or outgoing traffic. (Malware that targets actuation
capabilities and acts autonomously would fall into this category.)
Section \ref{sec:discussion} discusses this in more detail.

