\section{Related Work}
Unsupervised domain adaptation is a wildly used setting of transfer learning methods that aims to minimize the discrepancy between the source and target domains. To solve cross-domain classification tasks, these methods are based on deep feature representation~\citep{zhu2022weakly,zengtowards}, which maps different domains into a common feature space. Some recent studies have overcome the imbalance of domains and the label distribution shift of classes to transfer model well~\citep{jing2021towards,xu2023class,pu2024leveraging,zeng2024latent}. Some novel settings in domain adaption have also gotten a lot of attention, like source free domain adaption(SFDA)~\citep{yang2021generalized, xu2025unraveling}, test time domain adaption(TTDA)~\citep{wang2022continual}.
As for graph-structured data, several studies have been proposed for cross-graph knowledge transfer via GDA setting methods~\citep{shen2019network,dai2022graph,shi2024graph}. ACDNE~\citep{shen2020adversarial} adopt k-hop PPMI matrix to capture high-order proximity as global consistency for source information on graphs. CDNE~\citep{shen2020network} learning cross-network embedding from source and target data to minimize the maximum mean discrepancy (MMD) directly. GraphAE~\citep{yan2020graphae} analyzes node degree distribution shift in domain discrepancy and solves it by aligning message-passing routers. DM-GNN~\citep{shen2023domain} proposes a method to propagate node label information by combining its own and neighbors’ edge structure. 
UDAGCN~\citep{wu2020unsupervised} develops a dual graph convolutional network by jointly capturing knowledge from local and global levels to adapt it by adversarial training. ASN~\citep{zhang2021adversarial} separates domain-specific and domain-invariant variables by designing a private en-coder and uses the domain-specific features in the network to extract the domain-invariant shared features across networks.  SOGA~\citep{mao2021source} first time uses discriminability by encouraging the structural consistencies between target nodes in the same class for the SFDA in the graph. GraphAE~\citep{guo2022learning} focuses on how shifts in node degree distribution affect node embeddings by minimizing the discrepancy between router embedding to eliminate structural shifts.
SpecReg~\citep{you2022graph} used the optimal transport-based GDA bound for graph data and discovered that revising the GNNs’ Lipschitz constant can be achieved by spectral smoothness and maximum frequency response.  JHGDA~\citep{shi2023improving} studies the shifts in hierarchical graph structures, which are inherent properties of graphs by aggregating domain discrepancy from all hierarchy levels to derive a comprehensive discrepancy measurement. ALEX~\citep{yuan2023alex} first creates a label shift enhanced augmented graph view using a low-rank adjacency matrix obtained through singular value decomposition by driving contrasting loss. SGDA~\citep{qiao2023semi} enhances original source graphs by integrating trainable perturbations (adaptive shift parameters) into embeddings by conducting adversarial learning to simultaneously train both the graph encoder and perturbations, to minimize marginal shifts.