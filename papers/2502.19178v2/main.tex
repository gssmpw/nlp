%%
%% This is file `sample-sigconf.tex',
%% generated with the docstrip utility.
%%
%% The original source files were:
%%
%% samples.dtx  (with options: `all,proceedings,bibtex,sigconf')
%% 
%% IMPORTANT NOTICE:
%% 
%% For the copyright see the source file.
%% 
%% Any modified versions of this file must be renamed
%% with new filenames distinct from sample-sigconf.tex.
%% 
%% For distribution of the original source see the terms
%% for copying and modification in the file samples.dtx.
%% 
%% This generated file may be distributed as long as the
%% original source files, as listed above, are part of the
%% same distribution. (The sources need not necessarily be
%% in the same archive or directory.)
%%
%%
%% Commands for TeXCount
%TC:macro \cite [option:text,text]
%TC:macro \citep [option:text,text]
%TC:macro \citet [option:text,text]
%TC:envir table 0 1
%TC:envir table* 0 1
%TC:envir tabular [ignore] word
%TC:envir displaymath 0 word
%TC:envir math 0 word
%TC:envir comment 0 0
%%
%%
%% The first command in your LaTeX source must be the \documentclass
%% command.
%%
%% For submission and review of your manuscript please change the
%% command to \documentclass[manuscript, screen, review]{acmart}.
%%
%% When submitting camera ready or to TAPS, please change the command
%% to \documentclass[sigconf]{acmart} or whichever template is required
%% for your publication.
%%
%%
% \documentclass[sigconf,review]{acmart}

\documentclass[sigconf]{acmart}

\usepackage{threeparttable}
\usepackage{algorithm}
\usepackage{algorithmic}
\usepackage{booktabs}
\usepackage{enumitem}
\usepackage{multirow}
\usepackage{multicol}
\usepackage{array}
\usepackage{pifont}
\newcommand{\cmark}{\ding{51}} % 勾
\newcommand{\xmark}{\ding{55}} % 叉
\usepackage{listings}
\usepackage{amsmath}
\usepackage{color}
\usepackage{xcolor}
\newcommand{\lm}[1]{{\color{red} [llm: #1]}}

\usepackage{soul, color, xcolor,xspace}
\newcommand{\name}{UQABench\xspace}

% \usepackage{caption} % Include this in the preamble
% \captionsetup{skip=20pt} % Set the desired distance, for example, 10pt

% \usepackage{titlesec}
% % 调整 subsubsection 的间距
% \titlespacing{\subsubsection}{0pt}{1ex}{1.5ex}

%%
%% \BibTeX command to typeset BibTeX logo in the docs
\AtBeginDocument{%
  \providecommand\BibTeX{{%
    Bib\TeX}}}

%% Rights management information.  This information is sent to you
%% when you complete the rights form.  These commands have SAMPLE
%% values in them; it is your responsibility as an author to replace
%% the commands and values with those provided to you when you
%% complete the rights form.
\setcopyright{acmlicensed}
\copyrightyear{2018}
\acmYear{2018}
\acmDOI{XXXXXXX.XXXXXXX}

%% These commands are for a PROCEEDINGS abstract or paper.
\acmConference[Conference acronym 'XX]{Make sure to enter the correct
  conference title from your rights confirmation emai}{June 03--05,
  2018}{Woodstock, NY}
%%
%%  Uncomment \acmBooktitle if the title of the proceedings is different
%%  from ``Proceedings of ...''!
%%
%%\acmBooktitle{Woodstock '18: ACM Symposium on Neural Gaze Detection,
%%  June 03--05, 2018, Woodstock, NY}
\acmISBN{978-1-4503-XXXX-X/18/06}


%%
%% Submission ID.
%% Use this when submitting an article to a sponsored event. You'll
%% receive a unique submission ID from the organizers
%% of the event, and this ID should be used as the parameter to this command.
%%\acmSubmissionID{123-A56-BU3}

%%
%% For managing citations, it is recommended to use bibliography
%% files in BibTeX format.
%%
%% You can then either use BibTeX with the ACM-Reference-Format style,
%% or BibLaTeX with the acmnumeric or acmauthoryear sytles, that include
%% support for advanced citation of software artefact from the
%% biblatex-software package, also separately available on CTAN.
%%
%% Look at the sample-*-biblatex.tex files for templates showcasing
%% the biblatex styles.
%%

%%
%% The majority of ACM publications use numbered citations and
%% references.  The command \citestyle{authoryear} switches to the
%% "author year" style.
%%
%% If you are preparing content for an event
%% sponsored by ACM SIGGRAPH, you must use the "author year" style of
%% citations and references.
%% Uncommenting
%% the next command will enable that style.
%%\citestyle{acmauthoryear}


%%
%% end of the preamble, start of the body of the document source.
\begin{document}

%%
%% The "title" command has an optional parameter,
%% allowing the author to define a "short title" to be used in page headers.
\title{UQABench: Evaluating User Embedding for Prompting LLMs in Personalized Question Answering}

%%
%% The "author" command and its associated commands are used to define
%% the authors and their affiliations.
%% Of note is the shared affiliation of the first two authors, and the
%% "authornote" and "authornotemark" commands
%% used to denote shared contribution to the research.

% \author{Ben Trovato}
% \authornote{Both authors contributed equally to this research.}
% \email{trovato@corporation.com}
% \orcid{1234-5678-9012}
% \author{G.K.M. Tobin}
% \authornotemark[1]
% \email{webmaster@marysville-ohio.com}
% \affiliation{%
%   \institution{Institute for Clarity in Documentation}
%   \city{Dublin}
%   \state{Ohio}
%   \country{USA}
% }

\author{Langming Liu$^{\dagger}$, Shilei Liu$^{\dagger}$, Yujin Yuan, Yizhen Zhang, Bencheng Yan, Zhiyuan Zeng, Zihao Wang, Jiaqi Liu, Di Wang, Wenbo Su, Wang Pengjie, Jian Xu, Bo Zheng}
% \authornote{$^{\dagger}$Both authors contributed equally to this work.}
\affiliation{%
  \institution{Taobao \& Tmall Group of Alibaba}
  \city{}
  \country{}}

% \author{Haibin Chen}
% \affiliation{%
%   \institution{Taobao \& Tmall Group of Alibaba}
%   \city{Hangzhou}
%   \country{China}}

% \author{Yuhao Wang}
% \affiliation{%
%   \institution{City University of Hong Kong}
%   \city{Hong Kong}
%   \country{China}}

% \author{Yujin Yuan}
% \affiliation{%
%   \institution{Taobao \& Tmall Group of Alibaba}
%   \city{Hangzhou}
%   \country{China}}

% \author{Shilei Liu}
% \affiliation{%
%   \institution{Taobao \& Tmall Group of Alibaba}
%   \city{Hangzhou}
%   \country{China}}
  
% \author{Wenbo Su}
% \affiliation{%
%   \institution{Taobao \& Tmall Group of Alibaba}
%   \city{Hangzhou}
%   \country{China}}

% \author{Xiangyu	Zhao}
% \affiliation{%
%   \institution{City University of Hong Kong}
%   \city{Hong Kong}
%   \country{China}}

% \author{Bo Zheng}
% \affiliation{%
%   \institution{Taobao \& Tmall Group of Alibaba}
%   \city{Hangzhou}
%   \country{China}}

%%
%% By default, the full list of authors will be used in the page
%% headers. Often, this list is too long, and will overlap
%% other information printed in the page headers. This command allows
%% the author to define a more concise list
%% of authors' names for this purpose.
\renewcommand{\shortauthors}{Langming Liu, et al.}


%%
%% The abstract is a short summary of the work to be presented in the
%% article.
\begin{abstract}
Large language models (LLMs) achieve remarkable success in natural language processing (NLP). In practical scenarios like recommendations, as users increasingly seek personalized experiences, it becomes crucial to incorporate user interaction history into the context of LLMs to enhance personalization. 
However, from a practical utility perspective, user interactions' extensive length and noise present challenges when used directly as text prompts. 
A promising solution is to compress and distill interactions into compact embeddings, serving as soft prompts to assist LLMs in generating personalized responses. Although this approach brings efficiency, a critical concern emerges: Can user embeddings adequately capture valuable information and prompt LLMs?
To address this concern, we propose \name, a benchmark designed to evaluate the effectiveness of user embeddings in prompting LLMs for personalization. We establish a fair and standardized evaluation process, encompassing pre-training, fine-tuning, and evaluation stages. To thoroughly evaluate user embeddings, we design three dimensions of tasks: sequence understanding, action prediction, and interest perception. These evaluation tasks cover the industry's demands in traditional recommendation tasks, such as improving prediction accuracy, and its aspirations for LLM-based methods, such as accurately understanding user interests and enhancing the user experience.
We conduct extensive experiments on various state-of-the-art methods for modeling user embeddings. Additionally, we reveal the scaling laws of leveraging user embeddings to prompt LLMs.  
% The benchmark is available online at~\url{https://github.com/ming429778/UQABench}.
The benchmark is available online at~\url{https://github.com/OpenStellarTeam/UQABench}.

\end{abstract}

% Large language models (LLMs) have achieved remarkable success in natural language processing (NLP). In practical scenarios like recommendation systems, as users increasingly seek personalized experiences, it becomes crucial to incorporate user interaction history into the context of LLMs to enhance personalization. However, from a utility perspective, user interactions' extensive length and noise present challenges when used directly as text prompts. A promising solution is to compress and distill user interactions into compact embeddings, serving as soft prompts to assist LLMs in generating personalized responses. Although this approach brings efficiency, a critical concern emerges: Can user embeddings adequately capture valuable information and convey user interests to LLMs?
% To address this concern, we introduce \name, a comprehensive benchmark designed to evaluate the effectiveness of user embeddings in prompting LLMs for personalization. We establish a fair and standardized evaluation process, encompassing pre-training, fine-tuning, and evaluation stages. To thoroughly evaluate user embeddings, we design three dimensions of tasks: sequence understanding, action prediction, and interest perception. These evaluation tasks cover the industry's demands in traditional recommendation tasks, such as improving prediction accuracy, and its aspirations for LLM-based methods, such as accurately understanding user interests and enhancing the user experience.
% We conduct extensive experiments to compare various state-of-the-art methods for modeling user embeddings. Additionally, we reveal the scaling laws in modeling user interests using Transformers, the most widely adopted sequential model.

%%
%% The code below is generated by the tool at http://dl.acm.org/ccs.cfm.
%% Please copy and paste the code instead of the example below.
%%
\begin{CCSXML}
<ccs2012>
   <concept>
       <concept_id>10010147.10010178.10010179.10010186</concept_id>
       <concept_desc>Computing methodologies~Language resources</concept_desc>
       <concept_significance>500</concept_significance>
       </concept>
 </ccs2012>
\end{CCSXML}

\ccsdesc[500]{Computing methodologies~Language resources}

%%
%% Keywords. The author(s) should pick words that accurately describe
%% the work being presented. Separate the keywords with commas.
\keywords{Large Language Models, Recommendation, Personalization}
%% A "teaser" image appears between the author and affiliation
%% information and the body of the document, and typically spans the
%% page.

\maketitle

\sloppy

\section{Introduction}\label{sec:introduction}
% -- Outline
% ---- LLMs are popular
% ---- There're many stakeholders in the training and inference loop
% ---- Adversaries in the training loop are a problem -- malpractice, poisoning
% ---- Also, showing compliance
% ---- Need a framework to prove the integrity of the pipeline
% ---- Enter Atlas

% ---- LLMs are popular
In recent years, machine learning (ML) models, have become increasingly popular.
The pervasive use of large language models (LLMs), in particular, and multi-stakeholder
involvement in model creation and deployment exacerbate security and privacy risks.
These considerations are emphasized by the global nature and the complexity of
large-scale ML deployments with different lifecycle stages:
%gathering and sanitizing the data from different sources,
%training and inferencing across many data centers,
%compliance with local laws or corporate policies.

% ---- There're many stakeholders in the training and inference loop
%Additionally, different stages of the ML development pipeline come with their own stakeholders:
\begin{enumerate}[label=\arabic*)]
    \item Collection and sanitation of a \emph{training} dataset from several public and proprietary sources.
    %\item Solicitation and facilitation of training.
    \item Provisioning of the training environment (hardware and software).
    \item Execution of training across many data centers.
    \item Construction of a \emph{testing} dataset from several sources, and the evaluation.
    \item Deployment and use of the model for inference that is compliant with local laws or corporate policies.
    %\item Use of the model in compliance with local laws or corporate policies.
\end{enumerate}

% ---- Adversaries in the training loop are a problem -- malpractice, poisoning
Each of these stages is vulnerable to malicious or dishonest parties.
For example, data can be poisoned~\cite{biggio2012poisoning,carlini2024poisoning} during collection or training.
Service providers executing outsourced training can shorten or omit critical steps to reduce their cost.
Model providers can serve smaller models in SaaS, or even distribute malicious ones.

% ---- Also, showing compliance
On the other hand, responsible model builders and other stakeholders may be incentivised or required to provide security and trust guarantees.
They may want to prove low bias in their training data, offer easily verifiable performance claims, or guarantee end-to-end integrity of the model creation in high risk domains.

% ---- Need a framework to prove the integrity of the pipeline
To address these challenges, it is necessary to guarantee the integrity of the entire ML lifecycle --
beginning with the data, through the training, and finally, the evaluation and deployment.
Was the data modified?
Did the hardware and software environment adhere to the specification?
Did the contractor follow the specified training procedure?
Can I trust the evaluation?
How can I guarantee that I am interacting with the intended model?
These are example questions that showcase the breadth of the involved challenges that must be tackled to provide end-to-end security.

% --- Enter Atlas
In this work, we introduce \atlas, a framework for enhancing the security and transparency of the lifecycle of ML models.
\atlas establishes the baseline of fundamental components and capabilities needed for comprehensive provenance tracking
at each stage of the ML lifecycle.
Subsequently, \atlas defines the core integrity requirements for verifiable ML lifecycle transparency.
We provide a reference implementation that instantiates \atlas using hardware-based security mechanisms -- with trusted execution environment (TEE),
including attestations.% , and comprehensive metadata-based provenance tracking.
%Our implementation satisfies all \atlas requirements.

We claim the following contributions:
\begin{enumerate}[label=\arabic*.]\label{sec:introduction:contributions}
    \item We introduce \atlas, a framework designed for end-to-end ML lifecycle transparency.
    \item We instantiate \atlas using TEEs and metadata-based provenance tracking.
    \item We evaluate our \atlas prototype through two case studies:
        \begin{enumerate*}[label=\arabic*)]
            \item fine-tuning of a BERT model~\cite{lin2023metabert, lin2023metabertimpl};
            \item fine-tuning of a bge-reranker model~\cite{chen2023bge}
        \end{enumerate*}.
\end{enumerate}

%\msm{revise: Integrate this motivation into intro}
%Organizations frequently leverage pre-trained models, outsource training processes, and integrate components from multiple sources,
%making it difficult to verify the authenticity and trustworthiness of their ML systems. This complexity is further compounded
%by the potential for malicious modifications at various stages of the model lifecycle, from data preparation through deployment.
%The involvement of various third parties in ML model development and deployment
%creates critical challenges in ensuring supply chain integrity.
%
%While Software Bills of Materials (SBOMs) and AI Bills of Materials (AI BOMs) provide basic inventory tracking for model components,
%they fall short in addressing the dynamic nature of ML pipelines. These approaches typically offer point-in-time snapshots but
%fail to capture the complex transformations, fine-tuning operations, and runtime modifications that characterize modern ML workflows.
%Additionally, they lack cryptographic guarantees about the integrity of recorded information and cannot effectively track the provenance
% of model weights and training data.
%
% These approaches demonstrate the growing importance of ML supply chain security.
% However, they are typically applied in an ad-hoc fashion, highlighting the need
% for a more integrated approach that combines comprehensive lineage tracking,
% strong cryptographic properties, and practical integration capabilities with existing ML development and deployment pipelines.
%
%A comprehensive solution requires not just documentation of components, but verifiable evidence of their origins,
%transformations, and integrity throughout the entire model lifecycle. This need has driven interest in more robust
%provenance tracking mechanisms that can:
%
%\begin{itemize}
%\item Provide cryptographic proof of model lineage
%\item Track and verify all pipeline transformations
%\item Maintain tamper-evident records of training processes
%\item Ensure integrity of model artifacts across organizational boundaries
%\end{itemize}
%
%Several existing tools and frameworks
%commonly focusing on different components of the model lifecycle and provenance tracking.
%While these solutions offer valuable capabilities, they often address only specific parts of the end-to-end ML
%supply chain rather than providing comprehensive coverage.
%\msm{end-revise}
%
%\todo{add discussion of EU-CRA AI Act requirements for model documentation and FDA guidelines for AI/ML in healthcare}

%The remainder of this paper is organized as follows:
%in Section~\ref{sec:background-related} we provide an overview of the necessary background, and the related work;
%Section~\ref{sec:problem} presents the challenge of providing integrity in the ML pipeline, the threat model, and the system assumptions;
%in Section~\ref{sec:framework} we present \atlas -- our framework for providing ML integrity;
%Section~\ref{sec:implementation} covers implementation details;
%in Section~\ref{sec:eval}, we show that \atlas is effective across three dimensions: training overhead $<8\%$, the verification time increases linearly with the size of the model, and it is compatible with PyTorch and Tensorflow;
%in Section~\ref{sec:casestudies} we present the case studies;
%in Section~\ref{sec:discussion} we discuss additional considerations for \atlas,
%and Section~\ref{sec:conclusion} concludes the paper and provides directions for future work.

\begin{figure}[t]
    \centering
    % \vspace{-4mm}
    \includegraphics[width=1.0\linewidth]{pics/compare.pdf}
    % \vspace{-4mm}
    \caption{
    A brief comparison of SRs and GRs.
    }
    \label{fig:compare}
    \vspace{-2mm}
    % \captionsetup{belowskip=-20pt}
\end{figure}


\section{Preliminary}
We briefly introduce how both previous and current recommender systems utilize user interactions, focusing on two prevailing branches: sequential recommendations (SRs) and generative recommendations (GRs). The brief comparison is illustrated in Figure~\ref{fig:compare}.

\subsection{Sequential Recommendations (SRs)}
Given a set of users \(\mathcal{U}=\{u_1,u_2,\cdots,u_{\vert\mathcal{U}\vert}\}\) and a set of items \(\mathcal{V}=\{v_1,v_2,\cdots,v_{\vert\mathcal{V}\vert}\}\), consider that the interaction sequence of user \(u_i\) is denoted as \(s_i=[v_{1}^i, v_{2}^i, \cdots, v_{n_i}^i]\). For user \(u_i\), SRs take \(s_i\) as input and calculate recommendation scores (e.g., cosine similarity) for candidate items, then output the top-\(k\) items most likely to be interacted with in the subsequent time step.

\subsection{Generative Recommendations (GRs)}
GRs represent a shift towards utilizing LLMs to generate more personalized recommendation results, divided into two main branches:

\subsubsection{\textbf{Text-based GRs.}} 
Text-based GRs directly leverage the textual information of interactions. Interacted items' IDs and textual attributes, such as titles, categories, and descriptions, are chronologically placed in pre-designed prompts to serve as user context. Given questions like "What item will the user click next," the LLMs generate responses such as "item $j$."

\subsubsection{\textbf{Embedding-based GRs.}}
In contrast, embedding-based GRs transform lengthy, noisy interaction sequences into information-dense user embeddings. These user embeddings, aligned to the semantic space by an adapter, act as soft prompts for LLMs, personalizing them further. Compared to text-based GRs, this approach requires fewer tokens to convey user context, enhancing efficiency and scalability. However, a critical question remains: "Can this approach provide comparable personalized performance to text-based GRs?" This question forms one of the core research focuses of this paper.


% \section{Preliminary}
% We briefly introduce how both previous and current recommender systems utilize user interactions, focusing respectively on two prevailing branches: sequential recommendations (SRs) and generative recommendations (GRs).
% The brief comparison is shown in Figure~\ref{fig:compare}.

% \subsection{Sequential Recommendations (SRs)}
% Given a set of users \(\mathcal{U}=\{u_1,u_2,\cdots,u_{\vert\mathcal{U}\vert}\}\) and a set of items \(\mathcal{V}=\{v_1,v_2,\cdots,v_{\vert\mathcal{V}\vert}\}\), consider that the interaction sequence of user \(u_i\) is denoted as \(s_i=[v_{1}^i, v_{2}^i, \cdots, v_{n_i}^i]\). For user \(u_i\), SRs take \(s_i\) as input and calculate the recommendation scores (e.g., cosine similarity) for candidate items, then output the top-\(k\) items that are most likely to be interacted with in the subsequent time step.

% \subsection{Generative Recommendations (GRs)}
% \subsubsection{\textbf{Text-based GRs.}} 
% GRs employ LLMs to generate more personalized recommendation results than SRs.
% One branch of GRs, text-based GRs, directly leverages the textual information of interactions. The interacted items' IDs and textual information, such as title, category, and description, are filled in the reserved positions chronologically in the pre-designed prompt, serving as user context. Then, given questions like "What item will the user click next," the LLMs will generate responses such as "item $j$."

% \subsubsection{\textbf{Embedding-based GRs.}}
% As the user interactions are usually lengthy and noisy, embedding-based GRs transform the interaction sequence into an information-dense form, i.e., user embeddings. The user embedding will be aligned to the semantic space by the adapter and then act as the soft prompt for LLMs, prompting them to be more personalized. Compared to text-based GRs, this approach needs far fewer tokens for the user context, meaning improved efficiency and scalability. The remaining concern is, "Can this approach provide comparable performance to text-based GRs?" Which is also the core research point of this paper. 
\section{Methodology}



In this section, we present our proposed approach and the rationale behind it. First, in \S\ref{sec:prelim-hit}, we present the preliminaries for the multi-head attention mechanism and ViTs. Next, in \S\ref{sec:mha} we will show that attention and multi-head attention output can be decomposed into the individual contributions of the inputs. Finally, in \S\ref{sec:presenting-hit} we present our novel architecture, the Hindered Transformer (HiT). The core of our method is to minimise the mixing of patch-level information, which allows us to express the classification token (\CLS) in ViTs as the sum of individual tokens, a direct result of \S\ref{sec:mha}. In other words, this simplification allows us to check the contribution of each token.

\subsection{Preliminaries: Transformers, ViTs and Notations}\label{sec:prelim-hit}

The transformer architecture is built upon the Scaled Dot-Product Attention operation~\cite{vaswani2017attention}, commonly referred to as the \emph{attention}. Given a query token sequence ${x}^q \in \mathbb{R}^{L_q \times d_{model}}$ and a target sequence (or key-value sequence) ${x}^t \in \mathbb{R}^{L_t \times d_{model}}$, where $L_q$ and $L_t$ are their respective sequence lengths and $d_{model}$ is the token dimension, the attention mechanism is computed as follows:
\begin{equation}\label{eq:attn}
\begin{split}
    Q &= x^q\,W_Q + b_Q \\
    K &= x^t\,W_K + b_K \\
    V &= x^t\,W_V + b_V \\
    A(x^q, x^t) &= softmax\left(\frac{QK^T}{\sqrt{d_k}}\right)V
\end{split}
\end{equation}
where the output is a sequence of the same length as $x^q$, $d_k$ is the dimension of the linear transformations, and $W_i\in\mathbb{R}^{d_{model}\times d_k}$ and $b_i\in\mathbb{R}^{d_k}$ are the weights of the linear projection $i\in\{Q, K, V\}$. In addition, Vaswani \textit{et al.}~\cite{vaswani2017attention} proposed to compute the attention mechanism $h$ times in parallel, setting $d_k = d_{model} / h$ for each individual attention operation. The resulting vectors of each individual attention, formally called heads, are concatenated and linearly post-processed to obtain the final result. This operation is called multi-head attention, and it is described as follows:
\begin{equation}
    MHA(x^q, x^t) = \underbrace{[A^1(x^q, x^t); ...; A^h(x^q, x^t)]}_{\texttt{Concatenate $h$ times}}W_o + b_o,
\end{equation}
with $A^i$ being the $i^{th}$ attention mechanism in the MHA, and $W_o\in\mathbb{R}^{d_{model}\times d_{model}}$ and $b_o\in\mathbb{R}^{d_{model}}$ the linear transformation parameters. 

In computer vision, to incorporate image data into this sequence-based formulation, the ViT first partitions the input image into $N^2$ equal-sized patches and linearly projects them to create the patch token sequence\footnote{For the rest of the paper, we will use the terms token and patch interchangeably, referring to the image patch tokens.}.
Additionally, following standard practice, a learnable classification token \CLS is prepended to the patch sequence. Furthermore, each patch token is summed with a positional embedding to encode its spatial location within the image. For the remainder of the paper, the sequence $x \in \mathbb{R}^{(N^2 + 1) \times d_{model}}$ denotes the concatenation of the patch tokens and the \CLS token, where $x[0]$ corresponds to the \CLS token.

The main ViT block builds on the MHA operation, followed by a token-wise MLP block, as in text-based transformers. 
Formally, given a set of patches $x_l$ at layer $l$, the ViT block first computes a globalized set of tokens using the MHA block. 
The resulting output is summed with a skip connection. 
Then, the output is fed into a token-wise MLP to post-process each token, followed, again, by a skip connection.
This block is summarized as follows
\begin{equation}\label{eq:vit-block}
\begin{split}
    x_{l}' &= x_l + MHA(x_l, x_l) \\
    x_{l+1} &= x_l' + MLP(x_l')
\end{split}
\end{equation}
Note that before the MHA and MLP blocks, a LayerNorm~\cite{ba2016layer} operation is applied to the data sequence, but for simplicity, we omit this operation.
Finally, the \CLS token is fed into a LayerNorm followed by a linear classifier to produce the logits of the classification task.

\subsection{Multi-Head Attention and Patch Mixing in Transformers}\label{sec:mha}


In this section, we aim to decompose the MHA operation to demonstrate that it is possible to retrieve the individual contributions of each token. In this way, we aim to lay the foundation for our architecture, which is described in the next section. 

Let's start by focusing on the attention operation (Eq.~\ref{eq:attn}). Since we will focus on the \CLS token later, and to simplify the analysis, let's assume that the query sequence has length $L_q=1$. 
Consequently, the attention mechanism can be rewritten as
\begin{equation}\label{eq:attn-decomp}
    A(x^q, x^t) = \sum_{v\in x^t} a(v, x^q, x^t)(v\,W_V + b_v),
\end{equation}
where $a(v, x^q, x^t)$ is the attention of a single token $v\in x_t$. Here, Eq.~\ref{eq:attn-decomp} shows that we can decompose the attention mechanism into separately processed patches - each patch $v$ in $x^t$ adds $a(v, x^q, x^t)(v\,W_V + b_v)$. Accordingly, if $x^t$ contains purely local information, the output of the attention is \emph{a sum of local data}.


To continue, we incorporate the previous observation into multi-head attention and verify that we can still unroll this operation into a \emph{sum of separate vectors}. One might be concerned that the concatenation-linear operation will mix each token. However, we argue that the result is still valid, since concatenating and linearly transforming the resulting vector is equivalent to linearly transforming each head and adding them together. 
Formally, by denoting $W_v^i$ and $b_v^i$ as the weights of the linear transformation generating the value sequence of $i^{th}$ head, and breaking apart $W_o$ into $h$ separate matrices, $W_o = [W_o^1; W_o^2; ...; W_o^h]$, with $W_o^i\in\mathbb{R}^{d_k\times d_{model}}$, then, the MHA becomes
\begin{equation}
\begin{split}
    MHA(x^q, x^t) &= b_o + \sum_{v\in x^t} v'(v) \\
    \text{where} \quad v'(v) &= \sum_{i=1}^h a(v, x^q, x^t) (v\,W_v^i+b_v^i) W_o^i.
\end{split}
\end{equation}
The previous result implies that we can still decompose the MHA result as the sum of vector patches, regardless of the number of heads in the MHA. So the same conclusion holds as in Eq.~\ref{eq:attn-decomp}: if the content in $x^t$ is local, then we can unravel the MHA mechanisms into \emph{local contributions}. %

\subsection{Untangling Visual Transformers}\label{sec:presenting-hit}

Unlike single MHA layers, ViTs operate on global features.
To integrate local information, these architectures use two mechanisms: the MHA layers, which spread the information within tokens, and the nonlinear MLPs, which introduce complex correlations even when applied to a linear combination of local contributions. For better explainability, it would be ideal if the classifier's decision could be expressed as a combination of information from individual patches, allowing a more interpretable understanding of how local information contributes to global predictions.

\begin{figure}
    \centering
    \includegraphics[width=0.9\linewidth]{images/hit-block.pdf}
    \caption{\textbf{ViT and HiT blocks.} While the ViT block mixes the patch data, HiT uniquely updates the \CLS via the MHA, but avoids post-processing the classification token in the MLP, allowing the \CLS to be unrolled at the last layer as individual contributions.}
    \label{fig:hit}
\end{figure}

In this section, we describe our proposed architecture: Hindered Transformer (HiT). By constraining the image tokens to contain only local information along all inference blocks, and by avoiding mixing the \CLS token, our novel method is able to partition the \CLS token into each individual patch, a direct outcome of the previous section. Fig.~\ref{fig:hit} shows the difference between the ViT block, and our block.


The first challenge is then constraining the data flow between patches.
To do so, we create an intermediate architecture that uses \CLS token $x_l[0]$ as the query in the MHA operation, and the rest of the sequence $x_l$ as the key-value input.
So, the output from the MHA is a single token that is summed to $x_l[0]$.
Then, as in ViTs, we will post-process each token in the sequence with the MLP.
Thus, the ViT update function in Eq.~\ref{eq:vit-block} is transformed to
\begin{equation}\label{eq:pat-block}
\begin{split}
    x'_l[0] &= x_l[0] + MHA(x_l[0], x_l)\\
    x'_l[1:] &= x_l[1:] \\
    x_{l+1} &= x'_l + MLP(x'_l).
\end{split}
\end{equation}

The previous model solves one problem by limiting the merging of data in local patches. 
However, processing the \CLS token through the MLP mixes the local information provided by the MHA block, as well as the value and output operations. 
Since our goal is to disentangle the data flow into individual contributions, we need to further constrain this processing.
To do this, we simply avoid updating the \CLS token through the MLP and passing it to the target sequence.
So, our block inference is
\begin{equation}\label{eq:hit-block}
\begin{split}
    x_{l+1}[0] &= x_l[0] + MHA(x_l[0], x_l[1:])\\
    x_{l+1}[1:] &= x_l[1:] + MLP(x_l[1:])
\end{split}
\end{equation}
We call the final architecture the Hindered Transformer (HiT), as we hinder the connections of the ViT. 
In a nutshell, HiT only updates the \CLS token via the MHA, while the MLP blocks update the image patches. These restrictions help to preserve purely local information in each token, while allowing the \CLS token to be unrolled. 

\begin{figure}[t]
    \centering
    \includegraphics[width=0.95\linewidth]{images/hit-saliency.pdf}
    \caption{\textbf{Saliency Maps computation using HiT.} From the results from \S\ref{sec:mha} and the definition of our architecture, HiT enables to extract the individual contribution per token and per layer. By adding together all tokens per layer, we can rearrange the tokens in a spatial layout and use the linear layer \textit{\`a la} CAM~\cite{zhou2016learning} to extract the contribution of each token.}
    \label{fig:hit-saliency}
\end{figure}

Since the classification token is not post-processed with MLP or MHA, the final image classification is the sum of the individual tokens in all layers, as shown in \S\ref{sec:mha}. 
Therefore, the \CLS in the last layer is
\begin{equation}\label{eq:hit-cls}
\begin{split}
    x_L[0] &= x_0[0] + \sum_{l=0}^{L-1} MHA(x_l[0], x_l[1:]) \\
    &= x_0[0] + \sum_{l=0}^{L-1} \left[ b_o^l + \sum_{v\in x_l[1:]} v'_l(v) \right] \\
    &=\sum_{l=0}^{L-1} \sum_{v\in x_l[1:]} \left[v'_l(v) + \frac{b_o^l}{N^2} + \frac{x_0[0]}{LN^2} \right].
\end{split}
\end{equation}
Please note that we distribute the biases $b_o^l$ of the projection operation in the MHA head evenly to each patch $v'_l(v)$. 
In a similar fashion, we spread $x_0[0]$ into all tokens for all layers.

One advantage of this architecture is that we can easily compute saliency maps, as shown in Fig.~\ref{fig:hit-saliency}.
The double sum in Eq.~\ref{eq:hit-cls} can be decomposed as a tensor $\mathrm{R}^{L\times N^2 \times d_{model}}$, where the final image representation is the sum over the first and second dimensions, \ie the layer and token dimension, respectively.
Thus, and similarly to CAM~\cite{zhou2016learning}, to compute the regions of interest used by the model for an input image, we simply run the linear classifier on each patch to get the map.
This rationale is similar to the LRP~\cite{bach2015pixel} method in the sense that the sum of value in the saliency is equal to the output logit for that specific class.

\subsection{Token Pooling}\label{sec:pooling}


Token pooling \cite{tokenpooling}, which involves downsampling the number of tokens as one progresses through the layers, is commonly used to improve the computational efficiency of standard transformers. This pooling technique effectively addresses the issue of representation power (as empirically demonstrated in \S~\ref{sec:perf-loss}) by expanding the receptive field of the image tokens in the deeper layers. We choose to adopt this approach due to its significant advantages.

To achieve this, we first reorganize the tokens into their spatial layout and then perform the pooling operation. However, we need to adapt the explanation generation approach to accommodate the pooled tokens. Typically, this involves using the backward operation of the pooling operator. In our case, since we rely on average pooling, which is linear, the backward operation is simply the transposed operator, which replicates each output token across the associated \(2 \times 2\) block and divides by \(4\). In other words, we distribute the importance of the pooling step equally among the contributing tokens.




\section{Dataset Generation}
\label{sec:dataset}
\revise{
To train the proposed GNN, we constructed a dataset of building structures and a subset of these structures were subjected to fire simulations using FEA. The dataset generation process is illustrated in \figref{fig:dataset_generation_procedure}. Initially, a total of 33,000 building structures with geometrical details, material properties, and gravity loads were created. Due to randomness in generating these structures, a filter is applied to remove unreasonable data after gravity load simulation, which included 15,377 structures. A trade-off between computational feasibility and model performance is made among the remaining 17,623 structures. As further labeling structures with MIDR requires resource-intensive fire simulations via OpenSeesRT, a large proportion of 16,050 structures is selected as unlabeled dataset. On the other hand, each of the other 1,573 structures was further subjected to 30 different fire simulations, forming the labeled dataset containing $1,573\times 30 = 47,190$ fire cases.} This section details the step-by-step process for generating the dataset, including geometry creation, material property assignment, and simulations due to gravity loads and fire scenarios. 
% To train the proposed neural network, we constructed a dataset comprising building structure data and a subset of fire scenario data. The dataset generation process is illustrated in \figref{fig:dataset_generation_procedure}. 
% A total of 33,000 building structures with geometric details, material properties, and gravity loads were initially created. Out of these, 3,000 structures were selected as labeled data, and the remaining 30,000 were designated as unlabeled data. Further, about half of them filtered out due to instability under gravity loads only. 
\begin{figure*}[h!]
    \centering
    \includegraphics[width=0.8\linewidth]{figures/dataset_filter_procedure.pdf}
    \caption{Workflow for dataset generation (geometry, material property, gravity loads, and fire scenarios).}
    \label{fig:dataset_generation_procedure}
\end{figure*}

\subsection{Geometry Generation}
\label{subsec:geometry_generation}
The geometry of the building structures forms the foundation of the dataset. Regular 
\revise{3D structures} resembling multi-story parking structures or shopping malls were generated, with parameters such as building floor dimensions and story heights selected randomly. Each building structure is composed of multiple rooms, which serve as the basic unit in this study. A room herein is a cuboid space defined by specific length, width, and height. Within a structure, rooms of the same dimensions are uniformly arranged along the length, width, and height, corresponding to the $x$-, $y$-, and $z$-axes, respectively. Structures vary in room size and number of rooms along each axis. Specifically, the room length, width, and height are independently sampled from a uniform distribution within the interval $[2, 5]$ meters along the three directions of the structure. Similarly, the room number along each axis is uniformly sampled independently as an integer within the interval $[2, 7]$, i.e., the maximum number of stories of the buildings simulated in this study is 7.

To introduce variability and simulate real-world scenarios, approximately $8\%$ of structural elements (beams or columns) are randomly removed after initial geometry creation. 
\revise{Such removal is not fire-induced damage, but reflects functional diversity often observed in real buildings, such as open spaces designed for activities in shopping malls, e.g., ice skating rinks. Examples of the generated geometries are illustrated in \figref{fig:example_generated_geometry}, showcasing the diversity and realism of the dataset. This element removal does not affect the definition of room's geometry in the structure and nor does it affect the number of considered fire scenarios.} 

\revise{A range of coefficient of variation values ($3.3\%$ to $17.5\%$) was derived from prior studies that investigated the statistics of geometrical and material properties of structural components of buildings (e.g., \cite{mirza1979variations, lee2004probabilistic}). These studies provide empirical data on the natural variability in parameters such as Young's modulus, yield strength, and dimensions of structural elements due to manufacturing tolerances and material inconsistencies. By selecting $8\%$ for the removal of structural elements in our database, we aimed to maintain a level of variability that is representative of real-world uncertainties while ensuring computational feasibility. This choice ensures that the database captures realistic deviations without introducing extreme cases that may not be commonly encountered in practice.}

\begin{figure*}[h!]
    \centering
    \includegraphics[width=\linewidth]{figures/example_generated_geometry.pdf}
    \caption{Examples of generated structural geometry of different sizes (all dimensions in meters).}
    \label{fig:example_generated_geometry} 
\end{figure*}

{\blockRevise

In this study, we opted for a deterministic square, dimension of $0.1$ m, solid cross-sectional steel elements due to their simplicity in modeling and analysis. Square sections exhibit uniform geometrical properties in all directions, simplifying the computation of structural responses and avoiding complications associated with more complex shapes, such as wide-flange sections, facilitating the computational efficiency and scalability to generate a large dataset. This choice also helps to mitigate issues related to stress concentrations and facilitates a more straightforward representation of structural behavior under thermal loads. 

\textit{Remark:} The selected cross-section provides a comparable flexural rigidity to a $W 130 \times 130 \times 28.1$ wide-flange section (metric units), albeit with significantly higher axial rigidity. This cross-section is acceptable for gravity-load-designed frames under service loading conditions where the models assume fully rigid, moment-resisting beam-column connections for the evaluation of the IDR under thermal loading. This assumption is reasonable in this computational study where the primary interest is to understand the global deformation response of frames under fire conditions. The selection of uniform square cross-sections for both beams and columns, rather than adherence to standard capacity design principles, was made here primarily for computational efficiency and to reduce design parameters in the database generation process. This choice allows for simplified and scalable approach to analyze the fire-induced response of generic steel frames without the need for large section variations, where this study mainly focuses on the fire vulnerability assessment using ML-based predictions. However, if additional loading conditions, e.g., seismic or wind loads, were to be considered, larger sections, strong-column/weak-beam principle, and ductile detailing would be required in the generated buildings for realistic structural behavior under combined loading conditions. Future studies may also consider investigating the influence of variable cross-sectional dimensions and semi-rigid connections on the structural performance under fire conditions. 
} % blockRevise

\subsection{Material Properties}
Steel is chosen as the material for the structures. To reflect real-world variations, we randomly assign one of five slightly different steel material types to each structural element. \revise{
The ranges of material properties are provided in \tabref{tab:material_property_ranges} and the properties are sampled from uniform distributions of the corresponding ranges. These variations simulate differences arising from manufacturing batches or regional material properties. That these properties are at ambient temperature and change when the temperature rises due to a fire. The selection of materials with varying properties is aimed at increasing the diversity of the data. Our goal is to represent as wide a range of data as possible with a limited amount of building structure data, thereby enhancing the generalization ability of the GNN. Our assumed material property ranges are expected to be wider than the real-world conditions based on findings in \cite{mirza1979variations, lee2004probabilistic}. Therefore, we are essentially tackling a more challenging and general task. If we can solve this problem, we are confident that our method will perform equally well or even better in real-world scenarios.
}
\begin{table}[h!]
    \centering
    \caption{Material properties ranges for considered steel structures.}
    \begin{tabular}{lc}
        \toprule
        Property & Range \\
        \midrule
        Young's modulus & [168, 252] GPa \\
        Yield strength & [220, 330] MPa \\
        Strain-hardening ratio & [0.8, 1.2] \% \\
        \bottomrule
    \end{tabular}
    \label{tab:material_property_ranges}
\end{table}

\subsection{Gravity Loads}
Gravity loads are applied to columns and beams based on their \revise{influence (tributary) areas as typically conducted in structural analysis. The considered ``service'' load conditions include the column self-weight and the additional loads directly supported on the beams from their self-weight and weights of the reinforced concrete slabs, people as live load, and building content. An edge beam typically carries approximately half the gravity load supported by a parallel interior beam}. The ranges of gravity loads are listed in \tabref{tab:gravity_load_ranges}. \revise{The loads are sampled from uniform distributions of the corresponding ranges.} Structures that failed to meet an MIDR threshold of $1\%$ under gravity loads were deemed unacceptable designs and filtered out, as such configurations of randomly chosen geometry, material, and gravity load combinations were considered unrealistic from a regulatory and practicality points of view.
\begin{table}[h!]
    \centering
    \caption{Gravity load ranges for considered beams and columns.}
    \begin{tabular}{lc}
        \toprule
        Element & Range (kN/m)  \\
        \midrule
        Column & [0.5, 1.0]  \\
        Edge beam & [1.5, 4.5]  \\
        Interior beam & [3.0, 7.5]  \\
        \bottomrule
    \end{tabular}
    \label{tab:gravity_load_ranges}
\end{table} 

\subsection{Rule-based Thermal Load Generation}
\label{subsec:thermal_load_generation}
To evaluate a building's structural response during a fire event, we employed a simplified rule-based approach for thermal load generation. 
% Previous studies \cite{nan_structuralfire_2023} have demonstrated that steel structures rapidly equilibrate with surrounding gases temperatures due to efficient heat exchange. Consequently, gas temperatures can be directly used as inputs for FEA tools, e.g., OpenSees, simplifying the process of modeling thermal loads. 
% Accurately simulating temperature fields in fire scenarios poses significant challenges. Advanced thermodynamic simulations, such as those performed using Fire Dynamics Simulator (FDS) \cite{mcgrattan_fire_2000}, provide precise temperature predictions. However, these methods are hindered by high computational costs, prolonging execution times, and limited scalability, making them impractical for generating large datasets. Additionally, real-world fire loads often display substantial spatial variability across different rooms \cite{dundar_fire_2023}, resulting in scenario-specific temperature fields with limited generalizability. For example, studies on bridge fires \cite{he_study_2024} have demonstrated that environmental factors, such as wind speeds, can significantly influence temperature distributions. Furthermore, even within identical scenarios, variations in fire modeling methodologies can produce distinctly different temperature fields \cite{zhang_temperature_2020, du_new_2012}. These challenges emphasize the need for efficient and adaptable methods to generate fire temperature data.
% To address these issues, we adopted a rule-based approach to model temperature variations. 
According to \cite{spearpoint_fire_2008}, a typical fire development follows a predictable pattern. During the {\em{growth stage}}, the temperature rises slowly and approximately linearly after ignition. This is followed by the {\em{flashover stage}}, where temperatures increase rapidly to peak values. After reaching the peak, the temperature either stabilizes or continues to rise slowly until the {\em{decay stage}} begins. Inspired by this fire development pattern, we describe the temperature evolution in time, $t$, prior to the decay stage in two distinct stages:
\begin{enumerate}
    \item {\bf{Initial linear increase stage}}: For $t \in [0, t_1)$, temperature increases gradually and linearly as the fire spreads through the building. This stage represents the time before the fire directly affects a structural element.  
    \item {\bf{ISO 834 fire curve stage}}: For $t \in [t_1, t_{\thre}]$, temperature rises rapidly following the ISO 834 curve \cite{ISO834}, modeling the direct impact of the fire on the structural element. 
\end{enumerate}
The slope of the linear temperature increase, $c$, and the transition time, $t_1$, are influenced by the spatial relationship between the fire source and the structural element. For the second stage of temperature evolution, we utilize the ISO 834 curve, a widely accepted standard for fire resistance testing. This standardized fire curve describes the temperature rise over time, enabling rapid and consistent thermal fields across various scenarios. The duration of fire simulation in this study is set to $t_{\thre}=60$ minutes. This value represents the upper limit for the temperature evolution of each structural element, providing a consistent basis for analyzing the structural response to fire.

Let $(x, y, z)$ represents the midpoint of a structural element and $(x_{\subfire}, y_{\subfire}, z_{\subfire})$ the fire source point. \revise{Integer parameters $h$ and $h_{\subfire}$ correspond to the respective floor levels of the element and the fire source}. The temperature evolution for each element is expressed as follows:
\begin{enumerate}
    \item Linear increase stage ($0 < t < t_1$):
    \begin{equation}
    T(t) = c \cdot t,
    \end{equation}
    where $c$, the rate of temperature increase ($^\circ\mathrm{C}/\mathrm{min}$), depends on the height difference between the element, $h$, and the fire source, $h_{\subfire}$:
    \begin{equation}
        c = 
        \begin{cases} 
        5\left/\left(h - h_{\subfire} + 1\right)\right., & h \geq h_{\subfire}, \\
        2\left/\left(h_{\subfire} - h\right)\right., & h < h_{\subfire}.
        \end{cases}
    \end{equation}
     \item ISO 834 stage ($t \geq t_1$):
\begin{equation}
    T(t) = c \cdot t_1 + 345 \log_{10} \left(8 \left(t - t_1\right) + 1\right).
\end{equation}
\end{enumerate}

The transition (arrival) time $t_1$, marking the end of the linear stage, depends on the spatial distance between the fire source and the element. We define the following two Euclidean distances $L_p$ in the $xy$ plane and $L_s$ in the $xyz$ space:
\begin{eqnarray}
L_p & \triangleq & \sqrt{(x - x_{\subfire})^2 + (y - y_{\subfire})^2}, \\
\label{eq:Lp}
L_s & \triangleq & \sqrt{(x - x_{\subfire})^2 + (y - y_{\subfire})^2 + (z - z_{\subfire})^2}.
\label{eq:Ls}
\end{eqnarray}
Accordingly, the transition time, $t_1$, is expressed as follows:
\begin{equation}
    t_1 = 
    \begin{cases}
    \beta_{1} \cdot \left(1 - \exp\left\{- L_s\left/\alpha_{1}\right.\right\}\right), & h > h_{\subfire}, \\
    \beta_{2} \cdot \left(1 - \exp\left\{- L_p\left/\alpha_{2}\right.\right\}\right), & h = h_{\subfire}, \\
    \beta_{3} \cdot \left(1 - \exp\left\{- L_s\left/\alpha_{3}\right.\right\}\right), & h < h_{\subfire} .
    \end{cases}
    \label{eq:t1}
\end{equation}
The parameters $\beta_i$ and $\alpha_i$ for determining $t_1$ are summarized in Table~\ref{tab:fire_spread_parameters}. In this study, we take $r_{\mathrm{up}}=0.95$ and $r_{\mathrm{down}}=0.97$.
\begin{table}[ht]
    \centering
    \caption{Fire spread parameters for $t_1$ calculations.}
    \begin{tabular}{lcc}
        \toprule
        Case  & $\beta_i$ & $\alpha_i$  \\
        \midrule
        $i=1$, Upward spread & $16 \left.\left(1-r_{\mathrm{up}}^{\left|h-h_{\subfire}\right|}\right)\right/\left(1-r_{\mathrm{up}}\right)$ & $10$  \\
        $i=2$, Horizontal spread & $18$ & $18$  \\
        $i=3$, Downward spread & $30 \left.\left(1-r_{\mathrm{down}}^{\left|h-h_{\subfire}\right|}\right)\right/\left(1-r_{\mathrm{down}}\right)$ & $5$  \\
        \bottomrule
    \end{tabular}
    \label{tab:fire_spread_parameters}
\end{table}

\figref{fig:t1_curve} illustrates the $t_1$ curves for various fire scenarios: (1) fire originating on the lower floor, $h-h_{\subfire}=1$ with rapid upward spread, (2) fire on the same floor, $h=h_{\subfire}$ with the fastest spread, and (3) fire on the upper floor, $h_{\subfire}-h=1$ with slow downward spread. The exponential decay in $t_1$ reflects the accelerating fire propagation speed as the distance increases. \figref{fig:t1_curve} also indicates that the employed simplified model is consistent with the Markov chain-based dynamic model given by \cite{cheng_dynamic_2011}, where the rooms at the same floor of the fire point start flashover slightly before the corresponding upper floors. Additionally, $\beta_{1}$ and $\beta_{3}$ are the summation of a geometric sequence, where story level $h$ is the index. The common ratios $r_{\mathrm{up}}<1$ in $\beta_{1}$ and $r_{\mathrm{down}}<1$ in $\beta_{3}$ indicate that the fire speeds up to spread through the next story, which is consistent with the real-world fire spread mechanism given in \cite{hokugo_mechanism_2000}. The temperature profile within the range $t \in [0, t_{\thre}]$ is subsequently used as the thermal load in OpenSeesRT simulations to compute displacements at each structural node at time $t_{\thre}$.
\begin{figure}[h!]
    \centering
    \includegraphics[width=0.8\linewidth]{figures/m204_t1_curve.pdf}
    \caption{Three examples for the $t_1$ curve.}
    \label{fig:t1_curve}
\end{figure}

\revise{
\textit{Remark:} The effects of structural elements, such as concrete floor slabs and partitions, are not explicitly modeled in our approach. Instead, their influence is implicitly captured through the careful selection of the parameters $ \alpha, \beta, r_\mathrm{up} $, and $ r_\mathrm{down} $. This parameterization provides a unified framework for generating temperature fields. Indeed, fire propagation is governed by a multitude of factors and remains an open research question. For instance, if the fire resistance of a floor slab is enhanced by fire protective coating, the corresponding model can account for this by decreasing $\alpha_1$ \& $\alpha_3$, increasing $\beta_1$ \& $\beta_3$, and adopting larger values for $r_\mathrm{up}$ \& $r_\mathrm{down}$, which collectively slow down the vertical spread of fire. Conversely, scenarios involving higher amounts of combustible materials would warrant the opposite adjustments. This flexible and integrated approach avoids the need to design separate models for different fire propagation scenarios while still capturing the essential effects.
}

\revise{
In conclusion, our rule-based approach is a computationally efficient method for approximating fire temperature fields, enabling large-scale dataset generation to train predictive models. By combining ISO 834 fire curves with spatial considerations and embedding structural effects through parameter calibration, the method achieves a balanced trade-off between accuracy and scalability, making it a practical solution for thermal load modeling in fire scenarios. After generating the temperature of each beam or column according to the middle point, the temperature is applied as uniform thermal load to the elements of the structure in question using OpenSeesRT. 
}

% In conclusion, this rule-based approach is a computationally efficient method to approximate fire temperature fields, enabling large-scale dataset generation to train predictive models. By combining ISO 834 fire curves with spatial considerations, the method balances accuracy and scalability, making it a practical solution for thermal load modeling in fire scenarios.

% \subsection{Interstory Drift Ratio}
\subsection{OpenSeesRT Simulation}
\label{subsec:opensees_simulation}

The thermal and mechanical responses of 3D frame structures under combined fire and gravity loads are simulated using OpenSeesRT \cite{perez2024openseesrt}. \revise{In the simulation, the IDR of each node at $t_{\thre}$ is computed using the computed nodal displacements. Each structural model features six degrees of freedom per node (3 translational  and 3 rotational), with linear geometrical transformations (\texttt{geomTransf: Linear}) defining how the element local coordinate systems are mapped to the global coordinate system and assuming small displacements and rotations. Although OpenSeesRT allows a variety of options for modeling finite deformations, in the present simulations and mainly for simplicity, we did not consider large deformations. All bottom nodes (nodes on the ground) are fully constrained in all six degrees of freedom, while degrees of freedom os all other nodes are free.} Material behavior is temperature-dependent and modeled with \texttt{Steel01Thermal}, while fiber-based sections (\texttt{FiberThermal}) capture nonlinear interactions between thermal and mechanical responses at the cross-section level. \revise{Structural elements are represented as displacement-based Euler-Bernoulli beam-columns (\texttt{dispBeamColumnThermal}). This element  formulation accounts for thermal strains (temperature gradients) in the section, which is discretized into fibers. Numerical integration is used along the length of each element using three integration (Gauss) points, one at each end and the third in the middle of the element.}

{\revise{Thermal expansion of steel members plays a crucial role in IDR development. In reality, reinforced concrete floor slabs heat at a different rate than steel members due to their higher thermal mass and lower thermal conductivity. This differential heating can lead to restrained thermal expansion, introducing axial compression in beams and affecting the overall structural response. In this study, explicit {\em{composite action}} between steel members and concrete slabs is not modeled. Instead, our approach focuses on isolating the response of the steel structural frame, which is often the critical load-bearing component in fire scenarios. This assumption aligns with prior studies \cite{Possidente_2024} demonstrating that steel structures reach thermal equilibrium with surrounding gases quickly, allowing the use of uniform thermal loading in fire analysis. Future work could enhance this framework by incorporating slab-beam interaction effects, through a refined FEA for an extended dataset where constraints imposed by floor slabs are explicitly considered.}

The analysis begins with the application of gravity loads, followed by incremental thermal loads simulating the fire exposure. A static nonlinear solver using  \texttt{ExpressNewton} algorithm ensures convergence, while the \texttt{NormDispIncr} test maintains accuracy. An incremental \texttt{LoadControl} scheme with small step sizes is employed to guarantee numerical stability, using 10\% for gravity loads and 1\% for thermal loads. 

\revise{
In the thermal load analysis, uniform thermal load is applied to each beam or column, i.e., the temperature of each element is set to be that at the middle point, according to \secref{subsec:thermal_load_generation}. The \texttt{Steel01Thermal} material allows the properties (e.g., Young's modulus and yield strength) to be adjusted at increasing temperatures according to \cite{EN1993} using its Table 3.1: Reduction factors for the stress-strain relationship of carbon steel at elevated temperatures. For example, if the Young’s modulus at ambient temperature is $E_0$, then as the temperature ($T$) increases, the modulus changes as $E(T) = \eta (T) \times E_0$. \cite{EN1993} directly provides the values of $\eta(T) \in \left[0,1\right] $ at every $100 ^\circ\mathrm{C}$ interval and recommends using linear interpolation to obtain $\eta(T)$ for intermediate values of $T$.
} OpenSeesRT documentation \cite{OpenSeesThermalExamples} provides several examples of thermal analyses.

This modeling framework accommodates variations in material properties, cross-sectional geometries, and temperature profiles, providing robust simulations of structural behavior under fire conditions. The primary settings and configurations for the OpenSeesRT simulations are summarized in \tabref{tab:ops_detail}.
\begin{table}[h!]
    \centering
        \caption{Key settings of OpenSeesRT simulations.}
    \begin{tabular}{l|>{\raggedright\arraybackslash}p{0.6\linewidth}} %
    \toprule
    Modeling Aspect     & Details \\
    \midrule
    Geometry            & 3D models; 6 degrees of freedom per node \\
    Transformation      & geomTransf: Linear \\ 
    Material            & Steel01Thermal \\
    Section             & FiberThermal; Cross-section: $0.1$ m $\times$ $0.1$ m \\ 
    Element type        & {dispBeamColumnThermal} \\ 
    Loading             & Gravity loads: {beamUniform}; Thermal loads: {beamThermal} \\
    Integration scheme  & Incremental {LoadControl}; Step size: $10\%$ (gravity analysis), $1\%$ (thermal analysis) \\
    Nonlinear solver    & {ExpressNewton} algorithm; {UmfPack} solver; Convergence test: {NormDispIncr} tolerance: $10^{-8}$; Maximum \# iterations per step: $1000$. \\ 
    \bottomrule
    \end{tabular}
    \label{tab:ops_detail}
\end{table}

For each structure in the labeled dataset, 30 fire points are selected using a dual-granularity approach, \revise{i.e., two-stage sampling strategy,} to ensure they are well-distributed. Specifically, rooms are sequentially selected, with one fire point randomly chosen within each selected room. If a building is large and contains more than 30 rooms, we randomly select 30 rooms without replacement, i.e., ensuring that no more than one fire point is located in the same room. Conversely, if the building is small and has fewer than 30 rooms, all rooms are initially selected, with one fire point randomly assigned to each room. Additionally, rooms are then selected with replacement until a total of 30 fire points are assigned. \revise{The room-level sampling prioritizes selecting distinct rooms to avoid spatial clustering of fire points, while the point-level sampling ensures intra-room variability. This approach aligns with stratified sampling principles commonly used for efficient spatial representation, where multi-stage sampling strategies optimize coverage and variability, e.g., \cite{arunachalam_generalized_2023}, and enables a more comprehensive characterizing of how the structures respond under fire conditions.}
% This selection method prevents fire points from clustering too closely while maintaining an element of randomness. By distributing fire points in this manner, the 30 fire scenarios are effectively utilized, enabling a more comprehensive characterizing of how the structures respond under fire conditions.

\subsection{Summary of the Dataset Generation}
As discussed in this section and related to  \figref{fig:dataset_generation_procedure}, three key steps were considered in the development of the dataset: 
\begin{enumerate}
    \item {\bf{Filtering process}}: Structures with MIDR exceeding $1\%$ under gravity loads were excluded,  resulting in $1,573$ labeled structures retained for fire simulation and $16,050$ unlabeled structures for training the MFSP predictor.
    \item {\bf{Fire simulations}}: For each retained labeled structure, 30 fire scenarios were simulated using OpenSeesRT, yielding $47,190$ fire cases.
    \item {\bf{Data distribution check}}: MIDR distributions for labeled and unlabeled data under gravity loads were highly similar, because both datasets were generated using the same method. Under fire conditions, the MIDR distribution shifted, reflecting significant structural deformation with values reaching a maximum of about 6\%, an average of 1.70\%, and a standard deviation of 1.12\%. This step ensured a diverse and comprehensive dataset for the proposed predictive framework.
\end{enumerate}
The statistical distribution histograms for MIDR (after applying the $1\%$ filtering threshold \revise{for gravity load responses}) under different loading conditions are plotted in \figref{fig:histogram_mdr}. Figures \ref{fig:histogram_mdr}(a) and \ref{fig:histogram_mdr}(b) show the MIDR distributions of the labeled and unlabeled data, respectively, under gravity loads only. \figref{fig:histogram_mdr}(c) shows the MIDR distribution of the labeled data under the combined effects of gravity and fire loads. Fire load causes the structures to significantly deform, leading to a noticeably \revise{right-skewed} MIDR distribution.

\begin{figure*}[h!]
    \centering
    \includegraphics[width=\linewidth]{figures/histogram_mdr.pdf}
    \caption{Histograms of MIDR for labeled and unlabeled structures with gravity loads and fire cases.}
    \label{fig:histogram_mdr}
\end{figure*}

\revise{
This dataset provides the basis for training and testing the performance of the GNN-based framework. Although we employed a simplified rule-based thermal load generation method compared with conventional CFD-based simulations, the temperature field, the changes of the material properties, and the response of the structures, are all still highly nonlinear and complex. Therefore, it is still a challenging task for the NN to predict the MIDRs based on this dataset.
}
\section{Experiments}
\label{sec: exp}

In this section, we conduct experiments to answer the following research questions:
\begin{itemize}
\item \textbf{RQ1}: Can the proposed model effectively improve the performance of the original CDMs?  
\item \textbf{RQ2}: What is the impact of each component within the proposed method? 
\item \textbf{RQ3}: How does the proposed model perform on cold-start scenarios? 
% \item \textbf{RQ4}: What are the differences in diagnostic effectiveness when using different LLMs?
\item \textbf{RQ4}: How effective is the alignment of semantic and behavioral space embeddings during the cognitive level alignment process?
\end{itemize}

\subsection{Experimental Settings}

\subsubsection{Datasets}

\section{Baseline} \label{sec:splitgraph}

The baseline method for batch-$k$DP solves each query using flow-augmenting path-based methods, which rely on the concept of \textit{split-graphs}~\cite{baseline_moreverbose, baseline1step2, baselineOnlySplitP1}. 
% For each query, paths are iteratively found in a split-graph, which is updated after each iteration.
% A split-graph is constructed by two transformations of the original graph:
% (1) reversing result-set paths, simulating flow-augmentation, and 
% (2) splitting vertices within these paths, giving rise to the name ``split-graph."

\textbf{Definition: Split-Graph~\cite{baselineOnlySplitP1}} 
Given a graph \( G = (V, E) \) and a set \( P \) of disjoint paths from \( s \) to \( t \), the split-graph \( \iG_{G,P} = (\iV_{G,P}, \iE_{G,P}) \) is constructed as follows:
(1) Initializing \( \iV_{G,P} = V \) and \( \iE_{G,P} = E \).
(2) For each edge in \( E(P) \), reversing the corresponding edge in \( \iE_{G,P} \).
(3) Splitting vertices \(v \in V(P) \setminus \{s, t\}\) into \(v^{in}\) and \(v^{out}\), and connecting them accordingly.
(4) Replacing edges in \(\iE_{G,P}\) with updated vertex connections, preserving incoming and outgoing edges.

% \textbf{Example}: 
% Fig.~\ref{fig:eg_split} shows the split-graph construction for the graph \( G \) in Fig.~\ref{fig:g} with $P= \{p_1=\{a, e, d, h\}\}$. Changes are shown in red.


% \vspace{-10pt}
\begin{figure}[h!]
\newcommand{\mylinewidth}{\linewidth}
\centering
    \begin{subfigure}[t]{0.35\mylinewidth}
        \centering
        % \resizebox{\mylinewidth}{!}
        {\includegraphics[width=\linewidth]{pic/eg/g}}
        \caption{Disjoint paths for $(a, h)$.}
        \label{fig:g}
    \end{subfigure}
    \begin{subfigure}[t]{0.6\mylinewidth}
        \centering
        % \resizebox{\mylinewidth}{!}
        {\includegraphics[width=\linewidth]{pic/eg/steps_red_new.pdf}}
        \caption{Split-graph with $P= \{p_1=\{$a$, $e$, $d$, $h$\}\}$.}
        \label{fig:eg_split}
    \end{subfigure}
    \caption{Examples of disjoint paths and split-graph.}
    % \label{fig:fg_share_intuition}
\end{figure} 
% \vspace{-5pt}

% 删除 begin
Given a graph \( G \) and vertices \( s \) and \( t \), the algorithm proceeds as follows:
% (1) Initialize \( P = \emptyset \) and \( \iG_{G,P} = G \).
% (2) Find the first path \( p_1 \) using a path-finding algorithm (e.g., BFS) in \( \iG_{G,P} \) and update \( \iG_{G,P} \).
% (3) Find the second path \( p_2 \), update found paths following an approach similar to augmenting paths in the maximum flow problem~\cite{baseline_moreverbose}, then update \( \iG_{G,P} \). More paths are found in a similar manner.
(1) Initialize $P = \emptyset$ and $\iG_{G, P} = G$.
(2) Find the first path $p_1$ in $\iG_{G, P}$ using any path-finding algorithm (e.g., BFS), forming $P_1 = \{p_1\}$, and update $\iG_{G, P}$ to $\iG_{G, P_1}$.
(3) Search for $p_2$ in $\iG_{G, P_1}$, yielding $P_2 = \{p_1, p_2\}$, and adjust $P_2$ following an approach similar to augmenting flows~\cite{baseline_moreverbose}.
Then update $\iG_{G, P_1}$ to $\iG_{G, P_2}$.
(4) Search for $p_3$ in $\iG_{G, P_2}$. More paths are found in a similar manner.
% 删除 end
In our experiments, we utilize four courses, Python Programming (Python), Linux System (Linux), Database Technology and Application (Database), and Literature and History (Literature), from a publicly available dataset PTADisc~\cite{hu2023ptadisc}, which comes from real-world students' responses in the educational website PTA\footnote{\url{https://pintia.cn/}} and contains textual information of exercises and knowledge concepts. 
%Each response log in the dataset contains a student ID, an exercise ID, whether the student correctly answers the question, the content of the exercise, and the knowledge concepts related to the exercise.
The statistics of the datasets are presented in Table~\ref{tab: dataset}.
The datasets are divided into training, validation, and testing sets, with a ratio of 8:1:1.

\subsubsection{Evaluation Metrics}

Following previous works, we evaluate the students' cognitive status by predicting the performance of students on the testing set, as the cognitive status can not be directly observed. We adopt commonly used metrics, namely the Area Under a ROC Curve (AUC), the Prediction Accuracy (ACC), and the Root Mean Square Error (RMSE), to validate the effectiveness of the CDMs.
%In the subsequent tables, \textbf{bold} numbers represent the best performance, while \underline{underlined} numbers represent the second-best performance. 
For all the metrics, $\uparrow$ represents that a greater value is better, while $\downarrow$ represents the opposite.

\subsubsection{Baseline Methods}

To validate the effectiveness of the proposed method, we conduct experiments on several representative CDMs, including IRT~\cite{lord1952theory}, MIRT~\cite{reckase200618}, DINA~\cite{de2009dina}, NCD~\cite{wang2020neural}, RCD~\cite{gao2021rcd}, SCD~\cite{wang2023self} and ACD~\cite{wang2024boosting}.
 

\subsubsection{Implementation Details}

We utilize PyTorch to implement both the baseline methods and our proposed KCD framework. 
For the baseline models, We use the default hyper-parameters as stated in their papers and for KCD, we use the same hyper-parameter settings, such as training epoch, learning rate, and batch size.
We employ ChatGPT to represent LLMs (specifically, gpt-3.5-turbo-16k) and text-embedding-ada002 as the text embedding model. All the experiments are conducted on a GeForce RTX 3090 GPU.
We train the model on train set and at the end of each epoch, we evaluate the model on the validation set.
The hyper-parameter $\alpha$, $\beta$, and $\lambda$ was set to $0.04$, $0.015$, and $0.2$.
Since our dataset does not include affect labels, we utilize the unsupervised contrastive ACD model and employ NCD as the basic cognitive diagnosis module.
The behavioral space alignment approach is denoted as `-Beh' and the semantic space alignment approach is denoted as `-Sem'.
% We investigated the impact of the hyper-parameter $\lambda$, within the range $[0,0.2,\cdots,1]$ with a step size of $0.2$. Our analysis revealed that setting $\lambda$ to $0.1$ resulted in the best performance across all three datasets.

\begin{figure}[t]
  \centering
  
  \includegraphics[width=1.02\linewidth]{figs/experimentx.png}
  \caption{Performance comparison in cold (blue) and warm (red) scenarios on Python dataset.}
  \vspace{-2em}
\label{fig: experiment1}
\end{figure}

\subsection{Performance Comparison (RQ1)}
To demonstrate the effectiveness of our proposed method in improving cognitive diagnosis, we implement the framework on seven cognitive diagnosis models, and the results are shown in Table~\ref{tab:performance}. 
Additionally, we compared the performance of NCD in warm and cold scenarios, with the results illustrated in Figure~\ref{fig: experiment1}. Here we define the cold scenario as less than $3$ interactions in the training set for exercises and define the warm scenario as more than $10$ interactions in the training set for exercises. Following this definition, we divide the testing set into cold and warm subsets.
We have the following observations from the results: 

\begin{itemize}[leftmargin=*]
    \item[1)]  
    Both KCD-Beh and KCD-Sem achieve significant improvements compared to the basic CDMs.
    This indicates that our proposed framework is widely applicable to various CDMs, and both alignment methods can effectively align the behavioral space of CDMs and the semantic space of LLMs.
    In most models, the behavioral space alignment approach performs better, indicating that aligning in the behavioral space of CDMs can better align information from the semantic space of LLMs.
    \item[2)] Compared to basic CDMs, our proposed methods demonstrate improvements in both cold and warm scenarios, especially in cold scenarios. This indicates that our approach of introducing LLMs as knowledge enhancement effectively alleviates the cold-start issue.
\end{itemize}




\begin{table*}
  [t]
  \centering
  \resizebox{\textwidth}{!}{%
  \begin{tabular}{cccccccccccc}
    \toprule \multicolumn{2}{c}{Components}                                                             & \multicolumn{5}{c}{Re-executability Rate (\%)} & \multicolumn{5}{c}{Readability (\#)} \\
    \cmidrule(lr){1-2} \cmidrule(lr){3-7} \cmidrule(lr){8-12}        \hspace{8pt}\labelemoji\hspace{8pt}                                                                & \hspace{8pt}\toolemoji\hspace{8pt}                                      & O0                                 & O1             & O2             & O3             & AVG            & O0             & O1             & O2             & O3             & AVG            \\
    \hline
    \rowcolor[rgb]{0.93,0.93,0.93}\multicolumn{12}{c}{\textbf{Initialize with LLM4Decompile-End-6.7B~\citep{llm4decompile}}}   \\
    \xmark                                                                                              & \xmark                                    & 69.51                              & 46.95          & 50.61          & 46.34          & 53.35          & 3.98 & 3.41 & 3.44 & 3.38 & 3.55 \\
    \cmark                                                                                              & \xmark                                    & 75.61                              & 50.61          & 50.00          & 50.00          & 56.55          & 4.01 & 3.44 & 3.39 & \textbf{3.49} & 3.58 \\
    \xmark                                                                                              & \cmark                                    & 83.54                     & \textbf{56.10}          & 51.22          & 50.61 & 60.37 & 4.05 & 3.51 & 3.51 & 3.42 & 3.62 \\
    \cmark                                                                                              & \cmark                                    & \textbf{85.37}                            & \textbf{56.10}                     & \textbf{51.83} & \textbf{52.43}          & \textbf{61.43} & \textbf{4.13} & \textbf{3.60} & \textbf{3.54} & \textbf{3.49} & \textbf{3.69} \\

    \rowcolor[rgb]{0.93,0.93,0.93}\multicolumn{12}{c}{\textbf{Initialize with Deepseek-Coder-6.7B-base~\citep{deepseekcoder}}} \\
    \xmark                                                                                              & \xmark                                    & 59.15                              & 35.98          & 39.02          & 37.80          & 42.99          & 3.71 & 3.05 & 3.16 & 3.05 & 3.24 \\
    \cmark                                                                                              & \xmark                                    & 66.46                              & 41.46          & 38.41          & 36.59          & 45.73          & 3.76 & 3.17 & \textbf{3.21} & 3.08 & 3.31 \\
    \xmark                                                                                              & \cmark                                    & 70.73                              & 39.63          & 39.02          & 40.24          & 47.41          & 3.90 & 3.17 & 3.08 & 3.11 & 3.31 \\
    \cmark                                                                                              & \cmark                                    & \textbf{79.88}                     & \textbf{45.73} & \textbf{43.90} & \textbf{42.68} & \textbf{53.05} & \textbf{3.96} & \textbf{3.21} & 3.18 & \textbf{3.19} & \textbf{3.38} \\
    \bottomrule
  \end{tabular}%
  }
  \caption{The ablation study of different methods across four optimization levels
  (O0, O1, O2, O3), as well as their average scores (AVG). The results in bold represent the optimal performance. The ~\labelemoji~ and ~\toolemoji~ means Relabedling and Function Call. \textbf{Bold} denotes the best performance.}
  \label{tab:ablation}
\end{table*}
\subsection{Ablation Study (RQ2)}


To validate the effectiveness of different components of our proposed method, we conduct ablation experiments to verify several components utilized in LLM Diagnosis and Cognitive Level alignment, including the usage of collaborative information (denoted as `Coll. Info'), the local contrast and global contrast (denoted as `Local Con.' and `Global Con.'), and the dynamic masking strategy (denoted as `Dym. Mask').

Table~\ref{tab:ablation} demonstrates the results of the ablation study on Python dataset, comparing the model performance after removing specific components (denoted as `w/o'). `w/o Coll. Info' represents replacing collaborative information in the process of diagnosis generation and `w/o Dym. Mask' represents replacing dynamic masking strategy with a constant mask ratio.
Experimental results show that removing these components individually leads to a decline in the model's performance. This indicates that these components are crucial for the model's performance.


\begin{figure}[t]
  \centering
  
  \includegraphics[width=1\linewidth]{figs/drop.png}
  \caption{Performance on different dropout ratios.}
  
\label{fig: drop}
\end{figure}
\subsection{Performance on Cold-Start Scenarios (RQ3)}

we conduct additional experiments on sub-datasets with varying degrees of sparsity. Specifically, we apply random dropout to the training sets of the Python and Linux datasets at ratios of $10\%$, $20\%$, $30\%$, $40\%$, and $50\%$.

Figure~\ref{fig: drop} shows the results of the experiments on different dropout ratios. It is obvious that as the dropout ratio increases, both AUC and ACC decrease. This is because the training set becomes more sparse, approaching a cold-start scenario. 
Additionally, compared to ACC, AUC experiences a greater decline, which might be due to the different calculation methods of the two metrics. 
% For more sparse datasets, Python, AUC experience a more significant decrease compared to the Linux dataset. From the experimental results, it can be seen that our proposed method is effective across different dropout ratios, leading to significant improvements for CDMs. More specifically, from the different performances of NCD-Beh and NCD-Sem in the Linux and Python datasets, it can be seen that we can choose different alignment methods based on the dataset to achieve better diagnostic results.


\begin{figure}[t]
  \centering
  \vspace{-1em}
  \includegraphics[width=1\linewidth]{figs/experiment2.png}
  \caption{The t-SNE visualization of student embeddings on Literature dataset.}
  \vspace{-2em}
\label{fig: experiment2}
\end{figure}
\subsection{Visualization of Semantic and Behavioral Embeddings (RQ4)}


To validate the effectiveness of the two alignment processes, we utilize t-SNE~\cite{van2008visualizing} to visualize the distribution of features in LLMs semantic space and CDMs behavioral space. We randomly select 200 example students and map their behavioral embeddings and semantic embeddings to 2-dimensional space. NCD (w/o Alignment) represents the original CDMs without alignment.

Figure~\ref{fig: experiment2} demonstrates the integration of semantic and behavioral embeddings of NCD-Beh and NCD-Sem, with their distributions closely merged compared to original CDMs. This proves the effectiveness of the two alignment methods we proposed.

\begin{figure}[t]
  \centering
  
  \includegraphics[width=1\linewidth]{figs/case.png}
  \caption{The case study of a student on multiple knowledge concepts on Linux dataset.}
  \vspace{-2em}
\label{fig: case}
\end{figure}

\subsection{Case Study}


To more intuitively demonstrate the improvements our proposed methods bring to CDMs, we selected a diagnosis for a specific student in the Linux dataset and compared the prediction results of NCD with the diagnosis results of NCD-Beh.
As illustrated in Figure~\ref{fig: case}, we randomly choose a student, and list his mastery of some knowledge concepts predicted by NCD and our proposed NCD-Beh.
This student correctly answered the exercises related to `numerical encoding' and `process communication', showing mastery of these concepts. He answered other exercises incorrectly, indicating a lack of familiarity with the remaining knowledge concepts.
From the LLM's diagnostic results, it can be observed that the LLM captured similar question-answer information from the training set and made corresponding inferences. This played an important role in NCD-Beh's more accurate prediction of the student's mastery level.
\section{Related Work}
\subsection{Sequential Recommendations}
Sequential recommendations (SRs) learn the user representation from the historical interaction sequence, then calculate the recommendation scores of candidate items and choose top-$k$ as the recommendation result~\cite{fang2020deep,wang2019sequential}. 
Previous works try various deep learning modules to enhance user modeling performance. For example, GRU4Rec~\cite{hidasi2015session} leverages the GRU layers, Caser~\cite{tang2018personalized} uses the CNN layers, and HGN~\cite{ma2019hierarchical} adopts GLU layers. The recent research direction of SRs gradually converges to prevalent Transformer~\cite{vaswani2017attention}. The Transformer-based methods, such as SASRec~\cite{kang2018self}, BERT4Rec~\cite{sun2019bert4rec}, FDSA~\cite{zhang2019feature}, BST~\cite{chen2019behavior}, and CORE~\cite{hou2022core}, leverage attention layers to capture the node correlation within the interaction sequence, which remarkably elevates the representation learning.
Despite the performance effectiveness of Transformer-based methods, the main challenges are efficiency~\cite{tay2022efficient}. The attention module of Transformers brings quadratic computational complexity with sequence length, which is unrealistic for deployment in industrial recommender systems that need low inference time. The most recent works, such as LinRec~\cite{liu2023linrec}, LRURec~\cite{yue2024linear}, Mamba4Rec~\cite{liu2024mamba4rec}, provide efficient solutions that can par with or even outperform Transformer-based methods. However, one essential issue is that the CF framework of SRs may lead to information cocoons and jeopardize user experience.

% Sequential recommendations (SRs) learn the user representation from the historical interaction sequence, then calculate the recommendation scores of candidate items and choose top-$k$ as the recommendation result~\cite{fang2020deep,wang2019sequential}. 
% Previous works try various deep learning modules to enhance user modeling performance. For example, GRU4Rec~\cite{hidasi2015session} leverages the GRU layers, Caser~\cite{tang2018personalized} uses the CNN layers, and HGN~\cite{ma2019hierarchical} adopts GLU layers. The recent research direction of SRs gradually converges to prevalent Transformer~\cite{vaswani2017attention}. The Transformer-based methods, such as SASRec~\cite{kang2018self}, BERT4Rec~\cite{sun2019bert4rec}, FDSA~\cite{zhang2019feature}, BST~\cite{chen2019behavior}, and CORE~\cite{hou2022core}, leverage attention layers to capture the node correlation and dependency within the interaction sequence, which remarkably elevates the representation learning.
% Despite the performance effectiveness of Transformer-based methods, the main challenges are efficiency and scalability~\cite{tay2022efficient}. The attention module of Transformers brings quadratic computational complexity with sequence length, which is unrealistic for deployment in industrial recommender systems that need low inference time. The most recent works, such as LinRec~\cite{liu2023linrec}, LRURec~\cite{yue2024linear}, Mamba4Rec~\cite{liu2024mamba4rec}, provide some efficient solutions that can par with or even outperform transformer-based methods. However, one essential issue is that the CF framework of SRs may lead to information cocoons and jeopardize user experience.

\subsection{Generative Recommendations}
Generative recommendations (GRs) leverage the generative capabilities of LLMs to generate more personalized and diverse results besides candidate items~\cite{xu2024prompting,zhao2023recommender,zhang2023recommendation,wu2024survey}. The mainstream of GRs, text-based GRs, leverage the text information in user interaction sequences as the user context to prompt LLMs to generate personalized results ~\cite{liu2023chatgpt,petrov2023generative,kang2023llms,geng2022recommendation,lyu2023llm}. 
Some frequently used prompting methods in the LLM area can be adapted to text-based GRs, such as in-context learning (ICL)~\cite{gao2020making}, and chain-of-thought (CoT)~\cite{wei2022chain}. The ICL-based methods~\cite{liu2023chatgpt,li2023bookgpt} incorporate the task description and in-context demonstrations (i.e., few-shot). While the CoT-based approaches~\cite{huang2023recommender,wang2023recmind} design task-specific reasoning steps to assist LLMs in generating personalized responses step-by-step. 
Nevertheless, employing the whole user interaction sequence as the text context of LLMs is impractical and unscalable. A regular user interaction sequence with hundreds of items can produce 10k to 100k tokens, which may cause an unacceptably high inference time and cost and even exceed the LLMs' context window limit. Therefore, a line of GRs tries to transform the user interaction sequence into a compact form, user embedding, to softly prompt LLMs for personalization~\cite{li2023prompt,ning2024user,li2023personalized,hebert2024persoma}. The embedding-based GRs highly reduce the token number compared to directly using text information, improving the efficiency and scalability of GRs. 

% With rich worldwide knowledge and diversity, LLMs provide a chance to break the information cocoons of users and bring sufficient personalization~\cite{xu2024prompting,zhao2023recommender,zhang2023recommendation,wu2024survey}. One promising direction LLMs can take to provide personalization is incorporating user history interactions as contextual information~\cite{liu2023chatgpt,petrov2023generative,kang2023llms,geng2022recommendation,lyu2023llm}.

% GRs aim to employ LLMs to generate more diverse and personalized recommendation results than SRs.
% One branch of GRs, text-based GRs, directly leverages the text information of user interactions. The interacted items' text information, such as ID, title, and category, will be filled in the reserved positions chronologically in the pre-designed prompt, serving as the user's context. Then, given specific questions like "What item will the user click next," the LLMs will generate a response such as "item $j$."
% As the user interactions are usually lengthy and noisy, embedding-based GRs transform the interaction sequence into an information-dense and refined form, i.e., user embeddings. The user embedding will be aligned to semantic space by the adapter and then act as the soft prompt for LLMs, prompting them to be more personalized. Compared to text-based GRs, this approach needs fewer tokens for the user context, meaning improved efficiency and scalability. The remaining concern is, "Can user embedding-based GRs provide comparable performance to text-based GRs?" Which is also the core viewpoint of this paper. 



\section{Conclusion}
We have presented Digital Twin Buildings, a framework for extracting the 3D mesh of a building, for connecting the building to Google Maps Platform APIs, and for Multi-Agent Large Language Models data analytics. We demonstrate this by extracting visual description keywords and captions of the building from multi-view multi-scale images of the building. The framework can also be used to process different data modalities sourced from Google Cloud Services. This approach enables richer semantic understanding, seamless integration with geospatial data, and enhanced interaction with real-world structures, paving the way for advanced applications in urban analytics, navigation, and virtual environments.


\newpage
\bibliographystyle{ACM-Reference-Format}
\bibliography{custom}

\end{document}
\endinput
%%
%% End of file `sample-sigconf.tex'.
