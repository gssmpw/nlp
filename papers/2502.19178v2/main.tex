%%
%% This is file `sample-sigconf.tex',
%% generated with the docstrip utility.
%%
%% The original source files were:
%%
%% samples.dtx  (with options: `all,proceedings,bibtex,sigconf')
%% 
%% IMPORTANT NOTICE:
%% 
%% For the copyright see the source file.
%% 
%% Any modified versions of this file must be renamed
%% with new filenames distinct from sample-sigconf.tex.
%% 
%% For distribution of the original source see the terms
%% for copying and modification in the file samples.dtx.
%% 
%% This generated file may be distributed as long as the
%% original source files, as listed above, are part of the
%% same distribution. (The sources need not necessarily be
%% in the same archive or directory.)
%%
%%
%% Commands for TeXCount
%TC:macro \cite [option:text,text]
%TC:macro \citep [option:text,text]
%TC:macro \citet [option:text,text]
%TC:envir table 0 1
%TC:envir table* 0 1
%TC:envir tabular [ignore] word
%TC:envir displaymath 0 word
%TC:envir math 0 word
%TC:envir comment 0 0
%%
%%
%% The first command in your LaTeX source must be the \documentclass
%% command.
%%
%% For submission and review of your manuscript please change the
%% command to \documentclass[manuscript, screen, review]{acmart}.
%%
%% When submitting camera ready or to TAPS, please change the command
%% to \documentclass[sigconf]{acmart} or whichever template is required
%% for your publication.
%%
%%
% \documentclass[sigconf,review]{acmart}

\documentclass[sigconf]{acmart}

\usepackage{threeparttable}
\usepackage{algorithm}
\usepackage{algorithmic}
\usepackage{booktabs}
\usepackage{enumitem}
\usepackage{multirow}
\usepackage{multicol}
\usepackage{array}
\usepackage{pifont}
\newcommand{\cmark}{\ding{51}} % 勾
\newcommand{\xmark}{\ding{55}} % 叉
\usepackage{listings}
\usepackage{amsmath}
\usepackage{color}
\usepackage{xcolor}
\newcommand{\lm}[1]{{\color{red} [llm: #1]}}

\usepackage{soul, color, xcolor,xspace}
\newcommand{\name}{UQABench\xspace}

% \usepackage{caption} % Include this in the preamble
% \captionsetup{skip=20pt} % Set the desired distance, for example, 10pt

% \usepackage{titlesec}
% % 调整 subsubsection 的间距
% \titlespacing{\subsubsection}{0pt}{1ex}{1.5ex}

%%
%% \BibTeX command to typeset BibTeX logo in the docs
\AtBeginDocument{%
  \providecommand\BibTeX{{%
    Bib\TeX}}}

%% Rights management information.  This information is sent to you
%% when you complete the rights form.  These commands have SAMPLE
%% values in them; it is your responsibility as an author to replace
%% the commands and values with those provided to you when you
%% complete the rights form.
\setcopyright{acmlicensed}
\copyrightyear{2018}
\acmYear{2018}
\acmDOI{XXXXXXX.XXXXXXX}

%% These commands are for a PROCEEDINGS abstract or paper.
\acmConference[Conference acronym 'XX]{Make sure to enter the correct
  conference title from your rights confirmation emai}{June 03--05,
  2018}{Woodstock, NY}
%%
%%  Uncomment \acmBooktitle if the title of the proceedings is different
%%  from ``Proceedings of ...''!
%%
%%\acmBooktitle{Woodstock '18: ACM Symposium on Neural Gaze Detection,
%%  June 03--05, 2018, Woodstock, NY}
\acmISBN{978-1-4503-XXXX-X/18/06}


%%
%% Submission ID.
%% Use this when submitting an article to a sponsored event. You'll
%% receive a unique submission ID from the organizers
%% of the event, and this ID should be used as the parameter to this command.
%%\acmSubmissionID{123-A56-BU3}

%%
%% For managing citations, it is recommended to use bibliography
%% files in BibTeX format.
%%
%% You can then either use BibTeX with the ACM-Reference-Format style,
%% or BibLaTeX with the acmnumeric or acmauthoryear sytles, that include
%% support for advanced citation of software artefact from the
%% biblatex-software package, also separately available on CTAN.
%%
%% Look at the sample-*-biblatex.tex files for templates showcasing
%% the biblatex styles.
%%

%%
%% The majority of ACM publications use numbered citations and
%% references.  The command \citestyle{authoryear} switches to the
%% "author year" style.
%%
%% If you are preparing content for an event
%% sponsored by ACM SIGGRAPH, you must use the "author year" style of
%% citations and references.
%% Uncommenting
%% the next command will enable that style.
%%\citestyle{acmauthoryear}


%%
%% end of the preamble, start of the body of the document source.
\begin{document}

%%
%% The "title" command has an optional parameter,
%% allowing the author to define a "short title" to be used in page headers.
\title{UQABench: Evaluating User Embedding for Prompting LLMs in Personalized Question Answering}

%%
%% The "author" command and its associated commands are used to define
%% the authors and their affiliations.
%% Of note is the shared affiliation of the first two authors, and the
%% "authornote" and "authornotemark" commands
%% used to denote shared contribution to the research.

% \author{Ben Trovato}
% \authornote{Both authors contributed equally to this research.}
% \email{trovato@corporation.com}
% \orcid{1234-5678-9012}
% \author{G.K.M. Tobin}
% \authornotemark[1]
% \email{webmaster@marysville-ohio.com}
% \affiliation{%
%   \institution{Institute for Clarity in Documentation}
%   \city{Dublin}
%   \state{Ohio}
%   \country{USA}
% }

\author{Langming Liu$^{\dagger}$, Shilei Liu$^{\dagger}$, Yujin Yuan, Yizhen Zhang, Bencheng Yan, Zhiyuan Zeng, Zihao Wang, Jiaqi Liu, Di Wang, Wenbo Su, Wang Pengjie, Jian Xu, Bo Zheng}
% \authornote{$^{\dagger}$Both authors contributed equally to this work.}
\affiliation{%
  \institution{Taobao \& Tmall Group of Alibaba}
  \city{}
  \country{}}

% \author{Haibin Chen}
% \affiliation{%
%   \institution{Taobao \& Tmall Group of Alibaba}
%   \city{Hangzhou}
%   \country{China}}

% \author{Yuhao Wang}
% \affiliation{%
%   \institution{City University of Hong Kong}
%   \city{Hong Kong}
%   \country{China}}

% \author{Yujin Yuan}
% \affiliation{%
%   \institution{Taobao \& Tmall Group of Alibaba}
%   \city{Hangzhou}
%   \country{China}}

% \author{Shilei Liu}
% \affiliation{%
%   \institution{Taobao \& Tmall Group of Alibaba}
%   \city{Hangzhou}
%   \country{China}}
  
% \author{Wenbo Su}
% \affiliation{%
%   \institution{Taobao \& Tmall Group of Alibaba}
%   \city{Hangzhou}
%   \country{China}}

% \author{Xiangyu	Zhao}
% \affiliation{%
%   \institution{City University of Hong Kong}
%   \city{Hong Kong}
%   \country{China}}

% \author{Bo Zheng}
% \affiliation{%
%   \institution{Taobao \& Tmall Group of Alibaba}
%   \city{Hangzhou}
%   \country{China}}

%%
%% By default, the full list of authors will be used in the page
%% headers. Often, this list is too long, and will overlap
%% other information printed in the page headers. This command allows
%% the author to define a more concise list
%% of authors' names for this purpose.
\renewcommand{\shortauthors}{Langming Liu, et al.}


%%
%% The abstract is a short summary of the work to be presented in the
%% article.
\begin{abstract}
Large language models (LLMs) achieve remarkable success in natural language processing (NLP). In practical scenarios like recommendations, as users increasingly seek personalized experiences, it becomes crucial to incorporate user interaction history into the context of LLMs to enhance personalization. 
However, from a practical utility perspective, user interactions' extensive length and noise present challenges when used directly as text prompts. 
A promising solution is to compress and distill interactions into compact embeddings, serving as soft prompts to assist LLMs in generating personalized responses. Although this approach brings efficiency, a critical concern emerges: Can user embeddings adequately capture valuable information and prompt LLMs?
To address this concern, we propose \name, a benchmark designed to evaluate the effectiveness of user embeddings in prompting LLMs for personalization. We establish a fair and standardized evaluation process, encompassing pre-training, fine-tuning, and evaluation stages. To thoroughly evaluate user embeddings, we design three dimensions of tasks: sequence understanding, action prediction, and interest perception. These evaluation tasks cover the industry's demands in traditional recommendation tasks, such as improving prediction accuracy, and its aspirations for LLM-based methods, such as accurately understanding user interests and enhancing the user experience.
We conduct extensive experiments on various state-of-the-art methods for modeling user embeddings. Additionally, we reveal the scaling laws of leveraging user embeddings to prompt LLMs.  
% The benchmark is available online at~\url{https://github.com/ming429778/UQABench}.
The benchmark is available online at~\url{https://github.com/OpenStellarTeam/UQABench}.

\end{abstract}

% Large language models (LLMs) have achieved remarkable success in natural language processing (NLP). In practical scenarios like recommendation systems, as users increasingly seek personalized experiences, it becomes crucial to incorporate user interaction history into the context of LLMs to enhance personalization. However, from a utility perspective, user interactions' extensive length and noise present challenges when used directly as text prompts. A promising solution is to compress and distill user interactions into compact embeddings, serving as soft prompts to assist LLMs in generating personalized responses. Although this approach brings efficiency, a critical concern emerges: Can user embeddings adequately capture valuable information and convey user interests to LLMs?
% To address this concern, we introduce \name, a comprehensive benchmark designed to evaluate the effectiveness of user embeddings in prompting LLMs for personalization. We establish a fair and standardized evaluation process, encompassing pre-training, fine-tuning, and evaluation stages. To thoroughly evaluate user embeddings, we design three dimensions of tasks: sequence understanding, action prediction, and interest perception. These evaluation tasks cover the industry's demands in traditional recommendation tasks, such as improving prediction accuracy, and its aspirations for LLM-based methods, such as accurately understanding user interests and enhancing the user experience.
% We conduct extensive experiments to compare various state-of-the-art methods for modeling user embeddings. Additionally, we reveal the scaling laws in modeling user interests using Transformers, the most widely adopted sequential model.

%%
%% The code below is generated by the tool at http://dl.acm.org/ccs.cfm.
%% Please copy and paste the code instead of the example below.
%%
\begin{CCSXML}
<ccs2012>
   <concept>
       <concept_id>10010147.10010178.10010179.10010186</concept_id>
       <concept_desc>Computing methodologies~Language resources</concept_desc>
       <concept_significance>500</concept_significance>
       </concept>
 </ccs2012>
\end{CCSXML}

\ccsdesc[500]{Computing methodologies~Language resources}

%%
%% Keywords. The author(s) should pick words that accurately describe
%% the work being presented. Separate the keywords with commas.
\keywords{Large Language Models, Recommendation, Personalization}
%% A "teaser" image appears between the author and affiliation
%% information and the body of the document, and typically spans the
%% page.

\maketitle

\sloppy

\section{Introduction}

In today’s rapidly evolving digital landscape, the transformative power of web technologies has redefined not only how services are delivered but also how complex tasks are approached. Web-based systems have become increasingly prevalent in risk control across various domains. This widespread adoption is due their accessibility, scalability, and ability to remotely connect various types of users. For example, these systems are used for process safety management in industry~\cite{kannan2016web}, safety risk early warning in urban construction~\cite{ding2013development}, and safe monitoring of infrastructural systems~\cite{repetto2018web}. Within these web-based risk management systems, the source search problem presents a huge challenge. Source search refers to the task of identifying the origin of a risky event, such as a gas leak and the emission point of toxic substances. This source search capability is crucial for effective risk management and decision-making.

Traditional approaches to implementing source search capabilities into the web systems often rely on solely algorithmic solutions~\cite{ristic2016study}. These methods, while relatively straightforward to implement, often struggle to achieve acceptable performances due to algorithmic local optima and complex unknown environments~\cite{zhao2020searching}. More recently, web crowdsourcing has emerged as a promising alternative for tackling the source search problem by incorporating human efforts in these web systems on-the-fly~\cite{zhao2024user}. This approach outsources the task of addressing issues encountered during the source search process to human workers, leveraging their capabilities to enhance system performance.

These solutions often employ a human-AI collaborative way~\cite{zhao2023leveraging} where algorithms handle exploration-exploitation and report the encountered problems while human workers resolve complex decision-making bottlenecks to help the algorithms getting rid of local deadlocks~\cite{zhao2022crowd}. Although effective, this paradigm suffers from two inherent limitations: increased operational costs from continuous human intervention, and slow response times of human workers due to sequential decision-making. These challenges motivate our investigation into developing autonomous systems that preserve human-like reasoning capabilities while reducing dependency on massive crowdsourced labor.

Furthermore, recent advancements in large language models (LLMs)~\cite{chang2024survey} and multi-modal LLMs (MLLMs)~\cite{huang2023chatgpt} have unveiled promising avenues for addressing these challenges. One clear opportunity involves the seamless integration of visual understanding and linguistic reasoning for robust decision-making in search tasks. However, whether large models-assisted source search is really effective and efficient for improving the current source search algorithms~\cite{ji2022source} remains unknown. \textit{To address the research gap, we are particularly interested in answering the following two research questions in this work:}

\textbf{\textit{RQ1: }}How can source search capabilities be integrated into web-based systems to support decision-making in time-sensitive risk management scenarios? 
% \sq{I mention ``time-sensitive'' here because I feel like we shall say something about the response time -- LLM has to be faster than humans}

\textbf{\textit{RQ2: }}How can MLLMs and LLMs enhance the effectiveness and efficiency of existing source search algorithms? 

% \textit{\textbf{RQ2:}} To what extent does the performance of large models-assisted search align with or approach the effectiveness of human-AI collaborative search? 

To answer the research questions, we propose a novel framework called Auto-\
S$^2$earch (\textbf{Auto}nomous \textbf{S}ource \textbf{Search}) and implement a prototype system that leverages advanced web technologies to simulate real-world conditions for zero-shot source search. Unlike traditional methods that rely on pre-defined heuristics or extensive human intervention, AutoS$^2$earch employs a carefully designed prompt that encapsulates human rationales, thereby guiding the MLLM to generate coherent and accurate scene descriptions from visual inputs about four directional choices. Based on these language-based descriptions, the LLM is enabled to determine the optimal directional choice through chain-of-thought (CoT) reasoning. Comprehensive empirical validation demonstrates that AutoS$^2$-\ 
earch achieves a success rate of 95–98\%, closely approaching the performance of human-AI collaborative search across 20 benchmark scenarios~\cite{zhao2023leveraging}. 

Our work indicates that the role of humans in future web crowdsourcing tasks may evolve from executors to validators or supervisors. Furthermore, incorporating explanations of LLM decisions into web-based system interfaces has the potential to help humans enhance task performance in risk control.






\begin{figure}[t]
    \centering
    % \vspace{-4mm}
    \includegraphics[width=1.0\linewidth]{pics/compare.pdf}
    % \vspace{-4mm}
    \caption{
    A brief comparison of SRs and GRs.
    }
    \label{fig:compare}
    \vspace{-2mm}
    % \captionsetup{belowskip=-20pt}
\end{figure}


\section{Preliminary}
We briefly introduce how both previous and current recommender systems utilize user interactions, focusing on two prevailing branches: sequential recommendations (SRs) and generative recommendations (GRs). The brief comparison is illustrated in Figure~\ref{fig:compare}.

\subsection{Sequential Recommendations (SRs)}
Given a set of users \(\mathcal{U}=\{u_1,u_2,\cdots,u_{\vert\mathcal{U}\vert}\}\) and a set of items \(\mathcal{V}=\{v_1,v_2,\cdots,v_{\vert\mathcal{V}\vert}\}\), consider that the interaction sequence of user \(u_i\) is denoted as \(s_i=[v_{1}^i, v_{2}^i, \cdots, v_{n_i}^i]\). For user \(u_i\), SRs take \(s_i\) as input and calculate recommendation scores (e.g., cosine similarity) for candidate items, then output the top-\(k\) items most likely to be interacted with in the subsequent time step.

\subsection{Generative Recommendations (GRs)}
GRs represent a shift towards utilizing LLMs to generate more personalized recommendation results, divided into two main branches:

\subsubsection{\textbf{Text-based GRs.}} 
Text-based GRs directly leverage the textual information of interactions. Interacted items' IDs and textual attributes, such as titles, categories, and descriptions, are chronologically placed in pre-designed prompts to serve as user context. Given questions like "What item will the user click next," the LLMs generate responses such as "item $j$."

\subsubsection{\textbf{Embedding-based GRs.}}
In contrast, embedding-based GRs transform lengthy, noisy interaction sequences into information-dense user embeddings. These user embeddings, aligned to the semantic space by an adapter, act as soft prompts for LLMs, personalizing them further. Compared to text-based GRs, this approach requires fewer tokens to convey user context, enhancing efficiency and scalability. However, a critical question remains: "Can this approach provide comparable personalized performance to text-based GRs?" This question forms one of the core research focuses of this paper.


% \section{Preliminary}
% We briefly introduce how both previous and current recommender systems utilize user interactions, focusing respectively on two prevailing branches: sequential recommendations (SRs) and generative recommendations (GRs).
% The brief comparison is shown in Figure~\ref{fig:compare}.

% \subsection{Sequential Recommendations (SRs)}
% Given a set of users \(\mathcal{U}=\{u_1,u_2,\cdots,u_{\vert\mathcal{U}\vert}\}\) and a set of items \(\mathcal{V}=\{v_1,v_2,\cdots,v_{\vert\mathcal{V}\vert}\}\), consider that the interaction sequence of user \(u_i\) is denoted as \(s_i=[v_{1}^i, v_{2}^i, \cdots, v_{n_i}^i]\). For user \(u_i\), SRs take \(s_i\) as input and calculate the recommendation scores (e.g., cosine similarity) for candidate items, then output the top-\(k\) items that are most likely to be interacted with in the subsequent time step.

% \subsection{Generative Recommendations (GRs)}
% \subsubsection{\textbf{Text-based GRs.}} 
% GRs employ LLMs to generate more personalized recommendation results than SRs.
% One branch of GRs, text-based GRs, directly leverages the textual information of interactions. The interacted items' IDs and textual information, such as title, category, and description, are filled in the reserved positions chronologically in the pre-designed prompt, serving as user context. Then, given questions like "What item will the user click next," the LLMs will generate responses such as "item $j$."

% \subsubsection{\textbf{Embedding-based GRs.}}
% As the user interactions are usually lengthy and noisy, embedding-based GRs transform the interaction sequence into an information-dense form, i.e., user embeddings. The user embedding will be aligned to the semantic space by the adapter and then act as the soft prompt for LLMs, prompting them to be more personalized. Compared to text-based GRs, this approach needs far fewer tokens for the user context, meaning improved efficiency and scalability. The remaining concern is, "Can this approach provide comparable performance to text-based GRs?" Which is also the core research point of this paper. 
\begin{figure*}[t]
    \centering
    % \vspace{-4mm}
    \includegraphics[width=0.95\linewidth]{pics/data_construct_pipeline_2.pdf}
    %\vspace{-5mm}
    \caption{
    An overview of the data construction process of ChineseEcomQA.
    }
    \label{fig:data_pipeline}
    % \vspace{-2mm}
    % \captionsetup{belowskip=-20pt}
\end{figure*}

\begin{figure*}[t]
    \centering
    % \vspace{-4mm}
    \includegraphics[width=0.95\linewidth]{pics/EcomQA_data_construct.pdf}
     %\vspace{-2mm}
    \caption{
    Illustrative examples of the data construction process.
    }
\label{fig:data_pipeline_example}
    % \vspace{-2mm}
    % \captionsetup{belowskip=-20pt}
\end{figure*}

\section{ChineseEcomQA}
\subsection{Overview}
Figure \ref{fig:overview} shows the fundamental e-commerce
concepts. Figure \ref{fig:data_pipeline} shows the overview of the data construction process of ChineseEcomQA. Besides, we provide illustrative examples of the data construction process in Figure \ref{fig:data_pipeline_example}. Table \ref{tab:dataset_statistics} shows the statistics of ChineseEcomQA. Figure \ref{fig:model_radar} visualizes the results of some selected models on ten sub concept tasks. In Appendix ~\ref{sec:appendix A}, we provide some examples of ChineseEcomQA. In the following subsection, we will introduce fundamental e-commerce concepts, data construction process, dataset statistics and evaluation metrics.

\subsection{Fundamental E-commerce Concepts}
Starting from the basic elements of e-commerce such as user behavior and product information, we summarized the main types of e-commerce concepts, defined 10 sub-concepts from basic concepts to advanced concepts as follows:

\begin{itemize}[leftmargin=*]

\item \textbf{Industry Categorization.} Given e-commerce corpus (such as user queries or web corpus), the LLMs need to figure out which e-commerce industries and categories are involved. The difficulty lies in distinguishing similar categories in the e-commerce domain.

\item \textbf{Industry Concept.} The model needs to understand the specialized knowledge in different e-commerce industries. The difficulty lies in accurately memorizing professional factual knowledge.

\item \textbf{Category Concept.} The model must understand which category a common, standard product belongs to.

\item \textbf{Brand Concept.} The model needs to recognize major brands and understand some background information about them.

\item \textbf{Attribute Concept.} E-commerce text often describes products using basic attributes, like style or age group. The model must have the ability to pick out these specific attribute words.

\item \textbf{Spoken Concept.} The e-commerce field is closely related to daily life scenarios, and people often use casual and imprecise language to express what they want. The model needs to understand the true expression forms.

\item \textbf{Intent Concept.} Beyond just informal language, sometimes consumers just list a bunch of attributes. The model needs to figure out the consumer's true intention from these phrases (such as how to choose).

\item \textbf{Review Concept.} The model needs to understand common concepts in user comments, such as emotional tendencies, commonly used evaluation aspects, etc.

\item \textbf{Relevance Concept.} One of the most crucial concepts of e-commerce is figuring out how relevant a product is to what a user wants. The model needs to integrate basic concepts such as intent concept and category concept to determine the relevance among user expression and products.

\item \textbf{Personalized Concept.} Personalized concept is one of the most important parts of user experience. This requires combining basic e-commerce concepts with general reasoning skills to recommend new product categories that best match a user's recent preferences.

\end{itemize}

\subsection{Data Collection}
\subsubsection{QA-pair Generation}
We collect a large amount of knowledge-rich e-commerce corpus, rich in information and covering various related concepts. Then, we prompt the LLM (GPT-4o) to generate question-answer pairs faithfully based on the given contents. For more open questions, we require the LLM to simultaneously provide candidate answers that are highly confusing and difficult. Providing candidate options in some concepts is beneficial for the objectivity and uniqueness of the evaluation.

\subsubsection{LLM Verification}
In the previous subsection, we collected a large number of question-answer pairs. To ensure the basic quality of dataset, we use LLM (GPT-4o) to filter data that cannot meet the requirements of our predefined criteria. Specifically, the question-answer pairs must meet the following criteria.

\begin{itemize}[leftmargin=*]
\item Questions must be a clear question about an e-commerce concept. There should be no ambiguity in the problem statement. For example, "What is the most well-known brand of washing machine?" is a disqualification question, because "most familiar" may be controversial.

\item Questions must be answerable as of 2023. The e-commerce concept investigated in the question cannot use the industry knowledge after December 31, 2023.

\item Answers must be objective and unique. There should be only one clear and objective answer to the question raised. For example, "What do you think about the Dynamic Islang on iPhone? " is too subjective.

\item Answers must not change over time. Questions about current trends or the latest product series, which are constantly evolving, are not suitable.

\end{itemize}

\subsubsection{E-commerce Generality Verification}
Since we use e-commerce corpus to construct the question-answering pairs, the original data may contain platform-specific knowledge, which affects the generality of the dataset. Therefore, we deploy external retrieval tools (i.e., web search engines) to gather information from a wider range of sources. This helped the LLM to evaluate the generality of dataset. For some question types, like those requiring the model to choose the correct category from a list, direct web searching wasn't practical. In these cases, we first require the LLM to identify several knowledge points related to the question. We then used these knowledge points as search queries. After getting the search results, we used the LLM again to remove any data that does not conform to domain generality. As illustrated in Figure \ref{fig:data_pipeline_example}, "Are there screen protectors for iPhone 14 available on Taobao?" is not common knowledge in e-commerce.

\subsubsection{E-commerce Expertise Verification}
While web searches helped ensure e-commerce generality, we also needed to confirm that the questions had sufficient depth within the e-commerce domain. Therefore, we use e-commerce search engines (such as Taobao search and e-commerce encyclopedia) to obtain more specialized information. We require the LLM to judge the level of expertise, as concept that is too basic will be filtered out.

\subsubsection{E-commerce Expertise Verification}
Real-world e-commerce problems require to integrate domain expertise and general knowledge. To verify the factual correctness of question-answer pairs, we require the LLM consider information from both general web searches and specialized e-commerce searches. For example in Figure \ref{fig:data_pipeline_example}, regarding the question of iPhone 14 Pro, we used web search results and product encyclopedias to comprehensively determine the correctness of the facts.

\subsubsection{Difficulty Filtering}
We discover the knowledge boundaries of the LLMs by checking whether multiple LLMs answer correctly. Specifically, we selected models from the Qwen series, LLaMA series, and GPT-4o for evaluation. If all the models answer a question correctly, we considered it too easy and removed it.

\subsubsection{Human Verification}
Finally, we use manual annotation to verify the quality of the evaluation set. Manual annotation requires a comprehensive consideration of the characteristics mentioned above, such as domain generality, domain expertise, and overall quality. In fact, most of the problematic data has already been filtered by the previous process. Combining LLM's ability to integrate information, verify information, and perform manual verification, we believe it is a more scalable construction process.

\subsection{Dataset Statistics}
Figure \ref{tab:dataset_statistics} presents the statistics of ChineseEcomQA. With a total of 1,800 samples, ChineseSimpleQA \cite{he2024chinesesimpleqachinesefactuality} relatively evenly distributed 10 sub-concept types. Furthermore, the average length of reference answers is 18.26. The concise and unified format, consistent with the articles in the SimpleQA series~\cite{wei2024measuring,he2024chinese}, has the advantages of easy-to-evaluate and relatively low evaluation cost.

% \begin{table}[h]
% \centering
% \begin{tabular}{@{} l r @{}} % 用于取消左右两边的默认列间距
% \toprule
% \toprule
% \textbf{Statistics} & \textbf{Number} \\
% \midrule
% \textbf{Total Number} & 1,800 \\
% - Industry Categorization & 286 \\
% - Industry Concept & 111 \\
% - Category Concept & 194 \\
% - Brand Concept & 170 \\
% - Attribute Concept & 200 \\
% - Spoken Concept & 200 \\
% - Intent Concept& 90 \\
% - Review Concept& 71 \\
% - Relevance Concept& 305 \\
% - Personalized Concept& 173 \\
% \midrule
% \multicolumn{2}{@{}l}{\textbf{Average Length}} \\
% - Question Length & xx \\
% - Reference Answer Length & xx \\
% \bottomrule
% \bottomrule
% \end{tabular}
% \vspace{3mm}
% \caption{
%     Dataset statistics of ChineseEcomQA.
% }
%     \label{tab:dataset_statistics}
% \end{table}

\begin{figure}[t]
    \centering
    % \vspace{-4mm}
    \includegraphics[width=0.95\linewidth]{pics/domain_pie.png}
    % \vspace{-4mm}
    \caption{
    Dataset statistics of ChineseEcomQA.
    }
    \label{tab:dataset_statistics}
    % \vspace{-4mm}
    % \captionsetup{belowskip=-20pt}
\end{figure}

\subsection{Evaluation Metrics}
Given reference answer, LLM-as-a-judge is an effective and prevalence method \cite{gu2024survey}. Based on the questions, candidate answers, and reference answers, we use GPT-4o, Claude-3.5-Sonnet and Deepseek-V3 \cite{liu2024deepseek} as the judge models. The final judgment is determined by the voting results of three LLMs. There are three types of evaluation criteria: (1) Correct: The candidate answer fully includes the reference answer without introducing any contradictory elements. (2) Wrong: The candidate answer contains factual statements that contradict the reference answer. (3) Not Attempted: The LLM is not confident enough to provide specific answers, and there are no statements in the response that contradict the reference answer. The evaluation prompt can be found in the Appendix \ref{sec:appendix B}. In the results section that follows, we use the proportion of "Correct" answers as our accuracy metric.

\section{Experiment}
\label{s:experiment}

\subsection{Data Description}
We evaluate our method on FI~\cite{you2016building}, Twitter\_LDL~\cite{yang2017learning} and Artphoto~\cite{machajdik2010affective}.
FI is a public dataset built from Flickr and Instagram, with 23,308 images and eight emotion categories, namely \textit{amusement}, \textit{anger}, \textit{awe},  \textit{contentment}, \textit{disgust}, \textit{excitement},  \textit{fear}, and \textit{sadness}. 
% Since images in FI are all copyrighted by law, some images are corrupted now, so we remove these samples and retain 21,828 images.
% T4SA contains images from Twitter, which are classified into three categories: \textit{positive}, \textit{neutral}, and \textit{negative}. In this paper, we adopt the base version of B-T4SA, which contains 470,586 images and provides text descriptions of the corresponding tweets.
Twitter\_LDL contains 10,045 images from Twitter, with the same eight categories as the FI dataset.
% 。
For these two datasets, they are randomly split into 80\%
training and 20\% testing set.
Artphoto contains 806 artistic photos from the DeviantArt website, which we use to further evaluate the zero-shot capability of our model.
% on the small-scale dataset.
% We construct and publicly release the first image sentiment analysis dataset containing metadata.
% 。

% Based on these datasets, we are the first to construct and publicly release metadata-enhanced image sentiment analysis datasets. These datasets include scenes, tags, descriptions, and corresponding confidence scores, and are available at this link for future research purposes.


% 
\begin{table}[t]
\centering
% \begin{center}
\caption{Overall performance of different models on FI and Twitter\_LDL datasets.}
\label{tab:cap1}
% \resizebox{\linewidth}{!}
{
\begin{tabular}{l|c|c|c|c}
\hline
\multirow{2}{*}{\textbf{Model}} & \multicolumn{2}{c|}{\textbf{FI}}  & \multicolumn{2}{c}{\textbf{Twitter\_LDL}} \\ \cline{2-5} 
  & \textbf{Accuracy} & \textbf{F1} & \textbf{Accuracy} & \textbf{F1}  \\ \hline
% (\rownumber)~AlexNet~\cite{krizhevsky2017imagenet}  & 58.13\% & 56.35\%  & 56.24\%& 55.02\%  \\ 
% (\rownumber)~VGG16~\cite{simonyan2014very}  & 63.75\%& 63.08\%  & 59.34\%& 59.02\%  \\ 
(\rownumber)~ResNet101~\cite{he2016deep} & 66.16\%& 65.56\%  & 62.02\% & 61.34\%  \\ 
(\rownumber)~CDA~\cite{han2023boosting} & 66.71\%& 65.37\%  & 64.14\% & 62.85\%  \\ 
(\rownumber)~CECCN~\cite{ruan2024color} & 67.96\%& 66.74\%  & 64.59\%& 64.72\% \\ 
(\rownumber)~EmoVIT~\cite{xie2024emovit} & 68.09\%& 67.45\%  & 63.12\% & 61.97\%  \\ 
(\rownumber)~ComLDL~\cite{zhang2022compound} & 68.83\%& 67.28\%  & 65.29\% & 63.12\%  \\ 
(\rownumber)~WSDEN~\cite{li2023weakly} & 69.78\%& 69.61\%  & 67.04\% & 65.49\% \\ 
(\rownumber)~ECWA~\cite{deng2021emotion} & 70.87\%& 69.08\%  & 67.81\% & 66.87\%  \\ 
(\rownumber)~EECon~\cite{yang2023exploiting} & 71.13\%& 68.34\%  & 64.27\%& 63.16\%  \\ 
(\rownumber)~MAM~\cite{zhang2024affective} & 71.44\%  & 70.83\% & 67.18\%  & 65.01\%\\ 
(\rownumber)~TGCA-PVT~\cite{chen2024tgca}   & 73.05\%  & 71.46\% & 69.87\%  & 68.32\% \\ 
(\rownumber)~OEAN~\cite{zhang2024object}   & 73.40\%  & 72.63\% & 70.52\%  & 69.47\% \\ \hline
(\rownumber)~\shortname  & \textbf{79.48\%} & \textbf{79.22\%} & \textbf{74.12\%} & \textbf{73.09\%} \\ \hline
\end{tabular}
}
\vspace{-6mm}
% \end{center}
\end{table}
% 

\subsection{Experiment Setting}
% \subsubsection{Model Setting.}
% 
\textbf{Model Setting:}
For feature representation, we set $k=10$ to select object tags, and adopt clip-vit-base-patch32 as the pre-trained model for unified feature representation.
Moreover, we empirically set $(d_e, d_h, d_k, d_s) = (512, 128, 16, 64)$, and set the classification class $L$ to 8.

% 

\textbf{Training Setting:}
To initialize the model, we set all weights such as $\boldsymbol{W}$ following the truncated normal distribution, and use AdamW optimizer with the learning rate of $1 \times 10^{-4}$.
% warmup scheduler of cosine, warmup steps of 2000.
Furthermore, we set the batch size to 32 and the epoch of the training process to 200.
During the implementation, we utilize \textit{PyTorch} to build our entire model.
% , and our project codes are publicly available at https://github.com/zzmyrep/MESN.
% Our project codes as well as data are all publicly available on GitHub\footnote{https://github.com/zzmyrep/KBCEN}.
% Code is available at \href{https://github.com/zzmyrep/KBCEN}{https://github.com/zzmyrep/KBCEN}.

\textbf{Evaluation Metrics:}
Following~\cite{zhang2024affective, chen2024tgca, zhang2024object}, we adopt \textit{accuracy} and \textit{F1} as our evaluation metrics to measure the performance of different methods for image sentiment analysis. 



\subsection{Experiment Result}
% We compare our model against the following baselines: AlexNet~\cite{krizhevsky2017imagenet}, VGG16~\cite{simonyan2014very}, ResNet101~\cite{he2016deep}, CECCN~\cite{ruan2024color}, EmoVIT~\cite{xie2024emovit}, WSCNet~\cite{yang2018weakly}, ECWA~\cite{deng2021emotion}, EECon~\cite{yang2023exploiting}, MAM~\cite{zhang2024affective} and TGCA-PVT~\cite{chen2024tgca}, and the overall results are summarized in Table~\ref{tab:cap1}.
We compare our model against several baselines, and the overall results are summarized in Table~\ref{tab:cap1}.
We observe that our model achieves the best performance in both accuracy and F1 metrics, significantly outperforming the previous models. 
This superior performance is mainly attributed to our effective utilization of metadata to enhance image sentiment analysis, as well as the exceptional capability of the unified sentiment transformer framework we developed. These results strongly demonstrate that our proposed method can bring encouraging performance for image sentiment analysis.

\setcounter{magicrownumbers}{0} 
\begin{table}[t]
\begin{center}
\caption{Ablation study of~\shortname~on FI dataset.} 
% \vspace{1mm}
\label{tab:cap2}
\resizebox{.9\linewidth}{!}
{
\begin{tabular}{lcc}
  \hline
  \textbf{Model} & \textbf{Accuracy} & \textbf{F1} \\
  \hline
  (\rownumber)~Ours (w/o vision) & 65.72\% & 64.54\% \\
  (\rownumber)~Ours (w/o text description) & 74.05\% & 72.58\% \\
  (\rownumber)~Ours (w/o object tag) & 77.45\% & 76.84\% \\
  (\rownumber)~Ours (w/o scene tag) & 78.47\% & 78.21\% \\
  \hline
  (\rownumber)~Ours (w/o unified embedding) & 76.41\% & 76.23\% \\
  (\rownumber)~Ours (w/o adaptive learning) & 76.83\% & 76.56\% \\
  (\rownumber)~Ours (w/o cross-modal fusion) & 76.85\% & 76.49\% \\
  \hline
  (\rownumber)~Ours  & \textbf{79.48\%} & \textbf{79.22\%} \\
  \hline
\end{tabular}
}
\end{center}
\vspace{-5mm}
\end{table}


\begin{figure}[t]
\centering
% \vspace{-2mm}
\includegraphics[width=0.42\textwidth]{fig/2dvisual-linux4-paper2.pdf}
\caption{Visualization of feature distribution on eight categories before (left) and after (right) model processing.}
% 
\label{fig:visualization}
\vspace{-5mm}
\end{figure}

\subsection{Ablation Performance}
In this subsection, we conduct an ablation study to examine which component is really important for performance improvement. The results are reported in Table~\ref{tab:cap2}.

For information utilization, we observe a significant decline in model performance when visual features are removed. Additionally, the performance of \shortname~decreases when different metadata are removed separately, which means that text description, object tag, and scene tag are all critical for image sentiment analysis.
Recalling the model architecture, we separately remove transformer layers of the unified representation module, the adaptive learning module, and the cross-modal fusion module, replacing them with MLPs of the same parameter scale.
In this way, we can observe varying degrees of decline in model performance, indicating that these modules are indispensable for our model to achieve better performance.

\subsection{Visualization}
% 


% % 开始使用minipage进行左右排列
% \begin{minipage}[t]{0.45\textwidth}  % 子图1宽度为45%
%     \centering
%     \includegraphics[width=\textwidth]{2dvisual.pdf}  % 插入图片
%     \captionof{figure}{Visualization of feature distribution.}  % 使用captionof添加图片标题
%     \label{fig:visualization}
% \end{minipage}


% \begin{figure}[t]
% \centering
% \vspace{-2mm}
% \includegraphics[width=0.45\textwidth]{fig/2dvisual.pdf}
% \caption{Visualization of feature distribution.}
% \label{fig:visualization}
% % \vspace{-4mm}
% \end{figure}

% \begin{figure}[t]
% \centering
% \vspace{-2mm}
% \includegraphics[width=0.45\textwidth]{fig/2dvisual-linux3-paper.pdf}
% \caption{Visualization of feature distribution.}
% \label{fig:visualization}
% % \vspace{-4mm}
% \end{figure}



\begin{figure}[tbp]   
\vspace{-4mm}
  \centering            
  \subfloat[Depth of adaptive learning layers]   
  {
    \label{fig:subfig1}\includegraphics[width=0.22\textwidth]{fig/fig_sensitivity-a5}
  }
  \subfloat[Depth of fusion layers]
  {
    % \label{fig:subfig2}\includegraphics[width=0.22\textwidth]{fig/fig_sensitivity-b2}
    \label{fig:subfig2}\includegraphics[width=0.22\textwidth]{fig/fig_sensitivity-b2-num.pdf}
  }
  \caption{Sensitivity study of \shortname~on different depth. }   
  \label{fig:fig_sensitivity}  
\vspace{-2mm}
\end{figure}

% \begin{figure}[htbp]
% \centerline{\includegraphics{2dvisual.pdf}}
% \caption{Visualization of feature distribution.}
% \label{fig:visualization}
% \end{figure}

% In Fig.~\ref{fig:visualization}, we use t-SNE~\cite{van2008visualizing} to reduce the dimension of data features for visualization, Figure in left represents the metadata features before model processing, the features are obtained by embedding through the CLIP model, and figure in right shows the features of the data after model processing, it can be observed that after the model processing, the data with different label categories fall in different regions in the space, therefore, we can conclude that the Therefore, we can conclude that the model can effectively utilize the information contained in the metadata and use it to guide the model for classification.

In Fig.~\ref{fig:visualization}, we use t-SNE~\cite{van2008visualizing} to reduce the dimension of data features for visualization.
The left figure shows metadata features before being processed by our model (\textit{i.e.}, embedded by CLIP), while the right shows the distribution of features after being processed by our model.
We can observe that after the model processing, data with the same label are closer to each other, while others are farther away.
Therefore, it shows that the model can effectively utilize the information contained in the metadata and use it to guide the classification process.

\subsection{Sensitivity Analysis}
% 
In this subsection, we conduct a sensitivity analysis to figure out the effect of different depth settings of adaptive learning layers and fusion layers. 
% In this subsection, we conduct a sensitivity analysis to figure out the effect of different depth settings on the model. 
% Fig.~\ref{fig:fig_sensitivity} presents the effect of different depth settings of adaptive learning layers and fusion layers. 
Taking Fig.~\ref{fig:fig_sensitivity} (a) as an example, the model performance improves with increasing depth, reaching the best performance at a depth of 4.
% Taking Fig.~\ref{fig:fig_sensitivity} (a) as an example, the performance of \shortname~improves with the increase of depth at first, reaching the best performance at a depth of 4.
When the depth continues to increase, the accuracy decreases to varying degrees.
Similar results can be observed in Fig.~\ref{fig:fig_sensitivity} (b).
Therefore, we set their depths to 4 and 6 respectively to achieve the best results.

% Through our experiments, we can observe that the effect of modifying these hyperparameters on the results of the experiments is very weak, and the surface model is not sensitive to the hyperparameters.


\subsection{Zero-shot Capability}
% 

% (1)~GCH~\cite{2010Analyzing} & 21.78\% & (5)~RA-DLNet~\cite{2020A} & 34.01\% \\ \hline
% (2)~WSCNet~\cite{2019WSCNet}  & 30.25\% & (6)~CECCN~\cite{ruan2024color} & 43.83\% \\ \hline
% (3)~PCNN~\cite{2015Robust} & 31.68\%  & (7)~EmoVIT~\cite{xie2024emovit} & 44.90\% \\ \hline
% (4)~AR~\cite{2018Visual} & 32.67\% & (8)~Ours (Zero-shot) & 47.83\% \\ \hline


\begin{table}[t]
\centering
\caption{Zero-shot capability of \shortname.}
\label{tab:cap3}
\resizebox{1\linewidth}{!}
{
\begin{tabular}{lc|lc}
\hline
\textbf{Model} & \textbf{Accuracy} & \textbf{Model} & \textbf{Accuracy} \\ \hline
(1)~WSCNet~\cite{2019WSCNet}  & 30.25\% & (5)~MAM~\cite{zhang2024affective} & 39.56\%  \\ \hline
(2)~AR~\cite{2018Visual} & 32.67\% & (6)~CECCN~\cite{ruan2024color} & 43.83\% \\ \hline
(3)~RA-DLNet~\cite{2020A} & 34.01\%  & (7)~EmoVIT~\cite{xie2024emovit} & 44.90\% \\ \hline
(4)~CDA~\cite{han2023boosting} & 38.64\% & (8)~Ours (Zero-shot) & 47.83\% \\ \hline
\end{tabular}
}
\vspace{-5mm}
\end{table}

% We use the model trained on the FI dataset to test on the artphoto dataset to verify the model's generalization ability as well as robustness to other distributed datasets.
% We can observe that the MESN model shows strong competitiveness in terms of accuracy when compared to other trained models, which suggests that the model has a good generalization ability in the OOD task.

To validate the model's generalization ability and robustness to other distributed datasets, we directly test the model trained on the FI dataset, without training on Artphoto. 
% As observed in Table 3, compared to other models trained on Artphoto, we achieve highly competitive zero-shot performance, indicating that the model has good generalization ability in out-of-distribution tasks.
From Table~\ref{tab:cap3}, we can observe that compared with other models trained on Artphoto, we achieve competitive zero-shot performance, which shows that the model has good generalization ability in out-of-distribution tasks.


\section{Related Work}
\subsection{LLM Factuality}
LLMs have demonstrated exceptional capabilities in memorizing and utilizing factual knowledge through their strong parametric memories, enabling applications in knowledge-intensive domains. This potential has driven significant research into deploying LLMs for e-commerce applications, including product recommendation systems \cite{xu2024leveraginglargelanguagemodels,10.1145/3580305.3599519,10.1145/3616855.3635853}, search ranking \cite{10.1007/978-3-031-56060-6_24,rathee2025guidingretrievalusingllmbased}, and attribute extraction \cite{10.1007/978-3-031-78090-5_4, Baumann2024UsingLF,zou-etal-2024-eiven}. Recent domain-specific adaptations like EcomGPT \cite{10.1609/aaai.v38i17.29820} and eCeLLM \cite{10.5555/3692070.3693702} employ instruction tuning to align general-purpose LLMs with e-commerce scenarios. However, these approaches remain constrained by training data, which focus on narrow operational competencies (e.g., single-task attribute recognition) rather than comprehensive understanding.

\subsection{E-commerce Datasets}
Prior e-commerce datasets predominantly concentrate on isolated or narrowly related tasks. For instance, Amazon-M2 \cite{10.5555/3666122.3666473} focuses on session-based recommendation systems, Amazon-ESCI \cite{reddy2022shoppingqueriesdatasetlargescale} specializes in query-product matching, while EComInstruct \cite{10.1609/aaai.v38i17.29820} targets shopping concept understanding. These datasets exhibit limited coverage of the multifaceted skill requirements inherent to real-world e-commerce applications due to their constrained task diversity.

Shopping MMLU \cite{NEURIPS2024_2049d75d}, concurrently with our work, presents a multi-dimensional benchmark constructed from Amazon data. However, it exclusively focuses on the English-language domain. To address the absence of comprehensive evaluation resources for Chinese e-commerce ecosystems, we propose ChineseEcomQA - a benchmark systematically assessing diverse e-commerce capabilities in Chinese contexts.

\begin{figure}[t]
    \centering
    %\vspace{-2mm}
\includegraphics[width=1\columnwidth]{pics/rank_range.pdf}
    \vspace{-3mm}
    \caption{
     The rankings of different LLMs on ChineseSimpleQA and ChineseEcomQA.
    }
    \label{fig:rank_range}
    \vspace{-3mm}
\end{figure}

\vspace{-1mm}
\section{Conclusion}
In this paper, we propose ChineseEcomQA, a scalable question-answering benchmark designed to rigorously assess LLMs on fundamental e-commerce concepts. ChineseEcomQA is characterized by three core features: Focus on Fundamental Concept, E-Commerce Generalizability, and Domain-Specific Expertise, which collectively enable systematic evaluation of LLMs' e-commerce knowledge. Leveraging ChineseEcomQA, we conduct extensive evaluations on mainstream LLMs, yielding critical insights into their capabilities and limitations. Our findings not only highlight performance disparities across models but also delineate actionable directions for advancing LLM applications in the e-commerce domain.

\newpage
\bibliographystyle{ACM-Reference-Format}
\bibliography{custom}

\end{document}
\endinput
%%
%% End of file `sample-sigconf.tex'.
