\subsection{Related Work}\label{sec:related work}
\paragraph{Allocation with Non-Strategic Agents.}
When facing non-strategic agents whose utilities are revealed to the planner, non-monetary resource allocation mechanisms mainly focus on the trade-off between efficiency and fairness \citep{moulin2002proportional,procaccia2013approximate,caragiannis2019unreasonable}.
While fairness and efficiency can be optimized at the same time with non-strategic agents, it is in general impossible to ensure efficiency, fairness, and incentive-compatibility simultaneously \citep{budish2011combinatorial}; this paper focuses on efficiency and incentive-compatibility and sacrifices fairness.

\paragraph{One-Shot Allocation without Money.}
We move on to strategic agents, where agents submit reports instead of true utilities. However, the planner has access to agents' true utility distributions.
When the allocation is one-shot (\textit{i.e.}, $T=1$), ensuring both efficiency and incentive-compatibility is possible in specific setups: \citet{miralles2012cardinal} characterized the two-resource case where agents share the same utility distribution, and \citet{guo2010strategy}, \citet{han2011strategy}, and \citet{cole2013positive} studied the competitive ratio when agents' utilities are additive. Another slightly different model where the planner can charge the winners but not redistribute the charges, called \textit{money burning} setting, was also studied \citep{hartline2008optimal,hoppe2009theory,condorelli2012money}. Nevertheless, due to the Arrow's impossibility theorem \citep{arrow1950difficulty,gi73,s75}, when facing general utility distributions, one-shot mechanisms usually cannot converge to first-best allocations that maximize social welfare.

\paragraph{Repeated Allocation without Money.}
When the allocation is repeated for $T$ rounds and agents' utility distributions are still available, the folk theorem ensures the existence of a mechanism converging to the first-best allocation asymptotically \citep{fudenberg2009folk}, whose explicit characterization was given by \citet{guo2009competitive}. To sharpen the convergence rate, the \textit{artificial currencies} approach that allocates each agent an initial budget based on their utility distribution was proposed \citep{friedman2006efficiency,jackson2007overcoming,kash2007optimizing,kash2015equilibrium,budish2011combinatorial,johnson2014analyzing,gorokh2021monetary}. Among them, the most recent mechanism designed by \citet{gorokh2021monetary} ensured an $\O(1/T)$ convergence to an $\Otil(1/\sqrt T)$-approximate equilibrium. Targeting for perfect equilibria, the \textit{future promises} framework, dating back to the d'Aspremont--Gérard-Varet--Arrow mechanism \citep{d1979incentives,arrow1979property} and recently used by \citet{balseiro2019multiagent} and \citet{blanchard2024near}, ensured an $\O(1/\sqrt T)$-convergence to a Perfect Beyasian Equilibrium (PBE). This convergence rate was further sharpened to $\Otil(1/T)$ when agents' utility distributions enjoy specific properties \citep{blanchard2024near}.

\paragraph{Repeated Allocation without Money and Distributional Information.}
Finally, we introduce our setup where agents are strategic, monetary transfers are prohibited, and distributional information is unrevealed. In this setup, ensuring efficiency and incentive-compatibility is extraordinarily hard, and \citet{gorokh2019remarkable}, \citet{banerjee2023robust}, and \citet{fikioris2023online} opted for a \textit{$\beta$-Utopia} notion which focuses on ensuring a competitively high utility for every agent given a pre-determined allocation profile $\bm \alpha=(\alpha_i)_{i\in [K]}$ that says agent $i\in [K]$ should be allocated for $\alpha_i T$ rounds in expectation; however, we argue the generation of an optimal $\bm \alpha$ without distributions is highly non-trivial. In contrast, this paper stick to the notion of social welfare regret, which is a common metric in resource allocation literature \citep{devanur2009adwords,feldman2010online,agrawal2014dynamic,devanur2019near,balseiro2023best}, with the help of post-decision audits that reflects a central planner in a real-world resource allocation problem. Aligned with our social welfare objective, \citet{yin2022online} also designed a non-monetary mechanism ensuring incentive-compatibility and efficiency without distributional information; however, their approach critically relies on the assumption that agents have identical utility distributions, whereas those in our setup can be distinct.

\paragraph{Audits for Incentive-Compatibility.}
The idea of audits was utilized in various mechanism design setups for incentive-compatibility, for example, in the money-burning (\textit{i.e.}, money can be charged but not reallocated) repeated allocation problem where utility distributions are known \citep{lundy2019allocation}, in strategic classification problems \citep{estornell2021incentivizing,estornell2023incentivizing}, and in one-shot non-monetary allocation problem where utility distributions are known \citep{jalota2024catch}. However, none of these approaches applied to the unknown distribution case, which is much harder since a good audit plan ensuring incentive-compatibility requires agents' properties (for example, the fair shares that we use), whose estimation is coupled with agents' strategic reports in the past.

\paragraph{Comparison with \citep{yin2022online}.}
\citet{yin2022online} also utilized the idea of imposing future threats to design non-monetary allocation mechanisms where distributional information is missing, but our setups are very different:
\citet{yin2022online} made an important assumption that the agents have identical utility distributions, \textit{i.e.}, $\mV_i\equiv \mV_1$ for all $i\in [K]$. On the other hand, we allow the agents to have distinct utility distributions (the only requirement \Cref{assumption:min report}, as we argued, is weak). 
Importantly, their identical distribution assumption is not only for presentation simplicity but rather crucial for their design and analysis: due to their identical utility distribution assumption, \citet{yin2022online} threaten agents with elimination once their reports differ significantly from others (who are assumed to be honest since they aimed at proving honesty is approximately a PBE). On the other hand, due to agents' distinct distributions, we can impose threats via agents' previous reports, which is more complicated due to the coupling between strategic allocations and online estimates.
