\documentclass[11pt,fullpage,letterpaper]{article} % For LaTeX2e

\usepackage{natbib}

\bibliographystyle{plainnat}
% modification to natbib citations
\setcitestyle{authoryear,round,citesep={;},aysep={,},yysep={;}}
\usepackage[english]{babel}

\oddsidemargin 0in
\evensidemargin 0in
\textwidth 6.5in
\topmargin -0.5in
\textheight 9.0in
\usepackage{fancyhdr}
\pagestyle{fancy}
\fancyhead[C]{Non-Monetary Mechanism Design without Distributional Information}
\fancyhead[L]{}
\fancyhead[R]{}

\usepackage[utf8]{inputenc} % allow utf-8 input
\usepackage[T1]{fontenc}    % use 8-bit T1 fonts

\usepackage{titletoc}
\usepackage[toc, page, header]{appendix} %%% MAKE SURE TO PUT THIS BEFORE hyperref PACKAGE

\usepackage[colorlinks=true, linkcolor=blue, citecolor=blue,urlcolor=black]{hyperref}
\usepackage{url}            % simple URL typesetting
\usepackage{booktabs}       % professional-quality tables
\usepackage{amsfonts}       % blackboard math symbols
\usepackage{nicefrac}       % compact symbols for 1/2, etc.
\usepackage{microtype}      % microtypography
\usepackage{xcolor}         % colors

\usepackage{indentfirst}

\title{Non-Monetary Mechanism Design without Distributional Information: Using Scarce Audits Wisely}

\author{%
    Yan Dai~\thanks{Massachusetts Institute of Technology. Email: \texttt{yandai20@mit.edu}.}\and Mo\"ise Blanchard~\thanks{Columbia University. Email: \texttt{mb5414@columbia.edu}.}\and Patrick Jaillet~\thanks{Massachusetts Institute of Technology. Email: \texttt{jaillet@mit.edu}.}
}
\date{}

\usepackage{amsmath}
\usepackage{amsthm}
\usepackage{amssymb}
\usepackage{mathtools}

\usepackage{cleveref} % ===== BE SURE TO PUT THIS BEFORE \newtheorem's
\newtheorem{theorem}{Theorem}
\newtheorem{lemma}[theorem]{Lemma}
\newtheorem{proposition}[theorem]{Proposition}
\newtheorem{assumption}{Assumption}
\newtheorem{corollary}[theorem]{Corollary}
\newtheorem{claim}[theorem]{Claim}
\newtheorem{definition}[theorem]{Definition}
\newtheorem{remark}[theorem]{Remark}

\newcommand{\mathsc}[1]{{\normalfont\textsc{#1}}}
\newcommand{\E}{\operatornamewithlimits{\mathbb{E}}}
\newcommand{\argmax}{\operatornamewithlimits{\mathrm{argmax}}}
\newcommand{\argmin}{\operatornamewithlimits{\mathrm{argmin}}}
\newcommand{\tr}{\operatorname{\mathrm{Tr}}}
\renewcommand{\O}{\operatorname{\mathcal O}}
\newcommand{\Otil}{\operatorname{\tilde{\mathcal O}}}
\renewcommand{\P}{\operatorname{\mathbb P}}
\newcommand{\D}{\operatorname{\mathbb{D}}}
\newcommand{\T}{\operatorname{\mathcal{T}}}
\newcommand{\trans}{\mathsf{T}}
\newcommand{\mS}{\mathcal{S}}
\newcommand{\mA}{\mathcal{A}}
\newcommand{\mF}{\mathcal{F}}
\newcommand{\mW}{\mathcal{W}}
\newcommand{\mT}{\mathcal{T}}
\newcommand{\mH}{\mathcal{H}}
\newcommand{\mE}{\mathcal{E}}
\newcommand{\mD}{\mathcal{D}}
\newcommand{\mB}{\mathcal{B}}
\newcommand{\mI}{\mathcal{I}}
\newcommand{\mV}{\mathcal{U}}
\newcommand{\Gap}{\mathsc{Gap}}
\newcommand{\Unif}{\mathrm{Unif}}
\newcommand{\poly}{\mathrm{poly}}
\newcommand{\Cov}{\mathrm{Cov}}

\renewcommand{\tilde}{\widetilde}
\renewcommand{\hat}{\widehat}
\renewcommand{\bar}{\overline}
\newcommand{\tappa}{\mathfrak t}

\newcommand{\truth}{\textbf{truth}}

\newcommand{\paren}[1]{\left( #1 \right)}
\newcommand{\sqb}[1]{\left[ #1 \right]}
\newcommand{\set}[1]{\left\{ #1 \right\}}
\newcommand{\floor}[1]{\left\lfloor #1 \right\rfloor}
\newcommand{\ceil}[1]{\left\lceil #1 \right\rceil}
\newcommand{\abs}[1]{\left|#1\right|}
\newcommand{\1}{\mathbbm{1}}
\newcommand{\comment}[1]{}

\usepackage{nicefrac}
\usepackage{multirow}
\usepackage{footnote}
\usepackage{enumitem}
\setitemize{leftmargin=*, nosep, topsep=2pt, itemsep=2pt}
\setenumerate{leftmargin=*, nosep, topsep=2pt, itemsep=2pt}
\usepackage{comment}
\usepackage{bbm}
\usepackage{bm}

\allowdisplaybreaks

\usepackage{tikz,graphicx}
\usetikzlibrary{calc}
\usetikzlibrary{decorations.pathreplacing}
\usetikzlibrary{patterns}

%%%%% BE SURE TO REMOVE algorithm2e %%%%%
\newcommand{\theHalgorithm}{\arabic{algorithm}}
\newcommand{\alglinelabel}{%
\addtocounter{ALG@line}{-1}% Reduce line counter by 1
\refstepcounter{ALG@line}% Increment line counter with reference capability
\label% Regular \label
}
\usepackage{algorithm}
\usepackage[noEnd,commentColor=blue!70!white]{algpseudocodex}
\renewcommand{\algorithmicrequire}{\textbf{Input:}}
\renewcommand{\algorithmicensure}{\textbf{Output:}}
\Crefname{ALG@line}{Line}{Lines}
%%%%% END OF algpseudocodex %%%%%

\Crefname{assumption}{Assumption}{Assumptions}
\Crefformat{equation}{Eq. #2(#1)#3}
\Crefrangeformat{equation}{Eqs. #3(#1)#4 to #5(#2)#6}
\Crefmultiformat{equation}{Eqs. #2(#1)#3}{ and #2(#1)#3}{, #2(#1)#3}{ and #2(#1)#3}
\Crefrangemultiformat{equation}{Eqs. #3(#1)#4 to #5(#2)#6}{ and #3(#1)#4 to #5(#2)#6}{, #3(#1)#4 to #5(#2)#6}{ and #3(#1)#4 to #5(#2)#6}

\Crefname{enumi}{Restriction}{Restrictions}
\Crefname{footnote}{Footnote}{Footnotes}

\newcommand\scalemath[2]{\scalebox{#1}{\mbox{\ensuremath{\displaystyle #2}}}}
\let\originalmiddle=\middle
\def\middle#1{\mathrel{}\originalmiddle#1\mathrel{}}

\usepackage{xspace}
\newcommand{\mechname}{\texttt{AdaAudit}\xspace}

\renewcommand{\paragraph}[1]{\vspace{2pt}\noindent\textbf{#1}}

\begin{document}
\maketitle


\begin{abstract}
We study a repeated resource allocation problem with strategic agents where monetary transfers are disallowed and the central planner has no prior information on agents' utility distributions. In light of Arrow's impossibility theorem, acquiring information about agent preferences through some form of feedback is necessary. We assume that the central planner can request powerful but expensive audits on the winner in any round, revealing the true utility of the winner in that round. We design a mechanism achieving $T$-independent $\mathcal O(K^2)$ regret in social welfare while requesting $\mathcal O(K^3 \log T)$ audits in expectation, where $K$ is the number of agents and $T$ is the number of rounds. We also show an $\Omega(K)$ lower bound on the regret and an $\Omega(1)$ lower bound on the number of audits when having low regret.
Algorithmically, we show that incentive-compatibility can be mostly enforced with an accurate estimation of the winning probability of each agent under truthful reporting. To do so, we impose future punishments and introduce a \emph{flagging} component, allowing agents to flag any biased estimate (we show that doing so aligns with individual incentives). On the technical side, without monetary transfers and distributional information, the central planner cannot ensure that truthful reporting is exactly an equilibrium. Instead, we characterize the equilibrium via a reduction to a simpler \emph{auxiliary game}, in which agents cannot strategize until late in the $T$ rounds of the allocation problem. The tools developed therein may be of independent interest for other mechanism design problems in which the revelation principle cannot be readily applied. 
\end{abstract}


\section{Introduction}
\label{sec:introduction}
The business processes of organizations are experiencing ever-increasing complexity due to the large amount of data, high number of users, and high-tech devices involved \cite{martin2021pmopportunitieschallenges, beerepoot2023biggestbpmproblems}. This complexity may cause business processes to deviate from normal control flow due to unforeseen and disruptive anomalies \cite{adams2023proceddsriftdetection}. These control-flow anomalies manifest as unknown, skipped, and wrongly-ordered activities in the traces of event logs monitored from the execution of business processes \cite{ko2023adsystematicreview}. For the sake of clarity, let us consider an illustrative example of such anomalies. Figure \ref{FP_ANOMALIES} shows a so-called event log footprint, which captures the control flow relations of four activities of a hypothetical event log. In particular, this footprint captures the control-flow relations between activities \texttt{a}, \texttt{b}, \texttt{c} and \texttt{d}. These are the causal ($\rightarrow$) relation, concurrent ($\parallel$) relation, and other ($\#$) relations such as exclusivity or non-local dependency \cite{aalst2022pmhandbook}. In addition, on the right are six traces, of which five exhibit skipped, wrongly-ordered and unknown control-flow anomalies. For example, $\langle$\texttt{a b d}$\rangle$ has a skipped activity, which is \texttt{c}. Because of this skipped activity, the control-flow relation \texttt{b}$\,\#\,$\texttt{d} is violated, since \texttt{d} directly follows \texttt{b} in the anomalous trace.
\begin{figure}[!t]
\centering
\includegraphics[width=0.9\columnwidth]{images/FP_ANOMALIES.png}
\caption{An example event log footprint with six traces, of which five exhibit control-flow anomalies.}
\label{FP_ANOMALIES}
\end{figure}

\subsection{Control-flow anomaly detection}
Control-flow anomaly detection techniques aim to characterize the normal control flow from event logs and verify whether these deviations occur in new event logs \cite{ko2023adsystematicreview}. To develop control-flow anomaly detection techniques, \revision{process mining} has seen widespread adoption owing to process discovery and \revision{conformance checking}. On the one hand, process discovery is a set of algorithms that encode control-flow relations as a set of model elements and constraints according to a given modeling formalism \cite{aalst2022pmhandbook}; hereafter, we refer to the Petri net, a widespread modeling formalism. On the other hand, \revision{conformance checking} is an explainable set of algorithms that allows linking any deviations with the reference Petri net and providing the fitness measure, namely a measure of how much the Petri net fits the new event log \cite{aalst2022pmhandbook}. Many control-flow anomaly detection techniques based on \revision{conformance checking} (hereafter, \revision{conformance checking}-based techniques) use the fitness measure to determine whether an event log is anomalous \cite{bezerra2009pmad, bezerra2013adlogspais, myers2018icsadpm, pecchia2020applicationfailuresanalysispm}. 

The scientific literature also includes many \revision{conformance checking}-independent techniques for control-flow anomaly detection that combine specific types of trace encodings with machine/deep learning \cite{ko2023adsystematicreview, tavares2023pmtraceencoding}. Whereas these techniques are very effective, their explainability is challenging due to both the type of trace encoding employed and the machine/deep learning model used \cite{rawal2022trustworthyaiadvances,li2023explainablead}. Hence, in the following, we focus on the shortcomings of \revision{conformance checking}-based techniques to investigate whether it is possible to support the development of competitive control-flow anomaly detection techniques while maintaining the explainable nature of \revision{conformance checking}.
\begin{figure}[!t]
\centering
\includegraphics[width=\columnwidth]{images/HIGH_LEVEL_VIEW.png}
\caption{A high-level view of the proposed framework for combining \revision{process mining}-based feature extraction with dimensionality reduction for control-flow anomaly detection.}
\label{HIGH_LEVEL_VIEW}
\end{figure}

\subsection{Shortcomings of \revision{conformance checking}-based techniques}
Unfortunately, the detection effectiveness of \revision{conformance checking}-based techniques is affected by noisy data and low-quality Petri nets, which may be due to human errors in the modeling process or representational bias of process discovery algorithms \cite{bezerra2013adlogspais, pecchia2020applicationfailuresanalysispm, aalst2016pm}. Specifically, on the one hand, noisy data may introduce infrequent and deceptive control-flow relations that may result in inconsistent fitness measures, whereas, on the other hand, checking event logs against a low-quality Petri net could lead to an unreliable distribution of fitness measures. Nonetheless, such Petri nets can still be used as references to obtain insightful information for \revision{process mining}-based feature extraction, supporting the development of competitive and explainable \revision{conformance checking}-based techniques for control-flow anomaly detection despite the problems above. For example, a few works outline that token-based \revision{conformance checking} can be used for \revision{process mining}-based feature extraction to build tabular data and develop effective \revision{conformance checking}-based techniques for control-flow anomaly detection \cite{singh2022lapmsh, debenedictis2023dtadiiot}. However, to the best of our knowledge, the scientific literature lacks a structured proposal for \revision{process mining}-based feature extraction using the state-of-the-art \revision{conformance checking} variant, namely alignment-based \revision{conformance checking}.

\subsection{Contributions}
We propose a novel \revision{process mining}-based feature extraction approach with alignment-based \revision{conformance checking}. This variant aligns the deviating control flow with a reference Petri net; the resulting alignment can be inspected to extract additional statistics such as the number of times a given activity caused mismatches \cite{aalst2022pmhandbook}. We integrate this approach into a flexible and explainable framework for developing techniques for control-flow anomaly detection. The framework combines \revision{process mining}-based feature extraction and dimensionality reduction to handle high-dimensional feature sets, achieve detection effectiveness, and support explainability. Notably, in addition to our proposed \revision{process mining}-based feature extraction approach, the framework allows employing other approaches, enabling a fair comparison of multiple \revision{conformance checking}-based and \revision{conformance checking}-independent techniques for control-flow anomaly detection. Figure \ref{HIGH_LEVEL_VIEW} shows a high-level view of the framework. Business processes are monitored, and event logs obtained from the database of information systems. Subsequently, \revision{process mining}-based feature extraction is applied to these event logs and tabular data input to dimensionality reduction to identify control-flow anomalies. We apply several \revision{conformance checking}-based and \revision{conformance checking}-independent framework techniques to publicly available datasets, simulated data of a case study from railways, and real-world data of a case study from healthcare. We show that the framework techniques implementing our approach outperform the baseline \revision{conformance checking}-based techniques while maintaining the explainable nature of \revision{conformance checking}.

In summary, the contributions of this paper are as follows.
\begin{itemize}
    \item{
        A novel \revision{process mining}-based feature extraction approach to support the development of competitive and explainable \revision{conformance checking}-based techniques for control-flow anomaly detection.
    }
    \item{
        A flexible and explainable framework for developing techniques for control-flow anomaly detection using \revision{process mining}-based feature extraction and dimensionality reduction.
    }
    \item{
        Application to synthetic and real-world datasets of several \revision{conformance checking}-based and \revision{conformance checking}-independent framework techniques, evaluating their detection effectiveness and explainability.
    }
\end{itemize}

The rest of the paper is organized as follows.
\begin{itemize}
    \item Section \ref{sec:related_work} reviews the existing techniques for control-flow anomaly detection, categorizing them into \revision{conformance checking}-based and \revision{conformance checking}-independent techniques.
    \item Section \ref{sec:abccfe} provides the preliminaries of \revision{process mining} to establish the notation used throughout the paper, and delves into the details of the proposed \revision{process mining}-based feature extraction approach with alignment-based \revision{conformance checking}.
    \item Section \ref{sec:framework} describes the framework for developing \revision{conformance checking}-based and \revision{conformance checking}-independent techniques for control-flow anomaly detection that combine \revision{process mining}-based feature extraction and dimensionality reduction.
    \item Section \ref{sec:evaluation} presents the experiments conducted with multiple framework and baseline techniques using data from publicly available datasets and case studies.
    \item Section \ref{sec:conclusions} draws the conclusions and presents future work.
\end{itemize}
\section{Technical Overview}
\label{sec:TechnicalOverview}

In this section, we sketch the main ideas behind our proofs.
\paragraph*{Completeness of $k$-SUM for $\mathsf{FOP_{\mathbb{Z}}}(\exists^k)$}
With the right ingredients, proving that $k$-SUM is complete for $\FOPZ$ formulas with $k$ existential quantifiers (Theorem~\ref{existential_complete}) is possible via a simple approach: We observe that any $\FOPZ(\exists^k)$ formula $\phi$ can be rewritten such that we may assume that $\phi$  is a conjunction of $m$ inequalities. We then use a slight generalization of a bit-level trick of~\cite{DBLP:journals/siamcomp/WilliamsW13} to reduce each inequality to an equality, incurring only $O(\log n)$ overhead per inequality (intuitively, we need to guess the most significant bit position at which the left-hand side and the right-hand side differ).
Thus, we obtain $O(\log^m n)$ conjunctions of $m$ equalities; each such conjunction can be regarded as an instance of Vector $k$-SUM. Using a straightforward approach for reducing Vector $k$-SUM to $k$-SUM given in~\cite{DBLP:journals/corr/AbboudLW13}, the reduction to $k$-SUM follows. We give all details in Section~\ref{sec:existentialpa} and the full version of the paper.

 

\paragraph*{Counting witnesses and handling multisets}
While the reduction underlying Theorem \ref{existential_complete} preserves the existence of solutions, it fails 
to preserve the number of solutions. The challenge is that when applying the bit-level trick to reduce inequalities to equalities, we need to make sure that for each witness of a $\FOPZ(\exists^k)$ formula $\phi$, there is a unique witness in the $k$-SUM instances produced by the reduction. While it is straightforward to ensure that we do not produce multiple witnesses, the subtle issue arises that distinct witnesses for $\phi$ may be mapped to the same witness in the $k$-SUM instances. It turns out that it suffices to solve a \emph{multiset} version of \#$k$-SUM, i.e., to count all witnesses in a $k$-SUM instance in which each input number may occur multiple times. 

Thus, to obtain Theorem~\ref{thm:counting-witnesses}, we show a fine-grained equivalence of Multiset \#$k$-SUM and \#$k$-SUM, for all odd $k\ge 3$. This fine-grained equivalence, which we prove via a heavy-light approach, might be of independent interest.\footnote{We remark that it is plausible that the proof of the subquadratic equivalence of $3$-SUM and \#$3$-SUM due to Chan et al.~\cite{DBLP:journals/corr/abs-2303-14572} could be extended to establish subquadratic equivalence with Multiset \#$3$-SUM as well. Note, however, that a fine-grained equivalence of \#$k$-SUM and $k$-SUM is not known for any $k\ge 4$.}
Combining this equivalence with an inclusion-exclusion argument, we may thus lift Theorem~\ref{existential_complete} to a counting version for all odd $k \ge 3$.

In the reductions below, we will make crucial use of the immediate corollary of Theorem~\ref{thm:counting-witnesses} and~\cite{DBLP:journals/corr/abs-2303-14572} that for each $\FOPZ(\exists \exists \exists)$ formula $\phi$, there exists a subquadratic reduction from counting witnesses for $\phi$ to $3$-SUM (Corollary~\ref{Count3COMP}).




\paragraph*{On general quantifier structures}
We perform a systematic study on the different quantifier structures for $k=3$. 
Due to simple negation arguments, we only have to perform a systematic study on the classes of problems
$\mathsf{FOP}_{\mathbb{Z}}(\exists \exists \exists)$,
$\mathsf{FOP}_{\mathbb{Z}}(\forall \exists \exists)$,
$\mathsf{FOP}_{\mathbb{Z}}(\forall \forall \exists)$,
$\mathsf{FOP}_{\mathbb{Z}}(\exists \forall \exists)$.

First, we state a simple lemma establishing syntactic complete problems for the classes above.
\begin{lemma}[Syntactic Complete problems (Informal Version)]
Let $Q_1,Q_2 \in \{\exists,\forall\}$. We can reduce every formula of the class $\mathsf{FOP}_{\mathbb{Z}}(Q_1Q_2\exists)$ to the formula  
$$Q_1 \Tilde{a_1} \in \Tilde{A_1} Q_2 \Tilde{a_2}  \in \Tilde{A_2} \exists \Tilde{a_3}  \in \Tilde{A_3}: \Tilde{a_1} +\Tilde{a_2}  \leq \Tilde{a_3}.  $$
\end{lemma}

\paragraph*{On the quantifier change $\mathsf{FOP}_{\mathbb{Z}}(\forall \exists \exists) \to \mathsf{FOP}_{\mathbb{Z}}(\exists \exists \exists) $.}
We rely on the subquadratic equivalence between $3$-SUM and a functional version of $3$-SUM called All-ints $3$-SUM, which asks to determine 
for every $a \in A$  whether there is a solution involving $a$. A randomized subquadratic equivalence was given in~\cite{DBLP:conf/focs/WilliamsW10}, which can be turned deterministic~\cite{DBLP:conf/icalp/LincolnWWW16}.

This equivalence allows us to use the bit-level trick to turn inequalities to equalities, despite it seemingly not interacting well with the quantifier structure $\forall \exists \exists$ at first sight.
This results in a proof of the following hardness result.
\begin{restatable}{lemma}{allintshard}
	If $3$-SUM can be solved in time $O(n^{2-\epsilon})$ for an $\epsilon>0$,
	then all problems $P$ of $\mathsf{FOP}_{\mathbb{Z}}(\forall \exists \exists)$
	can be solved in time $O(n^{2-\epsilon_{P}})$ for an $\epsilon_{P}>0$. 
	\label{fopaee}
	\end{restatable}

\paragraph*{On the quantifier change $\mathsf{FOP}_{\mathbb{Z}}(\exists \exists \exists) \to \mathsf{FOP}_{\mathbb{Z}}(\forall \forall \exists) $.}
As a first result for the class $\mathsf{FOP}_{\mathbb{Z}}(\forall \forall \exists)$, 
we are able to show equivalence to $3$-SUM for a specific problem in this class,
thus introducing a $3$-SUM equivalent problem
with a different quantifier structure in comparison to $3$-SUM.
Specifically, we consider the problem of verifying additive $t$-approximation of sumsets.
We are able to precisely characterize the fine-grained complexity depending on $t$.

Formally, we show the following theorem.
\begin{restatable}{theorem}{sumsetapproxchar}
	Consider the Additive Sumset Approximation problem of deciding, given $A,B,C\subseteq \mathbb{Z}, t\in \mathbb{Z}$, whether
	$$A+B \subseteq C +\{0,\dots,t\}.$$
	This problem is
	\begin{itemize}
	\item solvable in time $O(n^{2-\delta})$ with $\delta>0$, whenever $t=O(n^{1-\epsilon})$ for any $\epsilon>0,$
	\item not solvable in time $O(n^{2-\epsilon})$, whenever $t =\Omega(n)$ assuming the Strong $3$-SUM hypothesis.
	\end{itemize}
	  Furthermore, subquadratic hardness holds under the standard 3-SUM Hypothesis if no restriction on $t$ is made.
	  \label{sumsetapproxTHM}
	\end{restatable}

The above theorem is essentially enabling a quantifier change transforming the $\exists \exists \exists$ quantifier structure for which $3$-SUM is complete 
into a subquadratic equivalent problem with a quantifier structure $\forall \forall \exists$.
Moreover, the $3$-SUM hardness is a witness to the hardness of the class $\mathsf{FOP}_{\mathbb{Z}}(\forall \forall \exists)$.

Let us remark a few interesting aspects: The algorithmic part follows from sparse convolution techniques
going back to Cole and Hariharan~\cite{DBLP:conf/stoc/ColeH02}, see~\cite{DBLP:journals/corr/abs-2107-07625} for a recent account and also \cite{DBLP:conf/stoc/ChanL15,DBLP:conf/icalp/BringmannN21,DBLP:conf/stoc/BringmannFN21}. Specifically, whenever $t = O(n^{1-\epsilon})$, it holds that $|C + \{0, \dots, t\}| = O(n^{2-\epsilon})$ and intuitively,
we can use an output-sensitive convolution algorithm to compute $A+B$ and compare it to $C+\{0, \dots, t\}$.\footnote{The argument is slightly more subtle, since we need to avoid computing $A+B$ if its size exceeds $O(n^{2-\epsilon})$.}
Our result indicates that an explicit construction of $C+\{0, \dots, t\}$ is required, since once it may get as large as $\Omega(n^2)$, we obtain a $n^{2-o(1)}$-time lower
bound assuming the Strong $3$-SUM Hypothesis.  

The lower bound follows from describing the $3$-SUM problem alternatively as $(A+B) \cap C \neq \emptyset$, which is equivalent to the negation of $(A+B)\subseteq \Bar{C}$, where $\Bar{C}$ denotes the complement of $C$.
Thus, we aim to cover the complement of $C$ by intervals of length $t$. While this appears impossible for $3$-SUM, we employ the subquadratic equivalence of $3$-SUM and its convolutional version due to Patrascu \cite{DBLP:conf/stoc/Patrascu10}. This problem will deliver us the necessary structure to represent this complement with the addition of few auxilliary points.


The reverse reduction from Additive Sumset Approximation to $3$-SUM follows from Theorem~\ref{three-sum-Completeness-all-quantifer} (as Additive Sumset Approximation has inequality dimension $2$).

\paragraph*{On completeness results for $\mathsf{FOP}_{\mathbb{Z}}^k$}
The above ingredients establish our completeness theorems by exhaustive search over remaining quantifiers. Specifically, by a combination of Theorem~\ref{sumsetapproxTHM}, which shows that Additive Sumset Approximation is $3$-SUM hard, and a combination of Lemma~\ref{fopaee} and Theorem~\ref{existential_complete},
we get:
\begin{restatable}[]{lemma}{verifhard}
There is a function $\epsilon(d)>0$ such that 
the Verification of Pareto Sum problem can be solved in time $O(n^{2-\epsilon(d)})$
if and only if all problems $P$ in the classes 
\begin{itemize}
\item $\mathsf{FOP}_{\mathbb{Z}}(Q_1\dots Q_{k-3}\exists \exists \exists)$,$\mathsf{FOP}_{\mathbb{Z}}(Q_1\dots Q_{k-3}\forall \forall \forall),$
\item $\mathsf{FOP}_{\mathbb{Z}}(Q_1\dots Q_{k-3}\forall \exists \exists)$,$\mathsf{FOP}_{\mathbb{Z}}(Q_1\dots Q_{k-3}\exists \forall \forall),$
\item $\mathsf{FOP}_{\mathbb{Z}}(Q_1\dots Q_{k-3}\forall \forall \exists)$,$\mathsf{FOP}_{\mathbb{Z}}(Q_1\dots Q_{k-3}\exists \exists \forall),$
\end{itemize}
where $Q_1, \dots Q_{k-3} \in \{ \exists, \forall \}$ and $k\geq 3$,
can be solved in time $O(n^{k-1-\epsilon_P})$ for an $\epsilon_P>0$.
\label{verif-complete-three}
\end{restatable}

Similarly, for quantifier structures ending in $\exists \forall \exists$ and $\forall \exists \forall$, we obtain the following completeness result.

\begin{restatable}[]{lemma}{hunthard}
	There is a function $\epsilon(d)>0$ such that 
	the Hausdorff Distance under $n$ Translations problem can be solved in time $O(n^{2-\epsilon(d)})$
	if and only if all problems $P$ in the classes
	\begin{itemize}
    \item $\mathsf{FOP}_{\mathbb{Z}}(Q_1\dots Q_{k-3}\exists \forall \exists), \mathsf{FOP}_{\mathbb{Z}}(Q_1\dots Q_{k-3}\forall \exists \forall),$
	\end{itemize}
	where $Q_1, \dots Q_{k-3} \in \{ \exists, \forall \}$ and $k\geq 3$,
	can be solved in time $O(n^{k-1-\epsilon_P})$ for an $\epsilon_P>0$.
	\label{hunthard}
	\end{restatable}

The combination of Lemma \ref{verif-complete-three} and Lemma \ref{hunthard}, thus suffice to 
prove Theorem \ref{completenesswholeFOP3}.
\paragraph*{The $3$-SUM completeness of formulas with inequality dimension at most $3$}

As a first idea, one could try to solve problems of different quantifier structures
by just counting witnesses. Consider in the following the example $\FOPZ(\forall \forall \exists)$. 

Assume we are promised that the formula $\forall a\in A \forall b\in B \exists c\in C \psi(a,b,c)$ 
satisfies a kind of \emph{disjointness} property, specifically that for every $(a,b) \in A \times B$ there exists at most one $c \in C$ such that $\psi(a,b,c)$. Then satisfying the formula boils down to checking whether the number of witnesses $(a,b,c)$ satisfiying $\psi(a,b,c)$ equals to $|A|\cdot |B|$.

To create this \emph{disjointness} effect, we use the following geometric approach: 
We show that one can re-interpret the formula as $\forall a\in A \forall b\in B: a+b\in \bigcup_{c'\in C'} V(c')$, where $A,B,C' \subseteq \mathbb{Z}^3$, $C'$ is a set of size $O(n)$ and $V(c')$ is an orthant associated to $c'$. Using an adapted variant of~\cite{DBLP:journals/dcg/ChewDEK99}, we decompose this union of orthants in $\mathbb{R}^3$ (which we may equivalently view as sufficiently large congruent cubes) into a set $\mathcal{R}$ of $O(n)$ \emph{disjoint} boxes. Thus, it remains to notice that the resulting problem  -- i.e., for all $a\in A, b\in B$ is there a box $R\in \mathcal{R}$ such that $a+b$ is contained in $R$ -- is a $\FOPZ(\forall \forall \exists)$ formula with the desired disjointness property, which can be handled as argued above.  
For the class $\mathsf{FOP}_{\mathbb{Z}}(\exists \forall \exists)$, we perform a slightly more involved argument.
The classes $\mathsf{FOP}_{\mathbb{Z}}(\exists \exists \exists)$ and $ \mathsf{FOP}_{\mathbb{Z}}(\forall \exists \exists)$
reduce to $3$-SUM regardless of the inequality dimension due to Theorem \ref{existential_complete} and Lemma \ref{fopaee}.












\newcommand{\tabincell}[2]{\begin{tabular}{@{}#1@{}}#2\end{tabular}}
\newcommand{\rowstyle}[1]{\gdef\currentrowstyle{#1}%
	#1\ignorespaces
}

\newcommand{\className}[1]{\textbf{\sf #1}}
\newcommand{\functionName}[1]{\textbf{\sf #1}}
\newcommand{\objectName}[1]{\textbf{\sf #1}}
\newcommand{\annotation}[1]{\textbf{#1}}
\newcommand{\todo}[1]{\textcolor{blue}{\textbf{[[TODO: #1]]}}}
\newcommand{\change}[1]{\textcolor{blue}{#1}}
\newcommand{\fetch}[1]{\textbf{\em #1}}
\newcommand{\phead}[1]{\vspace{1mm} \noindent {\bf #1}}
\newcommand{\wei}[1]{\textcolor{blue}{{\it [Wei says: #1]}}}
\newcommand{\peter}[1]{\textcolor{red}{{\it [Peter says: #1]}}}
\newcommand{\zhenhao}[1]{\textcolor{dkblue}{{\it [Zhenhao says: #1]}}}
\newcommand{\feng}[1]{\textcolor{magenta}{{\it [Feng says: #1]}}}
\newcommand{\jinqiu}[1]{\textcolor{red}{{\it [Jinqiu says: #1]}}}
\newcommand{\shouvick}[1]{\textcolor{violet(ryb)}{{\it [Shouvick says: #1]}}}
\newcommand{\pattern}[1]{\emph{#1}}
%\newcommand{\tool}{{{DectGUILag}}\xspace}
\newcommand{\tool}{{{GUIWatcher}}\xspace}


\newcommand{\guo}[1]{\textcolor{yellow}{{\it [Linqiang says: #1]}}}

\newcommand{\rqbox}[1]{\begin{tcolorbox}[left=4pt,right=4pt,top=4pt,bottom=4pt,colback=gray!5,colframe=gray!40!black,before skip=2pt,after skip=2pt]#1\end{tcolorbox}}

% Please add the following required packages to your document preamble:

% Beamer presentation requires \usepackage{colortbl} instead of \usepackage[table,xcdraw]{xcolor}
\begin{table*}[t]
\centering
\caption{Main Results. Eurus-2-7B-PRIME demonstrates the best reasoning ability.}
\label{tab:main_results}
\resizebox{\textwidth}{!}{
\begin{tabular}{lcccccc}
\toprule
\textbf{Model}                     & \textbf{AIME 2024}                           & \textbf{MATH-500} & \textbf{AMC}          & \textbf{Minerva Math} & \textbf{OlympiadBench} & \textbf{Avg.}          \\ \midrule
\textbf{GPT-4o}                    & 9.3                                          & 76.4              & 45.8                  & 36.8                  & \textbf{43.3}          & 43.3                   \\
\textbf{Llama-3.1-70B-Instruct}    & 16.7                                         & 64.6              & 30.1                  & 35.3                  & 31.9                   & 35.7                   \\
\textbf{Qwen-2.5-Math-7B-Instruct} & 13.3                                         & \textbf{79.8}     & 50.6                  & 34.6                  & 40.7                   & 43.8                   \\
\textbf{Eurus-2-7B-SFT}            & 3.3                                          & 65.1              & 30.1                  & 32.7                  & 29.8                   & 32.2                   \\
\textbf{Eurus-2-7B-PRIME}          & \textbf{26.7 {\color[HTML]{009901} (+23.3)}} & 79.2 {\color[HTML]{009901}(+14.1)}      & \textbf{57.8 {\color[HTML]{009901}(+27.7)}} & \textbf{38.6 {\color[HTML]{009901}(+5.9)}}  & 42.1 {\color[HTML]{009901}(+12.3) }          & \textbf{48.9 {\color[HTML]{009901}(+ 16.7)}} \\ \bottomrule
\end{tabular}
}
\end{table*}
\subsection{Proof sketch}

\begin{figure*}
\resizebox{\linewidth}{!}{
\begin{tikzpicture}[
>={Stealth[scale=1.2]}, 
    semithick, 
    group/.style={align=center,rounded corners=3pt,inner sep=5pt,path picture={\fill[left color=black!3, right color=black!0] (path picture bounding box.south west) rectangle (path picture bounding box.north east);}},
    lemma/.style={rounded corners=3pt, align=center,inner sep=5 pt,path picture={\fill[left color=blue!8, right color=blue!2] (path picture bounding box.south west) rectangle (path picture bounding box.north east);}},
    def/.style={rounded corners=3pt, align=center,inner sep=5 pt,        path picture={\fill[left color=yellow!8, right color=yellow!2] (path picture bounding box.south west) rectangle (path picture bounding box.north east);}},
    arrow/.style={->}
    % path picture={\fill[left color=red!10, right color=red!3] (path picture bounding box.south west) rectangle (path picture bounding box.north east);}]
    % path picture={\fill[left color=red!10, right color=red!3] (path picture bounding box.south west) rectangle (path picture bounding box.north east);}]
]
% Main decomposition
\node[group] (total) at (0,.7) {\textbf{Total regret}\\\large$\displaystyle \E\left[\sum_{t=1}^T r(s_t^m, \pi^m(s_t^m)) - \sum_{t=1}^T r(s_t, a_t)\right]$};

\node[group] (state) at (-4.5,-1.8) {\textbf{State-based regret}\\\large$\displaystyle \E\left[\sum_{t=1}^T r(s_t^m, \pi^m(s_t^m)) - \sum_{t=1}^T r(s_t, \pi^m(s_t))\right]$};

\node[group] (action) at (4,-1.8) {\textbf{Action-based regret}\\\large$\displaystyle \E\left[\sum_{t=1}^T r(s_t, \pi^m(s_t)) - \sum_{t=1}^T r(s_t, a_t)\right]$};

% State-based analysis
    \node[lemma] (lemma54) at (-10.3,-4) {\emph{\Cref{lem:split}} \\$\E[f(s_t^m) - f(s_t)] \leq \Delta_t$};

\node[lemma] (lemma55) at (-3,-4) {\emph{\Cref{lem:trajectories-induction}}\\$\Delta_t \leq \sum_{i=1}^{t-1} \sup_{X \subseteq S} \alpha_i(X)$};

\node[def] (defn41_x) at (-2.6,-6) {\Cref{def:ac} with\\$\mu_t(s,a) = P(s,a,X_t)$\\ for various $X_t\subseteq \s$};

\node[lemma] (lemma56) at (-6.4,-6) {\emph{\Cref{lem:trajectories}}\\$\Delta_t \leq R_T^{AC}$};

% \node[rounded corners=3pt, align=center,inner sep=5 pt] (interm) at (-10.6,-7.1) {$\E[r(s_t^m, \pi^m(s_t^m)) - r(s_t,\pi^m(s_t))] \le \Delta_t$};

\node[lemma] (statefinal) at (-5, -8) {\emph{\Cref{lem:state-regret}}\\ State-based regret $\le T R_T^{AC}$};

% \node[lemma] (statedelta) at (-10.5,-6.1) {$E[r(s_t^m, \pi^m(s_t^m)) - r(s_t, \pi^m(s_t))] \le \Delta_t$};

% Action-based analysis
\node[def] (defn41) at (3.5,-5) {\Cref{def:ac} with\\ $\mu_t(s,a) = r(s,a)$};

\node[lemma] (lemma53) at (3.5,-8) {\emph{\Cref{lem:action-regret}}\\Action-based regret $\leq R_T^{AC}$};

% Final result
\node[group,path picture={\fill[left color=green!7, right color=green!2] (path picture bounding box.south west) rectangle (path picture bounding box.north east);}] (final) at (0,-9.5) {Total regret $\leq (T+1)R_T^{AC} \in o(T)$};

% Main decomposition arrows
\draw[arrow] (total) -- (state);
\draw[arrow] (total) -- (action);

% State-based lemma dependency arrows
\draw[arrow] (lemma54) -- (lemma55) node[above,midway] {$f(s) = N_t^m(s,X)$};
\draw[arrow] (lemma55) -- (lemma56);
\draw[arrow] (defn41_x) -- (lemma56);
\draw[arrow] (lemma56) to (statefinal);
% \draw[arrow] (lemma54) -- (statedelta) node[midway,left] {$f(s) = r(s,\pi^m(s))$};
\draw[arrow] (lemma54) to[bend right=20] node[midway,left=.2cm] {$f(s) = r(s,\pi^m(s))$} (statefinal);
% \draw[arrow] (lemma54) to[bend right=20] (statefinal);
% \draw[arrow] (statedelta) to[bend right=20] (statefinal);
\draw[arrow] (statefinal) -- (final);
\draw[-,draw=black!30] (0,-1) -- (0,-8.5);

% Action-based arrows
\draw[arrow] (defn41) -- (lemma53);
\draw[arrow] (lemma53) -- (final);

\end{tikzpicture}
}
\caption{We decompose regret into state-based and action-based components by adding and subtracting $\E\big[\sum_{t=1}^T r(s_t,\pi^m(s_t))\big]$. Below, we show the dependencies of lemmas and applications of \Cref{def:ac} in the proof, leading to the final bound of $R_T \le (T+1)\rac$.}
\label{fig:proof}
\end{figure*}

The full proof is deferred to Appendix~\ref{sec:main-proof}, but we provide a visualization (\Cref{fig:proof}) and proof sketch here. 

First, we bound the action-based regret by a direct application of \Cref{def:ac}.\footnote{The fact that bounding the action-based regret is so simple is more a property of \Cref{def:ac} than of the decomposition.}

\begin{restatable}{lemma}{lemActionRegret}
\label{lem:action-regret}
If an algorithm satisfies \Cref{def:ac}, then $\E \big[\sum_{t=1}^T r(s_t, \pi^m(s_t)) - \sum_{t=1}^Tr(s_t, a_t)\big] \le  \rac$.
\end{restatable}

\begin{proof}
Let $\mu_t(s,a) := r(s,a)$ for all $t \in [T]$. Since $r$ satisfies local generalization, so does $\bfmu$. Hence by  \Cref{def:ac}, $\E \big[\sum_{t=1}^T r(s_t, \pi^m(s_t)) - \sum_{t=1}^T r(s_t, a_t)\big] \le  \rac$.
\end{proof}

It remains to bound the state-based regret. Rather than analyzing when $\smols$ is better or worse than $\sm$, we simply bound how much their distributions differ at all. Specifically, we will show that $\sup_{X\subseteq \s}(p_t^m(X) - p_t(X)) \le \rac$ for all $t \in [T]$. Let $\Delta_t = \sup_{X\subseteq \s}(p_t^m(X) - p_t(X))$; we will refer to this quantity frequently. 
First, we show that an entire class of expected values can be bounded by $\Delta_t$.


\begin{restatable}{lemma}{lemSplit}
\label{lem:split}
For any $t \in [T]$ and any measurable function $f: \s \to [0,1]$,  we have $\E[f(s_t^m) - f(s_t)] \le \Delta_t$.
\end{restatable}

The idea behind \Cref{lem:split} is that $\E[f(s_t^m) - f(s_t)]$ can only be large if $s_t^m$ is more concentrated than $s_t$ in states where $f$ is large. We use $\Delta_t$ to measure how large that difference in concentration can be. Also, since $f(s) \in [0,1]$, $f$ cannot be too much larger in these states where $s_t^m$ is more concentrated than $s_t$. Although there are some technical details, conceptually the proof amounts to:
\begin{align*}
\E[f(s_t^m) - f(s_t)] \le \sup_{s,s' \in \s} |f(s) - f(s')|\cdot  \Delta_t
\le  \Delta_t
\end{align*}
The most obvious application of \Cref{lem:split} is with $f(s) = r(s, \pi^m(s))$, and indeed, that will be one usage. However, we will also use this lemma to analyze the divergence between $\smols$ and $\sm$. Specifically, we will write $p_{t+1}^m(X) - p_{t+1}(X) = \E[f(s_t^m) - f(s_t)] + \alpha_t(X)$ for some functions $f$ and $\alpha_t$. Then \Cref{lem:split} will imply that
\begin{align*}
p_{t+1}^m(X) - p_{t+1}(X) \le&\ \Delta_t + \alpha_t(X) \quad \text{$\forall X \subseteq \s$}\\
\sup_{X\subseteq \s} (p_{t+1}^m(X) - p_{t+1}(X)) \le&\ \Delta_t + \sup_{X\subseteq \s} \alpha_t(X)\\
\Delta_{t+1} \le&\ \Delta_t + \sup_{X\subseteq \s} \alpha_t(X)
\end{align*}
The agent and mentor have the same initial state, so $\Delta_1 = 0$ and thus $\Delta_t \le \sum_{i=1}^{t-1} \sup_{X\subseteq \s} \alpha_t(X)$ by induction.

To enact this plan, we choose $f(s) = N_t^m(s,X)$ and $\alpha_t(X) = \E[N_t^m(s_t, X) - N_t(s_t, X)]$. Recall from \Cref{sec:model} that $N_t^m(s, X) = \Pr[s_{t+1}^m \in X \mid s_t^m = s]$ and $N_t(s,X) = \Pr[s_{t+1} \in X \mid s_t = s]$. The law of total expectation implies that $p_{t+1}^m(X) = \E[N_t^m(s_t^m, X)]$ and $p_{t+1}(X) = \E[N_t(s_t, X)]$, so
\begin{align*}
& \ \ p_{t+1}^m(X) - p_{t+1}(X)\\
=&\ \E[N_t^m(s_t^m, X)] - \E[N_t(s_t, X)]\\
=&\ \resizebox{\linewidth}{!}{%
\(\E[N_t^m(s_t^m, X) - N_t^m(s_t, X)] + \E[N_t^m(s_t, X) - N_t(s_t, X)]\)
}
\\
=&\ \E[N_t^m(s_t^m, X) - N_t^m(s_t, X)] + \alpha_t(X)
\end{align*}
(Interestingly, this decomposition is similar in structure to the state- vs action-based regret decomposition, although the terms here do not depend on $r$ at all.) 

This decomposition allows us to apply the aforementioned inductive strategy to obtain the following lemma:

\begin{restatable}{lemma}{lemTrajInd}\label{lem:trajectories-induction}
For any $t \in [T]$, $\Delta_t \le \sum_{i=1}^{t-1}\sup_{X\subseteq \s}\alpha_i(X)$.
\end{restatable}

This brings us to the trickiest part of the proof: bounding $\sum_{i=1}^{t-1}\sup_{X\subseteq \s}\alpha_i(X)$. The main idea is to invoke \Cref{def:ac} with $\mu_i(s, a) = P(s,a,X_i)$ for each $i \in [t-1]$ for every possible choice of $X_1,\dots,X_{t-1} \subseteq \s$. (It is also helpful to define $\mu_i(s,a)$ to be constant for $i > t$.) 

One can use the definition of total variation distance to show that if $P$ satisfies local generalization, so does this definition of $\bfmu$. Next, we can manipulate conditional expectations to show that $\E[N_i(s_i,X_i)] = \E[P(s_i,a_i,X_i)] = \E[\mu_i(s_i,a_i)]$ and $\E[N_i^m(s_i, X_i)] = \E[P(s_i, \pi^m(s_i), X_i)] = \E[\mu_i^m(s_i)]$. Thus applying \Cref{def:ac} to \Cref{lem:trajectories-induction} gives us
\begin{align*}
\Delta_t \le \sum_{i=1}^{t-1}\alpha_i(X_i) 
= \sum_{i=1}^T \E[\mu_i^m(s_i) - \mu_i(s_i, a_i)]
\le R_T^{AC}
\end{align*}
The result is \Cref{lem:trajectories}:

\begin{restatable}{lemma}{lemTraj}
\label{lem:trajectories}
If an algorithm satisfies \Cref{def:ac}, then $\Delta_t \le \rac$ for all $t \in [T]$.
\end{restatable}

Now we can bound the state-based regret by applying \Cref{lem:split} with $f(s) = r(s, \pi^m(s))$ followed by \Cref{lem:trajectories}. Specifically, we obtain
\[
\E\big[r(s_t^m, \pi^m(s_t^m)) - r(s_t, \pi^m(s_t))\big] \le \Delta_t \le \rac
\]
Summing this over $t$ produces \Cref{lem:state-regret}:

\begin{restatable}{lemma}{lemStateRegret}
\label{lem:state-regret}
If an algorithm satisfies \Cref{def:ac}, then \resizebox{1.02\linewidth}{!}{%
$\E\big[\sum_{t=1}^T r(s_t^m, \pi^m(s_t^m)) - \sum_{t=1}^T r(s_t, \pi^m(s_t))\big] \le T \rac$
.}
\end{restatable}

Since the state-based regret is at most $T \rac$ (\Cref{lem:state-regret}) and the action-based regret is at most $\rac$ (\Cref{lem:action-regret}), we conclude that $R_T \le (T+1)\rac$.

\bibliography{references}

%%%%%%%%%%%%%%%%%%%%%%%%%%%%%%%%%%%%%%%%%%%%%%%%%%%%%%%%%%%%%%%%%%%%%%%%%%%%%%%
%%%%%%%%%%%%%%%%%%%%%%%%%%%%%%%%%%%%%%%%%%%%%%%%%%%%%%%%%%%%%%%%%%%%%%%%%%%%%%%
% APPENDIX
%%%%%%%%%%%%%%%%%%%%%%%%%%%%%%%%%%%%%%%%%%%%%%%%%%%%%%%%%%%%%%%%%%%%%%%%%%%%%%%
%%%%%%%%%%%%%%%%%%%%%%%%%%%%%%%%%%%%%%%%%%%%%%%%%%%%%%%%%%%%%%%%%%%%%%%%%%%%%%%
\onecolumn
\newpage
\crefalias{section}{appendix}
\crefalias{subsection}{appendix}
\appendix

\startcontents[section]
\printcontents[section]{l}{1}{\setcounter{tocdepth}{2}}

\section{Characterization of a PBE for \mechname (\Cref{thm:equivalence of PBE})}\label{sec:appendix varying-p}

Towards proving our main result \Cref{thm:deterministic main thm}, we first aim to characterize a PBE for agent joint strategies in the game induced by \mechname. Following the proof structure described in \Cref{sec:sketch varying-p} (Part I), in this section we prove \Cref{thm:equivalence of PBE}.

\subsection{Additional Notations}\label{sec:appendix additional notations varying-p}

For convenience, we define a few additional notations for agent strategies.

\begin{definition}[Additional Notations for Agents' Strategies]
For any agent $i\in [K]$, in analog to $\bm \pi_i=(\pi_{t,i})_{t\in [T]}$, notation $\bm \pi_{-i}=(\pi_{t,j})_{t\in [T],j\ne i}$ contains the strategies of all agents other than $i$.

Given $\bm \pi_i^1=(\pi_{t,i}^1)_{t\in [T]}$ and $\bm \pi_{-i}^2=(\pi_{t,j}^2)_{t\in [T],j\ne i}$, we use $\bm \pi_i^1 \circ \bm \pi_{-i}^2$ to denote the corresponding joint strategy obtained by concatenation, that is, for every round $t\in [T]$, agent $i$ follows $\pi_{t,i}^1$ while agent $j\ne i$ follows $\pi_{t,j}^2$.

Given joint strategies $\bm \pi^1$ and $\bm \pi^2$, we use $\bm \pi^1\diamond_t \bm \pi^2$ to denote the joint strategy of following $\bm \pi^1$ up to round $t$ and following $\bm \pi^2$ afterwards, i.e., $\bm \pi^1\diamond_t \bm \pi^2=(\pi_{\tau,i})_{\tau\in [T],i\in [K]}$ where $\pi_{\tau,i}=\pi_{\tau,i}^1$ for $\tau\leq t$ and $\pi_{\tau,i}=\pi_{\tau,i}^2$ for $\tau>t$.
\end{definition}

Next, we recall the definition of the fair allocation probability $q_{t,i}$ for a round $t\in[T]$ and an agent $i\in[K]$, which we complement with the fair share $\mu_{t,i}$. By definition, for any round $t\in [T]$ and agent $i\in [K]$, we define
\begin{align}
\mu_{t,i}&= \E_{\bm u_t\sim \mV}\bigg [u_{t,i}\mathbbm{1}[i\in \mA_t] \1[u_{t,i}\geq c]\mathbbm{1}[u_{t,i}>u_{t,j},\forall j\in \mA_t\setminus \{i\}]\bigg ] \nonumber\\
q_{t,i}&=\Pr_{\bm u_t\sim \mV}\bigg \{(i\in \mA_t)\wedge (u_{t,i}\geq c) \wedge (u_{t,i}>u_{t,j},\forall j\in \mA_t\setminus \{i\})\bigg \}.\label{eq:fair share}
\end{align}
Although we write them as $\mu_{t,i}$ and $q_{t,i}$, they in fact both depend on the specific realization of $\mA_t$ during the execution of \Cref{alg:mechanism}, rather than only depending on the round number $t$.

We recall that the mechanism \mechname is formally described in \Cref{alg:mechanism}. For any strategy $\bm \pi=(\bm \pi^r,\bm \pi^f)$ under the mechanism \mechname, we can also rewrite the V-function defined in \Cref{eq:V-func general} through its \textit{Bellman form}: For any eliminated agent $i\not \in \mA_t$, $V_i^{\bm \pi}(\mH_t)=0$. For any alive agent $i\in \mA_t$, its V-function under $\bm \pi$ starting from history $\mH_t$ is
\begin{align}
V_i^{\bm \pi}(\mH_t)=\E\left [\sum_{\tau=t}^T u_{\tau,i}\mathbbm{1}[i_{\tau}=i]\middle \vert \mH_t\right ]=\E\big [u_{t,i}\mathbbm{1}[v_{t,i}>v_{t,j},\forall j\in \mA_t\setminus \{i\}]+V_i^{\bm \pi}(\mH_{t+1})~\big \vert ~ \mH_t\big ],\label{eq:V-func recursive}
\end{align}
where the randomness lies in the generation of utilities $\bm u_t\sim \mV$, reports $\bm v_t\sim \bm \pi_t^r$, flags $\bm f_t\sim \bm \pi_t^f$, $(i_t,o_t)$, and the corresponding next-round history $\mH_{t+1}$ generated according to \Cref{alg:mechanism}.

\subsection{Formal Setup for the Auxiliary Game (\Cref{def:auxiliary game})}\label{sec:appendix auxiliary game}
We recall that the auxiliary game was defined in \Cref{def:auxiliary game} as follows. We also include a formal pseudo-code of this mechanism in \Cref{alg:auxiliary game}. 

We specialize our definition of agents' strategies in \Cref{def:agent_strat_mechanism} to the auxiliary game as follows, where we recall that we defined $p_{t,i}:= \frac{2K^2}{(T-t)q_{t,i} c} $ for $t\in[T]$ and $i\in[K]$ within \Cref{def:auxiliary game}.
\begin{definition}[Agents' Strategies in Auxiliary Game]
We call $\tilde \mH_t:= (t,\mA_t)$ the simplified history at the beginning of round $t$, and $\tilde H_t:=\{t\}\times 2^{[K]}$ the corresponding simplified history space.
A {report strategy} in the auxiliary game \Cref{def:auxiliary game} for agent $i\in[K]$ is a sequence of measurable functions $\bm\pi_i^r:=(\pi^r_{t,i})_{t\in[T]}$ where
\begin{equation*}
\pi^r_{t,i}:[0,1]\times \tilde H_t\times \Xi \to \begin{cases}
[0,u_{t,i}],&\quad \text{if }p_{t,i}\leq 1\\
[0,1],&\quad \text{if }p_{t,i}> 1
\end{cases}.
\end{equation*}
Their report in round $t\in[T]$ is $v_{t,i}:=\pi^r_{t,i}(u_{t,i},\mH_t^i,\xi^r_{t,i})$, where $\xi^r_{t,i}\sim\mD_\xi$ is sampled independently from other random variables. There is no answers in this game, i.e., $F=\{0\}$.
A {joint strategy} for agents is a collection of all agent's report strategies $\bm\pi^r:=(\bm \pi_{i}^r)_{i\in[K]}$.
\end{definition}

\begin{algorithm}[!t]
\caption{Auxiliary Game}
\label{alg:auxiliary game}
\begin{algorithmic}[1]
\Require{Number of rounds $T$, number of agents $K$}
\Ensure{Allocations $i_1,i_2,\ldots,i_T\in \{0\}\cup [K]$}
\State Initialize the set of alive agents $\mA_1=[K]$.
\For{$t=1,2,\ldots,T$}
\State Collect reports $v_{t,i}\in [0,1]$ from $i\in \mA_t$ and allocate to agent $i_t=\argmax_{i\in \mA} v_{t,i}$.
\If{$v_{t,i_t}>u_{t,i_t}$}
\State Eliminate agent $i_t$ permanently by updating $\mA_{t+1}\gets \mA_t\setminus \{i_t\}$.
\Else
\State Everyone stays alive, that is, set $\mA_{t+1}\gets \mA_t$.
\EndIf
\EndFor
\end{algorithmic}
\end{algorithm}

Throughout the proof, to highlight the difference between the auxiliary game (\Cref{def:auxiliary game}) and the original game, we write down the following three restrictions that we described in the main text:
\begin{enumerate}
\item \textbf{Reports only depend on $(t,\mA_t)$.} Agents are forced to decide their reports $v_{t,i}$ only depending on their own private utilities $u_{t,i}$, the current round number $t$, and the set of alive agents $\mA_t$. \label{item:only simplified history formal}
\item \textbf{No Mark Up Unless $p_{t,i}>1$.} When $p_{t,i}=\frac{2K^2}{(T-t)q_{t,i} c} \leq 1$ (\textit{cf.} $\hat p_{t,i}=\min\big (\frac{8K^2}{(T-t) \hat q_i c},1\big )$ defined in \Cref{alg:mechanism}), agents must have their reports no larger than actual utilities, \textit{i.e.}, $v_{t,i}\le u_{t,i}$. \label{item:no mark up unless p>=1 formal}
\item \textbf{Eliminate Immediately on Mark Up.} In case an agent $i$ wins by marking up, \textit{i.e.}, $i_t=i$ and their report $v_{t,i}>u_{t,i}$, they are eliminated for all future rounds. In comparison, in the original game, agent $i$ would be eliminated with probability $\hat p_{t,i}\le 1$ (see \Cref{line:randomly_check} of \Cref{alg:mechanism}).\label{item:eliminate immediately formal}
\end{enumerate}

In the auxiliary game, the V-function only depends on the simplified history $\tilde \mH_t$ since agents' and the mechanism' actions only depend on $\tilde \mH_t$. Therefore, for a joint strategy ${\bm \pi}^r$, the agents' V-functions under ${\bm \pi}^r$ starting from simplified history $\tilde \mH_t$ can be written as
\begin{equation*}
\tilde V_i^{{\bm \pi}^r}(\tilde \mH_t)=\E\left [\sum_{\tau=t}^T u_{t,i}\mathbbm{1}[i_t=i]\middle \vert \tilde \mH_t\right ],
\end{equation*}
where agents utilities $\bm u_t,\bm u_{t+1},\ldots,\bm u_T\overset{\text{i.i.d.}}{\sim}\mV$, reports $\bm v_t\sim {\bm \pi}^r_t,\bm v_{t+1}\sim {\bm \pi}^r_{t+1},\ldots,\bm v_T\sim {\bm \pi}^r_T$, and histories $\tilde \mH_{t+1}=(t+1,\mA_{t+1}),\ldots,\tilde \mH_T=(T,\mA_T)$ are computed according to \Cref{def:auxiliary game}.

\subsection{Proof of Correspondence between Auxiliary Game and Original Game (\Cref{lem:equiv between actual and no-flagging})}\label{sec:equiv between actual and no-flagging formal}
For the ease of presentation, we restate the following definition that appeared as \Cref{def:well-behaved flagging}:
\begin{definition}[Well-Behaved Answer Strategy]\label{def:well-behaved flagging formal}
For an agent $i\in [K]$, define ${\bm \pi}_i^{\ast,f}=(\pi_{t,i}^{\ast,f})_{t\in [T]}$ as the strategy of agent $i$ that flags $f_{t,i}=1$ if and only if either of the following happens:
\begin{itemize}
\item $i_t\neq i$ and $\hat q_{t+1,i_t}>4q_{t+1,i_t}$ (the $\hat q_{t+1,i_t}$ defined in \Cref{line:estimate winning prob} can be computed via agent $i$'s observable history $\mH_t^i$ and is thus measurable),\footnote{As flagging only happens when nobody is eliminated in this round, we do not need to distinguish between $q_{t+1,i_t}$ and $q_{t,i_t}$ here.} or
\item $i_t=i$ and $\hat q_{t+1,i_t}<q_{t+1,i_t}/4$.
\end{itemize}
The well-behaved answer strategy ${\bm \pi}^{\ast,f}=({\bm \pi}_i^{\ast,f})_{i\in [K]}$ is the joint strategy where every agent $i\in [K]$ follows ${\bm \pi}_i^{\ast,f}$.
\end{definition}

We now present and prove the formal version of \Cref{lem:equiv between actual and no-flagging}:
\begin{lemma}[Correspondence between Auxiliary Game and Actual Game]\label{lem:equiv between actual and no-flagging formal}
Let ${\bm \pi}^r$ be a joint strategy for the auxiliary game from \Cref{def:auxiliary game}.
Consider the joint strategy $\bm \pi=({\bm \pi}^r,{\bm \pi}^{\ast,f})$ in the actual game that uses ${\bm \pi}^r$ as the reporting policy and ${\bm \pi}^{\ast,f}$ (see \Cref{def:well-behaved flagging formal}) as the flagging policy.

For every history $\mH_t$ such that $\hat q_{t,j}\le 4q_{t,j}$ for all $j\in [K]$ (which is a natural consequence of following ${\bm \pi}^{\ast,f}$ in the past) and the corresponding simplified history $\tilde \mH_t=(t,\mA_t)$,
\begin{equation}\label{eq:equiv between actual and no-flagging}
V_i^{\bm \pi}(\mH_{t})=\tilde V_i^{{\bm \pi}^r}(\tilde \mH_t),\quad \forall i\in [K].
\end{equation}
Furthermore, the allocation sequences $\{i_{\tau}\}_{\tau=t}^T$ in the actual and auxiliary games are the same.\footnote{By ``the same'', we mean there exists a coupling between the realization of $\{i_{\tau}\}_{\tau=t}^T$ in the actual game under joint strategy $\bm \pi$ and starting from $\mH_t$, and that in the auxiliary game under ${\bm \pi}^r$ and starting from $\tilde \mH_t$.}
\end{lemma}
\begin{proof}
We prove by backward induction. Let $t\in[T]$ and fix a common history $\mH_t$. Suppose that the claim holds for all possible next-round histories $\{\mH_{t+1}\}$ resulted from $\bm \pi$.
As the reporting policies and allocation mechanism are identical for both games, the reports $\bm v_t$ and winning agent $i_t$ can be coupled to be the same for both games.
To finish the proof, it suffices to show that the next-round history $\mH_{t+1}$ generated by $\bm \pi$ in the original game (LHS) has the same simplified history as the next-round simplified history $\tilde\mH_{t+1}$ generated by ${\bm \pi}^r$ in the auxiliary game (RHS).

Since $\hat q_{t,j}\le 4q_{t,j}$ for all $j\in [K]$, we have ($\hat p_{t,j}$ is defined in \Cref{line:check prob} and $p_{t,j}$ is in \Cref{def:auxiliary game})
\begin{equation*}
\hat p_{t,j}=\min\left (\frac{8K^2}{(T-t) \hat q_{t,j} c},1\right )\ge \min\left (\frac{8K^2}{(T-t) 4 q_{t,j} c},1\right )=\min(p_{t,j},1),\quad \forall j\in [K].
\end{equation*}

We now discuss whether the winner $i_t$ (the same in both games) marked up, that is $v_{t,i_t}>u_{t,i_t}$.
\begin{itemize}
\item If so, they are eliminated immediately in the RHS according to \Cref{item:eliminate immediately}. Thus, the next-round simplified history generated by ${\bm \pi}^r$ in the auxiliary game is $\tilde \mH_{t+1}=(t+1,\mA_t\setminus \{i_t\})$.

For the LHS, $i_t$ marked up using $\pi^r_{t,i_t}$, \Cref{item:no mark up unless p>=1 formal} implies that $p_{t,i_t}=\frac{2K^2}{(T-t) q_{t,i_t} c}\ge 1$ and thus $\hat p_{t,i_t}\ge \min(p_{t,i_t},1)=1$. Therefore, $i_t$ is audited and eliminated with probability 1 in the original game as well (\Cref{line:elimination} of \Cref{alg:mechanism}). The next-round history in the actual game $\mH_{t+1}$ hence corresponds to the same simplified history $\tilde \mH_{t+1}$ \textit{a.s.}
\item Otherwise, $i_t$ does not mark up. Thus, the next-round simplified history in the auxiliary game is $\tilde\mH_{t+1}=(t+1,\mA_t)$. Meanwhile, regardless of whether they are audited in the original game, $i_t$ stays alive because $v_{t,i_t}\le u_{t,i_t}$. Thus the next-round auxiliary histories are again identical.
\end{itemize}

In summary, there is a coupling in which both games generate the same round-$t$ allocation $i_t$ and next-round simplified history $\tilde \mH_{t+1}$.
The final step is to argue that the new $\mH_{t+1}$ always ensures $\hat q_{t+1,j}\le 4q_{t+1,j}$, $\forall j\in [K]$ so that the induction hypothesis is applicable:
\begin{itemize}
\item If someone is eliminated, then $\hat q_{t+1,j}=0$, which is no more than $4q_{t+1,j}$ for all $j$.
\item If no new $\hat q_{t+1,i_t}$ is generated or the new $\hat q_{t+1,i_t}\le 4q_{t+1,i_t}$, then the property follows from the condition that $\hat q_{t,j}\le 4q_{t,j}$ for all $j\in [K]$.
\item If a new $\hat q_{t+1,i_t}$ is generated and $\hat q_{t+1,i_t}>4q_{t+1,i_t}$, then ${\bm \pi}_{t,-i_t}^{\ast,f}$ would flag them. In this case $\hat q_{t+1,i_t}$ is replaced with 0 and the property also holds.
\end{itemize}
Hence, our claim follows from the backward induction on $t$.
\end{proof}

\subsection{Proof of V-Function Lower and Upper Bounds (\Cref{lem:u lower and upper bound})}
\begin{lemma}[Upper and Lower Bounds when Following $\bm \pi^\ast$]\label{lem:u lower and upper bound formal}
Fix any round $t\in [T]$ and any history $\mH_t$ generated via following $\bm \pi^\ast$ in the past. The following inequalities hold:
\begin{align}
(T-t+1) \mu_{t,i}-K &\leq V_i^{{\bm \pi}^\ast}(\mH_t)\leq (T-t+1) \mu_{t,i}+K^2,&&\quad \forall t\in [T],\text{history }\mH_t,i\in \mA_t,\tag{\ref{eq:for alive}}\\
(T-t+1) \mu_{1,i}-K &\leq V_i^{{\bm \pi}^\ast}(\mH_t)=0,&&\quad \forall t\in [T],\text{history }\mH_t,i\notin \mA_t, \tag{\ref{eq:for dead}}
\end{align}
where the definitions of $\mu_{t,i}$ and $\mu_{1,i}$ can be found in \Cref{eq:fair share}.
\end{lemma}
\begin{proof}
As $\mH_t$ is generated via following $\bm \pi^\ast$ in the past, we must have $\hat q_{t,j}\le 4q_{t,j}$ for all $j\in [K]$.
Applying \Cref{lem:equiv between actual and no-flagging formal} to the joint strategy $\bm \pi^\ast=({\bm \pi}^{\ast,r},{\bm \pi}^{\ast,f})$ and history $\mH_t$, we have
\begin{equation*}
V_i^{\bm\pi^\ast}(\mH_t) = \tilde V_i^{{\bm \pi}^{\ast,r}}(\tilde \mH_t),
\end{equation*}
where $\tilde \mH_t=(t,\mA_t)$ is the simplified history corresponding to $\mH_t$.

\paragraph{Lower Bounds for Alive Agents.} Fix an alive agent $i\in\mA_t$. We start by proving a lower bound on the utility of agent $i$ in the auxiliary game (\Cref{def:auxiliary game}). Consider the truthful reporting strategy $\truth_i$ for agent $i$ that always reports $v_{\tau,i}=u_{\tau,i}$ for any round $\tau\in[T]$. Since ${\bm \pi}^{\ast,r}$ is a PBE in the auxiliary game and $\truth_i\circ {\bm \pi}^{\ast,r}_{-i}$ is a valid joint strategy therein, we have
\begin{equation}\label{eq:pbe_for_lower_bound}
V_i^{\bm\pi^\ast}(\mH_t) = \tilde V_i^{{\bm \pi}^{\ast,r}}(\tilde \mH_t)\ge \tilde V_i^{\truth_i\circ {\bm \pi}^{\ast,r}_{-i}}(\tilde \mH_{t}).
\end{equation}

We now lower bound the RHS where agent $i$ always reports truthfully. If in any round $\tau\in\{t,\ldots,T\}$ agent $i$ would have won had other agents reported truthfully (\textit{i.e.}, agent $i$ had the largest true utility $u_{\tau,i}=\max_{j\in \mA_{\tau}} u_{\tau,j}$ and $u_{\tau,i}\ge c$), either agent $i$ won so that $i_{\tau}=i$, or the winner $i_{\tau}$ marked up and was eliminated. Since each agent can only be eliminated once and agent $i$ is never eliminated, the latter case occurs for at most $K-1$ rounds. Making it formal, we get
\begin{align*}
    &\quad \tilde V_i^{\truth_i\circ {\bm \pi}^{\ast,r}_{-i}}(\tilde \mH_{t}) = \E_{\truth_i\circ {\bm \pi}^{\ast,r}_{-i}}\left [ \sum_{\tau=t}^T u_{\tau,i} \1[i_{\tau}=i] \middle \vert \tilde \mH_t\right ]\\
    &\geq \E_{\truth_i\circ {\bm \pi}^{\ast,r}_{-i}} \sqb{ \sum_{\tau=t}^T u_{\tau,i} \paren{\1\sqb{u_{\tau,i}=\max_{j\in \mA_{\tau}}u_{\tau,j}} \1[u_{\tau,i}\ge c]  - \1[i_t\neq i] \1\sqb{v_{t,i_t}>u_{t,i_t}}} \1\sqb{i\in \mA_{\tau}}}\\
    &\overset{(a)}{\geq} (T-t+1)\mu_{t,i} + \E_{\truth_i\circ {\bm \pi}^{\ast,r}_{-i}} \sqb{ \sum_{t=1}^T |\mA_t|-|\mA_{t+1}|} \geq (T-t+1)\mu_{t,i} - (K-1).
\end{align*}
where (a) uses the fact that agent $i$ is always alive when adopting $\truth_i$ and that if $\mA_{\tau}\subseteq \mA_t$ and $i\in \mA_{\tau}$, then $\mu_{\tau,i}\ge \mu_{t,i}$.
Together with \Cref{eq:pbe_for_lower_bound}, this proves $V_i^{\bm\pi^\ast}(\mH_t)\ge (T-t+1)\mu_{t,i}-K$ for all alive agents $i\in\mA_t$, showing the lower bound of \Cref{eq:for alive}.

\paragraph{Lower Bounds for Eliminated Agents.}
Before proving the lower bound, we make the following claim: For any round $s\in[T]$, simplified history $\tilde\mH_s$, and alive agent $i\in\mA_s$ such that $(T-s)\mu_{s,i} > K$, agent $i$ remains alive (that is, $i\in \mA_{s+1}$) with probability $1$ under the joint strategy ${\bm \pi}^{\ast,r}$.

To prove this claim, we construct an alternative policy $\pi_{s,i}$ that never results in an elimination, and utilize the fact that ${\bm \pi}^{\ast,r}$ is a PBE in the auxiliary game. Specifically, define alternative policy
\begin{equation*}
\pi_{s,i}(u_{s,i};\tilde \mH_s)=\min(u_{s,i},v_{s,i}),\quad \text{where }v_{s,i}\sim \tilde{\pi}_{s,i}^\ast (u_{s,i};\tilde \mH_s),
\end{equation*}
that is, capping the reports suggested by $\pi_{s,i}^{\ast,r}$ at the true utility $u_{s,i}$.

As ${\bm \pi}^{\ast,r}$ is a PBE, agent $i$'s unilateral deviation to $\pi_{s,i}$ in round $s$ does not increase its gain:
\begin{equation*}
\tilde V_i^{{\bm \pi}^{\ast,r}}(\tilde\mH_s)\ge \tilde V_i^{(\pi_{s,i}\circ {\bm \pi}_{s,-i}^{\ast,r})\diamond_s {\bm \pi}^{\ast,r}}(\tilde \mH_s),
\end{equation*}
where we recall the notation $\bm \pi_1 \diamond_t \bm \pi_2$ means following $\bm \pi_1$ up to round $t$ and adopting $\bm \pi_2$ after that.

The only case where the two strategies yield different outcomes is when \textit{1)} agent $i$ marks up under $\pi^r_{s,i}$, \textit{i.e.}, $v_{s,i}>u_{s,i}$, and \textit{2)} agent $i$ wins at time $s$ by reporting $v_{s,i}$ but not by reporting $u_{s,i}$. Denote by $\mE_{s,i}$ this event. Under $\mE_{s,i}$, if agent $i$ followed $\pi_{s,i}^{\ast,r}$ by reporting $v_{s,i}$, its total utility is $u_{s,i}\leq 1$ since they are automatically eliminated at time $s$ (\Cref{item:eliminate immediately formal}). On the other hand, if agent $i$ followed $\pi_{s,i}$, they would report $u_{s,i}$ and obtain utility
\begin{equation*}
\tilde V_i^{{\bm \pi}^{\ast,r}}(\tilde \mH_{s+1})\geq (T-s)\mu_{s+1,i}-(K-1)\ge (T-s)\mu_{s,i}-(K-1),
\end{equation*}
where the first inequality is because $i\in \mA_{s+1}$ and the lower bound \Cref{eq:for alive} for those alive agents, and the second inequality is because $\mA_{s+1}\subseteq \mA_s$ and hence $\mu_{s+1,i}\ge \mu_{s,i}$. In summary,
\begin{equation*}
0\geq \tilde V_i^{(\pi_{s,i}\circ {\bm \pi}_{s,-i}^{\ast,r})\diamond_s {\bm \pi}^{\ast,r}}(\tilde \mH_s) - \tilde V_i^{{\bm \pi}^{\ast,r}}(\tilde \mH_{s}) \geq ((T-s)\mu_{s,i}-K)\Pr \{\mE_{s,i}\},
\end{equation*}
where the first inequality uses the fact that ${\bm \pi}^{\ast,r}$ is a PBE in the auxiliary game. As a result, $\Pr\{\mE_{s,i}\}=0$ and hence almost surely agent $i$ does not die in round $s$.

We now prove \Cref{eq:for dead} using this claim. Since $\tilde\mH_t$ has non-zero probability in the auxiliary game, the claim indicates that for all eliminated agents $i\notin \mA_t$ we have $(T-t+1)\mu_{1,i}\leq K$ because $\mu_{1,i}\le \mu_{s,i}$ for all $s\in [T]$. Therefore, $V_i^{\bm \pi^\ast}(\mH_t) = \tilde V_i^{{\bm \pi}^{\ast,r}}(\tilde\mH_t) = 0\ge (T-t)\mu_{1,i}-K$.

\paragraph{Upper Bounds for Alive Agents.}
Conditioning on $\mH_t$, only those agents in $\mA_t$ may receive allocations during rounds $\tau=t,t+1,\ldots,T$. Hence, the sum of V-functions of all agents can never exceed the mechanism that allocates the resource to the alive agent with the highest utility. That is,
\begin{align*}
\sum_{i=1}^K V_i^{{\bm \pi}^\ast}(\mH_{t}) &\le \sum_{i\in\mA_t} (T-t+1)\E_{\bm u_t\sim \mV} \bigg[u_{t,i}\mathbbm{1}[i\in \mA_t] \mathbbm{1}[u_{t,i}>u_{t,j},\forall j\in \mA_t\setminus \{i\}] \bigg]\\
&\le (T-t+1)\left (\sum_{i\in\mA_t} \mu_{t,i} + c\Pr_{\bm u\sim \mV}\{V_i<c,\forall i\in\mA_t\}\right ),
\end{align*}
where the second inequality is because the definition of $\mu_{t,i}$ in \Cref{eq:fair share} contains an indicator $\1[u_{t,i}\ge c]$.
We recall that by \Cref{assumption:min report}, there exists some (albeit unknown) agent $i_0\in [K]$ that ensures $\Pr_{u_{i_0}\sim\mu_{i_0}} \{u_{i_0}\geq c\}=1$. Thus, if $i_0\in \mA_t$ is still alive, we already proved the desired upper bound since $\Pr_{\bm u\sim \mV}\{V_i<c,\forall i\in\mA_t\}=0$. We now consider the case when $i_0\notin\mA_t$. Notice that
\begin{equation*}
    \sum_{i\notin\mA_t}\mu_{1,i}=\E_{\bm u\sim \mV}\sqb{\max_{i\in [K] u_{i}}\1\sqb{\argmax_{i\in [K]} V_i \notin \mA_t}}\ge c\Pr_{\bm u\sim \mV}\{V_i<c,\forall i\in\mA_t\},
\end{equation*}
where the inequality is because $i_0\notin \mA_t$ and \Cref{assumption:min report}.
As a result, we obtained
\begin{equation}\label{eq:upper_bound_max_welfare}
    \sum_{i=1}^K V_i^{{\bm \pi}^\ast}(\mH_{t}) \leq (T-t+1)\left (\sum_{i\in\mA_t} \mu_{t,i}+\sum_{i\notin\mA_t} \mu_{1,i}\right ) \leq (T-t+1)\sum_{i\in\mA_t} \mu_{t,i} + K\sum_{i=1}^K \1[i\notin \mA_t],
\end{equation}
where we used the proved claim that $(T-t+1)\mV_i\leq K$ for all $i\notin \mA_t$ (\Cref{eq:for dead}).
Utilizing the lower bound for those alive agents in \Cref{eq:for alive}, we know
\begin{equation*}
\sum_{j\in\mA_t\setminus\{i\}} u_{j}^{{\bm \pi}^\ast}(\mH_{t})\ge \sum_{j\in\mA_t\setminus\{i\}} \left ((T-t+1)\mu_{t,j}-K\right ),\quad \forall i\in \mA_t,
\end{equation*}
which consequently gives
\begin{align*}
V_i^{{\bm \pi}^\ast}(\mH_{t})&\le (T-t+1)\mu_{t,i}+K \sum_{j=1}^K (\1[j\notin \mA_t]+\1[j\in \mA_t\setminus \{i\}])\\&\le (T-t+1)\mu_{t,i}+K^2,\quad \forall i\in \mA_t.
\end{align*}
The upper bound involving $\mu_{1,i}$ follows from $\mu_{t,i}\geq \mu_{1,i}$.
\end{proof}

\subsection{Proof of Marking Up is Worse than Honesty (\Cref{lem:min report with u is good})}
\begin{lemma}[Marking Up is Worse than Honesty]\label{lem:min report with u is good formal}
Fix round $t\in[T]$ and history $\mH_t$ generated by following $\bm \pi^\ast$ in the past. Consider an alive agent $i\in\mA_t$ such that $p_{t,i}=\frac{2K^2}{(T-t)q_{t,i} c} \leq 1$. For any of its round-$t$ reporting strategy $\pi^r_{t,i}$ in the auxiliary game, let $\pi_{t,i}^{r,\prime}$ be the reporting strategy of
\begin{equation*}
\pi_{t,i}^{r,\prime}(u_{t,i};\tilde \mH_t)=\min(v_{t,i},u_{t,i}),\quad \text{where }v_{t,i}\sim \pi^r_{t,i}(u_{t,i};\tilde \mH_t),
\end{equation*}
which caps the reports suggested by $\pi^r_{t,i}$ with its true utility $u_{t,i}$.

Let $\pi_{t,i}=(\pi^r_{t,i},{\pi}_{t,i}^{\ast,f})$ and $\pi_{t,i}'=(\pi_{t,i}^{r,\prime},{\pi}_{t,i}^{\ast,f})$ be the corresponding round-$t$ strategies in the actual game.
When the opponents are following $\bm \pi_{t,-i}^\ast$, replacing $\pi_{t,i}$ with $\pi^r_{t,i}$ is better for agent $i$:
\begin{equation*}
V_i^{(\pi_{t,i}\circ \bm\pi^\ast_{t,-i})\diamond_t \bm\pi^\ast}(\mH_t) \geq V_i^{(\pi'_{t,i}\circ \bm\pi^\ast_{t,-i})\diamond_t \bm\pi^\ast}(\mH_t).
\end{equation*}
\end{lemma}
\begin{proof}
Throughout the proof, we denote by $u_{t,i}$ the utility of agent $i$ and by $v_{t,i}$ the report suggested by $\pi_{t,i}$ (both can be stochastic).
Since the opponents are following $\bm \pi_{t,-i}^\ast$, the flagging strategies ${\bm \pi}_{t,-i}^{\ast,f}$ are well-behaved (recall \Cref{def:well-behaved flagging}) and the next-round history $\mH_{t+1}$ ensures $\hat q_{t,j}\le 4q_{t,j}$, $\forall j\in [K]$. Applying \Cref{lem:equiv between actual and no-flagging formal} to any possible next-round history $\mH_{t+1}$ and $\bm \pi^\ast=({\bm \pi}^{\ast,r},{\bm \pi}^{\ast,f})$ thus gives $V_i^{\bm \pi^\ast}(\mH_{t+1})=\tilde V_i^{{\bm \pi}^{\ast,r}}(\tilde \mH_{t+1})$ --- that is, the gain of agent $i$ in the future only depends on $\tilde \mH_{t+1}$, and we only need to care about $i_t$ and $\tilde \mH_{t+1}$ when comparing the LHS and RHS.

Comparing $\pi_{t,i}$ and $\pi_{t,i}'$, the only event such that the allocation $i_t$ or the next-round simplified history $\tilde\mH_{t+1}$ can differ is that $v_{t,i}>u_{t,i}$. Denote this event by $\mE_{t,i}$.
We now assume $\mE_{t,i}$ holds and analyze the utility obtained by using $\pi_{t,i}$ or $\pi'_{t,i}$, conditioning on $u_{t,i}$ and $v_{t,i}$. 

We start with $\pi_{t,i}$. Since agent $i$ reports $v_{t,i}$, they win (so $i_t=i$) and are audited with probability $\hat p_{t,i}=\min(\frac{8K^2}{(T-t)\hat q_{t,i} c},1)$ (recall \Cref{line:check prob} from \Cref{alg:mechanism}). If audited, they are eliminated immediately because $v_{t,i}>u_{t,i}$. Otherwise, we move to a next-round simplified history $\tilde \mH_{t+1}=(t+1,\mA_{t+1})$ with $\mA_{t+1}=\mA_t$, and thus the V-function upper bound in \Cref{eq:for alive} ensures the utility is at most
\begin{equation*}
u_{t,i}+(1-\hat p_{i,t}) V_i^{\bm \pi^\ast}({\mH}_{t+1})\le 1 + (1-\hat p_{i,t})\paren{(T-t) \mu_{t,i}+K^2} := u_1.
\end{equation*}

Alternatively, if agent $i$ uses strategy $\pi'_{t,i}$ and reports $u_{t,i}$ instead, they remain alive as $i\in\mA_{t+1}$. Now applying the lower bound part in \Cref{eq:for alive} to the next-round history $\mH_{t+1}$ shows that
\begin{equation*}
u_{t,i}\mathbbm{1}[i_t=i]+V_i^{{\bm \pi}^\ast}({\mH}_{t+1})\ge (T-t) \mu_{t,i}-K:=u_2.
\end{equation*}

It suffices to show that $u_2\geq u_1$ for the desired conclusion.
Since $\hat q_{t,i}\leq 4q_{t,i}$, we have $\hat p_{t,i} \geq \min(p_{t,i},1)=p_{t,i}=\frac{2K^2}{(T-t)q_{t,i} c}$. As \Cref{assumption:min report} further implies $\mu_{t,i}\geq q_{t,i} c$, we obtain
\begin{equation*}
    u_2-u_1 \geq \hat p_{t,i}(T-t)\mu_{t,i} - K^2 - K - 1 \geq 2K^2 - K^2-K-1 \geq 0.
\end{equation*}
Therefore, under event $\mE_{t,i}$, the gain of $\pi_{t,i}'$ is no better than the that of $\pi_{t,i}$. Since they give the same allocations $i_t$ and next-round auxiliary histories $\tilde \mH_{t+1}$ under $\neg \mE_{t,i}$, our conclusion follows.
\end{proof}

\subsection{Proof of \Cref{thm:equivalence of PBE}}\label{sec:equivalence of PBE formal}

We are now ready to prove \Cref{thm:equivalence of PBE}, which characterizes a PBE under \mechname.

\begin{proof}[Proof of \Cref{thm:equivalence of PBE}]
Fix an agent $i\in[K]$. We aim to show that its unilateral deviation from $\bm \pi_i^\ast$ does not increase its utility. Formally, let ${\bm \pi}_i=({\bm \pi}_i^r,{\bm \pi}_i^f)$ be an alternative strategy for agent $i$.

\paragraph{Step 1. Isolating Deviations in Different Rounds.}
We apply the performance difference lemma \citep{kakade2002approximately} between the two joint strategies $\bm \pi_i\circ \bm \pi_{-i}^\ast$ and $\bm \pi^\ast$:
\begin{align}
&\quad V_i^{\bm \pi_i\circ \bm \pi_{-i}^\ast}(\mH_1)-V_i^{\bm \pi^\ast}(\mH_1) \nonumber\\
&=\sum_{t=1}^T \E_{\mH_t\sim \bm \pi_i\circ \bm \pi_{-i}^\ast}\left [\E_{u_{t,i}\sim \mV_i,v_{t,i}\sim \pi_{t,i}^r(u_{t,i};\mH_t)}\left [Q_i^{\bm \pi^\ast}(\mH_t,u_{t,i},v_{t,i},\pi_{t,i}^f)\right ]-V_i^{\bm \pi^\ast}(\mH_t)\right ],\label{eq:performance difference lemma}
\end{align}
where the Q-function (state-action utility function) for a joint strategy $\bm \pi$ in the actual game is defined similarly to V-function characterized in \Cref{eq:V-func recursive}, except that we fix agent $i$'s current-round utility $u_{t,i}$, report $v_{t,i}$, and flagging policy $\pi_{t,i}^f\colon (u_{t,i},\bm v_t,i_t;\mH_t)\mapsto \Delta(\{0,1\})$:
\begin{align}
Q_i^{\bm \pi}(\mH_t,u_{t,i},v_{t,i},\pi_{t,i}^f)=\E\left [u_{t,i}\mathbbm{1}[v_{t,i}>v_{t,j},\forall j\in \mA_t\setminus \{i\}]+V_i^{\bm \pi}(\mH_{t+1})\middle \vert \vphantom{\tilde V_i^{{\bm \pi}^{\ast,r}}(\tilde \mH_{t+1})} \mH_t\right ],\label{eq:Q-func recursive}
\end{align}
where in the expectation, we fix agent $i$'s utility and report as $u_{t,i}$ and $v_{t,i}$, and sample others' utilities and reports as $\bm u_{t,-i}\sim \mV_{-i}$, $\bm v_{t,-i}\sim \bm \pi_{t,-i}^r$. We calculate agent $i$'s flag as $f_{t,i}\sim \pi_{t,i}^f$ and others' as $\bm f_{t,-i}\sim \bm \pi_{t,-i}^f$. After that, the next-round history $\mH_{t+1}$ is generated according to \Cref{alg:mechanism}.

We now focus on a single history $\mH_t$ which is sampled from $\bm \pi_i\circ \bm \pi_{-i}^\ast$. Recalling the definition of well-behaved flagging policy ${\bm \pi}^{\ast,f}$ from \Cref{def:well-behaved flagging}, we immediately have $\hat q_{t,j}\le 4q_{t,j}$ for all $j\in [K]$ at $\mH_t$ since $\bm \pi_{-i}^\ast=({\bm \pi}_{-i}^{\ast,r},{\bm \pi}_{-i}^{\ast,f})$. In the rest of the proof, we show the following claim:
\begin{equation}\label{eq:obj after performance difference lemma}
\E_{u_{t,i}\sim \mV_i,v_{t,i}\sim \pi_{t,i}^r(u_{t,i};\mH_t)}\left [Q_i^{\bm \pi^\ast}(\mH_t,u_{t,i},v_{t,i},\pi_{t,i}^f)\right ]\le V_i^{\bm \pi^\ast}(\mH_t),\quad \forall \mH_t\text{ s.t. }\hat q_{t,j}\le 4q_{t,j}~\forall  j\in [K].
\end{equation}

That is, instead of considering the $T$-round deviation to $\bm \pi_i$, we now only need to focus on a single round deviation to $(\pi_{t,i}^r,\pi_{t,i}^f)$ and assume that the future joint strategy ($\bm \pi$ in the $V_i^{\bm \pi}(\mH_{t+1})$ terms in \Cref{eq:V-func recursive,eq:Q-func recursive}) is the well-behaved $\bm \pi^\ast=({\bm \pi}^{\ast,r},{\bm \pi}^{\ast,f})$, instead of the deviated $\bm \pi_i\circ \bm \pi_{-i}^\ast$.
Using the $\diamond_t$ notation for concatenating one policy to another, \Cref{eq:obj after performance difference lemma} can be rewritten as
\begin{equation*}
\E_{u_{t,i}\sim \mV_i,v_{t,i}\sim \pi_{t,i}^r(u_{t,i};\mH_t)}\left [Q_i^{\bm \pi^\ast}(\mH_t,u_{t,i},v_{t,i},\pi_{t,i}^f)\right ]=V_i^{(\bm \pi_{t,i}\circ \bm \pi_{t,-i}^\ast)\diamond_t \bm \pi^\ast}(\mH_t)\le V_i^{\bm \pi^\ast}(\mH_t).
\end{equation*}

\paragraph{Step 2. Eliminating the Deviation in Flagging Policies.}
We now eliminate the effect of replacing ${\pi}_{t,i}^f$ with $\pi_{t,i}^{\ast,f}$.
As the future joint strategy $\bm \pi^\ast$ is constructed from the joint strategy ${\bm \pi}^{\ast,r}$ in the auxiliary game (recall the construction in \Cref{lem:equiv between actual and no-flagging formal} and the definition of $\bm \pi^\ast$ in \Cref{thm:equivalence of PBE}),
\begin{equation*}
V_i^{\bm \pi^\ast}(\mH_{t+1})=\tilde V_i^{{\bm \pi}^{\ast,r}}(\tilde{\mH}_{t+1})\text{ as long as }\mH_{t+1}\text{ ensures }\hat q_{t+1,j}\le 4q_{t+1,j}\text{ for all } j\in [K].
\end{equation*}

Similar to the arguments when proving \Cref{lem:equiv between actual and no-flagging formal}, the condition $\hat q_{t+1,j}\le 4q_{t+1,j}$ for all $j\in [K]$ is always true under either $\pi_{t,i}^f$ or ${\pi}_{t,i}^{\ast,f}$, thanks to ${\bm \pi}_{t,-i}^{\ast,f}$: if someone dies, then all $\hat q_{t+1,j}$'s are zero; if $\hat q_{t+1,i_t}\le 4q_{t+1,i_t}$, then we are good; otherwise, due to ${\bm \pi}_{t,-i}^{\ast,f}$, such $\hat q_{t+1,i_t}$ would be reset to $0$.

Furthermore, ${\pi}_{t,i}^f$ and $\pi_{t,i}^{\ast,f}$ give the same next-round simplified history $\tilde \mH_{t+1}$ because flags only happen if $\mA_{t+1}=\mA_t$, which already indicates $\tilde \mH_{t+1}=(t+1,\mA_t)$. Therefore,
\begin{equation}\label{eq:eliminate_flag}
Q_i^{\bm \pi^\ast}(\mH_t,u_{t,i},v_{t,i},\pi_{t,i}^f) = Q_i^{\bm \pi^\ast}(\mH_t,u_{t,i},v_{t,i},\pi_{t,i}^{\ast,f}).
\end{equation}

\paragraph{Step 3. Eliminating the Deviation to Marking Up.}
We next introduce a reporting policy $\bar\pi^{r,1}_{t,i}$ that complies with \Cref{item:no mark up unless p>=1 formal}. If $p_{t,i}>1$, we simply use $\bar\pi^{r,1}_{t,i}=\bar\pi^{r}_{t,i}$. Otherwise, for given any utility $u_{t,i}$, we define an alternative strategy that caps the reports at the true utility, as in \Cref{lem:min report with u is good formal}:
\begin{equation*}
\pi_{t,i}^{r,1}(u_{t,i};\mH_t)=\min(v_{t,i},u_{t,i}),\quad \text{where }v_{t,i}\sim \pi_{t,i}^r (u_{t,i};\mH_t).
\end{equation*}
\Cref{lem:min report with u is good formal} then implies that
\begin{equation}\label{eq:obj after removing marking up}
V_i^{(({\color{red}\pi_{t,i}^{r,1}},{\pi}_{t,i}^{\ast,f})\circ {\bm \pi}_{-i}^\ast)\diamond_t \bm \pi^\ast}(\mH_t)  \geq V_i^{((\pi_{t,i}^{r},{\pi}_{t,i}^{\ast,f})\circ {\bm \pi}_{-i}^\ast)\diamond_t \bm \pi^\ast}(\mH_t) = V_i^{(\bm \pi_{t,i}\circ \bm \pi_{t,-i}^\ast)\diamond_t \bm \pi^\ast}(\mH_t), 
\end{equation}
where the equality comes from \Cref{eq:eliminate_flag}.

\paragraph{Step 4. Eliminating the History Dependency.}
We further remove the dependency on the full history $\mH_t$ in $\pi_{t,i}^{r,1}$ and get yet another intermediate policy $\pi_{t,i}^{r,2}$ that further complies with \Cref{item:only simplified history formal}.

To do so, since $\pi_{t,i}^{r,1}$ complies with \Cref{item:no mark up unless p>=1 formal}, we know
\begin{align*}
V_i^{(\pi_{t,i}^{r,1},{\pi}_{t,i}^{\ast,f})\circ {\bm \pi}_{-i}^\ast)\diamond_t \bm \pi^\ast}(\mH_t)&=\E_{u_{t,i}\sim \mV_i}\left [\E_{v_{t,i}\sim \pi_{t,i}^{r,1}(u_{t,i};\mH_t)}\left [Q_i^{\bm \pi^\ast}(\mH_t,u_{t,i},v_{t,i},\pi_{t,i}^{\ast,f})\right ]\right ]\\
&\le \E_{u_{t,i}\sim \mV_i}\left [\max_{v_{t,i}\in \text{Range}_{t,i}(u_{t,i})}\left [Q_i^{\bm \pi^\ast}(\mH_t,u_{t,i},v_{t,i},\pi_{t,i}^{\ast,f})\right ]\right ],
\end{align*}
where $\text{Range}_{t,i}(u_{t,i})$ is defined as $[0,u_{t,i}]$ if $\frac{2K^2}{(T-t) q_i c}\leq 1$ and $[0,1]$ otherwise. We define $\pi_{t,i}^{r,2}$ as the strategy attaining the RHS, and then justify that $\pi_{t,i}^{r,2}$ can be implemented only using $\tilde \mH_t$. That is, let
\begin{equation*}
\pi_{t,i}^{r,2}(u_{t,i};\mH_t)=\argmax_{v_{t,i}\in \text{Range}_{t,i}(u_{t,i})} Q_i^{{\bm \pi}^\ast}(\mH_t,u_{t,i},v_{t,i},{\pi}_{t,i}^{\ast,f}),
\end{equation*}
and we show that the $\argmax$ over $v_{t,i}\in \text{Range}_{t,i}(u_{t,i})$ can be evaluated only using $\tilde H_t$.\footnote{The statement is not true if we change $v_{t,i}\in \text{Range}_{t,i}(u_{t,i})$ to $v_{t,i}\in [0,1]$ because reporting $v_{t,i}>u_{t,i}$ when $p_{t,i}<1$ results in a different elimination probability from the auxiliary game \Cref{def:auxiliary game}. Therefore, we need Step 3 to generate a $\pi_{t,i}^{r,1}$ satisfying \Cref{item:no mark up unless p>=1 formal}.}
Recall the definition of the Q-function from \Cref{eq:Q-func recursive}:
\begin{align*}
Q_i^{\bm \pi^\ast}(\mH_t,u_{t,i},v_{t,i},{\pi_{t,i}}^{\ast,f})&=\E\bigg [u_{t,i}\mathbbm{1}[v_{t,i}>v_{t,j},\forall j\in \mA_t\setminus\{i\}]+ V_i^{{\bm \pi}^\ast}(\mH_{t+1})\bigg \vert\\
&\quad \qquad \bm u_{t,-i}\sim \mV_{-i}, \bm v_{t,-i}\sim {\bm \pi}_{t,-i}^{\ast,r},\mH_{t+1}\text{ from \Cref{alg:mechanism}}\bigg ]\bigg ].
\end{align*}

The indicator whether $v_{t,i}>v_{t,j}$ for all $j\in \mA_t\setminus\{i\}$ only depends on current-round reports $\bm v_t$ and the set of alive agents $\mA_t$. Meanwhile, thanks to \Cref{lem:equiv between actual and no-flagging formal}, $V_i^{\bm \pi^\ast}(\mH_{t+1})=\tilde V_i^{{\bm \pi}^{\ast,r}}(\tilde \mH_{t+1})$ only depends on $\tilde \mH_{t+1}=(t+1,\mA_{t+1})$. 
Thanks to the correspondence result in \Cref{lem:equiv between actual and no-flagging formal}, even in the actual game, whether agent $j\in \mA_t$ stays alive in the next round when reporting $v_{t,j}\in \text{Range}_{t,j}(u_{t,j})$ is also history-independent. Furthermore, opponents' strategies ${\bm \pi}_{t,-i}^{\ast,r}$ and $\text{Range}_{t,j}(u_{t,j})$ only depend on $\tilde \mH_t$. Hence, $\pi_{t,i}^{r,2}$ complies with \Cref{item:only simplified history formal}. Furthering the bound from \Cref{eq:obj after removing marking up},
\begin{equation}\label{eq:obj after removing history-dependency}
V_i^{(\bm \pi_{t,i}\circ \bm \pi_{t,-i}^\ast)\diamond_t \bm \pi^\ast}(\mH_t)\le V_i^{((\pi_{t,i}^{r,1},{\pi}_{t,i}^{\ast,f})\circ {\bm \pi}_{-i}^\ast)\diamond_t \bm \pi^\ast}(\mH_t) \leq V_i^{(({\color{red}\pi_{t,i}^{r,2}},{\pi}_{t,i}^{\ast,f})\circ {\bm \pi}_{-i}^\ast)\diamond_t \bm \pi^\ast}(\mH_t).
\end{equation}

\paragraph{Step 5. Utilizing ${\bm \pi}^{\ast,r}$ is a PBE.}
Because of Steps 2 -- 4, the joint strategy $(\pi_{t,i}^{r,2}\circ {\bm \pi}_{-i}^{\ast,r})\diamond_t {\bm \pi}^{\ast,r}$ is a valid in the auxiliary game (\Cref{def:auxiliary game}). Therefore, utilizing the equivalence lemma (\Cref{lem:equiv between actual and no-flagging formal}) and that ${\bm \pi}^{\ast,r}$ is a PBE in the auxiliary game, the second inequality in \Cref{eq:obj after removing history-dependency} is true:
\begin{equation*}
V_i^{((\pi_{t,i}^{r,2},{\pi}_{t,i}^{\ast,f})\circ {\bm \pi}_{-i}^\ast)\diamond_t \bm \pi^\ast}(\mH_t)\overset{(a)}{=}\tilde V_i^{(\pi_{t,i}^{r,2}\circ {\bm \pi}_{-i}^{\ast,r})\diamond_t {\bm \pi}^{\ast,r}}(\tilde \mH_t)\overset{(b)}{\le} \tilde V_i^{{\bm \pi}^{\ast,r}}(\tilde \mH_t)\overset{(c)}{=}V_i^{\bm \pi^\ast}(\mH_t),
\end{equation*}
where (a) applies \Cref{lem:equiv between actual and no-flagging formal} to $((\pi_{t,i}^{r,2},{\pi}_{t,i}^{\ast,f})\circ {\bm \pi}_{-i}^\ast)\diamond_t \bm \pi^\ast$ with respect to $\mH_t$, (b) uses the definition ${\bm \pi}^{\ast,r}$ is a PBE in the auxiliary game, and (c) applies \Cref{lem:equiv between actual and no-flagging formal} again to $\bm \pi^\ast$.

Together with \Cref{eq:obj after removing history-dependency}, this proves \Cref{eq:obj after performance difference lemma}. Last, plugging this into the performance difference lemma \Cref{eq:performance difference lemma} concludes the proof that any unilateral deviation $\bm \pi_i\circ {\bm \pi}_{-i}^\ast$ does not strictly improve over $\bm \pi^\ast$ for an arbitrary agent $i\in [K]$. Hence, $\bm \pi^\ast$ is a PBE in the actual game.
\end{proof}


\section{Guarantees on the Regret and Number of Audits for \mechname (\Cref{thm:deterministic main thm})}\label{sec:appendix varying-p-audits}

In this section, we prove our main result \Cref{thm:deterministic main thm} which bounds the regret and expected number of audits of the mechanism \mechname. We prove each bound separately in \Cref{subsec:proof_regret,subsec:proof_audits}.

\subsection{Proof of Regret Guarantee}\label{subsec:proof_regret}

Using \Cref{lem:u lower and upper bound formal}, we directly prove the following bound on the regret of \mechname under the PBE $\bm\pi^\ast$.

\begin{theorem}[Regret Guarantee of \mechname]\label{thm:regret main theorem formal}
Under the mechanism \mechname defined in \Cref{alg:mechanism}, there exists a PBE of agents' strategies, namely $\bm \pi^\ast=(\pi_{t,i}^\ast)_{t\in [T],i\in [K]}$, such that
$\mathcal R_T(\bm \pi^\ast,\mechname)\le K^2$.
\end{theorem}
\begin{proof}
From \Cref{thm:equivalence of PBE}, the $\bm \pi^\ast$ defined using ${\bm \pi}^{\ast,r}$ (the PBE in the auxiliary game \Cref{def:auxiliary game}) and ${\bm \pi}^{\ast,f}$ (the well-behaved flagging policy; \Cref{def:well-behaved flagging formal}) is a PBE.
Under ${\bm \pi}^\ast$, using the lower bound \Cref{eq:for alive} from \Cref{lem:u lower and upper bound formal}, the social welfare is at least
\begin{equation*}
\sum_{i=1}^K V_i^{{\bm \pi}^\ast}(\mH_1)\ge \sum_{i=1}^K (T\mu_{1,i}-K)=T\sum_{i=1}^K \mu_{1,i}-K^2,
\end{equation*}
where we used the fact that all agents are alive in the very beginning (that is, $\mA_1=[K]$).

Therefore, since the maximum possible social welfare is exactly $T\sum_{i=1}^K \mu_{1,i}$,
\begin{align*}
\mathcal R_T(\bm \pi^
\ast,\mechname)&=\E \left [\sum_{t=1}^T \left (\max_{i\in [K]} u_{t,i}-u_{t,i_t}\right )\right ]\\
&=\sum_{i=1}^K \E\left [\sum_{t=1}^T u_{t,i}\mathbbm{1}[u_{t,i}>u_{t,j},\forall j\ne i]\right ]-\sum_{i=1}^K V_i^{{\bm \pi}^\ast}(\mH_1)\\
&\overset{(a)}{=}\sum_{i=1}^K \E\left [\sum_{t=1}^T u_{t,i} \mathbbm{1}[u_{t,i}\ge c] \mathbbm{1}[u_{t,i}>u_{t,j},\forall j\ne i]\right ]-\sum_{i=1}^K V_i^{{\bm \pi}^\ast}(\mH_1)\\
&\le T\sum_{i=1}^K \mu_{1,i}-\left (T\sum_{i=1}^K \mu_{1,i}-K^2\right )=K^2,
\end{align*}
where (a) uses \Cref{assumption:min report}. The regret upper bound hence follows.
\end{proof}

\subsection{Proof of Number of Audits Guarantee}\label{subsec:proof_audits}

We next turn to proving the guarantee on the expected number of audits of \mechname from \Cref{thm:deterministic main thm}. This requires a few additional notations.

\begin{definition}[Estimation Epochs]
Let $1=t_1,t_2,\ldots$ be the rounds in which we eliminate everyone and reset all empirical estimations $\hat q_{t,i}$'s to $0$ (see \Cref{line:elimination}). Formally, we write
\begin{equation*}
    t_\ell=\min \set{t\in[T]\mid \lvert \mA_t\rvert =T-i+1}\cup\{+\infty\}, \quad \ell\in[K],
\end{equation*}
where $t_\ell=+\infty$ indicates that there were at most $\ell-1$ restarts of the estimation process. The $\ell$-th estimation epoch $E_\ell$ is all the rounds $E_\ell=\{t_\ell,t_\ell+1,\ldots,t_{\ell+1}-1\}$.
\end{definition}

For each agent $i\in[K]$ and each estimation epoch $\ell\in[K]$ in which agent $i$ is alive, we decompose the rounds in which agent $i$ wins on two phases. The first \emph{estimation} phase corresponds to the rounds where $\hat q_{t,i}=0$, and the second \emph{incentive-compatible} phase is when $\hat q_{t,i}>0$ (recall from \Cref{alg:mechanism} that $\hat q_{t,i}$ means the estimation $\hat q_i$ at the beginning of round $t$). To avoid confusions, we denote by $\hat q_{\ell,i}$ the final estimation $\hat q_{t,i}$ in epoch $\ell$ that no agent answer with $f_{t,i}=1$; this $\hat q_{\ell,i}$ is then fixed throughout the $\ell$-th incentive-compatible phase. In the special case where there is no incentive-compatible phase during the $\ell$-th epoch, we write $\hat q_{\ell,i}=0$.

As introduced in \Cref{sec:sketch varying-p} (Part II), bounding the number of audits requires bounding the number of events where agents voluntarily mark down. We prove the following complete version of \Cref{lem:number of marking down bound}.

\begin{lemma}\label{lem:number of marking down bound formal}
For every round $t\in [T]$ and agent $i\in [K]$, define indicator $D_{t,i}$ for the event that they are alive in round $t$, do not win, but would have won had they reported honestly, i.e.,
\begin{equation*}
D_{t,i}=\mathbbm{1}[i\in \mA_t]\mathbbm{1}[i_t\ne i]\mathbbm{1}[u_{t,i}\geq c]\mathbbm{1}[u_{t,i}>v_{t,j},\forall j\in \mA_t\setminus\{i\}].
\end{equation*}
Now we consider any history $\mH_{t_\ell}$ corresponding to the start of a new estimation epoch $E_\ell$. We have
\begin{equation*}
    \E\sqb{\sum_{i\in\mA_{t_\ell}}\sum_{t\in E_\ell} D_{t,i}\middle \vert\mH_{t_\ell}} \leq \frac{2K^2}{c}.
\end{equation*}
\end{lemma}

\begin{proof}
Fix $t\geq t_\ell$ and consider any history $\mH_{t}$ within $E_\ell$. Throughout this proof, the probabilities will always be taken over the allocation process starting from history $\mH_{t}$. We aim to show that
\begin{equation}\label{eq:lower_bound_proba_die}
    \Pr \{i_t\text{ eliminated at round } t\mid\mH_t \} \geq \frac{c}{2K^2} \E\sqb{\sum_{i\in\mA_t} D_{t,i} \middle \vert \mH_t}.
\end{equation}

To prove \Cref{eq:lower_bound_proba_die}, we consider an alternative policy $\pi_{t,i}$ for agent $i$ at time $t$:
\begin{equation*}
\pi_{t,i}=\max(v_{t,i},u_{t,i}),\quad \text{where }v_{t,i}\sim \pi_{t,i}^\ast(u_{t,i};\tilde \mH_{t}),
\end{equation*}
which removes all mark-downs suggested by $\bm \pi_i^\ast$. We now compare the utilities collected by agent $i$ starting from $\mH_{t}$ by $\bm\pi_i^\ast$ and $\pi_{t,i} \diamond_t \bm\pi_{i}^\ast$. Note that the only event making these two policies have different allocations $i_{t}$ or different next-round auxiliary histories $\tilde \mH_{t+1}$ is when $D_{t,i}=1$.

Suppose that $D_{t,i}=1$. If agent $i$ followed $\pi_{t,i} \diamond_t \bm\pi_{i}^\ast$, then they report $u_{t,i}$ in round $t$ and the next-round auxiliary history is $\tilde\mH_{t+1}^0:=(t+1,\mA_t)$. Using \Cref{lem:equiv between actual and no-flagging}, their utility is
\begin{equation*}
    V_{1,i}:=u_{t,i} + \tilde V_i^{{\bm \pi}^{\ast,r}}(\tilde\mH_{t+1}^0) \geq c + \tilde V_i^{{\bm \pi}^{\ast,r}}(\tilde\mH_{t+1}^0),
\end{equation*}
where we used the definition of $D_{t,i}=1$ to get $u_{t,i}\geq c$.

On the other hand, if agent $i$ used strategy $\bm \pi_{i}^\ast$, they report $v_{t,i}$ and lose in round $t$. There are two possibilities for the next-round auxiliary history: either the winner $i_{t}$ is eliminated which results in $\tilde\mH_{t+1}^1:=(t+1,\mA_t\setminus\{i_{t}\})$, or the winner remains alive and gives $\tilde\mH_{t+1}^0$. Again using \Cref{lem:equiv between actual and no-flagging}, the utility obtained is
\begin{align*}
    V_{2,i}&:=\tilde V_i^{{\bm \pi}^{\ast,r}}(\tilde\mH_{t+1}^0) + \1[i_{t}\text{ eliminated}] \paren{ \tilde V_i^{{\bm \pi}^{\ast,r}}(\tilde\mH_{t+1}^1) - \tilde V_i^{{\bm \pi}^{\ast,r}}(\tilde\mH_{t+1}^0) },
\end{align*}
where the inequality comes from the upper and lower bounds in \Cref{lem:u lower and upper bound}. For completeness, when $D_{t,i}=0$, let us pose $V_{1,i}:=V_{2,i}:=0$.
Altogether, we obtained
\begin{align*}
    V_i^{\bm\pi^\ast}(\mH_{t}) &- V_i^{(\pi_{t,i}\circ \bm \pi_{t,-i}^\ast)\diamond_t \bm \pi^\ast}(\mH_{t}) = \E[(V_{2,i}-V_{1,i})D_{t,i} \mid \mH_t]\\
    &\leq \E\sqb{D_{t,i}\1[i_t \text{ eliminated}] \paren{ \tilde V_i^{{\bm \pi}^{\ast,r}}(\tilde\mH_{t+1}^1) - \tilde V_i^{{\bm \pi}^{\ast,r}}(\tilde\mH_{t+1}^0) }\middle \vert \mH_t} - c\Pr\{D_{t,i}=1\mid\mH_t\}.
\end{align*}
Since $\bm\pi^\ast$ is a PBE, the left-hand side above is non-negative for all $i\in\mA_t$. Also, note that the auxiliary histories $\tilde\mH_{t+1}^0$ and $\tilde\mH_{t+1}^1$ only depends on $i_t$. In particular, if we fix any two realizable histories $\mH_{t+1}^0$ and $\mH_{t+1}^1$ such that their auxiliary histories are $\tilde\mH_{t+1}^0=(t+1,\mA_t)$ and $\tilde\mH_{t+1}^1=(t+1,\mA_t\setminus\{i_t\})$ respectively, the sum of the previous equations gives
\begin{align}
    0&\leq \sum_{i\in\mA_t} \left (V_i^{\bm\pi^\ast}(\mH_{t}) - V_i^{(\pi_{t,i}\circ \bm \pi_{t,-i}^\ast)\diamond_t \bm \pi^\ast}(\mH_{t})\right ) \notag \\
    &= \E\sqb{\1[i_t \text{ eliminated}]\sum_{i\in \mA_t}  D_{t,i} \paren{   V_i^{{\bm \pi}^{\ast}}(\mH_{t+1}^1) -  V_i^{{\bm \pi}^{\ast}}(\mH_{t+1}^0) }\middle \vert\mH_t} - c\E\sqb{\sum_{i\in\mA_t}D_{t,i}\middle \vert\mH_t} .\label{eq:summed_PBE_equations}
\end{align}
Now fix a specific realization for the reports at time $t$ such that $i_t$ was eliminated at that round (when all agents followed $\bm\pi^\ast$ at round $t$). We introduce the set $S:=\{i\in\mA_t\mid D_{t,i}=1\}$ and control the sum $\sum_{i\in \mA_t}D_{t,i}(V_i^{\bm \pi^\ast}(\mH_{t+1}^1)-V_i^{\bm \pi^\ast}(\mH_{t+1}^0))=\sum_{i\in S}(V_i^{\bm \pi^\ast}(\mH_{t+1}^1)-V_i^{\bm \pi^\ast}(\mH_{t+1}^0))$ in \Cref{eq:summed_PBE_equations}.

For $\sum_{i\in S} V_i^{\bm \pi^\ast}(\mH_{t+1}^0)$, we utilize the lower bound on V-functions for those alive agents from \Cref{lem:u lower and upper bound formal} (which is applicable to all $i\in S\subseteq \mA_{t+1}$) and get
\begin{equation}\label{eq:lower_bound_sum_values}
    \sum_{i\in S} V_i^{{\bm \pi}^{\ast}}(\mH_{t+1}^0) \geq (T-t)\sum_{i\in S}\mu_{t+1,i}^0 - K|S| = (T-t)\sum_{i\in S}\mu_{t,i} - K|S|.
\end{equation}
where by $\mu_{t+1,i}^0$ we denote the fair share of agent $i$ corresponding to the auxiliary history $\tilde\mH_{t+1}^0$. In the last equality, we use the fact that $\tilde \mH_{t+1}^0=(t+1,\mA_t)$ and thus $\mu_{t+1,i}^0=\mu_{t,i}$.

For $\sum_{i\in S}V_i^{\bm \pi^\ast}(\mH_{t+1}^1)$, we rewrite it as $\sum_{i\in \mA_{t+1}}V_i^{\bm \pi^\ast}(\mH_{t+1}^1)-\sum_{i\in \mA_{t+1}\setminus S}V_i^{\bm \pi^\ast}(\mH_{t+1}^1)$. For the first term, using the upper bounds on V-functions (specifically, \Cref{eq:upper_bound_max_welfare} in the proof of \Cref{lem:u lower and upper bound formal}),
\begin{equation*}
    \sum_{i\in \mA_{t+1}} V_i^{{\bm \pi}^{\ast}}(\mH_{t+1}^1)  \leq (T-t)\sum_{i\in\mA_{t+1}}\mu_{t+1,i}^1 + K^2.
\end{equation*}
Again, $\mu_{t+1,i}^1$ denotes the fair share corresponding to the auxiliary history $\tilde\mH_{t+1}^1$ which corresponds to the set of alive agents $\mA_t\setminus\{i_t\}$. Meanwhile, via the lower bound on V-functions in \Cref{lem:u lower and upper bound formal},
\begin{equation*}
     \sum_{i\in\mA_{t+1}\setminus S} V_i^{{\bm \pi}^{\ast}}(\mH_{t+1}^1) \geq (T-t)\sum_{i\in\mA_{t+1}\setminus S}\mu_{t+1,i}^1 -K|\mA_{t+1}\setminus S|.
\end{equation*}
Note that because agents in $S$ marked down at round $t$, they are not eliminated, that is $S\subset\mA_{t+1}$. Then, combining the two previous equations implies
\begin{align}
    \sum_{i\in S}  V_i^{{\bm \pi}^{\ast}}(\mH_{t+1}^1) &\leq (T-t)\sum_{i\in S}\mu_{t+1,i}^1 + K^2 + K(|\mA_t|-|S|-1)\notag\\
    &\leq (T-t)\paren{\sum_{i\in S}\mu_{t,i} + \mu_{t,i_t}} + K^2 + K(|\mA_t|-|S|-1)\label{eq:new_upper_bound_sum_S}
\end{align}
In the last inequality, we used the fact that the only difference between the fair shares $\mu_{t,i}$ and $\mu_{t+1,i}$ are that in the second one, agent $i_t$ is not alive anymore and as a result,
\begin{align*}
    \sum_{i\in S}\mu_{t+1,i}^1 
    &= \E_{\bm u\sim \mV}\sqb{\max_{i\in S}u_i \cdot \1\sqb{\max_{i\in S}u_i> \max_{i\in \mA_t\setminus(S\cup\{i_t\})}u_i}}\\
    &= \E_{\bm u\sim \mV}\sqb{\max_{i\in S}u_i \cdot \paren{ \1\sqb{\max_{i\in S}u_i> \max_{i\in \mA_t\setminus S}u_i} + \1\sqb{u_{i_t}>\max_{i\in S}u_i> \max_{i\in \mA_t\setminus S}u_i }}} \\
    &\leq \sum_{i\in S}\mu_{t,i} + \E_{\bm u}\sqb{u_{i_t} \cdot \1\sqb{u_{i_t}>\max_{i\in S}u_i> \max_{i\in \mA_t\setminus (S\cup\{i_t\})}u_i }} \leq \sum_{i\in S}\mu_{t,i} + \mu_{t,i_t}.
\end{align*} 
Next, the proof of \Cref{eq:for dead} in \Cref{lem:u lower and upper bound formal} showed that for all $i\in\mA_t$, if we had $(T-t)\mu_{t,i}>K$ then we would have $i\in\mA_{t+1}$ \textit{a.s.} As a result, with probability one, $i_t\notin\mA_{t+1}$ (\textit{i.e.}, $i_t$ is eliminated) implies $(T-t)\mu_{t,i_t}\leq K$. Therefore, with probability one, \Cref{eq:new_upper_bound_sum_S} gives
\begin{equation*}
    \1[i_t\text{ eliminated}] \sum_{i\in S}  V_i^{{\bm \pi}^{\ast}}(\mH_{t+1}^1) \leq \1[i_t\text{ eliminated}]\paren{ (T-t)\sum_{i\in S}\mu_{t,i} + K^2 + K(|\mA_t|-|S|)}.
\end{equation*}

Together with \Cref{eq:lower_bound_sum_values}, this shows that with probability one
\begin{align*}
    &\quad \1[i_t\text{ eliminated}]\sum_{i\in \mA_t} D_{t,i}\paren{  V_i^{{\bm \pi}^{\ast}}(\mH_{t+1}^1) - V_i^{{\bm \pi}^{\ast}}(\mH_{t+1}^0)} \\
    &=\1[i_t\text{ eliminated}]\sum_{i\in S} \left (V_i^{{\bm \pi}^{\ast}}(\mH_{t+1}^1) -  V_i^{{\bm \pi}^{\ast}}(\mH_{t+1}^0)\right )\leq 2K^2 \1[i_t\text{ eliminated}].
\end{align*}
Plugging this identity within \Cref{eq:summed_PBE_equations} exactly proves \Cref{eq:lower_bound_proba_die}.

As a result of \Cref{eq:lower_bound_proba_die}, the sequence formed by
\begin{equation*}
    \sum_{\tau=t_\ell}^{t} \sum_{i\in\mA_{\tau}}D_{\tau,i} - \frac{2K^2}{c} (|\mA_{t_\ell}| - |\mA_{t+1}|)
\end{equation*}
for $t\geq t_\ell$ is a supermartingale with increments bounded by $1$ in absolute utility. Then, we can apply Doob's stopping theorem to the stopping time $\min(t_{\ell+1}-1,T)$ and conclude
\begin{equation*}
    \E\sqb{\sum_{t\in E_\ell} \sum_{i\in\mA_t}D_{t,i} \middle \vert \mH_{t_l}} \leq \frac{2K^2}{c}.
\end{equation*}
Recalling that for all $t\geq t_\ell$ we have $\mA_t\subseteq\mA_{t_\ell}$ and that $D_{t,i}\leq \1[i\in\mA_t]$ ends the proof.
\end{proof}

We are now ready to bound the expected number of audits of \mechname under the PBE $\bm\pi^\ast$.

\begin{theorem}[Number of Audits Guarantee of \mechname]\label{thm:audit main theorem formal}
Under the same conditions as in \Cref{thm:regret main theorem formal}, the expected number of audits made by \mechname under the same PBE $\bm \pi^\ast$ ensures
\begin{equation*}
\mathcal B_T(\bm \pi^\ast,\mechname)=\O\paren{\frac{K^3}{c} \log T}.
\end{equation*}
\end{theorem}
\begin{proof}
We control the number of audits that we spend on every agent $i\in [K]$, every estimation epoch $\ell\in [K]$, and every corresponding estimation and incentive-compatible phase separately.

\paragraph{Estimation Phase.}
We use the same notation $D_{t,i}$ as introduced in \Cref{lem:number of marking down bound formal}. When agent $i$ wins in some round $t\in E_\ell$ during its estimation phase, at least one of these scenarios holds:
\begin{itemize}
\item Agent $i$ marked up. In this case, $t_{\ell+1}=t+1$ and $i$ is eliminated. This happens at most once.
\item Agent $i$ would have won had \emph{all} agents reported truthfully, which formalizes as $u_{t,i}>u_{t,j}$, $\forall j\in \mA_{t_\ell}\setminus \{i\}$. Denote by $F_{t,i}$ the indicator of this event.
\item Some other alive agent $j\in \mA_{t_\ell}\setminus \{i\}$ would have won if they reported truthfully but marked down. This corresponds to the case where $D_{t,j}=1$.
\end{itemize}
Therefore, we may upper bound the number of wins of agent $i$ from $t_\ell$ and until some $\tau\ge t_\ell$ as
\begin{align}
    \abs{\set{ t\in[t_\ell,\tau]\mid i_{t}=i }}
    &\leq \sum_{t=t_\ell}^{\tau} F_{t,i} + \underbrace{1+\sum_{t\in E_\ell} \1[i_{t}=i] \left (\bigvee_{j\in\mA_{t_\ell}\setminus\{i\}} D_{t,j} \right )}_{:=M_{\ell,i}}.\label{eq:upper_bound_nb_wins}
\end{align}
In words, the number of rounds that agent $i$ actually wins (LHS) is at most its fair winning rounds (those $t$'s with $F_{t,i}=1$) plus those other agents miss ($t$'s where any other $j\ne i$ has $D_{t,j}=1$).

On the other hand, we lower bound the same quantity in another way. In a round $t\in E_\ell$ where agent $i$ would have won if \emph{all} agents reported truthfully (so $F_{t,i}=1$), one of these scenarios holds
\begin{itemize}
    \item Agent $i$ indeed wins, \textit{i.e.}, $i_t=i$.
    \item Agent $i$ did not win because itself is marking down. That is, $v_{t,i}<\max_{j\in \mA_{t_\ell}\setminus \{i\}}v_{t,j}<u_{t,i}$. This corresponds to the case where $D_{t,i}=1$.
    \item The agent $i_t$ won by marking up. This means $t_{\ell+1}=t+1$ and $i_t$ is eliminated at that time. This happens at most once during $E_\ell$.
\end{itemize}
Therefore, for any $\tau\ge t_\ell$, we also have
\begin{equation}\label{eq:lower_bound_nb_wins}
    \abs{\set{ t\in[t_\ell,\tau]\mid i_{t}=i }} \geq \sum_{t=t_\ell}^{\tau} F_{t,i} - \underbrace{\paren{\sum_{t\in E_\ell}D_{t,i} + 1}}_{:=N_{\ell,i}}.
\end{equation}
In words, the number of rounds that agent $i$ actually wins (LHS) is at least its fair winning rounds minus the rounds that itself misses.

Last, we define ${\mathfrak t}_{\ell,i}$ as the first round $t$ such that agent $i$ wins for at least $\max(1,4M_{\ell,i}, 3N_{\ell,i})$ times during epoch $E_\ell$ and that
\begin{equation}\label{eq:constraint_F}
    \frac{1}{t-t_\ell+1}\sum_{\tau=t_\ell}^t F_{t,i} \in\sqb{\frac{q_{\ell,i}}{3}, 3q_{\ell,i}},
\end{equation}
which roughly means the estimation $\hat q_{t,i}$ generated using outcomes of rounds $t_\ell,t_\ell+1,\ldots,t$ is close to the actual $q_{t,i}$.
If such $t$ does not exist, we define ${\mathfrak t}_{\ell,i}=t_{\ell+1}-1$ as the last round of epoch $E_\ell$.
We formalize the intuition for \Cref{eq:constraint_F} as the following claim:

\begin{claim}
The estimation phase of agent $i\in [K]$ in epoch $\ell\in [K]$ ends on or before $\mathfrak t_{\ell,i}$.
\end{claim}
\begin{proof}
From \Cref{line:estimate winning prob} of \Cref{alg:mechanism}, the estimation phase ends in some round $\tau \in E_\ell$ once no agent flags the estimated $\hat q_{\tau,i}$. Recalling the well-behaved flagging policy ${\bm\pi}^{\ast,f}$ from \Cref{def:well-behaved flagging}, this corresponds to the first time that $\hat q_{\tau,i} \in [ q_{\tau,i}/4, 4 q_{\tau,i}]=[q_{\ell,i}/4,4q_{\ell,i}]$. We now show this happens on or before $\mathfrak t_{\ell,i}$.

The claim is immediate if ${\mathfrak t}_{\ell,i}$ is the last time of epoch $E_\ell$.
Otherwise, because agent $i$ won at least $4M_{\ell,i}$ times, we have
\begin{equation*}
    \sum_{\tau=t_\ell}^{{\mathfrak t}_{\ell,i}} F_{t,i}\geq \abs{\set{ \tau\in[t_\ell,{\mathfrak t}_{\ell,i}]: i_{\tau}=i }}-M_{i,j} \geq \frac{3\abs{\set{ \tau\in[t_\ell,{\mathfrak t}_{\ell,i}]: i_{\tau}=i }}}{4},
\end{equation*}
where we used \Cref{eq:upper_bound_nb_wins} in the first inequality. This implies
\begin{equation*}
    \frac{ \abs{\set{ \tau\in[t_\ell,{\mathfrak t}_{\ell,i}]: i_{\tau}=i }} }{{\mathfrak t}_{\ell,i}-t_\ell+1} \leq \frac{4}{3({\mathfrak t}_{\ell,i}-t_\ell+1)} \sum_{\tau=t_\ell}^{{\mathfrak t}_{\ell,i}} F_{t,i} \leq 4q_{\ell,i}.
\end{equation*}
In the last inequality, we used the definition of $\mathfrak t_{\ell,i}$ in \Cref{eq:constraint_F}. On the other hand, because at time $\mathfrak t_{\ell,i}$ agent $i$ won at least $3N_{\ell,i}$ times, \Cref{eq:lower_bound_nb_wins} implies
\begin{equation*}
    \sum_{\tau=t_\ell}^{{\mathfrak t}_{\ell,i}} F_{t,i} \leq \abs{\set{ \tau\in[t_\ell,{\mathfrak t}_{\ell,i}]: i_{\tau}=i }} + N_{\ell,i}\leq \frac{4}{3}\abs{\set{ \tau\in[t_\ell,{\mathfrak t}_{\ell,i}]: i_{\tau}=i }}.
\end{equation*}
Therefore,
\begin{equation*}
    \frac{ \abs{\set{ \tau\in[t_\ell,{\mathfrak t}_{\ell,i}]: i_{\tau}=i }} }{{\mathfrak t}_{\ell,i}-t_\ell+1} \geq \frac{3}{4({\mathfrak t}_{\ell,i}-t_\ell+1)} \sum_{\tau=t_\ell}^{{\mathfrak t}_{\ell,i}} F_{t,i} \geq \frac{q_{\ell,i}}{4}.
\end{equation*}
Again, the last inequality used \Cref{eq:constraint_F}. As a result, the estimation phase of agent $i$ during epoch $\ell$ ends at or before $\mathfrak t_{\ell,i}$.
\end{proof}

Using this claim, the number of audits that we spend on agent $i$ \textit{in their estimation phase} during epoch $\ell$ then satisfies
\begin{align*}
    A_{\ell,i}:=\sum_{t\in E_\ell} o_t \1[i_t=i] \1[\hat q_{t,i}=0] 
    &\leq \abs{\set{ \tau\in[t_\ell,{\mathfrak t}_{\ell,i}]: i_{\tau}=i }}.
\end{align*}

We make the following claim regarding the expectation of all alive agents' $A_{\ell,i}$'s:
\begin{claim}
The $A_{\ell,i}$ above satisfies $\E[\sum_{i\in\mA_{t_\ell}} A_{\ell,i}\mid \mH_{t_\ell}]=\O(K+\E[\sum_{i\in\mA_{t_\ell}} \sum_{t\in E_\ell} D_{t,i}\mid \mH_{t_\ell}])$.
\end{claim}
\begin{proof}
For an alive agent $i\in \mA_{t_\ell}$, to control $\abs{\set{ \tau\in[t_\ell,{\mathfrak t}_{\ell,i}]: i_{\tau}=i }}$, we additionally define $\tilde {\mathfrak t}_{\ell,i}$ as the first round when agent $i$ wins for the $\max(1,4M_{\ell,i}, 3N_{\ell,i})$-th time. If no such $t$ exists, we set $\tilde {\mathfrak t}_{\ell,i}=t_{\ell+1}-1$. The definition of $\tilde{\mathfrak t}_{\ell,i}$ is identical to that of $\mathfrak t_{\ell,i}$ except for the removal of \Cref{eq:constraint_F}.

If $\tilde {\mathfrak t}_{\ell,i}$ already satisfied the constraint from \Cref{eq:constraint_F}, then $\mathfrak t_{\ell,i}=\tilde{\mathfrak t}_{\ell,i}$ and we obtain directly
\begin{equation}\label{eq:case_1}
    A_{\ell,i}\leq \abs{\set{ \tau\in[t_\ell,\tilde {\mathfrak t}_{\ell,i}]: i_{\tau}=i }} = \max(1,4M_{\ell,i}, 3N_{\ell,i}).
\end{equation}
We suppose from now on that this is not the case. Applying \Cref{eq:upper_bound_nb_wins}, the upper bound on $\lvert \{t\in [t_\ell,\tau]\mid i_t=i\}\rvert$ for any $\tau\in E_\ell$, to the $\mathfrak t_{\ell,i}$ defined in \Cref{eq:constraint_F}, we have
\begin{equation}\label{eq:case_2a}
A_{\ell,i}\leq M_{\ell,i} +\sum_{\tau=t_\ell}^{{\mathfrak t}_{\ell,i}} F_{t,i}  \leq   M_{\ell,i} + 3q_{\ell,i}(\mathfrak t_{\ell,i}-t_\ell+1).
\end{equation}

Let $k_0$ be the integer such that $2^{k_0}\leq \mathfrak t_{\ell,i}-t_\ell+1 <2^{k_0+1}$. Since $\tilde {\mathfrak t}_{\ell,i}$ does not satisfy \Cref{eq:constraint_F}, we either have
\begin{equation}\label{eq:upper_bound_power_2}
    \frac{1}{ 2^{k_0+1}}\sum_{\tau=t_\ell}^{t_\ell + 2^{k_0+1}-1} F_{t,i} \geq \frac{1}{2(\tilde{\mathfrak t}_{\ell,i} -t_\ell + 1)}\sum_{\tau=t_\ell}^{\tilde{\mathfrak t}_{\ell,i}} F_{t,i} > \frac{3}{2}q_{\ell,i},
\end{equation}
if $t_\ell+2^{k_0+1}-1\in E_l$, or otherwise,
\begin{equation}\label{eq:lower_bound_power_2}
    \frac{1}{ 2^{k_0}}\sum_{\tau=t_\ell}^{t_\ell + 2^{k_0}-1} F_{t,i} < \frac{2}{\tilde{\mathfrak t}_{\ell,i} -t_\ell + 1}\sum_{\tau=t_\ell}^{\tilde{\mathfrak t}_{\ell,i}} F_{t,i} < \frac{2}{3}q_{\ell,i}.
\end{equation}

Now define $k_{\ell,i}$ as the smallest integer such that for all $k\geq k_{\ell,i}$, if $t_\ell+2^k-1\in E_\ell$ then
\begin{equation}\label{eq:def_k_il}
    \frac{1}{2^k}\sum_{\tau=t_\ell}^{t_\ell+2^k-1} F_{t,i} \in \sqb{\frac{2}{3}q_{\ell,i}, \frac{3}{2}q_{\ell,i} }.
\end{equation}
\Cref{eq:upper_bound_power_2,eq:lower_bound_power_2} show that $k_{\ell,i}>k_0$. Hence, furthering \Cref{eq:case_2a} gives $A_{\ell,i} \leq M_{\ell,i} + 6 q_{\ell,i} 2^{k_{\ell,i}}$.
Together with \Cref{eq:case_1} this shows that in all cases one has
\begin{equation}\label{eq:full_case}
    A_{\ell,i} \leq 1 + 4M_{\ell,i} + 3N_{\ell,i} + 6 q_{\ell,i} 2^{k_{\ell,i}}.
\end{equation}





We now bound $\E[2^{k_{\ell,i}}\mid\mH_{t_\ell}]$.
Note that conditionally on the history $\mH_{t_\ell}$ at the beginning of epoch $E_\ell$, the sequence of $F_{t,i}$ is an independent sequence of Bernoulli $\text{Ber}(q_{\ell,i})$. Hence, for any $k\geq 3+\log_2(1/q_{\ell,i})$, Chernoff's bound gives
\begin{equation*}
    \Pr\{k_{\ell,i} \geq k\} \leq \sum_{k'\geq k}\Pr\left \{\frac{1}{2^{k'}}\sum_{t=t_\ell}^{t_\ell+2^{k'}-1} F_{t,i} \notin \sqb{\frac{2}{3}q_{\ell,i},\frac{3}{2}q_{\ell,i}}\right \} \leq 2\sum_{k'\geq k}e^{-2^{k'}q_{\ell,i}/18} \leq 6e^{-2^{k}q_{\ell,i}/18}.
\end{equation*}

As a result,
\begin{equation*}
    \E[2^{k_{\ell,i}}\mid\mH_{t_\ell}] \leq \frac{8}{q_{\ell,i}} + \sum_{k\geq 3+\log_2(1/q_{\ell,i})}2^k \Pr \{k_{\ell,i} = k\} \leq \frac{98}{q_{\ell,i}}.
\end{equation*}

Plugging in the definitions of $M_{\ell,i}=1+\sum_{t\in E_\ell}\mathbbm{1}[i_t=i] (\bigvee_{j\in \mA_{t_\ell}\setminus \{i\}} D_{t,j})$ (from \Cref{eq:upper_bound_nb_wins}) and $N_{\ell,i}=1+\sum_{t\in E_\ell} D_{\ell,i}$ (from \Cref{eq:lower_bound_nb_wins}) into \Cref{eq:full_case} and observing that
\begin{equation}\label{eq:useful_inequality}
    \sum_{i\in \mA_{t_\ell}} M_{\ell,i} \leq K + \sum_{t\in E_\ell} \max_{j\in\mA_{t_\ell}} D_{t,j}\paren{ \sum_{i\in \mA_{t_\ell}} \1[i_{\tau}=i] } \leq K+\sum_{t\in E_\ell} \sum_{i\in \mA_{t_\ell}} D_{t,i},
\end{equation}
we get
\begin{equation*}
    \E\sqb{\sum_{i\in\mA_{t_\ell}} A_{\ell,i} \middle \vert \mH_{t_\ell}} \leq 600 K + 7 \E\sqb{\sum_{i\in \mA_{t_\ell}}\sum_{t\in E_\ell} D_{t,i} \middle \vert \mH_{t_\ell}},
\end{equation*}
which completes the proof of this claim.
\end{proof}

Together with the upper bound on $\E[\sum_{i\in \mA_{t_\ell}} \sum_{t\in E_\ell} D_{t,i}\mid \mH_{t_\ell}]$ from \Cref{lem:number of marking down bound formal}, we get
\begin{align*}
\E\sqb{\sum_{t=1}^T o_t \1[\hat q_{t,i}=0] }=\E\sqb{\sum_{\ell=1}^K \sum_{i\in\mA_{t_\ell}} A_{\ell,i} \middle \vert \mH_{t_\ell}}&\leq 600K^2 + 14\frac{K^3}{c}=\O\paren{\frac{K^3}{c}}.
\end{align*}

\paragraph{Incentive-Compatible Phase.} We now turn to the second phase. As before, we fix an epoch $\ell\in[K]$, a start time for the epoch $t_\ell\in[T]$ and a history $\mH_{t_\ell}$. We fix $i\in\mA_{t_\ell}$ and consider the incentive-compatible phase in which the estimation process already constructed an estimate $\hat q_{t,i}=\hat q_{\ell,i}\leq 4q_{t,i}$, $\forall t\in E_\ell$. For convenience, let us denote by ${\mathfrak t}_{\ell,i}^{ic}$ the start time of the incentive-compatible phase for $i$. During this phase, if agent $i$ wins, they are audited with probability $\hat p_{t,i}=\min(\frac{8K^2}{(T-t)\hat q_{\ell,i}c},1)$. Thus, the expected number of audits we spend on agent $i$ during their $\ell$-th incentive-compatible phase is (all expectations below are taken conditionally on $\mH_{t_{\ell,i}^{ic}}$)
\begin{align*}
B_{\ell,i}&:=\E\sqb{\sum_{t\in E_\ell}o_t \1[i_t=i] \1[\hat q_{t,i}>0] } = \E\sqb{\sum_{t\in E_\ell,t\geq {\mathfrak t}_{\ell,i}^{ic}}\1[i_t=i] \hat p_{i,t} \middle \vert \mH_{{\mathfrak t}_{\ell,i}^{ic}}}\\
&\leq \E\sqb{\sum_{t\in E_\ell,t\geq {\mathfrak t}_{\ell,i}^{ic}}(F_{t,i} + (1-F_{t,i})\1[i_t=i]) \hat p_{i,t} }\overset{(a)}{\leq} \E\sqb{\sum_{t\in E_\ell,t\geq {\mathfrak t}_{\ell,i}^{ic}}F_{t,i}\hat p_{i,t} } + \E[M_{\ell,i} ]\\
&\overset{(b)}{\leq } \frac{8K^2 }{c}\E\sqb{\sum_{t\in E_\ell,t\geq {\mathfrak t}_{\ell,i}^{ic}}\frac{q_{\ell,i}}{(T-t)\hat q_{\ell,i}} } + \E[M_{\ell,i} ],
\end{align*}
where \textit{(a)} uses the same arguments as in \Cref{eq:upper_bound_nb_wins} to bound by $M_{\ell,i}$ the number of times $i$ wins when they would not have won had all agents reported truthfully, and \textit{(b)} uses the fact that $F_{t,i}$ for $t\in E_\ell$ form an \textit{i.i.d.} sequence of $\text{Ber}(q_{\ell,i})$ as well as the definition of $\hat p_{t,i}$ in \Cref{line:check prob} of \Cref{alg:mechanism}. 

Since agent $i$ uses the flagging strategy ${\bm\pi}^{\ast,f}$, they ensure $\hat q_i\geq q_i/4$. 
Summing over all epochs $\ell\in[K]$ and $i\in\mA_{t_\ell}$ then gives
\begin{align*}
&\quad \E\sqb{\sum_{t=1}^T o_t \1[\hat q_{t,i}>0]}=\E\sqb{\sum_{\ell=1}^K B_{\ell,i}} \\
&\le \frac{32K^3}{c} \E\sqb{\sum_{t=1}^T \frac{1}{T-t} } + \E\sqb{\sum_{\ell\in[K]}\sum_{i\in\mA_{t_\ell}}M_{\ell,i}} \\
& =\mathcal O\paren{\frac{K^3}{c}\log T}.
\end{align*}
Here, we used the same arguments as in the estimation phase to bound the term summing the quantities $M_{\ell,i}$, that is \Cref{eq:useful_inequality} and \Cref{lem:number of marking down bound formal}.

Summing up the expected number of audits in the estimation and incentive-compatible phases completes the proof.
\end{proof}


%!TEX root=main.tex


\section{Simple Mechanism Auditing with Fixed Probability}\label{sec:appendix fixed-p}

The mechanism $\bm M^0(p)$ which audits with fixed a-priori probability $p\in(0,1]$ is formally defined in \Cref{alg:simple_mechanism}. We now prove \Cref{thm:simple mech main theorem} which bounds its regret and expected number of audits.

\begin{algorithm}[!t]
\caption{Simple Mechanism Auditing with Fixed Probability ($\bm M^0(p)$)}
\label{alg:simple_mechanism}
\begin{algorithmic}[1]
\Require{Number of rounds $T$, number of agents $K$, probability of auditing $p$}
\Ensure{Allocations $i_1,i_2,\ldots,i_T\in \{0\}\cup [K]$}
\State Initialize the set of alive agents $\mA_1=[K]$.
\For{$t=1,2,\ldots,T$}
\State Collect reports $v_{t,i}\in [0,1]$ from $i\in \mA_t$ and allocate to agent $i_t=\argmax_{i\in \mA} v_{t,i}$.
\State Decide whether to audit the winner by $o_t\sim \text{Ber}(p)$. Observe outcome $o_t u_{t,i_t}$.
\If{$o_t=1$ \textbf{and} $o_t u_{t,i_t} \neq v_{t,i_t}$} 
\State Eliminate agent $i_t$ permanently by updating $\mA_{t+1}\gets \mA_t\setminus \{i_t\}$.
\Else
\State Everyone stays alive, that is, set $\mA_{t+1}\gets \mA_t$.
\EndIf
\EndFor
\end{algorithmic}
\end{algorithm}

\begin{proof}[Proof of \Cref{thm:simple mech main theorem}]
Fix an agents' joint strategy PBE $\bm \pi$ under the mechanism $\bm M^0(p)$ defined in \Cref{alg:simple_mechanism}. Since the mechanism always independently audits with probability $p$,
\begin{equation*}
    \mathcal B_T(\bm \pi,\text{\Cref{alg:simple_mechanism}}) = \sum_{t=1}^T \Pr\{o_t=1\} = pT.
\end{equation*}

Next, we fix an agent $i\in [K]$ and prove a lower bound on its V-function $V_i^{\bm\pi}(\mH_1;\text{\Cref{alg:simple_mechanism}})$ obtained under joint strategy $\bm \pi$ and mechanism \Cref{alg:simple_mechanism}. Denote by $\textbf{truth}_i$ the truthful strategy of agent $i$ which always honestly reports $v_{t,i}=u_{t,i}$. Since $\bm\pi$ is a PBE, we have
\begin{align*}
    &\quad V_i^{\bm\pi}(\mH_1;\text{\Cref{alg:simple_mechanism}}) \geq V_i^{\textbf{truth}_i\circ \bm{\pi}_{-i}}(\mH_1;\text{\Cref{alg:simple_mechanism}})\\
    &= \E_{\textbf{truth}_i\circ \bm{\pi}_{-i}}\sqb{ \sum_{t=1}^T u_{t,i} \1\sqb{i_t=i}  }\\
    &\geq \E_{\textbf{truth}_i\circ \bm{\pi}_{-i}}\sqb{ \sum_{t=1}^T u_{t,i} \mathbbm{1}[i\in \mA_t] \paren{\1\sqb{u_{t,i}>u_{t,j},\forall j\in \mA_t\setminus \{i\}} -  \1[i_t\neq i]\1\sqb{v_{t,i_t}>u_{t,i_t}}}}.
\end{align*}

Note that under $\textbf{truth}_i$, agent $i$ is never eliminated. Hence, the indicator $\mathbbm{1}[i\in \mA_t]=1$ for all $t\in [T]$.
We now define the per-round fair share of agent $i$ as
\begin{equation*}
    \mu_i := \E\sqb{  u_i \1\sqb{u_i>u_j,\forall j\ne i}  }.
\end{equation*}
Since $\mA_t\subseteq [K]$, we know $(u_{t,i}>u_{t,j},\forall j\ne i)$ infers $(u_{t,i}>u_{t,j},\forall j\in \mA_t\setminus \{i\})$ for any $t\in [T]$ and $i\in \mA_t$. Therefore, we can further lower bound the V-function as
\begin{align*}
    V_i^{\bm\pi}(\mH_1;\text{\Cref{alg:simple_mechanism}}) &\geq \mu_i T - \sum_{j\neq i}\E \sqb{\sum_{t=1}^T \1[i_t=j] \1[v_j>u_j]}.
\end{align*}

Now we fix any opponent $j\neq i$. By construction of the mechanism, every round $t$ in which agent $j$ wins ($i_t=j$) by marking up ($v_j>u_j$), with probability $p$ independent of the past history, they are eliminated. Hence, $\sum_{t=1}^T \1[i_t=j] \1[v_j>u_j]$ is stochastically dominated by a geometric variable $\text{Geo}(p)$. Applying this to all $j\ne i$'s yields
\begin{equation*}
V_i^{\bm\pi}(\mH_1;\text{\Cref{alg:simple_mechanism}}) \geq \mu_i T - \frac{K-1}{p},\quad \forall i\in [K].
\end{equation*}

Given the lower bounds on V-functions, we are now ready to control the regret as
\begin{align*}
\mathcal R_T(\bm \pi,\text{\Cref{alg:simple_mechanism}}) &= T\E\sqb{\max_{i\in[K]}u_i} - \sum_{i\in[K]}V_i^{\bm\pi}(\mH_1;\text{\Cref{alg:simple_mechanism}}) \\ &\le T\sum_{i\in[K]}\mu_i -\sum_{i\in[K]}\left (\mu_i T - \frac{K-1}{p}\right ) = \frac{K(K-1)}{p}.
\end{align*}
This ends the proof.
\end{proof}

\section{Lower Bounds on Regret and Number of Audits}\label{sec:lower bounds}

In this section, we prove the lower bounds from \Cref{thm:lower_bounds}. We start by proving the lower bound $\Omega(K)$ on the regret, then prove the lower bound $\Omega(1)$ on the number of audits for having low regret.

\begin{theorem}[Lower Bound on Regret]\label{thm:lower_bound_regret}
Let $K\geq 2$ and $T\geq K$. Then, there exists $K$ utility distributions $\mV_1,\mV_2,\ldots,\mV_K$ over $[0,1]$, such that for any central planner mechanism $\bm M$ and corresponding agents' joint strategy PBE $\bm \pi=(\pi_{t,i})_{t\in [T],i\in[K]}$, we must have
\begin{equation*}
\mathcal R_T(\bm \pi,\bm M) \geq \frac{K-1}{64}=\Omega(K).
\end{equation*}
\end{theorem}
\begin{proof}
    We consider the following distributions: $\mV_1 = \delta_{2/3}$, that is, $u_1=\frac{2}{3}$ almost surely, and for all $i\geq 2$, we have $\mV_i \sim \frac{1}{3} + \frac{2}{3}\text{Ber}(\frac{\alpha}{T})$ where $\alpha=\frac{1}{8}$, that is, $u_i=1$ with probability $\alpha/T$ and $u_i=\frac{1}{3}$ otherwise.
    Now fix any PBE $\bm\pi$ corresponding to these utility distributions. We fix an agent $i\geq 2$ and aim to show that
    \begin{equation}\label{eq:bound_error_i}
        R_i:=\E_{\bm\pi}\sqb{\sum_{t=1}^T \1[u_{t,i}=1]\1[i_t\neq i] + \1[u_{t,i}=1/3]\1[i_t=i]} \geq \frac{\alpha}{2}.
    \end{equation}
    
    Suppose by contradiction that \Cref{eq:bound_error_i} does not hold. We can then bound the expected gain of agent $i$ as follows:
    \begin{equation}\label{eq:upper_bound_gain_i}
        V_i^{\bm\pi}(\mH_1) \leq \E\sqb{\sum_{t=1}^T \frac{1}{3}\1[u_{t,i}=1/3]\1[i_t=i] + \1[u_{t,i}=1]} \leq \alpha+\frac{R_i}{3}\leq \frac{7\alpha}{6}.
    \end{equation}
    On the other hand, we can also bound the expected gain of agent $i$ conditional on the event $\mE_i\colon \exists t\in[T]\text{ s.t. } u_{t,i}=1$ which excludes the case when agent $i$ always has utility $1/3$.
    \begin{align*}
        \E\sqb{\paren{\sum_{t=1}^T u_{t,i}\1[i_t=i]}\1[\mE_i]} &\geq \E\sqb{\paren{\sum_{t=1}^T \1[u_{t,i}=1]}\1[\mE_i]}- R_i\\
        &= \E\sqb{\sum_{t=1}^T \1[u_{t,i}=1]}- R_i =\alpha-R_i \geq \frac{\alpha}{2}.
    \end{align*}
    As a result,
    \begin{equation*}
        \E\sqb{\sum_{t=1}^T u_{t,i}\1[i_t=i] \middle \vert \mE_i} \geq \frac{\alpha}{2\Pr\{\mE_i\}} = \frac{\alpha}{2(1-(1-\alpha/T)^T)}\geq \frac{1}{2}.
    \end{equation*}
    We now define a new strategy $\tilde{\bm \pi}_i$ for agent $i$ which proceeds as follows: sample a "fake" sequence $\tilde{\bm u}_i=(\tilde u_{t,i})_{t\in[T]}$ from the distribution $\mV_i^{\otimes T}\mid\mE_i$, then run strategy $\pi_i$ throughout the whole allocation process by bidding as if the true utility of agent $i$ as $\tilde{\bm u}$. Then,
    \begin{align*}
        V_i^{(\bm\pi_{-i},\tilde\pi_i)}(\mH_1) &\geq \frac{1}{3}\Pr_{(\bm\pi_{-i},\tilde\pi_i)}\{\exists t\in[T]\text{ s.t. }i_t=i\} \geq \frac{1}{3} \E_{\bm \pi}\sqb{\sum_{t=1}^T u_{t,i}\1[i_t=i] \middle \vert \mE_i} \geq \frac{1}{6} > V_i^{\bm\pi}(\mH_1),
    \end{align*}
    where in the last inequality we used \Cref{eq:upper_bound_gain_i} and the fact that $\alpha<1/7$.
    This contradicts the fact that $\bm\pi$ is a PBE. In summary, we proved that \Cref{eq:bound_error_i} holds for all $i\geq 2$.

    We can then bound the regret of the mechanism at the PBE $\bm\pi$. First, note that any time $t$ for which $u_{t,i_t}=1/3$ induces a regret $1/3$ since agent $1$ always had utility $1$. Second, if at time $t$, there is a unique agent $i\geq 2$ which has utility $u_{t,i}=1$, the mechanism also incurs a regret at least $1/3$ if it does not allocate to agent $i$, that is if $i_t\neq i$. Formally,
    \begin{align*}
        \mathcal R_T(\bm \pi,\bm M) &\geq \E_{\bm u,\bm\pi}\sqb{\sum_{t=1}^T \sum_{i=2}^K\frac{1}{3}\paren{\1[u_{t,i}=1/3]\1[i_t=i]  + \1[u_{t,i}=1]\1[i_t\neq i]\1[u_{j,t}=1/3,\forall j\notin\{1,i\}]} }\\
        &\geq \frac{1}{3}\sum_{i=2}^K R_i - \frac{1}{3}\E\sqb{\sum_{t=1}^T \paren{\sum_{i=2}^K\1[u_{t,i}=1]}\1\sqb{\sum_{i=2}^K\1[u_{t,i}=1] >1}}\\
        &\geq \frac{(K-1)\alpha}{6} - \frac{T}{3}\paren{\frac{K\alpha}{T}-K\frac{\alpha}{T}\paren{1-\frac{\alpha}{T}}^{K-1}}\\
        &\geq \frac{(K-1)\alpha}{6}\paren{1-2\alpha},
    \end{align*}
    where in the last inequality, we used the fact that $1-(1-\alpha/T)^{K-1}\leq 1- (1-\alpha/T)^{T-1} \leq 1-e^{-\alpha}\leq \alpha$. Plugging in the utility $\alpha=1/8$ gives the desired result.
\end{proof}

\begin{theorem}[Lower Bound on Tradeoff Between Regret and Number of Audits]\label{thm:lower_bound_nb_checks}
Let $K\geq 2$ and $T\geq K$. Then, there exists $K$ utility distributions $\mV_1,\mV_2,\ldots,\mV_K$ over $[0,1]$, such that for any central planner mechanism $\bm M$ and corresponding agents' joint strategy PBE $\bm \pi=(\pi_{t,i})_{t\in [T],i\in[K]}$,
\begin{equation*}
    \mathcal B_T(\bm \pi,\bm M)\leq c_1\Longrightarrow \mathcal R_T(\bm \pi,\bm M)\ge c_2\sqrt {\frac{T}{\log T}},
\end{equation*}
where $c_1,c_2>0$ are some universal constants.
\end{theorem}
\begin{proof}
    We consider similar distributions as in the proof of \Cref{thm:lower_bound_regret}, but using only the first two agents. Precisely, we set $\mV_1=\delta_{2/3}$,  $\mV_2\sim\delta_{\frac{1}{3}} + \frac{2}{3}\text{Ber}(p)$, where $p=\frac{1}{3}$, and $\mV_i=\delta_{1/4}$ for $i\geq 3$. That is, $u_1=\frac{2}{3}$ always, $u_2$ equals $\frac{1}{3}$ or $1$ with equal probability $\frac{1}{2}$, and $u_i=\frac{1}{4}$ for all $i\geq 3$ (so their fair shares are all 0 as $u_i<u_1$ and $u_i<u_2$ \textit{a.s.}). By the revelation principle, we assume without loss of generality that under this specific utility distributions setup $\{\mV_i\}_{i\in [K]}$ and mechanism $\bm M$ truthful reporting $\textbf{truth}$ is the considered PBE. 

    By design, the central planner incurs a constant regret $\frac 13$ whenever they do not allocate the resource to agent $2$ if they had utility $1$. Also, because agent $1$ always has utility $\frac 23$, we have
    \begin{equation}\label{eq:lower_bound_regret}
        \mathcal R_T(\textbf{truth},\bm M) = \frac{1}{3}\E\sqb{\sum_{t=1}^T \1[u_{t,2}=1]\1[i_t\neq 2] + \1\sqb{u_{t,2}=\frac{1}{3} }\1[i_t\neq 1]}.
    \end{equation}

    From now, we focus on the agent $i=2$. For simplicity, we abbreviate the expectation
    \begin{equation*}
    \E_{\text{generate \textit{i.i.d.} samples }\bm u_i=(u_{t,i})_{t\in [T]}\text{ from }\mV_i} \left [\E_{\text{agents follow joint strategy }\bm \pi}\left [X\middle \vert \text{agent $i$ has true utilities }f(\bm u_i)\right ]\right ]
    \end{equation*}
    as $\E_{f(\bm u_i),\bm \pi}[X]$. We similarly define $\Pr_{f(\bm u_i),\bm \pi}[X]$. For example, if agent $i$'s true utilities are \textit{i.i.d.} samples from $\mV_i$ and all agents follow $\textbf{truth}$, then we write $\E_{\bm u_i,\textbf{truth}}$ as the expectation operator.
    We start by upper bounding the gain of agent $i$ under truthful reporting as follows.
    \begin{align}
        V_i^{\textbf{truth}}(\mH_1;\bm M) &\leq \E_{\bm u_i,\textbf{truth}} \sqb{\sum_{t=1}^T \1[u_{t,i}=1] + \frac{1}{3} \1\sqb{u_{t,i}=\frac{1}{3}}\1[i_t=i] } \notag\\
        &= Tp + \frac{1}{3} \E_{\bm u_i,\textbf{truth}} \sqb{\sum_{t=1}^T  \1\sqb{u_{t,i}=\frac{1}{3}}\1[i_t=i] }. \label{eq:upper_bound_truth}
    \end{align}
    
    We next construct an alternative strategy $\bm\pi_i$ for agent $i$ as follows.     
    In round $t\in[T]$, after observing their utility $u_{t,i}$,
    \begin{itemize}
    \item if $u_{t,i}=1$, then report $v_{t,i}=1$ truthfully as in \textbf{truth},
    \item otherwise, \textit{w.p.} $\epsilon=\frac{1}{\sqrt{T\log T}}$ independently report $v_{t,i}=1$ (\textit{i.e.}, pretend that they had true utility $u_{t,i}=1$), and otherwise report $v_{t,i}=\frac 13$ as in \textbf{truth}. In particular, for $T$ sufficiently large, we have $p+\epsilon\leq \frac{1}{2}$.
    \end{itemize}
    As a side note, the resource allocation setup defined in \Cref{sec:setup} allows the central planner to ask further information from the agents than only their utility report (see Step~\ref{item:additional_questions}). To these questions at round $t\in[T]$, $\bm\pi$ answers identically as what the PBE strategy $\textbf{truth}$ would have replied had their true utility sequence been the reported $(v_{\tau, i})_{\tau\leq t}$, and for the same public history. 
    In summary, the strategy $\bm\pi_i$ corresponds to using the PBE strategy $\textbf{truth}$ for the biaised utilities $(v_{t,i})_{t\in[T]}$. Note that by construction, these form an i.i.d. sequence of distribution $\tilde {\mathcal U}_i\sim\frac{1}{3} + \frac{2}{3}\text{Ber}(p+\alpha\sqrt p)$. 
    
    Importantly, the allocation process is identical under the following two scenarios:
    \begin{enumerate}
        \item agent $i$ had true utilities $\bm u_i=(u_{t,i})_{t\in [T]}$ but used strategy $\bm\pi_i$ to report $\bm v_i=(v_{t,i})_{t\in [T]}$, while all other agents follow $\textbf{truth}_{-i}$ (this corresponds to expectation operator $\E_{\bm u_i,\bm \pi_i\circ \textbf{truth}_{-i}}$), and
        \item agent $i$ had true utilities as the aforementioned $\bm v_i=(v_{t,i})_{t\in [T]}$ and used the PBE strategy $\textbf{truth}_i$, while all other agents still follow $\textbf{truth}_{-i}$ (we denote this case by $\E_{\bm v_i,\textbf{truth}}$),
    \end{enumerate}
    unless the central planner audits $i$ in a round $t$ where $v_{t,i}\neq u_{t,i}$ in the first scenario. Formally, for round $t\in[T]$, we denote by $F_{t,i}=\mathbbm{1}[u_{t,i}=\frac 13,v_{t,i}=1]$ the indicator that agent $i$ flips their report from $\frac 13$ to $1$ under strategy $\pi_{t,i}$. The two scenarios collide if $F_{t,i} \1[i_t=i] o_t=0$ for all $t\in[T]$.
    
    Putting everything together, we obtain
    \begin{align}
        V_i^{\bm\pi_i\circ \textbf{truth}_{-i}}(\mH_1;\bm M) &= \E_{\bm u_i,\bm \pi_i\circ \textbf{truth}_{-i}} \sqb{\sum_{t=1}^T u_{t,i} \mathbbm{1}[i_t=i] } \notag \\
        &\geq \E_{\bm v_i,\textbf{truth}} \sqb{ \1\left [\bigwedge_{t\in [T]} F_{t,i}\1[i_t=i]o_t=0\right ]\sum_{t=1}^T u_{t,i} \mathbbm{1}[i_t=i] } \notag  \\
        &\geq \E_{\bm v_i,\textbf{truth}} \sqb{\sum_{t=1}^T u_{t,i} \mathbbm{1}[i_t=i] } - T\E_{\bm v_i,\textbf{truth}}\sqb{\sum_{t=1}^T F_{t,i}\1[i_t=i]o_t}. \label{eq:bound_V_i_deviation_1}
    \end{align}

    We further bound each term of the last inequality as follows:
    \begin{align}
        \E_{\bm v_i,\textbf{truth}} \sqb{\sum_{t=1}^T u_{t,i} \mathbbm{1}[i_t=i] } &\geq \E_{\bm v_i,\textbf{truth}} \sqb{\sum_{t=1}^T u_{t,i} \mathbbm{1}[v_{t,i}=1] - \mathbbm{1}[v_{t,i}=1]\1[i_t\neq i] } \notag \\
        &=Tp +\frac{\epsilon T}{3}  - \E_{\bm v_i,\textbf{truth}} \sqb{\sum_{t=1}^T  \mathbbm{1}[v_{t,i}=1]\1[i_t\neq i] }. \label{eq:bound_V_i_deviation_2}
    \end{align}
    To bound the second term, note that in a round $t\in[T]$ in which the central planner allocated the resource to $i_t=i$, when the central planner makes the decision to audit agent $i$ or not, they only have access to the past history and the reports at round $t$, and $i_t$. Conditioning on these, the value of $u_{t,i}$ is in fact only dependent on the report $v_{t,i}$ as
    \begin{equation*}
        \Pr_{\bm v_i,\textbf{truth}}\set{ u_{t,i}=\frac{1}{3} \middle \vert \mH_t,\bm v_t, i_t, o_t} = \Pr\set{ u_{t,i}=\frac{1}{3} \middle \vert  v_{t,i}} = \begin{cases}
            1 & v_{t,i}=\frac{1}{3},\\
            \frac{\epsilon}{p}  & v_{t,i}=1.
        \end{cases}
    \end{equation*}
    In the last inequality, we used the construction of the variable $v_{t,i}$. 
    As a result, 
    \begin{align}
         \E_{\bm v_i,\textbf{truth}}\sqb{\sum_{t=1}^T F_{t,i}\1[i_t=i]o_t} &= \E_{\bm v_i,\textbf{truth}}\sqb{\sum_{t=1}^T \1[v_{t,i}=1,i_t=i]o_t \1\sqb{u_{t,i}=\frac 13}} \notag \\
         &= \frac{\epsilon}{p} \E_{\bm v_i,\textbf{truth}}\sqb{\sum_{t=1}^T \1[v_{t,i}=1,i_t=i]o_t}. \label{eq:bounding_nb_checks_for_catch}
    \end{align}
    
    In the remaining, we compare the expectation operators $\E_{\bm u_i,\textbf{truth}}$ and $\E_{\bm v_i,\textbf{truth}}$ which only differ in the fact that the true utility of agent $i$ was sampled i.i.d. following $\mathcal U_i$ and $\tilde{\mathcal U}_i$ respectively. That is,
    \begin{align}
    \E_{\bm u_i,\textbf{truth}}[X] &= \sum_{\bm z\in\set{\frac{1}{3},1}^T} \Pr_{\bm u\sim {\color{blue}{\mathcal U}_i^{\otimes T}}} \set{\bm u = \bm z} \E_{\textbf{truth}}[X\mid \text{agent $i$ has true utility }\bm z], \nonumber \\
    \E_{\bm v_i,\textbf{truth}}[X] &= \sum_{\bm z\in\set{\frac{1}{3},1}^T} \Pr_{\bm u\sim {\color{blue}\tilde{\mathcal U}_i^{\otimes T}}} \set{\bm u = \bm z} \E_{\textbf{truth}}[X\mid \text{agent $i$ has true utility }\bm z].\label{eq:comparing_expectations}
    \end{align}
    For any fixed $\bm z \in \set{\frac 13,1}^T$, denoting by $k_{\bm z}$ the number of components equal to $1$ within $\bm z$, we have
    
     \begin{align*}
         \log \frac{\Pr_{\bm u\sim \tilde{\mathcal U}_i^{\otimes T}} \set{\bm u = \bm z} }{\Pr_{\bm u\sim {\mathcal U}_i^{\otimes T}} \set{\bm u = \bm z} } &= k_{\bm z} \log\paren{\frac{p+\epsilon}{p}} +(T-k_{\bm z})\log \paren{\frac{1-p}{1-p-\epsilon}}\\
         &= T d_{KL}(p+\epsilon\parallel p) + (k_{\bm z}-T(p+\epsilon)) \log\paren{\frac{(p+\epsilon)(1-p)}{p(1-p-\epsilon)}}\\
         &\overset{(i)}{\leq} 1 + (k_{\bm z}-T(p+\epsilon)) \frac{\epsilon}{p(1-p-\epsilon)} \overset{(ii)}{\leq} 1 + \frac{2\epsilon}{p}   (k_{\bm z}-T(p+\epsilon)),
     \end{align*}
     where $d_{KL}(a\parallel b):=a\ln\frac{a}{b} + (1-a)\ln\frac{1-a}{1-b}$ is the binary KL divergence. In $(i)$ we used the fact that because $p+\epsilon\leq \frac{1}{2}$, one has $d_{KL}(p+\epsilon \parallel p)\leq \frac{\epsilon^2}{p}$ \citep[Lemma 16]{blanchard2024tight} as well as $\log(1+x)\leq x$ for $x>-1$. In $(ii)$ we used $p+\epsilon\leq \frac{1}{2}$. Next, by the Hoeffding bound, we have,
     \begin{align}
        \Pr_{\bm z\sim \tilde{\mathcal U}_i^{\otimes T}}\set{ \frac{\Pr_{\bm u\sim \tilde{\mathcal U}_i^{\otimes T}} \set{\bm u = \bm z} }{\Pr_{\bm u\sim {\mathcal U}_i^{\otimes T}} \set{\bm u = \bm z} } \leq e^{5}} &\geq \Pr_{\bm z\sim \tilde{\mathcal U}_i^{\otimes T}}\set{ k_{\bm z} \leq T(p+\epsilon) + \frac{2p}{\epsilon} } \notag \\
        &\geq 1-\exp \paren{-\frac{8 p^2 T}{\epsilon^2}} = 1-\frac{1}{T^2}.\label{eq:large_proba_similar_likelihood}
    \end{align}
    
    We denote by $\mathcal E_i$ the corresponding event. Plugging our estimates in \Cref{eq:comparing_expectations} then gives (which is the relationship between expectation operators)
    \begin{equation}\label{eq:complete_comparison_expectations}
        \E_{\bm v_i,\textbf{truth}} \leq e^{5} \E_{\bm u_i,\textbf{truth}} + \E_{\bm v_i,\textbf{truth}}\1[\neg \mathcal E_i].
    \end{equation}
    Using this identity twice within \Cref{eq:bound_V_i_deviation_2,eq:bounding_nb_checks_for_catch} then plugging the results within \Cref{eq:bound_V_i_deviation_1} yields
    \begin{align*}
        V_i^{\bm\pi_i\circ \textbf{truth}_{-i}}(\mH_1;\bm M) &\geq Tp + \frac{\epsilon T}{3} -\paren{\frac{\epsilon T^2}{p}+T}\Pr\set{\neg \mathcal E_i}- \\
        &\quad e^{5} \paren{ \frac{\epsilon T}{p} \E_{\bm u_i,\textbf{truth}}\sqb{\sum_{t=1}^T \1[v_{t,i}=1,i_t=i]o_t} +  \E_{\bm u_i,\textbf{truth}} \sqb{\sum_{t=1}^T  \mathbbm{1}[v_{t,i}=1]\1[i_t\neq i] }}.
    \end{align*}
    
    Next, by assumption \textbf{truth} is a PBE. Therefore, we must have $V_i^{\bm\pi_i\circ \textbf{truth}_{-i}}(\mH_1;\bm M) \leq V_i^{\textbf{truth}}(\mH_1;\bm M)$. Together with the previous lower bound for $V_i^{\bm\pi_i\circ \textbf{truth}_{-i}}(\mH_1;\bm M)$ and the upper bound on $V_i^{\textbf{truth}}(\mH_1;\bm M)$ from \Cref{eq:upper_bound_truth}, we obtain
    \begin{align*}
        &0\leq  V_i^{\textbf{truth}}(\mH_1;\bm M) - V_i^{\bm\pi_i\circ \textbf{truth}_{-i}}(\mH_1;\bm M) \\
        &\overset{(i)}{\leq} e^{5} \paren{ \frac{\epsilon T}{p} \E_{\bm u_i,\textbf{truth}}\sqb{\sum_{t=1}^T \1[v_{t,i}=1,i_t=i]o_t} +  \E_{\bm u_i,\textbf{truth}} \sqb{\sum_{t=1}^T \1\sqb{u_{t,i}=\frac{1}{3}}\1[i_t=i] + \mathbbm{1}[v_{t,i}=1]\1[i_t\neq i]  }} +\\
        &\qquad \frac{\epsilon}{p}+\frac{1}{T}  - \frac{\epsilon T}{3}\\
        &\leq e^5\paren{\frac{\epsilon T}{p}\mathcal B_T(\textbf{truth},\bm M) + 3\mathcal R_T(\textbf{truth},\bm M)} + 2 - \frac{\epsilon T}{3}.
    \end{align*}
    In $(i)$, we also used \Cref{eq:large_proba_similar_likelihood}. In $(ii)$, we used \Cref{eq:lower_bound_regret} and the definition of $\mathcal B_T(\textbf{truth},\bm M)$.

    As a result, if we have $B_T(\textbf{truth},\bm M) \leq \frac{p}{6e^5}$, the previous inequality implies
    \begin{equation*}
        \mathcal R_T(\textbf{truth},\bm M) \geq \frac{\epsilon T}{3}-2\geq \frac{\epsilon T}{4} = \frac{1}{4}\sqrt{\frac{T}{\log T}},
    \end{equation*}
    for $T$ sufficiently large. This ends the proof.
\end{proof}


\end{document}
