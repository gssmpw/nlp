\section{Related Work}
VA+VoicePop\cite{b2}, is a defense method against replay based spoofing attacks. It is built upon the idea, of using phonemes by using "pop-noises" to verify the claimed identity of the user\cite{b4}. This is done by identifying the unique way in which an individual pronounces certain types of phonemes. For example, when pronounced, the letter 'p' produces a distinct "pop" causing the human tongue and lips to push air out the mouth. The idea is that phonemes, within a given phrase, can be identified within audio. As a result, the intensity of these pop noises can be mapped out and used to authenticate a user. While a replay attack, in theory, has these pop noises present within the audio captured, playing these sounds over a speaker can make it harder for these pops to be recognized by the system. However, solely relying on this detection is not enough, and VA+VoicePop extends this work by using Gammatone Frequency Cepstral Coefficient (GFCC) to scrutinize voice samples further. While some replay attacks can be quickly rejected with the initial VA+VoicePop checks, some voice samples can confuse background noises for pop noises. To counter this issue, GFCC is used to filter down the audio to get the finer details of bursts of air being spoken into a microphone. This, in essence, captures the naturalness of a human speaking into a microphone.

In the VA+VoicePop system, once the passphrase has been correctly identified, the audio segment is then sent to the authentication server. The first part of this process is phoneme segmentation, where phonemes are extracted during the audio clip and where the pops occur in the audio. The audio is transformed via a Short Time Fourier Transform (STFT); this is used to capture the short burst of energy that is expelled at the time the pop noise occurs in the audio. Pop noises happen in the 100 Hz range, and the STFT helps narrow the low frequencies that occur below or equal to 100 Hz. GFCC feature extraction is used for replay attack detection. Pop noises from the previous stage are segmented at the start and end of the clip, with some overlap of background noise. This segment is then sent through a log function to normalize any spikes in audio and smoothens the shape of the energy bursts. A mean of the overall magnitudes is calculated to represent one of the feature vectors to the Support Vector Machines (SVM). In addition, two other features are calculated: the timing changes from pop noise to background ($\delta_1$) and the sudden shift in energy in the recording ($\delta_2$). $\delta_1$  indicates how the transition from pop noise to background (and vice versa) occurs by displaying the magnitude spike in energy being high during the beginning of the pop noise and the magnitude energy decreasing when going back to background noise. $\delta_2$, on the other hand, represents the rate of change that occurs during these two periods. A genuine pop noise will have both of these values at high rates as the voice captured can capture these more significant details because a microphone can pick up these harsh speech transitions naturally. A replay attack, on the other hand, will have much lower values for both deltas as the noise captured by the microphone cannot pick up on these details due to the audio sample being played from a speaker, which causes these details to be silenced or less pronounced. Once the GFCC features are extracted, a SVM classifier is used to identify if the audio sample is from a human or a replay attack.


\begin{figure}[htbp]
\centerline{\includegraphics[width=1.0\linewidth]{fig1.png}}
\caption{Overview of the VA+VoicePop architecture by Wang et al. \cite{b2}}
\label{fig:1}
\end{figure}



An overview of VA+VoicePop architecture is shown in \ref{fig:1}. While VA+VoicePop work reported a high accuracy of 93.5\% in detecting replay attacks, it has many limitations that require further investigation with logical attacks. In addition, their testing of the system only involved 18 participants, with a majority being men, which brings into question how well the dataset truly is over larger datasets. 

Our work addresses both these concerns, first by recreating VA+VoicePop system and testing it with large external datasets. Second, by deploying our SytheticPop attacks against it to assess robustness.