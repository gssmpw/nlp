Underwater remotely operated vehicles (ROVs) equipped with manipulator systems allow for advancements in ocean cleaning operations \cite{guimond_finding_plastic_2024}, deep sea specimen sampling \cite{mazzeo_marine_2022}, and human-robot collaboration in underwater environments \cite{feng_overview_2020}. 
Standard underwater manipulator systems are often controlled through teleoperation \cite{sivcev_underwater_2018}. 
Despite its success, reliance on teleoperation poses many challenges, primarily in operational cost and pilot training time.
These challenges associated with teleoperation have motivated research advances in autonomous underwater manipulation \cite{petillot_underwater_2019,micatka_grasp_volumes_2024}, where the operational cost is significantly lower.
Yet, many challenges remain in deploying autonomous mobile manipulator systems in environments such as the underwater domain.

Providing informative representations of the environment and objects that robots interact with is critical to achieving safe and reliable autonomy for robotic systems.
For this, a common approach is using sensors that provide rich environmental information to map the surrounding scene \cite{wang_real-time_2023, chen_monorun_2021, song2024turtlmap}. 
However, underwater sensing and perception suffer from several challenges, including image degradation \cite{mobley_light_1994, li_watergan_2017}, sensor noise \cite{song_uncertainty-aware_2023}, and uncertainty due to a lack of absolute position measurements \cite{rahman_svin2_2022}, which lead to challenges for robotics tasks that rely on accurate perceptual information.

A growing number of works have been successful in showing that incorporating uncertainty into perception systems, whether for detection \cite{kendall-what-uncertainties}, simultaneous localization and mapping~\cite{song_uncertainty-aware_2023}, or 3D reconstruction \cite{rosinol_probabilistic_2022,goli2023, ulusoy_patches_2016}, plays a crucial role in improving performance.
In addition to playing a crucial role in improving the robustness of perception systems, uncertainty plays a key role in embodied robotic tasks, such as planning and manipulation \cite{pmlr-v164-saund22a}. 

In this work, we model perceptual uncertainty in 3D reconstruction tasks to improve grasp selection and the manipulation of objects in underwater environments, using a method we call \textbf{P}erceptual \textbf{U}ncertainty for \textbf{G}rasp \textbf{S}election (PUGS). 
\begin{figure}[t!]
    \centering
    \includegraphics[width=1.0\linewidth]{figures/marc_figure_vis_new.pdf}
    \caption{Real-world underwater manipulation setup in the test tank at the University of Washington Applied Physics Laboratory (UW-APL). 
    The test system includes a Reach Robotics Bravo 7 electric manipulator and Trisect subsea stereo sensor.
    The gripper in red shows the grasp pose from an out-of-the-box grasp selection model \cite{player_real-time_2023}.
    The proposed method successfully leads to a more reliable grasping location, shown in green, by using perceptual uncertainty to weigh more favorable grasp locations.
    The visualizations are taken from evaluations of a real-world dataset and overlaid to match the photo showing the robot during regular operation.
    }
    \label{fig:hh101}
    \vspace{-3mm}
\end{figure}
We focus on modeling how uncertainty inherent from multi-view stereo can be leveraged for quantifying uncertainty in 3D reconstruction, specifically in representing occupancy in 3D space.
We then show that the uncertainty of the occupied regions can be a useful for improving existing grasp selection methods and guiding toward more reliable and robust grasp selection. We present the following contributions:
\begin{enumerate}
    \item We propose the construction of a fused occupancy field (FOF) informed by the uncertainty measurements and pose estimates. % from vSLAM methods,
    \item We develop a novel method to quantify the predictive uncertainty associated with occupancy in 3D space using probabilistic regression methods.
    \item We present an uncertainty fusion method to combine information from measurement and predictive uncertainty for modeling occupancy uncertainty.
    \item We provide an experimental evaluation in both simulation and real-world underwater environments to validate the proposed methods.
\end{enumerate}
