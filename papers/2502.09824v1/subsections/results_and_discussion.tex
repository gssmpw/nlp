\subsection{Simulation Results}
We use three metrics to evaluate our method in simulation experiments.
First, we evaluate grasp traversal success, which indicates whether the end effector can move to the grasp pose without colliding with the object of interest.
When evaluating this, we set an initial pose with a 0.2 m offset from the proposed grasp pose.
We count a successful grasp traversal if no collisions occur when moving between the initial and final grasp poses.
Secondly, we evaluate whether the gripper can close without collision after reaching the proposed grasp pose. 
We count a success if the gripper successfully closes with part of the object inside the convex hull of the closed gripper.
Lastly, we evaluate whether, after closing the gripper, the arm can pick up the object 0.1 m above the position and hold it there for 5 seconds.
% We note that these metrics are cascading: If the first metric fails, we also count the second and third as failures.

Table \ref{tab:quant_res_counting} shows the simulation experiments' results.
We point to the results in the partial and noisy reconstruction test setup.
TSGrasp consistently fails for partial reconstructions because the proposed grasp region is in an area with incomplete measurements.
This results in grasp candidates that consistently collide with the object.
Conversely, PUGS can determine that the edges of partially reconstructed geometry are unfavorable and guide the grasp to more reliable locations.
We show this occurring for the kettlebell and coffee mug in Fig. \ref{fig:sideways_fail}.
For the complete reconstruction, TSGrasp consistently outperforms PUGS.
This is expected, as the role of uncertainty as a signal to pull towards informative regions becomes less impactful when a complete reconstruction is obtained.

\subsection{Real World Results}
For the real-world results, we qualitatively evaluate the success of the grasp outputs for a kettlebell captured from different partial views.
Qualitative results from the grasp selection in real-world experiments are shown in Fig. \ref{fig:qualitative_results}.
We highlight that PUGS consistently leads the gripper pose to areas with more observations while retaining the grasping area's geometric feasibility.

The leftmost result in Fig. \ref{fig:qualitative_results} highlights the robustness that PUGS introduces when reasoning over very noisy pointclouds.
For this test, we tune the pointcloud filtering to be less aggressive by setting the outlier removal threshold to $\sigma_\text{thresh}=0.1$.
The proposed grasp pose from TSGrasp~\cite{player_real-time_2023} outputs the most confident grasp pose to be area captured due to noise, while PUGS successfully recovers to the kettlebell handle. 
This recovery of grasp success is shown with the green arrow in Fig. \ref{fig:qualitative_results} to indicate the correction.
