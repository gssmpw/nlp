In this section, we present important notations and results from bearing rigidity theory \cite{zhao2015bearing, trinh2019minimal}. For further details, we refer the reader to \cite{michieletto2021unified}.
\subsection{Bearing Rigidity Theory}
Consider a finite collection of $n$ points in $\mathbb{R}^d$ $(n\geq2, d\geq2)$ denoted as $\{p_i\}_{i=1}^n$. A configuration is defined as $\mathbf{p} = [p_1^\top, \dots, p_n^\top]^\top \in \mathbb{R}^{dn}$. Using a configuration $\mathbf{p}$ and an undirected graph $\mathcal{G}$, we define a framework as  
$\mathcal{G}(\mathbf{p})$, where each of vertex $i$ is uniquely mapped to each of the positions $p_i$, respectively.
% Given a finite collection of $n$ points $\{p_i\}_{i=1}^n$ in $\mathbb{R}^d$ $(n\geq2, d\geq2)$, we denote a configuration as $\mathbf{p} = [p_1^\top, \dots, p_n^\top]^\top \in \mathbb{R}^{dn}$. A framework is denoted as $\mathcal{G}(\mathbf{p})$ which is a combination of an undirected graph $\mathcal{G}$ and a given configuration $\mathbf{p}$, where each vertex $i \in \mathcal{V}$ in the graph is uniquely mapped to the position $p_i$. 
Intuitively, a framework combines an abstract graph structure with a topology defined by a configuration. Given a framework $\mathcal{G}(\mathbf{p})$, we define
\begin{equation}
    \label{eq:bearing}
    e_{ij}:=p_j - p_i, \quad g_{ij}:= \frac{e_{ij}}{\norm{e_{ij}}}, \quad \forall (i,j) \in \mathcal{E},
\end{equation}
where the unit vector $g_{ij}$ represents the relative bearing of $p_j$ to $p_i$. Note that $e_{ij} = -e_{ji}$ and $g_{ij} = - g_{ji}$. %For an oriented graph, it is easy to verify that $e$ satisfies $e = \bar{H}p$ where $e = [e_1^\top, \dots, e_m^\top]^\top$ and $\bar{H} = H \otimes I_d$ with $H$ as the incidence matrix. 
Define the bearing function $f_{B}: \mathbb{R}^{dn} \rightarrow \mathbb{R}^{dm}$ as: 
\begin{equation}
    \label{eq:bearing_func}
    f_B(\mathbf{p}) = [g_1^\top, \dots, g_m^\top]^\top,
\end{equation}
where $m=|\mathcal{E}|$ is the number of edges in the framework and $\mathbf{p} = [p_1^\top, \dots, p_n^\top]^\top \in \mathbb{R}^{dn}$.
The bearing function \eqref{eq:bearing_func} describes the bearing of the entire framework. The following Jacobian is defined as the \textit{bearing rigidity matrix}: 
\begin{equation}
    \label{eq:bearing_matrix}
    R_B(\mathbf{p}):= \frac{\partial f_B(\mathbf{p})}{\partial \mathbf{p}} \in \mathbb{R}^{dm\times dn}.
\end{equation}
% We define the orthogonal projection matrix operator $P:\mathbb{R}^d \rightarrow \mathbb{R}^{d\times d}$, for any nonzero vector $x\in \mathbb{R}^d$ as
% \begin{equation}
%     P(x):= I_d - \frac{x}{\norm{x}}\frac{x^\top}{\norm{x}}.
% \end{equation}
% For notational simplicity, we denote $P_x = P(x)$. It can be easily verified that $P_x^\top = P_x, P_x^2 = P_x$, and $P_x$ is positive semi-definite. It represents the orthogonal projection matrix that geometrically projects a vector onto the orthogonal complement of $x$. 

With the above notations defined, we now present some important properties and results of the bearing rigidity matrix.
% \begin{lemma}[\cite{zhao2015bearing}]
% The bearing rigidity matrix in \eqref{eq:bearing_matrix} can be expressed as $R_B(p) = \text{diag}(P_{g_k}/\norm{e_k})\bar{H}$.
% \end{lemma}
\begin{lemma}[Bearing Rigidity Criterion\cite{zhao2015bearing}]
For a given framework defined by $\mathcal{G}(\mathbf{p})$, the bearing rigidity matrix satisfies the following rank criterion: $\text{rank}(R_B) \leq dn-d-1$.
\end{lemma}

%Let $\delta \mathbf{p}$ denote the variation of $\mathbf{p}$. If $R_B(\mathbf{p})\delta \mathbf{p} = 0$, then $\delta \mathbf{p}$ is called an \textit{infinitesimal bearing motion} of $\mathcal{G}(\mathbf{p})$. The motion of a framework is called infinitesimal if and only if it preserves the bearing between any two pairs of neighbors in the framework. %An arbitrary framework always comprises two kinds of \textit{trivial} infinitesimal bearing motions: translation and scaling of the entire framework. 
We now formally present the definition of infinitesimal bearing rigidity and minimal bearing rigidity.
%These notions of rigidity will be used extensively later while designing resilient network topology for coverage control. }
\begin{definition}[Infinitesimal Bearing Rigidity \cite{zhao2015bearing}]
We call a framework infinitesimally bearing rigid if its bearing motions are trivial, i.e., translation and scaling.
%A framework is infinitesimally bearing rigid if the infinitesimal bearing motions of the framework are trivial, i.e., translation and scaling.
\end{definition}

%Infinitesimal bearing rigidity of a framework is a generic property, as it is mainly determined by the graph topology rather than the configuration \cite{trinh2019minimal}. 
The following definitions will introduce the concept of minimal bearing rigid graphs, which will be subsequently used to design network topologies with a minimum number of communication and sensing links composing a rigid structure. First, we define a generically bearing rigid graph. 
\begin{definition}[Generically Bearing Rigid Graph \cite{trinh2019minimal}]
A generically bearing rigid graph $\mathcal{G}$ is one that allows at least one configuration $\mathbf{p}\in \mathbb{R}^{dn}$ such that the framework $\mathcal{G}(\mathbf{p})$ is infinitesimally bearing rigid.
\end{definition}
\begin{definition}[Minimal Bearing Rigid Graph \cite{trinh2019minimal}]
A graph $\mathcal{G}$ composed of $n$ vertices and $m$ edges is minimally bearing rigid if and only if there does not exist any generically bearing rigid graph $\mathcal{H}$ in $\mathbb{R}^d$ with $n$ vertices and contains less than $m$ edges.   
\end{definition}
\subsection{Laman Graph and Henneberg Construction}
\label{subsec:henneberg}
We now review the iterative procedure to construct minimally bearing rigid graphs. First, we provide a combinatorial definition of Laman graphs, which represents a class of minimally rigid graphs.
\begin{definition}[Laman Graph \cite{zhao2017laman}]
A graph $\mathcal{G} = (\mathcal{V}, \mathcal{E})$ is a Laman graph if for all $k$, every subset of $k\geq2$ vertices spans at most $2k-3$ edges and the graph satisfies $|\mathcal{E}| = 2|\mathcal{V}| - 3$.
%A graph $\mathcal{G} = (\mathcal{V}, \mathcal{E})$ is Laman if $|\mathcal{E}| = 2|\mathcal{V}| - 3$ and every subset of $k\geq2$ vertices spans at most $2k-3$ edges.
\end{definition}

Intuitively, it implies that the graph's edges must be evenly distributed in a Laman graph. Another way to characterize Laman graphs is through Henneberg's construction. Figure \ref{fig:henneberg} illustrates the two operations of the Henneberg construction. 
\begin{definition}[Henneberg Construction \cite{zhao2017laman}]
For a given graph $\mathcal{G} = (\mathcal{V}, \mathcal{E})$, we form a new graph $\mathcal{G^{'}} = (\mathcal{V}^{'}, \mathcal{E}^{'})$ by appending a new vertex $v$ to the original graph $\mathcal{G}$ utilizing one of the following two operations:
% Given a graph $\mathcal{G} = (\mathcal{V}, \mathcal{E})$, a new graph $\mathcal{G^{'}} = (\mathcal{V}^{'}, \mathcal{E}^{'})$ is formed by adding a new vertex $v$ to the graph $\mathcal{G}$ and performing one of the following two operations:
\begin{enumerate}[(a)]
    \item Vertex Addition: Pick two vertices $i,j\in\mathcal{V}$ from graph $\mathcal{G}$ and connect it to $v$. This will lead to the resulting graph $\mathcal{G}^{'}$ with $\mathcal{V}^{'}= \mathcal{V} \cup v$ and $\mathcal{E}^{'} = \mathcal{E}\cup\{(v, i),(v, j)\}$. %as shown in fig. \ref{fig:henneberg}. 
    \item Edge Splitting: Pick three vertices $i,j, k\in\mathcal{V}$ and an edge $(i,j) \in \mathcal{E}$ from graph $\mathcal{G}$. Connect each vertex to $v$ and delete the edge $(i,j)$. In this case, the resulting graph $\mathcal{G}^{'}$ will have $\mathcal{V}^{'} = \mathcal{V} \cup v$ and $\mathcal{E}^{'} = \mathcal{E}\cup\{(v, i),(v, j), (v, k)\} \backslash \{ (i,j)\}$. % as shown in fig. \ref{fig:henneberg}.
\end{enumerate}
\end{definition}
\begin{figure}[h]
\centering
\includegraphics[width=0.5\textwidth,clip]{sections/images/henneberg.pdf}
\caption{Illustration of Henneberg construction to construct minimally rigid graphs utilizing two operations: (a) vertex-addition and (b) edge-splitting.}
\label{fig:henneberg}
\end{figure} 
The resulting graph obtained by applying Henneberg construction, starting with an edge connecting two vertices, generates a Laman graph, and its converse is also true \cite{zhao2017laman}.
%The converse is also true \cite{zhao2017laman}, i.e., if a graph is Laman, it can be generated using Henneberg construction. 