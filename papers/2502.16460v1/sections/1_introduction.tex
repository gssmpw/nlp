%% MOTIVATION
%Multi-robot systems (MRSs) have been actively studied and adopted for various applications such as search and rescue, surveillance, and wildlife monitoring because they can perform complicated tasks that are impossible with only single robots. 
Recent advances in multi-robot systems (MRSs), with their superior sensing, communication, and computational capabilities, allow them to perform complicated tasks otherwise impossible with only single-robot systems. MRSs have been widely adopted for numerous applications such as cooperative sensor coverage\cite{cortes2004coverage}, search and rescue\cite{kantor2006distributed}, and environmental monitoring \cite{burgard2005coordinated}. In recent catastrophic wildfires in Los Angeles, drone swarms have been actively utilized for monitoring and prevention of wildfires \cite{wildfire2025wong}. However, as the complexity of these systems increases, the number of failure modes affecting MRS performance and safety also increases. Furthermore, the sensing \cite{khan2023synthesis, pant2024adversarial}, and communication networks \cite{rezaee2024resilient} also open up new cyberattack surfaces, network vulnerabilities, and backdoors, which adversaries can exploit to degrade and disrupt the performance of the MRS. Thus, designing control architectures ensuring the system's resiliency under these unknown failure modes becomes essential. 

%have encouraged researchers to investigate yet another important aspect of autonomy: resiliency, i.e., the tendency to tolerate failure to unexpected failure modes. On the one hand, advanced sensing and communication capabilities allow these systems to accomplish their desired task more efficiently. However, these features introduce additional failure modes, which are difficult to model, disrupting their operations and leading to catastrophic failures. 

A key application of MRSs is to cover a desired area of interest, often denoted by a density function that indicates the importance of each point in the region. This task is termed as \textit{coverage control}. A group of robots deployed in an environment (in any arbitrary initial configuration) is moved to optimal locations that maximize the overall sensing/coverage performance.
%in an initial configuration with the objective of moving each robot to optimal locations so that the overall sensing/coverage performance is maximized.
Although extensive research on coverage control and its variants \cite{pierson2017adapting, schwager2009decentralized, todescato2017multi, lee2015multirobot, carron2020model, luo2019voronoi} have been done in the literature, the results often assume that all the robots remain healthy, connected, and free of localization errors throughout the operation, which may not be applicable in real-world situations. In many cases, the MRSs can fail due to known/unknown failure modes in navigation, communication, or control. For example, suppose the position estimates of some of the UAVs in a swarm are compromised by entering a GNSS-denied region or by an adversary injecting GNSS spoofing attack. One way to mitigate such failures is to take advantage of the inherent redundancy, i.e., locally sensed inter-robot measurements that exist by virtue of the configuration of robots \cite{vijay2025range}. To effectively utilize this information redundancy, it becomes essential for the network to maintain a resilient structure (or rigid structure), one where robot loss or anomalies can be easily detected and mitigated. 
%This ensures that robots remain well connected and collaboratively protect/guide each other under unknown localization drifts and failures.}
%In such cases, a single failure in one UAV can disrupt the entire operation of the UAV swarm. 

\begin{figure}[t]
\centering
\includegraphics[width=0.5\textwidth,clip]{sections/images/resilient_coverage.pdf}
\caption{Illustration of multi-robot coverage control for environmental monitoring during wildfire. Our proposed distributed approach ensures the coordination of the network of robots even when some robots become unhealthy while executing the mission. }
\label{fig:illustration}
\end{figure} 

%% Related works
In classical works, the coverage control has been rigorously studied with optimal solutions characterized by Centroidal Voronoi tesselation (CVT) \cite{du2006convergence}. In \cite{cortes2004coverage}, the authors proposed an iterative gradient-based control law using Lloyd's algorithm \cite{lloyd1982least} for the single integrator dynamics to converge to the Voronoi centroids. Coverage control with time-varying sensory function is analyzed in \cite{lee2015multirobot}, and a deep learning-based approach to cover an area with an unknown density function is proposed in \cite{rastgoftar2025deep}. The authors in \cite{schwager2009decentralized,todescato2017multi,rickenbach2024active} designed coverage algorithms that simultaneously estimate the sensory function along with the control design. Obstacle avoidance in coverage control is addressed in \cite{bai2021adaptive, bai2023safe}, where the authors designed a safe control strategy leveraging control barrier functions (CBFs). In \cite{rezaee2024resilient}, a distributed resilient strategy is proposed to mitigate the impact of misinformation in the network caused by cyberattacks on a subset of robots. Recently, the authors in \cite{carron2020model, rickenbach2024active} considered the coverage problem with the nonlinear robot dynamics with state and input constraints and formulated it as a reference tracking model predictive control (MPC). In most of the above approaches, the underlying network topology is implicitly assumed, and the robots stay healthy throughout the coverage task.
%The authors in \cite{bai2021adaptive, bai2023safe} considered obstacle avoidance in the coverage control problem and designed a safe control scheme that uses control barrier functions (CBFs) to encode the safety constraints. In \cite{rezaee2024resilient}, the authors have considered the coverage control problem in which certain robots are attacked, i.e., broadcasting incorrect positional information to their neighbors, and proposed a distributed resilient strategy to mitigate the effects of misinformation. All of the above-mentioned works consider either single integrator dynamics or stabilized passive systems without state and input constraints. Recently, the authors in \cite{carron2020model, rickenbach2024active} considered the coverage problem with the nonlinear robot dynamics with state and input constraints and formulated it as a reference tracking model predictive control (MPC). In all of the above approaches, the underlying network topology is either predefined or implicitly assumed.
%However, there is no In most of the above works, the underlying network topology of the MRS is considered static or predefined

Significant gaps exist in the literature where coverage control is studied while maintaining a resilient network topology. In \cite{luo2019voronoi}, a connectivity-constrained coverage controller is proposed that enforces connectivity by imposing hard constraints as barrier certificates using the algebraic network connectivity, i.e., eigenvalues of the graph Laplacians. However, such an approach can lead to infeasibility and deadlocks \cite{cavorsi2023multi}. To address this issue, the authors in \cite{cavorsi2023multi} utilized a bypass mechanism that considers an alternate CBF to relax connectivity constraints, minimizing deadlocks. However, it requires an exact representation of the environment, which may not be suitable for coverage control.
%Thus, it is challenging to design and maintain the network topology that simultaneously ensures the robots have the highest maneuverability and possess the ability to reconfigure themselves when a subset of robots becomes faulty or attacked.}
%However, enforcing such hard constraints can often lead to infeasibility and drive the robot configuration into a deadlock \cite{cavorsi2023multi}. To address this issue, the authors in \cite{cavorsi2023multi} utilized a bypass mechanism that considers an alternate CBF to relax connectivity constraints, minimizing deadlocks. However, it requires an exact representation of the environment, which may not be suitable for coverage control.  
% Therefore, simply ensuring connectivity while executing coverage motions can lead to overly conservative motion and inferior coverage performance \cite{luo2019voronoi, cavorsi2023multi}. 
% While connectivity ensures inter-robot communication necessary for efficient collaboration, it doesn't guarantee the structural resilience of the MRS network, an important property essential to detect and mitigate the effects of unexpected robot failures, e.g., localization errors.
% Thus, it is challenging to design and maintain the network topology that simultaneously ensures the robots have the highest maneuverability and possess the ability to reconfigure themselves when a subset of robots becomes faulty or attacked.

% None of the above-mentioned works have considered the system's recovery in the case of agent loss or localization errors in the coverage control setting. One of the main challenges in ensuring resilience is the design of the MRS sensing and communication network. The other challenge is to maintain the desired network throughout the coverage task.   
%Thus, it becomes crucial to systematically identify and remove faulty agents and swiftly reconfigure the MRS network to accommodate the robot's loss. 

%% OUR FOCUS
In this work, we focus on the aforementioned challenges and tackle the problem of enhancing the structural resilience of the underlying network topology of the MRS in a coverage control setting. Our goal is to move the robots to cover a desired area while ensuring that they remain well-connected and collaboratively protect/guide each other under unknown localization drifts/failures caused by faults or cyberattacks. We transform this problem into rigidity maintenance of a network (or structure) formed by robots (as joints) and their virtual sensing and communication links (as bars) as it is scaled and translated in space while viewing the localization failures and robot loss as perturbation forces acting on it.

We take advantage of the rigidity theory originally developed to analyze the flexibility (deformability) of rigid structures arising from external perturbations. Rigidity theory has been previously investigated in the MRS literature, notably for decentralized self-localization\cite{zhao2015control}, formation stabilization \cite{zhao2015bearing,schiano2016rigidity}, integrity monitoring \cite{vijay2025range} and reconfiguration \cite{anderson2008rigid}. In this work, for the first time, we leverage concepts from rigidity theory for coverage control. We impose the MRS's motion to satisfy the constant relative bearing between the robot's neighbors, i.e., bearing rigidity, which allows the network to scale and translate to achieve desired coverage performance.
%and leverage concepts from rigidity theory to enhance the resiliency of the MRS network for the coverage control task. 
% By doing so, we enhance the network's self-healing capabilities in the case of unexpected failures resulting from faults or cyberattacks. 
% \textcolor{red}{Why Rigid network topology and why recovery??}
As a consequence of maintaining the rigid structure, we show that the MRS can swiftly reconfigure itself when a robot leaves the network by making new local connections (among two-hop neighbors) while preserving the structural/topological network properties, i.e., resiliency and connectivity.

%% OUR CONTRIBUTIONS
%The objective of this paper is thus to develop a distributed coverage control algorithm that maintains a resilient bearing-rigid network topology. 
Our main contributions can be summarized as follows: 
\begin{enumerate}
    \item We propose a distributed coverage control algorithm utilizing nonlinear tracking MPC that converges to the centroidal Voronoi configuration while constantly maintaining a resilient bearing-rigid network topology.
    \item We prove the convergence of the proposed distributed approach to the CVT (if feasible), otherwise to the nearest position to the CVT satisfying state and input constraints. Furthermore, we provide a design procedure to compute the terminal ingredients necessary to ensure the recursive feasibility and closed-loop stability of the distributed nonlinear tracking MPC.  
    \item We design a proactive rigidity recovery algorithm that iteratively finds the set of new recovery edges for each robot in the event of losing its neighbor due to faults or cyberattacks. 
\end{enumerate}

%% ORGANISATION
The rest of this paper is organized as follows. Section \ref{sec:prelim} briefly introduces the concepts from bearing rigidity theory. Section \ref{sec:prob_form} presents the problem formulation for multi-robot resilient coverage control. Section \ref{sec:resilient_mpc} provides the design of our proposed control architecture that enforces a bearing-rigid formation for the coverage task to improve the resilience of the underlying network topology. We also present a rigidity recovery algorithm that equips each robot with a set of new edges (connections) to be made in the case of a robot failure. Section \ref{sec:sim} provides an illustrative numerical example of our proposed approach. Section \ref{sec:conc} concludes the paper.

\textit{Notations:} %For a given matrix $A_i \in \mathbb{R}^{p\times q}$ for $i = 1, \dots, n$, we denote $\text{diag}(A_i)\triangleq\text{blkdiag}\{A_1, \dots, A_n\} \in \mathbb{R}^{np\times nq}$. 
% For a given matrix $A \in \mathbb{R}^{p\times q}$, $\text{Null}(\cdot)$ and $\text{Range}(\cdot)$ denote the nullspace and range, respectively. Denote $I_d\in \mathbb{R}^{d\times d}$ as the identity matrix and $\mathbf{1} \triangleq [1, \dots, 1]^\top$. Let $\norm{ \cdot }$ be the Euclidean norm of a vector, and $\otimes$ denotes the Kronecker product. 
Let $\mathcal{G} = \{ \mathcal{V}, \mathcal{E}\}$ denote an undirected graph with $\mathcal{V}$ as its vertices and $\mathcal{E} \subseteq \mathcal{V}\times\mathcal{V}$ as its edge set with $n = |\mathcal{V}|$ and $m = |\mathcal{E}|$. The set of neighbours of a vertex $i$ is denoted as $\mathcal{N}_i \triangleq \{j\in \mathcal{V}:(i,j)\in \mathcal{E} \}$. For any real-valued vector $x \in \mathbb{R}^n$  and positive definite matrix W, $\norm{x}_{W}^2 := x^\top Wx$ and $\norm{ \cdot }$ denotes the Euclidean norm. $\mathbb{R}_{+}$ denotes the set of positive real numbers. A class $\mathcal{K}$ function $\alpha: \mathbb{R}_+ \rightarrow \mathbb{R}_+$ is a continuous and strictly increasing function with $\alpha(0) = 0$. A class $\mathcal{K}_\infty$ function $\beta: \mathbb{R}_+ \rightarrow \mathbb{R}_+$ is a class $\mathcal{K}$ function and is unbounded.
%An undirected graph with each of its edge associated with an orientation is called an oriented graph. The incident matrix of an oriented graph is denoted as $H\in \mathbb{R}^{n\times m}$. For a connected graph, the following property holds, $H \mathbf{1}= 0$ and $\text{rank}(H) = n-1$.