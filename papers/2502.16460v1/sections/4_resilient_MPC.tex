In this section, we will present our proposed control architecture and show that the resulting controller guarantees the convergence of the robot positions to the optimal Voronoi configuration. Our approach leverages nonlinear tracking MPC control for coverage control. Note that, we introduce an additional bearing maintenance cost in the optimization process to ensure that the robots maintain a bearing-rigid network topology. This ensures that even if a subset of robots becomes faulty or are lost during the mission, the rest of healthy robots can help mitigate localization failures or even reconfigure the MRS network (if the robots are lost) so that the desired coverage performance is maintained. This transforms the overall problem into a sequential multi-objective optimization problem, i.e., jointly minimizing the tracking and inter-robot bearing errors. 
\subsection{Nonlinear Tracking MPC with Bearing Maintenance}
We utilize the nonlinear tracking MPC framework to track the centroids defined in \eqref{eq:centroid} while maintaining a constant bearing. The centroids are updated iteratively based on a partition update criterion described later, ensuring robots converge to the optimal Voronoi partitions while satisfying recursive feasibility and closed-loop stability. 

The partition update step requires centralized coordination as it involves the knowledge of all robot positions to compute the Voronoi centroids. %\textcolor{red}{For the rest of the paper, we make the following notations: $w$ refers to the time when the Voronoi partitions are updated. It differs from time $k$, which indicates a single step.} 
We define the computed reference setpoint as $r_{i,k} \coloneq c_i \big( \mathcal{W}(\mathbf{p}_{k}) \big)$, where $\mathbf{p}_k$ denotes the robot positions at time $k$. Let the tracking error between the robot positions and the computed centroids at time $k$ be defined as
\begin{equation}
    e_{i,k} = \norm{p_{i,k} - r_{i,k}}.
\end{equation}
To guarantee convergence, we utilize the partition criterion developed in \cite[Sec. 4]{carron2020model}. The partition is updated only if,
\begin{itemize}
    \item$\norm{p_{i,k} - r_{i,k}} \leq   e_{i,k} \quad \forall i \in \{1,\dots,n\},$  
    \item $\exists j \in \{1,\dots,n\} \quad \text{s.t.} \  \norm{p_{j,k} - r_{i,k}} \leq   e_{j,k}.$
\end{itemize}
For bearing maintenance, we introduce an additional cost in the overall optimization. The computed Voronoi centroids \eqref{eq:centroid} are utilized to establish a bearing-rigid formation, which is then used to calculate the desired bearing vector. This can be described as follows. Consider a bearing rigid framework $\mathcal{G}(\mathbf{p}_k)$, defined by the robot's positions $\mathbf{p}_k$ at the time $k$ and the communication/sensing graph $\mathcal{G}$. Let us denote the desired minimally rigid framework as $\mathcal{G}(\mathbf{r}_{k})$, where $\mathbf{r}_{k}$ is the Voronoi centroids after the partition update. Thus, the desired bearing of robot $i$ is denoted as $g_{i,k}$, which is obtained using the bearing function $g_{i,k} = f_B(\mathbf{r}_{k})$.

In this work, we formulate the resilient coverage control strategy for simultaneous centroid tracking and bearing maintenance as a nonlinear output tracking MPC problem. We define the artificial steady-state state-input pair for each robot as $(\bar{x}_i, \bar{u}_i)$. The steady-state position of each robot is $\bar{r}_i$. Given a position reference $r_{i,k}$, which is set as the centroids of the partition for each robot and a bearing reference $g_{i,k}$ is computed using $r_{i,k}$, the optimization problem computes the control input $u_{i,k}$ along with the steady-state state-input pair $(\bar{x}_i, \bar{u}_i)$ and an artificial reference $\bar{r}_i$ that tracks $r_{i,k}$. 
%to steer the robots to the centroids of the partitions. 
\begin{remark}
By choosing the position reference for Voronoi centroids $\bar{r}_i$ as an additional decision variable in the optimal control problem, it is ensured that any changes in the centroids resulting from the partition update over time do not affect the feasibility of the optimization.
%The addition of an artificial position reference for Voronoi centroids $\bar{r}_i$ as an additional decision variable in the optimal control problem ensures the feasibility of the problem. Changes in the centroids and the optimal partitions over time do not affect the feasibility of the optimization.     
\end{remark}

To simplify the notations, we define $J_i(\cdot) \coloneq J_i(x_{i,k}, r_{i,k}, g_{i,k}; \mathbf{u}_i, \bar{r}_i)$, where $x_{i,k}, \ r_{i,k}$, and $g_{i,k}$ are inputs, and $\mathbf{u}_i$ and $\bar{r}_i$ are outputs. Let $N$ be the prediction horizon for the MPC. We denote $x_{i,l|k}$ and $u_{i,l|k}$ as the $l^{\text{th}}$ step forward prediction of the state and inputs of the robot $i$ at time $k$, respectively, with $l \in \{1,\dots, N\}$, where $x_{i,0|k} = x_{i,k}$ and $\mathbf{u}_i = [u_{i,0|k}^\top, u_{i,1|k}^\top, \dots, u_{i,N|k}^\top]^\top$. Then, the overall cost function for each robot can defined as:
\begin{align}
\label{eq:mpc_cost}
 J_i(\cdot) = \sum_{l=0}^{N-1} &\ell_{i,l}(x_{i,l|k} - \bar{x}_{i}, u_{i,l|k} -  \bar{u}_i) + \ell_{i,N}(x_{i,l|k} - \bar{x}_i, \bar{r}_i) \nonumber
 \\+ \mu &\ell_{i,r}(r_{i,k} - \bar{r}_i) + (1-\mu)\ell_{i,b}(g_{i,k}, \bar{r}_i),    
\end{align}
where $\ell_{i,l}: \mathbb{R}^{n_x} \times \mathbb{R}^{n_u} \rightarrow \mathbb{R}$ is the stage cost, $\ell_{i,N}: \mathbb{R}^{n_x} \times \mathbb{R}^{d} \rightarrow \mathbb{R}$ is the terminal cost, $\ell_{i,r}: \mathbb{R}^{d} \rightarrow \mathbb{R}$ is the reference centroid tracking cost, and $\ell_{i,b}: \mathbb{R}^{dn_x} \times \mathbb{R}^{d} \rightarrow \mathbb{R}$ is the bearing maintenance cost. All of the above are positive definite functions. $\mu\in(0,1]$ denotes a constant, which ensures that the combined cost is a convex combination of the reference tracking and bearing maintenance costs. Note that the first two terms of $J_i(\cdot)$ penalize the deviation with respect to the artificial position reference $\bar{r}_i$. In contrast, $\ell_{i,r}(\cdot)$ penalizes the deviation between the true centroid and the artificial reference $(r_{i,k} - \bar{r}_i)$. In our formulation, the bearing maintenance cost depends on the equilibrium reference $\bar{r}_i$ instead of introducing an additional equilibrium bearing output vector as a decision variable. This crucial step allows us to construct a coupling between the centroid reference tracking and bearing maintenance costs. Moreover, keeping $\mu>0$ ensures that the reference centroid tracking is always achieved. Furthermore, by constructing a convex combination of the two costs, the overall cost is convex and has a unique minimizer, i.e., one that ensures that the robots reach the centroidal Voronoi configuration.

We introduce the following restricted sets essential for deriving the convergence result of our proposed approach. In order to avoid the above steady-state variables for each robot $i$ with active constraints, we define the following restricted set
\begin{equation}
    \hat{\mathcal{Z}_i} = \{ z \ | \ z + e \in \mathcal{Z}_i, \forall \norm{e} \leq \epsilon \}, 
\end{equation}
where $\epsilon>0$ is a small positive constant. We define $\bar{\mathcal{Z}}_i$ as the set of steady-state state-input pairs satisfying constraints, and $\bar{\mathcal{R}}_i$ as the set of corresponding steady-state positions where the state and input constraints are not active as follows:
\begin{align}
    \bar{\mathcal{Z}}_i &= \{z \ | \ z = (x,u) \in \hat{\mathcal{Z}_i}, \forall \norm{e} \leq \epsilon \}, \\
    \bar{\mathcal{R}}_i &= \{p \ | \ p = C_ix, (x,u) \in \bar{\mathcal{Z}}_i\}.
\end{align}

Note that since $\epsilon$ can be arbitrarily small, $\bar{\mathcal{Z}}_i$ will contain all steady-state state-input pairs close to the boundary of hard constraints defined by the set $\mathcal{Z}_i$. These sets will be utilized later as we show the convergence of our proposed approach.

To derive the stability of the proposed controller, we define an invariant set for centroid tracking by extending the idea of an invariant output tracking set introduced in \cite{limon2018nonlinear, limon2008mpc}.
% extend the notion of an invariant set for tracking introduced in \cite{limon2018nonlinear, limon2008mpc} to the Voronoi centroid tracking case as follows:
\begin{definition}[Invariant Set for Centroid Tracking]
Let each robot $i\in \{1,\dots, n\}$ in the MRS have the constraint set on the state and the input as $\mathcal{Z}$, a set of feasible setpoints for the centroids $\mathcal{R}_i \subseteq \bar{\mathcal{R}}_i$, and a local control law $u_{i,k} = \kappa_i(x_{i,k}, \bar{r}_i)$. A set $\Gamma_i \subset \mathbb{R}^{n_x} \times \mathbb{R}^{d}$ is an invariant set for centroid tracking for the robot dynamics \eqref{eq:agent_dynamics} if for all $(x_{i,k}, \bar{r}_i) \in \Gamma_i$, we have $(x_{i,k}, \kappa_i(x_{i,k}, \bar{r}_i)) \in \mathcal{Z}$, $\bar{r}_i \in \mathcal{R}_i$, and $(x_{i,k+1}, \bar{r}_i) \in \Gamma_i$.
\end{definition}

Intuitively, this set represents the set of initial states and the artificial reference centroids $(x_i, \bar{r_i})$ that provides an admissible evolution of the robot dynamics defined by \eqref{eq:agent_dynamics} controlled by $u_{i,k} = \kappa_i(x_{i,k}, \bar{r}_i)$. Furthermore, once the system enters this set, it does not leave this set. The centroid tracking nonlinear constrained optimization problem, which simultaneously tracks the Voronoi centroids while minimizing bearing errors, is defined as:
\begin{subequations}
\label{eq:mpc}
\begin{align}
    \min_{u_i, \bar{x}_i, \bar{u}_i, \bar{r}_i} &J_i(x_{i,k}, r_{i,k}, g_{i,k}; \mathbf{u}_i, \bar{r}_i)\\
    &\text{s.t.} \quad l = \{1,\dots, N-1\}\\
    %\text{s.t.} \quad &l = \{1,\dots, N-1\}\\
    &x_{i,0|k} = x_{i,k}\\
    &x_{i,l+1|k} = f_i(x_{i,l|k}, u_{i,l|k})\\
    &(x_{i,l|k}, u_{i,l|k}) \in \mathcal{Z}_i\\
    &(x_{i,N|k}, \bar{r}_{i}) \in \Gamma_i\\
    &\bar{r}_i = C_i\bar{x}_i\\
    %&\bar{g}_i = f_B(\mathbf{c}(\mathcal{V}(\bar{r}_{i}))\\
    &\bar{x}_i = f_i(\bar{x}_i, \bar{u}_i),
\end{align}
\end{subequations}
where $\Gamma_i$ denotes the invariant set for centroid tracking.

Each robot then locally computes and applies the control input in a receding-horizon fashion. The decentralized control law for each robot at time $k$ is given by $ \kappa_{MPC}(x_{i,k}, r_{i,k}, g_{i,k}) = u_{i,0|k}^{*} \ \forall i \in \{1,\dots, n\}$, where $u^{*}$ is the optimal solution of \eqref{eq:mpc}. Algorithm \ref{alg:resilient_mpc} summarizes the overall control design procedure. Note that the centroid reference $r_{i,k}$ and the bearing reference $g_{i,k}$ do not affect the feasibility of the optimization problem defined in \eqref{eq:mpc} because they only appear in the cost function \eqref{eq:mpc_cost}.
\begin{algorithm}[t]
\caption{Resilient Multi-robot coverage control}\label{alg:resilient_mpc}
\begin{algorithmic}[1]
\Require Robot initial positions $\mathbf{p}_{0}$ and the MRS network topology $\mathcal{G} = (\mathcal{V}, \mathcal{E})$.
\State Compute $r_{i,0} = c_i(\mathcal{W}(\mathbf{p}_0))$, $g_{i,0} = f_B(r_{i,0})$ and $e_{i,0} = \norm{p_{i,0} - r_{i,0}} \ \forall i = \{ 1, \dots, n\}$.
\For {$k =1,2\dots,$}
    \For {$i =1,2\dots,n$}
        \State Take the measurement of $x_{i,k}$.
        \State Solve the optimization problem in \eqref{eq:mpc}.
        \State Apply the first control input $u_{i,0}^{*}$.
        \If{$\norm{p_{i,k} - r_{j,k}} \leq   e_{i,k}$ and $\exists j \in \{1,\dots,n\} \ \text{s.t.} \  \norm{p_{j,k} - r_{i,k}} \leq   e_{j,k}.$} 
            \State Update $r_{i,k} = c_i(\mathcal{W}(\mathbf{p}_k))$, $g_{i,k} = f_B(r_{i,k})$ and $e_{i,k} = \norm{p_{i,k} - r_{i,k}}$
        \EndIf 
    \EndFor
\EndFor
\end{algorithmic}
\end{algorithm}

Consider the following assumptions prevalent in the nonlinear tracking MPC literature that ensure recursive feasibility and convergence guarantees \cite{ferramosca2009mpc, limon2008mpc, limon2018nonlinear}. %Similar to \cite{carron2020model}, the assumptions on the terminal ingredients are extended to a set of equilibrium points while keeping them identical to the standard MPC literature. Generally, these assumptions cannot be trivially extended to equilibrium states for an arbitrary nonlinear system. However, as a result of Assumption \ref{ass:invariant} on robot dynamics, which makes any position a steady state, the conditions can be easily satisfied. 
The following assumptions must be satisfied by the stage cost function, reference centroid tracking cost, and the bearing maintenance cost:
\begin{assumption}
\label{ass:cost}
\begin{enumerate}
    \item The stage cost function must satisfy $\ell_i(z,v) \geq \alpha_{\ell}(\norm{z})$ for all $(z,v) \in \mathbb{R}^{n_x+n_u}$ for some class $\mathcal{K}_\infty$ function $\alpha_{\ell}(\cdot)$.
    %There exists a class $\mathcal{K}_\infty$ function $\alpha_{\ell}$ such that the stage cost function satisfies $\ell_i(z,v) \geq \alpha_{\ell}(\norm{z})$ for all $(z,v) \in \mathbb{R}^{n_x+n_u}$.
    \item The set $\mathcal{R}_i$, which represents the set of feasible centroids, is a convex set of $\bar{\mathcal{R}_i}$.
    \item The reference centroid tracking cost $\ell_{i,r}(r_{i,k}-\bar{r}_{i})$ and the bearing maintenance cost $\ell_{i,b}(g_{i,k},\bar{r}_{i})$ denote a convex, positive semi-definite and sub-differential function such that the minimizer of the convex combination of the two satisfies:
    \begin{equation}
    \label{eq:offset_cost}
        \bar{r}_i^{\dagger}= \argmin_{\bar{r}_i \in \mathcal{R}_i} \mu \ell_{i,r}(r_{i,k} - \bar{r}_i) + (1-\mu)\ell_{i,b}(g_{i,k}, \bar{r}_i)
    \end{equation}
    is unique. Moreover, there exist $\alpha_r(\cdot)$ and $\alpha_b(\cdot)$, both class $\mathcal{K}_\infty$ functions such that
    \begin{align*}
        \ell_{i,r}(r_{i,k}-\bar{r}_{i}) -  \ell_{i,r}(r_{i,k}-\bar{r}^*_{i}) &\geq \alpha_r(\norm{\bar{r}_i - \bar{r}^*_i}),\\
        \ell_{i,b}(g_{i,k},\bar{r}_{i}) -  \ell_{i,b}(g_{i,k},\bar{r}^*_{i}) &\geq \alpha_b(\norm{\bar{r}_i - \bar{r}^*_i}).
    \end{align*}
\end{enumerate}
\end{assumption}

To ensure closed-loop stability, the terminal ingredients must follow the following assumptions:
\begin{assumption}
\label{ass:terminal}
\begin{enumerate}
    \item For the robot dynamics $x_{i,k+1}=f_i(x_{i,k}, \kappa_i(x_{i,k}, \bar{r}_i))$, $\Gamma_i$ is an invariant set for centroid tracking.
    \item Let $\kappa_i(x_{i}, \bar{r}_i)$ be the control law for all $(x_{i}, \bar{r}_i) \in \Gamma_i$, the steady-state $\bar{x}_i$ is an asymptotically stable equilibrium state for the robot dynamics $x_{i,k+1}=f_i(x_{i,k}, \kappa_i(x_{i,k}, \bar{r}_i))$. Futhermore, $\kappa_i(x_i, \bar{r}_i)$ is continuous at $(x_i, \bar{r}_i)$ for all $\bar{r}_i \in \mathcal{R}_i$.
    \item Let $\ell_{i,N}(x_{i} - \bar{x}_i, \bar{r}_i)$ be a Lyapunov function for the robot dynamics $x_{i,k+1}=f_i(x_{i,k}, \kappa_i(x_{i,k}, \bar{r}_i))$ such that for all $(x_i, \bar{r}_i) \in \Gamma_i$, there exist constants $b>0$ and $\rho >0$ such that
    \begin{align}
    \label{eq:terminal_lyap_1}
        \ell_{i,N}(x_{i} - \bar{x}_i, \bar{r}_i) &\leq b\norm{x_i - \bar{x}_i}^\rho,\\
        \ell_{i,N}(f_i(x_{i,k}, \kappa_i(x_{i,k}))& - \bar{x}_i, \bar{r}_i) -  \ell_{i,N}(x_{i,k}-\bar{x}_i, \bar{r}_i)\nonumber \\
        \label{eq:terminal_lyap_2}
        \leq -\ell_{i}(x_{i,k}-\bar{x}_i, &\kappa_i(x_{i,k}, \bar{r}_i) - \bar{u}_{i}).
    \end{align}
\end{enumerate}
\end{assumption}

These assumptions require a local controller that asymptotically stabilizes the system to any steady-state position contained in $\mathcal{R}_i$. There are several existing methods for designing such a controller in the literature, such as LTV modelling \cite{limon2018nonlinear}, linearized LQR method \cite{ chen1998quasi, kohler2019nonlinear}, etc. In order to make this exposition self-contained, we will present the design of the terminal controller and terminal invariant set for centroid tracking using an LQR control design of the linearized system around the steady-state centroid position $\bar{r}_i$, which is presented in Appendix \ref{app:terminal}.    
% \begin{assumption}
% \label{ass:mpc}
% Consider the optimization problem \eqref{eq:mpc},
% \begin{itemize}
%     \item Let $\mathcal{R}$ be the set of admissible reference positions, i.e., $\mathcal{R} = \{(x,u) \ | \ r_{i,k} = C x_{i,k}, x_{i,k+1} = f_i(x_{i,k},u_{i,k}), (x_{i,k},u_{i,k}) \in (\mathcal{X}_i, \mathcal{U}_i), \ \forall i \in \{1, \dots, n\} \}$. Assume $\mathcal{R}$ to be convex.
%     \item Let $\kappa_i(x_{i,k}, r_{i,w}, g_{i,w})$ be a control law such that for all $r_{i,w} \in \mathcal{R}$, the steady state $(\bar{x}_i, \bar{u}_i)$ is asymptotically stable for the system $f_i(x_{i,k}, \kappa_i(x_{i,k}, r_{i,w}, g_{i,w}))$.
%     \item Let $\mathcal{X}_{\text{terminal}}$ be an invariant set for the system $ f_i(x_{i,k}, \kappa_i(x_{i,k}, r_{i,w}, g_{i,w}))$ under state and input constraints $x_{i,k} \in \mathcal{X}_i$ and $u_{i,k} \in \mathcal{U}_i$.
%     \item Let $\ell_{i,N}(x_{i,k},\bar{x}_i)$ be Lyapunov functions associated with the robot dynamics $ x_{i,l+1|k} = f_i(x_{i,l|k}, \kappa_i(x_{i,k}, r_{i,w}, g_{i,w}))$. 
%     \item Let $\ell_{i,l}(x_{i,l|k},u_{i,l|k}, \bar{x}_i, \bar{u}_i)$ be positive definite functions. 
%     \item Let $\ell_{i,r}(r_{i,w},\bar{r}_{i})$ be convex, sub-differential and positive definite function and and $\ell_{i,b}(g_{i,w},\bar{r}_{i})$  denote a convex, positive semi-definite function and sub-differential .
% \end{itemize}
% \end{assumption}
We now present the main convergence result in Theorem \ref{th:convergence}. It ensures the closed-loop stability of the decentralized controller that converges the robot positions to their respective Voronoi centroids. When the centroid is infeasible $r_{i,k} \notin \mathcal{R}_i$, the robots converge to the closest feasible position, which satisfies the state and the input constraints $\mathcal{Z}_i$. 
%We show that Algorithm \ref{alg:resilient_mpc} converges to centroidal Voronoi configuration while maintaining a bearing-rigid MRS network. 
%\textcolor{red}{Note that Assumptions \ref{ass:invariant} and \ref{ass:dynamics}, which are applicable for most mobile robots, ensure that Assumption \ref{ass:terminal} is met. In most practical cases, a quadratic stage cost is chosen satisfying Assumption \ref{ass:cost}.}
\begin{theorem}
\label{th:convergence}
Suppose Assumptions \ref{ass:invariant}, \ref{ass:dynamics}, \ref{ass:cost}, and \ref{ass:terminal} are met, and each robot computes its control inputs using Algorithm \ref{alg:resilient_mpc}. Assume the initial positions of the robots to be $p_{i,0} = C_ix_{i,0} \in \mathcal{Q} \ \forall i \in \{1,\dots, n\}$ subject to the dynamics \eqref{eq:agent_dynamics}, $r_{i,k}$ be the desired centroid position and $x_{i,0} \ \forall i \in \{1,\dots, n\}$ be the initial state satisfying the constraints in \eqref{eq:mpc}. Then, the controller obtained by solving the optimization problem \eqref{eq:mpc} is stable, recursively feasible, and converges to a steady-state position such that
%Suppose Assumptions \ref{ass:invariant}, \ref{ass:dynamics}, \ref{ass:cost}, and \ref{ass:terminal} are met, and each robot computes its control inputs using Algorithm \ref{alg:resilient_mpc}. Let $p_{i,0} = C_ix_{i,0} \in \mathcal{Q} \ \forall i \in \{1,\dots, n\}$ be the initial positions of the robots subject to the dynamics \eqref{eq:agent_dynamics}, $r_{i,k}$ be the desired centroid position and $x_{i,0} \ \forall i \in \{1,\dots, n\}$ be a feasible initial condition of problem \eqref{eq:mpc}. Then, the controller obtained by solving the optimization problem \eqref{eq:mpc} is stable, satisfies the constraints throughout the time, and converges to a steady-state position such that 
\begin{enumerate}
    \item If $r_{i,k}\in \mathcal{R}_i$, then $\lim_{k \rightarrow \infty} \norm{p_{i,k} - r_{i,k}} = 0 $.
    \item If $r_{i,k} \notin \mathcal{R}_i$, then $\lim_{k \rightarrow \infty} \norm{p_{i,k} - \bar{r}_{i}^{\dagger}} = 0 $, where
    \begin{equation*}
         \bar{r}_i^{\dagger}= \text{arg} \min_{\bar{r}_i \in \mathcal{R}_i} \mu \ell_{i,r}(r_{i,k} - \bar{r}_i) + (1-\mu)\ell_{i,b}(g_{i,k}, \bar{r}_i). 
    \end{equation*}
\end{enumerate}   
\end{theorem}
\begin{proof}
The proof is presented in Appendix \ref{app:convergence}.    
\end{proof}
% \begin{proposition}
% \label{prop:convergence}
% Suppose Assumptions \ref{ass:invariant} and \ref{ass:mpc} are met, and each robot computes its control inputs using Algorithm \ref{alg:resilient_mpc}. Let $p_{i,0} = C_ix_{i,0} \in \mathcal{Q} \ \forall i \in \{1,\dots, n\}$ be the initial positions of the robots subject to dynamics \eqref{eq:agent_dynamics} and $x_{i,0} \ \forall i \in \{1,\dots, n\}$ be a feasible initial condition of problem \eqref{eq:mpc}. Then, the optimization problem defined in \eqref{eq:mpc} is recursively feasible, and the robots converge to a centroidal Voronoi configuration.   
% \end{proposition}
% \begin{proof}
% Note that the optimization problem defined in \eqref{eq:mpc} has two objectives: minimizing the reference tracking error and bearing errors. This makes \eqref{eq:mpc} a multi-objective optimization problem. To resolve this issue, we create a coupling between the two cost functions, such that, $\ell_{i,r}(r_{i,w},\bar{r}_{i})$ and $\ell_{i,b}(g_{i,w},\bar{r}_{i})$ have the same optimization variable, i.e., $\bar{r}_i$. Furthermore, the overall cost function $J_i(\cdot)$ is constructed with a convex combination of the two costs with $\lambda>0$. Enforcing $\lambda>0$ ensures that the combined cost function denoting reference tracking and bearing maintenance are positive definite even when bearing cost reduces to zero. After these modifications and assumptions on the convexity of the reference tracking and bearing maintenance cost, our problem is transformed into the same form as \cite{carron2020model, ferramosca2009mpc}, where the combined cost resembles the reference tracking cost. 
% The rest of the proof for recursively feasibility and convergence directly follows from the Assumption \ref{ass:mpc} and steps in \cite[Th. 1]{ferramosca2009mpc}. Moreover, as a result of Assumption \ref{ass:invariant}, any centroid is a valid reference position. Hence, the control inputs obtained from the solution of the MPC problem \eqref{eq:mpc} iteratively steer the system to the centroidal Voronoi configuration.      
% \end{proof}
% Since the terminal cost and terminal set do not explicitly depend on the relative bearing of the MRS, we utilize the same procedure proposed in \cite[Alg. 2]{carron2020model} to compute the terminal ingredients, i.e., terminal control law $\kappa(x_{i,l|k}, \bar{x}_{i})$, the terminal cost $l_{i,N}(x_{i,l|k}, \bar{x}_{i})$ and the terminal set $\mathcal{X}_{\text{terminal}}$.
\subsection{Rigidity Recovery under Agent Loss}
We will now show how maintaining bearing rigidity allows us to design a systematic procedure to recover an MRS network in the case of robot loss. Moreover, we want to recover the network topology to remain minimally rigid even after reconfiguration. This will ensure that even after the reconfiguration, the robots maintain a unique topology with a minimum number of links, thereby ensuring resiliency with minimum additional connections. In rigidity theory literature, such problems are called \textit{minimal-rigidity preserving closing ranks} problem. This can be formally defined as follows. 

Given a minimally rigid graph $\mathcal{G} = (\mathcal{V}, \mathcal{E})$ and let  $\mathcal{G}_f = (\mathcal{V} \backslash \{v_r\}, \mathcal{E} \backslash \mathcal{E}_r)$ be the graph obtained when a robot representing $v_r \in \mathcal{V}$ and all edges incident to $v_r$ denoted as $\mathcal{E}_r$ are removed from the graph $\mathcal{G}$. The goal is to find a minimum set of new edges $\mathcal{E}_n$ which must be added to $\mathcal{G}_f$ to yield it minimally rigid. We now present an important result that establishes the existence and lower bounds on the number of edges to be added to yield a minimal rigidity.
\begin{lemma}[Local Repair, Th. 3\cite{fidan2010closing}]
\label{lem:closing_rank}
Consider a vertex $v$ of degree $\alpha$ being removed from a $d$-dimensional minimally rigid graph, where $d=2$ or $d=3$ and $\alpha\geq d$. To restore minimal rigidity, $\alpha - d$ edges (or none if $\alpha =d$) must be added. Furthermore, these new edges should only connect the neighbors of $v$, without using any other vertex as an endpoint.
% Suppose a vertex $v$ of degree $\alpha$ is removed from a $d$-dimensional minimally rigid graph with $d=2$ or $d=3$ and $\alpha$ satisfying $\alpha \geq d$. To regain minimal rigidity, $\alpha-d$ edges (or no edges if $\alpha=d$) must be inserted. Moreover, the new edges must be only inserted between the neighbors of $v$, without using any other vertex as an end-vertex.       
\end{lemma}
\begin{algorithm}[t]
\caption{Rigidity Recovery using Edge contraction}\label{alg:recovery}
\begin{algorithmic}[1]
\Require MRS network topology $\mathcal{G} = (\mathcal{V}, \mathcal{E})$.
\For {$i =1,2\dots,n$}
    \For {$j =1,2\dots, \mathcal{N}_i$}
        \If{($\alpha_j$ == $d$) degree of the vertex $v_j$}
        \Return
        \Else
            \State Check whether $e_{i,p} = (v_i,v_p) \ \forall p \in \{1, \dots, \mathcal{N}_j\}$ is contractible or not using Lemma \ref{lem:non_contract}.
            \State $\mathcal{E}^r_{ij} \leftarrow \big\{e_{i,q}, q\big\}  \ \forall q \in \{1, \dots, \mathcal{N}_j\}$ such that $e_{i,q}$ is contractible. 
        \EndIf
    \EndFor
\EndFor
\end{algorithmic}
\end{algorithm}

While the above lemma illustrates how to regain graph rigidity when a vertex is lost, it doesn't provide a constructive approach for finding the minimum edge set required to regain rigidity. A systematic procedure to find the minimum edge set can be obtained by using \textit{edge contraction approach} \cite{fidan2010closing}. The idea is to merge two adjacent vertices $v_1,v_2 \in \mathcal{V}$ of a graph $\mathcal{G} = (\mathcal{V}, \mathcal{E})$ into a single vertex by ``contracting" the edge between them. This new vertex should be adjacent to all the neighbors of $v_1$ and $v_2$ without redundant edges (if any would exist). If the resulting graph after the edge contraction operation on an edge $e$ is minimally rigid, then the edge $e$ is called a contraction edge. Figure \ref{fig:closing_rank} shows the edge contraction operation on a minimally rigid graph. We can see in the right side of the figure that when the vertex $v$ (in red) is removed in a minimally rigid graph, we can find new edges $(v_1,v_3), (v_1, v_4), (v_1, v_5)$ using edge contraction operation on $(v, v_1)$. The resulting graph is not minimally rigid when the edge contraction is performed on $(v, v_2)$. Hence, we get a minimally-rigid graph. The sufficient condition to check whether a given edge is contractible \cite{fidan2010closing} is given by the following lemma:
%We now present a sufficient condition to check whether a given edge is not contractible.
\begin{lemma}[Edge Contraction\cite{fidan2010closing}]
\label{lem:non_contract}
Consider a minimally rigid graph $\mathcal{G} = (\mathcal{V}, \mathcal{E})$ in $d=2$. Let $v,w \in \mathcal{V}$ and $(v,w) \in \mathcal{E}$. The edge $(v,w)$ is non-contractible in $\mathcal{G}$ if multiple vertices (more than one vertex) are adjacent to both $v$ and $w$.
% Consider a minimally rigid graph $\mathcal{G} = (\mathcal{V}, \mathcal{E})$ with $d=2$. Let $v,w \in \mathcal{V}$ and $(v,w) \in \mathcal{E}$. The edge $(v,w)$ is not contractible in $\mathcal{G}$ if more than one vertex is adjacent to $v$ and $w$.
\end{lemma}

Using the results from Lemmas \ref{lem:closing_rank} and \ref{lem:non_contract}, we design a proactive rigidity recovery algorithm as shown in Alg. \ref{alg:recovery}. Using the results from Lemma \ref{lem:closing_rank}, we are only required to search locally for the contractible edges among the $2$-hop neighbors of the robot. The algorithm checks all possible contractible edges for the neighbor of each robot in the MRS network. The idea is to compute recovery edge sets, $\mathcal{E}^r_{ij}: \big\{ \{(i,j), q\} | \ i,j \in \mathcal{V}, q \in \{i,j\} \big\} \quad \forall i \in \{1,\dots,n\},  \ \forall j \in \mathcal{N}_i$, comprising new recovery edges $\big\{(i,j), \ i,j \in \mathcal{V}\big\}$ and the corresponding contraction vertex $q \in \{i,j\}$ for each robot $i$ by anticipating a failure in each of the neighbors $j$. This ensures that each robot can reconfigure itself by making new connections with the help of new recovery edges in the event of robot loss \footnote{Note that these results are applicable only for $d=2$. The results from the planar case do not scale well for higher dimensions \cite{fidan2010closing}.}. Moreover, the set of recovery edges also guarantees a minimally rigid graph after reconfiguration. 
\begin{figure}[h]
\centering
\includegraphics[width=0.5\textwidth, trim={0.5cm 0.5cm 0.5cm 0.75cm}]{sections/images/edge_contraction.pdf}
\caption{Edge contraction operation for Rigidity recovery.}
\label{fig:closing_rank}
\end{figure} 
\begin{figure*}[t]
\centering
        \centering
        \includegraphics[width=0.8\textwidth, trim={2.8cm 1.5cm 1.5cm 1.5cm},clip]{sections/images/coverage.pdf}
\caption{Robot positions and Voronoi partitions along with bearing formation at time $k=0$, $50$, $140$.}
\label{fig:coverage}
\end{figure*}
\begin{figure}[ht]
\centering
\includegraphics[width=0.5\textwidth,trim={0.5cm 0.4cm 0.3cm 0.2cm},clip]{sections/images/coverage_cost.pdf}
\caption{Coverage cost evolution with time, for three different values of $\mu$: (a) without bearing maintenance $(\mu = 1)$ and (b) with bearing maintenance $(\mu = 0.7)$ and $(\mu = 0.1)$.}
\label{fig:cost}
\end{figure} 