% \title{Resilient Multi-Robot Coverage Control with Network Recovery Guarantees}
\title{On Enhancing Structural Resilience of Multirobot Coverage Control with Bearing Rigidity}
\author{ 
Kartik A. Pant, Vishnu Vijay, Minhyun Cho, and Inseok Hwang
% \thanks{This research is funded by the Secure Systems Research Center (SSRC) at the Technology Innovation Institute (TII), UAE. The authors are grateful to Dr. Shreekant (Ticky) Thakkar and his team members at the SSRC for their valuable comments and support.}
\thanks{The authors are with the School of Aeronautics and Astronautics, Purdue University,
West Lafayette, IN 47906. Email: ({\tt\small kpant}, {\tt\small vvijay}, {\tt\small cho515}, {\tt\small ihwang} {\tt \small@purdue.edu})}}
\maketitle
\thispagestyle{empty}
\begin{abstract}   
The problem of multi-robot coverage control has been widely studied to efficiently coordinate a team of robots to cover a desired area of interest. However, this problem faces significant challenges when some robots are lost or deviate from their desired formation during the mission due to faults or cyberattacks. Since a majority of multi-robot systems (MRSs) rely on communication and relative sensing for their efficient operation, a failure in one robot could result in a cascade of failures in the entire system. In this work, we propose a hierarchical framework for area coverage, combining centralized coordination by leveraging Voronoi partitioning with decentralized reference tracking model predictive control (MPC) for control design. In addition to reference tracking, the decentralized MPC also performs bearing maintenance to enforce a rigid MRS network, thereby enhancing the structural resilience, i.e., the ability to detect and mitigate the effects of localization errors and robot loss during the mission. Furthermore, we show that the resulting control architecture guarantees the recovery of the MRS network in the event of robot loss while maintaining a minimally rigid structure. The effectiveness of the proposed algorithm is validated through numerical simulations.     
\end{abstract}
\begin{keywords}
Model predictive control, multi-robot systems, coverage control, and bearing rigidity.
\end{keywords}