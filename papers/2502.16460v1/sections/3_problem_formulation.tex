In this section, we present the problem formulation for resilient multi-robot coverage control. %with recovery guarantees in the event of an agent loss. 
% We also present the robots' nonlinear constrained dynamics and communication/sensing networks. Furthermore, we present the idea of rigidity maintenance control to enforce connectivity constraints while simultaneously accomplishing the coverage task. This method provides a systematic procedure to ensure the system's recovery in the event of a robot's loss. 
We consider a multi-robot system with $n$ robots that can move in the region described by a polytope $\mathcal{Q}\subset \mathbb{R}^d$, with $d =2$ or $d = 3$. Each robot in the MRS is governed by the discrete-time nonlinear dynamics. The $i^\text{th}$ robot's dynamics is given by:
\begin{align} \label{eq:agent_dynamics}
    x_{i,k+1} &= f_{i}(x_{i,k}, u_{i,k})\\
     p_{i,k} &= C_{i}x_{i,k}
\end{align}
where  $x_{i,k} \in \mathbb{R}^{n_x}$, $u_{i,k} \in \mathbb{R}^{n_u}$ and $p_{i,k} \in \mathbb{R}^d$ are the state, the control input and the position of robot $i$, respectively. The matrix $C_i \in \mathbb{R}^{d \times n_x}$ picks the position from the state. All robots are subject to state and input constraints, i.e., $(x_{i,k},u_{i,k})  \in \mathcal{Z}_i$ for all $k\geq 0$, where $\mathcal{Z}_i \subset \mathbb{R}^{n_{x}+n_{u}}$ is a compact set.
%The state constraint also enforces constraints on the position of the robots such that $p_{i,k} \in \mathcal{P}_i$. 

The steady-state values of the state, input and reference position of the robots $\bar{x}_{i}, \bar{u}_{i}$ and $\bar{r}_{i}$ are defined such that the robot dynamics \eqref{eq:agent_dynamics} is satisfied, i.e.,
\begin{align} \label{eq:agent_dynamics_ss}
    \bar{x}_{i} &= f_{i}(\bar{x}_{i}, \bar{u}_{i})\\
     \bar{r}_{i} &= C_{i}\bar{x}_{i}
\end{align}

\begin{assumption}
\label{ass:invariant}
The dynamics of each robot $i$ \eqref{eq:agent_dynamics} is position invariant, that is, $x_{i,k+1} + \psi(p_{i,k}) = f_i(x_{i,k} + \psi(p_{i,k}),u_{i,k}) \ \forall p_{i,k} \in \mathbb{R}^d, x_{i,k}, u_{i,k}$, where $\psi(p): \mathbb{R}^d \rightarrow\mathbb{R}^{n_{x}}$ represents an operator that takes the position of robots as input and generates a vector of the dimension of the state of the robots, $\psi(p_{i,k}) = [p_{i,k}^\top, 0, \dots,0]^\top \in \mathbb{R}^{n_x}$.
\end{assumption}

As a consequence of Assumption \ref{ass:invariant}, the robots are restricted to have the dynamics where the position appears only in the integrator. In practice, most robotic systems, such as rovers, aerial vehicles, and underwater robots, satisfy this assumption.
% Each robot is subject to polytopic state and input constraints, i.e.,
% \begin{align}
%     \label{eq:st_ip_const}
%     \mathcal{X}_i &= \{\mathbf{x}_i, A_x \mathbf{x}_i \leq b_x \} \\
%     \mathcal{U}_i &= \{\mathbf{u}_i, A_u \mathbf{u}_i \leq b_u \}
% \end{align}
% where $A_x \in \mathbb{R}^{n_x\times n_x}, b_x \in \mathbb{R}^{n_x}, A_u \in \mathbb{R}^{n_x \times n_u}$ and $b_u \in \mathbb{R}^{n_u}$. 
\begin{assumption}
\label{ass:dynamics}
The robot's dynamics $f_i$ are Lipschitz continuous with Lipschitz constant $\mathcal{L}_i$. Furthermore, it is assumed that each robot's state can be measured perfectly.
\end{assumption}
%%% GRAPH TOPOLOGY: Vertices, Edges, Neighbors, Communication, Synchronization

For the sake of simplicity, we consider the MRS's sensing and communication network to be identical and described by a minimally rigid graph $\mathcal{G} = (\mathcal{V},\mathcal{E})$. The graph $\mathcal{G}$ is constructed using the Henneberg construction defined earlier in Section \ref{subsec:henneberg}. With the graph vertices representing the robots of the multi-robot system, the edges represent the measurements that can be computed, as well as the communication links between the robots. %Two robots are considered neighbors if an edge connects them, and the neighbors of robot $i \in \mathcal{V}$ are included in set $\mathcal{N}_i$. Using a similar notation, the set of edges connecting robot $i \in \mathcal{V}$ is denoted by $\mathcal{E}_i$.
\subsection{Voronoi-based Coverage Control}
Let the robot positions at time $k$ be denoted as $\mathbf{p}_k = [p_{1,k}^\top, \dots, p_{n,k}^\top]^\top \in \mathbb{R}^{n_x\times d}$. Let $\mathcal{O}_k = \{\mathcal{O}_{1,k}, \dots, \mathcal{O}_{n,k} \}$ be an arbitrary collection of polytopes that partition $\mathcal{Q}$. The coverage control problem can then be formalized as follows:
\begin{equation}
\label{eq:coverage_prob}
    \min_{\mathbf{p}_k,\mathcal{O}_k} \sum_{i=1}^n \int_{\mathcal{O}_{i,k}} \underbrace{\lambda(\norm{q - p_{i,k}}) \phi(q)dq}_{H(\mathbf{p}_k,\mathcal{O}_k)}, \quad p_{i,k} \in \mathcal{O}_{i,k},
\end{equation}
where $\phi: \mathcal{Q} \rightarrow\mathbb{R}_+$ is the \textit{density function} (or the information function) on the desired area $\mathcal{Q}$, $\lambda(\cdot)$ denotes the coverage performance of the robot at point $q$ from $p_{i,k}$, and $H(\mathbf{p}_k,\mathcal{O}_k)$ denotes the \textit{locational optimization cost}. Note that the optimal set $\mathcal{O}$ in \eqref{eq:coverage_prob}, when $\lambda(\cdot)$ is considered as the squared Euclidean norm, is given by the Voronoi tessellation \cite{cortes2004coverage,du2006convergence}. For a given position configuration $\mathbf{p}_k$, the partitions are denoted by $\mathcal{W}(\mathbf{p}_k) = \{ \mathcal{W}_{1}(p_{1,k}), \dots, \mathcal{W}_{n}(p_{n,k})\}$, with
\begin{equation}
    \mathcal{W}_{i}(\mathbf{p}_k) = \{ q \in \mathcal{Q} \ | \ \norm{q-p_{i,k}} \leq \norm{q - p_{j,k}}, \forall j  \neq i\}, 
\end{equation}
where $\norm{\cdot}$ denotes the Euclidean norm. If the configuration space $\mathcal{Q}$ is convex, all the sets in the partition $\mathcal{W}(\mathbf{p}_k)$ are also convex \cite{du2006convergence}. The optimal position of each robot is given by the centroids defined by $\mathbf{c}(\mathcal{W}(\mathbf{p}_k), \phi) = \big[c_1(\mathcal{W}_1(p_{1,k}), \dots, c_M(\mathcal{W}_n(p_{n,k})\big]$ with 
\begin{equation}
\label{eq:centroid}
    c_i(\mathcal{W}_i(\mathbf{p}_k), \phi) = \left( \int_{\mathcal{W}_i(\mathbf{p}_k)} \phi(q)dq\right)^{-1} \left(\int_{\mathcal{W}_i(\mathbf{p}_k)}q\phi(q)dq \right).
\end{equation}
The resulting locally optimal solution utilizing Voronoi tesselation for both partitioning and positions of the robots is called a \textit{centroidal Voronoi configuration}. Thus, the solution to the coverage problem defined in \eqref{eq:coverage_prob} reduces to ensuring that the position of each robot $p_i$ converges to the centroid of the Voronoi configuration $\mathbf{c}(\mathcal{W}_i(p_{i,k}), \phi)$. For the single integrator dynamics, this is achieved using Lloyd's algorithm \cite{cortes2004coverage}, which iteratively updates the position of each of the robots as $p_{i,k+1} = c_i(\mathcal{W}_i(p_{i,k}), \phi)$. For the nonlinear dynamics with input and state constraints, the convergence to the centroid can be obtained by utilizing a nonlinear output-tracking MPC controller \cite{carron2020model}. 
%For the robot dynamics with state and input constraints, the coverage control can be posed as a nonlinear output-tracking MPC problem \cite{carron2020model}. It has been shown that the system of robots converges to the centroidal Voronoi configuration while satisfying its dynamics, state, and input constraints. 

In most of the existing literature, the underlying network in a coverage control setting is assumed to be connected, and all robots are considered healthy, i.e., exhibit no failure/anomaly, which may not be practical. To this end, we focus on enhancing the resilience of an MRS network for coverage control tasks, particularly with minimal communication and sensing constraints, allowing robots to have a greater degree of freedom. The main challenge is to design a distributed algorithm that not only ensures optimal coverage but also ensures the robot stays connected and guarantees the system's recovery in case of a robot loss. This will allow the remaining robots to reorganize and accomplish the coverage task despite unknown failures on a subset of robots. Thus, the main goal is to design a distributed and resilient coverage control algorithm for the nonlinear robot dynamics, which ensures the optimality of coverage, connectivity, and self-healing capabilities in the event of robot failures. 
\begin{remark}
Note that in this work, we focus on a single robot loss at any given time and devise a recovery mechanism that ensures guaranteed network reconfiguration with minimal additional links. We posit that designing decentralized recovery mechanisms for simultaneous multiple robot losses is a significantly challenging problem, as it can result in highly abrupt changes in the network topology. We defer it for our future work.
\end{remark}
% \red{In this paper, however, we focus on guaranteeing the system's recovery in case of a robot failure so that the remaining robots can reorganize and accomplish the coverage task. To do so, we utilize bearing rigidity as a key concept in guaranteeing the MRS's recovery.}
%However, this work considers robot dynamics with polytopic state and input constraints, as well as network connectivity constraints. %Furthermore, we focus on the system's resilience in the event of an agent loss.
