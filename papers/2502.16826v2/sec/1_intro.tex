\section{Introduction}
\label{sec:intro}

With the proliferation of 3D scanners and depth cameras, the capture and processing of 3D point clouds have become commonplace. However, the captured data are often corrupted by noise due to factors like sensor errors and environmental conditions, which can significantly affects downstream tasks.
Removing noise from point clouds—which consist of discrete 3D points sampled irregularly from continuous surfaces—is a long-standing challenge due to their disordered nature and lack of connectivity information. Existing work includes traditional optimization-based approaches and deep learning approaches. Traditional methods often rely on surface normal estimation, predefined geometric priors, and cumbersome optimization techniques, which can lead to unsatisfactory results.

With the continuous emergence of neural network architectures designed for point clouds \cite{qi2017pointnet,qi2017pointnet2,KPconv,wang2019dynamic}, deep learning methods have recently shown promising denoising performance. 
In supervised settings, training denoising models requires pairs of noisy and clean data. Most of these models predict the displacement of noisy points relative to the underlying surface and then apply the inverse displacement to the noisy point cloud. However, in many real-world scenarios, obtaining clean data is difficult or  impractical. Therefore, unsupervised denoising has become a compelling research area.

Meanwhile, various unsupervised image denoising methods have been proposed, including Noise2Noise \cite{2018Noise2NoiseLI}, Noise2Void \cite{2018Noise2VoidL}, Noise2Self \cite{2019Noise2SelfBD}, Noise2Same \cite{Xie2020Noise2SameOA}, \textit{etc.} Such Noise2X methods are trained by minimizing variants of empirical risk variants.
Noise2Score \cite{kim_noise2score_2021} introduces a novel method for image denoising with the Tweedie formula for any exponential family noise. The Tweedie's formula \cite{efron2011tweedie} provides an explicit way to compute  posterior estimation from noisy measurements with exponential family noise given the score function (\textit{i.e.}, the gradient of the log-likelihood).
By reframing self-supervised image denoising as the problem of estimating a score function, Noise2Score offers a new perspective that provides important theoretical implications and flexibility in algorithmic implementation. 

Actually, score function estimation has long been a significant research topic in Bayesian statistics and machine learning \cite{hyvarinen2005estimation, vincent2011connection, alain2014regularized}. In particular, Alain and Bengio \cite{alain2014regularized} showed that minimizing the objective function of a denoising autoencoder (DAE) provides an explicit way to approximate the score function. The use of the amortized residual denoising autoencoder (AR-DAE)\cite{lim2020ar-dae} further extends this result by combining it with Tweedie's formula to obtain an unified framework for self-supervised image denoising using Bayesian methods.

Inspired by this, we propose unsupervised point cloud denoising method - Noise2Score3D that combines Tweedie's formula and score function estimation to effectively denoise point clouds. Our method achieves state-of-the-art performance among unsupervised methods and offers improved generalization ability for difference noise levels without re-training. Furthermore, the proposed framework can be easily adapted to other type of noise models, enhancing the method's applicability.

The main contributions of this paper are as follows:
\begin{itemize}
\item We propose an efficient unsupervised framework for point cloud denoising that learns the score function from noisy point clouds and performs denoising in only one step. The trained model can be applied to datasets with different noise levels and real world noise without re-training.
\item We introduce a metric to assess the quality of denoised point clouds, enabling estimation of unknown noise parameters thereby making our method broadly applicable to real-world data.
\item Our experiments demonstrate that, compared with existing unsupervised methods, Noise2Score3D achieves state-of-the-art results across different datasets and noise levels in terms of both quantitative metrics and visual quality.
\end{itemize}

%-------------------------------------------------------------------------
\begin{figure*}
  \centering
    \includegraphics[width=1.0\textwidth]{figures/KPCONV.pdf}

  \caption{Denoising workflow of Noise2Score3D with feature extraction and score prediction by KPConv architecture, followed by post-processing using Tweedie's formula.}
  \label{fig:kpconv}
\end{figure*}
%-------------------------------------------------------------------------