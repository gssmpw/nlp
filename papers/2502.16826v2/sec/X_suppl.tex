\clearpage
\renewcommand\thesection{\Alph{section}}
\renewcommand\thesubsection{\thesection.\arabic{subsection}}
\setcounter{page}{1}
\setcounter{section}{0}
\maketitlesupplementary


\section{Total Variation for Point cloud Denoising}
\label{sec:TV}

In this section, we provide a detailed explanation of Total Variation (TV) regularization technique for point cloud denoising, as it pertains to evaluating the quality of point clouds and images in our work. 

\subsection{Total Variation (TV) in image processing}
TV regularization is a commonly used method for image processing \cite{RUDIN1992}, designed to reduce noise while preserving important image features such as edges. The TV norm of an image \( O \in \mathbb{C}^{N \times N} \) is defined as the sum of the magnitudes of its discrete gradient, which can be expressed as \cite{An2022EnhancedTV}
\begin{align}
TV(O) &= \sum_{j,k} \nabla O_{j,k} \notag \\
&= \frac{1}{(j-1)k} \sum_{j-1,k} (O_x)_{j,k} + \frac{1}{j(k-1)} \sum_{j,k-1}(O_y)_{j,k}.
\end{align}
Here, the directional derivatives of the image \( O \) are computed in a pixel-wise manner as follows:
\begin{equation} 
(O_x)_{j,k} = O_{j+1,k} - O_{j,k},
(O_y)_{j,k} = O_{j,k+1} - O_{j,k}.
\end{equation}
The discrete gradients \( O_x \) and \( O_y \) represent the changes in pixel values in the horizontal and vertical directions, respectively.
The intuition behind Total Variation is that images with high-frequency details, such as noise and sharp edges, will have a large TV value. In our study, we adapt the concept of TV as a measure of quality. Specifically, TV serves as a penalty term to quantify the roughness or irregularity in the processed image or point cloud data. This adaptation allows us to assess whether the denoising method has effectively smoothed the data without introducing excessive distortion.

TV is primarily used as a quality metric to evaluate the denoised result by calculating the remaining high-frequency content. A lower TV value indicates a smoother output with fewer irregularities, whereas a higher TV value suggests the presence of residual noise or structural inconsistencies. In TV regularization, the optimization process seeks to minimize the total variation of an image \( O \), reducing sharp pixel intensity changes that correspond to noise, while preserving significant transitions corresponding to image edges.
%-------------------------------------------------------------------------
\begin{table}[tb]
    \begin{center}
     \resizebox{\columnwidth}{!}   &\multicolumn{2}{c}{2\%} &\multicolumn{2}{c|}{3\%} 
              &\multicolumn{2}{c}{1\%}   &\multicolumn{2}{c}{2\%} &\multicolumn{2}{c}{3\%}   \\  \midrule
    \multicolumn{1}{c|}{Metrics}  & CD & P2M   & CD  & P2M   & CD & P2M   & CD & P2M   & CD& P2M    & CD& P2M        \\
    \midrule
    \multicolumn{1}{c|}{\multirow{2}{*}{$\sigma_\text{true}$}}
    &\multicolumn{2}{c}{0.01} &\multicolumn{2}{c}{0.02}    &\multicolumn{2}{c|}{0.03}  
    &\multicolumn{2}{c}{0.01} &\multicolumn{2}{c}{0.02}    &\multicolumn{2}{c}{0.03} \\
    \multicolumn{1}{c|}{} 
    & \bf2.806   & \bf0.661      & \bf3.868  & \bf1.173    & \bf5.192 & \bf2.228   
    & 0.991 & \bf0.317        & \bf1.438  & \bf0.623       & \bf2.418  & \bf1.389 \\
    \midrule
    \multicolumn{1}{c|}{\multirow{2}{*}{$\sigma_\text{tv}$}} 
    
    &\multicolumn{2}{c}{0.0151}  &\multicolumn{2}{c}{0.0215}  &\multicolumn{2}{c|}{0.0303}     
    &\multicolumn{2}{c}{0.0122} &\multicolumn{2}{c}{0.0207}  &\multicolumn{2}{c}{0.0285}  \\
    \multicolumn{1}{c|}{} 
    & 2.848  & 1.106           & 4.190   & 1.818         & 5.583   & 2.947           
    & \bf0.833   & 0.461          & 1.532  & 0.970          & 2.434   & 1.704  \\
    \midrule
    \multicolumn{1}{c|}{Error(\%)}  &\multicolumn{2}{c}{51\%}  &\multicolumn{2}{c}{7.5\%} &\multicolumn{2}{c|}{1\%}   
    &\multicolumn{2}{c}{22\%}  &\multicolumn{2}{c}{3.5\%} &\multicolumn{2}{c}{5\%}     \\ 
\bottomrule
\end{tabular}
%-------------------------------------------------------------------------
% \begin{tabular}{cccc|ccc}
%     \toprule
%     \bf PU-Net & \multicolumn{3}{c|}{10K} & \multicolumn{3}{c}{50K}  \\
%     Inference & 1\%  & 2\%  & 3\%      & 1\%  & 2\%  & 3\%\\
%     \midrule
%     $\sigma_\text{true}$   & 0.01 & 0.02 & 0.03      & 0.01 & 0.02 & 0.03 \\
%     \midrule
%     $\sigma_\text{tv}$ &  0.0151 &  0.0215  &  0.0303     
%                 &  0.0122 &  0.0207  &  0.0285  \\
%     \midrule
%     Error(\%)  & 51\%  & 7.5\% & 1\%   & 22\%  & 3.5\% & 5\%      \\ 
% \bottomrule
% \end{tabular}
%------------------------------------------------------------------------- }%
    \end{center}
    \vspace{-0.05in}
    \caption{Estimated noise parameter ($\sigma$) by $TV_{PC}$ applied to the point cloud of the PU-Net \cite{yu2018PUNet} dataset vs. true values.}
    \label{table:sigma}
\end{table}
%-------------------------------------------------------------------------


\subsection{Total Variation for Point Cloud}

The concept of Total Variation has inspired us to propose similar metrics for point cloud data, which enables denoising with unknown noise parameters. In particular, we extend the idea of minimizing gradient magnitudes to point cloud data, where the goal is to reduce noise by minimizing the geometric differences between points and their neighbors. We introduce Total Variation for Point Cloud  ($TV_{PC}$), which is defined as: 
\begin{equation} 
\label{eq:TVPC}
TV_{PC} = \sum_{i=1}^N \sum_{j \in \text{neighbors}(i)} w_{i,j} \cdot \sqrt{\| \mathbf{p}_i - \mathbf{p}_j \|^2 + \epsilon^2}
\end{equation}
where \( w_{i,j} \) represents the weight between point \( \mathbf{p}_i \) and its neighbor \( \mathbf{p}_j \). 

$TV_{PC}$  measures the geometric difference between a point $p_i$ and its $k$ nearest neighboring points, $p_{\text{neighbors}[i]}$.
$\sqrt{\| \mathbf{p} \|^2 + \epsilon^2}$ in (\ref{eq:TVPC}) is a smooth approximation of the \(L_1\) norm, which is introduced in \cite{Charbonnier} to handle outliers and maintain robustness. Here, $\epsilon$ is a small positive constant, typically set to $10^{-4}$, which helps to smooth the variations and prevent over-penalization of boundary outliers, ensuring numerical stability. We note that $TV_{PC}$ is similar to the Graph Laplacian Regularizer but with \(L_1\) loss. 
For simplicity we set \( w_{i,j} \) to a constant in our method. Another choice of \( w_{i,j} \) is 
\begin{equation} 
w_{i,j} = \exp\left(-\frac{\| \mathbf{p}_i - \mathbf{p}_j \|^2}{2\sigma^2}\right),
\end{equation}
with \( \sigma \) controlling the scale of the Gaussian kernel \cite{chen2022deep} .


From \cref{table:sigma}, we observe that our proposed $TV_{PC}$ provides an effective denoising metric for Gaussian-distributed noisy point clouds, enabling the identification of noise parameters across different noise levels. This validates our method in blind estimation of noise parameters in real world datasets.

\section{Additional visual results}
\cref{fig:suppl_modelnet1,fig:suppl_modelnet2,fig:suppl_modelnet3,fig:suppl_modelnetsim05,fig:suppl_modelnetsim10,fig:suppl_modelnetsim15,fig:suppl_punet1,fig:suppl_punet2,fig:suppl_punet3} provide additional qualitative and visual results from ModelNet-40 and PU-Net. Points with yellower coloring are farther from the ground truth surface. Note that our main text \cref{fig:visualization} only shows the noise scale set to 2\% of the bounding sphere’s radius for Gaussian noise and 1.5\% for simulated LiDAR noise. 

%-------------------------------------------------------------------------
\begin{figure*}
\begin{center}
    \includegraphics[width=1.0\textwidth]{figures/suppl_modelnet010.pdf}
\end{center}
\vspace{-0.05in}
\caption{Visual comparison of additional denoising results of different algorithms on ModelNet-40 with Gaussian noise. The noise level is set to 1\%.}
\label{fig:suppl_modelnet1}
\end{figure*}

\begin{figure*}
\begin{center}
    \includegraphics[width=1.0\textwidth]{figures/suppl_modelnet020.pdf}
\end{center}
\vspace{-0.05in}
\caption{Visual comparison of additional denoising results of different algorithms on ModelNet-40 with Gaussian noise. The noise level is set to 2\%.}
\label{fig:suppl_modelnet2}
\end{figure*}

\begin{figure*}
\begin{center}
    \includegraphics[width=1.0\textwidth]{figures/suppl_modelnet030.pdf}
\end{center}
\vspace{-0.05in}
\caption{Visual comparison of additional denoising results of different algorithms on ModelNet-40 with Gaussian noise. The noise level is set to 3\%.}
\label{fig:suppl_modelnet3}
\end{figure*}
%-------------------------------------------------------------------------
%-------------------------------------------------------------------------
\begin{figure*}
\begin{center}
    \includegraphics[width=1.0\textwidth]{figures/suppl_punet010.pdf}
\end{center}
\vspace{-0.05in}
\caption{Visual comparison of additional denoising results from PU-Net dataset. The noise level is set to 1\%.}
\label{fig:suppl_punet1}
\end{figure*}

\begin{figure*}
\begin{center}
    \includegraphics[width=1.0\textwidth]{figures/suppl_punet020.pdf}
\end{center}
\vspace{-0.05in}
\caption{Visual comparison of additional denoising results from PU-Net dataset. The noise level is set to 2\%.}
\label{fig:suppl_punet2}
\end{figure*}

\begin{figure*}
\begin{center}
    \includegraphics[width=1.0\textwidth]{figures/suppl_punet030.pdf}
\end{center}
\vspace{-0.05in}
\caption{Visual comparison of additional denoising results from PU-Net dataset. The noise level is set to 3\%.}
\label{fig:suppl_punet3}
\end{figure*}
%-------------------------------------------------------------------------
%-------------------------------------------------------------------------
\begin{figure*}
\begin{center}
    \includegraphics[width=1.0\textwidth]{figures/suppl_sim005.pdf}
\end{center}
\vspace{-0.05in}
\caption{Visual comparison of additional denoising results from ModelNet-40 dataset with simulated LiDAR noise. The noise level is set to 0.5\%.}
\label{fig:suppl_modelnetsim05}
\end{figure*}

\begin{figure*}
\begin{center}
    \includegraphics[width=1.0\textwidth]{figures/suppl_sim010.pdf}
\end{center}
\vspace{-0.05in}
\caption{Visual comparison of additional denoising results from ModelNet-40 dataset with simulated LiDAR noise. The noise level is set to 1\%.}
\label{fig:suppl_modelnetsim10}
\end{figure*}

\begin{figure*}
\begin{center}
    \includegraphics[width=1.0\textwidth]{figures/suppl_sim015.pdf}
\end{center}
\vspace{-0.05in}
\caption{Visual comparison of additional denoising results from ModelNet-40 dataset with simulated LiDAR noise. The noise level is set to 1.5\%.}
\label{fig:suppl_modelnetsim15}
\end{figure*}
%-------------------------------------------------------------------------
