\section{Introduction}


\ac{OOD} detection represents a fundamental machine learning challenge focused on identifying data samples that deviate from expected inlier distributions. This capability is particularly crucial in safety-critical applications like robotics and autonomous driving, where accurate identification of anomalous motion trajectory \citep{diag_teaching} samples can prevent system failures.
Recent advances in \ac{OOD} detection have explored various unsupervised approaches to learn inlier data representations. These include likelihood-based methods that employ different likelihood measures for \ac{OOD} determination~\citep{NEURIPS2020_8965f766, NEURIPS2019_1e795968, choi2019generative, RAN2022199}, and reconstruction-based approaches that utilize pretrained generative models to assess sample similarity~\citep{denouden2018improvingreconstructionautoencoderoutofdistribution, Wyatt_2022_CVPR, Graham_2023_CVPR}. However, these methods typically require dataset-specific training, necessitating retraining for different \ac{ID} and \ac{OOD} datasets~\citep{heng2024out}. Recent research~\citep{xiao2021reallyneedlearnrepresentations} has addressed this limitation by exploring single discriminative models for \ac{OOD} detection. Our work similarly aims to develop unified OOD approaches that eliminate retraining requirements.

\begin{figure}[t]
    \centering
\fbox{\includegraphics[width=0.75\linewidth]{imgs/illustration/intro.png}}
    \caption{Sequences of rigid poses are abundant in displines that pertain to objects moving in the real world. We propose $\mathbf{DOSE3}$, a unified diffusion model over the $\mathbb{SE}(3)$ manifold to accurately detect out-of-distribution pose sequences.}
    \label{fig:intro}
\end{figure}

Current trajectory \ac{OOD} detection research primarily focuses on Latent Euclidean spaces, often overlooking explicit manifold space structures. Our work targets OOD detection for rigid body pose data, encompassing both position and orientation information. This type of data is fundamental to numerous applications in physics, engineering, and robotics that analyze object pose evolution over time \citep{unifying, darkgs}. We present theoretical insights and practical algorithms for detecting OOD data in rigid body pose sequences. Our framework, \emph{\textbf{D}iffusion-based \textbf{O}ut-of-distribution detection on $\mathbb{SE}(3)$} ($\mathbf{DOSE3}$), introduces a novel unified generative approach for trajectory space OOD detection. We define a manifold-specific diffusion process for rigid transformations on $\mathbb{SE}(3)$ and develop a high-dimensional OOD statistic for out-of-distribution sample identification.
We validate our approach using established robotics and automation datasets, creating benchmarks from Oxford RobotCar~\citep{RobotCarDatasetIJRR}, KITTI~\citep{Geiger2012CVPR}, and IROS20~\citep{wen2020se}. These datasets enable comprehensive evaluation across varying OOD similarity levels. Our key contributions include:
\begin{enumerate}
\item The $\mathbf{DOSE3}$ framework that \emph{diffuses} over $\mathbb{SE}(3)$ sequences, incorporating manifold structures into OOD detection.
\item A novel OOD statistic derived from our $\mathbb{SE}(3)$ manifold diffusion estimator for sample degree measurement.
\item Comprehensive empirical validation demonstrating $\mathbf{DOSE3}$'s effectiveness in distinguishing between in-distribution and OOD samples across diverse real-world trajectory datasets.
\end{enumerate}
By connecting diffusion models with trajectory OOD detection, $\mathbf{DOSE3}$ advances the development of robust and scalable methods for autonomous systems and 3D trajectory analysis applications.