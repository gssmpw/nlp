\section{Related Work}
\textbf{\ac{OOD} detection:}
\ac{OOD} detection plays a crucial role in safety-critical applications such as autonomous driving. Existing methods can generally be categorized into likelihood-based and reconstruction-based approaches. 

Likelihood-based \ac{OOD} detection methods involve training a model on \ac{ID} data and deriving a likelihood statistic from test samples to serve as an \ac{OOD} metric. Early work focused on learning discriminative representations to detect \ac{OOD} samples and identify distributional shifts~\citep{denouden2018improvingreconstructionautoencoderoutofdistribution, NEURIPS2020_8965f766}. More recent research has explored generative models due to their ability to model high-dimensional data and facilitate likelihood estimation~\citep{NEURIPS2020_eddea82a}. However, studies have shown that generative models may assign higher likelihoods to \ac{OOD} samples than to \ac{ID} ones~\citep{nalisnick2019a, hendrycks2019oe}. 

To address this issue, various refinements have been proposed, including likelihood ratios~\citep{NEURIPS2019_1e795968}, \ac{WAIC}~\citep{choi2019generative}, improved noise contrastive priors~\citep{RAN2022199}, and \ac{EBM}s~\citep{liu2020energy}. However, these enhancements remain ineffective in high-dimensional scenarios~\citep{Graham_2023_CVPR}. Another approach considers measuring how \emph{typical} a test input is~\citep{nalisnick2020detecting}, but this method suffers from poor performance at the sample level. Normalizing flows~\citep{NEURIPS2018_d139db6a} have also been investigated for \ac{OOD} detection as they provide direct likelihood estimation, yet they still suffer from overconfidence issues~\citep{NEURIPS2020_ecb9fe2f}. 

Reconstruction-based \ac{OOD} detection methods, on the other hand, aim to reconstruct input samples and compare them to their reconstructions to measure similarity. Early work used the reconstruction probability of VAEs~\citep{an2015variational, kingma2013auto} for anomaly detection. However, later studies found that \ac{OOD} samples can exhibit similar or even lower reconstruction errors compared to \ac{ID} samples, reducing the effectiveness of this approach~\citep{denouden2018improvingreconstructionautoencoderoutofdistribution}. 

\textbf{Diffusion-based OOD Detection:}
Diffusion models (DMs) have achieved remarkable performance in generative tasks across various modalities, including images~\citep{ho2020denoising, song2021scorebased}, videos~\citep{ho2022video}, and audio~\citep{chen2020wavegradestimatinggradientswaveform}. More recently, research has emphasized the robustness of DMs in sampling and their potential use in \ac{OOD} detection. Utilizing the reconstruction mean squared error (MSE) of DDPMs as an \ac{OOD} score has been shown to enhance image-space \ac{OOD} detection~\citep{Wyatt_2022_CVPR, Graham_2023_CVPR}. However, these models require retraining for different in-distribution datasets. 

A growing trend in machine learning research is the development of unified learning frameworks that generalize across various tasks~\citep{xiao2021reallyneedlearnrepresentations}. In an effort to construct a unified DM for \ac{OOD} detection, DiffPath~\citep{heng2024out} demonstrated that the rate of change and curvature of the forward diffusion trajectory can serve as effective \ac{OOD} metrics, eliminating the need for retraining on different datasets. 

The aforementioned \ac{OOD} detection methods—whether likelihood-based or reconstruction-based—are primarily focused on images or Euclidean space data. In contrast, some research from the robotics community implicitly incorporates \ac{OOD} detection under the framework of trajectory planning or optimization in 2D spaces~\citep{PDMP, 5354448, 5509799}. However, these models struggle to generalize to complex real-world scenarios that involve three-dimensional interactions~\citep{wang2023efficienttrajectorygenerationground, 10.1016/j.vehcom.2024.100733}. While there is a growing interest in extending diffusion models to non-Euclidean spaces ~\citep{huang2022riemannian, leach2022denoising}, these methods are limited to generating individual samples on manifolds rather than modeling entire trajectories. 

To the best of our knowledge, \textbf{DOSE3} is the first approach to leverage manifold-based diffusion over entire trajectories, enabling a unified \ac{OOD} detection framework.