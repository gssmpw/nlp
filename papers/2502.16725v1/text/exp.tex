\section{Experiments}
\subsection{Experiment Setup}
To evaluate $\mathbf{DOSE3}$'s validity, performance, and comprehensiveness, we conduct \ac{OOD} testing using the following $\mathbb{SE}(3)$ datasets:
\begin{itemize}
\item $\textbf{Oxford RobotCar}$~\citep{RobotCarDatasetIJRR}: This autonomous driving dataset encompasses over 1000 km of driving data from central Oxford, UK. It features multiple sensor modalities, including high-resolution stereo and monocular cameras, 2D and 3D LiDAR scans, and GPS/INS ground truth localization. Our experiments utilize the 3D LiDAR scans and ground-truth poses stored in $\mathbb{SE}(3)$ format.
\item $\textbf{KITTI}$~\citep{Geiger2012CVPR}: This comprehensive odometry dataset captures autonomous driving scenarios across urban, suburban, and rural environments. The dataset provides stereo and monocular camera imagery, 3D point clouds from a Velodyne LiDAR, and precise GPS/INS measurements. We utilize its pose data represented in $\mathbb{SE}(3)$
\item $\textbf{iros20-6d-pose-tracking}$~\citep{wen2020se}: This dataset advances research in 6D object pose estimation and tracking in dynamic environments. It is specifically designed to support the development and evaluation of algorithms for accurately determining and tracking six degrees of freedom (6D) poses in real-world scenarios.
\end{itemize}
For our diffusion model implementation, we standardize the input length for both $\mathbb{R}^{3}$ and $\mathbb{SE}(3)$ trajectory diffusion. Each trajectory in the datasets is segmented into fixed-length sub-paths of size 128 during experiments.
To standardize the translation data, we first center each trajectory by setting its starting coordinate to the origin, then normalize by dividing by the maximum translation value. This process constrains the translation data to the range [-1, 1], ensuring the model learns trajectory geometry independent of scale.
We evaluate $\mathbf{DOSE3}$ against leading \ac{OOD} detection methods, including \ac{JEM}\citep{Grathwohl2020Your} and Glow Model\citep{NEURIPS2018_d139db6a} with Likelihood Ratio~\citep{NEURIPS2019_1e795968}. These established baselines effectively handle high-dimensional inputs and are widely used for \ac{OOD} detection in image datasets.
\subsection{Quantitative evaluation of $\epsilon_{\theta}$ Distribution as an OOD Metric}
We analyze the statistical distribution of $\epsilon_{\theta}$ from inlier data to assess its effectiveness as an \ac{OOD} detection metric. Specifically, we investigate how the $\epsilon_{\theta}$ distribution of the $\mathbb{SO}(3)$ diffusion contributes to OOD sample identification. In~\cref{fig:eps_oxford_kitti}, we present a comparative analysis of $\epsilon$ distributions between Oxford RobotCar and KITTI datasets, using a model trained on KITTI. Our findings reveal that after translation data normalization, the translation $\epsilon$ distributions show substantial overlap across datasets, making them unsuitable as reliable OOD indicators. However, the rotation distribution, especially along the z-axis, demonstrates clear dataset separation. For the KITTI-trained model, we observe that KITTI's rotation distribution is centered at 0, aligning with standard Gaussian noise sampling characteristics. In contrast, the Oxford RobotCar dataset exhibits a notable rightward shift in its distribution, suggesting that reconstructing a KITTI sample from Oxford RobotCar input requires a non-Gaussian sampling distribution.

\begin{figure}[h!]
%\vskip 0.2in
\centering
    \subfigure[Translation Metric]{
        \includegraphics[width=0.45\linewidth]{imgs/eps_trans.png}
    }
    \hfill
    \subfigure[Rotation Z axis metric]{
        \includegraphics[width=0.45\linewidth]{imgs/eps_rot3.png}
    }
\vskip -0.2in
\caption{$\epsilon$ distribution retrieved by OOD testing on a model trained from KITTI dataset against the Oxford RoboCar dataset. We observe that the distributions of both datasets is particularly evident on the Z-axis rotation.}
\label{fig:eps_oxford_kitti}
\end{figure}

\subsection{Quantitative Results}
Table~\ref{tab:results} presents the OOD detection performance across datasets using the AUROC metric. All evaluated models underwent unsupervised training exclusively on the KITTI dataset. Our $\mathbb{SE}(3)$ model demonstrates exceptional performance, achieving near-perfect AUROC scores across all \ac{ID} and \ac{OOD} dataset combinations. In contrast, JEM, Glow-LR, and the $\mathbb{R}^{3}$ model show degraded performance when evaluating KITTI as an \ac{OOD} dataset, or when testing on the previously unseen Oxford Robot and IROS20 datasets. Additionally, we evaluate the impact of rotation metric splitting in $\mathbb{SE}(3)$ diffusion. The results indicate that separating the rotation metric space substantially enhances $\mathbf{DOSE3}$'s robustness and unified feature representation, particularly improving AUROC scores in scenarios where the training dataset, KITTI, serves as the \ac{OOD} data.
\begin{table}[t]
\caption{AUROC $\uparrow$ of OOD Detection. $\textbf{Bold}$ denotes the best
result.\\(O: Oxford, K: KITTI, I:IROS20)}
    \vskip 0.15in
    \begin{adjustbox}{width=0.68\linewidth,center}
    \begin{tabular}{lccccccc}
    \toprule
        \textbf{Method} & \textbf{O/K} & \textbf{K/O} & \textbf{O/I} & \textbf{I/O} & \textbf{K/I} & \textbf{I/K}\\
    \midrule
        JEM & 0.211 & 0.786 & 0.437 & 0.561 & 0.631 & 0.336\\
        Glow-LR  & 0.461 & 0.556 & 0.470 & 0.539 & 0.529 & 0.454\\
    \midrule
        $\mathbb{R}^{3}$-KITTI  & 0.362 & 0.770 & 0.417 & 0.585 & 0.793 & 0.409\\
        SE3-KITTI  & 0.845 & 0.952 & \textbf{1.000} & 0.398 & \textbf{1.000} & 0.234\\
          (No split) &&&&&&\\
        SE3-KITTI  & \textbf{1.000} & \textbf{0.956} & \textbf{1.000} & \textbf{0.931} & \textbf{1.000} & \textbf{0.897}\\
        % Path Sign-Oxford  & x & x & x & 0.xxx & x & x\\
    \bottomrule
    \end{tabular}
    \end{adjustbox}
    \vskip -0.1in
    \label{tab:results}
\end{table}

\subsection{Ablations}
\subsubsection{Trajectory Dataset used for Training}
$\mathbf{DOSE3}$ strives to develop a single unified model for effective OOD detection. We evaluate both $\mathbb{R}^3$ and $\mathbb{SE}(3)$-based diffusion models trained on different datasets. Table~\ref{tab:ablation_dataset} presents these results, highlighting two key findings:
\begin{enumerate}
\item $\mathbb{SE}(3)$ diffusion consistently demonstrates robust \ac{OOD} detection capabilities across various training datasets;
\item $\mathbb{SE}(3)$ diffusion successfully performs \ac{OOD} detection between two previously unseen datasets during training.
\end{enumerate}
We observe some performance degradation when training with the Oxford Robot Car dataset. This limitation primarily stems from the dataset's restricted trajectory diversity. Both IROS and KITTI datasets exhibit broader data distributions, encompassing more varied trajectory shapes. Consequently, when an Oxford-trained model attempts to distinguish between its own less diverse distribution and a highly varied dataset like IROS, the task becomes particularly challenging. Nevertheless, these results underscore the advantages of our unified diffusion approach to OOD detection. By requiring training on only a single dataset, our method significantly reduces the overall model training time.
\begin{table}[t]
    \caption{AUROC $\uparrow$ of OOD Detection over different train dataset on sequence length of 128 and 30 diffusion steps\\
    (O: Oxford, K: KITTI, I:IROS20)}
    \vskip 0.15in
\begin{adjustbox}{width=0.68\linewidth,center}
    \begin{tabular}{lcccccccc}
    \toprule
        \textbf{Method} & \textbf{O/K} & \textbf{K/O} & \textbf{O/I} & \textbf{I/O} & \textbf{K/I} & \textbf{I/K}\\
    \midrule
        $\mathbb{R}^{3}$-Oxford  & 0.897 & 0.327 & 0.890 & 0.378 & 0.464 & 0.532\\
        $\mathbb{SE}(3)$-Oxford  & 0.934 & \textbf{0.999} & \textbf{1.000} & 0.124 & \textbf{1.000} & 0.433\\
        % Path Sign-Oxford  & 0.996 & 0.995 & x & x & x & x\\
    \midrule
        $\mathbb{R}^{3}$-KITTI  & 0.362 & 0.770 & 0.417 & 0.585 & 0.793 & 0.409\\
        $\mathbb{SE}(3)$-KITTI  & \textbf{1.000} & 0.956 & \textbf{1.000} & \textbf{0.931} & \textbf{1.000} & \textbf{0.897}\\
        % Path Sign-KITTI  & x & x & x & x & x & x\\
    \bottomrule
    \end{tabular}
    \end{adjustbox}
    \vskip -0.1in
    \label{tab:ablation_dataset}
\end{table}

\subsubsection{Necessity of Rotational Diffusion Information}
We compare diffusion models trained on translation-only data versus those trained on complete $\mathbb{SE}(3)$ data to demonstrate the critical role of rotational information. Table~\ref{tab:ablation_dataset} reveals that OOD detection using only $\mathbb{R}^{3}$ data yields poor results, consistent with the overlapping statistical distributions shown in figure~\ref{fig:eps_oxford_kitti}. In contrast, $\mathbb{SE}(3)$ diffusion achieves superior performance by incorporating rotational components. This finding demonstrates that for complex trajectory analysis, orientation and rotation data provide richer discriminative features that vary significantly across different data distributions, thereby serving as robust indicators for OOD detection.

\subsubsection{Sequence length of the Trajectory}
Table~\ref{tab:ablation_seqlen} presents the \ac{OOD} detection performance for varying trajectory sequence lengths during KITTI dataset pre-training. The results demonstrate that $\mathbf{DOSE3}$ maintains consistently excellent performance with near-perfect AUROC scores across all \ac{ID} and \ac{OOD} pairs, independent of sequence length. This robustness to sequence length variation highlights the model's stability and generalization capabilities.
\begin{table}[t]
\caption{AUROC $\uparrow$ of OOD Detection over different sequence length for 30 diffusion steps with model trained on KITTI datatset\\
(O: Oxford, K: KITTI, I:IROS20)}
    \vskip 0.15in
    \begin{adjustbox}{width=0.68\linewidth,center}
    \begin{tabular}{lcccccr}
    \toprule
        \textbf{Seq Length} & \textbf{O/K} & \textbf{K/O} & \textbf{O/I} & \textbf{I/O} & \textbf{K/I} & \textbf{I/K}\\
    \midrule
        64   & \textbf{1.000} & \textbf{0.999} & \textbf{1.000} & \textbf{0.986} & 0.999 & \textbf{0.976} \\
        128  & \textbf{1.000} & 0.956 & \textbf{1.000} & 0.931 & \textbf{1.000} & 0.897\\
        256  & 0.980 & 0.961 & \textbf{1.000} & 0.941 & \textbf{1.000} & 0.932\\
        512  & \textbf{1.000} & \textbf{0.999} & \textbf{1.000} & 0.942 & \textbf{1.000} & 0.923\\
    \bottomrule
    \end{tabular}
    \end{adjustbox}
    \vskip -0.1in
    
    \label{tab:ablation_seqlen}
\end{table}

\subsubsection{DDPM Forward Steps}
Table~\ref{tab:ablation_steps} illustrates the relationship between $\mathbf{DOSE3}$ performance and the number of DDPM steps. The results indicate minimal variation in average AUROC scores across different step counts, demonstrating $\mathbf{DOSE3}$'s resilience to changes in the number of DDPM steps.
\begin{table}[t]
\caption{AUROC of OOD Detection over different numbers of diffusion steps on sequence length of 128 with model trained on KITTI datatset\\
(O: Oxford, K: KITTI, I:IROS20)}
    \vskip 0.15in
    \begin{adjustbox}{width=0.68\linewidth,center}
    \begin{tabular}{lcccccr}
    \toprule
        \textbf{Diffusion Step} & \textbf{O/K} & \textbf{K/O} & \textbf{O/I} & \textbf{I/O} & \textbf{K/I} & \textbf{I/K}\\
    \midrule
        5   & 0.916 & 0.969 & \textbf{1.000} & \textbf{0.955} & \textbf{1.000} & \textbf{0.945}\\
        10  & 0.914 & \textbf{0.975} & \textbf{1.000} & 0.938 & \textbf{1.000} & \textbf{0.945}\\
        15  & 0.948 & 0.964 & \textbf{1.000} & 0.919 & \textbf{1.000} & 0.912\\
        30  & \textbf{1.000} & 0.956 & 1.000 & 0.942 & \textbf{1.000} & 0.897\\
    \bottomrule
    \end{tabular}
    \end{adjustbox}
    \vskip -0.1in
    
    \label{tab:ablation_steps}
\end{table}

\section{Conclusions}
Out-of-Distribution (OOD) detection plays a vital role in machine learning, particularly in safety-critical domains like autonomous driving and robotics where systems must reliably interact with the physical world. In these applications, data typically consists of rigid object pose trajectories that capture both positional and rotational motion. While existing OOD detection approaches operate on assumed Euclidean latent spaces, we present $\mathbf{DOSE3}$, a novel unified diffusion-based \ac{OOD} detection framework specifically designed for $\mathbb{SE}(3)$ trajectory data.
$\mathbf{DOSE3}$ innovates by directly incorporating manifold operations into the diffusion model and introduces a novel architecture that extends DDPM to handle $\mathbb{SE}(3)$ manifold sequences. Through comprehensive empirical evaluation across diverse real-world safety-critical datasets, we demonstrate $\mathbf{DOSE3}$'s robust performance and effectiveness.