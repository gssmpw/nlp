\section{Preliminaries}
In this section, we first provide background on the architecture of diffusion models. We then discuss the recent advancements in constructing \emph{Unified} OOD detection models using diffusion models. Finally, we introduce the \emph{Special Euclidean Group in 3D}, $\mathbb{SE}(3)$, and elaborate on its geometric structure and related statistical foundations.

\subsection{Denoising Diffusion Probabilistic Model (DDPM)}

Diffusion models have gained widespread attention in generative modeling due to their strong ability to synthesize high-fidelity data. These models employ a forward diffusion process, where data $\mathbf{x_0}$ is gradually corrupted by adding Gaussian noise over $T$ timesteps, ultimately producing a noisy distribution $\mathbf{x_T}$ that approximates a standard normal distribution. The goal is to learn the reverse diffusion process, which systematically denoises $\mathbf{x_T}$ to recover the original data distribution. 

At the core of this reverse process is the $\epsilon$-model, typically implemented as a neural network trained to predict the noise $\boldsymbol{\epsilon}$ added at each timestep $t$. The forward diffusion process, expressed in equation~\ref{eq:forward}, illustrates how standard Gaussian noise is introduced to perturb the original sample $\mathbf{x}_0$. The backward process, given in equation~\ref{eq:backward}, employs the estimator model $\boldsymbol{\epsilon}_\theta$, which estimates the true Gaussian noise $\boldsymbol{\epsilon}$ and enables data recovery by removing the noise.

\begin{align}
\mathbf{x}_t &= \sqrt{\bar{\alpha}_t} \mathbf{x}_0 + \sqrt{1 - \bar{\alpha}_t} \boldsymbol{\epsilon}, \quad \boldsymbol{\epsilon} \sim \mathcal{N}(\mathbf{0}, \mathbf{I}) \label{eq:forward} \\
\mathbf{x}_{t-1} &= \frac{1}{\sqrt{\alpha_t}} \left(\mathbf{x}_t - \frac{\beta_t}{\sqrt{1 - \bar{\alpha}_t}} \boldsymbol{\epsilon}_\theta(\mathbf{x}_t, t) \right) + \sigma_t \mathbf{z} \nonumber\\
&\quad \mathbf{z} \sim \mathcal{N}(\mathbf{0}, \mathbf{I}) \label{eq:backward}
\end{align}
where $\alpha_t$, $\beta_t$, and $\bar{\alpha}_t$ are predefined noise schedule parameters, and $\mathbf{z} \sim \mathcal{N}(0, \mathbf{I})$. 

The theoretical foundation of diffusion models is grounded in variational inference, where the evidence lower bound (ELBO) in equation~\ref{eq:ELBO} is maximized to ensure that the learned reverse process closely approximates the true data distribution.
\begin{align}
\mathcal{L}_{\text{ELBO}} = \mathbb{E}_{q} [ & D_{\text{KL}}( q(\mathbf{x}_T | \mathbf{x}_0) \parallel p(\mathbf{x}_T) ) 
+ \sum_{t=2}^{T} D_{\text{KL}}( q(\mathbf{x}_{t-1} | \mathbf{x}_t, \mathbf{x}_0) \parallel p_\theta(\mathbf{x}_{t-1} | \mathbf{x}_t) ) - \log p_\theta(\mathbf{x}_0 | \mathbf{x}_1) ]
\label{eq:ELBO}
\end{align}
By leveraging the $\epsilon$-model within this framework, diffusion models effectively capture complex data manifolds, achieving state-of-the-art generative performance.

\subsection{Unified Out-of-Distribution Detection}
Traditional \ac{OOD} detection methods, such as likelihood-based and reconstruction-based approaches, require retraining a new model for each specific inlier data distribution. This results in significant computational costs when switching between different OOD tasks and distributions. Recently, \cite{heng2024out} introduced a new concept of Unified \ac{OOD} detection, where a single unconditional diffusion model is trained, and distributional information can be obtained from inlier distributions that were unseen during training.

The theoretical foundation of this approach builds on the variance-preserving formulation used in DDPM. The difference between each denoising timestep is given in equation~\ref{eq:diffusion_process} and can be rewritten as:
\begin{align}
    d\mathbf{x}_t &= -\frac{1}{2} \beta_t \mathbf{x}_t \, dt + \sqrt{\beta_t} \, d\mathbf{w}_t, \quad \mathbf{x}_0 \sim p_0(\mathbf{x}) \label{eq:diffusion_process} \\
    \frac{d \mathbf{x}_t}{dt} &= f(\mathbf{x}_t, t) + \frac{g(t)^2}{2 \sigma_t^2} \epsilon_p(\mathbf{x}_t, t) \label{eq:sgm_rewrite}
\end{align}
In equation~\ref{eq:kl_divergence}, we denote $\phi_T$ and $\psi_T$ as the marginals obtained by evolving two distinct distributions, $\phi_0$ and $\psi_0$, using their respective probability flow ordinary differential equations (ODEs) from equation~\ref{eq:sgm_rewrite}.
\begin{align}
    D_{\mathrm{KL}} (\phi_0 \parallel \psi_0) = \frac{1}{2} \int_{0}^{T} \mathbb{E}_{\mathbf{x}_t \sim \phi_t} \left[ \frac{g(t)^2}{\sigma_t^2} \left\| \epsilon_{\phi}(\mathbf{x}_t, t) - \epsilon_{\psi}(\mathbf{x}_t, t) \right\|^2 \right] dt + D_{\mathrm{KL}}(\phi_T \parallel \psi_T)\label{eq:kl_divergence}
\end{align}
However, the KL divergence remains dependent on the specific model estimators $\epsilon_\phi$ and $\epsilon_\psi$ in equation~\ref{eq:kl_divergence}. The key observation is that even when executing DDPM forward diffusion using an estimator $\epsilon_\theta$ trained on a third distribution $\theta$, the sample can still be successfully transformed into a standard Gaussian distribution. This insight motivates the use of $\epsilon_\theta$—metrics extracted from an arbitrary diffusion estimator—to perform OOD detection on an inlier distribution $\phi$.

\begin{figure*}[t]
    \centering
\includegraphics[width=0.98\linewidth]{imgs/illustration/system.png}
    \caption{System Diagram of DOSE3 processing flow. Sequences of pose data are diffused, where diffusion over rotational components is constrained to the $\mathbb{SO}(3)$ manifold. The resulting diffusion estimators are used to construct an OOD statistic.}
    \label{fig:system}
\end{figure*}

\subsection{The Special Euclidean Group in 3D} \label{subsec:SE(3)}

The Special Euclidean Group in 3D, denoted as \( \mathbb{SE}(3) \), represents the space of rigid body transformations, which consist of both rotations and translations. The transformation can be written as:
\[
T = \begin{bmatrix}
R & t \\
0 & 1
\end{bmatrix}
\]
where \( R \in \mathbb{SO}(3) \) is a rotation matrix, and \( t \in \mathbb{R}^{3} \) is a translation vector. The group \( \mathbb{SO}(3) \) consists of all \( 3 \times 3 \) real orthogonal matrices with determinant equal to one:
\begin{equation}
    \mathbb{SO}(3) = \{ R \in \mathbb{R}^{3 \times 3} \mid R^\top R = I, \ \det(R) = 1 \}
\end{equation}
where \( I \) is the \( 3 \times 3 \) identity matrix. The group \( \mathbb{SO}(3) \) represents all possible rotations about the origin in three-dimensional space.

The Lie algebra associated with \( \mathbb{SO}(3) \) is denoted as \( \mathfrak{so}(3) \) and consists of all \( 3 \times 3 \) skew-symmetric matrices. A general element \( \Omega \in \mathfrak{so}(3) \) can be written as:
\[
\Omega = \begin{bmatrix}
0 & -\omega_3 & \omega_2 \\
\omega_3 & 0 & -\omega_1 \\
-\omega_2 & \omega_1 & 0
\end{bmatrix}
\]
where \( \omega = [\omega_1, \omega_2, \omega_3]^\top \) is a vector in \( \mathbb{R}^3 \). The Lie algebra \( \mathfrak{so}(3) \) serves as the tangent space to the manifold \( \mathbb{SO}(3) \), providing a locally Euclidean structure that facilitates computations on \( \mathbb{SO}(3) \).

The exponential map, \( \exp: \mathfrak{so}(3) \rightarrow \mathbb{SO}(3) \), maps an element from the Lie algebra to the Lie group, enabling the representation of rotations in matrix form. Given \( \Omega \in \mathfrak{so}(3) \), the exponential map is defined as:
\begin{equation}
    \exp(\Omega) = I + \frac{\sin \theta}{\theta} \Omega + \frac{1 - \cos \theta}{\theta^2} \Omega^2
    \label{eq:exp}
\end{equation}
where \( \theta = \|\omega\| \) is the rotation angle, and \( \omega \) is the vector corresponding to \( \Omega \).

Conversely, the logarithmic map, \( \log: \mathbb{SO}(3) \rightarrow \mathfrak{so}(3) \), converts a rotation matrix into its corresponding Lie algebra representation. For any \( R \in \mathbb{SO}(3) \) that is not the identity matrix, the logarithmic map is given by:
\begin{equation}
    \log(R) = \frac{\theta}{2 \sin \theta} (R - R^\top)
    \label{eq:log}
\end{equation}
where the rotation angle \( \theta \) is computed as:
\begin{equation}
    \theta = \cos^{-1}\left( \frac{\text{trace}(R) - 1}{2} \right)
\end{equation}
Here, \( \mathfrak{so}(3) \), the \emph{tangent space} of \( \mathbb{SO}(3) \), lies within Euclidean space, allowing standard algebraic operations to be applied. This property is particularly useful for designing diffusion models over \( \mathbb{SO}(3) \), as it enables efficient computations and parameterizations of rotations.
