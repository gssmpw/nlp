\documentclass{article} 

\usepackage{amsmath}
\usepackage{amssymb}
\usepackage{amsthm}
\usepackage{mathtools}

\usepackage{hyperref}
\hypersetup{
    colorlinks=true,
    linkcolor=blue,
    urlcolor=magenta,
    citecolor=blue, 
    }
\usepackage{xcolor}
\usepackage{times}
\usepackage{fullpage}

\usepackage{natbib}

\newtheorem{property}{Property}
\newtheorem{claim}{Claim}
\newtheorem{example}{Example}
\newtheorem{theorem}{Theorem}
\newtheorem{definition}[theorem]{Definition}
\newtheorem{lemma}[theorem]{Lemma}
\newtheorem{remark}[theorem]{Remark}

\newtheorem{innercustomlem}{Restatement of Lemma}
\newenvironment{customlem}[1]
  {\renewcommand\theinnercustomlem{#1}\innercustomlem}
  {\endinnercustomlem}

\newtheorem{innercustomthm}{Restatement of Theorem}
\newenvironment{customthm}[1]
  {\renewcommand\theinnercustomthm{#1}\innercustomthm}
  {\endinnercustomthm}

\newtheorem{innercustomrmk}{Restatement of Remark}
\newenvironment{customrmk}[1]
  {\renewcommand\theinnercustomrmk{#1}\innercustomrmk}
  {\endinnercustomrmk}

\newtheorem{innercustomclm}{Restatement of Claim}
\newenvironment{customclm}[1]
  {\renewcommand\theinnercustomclm{#1}\innercustomclm}
  {\endinnercustomclm}

\newcommand{\rA}{\mathbf{A}}
\newcommand{\rt}{\mathbf{t}}
\newcommand{\ra}{\mathbf{a}}
\newcommand{\rv}{\mathbf{v}}
\newcommand{\rZ}{\mathbf{Z}}
\newcommand{\rN}{\mathbf{N}}
\newcommand{\rX}{\mathbf{X}}
\newcommand{\rY}{\mathbf{Y}}
\newcommand{\rx}{\mathbf{x}}
\newcommand{\ry}{\mathbf{y}}
\newcommand{\rS}{\mathbf{S}}
\newcommand{\rE}{\mathbf{E}}
\newcommand{\rM}{\mathbf{M}}
\newcommand{\rsigma}{\mathbf{\sigma}}

\newcommand{\cH}{\mathcal{S}^{d-1}}
\newcommand{\Rad}{\mathcal{R}}
\newcommand{\Loss}{\mathcal{L}}
\newcommand{\cD}{\mathcal{D}}
\newcommand{\Norm}{\mathcal{N}}
\newcommand{\Disc}{\mathcal{G}}
\newcommand{\ncH}{\mathcal{S}^d}
\newcommand{\embH}{\mathcal{H}}
\newcommand{\Ac}{\mathcal{A}}
\newcommand{\Xc}{\mathcal{X}}
\newcommand{\Net}{\mathcal{W}}
\newcommand{\Bc}{\mathcal{B}}
\newcommand{\AllOne}{\mathbf{1}}
\newcommand{\subH}{\mathcal{H}}


\newcommand{\R}{\mathbb{R}}
\newcommand{\N}{\mathbb{N}}
\newcommand{\Ball}{\mathbb{B}}
\newcommand{\Balld}{\mathbb{B}_2^d}
\newcommand{\E}{\mathbb{E}}
\newcommand{\Z}{\mathbb{Z}}

\newcommand{\eps}{\varepsilon}
\newcommand{\eqd}{\,{\buildrel d \over =}\,}
\DeclareMathOperator{\support}{supp}
\DeclareMathOperator{\sign}{sign}
\newcommand{\Ints}{\mathbb{Z}}
\newcommand{\finalH}{\mathcal{H}}
\newcommand{\finalX}{\mathcal{X}}
\DeclareMathOperator{\poly}{poly}

\usepackage{bbm}
\usepackage{dsfont}
\newcommand{\indi}[1]{\mathds{1}\{#1\}}
\DeclarePairedDelimiter{\floor}{\lfloor}{\rfloor}
\DeclarePairedDelimiter{\norm}{\parallel}{\parallel}
\DeclarePairedDelimiter{\ipr}{\langle}{\rangle}

\renewcommand{\Pr}{\mathbb{P}}

\title{Tight Generalization Bounds for Large-Margin Halfspaces}

\author{Kasper Green Larsen \qquad\qquad Natascha Schalburg \\ 
        \texttt{\{larsen,n.schalburg\}@cs.au.dk}\\
        Computer Science, Aarhus University
}
\date{}

\begin{document}

\maketitle

\begin{abstract}  
Test time scaling is currently one of the most active research areas that shows promise after training time scaling has reached its limits.
Deep-thinking (DT) models are a class of recurrent models that can perform easy-to-hard generalization by assigning more compute to harder test samples.
However, due to their inability to determine the complexity of a test sample, DT models have to use a large amount of computation for both easy and hard test samples.
Excessive test time computation is wasteful and can cause the ``overthinking'' problem where more test time computation leads to worse results.
In this paper, we introduce a test time training method for determining the optimal amount of computation needed for each sample during test time.
We also propose Conv-LiGRU, a novel recurrent architecture for efficient and robust visual reasoning. 
Extensive experiments demonstrate that Conv-LiGRU is more stable than DT, effectively mitigates the ``overthinking'' phenomenon, and achieves superior accuracy.
\end{abstract}  

\section{Introduction}


\begin{figure}[t]
\centering
\includegraphics[width=0.6\columnwidth]{figures/evaluation_desiderata_V5.pdf}
\vspace{-0.5cm}
\caption{\systemName is a platform for conducting realistic evaluations of code LLMs, collecting human preferences of coding models with real users, real tasks, and in realistic environments, aimed at addressing the limitations of existing evaluations.
}
\label{fig:motivation}
\end{figure}

\begin{figure*}[t]
\centering
\includegraphics[width=\textwidth]{figures/system_design_v2.png}
\caption{We introduce \systemName, a VSCode extension to collect human preferences of code directly in a developer's IDE. \systemName enables developers to use code completions from various models. The system comprises a) the interface in the user's IDE which presents paired completions to users (left), b) a sampling strategy that picks model pairs to reduce latency (right, top), and c) a prompting scheme that allows diverse LLMs to perform code completions with high fidelity.
Users can select between the top completion (green box) using \texttt{tab} or the bottom completion (blue box) using \texttt{shift+tab}.}
\label{fig:overview}
\end{figure*}

As model capabilities improve, large language models (LLMs) are increasingly integrated into user environments and workflows.
For example, software developers code with AI in integrated developer environments (IDEs)~\citep{peng2023impact}, doctors rely on notes generated through ambient listening~\citep{oberst2024science}, and lawyers consider case evidence identified by electronic discovery systems~\citep{yang2024beyond}.
Increasing deployment of models in productivity tools demands evaluation that more closely reflects real-world circumstances~\citep{hutchinson2022evaluation, saxon2024benchmarks, kapoor2024ai}.
While newer benchmarks and live platforms incorporate human feedback to capture real-world usage, they almost exclusively focus on evaluating LLMs in chat conversations~\citep{zheng2023judging,dubois2023alpacafarm,chiang2024chatbot, kirk2024the}.
Model evaluation must move beyond chat-based interactions and into specialized user environments.



 

In this work, we focus on evaluating LLM-based coding assistants. 
Despite the popularity of these tools---millions of developers use Github Copilot~\citep{Copilot}---existing
evaluations of the coding capabilities of new models exhibit multiple limitations (Figure~\ref{fig:motivation}, bottom).
Traditional ML benchmarks evaluate LLM capabilities by measuring how well a model can complete static, interview-style coding tasks~\citep{chen2021evaluating,austin2021program,jain2024livecodebench, white2024livebench} and lack \emph{real users}. 
User studies recruit real users to evaluate the effectiveness of LLMs as coding assistants, but are often limited to simple programming tasks as opposed to \emph{real tasks}~\citep{vaithilingam2022expectation,ross2023programmer, mozannar2024realhumaneval}.
Recent efforts to collect human feedback such as Chatbot Arena~\citep{chiang2024chatbot} are still removed from a \emph{realistic environment}, resulting in users and data that deviate from typical software development processes.
We introduce \systemName to address these limitations (Figure~\ref{fig:motivation}, top), and we describe our three main contributions below.


\textbf{We deploy \systemName in-the-wild to collect human preferences on code.} 
\systemName is a Visual Studio Code extension, collecting preferences directly in a developer's IDE within their actual workflow (Figure~\ref{fig:overview}).
\systemName provides developers with code completions, akin to the type of support provided by Github Copilot~\citep{Copilot}. 
Over the past 3 months, \systemName has served over~\completions suggestions from 10 state-of-the-art LLMs, 
gathering \sampleCount~votes from \userCount~users.
To collect user preferences,
\systemName presents a novel interface that shows users paired code completions from two different LLMs, which are determined based on a sampling strategy that aims to 
mitigate latency while preserving coverage across model comparisons.
Additionally, we devise a prompting scheme that allows a diverse set of models to perform code completions with high fidelity.
See Section~\ref{sec:system} and Section~\ref{sec:deployment} for details about system design and deployment respectively.



\textbf{We construct a leaderboard of user preferences and find notable differences from existing static benchmarks and human preference leaderboards.}
In general, we observe that smaller models seem to overperform in static benchmarks compared to our leaderboard, while performance among larger models is mixed (Section~\ref{sec:leaderboard_calculation}).
We attribute these differences to the fact that \systemName is exposed to users and tasks that differ drastically from code evaluations in the past. 
Our data spans 103 programming languages and 24 natural languages as well as a variety of real-world applications and code structures, while static benchmarks tend to focus on a specific programming and natural language and task (e.g. coding competition problems).
Additionally, while all of \systemName interactions contain code contexts and the majority involve infilling tasks, a much smaller fraction of Chatbot Arena's coding tasks contain code context, with infilling tasks appearing even more rarely. 
We analyze our data in depth in Section~\ref{subsec:comparison}.



\textbf{We derive new insights into user preferences of code by analyzing \systemName's diverse and distinct data distribution.}
We compare user preferences across different stratifications of input data (e.g., common versus rare languages) and observe which affect observed preferences most (Section~\ref{sec:analysis}).
For example, while user preferences stay relatively consistent across various programming languages, they differ drastically between different task categories (e.g. frontend/backend versus algorithm design).
We also observe variations in user preference due to different features related to code structure 
(e.g., context length and completion patterns).
We open-source \systemName and release a curated subset of code contexts.
Altogether, our results highlight the necessity of model evaluation in realistic and domain-specific settings.






\section{Overview}

\revision{In this section, we first explain the foundational concept of Hausdorff distance-based penetration depth algorithms, which are essential for understanding our method (Sec.~\ref{sec:preliminary}).
We then provide a brief overview of our proposed RT-based penetration depth algorithm (Sec.~\ref{subsec:algo_overview}).}



\section{Preliminaries }
\label{sec:Preliminaries}

% Before we introduce our method, we first overview the important basics of 3D dynamic human modeling with Gaussian splatting. Then, we discuss the diffusion-based 3d generation techniques, and how they can be applied to human modeling.
% \ZY{I stopp here. TBC.}
% \subsection{Dynamic human modeling with Gaussian splatting}
\subsection{3D Gaussian Splatting}
3D Gaussian splatting~\cite{kerbl3Dgaussians} is an explicit scene representation that allows high-quality real-time rendering. The given scene is represented by a set of static 3D Gaussians, which are parameterized as follows: Gaussian center $x\in {\mathbb{R}^3}$, color $c\in {\mathbb{R}^3}$, opacity $\alpha\in {\mathbb{R}}$, spatial rotation in the form of quaternion $q\in {\mathbb{R}^4}$, and scaling factor $s\in {\mathbb{R}^3}$. Given these properties, the rendering process is represented as:
\begin{equation}
  I = Splatting(x, c, s, \alpha, q, r),
  \label{eq:splattingGA}
\end{equation}
where $I$ is the rendered image, $r$ is a set of query rays crossing the scene, and $Splatting(\cdot)$ is a differentiable rendering process. We refer readers to Kerbl et al.'s paper~\cite{kerbl3Dgaussians} for the details of Gaussian splatting. 



% \ZY{I would suggest move this part to the method part.}
% GaissianAvatar is a dynamic human generation model based on Gaussian splitting. Given a sequence of RGB images, this method utilizes fitted SMPLs and sampled points on its surface to obtain a pose-dependent feature map by a pose encoder. The pose-dependent features and a geometry feature are fed in a Gaussian decoder, which is employed to establish a functional mapping from the underlying geometry of the human form to diverse attributes of 3D Gaussians on the canonical surfaces. The parameter prediction process is articulated as follows:
% \begin{equation}
%   (\Delta x,c,s)=G_{\theta}(S+P),
%   \label{eq:gaussiandecoder}
% \end{equation}
%  where $G_{\theta}$ represents the Gaussian decoder, and $(S+P)$ is the multiplication of geometry feature S and pose feature P. Instead of optimizing all attributes of Gaussian, this decoder predicts 3D positional offset $\Delta{x} \in {\mathbb{R}^3}$, color $c\in\mathbb{R}^3$, and 3D scaling factor $ s\in\mathbb{R}^3$. To enhance geometry reconstruction accuracy, the opacity $\alpha$ and 3D rotation $q$ are set to fixed values of $1$ and $(1,0,0,0)$ respectively.
 
%  To render the canonical avatar in observation space, we seamlessly combine the Linear Blend Skinning function with the Gaussian Splatting~\cite{kerbl3Dgaussians} rendering process: 
% \begin{equation}
%   I_{\theta}=Splatting(x_o,Q,d),
%   \label{eq:splatting}
% \end{equation}
% \begin{equation}
%   x_o = T_{lbs}(x_c,p,w),
%   \label{eq:LBS}
% \end{equation}
% where $I_{\theta}$ represents the final rendered image, and the canonical Gaussian position $x_c$ is the sum of the initial position $x$ and the predicted offset $\Delta x$. The LBS function $T_{lbs}$ applies the SMPL skeleton pose $p$ and blending weights $w$ to deform $x_c$ into observation space as $x_o$. $Q$ denotes the remaining attributes of the Gaussians. With the rendering process, they can now reposition these canonical 3D Gaussians into the observation space.



\subsection{Score Distillation Sampling}
Score Distillation Sampling (SDS)~\cite{poole2022dreamfusion} builds a bridge between diffusion models and 3D representations. In SDS, the noised input is denoised in one time-step, and the difference between added noise and predicted noise is considered SDS loss, expressed as:

% \begin{equation}
%   \mathcal{L}_{SDS}(I_{\Phi}) \triangleq E_{t,\epsilon}[w(t)(\epsilon_{\phi}(z_t,y,t)-\epsilon)\frac{\partial I_{\Phi}}{\partial\Phi}],
%   \label{eq:SDSObserv}
% \end{equation}
\begin{equation}
    \mathcal{L}_{\text{SDS}}(I_{\Phi}) \triangleq \mathbb{E}_{t,\epsilon} \left[ w(t) \left( \epsilon_{\phi}(z_t, y, t) - \epsilon \right) \frac{\partial I_{\Phi}}{\partial \Phi} \right],
  \label{eq:SDSObservGA}
\end{equation}
where the input $I_{\Phi}$ represents a rendered image from a 3D representation, such as 3D Gaussians, with optimizable parameters $\Phi$. $\epsilon_{\phi}$ corresponds to the predicted noise of diffusion networks, which is produced by incorporating the noise image $z_t$ as input and conditioning it with a text or image $y$ at timestep $t$. The noise image $z_t$ is derived by introducing noise $\epsilon$ into $I_{\Phi}$ at timestep $t$. The loss is weighted by the diffusion scheduler $w(t)$. 
% \vspace{-3mm}

\subsection{Overview of the RTPD Algorithm}\label{subsec:algo_overview}
Fig.~\ref{fig:Overview} presents an overview of our RTPD algorithm.
It is grounded in the Hausdorff distance-based penetration depth calculation method (Sec.~\ref{sec:preliminary}).
%, similar to that of Tang et al.~\shortcite{SIG09HIST}.
The process consists of two primary phases: penetration surface extraction and Hausdorff distance calculation.
We leverage the RTX platform's capabilities to accelerate both of these steps.

\begin{figure*}[t]
    \centering
    \includegraphics[width=0.8\textwidth]{Image/overview.pdf}
    \caption{The overview of RT-based penetration depth calculation algorithm overview}
    \label{fig:Overview}
\end{figure*}

The penetration surface extraction phase focuses on identifying the overlapped region between two objects.
\revision{The penetration surface is defined as a set of polygons from one object, where at least one of its vertices lies within the other object. 
Note that in our work, we focus on triangles rather than general polygons, as they are processed most efficiently on the RTX platform.}
To facilitate this extraction, we introduce a ray-tracing-based \revision{Point-in-Polyhedron} test (RT-PIP), significantly accelerated through the use of RT cores (Sec.~\ref{sec:RT-PIP}).
This test capitalizes on the ray-surface intersection capabilities of the RTX platform.
%
Initially, a Geometry Acceleration Structure (GAS) is generated for each object, as required by the RTX platform.
The RT-PIP module takes the GAS of one object (e.g., $GAS_{A}$) and the point set of the other object (e.g., $P_{B}$).
It outputs a set of points (e.g., $P_{\partial B}$) representing the penetration region, indicating their location inside the opposing object.
Subsequently, a penetration surface (e.g., $\partial B$) is constructed using this point set (e.g., $P_{\partial B}$) (Sec.~\ref{subsec:surfaceGen}).
%
The generated penetration surfaces (e.g., $\partial A$ and $\partial B$) are then forwarded to the next step. 

The Hausdorff distance calculation phase utilizes the ray-surface intersection test of the RTX platform (Sec.~\ref{sec:RT-Hausdorff}) to compute the Hausdorff distance between two objects.
We introduce a novel Ray-Tracing-based Hausdorff DISTance algorithm, RT-HDIST.
It begins by generating GAS for the two penetration surfaces, $P_{\partial A}$ and $P_{\partial B}$, derived from the preceding step.
RT-HDIST processes the GAS of a penetration surface (e.g., $GAS_{\partial A}$) alongside the point set of the other penetration surface (e.g., $P_{\partial B}$) to compute the penetration depth between them.
The algorithm operates bidirectionally, considering both directions ($\partial A \to \partial B$ and $\partial B \to \partial A$).
The final penetration depth between the two objects, A and B, is determined by selecting the larger value from these two directional computations.

%In the Hausdorff distance calculation step, we compute the Hausdorff distance between given two objects using a ray-surface-intersection test. (Sec.~\ref{sec:RT-Hausdorff}) Initially, we construct the GAS for both $\partial A$ and $\partial B$ to utilize the RT-core effectively. The RT-based Hausdorff distance algorithms then determine the Hausdorff distance by processing the GAS of one object (e.g. $GAS_{\partial A}$) and set of the vertices of the other (e.g. $P_{\partial B}$). Following the Hausdorff distance definition (Eq.~\ref{equation:hausdorff_definition}), we compute the Hausdorff distance to both directions ($\partial A \to \partial B$) and ($\partial B \to \partial A$). As a result, the bigger one is the final Hausdorff distance, and also it is the penetration depth between input object $A$ and $B$.


%the proposed RT-based penetration depth calculation pipeline.
%Our proposed methods adopt Tang's Hausdorff-based penetration depth methods~\cite{SIG09HIST}. The pipeline is divided into the penetration surface extraction step and the Hausdorff distance calculation between the penetration surface steps. However, since Tang's approach is not suitable for the RT platform in detail, we modified and applied it with appropriate methods.

%The penetration surface extraction step is extracting overlapped surfaces on other objects. To utilize the RT core, we use the ray-intersection-based PIP(Point-In-Polygon) algorithms instead of collision detection between two objects which Tang et al.~\cite{SIG09HIST} used. (Sec.~\ref{sec:RT-PIP})
%RT core-based PIP test uses a ray-surface intersection test. For purpose this, we generate the GAS(Geometry Acceleration Structure) for each object. RT core-based PIP test takes the GAS of one object (e.g. $GAS_{A}$) and a set of vertex of another one (e.g. $P_{B}$). Then this computes the penetrated vertex set of another one (e.g. $P_{\partial B}$). To calculate the Hausdorff distance, these vertex sets change to objects constructed by penetrated surface (e.g. $\partial B$). Finally, the two generated overlapped surface objects $\partial A$ and $\partial B$ are used in the Hausdorff distance calculation step.

\section{Main Proof}
\label{sec:mainproof}
We now set out to prove Theorem~\ref{thm:main} following the proof outline sketched in Section~\ref{sec:overview}. We start by a series of reductions that allow us to focus on a simpler task of establishing Theorem~\ref{thm:main} only for a small range of $\gamma$ and $\Loss_{\rS}^\gamma(w)$. We describe these reductions in Section~\ref{sec:reduct} and then proceed to the main arguments in Section~\ref{sec:mainargs}.

\newcommand{\tabincell}[2]{\begin{tabular}{@{}#1@{}}#2\end{tabular}}
\newcommand{\rowstyle}[1]{\gdef\currentrowstyle{#1}%
	#1\ignorespaces
}

\newcommand{\className}[1]{\textbf{\sf #1}}
\newcommand{\functionName}[1]{\textbf{\sf #1}}
\newcommand{\objectName}[1]{\textbf{\sf #1}}
\newcommand{\annotation}[1]{\textbf{#1}}
\newcommand{\todo}[1]{\textcolor{blue}{\textbf{[[TODO: #1]]}}}
\newcommand{\change}[1]{\textcolor{blue}{#1}}
\newcommand{\fetch}[1]{\textbf{\em #1}}
\newcommand{\phead}[1]{\vspace{1mm} \noindent {\bf #1}}
\newcommand{\wei}[1]{\textcolor{blue}{{\it [Wei says: #1]}}}
\newcommand{\peter}[1]{\textcolor{red}{{\it [Peter says: #1]}}}
\newcommand{\zhenhao}[1]{\textcolor{dkblue}{{\it [Zhenhao says: #1]}}}
\newcommand{\feng}[1]{\textcolor{magenta}{{\it [Feng says: #1]}}}
\newcommand{\jinqiu}[1]{\textcolor{red}{{\it [Jinqiu says: #1]}}}
\newcommand{\shouvick}[1]{\textcolor{violet(ryb)}{{\it [Shouvick says: #1]}}}
\newcommand{\pattern}[1]{\emph{#1}}
%\newcommand{\tool}{{{DectGUILag}}\xspace}
\newcommand{\tool}{{{GUIWatcher}}\xspace}


\newcommand{\guo}[1]{\textcolor{yellow}{{\it [Linqiang says: #1]}}}

\newcommand{\rqbox}[1]{\begin{tcolorbox}[left=4pt,right=4pt,top=4pt,bottom=4pt,colback=gray!5,colframe=gray!40!black,before skip=2pt,after skip=2pt]#1\end{tcolorbox}}


\subsection{Random Discretization}
\label{sec:mainargs}
We now set out to prove Lemma~\ref{lem:subgoal}. So let $0 < \delta < 1$, and fix a pair $(\Gamma_i, L_j)$. Following the proof outline in Section~\ref{sec:overview}, we now consider the following random discretization of hypotheses in $\subH(\Gamma_i, L_j)$: Let $k= k(i,j)$ be an integer parameter to be determined. Sample a random $k \times d$ matrix $\rA$ with each entry $\Norm(0,1/k)$ distributed as well as $k$ random offsets $\rt = (\rt_1,\dots,\rt_k)$ all independent and uniformly distributed in $[0,1]$.

Let $\Disc$ be the set of all vectors in $\R^k$ with coordinates in 
\[
\{(1/2)(10 \sqrt{k})^{-1} + z (10 \sqrt{k})^{-1}  \mid z \in \Ints\}.
\]
For $w \in \finalH$ and an outcome $(A,t)$ of $(\rA,\rt)$, define $h_{A,t}(w) \in \Disc$ as the vector obtained as follows: Consider each coordinate $(Aw)_i$ and let $z_i$ denote the integer such that 
\[
(1/2)(10 \sqrt{k})^{-1}  + z_i (10 \sqrt{k})^{-1} \leq (Aw)_i < (1/2)(10 \sqrt{k})^{-1}  + (z_i+1) (10\sqrt{k})^{-1}.
\]
Let $(h_{A,t}(w))_i$ equal $(1/2)(10 \sqrt{k})^{-1} + z_i (10\sqrt{k})^{-1}$ if $t_i \leq p(z_i)$ ($(Aw)_i$ rounded down) and otherwise let it equal $(1/2)(10 \sqrt{k})^{-1}  + (z_i + 1)(10\sqrt{k})^{-1}$. We choose $p(z_i) \in [0,1]$ such that 
\begin{align}
(Aw)_i &= 
p(z_i)\left(\frac{1}{2 \cdot 10 \sqrt{k}} +\frac{z_i}{10\sqrt{k}}\right) + (1-p(z_i))\left(\frac{1}{2 \cdot 10 \sqrt{k}} +\frac{z_i+1}{10\sqrt{k}}\right) \label{eq:expectround}
\end{align}
i.e.\ for fixed $A$, the expected value of the coordinates satisfy $\E_{\rt}[(h_{A,\rt}(w))_i]=(Aw)_i$. 
\begin{remark}
\label{rmk:pIsProb}
The value $p(z_i)$ satisfying~\eqref{eq:expectround} has $p(z_i) \in [0,1]$.
\end{remark}
We thus have that $p(z_i)$ is a well-defined probability. We prove Remark~\ref{rmk:pIsProb} in Appendix~\ref{sec:aux}. The random discretization has the desirable property that it approximately preserves margins/inner products as stated in the following
\begin{lemma}
\label{lem:concdiscretize}
    There is a constant $c>0$, such that for any integer $k \geq 1$, $w \in \finalH, x \in \finalX$ and any $\gamma \in (0,1]$, it holds that 
    $
    \Pr_{\rA,\rt}[|\langle h_{\rA,\rt}(w),\rA x\rangle - \langle w, x\rangle| > \gamma] < c\exp(-\gamma^2 k/c)
    $.
\end{lemma}
The proof of Lemma~\ref{lem:concdiscretize} follows the work by~\cite{AK17} in their work on lower bounds for the Johnson-Lindenstrauss transform, and has thus been deferred to Appendix~\ref{sec:aux}. We now observe that
\begin{align*}
    \Loss_{\cD}(w) &= \Loss^{\gamma_i/2}_{\rA \cD}(h_{\rA,\rt}(w)) + \Pr_{(\rx,\ry) \sim \cD}[y \langle h_{\rA,\rt}(w), \rA \rx\rangle > \gamma_i/2 \wedge \ry \langle w, \rx\rangle \leq 0] \\
    &- \Pr_{(\rx,\ry) \sim \cD}[\ry \langle h_{\rA,\rt}(w), \rA \rx\rangle \leq \gamma_i/2 \wedge \ry \langle w, \rx\rangle > 0].
\end{align*}
Similarly, we have for $\gamma \in \Gamma_i$ and any training set $S$ that
\begin{align*}
    \Loss_{S}^\gamma(w) &= \Loss^{\gamma_i/2}_{\rA S}(h_{\rA,\rt}(w)) + \Pr_{(\rx,\ry) \sim S}[\ry \langle h_{\rA,\rt}(w), \rA \rx\rangle > \gamma_i/2 \wedge \ry \langle w, \rx\rangle \leq \gamma] \\
    &- \Pr_{(\rx,\ry) \sim S}[\ry \langle h_{\rA,\rt}(w), \rA\rx\rangle \leq \gamma_i/2 \wedge \ry \langle w, \rx\rangle > \gamma].
\end{align*}
We now have for any $\gamma \in \Gamma_i$ that
\begin{align}
\sup_{w \in \subH(\Gamma_i,L_j)} \Loss_\cD(w) - \Loss^\gamma_S(w) =& \nonumber \\ 
\sup_{w \in \subH(\Gamma_i,L_j)} \big(\E_{\rA,\rt} [\Loss^{\gamma_i/2}_{\rA \cD}(h_{\rA,\rt}(w)) - \Loss^{\gamma_i/2}_{\rA S}(h_{\rA,\rt}(w))] +& \nonumber\\ \E_{\rA,\rt} [\Pr_{\cD}[\ry \langle h_{\rA,\rt}(w), \rA \rx\rangle > \gamma_i/2 \wedge \ry \langle w, \rx\rangle \leq 0] -\Pr_{S}[\ry \langle h_{\rA,\rt}(w), \rA \rx\rangle > \gamma_i/2 \wedge \ry \langle w, \rx\rangle \leq \gamma] ] +& \nonumber\\
\E_{\rA,\rt} [\Pr_{S}[\ry \langle h_{\rA,\rt}(w), \rA \rx\rangle \leq \gamma_i/2 \wedge \ry \langle w, \rx\rangle > \gamma] -\Pr_{\cD}[\ry \langle h_{\rA,\rt}(w), \rA \rx\rangle \leq \gamma_i/2 \wedge \ry \langle w, \rx\rangle > 0] ]\big).\label{eq:3terms}
\end{align}
A critical observation is that the distribution of $y \langle h_{\rA,\rt}(w),\rA x \rangle$ depends only on $y \langle w, x \rangle$. 
\begin{claim}
\label{clm:distDeterm}
For any $(x,y) \in \finalX \times \{-1,1\}$ and any $w \in \finalH$, the distribution of $y \langle h_{\rA,\rt}(w),\rA x \rangle$ is completely determined from $y \langle w, x \rangle$.
\end{claim}
We prove Claim~\ref{clm:distDeterm} in Section~\ref{sec:lip} by exploiting that the entries of $\rA$ are i.i.d.\ $\Norm(0,1/k)$ distributed and using the rotational invariance of the Gaussian distribution.

As outlined in the proof overview in Section~\ref{sec:overview}, we can now use Claim~\ref{clm:distDeterm} together with the contraction inequality of Rademacher complexity to bound several of the terms in~\eqref{eq:3terms}. Similarly to the introduction of the ramp loss in classic proofs of generalization for large-margin halfspaces, we need to introduce a continuous function upper bounding the probabilities above. With this in mind, we now define the following functions $\phi$ and $\rho$:
\[
\phi(\alpha) = \begin{cases} \Pr_{\rA, \rt}[y \langle h_{\rA,\rt}(w), \rA x\rangle > \gamma_i/2 \mid y \langle w, x \rangle = \alpha] & \text{if } -c_\gamma \leq \alpha \leq 0 \\
                      \frac{(\gamma_i-\alpha)}{\gamma_i}\Pr_{\rA, \rt}[y \langle h_{\rA,\rt}(w), \rA x\rangle > \gamma_i/2 \mid y \langle w, x \rangle = 0]                                    & \text{if } 0 < \alpha \leq \gamma_i      \\
                      0 & \text{if } \gamma_i < \alpha \leq c_\gamma
        \end{cases}
\]
\[
\rho(\alpha) = \begin{cases} \Pr_{\rA, \rt}[y \langle h_{\rA,\rt}(w), \rA x\rangle \leq \gamma_i/2 \mid y \langle w, x \rangle = \alpha] & \text{if } \gamma_i < \alpha \leq c_\gamma\\
                      \frac{\alpha}{\gamma_i}\Pr_{\rA, \rt}[y \langle h_{\rA,\rt}(w), \rA x\rangle \leq \gamma_i/2 \mid y \langle w, x \rangle = \gamma_i]                                    & \text{if } 0 < \alpha \leq \gamma_i      \\
                      0 & \text{if } -c_\gamma \leq \alpha \leq 0
        \end{cases}
\]
Here we slightly abuse notation and write $\Pr_{\rA, \rt}[y \langle h_{\rA,\rt}(w), \rA x\rangle > \gamma_i/2 \mid y \langle w, x \rangle = \alpha]$ to denote the probability $\Pr_{\rA, \rt}[y \langle h_{\rA,\rt}(w), \rA x\rangle > \gamma_i/2]$ for an arbitrary $w \in \finalH, (x,y) \in \finalX \times \{-1,1\}$ with $y \ipr{w,x}=\alpha$ and remark that this probability is the same for all such $w,x,y$ by Claim~\ref{clm:distDeterm}.

We now observe that $\phi$ and $\rho$ upper and lower bounds the terms in~\eqref{eq:3terms}
\begin{remark}
\label{rmk:phirho}
For any training set $S$ and distribution $\cD$ over $\finalX \times \{-1,1\}$, we have
\begin{align*}
    \E_{\rA,\rt} [\Pr_{(\rx,\ry) \sim \cD}[\ry \langle h_{\rA,\rt}(w), \rA \rx\rangle > \gamma_i/2 \wedge \ry \langle w, \rx\rangle \leq 0]] &\leq\E_{(\rx,\ry) \sim \cD}[\phi(\ry \langle w, \rx \rangle)] \\
    \E_{\rA,\rt} [\Pr_{(\rx,\ry) \sim S}[\ry \langle h_{\rA,\rt}(w), \rA \rx\rangle > \gamma_i/2 \wedge \ry \langle w, \rx\rangle \leq \gamma]] &\geq \E_{(\rx,\ry) \sim S}[\phi(\ry \langle w, \rx \rangle)] \\
    \E_{\rA,\rt} [\Pr_{(\rx,\ry) \sim S}[\ry \langle h_{\rA,\rt}(w), \rA \rx\rangle \leq \gamma_i/2 \wedge \ry \langle w, \rx\rangle > \gamma]] &\leq \E_{(\rx, \ry) \sim S}[\rho(\ry \langle w, \rx \rangle)]\\
    \E_{\rA,\rt} [\Pr_{(\rx,\ry) \sim \cD}[\ry \langle h_{\rA,\rt}(w), \rA \rx\rangle \leq \gamma_i/2 \wedge \ry \langle w, \rx\rangle > 0]] &\geq \E_{(\rx, \ry) \sim \cD}[\rho(\ry\langle w, \rx \rangle)].
\end{align*}
\end{remark}
The proof of Remark~\ref{rmk:phirho} follows from the definition of $\phi$ and $\rho$, along with monotonicity of $\Pr_{\rA,\rt}[y \ipr{h_{\rA,\rt}(w),\rA x} > \gamma_i \mid y\ipr{w,x}=\alpha]$ as a function of $\alpha$. The proofs have been deferred to Appendix~\ref{sec:aux}.
Continuing from~\eqref{eq:3terms} using Remark~\ref{rmk:phirho}, linearity of expectation and the triangle inequality, we have for any $\gamma \in \Gamma_i$ that
\begin{align}
\sup_{w \in \subH(\Gamma_i, L_j)} \Loss_\cD(w) - \Loss^\gamma_S(w) \leq&
\sup_{w \in \subH(\Gamma_i, L_j)} \left|\E_{\rA,\rt} [\Loss^{\gamma_i/2}_\cD(h_{\rA,\rt}(w)) - \Loss^{\gamma_i/2}_S(h_{\rA,\rt}(w))]\right| \label{eq:middle}\\
+&\sup_{w \in \subH(\Gamma_i, L_j)} \left| \E_{(\rx,\ry) \sim \cD}[\phi(\ry \langle w, \rx \rangle)] -\E_{(\rx,\ry) \sim S}[\phi(\ry \langle w, \rx \rangle)]\right| \label{eq:phi}\\
+&\sup_{w \in \subH(\Gamma_i, L_j)} \left|\E_{(\rx,\ry) \sim \cD}[\rho(\ry \langle w, \rx \rangle)] -\E_{(\rx,\ry) \sim S}[\rho(\ry \langle w, \rx \rangle)] \right|.\label{eq:rho}
\end{align}
In Section~\ref{sec:meetinmid}, we carefully use Bernstein's plus a (highly non-trivial) union bound over infinitely many grids of increasing size to bound~\eqref{eq:middle} as follows
\begin{lemma}
\label{lem:supW}
There is a constant $c>0$ such that with probability at least $1-\delta$ over $\rS \sim \cD^n$ we have
\begin{align*}
    \eqref{eq:middle} &\leq c \left(\sqrt{\frac{(\ell_{j+1}+ \exp(-\gamma_{i+1}^2k/c)) (k + \ln(e/\delta))}{n}} + \frac{(k + \ln(e/\delta))}{n} \right).
\end{align*}
\end{lemma}
In Section~\ref{sec:rademacher}, we then use Rademacher complexity and a bound on the Lipschitz constants of $\phi$ and $\rho$ to bound~\eqref{eq:phi} and~\eqref{eq:rho} as follows
\begin{lemma}
\label{lem:supPhi}
There are constants $c,c'>0$ such that with probability at least $1-\delta$ over $\rS \sim \cD^n$ we have
\begin{align*}
\max\{\eqref{eq:phi}, \eqref{eq:rho} \}\leq
c\exp(-\gamma_{i+1}^2k/c) \cdot \sqrt{(k + \gamma_{i+1}^{-2} + \ln(e/\delta))/n} .
\end{align*}
provided that $k \geq c' \gamma_{i+1}^{-2}$. 
\end{lemma}
To balance the expressions in Lemma~\ref{lem:supW} and Lemma~\ref{lem:supPhi}, we now set $k = c \gamma_{i+1}^{-2} \ln(e/\ell_{j+1})$ for a sufficiently large constant $c>0$ so that $\exp(-\gamma_{i+1}^2 k/c) \leq \ell_{j+1}/e$ and $k \geq c'\gamma_{i+1}^{-2}$. Combining Lemma~\ref{lem:supW} and Lemma~\ref{lem:supPhi} via a union bound with $\delta'=\delta/2$ and inserting into~\eqref{eq:middle},~\eqref{eq:phi} and~\eqref{eq:rho} gives
\begin{align*}
\sup_{w \in \subH(\Gamma_i, L_j)} \Loss_\cD(w) - \Loss^\gamma_S(w) \leq& \\c\left(\sqrt{\frac{\ell_{j+1}(\gamma_{i+1}^{-2}\ln(e/\ell_{j+1}) + \ln(e/\delta))}{n}}  + \frac{\gamma_{i+1}^{-2} \ln(e/\ell_{j+1})+\ln(e/\delta)}{n}\right) &+\\
c \left( \ell_{j+1} \sqrt{(\gamma_{i+1}^{-2} \ln(e/\ell_{j+1}) + \ln(e/\delta))/n} \right),
\end{align*}
for a constant $c>0$. This completes the proof of Lemma~\ref{lem:subgoal}, which together with Lemma~\ref{lem:subgoal2} completes the proof of our main result, Theorem~\ref{thm:main}.




\section{Regret bounds with sequential complexities in classical online learning}\label{rademacher}

Notions of complexity for a given hypothesis class have traditionally been studied within the batch learning framework and are often characterized by the Rademacher complexity \citep{Rademacher}.

\begin{definition}[Rademacher complexity]
    Let $\mathcal{X}$ be a sample space with an associated distribution $\mathcal D$ and $\mathcal{H}$ be the hypothesis class. Let $(x_j)_{j\in[m]} \sim \mathcal D^m$ be a sequence of samples, sampled i.i.d. from $\mathcal{X}$. The Rademacher complexity can then be defined as:
    \begin{equation*}
\label{eq:rad_comp}
    \mathcal{R}_m (\mathcal{H}) = \mathop{\mathbb{E}}_{(x_j)\sim \mathcal D^m} \Big[\frac{1}{m} \mathop{\mathbb{E}}_{\boldsymbol{\epsilon}} \Big[ \sup_{h \in \mathcal{H}} \sum_{j=1}^m \epsilon_j h(x_j) \Big]
 \Big],
\end{equation*}
where $\boldsymbol{\epsilon} = (\epsilon_1, \cdots, \epsilon_m)$ are called Rademacher variables, that satisfy $P(\epsilon = +1) = P(\epsilon = -1) = 1/2$.
\end{definition}
Perhaps not surprisingly, in addition to being an indicator for expressivity of a given hypothesis class, Rademacher complexity also upper bounds generalization error in the setting of batch learning \citep{Rademacher}.

Rademacher complexity generalises to sequential Rademacher complexity in the online setting \citep{rakhlin2015online}. In order to define sequential Rademacher complexity let us first define a $\mathcal{X}$-valued complete binary tree. 

\begin{definition}[Sequential Rademacher complexity]
    Let $\mathbf{x}$ be a $\mathcal{X}$-valued complete binary tree of depth $T$. The sequential Rademacher complexity of a hypothesis class $\mathcal{H}$ on the tree $\mathbf{x}$ is then given as:
    \begin{equation*}
    \label{eq:seq_rad_comp}
        \mathfrak{R}_T (\mathcal{H}, \mathbf{x}) = \Big[\frac{1}{T} \mathop{\mathbb{E}}_{\boldsymbol{\epsilon}} \Big[ \sup_{h \in \mathcal{H}} \sum_{t=1}^T \epsilon_t h(\mathbf{x}_t(\boldsymbol{\epsilon})) \Big]
    \Big].
    \end{equation*}
\end{definition}

The $\mathbf{x}$ dependence of the sequential Rademacher complexity can be subsequently removed by considering the supremum over all $\mathcal{X}$-valued trees of depth $T$: $\mathfrak{R}_T (\mathcal{H}) = \sup_{\mathbf{x}} \mathfrak{R}_T(\mathcal{H}, \mathbf{x})$. Similar to how Rademacher complexity upper bounds the generalization error, the sequential Rademacher complexity was shown to upper bound the minimax regret \citep{rakhlin2015online}. For the case of supervised learning, the following relation holds: 
\begin{equation}
\label{eq:reg_rad}
    \mathcal{V}_T \leq 2 L T \mathfrak{R}_T(\mathcal{H}).
\end{equation}
Here, $L$ comes from the fact that the loss function considered is $L$-Lipschitz. 

The growth of sequential Rademacher complexities has been shown to be influenced by other related notions of sequential complexities. One prominent example is the sequential fat-shattering dimension \citep{rakhlin2015online}. It was shown in \citet{rakhlin2015online} that this dimension serves as an upper bound to the sequential Rademacher complexity, which subsequently provides a bound on the minimax regret as per \cref{eq:reg_rad}. Similarly, recall that the minimax regret can be lower bounded by the sequential fat-shattering dimension (\cref{eq:Regret_LB}), provided that $\ell_t (h_t(x_t), y_t) = \vert h_t(x_t) - y_t \vert$ and that $\mathcal{P}$ is taken to be the whole set of all distributions on $\mathcal{X}$. In fact, it is also lower bounded by the Rademacher complexity \citep{rakhlin2015online, rakhlin2015sequential}. 
\begin{eqnarray}
\label{eq:Regret_LB_bis}
    \mathcal{V}_T &\geq& \Big< \sup_{\mathcal{D}_t \in \mathcal{P}} \ \mathop{\mathbb{E}}_{x_t \sim \mathcal{D}_t} \Big>_{t=1}^T  \ \mathop{\mathbb{E}}_{\boldsymbol{\epsilon}} \Big[ \sup_{h \in \mathcal{H}} \sum_{t=1}^T \epsilon_t h(\mathbf{x}_t(\boldsymbol{\epsilon})) \Big] \nonumber \\
    &\geq& \frac{1}{4 \sqrt{2}} \sup_{\delta > 0} \Big\{ \sqrt{\delta^2 T \min\{ \text{sfat}_\delta (\mathcal{H}, \mathcal{X}), T\}} \Big\}.
\end{eqnarray}
where ${\boldsymbol{\epsilon}} = (\epsilon_1, \epsilon_2, \cdots, \epsilon_T)$ are Rademacher variables. 

\section{Relation of MGD and Lipschitz Constant}\label{app:lip}


Regarding the relationship between the MGD and the Lipschitz constant of the vector field, we can consider the standard example of a linear ODE: $\dot{x}(t) = a \cdot x(t)$, for which the Lipschitz constant is precisely $a$. For simplicity, assume the initial condition $x(0) = 1$. Hence, the solution is $x(t) = e^{at}$. Then, by (4) we have $c = \frac{1}{T}(e^{aT} - 1)$, and hence:

\[
\begin{aligned}
\frac{1}{T} \int_0^T \|\dot{x}(t) - c\|^2 \, dt
&= \left( \frac{a}{2T} - \frac{1}{T^2} \right)e^{2aT} + \frac{1}{2T^2} e^{aT} - \frac{a}{2T} - \frac{1}{T^2}
\\&= \mathcal{O}\left(\frac{a}{T} e^{2aT}\right)
\\ \implies \text{MGD} &= \mathcal{O}\left(\sqrt{\frac{a}{T}} e^{aT}\right)
\end{aligned}
\]

In particular, for this example, the MGD grows super-exponentially with the Lipschitz constant. This result can be extended to an upper bound for general systems:

\begin{enumerate}
    \item By the Mean Value Theorem, there exists $\xi \in (0, T)$ such that $c = \frac{x(T) - x(0)}{T} = \dot{x}(\xi) = f(x(\xi))$.
    \item Therefore, $\|\dot{x}(t) - c\|^2 = \|\dot{x}(t) - \dot{x}(\xi)\|^2 = \|f(x(t)) - f(x(\xi))\|^2 \leq L^2 \|x(t) - x(\xi)\|^2$.
    \item We can bound the latter quantity via Grönwall's Lemma:
    \[
    \begin{aligned}
    \|x(t) - x(\xi)\|
    &= \left\|\int_{\xi}^t \dot{x}(s) \, ds \right\|
    \\&= \left\|\int_{\xi}^t f(x(s)) \, ds \right\|
    \\&\leq \int_{\xi}^t \|f(x(s))\| \, ds \quad \text{assuming } t \geq \xi
    \\&\leq \int_{\xi}^t \|f(x(s)) - f(x(\xi))\| + \|f(x(\xi))\| \, ds
    \\&\leq \underbrace{(t - \xi) \cdot \|f(x(\xi))\|}_{=\alpha(t)} + \int_{\xi}^t \underbrace{L}_{=\beta(s)} \cdot \underbrace{\|x(s) - x(\xi)\|}_{=u(s)} \, ds
    \\&\leq (t - \xi) \cdot \|f(x(\xi))\| \cdot e^{L(t - \xi)} \quad \text{via Grönwall}
    \end{aligned}
    \]

    And if $\xi \geq t$, we analogously get $\|x(t) - x(\xi)\| \leq (\xi - t) \cdot \|f(x(\xi))\| \cdot e^{L(\xi - t)}$.

    In particular, note that by the same method:
    \[
    \begin{aligned}
    \|f(x(\xi))\| = \frac{1}{T} \|x(T) - x(0)\| \leq \|f(x(0))\| e^{LT}
    \end{aligned}
    \]
\end{enumerate}

\begin{enumerate}
    \setcounter{enumi}{4}
    \item We use this result to bound the integral:
    \[
    \begin{aligned}
    \frac{1}{T} \int_0^T \|\dot{x}(t) - c\|^2 \, dt
    &\leq \frac{L^2}{T} \int_0^T \|x(t) - x(\xi)\|^2 \, dt
    \\&= \frac{L^2}{T} \left( \int_0^\xi \|x(t) - x(\xi)\|^2 \, dt + \int_\xi^T \|x(t) - x(\xi)\|^2 \, dt \right)
    \\&\leq \frac{L^2}{T} \|f(x(\xi))\|^2 \cdot \left( \int_0^\xi (\xi - t)^2 \cdot e^{2L(\xi - t)} \, dt + \int_\xi^T (t - \xi)^2 \cdot e^{2L(t - \xi)} \, dt \right)
    \\&\leq \dots
    \\&\leq \|f(x(\xi))\|^2 \left( (2LT^2 - T + \frac{1}{2L}) e^{2LT} - \frac{1}{2L} \right)
    \\&\lesssim \|f(x(0))\|^2 e^{2LT} \mathcal{O}(LT^2 e^{2LT})
    \\&= \mathcal{O}\left(\|f(x(0))\|^2 LTe^{4LT}\right)
    \end{aligned}
    \]
\end{enumerate}

\begin{enumerate}
    \setcounter{enumi}{5}
    \item Plugging this back into the definition of the MGD, we get:
    \[
    \begin{aligned}
    \text{MGD} &= \sqrt{\frac{1}{T} \int_0^T \|\dot{x}(t) - c\|^2 \, dt} = \mathcal{O}(\sqrt{LT} e^{2LT})
    \end{aligned}
    \]
\end{enumerate}


\section{Meet in the Middle Bound}
\label{sec:meetinmid}
The goal of this section is to prove the following
\begin{customlem}{\ref{lem:supW}}
There is a constant $c>0$ such that with probability at least $1-\delta$ over $\rS \sim \cD^n$ we have
\begin{align*}
    \sup_{w \in \subH(\Gamma_i,L_j)} \left|\E_{\rA,\rt} [\Loss^{\gamma_i/2}_{\rA \cD}(h_{\rA,\rt}(w)) - \Loss^{\gamma_i/2}_{\rA \rS}(h_{\rA,\rt}(w))]\right| &\leq\\ c \left(\sqrt{\frac{(\ell_{j+1}+ \exp(-\gamma_{i+1}^2k/c)) (k + \ln(e/\delta))}{n}} + \frac{(k + \ln(e/\delta))}{n} \right).
\end{align*}
\end{customlem}
Notice here that the two losses $\Loss^{\gamma_i/2}_{\rA \cD}(h_{\rA,\rt}(w))$ and $\Loss^{\gamma_i/2}_{\rA \rS}(h_{\rA,\rt}(w))$ refer to the same margin $\gamma_i/2$ and $h_{\rA,\rt}(w)$ has been discretized to have all coordinates of the form $(1/2)(10 \sqrt{k})^{-1} + z (10 \sqrt{k})^{-1}$ for integer $z$. Intuitively, we will try to exploit this discretization to union bound over a grid of finitely many hypotheses. Unfortunately, the random matrix $\rA$ may increase the norm of $w$ arbitrarily much, and thus a single grid is insufficient. Instead, we need an infinite sequence of grids. For this, let $\Disc_0$ denote the set of all vectors in $4 \Ball_2^k$ whose coordinates are of the form $(1/2)(10 \sqrt{k})^{-1} + z(10 \sqrt{k})^{-1}$ for integer $z$. More generally, let $\Disc_i$ for $i > 0$ denote the set of all vectors in $(2^i \cdot 4 \Ball_2^k)$ whose coordinates are of this form. Since $\|x\|_1 \leq \sqrt{k} \|x\|_2$ for any $x \in \R^k$, we have that $\Disc_i \subset (2^i \cdot 4 \Ball_2^k) \subseteq \sqrt{k}(2^i \cdot 4 \Ball_1^k)$. For a vector $x \in \Disc_i$, let $i(x)=(i_1,\dots,i_k)$ denote the integers so that $x = (10 \sqrt{k})^{-1} i(x) + (1/2)(10 \sqrt{k})^{-1} \AllOne$ with $\AllOne \in \R^k$ the all-1's vector. Then by the triangle inequality, we have $(10 \sqrt{k})^{-1}\|i(x)\|_2 \leq \|x\|_2 + (1/2)(10 \sqrt{k})^{-1}\|\AllOne\|_2 \leq 2^i \cdot 4 + 1/20$. This implies $\|i(x)\|_1 \leq (10 \sqrt{k}) \sqrt{k} (2^i \cdot 4 + 1/20) \leq (5 \cdot 2^{i+3}+1)k$. Since each coordinate of $i(x)$ is an integer, there are thus at most $2^k$ choices for the signs and $\sum_{t=0}^{(5 \cdot 2^{i+3}+1)k} \binom{k + t -1}{t}$ choices for the absolute values of the integers. That is, we have
\begin{align}
|\Disc_i| \leq 2^k \cdot \sum_{t=0}^{(5 \cdot 2^{i+3}+1) k} \binom{k + t -1}{t} \leq 2^{(5 \cdot 2^{i+3}+3)k} \leq 2^{2^{i+7}k}.\label{eq:netsize}
\end{align}
We now start by considering a fixed outcome $A$ of the random matrix $\rA$. For such a fixed $A$, the training set $\rS$ behaves well in the sense that $\Loss^\gamma_{A \cD}(w)$ and $\Loss^\gamma_{A \rS}(w)$ are close with high probability for any $w$. This is formalized in the following remark
\begin{remark}
\label{rmk:concentrationx}
For any distribution $\cD$ over $\finalX \times \{-1,1\}$, fixed $w \in \finalH$, margin $\gamma$ and any $A \in \R^{k \times d}$, it holds with probability at least $1-\delta$ over $\rS \sim \cD^n$ that
\[
|\Loss_{A\cD}^{\gamma}(w) - \Loss_{A\rS}^{\gamma}(w)| \leq \sqrt{\frac{8\Loss_{A \cD}^{\gamma}(w)\ln(1/\delta)}{n}} + \frac{2 \ln(1/\delta)}{n}.
\]
\end{remark}
The proof of Remark~\ref{rmk:concentrationx} is a simple application of Bernstein's and can be found in Appendix~\ref{sec:aux}.

In Lemma~\ref{lem:supW}, the matrix $\rA$ is not fixed but random. Thus we need to find a formal property of the training set $\rS$ under which $\Loss^{\gamma_i/2}_{\rA \cD}(h_{\rA,\rt}(w))$ and $\Loss^{\gamma_i/2}_{\rA \rS}(h_{\rA,\rt}(w))$ are close in expectation over the random choice of $\rA$. With this goal in mind, we now say that a matrix $A$ in the support of $\rA$ and a training set $S$ has \emph{distortion} at least $\beta$, if there is a grid $\Disc_a$ and a vector $w \in \Disc_a$ such that
\[
|\Loss_{A\cD}^{\gamma_i/2}(w) - \Loss_{A\rS}^{\gamma_i/2}(w)| > \beta \cdot \left(\sqrt{\frac{8\Loss_{A \cD}^{\gamma_i/2}(w)(2^{a+7}k + \ln(1/\delta))}{n}} + \frac{2 (2^{a+7}k + \ln(1/\delta))}{n}\right).
\]
For a training set $S$, we use $D_\beta(S)$ to denote the set of matrices $A$ with distortion at least $\beta$ for $S$.

We observe that for a fixed matrix $A$, grid $\Disc_a$ and $\beta>1$, we have by Remark~\ref{rmk:concentrationx} with $\delta'_a = (\delta/2^{2^{a+7}k})^{\beta}$
and a union bound over all $w \in \Disc_a$, that with probability at least $1-|\Disc_a|\delta'_a$, it holds for all $w \in \Disc_a$ that
\begin{align*}
    |\Loss_{A\cD}^{\gamma_i/2}(w) - \Loss_{A\rS}^{\gamma_i/2}(w)| &\leq \sqrt{\frac{8\Loss_{A \cD}^{\gamma_i/2}(w)\ln(1/\delta'_a)}{n}} + \frac{2 \ln(1/\delta'_a)}{n} \\
    &= \sqrt{\frac{8\Loss_{A \cD}^{\gamma_i/2}(w)(\beta 2^{a+7}k + \beta \ln(1/\delta))}{n}} + \frac{2 (\beta 2^{a+7}k + \beta \ln(1/\delta))}{n}  \\
    &\leq \beta \cdot \left(\sqrt{\frac{8\Loss_{A \cD}^{\gamma_i/2}(w)(2^{a+7}k + \ln(1/\delta))}{n}} + \frac{2 (2^{a+7}k + \ln(1/\delta))}{n}\right).
\end{align*}
Thus for $\beta \geq 2$, we have
\begin{align*}
\Pr_{\rS}[A \in D_\beta(\rS)] &\leq \sum_{a=0}^\infty |\Disc_a| \delta'_a \\
&\leq \sum_{a=0}^\infty \delta^\beta \cdot 2^{-(\beta-1)2^{a+7}k} \\
&\leq 2 \cdot \delta^{\beta} \cdot 2^{-(\beta-1)2^{7}k}.
\end{align*}
By Markov's inequality, we have
\begin{align*}
\Pr_{\rS}[\Pr_{\rA}[\rA \in D_\beta(\rS)] > 2 \cdot \delta^{\beta/2} \cdot 2^{-(\beta-1)\cdot 2^{6}k}] &\leq \frac{\E_{\rS}[\Pr_{\rA}[\rA \in D_\beta(\rS)]}{2 \cdot \delta^{\beta/2} \cdot 2^{-(\beta-1)\cdot 2^{6}k}} \\
&= \frac{\E_{\rA}[\Pr_{\rS}[\rA \in D_\beta(\rS)]}{2 \cdot \delta^{\beta/2} \cdot 2^{-(\beta-1)\cdot 2^{6}k}}\\
&\leq \delta^{\beta/2} \cdot 2^{-(\beta-1)2^{6}k}.
\end{align*}
Now call a training set $S$ \emph{representative} if it holds for every $\beta=2^h$ with integer $h \geq 1$ that
\[
\Pr_{\rA}[\rA \in D_\beta(\rS)] \leq 2 \cdot \delta^{\beta/2} \cdot 2^{-(\beta-1)\cdot 2^{6}k}.
\]
A union bound implies that $\rS$ is representative with probability at least
\[
1-\sum_{h=1}^\infty 2 \cdot \delta^{2^{h-1}} \cdot 2^{-(2^h-1)2^{6}k} \geq 1-\frac{\delta}{2^{2^6 k-2}} \geq 1-\delta.
\]
Now define for integer $h \geq 1$ the set
\[
K_h(S) = D_{2^h}(S) \setminus \left(\cup_{b=h+1}^{\infty} D_{2^b}(S) \right).
\]
Let $K_0(S)$ be defined as
\[
K_0(S) = \support(\rA) \setminus \left(\cup_{b=1}^{\infty} D_{2^b}(S) \right).
\]

For any $w \in \finalH$, we may use the triangle inequality to conclude
\begin{align*}
 \left|\E_{\rA,\rt} [\Loss^{\gamma_i/2}_{\rA \cD}(h_{\rA,\rt}(w)) - \Loss^{\gamma_i/2}_{\rA S}(h_{\rA,\rt}(w))]\right| &\leq \\
 \sum_{h=0}^\infty \E_{\rA,\rt}\left[\left|\Loss^{\gamma_i/2}_{\rA \cD}(h_{\rA,\rt}(w)) - \Loss^{\gamma_i/2}_{\rA S}(h_{\rA,\rt}(w))\right| \mid \rA \in K_h(S)\right] \Pr_{\rA}[\rA \in K_h(S)].
\end{align*}
Now consider an $A \in K_h(S)$. Then $A$ has distortion no more than $2^{h+1}$ by definition of $K_h(S)$. This implies that if $h_{A,t}(w)$ is in $\Disc_a$ but not $\Disc_b$ for $b < a$, then $\|h_{A,t}(w)\|_2 \geq 2^{a+1}$ by definition of $\Disc_b$ and we get
\begin{align*}
|\Loss_{A\cD}^{\gamma_i/2}(h_{A,t}(w)) - \Loss_{A\rS}^{\gamma_i/2}(h_{A,t}(w))| &\leq\\
2^{h+1} \cdot \left(\sqrt{\frac{8\Loss_{A \cD}^{\gamma_i/2}(w)(2^{a+7}k + \ln(1/\delta))}{n}} + \frac{2 (2^{a+7}k + \ln(1/\delta))}{n}\right) 
&\leq \\
2^{h+8}  \|h_{A,t}(w)\|_2 \cdot \left(\sqrt{\frac{8\Loss_{A \cD}^{\gamma_i/2}(w)(k + \ln(1/\delta))}{n}} + \frac{2 (k + \ln(1/\delta))}{n}\right).
\end{align*}
Using Cauchy-Schwartz, we thus get for any $w \in \finalH$ that
\begin{align*}
 \left|\E_{\rA,\rt} [\Loss^{\gamma_i/2}_{\rA \cD}(h_{\rA,\rt}(w)) - \Loss^{\gamma_i/2}_{\rA S}(h_{\rA,\rt}(w))]\right| &\leq \\
\sum_{h=0}^\infty 2^{h+8} \E_{\rA,\rt}\bigg[\|h_{\rA,\rt}(w)\|_2  \cdot  \bigg(\sqrt{\frac{8\Loss_{\rA \cD}^{\gamma_i/2}(w)(k + \ln(1/\delta))}{n}} &+\\
\frac{2 (k + \ln(1/\delta))}{n}\bigg)\mid \rA \in K_h(S)\bigg]\Pr_{\rA}[\rA \in K_h(S)]&\leq \\
\sum_{h=0}^\infty 2^{h+8} \sqrt{\E_{\rA,\rt}\left[\|h_{\rA,\rt}(w)\|^2_2  \mid \rA \in K_h(S) \right]} &\ \cdot \\
\sqrt{\E_{\rA,\rt}\left[ \left(\sqrt{\frac{8\Loss_{\rA \cD}^{\gamma_i/2}(w)(k + \ln(1/\delta))}{n}} + \frac{2 (k + \ln(1/\delta))}{n}\right)^2\mid \rA \in K_h(S)\right]}\Pr_{\rA}[\rA \in K_h(S)].
\end{align*}
By Cauchy-Schwartz, this is at most
\begin{align*}
\sqrt{\sum_{h=0}^\infty 2^{2h+16}\E_{\rA,\rt}[\|h_{\rA,\rt}(w)\|_2^2 \mid \rA \in K_h(S) ]  \Pr_{\rA}[\rA \in K_h(S)]} &\ \cdot\\
\sqrt{\sum_{h=0}^\infty \E_{\rA,\rt}\left[ \left(\sqrt{\frac{8\Loss_{\rA \cD}^{\gamma_i/2}(w)(k + \ln(1/\delta))}{n}} + \frac{2 (k + \ln(1/\delta))}{n}\right)^2\mid \rA \in K_h(S)\right]\Pr_{\rA}[\rA \in K_h(S)] }.
\end{align*}
Using Cauchy-Schwartz again and Jensen's inequality, the first sum is bounded by
\begin{align*}
\sum_{h=0}^\infty 2^{2h+16} \E_{\rA,\rt}[\|h_{\rA,\rt}(w)\|^2_2 \mid \rA \in K_h(S) ] \Pr_{\rA}[\rA \in K_h(S)] &\leq \\
\sqrt{ \sum_{h=0}^\infty 2^{4h + 64}\Pr_{\rA}[\rA \in K_h(S)] } \cdot \sqrt{\sum_{h=0}^\infty \E_{\rA,\rt}[\|h_{\rA,\rt}(w)\|^2_2 \mid \rA \in K_h(S) ]^2 \Pr_{\rA}[\rA \in K_h(S)] } &\leq \\
\sqrt{ \sum_{h=0}^\infty 2^{4h + 64}\Pr_{\rA}[\rA \in D_{2^h}(S)] } \cdot \sqrt{\sum_{h=0}^\infty \E_{\rA,\rt}[\|h_{\rA,\rt}(w)\|^4_2 \mid \rA \in K_h(S) ] \Pr_{\rA}[\rA \in K_h(S)] } &\leq \\
\sqrt{ \sum_{h=0}^\infty 2^{4h + 64} 2 (\delta/2^{2^7k+1})^{(2^h-1)/2} } \cdot \sqrt{\E_{\rA,\rt}[\|h_{\rA,\rt}(w)\|^4_2 ] } &\leq \\
2^{33} \cdot \sqrt{\E_{\rA,\rt}[\|h_{\rA,\rt}(w)\|^4_2 ] }.
\end{align*}
Using Jensen's inequality on the second sum, we find that
\begin{align*}
\sum_{h=0}^\infty \E_{\rA,\rt}\left[ \left(\sqrt{\frac{8\Loss_{\rA \cD}^{\gamma_i/2}(w)(k + \ln(1/\delta))}{n}} + \frac{2 (k + \ln(1/\delta))}{n} \right)^2\mid \rA \in K_h(S)\right]\Pr_{\rA}[\rA \in K_h(S)] &=\\
\E_{\rA,\rt}\left[ \left(\sqrt{\frac{8\Loss_{\rA \cD}^{\gamma_i/2}(w)(k + \ln(1/\delta))}{n}} + \frac{2 (k + \ln(1/\delta))}{n} \right)^2\right].
\end{align*}
For positive constants $c_0,c_1,c_2$, we have that the function $f(t)=(\sqrt{c_0 t + c_1} + c_2)^2$ is concave for $t \geq 0$. To see this, we compute its derivative 
\[
f'(t) = 2(\sqrt{c_0 t + c_1} + c_2) \cdot \frac{c_0}{2\sqrt{c_0 t + c_1}} = c_0 + \frac{c_0 c_2}{\sqrt{c_0 t + c_1}},
\]
and its second derivative
\begin{align*}
f''(t) &= \frac{-c_0^2 c_2}{2 (c_0 t + c_1)^{3/2}}.
\end{align*}
This is a negative function for $t \geq 0$. We thus use Jensen's inequality to conclude
\begin{align*}
\E_{\rA,\rt}\left[ \left(\sqrt{\frac{8\Loss_{\rA \cD}^{\gamma_i/2}(w)(k + \ln(1/\delta))}{n}} + \frac{2 (k + \ln(1/\delta))}{n} \right)^2\right] &\leq \\
 \left(\sqrt{\frac{8\E_{\rA,\rt}\left[\Loss_{\rA \cD}^{\gamma_i/2}(w)\right] (k + \ln(1/\delta))}{n}} + \frac{2 (k + \ln(1/\delta))}{n} \right)^2.
\end{align*}
Combining it all, we have thus shown
\begin{align*}
\left|\E_{\rA,\rt} [\Loss^{\gamma_i/2}_{\rA \cD}(h_{\rA,\rt}(w)) - \Loss^{\gamma_i/2}_{\rA S}(h_{\rA,\rt}(w))]\right| &\leq \\
\sqrt{2^{33} \cdot \sqrt{\E_{\rA,\rt}[\|h_{\rA,\rt}(w)\|_2^4]}} \cdot \sqrt{\left(\sqrt{\frac{8\E_{\rA,\rt}\left[\Loss_{\rA \cD}^{\gamma_i/2}(w)\right] (k + \ln(1/\delta))}{n}} + \frac{2 (k + \ln(1/\delta))}{n} \right)^2} &\leq \\
2^{17} \cdot \E_{\rA,\rt}[\|h_{\rA,\rt}(w)\|_2^4]^{1/4} \cdot \left(\sqrt{\frac{8\E_{\rA,\rt}\left[\Loss_{\rA \cD}^{\gamma_i/2}(w)\right] (k + \ln(1/\delta))}{n}} + \frac{2 (k + \ln(1/\delta))}{n} \right).
\end{align*}
We now bound $\E_{\rA,\rt}[\|h_{\rA,\rt}(w)\|_2^4]$ as follows
\begin{align*}
\E_{\rA,\rt}[\|h_{\rA,\rt}(w)\|_2^4] &= \\
\E_{\rA,\rt}[\|\rA w + (h_{\rA,\rt}(w)-\rA w)\|_2^4] &\leq \\
\E_{\rA,\rt}[\left(\|\rA w\|_2 + \|h_{\rA,\rt}(w)-\rA w\|_2\right)^4] &\leq\\
\E_{\rA,\rt}\left[\left(\|\rA w\|_2 + \sqrt{k (10\sqrt{k})^{-2}}\right)^4\right] &=\\
\E_{\rA,\rt}\left[\left(\|\rA w\|_2 + 1/10\right)^4\right] &=\\
\sum_{b=0}^4 \binom{4}{b} \E_{\rA,\rt}[\|\rA w\|_2^b] 10^{-(4-b)}.
\end{align*}
Recalling that $\|\rA w\|_2^2 \sim (1/k)\chi_k^2$, we have from the moments of the chi-square distribution that for even $k \geq 4$:
\[
\E_{\rA,\rt}[\|\rA w\|_2^b] \leq \E_{\rA,\rt}[\|\rA w\|_2^4] =k^{-2}\E_{\rA,\rt}[(k\|\rA w\|_2^2)^2] = k^{-2} 2^2 \frac{(2 + k/2)!}{(k/2)!} \leq 4.
\]
Hence
\begin{align*}
\E_{\rA,\rt}[\|h_{\rA,\rt}(w)\|_2^4] \leq
\sum_{b=0}^4 \binom{4}{b} 4 \cdot 10^{-(4-b)} \leq (4 + 1/10)^4 < 5^4.
\end{align*}
We thus have
\begin{align*}
\left|\E_{\rA,\rt} [\Loss^{\gamma_i/2}_{\rA \cD}(h_{\rA,\rt}(w)) - \Loss^{\gamma_i/2}_{\rA S}(h_{\rA,\rt}(w))]\right| &\leq \\
2^{20} \cdot \left(\sqrt{\frac{8\E_{\rA,\rt}\left[\Loss_{\rA \cD}^{\gamma_i/2}(w)\right] (k + \ln(1/\delta))}{n}} + \frac{2 (k + \ln(1/\delta))}{n} \right).
\end{align*}
Finally, we exploit that for any $w \in \subH(\Gamma_i, L_j)$, we have by definition that $\Loss_\cD^{(3/4)\gamma_{i}}(w)\leq \ell_{j+1}$. Thus for any such $w$, we have
\begin{align*}
\E_{\rA,\rt}[\Loss^{\gamma_i/2}_{\rA \cD}(w)] &=\E_{\rA,\rt}[\Pr_{(\rx,\ry)\sim \cD}[\ry \langle h_{\rA,\rt}(w), \rA\rx \rangle \leq \gamma_i/2]] \\
&=
\E_{(\rx,\ry)\sim \cD}[\Pr_{\rA,\rt}[\ry \langle h_{\rA,\rt}(w), \rA\rx \rangle \leq \gamma_i/2]]
\\&\leq
\Pr_{(\rx,\ry)\sim \cD}[\ry\langle w, \rx \rangle \leq (3/4)\gamma_{i}] \\
&+ \E_{(\rx,\ry)\sim \cD}[\Pr_{\rA,\rt}[\ry \langle h_{\rA,\rt}(w), \rA\rx \rangle \leq \gamma_i/2] \mid \ry\langle w, \rx \rangle > (3/4)\gamma_{i}] \\
&\leq \Loss_{\cD}^{(3/4)\gamma_{i}}(w)  + \sup_{\mu > (3/4)\gamma_i}[\Pr_{\rA,\rt}[\langle h_{\rA,\rt}(w), \rA x \rangle \leq \gamma_i/2 \mid y\ipr{w,x}=\mu].
\end{align*}
Using Lemma~\ref{lem:concdiscretize} and that $\Loss^{(3/4)\gamma_i}_\cD(w) \in L_j$ by definition of $\subH(\Gamma_i,L_j)$, there is a constant $c>0$ such that this is bounded by
\begin{align*}
&\leq \Loss_{\cD}^{(3/4)\gamma_{i}}(w) + c\exp(-k(\gamma_{i}/4)^2/c)\\
&\leq \ell_{j+1} + c \exp(-k \gamma_{i+1}^2/(16 c)).
\end{align*}
We have thus reached the conclusion that there is a constant $c>0$, such that with probability at least $1-\delta$ over $\rS \sim \cD^n$, it holds that
\begin{align*}
    \sup_{w \in \subH(\Gamma_i,L_j)} \left|\E_{\rA,\rt} [\Loss^{\gamma_i/2}_{\rA \cD}(h_{\rA,\rt}(w)) - \Loss^{\gamma_i/2}_{\rA S}(h_{\rA,\rt}(w))]\right| &\leq\\ c \cdot \left(\sqrt{\frac{(\ell_{j+1} + \exp(-k \gamma_{i+1}^2/c)) (k + \ln(1/\delta))}{n}} + \frac{k + \ln(1/\delta)}{n} \right).
\end{align*}
This completes the proof of Lemma~\ref{lem:supW}.

\section{Within Constant Factors}
\label{sec:withinconstant}
In this section we prove
\begin{customlem}{\ref{lem:subgoal2}}
There is a constant $c>1$, such that for any $0 < \delta < 1$ and any $\Gamma_i = (\gamma_i, \gamma_{i+1}]$, it holds with probability at least $1-\delta$ over a random sample $\rS \sim \cD^n$ that
\begin{align*}
\forall w \in \finalH : \Loss_{\rS}^{\gamma_i}(w) \geq \frac{\Loss_{ \cD}^{(3/4)\gamma_i}(w)}{4} - c \left(\frac{\ln(e\gamma_{i+1}^2 n)}{\gamma_{i+1}^2 n} - \frac{\ln(e/\delta)}{n}\right).
\end{align*}
\end{customlem}
The proof follows mostly the ideas in~\cite{SVMbest} that were outlined in the proof overview in Section~\ref{sec:overview}. 

\begin{proof}
Let $k \geq 1$ be a parameter to be determined and consider the random construction of $\rA$ and $\rt$ as defined in Section~\ref{sec:mainargs}. Let $\Disc_a$ be defined as in Section~\ref{sec:meetinmid}, i.e.\ $\Disc_a$ contains all vectors in $2^a \cdot 4 \Ball_2^k$. We say that a matrix $A$ in the support of $\rA$ and a training set $S$ is $\alpha$-\emph{unusual}, if there is a vector $w \in \Disc_0$ such that
\[
\Loss_{A S}^{(7/8)\gamma_i}(w) < \frac{\Loss_{A \cD}^{(7/8)\gamma_i}(w)}{2} - \frac{2^{11} k + \ln(1/\alpha)}{n}.
\]
For a fixed matrix $A$ and vector $w \in \Net_0$, we have by Bernstein's inequality and $\E_{\rS}[\Loss_{A S}^{(7/8)\gamma_i}(w)]= \Loss_{A \cD}^{(7/8)\gamma_i}(w)$ that
\begin{align*}
    \Pr_{\rS}\left[\left|\Loss_{A \rS}^{(7/8)\gamma_i}(w) -\Loss_{A \cD}^{(7/8)\gamma_i}(w)\right|>t/n \right] < \exp\left(- \frac{\frac{1}{2} t^2}{n\Loss_{A \cD}^{(7/8)\gamma_i}(w) + \frac{1}{3}t}\right).
\end{align*}
Setting
\[
t = n \cdot \left( \frac{\Loss_{A \cD}^{(7/8)\gamma_i}(w)}{2} + Z\right)
\]
with $Z=16 \ln(1/\alpha)/n$ gives
\begin{align*}
    \Pr_{\rS}\Bigg[\bigg|\Loss_{A \rS}^{(7/8)\gamma_i}(w) -\Loss&_{A \cD}^{(7/8)\gamma_i}(w)\bigg|>\left( \frac{\Loss_{A \cD}^{(7/8)\gamma_i}(w)}{2} + Z\right) \Bigg]\\ 
    &< \exp\left(- \frac{\frac{n^2}{2} \left( \frac{\Loss_{A \cD}^{(7/8)\gamma_i}(w)}{2} + Z\right)^2}{n\Loss_{A \cD}^{(7/8)\gamma_i}(w) + \frac{n}{3}\left( \frac{\Loss_{A \cD}^{(7/8)\gamma_i}(w)}{2} + Z\right)}\right) \\
    &\leq \exp\left(- \frac{\frac{n^2}{8} \max\{\Loss_{A \cD}^{(7/8)\gamma_i}(w), Z\}^2}{2n \max\{\Loss_{A \cD}^{(7/8)\gamma_i}(w), Z\}}\right) \\
    &\leq \exp\left(-\frac{n Z}{16} \right) \\
    &= \alpha.
\end{align*}
A union bound over all $w \in \Disc_0$ with $\alpha' = \alpha/e^{2^7 k}$ gives that a fixed matrix $A$ is $\alpha$-unusual for $\rS \sim \cD^n$ with probability at most
\[
|\Disc_0| \frac{\alpha}{e^{2^7 k}} < \alpha.
\]
Now call a training set $S$ $\alpha$-representative if $\rA$ is $\alpha$-unusual for $S$ with probability less than $1/4$. By Markov's inequality, we have
\begin{align*}
\Pr_{\rS}[\Pr_{\rA}[(\rS,\rA) \textrm{ is $\alpha$-unusual}] \geq 1/4] &\leq \frac{\E_{\rS}[\Pr_{\rA}[(\rS,\rA) \textrm{ is $\alpha$-unusual}]]}{1/4} \\
&= 4 \cdot \E_{\rA}[\Pr_{\rS}[(\rS,\rA) \textrm{ is $\alpha$-unusual}]]\\
&\leq 4 \alpha.
\end{align*}
Thus
\begin{align}
\Pr_\rS[\rS \textrm{ is $\alpha$-representative}] \geq 1-4 \alpha.\label{eq:oftenrep}
\end{align}
We claim that if the training set $S$ is $\delta$-representative, then it holds for all $w \in \finalH$ that
\[
\Loss^\gamma_{S}(w) \geq \frac{\Loss_{ \cD}^{(3/4)\gamma_i}(w)}{4} - \frac{2^{11} k + \ln(4/\delta)}{n} -30 \exp(-k \gamma_{i+1}^2/2^{14}).
\]
To see this, consider an arbitrary such $S$ and a $w \in \finalH$. Sample $\rA$ and $\rt$ as in the previous section. Call $\rA, \rt$ \emph{good} for $w$ if it satisfies both $\|h_{\rA,\rt}(w)\|_2 \leq 4$ and $\Loss^{(7/8)\gamma_i}_{\rA \cD}(h_{\rA,\rt}(w)) \geq \Loss^{(3/4)\gamma_i}_{\cD}(w) - 25 \exp(-k \gamma_{i+1}^2/2^{14})$. For ease of notation, let $G_w$ denote the set of $(A,t)$ that are good for $w$. Similarly, let $U_S$ denote the set of $A$ where $A$ is $\delta$-unusual for $S$.

For all $w \in \finalH$, $\gamma \in \Gamma_i$, $A$ and $t$, we have that
\begin{align*}
    \Loss_{S}^\gamma(w) &\geq \Loss^{(7/8)\gamma_i}_{A S}(h_{A,\rt}(w)) - \Pr_{(\rx,\ry)\sim S}[\ry \langle w, \rx \rangle > \gamma \wedge \ry \langle h_{A,\rt}(w), A \rx \rangle \leq (7/8)\gamma_i].
\end{align*}
Thus
\begin{align}
    \Loss_{S}^\gamma(w) &\geq \E_{\rA,\rt}[\Loss^{(7/8)\gamma_i}_{\rA S}(h_{\rA,\rt}(w)) - \Pr_{(\rx,\ry)\sim S}[\ry \langle w, \rx \rangle > \gamma \wedge \ry \langle h_{\rA,\rt}(w), \rA \rx \rangle \leq (7/8)\gamma_i]] \nonumber\\
    &\geq \E_{\rA,\rt}[\Loss^{(7/8)\gamma_i}_{\rA S}(h_{\rA,\rt}(w)) \mid (\rA,\rt) \in G_w \wedge \rA \notin U_S] \Pr_{\rA,\rt}[(\rA,\rt) \in G_w \wedge \rA \notin U_S] \label{eq:condGood}\\
    &-\E_{\rA,\rt}[\Pr_{(\rx,\ry)\sim S}[\ry \langle w, \rx \rangle > \gamma \wedge \ry \langle h_{\rA,\rt}(w), \rA \rx \rangle \leq (7/8)\gamma_i]]\label{eq:randroundoff}.
\end{align}
For the term~\eqref{eq:condGood}, we observe that conditioned on $(\rA,\rt) \in G_w$, we have that $h_{\rA,\rt}(w) \in \Disc_0$ since $\|h_{\rA,\rt}(w)\|_2 \leq 4$. Secondly, when $\rA \notin U_S$, this implies by the definition of $\delta$-unusual that
\[
\Loss_{A S}^{(7/8)\gamma_i}(h_{\rA,\rt}(w)) \geq \frac{\Loss_{A \cD}^{(7/8)\gamma_i}(h_{\rA,\rt}(w))}{2} - \frac{2^{11} k + \ln(1/\delta)}{n}.
\]
Hence
\begin{align}
    &\E_{\rA,\rt}[\Loss^{(7/8)\gamma_i}_{\rA S}(h_{\rA,\rt}(w)) \mid (\rA,\rt) \in G_w \wedge \rA \notin U_S] \Pr_{\rA,\rt}[(\rA,\rt) \in G_w \wedge \rA \notin U_S] \geq \nonumber\\
    &\E_{\rA,\rt}\Bigg[\frac{\Loss_{A \cD}^{(7/8)\gamma_i}(h_{\rA,\rt}(w))}{2} \Bigg|(\rA,\rt) \in G_w \wedge \rA \notin U_S\Bigg]\Pr_{\rA,\rt}[(\rA,\rt)\in G_w \wedge \rA \notin U_S] - \frac{2^{11} k + \ln(1/\delta)}{n}.\label{eq:adbound}
\end{align}
Using again that $(\rA, \rt) \in G_w$, we have that~\eqref{eq:adbound} is at least
\begin{align}
    \frac{\Loss_{ \cD}^{(3/4)\gamma_i}(w)}{2} \Pr_{\rA,\rt}[(\rA,\rt)\in G_w \wedge \rA \notin U_S] - \frac{2^{11} k + \ln(1/\delta)}{n} -25 \exp(-k \gamma_{i+1}^2/2^{14}). \label{eq:lastprop}
\end{align}
We now bound $\Pr[(\rA,\rt) \in G_w]$  and $\Pr[\rA \notin U_S]$. For this, we recall that $\|\rA w\|_2^2 \sim (1/k)\chi_2^k$. Thus $\E[\|\rA w\|_2^2] =1$ and by Markov's, we get $\Pr[\|\rA w\|_2^2 \geq 9] \leq 1/9$. Conditioned on $\|\rA w\|_2^2 < 9$, we have $\|h_{\rA,\rt}(w)\|_2 \leq \|\rA w\|_2 + \|h_{\rA,\rt}(w)-\rA w\|_2 \leq \sqrt{9} + \sqrt{k (10 \sqrt{k})^{-2}} < 4$. Next observe that
\begin{align*}
    \Loss_{\rA \cD}^{(7/8)\gamma_i}(h_{\rA,\rt}(w)) &\geq \Loss_{\cD}^{(3/4)\gamma_i}(w) - \Pr_{(\rx, \ry)\sim \cD}[\ry \langle w, \rx \rangle \leq (3/4)\gamma_i \wedge \ry \langle h_{\rA,\rt}(w), \rA \rx \rangle > (7/8)\gamma_i].
\end{align*}
We have by Lemma~\ref{lem:concdiscretize} that there is a constant $c>0$ so that
\begin{align*}
    \E_{\rA,\rt}[\Pr_{(\rx, \ry)\sim \cD}[\ry \langle w, \rx \rangle \leq (3/4)\gamma_i \wedge \ry \langle h_{\rA,\rt}(w), \rA \rx \rangle > (7/8)\gamma_i]] &= \\
    \E_{(\rx, \ry)\sim \cD}[\Pr_{\rA,\rt}[\ry \langle w, \rx \rangle \leq (3/4)\gamma_i \wedge \ry \langle h_{\rA,\rt}(w), \rA \rx \rangle > (7/8)\gamma_i]] &\leq \\
    \sup_{x \in \finalX : \langle w, x\rangle \leq (3/4)\gamma_i}\Pr_{\rA,\rt}[ \langle h_{\rA,\rt}(w), \rA x \rangle > (7/8)\gamma_i] &\leq \\
    c\exp(-k(\gamma_i/8)^2/c) &\leq \\
    c\exp(-k \gamma_{i+1}^2/(2^{8}c)).
\end{align*}
Thus by Markov's inequality, we conclude
\begin{align*}
\Pr_{\rA,\rt}[\Loss_{\rA \cD}^{(7/8)\gamma_i}(h_{\rA,\rt}(w)) <\Loss_{\cD}^{(3/4)\gamma_i}(w)- 5c \exp(-k \gamma_{i+1}^2/(2^{8}c))] &\leq \\
    \Pr_{\rA,\rt}[\Pr_{(\rx, \ry)\sim \cD}[\ry \langle w, \rx \rangle \leq (3/4)\gamma_i \wedge \ry \langle h_{\rA,\rt}(w), \rA \rx \rangle > (7/8)\gamma_i] > 5 c \exp(-k \gamma_{i+1}^2/(2^{8}c))] &< 1/5.
\end{align*}
Finally, since we assumed $S$ is $\delta$-representative, we have $\Pr_{\rA}[\rA \in U_S] \leq 1/4$ by definition of $\delta$-representative. We conclude by a union bound that
\begin{align*}
\Pr_{\rA,\rt}[(\rA,\rt)\in G_w \wedge \rA \notin U_S] &\geq 1-1/9-1/5-1/4\geq 1/2.
\end{align*}
In summary, we have shown that~\eqref{eq:lastprop} is at least
\[
\frac{\Loss_{ \cD}^{(3/4)\gamma_i}(w)}{2} \cdot \frac{1}{2} - \frac{2^{11} k + \ln(1/\delta)}{n} -5 c \exp(-k \gamma_{i+1}^2/(2^{8}c)).
\]
Recalling that~\eqref{eq:condGood}~$\geq$~\eqref{eq:lastprop} gives
\begin{align*}
\E_{\rA,\rt}[\Loss^{(7/8)\gamma_i}_{\rA S}(h_{\rA,\rt}(w)) \mid (\rA,\rt) \in G_w \wedge \rA \notin U_S] \Pr_{\rA,\rt}[(\rA,\rt) \in G_w \wedge \rA \notin U_S] &\geq \\
\frac{\Loss_{ \cD}^{(3/4)\gamma_i}(w)}{4} - \frac{2^{11} k + \ln(1/\delta)}{n} -5c \exp(-k \gamma_{i+1}^2/(2^{8}c)).
\end{align*}
The term~\eqref{eq:randroundoff} can be bounded using Lemma~\ref{lem:concdiscretize} by
\begin{align*}
\E_{\rA,\rt}[\Pr_{(\rx,\ry)\sim S}[\ry \langle w, \rx \rangle > \gamma \wedge \ry \langle h_{\rA,\rt}(w), \rA \rx \rangle \leq (7/8)\gamma_i]] &= \\
\E_{(\rx,\ry)\sim S}[\Pr_{\rA,\rt}[\ry \langle w, \rx \rangle > \gamma \wedge \ry \langle h_{\rA,\rt}(w), \rA \rx \rangle \leq (7/8)\gamma_i]] &\leq \\
\sup_{x \in \finalX : \langle w, x\rangle > \gamma }\Pr_{\rA,\rt}[\langle h_{\rA,\rt}(w), \rA x \rangle \leq (7/8)\gamma_i] &\leq\\
c\exp(-k(\gamma - (7/8)\gamma_i)^2/c) &\leq \\
c\exp(-k\gamma_i^2/(64 c))&\leq \\
c\exp(-k\gamma_{i+1}^2/(2^{8}c)).
\end{align*}
In summary, we have shown that for $(\delta/4)$-representative $S$, it holds for all $w \in \finalH$ that
\begin{align*}
    \Loss_S^\gamma(w) \geq \frac{\Loss_{ \cD}^{(3/4)\gamma_i}(w)}{4} - \frac{2^{11} k + \ln(4/\delta)}{n} -6c \exp(-k \gamma_{i+1}^2/(2^{8}c)).
\end{align*}
We finally conclude from~\eqref{eq:oftenrep} that with probability at least $1-\delta$ over $\rS$, it holds for all $w \in \finalH$ that
\begin{align*}
    \Loss_{\rS}^\gamma(w) \geq \frac{\Loss_{ \cD}^{(3/4)\gamma_i}(w)}{4} - \frac{2^{11} k + \ln(4/\delta)}{n} -6c \exp(-k \gamma_{i+1}^2/(2^{8}c)).
\end{align*}
Picking $k=2^{8}c \gamma_{i+1}^{-2} \ln(\gamma^2_{i+1} n)$ finally results in
\begin{align*}
    \Loss_{\rS}^\gamma(w) \geq \frac{\Loss_{ \cD}^{(3/4)\gamma_i}(w)}{4} - \frac{2^{20}c \ln(\gamma_{i+1}^2 n)}{\gamma_{i+1}^2 n} - \frac{2\ln(e/\delta)}{n}.
\end{align*}
This completes the proof.
\end{proof}

\section*{Acknowledgment}
The authors would like to thank Clement Svendsen for valuable measure theoretic insight. 

Kasper Green Larsen is co-funded by a DFF Sapere Aude Research Leader Grant No. 9064-00068B by the Independent Research Fund Denmark and co-funded by the European Union (ERC, TUCLA, 101125203). Natascha Schalburg is funded by the European Union (ERC, TUCLA, 101125203). Views and opinions expressed are however those of the author(s) only and do not necessarily reflect those of the European Union or the European Research Council. Neither the European Union nor the granting authority can be held responsible for them.

\bibliography{SVMBib}
\bibliographystyle{abbrvnat}

\appendix


\section{Auxiliary Results for Proofs}
\label{auxiliary}
In this subsection, we present some auxiliary results that are needed for our proof.
First, we present the estimation of the spectral norm of random matrices.
It can be easily derived from \cite{vershynin2018high} and we put it here for the completeness.

\begin{lemma}\citep[Adapted from Theorem 4.6.1]{vershynin2018high}
\label{lem:conrg}
    For a random sub-Gaussian matrix $\widetilde{\bm X} \in \mathbb{R}^{N \times d}$ whose rows are i.i.d. isotropic sub-gaussian random vector with sub-Gaussian norm $K$, then we have the following statement
\[
\mathbb{P} \left(   \left\|\frac{1}{N}\widetilde{\bm X}^{\!\top}\widetilde{\bm X}-\bm I_d\right\|_{op}  > \delta \right) \leq 2 \exp \left( -C N \min\left(\delta^2, \delta\right) \right)\,.
\]
for a universal constant $C$ depending only on $K$.
\end{lemma}

\begin{lemma}\citep[Adapted from Corollary 5.35]{vershynin2010introduction}
\label{lem:init-op-conct}
    For a random standard Gaussian matrix $\bm S\in\mathbb{R}^{d\times r}$ with $[\bm S]_{ij} \sim \mathcal{N}(0, 1)$, if $d > 2r$, we have 
    \begin{align}
        \label{norm-A0}
        \frac{\sqrt{d}}{2} \leq \|\bm S\|_{op} \leq (2 \sqrt{d} + \sqrt{r})\,,
    \end{align}
    with probability at least $1-C \operatorname{exp}(-d)$ for some positive constants $C$.
\end{lemma}

The following results are modified from the proof of \citet[Lemma 8.7]{stoger2021small}.
\begin{lemma}
\label{lem:min-singular-conct}
    Suppose $\bm S\in\mathbb{R}^{d\times r}$ is a random standard Gaussian matrix with $[\bm S]_{ij} \sim \mathcal{N}(0, 1)$ and $\bm U\in\mathbb{R}^{d\times r^*}$ has orthonormal columns. If $r\geq 2r^*$, with probability at least $1-C\operatorname{exp}(-r)$ for some positive constants $C$, we have
    \begin{align*}
        \lambda_{\operatorname{min}}(\bm U^{\!\top}\bm S) & \gtrsim 1\,.
    \end{align*}
    If $r^*\leq r < 2r^*$, by choosing $\xi>0$ appropriately, with probability at least $1-(C \xi)^{r-r^*+1}-C'\operatorname{exp}(-r)$ for some positive constants $C\,,C'$, we have
    \begin{align*}
        \lambda_{\operatorname{min}}(\bm U^{\!\top}\bm S) & \gtrsim \frac{\xi}{r}\,.
    \end{align*}
\end{lemma}

Next, we give a short description of the Hermite expansion of ReLU function via Hermite polynomials. Details can be found in \citet[A.1.1]{damian2022neural} and \cite{arous2021online}.
To be specific, the Hermite expansion of ReLU function $\sigma(x)$ is
\begin{align}
\label{Hermite-sigma}
    \sigma(x)=\sum_{j=1}^\infty \frac{c_j}{j!}\operatorname{He}_j(x) =\frac{1}{\sqrt{2\pi}}+\frac{1}{2}x+\frac{1}{\sqrt{2\pi}}\sum_{j\geq 1}\frac{(-1)^{j-1}}{j!2^j(2j-1)}\operatorname{He}_{2j}(x)\,,
\end{align}
which implies that we can express the Hermite coefficients as
\begin{align}
\label{Hermite-coef}
    \left\{\begin{aligned}
        c_0 & = \frac{1}{\sqrt{2\pi}}\,,\\
        c_1 & = \frac{1}{2}\,,\\
        c_{2j} & = \frac{(-1)^{j-1}}{\sqrt{2\pi}2^j(2j-1)}\quad \text{for }j\geq 1\,.
    \end{aligned}\right.
\end{align}
Furthermore, the derivative of $\sigma(x)$ admits
\begin{align}
\label{Hermite-sigma'}
    \sigma'(x)=\frac{1}{2}+\frac{1}{\sqrt{2\pi}}\sum_{j\geq 0}\frac{(-1)^{j}}{j!2^j(2j+1)}\operatorname{He}_{2j+1}(x)\,.
\end{align}

\begin{lemma}\citep[Corollary 9]{oko2024pretrained}\label{differential}
$\mathbb{E}_{\widetilde{\bm x}}[\nabla^k \sigma(\langle \bm w\,, \widetilde{\bm x}\rangle)] = c_k \bm w^{\otimes k}$ for any $k$ such that $c_k\neq 0$.
\end{lemma}

\begin{lemma}\label{vec-ineq}
For any vectors $\bm u$ and $\bm v$, we have
    \begin{align*}
        \left|\langle \bm u\,, \bm u \rangle^j - \langle \bm u\,, \bm v \rangle^j\right| & \leq j\,\max\left\{\left\|\bm u\right\|_2\,,\left\|\bm v\right\|_2\right\}^{2j-1} \left\|\bm u - \bm v\right\|_2\,.
    \end{align*}
\end{lemma}
\begin{proof}
    First, we analyze the following two scalar variables case
    \begin{align*}
        \left|x^j-y^j\right|\,.
    \end{align*}
    By algebraic identity $\sum_{j=1}^{t-1}x^{t-j-1}y^j=\frac{x^t-y^t}{x-y}$ which is valid for $\forall\,j\in\mathbb{N}^+$, we have
    \begin{align*}
        \left|x^j-y^j\right|&=\left|(x-y)\sum_{i=0}^{j-1}x^{j-i-1}y^i\right|
        \leq |x-y|\sum_{i=0}^{j-1}\max\left\{|x|\,,|y|\right\}^{j-1}
        = j|x-y|\max\left\{|x|\,,|y|\right\}^{j-1}\,.
    \end{align*}
    Now we define $x:=\langle \bm u\,, \bm u \rangle$ and $y:=\langle \bm u\,, \bm v \rangle$, then we can obtain
    \begin{align*}
        \left|\langle \bm u\,, \bm u \rangle^j - \langle \bm u\,, \bm v \rangle^j\right| & \leq j\,\max\left\{\left|\langle \bm u\,, \bm u \rangle\right|\,,\left|\langle \bm u\,, \bm v \rangle\right|\right\}^{j-1}\left|\langle \bm u\,, \bm u \rangle - \langle \bm u\,, \bm v \rangle\right|\\
        & \leq j\,\max\left\{\left\|\bm u\right\|_2^2\,,\left\|\bm u\right\|_2 \left\|\bm v\right\|_2\right\}^{j-1}\left\|\bm u\right\|_2 \left\|\bm u - \bm v\right\|_2\quad \tag*{\color{teal}[by Cauchy-Schwartz inequality]}\\
        & = j\,\max\left\{\left\|\bm u\right\|_2\,,\left\|\bm v\right\|_2\right\}^{2j-1} \left\|\bm u - \bm v\right\|_2\,.
    \end{align*}
\end{proof}

\end{document}
% \typeout{get arXiv to do 4 passes: Label(s) may have changed. Rerun}