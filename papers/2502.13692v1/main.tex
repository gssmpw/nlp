\documentclass{article} 

\usepackage{amsmath}
\usepackage{amssymb}
\usepackage{amsthm}
\usepackage{mathtools}

\usepackage{hyperref}
\hypersetup{
    colorlinks=true,
    linkcolor=blue,
    urlcolor=magenta,
    citecolor=blue, 
    }
\usepackage{xcolor}
\usepackage{times}
\usepackage{fullpage}

\usepackage{natbib}

\newtheorem{property}{Property}
\newtheorem{claim}{Claim}
\newtheorem{example}{Example}
\newtheorem{theorem}{Theorem}
\newtheorem{definition}[theorem]{Definition}
\newtheorem{lemma}[theorem]{Lemma}
\newtheorem{remark}[theorem]{Remark}

\newtheorem{innercustomlem}{Restatement of Lemma}
\newenvironment{customlem}[1]
  {\renewcommand\theinnercustomlem{#1}\innercustomlem}
  {\endinnercustomlem}

\newtheorem{innercustomthm}{Restatement of Theorem}
\newenvironment{customthm}[1]
  {\renewcommand\theinnercustomthm{#1}\innercustomthm}
  {\endinnercustomthm}

\newtheorem{innercustomrmk}{Restatement of Remark}
\newenvironment{customrmk}[1]
  {\renewcommand\theinnercustomrmk{#1}\innercustomrmk}
  {\endinnercustomrmk}

\newtheorem{innercustomclm}{Restatement of Claim}
\newenvironment{customclm}[1]
  {\renewcommand\theinnercustomclm{#1}\innercustomclm}
  {\endinnercustomclm}

\newcommand{\rA}{\mathbf{A}}
\newcommand{\rt}{\mathbf{t}}
\newcommand{\ra}{\mathbf{a}}
\newcommand{\rv}{\mathbf{v}}
\newcommand{\rZ}{\mathbf{Z}}
\newcommand{\rN}{\mathbf{N}}
\newcommand{\rX}{\mathbf{X}}
\newcommand{\rY}{\mathbf{Y}}
\newcommand{\rx}{\mathbf{x}}
\newcommand{\ry}{\mathbf{y}}
\newcommand{\rS}{\mathbf{S}}
\newcommand{\rE}{\mathbf{E}}
\newcommand{\rM}{\mathbf{M}}
\newcommand{\rsigma}{\mathbf{\sigma}}

\newcommand{\cH}{\mathcal{S}^{d-1}}
\newcommand{\Rad}{\mathcal{R}}
\newcommand{\Loss}{\mathcal{L}}
\newcommand{\cD}{\mathcal{D}}
\newcommand{\Norm}{\mathcal{N}}
\newcommand{\Disc}{\mathcal{G}}
\newcommand{\ncH}{\mathcal{S}^d}
\newcommand{\embH}{\mathcal{H}}
\newcommand{\Ac}{\mathcal{A}}
\newcommand{\Xc}{\mathcal{X}}
\newcommand{\Net}{\mathcal{W}}
\newcommand{\Bc}{\mathcal{B}}
\newcommand{\AllOne}{\mathbf{1}}
\newcommand{\subH}{\mathcal{H}}


\newcommand{\R}{\mathbb{R}}
\newcommand{\N}{\mathbb{N}}
\newcommand{\Ball}{\mathbb{B}}
\newcommand{\Balld}{\mathbb{B}_2^d}
\newcommand{\E}{\mathbb{E}}
\newcommand{\Z}{\mathbb{Z}}

\newcommand{\eps}{\varepsilon}
\newcommand{\eqd}{\,{\buildrel d \over =}\,}
\DeclareMathOperator{\support}{supp}
\DeclareMathOperator{\sign}{sign}
\newcommand{\Ints}{\mathbb{Z}}
\newcommand{\finalH}{\mathcal{H}}
\newcommand{\finalX}{\mathcal{X}}
\DeclareMathOperator{\poly}{poly}

\usepackage{bbm}
\usepackage{dsfont}
\newcommand{\indi}[1]{\mathds{1}\{#1\}}
\DeclarePairedDelimiter{\floor}{\lfloor}{\rfloor}
\DeclarePairedDelimiter{\norm}{\parallel}{\parallel}
\DeclarePairedDelimiter{\ipr}{\langle}{\rangle}

\renewcommand{\Pr}{\mathbb{P}}

\title{Tight Generalization Bounds for Large-Margin Halfspaces}

\author{Kasper Green Larsen \qquad\qquad Natascha Schalburg \\ 
        \texttt{\{larsen,n.schalburg\}@cs.au.dk}\\
        Computer Science, Aarhus University
}
\date{}

\begin{document}

\maketitle

\begin{abstract}
Retrieval-Augmented Generation (RAG) is often used with Large Language Models (LLMs) to infuse domain knowledge or user-specific information. In RAG, given a user query, a retriever extracts chunks of relevant text from a knowledge base. These chunks are sent to an LLM as part of the input prompt. Typically, any given chunk is repeatedly retrieved across user questions. However, currently, for every question, attention-layers in LLMs fully compute the key values (KVs) repeatedly for the input chunks, as state-of-the-art methods cannot reuse KV-caches when chunks appear at arbitrary locations with arbitrary contexts. Naive reuse leads to output quality degradation.  This leads to potentially redundant computations on expensive GPUs and increases latency. In this work, we propose \sys, a system for managing and reusing precomputed KVs corresponding to the text chunks (we call \textit{chunk-caches}) in RAG-based systems. We present how to identify \hl{\textit{chunk-caches} that are reusable}, how to efficiently perform a small fraction of recomputation to \textit{fix} the cache to maintain output quality, and how to efficiently store and evict \textit{chunk-caches} in the hardware for maximizing reuse while masking any overheads. With real production workloads as well as synthetic datasets, we show that \sys reduces redundant computation by \textbf{51\%} over SOTA prefix-caching and \textbf{75\%} over full recomputation.
\hl{Additionally, with continuous batching on a real production workload, we get a \textbf{1.6$\times$} speedup in throughput and a \textbf{2$\times$} reduction in end-to-end response latency over prefix-caching while maintaining quality, for both the \llama-3-8B and \llama-3-70B models. 
}
\end{abstract}






\section{Introduction}
\label{sec:intro}

\begin{figure*}[tb]
    \centering
    \includegraphics[width=0.848\linewidth]{figs/circuitnn.pdf} 
    \caption{Illustration of differentiable CircuitNN. CircuitNN is designed based on differentiable NAND gates. After DAS is guided by PI and PO pairs of the truth table, CircuitNN can get the precise circuit architecture logic equivalent to the truth table.}
    \label{fig:circuitnn}
\end{figure*}

% 1. Describe the importance of logic synthesis
% 2. Existing Problems
% (a) Neural Architecture Search: Unstable, Predefined Setting, etc.
% (b) Circuit Generation: Probabilistic Model, Logic Equivalence

With the rapid advancement of technology, the scale of integrated circuits (ICs) has expanded exponentially. 
This expansion has introduced significant challenges in chip manufacturing, particularly concerning power and area metrics.
A primary objective in IC design is achieving the same circuit function with fewer transistors, thereby reducing power usage and area occupancy.

Logic synthesis~\cite{hachtel2005logicsynth}, a critical step in electronic design automation (EDA), transforms behavioral-level circuit designs into optimized gate-level circuits, ultimately yielding the final IC layout. 
The primary goal of logic synthesis is to identify the physical implementation with the fewest gates for a given circuit function. 
This task constitutes a challenging NP-hard combinatorial optimization problem. 
Current logic synthesis tools~\cite{brayton2010abc, wolf2013yosys} rely on human-designed heuristics, often leading to sub-optimal outcomes.

Differentiable architecture search (DAS) techniques~\cite{liu2018darts, chu2020darts} offer novel perspectives on addressing challenges in this problem.
Circuit functions can be represented through truth tables, which map binary inputs to their corresponding outputs. 
Truth tables provide a precise representation of input-output relationships, ensuring the design of functionally equivalent circuits.
Inspired by this, researchers~\cite{deepmind2024ai4sys, wang2024tnet} have begun exploring the application of DAS to synthesize circuits directly from truth tables.
Specifically, \citet{deepmind2024ai4sys} proposed CircuitNN, a framework that learns differentiable connection structures with logic gates, enabling the automatic generation of logic circuits from truth tables.
This approach significantly reduces the complexity of traditional circuit generation. 
Building on this, \citet{wang2024tnet} introduced T-Net, a triangle-shaped variant of CircuitNN, incorporating regularization techniques to enhance the efficiency of DAS.

Despite these advancements, several challenges remain. 
The computational complexity of DAS grows quadratically with the number of gates, posing scalability issues.
Although triangle-shaped architecture~\cite{wang2024tnet} partially mitigates this problem, redundancy persists. 
%Additionally, DAS is susceptible to converging to local optima, limiting the ability to search architectures that satisfy the given truth tables~\cite{liu2018darts}. 
%Furthermore, hyperparameters (network depth and layer width) require extensive searches, introducing complexity and prolonging the synthesis process. 
Additionally, DAS is susceptible to converging to local optima~\cite{liu2018darts} and hyperparameters (network depth and layer width) require extensive searches. 
The challenges arise from the vast search space in DAS. 
% Even with predefined settings for CircuitNN, finding a configuration that meets the truth table requires extensive trial and error during the DAS process. 
Intuitively, limiting the search space through predefined parameters (network depth, gates per layer, and connection probabilities) can significantly reduce the complexity.

Recent advances~\cite{openai2023gpt4, abramson2024alphafold3, esser2024sd3, li2024mar} in conditional generative models have demonstrated remarkable performance across language, vision, and graph generation tasks. 
Motivated by these developments, we propose a novel approach to circuit generation that generates preliminary circuit structures to guide DAS in generating refined circuits matching specified truth tables. 
Firstly, we introduce CircuitVQ, a tokenizer with a discrete codebook for circuit tokenization. 
Built upon our Circuit AutoEncoder framework~\cite{hou2022graphmae,li2023maskgae,wu2025mgvga}, CircuitVQ is trained through a circuit reconstruction task. 
Specifically, the CircuitVQ encoder encodes input circuits into discrete tokens using a learnable codebook, while the decoder reconstructs the circuit adjacency matrix based on these tokens.
Subsequently, the CircuitVQ encoder serves as a circuit tokenizer for CircuitAR pretraining, which employs a masked autoregressive modeling paradigm~\cite{chang2022maskgit, li2023mage}. 
In this process, the discrete codes function as supervision signals. 
After training, CircuitAR can generate discrete tokens progressively, which can be decoded into initial circuit structures by the decoder of the CircuitVQ. 
These prior insights can guide DAS in producing refined circuits that match the target truth tables precisely.

Our key contributions can be summarized as follows:
\begin{itemize}
\item We introduce CircuitVQ, a circuit tokenizer that facilitates graph autoregressive modeling for circuit generation, based on our Circuit AutoEncoder framework;
\item Develop CircuitAR, a model trained using masked autoregressive modeling, which generates initial circuit structures conditioned on given truth tables;
\item Propose a refinement framework that integrates differentiable architecture search to produce functionally equivalent circuits guided by target truth tables;
\item Comprehensive experiments demonstrating the scalability and capability emergence of our CircuitAR and the superior performance of the proposed circuit generation approach.
\end{itemize}

% Motivation
% (a) Diffusion (Vision, Graph), Autoregressive (Language, Vision)
% (b) Circuit Generation for Predefined Setting
% (c) Neural Architecture Search for Strict Logic Equivalence

% Contribution
% (a) Circuit Tokenizer (new transformer arch, training strategy)
% (b) CircuitAR (train and gen strategies, post-ar strategy)
% (c) Extensive Evaluation including BitD (Bit Distance) for Scalability


\begin{figure*}[t]
\begin{center}
\includegraphics[width=.85\linewidth]{fig_overview_v3.pdf}
\end{center}
\caption{
FastAtlas Overview: In each frame, we compute charts spanning fully or partially visible triangles (a), determine texture space bounding boxes for the visible portions of the view-space projections of each chart, and tightly pack these boxes into atlases (b, here $2K \times 2K$). We simultaneously bijectively parameterize and shade the charts into their atlas boxes, obtaining high quality texture space shading (c), and use this shading to render the shaded frames (d).}
\label{fig:overview}
\label{fig:alg_overview}
\end{figure*}

\section{Overview}
\label{sec:overview}
Our work has two core contributions: a real-time, GPU-based algorithm for tight packing of general parameterized charts into compact atlases; and a real-time TSS method that
utilizes this packing.  

\paragraph*{FastAtlas Packing.}
FastAtlas runs entirely on the GPU as a series of compute shaders. It takes the bounding boxes of parameterized charts as input, and packs them into an atlas (Fig~\ref{fig:overview}b, Sec.~\ref{sec:pack}). As such, the only input it requires are the dimensions of the bounding boxes.
Its outputs are deterministic; identical input charts are packed into identical atlases. This is critical for TSS and similar applications, as it ensures that consecutive frames taken from the same camera view have the same shading. Even minute shading differences across such frames can cause sampling jitter, leading to undesirable flicker \cite{baker2012rock}. 
While prior methods such as \cite{mueller2018shading,hladky2019tessellated,hladky2021snakebinning,Neff2022MSA} cap the dimensions of the charts that can be packed as-is for a given atlas size, and scale down all charts that exceed these dimensions, we scale all charts by the same factor, and do so only when strictly necessary to achieve packing success (Figs~\ref{fig:atlas},~\ref{fig:sas_issues}). 

\paragraph*{TSS using FastAtlas.}
Our end-to-end TSS atlas generation method combines the packing method above with a novel approach for computing seamless per-frame charts. 
We define our charts as the connected components of the visible surfaces in each frame (Fig.~\ref{fig:overview}a), and efficiently compute them using a parallel union-find algorithm (Sec.~\ref{sec:visible}). Since the boundaries of these charts coincide with the contours of the rendered surface, they are {\em invisible} to the viewer. This approach 
eliminates the artifacts caused by shading discontinuities along visible seams (Fig.~\ref{fig:seams}). 

\begin{parWithWrapFigure}
\begin{wrapfigure}{l}{.27\columnwidth}%
\includegraphics[width=\linewidth]{fig_inset_view_plane.pdf}%
\end{wrapfigure}
We bijectively parametrize the {\em visible portions} of our charts by projecting them to view space (inset). This maps a constant number of texels to each pixel in the final rendered output, evenly distributing residual undersampling error across all image pixels. While conceptually straightforward, efficiently parameterizing charts containing partially visible triangles using viewspace projection is non-trivial, as the visible portions may no longer be triangular (e.g. green triangle in the inset); applying naive projection to triangles with vertices behind the camera may produce ill-posed results. Clipping triangles before projection is both computationally expensive and significantly complicates downstream operations. We avoid explicit clipping by observing that all that is required for atlas packing is the dimensions of, potentially conservative, bounding boxes of these projected visible portions. We compute such bounding boxes without explicit chart clipping by adapting a conservative screen coverage estimator \shortcite{Blinn:CalculatingScreenCoverage} (Sec.~\ref{sec:box}). We then pack the computed boxes using FastAtlas. 
\end{parWithWrapFigure}

Finally, we shade the visible portion of each chart into its corresponding atlas bounding box (Fig~\ref{fig:overview}c). 
The resulting texture is then used during rasterization as a standard texture map (Fig. ~\ref{fig:overview}d). 
Our framework is compatible with all existing approaches for texture space shading, including forward shading, raytraced illumination, or deferred shading in texture space \cite{baker:2016}. In the examples shown, we use the standard forward shading based rendering pipeline included in the G3D Innovation Engine \cite{G3D17}, a commercial grade renderer.


\section{Main Proof}
\label{sec:mainproof}
We now set out to prove Theorem~\ref{thm:main} following the proof outline sketched in Section~\ref{sec:overview}. We start by a series of reductions that allow us to focus on a simpler task of establishing Theorem~\ref{thm:main} only for a small range of $\gamma$ and $\Loss_{\rS}^\gamma(w)$. We describe these reductions in Section~\ref{sec:reduct} and then proceed to the main arguments in Section~\ref{sec:mainargs}.

\iffalse
\begin{table*}[htbp]
\tiny
\begin{center}
\begin{tabular}{lccccccccccccc}\toprule
Model, ft setting & mem & \#param & ARC-c & ARC-e & BoolQ & HS & OBQA & PIQA & rte & SIQA & WG & Avg
%\\\cmidrule(lr){2-3}\cmidrule(lr){4-5} \cmidrule(lr){6-7} \cmidrule(lr){8-9}\cmidrule(lr){10-11} \cmidrule(lr){12-13} \cmidrule(lr){14-15} \cmidrule(lr){16-17} 
\\\cmidrule(lr){1-13}
Llama2(7B), LoRA, $r=64$ & 23.46GB & 159.9M(2.37\%) & \textbf{44.97} & 77.02 & 77.43 & \textbf{57.75} & 32.0 & \textbf{78.45} & 62.09 & \textbf{47.75} & 68.75 & 60.69\\
Llama2(7B), SPruFT, $r=128$ & \textbf{17.62GB} & 145.8M(2.16\%) & 43.60 & \textbf{77.26} & \textbf{77.77} & 57.47 & \textbf{32.6} & 78.07 & \textbf{64.98} & 46.67 & \textbf{69.30} & \textbf{60.86} \\\cmidrule(lr){2-13}
Llama2(7B), FA-LoRA, $r=64$ & 17.25GB & 92.8M(1.38\%) & 43.77 & \textbf{77.57} & 77.74 & \textbf{57.45} & 31.0 & 77.86 & \textbf{66.06} & \textbf{47.13} & 69.06 & 60.85\\
Llama2(7B), FA-SPruFT, $r=128$ & \textbf{15.21GB} & 78.6M(1.17\%) & \textbf{43.94} & 77.22 & \textbf{77.83} & 57.11 & \textbf{32.0} & \textbf{78.18} & 65.70 & 46.47 & \textbf{69.38} & \textbf{60.87}\\\midrule
Llama3(8B), LoRA, $r=64$ & 30.37GB & 167.8M(2.09\%) & \textbf{53.07} & \textbf{81.40} & \textbf{82.32} & \textbf{60.67} & 34.2 & \textbf{79.98} & 69.68 & \textbf{48.52} & \textbf{73.56} & \textbf{64.82}\\
Llama3(8B), SPruFT, $r=128$ & \textbf{24.49GB} & 159.4M(1.98\%) & 52.47 & 81.10 & 81.28 & 60.29 & \textbf{34.6} & 79.76 & \textbf{70.04} & 47.75 & 73.24 & 64.50 \\\cmidrule(lr){2-13}
Llama3(8B), FA-LoRA, $r=64$ & 24.55GB & 113.2M(1.41\%) & \textbf{52.47} & \textbf{81.36} & \textbf{82.23} & 60.17 & \textbf{35.0} & \textbf{79.76} & \textbf{70.04} & \textbf{48.31} & \textbf{73.56} & \textbf{64.77}\\
Llama3(8B), FA-SPruFT, $r=128$ & \textbf{22.41GB} & 92.3M(1.15\%) & 52.22 & 81.19 & 81.35 & \textbf{60.20} & 34.2 & 79.71 & 69.31 & 47.13 & 73.01 & 64.26 \\\bottomrule
\end{tabular}
%\vspace{-0.2cm}
\caption{Fine-tuning Llama on Alpaca dataset for 5 epochs and evaluating on 9 tasks from EleutherAI LM Harness. "mem" represents the memory usage, with further details provided in Appendix~\ref{apdx:measure}. \#param is the number of trainable parameters, where the difference of \#param between the two approaches depends on the architecture of Llama, as some layers have $d_{in} \neq d_{out}$. Note that 10 million trainable parameters only account for less than 0.15GB of memory requirement. FA indicates that we freeze attention layers, but not including MLP layers followed by attention blocks. HS, OBQA, and WG represent HellaSwag, OpenBookQA, and WinoGrande datasets. More details of datasets can be found in Appendix~\ref{apdx:data}. The ablation study for different $r$ and the comparison with other LoRA variants can be found in Appendix~\ref{apdx:ablation}. All reported results are accuracies on the corresponding tasks. \textbf{Bold} indicates the best results of two approaches on the same task.} \label{tab:llm} 
\end{center}
\end{table*}
\fi

\begin{table*}[htbp]
\tiny
\begin{center}
\begin{tabular}{lccccccccccccc}\toprule
Model, ft setting & mem & \#param & ARC-c & ARC-e & BoolQ & HS & OBQA & PIQA & rte & SIQA & WG & Avg
\\\cmidrule(lr){1-13}
Llama2(7B)\\ \cmidrule(lr){1-1} 
LoRA, $r=64$ & 23.46GB & 159.9M(2.37\%) & \textbf{44.97} & 77.02 & 77.43 & 57.75 & 32.0 & \textbf{78.45} & 62.09 & 47.75 & 68.75 & 60.69\\
VeRA, $r=64$ & 22.97GB & 1.374M(0.02\%) & 43.26 & 76.43 & 77.40 & 57.26 & 31.6 & 78.02 & 62.09 & 45.85 & 68.75 & 60.07\\
DoRA, $r=64$ & 44.85GB & 161.3M(2.39\%) & 44.71 & 77.02 & 77.55 & \textbf{57.79} & 32.4 & 78.29 & 61.73 & \textbf{47.90} & 68.98 & 60.71\\
RoSA, $r=32, d=1.2\%$ & 44.69GB & 157.7M(2.34\%) & 43.86 & \textbf{77.48} & \textbf{77.86} & 57.42 & 32.2 & 77.97 & 63.90 &  47.29 & 69.06 & 60.78\\
SPruFT, $r=128$ & \textbf{17.62GB} & 145.8M(2.16\%) & 43.60 & 77.26 & 77.77 & 57.47 & \textbf{32.6} & 78.07 & \textbf{64.98} & 46.67 & \textbf{69.30} & \textbf{60.86} %\\\cmidrule(lr){2-13}
%FA-LoRA, $r=64$ & 17.25GB & 92.8M(1.38\%) & 43.77 & \textbf{77.57} & 77.74 & \textbf{57.45} & 31.0 & 77.86 & 66.06 & \textbf{47.13} & 69.06 & 60.85\\
%FA-DoRA, $r=64$ & 30.61GB & 93.6M(1.39\%) & 43.94 & 77.44 & 77.49 & 57.44 & 31.0 & 77.86 & \textbf{66.43} & 46.98 & 69.14 & 60.86\\
%FA-RoSA, $r=32, d=1.2\%$ & 38.34GB & 98.3M(1.46\%) & \textbf{44.28} & 77.02 & 77.68 & 57.22 & 31.0 & 77.97 & 64.26 & 46.32 & 69.22 & 60.55\\
%FA-SPruFT, $r=128$ & \textbf{15.21GB} & 78.6M(1.17\%) & 43.94 & 77.22 & \textbf{77.83} & 57.11 & \textbf{32.0} & \textbf{78.18} & 65.70 & 46.47 & \textbf{69.38} & \textbf{60.87}
\\\midrule
Llama3(8B)\\ \cmidrule(lr){1-1} 
LoRA, $r=64$ & 30.37GB & 167.8M(2.09\%) & 53.07 & 81.40 & 82.32 & 60.67 & 34.2 & 79.98 & 69.68 & 48.52 & 73.56 & 64.82\\
VeRA, $r=64$ & 29.49GB & 1.391M(0.02\%) & 50.26 & 80.30 & 81.41 & 60.16 & 34.4 & 79.60 & 69.31 & 46.93 & 72.77 & 63.90\\
DoRA, $r=64$ & 51.45GB & 169.1M(2.11\%) & \textbf{53.33} & \textbf{81.57} & \textbf{82.45} & \textbf{60.71} & 34.2 & \textbf{80.09} & 69.31 & \textbf{48.67} & \textbf{73.64} & \textbf{64.88}\\
RoSA, $r=32, d=1.2\%$ & 48.40GB & 167.6M(2.09\%) & 51.28 & 81.27 & 81.80 & 60.18 & 34.4 & 79.87 & 69.31 & 47.95 & 73.16 & 64.36\\
SPruFT, $r=128$ & \textbf{24.49GB} & 159.4M(1.98\%) & 52.47 & 81.10 & 81.28 & 60.29 & \textbf{34.6} & 79.76 & \textbf{70.04} & 47.75 & 73.24 & 64.50 %\\\cmidrule(lr){2-13}
%FA-LoRA, $r=64$ & 24.55GB & 113.2M(1.41\%) & 52.47 & 81.36 & 82.23 & 60.17 & \textbf{35.0} & 79.76 & 70.04 & 48.31 & \textbf{73.56} & 64.77\\
%FA-DoRA, $r=64$ & 40.62GB & 114.3M(1.42\%) & \textbf{52.56} & \textbf{81.69} & \textbf{82.26} & \textbf{60.20} & 34.4 & \textbf{79.82} & \textbf{70.40} & \textbf{48.46} & 73.40 & \textbf{64.80}\\
%FA-RoSA, $r=32, d=1.2\%$ & 42.31GB & 124.3M(1.55\%) & 52.22 & 81.19 & 82.05 & 60.11 & 34.4 & 79.76 & 69.31 & 47.70 & 73.16 & 64.43\\
%FA-SPruFT, $r=128$ & \textbf{22.41GB} & 92.3M(1.15\%) & 52.22 & 81.19 & 81.35 & \textbf{60.20} & 34.2 & 79.71 & 69.31 & 47.13 & 73.01 & 64.26 
\\\bottomrule
\end{tabular}
%\vspace{-0.2cm}
\caption{Fine-tuning Llama on Alpaca dataset for 5 epochs and evaluating on 9 tasks from EleutherAI LM Harness. ``mem" represents the memory usage, with further details provided in Appendix~\ref{apdx:measure}. \#param is the number of trainable parameters, where the difference of \#param between the two approaches depends on the architecture of Llama, as some layers have $d_{in} \neq d_{out}$. %FA indicates that we freeze attention layers, but not including MLP layers followed by attention blocks. 
HS, OBQA, and WG represent HellaSwag, OpenBookQA, and WinoGrande datasets. %More details of datasets can be found in Appendix~\ref{apdx:data}. 
The ablation study for different $r$ can be found in Appendix~\ref{apdx:ranks}. All reported results are accuracies on the corresponding tasks. \textbf{Bold} indicates the best result on the same task. } \label{tab:llm} 
\end{center}
\end{table*}

\section{Experimental Setup}\label{sec:setup}

%(0.5 page)
%Why the chosen framework?
%Some prior approaches

%- parameter settings
%- uniform across layers vs greedy ... 
%- potential transformer-specific details

%Equations about what these methods do.. 

%(0.5 page)
%Which NN architectures are used, why?
%Number of parameters, layers, ...

%(Potential prior work on compression -- )

\subsection{Datasets} \label{subsec:dataset}
We use multiple datasets for different tasks. For image classification, we fine-tune models on the training split and evaluate it on the validation split of Tiny-ImageNet~\citep{tavanaei2020embedded}, CIFAR100~\citep{alex2009learning}, and Caltech101~\citep{li_andreeto_ranzato_perona_2022}. For text generation, we fine-tune LLMs on 256 samples from Stanford-Alpaca~\citep{alpaca} and assess zero-shot performance on nine EleutherAI LM Harness tasks~\citep{gao2021framework}. See Appendix~\ref{apdx:data} for details.

\subsection{Models and Baselines} \label{subsec:models}

We fine-tune full-precision Llama-2-7B and Llama-3-8B (float32) using our SPruFT, LoRA~\citep{hulora}, VeRA~\citep{kopiczko2024vera}, DoRA~\citep{liu2024dora}, and RoSA~\citep{nikdan2024rosa}. RoSA is chosen as the representative SFT method and is the only SFT due to the high memory demands of other SFT approaches, while full fine-tuning is excluded for the same reason. We freeze Llama’s classification layers and fine-tune only the linear layers in attention and MLP blocks.

Next, we evaluate importance metrics by fine-tuning Llamas and image models, including DeiT~\citep{touvron2021training}, ViT~\citep{dosovitskiy2020image}, ResNet101~\citep{he2016deep}, and ResNeXt101~\citep{xie2017aggregated} on CIFAR100, Caltech101, and Tiny-ImageNet. For image tasks, we set the fine-tuning ratio at 5\%, meaning the trainable parameters are a total of 5\% of the backbone plus classification layers.

\subsection{Training Details} \label{subsec:training}
Our fine-tuning framework is built on torch-pruning\footnote{Torch-pruning is not required, all their implementations are based on PyTorch.}~\citep{fang2023depgraph}, PyTorch~\citep{paszke2019pytorch}, PyTorch-Image-Models~\citep{rw2019timm}, and HuggingFace Transformers~\citep{wolf2020transformers}. Most experiments run on a single A100-80GB GPU, while DoRA and RoSA use an H100-96GB GPU. We use the Adam optimizer~\citep{KingBa15} and fine-tune all models for a fixed number of epochs without validation-based model selection.

%Structured pruning often considers parameter dependencies in importance evaluation~\citep{liu2021group, fang2023depgraph, ma2023llmpruner}. This becomes the following process in our work: first, searching for dependencies by tracing the computation graph of gradient; next, evaluating the importance of parameter groups; and finally, fine-tuning the parameters within those important groups collectively. For instance, if $\W^{a}_{\cdot j}$ and $\W^{b}_{i\cdot}$ are dependent, where $\W^{a}_{\cdot j}$ is the $j$-th column in parameter matrix (or the $j$-th input channels/features) of layer $a$ and $\W^{b}_{i\cdot}$ is the $i$-th row in parameter matrix (or the $i$-th output channels/features) of layer $b$, then $\W^{a}_{\cdot j}$ and $\W^{b}_{i\cdot}$ will be fine-tuned simultaneously while the corresponding $\M^{a}_{dep}$ for $\W^{a}_{\cdot j}$ becomes column selection matrix and $\W^a_s$ becomes $\W^a_{f,dep}\M^a_{dep}$. Consequently, fine-tuning $2.5\%$ output channels for layer $b$ will result in fine-tuning additional $2.5\%$ input channels in each dependent layer. Therefore, for the $5\%$ of desired fine-tuning ratio, the fine-tuning ratio with considering dependencies is set to $2.5\%$\footnote{In some complex models, considering dependencies results in slightly more than twice the number of trainable parameters. However, in most cases, the factor is 2.} for the approach that includes dependencies. More details for dependencies of NN can be found in Appendix~\ref{apdx:dep}. 

\textbf{Image models}: The learning rate is set to $10^{-4}$ with cosine annealing decay~\citep{loshchilov2017sgdr}, where the minimum learning rate is $10^{-9}$. All image models used in this study are pre-trained on ImageNet. 

\textbf{Llama}: For LoRA and DoRA, we set $\alpha = 16$, a dropout rate of $0.1$, and a learning rate of $10^{-4}$  with linear decay (
$0.01$ decay rate). For SPruFT, we control trainable parameters using rank instead of fine-tuning ratio for direct comparison. The learning rate is $2 \cdot 10^{-5}$ with the same decay settings. Linear decay is applied after a warmup over the first $3$\% of training steps. The maximum sequence length is $2048$, with truncation for longer inputs and padding for shorter ones.



\subsection{Random Discretization}
\label{sec:mainargs}
We now set out to prove Lemma~\ref{lem:subgoal}. So let $0 < \delta < 1$, and fix a pair $(\Gamma_i, L_j)$. Following the proof outline in Section~\ref{sec:overview}, we now consider the following random discretization of hypotheses in $\subH(\Gamma_i, L_j)$: Let $k= k(i,j)$ be an integer parameter to be determined. Sample a random $k \times d$ matrix $\rA$ with each entry $\Norm(0,1/k)$ distributed as well as $k$ random offsets $\rt = (\rt_1,\dots,\rt_k)$ all independent and uniformly distributed in $[0,1]$.

Let $\Disc$ be the set of all vectors in $\R^k$ with coordinates in 
\[
\{(1/2)(10 \sqrt{k})^{-1} + z (10 \sqrt{k})^{-1}  \mid z \in \Ints\}.
\]
For $w \in \finalH$ and an outcome $(A,t)$ of $(\rA,\rt)$, define $h_{A,t}(w) \in \Disc$ as the vector obtained as follows: Consider each coordinate $(Aw)_i$ and let $z_i$ denote the integer such that 
\[
(1/2)(10 \sqrt{k})^{-1}  + z_i (10 \sqrt{k})^{-1} \leq (Aw)_i < (1/2)(10 \sqrt{k})^{-1}  + (z_i+1) (10\sqrt{k})^{-1}.
\]
Let $(h_{A,t}(w))_i$ equal $(1/2)(10 \sqrt{k})^{-1} + z_i (10\sqrt{k})^{-1}$ if $t_i \leq p(z_i)$ ($(Aw)_i$ rounded down) and otherwise let it equal $(1/2)(10 \sqrt{k})^{-1}  + (z_i + 1)(10\sqrt{k})^{-1}$. We choose $p(z_i) \in [0,1]$ such that 
\begin{align}
(Aw)_i &= 
p(z_i)\left(\frac{1}{2 \cdot 10 \sqrt{k}} +\frac{z_i}{10\sqrt{k}}\right) + (1-p(z_i))\left(\frac{1}{2 \cdot 10 \sqrt{k}} +\frac{z_i+1}{10\sqrt{k}}\right) \label{eq:expectround}
\end{align}
i.e.\ for fixed $A$, the expected value of the coordinates satisfy $\E_{\rt}[(h_{A,\rt}(w))_i]=(Aw)_i$. 
\begin{remark}
\label{rmk:pIsProb}
The value $p(z_i)$ satisfying~\eqref{eq:expectround} has $p(z_i) \in [0,1]$.
\end{remark}
We thus have that $p(z_i)$ is a well-defined probability. We prove Remark~\ref{rmk:pIsProb} in Appendix~\ref{sec:aux}. The random discretization has the desirable property that it approximately preserves margins/inner products as stated in the following
\begin{lemma}
\label{lem:concdiscretize}
    There is a constant $c>0$, such that for any integer $k \geq 1$, $w \in \finalH, x \in \finalX$ and any $\gamma \in (0,1]$, it holds that 
    $
    \Pr_{\rA,\rt}[|\langle h_{\rA,\rt}(w),\rA x\rangle - \langle w, x\rangle| > \gamma] < c\exp(-\gamma^2 k/c)
    $.
\end{lemma}
The proof of Lemma~\ref{lem:concdiscretize} follows the work by~\cite{AK17} in their work on lower bounds for the Johnson-Lindenstrauss transform, and has thus been deferred to Appendix~\ref{sec:aux}. We now observe that
\begin{align*}
    \Loss_{\cD}(w) &= \Loss^{\gamma_i/2}_{\rA \cD}(h_{\rA,\rt}(w)) + \Pr_{(\rx,\ry) \sim \cD}[y \langle h_{\rA,\rt}(w), \rA \rx\rangle > \gamma_i/2 \wedge \ry \langle w, \rx\rangle \leq 0] \\
    &- \Pr_{(\rx,\ry) \sim \cD}[\ry \langle h_{\rA,\rt}(w), \rA \rx\rangle \leq \gamma_i/2 \wedge \ry \langle w, \rx\rangle > 0].
\end{align*}
Similarly, we have for $\gamma \in \Gamma_i$ and any training set $S$ that
\begin{align*}
    \Loss_{S}^\gamma(w) &= \Loss^{\gamma_i/2}_{\rA S}(h_{\rA,\rt}(w)) + \Pr_{(\rx,\ry) \sim S}[\ry \langle h_{\rA,\rt}(w), \rA \rx\rangle > \gamma_i/2 \wedge \ry \langle w, \rx\rangle \leq \gamma] \\
    &- \Pr_{(\rx,\ry) \sim S}[\ry \langle h_{\rA,\rt}(w), \rA\rx\rangle \leq \gamma_i/2 \wedge \ry \langle w, \rx\rangle > \gamma].
\end{align*}
We now have for any $\gamma \in \Gamma_i$ that
\begin{align}
\sup_{w \in \subH(\Gamma_i,L_j)} \Loss_\cD(w) - \Loss^\gamma_S(w) =& \nonumber \\ 
\sup_{w \in \subH(\Gamma_i,L_j)} \big(\E_{\rA,\rt} [\Loss^{\gamma_i/2}_{\rA \cD}(h_{\rA,\rt}(w)) - \Loss^{\gamma_i/2}_{\rA S}(h_{\rA,\rt}(w))] +& \nonumber\\ \E_{\rA,\rt} [\Pr_{\cD}[\ry \langle h_{\rA,\rt}(w), \rA \rx\rangle > \gamma_i/2 \wedge \ry \langle w, \rx\rangle \leq 0] -\Pr_{S}[\ry \langle h_{\rA,\rt}(w), \rA \rx\rangle > \gamma_i/2 \wedge \ry \langle w, \rx\rangle \leq \gamma] ] +& \nonumber\\
\E_{\rA,\rt} [\Pr_{S}[\ry \langle h_{\rA,\rt}(w), \rA \rx\rangle \leq \gamma_i/2 \wedge \ry \langle w, \rx\rangle > \gamma] -\Pr_{\cD}[\ry \langle h_{\rA,\rt}(w), \rA \rx\rangle \leq \gamma_i/2 \wedge \ry \langle w, \rx\rangle > 0] ]\big).\label{eq:3terms}
\end{align}
A critical observation is that the distribution of $y \langle h_{\rA,\rt}(w),\rA x \rangle$ depends only on $y \langle w, x \rangle$. 
\begin{claim}
\label{clm:distDeterm}
For any $(x,y) \in \finalX \times \{-1,1\}$ and any $w \in \finalH$, the distribution of $y \langle h_{\rA,\rt}(w),\rA x \rangle$ is completely determined from $y \langle w, x \rangle$.
\end{claim}
We prove Claim~\ref{clm:distDeterm} in Section~\ref{sec:lip} by exploiting that the entries of $\rA$ are i.i.d.\ $\Norm(0,1/k)$ distributed and using the rotational invariance of the Gaussian distribution.

As outlined in the proof overview in Section~\ref{sec:overview}, we can now use Claim~\ref{clm:distDeterm} together with the contraction inequality of Rademacher complexity to bound several of the terms in~\eqref{eq:3terms}. Similarly to the introduction of the ramp loss in classic proofs of generalization for large-margin halfspaces, we need to introduce a continuous function upper bounding the probabilities above. With this in mind, we now define the following functions $\phi$ and $\rho$:
\[
\phi(\alpha) = \begin{cases} \Pr_{\rA, \rt}[y \langle h_{\rA,\rt}(w), \rA x\rangle > \gamma_i/2 \mid y \langle w, x \rangle = \alpha] & \text{if } -c_\gamma \leq \alpha \leq 0 \\
                      \frac{(\gamma_i-\alpha)}{\gamma_i}\Pr_{\rA, \rt}[y \langle h_{\rA,\rt}(w), \rA x\rangle > \gamma_i/2 \mid y \langle w, x \rangle = 0]                                    & \text{if } 0 < \alpha \leq \gamma_i      \\
                      0 & \text{if } \gamma_i < \alpha \leq c_\gamma
        \end{cases}
\]
\[
\rho(\alpha) = \begin{cases} \Pr_{\rA, \rt}[y \langle h_{\rA,\rt}(w), \rA x\rangle \leq \gamma_i/2 \mid y \langle w, x \rangle = \alpha] & \text{if } \gamma_i < \alpha \leq c_\gamma\\
                      \frac{\alpha}{\gamma_i}\Pr_{\rA, \rt}[y \langle h_{\rA,\rt}(w), \rA x\rangle \leq \gamma_i/2 \mid y \langle w, x \rangle = \gamma_i]                                    & \text{if } 0 < \alpha \leq \gamma_i      \\
                      0 & \text{if } -c_\gamma \leq \alpha \leq 0
        \end{cases}
\]
Here we slightly abuse notation and write $\Pr_{\rA, \rt}[y \langle h_{\rA,\rt}(w), \rA x\rangle > \gamma_i/2 \mid y \langle w, x \rangle = \alpha]$ to denote the probability $\Pr_{\rA, \rt}[y \langle h_{\rA,\rt}(w), \rA x\rangle > \gamma_i/2]$ for an arbitrary $w \in \finalH, (x,y) \in \finalX \times \{-1,1\}$ with $y \ipr{w,x}=\alpha$ and remark that this probability is the same for all such $w,x,y$ by Claim~\ref{clm:distDeterm}.

We now observe that $\phi$ and $\rho$ upper and lower bounds the terms in~\eqref{eq:3terms}
\begin{remark}
\label{rmk:phirho}
For any training set $S$ and distribution $\cD$ over $\finalX \times \{-1,1\}$, we have
\begin{align*}
    \E_{\rA,\rt} [\Pr_{(\rx,\ry) \sim \cD}[\ry \langle h_{\rA,\rt}(w), \rA \rx\rangle > \gamma_i/2 \wedge \ry \langle w, \rx\rangle \leq 0]] &\leq\E_{(\rx,\ry) \sim \cD}[\phi(\ry \langle w, \rx \rangle)] \\
    \E_{\rA,\rt} [\Pr_{(\rx,\ry) \sim S}[\ry \langle h_{\rA,\rt}(w), \rA \rx\rangle > \gamma_i/2 \wedge \ry \langle w, \rx\rangle \leq \gamma]] &\geq \E_{(\rx,\ry) \sim S}[\phi(\ry \langle w, \rx \rangle)] \\
    \E_{\rA,\rt} [\Pr_{(\rx,\ry) \sim S}[\ry \langle h_{\rA,\rt}(w), \rA \rx\rangle \leq \gamma_i/2 \wedge \ry \langle w, \rx\rangle > \gamma]] &\leq \E_{(\rx, \ry) \sim S}[\rho(\ry \langle w, \rx \rangle)]\\
    \E_{\rA,\rt} [\Pr_{(\rx,\ry) \sim \cD}[\ry \langle h_{\rA,\rt}(w), \rA \rx\rangle \leq \gamma_i/2 \wedge \ry \langle w, \rx\rangle > 0]] &\geq \E_{(\rx, \ry) \sim \cD}[\rho(\ry\langle w, \rx \rangle)].
\end{align*}
\end{remark}
The proof of Remark~\ref{rmk:phirho} follows from the definition of $\phi$ and $\rho$, along with monotonicity of $\Pr_{\rA,\rt}[y \ipr{h_{\rA,\rt}(w),\rA x} > \gamma_i \mid y\ipr{w,x}=\alpha]$ as a function of $\alpha$. The proofs have been deferred to Appendix~\ref{sec:aux}.
Continuing from~\eqref{eq:3terms} using Remark~\ref{rmk:phirho}, linearity of expectation and the triangle inequality, we have for any $\gamma \in \Gamma_i$ that
\begin{align}
\sup_{w \in \subH(\Gamma_i, L_j)} \Loss_\cD(w) - \Loss^\gamma_S(w) \leq&
\sup_{w \in \subH(\Gamma_i, L_j)} \left|\E_{\rA,\rt} [\Loss^{\gamma_i/2}_\cD(h_{\rA,\rt}(w)) - \Loss^{\gamma_i/2}_S(h_{\rA,\rt}(w))]\right| \label{eq:middle}\\
+&\sup_{w \in \subH(\Gamma_i, L_j)} \left| \E_{(\rx,\ry) \sim \cD}[\phi(\ry \langle w, \rx \rangle)] -\E_{(\rx,\ry) \sim S}[\phi(\ry \langle w, \rx \rangle)]\right| \label{eq:phi}\\
+&\sup_{w \in \subH(\Gamma_i, L_j)} \left|\E_{(\rx,\ry) \sim \cD}[\rho(\ry \langle w, \rx \rangle)] -\E_{(\rx,\ry) \sim S}[\rho(\ry \langle w, \rx \rangle)] \right|.\label{eq:rho}
\end{align}
In Section~\ref{sec:meetinmid}, we carefully use Bernstein's plus a (highly non-trivial) union bound over infinitely many grids of increasing size to bound~\eqref{eq:middle} as follows
\begin{lemma}
\label{lem:supW}
There is a constant $c>0$ such that with probability at least $1-\delta$ over $\rS \sim \cD^n$ we have
\begin{align*}
    \eqref{eq:middle} &\leq c \left(\sqrt{\frac{(\ell_{j+1}+ \exp(-\gamma_{i+1}^2k/c)) (k + \ln(e/\delta))}{n}} + \frac{(k + \ln(e/\delta))}{n} \right).
\end{align*}
\end{lemma}
In Section~\ref{sec:rademacher}, we then use Rademacher complexity and a bound on the Lipschitz constants of $\phi$ and $\rho$ to bound~\eqref{eq:phi} and~\eqref{eq:rho} as follows
\begin{lemma}
\label{lem:supPhi}
There are constants $c,c'>0$ such that with probability at least $1-\delta$ over $\rS \sim \cD^n$ we have
\begin{align*}
\max\{\eqref{eq:phi}, \eqref{eq:rho} \}\leq
c\exp(-\gamma_{i+1}^2k/c) \cdot \sqrt{(k + \gamma_{i+1}^{-2} + \ln(e/\delta))/n} .
\end{align*}
provided that $k \geq c' \gamma_{i+1}^{-2}$. 
\end{lemma}
To balance the expressions in Lemma~\ref{lem:supW} and Lemma~\ref{lem:supPhi}, we now set $k = c \gamma_{i+1}^{-2} \ln(e/\ell_{j+1})$ for a sufficiently large constant $c>0$ so that $\exp(-\gamma_{i+1}^2 k/c) \leq \ell_{j+1}/e$ and $k \geq c'\gamma_{i+1}^{-2}$. Combining Lemma~\ref{lem:supW} and Lemma~\ref{lem:supPhi} via a union bound with $\delta'=\delta/2$ and inserting into~\eqref{eq:middle},~\eqref{eq:phi} and~\eqref{eq:rho} gives
\begin{align*}
\sup_{w \in \subH(\Gamma_i, L_j)} \Loss_\cD(w) - \Loss^\gamma_S(w) \leq& \\c\left(\sqrt{\frac{\ell_{j+1}(\gamma_{i+1}^{-2}\ln(e/\ell_{j+1}) + \ln(e/\delta))}{n}}  + \frac{\gamma_{i+1}^{-2} \ln(e/\ell_{j+1})+\ln(e/\delta)}{n}\right) &+\\
c \left( \ell_{j+1} \sqrt{(\gamma_{i+1}^{-2} \ln(e/\ell_{j+1}) + \ln(e/\delta))/n} \right),
\end{align*}
for a constant $c>0$. This completes the proof of Lemma~\ref{lem:subgoal}, which together with Lemma~\ref{lem:subgoal2} completes the proof of our main result, Theorem~\ref{thm:main}.





\section{Rademacher Bounds}
\label{sec:rademacher}
In this section, we use Rademacher complexity and the contraction inequality to prove Lemma~\ref{lem:supPhi}. We focus on bounding~\eqref{eq:phi} and note that~\eqref{eq:rho} is handled symmetrically.

%\begin{customlem}{\ref{lem:supPhi}}
%There are constants $c,c'>0$ such that with probability at %least $1-\delta$ over $\rS \sim \cD^n$ we have
%\begin{align*}
%\sup_{w \in \subH(\Gamma_i,L_j)} \left| \E_{(\rx,\ry) \sim \cD}%[\phi(\ry \langle w, \rx \rangle)] -\E_{(\rx,\ry) \sim \rS}%[\phi(\ry \langle w, \rx \rangle)]\right| &\leq\\
%c\exp(-\gamma_{i+1}^2k/c) \cdot \sqrt{\frac{k + %\gamma_{i+1}^{-2} + \ln(e/\delta)}{n}},
%\end{align*}
%provided that $k \geq c' \gamma_{i+1}^2$. The same bound holds %with $\rho$ in place of $\phi$.
%\end{customlem}
For a training set $S \in (\finalX \times \{-1,1\})^n$, consider the empirical Rademacher complexity (for $\rsigma=(\rsigma_1,\dots,\rsigma_{n})$ a vector of independent and uniform variables in $\{-1,1\}$):
\begin{align*}
\hat{\Rad}_{\phi, \subH(\Gamma_i,L_j)}(S) &= \frac{1}{n} \cdot \E_{\rsigma}\left[ \sup_{w \in \subH(\Gamma_i,L_j)} \sum_{(x_i,y_i) \in S} \rsigma_i \phi(y_i \langle w, x_i \rangle )\right] \\
&\leq \frac{1}{n} \cdot \E_{\rsigma}\left[ \sup_{w \in \finalH} \sum_{(x_i,y_i) \in S} \rsigma_i \phi(y_i \langle w, x_i \rangle )\right]
\end{align*}
If $\phi$ is $L_\phi$-Lipschitz, then the contraction inequality from~\cite{ledoux1991probability} gives that
\[
\hat{\Rad}_{\phi, \finalH}(S) \leq \frac{L_\phi}{n} \cdot \E_{\rsigma}\left[ \sup_{w \in \finalH} \sum_{(x_i,y_i) \in S} \rsigma_i y_i \langle w, x_i \rangle  \right].
\]
Using Cauchy-Schwartz, this is bounded by
\begin{align*}
\hat{\Rad}_{\phi, \finalH}(S) \leq& \frac{L_\phi}{n} \cdot  \E_{\rsigma}\left[ \sup_{w \in \finalH}  \left\langle w, \sum_{(x_i,y_i) \in S}\rsigma_i y_i x_i \right\rangle  \right] \\
\leq& \frac{L_\phi}{n} \cdot \left(\sup_{w \in \finalH} \|w\|_2\right) \cdot \E_{\rsigma}\left[\left\| \sum_{(x_i,y_i) \in S}\rsigma_i y_i x_i\right\|_2 \right] \\
\leq& \frac{L_\phi}{n} \cdot \sqrt{\E_{\rsigma}\left[\left\| \sum_{(x_i,y_i) \in S}\rsigma_i y_i x_i\right\|_2^2 \right]}\\
=& \frac{L_\phi}{\sqrt{n}} \cdot \sqrt{\sum_{(x_i,y_i) \in S} \sum_{(x_j,y_j) \in S} \E_\sigma[\sigma_i \sigma_j] y_i y_j \ipr{x_i, x_j}} \\
=& \frac{L_\phi}{\sqrt{n}}.
\end{align*}
Since this inequality holds for all $S$ with each $(x,y) \in S$ satisfying $\|x\|_2 = 1$, we have for the distribution $\cD$ that the Rademacher complexity
\[
\Rad_{\cD,\phi,\finalH}(n) = \E_{\rS \sim \cD^n}[\hat{\Rad}_{\phi,\finalH}(\rS)]
\]
satisfies $\Rad_{\cD,\phi,\finalH}(n) \leq L_\phi/\sqrt{n}$. By Lemma~\ref{lem:concdiscretize} and $\gamma_i = \gamma_{i+1}/2$, we have that $\phi$ is bounded by 
\[
0 \leq \phi(\alpha) \leq \max_{-c_\gamma \leq \alpha \leq 0}\Pr_{\rA, \rt}[y \langle h_{\rA,\rt}(w), \rA x\rangle > \gamma_i/2 \mid y \langle w, x \rangle = \alpha] \leq c\exp(-k\gamma_{i+1}^2/c),
\]
for a constant $c>0$. We conclude from standard results on Rademacher complexity (see e.g.~\cite{rademacherbound}), that with probability $1-\delta$ over a sample $\rS \sim \cD^n$ it holds that
\begin{align*}
\sup_{w \in \subH(\Gamma_i,L_j)} \left| \E_{(\rx,\ry) \sim \cD}[\phi(\ry \langle w, \rx \rangle)] -\E_{(\rx,\ry) \sim \rS}[\phi(\ry \langle w, \rx \rangle)]\right| \leq& \\
2 \Rad_{\cD,\phi,\finalH}(n) + c_R\left(c\exp(-k\gamma_{i+1}^2/c)\sqrt{\frac{\ln(1/\delta)}{n}} \right) \leq&\\
\frac{2 L_\phi}{\sqrt{n}} + c_R\left(c\exp(-k\gamma_{i+1}^2/c)\sqrt{\frac{\ln(1/\delta)}{n}} \right).
\end{align*}
where $c_R>0$ is a constant. Symmetric arguments bounds $\rho$ by the same, with the Lipschitz constant $L_\rho$ of $\rho$ in place of $L_\phi$.

We now use the following bound on the Lipschitz constants of $\phi$ and $\rho$
\begin{lemma}
\label{lem:finalLip}
There are constants $c_L, c>0$ such that the Lipschitz constants $L_\phi$ and $L_\rho$ of $\phi$ and $\rho$ are bounded by
\[
c_L \exp(-\gamma_{i+1}^2 k/c_L)\cdot \left(\sqrt{k} + \gamma^{-1}_{i+1}\right),
\]
when $k \geq c \gamma_{i+1}^{-2}$.
\end{lemma}
We prove this lemma in the next section. We thus conclude that with probability at least $1-\delta$ over $\rS \sim \cD^n$, we have
\begin{align*}
\sup_{w \in \subH(\Gamma_i,L_j)} \left| \E_{(\rx,\ry) \sim \cD}[\phi(\ry \langle w, \rx \rangle)] -\E_{(\rx,\ry) \sim \rS}[\phi(\ry \langle w, \rx \rangle)]\right| \leq& \\
2 \cdot \frac{c_L \exp(-\gamma_{i+1}^2k/c_L)\left(\sqrt{k} +\gamma^{-1}_{i+1}\right)}{\sqrt{n}} + c_R \left(c \exp(-k \gamma_{i+1}^2/c) \sqrt{\frac{\ln(1/\delta)}{n}} \right).
\end{align*}
The same bound holds for $\rho$ via Lemma~\ref{lem:finalLip}, which completes the proof of Lemma~\ref{lem:supPhi}.

The method outlined in the previous section utilizes a routine and tailored data to regularize models during training, aligning them with human attention. However, collecting such data can be laborious and sometimes impossible. In this section, we pivot to an alternative approach that shifts \textbf{from regularizing to constraining the model}. We employ 1-Lipschitz networks, trained with a transport loss. In \autoref{sec:attributions:mege}, we used our metric of algorithmic stability to demonstrate that 1-Lipschitz models provide more general explanations. Here, we will illustrate that the gradient of these models has a compelling interpretation: it points towards the counterfactual, meaning the closest real point belonging to a different class (see Figure~\ref{fig:lipschitz:big_picture}). We will show that these models, originally designed for robustness against adversarial attacks, are \textit{also} naturally aligned.

\begin{figure*}[ht]
  \centering
  \includegraphics[width=0.99\linewidth]{assets/lipschitz/big_picturev_2.jpg}
\caption{
\textbf{Illustration of the beneficial properties of $\losshkr$ gradients.} Examples \textbf{a)} and \textbf{b)} show that the gradients naturally provide a direction that enables the generation of adversarial images - a theoretical justification based on optimal transport is provided in the \autoref{sec:lipschitz:theory}. By applying the gradient $\rvx' = \rvx  -\alpha \nabla_{\rvx}  \f(\rvx)$ to the original image $\rvx$ (on the left), any digit from MNIST can be transformed into its counterfactual $\rvx'$ (e.g., turning a 0 into a 5). 
In \textbf{b)}, we illustrate that this approach can be applied to larger datasets, such as Celeb-A, by creating two counterfactual examples for the closed-mouth and blonde  classes. In \textbf{c)}, we compare the Saliency Map of a classical model with those of $\losshkr$ gradients, which are more focused on relevant elements. Finally, in \textbf{d)}, we show that following the gradients of $\losshkr$ could generate convincing feature visualizations that ease the understanding of the model's features. 
}
\label{fig:lipschitz:big_picture}
\end{figure*}

\subsection{Background}

Let us consider a probability space $(\Omega, \mathcal{F}, \P)$, where $\Omega$ represents the set of outcomes, $\mathcal{F}$ a $\sigma$-algebra of events, and $\P$ a probability measure. The space of all probability measures on a metric space $(\sx, \norm{\cdot})$ is denoted as $\s{P}(\sx)$. Here, $\sx \subseteq \Real^d$ signifies the input space, and $\sy = \{\pm 1\}$ the output space. The input data $\rvx : \Omega \rightarrow \sx$ and target label $\ry : \Omega \rightarrow \sy$ are modeled as random variables with distributions $\P_{\rvx}$ and $\P_{\ry}$, respectively, with $\P_{\rvx,\ry}$ representing their joint distribution over $\sx \times \sy$.

The Wasserstein distance, inspired by the theory of optimal transport~\cite{villani2009optimal}, measures the minimal cost required to transform one probability distribution into another. It roots back to the work of Gaspard Monge in the 18th century~\cite{monge1781memoire}, and was originally defined as:

\begin{equation}
    \wasserstein_1(\mu, \nu) = \inf_{\pi \in \Pi(\mu, \nu)} \int_{\sx \times \sx} \norm{\rvx - \rv{z}} \,d\pi(\rvx, \rv{z}),
\end{equation}

Where $\Pi(\mu, \nu)$ is the set of all couplings of $\mu$ and $\nu$, $\wasserstein_1$ denote the 1-Wasserstein distance, also known as the Earth-Mover's distance, between two probability measures $\mu$ and $\nu$ over $\sx$.
Moreover, it can be shown that dual representation of $\wasserstein_1$ is a special case of the duality theorem of Kantorovich and Rubinstein (\cite{kantorovich1960mathematical}) and is defined as:

\begin{align}\label{eq:lip:kantorovich}
    \wasserstein_1(\mu, \nu) &= \sup_{\f \in \lip_1(\sx)} \left( \int_{\sx} \f(\rvx) \,d\mu(\rvx) - \int_{\sx} \f(\rvx) \,d\nu(\rvx) \right) \\
    & = \sup_{\f \in \lip_1(\sx)} ~ ~ \underset{\rvx \sim \mu}{\E}(\f(\rvx)) ~ - ~ \underset{\rvx \sim \nu}{\E}(\f(\rvx)).
\end{align}

Where $\lip_1(\sx)$ denotes the space of 1-Lipschitz functions on $\sx$. For reference, a function is considered L-Lipschitz if for all pairs $(\rvx, \rv{z}) \in \sx^2$, the norm of the difference between $\rvx$ and $\rv{z}$ is less than or equal to $L$ times the norm of the difference between $\f(\rvx)$ and $\f(\rv{z})$ :

$$
\forall (\rvx, \rv{z}) \in \sx^2, \norm{\f(\rvx) - \f(\rv{z})} \leq L \norm{\rvx - \rv{z}}.
$$

This formulation in Equation~\ref{eq:lip:kantorovich} is particularly intriguing, as it renders the computation of the Wasserstein distance tractable if one can correctly parametrize to optimize over the space of 1-Lipschitz functions.
Recent works have proposed to use deep neural network to parametrize the function $\f(\cdot, \parameters)$ and have found various ways to constraint the function space such that $\f \in \lip_1$ at every step in the training process. We refer the reader to \cite{serrurier2022adversarial,hein_formal_2017,Sokolic_2017,tsipras2018robustness,salimans2016weight,miyato2018spectral} for more information.

\paragraph{\hkr: Robust Classification via Transport-Based Loss Function}

Building on these foundations, the \hkr~Loss introduced in~\cite{serrurier2021achieving}~incorporates a hinge regularization term to the Kantorovich-Rubinstein optimization objective, aiming to enhance binary classification performance. It is formulated as:
\begin{equation}
    \losshkr(\f) = 
    \underset{\rvx \sim \mu}{\E}(\f(\rvx)) ~ - ~ \underset{\rvx \sim \nu}{\E}(\f(\rvx)) 
    ~ +  \underset{(\rvx, \ry) \sim \P_{\rvx, \ry}}{\lambda ~ \E}\big(\margin - \ry \f(\rvx)\big)^+
\end{equation}
With $\margin > 0$, the margin introduces a significant contribution to the model's robustness and interpretability by promoting separation between the distributions of positive and negative classes. This loss has been thoroughly analyzed in~\cite{bethune2022pay}, providing insights into its interpretation, limitations, and advantages, especially in controlling the Lipschitz constant. Moreover, the HKR loss has been applied in computing SDF functions~\cite{bethune2023robust} and in DP-training~\cite{bethune2023dp}. In practice, the model is trained using the DeelLip\footnote{https://github.com/deel-ai/deel-lip} library (\cite{deelLip}).

\subsection{An optimal transport perspective of Saliency}
\label{sec:lipschitz:theory}

Models trained with the previously introduced \hkr~loss exhibit interesting properties from a transport perspective: the gradient points towards a point of the opposite class, a counterfactual. We will revisit these propositions and interpret the significance of this gradient, then explore how this translates into terms of alignment.

We note $\optiplan$ the optimal transport plan corresponding to the minimizer of the \hkr loss. In the most general setting, $\optiplan$ is a joint distribution over $\mu,\nu$ pairs. However, when $\mu$ and $\nu$ admit a density function~\cite{peyre2018computational} with respect to Lebesgue measure, then the joint density describes a deterministic mapping, i.e. a Monge map. Given $\rvx \sim \mu$ 
(resp. $\nu$) we note $\rv{z} = \transport(\rvx) \in \nu$ (resp. $\mu$) the image of $\rvx$ with respect to $\optiplan$. When $\optiplan$ is not deterministic (on real datasets that are defined as a discrete collection of Diracs), we take $\transport(\rvx)$ as the point of maximal mass with respect to $\optiplan$.

\begin{theorem}[Transportation plan direction~\cite{serrurier2024explainable}]\label{th:gradient_transport_plan}
Let $\f^\star$ an optimal solution minimizing the $\losshkr$. Given $\rvx \sim \mu$ (resp. $\nu$)  and  $\rv{z} = \transport(\rvx)$, then $\exists \alpha \geq 0$ (resp. $\alpha \leq 0$) such that $\transport(\rvx) = \rvx  -\alpha \nabla_{\rvx}  \f^\star(\rvx)$ almost surely. 
\end{theorem}

This proposition also holds for the Kantorovich-Rubinstein dual problem without hinge regularization, demonstrating that for $\rvx \sim \P_{\rvx,\ry}$, the gradient $\nabla_{\rvx} \f^{\star}(\rvx)$ indicates the direction in the transportation plan almost surely.

\begin{theorem}[Decision boundary~\cite{serrurier2024explainable}]\label{boundary_distance}
Let $\mu$ and $\nu$ two distributions with disjoint supports with minimal distance~$\xi$ and  $\f^\star$ an optimal solution minimizing the $\losshkr$~with~$\delta < 2\xi$. Given $\rvx \sim \P_{\rvx,\ry}$, $\rvx_\delta = \rvx -\f^\star(\rvx) \nabla_{\rvx} \f^\star(\rvx) \in \boundary $
where $\boundary = \left\{\rvx' \in \sx | \f^\star(\rvx') = 0 \right\}$  is the decision boundary (i.e. the 0 level set of $\f^\star$).
\end{theorem}

Experiments perform in \cite{serrurier2024explainable} suggest this probably remains true when the supports of $\mu$ and $\nu$  are not disjoint. 

\begin{corollary}[\cite{serrurier2024explainable}]\label{fx_grad_adversarial}
Let $\mu$ and $\nu$ two separable distributions with minimal distance $\xi$ and  $\f^\star$ an optimal solution minimizing the $\losshkr$ with $\delta <2\xi$, given $\rvx \sim \P_{\rvx, \ry}$, 
$adv(\f^\star,\rvx) = \rvx_{\delta}$ 
almost surely, where $\rvx_\delta = \rvx -\f^\star(\rvx) \nabla_{\rvx} \f^\star(\rvx)$.
\end{corollary}

\begin{figure*}[ht]
    \centering
    \includegraphics[width=.9\textwidth]{assets/lipschitz/koch_v4.jpg}
    \caption{Level sets of an 1-Lipschitz classifier train with $\losshkr$ for two concentric Koch snowflakes \textbf{(a)}.  The decision boundary (denoted $\boundary$, also called the 0-level set) is the red dashed line. Figure \textbf{(b)} (resp. \textbf{(c)}) represents the translation of the form $\rvx'= \rvx -\f(\rvx)\nabla_{\rvx} \f(\rvx)$ of each point $\rvx$ of the first class (resp second class). 
    $(\rvx,\rvx')$ pairs are represented by blue (resp. orange)  segments.}
    \label{fig:lipschitz:koch}
\end{figure*}

This corollary is of great interest as it shows that adversarial examples are precisely identified for the classifier based on $\losshkr$: the direction is given by the gradient $\nabla_{\rvx} \f^\star(\rvx)$ and the distance by $\norm{\f^\star(\rvx)}$. In this scenario, the optimal adversarial attacks align with the gradient direction.

To illustrate these propositions, we learned a dense binary classifier with $\losshkr$ to separate two complex distributions, following two concentric Koch snowflakes. Figure~\ref{fig:lipschitz:koch}) \textbf{(a)} shows the two distributions (blue and orange snowflakes), the learned boundary ($0$-level set) (red dashed line). In the same figure, \textbf{(b)} and \textbf{(c)} show, for random samples $\rvx$ from the two distributions, the segments $[\rvx,\rvx_\delta]$ where $\rvx_\delta$ is defined in Proposition.~\ref{boundary_distance}.

\paragraph{Alignment induced by $\losshkr$.}

Thus, the learning process of those models induces a strong constraint on the gradients of the neural network, aligning them to the optimal transport plan. We claim that is the reason why the simple Saliency Maps have very good properties for those networks.

By adopting the metric we have proposed in~\autoref{sec:harmonization}, we have computed the human feature alignment of $\losshkr$ Saliency Maps and compare with the others models tested in ~\cite{fel2022aligning}-- more than 100 recent deep neural networks. In Figure ~\ref{fig:lipschitz:human_alignement}, we demonstrate that those model's Saliency Maps, do not only carry strong theoretical interpretation as the direction of the transport plan, it is also more aligned with human attention than any other tested models and significantly surpasses the Pareto front discovered previously. Perhaps the most surprising is that no clickmap or any specific routine like the harmonization one was used: the OTNN model is even more aligned than a ResNet50 model trained with the specific alignment objective proposed in~\autoref{sec:harmonization}.
The implications of these results are crucial for both cognitive science and industrial applications. A model that more closely aligns with human attention and visual strategies can provide a more comprehensive understanding of how vision operates for humans, and also enhance the predictability, interpretability, and performance of object recognition models in industry settings. 
Furthermore, the drop in alignment observed in recent models highlights the necessity of considering the alignment of model visual strategies with human attention while developing object recognition models to reduce the reliance on spurious correlations and ensure that our models are accurate for the right reasons.


\begin{figure*}[ht]
\centering
\includegraphics[width=.99\textwidth]{assets/lipschitz/human_alignment_flat_2.png}
\caption{\textbf{$\losshkr$~ naturally align gradients with Human attention.} Our study shows that the Saliency Map of $\losshkr$ model (denoted OTNN, for Optimal Transport Neural Network) is highly aligned with human attention. The degree of alignment between human and DNN saliency is measured using the mean Spearman correlation, normalized by the average inter-rater alignment of humans. 
}
\label{fig:lipschitz:human_alignement}
\end{figure*}



\section{Meet in the Middle Bound}
\label{sec:meetinmid}
The goal of this section is to prove the following
\begin{customlem}{\ref{lem:supW}}
There is a constant $c>0$ such that with probability at least $1-\delta$ over $\rS \sim \cD^n$ we have
\begin{align*}
    \sup_{w \in \subH(\Gamma_i,L_j)} \left|\E_{\rA,\rt} [\Loss^{\gamma_i/2}_{\rA \cD}(h_{\rA,\rt}(w)) - \Loss^{\gamma_i/2}_{\rA \rS}(h_{\rA,\rt}(w))]\right| &\leq\\ c \left(\sqrt{\frac{(\ell_{j+1}+ \exp(-\gamma_{i+1}^2k/c)) (k + \ln(e/\delta))}{n}} + \frac{(k + \ln(e/\delta))}{n} \right).
\end{align*}
\end{customlem}
Notice here that the two losses $\Loss^{\gamma_i/2}_{\rA \cD}(h_{\rA,\rt}(w))$ and $\Loss^{\gamma_i/2}_{\rA \rS}(h_{\rA,\rt}(w))$ refer to the same margin $\gamma_i/2$ and $h_{\rA,\rt}(w)$ has been discretized to have all coordinates of the form $(1/2)(10 \sqrt{k})^{-1} + z (10 \sqrt{k})^{-1}$ for integer $z$. Intuitively, we will try to exploit this discretization to union bound over a grid of finitely many hypotheses. Unfortunately, the random matrix $\rA$ may increase the norm of $w$ arbitrarily much, and thus a single grid is insufficient. Instead, we need an infinite sequence of grids. For this, let $\Disc_0$ denote the set of all vectors in $4 \Ball_2^k$ whose coordinates are of the form $(1/2)(10 \sqrt{k})^{-1} + z(10 \sqrt{k})^{-1}$ for integer $z$. More generally, let $\Disc_i$ for $i > 0$ denote the set of all vectors in $(2^i \cdot 4 \Ball_2^k)$ whose coordinates are of this form. Since $\|x\|_1 \leq \sqrt{k} \|x\|_2$ for any $x \in \R^k$, we have that $\Disc_i \subset (2^i \cdot 4 \Ball_2^k) \subseteq \sqrt{k}(2^i \cdot 4 \Ball_1^k)$. For a vector $x \in \Disc_i$, let $i(x)=(i_1,\dots,i_k)$ denote the integers so that $x = (10 \sqrt{k})^{-1} i(x) + (1/2)(10 \sqrt{k})^{-1} \AllOne$ with $\AllOne \in \R^k$ the all-1's vector. Then by the triangle inequality, we have $(10 \sqrt{k})^{-1}\|i(x)\|_2 \leq \|x\|_2 + (1/2)(10 \sqrt{k})^{-1}\|\AllOne\|_2 \leq 2^i \cdot 4 + 1/20$. This implies $\|i(x)\|_1 \leq (10 \sqrt{k}) \sqrt{k} (2^i \cdot 4 + 1/20) \leq (5 \cdot 2^{i+3}+1)k$. Since each coordinate of $i(x)$ is an integer, there are thus at most $2^k$ choices for the signs and $\sum_{t=0}^{(5 \cdot 2^{i+3}+1)k} \binom{k + t -1}{t}$ choices for the absolute values of the integers. That is, we have
\begin{align}
|\Disc_i| \leq 2^k \cdot \sum_{t=0}^{(5 \cdot 2^{i+3}+1) k} \binom{k + t -1}{t} \leq 2^{(5 \cdot 2^{i+3}+3)k} \leq 2^{2^{i+7}k}.\label{eq:netsize}
\end{align}
We now start by considering a fixed outcome $A$ of the random matrix $\rA$. For such a fixed $A$, the training set $\rS$ behaves well in the sense that $\Loss^\gamma_{A \cD}(w)$ and $\Loss^\gamma_{A \rS}(w)$ are close with high probability for any $w$. This is formalized in the following remark
\begin{remark}
\label{rmk:concentrationx}
For any distribution $\cD$ over $\finalX \times \{-1,1\}$, fixed $w \in \finalH$, margin $\gamma$ and any $A \in \R^{k \times d}$, it holds with probability at least $1-\delta$ over $\rS \sim \cD^n$ that
\[
|\Loss_{A\cD}^{\gamma}(w) - \Loss_{A\rS}^{\gamma}(w)| \leq \sqrt{\frac{8\Loss_{A \cD}^{\gamma}(w)\ln(1/\delta)}{n}} + \frac{2 \ln(1/\delta)}{n}.
\]
\end{remark}
The proof of Remark~\ref{rmk:concentrationx} is a simple application of Bernstein's and can be found in Appendix~\ref{sec:aux}.

In Lemma~\ref{lem:supW}, the matrix $\rA$ is not fixed but random. Thus we need to find a formal property of the training set $\rS$ under which $\Loss^{\gamma_i/2}_{\rA \cD}(h_{\rA,\rt}(w))$ and $\Loss^{\gamma_i/2}_{\rA \rS}(h_{\rA,\rt}(w))$ are close in expectation over the random choice of $\rA$. With this goal in mind, we now say that a matrix $A$ in the support of $\rA$ and a training set $S$ has \emph{distortion} at least $\beta$, if there is a grid $\Disc_a$ and a vector $w \in \Disc_a$ such that
\[
|\Loss_{A\cD}^{\gamma_i/2}(w) - \Loss_{A\rS}^{\gamma_i/2}(w)| > \beta \cdot \left(\sqrt{\frac{8\Loss_{A \cD}^{\gamma_i/2}(w)(2^{a+7}k + \ln(1/\delta))}{n}} + \frac{2 (2^{a+7}k + \ln(1/\delta))}{n}\right).
\]
For a training set $S$, we use $D_\beta(S)$ to denote the set of matrices $A$ with distortion at least $\beta$ for $S$.

We observe that for a fixed matrix $A$, grid $\Disc_a$ and $\beta>1$, we have by Remark~\ref{rmk:concentrationx} with $\delta'_a = (\delta/2^{2^{a+7}k})^{\beta}$
and a union bound over all $w \in \Disc_a$, that with probability at least $1-|\Disc_a|\delta'_a$, it holds for all $w \in \Disc_a$ that
\begin{align*}
    |\Loss_{A\cD}^{\gamma_i/2}(w) - \Loss_{A\rS}^{\gamma_i/2}(w)| &\leq \sqrt{\frac{8\Loss_{A \cD}^{\gamma_i/2}(w)\ln(1/\delta'_a)}{n}} + \frac{2 \ln(1/\delta'_a)}{n} \\
    &= \sqrt{\frac{8\Loss_{A \cD}^{\gamma_i/2}(w)(\beta 2^{a+7}k + \beta \ln(1/\delta))}{n}} + \frac{2 (\beta 2^{a+7}k + \beta \ln(1/\delta))}{n}  \\
    &\leq \beta \cdot \left(\sqrt{\frac{8\Loss_{A \cD}^{\gamma_i/2}(w)(2^{a+7}k + \ln(1/\delta))}{n}} + \frac{2 (2^{a+7}k + \ln(1/\delta))}{n}\right).
\end{align*}
Thus for $\beta \geq 2$, we have
\begin{align*}
\Pr_{\rS}[A \in D_\beta(\rS)] &\leq \sum_{a=0}^\infty |\Disc_a| \delta'_a \\
&\leq \sum_{a=0}^\infty \delta^\beta \cdot 2^{-(\beta-1)2^{a+7}k} \\
&\leq 2 \cdot \delta^{\beta} \cdot 2^{-(\beta-1)2^{7}k}.
\end{align*}
By Markov's inequality, we have
\begin{align*}
\Pr_{\rS}[\Pr_{\rA}[\rA \in D_\beta(\rS)] > 2 \cdot \delta^{\beta/2} \cdot 2^{-(\beta-1)\cdot 2^{6}k}] &\leq \frac{\E_{\rS}[\Pr_{\rA}[\rA \in D_\beta(\rS)]}{2 \cdot \delta^{\beta/2} \cdot 2^{-(\beta-1)\cdot 2^{6}k}} \\
&= \frac{\E_{\rA}[\Pr_{\rS}[\rA \in D_\beta(\rS)]}{2 \cdot \delta^{\beta/2} \cdot 2^{-(\beta-1)\cdot 2^{6}k}}\\
&\leq \delta^{\beta/2} \cdot 2^{-(\beta-1)2^{6}k}.
\end{align*}
Now call a training set $S$ \emph{representative} if it holds for every $\beta=2^h$ with integer $h \geq 1$ that
\[
\Pr_{\rA}[\rA \in D_\beta(\rS)] \leq 2 \cdot \delta^{\beta/2} \cdot 2^{-(\beta-1)\cdot 2^{6}k}.
\]
A union bound implies that $\rS$ is representative with probability at least
\[
1-\sum_{h=1}^\infty 2 \cdot \delta^{2^{h-1}} \cdot 2^{-(2^h-1)2^{6}k} \geq 1-\frac{\delta}{2^{2^6 k-2}} \geq 1-\delta.
\]
Now define for integer $h \geq 1$ the set
\[
K_h(S) = D_{2^h}(S) \setminus \left(\cup_{b=h+1}^{\infty} D_{2^b}(S) \right).
\]
Let $K_0(S)$ be defined as
\[
K_0(S) = \support(\rA) \setminus \left(\cup_{b=1}^{\infty} D_{2^b}(S) \right).
\]

For any $w \in \finalH$, we may use the triangle inequality to conclude
\begin{align*}
 \left|\E_{\rA,\rt} [\Loss^{\gamma_i/2}_{\rA \cD}(h_{\rA,\rt}(w)) - \Loss^{\gamma_i/2}_{\rA S}(h_{\rA,\rt}(w))]\right| &\leq \\
 \sum_{h=0}^\infty \E_{\rA,\rt}\left[\left|\Loss^{\gamma_i/2}_{\rA \cD}(h_{\rA,\rt}(w)) - \Loss^{\gamma_i/2}_{\rA S}(h_{\rA,\rt}(w))\right| \mid \rA \in K_h(S)\right] \Pr_{\rA}[\rA \in K_h(S)].
\end{align*}
Now consider an $A \in K_h(S)$. Then $A$ has distortion no more than $2^{h+1}$ by definition of $K_h(S)$. This implies that if $h_{A,t}(w)$ is in $\Disc_a$ but not $\Disc_b$ for $b < a$, then $\|h_{A,t}(w)\|_2 \geq 2^{a+1}$ by definition of $\Disc_b$ and we get
\begin{align*}
|\Loss_{A\cD}^{\gamma_i/2}(h_{A,t}(w)) - \Loss_{A\rS}^{\gamma_i/2}(h_{A,t}(w))| &\leq\\
2^{h+1} \cdot \left(\sqrt{\frac{8\Loss_{A \cD}^{\gamma_i/2}(w)(2^{a+7}k + \ln(1/\delta))}{n}} + \frac{2 (2^{a+7}k + \ln(1/\delta))}{n}\right) 
&\leq \\
2^{h+8}  \|h_{A,t}(w)\|_2 \cdot \left(\sqrt{\frac{8\Loss_{A \cD}^{\gamma_i/2}(w)(k + \ln(1/\delta))}{n}} + \frac{2 (k + \ln(1/\delta))}{n}\right).
\end{align*}
Using Cauchy-Schwartz, we thus get for any $w \in \finalH$ that
\begin{align*}
 \left|\E_{\rA,\rt} [\Loss^{\gamma_i/2}_{\rA \cD}(h_{\rA,\rt}(w)) - \Loss^{\gamma_i/2}_{\rA S}(h_{\rA,\rt}(w))]\right| &\leq \\
\sum_{h=0}^\infty 2^{h+8} \E_{\rA,\rt}\bigg[\|h_{\rA,\rt}(w)\|_2  \cdot  \bigg(\sqrt{\frac{8\Loss_{\rA \cD}^{\gamma_i/2}(w)(k + \ln(1/\delta))}{n}} &+\\
\frac{2 (k + \ln(1/\delta))}{n}\bigg)\mid \rA \in K_h(S)\bigg]\Pr_{\rA}[\rA \in K_h(S)]&\leq \\
\sum_{h=0}^\infty 2^{h+8} \sqrt{\E_{\rA,\rt}\left[\|h_{\rA,\rt}(w)\|^2_2  \mid \rA \in K_h(S) \right]} &\ \cdot \\
\sqrt{\E_{\rA,\rt}\left[ \left(\sqrt{\frac{8\Loss_{\rA \cD}^{\gamma_i/2}(w)(k + \ln(1/\delta))}{n}} + \frac{2 (k + \ln(1/\delta))}{n}\right)^2\mid \rA \in K_h(S)\right]}\Pr_{\rA}[\rA \in K_h(S)].
\end{align*}
By Cauchy-Schwartz, this is at most
\begin{align*}
\sqrt{\sum_{h=0}^\infty 2^{2h+16}\E_{\rA,\rt}[\|h_{\rA,\rt}(w)\|_2^2 \mid \rA \in K_h(S) ]  \Pr_{\rA}[\rA \in K_h(S)]} &\ \cdot\\
\sqrt{\sum_{h=0}^\infty \E_{\rA,\rt}\left[ \left(\sqrt{\frac{8\Loss_{\rA \cD}^{\gamma_i/2}(w)(k + \ln(1/\delta))}{n}} + \frac{2 (k + \ln(1/\delta))}{n}\right)^2\mid \rA \in K_h(S)\right]\Pr_{\rA}[\rA \in K_h(S)] }.
\end{align*}
Using Cauchy-Schwartz again and Jensen's inequality, the first sum is bounded by
\begin{align*}
\sum_{h=0}^\infty 2^{2h+16} \E_{\rA,\rt}[\|h_{\rA,\rt}(w)\|^2_2 \mid \rA \in K_h(S) ] \Pr_{\rA}[\rA \in K_h(S)] &\leq \\
\sqrt{ \sum_{h=0}^\infty 2^{4h + 64}\Pr_{\rA}[\rA \in K_h(S)] } \cdot \sqrt{\sum_{h=0}^\infty \E_{\rA,\rt}[\|h_{\rA,\rt}(w)\|^2_2 \mid \rA \in K_h(S) ]^2 \Pr_{\rA}[\rA \in K_h(S)] } &\leq \\
\sqrt{ \sum_{h=0}^\infty 2^{4h + 64}\Pr_{\rA}[\rA \in D_{2^h}(S)] } \cdot \sqrt{\sum_{h=0}^\infty \E_{\rA,\rt}[\|h_{\rA,\rt}(w)\|^4_2 \mid \rA \in K_h(S) ] \Pr_{\rA}[\rA \in K_h(S)] } &\leq \\
\sqrt{ \sum_{h=0}^\infty 2^{4h + 64} 2 (\delta/2^{2^7k+1})^{(2^h-1)/2} } \cdot \sqrt{\E_{\rA,\rt}[\|h_{\rA,\rt}(w)\|^4_2 ] } &\leq \\
2^{33} \cdot \sqrt{\E_{\rA,\rt}[\|h_{\rA,\rt}(w)\|^4_2 ] }.
\end{align*}
Using Jensen's inequality on the second sum, we find that
\begin{align*}
\sum_{h=0}^\infty \E_{\rA,\rt}\left[ \left(\sqrt{\frac{8\Loss_{\rA \cD}^{\gamma_i/2}(w)(k + \ln(1/\delta))}{n}} + \frac{2 (k + \ln(1/\delta))}{n} \right)^2\mid \rA \in K_h(S)\right]\Pr_{\rA}[\rA \in K_h(S)] &=\\
\E_{\rA,\rt}\left[ \left(\sqrt{\frac{8\Loss_{\rA \cD}^{\gamma_i/2}(w)(k + \ln(1/\delta))}{n}} + \frac{2 (k + \ln(1/\delta))}{n} \right)^2\right].
\end{align*}
For positive constants $c_0,c_1,c_2$, we have that the function $f(t)=(\sqrt{c_0 t + c_1} + c_2)^2$ is concave for $t \geq 0$. To see this, we compute its derivative 
\[
f'(t) = 2(\sqrt{c_0 t + c_1} + c_2) \cdot \frac{c_0}{2\sqrt{c_0 t + c_1}} = c_0 + \frac{c_0 c_2}{\sqrt{c_0 t + c_1}},
\]
and its second derivative
\begin{align*}
f''(t) &= \frac{-c_0^2 c_2}{2 (c_0 t + c_1)^{3/2}}.
\end{align*}
This is a negative function for $t \geq 0$. We thus use Jensen's inequality to conclude
\begin{align*}
\E_{\rA,\rt}\left[ \left(\sqrt{\frac{8\Loss_{\rA \cD}^{\gamma_i/2}(w)(k + \ln(1/\delta))}{n}} + \frac{2 (k + \ln(1/\delta))}{n} \right)^2\right] &\leq \\
 \left(\sqrt{\frac{8\E_{\rA,\rt}\left[\Loss_{\rA \cD}^{\gamma_i/2}(w)\right] (k + \ln(1/\delta))}{n}} + \frac{2 (k + \ln(1/\delta))}{n} \right)^2.
\end{align*}
Combining it all, we have thus shown
\begin{align*}
\left|\E_{\rA,\rt} [\Loss^{\gamma_i/2}_{\rA \cD}(h_{\rA,\rt}(w)) - \Loss^{\gamma_i/2}_{\rA S}(h_{\rA,\rt}(w))]\right| &\leq \\
\sqrt{2^{33} \cdot \sqrt{\E_{\rA,\rt}[\|h_{\rA,\rt}(w)\|_2^4]}} \cdot \sqrt{\left(\sqrt{\frac{8\E_{\rA,\rt}\left[\Loss_{\rA \cD}^{\gamma_i/2}(w)\right] (k + \ln(1/\delta))}{n}} + \frac{2 (k + \ln(1/\delta))}{n} \right)^2} &\leq \\
2^{17} \cdot \E_{\rA,\rt}[\|h_{\rA,\rt}(w)\|_2^4]^{1/4} \cdot \left(\sqrt{\frac{8\E_{\rA,\rt}\left[\Loss_{\rA \cD}^{\gamma_i/2}(w)\right] (k + \ln(1/\delta))}{n}} + \frac{2 (k + \ln(1/\delta))}{n} \right).
\end{align*}
We now bound $\E_{\rA,\rt}[\|h_{\rA,\rt}(w)\|_2^4]$ as follows
\begin{align*}
\E_{\rA,\rt}[\|h_{\rA,\rt}(w)\|_2^4] &= \\
\E_{\rA,\rt}[\|\rA w + (h_{\rA,\rt}(w)-\rA w)\|_2^4] &\leq \\
\E_{\rA,\rt}[\left(\|\rA w\|_2 + \|h_{\rA,\rt}(w)-\rA w\|_2\right)^4] &\leq\\
\E_{\rA,\rt}\left[\left(\|\rA w\|_2 + \sqrt{k (10\sqrt{k})^{-2}}\right)^4\right] &=\\
\E_{\rA,\rt}\left[\left(\|\rA w\|_2 + 1/10\right)^4\right] &=\\
\sum_{b=0}^4 \binom{4}{b} \E_{\rA,\rt}[\|\rA w\|_2^b] 10^{-(4-b)}.
\end{align*}
Recalling that $\|\rA w\|_2^2 \sim (1/k)\chi_k^2$, we have from the moments of the chi-square distribution that for even $k \geq 4$:
\[
\E_{\rA,\rt}[\|\rA w\|_2^b] \leq \E_{\rA,\rt}[\|\rA w\|_2^4] =k^{-2}\E_{\rA,\rt}[(k\|\rA w\|_2^2)^2] = k^{-2} 2^2 \frac{(2 + k/2)!}{(k/2)!} \leq 4.
\]
Hence
\begin{align*}
\E_{\rA,\rt}[\|h_{\rA,\rt}(w)\|_2^4] \leq
\sum_{b=0}^4 \binom{4}{b} 4 \cdot 10^{-(4-b)} \leq (4 + 1/10)^4 < 5^4.
\end{align*}
We thus have
\begin{align*}
\left|\E_{\rA,\rt} [\Loss^{\gamma_i/2}_{\rA \cD}(h_{\rA,\rt}(w)) - \Loss^{\gamma_i/2}_{\rA S}(h_{\rA,\rt}(w))]\right| &\leq \\
2^{20} \cdot \left(\sqrt{\frac{8\E_{\rA,\rt}\left[\Loss_{\rA \cD}^{\gamma_i/2}(w)\right] (k + \ln(1/\delta))}{n}} + \frac{2 (k + \ln(1/\delta))}{n} \right).
\end{align*}
Finally, we exploit that for any $w \in \subH(\Gamma_i, L_j)$, we have by definition that $\Loss_\cD^{(3/4)\gamma_{i}}(w)\leq \ell_{j+1}$. Thus for any such $w$, we have
\begin{align*}
\E_{\rA,\rt}[\Loss^{\gamma_i/2}_{\rA \cD}(w)] &=\E_{\rA,\rt}[\Pr_{(\rx,\ry)\sim \cD}[\ry \langle h_{\rA,\rt}(w), \rA\rx \rangle \leq \gamma_i/2]] \\
&=
\E_{(\rx,\ry)\sim \cD}[\Pr_{\rA,\rt}[\ry \langle h_{\rA,\rt}(w), \rA\rx \rangle \leq \gamma_i/2]]
\\&\leq
\Pr_{(\rx,\ry)\sim \cD}[\ry\langle w, \rx \rangle \leq (3/4)\gamma_{i}] \\
&+ \E_{(\rx,\ry)\sim \cD}[\Pr_{\rA,\rt}[\ry \langle h_{\rA,\rt}(w), \rA\rx \rangle \leq \gamma_i/2] \mid \ry\langle w, \rx \rangle > (3/4)\gamma_{i}] \\
&\leq \Loss_{\cD}^{(3/4)\gamma_{i}}(w)  + \sup_{\mu > (3/4)\gamma_i}[\Pr_{\rA,\rt}[\langle h_{\rA,\rt}(w), \rA x \rangle \leq \gamma_i/2 \mid y\ipr{w,x}=\mu].
\end{align*}
Using Lemma~\ref{lem:concdiscretize} and that $\Loss^{(3/4)\gamma_i}_\cD(w) \in L_j$ by definition of $\subH(\Gamma_i,L_j)$, there is a constant $c>0$ such that this is bounded by
\begin{align*}
&\leq \Loss_{\cD}^{(3/4)\gamma_{i}}(w) + c\exp(-k(\gamma_{i}/4)^2/c)\\
&\leq \ell_{j+1} + c \exp(-k \gamma_{i+1}^2/(16 c)).
\end{align*}
We have thus reached the conclusion that there is a constant $c>0$, such that with probability at least $1-\delta$ over $\rS \sim \cD^n$, it holds that
\begin{align*}
    \sup_{w \in \subH(\Gamma_i,L_j)} \left|\E_{\rA,\rt} [\Loss^{\gamma_i/2}_{\rA \cD}(h_{\rA,\rt}(w)) - \Loss^{\gamma_i/2}_{\rA S}(h_{\rA,\rt}(w))]\right| &\leq\\ c \cdot \left(\sqrt{\frac{(\ell_{j+1} + \exp(-k \gamma_{i+1}^2/c)) (k + \ln(1/\delta))}{n}} + \frac{k + \ln(1/\delta)}{n} \right).
\end{align*}
This completes the proof of Lemma~\ref{lem:supW}.

\section{Within Constant Factors}
\label{sec:withinconstant}
In this section we prove
\begin{customlem}{\ref{lem:subgoal2}}
There is a constant $c>1$, such that for any $0 < \delta < 1$ and any $\Gamma_i = (\gamma_i, \gamma_{i+1}]$, it holds with probability at least $1-\delta$ over a random sample $\rS \sim \cD^n$ that
\begin{align*}
\forall w \in \finalH : \Loss_{\rS}^{\gamma_i}(w) \geq \frac{\Loss_{ \cD}^{(3/4)\gamma_i}(w)}{4} - c \left(\frac{\ln(e\gamma_{i+1}^2 n)}{\gamma_{i+1}^2 n} - \frac{\ln(e/\delta)}{n}\right).
\end{align*}
\end{customlem}
The proof follows mostly the ideas in~\cite{SVMbest} that were outlined in the proof overview in Section~\ref{sec:overview}. 

\begin{proof}
Let $k \geq 1$ be a parameter to be determined and consider the random construction of $\rA$ and $\rt$ as defined in Section~\ref{sec:mainargs}. Let $\Disc_a$ be defined as in Section~\ref{sec:meetinmid}, i.e.\ $\Disc_a$ contains all vectors in $2^a \cdot 4 \Ball_2^k$. We say that a matrix $A$ in the support of $\rA$ and a training set $S$ is $\alpha$-\emph{unusual}, if there is a vector $w \in \Disc_0$ such that
\[
\Loss_{A S}^{(7/8)\gamma_i}(w) < \frac{\Loss_{A \cD}^{(7/8)\gamma_i}(w)}{2} - \frac{2^{11} k + \ln(1/\alpha)}{n}.
\]
For a fixed matrix $A$ and vector $w \in \Net_0$, we have by Bernstein's inequality and $\E_{\rS}[\Loss_{A S}^{(7/8)\gamma_i}(w)]= \Loss_{A \cD}^{(7/8)\gamma_i}(w)$ that
\begin{align*}
    \Pr_{\rS}\left[\left|\Loss_{A \rS}^{(7/8)\gamma_i}(w) -\Loss_{A \cD}^{(7/8)\gamma_i}(w)\right|>t/n \right] < \exp\left(- \frac{\frac{1}{2} t^2}{n\Loss_{A \cD}^{(7/8)\gamma_i}(w) + \frac{1}{3}t}\right).
\end{align*}
Setting
\[
t = n \cdot \left( \frac{\Loss_{A \cD}^{(7/8)\gamma_i}(w)}{2} + Z\right)
\]
with $Z=16 \ln(1/\alpha)/n$ gives
\begin{align*}
    \Pr_{\rS}\Bigg[\bigg|\Loss_{A \rS}^{(7/8)\gamma_i}(w) -\Loss&_{A \cD}^{(7/8)\gamma_i}(w)\bigg|>\left( \frac{\Loss_{A \cD}^{(7/8)\gamma_i}(w)}{2} + Z\right) \Bigg]\\ 
    &< \exp\left(- \frac{\frac{n^2}{2} \left( \frac{\Loss_{A \cD}^{(7/8)\gamma_i}(w)}{2} + Z\right)^2}{n\Loss_{A \cD}^{(7/8)\gamma_i}(w) + \frac{n}{3}\left( \frac{\Loss_{A \cD}^{(7/8)\gamma_i}(w)}{2} + Z\right)}\right) \\
    &\leq \exp\left(- \frac{\frac{n^2}{8} \max\{\Loss_{A \cD}^{(7/8)\gamma_i}(w), Z\}^2}{2n \max\{\Loss_{A \cD}^{(7/8)\gamma_i}(w), Z\}}\right) \\
    &\leq \exp\left(-\frac{n Z}{16} \right) \\
    &= \alpha.
\end{align*}
A union bound over all $w \in \Disc_0$ with $\alpha' = \alpha/e^{2^7 k}$ gives that a fixed matrix $A$ is $\alpha$-unusual for $\rS \sim \cD^n$ with probability at most
\[
|\Disc_0| \frac{\alpha}{e^{2^7 k}} < \alpha.
\]
Now call a training set $S$ $\alpha$-representative if $\rA$ is $\alpha$-unusual for $S$ with probability less than $1/4$. By Markov's inequality, we have
\begin{align*}
\Pr_{\rS}[\Pr_{\rA}[(\rS,\rA) \textrm{ is $\alpha$-unusual}] \geq 1/4] &\leq \frac{\E_{\rS}[\Pr_{\rA}[(\rS,\rA) \textrm{ is $\alpha$-unusual}]]}{1/4} \\
&= 4 \cdot \E_{\rA}[\Pr_{\rS}[(\rS,\rA) \textrm{ is $\alpha$-unusual}]]\\
&\leq 4 \alpha.
\end{align*}
Thus
\begin{align}
\Pr_\rS[\rS \textrm{ is $\alpha$-representative}] \geq 1-4 \alpha.\label{eq:oftenrep}
\end{align}
We claim that if the training set $S$ is $\delta$-representative, then it holds for all $w \in \finalH$ that
\[
\Loss^\gamma_{S}(w) \geq \frac{\Loss_{ \cD}^{(3/4)\gamma_i}(w)}{4} - \frac{2^{11} k + \ln(4/\delta)}{n} -30 \exp(-k \gamma_{i+1}^2/2^{14}).
\]
To see this, consider an arbitrary such $S$ and a $w \in \finalH$. Sample $\rA$ and $\rt$ as in the previous section. Call $\rA, \rt$ \emph{good} for $w$ if it satisfies both $\|h_{\rA,\rt}(w)\|_2 \leq 4$ and $\Loss^{(7/8)\gamma_i}_{\rA \cD}(h_{\rA,\rt}(w)) \geq \Loss^{(3/4)\gamma_i}_{\cD}(w) - 25 \exp(-k \gamma_{i+1}^2/2^{14})$. For ease of notation, let $G_w$ denote the set of $(A,t)$ that are good for $w$. Similarly, let $U_S$ denote the set of $A$ where $A$ is $\delta$-unusual for $S$.

For all $w \in \finalH$, $\gamma \in \Gamma_i$, $A$ and $t$, we have that
\begin{align*}
    \Loss_{S}^\gamma(w) &\geq \Loss^{(7/8)\gamma_i}_{A S}(h_{A,\rt}(w)) - \Pr_{(\rx,\ry)\sim S}[\ry \langle w, \rx \rangle > \gamma \wedge \ry \langle h_{A,\rt}(w), A \rx \rangle \leq (7/8)\gamma_i].
\end{align*}
Thus
\begin{align}
    \Loss_{S}^\gamma(w) &\geq \E_{\rA,\rt}[\Loss^{(7/8)\gamma_i}_{\rA S}(h_{\rA,\rt}(w)) - \Pr_{(\rx,\ry)\sim S}[\ry \langle w, \rx \rangle > \gamma \wedge \ry \langle h_{\rA,\rt}(w), \rA \rx \rangle \leq (7/8)\gamma_i]] \nonumber\\
    &\geq \E_{\rA,\rt}[\Loss^{(7/8)\gamma_i}_{\rA S}(h_{\rA,\rt}(w)) \mid (\rA,\rt) \in G_w \wedge \rA \notin U_S] \Pr_{\rA,\rt}[(\rA,\rt) \in G_w \wedge \rA \notin U_S] \label{eq:condGood}\\
    &-\E_{\rA,\rt}[\Pr_{(\rx,\ry)\sim S}[\ry \langle w, \rx \rangle > \gamma \wedge \ry \langle h_{\rA,\rt}(w), \rA \rx \rangle \leq (7/8)\gamma_i]]\label{eq:randroundoff}.
\end{align}
For the term~\eqref{eq:condGood}, we observe that conditioned on $(\rA,\rt) \in G_w$, we have that $h_{\rA,\rt}(w) \in \Disc_0$ since $\|h_{\rA,\rt}(w)\|_2 \leq 4$. Secondly, when $\rA \notin U_S$, this implies by the definition of $\delta$-unusual that
\[
\Loss_{A S}^{(7/8)\gamma_i}(h_{\rA,\rt}(w)) \geq \frac{\Loss_{A \cD}^{(7/8)\gamma_i}(h_{\rA,\rt}(w))}{2} - \frac{2^{11} k + \ln(1/\delta)}{n}.
\]
Hence
\begin{align}
    &\E_{\rA,\rt}[\Loss^{(7/8)\gamma_i}_{\rA S}(h_{\rA,\rt}(w)) \mid (\rA,\rt) \in G_w \wedge \rA \notin U_S] \Pr_{\rA,\rt}[(\rA,\rt) \in G_w \wedge \rA \notin U_S] \geq \nonumber\\
    &\E_{\rA,\rt}\Bigg[\frac{\Loss_{A \cD}^{(7/8)\gamma_i}(h_{\rA,\rt}(w))}{2} \Bigg|(\rA,\rt) \in G_w \wedge \rA \notin U_S\Bigg]\Pr_{\rA,\rt}[(\rA,\rt)\in G_w \wedge \rA \notin U_S] - \frac{2^{11} k + \ln(1/\delta)}{n}.\label{eq:adbound}
\end{align}
Using again that $(\rA, \rt) \in G_w$, we have that~\eqref{eq:adbound} is at least
\begin{align}
    \frac{\Loss_{ \cD}^{(3/4)\gamma_i}(w)}{2} \Pr_{\rA,\rt}[(\rA,\rt)\in G_w \wedge \rA \notin U_S] - \frac{2^{11} k + \ln(1/\delta)}{n} -25 \exp(-k \gamma_{i+1}^2/2^{14}). \label{eq:lastprop}
\end{align}
We now bound $\Pr[(\rA,\rt) \in G_w]$  and $\Pr[\rA \notin U_S]$. For this, we recall that $\|\rA w\|_2^2 \sim (1/k)\chi_2^k$. Thus $\E[\|\rA w\|_2^2] =1$ and by Markov's, we get $\Pr[\|\rA w\|_2^2 \geq 9] \leq 1/9$. Conditioned on $\|\rA w\|_2^2 < 9$, we have $\|h_{\rA,\rt}(w)\|_2 \leq \|\rA w\|_2 + \|h_{\rA,\rt}(w)-\rA w\|_2 \leq \sqrt{9} + \sqrt{k (10 \sqrt{k})^{-2}} < 4$. Next observe that
\begin{align*}
    \Loss_{\rA \cD}^{(7/8)\gamma_i}(h_{\rA,\rt}(w)) &\geq \Loss_{\cD}^{(3/4)\gamma_i}(w) - \Pr_{(\rx, \ry)\sim \cD}[\ry \langle w, \rx \rangle \leq (3/4)\gamma_i \wedge \ry \langle h_{\rA,\rt}(w), \rA \rx \rangle > (7/8)\gamma_i].
\end{align*}
We have by Lemma~\ref{lem:concdiscretize} that there is a constant $c>0$ so that
\begin{align*}
    \E_{\rA,\rt}[\Pr_{(\rx, \ry)\sim \cD}[\ry \langle w, \rx \rangle \leq (3/4)\gamma_i \wedge \ry \langle h_{\rA,\rt}(w), \rA \rx \rangle > (7/8)\gamma_i]] &= \\
    \E_{(\rx, \ry)\sim \cD}[\Pr_{\rA,\rt}[\ry \langle w, \rx \rangle \leq (3/4)\gamma_i \wedge \ry \langle h_{\rA,\rt}(w), \rA \rx \rangle > (7/8)\gamma_i]] &\leq \\
    \sup_{x \in \finalX : \langle w, x\rangle \leq (3/4)\gamma_i}\Pr_{\rA,\rt}[ \langle h_{\rA,\rt}(w), \rA x \rangle > (7/8)\gamma_i] &\leq \\
    c\exp(-k(\gamma_i/8)^2/c) &\leq \\
    c\exp(-k \gamma_{i+1}^2/(2^{8}c)).
\end{align*}
Thus by Markov's inequality, we conclude
\begin{align*}
\Pr_{\rA,\rt}[\Loss_{\rA \cD}^{(7/8)\gamma_i}(h_{\rA,\rt}(w)) <\Loss_{\cD}^{(3/4)\gamma_i}(w)- 5c \exp(-k \gamma_{i+1}^2/(2^{8}c))] &\leq \\
    \Pr_{\rA,\rt}[\Pr_{(\rx, \ry)\sim \cD}[\ry \langle w, \rx \rangle \leq (3/4)\gamma_i \wedge \ry \langle h_{\rA,\rt}(w), \rA \rx \rangle > (7/8)\gamma_i] > 5 c \exp(-k \gamma_{i+1}^2/(2^{8}c))] &< 1/5.
\end{align*}
Finally, since we assumed $S$ is $\delta$-representative, we have $\Pr_{\rA}[\rA \in U_S] \leq 1/4$ by definition of $\delta$-representative. We conclude by a union bound that
\begin{align*}
\Pr_{\rA,\rt}[(\rA,\rt)\in G_w \wedge \rA \notin U_S] &\geq 1-1/9-1/5-1/4\geq 1/2.
\end{align*}
In summary, we have shown that~\eqref{eq:lastprop} is at least
\[
\frac{\Loss_{ \cD}^{(3/4)\gamma_i}(w)}{2} \cdot \frac{1}{2} - \frac{2^{11} k + \ln(1/\delta)}{n} -5 c \exp(-k \gamma_{i+1}^2/(2^{8}c)).
\]
Recalling that~\eqref{eq:condGood}~$\geq$~\eqref{eq:lastprop} gives
\begin{align*}
\E_{\rA,\rt}[\Loss^{(7/8)\gamma_i}_{\rA S}(h_{\rA,\rt}(w)) \mid (\rA,\rt) \in G_w \wedge \rA \notin U_S] \Pr_{\rA,\rt}[(\rA,\rt) \in G_w \wedge \rA \notin U_S] &\geq \\
\frac{\Loss_{ \cD}^{(3/4)\gamma_i}(w)}{4} - \frac{2^{11} k + \ln(1/\delta)}{n} -5c \exp(-k \gamma_{i+1}^2/(2^{8}c)).
\end{align*}
The term~\eqref{eq:randroundoff} can be bounded using Lemma~\ref{lem:concdiscretize} by
\begin{align*}
\E_{\rA,\rt}[\Pr_{(\rx,\ry)\sim S}[\ry \langle w, \rx \rangle > \gamma \wedge \ry \langle h_{\rA,\rt}(w), \rA \rx \rangle \leq (7/8)\gamma_i]] &= \\
\E_{(\rx,\ry)\sim S}[\Pr_{\rA,\rt}[\ry \langle w, \rx \rangle > \gamma \wedge \ry \langle h_{\rA,\rt}(w), \rA \rx \rangle \leq (7/8)\gamma_i]] &\leq \\
\sup_{x \in \finalX : \langle w, x\rangle > \gamma }\Pr_{\rA,\rt}[\langle h_{\rA,\rt}(w), \rA x \rangle \leq (7/8)\gamma_i] &\leq\\
c\exp(-k(\gamma - (7/8)\gamma_i)^2/c) &\leq \\
c\exp(-k\gamma_i^2/(64 c))&\leq \\
c\exp(-k\gamma_{i+1}^2/(2^{8}c)).
\end{align*}
In summary, we have shown that for $(\delta/4)$-representative $S$, it holds for all $w \in \finalH$ that
\begin{align*}
    \Loss_S^\gamma(w) \geq \frac{\Loss_{ \cD}^{(3/4)\gamma_i}(w)}{4} - \frac{2^{11} k + \ln(4/\delta)}{n} -6c \exp(-k \gamma_{i+1}^2/(2^{8}c)).
\end{align*}
We finally conclude from~\eqref{eq:oftenrep} that with probability at least $1-\delta$ over $\rS$, it holds for all $w \in \finalH$ that
\begin{align*}
    \Loss_{\rS}^\gamma(w) \geq \frac{\Loss_{ \cD}^{(3/4)\gamma_i}(w)}{4} - \frac{2^{11} k + \ln(4/\delta)}{n} -6c \exp(-k \gamma_{i+1}^2/(2^{8}c)).
\end{align*}
Picking $k=2^{8}c \gamma_{i+1}^{-2} \ln(\gamma^2_{i+1} n)$ finally results in
\begin{align*}
    \Loss_{\rS}^\gamma(w) \geq \frac{\Loss_{ \cD}^{(3/4)\gamma_i}(w)}{4} - \frac{2^{20}c \ln(\gamma_{i+1}^2 n)}{\gamma_{i+1}^2 n} - \frac{2\ln(e/\delta)}{n}.
\end{align*}
This completes the proof.
\end{proof}

\section*{Acknowledgment}
The authors would like to thank Clement Svendsen for valuable measure theoretic insight. 

Kasper Green Larsen is co-funded by a DFF Sapere Aude Research Leader Grant No. 9064-00068B by the Independent Research Fund Denmark and co-funded by the European Union (ERC, TUCLA, 101125203). Natascha Schalburg is funded by the European Union (ERC, TUCLA, 101125203). Views and opinions expressed are however those of the author(s) only and do not necessarily reflect those of the European Union or the European Research Council. Neither the European Union nor the granting authority can be held responsible for them.

\bibliography{SVMBib}
\bibliographystyle{abbrvnat}

\appendix


\section{Auxiliary Results for Proofs}
\label{auxiliary}
In this subsection, we present some auxiliary results that are needed for our proof.
First, we present the estimation of the spectral norm of random matrices.
It can be easily derived from \cite{vershynin2018high} and we put it here for the completeness.

\begin{lemma}\citep[Adapted from Theorem 4.6.1]{vershynin2018high}
\label{lem:conrg}
    For a random sub-Gaussian matrix $\widetilde{\bm X} \in \mathbb{R}^{N \times d}$ whose rows are i.i.d. isotropic sub-gaussian random vector with sub-Gaussian norm $K$, then we have the following statement
\[
\mathbb{P} \left(   \left\|\frac{1}{N}\widetilde{\bm X}^{\!\top}\widetilde{\bm X}-\bm I_d\right\|_{op}  > \delta \right) \leq 2 \exp \left( -C N \min\left(\delta^2, \delta\right) \right)\,.
\]
for a universal constant $C$ depending only on $K$.
\end{lemma}

\begin{lemma}\citep[Adapted from Corollary 5.35]{vershynin2010introduction}
\label{lem:init-op-conct}
    For a random standard Gaussian matrix $\bm S\in\mathbb{R}^{d\times r}$ with $[\bm S]_{ij} \sim \mathcal{N}(0, 1)$, if $d > 2r$, we have 
    \begin{align}
        \label{norm-A0}
        \frac{\sqrt{d}}{2} \leq \|\bm S\|_{op} \leq (2 \sqrt{d} + \sqrt{r})\,,
    \end{align}
    with probability at least $1-C \operatorname{exp}(-d)$ for some positive constants $C$.
\end{lemma}

The following results are modified from the proof of \citet[Lemma 8.7]{stoger2021small}.
\begin{lemma}
\label{lem:min-singular-conct}
    Suppose $\bm S\in\mathbb{R}^{d\times r}$ is a random standard Gaussian matrix with $[\bm S]_{ij} \sim \mathcal{N}(0, 1)$ and $\bm U\in\mathbb{R}^{d\times r^*}$ has orthonormal columns. If $r\geq 2r^*$, with probability at least $1-C\operatorname{exp}(-r)$ for some positive constants $C$, we have
    \begin{align*}
        \lambda_{\operatorname{min}}(\bm U^{\!\top}\bm S) & \gtrsim 1\,.
    \end{align*}
    If $r^*\leq r < 2r^*$, by choosing $\xi>0$ appropriately, with probability at least $1-(C \xi)^{r-r^*+1}-C'\operatorname{exp}(-r)$ for some positive constants $C\,,C'$, we have
    \begin{align*}
        \lambda_{\operatorname{min}}(\bm U^{\!\top}\bm S) & \gtrsim \frac{\xi}{r}\,.
    \end{align*}
\end{lemma}

Next, we give a short description of the Hermite expansion of ReLU function via Hermite polynomials. Details can be found in \citet[A.1.1]{damian2022neural} and \cite{arous2021online}.
To be specific, the Hermite expansion of ReLU function $\sigma(x)$ is
\begin{align}
\label{Hermite-sigma}
    \sigma(x)=\sum_{j=1}^\infty \frac{c_j}{j!}\operatorname{He}_j(x) =\frac{1}{\sqrt{2\pi}}+\frac{1}{2}x+\frac{1}{\sqrt{2\pi}}\sum_{j\geq 1}\frac{(-1)^{j-1}}{j!2^j(2j-1)}\operatorname{He}_{2j}(x)\,,
\end{align}
which implies that we can express the Hermite coefficients as
\begin{align}
\label{Hermite-coef}
    \left\{\begin{aligned}
        c_0 & = \frac{1}{\sqrt{2\pi}}\,,\\
        c_1 & = \frac{1}{2}\,,\\
        c_{2j} & = \frac{(-1)^{j-1}}{\sqrt{2\pi}2^j(2j-1)}\quad \text{for }j\geq 1\,.
    \end{aligned}\right.
\end{align}
Furthermore, the derivative of $\sigma(x)$ admits
\begin{align}
\label{Hermite-sigma'}
    \sigma'(x)=\frac{1}{2}+\frac{1}{\sqrt{2\pi}}\sum_{j\geq 0}\frac{(-1)^{j}}{j!2^j(2j+1)}\operatorname{He}_{2j+1}(x)\,.
\end{align}

\begin{lemma}\citep[Corollary 9]{oko2024pretrained}\label{differential}
$\mathbb{E}_{\widetilde{\bm x}}[\nabla^k \sigma(\langle \bm w\,, \widetilde{\bm x}\rangle)] = c_k \bm w^{\otimes k}$ for any $k$ such that $c_k\neq 0$.
\end{lemma}

\begin{lemma}\label{vec-ineq}
For any vectors $\bm u$ and $\bm v$, we have
    \begin{align*}
        \left|\langle \bm u\,, \bm u \rangle^j - \langle \bm u\,, \bm v \rangle^j\right| & \leq j\,\max\left\{\left\|\bm u\right\|_2\,,\left\|\bm v\right\|_2\right\}^{2j-1} \left\|\bm u - \bm v\right\|_2\,.
    \end{align*}
\end{lemma}
\begin{proof}
    First, we analyze the following two scalar variables case
    \begin{align*}
        \left|x^j-y^j\right|\,.
    \end{align*}
    By algebraic identity $\sum_{j=1}^{t-1}x^{t-j-1}y^j=\frac{x^t-y^t}{x-y}$ which is valid for $\forall\,j\in\mathbb{N}^+$, we have
    \begin{align*}
        \left|x^j-y^j\right|&=\left|(x-y)\sum_{i=0}^{j-1}x^{j-i-1}y^i\right|
        \leq |x-y|\sum_{i=0}^{j-1}\max\left\{|x|\,,|y|\right\}^{j-1}
        = j|x-y|\max\left\{|x|\,,|y|\right\}^{j-1}\,.
    \end{align*}
    Now we define $x:=\langle \bm u\,, \bm u \rangle$ and $y:=\langle \bm u\,, \bm v \rangle$, then we can obtain
    \begin{align*}
        \left|\langle \bm u\,, \bm u \rangle^j - \langle \bm u\,, \bm v \rangle^j\right| & \leq j\,\max\left\{\left|\langle \bm u\,, \bm u \rangle\right|\,,\left|\langle \bm u\,, \bm v \rangle\right|\right\}^{j-1}\left|\langle \bm u\,, \bm u \rangle - \langle \bm u\,, \bm v \rangle\right|\\
        & \leq j\,\max\left\{\left\|\bm u\right\|_2^2\,,\left\|\bm u\right\|_2 \left\|\bm v\right\|_2\right\}^{j-1}\left\|\bm u\right\|_2 \left\|\bm u - \bm v\right\|_2\quad \tag*{\color{teal}[by Cauchy-Schwartz inequality]}\\
        & = j\,\max\left\{\left\|\bm u\right\|_2\,,\left\|\bm v\right\|_2\right\}^{2j-1} \left\|\bm u - \bm v\right\|_2\,.
    \end{align*}
\end{proof}

\end{document}
% \typeout{get arXiv to do 4 passes: Label(s) may have changed. Rerun}