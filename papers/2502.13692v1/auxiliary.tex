\section{Auxiliary Results}
\label{sec:aux}
In this section, we prove a number of auxiliary results used throughout the paper. For this, we need the following concentration inequality:

\begin{theorem}[\cite{Wainwright_2019}, example 2.11]
\label{thm:chibound}
Let $Y \sim \chi^2_k$, then for any $x \in (0,1)$ it holds that
\[
\Pr\left[\left|\frac{Y}{k} - 1\right| \ge x\right] \le 2\exp(-k x^2/8).
\]
\end{theorem}

\begin{customclm}{\ref{clm:union}}
    For any $0 < \delta < 1$, it holds with probability $1-\delta$ over $\rS \sim \cD^n$ that~\eqref{eq:subgoal} and~\eqref{eq:subgoal2} simultaneously hold for all $(\Gamma_i,L_j)$ and $\Gamma_i$, with slightly different constants $c$.
\end{customclm}
\begin{proof}
Let $(\gamma_i, \gamma_{i+1}]$ be such that $\gamma_{i+1} := 2^{i}n^{-1/2}$. Similarly, let $(\ell_j, \ell_{j+1}]$ be such that $\ell_{j+1} := 2^j n^{-1}$. Do a union bound over all $(\Gamma_i, L_j)$ for $i=1,\dots,\lg_2(c_\gamma n^{1/2})$ and $j=0,\dots,\lg_2 n$ with $\delta_{i,j} := (\delta/e)^3 \exp(-\gamma_{i+1}^{-2} \ln(e/\ell_{j+1}))$ in~\eqref{eq:subgoal}. We see that
\begin{align*}
    \sum_{i=1}^{\lg_2(c_\gamma n^{1/2})} \sum_{j=0}^{\lg_2 n} \delta_{i,j} &= \sum_{i=1}^{\lg_2(c_\gamma n^{1/2})} \sum_{j=0}^{\lg_2 n} (\delta/e)^3 \exp(-\gamma_{i+1}^{-2} \ln(e/\ell_{j+1})) \\
    &= \sum_{i=1}^{\lg_2(c_\gamma n^{1/2})} \sum_{j=0}^{\lg_2 n}  (\delta/e)^3 \exp(- 2^{-2i}n \ln(e n 2^{-j})) \\
    &= \sum_{i=1}^{\lg_2(c_\gamma n^{1/2})} \sum_{j=0}^{\lg_2 n}  (\delta/e)^3 (e n 2^{-j})^{-2^{-2i}n} \\
\end{align*}
Doing the substitutions $j \gets \lg_2 n-j$ and $i \gets \lg_2(c_\gamma n^{1/2}) +1 - i$, this equals
\begin{align*}
    &= \sum_{i=1}^{\lg_2(c_\gamma n^{1/2})} \sum_{j=0}^{\lg_2 n}  (\delta/e)^3 (e 2^{j})^{-2^{2i-2}c_\gamma^{-2}} \\
    &\leq \sum_{i=1}^{\lg_2(c_\gamma n^{1/2})} \sum_{j=0}^{\lg_2 n}  (\delta/e)^3 e^{-2^{2i-2}} 2^{-j}\\ 
    &\leq \sum_{i=1}^{\lg_2(c_\gamma n^{1/2})} 2(\delta/e)^3 e^{-2^{2i-2}}\ \\
    &\leq \delta/2.
\end{align*}
Similarly, do a union bound over all $\Gamma_i$ with $\delta_i := (\delta/e)^3 \exp(-\gamma_{i+1}^{-2} \ln(e \gamma^2_{i+1} n))$ in~\eqref{eq:subgoal2}. We have
\begin{align*}
\sum_{i=1}^{\lg_2(c_\gamma n^{1/2})} \delta_i &= \sum_{i=1}^{\lg_2(c_\gamma n^{1/2})} (\delta/e)^3 \exp(-\gamma_{i+1}^{-2} \ln(e \gamma^2_{i+1} n)) \\
&\leq\sum_{i=1}^{\lg_2(c_\gamma n^{1/2})}(\delta/e)^3 \exp(-\ln(e \gamma^2_{i+1} n))\\
&= \sum_{i=1}^{\lg_2(c_\gamma n^{1/2})}(\delta/e)^3 \frac{1}{e \gamma_{i+1}^2 n} \\
&= \sum_{i=1}^{\lg_2(c_\gamma n^{1/2})}(\delta/e)^3 \frac{n}{e 2^{2i} n} \\
&\leq (\delta/e)^3 \\
&\leq \delta/2.
\end{align*}
We thus have that with probability at least $1-\delta$ that for all $(\Gamma_i, L_j)$, we have
\begin{align*}
\sup_{w \in \subH(\Gamma_i, L_j), \gamma \in \Gamma_i} \left|\Loss_\cD(w) - \Loss^{\gamma}_\rS(w) \right| &\leq \nonumber\\
c \left(\sqrt{\ell_{j+1}\left( \frac{ \ln(e/\ell_{j+1})}{\gamma_{i+1}^2 n} + \frac{\ln(e/\delta_{i,j})}{n} \right)} + \frac{\ln(e/\ell_{j+1})}{\gamma_{i+1}^2 n} + \frac{\ln(e/\delta_{i,j})}{n} \right) &=\\
c \left(\sqrt{\ell_{j+1}\left( \frac{ \ln(e/\ell_{j+1})}{\gamma_{i+1}^2 n} + \frac{\ln(e/\delta_{i,j})}{n} \right)} + 2\cdot \frac{\ln(e/\ell_{j+1})}{\gamma_{i+1}^2 n} + 3 \cdot \frac{\ln(e/\delta)}{n} \right).
\end{align*}
and for all $\Gamma_i$, we have
\begin{align*}
\inf_{w \in \finalH} \Loss_{\rS}^{\gamma_i}(w) &\geq \frac{\Loss_{ \cD}^{(3/4)\gamma_i}(w)}{4} - c \left(\frac{\ln(e\gamma_{i+1}^2 n)}{\gamma_{i+1}^2 n} - \frac{\ln(e/\delta_i)}{n}\right)\\ &=
\frac{\Loss_{ \cD}^{(3/4)\gamma_i}(w)}{4} - c \left(2 \cdot \frac{\ln(e\gamma_{i+1}^2 n)}{\gamma_{i+1}^2 n} - 3 \cdot \frac{\ln(e/\delta)}{n}\right).
\end{align*}
\end{proof}

\begin{customclm}{\ref{clm:combine}}
    For any $0 < \delta < 1$ and training set $S$, if~\eqref{eq:subgoal} and~\eqref{eq:subgoal2} hold simultaneously for all $(\Gamma_i, L_j)$ and $\Gamma_i$, then~\eqref{eq:maingoal} holds for all $\gamma \in (n^{-1/2}, c_\gamma]$ and all $w \in \finalH$ for large enough constant $c>1$ in~\eqref{eq:maingoal}.
\end{customclm}
\begin{proof}
Let $0 < \delta < 1$ and assume as in the claim that~\eqref{eq:subgoal} and~\eqref{eq:subgoal2} holds for all $(\Gamma_i, L_j)$ and $\Gamma_i$. Now consider an arbitrary $\gamma \in (n^{-1/2},c_\gamma]$ and $w \in \finalH$. Let $i$ and $j$ be such that $\gamma \in (\gamma_i, \gamma_{i+1}]$ and $\Loss^{(3/4)\gamma_{i}}_\cD(w) \in (\ell_j, \ell_{j+1}]$ with $\gamma_{i+1}=2^i n^{-1/2}$ and $\ell_{j+1} = 2^j n^{-1}$. We consider two cases. Let $c_{\ref{lem:subgoal2}}>1$ be the constant in Lemma~\ref{lem:subgoal2}. First, if 
\[
\Loss_\cD^{(3/4)\gamma_{i}}(w) \leq 16 \cdot c_{\ref{lem:subgoal2}} \cdot \left(\frac{\ln(e\gamma_{i+1}^2 n)}{\gamma_{i+1}^2 n} + \frac{\ln(e/\delta)}{n} \right),
\]
then since $\Loss_\cD(w) \leq \Loss_\cD^{(3/4)\gamma_{i}}(w)$ and $\gamma \leq \gamma_{i+1}$ (using that $\ln(e \gamma^2 n)/(\gamma^2 n)$ is decreasing in $\gamma$ for $\gamma \geq n^{-1/2}$), we have already shown~\eqref{eq:maingoal} for sufficiently large constant $c$ in~\eqref{eq:maingoal}. So assume this is not the case. Our goal is to show that $\ell_{j+1}$ and $\Loss_S^\gamma(w)$ are within constant factors of each other so that we may replace occurrences of $\ell_{j+1}$ by $\Loss_S^\gamma(w)$ in~\eqref{eq:subgoal}. We first see that our assumption implies
\begin{align}
\ell_{j+1} \geq \Loss_\cD^{(3/4)\gamma_{i}}(w) \geq 16 \cdot c_{\ref{lem:subgoal2}} \cdot \left(\frac{\ln(e\gamma_{i+1}^2 n)}{\gamma_{i+1}^2 n} + \frac{\ln(e/\delta)}{n} \right) \geq   \frac{1}{\gamma_{i+1}^2 n} > n^{-1}.\label{eq:ljrelate}
\end{align}
This also implies $j \neq 0$ and hence $\ell_{j+1}=2\ell_j$ and therefore $\ell_{j+1} \leq 2 \Loss_{\cD}^{(3/4)\gamma_{i}}(w)$. Letting $c_{\ref{lem:subgoal}}$ be the constant in Lemma~\ref{lem:subgoal}, we get from~\eqref{eq:subgoal} and~\eqref{eq:ljrelate}, that
\begin{align*}
    \Loss^\gamma_S(w) &\leq \Loss_\cD(w) + c_{\ref{lem:subgoal}} \cdot \left(\sqrt{\ell_{j+1}\left( \frac{ \ln(e/\ell_{j+1})}{\gamma_{i+1}^2 n} + \frac{\ln(e/\delta)}{n} \right)} + \frac{\ln(e/\ell_{j+1})}{\gamma_{i+1}^2 n} + \frac{\ln(e/\delta)}{n} \right) \\
    \leq \ & \Loss^{(3/4)\gamma_{i}}_\cD(w) + c_{\ref{lem:subgoal}} \cdot \left(\sqrt{2 \Loss^{(3/4)\gamma_{i}}_\cD(w)\left( \frac{ \ln(e\gamma_{i+1}^2 n)}{\gamma_{i+1}^2 n} + \frac{\ln(e/\delta)}{n} \right)} + \frac{\ln(e \gamma_{i+1}^2 n)}{\gamma_{i+1}^2 n} + \frac{\ln(e/\delta)}{n} \right)\\
    \leq \ & \Loss^{(3/4)\gamma_{i}}_\cD(w) + c_{\ref{lem:subgoal}} \cdot \left(\sqrt{2 \Loss^{(3/4)\gamma_{i}}_\cD(w)\Loss^{(3/4)\gamma_{i}}_\cD(w)} + \Loss^{(3/4)\gamma_{i}}_\cD(w) \right)\\
    \leq \ & 3\cdot c_{\ref{lem:subgoal}} \cdot \Loss^{(3/4)\gamma_{i}}_\cD(w).
\end{align*}
We thus also have $\ell_{j+1} \geq \Loss^{(3/4)\gamma_{i}}_\cD(w) \geq (3\cdot c_{\ref{lem:subgoal}})^{-1} \Loss_S^\gamma(w)$. Inserting this and~\eqref{eq:ljrelate} in~\eqref{eq:subgoal} gives
\begin{align}
\Loss_{\cD}(w) \leq \Loss_{S}^\gamma(w) + c_{\ref{lem:subgoal}} \left( \sqrt{\ell_{j+1}\left( \frac{ \ln(3e c_{\ref{lem:subgoal}}/\Loss_{S}^\gamma(w))}{\gamma_{i+1}^2 n} + \frac{\ln(e/\delta)}{n} \right)} + \frac{\ln(e\gamma_{i+1}^2 n)}{\gamma_{i+1}^2 n} + \frac{\ln(e/\delta)}{n} \right).\label{eq:almostdone}
\end{align}
Finally from~\eqref{eq:subgoal2} and $\gamma \geq \gamma_i$, we have
\begin{align*}
\Loss_{S}^\gamma(w) &\geq \Loss_{S}^{\gamma_i}(w) \\
&\geq \frac{\Loss_{\cD}^{(3/4)\gamma_i}(w)}{4} - c_{\ref{lem:subgoal2}}\left( \frac{\ln(e\gamma_{i+1}^2 n)}{\gamma_{i+1}^2 n} - \frac{\ln(e/\delta)}{n}\right)\\
&\geq \frac{\ell_{j+1}}{8} - c_{\ref{lem:subgoal2}}\left( \frac{\ln(e\gamma_{i+1}^2 n)}{\gamma_{i+1}^2 n} - \frac{\ln(e/\delta)}{n}\right).
\end{align*}
From~\eqref{eq:ljrelate}, this is at least $\ell_{j+1}/16$ and thus $\ell_{j+1} \leq 16 \Loss^{\gamma}_S(w)$. Inserting this in~\eqref{eq:almostdone} finally gives us
\begin{align*}
\Loss_{\cD}(w) \leq \Loss_{S}^\gamma(w) + c_{\ref{lem:subgoal}} \left( \sqrt{16 \Loss^\gamma_S(w) \left( \frac{ \ln(2e c_{\ref{lem:subgoal}}/\Loss_{S}^\gamma(w))}{\gamma_{i+1}^2 n} + \frac{\ln(e/\delta)}{n} \right)} + \frac{\ln(e\gamma_{i+1}^2 n)}{\gamma_{i+1}^2 n} + \frac{\ln(e/\delta)}{n} \right).
\end{align*}
Since $\gamma \leq \gamma_{i+1}$, this completes the proof of Claim~\ref{clm:combine} for sufficiently large $c>0$ in~\eqref{eq:maingoal}.
\end{proof}

\begin{customlem}{\ref{lem:concdiscretize}}
    There is a constant $c>0$, such that for any integer $k \geq 1$, $w \in \finalH, x \in \finalX$ and any $\gamma \in (0,1]$, it holds that 
    $
    \Pr_{\rA,\rt}[|\langle h_{\rA,\rt}(w),\rA x\rangle - \langle w, x\rangle| > \gamma] < c\exp(-\gamma^2 k/c)
    $.    
\end{customlem}
\begin{proof} %(of Lemma~\ref{lem:concdiscretize})
We start by observing that $\|\rA w\|_2^2$, $\|\rA x\|_2^2$ and $\|\rA (w-x)\|_2^2/\|w-x\|_2^2$ are all $(1/k)\chi^2_k$ distributed. Using Theorem~\ref{thm:chibound} with $x=\gamma/3$, we have with probability at least $1-6 \exp(-k \gamma^2/72)$ that $\|\rA w\|_2^2 \in 1 \pm \gamma/3$, $\|\rA x\|_2^2 \in 1 \pm \gamma/3$ and $\|\rA(w-x)\|_2^2 \in \|w-x\|_2^2(1 \pm \gamma/3)$. By the polar identity, this implies
\begin{align*}
\langle \rA w, \rA w \rangle &= \frac{1}{4} \left(\|\rA w\|_2^2 + \|\rA x\|_2^2 - \|\rA(w-x)\|_2^2 \right) \\
&\in \frac{1}{4}\left(\|w\|_2^2 + \|x\|_2^2 - \|w-x\|_2^2 \right) \pm \frac{\gamma}{12}\left(\|w\|_2^2 + \|x\|_2^2 + \|w-x\|_2^2 \right) \\
&\subseteq \langle w, x \rangle \pm \frac{\gamma}{12}\left(1 + 1 + 4 \right) \\
&= \langle w, x \rangle \pm \frac{\gamma}{2}.
\end{align*}
Let us condition on an outcome $A$ of $\rA$ satisfying the above. We then observe that
\begin{align*}
\langle h_{A, \rt}(w) , A x\rangle &= \langle h_{A, \rt}(w) - A w, A x\rangle + \langle A w, A x\rangle.
\end{align*}
By the randomized rounding procedure, we have that each coordinate $i$ satisfies $\E_{\rt_i}[(h_{A,\rt}(w))_i] = (A w)_i$. Moreover, these coordinates are independent. Letting $\Delta_i = (h_{A,\rt}(w))_i - (A w)_i$, we then have that $\E[\Delta_i] = 0$ and that $\Delta_i$ lies in an interval of length $(10 \sqrt{k})^{-1}$. Hoeffding's inequality implies
\begin{align*}
\Pr_{\rt}[\left|\langle h_{A, \rt}(w) - A w, A x\rangle\right| > \gamma/2] &= \Pr_{\Delta_1,\dots,\Delta_k}\left[\left|\sum_{i=1}^k \Delta_i(A x)_i\right| > \gamma/2\right] \\
&< 2 \exp\left( -\frac{2 (\gamma/2)^2}{\sum_{i=1}^k (10 \sqrt{k})^{-2}(Ax)_i^2} \right) \\
&= 2 \exp\left( -\frac{50 \gamma^2 k}{\|A x\|_2^2} \right) \\
&\leq 2 \exp\left( - 25 \gamma^2 k \right)
\end{align*}
In summary, it holds with probability at least $1-6\exp(-k \gamma^2/72) - 2 \exp(-25 \gamma^2 k) \geq 1-7 \exp(-k \gamma^2/72)$ that
\begin{align*}
|\langle h_{\rA,\rt}(w),\rA x\rangle - \langle w, x\rangle| &\leq |\langle h_{\rA,\rt}(w),\rA x\rangle - \langle \rA w, \rA x \rangle | + |\langle \rA w, \rA x \rangle - \langle w, x\rangle|\\
&\leq |\langle h_{\rA,\rt}(w)-\rA w,\rA x\rangle| + \gamma/2\\
&< \gamma.
\end{align*}
\end{proof}

\begin{customrmk}{\ref{rmk:pIsProb}}
The value $p(z_i)$ satisfying~\eqref{eq:expectround} has $p(z_i) \in [0,1]$.
\end{customrmk}
\begin{proof} 
Recall that~\eqref{eq:expectround} states that $p(z_i)$ satisfies
\begin{align*}
(Aw)_i &= 
p(z_i)\left(\frac{1}{2 \cdot 10 \sqrt{k}} +\frac{z_i}{10\sqrt{k}}\right) + (1-p(z_i))\left(\frac{1}{2 \cdot 10 \sqrt{k}} +\frac{z_i+1}{10\sqrt{k}}\right)
\end{align*}
where $z_i$ is such that
\[
(1/2)(10 \sqrt{k})^{-1}  + z_i (10 \sqrt{k})^{-1} \leq (Aw)_i < (1/2)(10 \sqrt{k})^{-1}  + (z_i+1) (10\sqrt{k})^{-1}.
\]
This implies
\begin{align*}
    ((1/2)(10 \sqrt{k})^{-1} +z_i (10\sqrt{k})^{-1}) + (1-p(z_i))(10\sqrt{k})^{-1} &= (Aw)_i \Rightarrow \\
    (Aw)_i -((1/2)(10 \sqrt{k})^{-1} +z_i (10\sqrt{k})^{-1})&=(1-p(z_i))(10\sqrt{k})^{-1}.
\end{align*}
By definition of $z_i$, we have that the left hand side is a number in $[0,(10\sqrt{k})^{-1}]$ and thus we conclude
\[
(1-p(z_i)) \in [0,1] \Rightarrow p(z_i) \in [0,1].
\]
\end{proof}

\begin{customrmk}{\ref{rmk:phirho}}
For any training set $S$ and distribution $\cD$ over $\finalX \times \{-1,1\}$, we have
\begin{align*}
    \E_{\rA,\rt} [\Pr_{(\rx,\ry) \sim \cD}[\ry \langle h_{\rA,\rt}(w), \rA \rx\rangle > \gamma_i/2 \wedge \ry \langle w, \rx\rangle \leq 0]] &\leq\E_{(\rx,\ry) \sim \cD}[\phi(\ry \langle w, \rx \rangle)] \\
    \E_{\rA,\rt} [\Pr_{(\rx,\ry) \sim S}[\ry \langle h_{\rA,\rt}(w), \rA \rx\rangle > \gamma_i/2 \wedge \ry \langle w, \rx\rangle \leq \gamma]] &\geq \E_{(\rx,\ry) \sim S}[\phi(\ry \langle w, \rx \rangle)] \\
    \E_{\rA,\rt} [\Pr_{(\rx,\ry) \sim S}[\ry \langle h_{\rA,\rt}(w), \rA \rx\rangle \leq \gamma_i/2 \wedge \ry \langle w, \rx\rangle > \gamma]] &\leq \E_{(\rx, \ry) \sim S}[\rho(\ry \langle w, \rx \rangle)]\\
    \E_{\rA,\rt} [\Pr_{(\rx,\ry) \sim \cD}[\ry \langle h_{\rA,\rt}(w), \rA \rx\rangle \leq \gamma_i/2 \wedge \ry \langle w, \rx\rangle > 0]] &\geq \E_{(\rx, \ry) \sim \cD}[\rho(\ry\langle w, \rx \rangle)].
\end{align*}
\end{customrmk}
In the proof, we will need the following monotonicity properties 
\begin{claim}
\label{clm:monotone1}
We have $\Pr_{\rA,\rt}[y \ipr{h_{\rA,\rt}(w),\rA x} > \gamma_i/2 \mid y \ipr{w,x}=\alpha_1] \leq \Pr_{\rA,\rt}[y \ipr{h_{\rA,\rt}(w),\rA x} > \gamma_i/2 \mid y \ipr{w,x}=\alpha_2]$ for any $0 \leq \alpha_1 \leq \alpha_2 \leq \gamma_i$.
\end{claim}
\begin{claim}
\label{clm:monotone2}
We have $\Pr_{\rA,\rt}[y \ipr{h_{\rA,\rt}(w),\rA x} \leq \gamma_i/2 \mid y \ipr{w,x}=\alpha_2] \leq \Pr_{\rA,\rt}[y \ipr{h_{\rA,\rt}(w),\rA x} \leq \gamma_i/2 \mid y \ipr{w,x}=\alpha_1]$ for any $0 < \alpha_1 \leq \alpha_2 \leq \gamma_i$.
\end{claim}
First we will prove Remark~\ref{rmk:phirho} using the two claims. Afterward, we will prove Claim~\ref{clm:monotone1} and Claim~\ref{clm:monotone2}.

\begin{proof}[Proof of Remark~\ref{rmk:phirho}]
For convenience, let us recall the definitions of $\phi$ and $\rho$:
\[
\phi(\alpha) = \begin{cases} \Pr_{\rA, \rt}[y \langle h_{\rA,\rt}(w), \rA x\rangle > \gamma_i/2 \mid y \langle w, x \rangle = \alpha] & \text{if } -c_\gamma \leq \alpha \leq 0 \\
                      \frac{(\gamma_i-\alpha)}{\gamma_i}\Pr_{\rA, \rt}[y \langle h_{\rA,\rt}(w), \rA x\rangle > \gamma_i/2 \mid y \langle w, x \rangle = 0]                                    & \text{if } 0 < \alpha \leq \gamma_i      \\
                      0 & \text{if } \gamma_i < \alpha \leq c_\gamma
        \end{cases}
\]
\[
\rho(\alpha) = \begin{cases} \Pr_{\rA, \rt}[y \langle h_{\rA,\rt}(w), \rA x\rangle \leq \gamma_i/2 \mid y \langle w, x \rangle = \alpha] & \text{if } \gamma_i < \alpha \leq c_\gamma \\
                      \frac{\alpha}{\gamma_i}\Pr_{\rA, \rt}[y \langle h_{\rA,\rt}(w), \rA x\rangle \leq \gamma_i/2 \mid y \langle w, x \rangle = \gamma_i]                                    & \text{if } 0 < \alpha \leq \gamma_i      \\
                      0 & \text{if } -c_\gamma \leq \alpha \leq 0
        \end{cases}
\]
We handle each of the inequalities in turn. First we see that
\begin{align*}
\E_{\rA,\rt} [\Pr_{(\rx,\ry) \sim \cD}[\ry \langle h_{\rA,\rt}(w), \rA \rx\rangle > \gamma_i/2 \wedge \ry \langle w, \rx\rangle \leq 0] &= \\
\E_{(\rx,\ry) \sim \cD}[\Pr_{\rA,\rt} [\ry \langle h_{\rA,\rt}(w), \rA \rx\rangle > \gamma_i/2 \wedge \ry \langle w, \rx\rangle \leq 0]] 
&\leq \\
\E_{(\rx,\ry) \sim \cD}[\phi(\ry \langle w, \rx \rangle)].
\end{align*}
Here the inequality follows from the observations that $\phi(y \langle w, x \rangle) \geq 0$ for $y \langle w, x \rangle > 0$, whereas $\Pr_{\rA,\rt} [\ry \langle h_{\rA,\rt}(w), \rA \rx\rangle > \gamma_i/2 \wedge \ry \langle w, \rx\rangle \leq 0] = 0$ for such $y \langle w, x \rangle$. Similarly for $y\langle w, x \rangle = \alpha \leq 0$, we have $\phi(y \langle w, x \rangle) = \Pr_{\rA,\rt}[y\ipr{h_{\rA,\rt}(w),\rA x} > \gamma_i/2 \mid y\langle w, x \rangle = \alpha] = \Pr_{\rA,\rt}[y\ipr{h_{\rA,\rt}(w),\rA x} > \gamma_i/2 \wedge y \ipr{w,x} \leq 0 \mid y\langle w, x \rangle = \alpha]$.

Similarly, we have
\begin{align*}
\E_{\rA,\rt} [\Pr_{(\rx,\ry) \sim S}[\ry \langle h_{\rA,\rt}(w), \rA \rx\rangle > \gamma_i/2 \wedge \ry \langle w, \rx\rangle \leq \gamma] &= \\
\E_{(\rx,\ry) \sim S}[\Pr_{\rA,\rt} [\ry \langle h_{\rA,\rt}(w), \rA x\rangle > \gamma_i/2 \wedge \ry \langle w, \rx\rangle \leq \gamma]] &\geq \\
\E_{(\rx,\ry) \sim S}[\Pr_{\rA,\rt} [\ry \langle h_{\rA,\rt}(w), \rA \rx\rangle > \gamma_i/2 \wedge \ry \langle w, \rx\rangle \leq \gamma_{i}]] &\geq \\
\E_{(\rx,\ry) \sim S}[\phi(\ry \langle w, \rx \rangle)].
\end{align*}
The last inequality follows by observing that if $y \langle w, x \rangle > \gamma_i$, we have $\phi(y \ipr{w,x})=0$ and $\Pr_{\rA,\rt}[y \ipr{h_{\rA,\rt}(w),\rA x} > \gamma_i/2 \wedge y \ipr{w,x} \leq \gamma_i]=0$. For $\alpha = y \ipr{w,x}$ with $0 < \alpha \leq \gamma_i$, we have $\phi(\alpha)=\tfrac{\gamma_i-\alpha}{\gamma_i}\Pr_{\rA,\rt}[y\ipr{h_{\rA,\rt}(w),\rA x } > \gamma_i/2 \mid y \ipr{w,x}=0] \leq \Pr_{\rA,\rt}[y\ipr{h_{\rA,\rt}(w),\rA x } > \gamma_i/2 \mid y \ipr{w,x}=\alpha] = \Pr_{\rA,\rt}[y\ipr{h_{\rA,\rt}(w),\rA x } > \gamma_i/2 \wedge y\ipr{w,x}\leq \gamma_i \mid y \ipr{w,x}=\alpha]$. This uses the monotonicity in Claim~\ref{clm:monotone1}. Finally for $y\ipr{w,x} = \alpha \leq 0$, the two coincide as in the above argument.

Symmetric arguments for $\rho$ gives
\begin{align*}
\E_{\rA,\rt} [\Pr_{(\rx,\ry) \sim S}[\ry \langle h_{\rA,\rt}(w), \rA \rx\rangle \leq \gamma_i/2 \wedge \ry \langle w, \rx\rangle > \gamma] &= \\
\E_{(\rx,\ry) \sim S} [\Pr_{\rA,\rt}[\ry \langle h_{\rA,\rt}(w), \rA \rx\rangle \leq \gamma_i/2 \wedge \ry \langle w, \rx\rangle > \gamma] &\leq \\
\E_{(\rx,\ry) \sim S} [\Pr_{\rA,\rt}[\ry \langle h_{\rA,\rt}(w), \rA \rx\rangle \leq \gamma_i/2 \wedge \ry \langle w, \rx\rangle > \gamma_i] &\leq \\
\E_{(\rx, \ry) \sim S}[\rho(\ry \langle w, \rx \rangle)].
\end{align*}
Here the last inequality follows from the following considerations. For $y \ipr{w,x}=\alpha$ with $\alpha \leq \gamma_i$, we have that $\Pr_{\rA,\rt}[y \langle h_{\rA,\rt}(w), \rA x\rangle \leq \gamma_i/2 \wedge y \langle w, x\rangle > \gamma_i] = 0$ and $\rho$ is always non-negative. For $\alpha > \gamma_i$, we have $\rho(\alpha) = \Pr_{\rA,\rt}[y \langle h_{\rA,\rt}(w), \rA x\rangle \leq \gamma_i/2 \mid y \langle w, x\rangle = \alpha] = \Pr_{\rA,\rt}[y \langle h_{\rA,\rt}(w), \rA x\rangle \leq \gamma_i/2 \wedge y \ipr{w,x} > \gamma_i \mid y \langle w, x\rangle = \alpha]$ and the two coincide.

Finally, we have
\begin{align*}
\E_{\rA,\rt} [\Pr_{(\rx,\ry) \sim \cD}[\ry \langle h_{\rA,\rt}(w), \rA \rx\rangle \leq \gamma_i/2 \wedge \ry \langle w, \rx\rangle > 0] &= \\
\E_{(\rx,\ry) \sim \cD} [\Pr_{\rA,\rt}[\ry \langle h_{\rA,\rt}(w), \rA \rx\rangle \leq \gamma_i/2 \wedge \ry \langle w, \rx\rangle > 0] &\geq \\
\E_{(\rx, \ry) \sim \cD}[\rho(\ry\langle w, \rx \rangle)].
\end{align*}
Here the inequality follows by observing that for $y\ipr{w,x} = \alpha$ with $\alpha \leq 0$, both $\rho(\alpha)$ and $\Pr_{\rA,\rt}[y \langle h_{\rA,\rt}(w), \rA x\rangle \leq \gamma_i/2 \wedge y \langle w, x\rangle > 0]$ are $0$. For $0 \leq \alpha \leq \gamma_i$ we have by definition that $\rho(\alpha) = \tfrac{\alpha}{\gamma_i} \Pr_{\rA,\rt}[y\ipr{h_{\rA,\rt}(w),\rA x} \leq \gamma_i/2 \mid y\ipr{w,x} = \gamma_i] \leq \Pr_{\rA,\rt}[y\ipr{h_{\rA,\rt}(w),\rA x} \leq \gamma_i/2 \mid y\ipr{w,x} = \alpha] = \Pr_{\rA,\rt}[y\ipr{h_{\rA,\rt}(w),\rA x} \leq \gamma_i/2 \wedge y \ipr{w,x}>0 \mid y\ipr{w,x} = \alpha]$, where we used that $\Pr_{\rA,\rt}[y\ipr{h_{\rA,\rt}(w),\rA x} \leq \gamma_i/2 \mid y\ipr{w,x} = \alpha]$ is decreasing in $\alpha$ (as stated in Claim~\ref{clm:monotone2}). Finally, for $\alpha > \gamma_i$, the two coincide as above.
\end{proof}

\begin{proof}[Proof of Claim~\ref{clm:monotone1}]
Let $w_1,x_1,y_1$ be such that $\alpha_1 := y_1\langle w_1,x_1\rangle$ and let $w_2,x_2,y_2$ be such that $\alpha_2 := y_2 \ipr{w_2,x_2}$. Consider sampling $\rX_i,\rY_i \sim \Norm(0,1/k)$ independently. Also sample offsets $\rt_1',\dots,\rt_k'$ uniformly and independently in $[0,1]$ and let $\rX'_i$ be $\rX_i$ rounded based on $\rt_i'$ as above. Let $\rZ_1 = \rY =\alpha_1 \rX + \sqrt{1-\alpha_1^2}\rY$ and $\rZ_2 = \alpha_2 \rX + \sqrt{1-\alpha_2^2}\rY$. Then the marginal distribution of $\langle \rX', \rZ_j\rangle$ equals the distribution of $\ipr{h_{\rA,\rt}(w_j), y_j \rA x_j} = y_j\ipr{h_{\rA,\rt}(w_j), \rA x_j} $. 

Consider now an arbitrary outcome $X',X$ of $\rX, \rX'$. We have $\langle \rZ_j , X' \rangle \geq \gamma_i/2$ if and only if $\alpha_j \langle X, X'\rangle + \sqrt{1-\alpha_j^2} \langle \rY, X'\rangle \geq \gamma_i/2$. We also have that $\langle \rY, X'\rangle \sim \Norm(0,\|X'\|_2^2/k)$ and thus
\begin{align}
\Pr[\ipr{\rZ_2,X'}\geq \gamma_i/2] - \Pr[\ipr{\rZ_1,X'}\geq \gamma_i/2] &= \nonumber\\
\left(1-\Phi\left(\sqrt{k} \cdot \frac{\gamma_i/2-\alpha_2\ipr{X,X'}}{\sqrt{1-\alpha_2^2}}\right)\right)-\left(1-\Phi\left(\sqrt{k} \cdot \frac{\gamma_i/2-\alpha_1\ipr{X,X'}}{\sqrt{1-\alpha_1^2}}\right)\right)&= \nonumber\\
\Phi\left(\sqrt{k} \cdot \frac{\gamma_i/2-\alpha_1\ipr{X,X'}}{\sqrt{1-\alpha_1^2}}\right)-\Phi\left(\sqrt{k} \cdot \frac{\gamma_i/2-\alpha_2\ipr{X,X'}}{\sqrt{1-\alpha_2^2}}\right). \label{eq:diffcdf}
\end{align}
Here $\Phi(\cdot)$ denotes the cumulative density function of the normal distribution with mean $0$ and variance $1$. Now let
\[
u := \sqrt{k} \cdot \frac{\gamma_i/2-\alpha_1\ipr{X,X'}}{\sqrt{1-\alpha_1^2}}.
\]
and
\[
\ell := \sqrt{k} \cdot \frac{\gamma_i/2-\alpha_2\ipr{X,X'}}{\sqrt{1-\alpha_2^2}}
\]
Consider now the derivative
\begin{align*}
    \frac{\partial}{\partial \alpha} \sqrt{k} \cdot \frac{\gamma_i/2-\alpha\ipr{X,X'}}{\sqrt{1-\alpha^2}} &= \sqrt{k} \cdot
    \frac{\alpha \gamma_i/2 - \ipr{X,X'}}{(1-\alpha^2)^{3/2}}.
\end{align*}
Assume first that $\|X\|_2^2 \geq 9/10$. Then $\ipr{X,X'} \geq 8/9$ by Remark~\ref{rmk:ipXX'}. Now since $\alpha \gamma_i/2 \leq \gamma_i^2/2 \leq c_\gamma^2/8 \leq 1/9$ for $c_\gamma$ small enough. Thus the derivative when $\|X\|_2^2 \geq 9/10$ is no more than
\begin{align*}
    \sqrt{k} \cdot (1/9 - 8/9) \leq -7\sqrt{k}/9.
\end{align*}
This implies $u - \ell \geq 7(\alpha_2-\alpha_1) \sqrt{k}/9 > 0$ and therefore 
\[
\Pr[\ipr{\rZ_2,X'}\geq \gamma_i/2] \geq \Pr[\ipr{\rZ_1,X'}\geq \gamma_i/2].
\]
If we in addition have that $\|X\|_2^2 \leq 4/3$, then we may even show that the difference in probabilities is large as a function of $\alpha_2-\alpha_1$ as follows
\begin{align*}
    \Pr[\ipr{\rZ_2,X'}\geq \gamma_i/2] - \Pr[\ipr{\rZ_1,X'}\geq \gamma_i/2] &=\frac{1}{\sqrt{2 \pi}} \cdot \int_{x=\ell}^u e^{-x^2/2} dx \\
    &\geq e^{-\max_{a \in [\ell ,u]} a^2/2} \frac{7 \sqrt{k} (\alpha_2-\alpha_1)}{9 \sqrt{2 \pi}}.
\end{align*}
Observing that
\[
\max_{a \in [\ell, u]} a^2 \leq \frac{k}{1-c_\gamma^2} \cdot \max\{\gamma^2_i/2, \gamma_i^2 \ipr{X,X'}^2\}
\]
we use Remark~\ref{rmk:ipXX'} to conclude $\ipr{X,X'} \leq (10/9)\|X\|_2^2$ and thus $u^2 \leq 2k \gamma_i^2 (10/9)^2 \leq 3k \gamma_i^2$ for $c_\gamma \leq 1/\sqrt{2}$. This gives us that for any $X$ with $9/10 \leq \|X\|_2^2 \leq 4/3$, it holds that
\begin{align*}
    \Pr[\ipr{\rZ_2,X'}\geq \gamma_i/2] - \Pr[\ipr{\rZ_1,X'}\geq \gamma_i/2] &\geq e^{-3k \gamma_i^2/2} \frac{7 \sqrt{k} (\alpha_2-\alpha_1)}{9 \sqrt{2 \pi}}.
\end{align*}
For $\|X\|_2^2 < 9/10$, we have $\|X'\|_2 = \|X'-X + X\|_2 \leq \|X'-X\|_2 + \|X\|_2 \leq \sqrt{k (10 \sqrt{k})^{-2}} + \sqrt{9/10} \leq 11/10$. It follows by Cauchy-Schwartz that $|\ipr{X,X'}| \leq \|X\|_2 \cdot \|X'\|_2 \leq \sqrt{9/10} \cdot 11/10 \leq 11/10$. For $0 \leq \alpha \leq \gamma_i \leq c_\gamma/2 \leq 1/\sqrt{8}$ for $c_\gamma \leq 1/\sqrt{2}$, this upper bounds the derivative by
\[
\sqrt{k} \cdot \frac{\gamma_i^2/2 + 11/10}{(1-1/8)^{3/2}} < 2 \sqrt{k}.
\]
If $u \geq \ell$, we already have that 
\[
\Pr[\ipr{\rZ^2,X'}\geq \gamma_i/2] - \Pr[\ipr{\rZ^1,X'}\geq \gamma_i/2] \geq 0
\]
So assume $u < \ell$. The bound on the derivative gives us that $\ell-u \leq 2 \sqrt{k} (\alpha_2-\alpha_1)$ and we conclude
\begin{align*}
    \Pr[\ipr{\rZ^2,X'}\geq \gamma_i/2] - \Pr[\ipr{\rZ^1,X'}\geq \gamma_i/2] &=-\frac{1}{\sqrt{2 \pi}} \cdot \int_{x=u}^\ell e^{-x^2/2} dx \\
    &\geq -e^{-\min_{a \in [u,\ell]} a^2/2} \cdot \frac{2 \sqrt{k} (\alpha_2-\alpha_1)}{\sqrt{2 \pi}} \\
    &\geq -\frac{2 \sqrt{k} (\alpha_2-\alpha_1)}{\sqrt{2 \pi}}.
\end{align*}
We finally conclude
\begin{align}
    \Pr[\ipr{\rZ^2,X'}\geq \gamma_i/2] - \Pr[\ipr{\rZ^1,X'}\geq \gamma_i/2] &\geq \nonumber\\
    \Pr[9/10 \leq \|\rX\|_2^2 \leq 4/3] \cdot e^{- 3k\gamma_i^2/2} \cdot \frac{7 \sqrt{k}(\alpha_2-\alpha_1)}{9 \sqrt{2 \pi}} - \Pr[\|\rX\|_2^2 < 9/10] \cdot \frac{2 \sqrt{k} (\alpha_2-\alpha_1)}{\sqrt{2 \pi}}.\label{eq:propdiff}
\end{align}
Using Theorem~\ref{thm:chibound}, we get
\begin{align}
\Pr[9/10 \leq \|\rX\|_2^2 \leq 4/3] \geq 1-2\exp(-k/800), \label{eq:likelyCase}
\end{align}
and
\[
\Pr[\|\rX\|_2^2 < 9/10] \leq 2\exp(-k/800).
\]
For $k$ at least a sufficiently large constant, we have that~\eqref{eq:likelyCase} is at least $1/2$ and we get that~\eqref{eq:propdiff} is at least
\begin{align*}
    e^{- 3k\gamma_i^2/2} \cdot \frac{7 \sqrt{k}(\alpha_2-\alpha_1)}{18 \sqrt{2 \pi}} - e^{-k/800} \cdot \frac{4 \sqrt{k} (\alpha_2-\alpha_1)}{\sqrt{2 \pi}}.
\end{align*}
For the constant $c_\gamma$ sufficiently small, this is positive as $\gamma_i \leq c_\gamma$.
\end{proof}

\begin{proof}[Proof of Claim~\ref{clm:monotone2}]
Similarly to the proof of Claim~\ref{clm:monotone1}, let $w_1,x_1,y_1$ by such that $\alpha_1 = y_1 \ipr{w_1,x_1}$ and let $w_2,x_2,y_2$ be such that $\alpha_2 = y_2 \ipr{w_2,x_2}$. Draw $\rX$, $\rX'$ and $\rZ_1,\rZ_2$ as above. Consider again an arbitrary outcome $X',X$ of $\rX,\rX'$. We have $\ipr{\rZ_j,X'} \leq \gamma_i/2$ if and only if $\alpha_j \ipr{X,X'} + \sqrt{1-\alpha_j^2} \ipr{\rY,X'} \leq \gamma_i/2$. Hence
\begin{align*}
    \Pr[\ipr{\rZ_1,X'} \leq \gamma_i/2] - \Pr[\ipr{\rZ_2,X'} \leq \gamma_i/2] &=\\
\Phi\left(\sqrt{k} \cdot \frac{\gamma_i/2-\alpha_1\ipr{X,X'}}{\sqrt{1-\alpha_1^2}}\right)-\Phi\left(\sqrt{k} \cdot \frac{\gamma_i/2-\alpha_2\ipr{X,X'}}{\sqrt{1-\alpha_2^2}}\right)
\end{align*}
This has the exact same constraints $0 \leq \alpha_1 \leq \alpha_2 \leq \gamma_i$ and exact same form as~\eqref{eq:diffcdf}. The conclusion thus follows from the proof of Claim~\ref{clm:monotone1}.
\end{proof}

\begin{customrmk}{\ref{rem:case_ii_norm_bound}}
   If $\|X\|_2^2 \leq 4/3$, then $\norm{X'}_2^2 < 2$.
\end{customrmk}
%the following are proofs used in subsection _  to bound
\begin{proof}
By the triangle inequality, and using that all coordinates of $X-X'$ are bounded by $(10 \sqrt{k})^{-1}$ in absolute value, we have
\begin{align*}
    \norm{X'}_2^2 &= \norm{X'-X+X}_2^2 \\
    &\leq \left(\|X'-X\|_2 + \|X\|_2\right)^2 \\
    &\leq \left(\sqrt{k (10 \sqrt{k})^{-2}} + \sqrt{4/3}\right)^2 \\
    &= (1/10 + \sqrt{4/3})^2 \\
    &< 2.
\end{align*}
\end{proof}

\begin{customrmk}{\ref{rmk:ipXX'}}
    If $\norm{X}_2^2\ge 9/10$, then $(8/9)\norm{X}_2^2\le \ipr{X,X'}\le (10/9)\norm{X}_2^2$
\end{customrmk}

\begin{proof}
We have:
\begin{align*}
    \ipr{X',X} &= \ipr{X'-X + X,X} \\
    &= \ipr{X'-X,X} + \|X\|_2^2.
\end{align*}
Since each coordinate of $X'-X$ is bounded by $(10 \sqrt{k})^{-1}$ in absolute value, it follows by Cauchy-Schwartz that
\begin{align*}
    |\ipr{X'-X,X}| &\leq \|X'-X\|_2 \cdot \|X\|_2 \\
    &\leq \sqrt{k (10 \sqrt{k})^{-2}} \cdot \frac{\|X\|^2_2}{\|X\|_2} \\
    &\leq \frac{\|X\|_2^2}{10 \sqrt{9/10}} \\
    &\leq \|X\|_2^2/9.
\end{align*}
The conclusion follows.
\end{proof}

\begin{customrmk}{\ref{rmk:concentrationx}}
For any distribution $\cD$ over $\finalX \times \{-1,1\}$, fixed $w \in \finalH$, margin $\gamma$ and any $A \in \R^{k \times d}$, it holds with probability at least $1-\delta$ over $\rS \sim \cD^n$ that
\[
|\Loss_{A\cD}^{\gamma}(w) - \Loss_{A\rS}^{\gamma}(w)| \leq \sqrt{\frac{8\Loss_{A \cD}^{\gamma}(w)\ln(1/\delta)}{n}} + \frac{2 \ln(1/\delta)}{n}.
\]
\end{customrmk}
\begin{proof}
Since $\Loss^\gamma_{A \rS}(w)$ is an average of $n$ i.i.d.\ $0/1$ random variables with mean $\Loss^\gamma_{A \cD}(w)$, we get from Bernstein's inequality that 
\begin{align*}
    \Pr_{\rS \sim \cD}\left[|\Loss_{A\cD}^{\gamma}(w) - \Loss_{A\rS}^{\gamma}(w)| >\sqrt{\frac{8\Loss_{A \cD}^{\gamma}(w)\ln(1/\delta)}{n}} + \frac{2 \ln(1/\delta)}{n} \right] &\leq \\
    \exp\left(- \frac{\tfrac{1}{2} \cdot \left(\sqrt{ 8\Loss_{A \cD}^{\gamma}(w) n \ln(1/\delta) } + 2 \ln(1/\delta)\right)^2}{n \Loss_{A \cD}^\gamma(w) + \tfrac{1}{3} \cdot \left(\sqrt{ 8\Loss_{A \cD}^{\gamma}(w)\ln(1/\delta) n} + 2 \ln(1/\delta)\right) }\right) &\leq\\
    \exp\left(- \frac{\tfrac{1}{2} \cdot \max\left\{8\Loss_{A \cD}^{\gamma}(w) n,  4 \ln(1/\delta)\right\} \ln(2/\delta)}{\tfrac{1}{8} \max\{n \Loss_{A \cD}^\gamma(w), 4\ln(1/\delta)\} + \tfrac{1}{3} \cdot \sqrt{2 \cdot \max\{ 8\Loss_{A \cD}^{\gamma}(w), 4 \ln(1/\delta)\} \cdot \ln(1/\delta)}}\right).
\end{align*}
Using that $\ln(1/\delta) \leq \tfrac{1}{4} \max\{ 8\Loss_{A \cD}^{\gamma}(w), 4 \ln(1/\delta)\}$, this is at most
\[
    \exp\left(- \frac{\tfrac{1}{2} \ln(1/\delta)}{\tfrac{1}{8}  + \tfrac{1}{3} \cdot \sqrt{\tfrac{1}{2} }}\right) \leq \exp(-\ln(1/\delta)) = \delta.
\]
\end{proof}