\vspace{-2mm}
\section{Introduction}
Medical time series data, such as Electroencephalography (EEG) and Electrocardiography (ECG) signals, has been playing a vital role in real-world healthcare systems by providing valuable information in monitoring health conditions of patients. 
EEG signals, which measure the electrical activity of the brain, are widely used to diagnose and monitor various neurological disorders, including epilepsy, Alzheimer's disease, and sleep disorders \cite{cohen2017does}. Similarly, ECG signals, which record the electrical activity of the heart, are essential for diagnosing and monitoring cardiovascular diseases, such as arrhythmias, myocardial infarction, and congestive heart failure \cite{berkaya2018survey}. Classifying these medical time series is of paramount importance as it could enable the early detection of abnormalities, accurate diagnosis, and personalized treatment. By identifying patterns and features indicative of specific conditions, medical time series classification can assist clinicians in making timely diagnoses \cite{ismail2019deep} and facilitate adapting treatment plans accordingly, potentially leading to improved patient outcomes and quality of life.

Traditionally, medical time series classification has been primarily relied on handcrafted feature extraction, which often involve domain expertise to identify relevant features from the raw data. For instance, for ECG analysis, many features such as R-peak amplitude and heart rate variability would be mannually extracted~\cite{berkaya2018survey}.  %in EEG analysis, other features including band power, spectral entropy, and fractal dimensions are commonly used. 
Later, statistical methods have also been applied to medical time series classification: the autoregressive models, hidden Markov models, and Gaussian mixture models have been used to capture the temporal dependencies and dynamics in ECG and EEG signals~\cite{schaffer2021interrupted,vincent2010spatially,turner2020design}. Though statistical methods can provide robust results and handle uncertainty, they often make strong assumptions about the data distribution and may struggle with complex, non-linear patterns. 
With the advent of artificial intelligence, various deep learning methods have been applied to medical time series classification~\cite{morid2023time}: convolution neural networks have been particularly successful in learning representations directly from raw time series, such as EEGNet~\cite{eegnet_2016}; Transformer-based methods have been applied into the medical time series classification~\cite{medformer_2024}. In addition, graph neural network has also been adopted for multivariate time series classification~\cite{ZhaLZH22,YounisHA24}.


However, these methods often fail to fully model the complex spatial (channel) dynamics under different scales, which ignore the dynamic multi-resolution spatial and temporal joint interdependencies. Moreover, most of them are usually for general classification, without considering the special problem, such as baseline wander, as well as the multi-view characteristics in medical time series, which largely hinders their prediction performance.
To address these limitations, we aim to propose a novel framework to learn the multi-scale and multi-view representations for medical time series classification. %, in order to capture the intricate patterns and dependencies at different temporal and spatial resolutions. 
However, several challenges arise in achieving this goal: 
(i) \textit{{how to model the dynamic spatial structures between different time series channels with multiple resolutions?}} Since the medical time series usually includes multiple channels, the dynamic spatial dependencies keep changing with the resolutions or scales of time series, which need to be properly modeled for accurate classification;
(ii) \textit{how to learn the the multi-view characteristics of medical time series while addressing the baseline wander problem?} The baseline wander problem~\cite{blanco2008ecg}, i.e., the constant offsets or slow drifts towards baseline measurements of medical series would always hinders the models in learning the key patterns and fluctuations; meanwhile the multi-view characteristics of medical time series based on the features from both the time domain and the frequency domain are usually ignored, largely hindering the classification.


To tackle these challenges, we introduce a Multi-resolution Spatiotemporal Graph Learning framework, \textit{{MedGNN}}.
% a Multi-resolution Spatiotemporal Graph Learning (MedGNN) framework.
% Multi-resolution Spatiotemporal Adaptive Graph Learning (MSAGL) framework. 
Specifically, in our MedGNN framework, we first propose to construct multi-resolution adaptive graph structures to learn dynamic spatial temporal representations, where we utilize different kernels of convolutions to extract multi-scale medical time series embeddings to cover the local and global patterns. We construct multi-resolution graphs based on the learned embeddings to model the dynamic spatial dependencies among different channels; the graph structures are adaptively learned to reflect the changing correlations at different resolutions. 
Then, to address the baseline wander problem and learn the multi-view characteristics, we propose two novel networks for temporal modeling: (i) the Difference Attention Networks focus on the temporal changes in the medical time series, which operates self-attention machenisms on the finite difference (e.g., first-order difference) along the temporal dimension, targeting capturing key temporal patterns while mitigating the impact of baseline wander; (ii) the Frequency Convolution Networks captures complementary information in the frequency domain by applying Fourier transform and frequency-domain convolutions~\cite{yi2024frequency, yi2024filternet}, providing a multi-view perspective of the temporal dynamics for medical time series.
In addition, to learn the complicated multi-resolution spatiotemporal graph representations, we utilize the Multi-resolution Graph Transformer architecture to model the dynamic spatial dependencies and fuse the information from different resolutions. Our main contributions are mainly as follows:
\vspace{-0mm}
\begin{itemize}[leftmargin=*]
    \item We propose a novel approach for medical time series classification to capture the complex multi-view spatiotemporal dependencies and multi-scale dynamics of medical time series through multi-resolution learning.
    \item We construct adaptive graph structures at different resolutions to model spatial correlations among time series channels and utilize a Multi-resolution Graph Transformer architecture for the resolution learning and information fusion.
    \item We propose Difference Attention Network and Frequency Convolution Network, for temporal modeling to overcome baseline wander problem of medical time series and meanwhile capture multi-view characteristics from the time and frequency domain.
    \item We have conducted extensive experiments on multiple medical time series datasets, including both ECG and EEG signals, to demonstrate the superior performance of our proposed framework compared to state-of-the-art methods, highlighting its great potential for the real-world clinical applications.
\end{itemize}

