\vspace{-2mm}
\section{Problem Formulation}
% In this section, we present some important backgrounds and definitions, and then outline the problem statement with notations.




% \subsection{Problem Formulation}

% Given a collection of medical time series from $N$ participants denoted by $\{P_1, \cdots, P_N\}$,  each participant has multiple samples of collected series $\{\mathbf{X}^{1}_{P_n},  \cdots, \mathbf{X}^{k_n}_{P_n}  \}  $ where $k_n$ is the number of samples for $n$-th participant.

Given a collection of medical time series from $N$ participants denoted by $\{P_1, \cdots, P_N \}$, each participant has multiple samples of collected series $\{\mathbf{X}^{1}_{P_n}, \cdots, \mathbf{X}^{k_n}_{P_n}\}$, where $k_n$ is the number of samples for the $n$-th participant. Each data sample $\mathbf{X}^{i}_{P_n} \in \mathbb{R}^{T \times C}$ represents the collected multivariate medical time series (e.g., multi-lead ECG) in one participation, where $T$ denotes the number of timestamps and $C$ is the number of channels; the corresponding label for sample $\mathbf{X}^{i}_{P_n}$ is represented as $y^{i}_{P_n} \in \{0, 1\}$ for binary medical classification problems where  0 indicates a healthy participant and 1 indicates a participant diagnosed with a specific disease, or represented as $y^{i}_{P_n} \in \{1,2, \cdots, c\}$ for multi-class medical classification problems where each class corresponds to one kind of diseases or conditions.
The objective of medical time series classification problem is to learn a mapping function %$f: { \mathbf{X}^{i}_{P_n}} \rightarrow y^i_{P_n}$
$f: \mathbb{R}^{T*C} \rightarrow \mathbb{R}^1$
that can accurately predict the label based on medical time series samples in each participation. Formally, given a participant's time series sample $\mathbf{X}^{i}_{P_n}$, the goal is to predict the corresponding label $y^i_{P_n}$ to indicate the disease or condition, which can be written as:
\begin{equation}
\hat{y}^i_{P_n} = f(\mathbf{X}^{i}_{P_n}; \theta),
\end{equation}
where $\hat{y}^i_{P_n}$ is label for $i$-th sample for $n$-th participant; the mapping function $f$ is parameterized by the learnable parameters $\theta$.
% The performance of the learned mapping function $f$ can be evaluated using various metrics, such as accuracy, precision, recall, F1-score, and area under the receiver operating characteristic curve (AUROC). The goal is to maximize these metrics on a held-out test set to ensure the generalization ability of the learned model.
% In summary, the problem of medical time series classification can be formulated as learning a mapping function that can accurately predict the label (healthy or diseased) of a participant based on their multiple time series samples, where each sample is represented as a 2-D array with dimensions corresponding to the number of timestamps and channels. The key challenge lies in effectively leveraging the information from multiple samples per participant to make accurate predictions.

To ensure the real-world clinical utility of the model, it is important to evaluate the generalization ability on unseen participants. Thus we split the dataset based on participants: specifically, we let $\mathcal{P}{{train}}$, $\mathcal{P}{{val}}$, and $\mathcal{P}{{test}}$ denote the disjoint sets of participants used for training, validation, and testing, respectively. With training and testing are on different participants, the generalization of the developed model can be evaluated on unseen participants or patients, which can simulate a more realistic estimate of its potential performance in real-world clinical settings. 
In addition, since one patient may visit hospitals for tests for many times, we also consider a complementary settings that split the dataset into training, validation and test data only relying on individual samples that can be represented by $\mathcal{X}_{train}, \mathcal{X}_{val}$ and $\mathcal{X}_{test}$ as disjoint sets of samples.




% \subsection{Backgrounds and Definitions}

