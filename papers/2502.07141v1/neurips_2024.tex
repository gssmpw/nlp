\documentclass{article}


% if you need to pass options to natbib, use, e.g.:
\PassOptionsToPackage{numbers, compress}{natbib}
% before loading neurips_2024


% ready for submission
%\usepackage{neurips_2024}


% to compile a preprint version, e.g., for submission to arXiv, add add the
% [preprint] option:
%     \usepackage[preprint]{neurips_2024}


% to compile a camera-ready version, add the [final] option, e.g.:
\usepackage[final]{neurips_2024}


% to avoid loading the natbib package, add option nonatbib:
%    \usepackage[nonatbib]{neurips_2024}


\usepackage[utf8]{inputenc} % allow utf-8 input
\usepackage[T1]{fontenc}    % use 8-bit T1 fonts
\usepackage{hyperref}       % hyperlinks
\usepackage{url}            % simple URL typesetting
\usepackage{booktabs}       % professional-quality tables
\usepackage{amsfonts}       % blackboard math symbols
\usepackage{nicefrac}       % compact symbols for 1/2, etc.
\usepackage{microtype}      % microtypography
\usepackage{xcolor}         % colors


%\title{Any Learning Rate Succeeds for Stochastic Gradient Bandit}
%\title{Stochastic Gradient Bandit Succeeds with Any Learning Rate}
%\title{The Stochastic Gradient Bandit Algorithm Succeeds with Any Learning Rate}
\title{Small steps no more: Global convergence of stochastic gradient bandits for arbitrary learning rates}


% The \author macro works with any number of authors. There are two commands
% used to separate the names and addresses of multiple authors: \And and \AND.
%
% Using \And between authors leaves it to LaTeX to determine where to break the
% lines. Using \AND forces a line break at that point. So, if LaTeX puts 3 of 4
% authors names on the first line, and the last on the second line, try using
% \AND instead of \And before the third author name.


\author{%
  Jincheng Mei$^{\, 1}$
  \hspace{5mm}
  Bo Dai$^{\, 1 \, 3}$ 
  \hspace{5mm}
  Alekh Agarwal$^{\, 2}$
  \hspace{5mm}
  \textbf{Sharan Vaswani}$^{\, 5}$
  \hspace{5mm}
  \textbf{Anant Raj}$^{\, 6}$ \\
  \hspace{7mm}
  \textbf{Csaba Szepesv{\'a}ri}$^{\, 1 \, 4}$
  \hspace{7mm}
  \textbf{Dale Schuurmans}$^{\, 1 \, 4}$ \\
  %\AND
  \\
  \hspace{-7mm} $^1$\normalfont{Google DeepMind} \hspace{2mm} $^2$Google Research \hspace{2mm} $^3$Georgia Institute of Technology \hspace{2mm} $^4$University of Alberta \\ $^5$Simon Fraser University \hspace{2mm} $^6$Indian Institute of Science \\
  \\
  \hspace{-2mm}
  \texttt{\{jcmei,bodai,alekhagarwal,szepi,schuurmans\}@google.com} \\
  \texttt{vaswani.sharan@gmail.com} \hspace{2mm} \texttt{anantraj@iisc.ac.in} 
}

\usepackage{enumitem}


%!TEX root =  neurips_2024.tex
%%%%% NEW MATH DEFINITIONS %%%%%

\usepackage{amsmath,amsfonts,bm}
\usepackage{amssymb,amsthm}
\usepackage{mathtools}
\usepackage{algorithm,algorithmic}
%\usepackage{natbib}
\usepackage{xcolor}
\usepackage{hyperref}
\usepackage{url}
\usepackage{multirow}
\usepackage{array}
\usepackage{cancel}
\usepackage{etoc}
%\usepackage{arev}
\usepackage{pifont}
\usepackage[capitalize]{cleveref}


\renewcommand{\algorithmicrequire}{\textbf{Input:}}
\renewcommand{\algorithmicensure}{\textbf{Output:}}

\crefname{proposition}{Proposition}{Propositions}
\crefname{theorem}{Theorem}{Theorems}
\crefname{lemma}{Lemma}{Lemmas}
\crefname{update_rule}{Update}{Updates}
\crefname{algorithm}{Algorithm}{Algorithms}
\crefname{figure}{Figure}{Figures}

% Mark sections of captions for referring to divisions of figures
\newcommand{\figleft}{{\em (Left)}}
\newcommand{\figcenter}{{\em (Center)}}
\newcommand{\figright}{{\em (Right)}}
\newcommand{\figtop}{{\em (Top)}}
\newcommand{\figbottom}{{\em (Bottom)}}
\newcommand{\captiona}{{\em (a)}}
\newcommand{\captionb}{{\em (b)}}
\newcommand{\captionc}{{\em (c)}}
\newcommand{\captiond}{{\em (d)}}

% Highlight a newly defined term
\newcommand{\newterm}[1]{{\bf #1}}

% Figure reference, lower-case.
\def\figref#1{figure~\ref{#1}}
% Figure reference, capital. For start of sentence
\def\Figref#1{Figure~\ref{#1}}
\def\twofigref#1#2{figures \ref{#1} and \ref{#2}}
\def\quadfigref#1#2#3#4{figures \ref{#1}, \ref{#2}, \ref{#3} and \ref{#4}}
% Section reference, lower-case.
\def\secref#1{section~\ref{#1}}
% Section reference, capital.
\def\Secref#1{Section~\ref{#1}}
% Reference to two sections.
\def\twosecrefs#1#2{sections \ref{#1} and \ref{#2}}
% Reference to three sections.
\def\secrefs#1#2#3{sections \ref{#1}, \ref{#2} and \ref{#3}}
% Reference to an equation, lower-case.
\def\eqref#1{equation~\ref{#1}}
% Reference to an equation, upper case
\def\Eqref#1{Equation~\ref{#1}}
% A raw reference to an equation---avoid using if possible
\def\plaineqref#1{\ref{#1}}
% Reference to a chapter, lower-case.
\def\chapref#1{chapter~\ref{#1}}
% Reference to an equation, upper case.
\def\Chapref#1{Chapter~\ref{#1}}
% Reference to a range of chapters
\def\rangechapref#1#2{chapters\ref{#1}--\ref{#2}}
% Reference to an algorithm, lower-case.
\def\algref#1{algorithm~\ref{#1}}
% Reference to an algorithm, upper case.
\def\Algref#1{Algorithm~\ref{#1}}
\def\twoalgref#1#2{algorithms \ref{#1} and \ref{#2}}
\def\Twoalgref#1#2{Algorithms \ref{#1} and \ref{#2}}
% Reference to a part, lower case
\def\partref#1{part~\ref{#1}}
% Reference to a part, upper case
\def\Partref#1{Part~\ref{#1}}
\def\twopartref#1#2{parts \ref{#1} and \ref{#2}}

\def\ceil#1{\lceil #1 \rceil}
\def\floor#1{\lfloor #1 \rfloor}
\def\1{\bm{1}}
\newcommand{\train}{\mathcal{D}}
\newcommand{\valid}{\mathcal{D_{\mathrm{valid}}}}
\newcommand{\test}{\mathcal{D_{\mathrm{test}}}}

\def\eps{{\epsilon}}


% Random variables
\def\reta{{\textnormal{$\eta$}}}
\def\ra{{\textnormal{a}}}
\def\rb{{\textnormal{b}}}
\def\rc{{\textnormal{c}}}
\def\rd{{\textnormal{d}}}
\def\re{{\textnormal{e}}}
\def\rf{{\textnormal{f}}}
\def\rg{{\textnormal{g}}}
\def\rh{{\textnormal{h}}}
\def\ri{{\textnormal{i}}}
\def\rj{{\textnormal{j}}}
\def\rk{{\textnormal{k}}}
\def\rl{{\textnormal{l}}}
% rm is already a command, just don't name any random variables m
\def\rn{{\textnormal{n}}}
\def\ro{{\textnormal{o}}}
\def\rp{{\textnormal{p}}}
\def\rq{{\textnormal{q}}}
\def\rr{{\textnormal{r}}}
\def\rs{{\textnormal{s}}}
\def\rt{{\textnormal{t}}}
\def\ru{{\textnormal{u}}}
\def\rv{{\textnormal{v}}}
\def\rw{{\textnormal{w}}}
\def\rx{{\textnormal{x}}}
\def\ry{{\textnormal{y}}}
\def\rz{{\textnormal{z}}}

% Random vectors
\def\rvepsilon{{\mathbf{\epsilon}}}
\def\rvtheta{{\mathbf{\theta}}}
\def\rva{{\mathbf{a}}}
\def\rvb{{\mathbf{b}}}
\def\rvc{{\mathbf{c}}}
\def\rvd{{\mathbf{d}}}
\def\rve{{\mathbf{e}}}
\def\rvf{{\mathbf{f}}}
\def\rvg{{\mathbf{g}}}
\def\rvh{{\mathbf{h}}}
\def\rvu{{\mathbf{i}}}
\def\rvj{{\mathbf{j}}}
\def\rvk{{\mathbf{k}}}
\def\rvl{{\mathbf{l}}}
\def\rvm{{\mathbf{m}}}
\def\rvn{{\mathbf{n}}}
\def\rvo{{\mathbf{o}}}
\def\rvp{{\mathbf{p}}}
\def\rvq{{\mathbf{q}}}
\def\rvr{{\mathbf{r}}}
\def\rvs{{\mathbf{s}}}
\def\rvt{{\mathbf{t}}}
\def\rvu{{\mathbf{u}}}
\def\rvv{{\mathbf{v}}}
\def\rvw{{\mathbf{w}}}
\def\rvx{{\mathbf{x}}}
\def\rvy{{\mathbf{y}}}
\def\rvz{{\mathbf{z}}}

% Elements of random vectors
\def\erva{{\textnormal{a}}}
\def\ervb{{\textnormal{b}}}
\def\ervc{{\textnormal{c}}}
\def\ervd{{\textnormal{d}}}
\def\erve{{\textnormal{e}}}
\def\ervf{{\textnormal{f}}}
\def\ervg{{\textnormal{g}}}
\def\ervh{{\textnormal{h}}}
\def\ervi{{\textnormal{i}}}
\def\ervj{{\textnormal{j}}}
\def\ervk{{\textnormal{k}}}
\def\ervl{{\textnormal{l}}}
\def\ervm{{\textnormal{m}}}
\def\ervn{{\textnormal{n}}}
\def\ervo{{\textnormal{o}}}
\def\ervp{{\textnormal{p}}}
\def\ervq{{\textnormal{q}}}
\def\ervr{{\textnormal{r}}}
\def\ervs{{\textnormal{s}}}
\def\ervt{{\textnormal{t}}}
\def\ervu{{\textnormal{u}}}
\def\ervv{{\textnormal{v}}}
\def\ervw{{\textnormal{w}}}
\def\ervx{{\textnormal{x}}}
\def\ervy{{\textnormal{y}}}
\def\ervz{{\textnormal{z}}}

% Random matrices
\def\rmA{{\mathbf{A}}}
\def\rmB{{\mathbf{B}}}
\def\rmC{{\mathbf{C}}}
\def\rmD{{\mathbf{D}}}
\def\rmE{{\mathbf{E}}}
\def\rmF{{\mathbf{F}}}
\def\rmG{{\mathbf{G}}}
\def\rmH{{\mathbf{H}}}
\def\rmI{{\mathbf{I}}}
\def\rmJ{{\mathbf{J}}}
\def\rmK{{\mathbf{K}}}
\def\rmL{{\mathbf{L}}}
\def\rmM{{\mathbf{M}}}
\def\rmN{{\mathbf{N}}}
\def\rmO{{\mathbf{O}}}
\def\rmP{{\mathbf{P}}}
\def\rmQ{{\mathbf{Q}}}
\def\rmR{{\mathbf{R}}}
\def\rmS{{\mathbf{S}}}
\def\rmT{{\mathbf{T}}}
\def\rmU{{\mathbf{U}}}
\def\rmV{{\mathbf{V}}}
\def\rmW{{\mathbf{W}}}
\def\rmX{{\mathbf{X}}}
\def\rmY{{\mathbf{Y}}}
\def\rmZ{{\mathbf{Z}}}

% Elements of random matrices
\def\ermA{{\textnormal{A}}}
\def\ermB{{\textnormal{B}}}
\def\ermC{{\textnormal{C}}}
\def\ermD{{\textnormal{D}}}
\def\ermE{{\textnormal{E}}}
\def\ermF{{\textnormal{F}}}
\def\ermG{{\textnormal{G}}}
\def\ermH{{\textnormal{H}}}
\def\ermI{{\textnormal{I}}}
\def\ermJ{{\textnormal{J}}}
\def\ermK{{\textnormal{K}}}
\def\ermL{{\textnormal{L}}}
\def\ermM{{\textnormal{M}}}
\def\ermN{{\textnormal{N}}}
\def\ermO{{\textnormal{O}}}
\def\ermP{{\textnormal{P}}}
\def\ermQ{{\textnormal{Q}}}
\def\ermR{{\textnormal{R}}}
\def\ermS{{\textnormal{S}}}
\def\ermT{{\textnormal{T}}}
\def\ermU{{\textnormal{U}}}
\def\ermV{{\textnormal{V}}}
\def\ermW{{\textnormal{W}}}
\def\ermX{{\textnormal{X}}}
\def\ermY{{\textnormal{Y}}}
\def\ermZ{{\textnormal{Z}}}

% Vectors
\def\vzero{{\bm{0}}}
\def\vone{{\bm{1}}}
\def\vmu{{\bm{\mu}}}
\def\vtheta{{\bm{\theta}}}
\def\va{{\bm{a}}}
\def\vb{{\bm{b}}}
\def\vc{{\bm{c}}}
\def\vd{{\bm{d}}}
\def\ve{{\bm{e}}}
\def\vf{{\bm{f}}}
\def\vg{{\bm{g}}}
\def\vh{{\bm{h}}}
\def\vi{{\bm{i}}}
\def\vj{{\bm{j}}}
\def\vk{{\bm{k}}}
\def\vl{{\bm{l}}}
\def\vm{{\bm{m}}}
\def\vn{{\bm{n}}}
\def\vo{{\bm{o}}}
\def\vp{{\bm{p}}}
\def\vq{{\bm{q}}}
\def\vr{{\bm{r}}}
\def\vs{{\bm{s}}}
\def\vt{{\bm{t}}}
\def\vu{{\bm{u}}}
\def\vv{{\bm{v}}}
\def\vw{{\bm{w}}}
\def\vx{{\bm{x}}}
\def\vy{{\bm{y}}}
\def\vz{{\bm{z}}}

% Elements of vectors
\def\evalpha{{\alpha}}
\def\evbeta{{\beta}}
\def\evepsilon{{\epsilon}}
\def\evlambda{{\lambda}}
\def\evomega{{\omega}}
\def\evmu{{\mu}}
\def\evpsi{{\psi}}
\def\evsigma{{\sigma}}
\def\evtheta{{\theta}}
\def\eva{{a}}
\def\evb{{b}}
\def\evc{{c}}
\def\evd{{d}}
\def\eve{{e}}
\def\evf{{f}}
\def\evg{{g}}
\def\evh{{h}}
\def\evi{{i}}
\def\evj{{j}}
\def\evk{{k}}
\def\evl{{l}}
\def\evm{{m}}
\def\evn{{n}}
\def\evo{{o}}
\def\evp{{p}}
\def\evq{{q}}
\def\evr{{r}}
\def\evs{{s}}
\def\evt{{t}}
\def\evu{{u}}
\def\evv{{v}}
\def\evw{{w}}
\def\evx{{x}}
\def\evy{{y}}
\def\evz{{z}}

% Matrix
\def\mA{{\bm{A}}}
\def\mB{{\bm{B}}}
\def\mC{{\bm{C}}}
\def\mD{{\bm{D}}}
\def\mE{{\bm{E}}}
\def\mF{{\bm{F}}}
\def\mG{{\bm{G}}}
\def\mH{{\bm{H}}}
\def\mI{{\bm{I}}}
\def\mJ{{\bm{J}}}
\def\mK{{\bm{K}}}
\def\mL{{\bm{L}}}
\def\mM{{\bm{M}}}
\def\mN{{\bm{N}}}
\def\mO{{\bm{O}}}
\def\mP{{\bm{P}}}
\def\mQ{{\bm{Q}}}
\def\mR{{\bm{R}}}
\def\mS{{\bm{S}}}
\def\mT{{\bm{T}}}
\def\mU{{\bm{U}}}
\def\mV{{\bm{V}}}
\def\mW{{\bm{W}}}
\def\mX{{\bm{X}}}
\def\mY{{\bm{Y}}}
\def\mZ{{\bm{Z}}}
\def\mBeta{{\bm{\beta}}}
\def\mPhi{{\bm{\Phi}}}
\def\mLambda{{\bm{\Lambda}}}
\def\mSigma{{\bm{\Sigma}}}

% Tensor
\DeclareMathAlphabet{\mathsfit}{\encodingdefault}{\sfdefault}{m}{sl}
\SetMathAlphabet{\mathsfit}{bold}{\encodingdefault}{\sfdefault}{bx}{n}
\newcommand{\tens}[1]{\bm{\mathsfit{#1}}}
\def\tA{{\tens{A}}}
\def\tB{{\tens{B}}}
\def\tC{{\tens{C}}}
\def\tD{{\tens{D}}}
\def\tE{{\tens{E}}}
\def\tF{{\tens{F}}}
\def\tG{{\tens{G}}}
\def\tH{{\tens{H}}}
\def\tI{{\tens{I}}}
\def\tJ{{\tens{J}}}
\def\tK{{\tens{K}}}
\def\tL{{\tens{L}}}
\def\tM{{\tens{M}}}
\def\tN{{\tens{N}}}
\def\tO{{\tens{O}}}
\def\tP{{\tens{P}}}
\def\tQ{{\tens{Q}}}
\def\tR{{\tens{R}}}
\def\tS{{\tens{S}}}
\def\tT{{\tens{T}}}
\def\tU{{\tens{U}}}
\def\tV{{\tens{V}}}
\def\tW{{\tens{W}}}
\def\tX{{\tens{X}}}
\def\tY{{\tens{Y}}}
\def\tZ{{\tens{Z}}}


% Graph
\def\gA{{\mathcal{A}}}
\def\gB{{\mathcal{B}}}
\def\gC{{\mathcal{C}}}
\def\gD{{\mathcal{D}}}
\def\gE{{\mathcal{E}}}
\def\gF{{\mathcal{F}}}
\def\gG{{\mathcal{G}}}
\def\gH{{\mathcal{H}}}
\def\gI{{\mathcal{I}}}
\def\gJ{{\mathcal{J}}}
\def\gK{{\mathcal{K}}}
\def\gL{{\mathcal{L}}}
\def\gM{{\mathcal{M}}}
\def\gN{{\mathcal{N}}}
\def\gO{{\mathcal{O}}}
\def\gP{{\mathcal{P}}}
\def\gQ{{\mathcal{Q}}}
\def\gR{{\mathcal{R}}}
\def\gS{{\mathcal{S}}}
\def\gT{{\mathcal{T}}}
\def\gU{{\mathcal{U}}}
\def\gV{{\mathcal{V}}}
\def\gW{{\mathcal{W}}}
\def\gX{{\mathcal{X}}}
\def\gY{{\mathcal{Y}}}
\def\gZ{{\mathcal{Z}}}

% Sets
\def\sA{{\mathbb{A}}}
\def\sB{{\mathbb{B}}}
\def\sC{{\mathbb{C}}}
\def\sD{{\mathbb{D}}}
% Don't use a set called E, because this would be the same as our symbol
% for expectation.
\def\sF{{\mathbb{F}}}
\def\sG{{\mathbb{G}}}
\def\sH{{\mathbb{H}}}
\def\sI{{\mathbb{I}}}
\def\sJ{{\mathbb{J}}}
\def\sK{{\mathbb{K}}}
\def\sL{{\mathbb{L}}}
\def\sM{{\mathbb{M}}}
\def\sN{{\mathbb{N}}}
\def\sO{{\mathbb{O}}}
\def\sP{{\mathbb{P}}}
\def\sQ{{\mathbb{Q}}}
\def\sR{{\mathbb{R}}}
\def\sS{{\mathbb{S}}}
\def\sT{{\mathbb{T}}}
\def\sU{{\mathbb{U}}}
\def\sV{{\mathbb{V}}}
\def\sW{{\mathbb{W}}}
\def\sX{{\mathbb{X}}}
\def\sY{{\mathbb{Y}}}
\def\sZ{{\mathbb{Z}}}

% Entries of a matrix
\def\emLambda{{\Lambda}}
\def\emA{{A}}
\def\emB{{B}}
\def\emC{{C}}
\def\emD{{D}}
\def\emE{{E}}
\def\emF{{F}}
\def\emG{{G}}
\def\emH{{H}}
\def\emI{{I}}
\def\emJ{{J}}
\def\emK{{K}}
\def\emL{{L}}
\def\emM{{M}}
\def\emN{{N}}
\def\emO{{O}}
\def\emP{{P}}
\def\emQ{{Q}}
\def\emR{{R}}
\def\emS{{S}}
\def\emT{{T}}
\def\emU{{U}}
\def\emV{{V}}
\def\emW{{W}}
\def\emX{{X}}
\def\emY{{Y}}
\def\emZ{{Z}}
\def\emSigma{{\Sigma}}

% entries of a tensor
% Same font as tensor, without \bm wrapper
\newcommand{\etens}[1]{\mathsfit{#1}}
\def\etLambda{{\etens{\Lambda}}}
\def\etA{{\etens{A}}}
\def\etB{{\etens{B}}}
\def\etC{{\etens{C}}}
\def\etD{{\etens{D}}}
\def\etE{{\etens{E}}}
\def\etF{{\etens{F}}}
\def\etG{{\etens{G}}}
\def\etH{{\etens{H}}}
\def\etI{{\etens{I}}}
\def\etJ{{\etens{J}}}
\def\etK{{\etens{K}}}
\def\etL{{\etens{L}}}
\def\etM{{\etens{M}}}
\def\etN{{\etens{N}}}
\def\etO{{\etens{O}}}
\def\etP{{\etens{P}}}
\def\etQ{{\etens{Q}}}
\def\etR{{\etens{R}}}
\def\etS{{\etens{S}}}
\def\etT{{\etens{T}}}
\def\etU{{\etens{U}}}
\def\etV{{\etens{V}}}
\def\etW{{\etens{W}}}
\def\etX{{\etens{X}}}
\def\etY{{\etens{Y}}}
\def\etZ{{\etens{Z}}}

% The true underlying data generating distribution
\newcommand{\pdata}{p_{\rm{data}}}
% The empirical distribution defined by the training set
\newcommand{\ptrain}{\hat{p}_{\rm{data}}}
\newcommand{\Ptrain}{\hat{P}_{\rm{data}}}
% The model distribution
\newcommand{\pmodel}{p_{\rm{model}}}
\newcommand{\Pmodel}{P_{\rm{model}}}
\newcommand{\ptildemodel}{\tilde{p}_{\rm{model}}}
% Stochastic autoencoder distributions
\newcommand{\pencode}{p_{\rm{encoder}}}
\newcommand{\pdecode}{p_{\rm{decoder}}}
\newcommand{\precons}{p_{\rm{reconstruct}}}

\newcommand{\laplace}{\mathrm{Laplace}} % Laplace distribution

\newcommand{\E}{\mathbb{E}}
\newcommand{\Ls}{\mathcal{L}}
\newcommand{\R}{\mathbb{R}}
\newcommand{\emp}{\tilde{p}}
\newcommand{\lr}{\alpha}
\newcommand{\reg}{\lambda}
\newcommand{\rect}{\mathrm{rectifier}}
\newcommand{\softmax}{\mathrm{softmax}}
\newcommand{\sigmoid}{\sigma}
\newcommand{\softplus}{\zeta}
\newcommand{\KL}{D_{\mathrm{KL}}}
\newcommand{\Var}{\mathrm{Var}}
\newcommand{\standarderror}{\mathrm{SE}}
\newcommand{\Cov}{\mathrm{Cov}}
% Wolfram Mathworld says $L^2$ is for function spaces and $\ell^2$ is for vectors
% But then they seem to use $L^2$ for vectors throughout the site, and so does
% wikipedia.
\newcommand{\normlzero}{L^0}
\newcommand{\normlone}{L^1}
\newcommand{\normltwo}{L^2}
\newcommand{\normlp}{L^p}
\newcommand{\normmax}{L^\infty}

\newcommand{\parents}{Pa} % See usage in notation.tex. Chosen to match Daphne's book.

\DeclareMathOperator*{\argmax}{arg\,max}
\DeclareMathOperator*{\argmin}{arg\,min}
\DeclareMathOperator*{\argsup}{arg\,sup}


\DeclareMathOperator{\sign}{sign}
\DeclareMathOperator{\Tr}{Tr}
\let\ab\allowbreak

\newtheorem{theorem}{Theorem}
\newtheorem{lemma}{Lemma}
\newtheorem{definition}{Definition}
\newtheorem{proposition}{Proposition}
\newtheorem{remark}{Remark}
\newtheorem{example}{Example}
\newtheorem{assumption}{Assumption}
\newtheorem{corollary}{Corollary}
\newtheorem{update_rule}{Update}
\newtheorem{claim}[theorem]{Claim}
\newtheorem{conjecture}{Conjecture}


\DeclareMathOperator*{\expectation}{\mathbb{E}}
\def\rvpi{{\boldsymbol{\pi}}}
\def\rvdelta{{\boldsymbol{\delta}}}
\def\rvtheta{{\boldsymbol{\theta}}}
\def\rvTheta{{\boldsymbol{\Theta}}}

\def\rvone{{\mathbf{1}}}
\def\rvzero{{\mathbf{0}}}

\def\identitymatrix{\mathbf{Id}}
\def\diagonalmatrix{\text{diag}}
\def\linearspan{\text{Span}}

\DeclareMathOperator*{\subjecto}{subject\,to}
\DeclareMathOperator*{\probability}{Pr}
\DeclareMathOperator*{\cov}{Cov}

\newcommand{\cS}{\mathcal{S}}
\newcommand{\cA}{\mathcal{A}}
\newcommand{\chE}{\mathbb{E}}
\newcommand{\EE}[1]{\mathbb{E}[#1]}
\newcommand{\EEt}[1]{\mathbb{E}_t[#1]}

\newcommand\scalemath[2]{\scalebox{#1}{\mbox{\ensuremath{\displaystyle #2}}}}


\newlength\tocrulewidth
\setlength{\tocrulewidth}{1.5pt}



%\usepackage{cases}
\usepackage{empheq}

\usepackage{subcaption}


\begin{document}


\maketitle


\begin{abstract}
\begin{abstract}  
Test time scaling is currently one of the most active research areas that shows promise after training time scaling has reached its limits.
Deep-thinking (DT) models are a class of recurrent models that can perform easy-to-hard generalization by assigning more compute to harder test samples.
However, due to their inability to determine the complexity of a test sample, DT models have to use a large amount of computation for both easy and hard test samples.
Excessive test time computation is wasteful and can cause the ``overthinking'' problem where more test time computation leads to worse results.
In this paper, we introduce a test time training method for determining the optimal amount of computation needed for each sample during test time.
We also propose Conv-LiGRU, a novel recurrent architecture for efficient and robust visual reasoning. 
Extensive experiments demonstrate that Conv-LiGRU is more stable than DT, effectively mitigates the ``overthinking'' phenomenon, and achieves superior accuracy.
\end{abstract}  
\end{abstract}

\section{Introduction}
\label{sec:introduction}
The business processes of organizations are experiencing ever-increasing complexity due to the large amount of data, high number of users, and high-tech devices involved \cite{martin2021pmopportunitieschallenges, beerepoot2023biggestbpmproblems}. This complexity may cause business processes to deviate from normal control flow due to unforeseen and disruptive anomalies \cite{adams2023proceddsriftdetection}. These control-flow anomalies manifest as unknown, skipped, and wrongly-ordered activities in the traces of event logs monitored from the execution of business processes \cite{ko2023adsystematicreview}. For the sake of clarity, let us consider an illustrative example of such anomalies. Figure \ref{FP_ANOMALIES} shows a so-called event log footprint, which captures the control flow relations of four activities of a hypothetical event log. In particular, this footprint captures the control-flow relations between activities \texttt{a}, \texttt{b}, \texttt{c} and \texttt{d}. These are the causal ($\rightarrow$) relation, concurrent ($\parallel$) relation, and other ($\#$) relations such as exclusivity or non-local dependency \cite{aalst2022pmhandbook}. In addition, on the right are six traces, of which five exhibit skipped, wrongly-ordered and unknown control-flow anomalies. For example, $\langle$\texttt{a b d}$\rangle$ has a skipped activity, which is \texttt{c}. Because of this skipped activity, the control-flow relation \texttt{b}$\,\#\,$\texttt{d} is violated, since \texttt{d} directly follows \texttt{b} in the anomalous trace.
\begin{figure}[!t]
\centering
\includegraphics[width=0.9\columnwidth]{images/FP_ANOMALIES.png}
\caption{An example event log footprint with six traces, of which five exhibit control-flow anomalies.}
\label{FP_ANOMALIES}
\end{figure}

\subsection{Control-flow anomaly detection}
Control-flow anomaly detection techniques aim to characterize the normal control flow from event logs and verify whether these deviations occur in new event logs \cite{ko2023adsystematicreview}. To develop control-flow anomaly detection techniques, \revision{process mining} has seen widespread adoption owing to process discovery and \revision{conformance checking}. On the one hand, process discovery is a set of algorithms that encode control-flow relations as a set of model elements and constraints according to a given modeling formalism \cite{aalst2022pmhandbook}; hereafter, we refer to the Petri net, a widespread modeling formalism. On the other hand, \revision{conformance checking} is an explainable set of algorithms that allows linking any deviations with the reference Petri net and providing the fitness measure, namely a measure of how much the Petri net fits the new event log \cite{aalst2022pmhandbook}. Many control-flow anomaly detection techniques based on \revision{conformance checking} (hereafter, \revision{conformance checking}-based techniques) use the fitness measure to determine whether an event log is anomalous \cite{bezerra2009pmad, bezerra2013adlogspais, myers2018icsadpm, pecchia2020applicationfailuresanalysispm}. 

The scientific literature also includes many \revision{conformance checking}-independent techniques for control-flow anomaly detection that combine specific types of trace encodings with machine/deep learning \cite{ko2023adsystematicreview, tavares2023pmtraceencoding}. Whereas these techniques are very effective, their explainability is challenging due to both the type of trace encoding employed and the machine/deep learning model used \cite{rawal2022trustworthyaiadvances,li2023explainablead}. Hence, in the following, we focus on the shortcomings of \revision{conformance checking}-based techniques to investigate whether it is possible to support the development of competitive control-flow anomaly detection techniques while maintaining the explainable nature of \revision{conformance checking}.
\begin{figure}[!t]
\centering
\includegraphics[width=\columnwidth]{images/HIGH_LEVEL_VIEW.png}
\caption{A high-level view of the proposed framework for combining \revision{process mining}-based feature extraction with dimensionality reduction for control-flow anomaly detection.}
\label{HIGH_LEVEL_VIEW}
\end{figure}

\subsection{Shortcomings of \revision{conformance checking}-based techniques}
Unfortunately, the detection effectiveness of \revision{conformance checking}-based techniques is affected by noisy data and low-quality Petri nets, which may be due to human errors in the modeling process or representational bias of process discovery algorithms \cite{bezerra2013adlogspais, pecchia2020applicationfailuresanalysispm, aalst2016pm}. Specifically, on the one hand, noisy data may introduce infrequent and deceptive control-flow relations that may result in inconsistent fitness measures, whereas, on the other hand, checking event logs against a low-quality Petri net could lead to an unreliable distribution of fitness measures. Nonetheless, such Petri nets can still be used as references to obtain insightful information for \revision{process mining}-based feature extraction, supporting the development of competitive and explainable \revision{conformance checking}-based techniques for control-flow anomaly detection despite the problems above. For example, a few works outline that token-based \revision{conformance checking} can be used for \revision{process mining}-based feature extraction to build tabular data and develop effective \revision{conformance checking}-based techniques for control-flow anomaly detection \cite{singh2022lapmsh, debenedictis2023dtadiiot}. However, to the best of our knowledge, the scientific literature lacks a structured proposal for \revision{process mining}-based feature extraction using the state-of-the-art \revision{conformance checking} variant, namely alignment-based \revision{conformance checking}.

\subsection{Contributions}
We propose a novel \revision{process mining}-based feature extraction approach with alignment-based \revision{conformance checking}. This variant aligns the deviating control flow with a reference Petri net; the resulting alignment can be inspected to extract additional statistics such as the number of times a given activity caused mismatches \cite{aalst2022pmhandbook}. We integrate this approach into a flexible and explainable framework for developing techniques for control-flow anomaly detection. The framework combines \revision{process mining}-based feature extraction and dimensionality reduction to handle high-dimensional feature sets, achieve detection effectiveness, and support explainability. Notably, in addition to our proposed \revision{process mining}-based feature extraction approach, the framework allows employing other approaches, enabling a fair comparison of multiple \revision{conformance checking}-based and \revision{conformance checking}-independent techniques for control-flow anomaly detection. Figure \ref{HIGH_LEVEL_VIEW} shows a high-level view of the framework. Business processes are monitored, and event logs obtained from the database of information systems. Subsequently, \revision{process mining}-based feature extraction is applied to these event logs and tabular data input to dimensionality reduction to identify control-flow anomalies. We apply several \revision{conformance checking}-based and \revision{conformance checking}-independent framework techniques to publicly available datasets, simulated data of a case study from railways, and real-world data of a case study from healthcare. We show that the framework techniques implementing our approach outperform the baseline \revision{conformance checking}-based techniques while maintaining the explainable nature of \revision{conformance checking}.

In summary, the contributions of this paper are as follows.
\begin{itemize}
    \item{
        A novel \revision{process mining}-based feature extraction approach to support the development of competitive and explainable \revision{conformance checking}-based techniques for control-flow anomaly detection.
    }
    \item{
        A flexible and explainable framework for developing techniques for control-flow anomaly detection using \revision{process mining}-based feature extraction and dimensionality reduction.
    }
    \item{
        Application to synthetic and real-world datasets of several \revision{conformance checking}-based and \revision{conformance checking}-independent framework techniques, evaluating their detection effectiveness and explainability.
    }
\end{itemize}

The rest of the paper is organized as follows.
\begin{itemize}
    \item Section \ref{sec:related_work} reviews the existing techniques for control-flow anomaly detection, categorizing them into \revision{conformance checking}-based and \revision{conformance checking}-independent techniques.
    \item Section \ref{sec:abccfe} provides the preliminaries of \revision{process mining} to establish the notation used throughout the paper, and delves into the details of the proposed \revision{process mining}-based feature extraction approach with alignment-based \revision{conformance checking}.
    \item Section \ref{sec:framework} describes the framework for developing \revision{conformance checking}-based and \revision{conformance checking}-independent techniques for control-flow anomaly detection that combine \revision{process mining}-based feature extraction and dimensionality reduction.
    \item Section \ref{sec:evaluation} presents the experiments conducted with multiple framework and baseline techniques using data from publicly available datasets and case studies.
    \item Section \ref{sec:conclusions} draws the conclusions and presents future work.
\end{itemize}

\section{Setting and Background}

We consider the stochastic multi-armed bandit problem \citep{lattimore2020bandit}, specified by $K$ actions and a true mean reward vector $r \in \sR^K$, where for each action $a \in [K] \coloneqq \{1, 2, \dots, K \}$,
\begin{align}
\label{eq:true_mean_reward_expectation_bounded_sampled_reward}
    r(a) = \int_{-R_{\max}}^{R_{\max}}{ x \cdot P_a(x) \mu(d x)},
\end{align}
where $R_{\max} > 0$ is the reward range, $\mu$ is a finite measure over $[-R_{\max}, R_{\max}]$, and $P_a(x) \ge 0$ is the probability density function with respect to $\mu$.
We use $R_a$ to denote the reward distribution for action $a$ defined by the density $P_a$ and base measure $\mu$. The goal is to find a policy $\pi_{\theta} \in [0, 1]^K$ to achieve high expected reward,
\begin{align}
\label{eq:expected_reward}
    \max_{\theta \in \sR^K}{ \pi_{\theta}^\top r},
\end{align}
where $\pi_\theta$ is parameterized by $\theta \in \sR^K$. 

\textbf{The gradient bandit algorithm.} A natural idea to optimize \cref{eq:expected_reward} is to use stochastic gradient ascent, which is shown in \cref{alg:gradient_bandit_algorithm_sampled_reward} and known as the gradient bandit algorithm \citep[Section 2.8]{sutton2018reinforcement}. In \cref{alg:gradient_bandit_algorithm_sampled_reward}, in each iteration $t \ge 1$, the probability of pulling arm $a \in [K]$ is given as 
%$\probability{\left( a_t = a \right)} = \pi_{\theta_t}(a)$ such that $\pi_{\theta_t} = \softmax(\theta_t)$, where
\begin{align}
\label{eq:softmax}
    \pi_{\theta_t}(a) = [ \softmax(\theta_t) ](a) \coloneqq
    \frac{ \exp\{ \theta_t(a) \} }{ \sum_{a^\prime \in [K]}{ \exp\{ \theta_t(a^\prime) } \} }, \mbox{ \quad   for all } a \in [K],
\end{align}
where $\theta_t \in \sR^{K}$ is the parameter vector to be updated. The following proposition shows that \cref{alg:gradient_bandit_algorithm_sampled_reward} is an instance of stochastic gradient ascent with an unbiased gradient estimator \citep{sutton2018reinforcement,mei2024stochastic}.

\begin{figure}[t]
\centering
\vskip -0.1in
\begin{minipage}{.7\linewidth}
    \begin{algorithm}[H]
    \caption{Gradient bandit algorithm (without baselines)}
    \label{alg:gradient_bandit_algorithm_sampled_reward}
    \begin{algorithmic}
    \STATE {\bfseries Input:} initial parameters $\theta_1 \in \sR^K$, learning rate $\eta > 0$.
    \STATE {\bfseries Output:} policies $\pi_{\theta_t} = \softmax(\theta_t)$.
    \WHILE{$t \ge 1$}
   \STATE Sample an action $a_t \sim \pi_{\theta_t}(\cdot)$ and observe reward $R_t(a_t)\sim P_{a_t}$.
   \FOR{all $a \in [K]$}
   \IF{$a = a_t$}
   \STATE $\theta_{t+1}(a) \gets \theta_t(a) + \eta \cdot \left( 1 - \pi_{\theta_t}(a) \right) \cdot R_t(a_t)$.
   \ELSE
   \STATE $\theta_{t+1}(a) \gets \theta_t(a) - \eta \cdot \pi_{\theta_t}(a) \cdot R_t(a_t)$.
   \ENDIF
   \ENDFOR
   \ENDWHILE
   \end{algorithmic}
    \end{algorithm}
\end{minipage}
\end{figure}

\begin{proposition}[Proposition 2.3 of \citep{mei2024stochastic}]
\label{prop:gradient_bandit_algorithm_equivalent_to_stochastic_gradient_ascent_sampled_reward}
\cref{alg:gradient_bandit_algorithm_sampled_reward} is equivalent to the following update,
\begin{align}
\label{eq:stochastic_gradient_ascent_sampled_reward}
    \theta_{t+1} &\gets  \theta_{t} + \eta \cdot \frac{d \ \pi_{\theta_t}^\top \hat{r}_t}{d \theta_t} = \theta_t + \eta \cdot \left(  \diagonalmatrix{(\pi_{\theta_t})} - \pi_{\theta_t} \pi_{\theta_t}^\top \right) \hat{r}_t,
\end{align}
where $\mathbb{E}_t{ \Big[ \frac{d \ \pi_{\theta_t}^\top \hat{r}_t }{d \theta_t} \Big] } = \frac{d \ \pi_{\theta_t}^\top r}{d \theta_t }$, and $\EEt{\cdot}$ is defined with respect to randomness from on-policy sampling $a_t \sim \pi_{\theta_t}(\cdot)$ and reward sampling $R_t(a_t)\sim P_{a_t}$. The Jacobian of $\theta \mapsto \pi_\theta \coloneqq \softmax(\theta)$ is $\left( \frac{d \ \pi_{\theta}}{d \theta} \right)^\top = \diagonalmatrix{(\pi_{\theta})} - \pi_{\theta} \pi_{\theta}^\top \in \sR^{K \times K}$, and 
$\hat{r}_t(a) \coloneqq \frac{ \sI\left\{ a_t = a \right\} }{ \pi_{\theta_t}(a) } \cdot R_t(a)$ for all $a \in [K]$ is the importance sampling (IS) estimator, and we set $R_t(a)=0$ for all $a \not= a_t$. 
\end{proposition}
\textbf{Known results on the convergence of the gradient bandit algorithm.}
Since \cref{eq:expected_reward} corresponds to a smooth non-concave maximization problem over $\theta \in \sR^K$ \citep{mei2020global}, using \cref{alg:gradient_bandit_algorithm_sampled_reward} with decaying learning rates is sufficient to guarantee convergence to a stationary point \citep{robbins1951stochastic,nemirovski2009robust,ghadimi2013stochastic,zhang2020global}. However, this is insufficient to ensure the  globally optimal solution of \cref{eq:expected_reward} is reached, since there exist multiple stationary points. More recently, guarantees of convergence to a globally optimal policy have been developed for PG methods in the true gradient setting \citep{agarwal2021theory,mei2020global}, where the algorithm has access to exact mean rewards. These results were later extended to achieve global convergence guarantees (almost surely) in the stochastic setting \citep{zhang2020sample,ding2021beyond,zhang2021convergence,yuan2022general,mei2024stochastic,lu2024towards}. However, these extended results have required decaying or sufficiently small learning rates, motivated by exploiting smoothness and combating the inherent noise in stochastic gradients.

Despite these previous assumptions, there exists empirical and theoretical evidence that using a large learning rate in the stochastic gradient bandit algorithm is a viable option. For example, it has been observed that softmax policies learn even with extremely large learning rates such as $2^{14}$ \citep{garg2021alternate}. For logistic regression on linearly separable data, the objective has an exponential tail and the minimizer is unbounded, yet it has been shown that gradient descent with iteration dependent learning rate $\eta \in \Theta(t)$ achieves accelerated $O(1/t^2)$ convergence \citep{wu2024large}. Though the objective in \cref{eq:expected_reward} has similar properties, unlike logistic regression, the problem we are considering is non-concave, so the same techniques cannot be directly applied. % and hence similar techniques are not applicable here. 
The most related results are from \citep{mei2024stochastic}, which proved that with a small problem specific constant learning rate, \cref{alg:gradient_bandit_algorithm_sampled_reward} achieves convergence to a globally optimal policy almost surely. However, as mentioned, the learning rate choices in \citep{mei2024stochastic} rely on assumptions of (non-uniform) smoothness and noise growth conditions (their Lemmas 4.2, 4.3, and 4.6), which cannot be directly applied here for a large learning rate.

Consequently, the use of large learning rates appear to render existing results and techniques inapplicable. Furthermore, with a large constant learning rate, it is unclear whether \cref{alg:gradient_bandit_algorithm_sampled_reward} will converge to any stationary point, or the iterates will keep oscillating. If the algorithm does converge, it is also not clear what effect large step-sizes have on exploration, and whether the algorithm will converge to the optimal arm in such cases. Resolving these questions requires new results that characterize the behavior of \cref{alg:gradient_bandit_algorithm_sampled_reward}, since the classical optimization and stochastic approximation convergence theories are no longer applicable, as explained. % (smoothness and noise control) are no longer applicable as explained.

%It has been recently discovered that with small step-sizes, exploration is handled automatically and the algorithm converges to the optimal arm without any explicit exploration \citep{mei2024stochastic}.




\section{Asymptotic Global Convergence of Gradient Bandit Algorithm}

We have seen that solving the non-concave maximization problem \cref{eq:expected_reward} using \cref{alg:gradient_bandit_algorithm_sampled_reward} with any constant (potentially large) learning rate requires ideas beyond classical optimization theory. Here, we take a different perspective to investigate how \cref{alg:gradient_bandit_algorithm_sampled_reward} samples actions. For analysis, we make the following assumption about the reward distribution.
\begin{assumption}[True mean reward has no ties]
\label{assp:reward_no_ties}
For all $i, j \in [K]$, if $i \not= j$, then $r(i) \not= r(j)$.
\end{assumption}
\begin{remark}
Removing \cref{assp:reward_no_ties} remains an open question for future work, while we believe that \cref{alg:gradient_bandit_algorithm_sampled_reward} works without \cref{assp:reward_no_ties}. One piece of evidence to support this conjecture is that even in the exact gradient setting, the set of initializations where Softmax PG approaches non-strict one-hot policies has zero measure.
\end{remark}

\subsection{Failure Mode of Aggressive Updates}
\label{subsec:failure_model_aggressive_updates}

It has been observed that several accelerated PG methods in the true gradient setting, including natural PG \citep{kakade2002natural,agarwal2021theory} and normalized PG \citep{mei2021leveraging}, obtain worse results than standard softmax PG if combined with online sampling $a_t \sim \pi_{\theta_t}(\cdot)$ using constant learning rates \citep{mei2021understanding}. The failure mode in these cases is that the update is too aggressive and commits to a sub-optimal arm without sufficiently exploring all arms. This results in a non-trivial probability of sampling one action forever, i.e., there exists a potentially sub-optimal action $a \in [K]$, such that with some constant probability, $a_t = a$ for all $t \ge 1$. Such an outcome implies that $\pi_{\theta_t}(a) \to 1$ as $t \to \infty$ \citep[Theorem 3]{mei2021understanding}. Since $a \in [K]$ could be a sub-optimal action with $r(a) < r(a^*) = \max_{a \in [K]} r(a)$, this results in a lack of exploration, and consequently, methods such as natural PG and normalized PG are not guaranteed to converge to the optimal action $a^* \coloneqq \argmax_{a \in [K]}{ r(a) }$ with probability $1$.

\subsection{Stochastic Gradient Automatically Avoids Lack of Exploration}
\label{subsec:avoid_lack_of_exploration}

Our first key finding is that \cref{alg:gradient_bandit_algorithm_sampled_reward} does not keep sampling one action forever, no matter how large the constant learning rate is.  This property avoids the problem of a lack of exploration, in the sense that \cref{alg:gradient_bandit_algorithm_sampled_reward} will at least explore more than one action infinitely often. At first glance, this might not seem like a strong property, since the algorithm might somehow explore only sub-optimal actions forever.  %one might wonder what if the algorithm just explores over sub-optimal actions? 
However, we will argue below that this property coupled with additional arguments is sufficient to guarantee convergence to the globally optimal policy.

Let us now formally prove the above property. By \cref{alg:gradient_bandit_algorithm_sampled_reward}, for all $a \in [K]$, for all $t \ge 1$,
\begin{empheq}[left={\theta_{t+1}(a) \gets \theta_t(a) +\empheqbiglbrace~}]{align}
\label{eq:first_update}
    \eta \cdot \left( 1 - \pi_{\theta_t}(a) \right) \cdot R_t(a), & \quad \text{ if} \ a_t = a, \\ 
\label{eq:second_update}
    - \eta \cdot \pi_{\theta_t}(a) \cdot R_t(a_t), & \quad \text{ otherwise}.
\end{empheq}
We define $N_t(a)$ as the number of times action $a \in [K]$ is sampled up to iteration $t \ge 1$, i.e.,
\begin{align}
\label{eq:finite_sample_count}
    N_t(a) \coloneqq \sum_{s=1}^{t}{ \sI\left\{ a_s = a \right\} },
\end{align}
and its asymptotic limit $N_\infty(a) \coloneqq \lim_{t \to \infty}{ N_t(a) }$, which could possibly be infinity. For all $a \in [K]$, we have either $N_\infty(a) = \infty$ or $N_\infty(a) < \infty$, meaning that $a \in [K]$ is sampled infinitely often or only finitely many times asymptotically. First, we prove the following \cref{lem:finite_sample_time_implies_finite_parameter}, which shows that if an action $a \in [K]$ is sampled only finitely many times as $t \to \infty$, then the parameter corresponding to action $a$ is also finite, i.e., $\sup_{t \ge 1}{| \theta_t(a) |} < \infty$.
\begin{lemma}
\label{lem:finite_sample_time_implies_finite_parameter}
Using \cref{alg:gradient_bandit_algorithm_sampled_reward} with any constant $\eta \in \Theta(1)$, if $N_\infty(a) < \infty$ for an action $a \in [K]$, then we have, almost surely,
\begin{align}
\label{lem:finite_sample_time_implies_finite_parameter_claim}
    \sup_{t \ge 1}{ \theta_t(a) } < \infty, \text{ and } \inf_{t \ge 1}{ \theta_t(a) } > -\infty.
\end{align}
\end{lemma}
\cref{lem:finite_sample_time_implies_finite_parameter} will be used multiple times in the subsequent convergence arguments.

\paragraph{Proof sketch.} Since we assume action $a \in [K]$ is sampled finitely many times, the update given in the case depicted by~\cref{eq:first_update} happens finitely many times. Each update is bounded since the sampled reward is in $[-R_{\max}, R_{\max}]$ by \cref{eq:true_mean_reward_expectation_bounded_sampled_reward}, and the learning rate is a constant, i.e., $\eta \in \Theta(1)$. In \cref{alg:gradient_bandit_algorithm_sampled_reward}, $\theta_t(a)$ is still updated even when $a_t \ne a$, with the corresponding update given by the case depicted by \cref{eq:second_update}. Therefore, whether $\theta_t(a)$ is bounded depends on the cumulative probability $\sum_{s=1}^{t}{ \pi_{\theta_s}(a) }$ being summable as $t \to \infty$. According to the extended Borel-Cantelli lemma (\cref{lem:ebc}), we have, almost surely,
\begin{align}
\label{eq:ebc_result}
    \Big\{ \sum_{t \ge 1} \pi_{\theta_t}(a) =\infty \Big\} = \left\{ N_\infty(a )=\infty \right\},
\end{align}
which implies (by taking complements) that $\sum_{t \ge 1} \pi_{\theta_t}(a) < \infty$ if and only if $N_\infty(a) < \infty$. Therefore, if $a \in [K]$ is sampled finitely often, $\theta_t(a)$ will be updated in a bounded manner (using \cref{eq:first_update,eq:second_update}) as $t \to \infty$, hence establishing \cref{lem:finite_sample_time_implies_finite_parameter}. Detailed proofs for this lemma, as well as for all other results in this paper can be found in the appendix. % due to space constraints.

Given \cref{lem:finite_sample_time_implies_finite_parameter}, we can then establish 
%are ready to present  \cref{lem:at_least_two_actions_infinite_sample_time}, which formally states 
the above-mentioned finding about the exploration effect of~\cref{alg:gradient_bandit_algorithm_sampled_reward} in \cref{lem:at_least_two_actions_infinite_sample_time}.
\begin{lemma}[Avoiding a lack of exploration]
\label{lem:at_least_two_actions_infinite_sample_time}
Using \cref{alg:gradient_bandit_algorithm_sampled_reward} with any $\eta \in \Theta(1)$, there exists at least a pair of distinct actions $i, j \in [K]$ and $i \ne j$, such that, almost surely,
\begin{align}
\label{lem:at_least_two_actions_infinite_sample_time_claim}
    N_\infty(i) = \infty, \text{ and } N_\infty(j) = \infty.
\end{align}
\end{lemma}
\paragraph{Proof sketch.} 
The argument for the existence of one such action is straightforward, since by the pigeonhole principle, if there are finitely many actions, i.e., $K < \infty$, there must be at least one action $i \in [K]$ that is sampled infinitely often as $t \to \infty$.

The argument for the existence of a second such action is by contradiction. Suppose that all the other actions $j \in [K]$ with $j \ne i$ are sampled only finitely many times as $t \to \infty$. According to \cref{lem:finite_sample_time_implies_finite_parameter}, their corresponding parameters must remain finite, i.e., $\sup_{t \ge 1}{ |\theta_t(j)| } < \infty$ for all $j \in [K]$ with $j \ne i$. Now consider  $\theta_t(i)$. By assumption, the second update case for this parameter, \cref{eq:second_update}, happens only finitely often, since \cref{eq:second_update} can only occur when $a_t \ne i$. Therefore, the key question is whether the cumulative probability $\sum_{s=1}^{t} \left( 1 - \pi_{\theta_s}(i) \right)$ involved in the first case of the update,~\cref{eq:first_update}, is summable as $t \to \infty$. Note that $\sum_{s=1}^{t} \left( 1 - \pi_{\theta_s}(i) \right) = \sum_{s=1}^{t} \sum_{j \ne i}{ \pi_{\theta_s}(j) }$, which is indeed summable as $t \to \infty$, by the assumption and~\cref{eq:ebc_result}. This implies that   action $i$, which is sampled infinitely often, achieves a parameter magnitude, $\sup_{t \ge 1}{ |\theta_t(i)| } < \infty$, that remains bounded as $t\rightarrow\infty$. Using the softmax parameterization \cref{eq:softmax} in the above argument, we conclude that for all $a \in [K]$, $\inf_{t \ge 1}{ \pi_{\theta_t}(a)} > 0$, i.e., every action's probability remains bounded away from zero, and hence is not summable. Using~\cref{eq:ebc_result}, this implies that every action is sampled infinitely often, which contradicts the assumption that only action $i$ is sampled infinitely often as $t \to \infty$. 

\paragraph{Discussion.}
\cref{lem:at_least_two_actions_infinite_sample_time} implies that~\cref{alg:gradient_bandit_algorithm_sampled_reward} is not an aggressive method in the sense of \citep{mei2021understanding}, no matter how large the learning rate is, as long as it is constant, i.e., $\eta \in \Theta(1)$. According to \citep[Theorem 7]{mei2021understanding}, even if we fix the sampling in \cref{alg:gradient_bandit_algorithm_sampled_reward} to a sub-optimal action $a \in [K]$ forever, i.e., $a_t = a$ for all $t \ge 1$, its probability will not approach $1$ faster than $O(1/t)$, i.e., $1 - \pi_{\theta_t}(a) \in \Omega(1/t)$.  This means that there must be at least one another action $a^\prime \in [K]$ with $a^\prime \ne a$, such that $a^\prime$ will also be sampled  infinitely often. A more intuitive explanation is that the $\left( 1 - \pi_{\theta_t}(a) \right)$ term in \cref{eq:first_update} will be near $0$, which slows the speed of committing to a  deterministic policy on $a$ whenever $\pi_{\theta_t}(a)$ is close to $1$, % for any $a \in [K]$, 
which encourages exploration. Such natural exploratory behavior arises in \cref{alg:gradient_bandit_algorithm_sampled_reward} because of the softmax Jacobian $  \diagonalmatrix{(\pi_{\theta})} - \pi_{\theta} \pi_{\theta}^\top $ in the update shown in \cref{prop:gradient_bandit_algorithm_equivalent_to_stochastic_gradient_ascent_sampled_reward}, which determines the growth order of $\theta_t(a)$ for all $a \in [K]$ as $t \to \infty$, making the effect of a constant learning rate $\eta \in \Theta(1)$  asymptotically inconsequential.


\subsection{Warm up: Global Asymptotic Convergence when $K = 2$}

We now consider the simplest case, where we have only two possible actions. According to \cref{lem:at_least_two_actions_infinite_sample_time}, each of the two actions must be sampled infinitely often as $t \to \infty$. We now illustrate the second key result, that for both actions $a \in [K]$, the random sequence $\{ \theta_t(a) \}_{t \ge 1}$ follows the direction of the expected gradient for sufficiently large $t \ge 1$ almost surely. The proof uses a technique that has been previously used in \citep{mei2022role,mei2024stochastic} for small learning rates, but here we observe that the same technique continues to work for \cref{alg:gradient_bandit_algorithm_sampled_reward} no matter how large the learning rate is, as long as $\eta \in \Theta(1)$.

\begin{theorem}
\label{thm:two_action_global_convergence}
Let $K = 2$ and $r(1) > r(2)$. Using \cref{alg:gradient_bandit_algorithm_sampled_reward} with any $\eta \in \Theta(1)$, we have, almost surely, $\pi_{\theta_t}(a^*) \to 1$ as $t \to \infty$, where $a^* \coloneqq \argmax_{a \in [K]}{ r(a) }$ (equal to Action $1$ in this case).
\end{theorem}
\paragraph{Proof sketch.} According to \cref{lem:at_least_two_actions_infinite_sample_time}, $N_\infty(1) = N_\infty(2) = \infty$. Denote the the reward gap as $\Delta \coloneqq r(a^*) - \max_{a \not= a^*}{ r(a) } > 0$, which becomes $\Delta = r(1) - r(2)$ for two actions. Since the stochastic gradient is unbiased (\cref{prop:gradient_bandit_algorithm_equivalent_to_stochastic_gradient_ascent_sampled_reward}), we have, for all $t \ge 1$ (detailed calculations omitted),
\begin{align}
\label{eq:two_action_case_optimal_action_param_submartingale_a}
    \EEt{\theta_{t+1}(a^*)} &= \theta_t(a^*) + \eta \cdot \pi_{\theta_t}(a^*) \cdot \left(  r(a^*) - \pi_{\theta_t}^\top r \right) \\
\label{eq:two_action_case_optimal_action_param_submartingale_b}
    &= \theta_t(a^*) + \eta \cdot \pi_{\theta_t}(a^*) \cdot \Delta \cdot \left( 1 - \pi_{\theta_t}(a^*) \right) > \theta_t(a^*).
\end{align}
A similar calculation shows that,
\begin{align}
\label{eq:two_action_case_suboptimal_action_param_supermartingale}
    \EEt{\theta_{t+1}(2)} = \theta_t(2) - \eta \cdot \pi_{\theta_t}(2) \cdot \Delta \cdot \left( 1 - \pi_{\theta_t}(2) \right) &< \theta_t(2),
\end{align}
which means that $\theta_t(a^*)$ is monotonically increasing in expectation and $\theta_t(2)$ is monotonically decreasing in expectation. In other words, $\{ \theta_t(a^*) \}_{t \ge 1}$ is a sub-martingale, while $\{ \theta_t(2) \}_{t \ge 1}$ is a  super-martingale. However, since $\theta_t \in \sR^K$ is unbounded, Doob's martingale convergence results cannot be directly applied, so we pursue a different argument. Following \citep{mei2022role,mei2024stochastic}, given an action $a \in [K]$, we define $P_t(a) \coloneqq \EEt{\theta_{t+1}(a)} -\theta_t(a) $ as the ``progress'', and define $W_t(a) \coloneqq \theta_t(a) - \chE_{t-1}{[ \theta_t(a)]}$ as the ``noise'', where $\theta_{t}(a) = W_t(a) + P_{t-1}(a) + \theta_{t-1}(a)$.  By recursion we can determine that,
\begin{align}
\label{eq:cumulative_noise_progress}
    \theta_t(a) = \EE{\theta_1(a)} + \sum_{s=1}^{t}{W_s(a)} + \sum_{s=1}^{t-1}{P_s(a)},
\end{align}
i.e., $\theta_t(a)$ is the result of ``cumulative progress'' and ``cumulative noise''. According to \citep[Theorem C.3]{mei2024stochastic}, the cumulative noise term can be bounded by using martingale concentration, where the order of the corresponding confidence interval is smaller than the order of the cumulative progress. 
% In particular, the noise term will appear under a square root 
Therefore, the summation will always be determined by the cumulative progress as $t \to \infty$. According to the calculations in~\cref{eq:two_action_case_optimal_action_param_submartingale_b,eq:two_action_case_suboptimal_action_param_supermartingale}, we have $P_t(a^*) > 0$ and $P_t(2) < 0$,  both of which are not summable. As a result, $\theta_t(a^*) \to \infty$ and $\theta_t(2) \to -\infty$ as $t \to \infty$, which implies that $\frac{ \pi_{\theta_t}(a^*)}{\pi_{\theta_t}(2)} = \exp\{ \theta_t(a^*) - \theta_t(a)\} \to \infty$, hence $\pi_{\theta_t}(a^*) \to 1$ as $t \to \infty$.

\subsection{Global Asymptotic Convergence for all $K \ge 2$}

The illustrative two-action case shows that if $\pi_{\theta_t}^\top r \in (r(2), r(a^*))$ and if both actions are sampled infinitely often, then we have, almost surely $\theta_t(a^*) \to \infty$ and $\theta_t(2) \to - \infty$ as $t \to \infty$. However, the question at the beginning of \cref{subsec:avoid_lack_of_exploration} remains: when $K > 2$, if the two actions sampled infinitely often in \cref{lem:at_least_two_actions_infinite_sample_time} are both sub-optimal, will that result in a similar failure mode to the one described in \cref{subsec:failure_model_aggressive_updates}? The answer is no, which follows from our third key finding, which is based on another contradiction-based argument that establishes almost sure convergence to a globally optimal policy in the general $K > 2$ case.

\begin{theorem}
\label{thm:general_action_global_convergence}
Given $K \ge 2$, using \cref{alg:gradient_bandit_algorithm_sampled_reward} with any $\eta \in \Theta(1)$, we have, almost surely, $\pi_{\theta_t}(a^*) \to 1$ as $t \to \infty$, where $a^* = \argmax_{a \in [K]}{ r(a) }$ is the optimal action.
\end{theorem}
\paragraph{Proof sketch.} We consider two cases: $N_\infty(a^*) < \infty$ and $N_\infty(a^*) = \infty$, corresponding to whether the optimal action is sampled finitely or infinitely often as $t \to \infty$. We argue that the first case ($N_\infty(a^*) < \infty$) is impossible, while for the second case ($N_\infty(a^*) = \infty$) we prove that $\theta_t(a^*) - \theta_t(a) \to \infty$ for all $a \in [K]$ with $r(a) < r(a^*)$, which implies $\pi_{\theta_t}(a^*) \to 1$ as $t \to \infty$.

%\textit{First case.} Suppose that $N_\infty(a^*) < \infty$. We argue that this is not possible via contradiction. From our assumption and according to \cref{lem:finite_sample_time_implies_finite_parameter}, we have that $\inf_{t \ge 1}{ \theta_t(a^*)} > -\infty$. According to \cref{lem:expected_reward_range}, we have that for all large enough $t \ge 1$, $r(i_1) < \pi_{\theta_t}^\top r < r(i_2)$, where $i_1, i_2 \in [K]$ such that $N_\infty(i_1) = N_\infty(i_2) = \infty$ and $i_1 \ne i_2$. Since~\cref{lem:at_least_two_actions_infinite_sample_time} guarantees that there are at least two actions that are sampled infinitely often and since $N_\infty(a^*) < \infty$ from our assumption, we can conclude that $r(i_1) < r(i_2) < r(a^*)$. Now consider the sub-optimal action $i_1 \in [K]$. We have,
%\begin{align}
%\label{eq:general_case_first_case_suboptimal_action_param_supermartingale}
%    \EEt{\theta_{t+1}(i_1)} &= \theta_t(i_1) + \eta \cdot \pi_{\theta_t}(i_1) \cdot \left(  r(i_1) - \pi_{\theta_t}^\top r \right) < \theta_t(i_1),
%\end{align}
%After using similar arguments of \cref{eq:two_action_case_suboptimal_action_param_supermartingale,eq:cumulative_noise_progress} in the two-action case (\cref{thm:two_action_global_convergence}), implies that $\sup_{t \ge 1}{ \theta_t(i_1) } < \infty$. Therefore, we have,
%\begin{align}
%\label{eq:general_case_first_case_bounded_prob_ratio}
%    \sup_{t \ge 1}{ \frac{\pi_{\theta_t}(i_1)}{\pi_{\theta_t}(a^*)} } = \sup_{t \ge 1} \, \exp\{ \theta_t(i_1) - \theta_t(a^*) \}  < \infty \,.
%\end{align}
%On the other hand, since $N_\infty(i_1) = \infty$ (by \cref{lem:expected_reward_range}) and $N_\infty(a^*) < \infty$ (by assumption),~\cref{lem:unbouned_prob_ratio} implies that $\sup_{t \ge 1}{ \exp\{ \theta_t(i_1) - \theta_t(a^*)} \} = \infty$. This is a contradiction to~\cref{eq:general_case_first_case_bounded_prob_ratio}. 

\textit{First case.} Suppose that $N_\infty(a^*) < \infty$. We argue that this is impossible via contradiction. 
Given the assumption and \cref{lem:at_least_two_actions_infinite_sample_time} we know there must be at least two other sub-optimal actions $i_1, i_2 \in [K]$, $i_1 \ne i_2$, such that $N_\infty(i_1) = N_\infty(i_2) = \infty$.
In particular, let $i_1=\argmin_{a \in [K], N_\infty(a) = \infty} r(a)$
and $i_2=\argmax_{a \in [K], N_\infty(a) = \infty} r(a)$,
hence $r(i_1) < r(i_2) < r(a^*)$.
By \cref{lem:expected_reward_range} (see Appendix) we will also have $r(i_1) < \pi_{\theta_t}^\top r < r(i_2)$ for sufficiently large $t \ge 1$ , which implies for  action $i_1$,
\begin{align}
\label{eq:general_case_first_case_suboptimal_action_param_supermartingale}
    \EEt{\theta_{t+1}(i_1)} &= \theta_t(i_1) + \eta \cdot \pi_{\theta_t}(i_1) \cdot \left(  r(i_1) - \pi_{\theta_t}^\top r \right) < \theta_t(i_1),
\end{align}
for sufficiently large $t \ge 1$, which further implies
%A similar sequence of arguments to \cref{eq:two_action_case_suboptimal_action_param_supermartingale,eq:cumulative_noise_progress} in the two-action case (\cref{thm:two_action_global_convergence}) then imply 
that $\sup_{t \ge 1}{ \theta_t(i_1) } < \infty$. 
Meanwhile, for the optimal action $a^*$, the assumption and \cref{lem:finite_sample_time_implies_finite_parameter} imply that $\inf_{t \ge 1}{ \theta_t(a^*)} > -\infty$.
Combining these two observations gives,
\begin{align}
\label{eq:general_case_first_case_bounded_prob_ratio}
    \sup_{t \ge 1}{ \frac{\pi_{\theta_t}(i_1)}{\pi_{\theta_t}(a^*)} } = \sup_{t \ge 1} \, \exp\{ \theta_t(i_1) - \theta_t(a^*) \}  < \infty \,.
\end{align}
On the other hand, since $N_\infty(i_1) = \infty$ and $N_\infty(a^*) < \infty$ (by assumption), we then have
$\sup_{t \ge 1}{ \exp\{ \theta_t(i_1) - \theta_t(a^*)} \} = \infty$
by \cref{lem:unbouned_prob_ratio} (see Appendix), which contradicts \cref{eq:general_case_first_case_bounded_prob_ratio}. 



\textit{Second case.} Suppose that $N_\infty(a^*) = \infty$. We will argue that $\pi_{\theta_t}(a^*) \to 1$ as $t \to \infty$ almost surely.
First, according to \cref{lem:at_least_two_actions_infinite_sample_time}, there exists at least one sub-optimal action $i_1\in[K]$, $i_1\neq a^*$, such that $N_\infty(i_1) = \infty$.
Let $i_1=\argmin_{a \in [K], N_\infty(a) = \infty} r(a)$.
By \cref{lem:expected_reward_range} and the definition of $a^*$, we have $r(i_1) < \pi_{\theta_t}^\top r < r(a^*)$ for all sufficiently large $t \ge 1$. Since $N_\infty(i_1) = \infty$, using similar calculations to~\cref{eq:two_action_case_suboptimal_action_param_supermartingale,eq:cumulative_noise_progress} in \cref{thm:two_action_global_convergence}, we have, $\theta_t(i_1) \to -\infty$ as $t \to \infty$. We also have, $\inf_{t \ge 1} \theta_t(a^*) > - \infty$ as $t \to \infty$. Hence, $\frac{ \pi_{\theta_t}(a^*) }{ \pi_{\theta_t}(i_1)} = \exp\{ \theta_t(a^*) - \theta_t(i_1) \} \to \infty$ as $t \to \infty$. 

Define $\gA_\infty \coloneqq \left\{ a \in [K] \ | \ N_\infty(a) = \infty \right\}$ as the set of actions that are sampled infinitely often, and note that $| \gA_\infty | \ge 2$ by \cref{lem:at_least_two_actions_infinite_sample_time}. Sort the action indices in $\gA_\infty$ according to their expected reward values in descending order, i.e.,
\begin{align}
\label{eq:general_case_second_case_descending_action_indices}
    r(a^*) > r(i_{|\gA_\infty| - 1}) > r(i_{|\gA_\infty| - 2}) > \cdots > r(i_2) > r(i_1).
\end{align}
\cref{assp:reward_no_ties} is used here to prevent two arms from having the same reward and thus guarantee the inequalities are strict in \cref{eq:general_case_second_case_descending_action_indices}. Next, using similar calculations as in \cref{lem:expected_reward_range}, we have,
\begin{align}
\label{eq:general_case_second_case_reward_difference}
    \pi_{\theta_t}^\top r - r(i_2) > \pi_{\theta_t}(a^*) \cdot \bigg[ r(a^*) - r(i_2) - \sum_{a^- \in \gA^-(i_2)}{ \frac{ \pi_{\theta_t}(a^-)}{ \pi_{\theta_t}(a^*) } \cdot ( r(i_2) - r(a^-)) } \bigg],
\end{align}
where $\gA^-(i_2) \coloneqq \left\{ a^- \in [K]: r(a^-) < r(i_2) \right\}$ is the set of actions that have lower mean reward than $i_2 \in [K]$, and note that $i_1 \in \gA^-(i_2)$. Using the above definitions, we can conclude that $i_1$ is the only arm in $\gA^-(i_2)$ that has been sampled infinitely often. According to \cref{lem:unbouned_prob_ratio}, for all $a^- \in \gA^-(i_2)$ with $a^- \ne i_1$, we have, $\frac{ \pi_{\theta_t}(a^*) }{ \pi_{\theta_t}(a^-)} \to \infty$ as $t \to \infty$, since $N_\infty(a^*) = \infty$ (by assumption) and $N_\infty(a^-) < \infty$ (by \cref{eq:general_case_second_case_descending_action_indices}). Therefore, for all sufficiently large $t$, the probability ratio in \cref{eq:general_case_second_case_reward_difference} $ \frac{ \pi_{\theta_t}(a^*)}{ \pi_{\theta_t}(a^-)} \to \infty$ for all $a^- \in \gA^-(i_2)$, which implies that, for all sufficiently large $t \ge 1$,
\begin{align}
\label{eq:general_case_second_case_positive_reward_difference_a}
    \pi_{\theta_t}^\top r - r(i_2) > 0.5 \cdot \pi_{\theta_t}(a^*) \cdot \big( r(a^*) - r(i_2) \big) > 0.
\end{align}
We have thus shown that $\pi_{\theta_t}^\top r > r(i_2)$. Recall that we had previously proved that $\pi_{\theta_t}^\top r > r(i_1)$. Hence, we will apply this argument recursively: after this point, $i_2 \in [K]$ will become the new ``$i_1 \in [K]$'', and a similar inequality to \cref{eq:two_action_case_suboptimal_action_param_supermartingale} will then hold for $i_2 \in [K]$ from similar calculations to \cref{eq:two_action_case_suboptimal_action_param_supermartingale,eq:cumulative_noise_progress} in \cref{thm:two_action_global_convergence},
establishing $\theta_t(i_2) \to -\infty$ as $t \to \infty$. 
This will imply that for all sufficiently large $t \ge 1$,
\begin{align}
\label{eq:general_case_second_case_positive_reward_difference_b}
    \pi_{\theta_t}^\top r - r(i_3) > 0.5 \cdot \pi_{\theta_t}(a^*) \cdot \big( r(a^*) - r(i_3) \big) > 0.
\end{align}
Continuing the recursive argument, we can conclude for all actions $a \in \gA_\infty$ with $a \ne a^*$ that $\frac{ \pi_{\theta_t}(a^*) }{ \pi_{\theta_t}(a)} \to \infty$ as $t \to \infty$. 
Meanwhile, for all actions $a \not\in \gA_\infty$, \cref{lem:unbouned_prob_ratio} also shows that
$\frac{ \pi_{\theta_t}(a^*) }{ \pi_{\theta_t}(a)} \to \infty$ as $t \to \infty$. Combining these two results yields the conclusion that for all sub-optimal actions $a \in [K]$ with $r(a) < r(a^*)$ we have $\frac{ \pi_{\theta_t}(a^*) }{ \pi_{\theta_t}(a)} \to \infty$ as $t \to \infty$, which implies $\pi_{\theta_t}(a^*) \to 1$ as $t \to \infty$. Thus, we have established almost sure convergence to the globally optimal policy.

\paragraph{Discussion.} \cref{lem:at_least_two_actions_infinite_sample_time} is important to prove that the optimal arm will be sampled infinitely often. In particular,~\cref{lem:at_least_two_actions_infinite_sample_time} guarantees $| \gA_\infty | \ge 2$ and the existence of $i_1$ in \cref{eq:general_case_first_case_suboptimal_action_param_supermartingale}, which can then be used to construct the contradiction in \cref{eq:general_case_first_case_bounded_prob_ratio}. Without \cref{lem:at_least_two_actions_infinite_sample_time}, 
% the relation $r(i_1) < \pi_{\theta_t}^\top r < r(i_2)$ above \cref{eq:general_case_first_case_suboptimal_action_param_supermartingale} does not necessarily hold (since 
$| \gA_\infty |$ might be equal to $1$ and the failure mode in \cref{subsec:failure_model_aggressive_updates} can occur, resulting in \cref{alg:gradient_bandit_algorithm_sampled_reward} not sampling the optimal action infinitely often as $t \to \infty$.

\subsection{Asymptotic Rate of Convergence}

According to \cref{thm:general_action_global_convergence}, almost surely, $\pi_{\theta_t}(a^*) \to 1$ as $t \to \infty$. Therefore, after a large enough time $\tau < \infty$, we have $\pi_{\theta_t}(a^*) \ge 1/2$, which implies that the ``progress'' term in \cref{eq:cumulative_noise_progress} can be lower bounded. With this, an asymptotic rate of convergence can be proved as follows.
\begin{theorem}
\label{thm:asymptotic_rate_of_convergence}
For a large enough $\tau > 0$, for all $T > \tau$, the average sub-optimality decreases at an $O\left(\frac{\ln(T)}{T} \right)$ rate. Formally, if $a^*$ is the optimal arm, then, for a constant $c$,
\begin{align*}
\frac{\sum_{s=\tau}^{T} r(a^*) - \langle \pi_s, r \rangle}{T}  & \leq \frac{c \, \ln(T)}{T - \tau}.
\end{align*}
\end{theorem}




% \section{\textcolor{red}{Finite-time Convergence}}


\begin{itemize}
    \item high-level intuitions: transferring asymptotic arguments to their finite-time correspondences.
    \item a pair of two actions get sampled for $\Omega(\log{T})$ times with high probability.
    \item with high probability then actions' parameters have to shift largely if actions are sampled for enough many times with high probability.
    \item optimal action has to be sampled for enough many times with high probability (contradiction based arguments)
    \item then with high probability, optimal action wins over other sub-optimal actions.
\end{itemize}

%
%|>>|///////////////////////////////////////////////////////////////////////////////////////////|>>|
%|>>|///////////////////////////////////////////////////////////////////////////////////////////|>>|
%|>>|///////////////////////////////////////////////////////////////////////////////////////////|>>|
% \begin{figure}
% %
% %\begin{subfigure}{0.66\textwidth}
% %
% \includegraphics[width=\textwidth]{errors-and-inconsistent--logistic--nbiht--beta=1.png}
% \caption{\label{fig:error-decay-plot:logistic-regression}
% %This experiment was run under the logistic regression model with \betaXnamelr \(  \betaX = 1  \).
% %The \LHS shows the error decay of BIHT approximations empirically and theoretically.
% %%The empirical results were obtained by running \(  100  \) trials of \(  50  \) iterations of the normalized BIHT algorithm when recovering random \(  k  \)-sparse unit vectors.
% %The \RHS displays the fraction of covariates which fall onto opposite sides of
% %the hyperplanes associated with
% %%the hyperplane whose normal vector is
% %the true parameter, \(  \thetaStar  \), and the approximations.
% %The empirical results were obtained by running \(  100  \) trials of recovering random \(  k  \)-sparse unit vectors via the normalized BIHT algorithm for \(  25  \) iterations.
% %The parameters were set as: \(  d = 2000  \), \(  k = 5  \), \(  n = 3000  \), \(  \epsilonX = 0.25  \), and \(  \rhoX = 0.25  \).%
% This experiment demonstrates the convergence behavior of the normalized BIHT algorithm under logistic regression with \betaXnamelr \(  \betaX = 1  \).
% The left plot shows the empirical and theoretical error decay of the BIHT approximations.
% The right plot shows the fraction of covariates which fall onto opposite sides of
% the hyperplanes associated with
% %the hyperplane whose normal vector is
% the true parameter, \(  \thetaStar  \), and the approximations.
% The experiment ran \(  100  \) trials of recovery for \(  25  \) iterations with parameters: \(  d = 2000  \), \(  k = 5  \), \(  n = 3000  \), \(  \epsilonX = 0.25  \), and \(  \rhoX = 0.25  \).%
% }
% %
% %\Description[Two box plots showing the rates of decay of (1) the error, and (2) the fraction of mismatches in a numerical experiment.]{Two side-by-side box plots displaying the number of iterations of BIHT versus (1) the error, and (2) the fraction of mismatches in a numerical experiment, which are seen to track with the theoretically predicted rates of decay. Results are shown for each of the first 25 iterations.}
% %
% %\end{subfigure}
% \end{figure}
% %|>>|///////////////////////////////////////////////////////////////////////////////////////////|>>|
% %|>>|///////////////////////////////////////////////////////////////////////////////////////////|>>|
% %|>>|///////////////////////////////////////////////////////////////////////////////////////////|>>|

% %|<<|\\\\\\\\\\\\\\\\\\\\\\\\\\\\\\\\\\\\\\\\\\\\\\\\\\\\\\\\\\\\\\\\\\\\\\\\\\\\\\\\\\\\\\\\\\\|<<|
% %|<<|\\\\\\\\\\\\\\\\\\\\\\\\\\\\\\\\\\\\\\\\\\\\\\\\\\\\\\\\\\\\\\\\\\\\\\\\\\\\\\\\\\\\\\\\\\\|<<|
% %|<<|\\\\\\\\\\\\\\\\\\\\\\\\\\\\\\\\\\\\\\\\\\\\\\\\\\\\\\\\\\\\\\\\\\\\\\\\\\\\\\\\\\\\\\\\\\\|<<|
% \begin{figure}
% %\begin{subfigure}{0.33\textwidth}
% %
% \centering
% \includegraphics[width=0.85\textwidth]{m-vs-errors--logistic--nbiht--beta=1.png}
% \caption{\label{fig:m-vs-error-plot:logistic-regression}%
% %This plot shows the (roughly linear) relationship between the number of covariates, \(  d  \), (\(  x  \)-axis) and the reciprocal of the squared error (\(  y  \)-axis), where the error is the \(  \lnorm{2}  \)-distance between the true parameter and the approximation obtained after \(  25  \) iterations of the normalized BIHT algorithm under the logistic regression model with \betaXnamelr \(  \betaX = 1  \).
% %The sparsity and dimension parameters were set, respectively, as: \(  d = 2000  \) and \(  k = 5  \).%
% This experiment studies the relationship between the \errorrate and sample complexity for the normalized BIHT algorithm under logistic regression with \betaXnamelr \(  \betaX = 1  \).
% This plot shows the reciprocal of the squared error (\(  y  \)-axis) scaling roughly linearly with the number of covariates, \(  d  \), (\(  x  \)-axis), where the error is the \(  \lnorm{2}  \)-distance between the true parameter and the approximation obtained after \(  25  \) iterations of the algorithm.
% The experiment ran 100 trials of recovery with dimension and sparsity parameters: \(  d = 2000  \) and \(  k = 5  \).%
% }
% %
% %\Description[A box plot of the relationship between the number of measurements and reciprocal of the error in a numerical experiment.]{A box plot displaying the approximately linear scaling of the number of measurements versus the reciprocal of the error in a numerical experiment, where the number of measurements is varied between 100 and 2000 at increments of 100.}
% %\end{subfigure}
% \end{figure}
% %|<<|\\\\\\\\\\\\\\\\\\\\\\\\\\\\\\\\\\\\\\\\\\\\\\\\\\\\\\\\\\\\\\\\\\\\\\\\\\\\\\\\\\\\\\\\\\\|<<|
% %|<<|\\\\\\\\\\\\\\\\\\\\\\\\\\\\\\\\\\\\\\\\\\\\\\\\\\\\\\\\\\\\\\\\\\\\\\\\\\\\\\\\\\\\\\\\\\\|<<|
% %|<<|\\\\\\\\\\\\\\\\\\\\\\\\\\\\\\\\\\\\\\\\\\\\\\\\\\\\\\\\\\\\\\\\\\\\\\\\\\\\\\\\\\\\\\\\\\\|<<|

% %|>>|///////////////////////////////////////////////////////////////////////////////////////////|>>|
% %|>>|///////////////////////////////////////////////////////////////////////////////////////////|>>|
% %|>>|///////////////////////////////////////////////////////////////////////////////////////////|>>|
% \begin{figure}
% %
% %\begin{subfigure}{0.66\textwidth}
% %
% \includegraphics[width=\textwidth]{errors-and-inconsistent--probit--nbiht--beta=1.png}
% \caption{\label{fig:error-decay-plot:probit}
% %This experiment was run under the probit regression model with \betaXnamepr \(  \betaX = 1  \).
% %The \LHS shows the empirical and theoretical error decay of the BIHT approximations.
% %The \RHS shows the fraction of covariates which fall onto opposite sides of
% %the hyperplanes associated with
% %%the hyperplane whose normal vector is
% %the true parameter, \(  \thetaStar  \), and the approximations.
% %The experiment ran \(  100  \) trials of recovery with the normalized BIHT algorithm for \(  25  \) iterations under the probit regression model with \betaXnamepr \(  \betaX = 1  \) with the parameters: \(  d = 2000  \), \(  k = 5  \), \(  n = 3000  \), \(  \epsilonX = 0.25  \), and \(  \rhoX = 0.25  \).%
% This experiment demonstrates the convergence behavior of the normalized BIHT algorithm under probit regression with \betaXnamepr \(  \betaX = 1  \).
% The left plot shows the empirical and theoretical error decay of the BIHT approximations.
% The right plot shows the fraction of covariates which fall onto opposite sides of
% the hyperplanes associated with
% %the hyperplane whose normal vector is
% the true parameter, \(  \thetaStar  \), and the approximations.
% The experiment ran \(  100  \) trials of recovery for \(  25  \) iterations with parameters: \(  d = 2000  \), \(  k = 5  \), \(  n = 3000  \), \(  \epsilonX = 0.25  \), and \(  \rhoX = 0.25  \).%
% }
% %
% %\Description[Two box plots showing the rates of decay of (1) the error, and (2) the fraction of mismatches in a numerical experiment.]{Two side-by-side box plots displaying the number of iterations of BIHT versus (1) the error, and (2) the fraction of mismatches in a numerical experiment, which are seen to track with the theoretically predicted rates of decay. Results are shown for each of the first 25 iterations.}
% %
% %\end{subfigure}
% \end{figure}
% %|>>|///////////////////////////////////////////////////////////////////////////////////////////|>>|
% %|>>|///////////////////////////////////////////////////////////////////////////////////////////|>>|
% %|>>|///////////////////////////////////////////////////////////////////////////////////////////|>>|

% %|<<|\\\\\\\\\\\\\\\\\\\\\\\\\\\\\\\\\\\\\\\\\\\\\\\\\\\\\\\\\\\\\\\\\\\\\\\\\\\\\\\\\\\\\\\\\\\|<<|
% %|<<|\\\\\\\\\\\\\\\\\\\\\\\\\\\\\\\\\\\\\\\\\\\\\\\\\\\\\\\\\\\\\\\\\\\\\\\\\\\\\\\\\\\\\\\\\\\|<<|
% %|<<|\\\\\\\\\\\\\\\\\\\\\\\\\\\\\\\\\\\\\\\\\\\\\\\\\\\\\\\\\\\\\\\\\\\\\\\\\\\\\\\\\\\\\\\\\\\|<<|
% \begin{figure}
% %\begin{subfigure}{0.33\textwidth}
% %
% \centering
% \includegraphics[width=0.85\textwidth]{m-vs-errors--probit--nbiht--beta=1.png}
% \caption{\label{fig:m-vs-error-plot:probit}%
% %This plot shows the (roughly linear) relationship between the number of covariates, \(  d  \), (\(  x  \)-axis) and the reciprocal of the squared error (\(  y  \)-axis), where the error is the \(  \lnorm{2}  \)-distance between the true parameter and the approximation obtained after \(  25  \) iterations of the normalized BIHT algorithm under the probit regression model with \betaXnamepr \(  \betaX = 1  \).
% %The sparsity and dimension parameters were set, respectively, as: \(  d = 2000  \) and \(  k = 5  \).%
% This experiment studies the relationship between the \errorrate and sample complexity for the normalized BIHT algorithm under probit regression with \betaXnamepr \(  \betaX = 1  \).
% This plot shows the reciprocal of the squared error (\(  y  \)-axis) scaling roughly linearly with the number of covariates, \(  d  \), (\(  x  \)-axis), where the error is the \(  \lnorm{2}  \)-distance between the true parameter and the approximation obtained after \(  25  \) iterations of the algorithm.
% The experiment ran 100 trials of recovery with dimension and sparsity parameters: \(  d = 2000  \) and \(  k = 5  \).%
% }
% %
% %\Description[A box plot of the relationship between the number of measurements and reciprocal of the error in a numerical experiment.]{A box plot displaying the approximately linear scaling of the number of measurements versus the reciprocal of the error in a numerical experiment, where the number of measurements is varied between 100 and 2000 at increments of 100.}
% %\end{subfigure}
% \end{figure}
% %|<<|\\\\\\\\\\\\\\\\\\\\\\\\\\\\\\\\\\\\\\\\\\\\\\\\\\\\\\\\\\\\\\\\\\\\\\\\\\\\\\\\\\\\\\\\\\\|<<|
% %|<<|\\\\\\\\\\\\\\\\\\\\\\\\\\\\\\\\\\\\\\\\\\\\\\\\\\\\\\\\\\\\\\\\\\\\\\\\\\\\\\\\\\\\\\\\\\\|<<|
% %|<<|\\\\\\\\\\\\\\\\\\\\\\\\\\\\\\\\\\\\\\\\\\\\\\\\\\\\\\\\\\\\\\\\\\\\\\\\\\\\\\\\\\\\\\\\\\\|<<|

% %|>>|///////////////////////////////////////////////////////////////////////////////////////////|>>|
% %|>>|///////////////////////////////////////////////////////////////////////////////////////////|>>|
% %|>>|///////////////////////////////////////////////////////////////////////////////////////////|>>|
\begin{figure}
%
%\begin{subfigure}{0.66\textwidth}
%
\includegraphics[width=\textwidth]{errors--logistic--nbiht-beta=1.png}
\includegraphics[width=\textwidth]{errors--logistic--biht-beta=1.png}
\caption{\label{fig:error-decay-nbiht-biht-comparison-beta=1-plot:logistic-regression}
This experiment compares the iterative approximation errors for BIHT with (top plot) and without (bottom plot) the normalization step of the algorithm under logistic regression with \betaXnamelr \(  \betaX = 1  \).
The error is the \(  \lnorm{2}  \)-distance between the normalized approximation and the true parameter.
In both plots, the theoretical error decay for the normalized version of BIHT with logistic regression is displayed for reference.
The experiment ran \(  100  \) trials of recovery for \(  30  \) iterations with parameters: \(  d = 2000  \), \(  k = 5  \), \(  n = 3000  \), \(  \epsilonX = 0.25  \), and \(  \rhoX = 0.25  \).%
}
%
%\end{subfigure}
\end{figure}
%|>>|///////////////////////////////////////////////////////////////////////////////////////////|>>|
%|>>|///////////////////////////////////////////////////////////////////////////////////////////|>>|
%|>>|///////////////////////////////////////////////////////////////////////////////////////////|>>|

%|>>|///////////////////////////////////////////////////////////////////////////////////////////|>>|
%|>>|///////////////////////////////////////////////////////////////////////////////////////////|>>|
%|>>|///////////////////////////////////////////////////////////////////////////////////////////|>>|
\begin{figure}
%
%\begin{subfigure}{0.66\textwidth}
%
\includegraphics[width=\textwidth]{errors--sign--nbiht.png}
\includegraphics[width=\textwidth]{errors--sign--biht.png}
\caption{\label{fig:error-decay-nbiht-biht-comparison-beta=1-plot:sign}
This experiment compares the iterative approximation errors for BIHT with (top plot) and without (bottom plot) the normalization step of the algorithm under the noiseless model.
The error is the \(  \lnorm{2}  \)-distance between the normalized approximation and the true parameter.
In both plots, the theoretical error decay for the normalized version of BIHT in the noiseless setting---established by \cite{matsumoto2022binary}---is displayed for reference.
The experiment ran \(  100  \) trials of recovery for \(  30  \) iterations with parameters: \(  d = 2000  \), \(  k = 5  \), \(  n = 700  \), \(  \epsilonX = 0.25  \), and \(  \rhoX = 0.25  \).%
}
%
%\end{subfigure}
\end{figure}
%|>>|///////////////////////////////////////////////////////////////////////////////////////////|>>|
%|>>|///////////////////////////////////////////////////////////////////////////////////////////|>>|
%|>>|///////////////////////////////////////////////////////////////////////////////////////////|>>|

\begin{comment}

%|>>|///////////////////////////////////////////////////////////////////////////////////////////|>>|
%|>>|///////////////////////////////////////////////////////////////////////////////////////////|>>|
%|>>|///////////////////////////////////////////////////////////////////////////////////////////|>>|
\begin{figure}
%
%\begin{subfigure}{0.66\textwidth}
%
\includegraphics[width=\textwidth]{lr-nbiht-errors-and-inconsistent-n=2000-k=5-m=1000-epsilon=0.05-rho=0.05-num_trials=100-figsize=(28, 6)--2-1--cropped.png}%{lr-nbiht-errors-and-inconsistent-n=2000-k=5-m=1000-epsilon=0.05-rho=0.05-num_trials=100--2-1b.png}
\caption{\label{fig:error-decay-plot:logistic-regression}
This experiment was run under the logistic regression model with \betaXnamelr \(  \betaX = 1  \).
The \LHS shows the error decay of BIHT approximations empirically and theoretically.
%The empirical results were obtained by running \(  100  \) trials of \(  50  \) iterations of the normalized BIHT algorithm when recovering random \(  k  \)-sparse unit vectors.
The \RHS displays the fraction of covariates which fall onto opposite sides of
the hyperplanes associated with
%the hyperplane whose normal vector is
the true parameter, \(  \thetaStar  \), and the approximations.
The empirical results were obtained by running \(  100  \) trials of recovering random \(  k  \)-sparse unit vectors via the normalized BIHT algorithm for \(  25  \) iterations.
The parameters were set as: \(  d = 2000  \), \(  k = 5  \), \(  n = 1000  \), \(  \epsilonX = 0.05  \), and \(  \rhoX = 0.05  \).%
}
%
%\Description[Two box plots showing the rates of decay of (1) the error, and (2) the fraction of mismatches in a numerical experiment.]{Two side-by-side box plots displaying the number of iterations of BIHT versus (1) the error, and (2) the fraction of mismatches in a numerical experiment, which are seen to track with the theoretically predicted rates of decay. Results are shown for each of the first 25 iterations.}
%
%\end{subfigure}
\end{figure}
%|>>|///////////////////////////////////////////////////////////////////////////////////////////|>>|
%|>>|///////////////////////////////////////////////////////////////////////////////////////////|>>|
%|>>|///////////////////////////////////////////////////////////////////////////////////////////|>>|

%|<<|\\\\\\\\\\\\\\\\\\\\\\\\\\\\\\\\\\\\\\\\\\\\\\\\\\\\\\\\\\\\\\\\\\\\\\\\\\\\\\\\\\\\\\\\\\\|<<|
%|<<|\\\\\\\\\\\\\\\\\\\\\\\\\\\\\\\\\\\\\\\\\\\\\\\\\\\\\\\\\\\\\\\\\\\\\\\\\\\\\\\\\\\\\\\\\\\|<<|
%|<<|\\\\\\\\\\\\\\\\\\\\\\\\\\\\\\\\\\\\\\\\\\\\\\\\\\\\\\\\\\\\\\\\\\\\\\\\\\\\\\\\\\\\\\\\\\\|<<|
\begin{figure}
%\begin{subfigure}{0.33\textwidth}
%
\centering
\includegraphics[width=0.85\textwidth]{lr-nbiht-m-vs-errors--n=2000-k=5-m=[100-2000]-epsilon=0.1-rho=0.05-num_trials=100-figsize=(24, 6)--2-1--cropped.png}%{lr-nbiht-m-vs-errors--n=2000-k=5-m=[100-2000]-epsilon=0.1-rho=0.05-num_trials=100--2-1b.png}
\caption{\label{fig:m-vs-error-plot:logistic-regression}%
This plot shows the (roughly linear) relationship between the number of covariates, \(  d  \), (\(  x  \)-axis) and the reciprocal of the squared error (\(  y  \)-axis), where the error is the \(  \lnorm{2}  \)-distance between the true parameter and the approximation obtained after \(  25  \) iterations of the normalized BIHT algorithm under the logistic regression model with \betaXnamelr \(  \betaX = 1  \).
The sparsity and dimension parameters were set, respectively, as: \(  d = 2000  \) and \(  k = 5  \).%
}
%
%\Description[A box plot of the relationship between the number of measurements and reciprocal of the error in a numerical experiment.]{A box plot displaying the approximately linear scaling of the number of measurements versus the reciprocal of the error in a numerical experiment, where the number of measurements is varied between 100 and 2000 at increments of 100.}
%\end{subfigure}
\end{figure}
%|<<|\\\\\\\\\\\\\\\\\\\\\\\\\\\\\\\\\\\\\\\\\\\\\\\\\\\\\\\\\\\\\\\\\\\\\\\\\\\\\\\\\\\\\\\\\\\|<<|
%|<<|\\\\\\\\\\\\\\\\\\\\\\\\\\\\\\\\\\\\\\\\\\\\\\\\\\\\\\\\\\\\\\\\\\\\\\\\\\\\\\\\\\\\\\\\\\\|<<|
%|<<|\\\\\\\\\\\\\\\\\\\\\\\\\\\\\\\\\\\\\\\\\\\\\\\\\\\\\\\\\\\\\\\\\\\\\\\\\\\\\\\\\\\\\\\\\\\|<<|

%|>>|///////////////////////////////////////////////////////////////////////////////////////////|>>|
%|>>|///////////////////////////////////////////////////////////////////////////////////////////|>>|
%|>>|///////////////////////////////////////////////////////////////////////////////////////////|>>|
\begin{figure}
%
%\begin{subfigure}{0.66\textwidth}
%
\includegraphics[width=\textwidth]{pr-nbiht-errors-and-inconsistent-n=2000-k=5-m=1000-epsilon=0.05-rho=0.05-num_trials=100-figsize=(28, 6)--2-1--cropped.png}%{pr-nbiht-errors-and-inconsistent-n=2000-k=5-m=1000-epsilon=0.05-rho=0.05-num_trials=100--2-1c.png}
\caption{\label{fig:error-decay-plot:probit}
This experiment was run under the probit regression model with \betaXnamepr \(  \betaX = 1  \).
The \LHS shows the error decay of BIHT approximations empirically and theoretically.
%The empirical results were obtained by running \(  100  \) trials of \(  50  \) iterations of the normalized BIHT algorithm when recovering random \(  k  \)-sparse unit vectors.
The \RHS displays the fraction of covariates which fall onto opposite sides of
the hyperplanes associated with
%the hyperplane whose normal vector is
the true parameter, \(  \thetaStar  \), and the approximations.
The empirical results were obtained by running \(  100  \) trials of recovering random \(  k  \)-sparse unit vectors via the normalized BIHT algorithm for \(  25  \) iterations.
The parameters were set as: \(  d = 2000  \), \(  k = 5  \), \(  n = 1000  \), \(  \epsilonX = 0.05  \), and \(  \rhoX = 0.05  \).%
}
%
%\Description[Two box plots showing the rates of decay of (1) the error, and (2) the fraction of mismatches in a numerical experiment.]{Two side-by-side box plots displaying the number of iterations of BIHT versus (1) the error, and (2) the fraction of mismatches in a numerical experiment, which are seen to track with the theoretically predicted rates of decay. Results are shown for each of the first 25 iterations.}
%
%\end{subfigure}
\end{figure}
%|>>|///////////////////////////////////////////////////////////////////////////////////////////|>>|
%|>>|///////////////////////////////////////////////////////////////////////////////////////////|>>|
%|>>|///////////////////////////////////////////////////////////////////////////////////////////|>>|

%|<<|\\\\\\\\\\\\\\\\\\\\\\\\\\\\\\\\\\\\\\\\\\\\\\\\\\\\\\\\\\\\\\\\\\\\\\\\\\\\\\\\\\\\\\\\\\\|<<|
%|<<|\\\\\\\\\\\\\\\\\\\\\\\\\\\\\\\\\\\\\\\\\\\\\\\\\\\\\\\\\\\\\\\\\\\\\\\\\\\\\\\\\\\\\\\\\\\|<<|
%|<<|\\\\\\\\\\\\\\\\\\\\\\\\\\\\\\\\\\\\\\\\\\\\\\\\\\\\\\\\\\\\\\\\\\\\\\\\\\\\\\\\\\\\\\\\\\\|<<|
\begin{figure}
%\begin{subfigure}{0.33\textwidth}
%
\centering
\includegraphics[width=0.85\textwidth]{pr-nbiht-m-vs-errors--n=2000-k=5-m=[100-2000]-epsilon=0.1-rho=0.05-num_trials=100-figsize=(24, 6)--2-1--cropped.png}%{pr-nbiht-m-vs-errors--n=2000-k=5-m=[100-2000]-epsilon=0.1-rho=0.05-num_trials=100--2-1f.png}
\caption{\label{fig:m-vs-error-plot:probit}%
This plot shows the (roughly linear) relationship between the number of covariates, \(  d  \), (\(  x  \)-axis) and the reciprocal of the squared error (\(  y  \)-axis), where the error is the \(  \lnorm{2}  \)-distance between the true parameter and the approximation obtained after \(  25  \) iterations of the normalized BIHT algorithm under the probit regression model with \betaXnamepr \(  \betaX = 1  \).
The sparsity and dimension parameters were set, respectively, as: \(  d = 2000  \) and \(  k = 5  \).%
}
%
%\Description[A box plot of the relationship between the number of measurements and reciprocal of the error in a numerical experiment.]{A box plot displaying the approximately linear scaling of the number of measurements versus the reciprocal of the error in a numerical experiment, where the number of measurements is varied between 100 and 2000 at increments of 100.}
%\end{subfigure}
\end{figure}
%|<<|\\\\\\\\\\\\\\\\\\\\\\\\\\\\\\\\\\\\\\\\\\\\\\\\\\\\\\\\\\\\\\\\\\\\\\\\\\\\\\\\\\\\\\\\\\\|<<|
%|<<|\\\\\\\\\\\\\\\\\\\\\\\\\\\\\\\\\\\\\\\\\\\\\\\\\\\\\\\\\\\\\\\\\\\\\\\\\\\\\\\\\\\\\\\\\\\|<<|
%|<<|\\\\\\\\\\\\\\\\\\\\\\\\\\\\\\\\\\\\\\\\\\\\\\\\\\\\\\\\\\\\\\\\\\\\\\\\\\\\\\\\\\\\\\\\\\\|<<|

\end{comment}

%!TEX root =  neurips_2024.tex
\vskip-2ex
\section{Simulation Study}
\label{sec:sim}
\vskip-1ex

To validate and enhance the theoretical findings, we ran experiments with a four action bandit environment
($K=4$)
with a true mean reward vector of $r = (0.2, 0.05, -0.1, -0.4)^\top \in \sR^4$.
The reward distribution $P_a$ for arm $1\le a \le 4$ is Gaussian, centered at $r(a)$
and with a standard deviation of $0.1$.
The environment is chosen to illustrate various phenomenon, which we discuss after presenting the results.
%We conduct a few simple simulations to empirically verify the theoretical findings.
The algorithm is \cref{alg:gradient_bandit_algorithm_sampled_reward}
with 
$\theta_1 = \rvzero \in \sR^K$.
%We  consider a stochastic bandit
%problem with $K = 4$ actions and . In each iteration $t \ge 1$ of \cref{alg:gradient_bandit_algorithm_sampled_reward}, given a sampled action $a_t \sim \pi_{\theta_t}(\cdot)$, the observed reward is generated as $R_t(a_t) = r(a_t) + 0.1 \cdot Z_t$, where $Z_t \sim \gN(0, 1)$ is random Gaussian noise. 

For comparison, assuming that the random rewards belong to the $[-1,1]$ interval,
the only result for the stochastic gradient bandit algorithm~\citep[Lemma 4.6]{mei2024stochastic}
that allowed a constant learning rate
required that the learning rate be less 
than $\eta_c = \frac{\Delta^2}{40 \cdot K^{3/2} \cdot R_{\max}^3 } = \frac{9}{128000} \approx 0.00007$, where we used $\Delta = 0.15$, $K = 4$, and $R_{\max} = 1$.
While technically, the result does not apply to our case where the reward distributions have unbounded support, the probability of the reward landing outside of $[-1,1]$ is in the order of $10^{-9}$.
Choosing $R_{\max}$ to be larger, this probability falls extremely quickly, which suggests that the above threshold is generous.
For the experiments
 we use the learning rates $\eta \in \{1, 10, 100, 1000\}$, that are several orders of magnitudes larger than $\eta_c$.
 For each learning rate, we plot the outcome of $10$ runs, corresponding to different random seeds.
Each run lasts $10^6 (\approx e^{14})$ iterations. 
The log-suboptimality gaps for the $4\times 10$ cases are shown on
Figures~\ref{fig::sub_optimality_gap_general_action_case_eta_1}-\ref{fig::sub_optimality_gap_general_action_case_eta_1000}, 
where they are plotted against the logarithm of time. 
Additional results for $K=2$ arms are shown in Appendix~\ref{app:sim}.
Note that in this example small sub-optimality implies that the optimal arm is chosen with high probability.
In what follows, we discuss the results in the plots. 

\paragraph{Asymptotic convergence.} %We run \cref{alg:gradient_bandit_algorithm_sampled_reward} on 
For the smaller learning rates of $\eta = 1$ and $10$, all $10$ seeds rapidly and steadily converge, reaching a sub-optimality of $e^{-14}$ or less. For $\eta = 100$ and $1000$, most of the runs reach even small error even faster, but some runs are ``stuck'' even after $10^6$ steps. Note that this does not contradict the theoretical result; nor do we suspect numerical issues. As seen for the case of $\eta=100$, even after a long phase with little to no progress, a run can ``recover'' (see the grey curve). In fact, it is reasonable to expect that the price of increasing the learning rate is larger variance; as seen in these plots (subplots (a) and (b) are also attesting to this). Differences between learning rates are further discussed below.


% , while a few take longer to converge as we show in Appendix~\ref{app:sim}. 

\paragraph{Non-monotone objective value.} Using a very small learning rate guarantees monotonic improvement (in expectation) in the policy's expected reward~\citep{mei2024stochastic}. 
Conversely, a large learning rate results in non-monotonic evolution of the expected rewards $\{ \pi_{\theta_t}^\top r \}_{t \ge 1}$, even in the final stages of convergence, as can be seen clearly for 
the learning rates of $\eta = 1$ and $10$ in Figures~\ref{fig::sub_optimality_gap_general_action_case_eta_1} and~\ref{fig::sub_optimality_gap_general_action_case_eta_10}. For larger $\eta$, the non-monotone behavior happens over longer periods and is less visible in the plots. This is because using large learning rates causes the policy to rapidly increase the parameters for some action, after which the gradient becomes small, limiting further progress.
% and the recovery to the optimal arm is slower. This is related to the above-mentioned slower convergence of some runs for larger $\eta$.

\begin{figure}
\centering
\begin{subfigure}[b]{.39\linewidth}
\includegraphics[width=\linewidth]{figs/large_learning_rate_general_action_log_subopt_eta_1.00.pdf}
\caption{$\eta = 1$.}\label{fig::sub_optimality_gap_general_action_case_eta_1}
\end{subfigure}
\begin{subfigure}[b]{.39\linewidth}
\includegraphics[width=\linewidth]{figs/large_learning_rate_general_action_log_subopt_eta_10.00.pdf}
\caption{$\eta = 10$.}\label{fig::sub_optimality_gap_general_action_case_eta_10}
\end{subfigure}\\
\begin{subfigure}[b]{.39\linewidth}
\includegraphics[width=\linewidth]{figs/large_learning_rate_general_action_log_subopt_eta_100.00.pdf}
\caption{$\eta = 100$.}\label{fig::sub_optimality_gap_general_action_case_eta_100}
\end{subfigure}
\begin{subfigure}[b]{.39\linewidth}
\includegraphics[width=\linewidth]{figs/large_learning_rate_general_action_log_subopt_eta_1000.00.pdf}
\caption{$\eta = 1000$.}\label{fig::sub_optimality_gap_general_action_case_eta_1000}
\end{subfigure}
\caption{
Log sub-optimality gap, 
$\log{ (r(a^*) - \pi_{\theta_t}^\top r) } $, plotted against the logarithm of time,  $\log{t}$, in a $4$-action problem with various learning rates, $\eta$. 
Each subplot shows a run with a specific learning rate. The curves in a subplot correspond to 10 different random seeds. Theory predicts that essentially all seeds will lead to a curve converging to zero ($-\infty$ in these plots). For a discussion of the results, see the text.,
}
\label{fig:visualization_general_action_case}
\vspace{-10pt}
\end{figure}
\paragraph{Rate of convergence.} Figures~\ref{fig::sub_optimality_gap_general_action_case_eta_1} and~\ref{fig::sub_optimality_gap_general_action_case_eta_10}, where the log-log plot has a slope of nearly $-1$, give some evidence that an $O(1/t)$ asymptotic rate is achieved. In general, such a rate cannot be improved in terms of $t$ \citep{lai1985asymptotically}. \cref{thm:asymptotic_rate_of_convergence} gives a weaker version of convergence rate over averaged iterates (not last iterate), which is slightly worse than $O(1/t)$. More work is needed to verify if the asymptotic convergence rate in \cref{thm:asymptotic_rate_of_convergence} is improvable or not. 

\paragraph{Different learning rates.} 
% The results do not characterize the detailed effects of different learning rates. Presumably, using $\eta = 1$ would lead to very different behavior in practice compared to using $\eta = 1000$. 
Two observations can be made from Figure~\ref{fig:visualization_general_action_case} regarding the effect of using different $\eta$ values: \textbf{First}, during the final stage of convergence when $r(a^*) - \pi_{\theta_t}^\top r \approx 0$, using larger $\eta$ results in faster convergence on average. As $\eta$ increases, the order of $\log{ (r(a^*) - \pi_{\theta_t}^\top r) } $ also changes from $e^{-14}$ ($\eta = 1$), to $e^{-20}$ ($\eta = 100$), and $e^{-200}$ ($\eta = 1000$). We conjecture that the asymptotic rate of convergence has an $O(1/\eta)$ dependence. 
% However, this can only be verified after proving a rate of convergence result, which still remains open. 
\textbf{Second}, using larger learning rates can take a longer time to enter the final stage of convergence. When $\eta = 1$ or $10$, all curves quickly enter the final stage of $r(a^*) - \pi_{\theta_t}^\top r \approx 0$. However, for larger $\eta$ values, $1/10$ runs ($\eta = 100$) and $3/10$ runs ($\eta = 1000$) result in $r(a^*) - \pi_{\theta_t}^\top r $ values far from $0$ even after $10^6$ iterations. These runs take orders of magnitude more iterations to eventually achieve $r(a^*) - \pi_{\theta_t}^\top r \approx 0$. These situations correspond to the policy $\pi_{\theta_t}$ getting stuck near sub-optimal corners of the simplex, meaning that $\pi_{\theta_t}(i) \approx 1$ for a sub-optimal action $i \in [K]$ with $r(i) < r(a^*)$. In such cases, even Softmax PG with the true gradient can remain stuck on a sub-optimal plateau for an extremely long time~\citep{mei2020global}. However, the reason why larger learning rates lead to longer plateaus in the stochastic setting remains unclear.

\paragraph{Trade-offs and multi-stage chracterizations of convergence.} Given the above observations, there appears to exist a trade-off for $\eta$: larger $\eta$ values result in faster convergence during the final stage where $r(a^*) - \pi_{\theta_t}^\top r \approx 0$, but at the the cost of taking far longer to enter this final stage of convergence. Since asymptotic convergence results are insufficient for explaining these subtleties in a satisfactory manner, a more refined analysis that considers the different stages of convergence is required. 
% needed to better understand the effect of using different $\eta$ values.



%!TEX root =  neurips_2024.tex

\vskip-2ex
\section{Conclusions and Future Directions}
\label{sec:conclusion}
\vskip-1ex

This work refines our understanding of stochastic gradient bandit algorithms by proving that it converges to a globally optimal policy almost surely with \emph{any} constant learning rate. Our new proof strategy based on the asymptotics of sample counts opens new directions for better characterizing exploration effects of stochastic gradient methods, while also suggesting interesting new questions. 
Characterizing the multiple stages of convergence remains another interesting future direction.
One interesting possibility is  that there might exist an optimal time-dependent scheme for \emph{increasing} the learning rate (such as $\eta \in O(\log{t})$) to accelerate convergence, rather than use a constant $\eta \in O(1)$. This is corroborated by our experiments:
As seen in \cref{fig:visualization_general_action_case}, small learning rates perform better during the early stages of optimization, while larger learning rates achieve faster convergence during the final stage. Other directions include extending our bandit results to the more general RL setting \citep{williams1992simple}, as well as extending our results for the softmax tabular parameterization to handle function approximation~\citep{agarwal2021theory}.

\textbf{Limitations:} While this work establishes a surprising asymptotic convergence result for any constant learning rate, it does not shed light on the effect of different learning rates on the convergence. Moreover, our analysis is limited to multi-armed bandits, and does not immediately extend to the general RL setting. These aspects are the main limitations of this paper.  
% , along with the lack of a more detailed empirical study are the main limitations of this paper. 
% Extensions to more general RL scenarios is another important direction for future work.

\textbf{Broader impact:} This is primarily theoretical work on a fundamental algorithm that is used broadly in RL applications. We expect these results to improve the research community's understanding of the basic stochastic gradient bandit method.



% \smallskip
% \myparagraph{Acknowledgments} We thank the reviewers for their comments.
% The work by Moshe Tennenholtz was supported by funding from the
% European Research Council (ERC) under the European Union's Horizon
% 2020 research and innovation programme (grant agreement 740435).


{\small
\bibliography{neurips_refs}
}
\bibliographystyle{plain}

%%%%%%%%%%%%%%%%%%%%%%%%%%%%%%%%%%%%%%%%%%%%%%%%%%%%%%%%%%%%

\appendix
\newpage
% \input{not_used_asymptotic_rate}

% \subsection{Lloyd-Max Algorithm}
\label{subsec:Lloyd-Max}
For a given quantization bitwidth $B$ and an operand $\bm{X}$, the Lloyd-Max algorithm finds $2^B$ quantization levels $\{\hat{x}_i\}_{i=1}^{2^B}$ such that quantizing $\bm{X}$ by rounding each scalar in $\bm{X}$ to the nearest quantization level minimizes the quantization MSE. 

The algorithm starts with an initial guess of quantization levels and then iteratively computes quantization thresholds $\{\tau_i\}_{i=1}^{2^B-1}$ and updates quantization levels $\{\hat{x}_i\}_{i=1}^{2^B}$. Specifically, at iteration $n$, thresholds are set to the midpoints of the previous iteration's levels:
\begin{align*}
    \tau_i^{(n)}=\frac{\hat{x}_i^{(n-1)}+\hat{x}_{i+1}^{(n-1)}}2 \text{ for } i=1\ldots 2^B-1
\end{align*}
Subsequently, the quantization levels are re-computed as conditional means of the data regions defined by the new thresholds:
\begin{align*}
    \hat{x}_i^{(n)}=\mathbb{E}\left[ \bm{X} \big| \bm{X}\in [\tau_{i-1}^{(n)},\tau_i^{(n)}] \right] \text{ for } i=1\ldots 2^B
\end{align*}
where to satisfy boundary conditions we have $\tau_0=-\infty$ and $\tau_{2^B}=\infty$. The algorithm iterates the above steps until convergence.

Figure \ref{fig:lm_quant} compares the quantization levels of a $7$-bit floating point (E3M3) quantizer (left) to a $7$-bit Lloyd-Max quantizer (right) when quantizing a layer of weights from the GPT3-126M model at a per-tensor granularity. As shown, the Lloyd-Max quantizer achieves substantially lower quantization MSE. Further, Table \ref{tab:FP7_vs_LM7} shows the superior perplexity achieved by Lloyd-Max quantizers for bitwidths of $7$, $6$ and $5$. The difference between the quantizers is clear at 5 bits, where per-tensor FP quantization incurs a drastic and unacceptable increase in perplexity, while Lloyd-Max quantization incurs a much smaller increase. Nevertheless, we note that even the optimal Lloyd-Max quantizer incurs a notable ($\sim 1.5$) increase in perplexity due to the coarse granularity of quantization. 

\begin{figure}[h]
  \centering
  \includegraphics[width=0.7\linewidth]{sections/figures/LM7_FP7.pdf}
  \caption{\small Quantization levels and the corresponding quantization MSE of Floating Point (left) vs Lloyd-Max (right) Quantizers for a layer of weights in the GPT3-126M model.}
  \label{fig:lm_quant}
\end{figure}

\begin{table}[h]\scriptsize
\begin{center}
\caption{\label{tab:FP7_vs_LM7} \small Comparing perplexity (lower is better) achieved by floating point quantizers and Lloyd-Max quantizers on a GPT3-126M model for the Wikitext-103 dataset.}
\begin{tabular}{c|cc|c}
\hline
 \multirow{2}{*}{\textbf{Bitwidth}} & \multicolumn{2}{|c|}{\textbf{Floating-Point Quantizer}} & \textbf{Lloyd-Max Quantizer} \\
 & Best Format & Wikitext-103 Perplexity & Wikitext-103 Perplexity \\
\hline
7 & E3M3 & 18.32 & 18.27 \\
6 & E3M2 & 19.07 & 18.51 \\
5 & E4M0 & 43.89 & 19.71 \\
\hline
\end{tabular}
\end{center}
\end{table}

\subsection{Proof of Local Optimality of LO-BCQ}
\label{subsec:lobcq_opt_proof}
For a given block $\bm{b}_j$, the quantization MSE during LO-BCQ can be empirically evaluated as $\frac{1}{L_b}\lVert \bm{b}_j- \bm{\hat{b}}_j\rVert^2_2$ where $\bm{\hat{b}}_j$ is computed from equation (\ref{eq:clustered_quantization_definition}) as $C_{f(\bm{b}_j)}(\bm{b}_j)$. Further, for a given block cluster $\mathcal{B}_i$, we compute the quantization MSE as $\frac{1}{|\mathcal{B}_{i}|}\sum_{\bm{b} \in \mathcal{B}_{i}} \frac{1}{L_b}\lVert \bm{b}- C_i^{(n)}(\bm{b})\rVert^2_2$. Therefore, at the end of iteration $n$, we evaluate the overall quantization MSE $J^{(n)}$ for a given operand $\bm{X}$ composed of $N_c$ block clusters as:
\begin{align*}
    \label{eq:mse_iter_n}
    J^{(n)} = \frac{1}{N_c} \sum_{i=1}^{N_c} \frac{1}{|\mathcal{B}_{i}^{(n)}|}\sum_{\bm{v} \in \mathcal{B}_{i}^{(n)}} \frac{1}{L_b}\lVert \bm{b}- B_i^{(n)}(\bm{b})\rVert^2_2
\end{align*}

At the end of iteration $n$, the codebooks are updated from $\mathcal{C}^{(n-1)}$ to $\mathcal{C}^{(n)}$. However, the mapping of a given vector $\bm{b}_j$ to quantizers $\mathcal{C}^{(n)}$ remains as  $f^{(n)}(\bm{b}_j)$. At the next iteration, during the vector clustering step, $f^{(n+1)}(\bm{b}_j)$ finds new mapping of $\bm{b}_j$ to updated codebooks $\mathcal{C}^{(n)}$ such that the quantization MSE over the candidate codebooks is minimized. Therefore, we obtain the following result for $\bm{b}_j$:
\begin{align*}
\frac{1}{L_b}\lVert \bm{b}_j - C_{f^{(n+1)}(\bm{b}_j)}^{(n)}(\bm{b}_j)\rVert^2_2 \le \frac{1}{L_b}\lVert \bm{b}_j - C_{f^{(n)}(\bm{b}_j)}^{(n)}(\bm{b}_j)\rVert^2_2
\end{align*}

That is, quantizing $\bm{b}_j$ at the end of the block clustering step of iteration $n+1$ results in lower quantization MSE compared to quantizing at the end of iteration $n$. Since this is true for all $\bm{b} \in \bm{X}$, we assert the following:
\begin{equation}
\begin{split}
\label{eq:mse_ineq_1}
    \tilde{J}^{(n+1)} &= \frac{1}{N_c} \sum_{i=1}^{N_c} \frac{1}{|\mathcal{B}_{i}^{(n+1)}|}\sum_{\bm{b} \in \mathcal{B}_{i}^{(n+1)}} \frac{1}{L_b}\lVert \bm{b} - C_i^{(n)}(b)\rVert^2_2 \le J^{(n)}
\end{split}
\end{equation}
where $\tilde{J}^{(n+1)}$ is the the quantization MSE after the vector clustering step at iteration $n+1$.

Next, during the codebook update step (\ref{eq:quantizers_update}) at iteration $n+1$, the per-cluster codebooks $\mathcal{C}^{(n)}$ are updated to $\mathcal{C}^{(n+1)}$ by invoking the Lloyd-Max algorithm \citep{Lloyd}. We know that for any given value distribution, the Lloyd-Max algorithm minimizes the quantization MSE. Therefore, for a given vector cluster $\mathcal{B}_i$ we obtain the following result:

\begin{equation}
    \frac{1}{|\mathcal{B}_{i}^{(n+1)}|}\sum_{\bm{b} \in \mathcal{B}_{i}^{(n+1)}} \frac{1}{L_b}\lVert \bm{b}- C_i^{(n+1)}(\bm{b})\rVert^2_2 \le \frac{1}{|\mathcal{B}_{i}^{(n+1)}|}\sum_{\bm{b} \in \mathcal{B}_{i}^{(n+1)}} \frac{1}{L_b}\lVert \bm{b}- C_i^{(n)}(\bm{b})\rVert^2_2
\end{equation}

The above equation states that quantizing the given block cluster $\mathcal{B}_i$ after updating the associated codebook from $C_i^{(n)}$ to $C_i^{(n+1)}$ results in lower quantization MSE. Since this is true for all the block clusters, we derive the following result: 
\begin{equation}
\begin{split}
\label{eq:mse_ineq_2}
     J^{(n+1)} &= \frac{1}{N_c} \sum_{i=1}^{N_c} \frac{1}{|\mathcal{B}_{i}^{(n+1)}|}\sum_{\bm{b} \in \mathcal{B}_{i}^{(n+1)}} \frac{1}{L_b}\lVert \bm{b}- C_i^{(n+1)}(\bm{b})\rVert^2_2  \le \tilde{J}^{(n+1)}   
\end{split}
\end{equation}

Following (\ref{eq:mse_ineq_1}) and (\ref{eq:mse_ineq_2}), we find that the quantization MSE is non-increasing for each iteration, that is, $J^{(1)} \ge J^{(2)} \ge J^{(3)} \ge \ldots \ge J^{(M)}$ where $M$ is the maximum number of iterations. 
%Therefore, we can say that if the algorithm converges, then it must be that it has converged to a local minimum. 
\hfill $\blacksquare$


\begin{figure}
    \begin{center}
    \includegraphics[width=0.5\textwidth]{sections//figures/mse_vs_iter.pdf}
    \end{center}
    \caption{\small NMSE vs iterations during LO-BCQ compared to other block quantization proposals}
    \label{fig:nmse_vs_iter}
\end{figure}

Figure \ref{fig:nmse_vs_iter} shows the empirical convergence of LO-BCQ across several block lengths and number of codebooks. Also, the MSE achieved by LO-BCQ is compared to baselines such as MXFP and VSQ. As shown, LO-BCQ converges to a lower MSE than the baselines. Further, we achieve better convergence for larger number of codebooks ($N_c$) and for a smaller block length ($L_b$), both of which increase the bitwidth of BCQ (see Eq \ref{eq:bitwidth_bcq}).


\subsection{Additional Accuracy Results}
%Table \ref{tab:lobcq_config} lists the various LOBCQ configurations and their corresponding bitwidths.
\begin{table}
\setlength{\tabcolsep}{4.75pt}
\begin{center}
\caption{\label{tab:lobcq_config} Various LO-BCQ configurations and their bitwidths.}
\begin{tabular}{|c||c|c|c|c||c|c||c|} 
\hline
 & \multicolumn{4}{|c||}{$L_b=8$} & \multicolumn{2}{|c||}{$L_b=4$} & $L_b=2$ \\
 \hline
 \backslashbox{$L_A$\kern-1em}{\kern-1em$N_c$} & 2 & 4 & 8 & 16 & 2 & 4 & 2 \\
 \hline
 64 & 4.25 & 4.375 & 4.5 & 4.625 & 4.375 & 4.625 & 4.625\\
 \hline
 32 & 4.375 & 4.5 & 4.625& 4.75 & 4.5 & 4.75 & 4.75 \\
 \hline
 16 & 4.625 & 4.75& 4.875 & 5 & 4.75 & 5 & 5 \\
 \hline
\end{tabular}
\end{center}
\end{table}

%\subsection{Perplexity achieved by various LO-BCQ configurations on Wikitext-103 dataset}

\begin{table} \centering
\begin{tabular}{|c||c|c|c|c||c|c||c|} 
\hline
 $L_b \rightarrow$& \multicolumn{4}{c||}{8} & \multicolumn{2}{c||}{4} & 2\\
 \hline
 \backslashbox{$L_A$\kern-1em}{\kern-1em$N_c$} & 2 & 4 & 8 & 16 & 2 & 4 & 2  \\
 %$N_c \rightarrow$ & 2 & 4 & 8 & 16 & 2 & 4 & 2 \\
 \hline
 \hline
 \multicolumn{8}{c}{GPT3-1.3B (FP32 PPL = 9.98)} \\ 
 \hline
 \hline
 64 & 10.40 & 10.23 & 10.17 & 10.15 &  10.28 & 10.18 & 10.19 \\
 \hline
 32 & 10.25 & 10.20 & 10.15 & 10.12 &  10.23 & 10.17 & 10.17 \\
 \hline
 16 & 10.22 & 10.16 & 10.10 & 10.09 &  10.21 & 10.14 & 10.16 \\
 \hline
  \hline
 \multicolumn{8}{c}{GPT3-8B (FP32 PPL = 7.38)} \\ 
 \hline
 \hline
 64 & 7.61 & 7.52 & 7.48 &  7.47 &  7.55 &  7.49 & 7.50 \\
 \hline
 32 & 7.52 & 7.50 & 7.46 &  7.45 &  7.52 &  7.48 & 7.48  \\
 \hline
 16 & 7.51 & 7.48 & 7.44 &  7.44 &  7.51 &  7.49 & 7.47  \\
 \hline
\end{tabular}
\caption{\label{tab:ppl_gpt3_abalation} Wikitext-103 perplexity across GPT3-1.3B and 8B models.}
\end{table}

\begin{table} \centering
\begin{tabular}{|c||c|c|c|c||} 
\hline
 $L_b \rightarrow$& \multicolumn{4}{c||}{8}\\
 \hline
 \backslashbox{$L_A$\kern-1em}{\kern-1em$N_c$} & 2 & 4 & 8 & 16 \\
 %$N_c \rightarrow$ & 2 & 4 & 8 & 16 & 2 & 4 & 2 \\
 \hline
 \hline
 \multicolumn{5}{|c|}{Llama2-7B (FP32 PPL = 5.06)} \\ 
 \hline
 \hline
 64 & 5.31 & 5.26 & 5.19 & 5.18  \\
 \hline
 32 & 5.23 & 5.25 & 5.18 & 5.15  \\
 \hline
 16 & 5.23 & 5.19 & 5.16 & 5.14  \\
 \hline
 \multicolumn{5}{|c|}{Nemotron4-15B (FP32 PPL = 5.87)} \\ 
 \hline
 \hline
 64  & 6.3 & 6.20 & 6.13 & 6.08  \\
 \hline
 32  & 6.24 & 6.12 & 6.07 & 6.03  \\
 \hline
 16  & 6.12 & 6.14 & 6.04 & 6.02  \\
 \hline
 \multicolumn{5}{|c|}{Nemotron4-340B (FP32 PPL = 3.48)} \\ 
 \hline
 \hline
 64 & 3.67 & 3.62 & 3.60 & 3.59 \\
 \hline
 32 & 3.63 & 3.61 & 3.59 & 3.56 \\
 \hline
 16 & 3.61 & 3.58 & 3.57 & 3.55 \\
 \hline
\end{tabular}
\caption{\label{tab:ppl_llama7B_nemo15B} Wikitext-103 perplexity compared to FP32 baseline in Llama2-7B and Nemotron4-15B, 340B models}
\end{table}

%\subsection{Perplexity achieved by various LO-BCQ configurations on MMLU dataset}


\begin{table} \centering
\begin{tabular}{|c||c|c|c|c||c|c|c|c|} 
\hline
 $L_b \rightarrow$& \multicolumn{4}{c||}{8} & \multicolumn{4}{c||}{8}\\
 \hline
 \backslashbox{$L_A$\kern-1em}{\kern-1em$N_c$} & 2 & 4 & 8 & 16 & 2 & 4 & 8 & 16  \\
 %$N_c \rightarrow$ & 2 & 4 & 8 & 16 & 2 & 4 & 2 \\
 \hline
 \hline
 \multicolumn{5}{|c|}{Llama2-7B (FP32 Accuracy = 45.8\%)} & \multicolumn{4}{|c|}{Llama2-70B (FP32 Accuracy = 69.12\%)} \\ 
 \hline
 \hline
 64 & 43.9 & 43.4 & 43.9 & 44.9 & 68.07 & 68.27 & 68.17 & 68.75 \\
 \hline
 32 & 44.5 & 43.8 & 44.9 & 44.5 & 68.37 & 68.51 & 68.35 & 68.27  \\
 \hline
 16 & 43.9 & 42.7 & 44.9 & 45 & 68.12 & 68.77 & 68.31 & 68.59  \\
 \hline
 \hline
 \multicolumn{5}{|c|}{GPT3-22B (FP32 Accuracy = 38.75\%)} & \multicolumn{4}{|c|}{Nemotron4-15B (FP32 Accuracy = 64.3\%)} \\ 
 \hline
 \hline
 64 & 36.71 & 38.85 & 38.13 & 38.92 & 63.17 & 62.36 & 63.72 & 64.09 \\
 \hline
 32 & 37.95 & 38.69 & 39.45 & 38.34 & 64.05 & 62.30 & 63.8 & 64.33  \\
 \hline
 16 & 38.88 & 38.80 & 38.31 & 38.92 & 63.22 & 63.51 & 63.93 & 64.43  \\
 \hline
\end{tabular}
\caption{\label{tab:mmlu_abalation} Accuracy on MMLU dataset across GPT3-22B, Llama2-7B, 70B and Nemotron4-15B models.}
\end{table}


%\subsection{Perplexity achieved by various LO-BCQ configurations on LM evaluation harness}

\begin{table} \centering
\begin{tabular}{|c||c|c|c|c||c|c|c|c|} 
\hline
 $L_b \rightarrow$& \multicolumn{4}{c||}{8} & \multicolumn{4}{c||}{8}\\
 \hline
 \backslashbox{$L_A$\kern-1em}{\kern-1em$N_c$} & 2 & 4 & 8 & 16 & 2 & 4 & 8 & 16  \\
 %$N_c \rightarrow$ & 2 & 4 & 8 & 16 & 2 & 4 & 2 \\
 \hline
 \hline
 \multicolumn{5}{|c|}{Race (FP32 Accuracy = 37.51\%)} & \multicolumn{4}{|c|}{Boolq (FP32 Accuracy = 64.62\%)} \\ 
 \hline
 \hline
 64 & 36.94 & 37.13 & 36.27 & 37.13 & 63.73 & 62.26 & 63.49 & 63.36 \\
 \hline
 32 & 37.03 & 36.36 & 36.08 & 37.03 & 62.54 & 63.51 & 63.49 & 63.55  \\
 \hline
 16 & 37.03 & 37.03 & 36.46 & 37.03 & 61.1 & 63.79 & 63.58 & 63.33  \\
 \hline
 \hline
 \multicolumn{5}{|c|}{Winogrande (FP32 Accuracy = 58.01\%)} & \multicolumn{4}{|c|}{Piqa (FP32 Accuracy = 74.21\%)} \\ 
 \hline
 \hline
 64 & 58.17 & 57.22 & 57.85 & 58.33 & 73.01 & 73.07 & 73.07 & 72.80 \\
 \hline
 32 & 59.12 & 58.09 & 57.85 & 58.41 & 73.01 & 73.94 & 72.74 & 73.18  \\
 \hline
 16 & 57.93 & 58.88 & 57.93 & 58.56 & 73.94 & 72.80 & 73.01 & 73.94  \\
 \hline
\end{tabular}
\caption{\label{tab:mmlu_abalation} Accuracy on LM evaluation harness tasks on GPT3-1.3B model.}
\end{table}

\begin{table} \centering
\begin{tabular}{|c||c|c|c|c||c|c|c|c|} 
\hline
 $L_b \rightarrow$& \multicolumn{4}{c||}{8} & \multicolumn{4}{c||}{8}\\
 \hline
 \backslashbox{$L_A$\kern-1em}{\kern-1em$N_c$} & 2 & 4 & 8 & 16 & 2 & 4 & 8 & 16  \\
 %$N_c \rightarrow$ & 2 & 4 & 8 & 16 & 2 & 4 & 2 \\
 \hline
 \hline
 \multicolumn{5}{|c|}{Race (FP32 Accuracy = 41.34\%)} & \multicolumn{4}{|c|}{Boolq (FP32 Accuracy = 68.32\%)} \\ 
 \hline
 \hline
 64 & 40.48 & 40.10 & 39.43 & 39.90 & 69.20 & 68.41 & 69.45 & 68.56 \\
 \hline
 32 & 39.52 & 39.52 & 40.77 & 39.62 & 68.32 & 67.43 & 68.17 & 69.30  \\
 \hline
 16 & 39.81 & 39.71 & 39.90 & 40.38 & 68.10 & 66.33 & 69.51 & 69.42  \\
 \hline
 \hline
 \multicolumn{5}{|c|}{Winogrande (FP32 Accuracy = 67.88\%)} & \multicolumn{4}{|c|}{Piqa (FP32 Accuracy = 78.78\%)} \\ 
 \hline
 \hline
 64 & 66.85 & 66.61 & 67.72 & 67.88 & 77.31 & 77.42 & 77.75 & 77.64 \\
 \hline
 32 & 67.25 & 67.72 & 67.72 & 67.00 & 77.31 & 77.04 & 77.80 & 77.37  \\
 \hline
 16 & 68.11 & 68.90 & 67.88 & 67.48 & 77.37 & 78.13 & 78.13 & 77.69  \\
 \hline
\end{tabular}
\caption{\label{tab:mmlu_abalation} Accuracy on LM evaluation harness tasks on GPT3-8B model.}
\end{table}

\begin{table} \centering
\begin{tabular}{|c||c|c|c|c||c|c|c|c|} 
\hline
 $L_b \rightarrow$& \multicolumn{4}{c||}{8} & \multicolumn{4}{c||}{8}\\
 \hline
 \backslashbox{$L_A$\kern-1em}{\kern-1em$N_c$} & 2 & 4 & 8 & 16 & 2 & 4 & 8 & 16  \\
 %$N_c \rightarrow$ & 2 & 4 & 8 & 16 & 2 & 4 & 2 \\
 \hline
 \hline
 \multicolumn{5}{|c|}{Race (FP32 Accuracy = 40.67\%)} & \multicolumn{4}{|c|}{Boolq (FP32 Accuracy = 76.54\%)} \\ 
 \hline
 \hline
 64 & 40.48 & 40.10 & 39.43 & 39.90 & 75.41 & 75.11 & 77.09 & 75.66 \\
 \hline
 32 & 39.52 & 39.52 & 40.77 & 39.62 & 76.02 & 76.02 & 75.96 & 75.35  \\
 \hline
 16 & 39.81 & 39.71 & 39.90 & 40.38 & 75.05 & 73.82 & 75.72 & 76.09  \\
 \hline
 \hline
 \multicolumn{5}{|c|}{Winogrande (FP32 Accuracy = 70.64\%)} & \multicolumn{4}{|c|}{Piqa (FP32 Accuracy = 79.16\%)} \\ 
 \hline
 \hline
 64 & 69.14 & 70.17 & 70.17 & 70.56 & 78.24 & 79.00 & 78.62 & 78.73 \\
 \hline
 32 & 70.96 & 69.69 & 71.27 & 69.30 & 78.56 & 79.49 & 79.16 & 78.89  \\
 \hline
 16 & 71.03 & 69.53 & 69.69 & 70.40 & 78.13 & 79.16 & 79.00 & 79.00  \\
 \hline
\end{tabular}
\caption{\label{tab:mmlu_abalation} Accuracy on LM evaluation harness tasks on GPT3-22B model.}
\end{table}

\begin{table} \centering
\begin{tabular}{|c||c|c|c|c||c|c|c|c|} 
\hline
 $L_b \rightarrow$& \multicolumn{4}{c||}{8} & \multicolumn{4}{c||}{8}\\
 \hline
 \backslashbox{$L_A$\kern-1em}{\kern-1em$N_c$} & 2 & 4 & 8 & 16 & 2 & 4 & 8 & 16  \\
 %$N_c \rightarrow$ & 2 & 4 & 8 & 16 & 2 & 4 & 2 \\
 \hline
 \hline
 \multicolumn{5}{|c|}{Race (FP32 Accuracy = 44.4\%)} & \multicolumn{4}{|c|}{Boolq (FP32 Accuracy = 79.29\%)} \\ 
 \hline
 \hline
 64 & 42.49 & 42.51 & 42.58 & 43.45 & 77.58 & 77.37 & 77.43 & 78.1 \\
 \hline
 32 & 43.35 & 42.49 & 43.64 & 43.73 & 77.86 & 75.32 & 77.28 & 77.86  \\
 \hline
 16 & 44.21 & 44.21 & 43.64 & 42.97 & 78.65 & 77 & 76.94 & 77.98  \\
 \hline
 \hline
 \multicolumn{5}{|c|}{Winogrande (FP32 Accuracy = 69.38\%)} & \multicolumn{4}{|c|}{Piqa (FP32 Accuracy = 78.07\%)} \\ 
 \hline
 \hline
 64 & 68.9 & 68.43 & 69.77 & 68.19 & 77.09 & 76.82 & 77.09 & 77.86 \\
 \hline
 32 & 69.38 & 68.51 & 68.82 & 68.90 & 78.07 & 76.71 & 78.07 & 77.86  \\
 \hline
 16 & 69.53 & 67.09 & 69.38 & 68.90 & 77.37 & 77.8 & 77.91 & 77.69  \\
 \hline
\end{tabular}
\caption{\label{tab:mmlu_abalation} Accuracy on LM evaluation harness tasks on Llama2-7B model.}
\end{table}

\begin{table} \centering
\begin{tabular}{|c||c|c|c|c||c|c|c|c|} 
\hline
 $L_b \rightarrow$& \multicolumn{4}{c||}{8} & \multicolumn{4}{c||}{8}\\
 \hline
 \backslashbox{$L_A$\kern-1em}{\kern-1em$N_c$} & 2 & 4 & 8 & 16 & 2 & 4 & 8 & 16  \\
 %$N_c \rightarrow$ & 2 & 4 & 8 & 16 & 2 & 4 & 2 \\
 \hline
 \hline
 \multicolumn{5}{|c|}{Race (FP32 Accuracy = 48.8\%)} & \multicolumn{4}{|c|}{Boolq (FP32 Accuracy = 85.23\%)} \\ 
 \hline
 \hline
 64 & 49.00 & 49.00 & 49.28 & 48.71 & 82.82 & 84.28 & 84.03 & 84.25 \\
 \hline
 32 & 49.57 & 48.52 & 48.33 & 49.28 & 83.85 & 84.46 & 84.31 & 84.93  \\
 \hline
 16 & 49.85 & 49.09 & 49.28 & 48.99 & 85.11 & 84.46 & 84.61 & 83.94  \\
 \hline
 \hline
 \multicolumn{5}{|c|}{Winogrande (FP32 Accuracy = 79.95\%)} & \multicolumn{4}{|c|}{Piqa (FP32 Accuracy = 81.56\%)} \\ 
 \hline
 \hline
 64 & 78.77 & 78.45 & 78.37 & 79.16 & 81.45 & 80.69 & 81.45 & 81.5 \\
 \hline
 32 & 78.45 & 79.01 & 78.69 & 80.66 & 81.56 & 80.58 & 81.18 & 81.34  \\
 \hline
 16 & 79.95 & 79.56 & 79.79 & 79.72 & 81.28 & 81.66 & 81.28 & 80.96  \\
 \hline
\end{tabular}
\caption{\label{tab:mmlu_abalation} Accuracy on LM evaluation harness tasks on Llama2-70B model.}
\end{table}

%\section{MSE Studies}
%\textcolor{red}{TODO}


\subsection{Number Formats and Quantization Method}
\label{subsec:numFormats_quantMethod}
\subsubsection{Integer Format}
An $n$-bit signed integer (INT) is typically represented with a 2s-complement format \citep{yao2022zeroquant,xiao2023smoothquant,dai2021vsq}, where the most significant bit denotes the sign.

\subsubsection{Floating Point Format}
An $n$-bit signed floating point (FP) number $x$ comprises of a 1-bit sign ($x_{\mathrm{sign}}$), $B_m$-bit mantissa ($x_{\mathrm{mant}}$) and $B_e$-bit exponent ($x_{\mathrm{exp}}$) such that $B_m+B_e=n-1$. The associated constant exponent bias ($E_{\mathrm{bias}}$) is computed as $(2^{{B_e}-1}-1)$. We denote this format as $E_{B_e}M_{B_m}$.  

\subsubsection{Quantization Scheme}
\label{subsec:quant_method}
A quantization scheme dictates how a given unquantized tensor is converted to its quantized representation. We consider FP formats for the purpose of illustration. Given an unquantized tensor $\bm{X}$ and an FP format $E_{B_e}M_{B_m}$, we first, we compute the quantization scale factor $s_X$ that maps the maximum absolute value of $\bm{X}$ to the maximum quantization level of the $E_{B_e}M_{B_m}$ format as follows:
\begin{align}
\label{eq:sf}
    s_X = \frac{\mathrm{max}(|\bm{X}|)}{\mathrm{max}(E_{B_e}M_{B_m})}
\end{align}
In the above equation, $|\cdot|$ denotes the absolute value function.

Next, we scale $\bm{X}$ by $s_X$ and quantize it to $\hat{\bm{X}}$ by rounding it to the nearest quantization level of $E_{B_e}M_{B_m}$ as:

\begin{align}
\label{eq:tensor_quant}
    \hat{\bm{X}} = \text{round-to-nearest}\left(\frac{\bm{X}}{s_X}, E_{B_e}M_{B_m}\right)
\end{align}

We perform dynamic max-scaled quantization \citep{wu2020integer}, where the scale factor $s$ for activations is dynamically computed during runtime.

\subsection{Vector Scaled Quantization}
\begin{wrapfigure}{r}{0.35\linewidth}
  \centering
  \includegraphics[width=\linewidth]{sections/figures/vsquant.jpg}
  \caption{\small Vectorwise decomposition for per-vector scaled quantization (VSQ \citep{dai2021vsq}).}
  \label{fig:vsquant}
\end{wrapfigure}
During VSQ \citep{dai2021vsq}, the operand tensors are decomposed into 1D vectors in a hardware friendly manner as shown in Figure \ref{fig:vsquant}. Since the decomposed tensors are used as operands in matrix multiplications during inference, it is beneficial to perform this decomposition along the reduction dimension of the multiplication. The vectorwise quantization is performed similar to tensorwise quantization described in Equations \ref{eq:sf} and \ref{eq:tensor_quant}, where a scale factor $s_v$ is required for each vector $\bm{v}$ that maps the maximum absolute value of that vector to the maximum quantization level. While smaller vector lengths can lead to larger accuracy gains, the associated memory and computational overheads due to the per-vector scale factors increases. To alleviate these overheads, VSQ \citep{dai2021vsq} proposed a second level quantization of the per-vector scale factors to unsigned integers, while MX \citep{rouhani2023shared} quantizes them to integer powers of 2 (denoted as $2^{INT}$).

\subsubsection{MX Format}
The MX format proposed in \citep{rouhani2023microscaling} introduces the concept of sub-block shifting. For every two scalar elements of $b$-bits each, there is a shared exponent bit. The value of this exponent bit is determined through an empirical analysis that targets minimizing quantization MSE. We note that the FP format $E_{1}M_{b}$ is strictly better than MX from an accuracy perspective since it allocates a dedicated exponent bit to each scalar as opposed to sharing it across two scalars. Therefore, we conservatively bound the accuracy of a $b+2$-bit signed MX format with that of a $E_{1}M_{b}$ format in our comparisons. For instance, we use E1M2 format as a proxy for MX4.

\begin{figure}
    \centering
    \includegraphics[width=1\linewidth]{sections//figures/BlockFormats.pdf}
    \caption{\small Comparing LO-BCQ to MX format.}
    \label{fig:block_formats}
\end{figure}

Figure \ref{fig:block_formats} compares our $4$-bit LO-BCQ block format to MX \citep{rouhani2023microscaling}. As shown, both LO-BCQ and MX decompose a given operand tensor into block arrays and each block array into blocks. Similar to MX, we find that per-block quantization ($L_b < L_A$) leads to better accuracy due to increased flexibility. While MX achieves this through per-block $1$-bit micro-scales, we associate a dedicated codebook to each block through a per-block codebook selector. Further, MX quantizes the per-block array scale-factor to E8M0 format without per-tensor scaling. In contrast during LO-BCQ, we find that per-tensor scaling combined with quantization of per-block array scale-factor to E4M3 format results in superior inference accuracy across models. 

\section{Asymptotic Convergence}

\textbf{\cref{lem:finite_sample_time_implies_finite_parameter}.}
Using \cref{alg:gradient_bandit_algorithm_sampled_reward} with any constant $\eta \in \Theta(1)$, if $N_\infty(a) < \infty$ for an action $a \in [K]$, then we have, almost surely,
\begin{align}
    \sup_{t \ge 1}{ \theta_t(a) } < \infty, \text{ and } \inf_{t \ge 1}{ \theta_t(a) } > -\infty.
\end{align}
\begin{proof}
Suppose $N_\infty(a) < \infty$ for an action $a \in [K]$. According to \cref{alg:gradient_bandit_algorithm_sampled_reward},
\begin{align}
\label{eq:finite_sample_time_implies_finite_parameter_proof_1}
    \theta_{t+1}(a) &\gets \theta_t(a) + \begin{cases}
		\eta \cdot \left( 1 - \pi_{\theta_t}(a) \right) \cdot R_t(a), & \text{if } a_t = a\, , \\
		- \eta \cdot \pi_{\theta_t}(a) \cdot R_t(a_t), & \text{otherwise}\, ,
    \end{cases}
\end{align}
and let
\begin{align}
\label{eq:finite_sample_time_implies_finite_parameter_proof_2}
    I_t(a) \coloneqq 
    \begin{cases}
	1, & \text{if } a_t = a\, , \\
	0, & \text{otherwise}\,.
    \end{cases}
\end{align}
According to \cref{eq:finite_sample_time_implies_finite_parameter_proof_1}, we have, for all $t \ge 1$,
\begin{align}
\label{eq:finite_sample_time_implies_finite_parameter_proof_3}
    \theta_t(a) - \theta_1(a) = \sum_{s=1}^{t-1}{ I_s(a) \cdot \eta \cdot \left( 1 - \pi_{\theta_s}(a) \right) \cdot R_s(a)} + \sum_{s=1}^{t-1}{ \left( 1 - I_s(a) \right) \cdot (- \eta) \cdot \pi_{\theta_s}(a) \cdot R_s(a_s) }.
\end{align}
Using triangle inequality, we have,
\begin{align}
\label{eq:finite_sample_time_implies_finite_parameter_proof_4}
    \left| \theta_t(a) - \theta_1(a) \right| &\le \sum_{s=1}^{t-1}{ \Big| I_s(a) \cdot \eta \cdot \left( 1 - \pi_{\theta_s}(a) \right) \cdot R_s(a) \Big| } + \sum_{s=1}^{t-1}{ \Big|\left( 1 - I_s(a) \right) \cdot (- \eta) \cdot \pi_{\theta_s}(a) \cdot R_s(a_s) \Big| } \\
    &\le \eta \cdot R_{\max} \cdot \sum_{s=1}^{t-1}{ I_s(a) } + \eta \cdot R_{\max} \cdot \sum_{s=1}^{t-1}{ \pi_{\theta_s}(a) } \\
    &= \eta \cdot R_{\max} \cdot \Big( N_{t-1}(a) + \sum_{s=1}^{t-1}{ \pi_{\theta_s}(a) } \Big).
\end{align}
By assumption, we have,
\begin{align}
\label{eq:finite_sample_time_implies_finite_parameter_proof_5}
    N_\infty(a) \coloneqq \lim_{t \to \infty}{N_t(a)} < \infty.
\end{align}
According to the extended Borel-Cantelli \cref{lem:ebc}, we have, almost surely,
\begin{align}
\label{eq:finite_sample_time_implies_finite_parameter_proof_6}
    \sum_{t=1}^{\infty}{\pi_{\theta_t}(a)} \coloneqq \lim_{t \to \infty}{ \sum_{s=1}^{t}{ \pi_{\theta_s}(a) } } < \infty.
\end{align}
Combining \cref{eq:finite_sample_time_implies_finite_parameter_proof_4,eq:finite_sample_time_implies_finite_parameter_proof_5,eq:finite_sample_time_implies_finite_parameter_proof_6}, we have, almost surely,
\begin{align}
\label{eq:finite_sample_time_implies_finite_parameter_proof_7}
    \sup_{t \ge 1}{ \left| \theta_t(a) - \theta_1(a) \right| } < \infty,
\end{align}
which implies that, almost surely,
\begin{equation*}
    \sup_{t \ge 1}{ \left| \theta_t(a) \right| } \le \sup_{t \ge 1}{ \left| \theta_t(a) - \theta_1(a) \right| + \left| \theta_1(a) \right| } < \infty. \qedhere
\end{equation*}
\end{proof}

\textbf{\cref{lem:at_least_two_actions_infinite_sample_time} }(Avoiding lack of exploration)\textbf{.}
Using \cref{alg:gradient_bandit_algorithm_sampled_reward} with any $\eta \in \Theta(1)$, there exists at least a pair of distinct actions $i, j \in [K]$ and $i \ne j$, such that, almost surely,
\begin{align}
    N_\infty(i) = \infty, \text{ and } N_\infty(j) = \infty.
\end{align}
\begin{proof}
First, we have, for all $t \ge 1$,
\begin{align}
\label{eq:at_least_two_actions_infinite_sample_time_proof_1}
    t &= \sum_{s=1}^{t}{ \sum_{a \in [K]}{ I_s(a) } } \qquad \Big( \sum_{a \in [K]}{ I_t(a) } = 1 \text{ for all } t \ge 1 \Big) \\
    &= \sum_{a \in [K]} \sum_{s=1}^{t} I_s(a) \\
    &= \sum_{a \in [K]}{ N_t(a) }.
\end{align}
By pigeonhole principle, there exists at least one action $i \in [K]$, such that, almost surely,
\begin{align}
\label{eq:at_least_two_actions_infinite_sample_time_proof_2}
    N_\infty(i) \coloneqq \lim_{t \to \infty}{ N_t(i) } = \infty.
\end{align}
We argue the existence of another action by contradiction. Suppose for all the other actions $j \in [K]$ and $j \ne i$, we have $N_\infty(j) < \infty$. According to the extended Borel-Cantelli \cref{lem:ebc}, we have, almost surely,
\begin{align}
\label{eq:at_least_two_actions_infinite_sample_time_proof_3}
    \sum_{t=1}^{\infty}{\pi_{\theta_t}(j)} \coloneqq \lim_{t \to \infty}{ \sum_{s=1}^{t}{ \pi_{\theta_s}(j) } } < \infty.
\end{align}
According to the update \cref{eq:finite_sample_time_implies_finite_parameter_proof_1}, we have, for all $t \ge 1$,
\begin{align}
\label{eq:at_least_two_actions_infinite_sample_time_proof_4}
    \theta_t(i) - \theta_1(i) = \sum_{s=1}^{t-1}{ I_s(i) \cdot \eta \cdot \left( 1 - \pi_{\theta_s}(i) \right) \cdot R_s(i)} + \sum_{s=1}^{t-1}{ \left( 1 - I_s(i) \right) \cdot (- \eta) \cdot \pi_{\theta_s}(i) \cdot R_s(a_s) }.
\end{align}
By triangle inequality, we have,
\begin{align}
\label{eq:at_least_two_actions_infinite_sample_time_proof_5}
    \left| \theta_t(i) - \theta_1(i) \right| &\le \sum_{s=1}^{t-1}{ \Big| I_s(i) \cdot \eta \cdot \left( 1 - \pi_{\theta_s}(i) \right) \cdot R_s(i) \Big| } + \sum_{s=1}^{t-1}{ \Big| \left( 1 - I_s(i) \right) \cdot (- \eta) \cdot \pi_{\theta_s}(i) \cdot R_s(a_s) \Big| } \\
    &\le \eta \cdot R_{\max} \cdot \sum_{s=1}^{t-1}{ \left( 1 - \pi_{\theta_s}(i) \right) } + \eta \cdot R_{\max} \cdot \sum_{s=1}^{t-1}{ \left( 1 - I_s(i) \right) } \\
    &= \eta \cdot R_{\max} \cdot \Big( \sum_{s=1}^{t-1} \sum_{j \ne i}{ \pi_{\theta_s}(j) } +  \sum_{s=1}^{t-1} \sum_{j \ne i}{ I_s(j) } \Big) \\
    &= \eta \cdot R_{\max} \cdot \Big( \sum_{j \ne i} \sum_{s=1}^{t-1}{\pi_{\theta_s}(j)} + \sum_{j \ne i}{ N_{t-1}(j) } \Big).
\end{align}
Combing \cref{eq:at_least_two_actions_infinite_sample_time_proof_3,eq:at_least_two_actions_infinite_sample_time_proof_5} and the assumption of $N_\infty(j) < \infty$ for all $j \ne i$, we have, almost surely,
\begin{align}
\label{eq:at_least_two_actions_infinite_sample_time_proof_6}
    \sup_{t \ge 1}{ \left| \theta_t(i) \right| } \le \sup_{t \ge 1}{ \left| \theta_t(i) - \theta_1(i) \right| + \left| \theta_1(i) \right| } < \infty.
\end{align}
Since $N_\infty(j) < \infty$ for all $j \ne i$ by assumption, and according to \cref{lem:finite_sample_time_implies_finite_parameter}, we have, almost surely,
\begin{align}
\label{eq:at_least_two_actions_infinite_sample_time_proof_7}
    \sup_{t \ge 1}{ \left| \theta_t(j) \right| } < \infty.
\end{align}
Combining \cref{eq:at_least_two_actions_infinite_sample_time_proof_6,eq:at_least_two_actions_infinite_sample_time_proof_7}, we have,  for all action $a \in [K]$,
\begin{align}
\label{eq:at_least_two_actions_infinite_sample_time_proof_8}
    \sup_{t \ge 1}{ \left| \theta_t(a) \right| } < \infty,
\end{align}
which implies that, there exists $c > 0$ and $c \in O(1)$, such that, for all $a \in [K]$,
\begin{align}
\label{eq:at_least_two_actions_infinite_sample_time_proof_9}
    \inf_{t \ge 1}{ \pi_{\theta_t}(a) } = \inf_{t \ge 1}{ \frac{ \exp\{ \theta_t(a) \} }{ \sum_{a^\prime \in [K]}{ \exp\{ \theta_t(a^\prime) } \} } } \ge c > 0.
\end{align}
Therefore, for all action $a \in [K]$,
\begin{align}
\label{eq:at_least_two_actions_infinite_sample_time_proof_10}
    \sum_{t=1}^{\infty}{\pi_{\theta_t}(a)} &\coloneqq \lim_{t \to \infty}{ \sum_{s=1}^{t}{ \pi_{\theta_s}(a) } } \\
    &\ge \lim_{t \to \infty}{ \sum_{s=1}^{t}{ c } } \\
    &= \lim_{t \to \infty}{ t \cdot c } \\
    &= \infty.
\end{align}
According to the extended Borel-Cantelli \cref{lem:ebc}, we have, almost surely, for all action $a \in [K]$,
\begin{align}
\label{eq:at_least_two_actions_infinite_sample_time_proof_11}
    N_\infty(a) = \infty,
\end{align}
which is a contradiction with the assumption of $N_\infty(j) < \infty$ for all the other actions $j \in [K]$ with $j \ne i$. Therefore, there exists another action $j \in [K]$ with $j \ne i$, such that $N_\infty(j) = \infty$.
\end{proof}

\textbf{\cref{thm:two_action_global_convergence}.}
Let $K = 2$ and $r(1) > r(2)$. Using \cref{alg:gradient_bandit_algorithm_sampled_reward} with any $\eta \in \Theta(1)$, we have, almost surely, $\pi_{\theta_t}(a^*) \to 1$ as $t \to \infty$, where $a^* \coloneqq \argmax_{a \in [K]}{ r(a) }$ (equal to Action $1$ in this case).
\begin{proof}
The proof uses the same notations as of~\citep[Theorem 5.1]{mei2024stochastic}.

According to \cref{lem:at_least_two_actions_infinite_sample_time}, we have, $N_\infty(1) = \infty$ and $N_\infty(2) = \infty$, i.e., both of the two actions are sampled for infinitely many times as $t \to \infty$.

Let $\gF_t$ be the $\sigma$-algebra generated by $a_1$, $R_1(a_1)$, $\cdots$, $a_{t-1}$, $R_{t-1}(a_{t-1})$:
\begin{align}
\label{eq:two_action_global_convergence_proof_1}
    \gF_t = \sigma( \{ a_1, R_1(a_1), \cdots, a_{t-1}, R_{t-1}(a_{t-1}) \} )\,.
\end{align}
Note that $\theta_{t}$ and $I_t$ (defined by \cref{eq:finite_sample_time_implies_finite_parameter_proof_2}) are $\gF_t$-measurable for all $t\ge 1$. Let $\EEt{\cdot}$ denote the conditional expectation with respect to $\gF_t$: $\mathbb{E}_t[X] = \mathbb{E}[X|\gF_t]$. Define the following notations,
\begin{align}
%\label{eq:two_action_global_convergence_proof_2a}
    %Z_t(a) &\coloneqq W_1(a) + \cdots + W_t(a), \qquad \left( \text{``cumulative noise''} \right) \\
\label{eq:two_action_global_convergence_proof_2b}
    W_t(a) &\coloneqq \theta_t(a) - \chE_{t-1}{[ \theta_t(a)]}, \qquad \left( \text{``noise''} \right) \\
\label{eq:two_action_global_convergence_proof_2c}
    P_t(a) &\coloneqq \EEt{\theta_{t+1}(a)} - \theta_t(a). \qquad \left( \text{``progress''} \right)
\end{align}
For each action $a \in [K]$, for $t \ge 2$, we have the following decomposition,
\begin{align}
\label{eq:two_action_global_convergence_proof_3}
    \theta_{t}(a) = W_t(a) + P_{t-1}(a) + \theta_{t-1}(a).
\end{align}
By recursion we can determine that,
\begin{align}
\label{eq:two_action_global_convergence_proof_4}
    \theta_t(a) = \EE{\theta_1(a)} + \sum_{s=1}^{t}{W_s(a)} + \sum_{s=1}^{t-1}{P_s(a)},
\end{align}
while we also have,
\begin{align}
\label{eq:two_action_global_convergence_proof_5}
    \theta_1(a) = \underbrace{ \theta_{1}(a) - \EE{\theta_{1}(a)} }_{ W_1(a) } + \EE{\theta_{1}(a)},
\end{align}
and $\EE{\theta_{1}(a)}$ accounts for potential randomness in initializing $\theta_1 \in \sR^K$. According to \cref{prop:gradient_bandit_algorithm_equivalent_to_stochastic_gradient_ascent_sampled_reward}, we have,
\begin{align}
\label{eq:two_action_global_convergence_proof_6}
    P_t(a) &= \EEt{\theta_{t+1}(a)} - \theta_t(a) 
    = \eta \cdot \pi_{\theta_t}(a) \cdot \left(  r(a) - \pi_{\theta_t}^\top r \right),
\end{align}
which implies that, for all $t \ge 1$,
\begin{align}
\label{eq:two_action_global_convergence_proof_7}
    P_t(a^*) > 0 > P_t(2).
\end{align}
Next, we show that $\sum_{s=1}^t P_s(a^*) \to \infty$ as $t \to \infty$ by contradiction. Suppose that,
\begin{align}
\label{eq:two_action_global_convergence_proof_8}
    \sum_{t=1}^{\infty} P_t(a^*) < \infty.
\end{align}
For the optimal action $a^* = 1$,
\begin{align}
\label{eq:two_action_global_convergence_proof_9}
    \sum_{s=1}^t P_s(a^*) &=  \sum_{s=1}^t \eta \cdot \pi_{\theta_s}(a^*) \cdot ( r(a^*) - \pi_{\theta_s}^\top r ) \\
    &= \eta \cdot \Delta \cdot  \sum_{s=1}^t  \pi_{\theta_s}(a^*) \cdot ( 1 - \pi_{\theta_s}(a^*)  ),
\end{align}
where $\Delta \coloneqq r(a^*) - \max_{a \not= a^*}{ r(a) } = r(1) - r(2) > 0$ is the reward gap. Denote that, for all $t 
\ge 1$, 
\begin{align} 
\label{eq:two_action_global_convergence_proof_10}
    V_t(a^*) &\coloneqq \frac{5}{18} \cdot \sum_{s=1}^{t-1}  \pi_{\theta_s}(a^*) \cdot (1-\pi_{\theta_s}(a^*)).
\end{align}
According to \cref{lem:bounded_progress_bounded_parameter} (using \cref{eq:two_action_global_convergence_proof_7,eq:two_action_global_convergence_proof_8,eq:two_action_global_convergence_proof_9,eq:two_action_global_convergence_proof_10}), we have, almost surely,
\begin{align}
\label{eq:two_action_global_convergence_proof_11}
    \sup_{t \ge 1}{ |\theta_t(a^*)| } < \infty.
\end{align}
For the sub-optimal action (Action $2$), we have,
\begin{align}
\label{eq:two_action_global_convergence_proof_12}
    \sum_{s=1}^t P_s(2) &=  \sum_{s=1}^t \eta \cdot \pi_{\theta_s}(2) \cdot ( r(2) - \pi_{\theta_s}^\top r ) \\
    &= - \eta \cdot \Delta \cdot  \sum_{s=1}^t  \pi_{\theta_s}(2) \cdot ( 1 - \pi_{\theta_s}(2)  ) \\
    &= - \sum_{s=1}^t P_s(a^*),
\end{align}
and also denote, for all $t \ge 1$,
\begin{align}
\label{eq:two_action_global_convergence_proof_13}
    V_t(2) &\coloneqq \frac{5}{18} \cdot \sum_{s=1}^{t-1}  \pi_{\theta_s}(2) \cdot (1-\pi_{\theta_s}(2)) = V_t(a^*).
\end{align}
According to \cref{lem:bounded_progress_bounded_parameter} (using \cref{eq:two_action_global_convergence_proof_7,eq:two_action_global_convergence_proof_8,eq:two_action_global_convergence_proof_12,eq:two_action_global_convergence_proof_13}), we have, almost surely,
\begin{align}
\label{eq:two_action_global_convergence_proof_14}
    \sup_{t \ge 1}{ \big| \theta_t(2) \big| } < \infty.
\end{align}
Combining \cref{eq:two_action_global_convergence_proof_11,eq:two_action_global_convergence_proof_14}, there exists $c > 0$ and $c \in O(1)$, such that, for all $a \in \{ a^*, 2 \}$,
\begin{align}
\label{eq:two_action_global_convergence_proof_15}
    \inf_{t \ge 1}{ \pi_{\theta_t}(a) } = \inf_{t \ge 1}{ \frac{ \exp\{ \theta_t(a) \} }{ \sum_{a^\prime \in [K]}{ \exp\{ \theta_t(a^\prime) } \} } } \ge c > 0,
\end{align}
which implies that,
\begin{align}
\label{eq:two_action_global_convergence_proof_16}
    \sum_{t=1}^{\infty}{\pi_{\theta_t}(a^*) \cdot ( 1 - \pi_{\theta_t}(a^*)  )} &\coloneqq \lim_{t \to \infty}{ \sum_{s=1}^{t}{ \pi_{\theta_s}(a^*) \cdot ( 1 - \pi_{\theta_s}(a^*)  ) } } \\
    &\ge \lim_{t \to \infty}{ \sum_{s=1}^{t}{ c \cdot c} } \\
    &= \lim_{t \to \infty}{ t \cdot c^2 } \\
    &= \infty,
\end{align}
which is a contradiction with the assumption of 
\cref{eq:two_action_global_convergence_proof_8}. Therefore, we have,
\begin{align}
\label{eq:two_action_global_convergence_proof_17}
    \sum_{s=1}^t P_s(a^*) \to \infty, \text{ as } t \to \infty.
\end{align}
According to \cref{lem:positive_unbounded_progress_unbounded_parameter} (using \cref{eq:two_action_global_convergence_proof_7,eq:two_action_global_convergence_proof_17,eq:two_action_global_convergence_proof_9,eq:two_action_global_convergence_proof_10}), we have, almost surely,
\begin{align}
\label{eq:two_action_global_convergence_proof_18}
    \theta_t(a^*) \to \infty, \text{ as } t \to \infty.
\end{align}
Similarly, according to \cref{lem:negative_unbounded_progress_unbounded_parameter} (using \cref{eq:two_action_global_convergence_proof_7,eq:two_action_global_convergence_proof_17,eq:two_action_global_convergence_proof_12,eq:two_action_global_convergence_proof_13}), we have, almost surely,
\begin{align}
\label{eq:two_action_global_convergence_proof_19}
     \lim_{t \to \infty}{ \theta_t(2) } = - \infty.
\end{align}
Note that,
\begin{align}
\label{eq:two_action_global_convergence_proof_20}
    \pi_{\theta_t}(a^*)
    &= \frac{\pi_{\theta_t}(a^*)}{ \pi_{\theta_t}(2) +  \pi_{\theta_t}(a^*)} \\
    &= \frac{1}{ \exp\{ \theta_t(2) - \theta_t(a^*) \} + 1}.
\end{align}
Combining \cref{eq:two_action_global_convergence_proof_18,eq:two_action_global_convergence_proof_19,eq:two_action_global_convergence_proof_20}, we have, almost surely,
\begin{equation*}
    \pi_{\theta_t}(a^*) \to 1, \text{ as } t \to \infty. \qedhere
\end{equation*}
\end{proof}

\textbf{\cref{thm:general_action_global_convergence}.}
Under~\cref{assp:reward_no_ties}, given $K \ge 2$, using \cref{alg:gradient_bandit_algorithm_sampled_reward} with any $\eta \in \Theta(1)$, we have, almost surely, $\pi_{\theta_t}(a^*) \to 1$ as $t \to \infty$, where $a^* = \argmax_{a \in [K]}{ r(a) }$ is the optimal action.
\begin{proof}
Similar to the proof of~\cref{thm:two_action_global_convergence}, we define $\gF_t$ to be the $\sigma$-algebra generated by $a_1$, $R_1(a_1)$, $\cdots$, $a_{t-1}$, $R_{t-1}(a_{t-1})$ i.e. 
\begin{align}
\label{eq:general_action_global_convergence_proof_1}
    \gF_t = \sigma( \{ a_1, R_1(a_1), \cdots, a_{t-1}, R_{t-1}(a_{t-1}) \} )\,.
\end{align}
Note that $\theta_{t}$ and $I_t$ (defined by \cref{eq:finite_sample_time_implies_finite_parameter_proof_2}) are $\gF_t$-measurable for all $t\ge 1$. Let $\EEt{\cdot}$ denote the conditional expectation with respect to $\gF_t$: $\mathbb{E}_t[X] = \mathbb{E}[X|\gF_t]$. Define the following notations,
\begin{align}
W_t(a) &\coloneqq \theta_t(a) - \chE_{t-1}{[ \theta_t(a)]}, \qquad \left( \text{``noise''} \right) \\
P_t(a) &\coloneqq \EEt{\theta_{t+1}(a)} - \theta_t(a). \qquad \left( \text{``progress''} \right)
\end{align}
For each action $a \in [K]$, for $t \ge 2$, we have the following decomposition,
\begin{align}
\theta_{t}(a) = W_t(a) + P_{t-1}(a) + \theta_{t-1}(a).
\end{align}
By recursion we can determine that,
\begin{align}
\theta_t(a) = \EE{\theta_1(a)} + \sum_{s=1}^{t}{W_s(a)} + \sum_{s=1}^{t-1}{P_s(a)},
\end{align}
while we also have,
\begin{align}
\theta_1(a) = \underbrace{ \theta_{1}(a) - \EE{\theta_{1}(a)} }_{ W_1(a) } + \EE{\theta_{1}(a)},
\end{align}
and $\EE{\theta_{1}(a)}$ accounts for potential randomness in initializing $\theta_1 \in \sR^K$. For an action $a \in [K]$, 
\begin{align}
\label{eq:general_action_global_convergence_first_part_proof_6}
    \gA^+(a) &\coloneqq \left\{ a^+ \in [K]: r(a^+) > r(a) \right\}, \\
    \gA^-(a) &\coloneqq \left\{ a^- \in [K]: r(a^-) < r(a) \right\}. 
\end{align}
Define $\gA_\infty$ as the set of actions which are sampled for infinitely many times as $t \to \infty$, i.e.,
\begin{align}
    \gA_\infty \coloneqq \left\{ a \in [K] \ | \ N_\infty(a) = \infty \right\}.
\end{align}

\textbf{First part.} We show that $N_\infty(a^*) = \infty$ by contradiction.

Suppose $a^* \not\in \gA_\infty$ i.e. $N_\infty(a^*) < \infty$. According to \cref{lem:at_least_two_actions_infinite_sample_time}, we have, $| \gA_\infty | \ge 2$. Since $a^* \not\in \gA_\infty$ by assumption, there must be at least two other sub-optimal actions $i_1, i_2 \in [K]$, $i_1 \ne i_2$, such that,
\begin{align}
\label{eq:general_action_global_convergence_first_part_proof_1}
    N_\infty(i_1) = N_\infty(i_2) = \infty.
\end{align}
In particular, define
\begin{align}
\label{eq:general_action_global_convergence_first_part_proof_2}
    i_1 &\coloneqq \argmin_{\substack{a \in [K], \\ N_\infty(a) = \infty}} r(a), \\
    i_2 &\coloneqq \argmax_{\substack{a \in [K], \\ N_\infty(a) = \infty}} r(a).
\end{align}
Using the update for arm $i_1$, we know that, for all $s \geq 1$, 
\begin{align}
P_s(i_1) &= \eta \cdot \pi_{\theta_s}(i_1) \cdot ( r(i_1) - \pi_{\theta_s}^\top r ) 
\end{align}
By definition and because of~\cref{assp:reward_no_ties}, we have that $r(i_1) < r(i_2) < r(a^*)$. Furthermore, according to \cref{lem:expected_reward_range}, we have, for sufficiently large $\tau \ge 1$,
\begin{align}
\label{eq:general_action_global_convergence_first_part_proof_3}
    r(i_1) < \pi_{\theta_t}^\top r < r(i_2),
\end{align}
which implies that after some large enough $\tau \ge 1$,
\begin{align}
    \sum_{s=\tau}^t P_s(i_1) &= \sum_{s=\tau}^t \eta \cdot \pi_{\theta_s}(i_1) \cdot ( r(i_1) - \pi_{\theta_s}^\top r ) \\
    & < 0. \qquad \left(\text{by \cref{eq:general_action_global_convergence_first_part_proof_3}}\right) \label{eq:general_action_global_convergence_first_part_proof_4}
\end{align}

\begin{align}
    r(i_1) - \pi_{\theta_s}^\top r &= \sum_{a \neq i_1} \pi_{\theta_s}(a) \, [r(i_1) - r(a)] \nonumber \\
    &= - \sum_{a^+ \in \gA^+(i_1)}{ \pi_{\theta_s}(a^+) \cdot (r(a^+) - r(i_1) ) } + \sum_{a^- \in \gA^-(i_1)}{ \pi_{\theta_s}(a^-) \cdot ( r(i_1) - r(a^-)) }. \label{eq:general_action_global_convergence_first_part_proof_5}
\end{align}
From~\cref{eq:general_action_global_convergence_first_part_proof_2}, we have,
\begin{align}
\label{eq:general_action_global_convergence_first_part_proof_8}
    \gA_\infty \subseteq \gA^+(i_1) \cup \{ i_1 \},
\end{align}
which implies that, for all $a^- \in \gA^-(i_1)$, we have,
\begin{align}
\label{eq:general_action_global_convergence_first_part_proof_9}
    N_\infty(a^-) < \infty.
\end{align}
Note that, by definition, $i_2 \in \gA^+(i_1)$, and 
\begin{align}
\label{eq:general_action_global_convergence_first_part_proof_10}
    N_\infty(i_2) = \infty.
\end{align}
In order to bound~\cref{eq:general_action_global_convergence_first_part_proof_5} using the above relations, note that, 
\begin{align}
\MoveEqLeft
    \sum_{a^- \in \gA^-(i_1)}{ \frac{\pi_{\theta_s}(a^-)}{ \sum_{a^+ \in \gA^+(i_1)}{ \pi_{\theta_s}(a^+) \cdot (r(a^+) - r(i_1) ) } } \cdot ( r(i_1) - r(a^-)) } \\
    &\qquad <  \sum_{a^- \in \gA^-(i_1)}{ \frac{\pi_{\theta_s}(a^-)}{ \pi_{\theta_s}(i_2) \cdot (r(i_2) - r(i_1) ) } \cdot ( r(i_1) - r(a^-)) } \tag{Fewer terms in the denominator} \\    
    \intertext{By \cref{lem:unbouned_prob_ratio}, for large enough $\tau \geq 1$, for all $a^-$, $\frac{\pi_{\theta_s}(a^-)}{ \pi_{\theta_s}(i_2)} \leq \frac{1}{\left| \gA^-(i_1) \right|} \cdot \frac{r(i_2) - r(i_1)}{r(i_1) - r(a^-)}$. Hence,}
    &\qquad \le \sum_{a^- \in \gA^-(i_1)}{ \frac{1}{2} \cdot \frac{1}{\left| \gA^-(i_1) \right|} \cdot \frac{r(i_2) - r(i_1)}{r(i_1) - r(a^-)}  \cdot \frac{r(i_1) - r(a^-)}{r(i_2) - r(i_1)} } \\
    &\qquad = \frac{1}{2} \\
    \implies & \sum_{a^- \in \gA^-(i_1)} \pi_{\theta_s}(a^-) \cdot ( r(i_1) - r(a^-)) \leq \frac{1}{2} \, \sum_{a^+ \in \gA^+(i_1)} \pi_{\theta_s}(a^+) \cdot (r(a^+) - r(i_1) ). \label{eq:general_action_global_convergence_first_part_proof_11}
\end{align}
Combining \cref{eq:general_action_global_convergence_first_part_proof_4,eq:general_action_global_convergence_first_part_proof_5,eq:general_action_global_convergence_first_part_proof_11}, we have,
\begin{align}
    \sum_{s=\tau}^t P_s(i_1) &= \sum_{s=\tau}^t \eta \cdot \pi_{\theta_s}(i_1) \cdot ( r(i_1) - \pi_{\theta_s}^\top r ) \\
    &\le - \frac{\eta}{2} \cdot \sum_{s=\tau}^t \pi_{\theta_s}(i_1) \sum_{a^+ \in \gA^+(i_1)}{ \pi_{\theta_s}(a^+) \cdot (r(a^+) - r(i_1) ) } \\
    &\le - \frac{\eta \cdot \Delta}{2} \cdot \sum_{s=\tau}^t \pi_{\theta_s}(i_1) \sum_{a^+ \in \gA^+(i_1)}{ \pi_{\theta_s}(a^+) } \label{eq:general_action_global_convergence_first_part_proof_12} \,,
\end{align}
where
\begin{align}
\label{eq:general_action_global_convergence_first_part_proof_13}
    \Delta \coloneqq \min_{i, j \in [K], \ i \not= j}{ |r(i) - r(j)| } > 0. \qquad \left( \text{by \cref{assp:reward_no_ties}} \right)
\end{align}
Bounding the variance for arm $i_1$, we have that for all large enough $\tau \ge 1$,
\begin{align}
    V_t(i_1) &\coloneqq \frac{5}{18} \cdot \sum_{s=\tau}^{t-1}  \pi_{\theta_s}(i_1) \cdot (1-\pi_{\theta_s}(i_1)) \\
    &= \frac{5}{18} \cdot \sum_{s=\tau}^{t-1}  \pi_{\theta_s}(i_1) \cdot \left( \sum_{a^- \in \gA^-(i_1)} \pi_{\theta_s}(a^-) + \sum_{a^+ \in \gA^+(i_1)} \pi_{\theta_s}(a^+) \right) \label{eq:general_action_global_convergence_first_part_proof_14}
\end{align}
Using similar calculations as for the progress term, we have,
\begin{align}
    \sum_{a^- \in \gA^-(i_1)} \frac{ \pi_{\theta_s}(a^-) }{ \sum_{a^+ \in \gA^+(i_1)} \pi_{\theta_s}(a^+) } &< \sum_{a^- \in \gA^-(i_1)} \frac{ \pi_{\theta_s}(a^-) }{  \pi_{\theta_s}(i_2) } \tag{Fewer terms in the denominator} \\
    \intertext{By \cref{lem:unbouned_prob_ratio}, for large enough $\tau \geq 1$, for all $a^-$, $\frac{\pi_{\theta_s}(a^-)}{ \pi_{\theta_s}(i_2)} \leq \frac{13}{5} \, \frac{1}{\left| \gA^-(i_1) \right|}$. Hence,}
    &\le \sum_{a^- \in \gA^-(i_1)}{ \frac{13}{5} \cdot \frac{1}{\left| \gA^-(i_1) \right|} } \\
    &= \frac{13}{5} \\
\implies \sum_{a^- \in \gA^-(i_1)} \pi_{\theta_s}(a^-)  & \leq  \frac{13}{5} \, \sum_{a^+ \in \gA^+(i_1)} \pi_{\theta_s}(a^+)  \label{eq:general_action_global_convergence_first_part_proof_15}
\end{align}
Combining \cref{eq:general_action_global_convergence_first_part_proof_14,eq:general_action_global_convergence_first_part_proof_15}, we have that for large enough $\tau \geq 1$, 
\begin{align}
    V_t(i_1) &= \frac{5}{18} \cdot \sum_{s=\tau}^{t-1}  \pi_{\theta_s}(i_1) \cdot (1-\pi_{\theta_s}(i_1)) \\
    &\le \sum_{s=\tau}^{t-1}  \pi_{\theta_s}(i_1) \sum_{a^+ \in \gA^+(i_1)}{ \pi_{\theta_s}(a^+) }. \label{eq:general_action_global_convergence_first_part_proof_16}
\end{align}
According to \cref{lem:negative_progress_upper_bounded_parameter} (using \cref{eq:general_action_global_convergence_first_part_proof_4,eq:general_action_global_convergence_first_part_proof_12,eq:general_action_global_convergence_first_part_proof_16}), we have, almost surely,
\begin{align}
\label{eq:general_action_global_convergence_first_part_proof_17}
    \sup_{t \ge 1}{ \theta_t(i_1) } < \infty.
\end{align}
Since $N_\infty(a^*) < \infty$ by assumption, and according to \cref{lem:finite_sample_time_implies_finite_parameter}, we have, almost surely, 
\begin{align}
\label{eq:general_action_global_convergence_first_part_proof_18}
    \inf_{t \ge 1}{ \theta_t(a^*)} > -\infty.
\end{align}
Combining \cref{eq:general_action_global_convergence_first_part_proof_17,eq:general_action_global_convergence_first_part_proof_18}, we have,
\begin{align}
\label{eq:general_action_global_convergence_first_part_proof_19}
    \sup_{t \ge 1}{ \frac{\pi_{\theta_t}(i_1)}{\pi_{\theta_t}(a^*)} } = \sup_{t \ge 1} \, \exp\{ \theta_t(i_1) - \theta_t(a^*) \}  < \infty \,.
\end{align}
On the other hand, by \cref{lem:unbouned_prob_ratio}, we have,
\begin{align}
\label{eq:general_action_global_convergence_first_part_proof_20}
    \sup_{t \ge 1}{ \frac{\pi_{\theta_t}(i_1)}{\pi_{\theta_t}(a^*)} } = \infty
\end{align}
which contradicts \cref{eq:general_action_global_convergence_first_part_proof_19}. Hence, the assumption that $N_\infty(a^*) < \infty$ cannot hold. This completes the proof by contradiction, and implies that $N_\infty(a^*) = \infty$. 

\clearpage
\textbf{Second part.} With $a^* \in \gA_\infty$ i.e. $N_\infty(a^*) =  \infty$, we now argue that $\pi_{\theta_t}(a^*) \to 1$ as $t \to \infty$ almost surely.

According to \cref{lem:at_least_two_actions_infinite_sample_time}, we have, $| \gA_\infty | \ge 2$. Since $a^* \in \gA_\infty$ by assumption, there must be at least one sub-optimal action $i_1\in[K]$ with $r(i_1) < r(a^*)$, such that,
\begin{align}
\label{eq:general_action_global_convergence_second_part_proof_1}
    N_\infty(i_1) = \infty.
\end{align}
In particular, define
\begin{align}
\label{eq:general_action_global_convergence_second_part_proof_2}
    i_1 \coloneqq \argmin_{\substack{a \in [K], \\ N_\infty(a) = \infty}} r(a).
\end{align}
According to \cref{lem:expected_reward_range},  we have, for all sufficiently large $\tau \ge 1$,
\begin{align}
\label{eq:general_action_global_convergence_second_part_proof_3}
    r(i_1) < \pi_{\theta_t}^\top r < r(a^*).
\end{align}
Using the same arguments as in the first part (except that $i_2$ in \cref{eq:general_action_global_convergence_first_part_proof_10} is replaced with $a^*$), both
\cref{eq:general_action_global_convergence_first_part_proof_12,eq:general_action_global_convergence_first_part_proof_16} hold. 

Next, we argue that $\sum_{s=\tau}^t \pi_{\theta_s}(i_1) \sum_{a^+ \in \gA^+(i_1)}{ \pi_{\theta_s}(a^+) } \to \infty$ as $t \to \infty$ by contradiction. 

Suppose
\begin{align}
\label{eq:general_action_global_convergence_second_part_proof_4}
    \sum_{t=\tau}^{\infty} \pi_{\theta_t}(i_1) \sum_{a^+ \in \gA^+(i_1)}{ \pi_{\theta_t}(a^+) } < \infty.
\end{align}
According to \cref{lem:bounded_progress_bounded_parameter} (using \cref{eq:general_action_global_convergence_first_part_proof_4,eq:general_action_global_convergence_first_part_proof_12,eq:general_action_global_convergence_first_part_proof_16,eq:general_action_global_convergence_second_part_proof_4}), we have,
\begin{align}
\label{eq:general_action_global_convergence_second_part_proof_5}
    \sup_{t \ge 1}{ |\theta_t(i_1)| } < \infty.
\end{align}
Calculating the progress and variance for arm $a^*$, for $t \geq 1$, 
\begin{align}
\label{eq:general_action_global_convergence_second_part_proof_5b}
    P_t(a^*) &= \eta \cdot \pi_{\theta_t}(a^*) \cdot (r(a^*) -  \pi_{\theta_t}^\top r ) \\
    &\geq \eta \cdot \Delta \cdot \pi_{\theta_t}(a^*) \cdot \big( 1 - \pi_{\theta_t}(a^*) \big). \qquad \big( \Delta \coloneqq r(a^*) - \max_{a \neq a^*}{r(a)} \big) \\
    &\ge 0.
\end{align}
Denote that, for all $t 
\ge 1$, 
\begin{align}
\label{eq:general_action_global_convergence_second_part_proof_5c}
    V_t(a^*) &\coloneqq \frac{5}{18} \cdot \sum_{s=1}^{t-1}  \pi_{\theta_s}(a^*) \cdot (1-\pi_{\theta_s}(a^*)).
\end{align}
According to \cref{lem:positive_progress_lower_bounded_parameter} (using \cref{eq:general_action_global_convergence_second_part_proof_5b,eq:general_action_global_convergence_second_part_proof_5c}), we have,
\begin{align}
\label{eq:general_action_global_convergence_second_part_proof_6}
    \inf_{t \ge 1}{\theta_t(a^*)} > -\infty.
\end{align}
Combining \cref{eq:general_action_global_convergence_second_part_proof_5,eq:general_action_global_convergence_second_part_proof_6}, we have,
\begin{align}
\label{eq:general_action_global_convergence_second_part_proof_7}
    \sup_{t \ge 1}{ \frac{\pi_{\theta_t}(i_1)}{\pi_{\theta_t}(a^*)}} &= \sup_{t \ge 1} {\exp\{ \theta_t(i_1) - \theta_t(a^*) \}} < \infty,
\end{align}
For all $t \geq 1$, $|\theta_t(i_1)| < \infty$ and since there is at least one arm ($a^*$) s.t. $\inf_{t \geq 1} \theta_t(a) > -\infty$, there exists $\epsilon > 0$  and $\epsilon \in O(1)$, such that,
\begin{align}
\label{eq:general_action_global_convergence_second_part_proof_8}
    \sup_{t \ge 1}{ \pi_{\theta_t}(i_1)} < 1 - 2 \, \epsilon.
\end{align}
According to \cref{eq:general_action_global_convergence_second_part_proof_1}, we know that $N_\infty(i_1) = \infty$ and for all $a^- \in \gA^{-}(i_1)$, $N_\infty(a^-) < \infty$. Using~\cref{lem:unbouned_prob_ratio}, we have, for all large enough $t \ge 1$, $\frac{\pi_{\theta_t}(a^-)}{\pi_{\theta_t}(i_1)} < \frac{\epsilon}{|\gA^{-}(i_1)|}$. 
\begin{align}
    \pi_{\theta_t}(i_1) + \sum_{a^+ \in \gA^+(i_1)}{ \pi_{\theta_t}(a^+) } &= 1 - \pi_{\theta_t}(i_1) \,  \sum_{a^- \in \gA^-(i_1)}{ \frac{\pi_{\theta_t}(a^-)}{\pi_{\theta_t}(i_1)} } > 1 - \pi_{\theta_t}(i_1) \, \epsilon \\
    \implies \pi_{\theta_t}(i_1) + \sum_{a^+ \in \gA^+(i_1)}{ \pi_{\theta_t}(a^+) } & \ge 1 - \epsilon. \label{eq:general_action_global_convergence_second_part_proof_9}
\end{align}
Combining \cref{eq:general_action_global_convergence_second_part_proof_8,eq:general_action_global_convergence_second_part_proof_9}, we have, for all large enough $t \ge 1$,
\begin{align}
\label{eq:general_action_global_convergence_second_part_proof_10}
    \sum_{a^+ \in \gA^+(i_1)}{ \pi_{\theta_t}(a^+) } \ge \epsilon,
\end{align}
which implies that,
\begin{align}
    \sum_{s=\tau}^{\infty} \pi_{\theta_s}(i_1) \sum_{a^+ \in \gA^+(i_1)}{ \pi_{\theta_s}(a^+) } &\ge \epsilon \cdot \sum_{s=\tau}^{\infty} \pi_{\theta_s}(i_1) \\
    &= \infty, \qquad \left( \text{since $N_\infty(i_1) = \infty$ and by \cref{lem:ebc}} \right) \label{eq:general_action_global_convergence_second_part_proof_11}
\end{align}
which contradicts the assumption of \cref{eq:general_action_global_convergence_second_part_proof_4}. This completes the proof by contradiction, and therefore, we have,
\begin{align}
\label{eq:general_action_global_convergence_second_part_proof_12}
    \sum_{s=\tau}^t \pi_{\theta_s}(i_1) \sum_{a^+ \in \gA^+(i_1)}{ \pi_{\theta_s}(a^+) } \to \infty, \text{ as } t \to \infty.
\end{align}
According to \cref{lem:negative_unbounded_progress_unbounded_parameter} (using \cref{eq:general_action_global_convergence_first_part_proof_12,eq:general_action_global_convergence_first_part_proof_16,eq:general_action_global_convergence_second_part_proof_12}), we have, almost surely, 
\begin{align}
\label{eq:general_action_global_convergence_second_part_proof_13}
    \theta_t(i_1) \to  - \infty, \text{ as } t \to \infty.
\end{align}
Combining \cref{eq:general_action_global_convergence_second_part_proof_6,eq:general_action_global_convergence_second_part_proof_13}, we have, almost surely,
\begin{align}
\label{eq:general_action_global_convergence_second_part_proof_14}
    \frac{ \pi_{\theta_t}(a^*) }{ \pi_{\theta_t}(i_1)} = \exp\{ \theta_t(a^*) - \theta_t(i_1) \} \to \infty, \text{ as } t \to \infty.
\end{align}
Hence, we have proved that if $N_\infty(i_1) = \infty$ and $r(i_1) < \pi_{\theta_t}^\top r < r(a^*)$ for sufficiently large $\tau \geq 1$, then, $\frac{ \pi_{\theta_t}(a^*) }{ \pi_{\theta_t}(i_1)} \to \infty, \text{ as } t \to \infty$. 

In order to use this argument recursively, consider sorting the action indices in $\gA_\infty$ according to their descending expected reward values,
\begin{align}
\label{eq:general_action_global_convergence_second_part_proof_15}
    r(a^*) > r(i_{|\gA_\infty| - 1}) > r(i_{|\gA_\infty| - 2}) > \cdots > r(i_2) > r(i_1).
\end{align}
We know that, 
\begin{align}
    \pi_{\theta_t}^\top r - r(i_2) &= \sum_{a \neq i_2} \pi_{\theta_t}(a) \cdot (r(a) - r(i_2)) \\
    & = \sum_{a^- \in \gA^-(i_2)}{ \pi_{\theta_t}(a^-) \cdot (r(a^-) - r(i_2)) } + \sum_{a^+ \in \gA^+(i_2)}{ \pi_{\theta_t}(a^+) \cdot ( r(a^+) - r(i_2)) }
    \\
    &> \pi_{\theta_t}(a^*) \cdot (r(a^*) - r(i_2) ) - \sum_{a^- \in \gA^-(i_2)}{ \pi_{\theta_t}(a^-) \cdot ( r(i_2) - r(a^-)) } \\
    &= \pi_{\theta_t}(a^*) \cdot \bigg[ r(a^*) - r(i_2) - \sum_{a^- \in \gA^-(i_2)}{ \frac{ \pi_{\theta_t}(a^-)}{ \pi_{\theta_t}(a^*) } \cdot ( r(i_2) - r(a^-)) } \bigg], \label{eq:general_action_global_convergence_second_part_proof_16}
\end{align}
According to \cref{lem:unbouned_prob_ratio}, for all $a^- \in \gA^-(i_2)$ with $a^- \ne i_1$, we have, 
\begin{align}
\label{eq:general_action_global_convergence_second_part_proof_18}
    \frac{ \pi_{\theta_t}(a^*) }{ \pi_{\theta_t}(a^-)} \to \infty, \text{ as } t \to \infty.
\end{align}
Combining \cref{eq:general_action_global_convergence_second_part_proof_14,eq:general_action_global_convergence_second_part_proof_18}, we have, for all $a^- \in \gA^-(i_2)$, 
\begin{align}
    \frac{ \pi_{\theta_t}(a^*) }{ \pi_{\theta_t}(a^-)} \to \infty, \text{ as } t \to \infty.
\end{align}
Hence, for all sufficiently large $\tau \ge 1$, for all $a^- \in \gA^{-}(i_2)$, 
\begin{align}
\frac{ \pi_{\theta_t}(a^-) }{ \pi_{\theta_t}(a^*)} \leq \frac{1}{2 \, |\gA^{-}(i_2)|} \, \frac{r(a^*) - r(i_2)}{r(i_2) - r(a^-)}.
\label{eq:general_action_global_convergence_second_part_proof_19}
\end{align}
Combining \cref{eq:general_action_global_convergence_second_part_proof_16,eq:general_action_global_convergence_second_part_proof_19}, we have, for all sufficiently large $t \ge 1$,
\begin{align}
\label{eq:general_action_global_convergence_second_part_proof_20}
    \pi_{\theta_t}^\top r - r(i_2) > \pi_{\theta_t}(a^*) \cdot \frac{ r(a^*) - r(i_2)}{2} > 0.
\end{align}
Hence we have, for all sufficiently large $\tau \ge 1$,
\begin{align}
\label{eq:general_action_global_convergence_second_part_proof_21}
    r(i_2) < \pi_{\theta_t}^\top r < r(a^*)
\end{align}
Comparing \cref{eq:general_action_global_convergence_second_part_proof_21,eq:general_action_global_convergence_second_part_proof_3}, we can use a similar argument for $i_2$ and conclude that, for sufficiently large $\tau \geq 1$, then, $\frac{ \pi_{\theta_t}(a^*) }{ \pi_{\theta_t}(i_2)} \to \infty, \text{ as } t \to \infty$. This further implies that
\begin{align}
r(i_3) < \pi_{\theta_t}^\top r < r(a^*).    
\end{align}
Continuing this recursive argument, we have, for all actions $a \in \gA_\infty$ with $a \ne a^*$,
\begin{align}
\label{eq:general_action_global_convergence_second_part_proof_24}
    \frac{ \pi_{\theta_t}(a^*) }{ \pi_{\theta_t}(a)} \to \infty, \text{ as } t \to \infty.
\end{align}
Meanwhile, according to \cref{lem:unbouned_prob_ratio}, we have, for all actions $a \not\in \gA_\infty$,
\begin{align}
\label{eq:general_action_global_convergence_second_part_proof_25}
    \frac{ \pi_{\theta_t}(a^*) }{ \pi_{\theta_t}(a)} \to \infty, \text{ as } t \to \infty.
\end{align}
Combining \cref{eq:general_action_global_convergence_second_part_proof_24,eq:general_action_global_convergence_second_part_proof_25}, we have, for all sub-optimal actions $a \in [K]$ with $r(a) < r(a^*)$,
\begin{align}
\label{eq:general_action_global_convergence_second_part_proof_26}
    \frac{ \pi_{\theta_t}(a^*) }{ \pi_{\theta_t}(a)} \to \infty, \text{ as } t \to \infty.
\end{align}
Finally, note that, 
\begin{align}
    \pi_{\theta_t}(a^*)
    &= \frac{\pi_{\theta_t}(a^*)}{ \sum_{a \in [K]: \ r(a) < r(a^*)} \pi_{\theta_t}(a) +  \pi_{\theta_t}(a^*)} \\
    &= \frac{1}{ \sum_{a \in [K]: \ r(a) < r(a^*)} \frac{\pi_{\theta_t}(a)}{\pi_{\theta_t}(a^*)}  +  1} \label{eq:general_action_global_convergence_second_part_proof_27}.
\end{align}
Combining \cref{eq:general_action_global_convergence_second_part_proof_26,eq:general_action_global_convergence_second_part_proof_27}, we have, almost surely,
\begin{equation*}
    \pi_{\theta_t}(a^*) \to 1, \text{ as } t \to \infty. \qedhere
\end{equation*}
\end{proof}

\clearpage
\section{Miscellaneous Extra Supporting Results}

\begin{lemma}[Extended Borel-Cantelli Lemma, Corollary 5.29 of \citep{breiman1992probability}]
\label{lem:ebc}
Let $( \gF_n)_{n \ge 1}$ be a filtration, $A_n \in \gF_n$.
Then, almost surely, 
\begin{align}
\{ \omega \,: \, \omega \in A_n \text{ infinitely often } \} = \left\{ \omega \, : \, 
\sum_{n=1}^\infty \sP(A_n|\gF_n) = \infty \right\}\,.
\end{align}
\end{lemma}

\begin{lemma}[Freedman’s inequality \citep{freedman1975on,cesa2005improved}, Theorem C.3 of \citep{mei2024stochastic}]
\label{lem:conc_new}
    Let $X_1, X_2, \dots$ be a sequence of random variables, such that for all $t \ge 1$, $|X_t|\le 1/2 $. Define 
\begin{align}
    S_n \coloneqq \left| \sum_{t=1}^n \EE{ X_t | X_1, \dots, X_{t-1} } - X_t \right|
    \quad \text{ and } \quad 
    V_n \coloneqq \sum_{t=1}^n \mathrm{Var}[ X_t | X_1, \ldots, X_{t-1} ].
\end{align}
Then, for all $\delta> 0$,
\begin{align}
    \probability{ \left( \exists \ n:\ S_n\ge  6 \ \sqrt{  \left(V_n+ \frac{4}{3} \right) \ \log \left( \frac{  V_n+1  }{ \delta } \right) } + 2\log \left( \frac{1}{\delta} \right)  + \frac{4}{3} \log 3
    ~ \right) } \le \delta.
\end{align}
\end{lemma}

\begin{lemma}
\label{lem:unbouned_prob_ratio}
Using \cref{alg:gradient_bandit_algorithm_sampled_reward}, for any two different actions $i, j \in [K]$ with $i \ne j$, if $N_\infty(i) = \infty$ and $N_\infty(j) < \infty$, then we have, almost surely,
\begin{align}
    \sup_{t \ge 1}{ \frac{ \pi_{\theta_t}(i) }{  \pi_{\theta_t}(j) } } = \infty.
\end{align}
\end{lemma}
\begin{proof}
We prove the result by contradiction. Suppose
\begin{align}
\label{eq:unbouned_prob_ratio_proof_1}
    c \coloneqq \sup_{t \ge 1}{ \frac{ \pi_{\theta_t}(i) }{  \pi_{\theta_t}(j) } } < \infty.
\end{align}
According to the extended Borel-Cantelli \cref{lem:ebc}, we have, for all $a \in [K]$, almost surely,
\begin{align}
\label{eq:unbouned_prob_ratio_proof_2}
    \Big\{ \sum_{t \ge 1} \pi_{\theta_t}(j)=\infty \Big\} = \left\{ N_\infty(j)=\infty \right\}.
\end{align}
Hence, taking complements, we have,
\begin{align}
\label{eq:unbouned_prob_ratio_proof_3}
    \Big\{ \sum_{t \ge 1} \pi_{\theta_t}(j)<\infty \Big\} = \left\{N_\infty(j)<\infty\right\}
\end{align}
also holds almost surely, which implies that,
\begin{align}
\label{eq:unbouned_prob_ratio_proof_4a}
    \sum_{t=1}^{\infty} \pi_{\theta_t}(i) &= \infty, \text{ and} \\
\label{eq:unbouned_prob_ratio_proof_4b}
    \sum_{t=1}^{\infty} \pi_{\theta_t}(j) &< \infty.
\end{align}
Therefore, we have,
\begin{align}
\label{eq:unbouned_prob_ratio_proof_5}
    \sum_{t=1}^{\infty} \pi_{\theta_t}(i) &= \sum_{t=1}^{\infty} \pi_{\theta_t}(j) \cdot \frac{ \pi_{\theta_t}(i) }{ \pi_{\theta_t}(j) } \\
    &\le c \cdot \sum_{t=1}^{\infty} \pi_{\theta_t}(j) \\
    &< \infty,
\end{align}
which is a contradiction with \cref{eq:unbouned_prob_ratio_proof_4a}.
\end{proof}

\begin{lemma}
\label{lem:expected_reward_range}
Using \cref{alg:gradient_bandit_algorithm_sampled_reward}, we have, almost surely, for all large enough $t \ge 1$,
\begin{align}
\label{eq:expected_reward_range_claim_1}
    r(i_1) < \pi_{\theta_t}^\top r < r(i_2),
\end{align}
where $i_1, i_2 \in [K]$ and $i_1 \ne i_2$ are the action indices defined as,
\begin{align}
\label{eq:expected_reward_range_claim_2a}
    i_1 &\coloneqq \argmin_{\substack{a \in [K], \\ N_\infty(a) = \infty}} r(a), \\
\label{eq:expected_reward_range_claim_2b}
    i_2 &\coloneqq \argmax_{\substack{a \in [K], \\ N_\infty(a) = \infty}} r(a).
\end{align}
\end{lemma}
\begin{proof}
Define $\gA_\infty$ as the set of actions which are sampled for infinitely many times as $t \to \infty$, i.e.,
\begin{align}
\label{eq:expected_reward_range_proof_1}
    \gA_\infty \coloneqq \left\{ a \in [K] \ | \ N_\infty(a) = \infty \right\}.
\end{align}
According to \cref{lem:at_least_two_actions_infinite_sample_time}, we have $| \gA_\infty | \ge 2$, which implies that $i_1 \ne i_2$.

Given any sub-optimal action $i \in [K]$ with $r(i) < r(a^*)$, we partition the remaining actions into two parts using $r(i)$.
\begin{align}
\label{eq:expected_reward_range_proof_2a}
    \gA^+(i) &\coloneqq \left\{ a^+ \in [K]: r(a^+) > r(i) \right\}, \\
\label{eq:expected_reward_range_proof_2b}
    \gA^-(i) &\coloneqq \left\{ a^- \in [K]: r(a^-) < r(i) \right\}.
\end{align}
By definition, we have,
\begin{align}
\label{eq:expected_reward_range_proof_3a}
    &\gA_\infty \subseteq \gA^+(i_1) \cup \{ i_1 \}, \text{ and} \\
\label{eq:expected_reward_range_proof_3b}
    &\gA_\infty \subseteq \gA^-(i_2) \cup \{ i_2 \}.
\end{align}

\textbf{First part.} $r(i_1) < \pi_{\theta_t}^\top r$. 

If $r(i_1) = \min_{a \in [K]} r(a)$, i.e., $i_1$ is the ``worst action'', then $r(i_1) < \pi_{\theta_t}^\top r$ holds trivially. Otherwise, suppose $r(i_1) \ne \min_{a \in [K]} r(a)$. We have,
\begin{align}
\label{eq:expected_reward_range_proof_4}
    \pi_{\theta_t}^\top r - r(i_1) &= \sum_{a^+ \in \gA^+(i_1)}{ \pi_{\theta_t}(a^+) \cdot (r(a^+) - r(i_1) ) } - \sum_{a^- \in \gA^-(i_1)}{ \pi_{\theta_t}(a^-) \cdot ( r(i_1) - r(a^-)) }.
\end{align}
Consider the non-empty set $\gA_\infty \cap \gA^+(i_1)$. Pick an action $j_1 \in \gA_\infty \cap \gA^+(i_1)$, and ignore all the other actions $a^+ \in \gA^+(i_1)$ with $a^+ \ne j_1$ in the above equation. We have,
\begin{align}
\label{eq:expected_reward_range_proof_5}
    \pi_{\theta_t}^\top r - r(i_1) &> \pi_{\theta_t}(j_1) \cdot (r(j_1) - r(i_1) ) - \sum_{a^- \in \gA^-(i_1)}{ \pi_{\theta_t}(a^-) \cdot ( r(i_1) - r(a^-)) } \\
    &= \pi_{\theta_t}(j_1) \cdot \bigg[ r(j_1) - r(i_1) - \sum_{a^- \in \gA^-(i_1)}{ \frac{ \pi_{\theta_t}(a^-)}{ \pi_{\theta_t}(j_1) } \cdot ( r(i_1) - r(a^-)) } \bigg],
\end{align}
where the first inequality is because of $r(a^+) - r(i_1) > 0$ for all $a^+ \in \gA^+(i_1)$ by \cref{eq:expected_reward_range_proof_2a}. Note that $N_\infty(j_1) = \infty$ and $N_\infty(a^-) < \infty$. According to \cref{lem:unbouned_prob_ratio}, we have, for all large enough $t \ge 1$, 
\begin{align}
\label{eq:expected_reward_range_proof_6}
\MoveEqLeft
    \sum_{a^- \in \gA^-(i_1)}{ \frac{ \pi_{\theta_t}(a^-)}{ \pi_{\theta_t}(j_1) } \cdot ( r(i_1) - r(a^-)) } = \sum_{a^- \in \gA^-(i_1)}{ ( r(i_1) - r(a^-)) \Big/ \frac{ \pi_{\theta_t}(j_1) }{ \pi_{\theta_t}(a^-)} } \\
    &< \sum_{a^- \in \gA^-(i_1)}{ ( r(i_1) - r(a^-)) \cdot \frac{1}{\big| \gA^-(i_1) \big| } \cdot \frac{r(j_1) - r(i_1)}{r(i_1) - r(a^-)} \cdot \frac{1}{2} } \\
    &= \frac{r(j_1) - r(i_1)}{2}.
\end{align}
Combining \cref{eq:expected_reward_range_proof_5,eq:expected_reward_range_proof_6}, we have,
\begin{align}
\label{eq:expected_reward_range_proof_7}
    \pi_{\theta_t}^\top r - r(i_1) > \pi_{\theta_t}(j_1) \cdot \frac{r(j_1) - r(i_1)}{2} > 0,
\end{align}
where the last inequality is because  $j_1 \in \gA^+(i_1)$.

\textbf{Second part.} $\pi_{\theta_t}^\top r < r(i_2)$.

The arguments are similar to the first part. If $r(i_2) = r(a^*)$, i.e., $i_2$ is the optimal action, then $\pi_{\theta_t}^\top r < r(i_2)$ holds trivially. Otherwise, suppose $r(i_2) \ne r(a^*)$. We have,
\begin{align}
\label{eq:expected_reward_range_proof_8}
    r(i_2) - \pi_{\theta_t}^\top r &= \sum_{a^- \in \gA^-(i_2)}{ \pi_{\theta_t}(a^-) \cdot (r(i_2) - r(a^-) ) } - \sum_{a^+ \in \gA^+(i_2)}{ \pi_{\theta_t}(a^+) \cdot ( r(a^+) - r(i_2)) }.
\end{align}
Consider the non-empty set $\gA_\infty \cap \gA^-(i_2)$. Pick an action $j_2 \in \gA_\infty \cap \gA^-(i_2)$, and ignore all the other actions $a^- \in \gA^-(i_2)$ with $a^- \ne j_2$ in the above equation. We have,
\begin{align}
\label{eq:expected_reward_range_proof_9}
    r(i_2) - \pi_{\theta_t}^\top r &\ge \pi_{\theta_t}(j_2) \cdot (r(i_2) - r(j_2) ) - \sum_{a^+ \in \gA^+(i_2)}{ \pi_{\theta_t}(a^+) \cdot ( r(a^+) - r(i_2)) } \\
    &=\pi_{\theta_t}(j_2) \cdot \bigg[ r(i_2) - r(j_2) - \sum_{a^+ \in \gA^+(i_2)}{ \frac{ \pi_{\theta_t}(a^+)}{\pi_{\theta_t}(j_2)} \cdot ( r(a^+) - r(i_2)) } \bigg],
\end{align}
where the first inequality is because of $r(i_2) - r(a^-) > 0$ for all $a^- \in \gA^-(i_2)$ by \cref{eq:expected_reward_range_proof_2a}. Note that $N_\infty(j_2) = \infty$ and $N_\infty(a^+) < \infty$. According to \cref{lem:unbouned_prob_ratio}, we have, for all large enough $t \ge 1$, 
\begin{align}
\label{eq:expected_reward_range_proof_10}
\MoveEqLeft
    \sum_{a^+ \in \gA^+(i_2)}{ \frac{ \pi_{\theta_t}(a^+)}{\pi_{\theta_t}(j_2)} \cdot ( r(a^+) - r(i_2)) } = \sum_{a^+ \in \gA^+(i_2)}{ ( r(a^+) - r(i_2)) \Big/ \frac{\pi_{\theta_t}(j_2)}{ \pi_{\theta_t}(a^+)}  } \\
    &< \sum_{a^+ \in \gA^+(i_2)}{ ( r(a^+) - r(i_2)) \cdot \frac{1}{\big| \gA^+(i_2) \big| } \cdot \frac{r(i_2) - r(j_2)}{r(a^+) - r(i_2)} \cdot \frac{1}{2} } \\
    &= \frac{r(i_2) - r(j_2)}{2}.
\end{align}
Combining \cref{eq:expected_reward_range_proof_9,eq:expected_reward_range_proof_10}, we have,
\begin{align}
    r(i_2) - \pi_{\theta_t}^\top r \ge \pi_{\theta_t}(j_2) \cdot \frac{r(i_2) - r(j_2)}{2} > 0,
\end{align}
where the last inequality is because  $j_2 \in \gA^-(i_2)$.
\end{proof}

For the following lemmas, we will use the notation defined in the proofs for \cref{thm:two_action_global_convergence,thm:general_action_global_convergence}.

\begin{lemma}[Concentration of noise]
\label{lem:parameter_noise_concentration}
Given an action $a \in [K]$. We have, with probability at least $1 - \delta$,
\begin{align}
\label{eq:parameter_noise_concentration_claim_1}
\MoveEqLeft
    \forall t:\quad \left| \sum_{s=1}^t W_{s+1}(a) \right| \le  36 \ \eta \ R_{\max} \ \sqrt{  (V_t(a)+4/3)\log \left( \frac{  V_t(a)+1  }{ \delta } \right) } + 12 \ \eta \ R_{\max} \ \log(1/\delta) +  8 \ \eta \ R_{\max}\log 3,
\end{align}
where
\begin{align}
\label{eq:parameter_noise_concentration_claim_2}
    V_t(a) \coloneqq \frac{5}{18} \cdot \sum_{s=1}^t  \pi_{\theta_s}(a)\cdot (1-\pi_{\theta_s}(a)),
\end{align}
and 
\begin{align}
\label{eq:parameter_noise_concentration_claim_3}
    W_t(a) \coloneqq \theta_t(a) - \chE_{t-1}{[ \theta_t(a)]}
\end{align}
is originally defined by \cref{eq:two_action_global_convergence_proof_2b}.
\end{lemma}
\begin{proof}
First, note that,
\begin{align}
\label{eq:parameter_noise_concentration_proof_1}
    \EEt{W_{t+1}(a)}=0, \text{ for all } t \ge 0.
\end{align}
Using the update \cref{eq:finite_sample_time_implies_finite_parameter_proof_1}, we have,
\begin{align}
\label{eq:parameter_noise_concentration_proof_2}
\MoveEqLeft
    W_{t+1}(a) = \theta_{t+1}(a) - \EEt{\theta_{t+1}(a)} \\
    &=  \theta_{t}(a) + \eta \cdot \left( I_t(a) - \pi_{\theta_t}(a) \right) \cdot R_t(a_t)   - \left( \theta_{t}(a) + \eta \cdot \pi_{\theta_t}(a) \cdot \left( r(a) - \pi_{\theta_t}^\top r \right) \right) \\
    &= \eta \cdot \left( I_t(a) - \pi_{\theta_t}(a) \right) \cdot R_t(a_t)   - \eta \cdot \pi_{\theta_t}(a) \cdot \left( r(a) - \pi_{\theta_t}^\top r \right).
\end{align}
According to 
\cref{eq:true_mean_reward_expectation_bounded_sampled_reward}, we have,
\begin{align}
\label{eq:parameter_noise_concentration_proof_3}
    |W_{t+1}(a)| \le 3 \, \eta \cdot R_{\max}.
\end{align}
The conditional variance of noise is,
\begin{align}
\label{eq:parameter_noise_concentration_proof_4}
\MoveEqLeft
    \mathrm{Var}[ W_{t+1}(a) | \gF_t ] \coloneqq \EEt{( W_{t+1}(a) )^2 }  \\
    &\le 2 \ \eta^2 \cdot \EEt { \left( I_t(a) - \pi_{\theta_t}(a) \right)^2 \cdot R_t(a_t)^2 } + 2 \ \eta^2 \cdot \pi_{\theta_t}(a)^2 \cdot \left( r(a) - \pi_{\theta_t}^\top r \right)^2,
\end{align}
where the inequality is by $(a+b)^2 \le 2a^2 + 2 b^2$. Next, we have,
\begin{align}
\label{eq:parameter_noise_concentration_proof_5}
    \EEt{ \left( I_t(a) - \pi_{\theta_t}(a) \right)^2 \cdot R_t(a_t)^2 } &= \pi_t(a) \cdot ( 1 - \pi_t(a) )^2 \cdot r(a)^2 + \sum_{a'\ne a} \pi_{\theta_t}(a^\prime) \cdot \pi_t(a)^2 \cdot r(a^\prime)^2 \\
    &\le R_{\max}^2 \cdot \Big( \pi_t(a) \cdot ( 1 - \pi_t(a) )^2 + ( 1 - \pi_t(a) ) \cdot \pi_t(a)^2  \Big) \\
    &= R_{\max}^2 \cdot  \pi_{\theta_t}(a) \cdot ( 1 - \pi_{\theta_t}(a) ),
\end{align}
and,
\begin{align}
\label{eq:parameter_noise_concentration_proof_6}
    \left| r(a) - \pi_{\theta_t}^\top r  \right|  
    &= \bigg| \sum_{a^\prime \ne a} \pi_{\theta_t}(a^\prime) \cdot \left( r(a)-r(a^\prime) \right) \bigg| \\ 
    &\le \sum_{a^\prime \ne a} \pi_{\theta_t}(a^\prime) \cdot \left|  r(a)-r(a^\prime)  \right| 
    \\
    &\le 2 \ R_{\max} \cdot \sum_{a^\prime \ne a} \pi_{\theta_t}(a^\prime) \\
    &= 2 \ R_{\max} \cdot \left( 1-\pi_{\theta_t}(a) \right).
\end{align}
Combining \cref{eq:parameter_noise_concentration_proof_4,eq:parameter_noise_concentration_proof_5,eq:parameter_noise_concentration_proof_6}, we have,
\begin{align}
\label{eq:parameter_noise_concentration_proof_7}
    \mathrm{Var}[ W_{t+1}(a) | \gF_t ]
    &\le 2 \ \eta^2 \cdot R_{\max}^2 \cdot \pi_{\theta_t}(a) \cdot (1-\pi_{\theta_t}(a)) + 8 \ \eta^2 \cdot R_{\max}^2 \cdot \pi_{\theta_t}(a)^2 \cdot ( 1-\pi_{\theta_t}(a) )^2 \\
    &\le 10 \ \eta^2 \cdot R_{\max}^2 \cdot \pi_{\theta_t}(a) \cdot (1-\pi_{\theta_t}(a)).
\end{align}
Let $X_{t+1}(a) \coloneqq \frac{ W_{t+1}(a) }{  6 \ \eta \cdot R_{\max} }$. Then we have, 
\begin{align}
\label{eq:parameter_noise_concentration_proof_8a}
    |X_{t+1}(a)| &\le 1/2, \text{ and} \\
\label{eq:parameter_noise_concentration_proof_8b}
    \mathrm{Var}[ X_{t+1}(a) | \gF_t ] &\le \frac{5}{18} \cdot \pi_{\theta_t}(a) \cdot (1-\pi_{\theta_t}(a)).
\end{align}
According to \cref{lem:conc_new}, there exists an event $\gE_1 $ such that $\probability(\gE_1 ) \ge 1- \delta$, and when $\gE_1 $ holds, 
\begin{align}
\label{eq:parameter_noise_concentration_proof_9}
    &\forall t: \quad \left| \sum_{s=1}^t X_{s+1}(a) \right| \le  6 \ \sqrt{  (V_t(a)+4/3)\log \left( \frac{ V_t(a)+1  }{ \delta } \right) } + 2 \ \log(1/\delta)   + \frac{4}{3} \log 3,
\end{align}
which implies that, 
\begin{align}
\label{eq:parameter_noise_concentration_proof_10}
\MoveEqLeft
    \forall t:\quad \left| \sum_{s=1}^t W_{s+1}(a) \right| \le  36 \ \eta \ R_{\max} \ \sqrt{  (V_t(a)+4/3)\log \left( \frac{  V_t(a)+1  }{ \delta } \right) } + 12 \ \eta \ R_{\max} \ \log(1/\delta) +  8 \ \eta \ R_{\max}\log 3,
\end{align}
where $V_t(a) \coloneqq \frac{5}{18} \cdot \sum_{s=1}^t  \pi_{\theta_s}(a)\cdot (1-\pi_{\theta_s}(a))$.
\end{proof}

\begin{lemma}[Bounded progress]
\label{lem:bounded_progress_bounded_parameter}
For any action $a \in [K]$ with $N_\infty(a) = \infty$, if there exists $c > 0$ and $\tau < \infty$, such that, for all $t \ge \tau$,
\begin{align}
\label{eq:bounded_progress_bounded_parameter_claim_1}
    \sum_{s=1}^{t} \big| P_s(a) \big| &\ge c \cdot V_t(a),
\end{align}
where $V_t(a)$ is defined in \cref{eq:parameter_noise_concentration_claim_2}, and if also
\begin{align}
\label{eq:bounded_progress_bounded_parameter_claim_2}
    \sum_{t=1}^{\infty}{ \big| P_t(a) \big| } &< \infty,
\end{align}
then we have, almost surely,
\begin{align}
\label{eq:bounded_progress_bounded_parameter_claim_3}
    \sup_{t \ge 1}{ |\theta_t(a)| } < \infty.
\end{align}
\end{lemma}
\begin{proof}
According to \cref{eq:two_action_global_convergence_proof_4} and triangle inequality, we have,
\begin{align}
\label{eq:bounded_progress_bounded_parameter_proof_1}
    \big| \theta_t(a) \big| &\le \Big| \EE{\theta_1(a)} \Big| + \bigg| \sum_{s=1}^{t}{W_s(a)} \bigg| + \bigg| \sum_{s=1}^{t-1}{P_s(a)} \bigg|.
\end{align}
According to 
\cref{lem:parameter_noise_concentration}, there exists an event $\gE_1 $, such that $\probability(\gE_1 ) \ge 1- \delta$, and when $\gE_1 $ holds, we have, for all $t \ge \tau+1$,
\begin{align}
\label{eq:bounded_progress_bounded_parameter_proof_2}
    \bigg| \sum_{s=1}^{t}{W_s(a)} \bigg| &\le 36 \ \eta \ R_{\max} \ \sqrt{  (V_{t-1}(a)+4/3) \cdot \log \left( \frac{  V_{t-1}(a)+1  }{ \delta } \right) } + 12 \ \eta \ R_{\max} \ \log(1/\delta) +  8 \ \eta \ R_{\max}\log 3.
\end{align}
According to \cref{eq:bounded_progress_bounded_parameter_claim_2}, as $t \to \infty$,
\begin{align}
\label{eq:bounded_progress_bounded_parameter_proof_3}
    \bigg| \sum_{s=1}^{t-1}{P_s(a)} \bigg| \le \sum_{s=1}^{t-1}{ \big| P_s(a) \big| } < \infty.
\end{align}
According to \cref{eq:bounded_progress_bounded_parameter_proof_2,eq:bounded_progress_bounded_parameter_proof_3,eq:bounded_progress_bounded_parameter_claim_1}, we have,
\begin{align}
\label{eq:bounded_progress_bounded_parameter_proof_4}
    \bigg| \sum_{s=1}^{t}{W_s(a)} \bigg| < \infty.
\end{align}
Combining \cref{eq:bounded_progress_bounded_parameter_proof_1,eq:bounded_progress_bounded_parameter_proof_3,eq:bounded_progress_bounded_parameter_proof_4}, we have, as $t \to \infty$,
\begin{align}
\label{eq:bounded_progress_bounded_parameter_proof_5}
    |\theta_t(a)| < \infty.
\end{align}
Take any $\omega \in \gE \coloneqq \{ N_\infty(a) = \infty \}$. Because $\sP{\left( \gE  \setminus \left( \gE  \cap \gE_1  \right) \right)} \le \sP{\left( \Omega \setminus \gE_1  \right)} \le \delta \to 0$  as $\delta\to 0$, we have that $\sP$-almost surely for all $\omega \in \gE $
there exists $\delta > 0$ such that $\omega \in \gE \cap \gE_1 $
while
\cref{eq:bounded_progress_bounded_parameter_proof_2} also holds for this $\delta$.
Take such a $\delta$. We have, almost surely,
\begin{equation*}
    \sup_{t \ge 1}{ |\theta_t(a)| } < \infty. \qedhere
\end{equation*}
\end{proof}

\begin{lemma}[Unbounded positive progress]
\label{lem:positive_unbounded_progress_unbounded_parameter}
For any action $a \in [K]$ with $N_\infty(a) = \infty$, if there exists $c > 0$ and $\tau < \infty$, such that, for all $t \ge \tau$,
\begin{align}
\label{eq:positive_unbounded_progress_unbounded_parameter_claim_1a}
    P_t(a) &> 0, \text{ and} \\
\label{eq:positive_unbounded_progress_unbounded_parameter_claim_1b}
    \sum_{s=\tau}^{t} P_s(a) &\ge c \cdot V_t(a),
\end{align}
where $V_t(a)$ is defined in \cref{eq:parameter_noise_concentration_claim_2}, and if also,
\begin{align}
\label{eq:positive_unbounded_progress_unbounded_parameter_claim_2}
    \sum_{t=\tau}^{\infty}{ P_t(a) } = \infty,
\end{align}
then we have, almost surely,
\begin{align}
\label{eq:positive_unbounded_progress_unbounded_parameter_claim_3}
    \theta_t(a) \to \infty, \text{ as } t \to \infty.
\end{align}
\end{lemma}
\begin{proof}
According to \cref{eq:two_action_global_convergence_proof_4}, 
\begin{align}
\label{eq:positive_unbounded_progress_unbounded_parameter_proof_1}
    \theta_t(a) = \EE{\theta_1(a)} + \sum_{s=1}^{t}{W_s(a)} + \sum_{s=1}^{t-1}{P_s(a)}.
\end{align}
According to 
\cref{lem:parameter_noise_concentration}, there exists an event $\gE_1 $, such that $\probability(\gE_1 ) \ge 1- \delta$, and when $\gE_1 $ holds, we have, for all $t \ge \tau+1$,
\begin{align}
\label{eq:positive_unbounded_progress_unbounded_parameter_proof_2}
    \sum_{s=1}^{t}{W_s(a)} &\ge - 36 \ \eta \ R_{\max} \ \underbrace{\sqrt{  (V_{t-1}(a)+4/3) \cdot \log \left( \frac{  V_{t-1}(a)+1  }{ \delta } \right) }}_{\heartsuit
    } - 12 \ \eta \ R_{\max} \ \log(1/\delta) -  8 \ \eta \ R_{\max}\log 3.
\end{align}
By \cref{eq:positive_unbounded_progress_unbounded_parameter_claim_2}, as $t \to \infty$,
\begin{align}
\label{eq:positive_unbounded_progress_unbounded_parameter_proof_3}
    \sum_{s=1}^{t-1}{P_s(a)} \to \infty,
\end{align}
and the speed of $\sum_{s=1}^{t-1}{P_s(a)} \to \infty$ is strictly faster than $\heartsuit \to \infty$, according to \cref{eq:positive_unbounded_progress_unbounded_parameter_claim_1b}. This implies that $\theta_t(a) \to \infty$, as a result of ``cumulative progress'' dominates ``cumulative noise''.

Take any $\omega \in \gE \coloneqq \{ N_\infty(a) = \infty \}$. Because $\sP{\left( \gE  \setminus \left( \gE  \cap \gE_1  \right) \right)} \le \sP{\left( \Omega \setminus \gE_1  \right)} \le \delta \to 0$  as $\delta\to 0$, we have that $\sP$-almost surely for all $\omega \in \gE $
there exists $\delta > 0$ such that $\omega \in \gE \cap \gE_1 $
while
\cref{eq:positive_unbounded_progress_unbounded_parameter_proof_2} also holds for this $\delta$.
Take such a $\delta$. We have, almost surely,
\begin{equation*}
    \theta_t(a) \to \infty, \text{ as } t \to \infty. \qedhere
\end{equation*}
\end{proof}

\begin{lemma}[Unbounded negative progress]
\label{lem:negative_unbounded_progress_unbounded_parameter}
For any action $a \in [K]$ with $N_\infty(a) = \infty$, if there exists $c > 0$ and $\tau < \infty$, such that, for all $t \ge \tau$,
\begin{align}
\label{eq:negative_unbounded_progress_unbounded_parameter_claim_1a}
    P_t(a) &< 0, \text{ and} \\
\label{eq:negative_unbounded_progress_unbounded_parameter_claim_1b}
    - \sum_{s=\tau}^{t} P_s(a) &\ge c \cdot V_t(a),
\end{align}
where $V_t(a)$ is defined in \cref{eq:parameter_noise_concentration_claim_2}, and if also,
\begin{align}
\label{eq:negative_unbounded_progress_unbounded_parameter_claim_2}
    \sum_{t=\tau}^{\infty}{ P_t(a) } = - \infty,
\end{align}
then we have, almost surely,
\begin{align}
\label{eq:negative_unbounded_progress_unbounded_parameter_claim_3}
    \theta_t(a) \to - \infty, \text{ as } t \to \infty.
\end{align}
\end{lemma}
\begin{proof}
The proof follows almost the same arguments for \cref{lem:positive_unbounded_progress_unbounded_parameter}.
\end{proof}


\begin{lemma}[Positive progress]
\label{lem:positive_progress_lower_bounded_parameter}
For any action $a \in [K]$ with $N_\infty(a) = \infty$, if there exists $c > 0$ and $\tau < \infty$, such that, for all $t \ge \tau$,
\begin{align}
\label{eq:positive_progress_lower_bounded_parameter_claim_1a}
    P_t(a) &> 0, \text{ and} \\
\label{eq:positive_progress_lower_bounded_parameter_claim_1b}
    \sum_{s=\tau}^{t} P_s(a) &\ge c \cdot V_t(a),
\end{align}
where $V_t(a)$ is defined in \cref{eq:parameter_noise_concentration_claim_2}, 
then we have, almost surely,
\begin{align}
\label{eq:positive_progress_lower_bounded_parameter_claim_2}
    \inf_{t \ge 1}{ \theta_t(a) } > -\infty.
\end{align}
\end{lemma}
\begin{proof}
\textbf{First case:} if $\sum_{t=1}^{\infty} P_t(a) < \infty$, then according to \cref{lem:bounded_progress_bounded_parameter}, we have, almost surely,
\begin{align}
\label{eq:positive_progress_lower_bounded_parameter_proof_1}
    \sup_{t \ge 1}{ |\theta_t(a)| } < \infty,
\end{align}
which implies \cref{eq:positive_progress_lower_bounded_parameter_claim_2}.

\textbf{Second case:} if $\sum_{t=1}^{\infty} P_t(a) = \infty$, then according to \cref{lem:positive_unbounded_progress_unbounded_parameter}, we have, almost surely,
\begin{align}
\label{eq:positive_progress_lower_bounded_parameter_proof_2}
    \theta_t(a) \to \infty, \text{ as } t \to \infty,
\end{align}
which also implies \cref{eq:positive_progress_lower_bounded_parameter_claim_2}.
\end{proof}

\begin{lemma}[Negative progress]
\label{lem:negative_progress_upper_bounded_parameter}
For any action $a \in [K]$ with $N_\infty(a) = \infty$, if there exists $c > 0$ and $\tau < \infty$, such that, for all $t \ge \tau$,
\begin{align}
\label{eq:negative_progress_upper_bounded_parameter_claim_1a}
    P_t(a) &< 0, \text{ and} \\
\label{eq:negative_progress_upper_bounded_parameter_claim_1b}
    - \sum_{s=\tau}^{t} P_s(a) &\ge c \cdot V_t(a),
\end{align}
where $V_t(a)$ is defined in \cref{eq:parameter_noise_concentration_claim_2}, 
then we have, almost surely,
\begin{align}
\label{eq:negative_progress_upper_bounded_parameter_claim_2}
    \sup_{t \ge 1}{ \theta_t(a) } < \infty.
\end{align}
\end{lemma}
\begin{proof}
\textbf{First case:} if $ - \sum_{t=1}^{\infty} P_t(a) < \infty$, then according to \cref{lem:bounded_progress_bounded_parameter}, we have, almost surely,
\begin{align}
\label{eq:negative_progress_upper_bounded_parameter_proof_1}
    \sup_{t \ge 1}{ |\theta_t(a)| } < \infty,
\end{align}
which implies \cref{eq:negative_progress_upper_bounded_parameter_claim_2}.

\textbf{Second case:} if $- \sum_{t=1}^{\infty} P_t(a) = \infty$, then according to \cref{lem:negative_unbounded_progress_unbounded_parameter}, we have, almost surely,
\begin{align}
\label{eq:negative_progress_upper_bounded_parameter_proof_2}
    \theta_t(a) \to - \infty, \text{ as } t \to \infty,
\end{align}
which also implies \cref{eq:negative_progress_upper_bounded_parameter_claim_2}.
\end{proof}

\clearpage
\section{Rate of Convergence}


\textbf{\cref{thm:asymptotic_rate_of_convergence}.}
For a large enough $\tau > 0$, for all $T > \tau$, the average sub-optimality decreases at an $O\left(\frac{\ln(T)}{T} \right)$ rate. Formally, if $a^*$ is the optimal arm, then, for a constant $c$
\begin{align*}
\frac{\sum_{s=\tau}^{T} r(a^*) - \langle \pi_s, r \rangle}{T}  & \leq \frac{c \, \ln(T)}{T - \tau}
\end{align*}
\begin{proof}
The progress for the optimal action is,
\begin{align}
    P_t(a^*) &= \eta \cdot \pi_{\theta_t}(a^*) \cdot (r(a^*) -  \pi_{\theta_t}^\top r ) \\
    &\geq \eta \cdot \Delta \cdot \pi_{\theta_t}(a^*) \cdot \big( 1 - \pi_{\theta_t}(a^*) \big). \qquad \big( \Delta \coloneqq r(a^*) - \max_{a \neq a^*}{r(a)} \big) \\
    &\ge 0.
\end{align}
Since $\lim_{t \to \infty}{\pi_{\theta_t}(a^*)} = 1$, we have for all large enough $t \ge 1$,
\begin{align}
    \pi_{\theta_t}(a^*) \ge 1/2.
\end{align}
Since $N_\infty(a^*) = \infty$ (\cref{thm:general_action_global_convergence}
) and $|\gA_\infty| \ge 2$ (\cref{lem:at_least_two_actions_infinite_sample_time}), we have,
\begin{align}
    \sum_{t=1}^{\infty}{(1 - \pi_t(a^*))} \ge \sum_{t=1}^{\infty}{\pi_t(i_1)} = \infty,
\end{align}
where $N_\infty(i_1) = \infty$ and $r(i_1) < r(a^*)$. Therefore, we have,
\begin{align}
    \sum_{t=1}^{\infty}{P_t(a^*)} = \infty.
\end{align}
The variance of noise is,
\begin{align}
    V_t(a^*) &\coloneqq \frac{5}{18} \cdot \sum_{s=1}^{t-1}  \pi_{\theta_s}(a^*) \cdot (1-\pi_{\theta_s}(a^*)),
\end{align}
which will be dominated by sum of $P_t(a^*)$ since $V_t(a^*)$ appears under square root (\cref{lem:parameter_noise_concentration}). Therefore, for all large enough $t \ge \tau$,
\begin{align}
    \theta_t(a^*) \ge C \cdot \sum_{s=\tau}^{t}{(1 - \pi_s(a^*))}.
\end{align}
%\begin{align}
%    \liminf_{t \to \infty}{ \frac{\theta_t(a^*)}{\sum_{s=\tau}^{t}{(1 - \pi_s(a^*))}}} \ge C   
%\end{align}
On the other hand, we argued recursively (in the proofs for \cref{thm:general_action_global_convergence})  that for all sub-optimal action $a \in [K]$ with $r(a) < r(a^*)$,
\begin{align}
    \sup_{t \ge 1} \theta_t(a) < \infty.
\end{align}
Therefore, we have, for all large enough $t \ge \tau$,
\begin{align}
    \theta_t(a^*) - \theta_t(a) \ge C \cdot \sum_{s=\tau}^{t}{(1 - \pi_s(a^*))},
\end{align}
which implies that,
\begin{align}
    \sum_{a \neq a^*}{ \exp\{ \theta_t(a) - \theta_t(a^*) \}} \le (K - 1) \cdot \exp\bigg\{ - C \cdot \sum_{s=\tau}^{t}{(1 - \pi_s(a^*))} \bigg\}.
\end{align}
Therefore, we have,
\begin{align}
    1 - \pi_t(a^*) \le \frac{1 - \pi_t(a^*)}{\pi_t(a^*)} = \sum_{a \neq a^*}{ \frac{\pi_t(a)}{\pi_t(a^*)}} \le (K - 1) \cdot \exp\bigg\{ - C \cdot \sum_{s=\tau}^{t-1}{(1 - \pi_s(a^*))} \bigg\}.
\end{align} 
Using~\cref{lemma:partial-sum-combination} with $x_n = \sum_{s=\tau}^{t-1}{(1 - \pi_s(a^*))} > 0$, $x_{n+1} = \sum_{s=\tau}^{t}{(1 - \pi_s(a^*))} > 0$, $c = C > 0$ and $B = K - 1 \geq 1$ gives us that for all $t > \tau$, 
\begin{align}
\sum_{s=\tau}^{t}{(1 - \pi_s(a^*))} & \leq \frac1{C} \ln(C t + e^{C M} ) + \frac{\pi^2}{12C} \,,
\label{eq:rate-inter}
\end{align}
where $M = \max\{K - 1, \frac{1}{C} \ln ((K - 1) \, C)), (1 - \pi_\tau(a^*)) \} = K - 1$. 

Finally, we use~\cref{eq:rate-inter} to bound the average sub-optimality. For any $s \geq \tau$ and $T > \tau$,
\begin{align*}
r(a^*) - \langle \pi_s, r \rangle &= \sum_{a \neq a^*} \pi_{s} \, [r(a^*) - r(a)] \leq 2 \, R_{\max} \, (1 - \pi_s(a^*)) \tag{Since $r(a) \in [-R_{\max},R_{\max}]$}. \\
\intertext{Summing from $s = \tau$ to $T$,}
\implies \frac{\sum_{s=\tau}^{T} r(a^*) - \langle \pi_s, r \rangle}{T} & \leq \frac{2 \, R_{\max} \, \left[\frac1{C} \ln(C \, T + e^{C M} ) + \frac{\pi^2}{12C} \right]}{T - \tau}. \qedhere
\end{align*}
\end{proof}

\begin{lemma}
Let $\{y_n\}$ be the solution to the difference equation $y_{n+1} = y_n + B \, e^{-cy_n}$ with $B \geq 1$, $c > 0$ and $y_0 \ge \max(B,\frac{1}{c}\ln (B \, c))$. Let $\{x_n\}$ be a nonnegative valued sequence such that $x_0 \le y_0$ and $x_{n+1} \le x_n + B \, e^{-cx_n}$ for all $n \ge 0$. Then, $x_n \le y_n$ for all $n \ge 0$.
\label{lemma:difference-inequality}
\end{lemma}

\begin{proof}
Define the function $f(y) = \max \{ x + B \, e^{-cx} : 0\le x \le y \}$. Clearly, $f$ is an increasing function of its argument.

\noindent \underline{Claim:} For $y\ge \max(B, \frac{1}{c} \ln (B \, c))$, $f(y) = y + B \, e^{-cy}$.
To prove this claim, for $x\in \mathbb{R}$, define
\begin{align}\label{eq:gdef}
g(x) := x + B \, e^{-cx}\,.
\end{align}
Function $g$ is increasing on $(\frac{1}{c} \ln (B \, c), \infty)$ and decreasing on $(-\infty,\frac{1}{c}\ln (B \, c))$. 

If $c < \nicefrac{1}{B}$, $\frac{1}{c}\ln (Bc) < 0$, hence $g$ is increasing on $(0, \infty)$. Hence, $f(y)=g(y)$, proving the claim. 

If $c \ge \nicefrac{1}{B}$, then $\frac{1}{c}\ln (B \, c) \ge 0$. Hence, $g$ is decreasing on $(0, \nicefrac{1}{c}\ln (B \, c))$ and then increasing on $(\nicefrac{1}{c}\ln (B \, c) )$. Since $y\ge \frac{1}{c} \ln(B \, c)$, $f(y) = \max(g(0),g(y))$. Since we also have $y\ge B > \frac{1}{c}\ln (B \, c)$, $g(y) \ge g(B) > B = g(0)$ and thus $f(y)=g(y)$, finishing the proof of the claim.

The difference equation for $y_n$ is $y_{n+1} = y_n + B \, e^{-cy_n}$. Since $y_0 \ge \max(B,\frac{1}{c}\ln (B \,c))$ and $y_n$ is increasing, we have $y_n \ge \max(B,\frac{1}{c}\ln (B \,c))$ for all $n$. 
Therefore, we can apply the equality from the previous claim to get 
\[
y_{n+1} = y_n + B \, e^{-cy_n} = f(y_n)\,.
\]
% S: commenting since we are not using this
% This implies 
% \[
% y_n = f(f(\cdots f(y_0) \cdots )) \qquad \text{(apply $f$ $n$ times)}
% \]
The sequence $(x_n)$ satisfies the inequality $x_{n+1} \le x_n + B \, e^{-cx_n} \le f(x_n)$.

Now, let us prove $x_n \le y_n$ using induction. The base case is $x_0 \le y_0$ (given). Assume $x_k \le y_k$ for some $k \ge 0$. Since $f$ is increasing, $x_k \le y_k$ implies $f(x_k) \le f(y_k)$. Using the properties $x_{k+1} \le f(x_k)$ and $f(y_k) = y_{k+1}$, we get $x_{k+1} \le f(x_k) \le f(y_k) = y_{k+1}$. By the principle of mathematical induction, $x_n \le y_n$ for all $n \ge 0$.
\end{proof}

\begin{lemma}
Let $c > 0$ and $B \geq 1$, and let $\{y_n\}_{n=0}^\infty$ be a sequence defined by the recurrence relation
\[
y_{n+1} = y_n + B \, e^{-c y_n}
\]
s.t. $y_0 \ge \max(B,\frac{1}{c}\ln (B \, c))$. Then, 
\[
y_n \leq \frac{1}{c} \ln(c n + e^{c y_0} ) + \frac{\pi^2}{12c}.
\]
\label{lemma:partial-sum-bound}
\end{lemma}

\begin{proof}
Define the  function:
\begin{align*}
    y(t) & := \frac1{c} \ln(c t + e^{c y_0} )\,, \qquad t\ge 0\,,
\end{align*}
and note that it is an increasing function, Moreover, $\dot y(t) (= \frac{d}{dt} y(t)) = e^{-c y(t)}$ for any $t \ge 0$. Hence,
\begin{align*}
y(t) = y_0 + \int_0^t e^{-c y(s)} \, ds\,, \qquad t\ge 0
\end{align*}

Define the function:
\begin{align*}
g(x) := x + B \, e^{-cx}\,,
\end{align*}
and note that $g$ is increasing when $y\ge \frac{1}{c} \ln (B \,c)$. Moreover, $y(0) = y_0$ and $y_{n+1} = g(y_n)$. 

Since $y(0) = y_0 \ge \frac{1}{c} \ln (B \,c))$ and both $\{y_n\}$ and $y(t)$ are increasing, $y_n \geq \frac{1}{c} \ln (B \, c)$ for all $n$ and , $y(t) \geq \frac{1}{c} \ln (B \, c)$ for all $t \geq 0$.

We first prove that
\begin{align}
    y(n)\le y_n\,, \qquad n=0,1,\dots\,.
    \label{eq:ynlb}
\end{align}
We prove \cref{eq:ynlb} by induction. The claim holds for $n=0$ by construction. Now assume that $y(n)\le y_n$ holds 
for some $n\ge 0$. Let us show that that $y(n+1)\le y_{n+1}$ also holds. For this note that
\begin{align*}
    y(n+1) 
    &=      y(n) + \int_n^{n+1} e^{-c y(s)} ds \\
    &\le    y(n) + \int_n^{n+1} e^{-c y(n)} ds  \tag{$t\mapsto y(t)$ is increasing}\\
    &=      y(n) + e^{-c y(n)} \\
    & \leq  y(n) + B \, e^{-c y(n)} \tag{since $B \geq 1$} \\
    &=   g(y(n)) \tag{definition of $g$} \\
    &\le    g(y_n) \tag{induction hypothesis, $g$ is increasing for $y\ge \frac{1}{c} \ln (B \,c)$ and $y_n, y(n) \ge \frac{1}{c} \, \ln (B \, c)$} \\
    &=      y_{n+1}\,. \tag{By definition of $y_{n+1}$}
\end{align*}    
By the principle of mathematical induction, $y(n) \le y_{n}$ for all $n \ge 0$.

Define $\Delta_n := y_n - y(n)$. From our previous inequality, we know that $\Delta_n \ge 0$. We now show that $\{\Delta_n\}_n$ is bounded, from which the desired statement follows immediately.
To show that $\{\Delta_n\}_n$ is bounded we will show that $\Delta_{n+1}-\Delta_n$ is summable.
To show this, we start by obtaining an expression for $\Delta_{n+1}-\Delta_n$. 
Let $A = e^{c y_0}$. Let $n\ge 0$.
Direct calculation gives
\begin{align*}
    \Delta_{n+1}-\Delta_n 
    & =  e^{-c y_n} - \frac{1}{c} \ln\left( 1 + \frac{c}{cn+A} \right)\,.
\end{align*}
Using that for all $x>0$, $\ln(1+x)\ge \frac{x}{1+\frac{x}{2}}$, we get 
\begin{align*}
    \Delta_{n+1}-\Delta_n 
    & \le e^{-c y_n} - \frac{1}{cn + A + \frac{c}{2}} \\
    & \le e^{-c y(n)} - \frac{1}{cn + A + \frac{c}{2}} \tag{from \cref{eq:ynlb}} \\
    & \le \frac{1}{cn +A } - \frac{1}{cn + A + \frac{c}{2}} \tag{definition of $y(n)$} \\
    & = \frac{1}{2c} \, \frac{1}{(n + \nicefrac{A}{c}) \, (n + \nicefrac{A}{c} + \frac{1}{2})}  \\
    & \le \frac{1}{2c} \frac{1}{(n+1)^2}\,. \qquad \tag{Since $y_0 \geq \frac{1}{c} \ln (B \,c))$, $A/c = B \ge 1$}
\end{align*}
For a fixed $m > 0$, summing up the above inequality from $n = 0$ to $m-1$,
\begin{align*}
\Delta_{m}  - \Delta_{0} & \leq \frac{1}{2c} \sum_{n = 0}^{m-1} \frac{1}{(n+1)^2} \leq \frac{\pi^2}{12c} \tag{Since $\sum_{i = 1}^{\infty} \frac{1}{i^2} = \frac{\pi^2}{6}$} \\
\implies \Delta_{m} & \leq \frac{\pi^2}{12c}.  \tag{Since $\Delta_0 = 0$}
\end{align*}
where $\pi = 3.14159\dots$. Hence, it follows that for any $n\ge 0$,
\begin{equation*}
y_n = y(n) + \Delta_n \leq y(n) + \frac{\pi^2}{12c} = \frac1{c} \ln(c n + e^{c y_0} ) + \frac{\pi^2}{12c}. \qedhere
\end{equation*}
\end{proof}

\begin{lemma}
Let $\{x_n\}$ be a nonnegative valued sequence such that $x_{n+1} \le x_n + B \, e^{-cx_n}$ for all $n \ge 0$ with $B \geq 1$, $c > 0$. Then, for all $n \ge 0$, 
\begin{align*}
x_n \le  \frac1{c} \ln(c n + e^{c M} ) + \frac{\pi^2}{12c} \,,
\end{align*}
where $M = \max\{B, \frac{1}{c} \ln (B \,c)), x_{0} \}$. 
\label{lemma:partial-sum-combination}    
\end{lemma}
\begin{proof}
Let $\{y_n\}$ be the solution to the difference equation $y_{n+1} = y_n + B \, e^{-cy_n}$ where $y_0 = \max\{B, \frac{1}{c} \ln (B \,c)), x_{0} \}$. Since $y_0 \geq x_0$ and $y_0 \ge \max(B,\frac{1}{c}\ln (B \, c))$, we can use~\cref{lemma:difference-inequality} to conclude that for all $n \ge 0$, 
\[
x_n \le y_n \,.
\] 

Furthermore, using~\cref{lemma:partial-sum-bound} we can conclude that, 
\begin{align*}
y_n \leq \frac{1}{c} \ln(c n + e^{c y_0} ) + \frac{\pi^2}{12c}.
\end{align*} 
Combining the above inequalities completes the proof. 
\end{proof}

% This section shows that \cref{alg:gradient_bandit_algorithm_sampled_reward} achieves a $O(1/t)$ asymptotic convergence rate, which is optimal and thus not improvable in terms of $t$ \citep{lai1985asymptotically}.
% \begin{theorem}
% \label{thm:asymptotic_rate_two_action_case}
% Let $K = 2$ and $r(1) > r(2)$. We have,
% \begin{align}
%     \EE{ r(a^*) - \pi_{\theta_t}^\top r } \in O(1/t).
% \end{align}
% \end{theorem}


% \textbf{\cref{thm:asymptotic_rate_two_action_case}.}
% Let $K = 2$ and $r(1) > r(2)$. We have,
% \begin{align}
%     \EE{ r(a^*) - \pi_{\theta_t}^\top r } \in O(1/t).
% \end{align}
% \begin{proof}
% Define the following quantity, for all $t \ge 1$,
% \begin{align}
%     q_t &\coloneqq \frac{1 - \pi_{\theta_t}(a^*) }{\pi_{\theta_t}(a^*)} \\
%     &= \exp\{ \theta_t(2) - \theta_t(a^*) \}.
% \end{align}
% We have, for all $t \ge 1$,
% \begin{align}
%     \EEt{ \log{q_{t+1}} } &= \EEt{ \theta_{t+1}(2) - \theta_{t+1}(a^*) } \\
%     &= \theta_t(2) - \theta_t(a^*) + \eta \cdot \big[ \pi_{\theta_t}(2) \cdot (r(2) - \pi_{\theta_t}^\top r) -  \pi_{\theta_t}(a^*) \cdot (r(a^*) - \pi_{\theta_t}^\top r)\big] \\
%     &= \log{q_t} - 2 \ \eta \cdot \Delta \cdot \pi_{\theta_t}(a^*) \cdot (1 - \pi_{\theta_t}(a^*)) \\
%     &= \log{q_t} - 2 \ \eta \cdot \Delta \cdot q_t \cdot \pi_{\theta_t}(a^*)^2.
% \end{align}
% Taking expectation on both sides in the above equation, we have,
% \begin{align}
%     \EE{ \log{q_{t+1}} } &= \EE{ \log{q_t} } - 2 \ \eta \cdot \Delta \cdot \EE{ q_t \cdot \pi_{\theta_t}(a^*)^2 } \\
%     &\le \EE{ \log{q_t} } - 2 \ \eta \cdot \Delta \cdot (1 - \epsilon)^2 \cdot \EE{ q_t },
% \end{align}
% where the last inequality is because of  $\pi_{\theta_t}(a^*) \to 1$ as $t \to \infty$ almost surely, which implies that $\pi_{\theta_t}(a^*) \geq 1 - \epsilon$ for all $t \ge t_0$. Summing up from $t = t_0$ to $T$ and by telescoping, we have,
% \begin{align}
%     2 \ \eta \cdot \Delta \cdot (1 - \epsilon)^2 \cdot \sum_{t=t_0}^{T}{ \EE{ q_t } } &\leq \EE{ \log{q_{t_0}} } - \EE{ \log{q_{T+1}} } \\
%     &\leq - \EE{ \log{q_{T+1}} },
% \end{align}
% where the last inequality is because of $q_t < 1$ for all $t \ge t_0$. If $- \EE{ \log{q_{T+1}} } \leq \log{T}$, then we have,
% \begin{align}
%     \frac{1}{T - t_0} \cdot \sum_{t=t_0}^{T}{ \EE{ q_t } } \le \frac{1}{2 \ \eta \cdot \Delta \cdot (1 - \epsilon)^2} \cdot \frac{\log{T}}{T - t_0},
% \end{align}
% which is an average rate of $O(\log{T} / (T - t_0))$. Otherwise, if $- \EE{ \log{q_{T+1}} } \geq \log{T}$, we have,
% \begin{align}
%     - \log{( \EE{ q_{T+1} } )} \leq - \EE{ \log{q_{T+1}} } \geq \log{T}
% \end{align}

% \textcolor{red}{which implies that,}
% \begin{align}
%     \EE{q_t} \le \frac{C}{t}.
% \end{align}
% Therefore, we have,
% \begin{align}
%     \EE{ r(a^*) - \pi_{\theta_t}^\top r } = \Delta \cdot \EE{ 1 - \pi_{\theta_t}(a^*) } \le \frac{C}{t}.
% \end{align}

% Define
% \begin{align}
%     p_t = \theta_t(a^*) - \theta_t(2).
% \end{align}
% We have,
% \begin{align}
%     \EEt{ p_{t+1} } &=  p_t + 2 \ \Delta \ \eta \cdot \pi_{\theta_t}(a^*) \cdot (1 - \pi_{\theta_t}(a^*) ) \\
%     &= p_t + 2 \ \Delta \ \eta \cdot \pi_{\theta_t}(a^*)^2 \cdot \exp\{  - p_t \} \\
%     &\ge p_t + 2 \ \Delta \ \eta \cdot c^2 \cdot \exp\{  - p_t \}.
% \end{align}
% Taking expectation, we have,
% \begin{align}
%     \EE{ p_{t+1} } &\ge \EE{ p_t } + 2 \ \Delta \ \eta \cdot c^2 \cdot \EE{ \exp\{  - p_t \} } \\
%     &\ge \EE{ p_t } + 2 \ \Delta \ \eta \cdot c^2 \cdot \exp\{  - \EE{  p_t } \} 
% \end{align}
% \textcolor{red}{If we have,}
% \begin{align}
%     \exp\{ \EE{ p_t } \} \ge C \cdot t,
% \end{align}
% then,
% \begin{align}
%      \exp\{ - \EE{ p_t } \} \in O(1/t).
% \end{align}
% What we need:
% \begin{align}
%     \frac{1 - \EE{\pi_{\theta_t}(a^*)}}{\EE{\pi_{\theta_t}(a^*)}}
% \end{align}
% or
% \begin{align}
%     1 - \EE{\pi_{\theta_t}(a^*)}.
% \end{align}
% \end{proof}


% \section{Two action no reward noise case}

% First we note that for two actions, we have 

% \begin{align}
%     P_t(a_1) =& \eta \pi_t(a_1)(r(a_1) - \pi_t(a_1) r(a_1) - (1-\pi_t(a_1)) r(a_2))\\ =& \eta \pi_t(a_1)(r(a_1) - r(a_2))(1-\pi_t(a_1))\\ 
%     =& \eta \Delta \pi_t(a_1)(1-\pi_t(a_1)).
% \end{align}

% For the noise term, we have 

% \begin{align}
%     W_t(a_1)/\eta =& (I_t(a_1) - \pi_t(a_1))r(a_t) - \pi_t(a_1)\Delta (1-\pi_t(a_1))\\
%     =& I_t(a_1)(1-\pi_t(a_1)) r(a_1) + (1-I_t(a_1))(-\pi_t(a_1))r(a_2) - \pi_t(a_1)\Delta (1-\pi_t(a_1))\\
%     =& I_t(a_1)(1-\pi_t(a_1))r(a_1) + I_t(a_1)(\pi_t(a_1) - 1)r(a_2) + (I_t(a_1) - \pi_t(a_1))r(a_2) - \pi_t(a_1)\Delta (1-\pi_t(a_1))\\
%     =& I_t(a_1)(1-\pi_t(a_1))\Delta+ (I_t(a_1) - \pi_t(a_1))r(a_2) - \pi_t(a_1)\Delta (1-\pi_t(a_1))\\
%     =& (I_t(a_1) - \pi_t(a_1))(1-\pi_t(a_1))\Delta+ (I_t(a_1) - \pi_t(a_1))r(a_2)
% \end{align}

% \begin{align}
%     \theta_{t+1}(a^*) - \theta_{t+1}(2) &= \theta_{t}(a^*) - \theta_{t}(2) + 2 \cdot \eta \cdot \pi_{\theta_t}(a^*) \cdot (1 - \pi_{\theta_t}(a^*)) \cdot \Delta
% \end{align}
% or
% \begin{align}
%     \log{\bigg( \frac{\pi_{\theta_{t+1}}(a^*)}{1 - \pi_{\theta_{t+1}}(a^*)} \bigg)} &= \log{\bigg( \frac{\pi_{\theta_{t}}(a^*)}{1 - \pi_{\theta_{t}}(a^*)} \bigg)} + + 2 \cdot \eta \cdot \pi_{\theta_t}(a^*) \cdot (1 - \pi_{\theta_t}(a^*)) \cdot \Delta
% \end{align}



\clearpage
\section{Additional simulation results}
\label{app:sim}

\begin{figure}[h]
\centering
\begin{subfigure}[b]{.328\linewidth}
\includegraphics[width=\linewidth]{figs/large_learning_rate_two_action_prob_astar_eta_100.00.pdf}
\caption{$\pi_{\theta_t}(a^*)$, $\, \eta = 100$.}\label{fig:optimal_action_prob_two_action_case_eta_100}
\end{subfigure}
\begin{subfigure}[b]{.328\linewidth}
\includegraphics[width=\linewidth]{figs/large_learning_rate_two_action_subopt_eta_100.00.pdf}
\caption{$r(a^*) - \pi_{\theta_t}^\top r$, $\, \eta = 100$.}\label{fig:sub_optimality_gap_two_action_case_eta_100}
\end{subfigure}
\begin{subfigure}[b]{.328\linewidth}
\includegraphics[width=\linewidth]{figs/large_learning_rate_two_action_log_subopt_eta_10.00.pdf}
\caption{$\log{ ( r(a^*) - \pi_{\theta_t}^\top r ) }$, $\, \eta = 10$.}\label{fig:sub_optimality_gap_two_action_case_eta_10}
\end{subfigure}
\caption{Visualization in a two-action stochastic bandit problem. Here the rewards are defined as $(-0.05, -0.25)$. Other details are same as for \cref{fig:visualization_general_action_case}. Figures~\ref{fig:optimal_action_prob_two_action_case_eta_100} and~\ref{fig:optimal_action_prob_two_action_case_eta_100} are based on a single run, while Figure~\ref{fig:sub_optimality_gap_two_action_case_eta_10} averages across 10 runs. 
Note that $\log{ ( r(a^*) - \pi_{\theta_t}^\top r ) } 
\approx 10^{-33}$ at the final stages on \cref{fig:sub_optimality_gap_two_action_case_eta_100}.} 
\label{fig:visualization_two_action_case}
\vspace{-10pt}
\end{figure}
 %Sharan: I edited and put the new cleaned up proof of Thm 2 in this file. The old proof is still in appendix.tex. 


%%%%%%%%%%%%%%%%%%%%%%%%%%%%%%%%%%%%%%%%%%%%%%%%%%%%%%%%%%%%

\end{document}