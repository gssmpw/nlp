\section{Related Works}
\label{sec2}
The field of brain tumor classification and segmentation has witnessed significant advancements, with various DL approaches being developed to enhance accuracy and efficiency. Balamurugan and Gnanamanoharan \cite{balamurugan2023brain} introduced a hybrid deep CNN with LuNetClassifier, demonstrating an effective method for brain tumor segmentation and classification. This model focuses on extracting robust features, which contribute to improved diagnostic accuracy. Similarly, Ali et al. \cite{ali2024segmentation} proposed the PG-OneShot learning CNN model, which utilizes a one-shot learning approach to segment and identify brain tumors in MRI images, addressing the challenge of limited labeled data. Their model's ability to perform well with minimal training data represents a significant step forward in medical image analysis.\\
Further contributions to this domain include the work by Ramakrishnan et al. \cite{ramakrishnan2024optimizing}, who developed a hybrid CNN architecture optimized with oneAPI to balance accuracy and computational efficiency. Their approach highlights the importance of optimizing neural networks to meet the demands of real-time medical diagnostics. Kumar and Kumar \cite{kumar2023human} also explored the use of CNNs for brain tumor classification and segmentation, focusing on enhancing model performance through innovative network designs. Additionally, Alturki et al. \cite{alturki2023combining} combined CNN features with voting classifiers to optimize brain tumor classification performance, showcasing the potential of ensemble methods in improving model robustness. Sharma and Vardhan \cite{sharma2024mtjnet} introduced MTJNet, a multi-task joint learning network, to advance classification tasks, though their focus was on medicinal plant and leaf classification, their techniques are highly relevant to brain tumor classification, illustrating the versatility and potential cross-application of DL models in various domains. \\
Our research compares the performance of our model with several recent works in the field of brain tumor classification, highlighting the advancements and challenges addressed by each. Khushi et al. \cite{khushi2024performance} conducted a performance analysis of state-of-the-art CNN architectures, providing a benchmark for our model’s superior accuracy and efficiency. Bose and Garg \cite{bose2024optimized} employed an optimized CNN using Manta-Ray Foraging Optimization, which demonstrated significant accuracy improvements but did not surpass the robustness of our approach. Similarly, Jacob et al. \cite{jacob2023brain} proposed a deep CNN combined with a modified butterfly optimization algorithm, which, while effective, showed limitations in scalability that our model overcomes. Kordemir et al. \cite{kordemir2024mask} utilized a Mask R-CNN for brain tumor detection, achieving notable precision but at the cost of higher computational complexity, which our model mitigates through efficient architecture design. Venkatachalam et al. \cite{venkatachalam2024ensemble} introduced an ensemble of 3D CNN and U-Net models, focusing on segmentation and classification, yet our model’s streamlined design offers a more balanced solution in terms of both accuracy and resource consumption. Lastly, Shajin et al. \cite{shajin2023efficient} developed a hierarchical DL framework, which, despite its layered approach, did not achieve the overall performance metrics that our model consistently delivers. These comparisons underscore the strengths of our model in delivering high accuracy, reduced loss, and computational efficiency across diverse tumor types.