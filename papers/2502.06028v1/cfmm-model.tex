\section{PDLPs and Constant Function Market Makers}\label{app:CFMM}
We consider a model where the implied swap price $g(\delta)$ provided by a PDLP is represented by a CFMM (see~\S\ref{subsec:arb-single}).
One way that this can occur is if the oracle $p$ is replaced by a constant function market maker (CFMM).
In particular, when a user tenders a portfolio $\delta$, there is a trade made against a CFMM and the post trade price is used as an oracle price for the PDLP.

\paragraph{Constant Function Market Makers.}
One can often view pieces of a PDLP a constant function market maker~\cite{angeris2020improved, angeris2023replicating}.
This is because any system aiming to have target portfolio weights can be viewed as having a concave payoff function, which can be replicated by a CFMM~\cite{angeris2023replicating}.
By representing the pool as a CFMM, we will be able to construct precise formulas for arbitrage profits for borrowers, lenders, and LPs.
We will model the oracle price of the perpetuals exchange to be equal (in no-arbitrage) to the mark price implied by the PDLP viewed as a CFMM.

We provide a brief introduction to CFMMs, referring interested readers to the book chapter~\cite{angeris2023replicating} for more details.
A CFMM is defined by a \emph{trading function} $\varphi : \reals^n_+ \rightarrow \reals$, where $\varphi(R)$ maps reserves to an invariant value.
Given a set of reserves $R \in \reals^n_+$, a trading function implies a price between any pair of assets.
If $\varphi$ is smooth, this price can be represented as~\cite[\S2.5]{angeris2023replicating}
\[
p_{ij}(R) = \frac{\nabla_i \varphi(R)}{\nabla_j \varphi(R)}
\]
Thus, a CFMM is defined as a set of reserves $R$ and a trading function $\varphi$, which implies all asset pairs can be prices.

$\varphi$ can be assumed, without the loss of generality, to be concave, non-decreasing, and homogeneous (see~\cite{angeris2023geometry, angeris2023replicating}).
If not true, then there exists arbitrage, as demonstrated in~\cite{angeris2020improved, angeris2023geometry}.
A point $\Delta \in \reals^n$ is said to be a \emph{valid trade} in a CFMM if $\varphi(R + \Delta) = \varphi(R)$.
We can view the interactions that arbitrageurs have with the PDLP (in order to return the PDLP towards the target weighting) as a sequence of valid trades.
If there is an external oracle price $p$, then in the presence of arbitrageurs, the difference between a CFMM price and an oracle price will be bounded and decrease as $\Vert R \Vert_1$ increases~\cite{angeris2020improved}.
We can therefore define the mark price at time $t$ between a pair of assets $i$ and $j$ to be $p^{\dagger}_t(i,j) = p_{ij}(R_t)$.

 
\paragraph{Portfolio Value.}
For a CFMM $\varphi(R)$, one can define the \emph{portfolio value} function, $V(p, R)$ given a set of oracle prices $p \in \reals^n_+$.
This is the value of CFMM assets given an oracle price $p$, similar to what we defined for PDLPs.
Given the reserves, the portfolio value of a CFMM is defined as $V(p, R) = p \cdot R$.
One can also write the portfolio value directly in terms of $\varphi$ thanks to convex duality~\cite{angeris2023geometry}.
This representation is useful for analytic calculations.
For this section, we will only consider the portfolio value for the Uniswap CFMM, $\varphi(R) = \left(\prod_{i=1}^n R_i\right)^{1/n}$.
For two-assets, it is known that~\cite{angeris2020improved}
\begin{equation}\label{eq:uni-pv}
V(R, p) = 2 \sqrt{k p} = 2\sqrt{R_1 R_2 p}
\end{equation}

\paragraph{Price Impact and Exchange Functions.}
After a user submits a valid trade $\Delta$ to an initial reserve $R$, the the prices of assets change from $p_{ij}(R)$ to $p_{ij}(R+\Delta)$.
The \emph{price impact function} for a trade $\Delta$, $g(\delta)$ is defined as $g(\Delta) = p_{ij}(R+\Delta)-p_{ij}(R)$.
When $\varphi$ is smooth, there can often be simple expressions for $g(\Delta)$.
For instance, when $\varphi(R_1, R_2) = \sqrt{R_1 R_2}$ (\ie~Uniswap), we have
\begin{equation}\label{eq:uniswap}
g(\Delta) = \frac{R_1 R_2}{(R_2-\Delta)^2}
\end{equation}
The forward and backward exchange functions, $F(\Delta), G(\Delta)$, represent the quantity of assets tendered in exchange for an input trade of size $\Delta$.
One can view $G(\Delta), g(\Delta)$ as the functions defined for PDLPs in~\S\ref{subsec:arb-single}.

% We briefly summarize the main definitions and results of~\cite{angeris2022does} here.
% Suppose that the trading function $\psi$ is differentiable (as most trading functions in practice are), then
% the \emph{forward exchange rate} for a trade of size $\Delta$ is
% $
% g(\Delta) = \frac{\partial_3 \psi(R, R', \Delta, \Delta')}{\partial_4 \psi(R, R', \Delta, \Delta')}.
% $
% Here $\partial_i$ denotes the partial derivative with respect to the $i$th argument,
% and $\Delta'$ is specified by the implicit condition $\psi(R, R', \Delta, \Delta') = \psi(R, R', 0, 0)$; \ie, the trade
% $(\Delta, \Delta')$ is assumed to be valid. Additionally, the reserves $R, R'$ are assumed to be fixed.
% Matching the notation of Section~\ref{sec:uniswap}, the function $g$ represents the marginal forward exchange rate of a positive-sized trade.
% We say that a CFMM is \emph{$\alpha$-stable} if it satisfies 
% \begin{align*}
%     g(0) - g(-\Delta) \leq \alpha \Delta
% \end{align*}
% for all $\Delta \in [0, M]$ for some positive $M$. This condition provides a linear upper bound on the maximum price impact that a trade bounded by $M$ can have.
% Similarly, we say that a CFMM is $\beta$-liquid if it satisfies
% \begin{align*}
%     g(0) - g(-\Delta) \geq \beta \Delta
% \end{align*}
% for all $\Delta \in [0, K]$ for some positive $K$. One important property of $g$ is that it can be used to compute $\Delta'$~\cite[\S2.1]{angerisCurvature}:
% \begin{equation}\label{eq:curvOut}
%     \Delta' = \int_{0}^{-\Delta} g(t) dt.
% \end{equation}
% Simple methods for computing $\alpha$ and $\beta$ in common CFMMs are presented
% in~\cite[\S1.1]{angerisCurvature} and~\cite[\S4]{angeris2021cfmm}. We define $\de' = G(\Delta)$ to be the \emph{forward exchange function}, which is the amount of output token received for an input of size $\Delta$. Whenever we reference the function $G(\Delta)$ for a given CFMM, we always make clear the reserves associated with that CFMM. We note that $G(\Delta)$ was shown to be concave and increasing in \cite{angeris2021cfmm}. 


\paragraph{Fees and Costs.}
Perpetual exchanges charge a fixed percentage fee of the notional position size, which is a fixed charge beyond the funding rates.
PDLPs charge a fee proportional to the borrowed quantity.  
Given that PDLP borrowers can only utilize their loans on a particular perpetuals exchange, borrowers must pay both of these fees.
We will represent the fees for the perpetuals exchange, PDLP, and CFMM as $f^{p}, f^{PDLP}, f^{CFMM} \in (0, 1)$, respectively.


\subsection{Two-asset, single period demand loan pool.} 
We first consider a two-asset PDLP, modeled via the following steps:
\begin{enumerate}
    \item The oracle price moves from $p_0$ to $p_1$
    \item Funding rate arbitrageur borrows from the pool, opening a long or short position to capture the funding rate until mark and oracle prices converge
    \item Borrow changes the price impact function implying that optimal arbitrage~\eqref{eq:arb-qty} changes
    \item CFMM arbitrageur trades against the pool updating the mark price to $p_1$
\end{enumerate}
We will first show how to compute bounds on how large $f^{PDLP}$ needs to be relative to the price change in order for no-arbitrage to hold.
This will allow us to see how to utilize the different components introduced in the last section to model the performance of a PDLP.
We initially assume that all prices are equilibrated and that at time $0$ there is no-arbitrage between the external market at price $p_0$ and the PDLP and perpetuals markets at price $p^{\dagger}_0$.
In particular, this implies that the funding rate $\gamma_0 = 0$ by equation~

% \paragraph{Arbitraging the PDLP pool.}
% We assume there is a single perpetuals arbitrageur who is aiming to open a long position of size $\ell$ and a short position of size $\psi$ such they earn the funding rate in the period before the PDLP is arbitraged by a CFMM arbitrageur.
% We first suppose that $p_1 > p_0$ and note that an entirely symmetric derivation can be utilized for the case $p_0 < p_1$.
% When this is true, note that we have
% \[
% \gamma(L_0, S_0, p_1, p_0) = \kappa\left(\frac{L_0}{S_0} - \frac{p_1}{p_0}\right) = \kappa\left(1 - \frac{p_1}{p_0}\right) < 0
% \]
% where the equality $\frac{L_0}{S_0} = 1$ is implied by $\gamma_0 = 0$.
% This implies that the shorts pay the long and that a long position would be profitable and realize funding payment $\gamma_1$.
% We want to compute the position that an arbitrageur trying to earn the funding rate prior to the mark price converging to the oracle price by borrowing from the PDLP.

% What is the maximal size of a position that can be constructed?
% Given that the funding rate is negative, short positions are penalized negatively, so that the profit from the position $(\ell, \psi)$ is $(1+\gamma)\ell + (1-\gamma) \psi$.
% As such, the maximal profit position is necessarily long only position and ensures that funding rate goes from negative to zero is the maximal profitable position.
% We can compute the size of this position by finding $\ell$ such that $\gamma(L_0+\ell, S_0, p_1, p_0) = 0$, which occurs when
% \[
% \ell = \max\left(\frac{S_0 p_1}{p_0}-L_0, 0\right)
% \]
% Hence, the profit realized by the arbitrageur is $(1-f^{PDLP})\ell$ which is positive for all fees $f^{PDLP} < 1$ if $\ell > 0$.

% This analysis implies that provided that there is any arbitrage between the oracle price and mark price, there will always be positive fees for the borrowing pool.\footnote{A more careful analysis would argue that this is only true if the arbitrageur didn't have the capital to do the arbitrage themselves and/or if the fee $f^{FDLP} < f^{OC}$ where $f^{OC}$ is the opportunity cost rate (\eg~staking rate) for using their own capital. We should likely spell this out in the full paper}

\paragraph{Arbitraging the PDLP's CFMM.}
We assume that there exists a Uniswap-style CFMM, $\varphi(R_1, R_2) = \sqrt{R_1, R_2}$ for the PDLP. 
For this CFMM, the price impact function $g(\Delta)$ has the form~\eqref{eq:uniswap}.
Suppose we consider a one-period update where the oracle price changes from $p_0$ to $p_1$.
Using this function, we can compute the quantity traded to ensure no-arbitrage, $\Delta^{arb}$, \ie~the amount of asset to be traded to ensure that the CFMM price of the PDLP is $p_1$.
The arbitrage condition is defined by the equation, $p_1 = g(\Delta^{arb}, R_1, R_2)$, which, in words, states that the amount traded in arbitrage ensures that the price implied by the PDLP's CFMM is $p_1$.
Solving this equation for $\Delta^{arb}$, we find that 
\begin{equation}\label{eq:arb-qty}
\Delta^{arb} = R_2 - \sqrt{\frac{R_1 R_2}{p_1}}
\end{equation}

When the perpetual arbitrageur borrows a position of size $\ell$ to go long in the perpetuals exchange, the CFMM reserves are now $(\tilde{R}_1, \tilde{R}_2) = (R_1 - \ell, R_2)$.
If we similarly define $\tilde{\Delta}^{arb}$ via the no-arbitrage equation $p_1 = g(\Delta^{arb}, \tilde{R}_1, \tilde{R}_2)$, we have \TD{remove}
\begin{align*}
    \tilde{\Delta}^{arb} &= \tilde{R}_2 - \sqrt{\frac{\tilde{R}_1 \tilde{R}_2}{p_1}} = R_2 - \sqrt{\frac{(R_1 - \ell)R_2}{p_1}} < \Delta^{arb}
\end{align*}
where $\ell$ is the quantity implied by funding rate arbitrage (see~\S\ref{subsec:arb-single}).
This implies that the output from the pool, in terms of arbitrage profit for the CFMM arbitrageur is smaller when the pool lends out to the perpetuals exchange.
In particular, the net profit for the PDLP is
\begin{align}\label{eq:profit-lvr}
\mathsf{Profit} &= \overbrace{f^{PDLP} \ell}^{\text{Profit from Borrow}} - \overbrace{(V(p_1, R_1, R_2) - V(p_1, \tilde{R}_1 + F(\tilde{\Delta}^{arb}), \tilde{R}_2 - \tilde{\Delta}^{arb}))}^{\mathsf{LVR}(p_1, R, \tilde{R})}
\end{align}
where $F(\Delta)$ is the forward exchange function representing the amount of asset 1 added in order to faciliate an arbitrage trade of size $\Delta$ in asset 2.
The first term represents the profit from lending out $\ell$ to the perpetuals arbitrage trader, whereas the second term represent the loss-versus-rebalancing (LVR)~\cite{milionis2022automated} common to CFMMs.
This arbitrage loss exists as the PDLP effectively pays a cost to ensure that the target weights and true weights are kept close.

\paragraph{Choosing Fees.}
A natural question that an operator of a pool might ask is how to set the fees $f^{PDLP}$ and $f^{CFMM}$, which are at the control of the PDLP creator.
We showed that profitability for a PDLP can be viewed as the difference between the profit from borrowing the asset and the so-called loss-versus-rebalancing~\cite{milionis2022automated} for CFMMs.
This net profit then provides a number of conditions for how high the borrow fee $f^{PDLP}$ needs to be in order for the PDLP to be profitable in terms of the prices and CFMM parameters.
We find that provided there is a sufficiently large borrow size relative to the arbitrage loss, the PDLP can always be profitable at sufficiently large fees.
% TD: constant linear, bound the tick size, otherwise you blow up

According to~\eqref{eq:profit-lvr}, we see that the PDLP is profitable provided that the fee charged satisfies
\begin{align}
f^{PDLP} &\geq \frac{V(p_1, R_1, R_2) - V(p_1, \tilde{R}_1 + F(\tilde{\Delta}^{arb}), \tilde{R}_2 - \tilde{\Delta}^{arb})}{\ell} \nonumber \\
& = \frac{\sqrt{2p_1}}{\ell}\left(\sqrt{R_1 R_2} - \sqrt{(\tilde{R}_1 + F(\tilde{\Delta}^{arb}))(\tilde{R}_2 - \tilde{\Delta}^{arb}))}\right) \nonumber \\
&= \frac{\sqrt{2p_1}}{\ell}\left(\sqrt{R_1 R_2} - \sqrt{(R_1 - \ell + F(\tilde{\Delta}^{arb}))(R_2 - \tilde{\Delta}^{arb}))}\right) \label{eq:pdlp-necessary-and-sufficient}
\end{align}
We get a useful sufficient for the fee when $F(\tilde{\Delta}^{arb}) \leq \ell$ (\ie~the borrowed quantity is higher than the arbitrage loss).
This is because this allows us to upper bound the right-hand side of~\eqref{eq:pdlp-necessary-and-sufficient} which then implies a sufficient condition on $f^{PDLP}$ for profitability.
We first claim the following upper bound on $\mathsf{LVR}$
\begin{claim}\label{claim:ab-lb}
Let $p = \frac{\ell-F(\tilde{\Delta}^{arb})}{R_1}$ and $q = \frac{\tilde{\Delta}^{arb}}{R_2}$.
If for some $\epsilon > 0$, $p + q - pq \in (0, 1)$, then
\[
\frac{\mathsf{LVR}(p_1, R, \tilde{R})}{\sqrt{2p_1 R_1 R_2}} \leq 
 \sqrt{\frac{\ell-F(\tilde{\Delta}^{arb})}{R_1} + \frac{\tilde{\Delta}^{arb}}{R_2}} = \sqrt{p + q}
\]
\end{claim}
\noindent We prove this claim in Appendix~\ref{app:ineq-proof}.
This allows us to turn the necessary and sufficient condition~\eqref{eq:pdlp-necessary-and-sufficient} into an easier to handle sufficient condition.
In particular, the claim implies that if the hypotheses of the claim are satisfied, then a fee satisfying
\[
f^{PDLP} \geq \frac{\sqrt{2p_1 R_1 R_2}}{\ell} \sqrt{\frac{\ell-F(\tilde{\Delta}^{arb})}{R_1} + \frac{\tilde{\Delta}^{arb}}{R_2}} 
= \Omega\left(\frac{1}{\sqrt{\ell}}\right)
\]
suffices to ensure the PDLP is profitable.
If, in addition, we have $\frac{\ell-F(\tilde{\Delta}^{arb})}{R_1} > \tilde{\Delta}^{arb}{R_2}$, then we have a simpler condition
\[
f^{PDLP} \geq 2 \sqrt{\frac{p_1 R_2}{\ell R_1}}\left(1 - \frac{F(\tilde{\Delta}^{arb})}{\ell}\right) = 2 \sqrt{\frac{p_1}{p_0 \ell}}\left(1 - \frac{F(\tilde{\Delta}^{arb})}{\ell}\right)
\]
If $p_0, p_1$ were drawn using a Geometric Brownian Motion with mean $\mu$ and variance $\sigma^2$, then the results of~\cite[Appendix C]{angeris2021analysis} show that expected version of this condition is
\[
f^{PDLP} \geq \frac{2 e^{\frac{\mu}{2}-\frac{\sigma^2}{8}}}{\sqrt{\ell}}
\]


\subsection{Proof of Claim~\ref{claim:ab-lb}}\label{app:ineq-proof}
From eq.~\eqref{eq:pdlp-necessary-and-sufficient}, we know that
\begin{align*}
\frac{\mathsf{LVR}(p_1, R, \tilde{R})}{\sqrt{2 p_1 R_1 R_2}} &= 1 - \sqrt{1 - \frac{\ell - F(\tilde{\Delta}^{arb})}{R_1} - \frac{\tilde{\Delta}^{arb}}{R_2} + \frac{(\ell - F(\tilde{\Delta}^{arb}))\tilde{\Delta}^{arb}}{R_1 R_2}} \\ 
& = 1 - \sqrt{1 - p - q + pq} \\
& = 1 - \sqrt{1 - x}
\end{align*}
where we define $x = p + q - pq$.
We now claim that $1-\sqrt{1-x} \leq \sqrt{x}$ holds for any $x \in (0, 1)$.
Rearranging this inequality and squaring gives $1 - 2c\sqrt{x} + c^2 x < 1 - x$ or $(c^2+1)x < 2 c \sqrt{x}$.
This is equal to the condition $\sqrt{x} \leq \frac{2c}{c^2 +1}$.
Since $x \in (0,1)$, as long as $\frac{2c}{c^2 + 1} \geq 1$, we are done.
Note that this inequality holds, in particular at $c = 1$,
This inequality implies that 
\begin{align*}
\frac{\mathsf{LVR}(p_1, R, \tilde{R})}{\sqrt{2 p_1 R_1 R_2}} &\leq \sqrt{x} = \sqrt{p+q - pq} < \sqrt{p+q}
\end{align*}
as claimed.

% To show this note the following
% \begin{align*}
%     \sqrt{ab} - \sqrt{(a-\delta)(b-\gamma)} &= \sqrt{ab}\left(1 - \sqrt{1 - \frac{\delta}{a} - \frac{\gamma}{b} + \frac{\delta \gamma}{ab}}\right)
% \end{align*}
% Let $x = \frac{\delta}{a} + \frac{\gamma}{b} - \frac{\delta\gamma}{ab}$ and $y = 1-x$ so that we have $\sqrt{ab} - \sqrt{(a-\delta)(b-\gamma)} = \sqrt{ab}(1-\sqrt{y})$.
% Recall the elementary bound $\sqrt{y} \geq \frac{y+1}{2} - \frac{(y-1)^2}{2}$ for $y < 1$.
% This implies that
% \[
% 1-\sqrt{y} \leq 1 - \frac{y+1}{2} + \frac{(y-1)^2}{2} < \frac{x}{2} + \frac{x^2}{2} < x
% \]
% since $x^2 < x$ for $x \in (0,1)$.
% Since $x = \frac{\delta}{a} + \frac{\gamma}{b} - \frac{\delta\gamma}{ab} < \frac{\delta}{a} + \frac{\gamma}{b}$, this implies the claim.
% This implies that 
% \begin{align*}
%     \sqrt{ab} - \sqrt{(a-\delta)(b-\gamma)} &= \sqrt{ab}\left(1 - \sqrt{1 - \frac{\delta}{a} - \frac{\gamma}{b} + \frac{\delta \gamma}{ab}}\right) \\
%     &\geq \frac{\sqrt{ab}}{2} \left(\frac{\delta}{a} + \frac{\gamma}{b} - \frac{\delta\gamma}{ab} \right) \\
%     &\geq \frac{\sqrt{ab}}{4} \left(\frac{\delta}{a} + \frac{\gamma}{b} \right)
% \end{align*}
% where the last inequality follows from noting that for $p, q < 1$ we have $pq < \frac{(p+q)^2}{2} < \frac{p+q}{2}$ and applying this to $p = \frac{\delta}{a}$ and $q = \frac{\gamma}{b}$.