% \section{Proof of Claim~\ref{claim:delta-bounded}}\label{app:delta-bounded}
% From equation~\eqref{eq:pdlp-portfolio-value}, we have
%  \begin{align*}
% \frac{\partial (\tilde{V}-V)}{\partial p} &= -\frac{1}{(1+F)^2} \frac{\partial F}{\partial p} V(p, R) + \left(\frac{1}{1+F} - 1\right)R + \frac{\delta}{1+F} - \frac{\delta}{(1+F)^2} \frac{\partial F}{\partial p} + f(R-R^A) \\
% &\leq -\frac{R}{2} - \frac{1}{4}\frac{\partial F}{\partial p}(V(p, R)+\delta) + \delta + f(R-R^A) \\ 
% &\leq \left(f - \frac{1}{2}\right)R  + R - \frac{1}{4}\frac{\partial F}{\partial p} \left[p^T R + \delta\right] \\
% &\leq \left(\frac{1}{2}-f\right)R
% \end{align*}
% The first inequality holds from the assumption that $F \leq 1$, the second inequality holds from $\delta \leq R$, and the final inequality comes from $\frac{\partial F}{\partial p} \geq \frac{8R}{p^T R}$ 

% % \begin{align*}
% % \frac{\partial (\tilde{V}-V)}{\partial p} &= \delta + (\nabla_p F)(p^T(R^U+\delta)) + F(p, w^{\star}, R, \delta)(R^U + \delta) + f(R-R^U) \\
% % &= (1+F(p, w^{\star}, R^U, \delta)) \delta + (\nabla_p F) (p^T(R^U+\delta)) + fR + (F(p, w^{\star}, R, \delta)-f)R^U \\
% % &\leq 2\delta + \frac{f p^T(R^U+\delta)}{p^T R}R + fR + (1-f)R^U \\
% % &\leq 2\delta + 2fR + (1-f)R^U = 2\delta + 2fR + (1-f)(R-\ell) \\
% % & \leq (1-f)R
% % \end{align*}
% % as claimed.
% The first equality is rearranging terms, the second inequality uses the upper bound on $\nabla_p F$ and $F \leq 1$, the third inequality uses the fact that $R^A+\delta < R$ if $\delta$ is admissible, and the final inequality uses the condition on $\ell$.
% We note that for GMX's PDLP~\eqref{eq:gmx-discounting}, the subgradient of $F$ is bounded everywhere by $\gamma_t \frac{\partial w_i}{\partial p_j}$, where $w_i = \frac{p_i R_i}{p^T R}$ which naturally has $\frac{\partial w_i}{\partial p_j} \geq \frac{c R}{p^T R}$.
% Therefore, for appropriately chosen $\gamma_t$, the conditions of the claim hold for the GMX discount function.


% % From the definition of $\tilde{V}$ and linearity of the derivative, we have
% % \begin{align*}
% % \Delta(\tilde{V} - V) &= \frac{\partial}{\partial p}\left(F(p, w^{\star}, \pi, \delta)V(p, \pi+\delta)\right) + f \ell \\ 
% % &= \frac{\partial F}{\partial w} \frac{\partial w}{\partial p} V(p, \pi+\delta) + F(p, w^{\star}, \delta) \Delta(V) + f \ell \\
% % &\leq \Delta(V) - \frac{\partial w}{\partial p} V(p, \pi+\delta) + f\ell
% % \end{align*}
% % From equation~\eqref{eq:weight-derivative}, we have
% % \[
% % \frac{\partial w_i }{\partial p_i} = \frac{\pi_i}{V(p, \pi)} - \frac{p_i \pi_i^2}{V(p, \pi)^2} \geq \frac{\pi_i}{V(p, \pi)} - \frac{1}{\min_i p_i} \geq c \frac{\pi_i}{V(p, \pi)}
% % \]
% % for $c > 0$.
% % This implies
% % \[
% % \frac{\partial w}{\partial p} V(p, \pi+\delta) \geq c\frac{\pi V(p, \pi+\delta)}{V(p, \pi)} = c\Delta(V)\frac{V(p, \pi+\delta)}{V(p, \pi)} = c\Delta(V) \left(1 + \frac{p^T \delta}{p^T \pi}\right)
% % \]
% % for $c < 1$ as $\Delta(V) = \pi$.
% % Therefore, we have
% % \[
% % \Delta(\tilde{V}-V) \leq \Delta(V) - c\left(1 + \frac{p^T \delta}{p^T \pi}\right)\Delta(V) + f \ell = K \Delta(V) + f\ell
% % \]
% % for $K < 1$.
% % Finally, note that 