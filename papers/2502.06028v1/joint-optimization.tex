% \section{Joint Optimization}\label{app:joint-optimization}
% % \paragraph{Joint Optimization.}
% As constructed, creation arbitrage from~\S\ref{subsec:creation} adds assets to the pool whereas redemption arbitrage from~\S\ref{subsec:redemption} removes assets from the pool profitably.
% As one might expect, jointly optimizing the creation and redemption processes is likely to maximize output.
% In the general case, this problem is described by a pair of bilevel optimization problems~\cite{dempe2002foundations}, one where creation arbitrage is performed first and the other with redemption arbitrage performed first.
% Let $(\delta^{\star}_1, \sigma^{\star}_1, \lambda^{\star}_1)$ be the solution to the optimization problem 
% \begin{equation}\label{eq:bilevel-arb-1}
% \begin{aligned}
%     &\text{maximize}_{\sigma, \lambda} && \sigma\left((1+f(\lambda-\delta)) \frac{V(p, R)}{S} - p^{\mathcal{P}}\right),\; \\
%     &\text{s.t.} && \lambda \leq R^U + \delta \\ 
%     &&&  0 \leq \sigma \leq S \\
%     &&& \delta \in \argmax_{\delta'} F(p, w^{\star}, R, \delta')(p^T \delta') + \ones_{\delta' \geq - R^U}
% \end{aligned}
% \end{equation}
% This represents optimizing the creation arbitrage first solving for $\delta_1^{\star}$ and then adding the liquidity from $\delta_1^{\star}$ to the unutilized reserves and then solving for $(\sigma_1^{\star}, \lambda_1^{\star})$.
% Similarly, let $(\delta^{\star}_2, \sigma^{\star}_2, \lambda^{\star}_2)$ be the solution to 
% \begin{equation}\label{eq:bilevel-arb-2}
% \begin{aligned}
%     &\text{maximize}_{\delta} &&  F(p, w^{\star}, R-\lambda, \delta')(p^T \delta'),\; \\
%     &\text{s.t.} &&  \delta \geq -R^U + \lambda \\
%     &&& (\sigma, \lambda) \in \argmax_{\sigma, \lambda} \sigma\left((1+f(\lambda)) \frac{V(p, R)}{S} - p^{\mathcal{P}}\right) + \ones_{\lambda \leq R^U} + \ones_{\sigma \in [0, S]}
% \end{aligned}
% \end{equation}
% This corresponds to solving the redemption arbitrage first, before solving the creation arbitrage problem with the redemption portfolio removed.
% Define the net profit function
% \[
% \mathsf{PNL}(\delta, \sigma, \lambda) = \sigma\left((1+f(\lambda-\delta)) \frac{V(p, R)}{S} - p^{\mathcal{P}}\right) + F(p, w^{\star}, R-\lambda, \delta')(p^T \delta')
% \]
% If $\mathsf{PNL}(\delta_1^{\star}, \sigma_1^{\star}, \lambda_1^{\star}) > \mathsf{PNL}(\delta_2^{\star}, \sigma_2^{\star}, \lambda_2^{\star})$, then use $(\delta^{\star}_1, \sigma^{\star}_1, \lambda^{\star}_1)$ as the optimal arbitrage; otherwise use $(\delta^{\star}_2, \sigma^{\star}_2, \lambda^{\star}_2)$.
% We note that the uniqueness and existence of the solutions to such bilevel optimizations problems are well studied and hold under a number of known smoothness conditions (\cf~\cite[Ch. 3]{dempe2015bilevel}).
