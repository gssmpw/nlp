\section{When is it better to have a multiple pools?}\label{app:multipool}
Recall that GMX moved from a single-pool PDLP model to a multiple pool PDLP system.
In this system, a PDLP portfolio $R$ is partitioned into sub-portfolios $R = \sum_{i=1}^k R_i$ where $R_i$ represents the assets in pool $i \in [k]$.
Each pool services a different set of perpetuals with a unique target weight, among other parameters.
A natural question to ask is when is it more efficient to aggregate the $k$ pools into a single pool.

We will consider the simplest model to analyze this question with $k = 2$ and $\mathsf{supp}(R_1) \cap \mathsf{supp}(R_2) = \emptyset$, \ie,~the two PDLP pools have no assets in common.
We will assume that $|\mathsf{supp}(R_1)| = m_1, |\mathsf{supp}(R_2)| = m_2$ and that $m_1+m_2 = n$.
We will abuse notation slightly as also refer to the set of assets in $R_i$ as $R_i$.
We consider the full symmetric stochastic covariance matrix $\Sigma \in \reals^{n \times n}$ and write
\[
\Sigma = \left[
\begin{matrix}
    A & B \\ 
    B^T & C
\end{matrix}
\right]
\]
where $A \in \reals^{m_1 \times m_1}, B \in \reals^{m_1 \times m_2}, C \in \reals^{m_2 \times m_2}$ are matrices with $A, C$, non-singular and positive definite.

For mean-variance optimization, one computes the conditional covariance\footnote{Note that this is technically only the conditional covariance for multivariate normal distributions; for generic distributions it corresponds to the partial correlation~\cite[\S6.2.3]{zhang2006schur}.} using Schur Complements~\cite{ouellette1981schur, zhang2006schur}.
We denote the Schur complements as $\Sigma / A = A - B C^{-1} B^{T}$ and $\Sigma / C = C - B^T A^{-1} B$.
$\Sigma / A$ represents the covariance matrix after marginalizing the variables in $R_2$, and similarly for $\Sigma / C$~\cite[\S6.2.3]{zhang2006schur}.
For a covariance matrix $S$, we define the portfolio values $V_t^S$ as the delta hedged portfolio values when the covariance matrix is $S$.

Our goal is to find conditions under which the delta hedged portfolio for a single pool is better than an isolated pool.
We can view the isolated pool as having either the covariance $A$ or $C$ without the impact of the other, whereas the single pool needs to incorporate information from both pools.
We can view the conditional covariance matrices $\Sigma / A$ and $\Sigma / C$ as incorportating information from both pools and hence solving the mean-variance problem for these matrices represents the single pool solution.

If the single pool provides better returns than multiple pools, this corresponds to the following conditions: $\Expect[V_t^{\Sigma / A} - V(\mathcal{P}, p)] \geq \Expect[V_t^{A}-V(\mathcal{P}, p)]$ and $\Expect[V_t^{\Sigma / C} - V(\mathcal{P}, p)] \geq \Expect[V_t^{C}-V(\mathcal{P}, p)]$
which reduces to
\begin{align}\label{eq:single-pool-win-condition}
     \Expect[V_t^{\Sigma / A}] \geq \Expect[V_t^{A}] &&    \Expect[V_t^{\Sigma / C}] \geq \Expect[V_t^{C}]
\end{align}
We prove the following sufficent condition for when this occurs:
\begin{claim}\label{claim:schur}
    Suppose that $\sigma_{\min}(\Sigma_X) > \sigma_{\max}(A)$ and $\sigma_{\min}(\Sigma_Y) > \sigma_{\max}(B)$.
    Then~\eqref{eq:single-pool-win-condition} holds and a single pool provides better delta hedged returns than multiple pools.
\end{claim}
\begin{proof}
%\section{Proof of Claim~\ref{claim:schur}}\label{app:schur}
Conditions~\eqref{eq:single-pool-win-condition} are equivalent to
\begin{equation}\label{eq:delta-portfolio}
\Expect[p^T(1, \pi_t^{\Sigma / A})] \geq \Expect[p^T(1, \pi_t^{A})]
\end{equation}
where $\pi^{\Sigma}_t$ is the optimal portfolio with covariance $\Sigma$.
We can proceed analogously to~\ref{subsec:sharpe}, which makes~\eqref{eq:delta-portfolio} equivalent to the condition
\[
\frac{f}{\gamma} \tilde{p}^T (\Sigma/A) \tilde{\ell} - \tilde{p}^T \tilde{\Delta}(\mathcal{P}) \geq \frac{f}{\gamma}\tilde{p}^T A \tilde{\ell} - \tilde{p}^T \tilde{\Delta}(\mathcal{P})
\]
which is equivalent to
\[
\frac{f}{\gamma}\tilde{p}^T (\Sigma / A) \tilde{\ell} \geq \frac{f}{\gamma}\tilde{p}^T A \tilde{\ell}
\]
A sufficient condition for this to hold for all $\tilde{p}, \tilde{\ell}$ is for $\sigma_{\max}(A) < \sigma_{\min}(\Sigma / A)$.
We note that this doesn't violate Cauchy's interlacing theorem style results which show that $ \sigma_{n-m_1}(\Sigma)< \sigma_{\max}(\Sigma / A) \leq \sigma_{\max}(\Sigma)$
where $\sigma_{k}(X)$ is the $k$th eigenvalue of a matrix $X$ since those are between the spectral of the full matrix $\Sigma$ and the Schur complement as opposed to the subcomponent~\cite[Ch. 2]{zhang2006schur}. 
The same proof applies for the other condition.
\end{proof}

\noindent We note these conditions are similar to hierarchical risk-parity methods~\cite{cotton2024schur, lopez2016building}.
We also note that these conditions suggest that GMX V2's recent dynamic pricing upgrade~\cite{gmx-dynamic-price-impact} is not sufficient alone for ensuring that V2 vaults are easily delta hedgeable as it doesn't take into account pool asset covariance as is done here.
We leave it for future work to connect the dynamic pricing model with the cost of delta hedging.