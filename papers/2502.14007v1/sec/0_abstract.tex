\vspace{-0.8em}

\begin{abstract}
Structural guidance in an image-to-image translation allows intricate control over the shapes of synthesized images. Generating high-quality realistic images from user-specified rough hand-drawn sketches is one such task that aims to impose a structural constraint on the conditional generation process. While the premise is intriguing for numerous use cases of content creation and academic research, the problem becomes fundamentally challenging due to substantial ambiguities in freehand sketches. Furthermore, balancing the trade-off between shape consistency and realistic generation contributes to additional complexity in the process. Existing approaches based on Generative Adversarial Networks (GANs) generally utilize conditional GANs or GAN inversions, often requiring application-specific data and optimization objectives. The recent introduction of Denoising Diffusion Probabilistic Models (DDPMs) achieves a generational leap for low-level visual attributes in general image synthesis. However, directly retraining a large-scale diffusion model on a domain-specific subtask is often extremely difficult due to demanding computation costs and insufficient data. In this paper, we introduce a technique for sketch-to-image translation by exploiting the feature generalization capabilities of a large-scale diffusion model without retraining. In particular, we use a learnable lightweight mapping network to achieve latent feature translation from source to target domain. Experimental results demonstrate that the proposed method outperforms the existing techniques in qualitative and quantitative benchmarks, allowing high-resolution realistic image synthesis from rough hand-drawn sketches.

\keywords{Sketch-to-Image translation \and Latent diffusion models.}
\end{abstract}
