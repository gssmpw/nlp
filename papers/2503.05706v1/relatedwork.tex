\section{Related Work}
\subsection{Intersection as Critical Zones for Road Accidents}

Intersections, with their heightened number of conflict points and requirements for simultaneous interactions among multiple road agents, are well recognized as significant hotspots for road safety concerns. The likelihood of collisions at intersections is reported to be approximately 335 times higher than at other road segments, predominantly caused by vehicles turning with obstructed sightlines \cite{congress2021identifying}. Furthermore, the National Highway Traffic Safety Administration reports that over 80\% of pedestrian accidents can be categorized into specific types, with a substantial portion being intersection-related \cite{nhtsa}.

Several studies have investigated this intersection-accident nexus. A study by \cite{al2003analysis} analyzed the triggers of intersection-related accidents, identifying key traffic safety issues. However, this study did not dissect the unique characteristics of intersection and non-intersection-related accidents and lacked an in-depth exploration of geographic features and traffic characteristics. Similarly, \citet{kumar2016data} used data mining techniques to classify accident locations based on accident frequency, but their analysis was limited by the lack of additional data, such as vehicle speed, weather information, and road conditions.

\subsection{Exploration into Intersection Accident Factors}

The complexity of intersection-related traffic accidents has prompted a multitude of research efforts aimed at understanding the contributing factors. These factors can be broadly divided into two major categories: human factors and environmental factors.

Investigations into human factors have largely centered around the behaviors of drivers and pedestrians. Studies, such as those by \citet{lee2014analysis} and \citet{tomoda2022analysis}, have probed the correlation between population size, non-motorized transport users, children's behaviors at intersections, and accident rates. However, these studies face limitations due to the reliance on self-reported data and the lack of geographical and cultural standardization. While qualitative data offer insights into individual experiences, perceptions, and subjective interpretations, the integration of quantitative metrics, such as accident rates, can provide a more comprehensive and measurable understanding of human factors contributing to intersection-related accidents.

On the other hand, research into environmental factors in intersection-related accidents employs various quantitative methodologies and models. These studies consider variables like accident severity, road type, external conditions, intersection attributes, time factors, infrastructure elements, and vehicle factors \cite{dixon2015improved, eboli2020factors, huang2008severity, oh2004development, mussone2017analysis, gong2020application, chen2012analysis, park2016random, lam2004environmental}. However, many of these studies overlook less tangible factors such as geographical characteristics of accidents and other potential contributors that are difficult to directly observe or record.

\subsection{Intersection Visibility and Road Accidents}

Intersection visibility, encompassing the driver's visual field and sight distance, plays a pivotal role in road safety \cite{todd2017quantitative, bauer1996statistical}. Sight distance, or the length of road visible to a driver, is particularly crucial at intersections to ensure safe maneuvering. However, various factors such as road curvature or roadside obstructions can compromise this visibility. In response to these challenges, recent studies have employed high-resolution lidar data, simulations, and Digital Elevation Models (DEM) to explore these elements and assess their impact on sight distance \cite{chen2012analysis}.

Despite these advancements, there exists a notable gap in comprehensive research linking sight distances, fields of view, and intersection-related risks. Our research aims to bridge this gap by employing information. By doing so, we hope to offer a more holistic investigation of intersection visibility and its implications for road safety \cite{mussone1999analysis}.

\subsection{The Role of Geographic Information Systems (GIS) in Accident Analysis}

Intersection accident analyses often use macro-factor datasets from broad geographical areas. They typically account for accident counts and severity over specific periods. Many past studies target specific regions, but a few have analyzed accident datasets at a macro level \cite{ma2014quasi, hu2020investigation, chen2019multinomial, gomes2013influence}. However, this might lead to spatial autocorrelation issues, challenging the independent observation regions assumption \cite{dale2002spatial}. Thus, spatial considerations are vital for these investigations.

GIS tools, essential for geospatial data analysis, are becoming more prevalent in road safety studies. Existing literature mainly discusses two GIS applications: identifying accident hotspots and geocoding accidents for spatial categorization and analysis. In hotspot identification, techniques include kernel density estimation, nearest neighbor distances, and spatial indices like Moran's I. For example, \cite{al2021mapping} used the SANET toolkit to assess hotspot locations, while \citet{cheng2018traffic} applied the Getis-Ord Gi* technique for spatial autocorrelation. The second category, geocoding, facilitates statistical evaluations. \citet{ma2014quasi} visualized geocoded road and accident datasets, linking accidents to road segments. Similarly, \cite{loo2006validating} integrated the Traffic Accident Data System with road networks to position accident data accurately. \cite{noland2004spatially} spatially organized England's road casualty data, connecting it to land-use types in electoral districts.

In conclusion, while previous studies have contributed significantly to understanding the factors influencing intersection-related accidents, a gap persists in the literature with respect to the incorporation of spatial and visibility factors at the micro-level. Our research aims to bridge this gap by integrating human factors, environmental factors, and GIS capabilities, with particular emphasis on the quantification of visibility at intersections. By doing so, we introduce a novel perspective to the study of intersection-related accidents. This approach allows us to capture the subtle yet significant influence of intersection visibility on accident occurrence, thereby contributing to a more nuanced understanding that could inform more effective road safety measures.