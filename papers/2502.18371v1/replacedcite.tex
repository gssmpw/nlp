\section{Related Work}
\subsection{Factors influencing Memorability}
Bainbridge et al. explored how humans process and retain visual stimuli, emphasizing the importance of emotionally salient and visually distinctive elements in enhancing memorability ____. Their findings suggested that humans are more likely to remember visual scenes that contain unique or emotionally charged content, as opposed to mundane or repetitive scenes. Additionally, the other study focused on the concept of intrinsic memorability revealing that certain visual characteristics, such as color, object saliency, and scene composition, naturally influence memory retention, independent of individual viewer biases ____. These studies laid the groundwork for understanding the cognitive processes involved in memorizing visual information and provided key insights into the types of visual content that are more likely to be remembered.
Several studies have aimed to identify the specific features or characteristics that contribute to the memorability of visual content. One study assessed the memorability of various objects within scenes, and found that certain object categories, like faces and animals, are inherently more memorable than others, such as buildings or landscapes ____. It highlighted the role of object prominence and scene context in shaping human memory. Similarly, it has been shown that the memorability of a scene is largely driven by its most memorable object ____. Despite these valuable insights, these studies were limited in their focus on static images and often failed to account for the dynamic, multimodal nature of real-world stimuli, such as advertisements or videos. Moreover, these works largely overlooked the temporal and emotional dimensions that play a critical role in memory formation.

\subsection{Machine Learning Approaches for Multimodal Memorability Prediction}
More recently, multimodal approaches have emerged as a powerful method for memorability prediction, integrating visual, textual, and audio features to capture a broader spectrum of the factors that contribute to memory retention. Several studies have investigated predicting memorability from video content, integrating audio and emotional cues to enhance model accuracy ____. Other study leveraged video-triggered Electroencephalogram (EEG) data to examine how emotions evoked by videos influence memorability ____. Another study has integrated LLMs with deep learning to process not just visual features, but also audio and textual elements, highlighting the benefit of capturing the complex interactions across modalities in advertisements ____. Although these models have improved prediction accuracy, they often fail to fully capture the complexity of human cognition, as they process modalities separately rather than integrating them into cohesive multimodal representations: an essential aspect for modeling human memory, particularly in scenarios requiring temporal processing and adaptability.