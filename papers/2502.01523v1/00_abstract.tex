Large language models (LLMs) are prone to hallucinations in question-answering (QA) tasks when faced with ambiguous questions. Users often assume that LLMs share their cognitive alignment, a mutual understanding of context, intent, and implicit details, leading them to omit critical information in the queries. However, LLMs generate responses based on assumptions that can misalign with user intent, which may be perceived as hallucinations if they misalign with the user's intent. Therefore, identifying those implicit assumptions is crucial to resolve ambiguities in QA. Prior work, such as AmbigQA, reduces ambiguity in queries via human-annotated clarifications, which is not feasible in real application. Meanwhile, ASQA compiles AmbigQA's short answers into long-form responses but inherits human biases and fails capture explicit logical distinctions that differentiates the answers. We introduce Conditional Ambiguous Question-Answering (CondAmbigQA), a benchmark with 200 ambiguous queries and condition-aware evaluation metrics. Our study pioneers the concept of ``conditions'' in ambiguous QA tasks, where conditions stand for contextual constraints or assumptions that resolve ambiguities. The retrieval-based annotation strategy uses retrieved Wikipedia fragments to identify possible interpretations for a given query as its conditions and annotate the answers through those conditions. Such a strategy minimizes human bias introduced by different knowledge levels among annotators. By fixing retrieval results, CondAmbigQA evaluates how RAG systems leverage conditions to resolve ambiguities. Experiments show that models considering conditions before answering improve performance by $20\%$, with an additional $5\%$ gain when conditions are explicitly provided. These results underscore the value of conditional reasoning in QA, offering researchers tools to rigorously evaluate ambiguity resolution.