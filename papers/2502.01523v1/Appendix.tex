
\appsection{Appendix}
% 各个章节以 A, B, C 等编号

\renewcommand{\thesection}{\Alph{section}}  
\section{Dataset Examples}
\label{appendix_label}


\begin{table}[htbp]
\centering
\begin{tabular}{l p{0.8\textwidth}}
\hline
\multicolumn{2}{c}{Data example, Rainbow Six Siege} \\
\hline
Question & How long is a Rainbow Six Siege game? \\
\hline
Property1  & \\ \textbf{Condition}: & The game \textit{Tom Clancy's Rainbow Six Siege} includes several modes such as casual and ranked, with an emphasis on close-quarters combat. Players cannot respawn during a round, and each round begins with a one-minute preparation phase where attackers use drones to scout enemy positions while defenders set up their defenses. The game also features a spectator mode and large-scale, time-limited seasonal events that differ from standard modes. The addition of DLC has expanded the variety of maps and operators in the game. \\
\hline
Property1 & \\ \textbf{Ground Truth}:  &In casual matches, each round lasts four minutes, while ranked matches have three-minute rounds. The one-minute preparation phase extends the overall duration of each round slightly, but does not impact the core timing. The spectator mode allows for additional observation angles, without affecting the match length. Seasonal events can extend match durations, although precise times are not specified. Despite updates and the introduction of new maps and operators, the round durations remain four minutes for casual and three minutes for ranked modes. \\
\hline
Citations &\textbf{Fragment 2}: "title: Tom Clancy's Rainbow Six Siege, text: When the round begins in an online match, the attackers choose one of several spawn points. A one-minute preparatory period will commence, allowing the attackers to scout the map, while defenders establish their defenses. Matches last only four minutes for casual and three minutes for ranked." \newline\newline
\textbf{Fragment 5}: "title: Tom Clancy's Rainbow Six Siege, text: Seasonal Events: Limited-time events are available for one season, featuring game modes unique to the regular Bomb, Secure Area, or Hostage modes." \newline\newline
\textbf{Fragment 1}: "title: Tom Clancy's Rainbow Six Siege, text: At launch, the game featured 11 maps and 5 different gameplay modes spanning both PVE and PVP. With the downloadable content (DLC) released post-launch, there are currently 20 playable maps." \\
\hline
Retrieval Fragments & \textbf{Fragment 1}: "title: Tom Clancy's Rainbow Six Siege, text: At launch, the game featured 11 maps and 5 different gameplay modes spanning both PVE and PVP. With the downloadable content (DLC) released post-launch, there are currently 20 playable maps." \newline\newline
\textbf{Fragment 2}: "title: Tom Clancy's Rainbow Six Siege, text: When the round begins in an online match, the attackers choose one of several spawn points. A one-minute preparatory period will commence, allowing the attackers to scout the map, while defenders establish their defenses. Matches last only four minutes for casual and three minutes for ranked." \newline\newline
\textbf{Fragment 3}: "title: Allegiance (video game), text: A typical game lasts between thirty and forty-five minutes, although games of more than two hours in length are not uncommon." \newline\newline
\textbf{Fragment 4}: "title: Tom Clancy's Rainbow Six (video game), text: Rainbow Six is a tactical shooter, in which characters are affected by realistic factors and can be killed with a single bullet; therefore, wise tactics and planning are encouraged to complete missions over sheer force and firepower." \newline\newline
\textbf{Fragment 5}: "title: Tom Clancy's Rainbow Six Siege, text: Seasonal Events: Limited-time events are available for one season, featuring game modes unique to the regular Bomb, Secure Area, or Hostage modes." \\

\hline
\end{tabular}
\end{table}

\clearpage

\section{Query Prompts Template}
\label{appendix_labelb}
\begin{tabular}{p{0.9\textwidth}}
\hline
\textbf{Query Analysis Instructions Template} \\
\hline
You are a professional question analysis assistant. Your task is to analyze questions and their previous incomplete annotations, determining whether these questions contain ambiguities or have multiple possible answers. Please carefully read the following instructions and complete the analysis as required. First, you will receive two inputs:
\verb|<questions> {{QUESTIONS}} </questions>| \newline
\verb|<previous_annotations> {{PREVIOUS_ANNOTATIONS}} </previous_annotations>| \\
\hline
Please follow these steps: 1) Read each question and annotation carefully. 2) Analyze each question for: a) ambiguity - explain different interpretations, b) multiple possible answers - provide examples. 3) Consider: question clarity, vague terms, context sufficiency, subjective elements. 4) Use format: \newline
\verb|<analysis>| \newline
\verb|<question_number>Number</question_number>| \newline
\verb|<question_text>Text</question_text>| \newline
\verb|<ambiguity_analysis>Results</ambiguity_analysis> |\newline
\verb|<multiple_answers>Results</multiple_answers>|\newline
\verb|</analysis>| \newline
5) Complete in English. 6) Compare with previous annotations. \\
\hline
\end{tabular}


\clearpage
\section{Dataset Prompts}
\label{appendix_labelc}
\begin{tabular}{l p{0.75\textwidth}}
\hline
\multicolumn{2}{c}{\textbf{Dataset Prompts (Part 1)}} \\
\hline
Question Answering & You are tasked with providing a structured answer to a question based on the given text fragments. Your goal is to present possible interpretations supported by the fragments, clearly distinguishing between preconditions and detailed answers. \newline
Question: <question> [INSERT QUESTION HERE] </question> \newline
Text fragments: \newline
<fragments> \newline
[INSERT FRAGMENTS HERE] \newline
</fragments> \newline
Please provide your answer using the following format: \newline
<answer> \newline
[English Answer] \newline
Interpretation [X]: \newline
Preconditions: \newline
* [Necessary background information or assumptions, not directly answering the question] [Fragment X] \newline
* [Necessary background information or assumptions, not directly answering the question] [Fragment Y] \newline
Detailed answer: \newline
[Specific information directly answering the question] [Fragment Z] \newline
[Specific information directly answering the question] [Fragment A, Fragment B] \newline
[Repeat the Interpretation structure for as many interpretations as necessary] \newline
</answer> \newline
Provide all possible interpretations, ensuring that preconditions and detailed answers are clearly distinct. Every statement must be supported by at least one fragment citation. If you find conflicting information, present all viewpoints and clearly indicate the source of each. \\

\hline
Ambiguity Analysis & Based on the question "[INSERT QUESTION HERE]" and the interpretations you provided in your previous answer, analyze potential ambiguities in the question and how they lead to different answers. Consider how different contexts might influence these interpretations. \newline
<analysis> \newline
Ambiguity point [X]: [Describe ambiguity that could lead to different interpretations] \newline
Impact: \newline
1. [Impact on Interpretation 1] [Based on Fragment X, Y] \newline
2. [Impact on Interpretation 2] [Based on Fragment Z, A] \newline
Contextual considerations: [How different backgrounds might affect understanding] \newline
[Repeat the Ambiguity point structure for as many ambiguities as necessary] \newline
</analysis> \newline
Explain how each ambiguity leads to different valid answers, citing relevant fragments for each interpretation. \\
\hline
Evidence Evaluation & For each interpretation of the question "[INSERT QUESTION HERE]" that you've provided, evaluate the strength of the supporting evidence. Consider the reliability of sources, consistency across fragments, and potential biases or limitations in the available information. \newline
<evaluation> \newline
Interpretation [X]: [Brief summary of Interpretation X] \newline
Evidence assessment: \newline
* Strengths: [List strong evidence supporting this interpretation] [Fragment X, Y] \newline
* Weaknesses: [Point out potential issues or shortcomings] [Fragment Z] \newline
* Consistency: [Evaluate the consistency of information across fragments] \newline
Overall credibility: [Provide an overall assessment, e.g., "High", "Medium", or "Low"] \newline
[Repeat the Interpretation structure for as many interpretations as necessary] \newline
</evaluation> \newline
Provide a balanced assessment of the evidence for each interpretation, citing specific fragments to support your evaluation. \\
\hline
\end{tabular}

\begin{tabular}{l p{0.75\textwidth}}
\hline
\multicolumn{2}{c}{\textbf{Dataset Prompts (Part 2)}} \\
\hline
\textbf{Structured Answer} & Please provide your answer using the following format:

<answer>
Interpretation [X]:
Preconditions:
* [Necessary background information or assumptions, not directly answering the question] [Fragment X]
* [Necessary background information or assumptions, not directly answering the question] [Fragment Y]
Detailed answer:
* [Specific information directly answering the question] [Fragment Z]
* [Specific information directly answering the question] [Fragment A, Fragment B]
[Repeat the Interpretation structure for as many interpretations as necessary]
</answer>

Provide all possible interpretations, ensuring that preconditions and detailed answers are clearly distinct. Every statement must be supported by at least one fragment citation. If you find conflicting information, present all viewpoints and clearly indicate the source of each.\\

\hline
\textbf{Calibration} & You are tasked with generating a response based strictly on the provided retrieved fragments. Do not introduce any external knowledge or assumptions. Your job is to fill out the following fields using only the information present in the fragments. If any information is missing, leave that field blank.

1. Condition: Summarize the context of the question strictly using the provided fragments. Do not speculate beyond the given information.
2. Groundtruth: Provide the exact answer to the question based on the retrieved fragments. Use only what is explicitly stated.
3. Citations: List the relevant fragments that support your answer. Include the title and text of the fragments that were used.
4. Reason: Explain how the answer was derived solely from the fragments, and mention why any gaps in information were left unfilled.

Fragments:
{retrieved fragments}

Output format:
{
  "condition": "<summary based on fragments>",
  "groundtruth": ["<answer derived from fragments>"],
  "citations": [
    {
      "title": "<fragment title>",
      "text": "<fragment text>"
    }
  ],
  "reason": "<explanation>"
} \\
\hline
\textbf{Merging} & You are provided with a question and several annotated dictionaries. Your task is to merge all the dictionaries without changing the structure or key names. Consolidate similar information, eliminate redundancy, and ensure that the final output accurately reflects the content of all dictionaries. Do not introduce external knowledge or assumptions.

Question:
{question}

Dictionaries:
{dictionaries}

Instructions:
- Merge the "condition" fields from all dictionaries into one, keeping only unique and relevant information.
- Merge the "groundtruth" fields into a single list, ensuring no redundant entries.
- Combine the "citations" fields from all dictionaries, ensuring all relevant citations are included without duplication.
- Leave the "reason" field as an empty string.

Output format:
{
  "condition": "<merged condition from all dictionaries>",
  "groundtruth": ["<merged groundtruth from all dictionaries>"],
  "citations": [
    {
      "title": "<citation title from any dictionary>",
      "text": "<citation text from any dictionary>"
    },
    {
      "title": "<citation title from another dictionary>",
      "text": "<citation text from another dictionary>"
    }
  ],
  "reason": ""
} \\
\hline
\end{tabular}

\section{Evaluation Prompts}
\label{appendix_labeld}
\begin{tabular}{l p{0.75\textwidth}}
\hline
\textbf{RAG with} & Question: \{question\} \\
\textbf{Conditions} & Retrieved fragments: \\
\textbf{Prompt} & \{Fragment 1 - \{title\}: \\
& \{text\}\} \\
& ... \\
& Please complete the following tasks: \\
& 1. Identify up to THREE key conditions related to the question based solely on the provided fragments. \\
& 2. For each condition, provide a corresponding detailed answer. \\
& 3. Cite the sources (fragment numbers) that support each condition and answer. \\
& 4. Output the results in JSON format with the following structure: \\
& \# ... JSON structure and instructions ... \\
\hline
\textbf{Modified} & Question: \{question\} \\
\textbf{Condition-based} & Context fragments: \\
\textbf{Prompt} & \{Fragment 1 - \{title\}: \\
& \{text\}\} \\
& ... \\
& Conditions to address: \\
& Condition 1: \{condition\} \\
& ... \\
& IMPORTANT: Respond with ONLY the following JSON format, no other text: \\
& \# ... JSON structure and instructions ... \\
\hline
\textbf{Standard RAG} & Question: \{question\} \\
\textbf{Prompt} & Retrieved fragments: \\
& \{Fragment 1 - \{title\}: \\
& \{text\}\} \\
& ... \\
& Please complete the following tasks: \\
& 1. Answer the question based solely on the provided fragments. \\
& 2. Cite up to THREE sources (fragment numbers) that support your answer. \\
& \# ... Additional instructions and JSON structure ... \\
\hline
\textbf{Evaluation} & \textbf{Condition Correctness}: \\
\textbf{Metrics} & - Name: ``Condition Correctness'' \\
& - Criteria: ``Determine whether the actual condition is factually correct based on the expected condition.'' \\
& - Evaluation steps: \\
& 1. Check whether the facts in 'actual condition' contradicts any facts in 'expected condition'. \\
& 2. Heavily penalize omission of critical details in the condition. \\
& 3. Ensure that the condition is clear and unambiguous. \\
& \\
\hline
\textbf{Evaluation} & \textbf{Answer Correctness}: \\
\textbf{Metrics} & - Name: ``Answer Correctness'' \\
& - Name: ``Answer Correctness'' \\
& - Criteria: ``Determine whether the actual answer is factually correct based on the expected answers.'' \\
& - Evaluation steps: \\
& 1. Check whether the facts in 'actual answer' contradicts any facts in 'expected answers'. \\
& 2. Heavily penalize omission of critical details in the answer. \\
& 3. Ensure that the answer directly addresses the question without irrelevant information. \\
\hline
\end{tabular}