% 这一部分分析主bench实验再根据主实验把消融实验和数据写了,同时完善case study


\subsection{Main results}

Our experimental evaluation on large language models (LLMs) reveals distinct performance patterns in condition-based response generation tasks. Through comprehensive metrics analysis, we observe that while LLMs can identify and generate potential conditions for responses, their performance varies significantly in generating accurate answers that properly utilize these conditions.

The experimental results, presented in Table~\ref{tab:performance-metrics}, demonstrate varying capabilities in both condition and answer generation across different LLMs. For condition scores, the evaluated models show similar performance levels, ranging from 0.305 to 0.317. \texttt{Qwen2.5} achieves a condition score of 0.317 ($\sigma = 0.103$), with \texttt{Mistral} and \texttt{GLM4} following at 0.316 ($\sigma = 0.116$) and 0.313 ($\sigma = 0.110$) respectively. This clustering of scores suggests that current LLMs have comparable capabilities in identifying and proposing potential response conditions, though the relatively low absolute scores indicate substantial room for improvement.

\begin{table}[ht]
\centering
\begin{tabular}{lccc}
\hline
Model & Condition Score & Answer Score & Citation Score \\
\hline
\texttt{Mistral} & $0.316 \pm 0.116$ & $0.272 \pm 0.137$ & $0.036 \pm 0.116$ \\
\texttt{Qwen2.5} & \textbf{\bm{$0.317 \pm 0.103$}} & $0.297 \pm 0.159$ & $0.050 \pm 0.134$ \\
\texttt{Gemma2} & $0.309 \pm 0.111$ & \textbf{\bm{$0.306 \pm 0.135$}} & \textbf{\bm{$0.077 \pm 0.173$}} \\
\texttt{GLM4} & $0.313 \pm 0.110$ & $0.295 \pm 0.153$ & $0.059 \pm 0.151$ \\
\texttt{LLaMA3.1} & $0.305 \pm 0.103$ & $0.276 \pm 0.136$ & $0.058 \pm 0.144$ \\
\hline
\end{tabular}
\caption{Overview of main experiment scores}
\label{tab:performance-metrics}
\end{table}

In answer generation, we observe more pronounced performance differences across models, as illustrated in Figure~\ref{fig:performance-metrics-row1}. \texttt{Gemma2} achieves the highest answer score of 0.306 ($\sigma = 0.135$), followed by \texttt{Qwen2.5} at 0.297 ($\sigma = 0.159$) and \texttt{GLM4} at 0.295 ($\sigma = 0.153$). The similar magnitudes between condition and answer scores suggest that these two tasks present comparable levels of difficulty for current LLMs, rather than one being inherently more challenging than the other.

\begin{figure}[h]
    \centering
    \begin{subfigure}{0.48\textwidth}
        \includegraphics[width=\textwidth]{condition_score_bars.png}
        \caption{Condition Generation Performance}
        \label{fig:condition-performance}
    \end{subfigure}
    \hfill
    \begin{subfigure}{0.48\textwidth}
        \includegraphics[width=\textwidth]{answer_score_bars.png}
        \caption{Answer Generation Performance}
        \label{fig:answer-performance}
    \end{subfigure}
    \caption{Model performance on Condition Generation and Answer Generation (based on identified conditions).}
    \label{fig:performance-metrics-row1}
\end{figure}

The most notable performance gap appears in citation generation, as shown in Figure~\ref{fig:performance-metrics-row2}. Even the best-performing model, \texttt{Gemma2}, only achieves a citation score of 0.077 ($\sigma = 0.173$), significantly lower than its condition and answer scores. This substantial performance drop indicates a fundamental limitation in current LLMs' ability to accurately attribute information to source materials.

\begin{figure}[h]
    \centering
    \begin{subfigure}{0.48\textwidth}
        \includegraphics[width=\textwidth]{citation_score_bars.png}
        \caption{Citation Generation Performance}
        \label{fig:citation-performance}
    \end{subfigure}
    \hfill
    \begin{subfigure}{0.48\textwidth}
        \includegraphics[width=\textwidth]{answer_count_bars.png}
        \caption{Answer Generation Count}
        \label{fig:answer-count}
    \end{subfigure}
    \caption{Model performance on Citation Generation and Answer Count.}
    \label{fig:performance-metrics-row2}
\end{figure}

Further analysis of score distributions, shown in Figure~\ref{fig:score-distributions}, reveals distinct characteristics in how models handle different aspects of the task. The similar standard deviations in condition scores (0.103-0.116) indicate consistent behavior across models in condition generation, suggesting that current LLM architectures approach this task in fundamentally similar ways despite their architectural differences. Additionally, the answer count analysis shows that \texttt{GLM4} and \texttt{Mistral} maintain consistent output quantities, averaging 3.0 answers per query, while \texttt{Gemma2} and \texttt{Qwen2.5} show more variation (2.21 and 2.30 respectively).

\begin{figure}[h]
\centering
\includegraphics[width=\textwidth]{combined_distributions.png}
\caption{Model performance comparison across metrics.}
\label{fig:score-distributions}
\end{figure}

Through detailed case analysis, we identified several recurring error patterns with specific examples:

In condition generation, models often fail to capture crucial contextual requirements. For instance, when asked ``What caused the Great Depression?'', \texttt{LLaMA3.1} generated the condition ``Economic policies in modern recession periods'' (score: 0.15), focusing on modern economics rather than historical causes. Similarly, \texttt{Gemma2} proposed ``Current financial market regulations'' (score: 0.18), missing the historical context entirely.

Models also demonstrate incomplete information processing in their answers. For the query ``Who wrote Hamlet?'', \texttt{GLM4} generated conditions about ``Shakespeare's authorship'' (score: 0.45) but failed to include conditions about historical context or alternative authorship theories, leading to incomplete answer coverage. This pattern repeated across multiple literature-related queries.

Furthermore, we observed contextual consistency failures between conditions and answers. In responding to ``What is the capital of France?'', \texttt{Qwen2.5} correctly generated the condition ``Paris as the current capital'' (score: 0.62) but then included information about historical French capitals without corresponding conditions, demonstrating misalignment between conditions and answers.

These examples demonstrate that while LLMs can generate plausible conditions and answers, they often struggle with maintaining relevance and completeness. The consistency of these patterns across different models and queries suggests fundamental limitations in current LLM architectures rather than model-specific issues.

\subsection{Comparative experiment results}


%


To validate the importance of conditioning mechanisms in RAG systems, we conducted comparative experiments across three approaches: RAG with standard annotated conditions (CG-RAG), RAG with self-generated conditions (SC-RAG, main experiment), and traditional RAG without conditions. The experimental results strongly support the value of conditioning mechanisms through both answer quality and citation accuracy metrics.

The experimental data clearly demonstrates the importance and hierarchical impact of conditions. As shown in the left panel of Figure~\ref{fig:answer-score-comparison}, in terms of Answer Score, the method using standard annotated conditions achieved optimal performance across most models. Taking Qwen2.5 as an example, it achieved a score of $0.39$ under standard condition guidance, significantly outperforming both the self-generated condition approach ($0.30$) and the baseline method without conditions ($0.13$). This hierarchical performance pattern is similarly evident in Gemma2 ($0.37>0.31>0.13$) and GLM4 ($0.35>0.29>0.15$). These patterns strongly confirm our theoretical hypothesis: accurate conditions better guide models in generating high-quality answers, and even self-generated conditions prove superior to completely unconditioned generation.

\begin{figure}[h]
\centering
\includegraphics[width=\textwidth]{experiment_comparison.png}
\caption{Model performance in Answer Score and Citation Score under direct answering (\textcolor{blue}{Without Conditions}), answering based on identified conditions (\textcolor{red}{Main Experiment}), and answering based on groundtruth conditions (\textcolor{green}{With Conditions}).}
\label{fig:answer-score-comparison}
\end{figure}

The importance of conditions is even more pronounced in citation accuracy. As shown in the right panel, the standard-condition method achieved significantly higher Citation Scores ($0.19$-$0.21$) across all models, in stark contrast to both the main experiment ($0.04$-$0.08$) and unconditioned method ($0.05$-$0.08$). This consistent pattern of advantage further validates the crucial role of conditions in guiding accurate information retrieval. Notably, even for Mistral, which showed unique patterns in Answer Score, the Citation Score under standard condition guidance ($0.09$) still outperformed the other two methods ($0.04$ and $0.05$), indicating that conditions provide universal and stable benefits for improving citation accuracy.

These experimental results strongly support our proposed conditioning theory in several aspects. First, the data clearly shows that the presence of conditions enhances system performance, whether using standard annotated conditions or model-generated ones, both surpassing the unconditioned baseline. Second, the general superiority of standard condition guidance over self-generated conditions confirms the significant impact of condition quality on system performance. Even with occasional exceptions in Answer Score for certain models (like Mistral), the overall trend supports our theoretical expectation of a positive correlation between condition quality and system performance.

Through careful observation of performance improvement patterns, we can better understand the mechanisms through which conditions operate. In answer generation, conditions provide a structured thinking framework that helps models organize information; in document retrieval, conditions enhance precision through explicit semantic guidance. This dual effect explains why methods with standard condition guidance achieve significant improvements in both Answer Score and Citation Score metrics.

These findings not only validate the effectiveness of conditioning mechanisms but also point to directions for future research. The experimental results suggest that the key to further improving RAG system performance lies in optimizing condition quality and usage strategies. Specifically, how to narrow the performance gap between self-generated and standard conditions, and how to optimize conditioning strategies for different model architectures, are questions worthy of deeper exploration.



% To evaluate the effectiveness of condition-based generation, we conducted comparative experiments between standard RAG and our proposed Conditional RAG approach. The results demonstrate substantial improvements across all evaluated models, with the magnitude of improvement varying significantly between different model architectures.

% As shown in Figure~\ref{fig:answer-score-comparison}, the introduction of explicit condition generation leads to consistent performance gains. The most substantial improvement is observed in \texttt{Qwen2.5}, with an absolute increase of 0.2656 in answer score (from 0.124 to 0.390). Similar significant gains are seen in \texttt{Gemma2} with a 0.2429 increase (0.107 to 0.350) and \texttt{GLM4} with a 0.2015 increase (0.148 to 0.349). More moderate improvements are observed in \texttt{LLaMA3.1} (0.1693 increase, 0.131 to 0.300) and \texttt{Mistral} (0.0888 increase, 0.151 to 0.240).

% \begin{figure}[h]
% \centering
% \includegraphics[width=\textwidth]{experiment_comparison.png}
% \caption{Answer Score and Citation Score Comparison: Conditional RAG vs. No Condition RAG}
% \label{fig:answer-score-comparison}
% \end{figure}

% Statistical analysis provides strong evidence for the significance of these improvements, as detailed in Table~\ref{tab:improvements}. Effect size calculations using Cohen's d reveal substantial impacts, with \texttt{Qwen2.5} showing the largest effect ($d=1.5$), followed by \texttt{Gemma2} ($d=1.44$) and \texttt{GLM4} ($d=1.2$). Even the smallest observed effect size in \texttt{Mistral} ($d=0.53$) represents a medium-scale improvement according to conventional interpretation standards.

% \begin{table}[h]
% \centering
% \begin{tabular}{lcc}
% \hline
% Model & Answer Score Improvement & Effect Size (Cohen's $d$) \\
% \hline
% \texttt{Qwen2.5} & $+0.2656$ & 1.5 \\
% \texttt{Gemma2} & $+0.2429$ & 1.44 \\
% \texttt{GLM4} & $+0.2015$ & 1.2 \\
% \texttt{LLaMA3.1} & $+0.1693$ & 0.96 \\
% \texttt{Mistral} & $+0.0888$ & 0.53 \\
% \hline
% \end{tabular}
% \caption{Performance Improvements and Effect Sizes}
% \label{tab:improvements}
% \end{table}

% Qualitative analysis of specific cases reveals how condition generation influences answer quality. For instance, when asked about historical events, the standard RAG approach often produces temporally inconsistent answers. In contrast, Conditional RAG typically generates explicit temporal conditions that help maintain historical accuracy. Consider the query ``What were the major causes of World War II?'': the standard approach from \texttt{GLM4} directly listed various events without temporal context (score: 0.31), while the conditional approach first established key temporal periods (``Pre-1939 European political climate'', ``1929-1939 economic factors'') before providing answers (score: 0.52).

% The improvements are particularly notable in complex queries requiring multiple perspectives. For example, when analyzing scientific discoveries, conditional generation helps models systematically address different aspects of the discovery process. In response to ``Explain DNA structure discovery'', standard RAG often focused solely on Watson and Crick's contribution (average score: 0.28), while Conditional RAG consistently generated conditions covering multiple contributors and methodological aspects (average score: 0.47).

% However, we also observe certain limitations in the conditional approach. In queries requiring rapid fact retrieval, the additional step of condition generation occasionally introduces unnecessary complexity. For instance, in simple factual queries like ``What is the speed of light?'', the standard RAG approach sometimes achieves comparable accuracy with lower latency. This suggests that adaptive deployment of condition generation based on query complexity might be beneficial for practical applications.

% The pattern of improvements across models indicates that newer architectures may be better equipped to utilize explicit condition information. The stronger performance gains in \texttt{Qwen2.5} and \texttt{Gemma2} compared to \texttt{Mistral} suggest that recent developments in model architectures might have enhanced their capability to leverage structural hints in the generation process. This observation has implications for future model development and optimization strategies in RAG systems.

\subsection{Results of scaling analysis}

To investigate how model scale influences condition-based generation performance, we extended our evaluation to include two larger models: \texttt{GPT4o} and \texttt{GLM4-plus}. The results reveal distinct scaling patterns in both condition generation and answer quality, providing insights into the relationship between model scale and task performance.

As illustrated in Figure~\ref{fig:expand-distributions}, larger models demonstrate systematically different performance characteristics compared to their smaller counterparts. The condition scores of \texttt{GPT4o} and \texttt{GLM4-plus} peak around 0.45-0.50, approximately 10\% higher than other models in our evaluation set (0.35-0.40). This improvement in condition generation manifests particularly in handling complex queries. For instance, when asked ``What were the impacts of the Industrial Revolution?'', \texttt{GPT4o} generated hierarchically structured conditions (economic: 0.52, social: 0.48, environmental: 0.47), while smaller models typically produced less organized conditions with lower scores (average 0.31).

\begin{figure}[h]
\centering
\includegraphics[width=\textwidth]{combined_distributions_add.png}
\caption{Performance comparison of models with different scales across metrics.}
\label{fig:expand-distributions}
\end{figure}

The answer score distributions for larger models exhibit a distinctive bimodal pattern, with peaks at 0.5-0.7, notably higher than the single peak at 0.25 observed in smaller models. This bimodal distribution suggests that larger models can achieve substantially better performance on certain query types while maintaining baseline performance on others. Analysis of specific cases reveals that the higher peak (0.6-0.7) typically corresponds to responses to complex, multi-faceted queries, while the lower peak (0.5-0.6) aligns with performance on simpler, fact-based queries.

However, the relationship between model scale and performance is not uniformly positive across all metrics. While condition and answer scores show clear scaling benefits, citation accuracy improvements are less pronounced. Even the largest models in our evaluation show only modest improvements in citation scores (0.08-0.09) compared to smaller models (0.05-0.07), suggesting that citation generation may face fundamental challenges not easily addressed by increased model scale alone.

The performance patterns observed in larger models provide several insights into scaling behavior:

First, the improved condition scores in larger models primarily manifest in better structure and comprehensiveness rather than just higher accuracy. For example, when analyzing historical events, \texttt{GLM4-plus} consistently generates conditions that capture both immediate and long-term factors (average score 0.48), while smaller models tend to focus on more immediate causes (average score 0.33).

Second, the bimodal distribution in answer scores suggests that larger models are particularly effective at handling complex queries that require integration of multiple conditions. In the query ``How does climate change affect agriculture?'', \texttt{GPT4o} maintained consistent high performance (0.65) across different aspects (crop yields, water resources, farming practices), while smaller models showed more variable performance (0.25-0.35) across these aspects.

Third, the limited improvement in citation scores across model scales indicates that accurate source attribution remains challenging regardless of model size. This suggests that citation generation may require architectural innovations beyond simple scaling of existing models. Both \texttt{GPT4o} and \texttt{GLM4-plus} show similar patterns of citation errors to smaller models, primarily in terms of source conflation and imprecise attribution.

Importantly, we observe diminishing returns in performance improvements as model scale increases. The gap between the largest and smallest models in our study (approximately 0.15 in condition scores) is smaller than would be predicted by standard scaling laws \cite{kaplan2020scaling}. This suggests that current model architectures may be approaching practical limits in their ability to handle condition-based generation tasks, indicating a potential need for architectural innovations rather than just increased scale.

\subsection{Case studies}

Through detailed analysis of specific queries, we examine how different models handle condition generation and answer formulation in practice. Our case studies focus on the question ``Where is the TV show \textit{The Ranch} located?'', revealing systematic patterns in model behavior across different scales and architectures.

As shown in Table~\ref{tab:combined-analysis}, smaller models frequently generate conditions that fail to capture the key aspects of the query. \texttt{LLaMA3.1} produces irrelevant conditions such as ``Other types of ranches and related concepts remain undeveloped in terms of their broader societal implications'' (score: 0.11) and ``Movie ranches and TV series sets in California'' (score: 0.22). Similarly, \texttt{Gemma2}'s conditions like ``Definition of Ranching'' (score: 0.24) and ``Production of \textit{The Ranch} (2018 TV Series)'' (score: 0.38) demonstrate limited focus on the location-specific nature of the query.
\begin{table}[htbp]
\centering
\begin{tabular}{|l|l|p{6cm}|c|l|}
\hline
\textbf{Scale} & \textbf{Model} & \textbf{Generated Condition} & \textbf{G-Eval Score} & \textbf{Analysis} \\
\hline
\multicolumn{5}{|c|}{\textbf{Ground Truth Conditions}} \\
\hline
\multirow{3}{*}{Ground Truth} & GT1 & \multicolumn{3}{p{9cm}|}{The show \textit{The Ranch} is primarily set in a fictional small town called Garrison in Colorado. The show's story revolves around the Bennett family and their Iron River Ranch.} \\
\cline{2-5}
& GT2 & \multicolumn{3}{p{9cm}|}{While set in Colorado, the show was primarily filmed at a sound stage in Burbank, California. The town of Ouray, Colorado appears in the opening sequence.} \\
\cline{2-5}
& GT3 & \multicolumn{3}{p{9cm}|}{The show features both interior shots (filmed in California) and exterior establishing shots (filmed in Colorado).} \\
\hline
\multicolumn{5}{|c|}{\textbf{Model Evaluations}} \\
\hline
\multirow{8}{*}{Small Models} & LLaMA 3.1 & Other types of ranches and related concepts remain undeveloped in terms of their broader societal implications. & 0.11 & Completely irrelevant \\
\cline{2-5}
& LLaMA 3.1 & Movie ranches and TV series sets in California remain undeveloped.  & 0.22 & Incorrect context \\
\cline{2-5}
& Gemma 2 & Definition of Ranching & 0.24 & Generic definition \\
\cline{2-5}
& Gemma 2 & Production of \textit{The Ranch} (2018 TV Series) & 0.38 & Not location-focused \\
\cline{2-5}
& GLM4 & The term 'ranch' refers to land primarily used for raising grazing livestock and is a subtype of farm.  & 0.30 & Generic definition \\
\cline{2-5}
& GLM4 & Sable Ranch in Santa Clarita was a filming location used for various film and television series before being destroyed in a wildfire. & 0.17 & Wrong location \\
\cline{2-5}
& Qwen 2.5 & The destruction of Sable Ranch during the Sand Fire wildfire. & 0.13 & Wrong location \\
\cline{2-5}
& Qwen 2.5 & The plot and characters of the TV series \textit{The Ranch} (2006) & 0.35 & Not location-focused \\
\hline
\multirow{4}{*}{Large Models} & GPT4o & Setting of the TV show \textit{The Ranch} & 0.45 & Clear setting focus \\
\cline{2-5}
& GPT4o & Filming locations for \textit{The Ranch} & 0.44 & Location specific \\
\cline{2-5}
& GLM4-plus & Filming Location of \textit{The Ranch} & 0.41 & Direct focus \\
\cline{2-5}
& GLM4-plus & Setting of \textit{The Ranch} in Colorado & 0.53 & Abstract but accurate \\
\hline
\end{tabular}
\caption{Case study: comprehensive analysis of model responses vs ground truth for the query ``Where is the TV show \textit{The Ranch} located?''}
\label{tab:combined-analysis}
\end{table}

% \begin{table}[htbp]
% \centering
% \begin{tabular}{|l|p{6cm}|c|l|}
% \hline
% \textbf{Model} & \textbf{Generated Condition} & \textbf{G-Eval Score} & Analysis \\
% \hline
% LLaMA 3.1 & Other types of ranches and related concepts remain undeveloped in terms of their broader societal implications. & 0.11 & Completely irrelevant \\
% \hline
% LLaMA 3.1 & Movie ranches and TV series sets in California remain undeveloped. & 0.22 & Incorrect context, low relevance \\
% \hline
% Gemma 2 & Definition of Ranching & 0.24 & Generic definition, not location-specific \\
% \hline
% Gemma 2 & Production of \textit{The Ranch} (2018 TV Series) & 0.38 & Relevant to show but not location \\
% \hline
% GLM4 & The term 'ranch' refers to land primarily used for raising grazing livestock and is a subtype of farm. & 0.3 & Generic ranch definition, not show-specific \\
% \hline
% GLM4 & Sable Ranch in Santa Clarita was a filming location used for various film and television series before being destroyed in a wildfire. & 0.17 & Wrong filming location, below threshold \\
% \hline
% Qwen 2.5 & The destruction of Sable Ranch during the Sand Fire wildfire. & 0.13 & Wrong location\\
% \hline
% Qwen 2.5 & The plot and characters of the TV series \textit{The Ranch} (2006) & 0.35 & Show-related but not location-focused \\
% \hline
% \end{tabular}
% \caption{Analysis of Small Models' Generated Conditions}
% \label{tab:small-models}
% \end{table}

% In contrast, larger models show more focused condition generation, as evidenced in Table~\ref{tab:large-models}. \texttt{GPT4o} generates directly relevant conditions such as ``Setting of the TV show \textit{The Ranch}'' (score: 0.45) and ``Filming locations for \textit{The Ranch}'' (score: 0.44). Similarly, \texttt{GLM4-plus} produces focused conditions about ``Filming Location of \textit{The Ranch}'' (score: 0.41) and ``Setting of \textit{The Ranch}'' (score: 0.53).

% \begin{table}[htbp]
% \centering
% \begin{tabular}{|l|p{6cm}|c|l|}
% \hline
% Model & Generated Condition & G-Eval Score \\
% \hline
% GPT4o & Setting of the TV show \textit{The Ranch} & 0.45  \\
% \hline
% GPT4o & Filming locations for \textit{The Ranch} & 0.44  \\
% \hline
% GLM4-plus & Filming Location of \textit{The Ranch} & 0.41  \\
% \hline
% GLM4-plus & Setting of \textit{The Ranch} & 0.53  \\
% \hline
% \end{tabular}
% \caption{Advanced LLMs' Generated Conditions}
% \label{tab:large-models}
% \end{table}


% \begin{table}[htbp]
% \centering
% \begin{tabular}{|l|p{6cm}|}
% \hline
% Model & Ground Truth Condition \\
% \hline
% Ground Truth 1 & The show \textit{The Ranch} is an American sitcom. The show mainly revolves around the Bennett family and their ranch in Colorado. \\
% \hline
% Ground Truth 2 & The show is set in a fictional town in Colorado. The town of Ouray, Colorado, and its surrounding areas appear in the opening sequence. \\
% \hline
% Ground Truth 3 & The show \textit{The Ranch} is a Polish television comedy series. It follows Lucy Wilska, a Polish-American who inherits her grandmother's house in a small village in Poland. \\
% \hline
% \end{tabular}
% \caption{Ground Truth Conditions }
% \label{tab:ground-truth}
% \end{table}
To validate these patterns, we examined responses to similar queries. For instance, with the question ``Who set the fire in \textit{One Tree Hill}?'', we observe comparable behavior: smaller models generate tangential conditions about general fire safety or unrelated plot points (scores: 0.15-0.30), while larger models produce more focused conditions about ``Dan Scott's Memory of the Fire'' (score: 0.50) and ``Dan Scott's Confession and Guilt'' (score: 0.47).

Through these case studies, we identify clear score thresholds that correspond to different levels of response quality:

1. Scores below 0.20 consistently indicate irrelevant or incorrect responses, often failing to address the core query. For example, \texttt{LLaMA3.1}'s condition about ``broader societal implications'' (score: 0.11) completely misses the query's intent.

2. Scores between 0.20 and 0.35 typically represent partially relevant but imprecise responses. \texttt{Gemma2}'s generic ``Definition of Ranching'' (score: 0.24) exemplifies this category, touching on related concepts without addressing the specific question.

3. Scores between 0.35 and 0.50 indicate accurate but insufficiently detailed responses. \texttt{Qwen2.5}'s condition about ``plot and characters'' (score: 0.35) demonstrates relevance but lacks location specificity.

4. Scores above 0.50 represent high-quality, focused responses, as seen in \texttt{GLM4-plus}'s ``Setting of \textit{The Ranch}'' (score: 0.53).

These thresholds remain consistent across different queries and models, providing a reliable framework for evaluating condition quality. However, our analysis also reveals that even high-scoring conditions from larger models tend toward abstraction, potentially limiting their practical utility. This tendency toward abstraction represents a key challenge in condition generation that persists across model scales, suggesting that architectural improvements beyond simple scaling may be necessary for further advancement.


% \subsection{discussion}

% 2000 expansions
%大模型回答在复杂任务上答案相对准确,但整体可靠度下降[nature],大参数模型错误回答比例相对于小参数模型有所上升,即死不承认自己错了,甚至在简单任务会出现很多低级错误,比如GPT4在处理简单加法或者字谜任务错误率比小模型高15%,大部分原因是大模型不承认自己不知道或者拒绝回答而是瞎达,难度一致性,任务回避,提示稳定性,难度不一致现象. 对齐后,会出现减少回避回答,给出看似合理,但实际错误,而且用户无法甄别,而且人类对于任务的复杂和机器不是一个概念. larger and more instreuctable language model become less reliable