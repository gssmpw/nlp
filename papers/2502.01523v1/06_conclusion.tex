%主要总结这篇文章新定义了什么任务,同时基于提出的理论目前有啥缺陷,未来的研究方向有什么

\subsection{Conclusions}
In this work, we introduced CondAmbigQA, a novel framework and benchmark for addressing ambiguity in question answering through explicit condition identification. Our approach addresses a fundamental challenge in QA systems: the misalignment between implicit user assumptions and model capabilities that often leads to hallucinations and inconsistent responses.
Our experimental findings reveal several key insights. First, explicit condition identification significantly improves answer quality across different model architectures (scores ranging from $0.305$ to $0.317$), with condition-guided generation showing substantial improvements over standard RAG approaches.% (effect sizes $d=0.53$-$1.5$). 
Second, through scaling analysis, we observed that larger models demonstrate superior condition processing capabilities, suggesting that model scale remains a crucial factor in handling complex disambiguation tasks. Third, our human-LLM collaborative annotation process successfully produced high-quality, consistent data while minimizing both human subjectivity and model biases.
The CondAmbigQA framework advances beyond existing approaches \cite{min-etal-2020-ambigqa,stelmakh-etal-2022-asqa} in three key aspects: (1) introducing a theoretical foundation that identifies implicit conditions as the root cause of QA ambiguity, (2) developing a structured condition-based framework for systematic disambiguation, and (3) demonstrating an effective human-LLM collaborative annotation methodology. These contributions establish a new paradigm for handling ambiguity in QA systems, grounded in both theoretical understanding and practical implementation.
\subsection{Limitations}
\textbf{Dataset Constraints.} From a dataset perspective, our current collection of 200 annotated instances, while carefully curated, may not fully capture the diverse range of ambiguity patterns encountered in real-world scenarios. The \textit{human-LLM collaborative annotation} process, though designed to minimize biases, involves complex iterative calibration that may still introduce subtle inconsistencies in condition-answer relationships.

\textbf{Methodological Challenges.} Our evaluation framework, while comprehensive, may not fully capture the nuanced interactions between different types of conditions and their collective impact on answer quality. This challenge is particularly evident in cases where multiple conditions exhibit complex interdependencies, suggesting that the benefits of our approach may be partially dependent on underlying model capabilities.

\subsection{Future work}
\textbf{Enhanced Reasoning Architecture.} Our findings suggest that explicit \textit{condition representation} could serve as a foundation for enhancing language models' advanced reasoning capabilities. Future research should explore how condition-based frameworks can be integrated into \texttt{LLMs'} architecture to improve their \textit{logical reasoning} abilities, particularly in handling complex, multi-step deduction tasks. This could involve developing specialized \textit{attention mechanisms} that explicitly model condition dependencies and their logical implications.

\textbf{Multi-Agent Framework.} Beyond individual models, the condition framework could be extended to \textit{multi-agent systems}, where different agents specialize in condition identification, logical reasoning, and answer generation respectively. Such architectural innovations could significantly advance models' capabilities in tasks requiring sophisticated \textit{logical reasoning} and \textit{contextual understanding}, potentially offering a new paradigm for enhancing \texttt{LLMs'} higher-order cognitive abilities.