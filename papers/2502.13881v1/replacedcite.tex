\section{Related Work}
%In this section, we review the most closely related work. % and include additional related work in Appendix \ref{sec:addrelatedwork}.
% \vspace{-1em}
\subsection{Conversational Product Search}
Recently, there has been a growing interest regarding \ac{CPS} in the information retrieval community ____. In the early stage,____ proposed a first \ac{CPS} model by introducing a unified framework for conversational search and recommendation. %They proposed a first \ac{CPS} model that incorporates a multi-memory network architecture. 
This model engages users by posing questions about various aspects derived from user reviews and collects their feedback on these aspect values. Similarly,____ simulated conversations and solicited explicit user responses to aspect-value pairs mined from product reviews, with a particular emphasis on the importance of negative feedback in the context of \ac{CPS}. In contrast to this approach,____ focused on querying users about informative terms, typically entities, extracted from item-related descriptions and reviews. They introduced a sequential Bayesian method that employs cross-user duet training for enhancing \ac{CPS}. Based on that, ____ conducted an empirical study to measure user willingness and the accuracy of answers provided in question-based product search systems, shedding light on the effectiveness of existing question-based product search approaches. More recently,____ asked clarifying questions ____ on item aspects and proposed a \ac{CPS} model via representation learning. 

The aforementioned studies on \ac{CPS} all simulate user conversations when interacting with the product search system. That is, their conversations are based on template-based questions and simulated user answers. Although they demonstrate CPS is a promising research direction,
their deployment of simulated conversations is suboptimal as this is not a human-like setting____. This is mainly due to the lack of real conversations for product search. In this work, we fill this gap by proposing a dataset for \ac{CPS} with real human-like conversations. 
% \vspace{-1em}
\subsection{Conversational Datasets}
There are several conversational datasets available for different tasks, such as conversational recommendation____, and conversational search____. The majority of conversational recommendation datasets focus on the movie domain, which involves a collection of annotated dialogs where a seeker requests movie suggestions from the recommender____. In contrast, conversational search datasets____ focus on conversations to assist information seeking in the search scenarios. For instance, ____ and ____ released the CAsT and MISC datasets, which are created by volunteers or experts with 80 and 110 conversations, respectively. %____ proposed a dataset to explore how to ask clarifying questions for information-seeking by using crowdsourcing workers. 
____ proposed a conversational information-seeking dataset with the Wizard-of-Oz (WoZ) setup. In this paper, we use a human-human data collection protocol based on participants recruited from universities, which is more reliable ____. 
Compared with conversational search for locating relevant documents, product search focuses on locating the potential products for purchase____, which is a subset of relevant products and thus more challenging. Moreover, user queries in product search are often short and thus preference elicitation plays an important role in \ac{CPS} due to its goal of locating the purchased products. 

As for product search and e-commerce, there are only a few conversational datasets available. ____ built a conversational dataset for e-commerce constructed from Amazon reviews. However, they use simulated users and conversations. ____ introduced a conversational dataset from the E-commerce domain with user profiles, but they are in Chinese only and involve a single market only. ____ collected a dataset of multi-goal conversations for e-commerce, which is the closest dataset to ours. However, they are in a sole language, i.e., English, and involve a single market only. Also, they only contain less than 20 conversations per category. Different from the above publications, we build a \ac{CPS} dataset through real human-human conversations, which is on a suitable scale. Moreover, it can support dual markets and two languages, along with a knowledge graph.

\subsection{Conversational Recommender System for e-Commerce} Conversational recommender systems for e-commerce are also associated with product search. For instance,____ propose a unified paradigm for product search and recommendation. Conversational recommender systems utilize human-like natural language to deliver personalized and engaging recommendations through conversational interfaces like chatbots and intelligence assistants ____. In general, the existing research on conversational recommender systems for e-commerce is usually for anchor-based conversational recommender systems, which are in the form of ``system ask--user respond'' mode. In this context, the systems ask questions, prompting users to provide feedback, subsequently utilizing this user-provided information to refine their recommendations. They simulate conversations based on the predefined anchors to characterize items, including intent slots (e.g., item aspects and facets)____, entities____, and attributes____. 

In contrast to the existing work for conversational recommendation that mainly focuses on preference elicitation and top-k recommendation____, our \ac{CPS} focuses on item ranking and incorporates user intent detection, keyword extraction, system action prediction, and response generation. That is, they only cover part of aspects of the \ac{CPS} pipeline. 

\begin{figure}[t]
\centering
\includegraphics[width=\columnwidth]{pipline.pdf}
\caption{Study protocol overview.}
\label{fig:pipline}
\end{figure}

\begin{figure*}[tb]
\centering
\begin{subfigure}{0.228\textwidth}%0.245
    \includegraphics[width=\textwidth]{SearchInterface.pdf}
    \caption{Search interface.}
\end{subfigure}
\begin{subfigure}{0.62\textwidth}%0.66
    \includegraphics[width=\textwidth]{chat-labelling-system.pdf}
    \caption{Chat interface.}
\end{subfigure}
% \includegraphics[width=0.5\columnwidth]{SearchInterface.pdf}
% \includegraphics[width=1.4\columnwidth]{chat-labelling-system.pdf}
\caption{The interface of the chat room (the system role).}
\label{fig:sysinterface}
\end{figure*}

% \vspace{-1em}