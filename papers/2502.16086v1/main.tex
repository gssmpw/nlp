% This must be in the first 5 lines to tell arXiv to use pdfLaTeX, which is strongly recommended.
\pdfoutput=1
% In particular, the hyperref package requires pdfLaTeX in order to break URLs across lines.

\documentclass[11pt]{article}

% Change "review" to "final" to generate the final (sometimes called camera-ready) version.
% Change to "preprint" to generate a non-anonymous version with page numbers.
% \usepackage[review]{acl}
\usepackage{acl}
% Standard package includes
\usepackage{times}
\usepackage{latexsym}
\usepackage{enumitem}
\usepackage{amsmath}
\usepackage{tcolorbox}
\usepackage{tabularx} % 支持自动调整列宽
\usepackage{lipsum}   % 仅用于生成示例文本
\usepackage{bm} % 导入bm宏包

% For proper rendering and hyphenation of words containing Latin characters (including in bib files)
\usepackage[T1]{fontenc}
% For Vietnamese characters
% \usepackage[T5]{fontenc}
% See https://www.latex-project.org/help/documentation/encguide.pdf for other character sets

% This assumes your files are encoded as UTF8
\usepackage[utf8]{inputenc}

% This is not strictly necessary, and may be commented out,
% but it will improve the layout of the manuscript,
% and will typically save some space.
\usepackage{microtype}

% This is also not strictly necessary, and may be commented out.
% However, it will improve the aesthetics of text in
% the typewriter font.
\usepackage{inconsolata}

%Including images in your LaTeX document requires adding
%additional package(s)
\usepackage{graphicx}
\usepackage{multirow}
\usepackage{booktabs}
\usepackage{array}
\usepackage{float}


% If the title and author information does not fit in the area allocated, uncomment the following
%
%\setlength\titlebox{<dim>}
%
% and set <dim> to something 5cm or larger.

\title{Stealing Training Data from Large Language Models \\ in Decentralized Training through Activation Inversion Attack}

% Author information can be set in various styles:
% For several authors from the same institution:
% \author{Author 1 \and ... \and Author n \\
%         Address line \\ ... \\ Address line}
% if the names do not fit well on one line use
%         Author 1 \\ {\bf Author 2} \\ ... \\ {\bf Author n} \\
% For authors from different institutions:
% \author{Author 1 \\ Address line \\  ... \\ Address line
%         \And  ... \And
%         Author n \\ Address line \\ ... \\ Address line}
% To start a separate ``row'' of authors use \AND, as in
% \author{Author 1 \\ Address line \\  ... \\ Address line
%         \AND
%         Author 2 \\ Address line \\ ... \\ Address line \And
%         Author 3 \\ Address line \\ ... \\ Address line}

\author{Chenxi Dai\thanks{Equal contribution} \and
        Lin Lu\footnotemark[1] \and 
        Pan Zhou\thanks{Corresponding author} \\
  Huazhong University of Science of Technology \\
  \{dcx001,loserlulin,panzhou\}@hust.edu.cn}

% \usepackage[T1]{fontenc}
% \usepackage[utf8]{inputenc}
% \usepackage{authblk}

% \author[a]{Chenxi Dai}
% \author[a]{Lin Lu}
% \author[a]{Pan Zhou C \thanks{Corresponding author: panzhou@hust.edu.cn}}
% \affil[a]{Huazhong University of Science of Technology}




%\author{
%  \textbf{First Author\textsuperscript{1}},
%  \textbf{Second Author\textsuperscript{1,2}},
%  \textbf{Third T. Author\textsuperscript{1}},
%  \textbf{Fourth Author\textsuperscript{1}},
%\\
%  \textbf{Fifth Author\textsuperscript{1,2}},
%  \textbf{Sixth Author\textsuperscript{1}},
%  \textbf{Seventh Author\textsuperscript{1}},
%  \textbf{Eighth Author \textsuperscript{1,2,3,4}},
%\\
%  \textbf{Ninth Author\textsuperscript{1}},
%  \textbf{Tenth Author\textsuperscript{1}},
%  \textbf{Eleventh E. Author\textsuperscript{1,2,3,4,5}},
%  \textbf{Twelfth Author\textsuperscript{1}},
%\\
%  \textbf{Thirteenth Author\textsuperscript{3}},
%  \textbf{Fourteenth F. Author\textsuperscript{2,4}},
%  \textbf{Fifteenth Author\textsuperscript{1}},
%  \textbf{Sixteenth Author\textsuperscript{1}},
%\\
%  \textbf{Seventeenth S. Author\textsuperscript{4,5}},
%  \textbf{Eighteenth Author\textsuperscript{3,4}},
%  \textbf{Nineteenth N. Author\textsuperscript{2,5}},
%  \textbf{Twentieth Author\textsuperscript{1}}
%\\
%\\
%  \textsuperscript{1}Affiliation 1,
%  \textsuperscript{2}Affiliation 2,
%  \textsuperscript{3}Affiliation 3,
%  \textsuperscript{4}Affiliation 4,
%  \textsuperscript{5}Affiliation 5
%\\
%  \small{
%    \textbf{Correspondence:} \href{mailto:email@domain}{email@domain}
%  }
%}

\begin{document}
\maketitle
\begin{abstract}
Decentralized training has become a resource-efficient framework to democratize the training of large language models (LLMs). However, the privacy risks associated with this framework, particularly due to the potential inclusion of sensitive data in training datasets, remain unexplored. This paper identifies a novel and realistic attack surface: the privacy leakage from training data in decentralized training, and proposes \textit{activation inversion attack} (AIA) for the first time. AIA first constructs a shadow dataset comprising text labels and corresponding activations using public datasets. Leveraging this dataset, an attack model can be trained to reconstruct the training data from activations in victim decentralized training. We conduct extensive experiments on various LLMs and publicly available datasets to demonstrate the susceptibility of decentralized training to AIA. These findings highlight the urgent need to enhance security measures in decentralized training to mitigate privacy risks in training LLMs.
\end{abstract}


% Previous research has focused on fully trained and fixed models, whereas we explore a part of the model that changes dynamically during pipeline fine-tuning. To reduce the cost of the attack, we directly use pre-trained models as shadow models, and experiments validate the rationality and effectiveness of this approach. 

% \section{Introduction}
% Deep neural networks (DNNs), particularly large language models (LLMs)~\cite{gpt3,chen2023extending,mistral,gemma2}, have achieved remarkable success and found widespread applications across various domains due to their exceptional performance~\cite{li2024ecomgpt,wu2024chateda,lu2024chameleon}. However, the performance of these models is closely tied to both their size and the scale of the datasets they are trained on~\cite{bahri2024explaining}, which in turn leads to increasingly demanding requirements in terms of device memory and computational time. For instance, the training of the DeepSeek-V3 model, with 671 billion parameters, on a dataset of 14.8 trillion tokens required 2.664 million H800 GPU hours~\cite{liu2024deepseek}. As a result, distributed training across multiple devices has become a common approach for training modern large-scale models.

% To address this challenge, existing DNN training systems often employ model parallelism~\cite{Lepikhin2020Gshard,narayanan2021efficient,zheng2022alpa}, which divides a DNN into multiple segments and places them on devices with sufficient memory. Pipeline parallelism~\cite{huang2019gpipe,narayanan2019pipedream,narayanan2021memory}, on the other hand, assigns different stages of the model to different devices in a sequential manner. Since each device can simultaneously process different stages of different data, this approach significantly enhances resource utilization. Compared to traditional data parallelism~\cite{li2014scaling,luo2020prague}, pipeline parallelism alleviates the issue of memory constraints and computation bottlenecks, making it a prevalent paradigm for training large models today.

% Deep neural networks (DNNs), particularly large language models (LLMs)~\cite{gpt3,chen2023extending,mistral,gemma2}, have achieved significant success. However, their computational demands have increased exponentially due to the massive scale of their parameters and the high requirements for training data~\cite{naveed2023comprehensive,raiaan2024review}. For instance, the DeepSeek-V3 model, which trained 671 billion parameters on 14.8 trillion tokens, consumed 2.664 million H800 GPU hours~\cite{liu2024deepseek}. This exponential growth necessitates distributed training across multiple devices, and decentralized training based on pipeline parallelism has become a key solution. Unlike traditional data parallelism~\cite{li2014scaling,luo2020prague}, pipeline parallelism~\cite{narayanan2019pipedream} strategically divides model layers across devices, enabling concurrent processing of different data batches in successive stages. This approach not only optimizes memory utilization but also alleviates computational bottlenecks. Represented by frameworks such as GPipe~\cite{huang2019gpipe} and Megatron-LM~\cite{narayanan2021efficient}, this distributed training paradigm effectively balances resource limitations with training efficiency, forming the foundation of modern LLM development.

% However, the training paradigm of pipeline parallelism differs from that of data parallelism or local training, and introduces several new risks. Existing research~\cite{thorpe2023bamboo,jang2023oobleck,duan2024parcae} predominantly focuses on addressing fault tolerance related to hardware crashes in pipeline parallelism, largely overlooking the impact of human threats. While the potential of poisoning attacks in pipeline parallelism has been explored, these studies~\cite{lu2024position} typically assume the presence of an adversary capable of randomly controlling a specific stage within the pipeline. Such an adversary could manipulate activations or gradient values at that stage, potentially delaying or completely preventing model convergence. Although these manipulations can indeed disrupt the training process, they are generally easy to detect through standard anomaly detection methods. As a result, these attacks are unlikely to cause significant, long-term damage to the training process or the model itself.

% Despite this, there are subtler forms of threats in pipeline parallelism that require further attention. Inspired by leakage from gradients in federated learning~\cite{zhu2019deep,zhao2020idlg} and embedding inversion attacks~\cite{li2023sentence,chen2024text}, this paper introduces the first activation inversion attack under decentralized training. Unlike previous work, this attack does not require modifying the transmitted values during decentralized training; instead, it only requires access to the transmitted values to achieve the attack goal. Specifically, we consider a system in which a pipeline fine-tunes a large language model, with one of the stages being honest-but-curious. During the training process, this stage performs forward propagation and gradient calculation as usual but attempts to reconstruct the original training data from intermediate data. Since this stage only possesses a portion of the model, it cannot directly obtain activation values from sentence tokens, nor can it easily reconstruct data via gradients. Our attack aims to achieve the following two goals: 1) \textbf{the shadow model} is used to mimic the behavior of the victim model, even as the victim model evolves during fine-tuning; 2) \textbf{the attack model} is trained using a dataset constructed from the shadow model and then attempts to attack the victim model, attempting to reverse-engineer the original text from the activation values.

% To achieve the first goal, we used existing pre-trained models as shadow models and demonstrated the validity of this approach through experimentation. For the second goal, we employed a variety of model architectures as attack models to explore the effectiveness of text reconstruction.

% To validate the effectiveness of our attack, we conducted extensive experiments on three popular models: GPT2-XL~\cite{gpt2}, Bloom-7B1~\cite{bloom}, and LLaMA3-8B~\cite{llama3}. The experimental results show that the perplexity of the reconstructed text is consistently around 10, indicating that the reconstructed text closely resembles the original text. To further investigate the potential harm of activation inversion attacks, we constructed a Personally Identifiable Information (PII) dataset for privacy item extraction experiments. The results demonstrate that a large number of private items can be accurately reconstructed, with near 100\% recovery rates for the PII type of birthday and job.

\section{Introduction}

Large language models (LLMs)~\cite{gpt3, chen2023extending, mistral, gemma2} have demonstrated remarkable efficacy across diverse domains~\cite{li2024ecomgpt, wu2024chateda, lu2024chameleon} due to their advanced capabilities in semantic understanding and text generation. However, their emergent abilities follow the scaling law~\cite{bahri2024explaining, naveed2023comprehensive, raiaan2024review}, which leads to state-of-the-art LLMs typically comprising billions of parameters. For instance, the DeepSeek-V3~\cite{liu2024deepseek} model, with its 671 billion parameters, requires 2,664 million H800 GPU hours for training. This resource-intensive training and fine-tuning process presents significant barriers to the democratization of LLMs. As a result, decentralized training~\cite{yuan2022decentralized, ryabinin2023swarm} is gaining increasing attention as a promising solution to mitigate these resource challenges.

Decentralized training is mainly based on parallel training (e.g., \textit{pipeline parallelism}~\cite{narayanan2019pipedream}), which distributes training computations across heterogeneous computing devices (typically GPUs) in a pipeline, with each device acting as a distinct stage. Unlike traditional federated learning (FL), which is based on data parallelism~\cite{li2014scaling, luo2020prague}, pipeline parallelism allocates model layers across devices, facilitating the concurrent processing of multiple data batches over successive stages. During decentralized training, each stage transmits activations during forward propagation and gradients during backward propagation to iteratively update model parameters. This approach enhances memory utilization and alleviates computational bottlenecks. Frameworks such as GPipe~\cite{huang2019gpipe} and Megatron-LM~\cite{narayanan2021efficient} effectively balance resource constraints with training efficiency, supporting the democratization of LLMs.

As research on the robustness of decentralized training progresses, the security vulnerabilities of this framework have become increasingly evident. However, most existing studies~\cite{thorpe2023bamboo, jang2023oobleck, duan2024parcae} primarily focus on addressing fault tolerance issues related to hardware failures in pipeline parallelism, often neglecting the impact of human threats. While some research~\cite{lu2024position} has examined the role of attackers, demonstrating that malicious stages in decentralized training can significantly disrupt training outcomes and hinder model convergence, this study typically assumes that attackers can control any stage of decentralized training. Such strong assumptions about the attackers' capabilities make the attack methods impractical in real-world training scenarios, where tampering with transmitted values is highly likely to be detected by the training initiator. Furthermore, the above studies fail to address privacy risks, which could lead to more severe consequences~\cite{bethany2024large}.

Motivated by this gap, we aim to investigate whether malicious stages in decentralized training can steal privacy without disrupting the training process. However, implementing this privacy reconstruction attack presents a significant challenge: decentralized training differs substantially from traditional training methods, such as localized training or FL. In traditional training, attackers may have access to a complete model copy~\cite{li2023sentence,morris2023text} or its inputs and corresponding outputs~\cite{huang2024transferable}. In contrast, within the decentralized training, malicious stages can only access the transmitted values between stages. This raises a critical research question: \textit{How to steal privacy, such as training data, solely through transmitted values in decentralized training?}


To address this critical research question, this paper first introduces the \textbf{\textit{\underline{A}ctivation \underline{I}nversion \underline{A}ttack}} (AIA) targeting decentralized training. Specifically, we demonstrate how a malicious stage in decentralized training can steal training data by exploiting activations through a two-step process. In the first step: \textbf{Shadow Dataset Construction}, the attacker creates a shadow dataset of text-activation pairs using a public dataset, aiming to align the data distribution of the shadow dataset with that of the actual training process. In the second step: \textbf{Attack Model Training}, the attacker trains a generative model using the shadow dataset to learn the mapping from activations to text labels. The attacker then reconstructs the corresponding training data from victim activations. In summary, the contributions of this paper are as follows:


\begin{itemize}[nolistsep, leftmargin=*, topsep=0pt]

    \item We identify a novel attack surface, marking the first attempt to steal private training data within decentralized training frameworks.

    \item We propose a two-step attack framework, AIA, that steals training data through activations in decentralized training without detection.

    \item We conduct a comprehensive evaluation of the effectiveness of AIA, demonstrating its character-level capability for training data reconstruction. Specifically, AIA achieves 62\% accuracy in stealing private emails when fine-tuning GPT2-XL.
    
\end{itemize}


\section{Related Work}

\subsection{Decentralized Training Safety}

\citet{yuan2022decentralized} initially explores decentralized training for LLMs. Several studies then examine decentralized training in slow networks~\cite{ryabinin2023swarm, wang2023cocktailsgd} and explore the development of geo-distributed training systems tailored for LLMs~\cite{gandhi2024improving, tang2024fusionllm}. While safety concerns in decentralized training have been identified in previous works~\cite{tang2023fusionai, borzunov2022training}, most existing research focuses mainly on ensuring seamless pipeline operations on preemptible devices, employing techniques such as model backup and redundant computation~\cite{thorpe2023bamboo, jang2023oobleck}. \citet{lu2024position} comprehensively evaluate the potential threats in decentralized training. However, the proposed \textit{forward attack} can be easily mitigated by detection methods, making it impractical in real-world scenarios.


% Decentralized training has emerged as the mainstream approach for training large models. \citet{yuan2022decentralized} investigates model parallelism-based training of large models in heterogeneous environments. Some research examines distributed training in the presence of slow network conditions~\cite{ryabinin2023swarm,wang2023cocktailsgd}. Other efforts focus on pipeline training on preemptible devices, employing techniques such as model backup and redundant computation~\cite{thorpe2023bamboo,jang2023oobleck}. Several works explore the development of geo-distributed training systems tailored for large language models~\cite{gandhi2024improving,tang2024fusionllm}. Furthermore, \citet{tang2023fusionai} utilizes consumer-grade GPUs for training large models.


\subsection{Data Leakage from Transmitted Values}

\noindent{\textbf{Data leakage from gradients.}}
In the context of FL, researchers such as \citet{zhu2019deep} have explored deep gradient leakage attacks on both visual and language models. 
\citet{balunovic2022lamp} uses auxiliary language models to model prior probabilities, reducing the loss through alternating continuous and discrete optimization. \citet{gupta2022recovering} first recovers a set of words from gradients, and then reconstructs the sentence from this set of words using beam search. \citet{fowl2022decepticons} and \citet{boenisch2023curious} propose a powerful threat model in which the server is malicious and can manipulate model weights, easily reconstructing the data.
\citet{wu2023learning} proposes a simple adaptive attack method that can bypass various defense mechanisms, including differential privacy and gradient compression, and successfully reconstruct the original text.
% \citet{li2022you} leverages the hidden states of conversational models to conduct attribute inference attacks. Several studies assume that attackers can access the language model's output tokens and logits, which they exploit to induce the model to inadvertently leak private data.

% Large language models perform well across a variety of tasks but also expose increasing privacy risks~\cite{das2024security,yan2024protecting}. In the context of federated learning, researchers such as \citet{zhu2019deep} have studied deep gradient leakage attacks on both visual and language models, while \citet{gupta2022recovering} and \citet{balunovic2022lamp} specifically investigate deep gradient leakage attacks in language models. \citet{li2022you} utilizes the hidden states of conversational models to conduct attribute inference attacks. Several studies assume that attackers can access the language model and obtain output tokens and logits, which they use to induce the model to leak private data. Research by \citet{carlini2021extracting} shows that large models retain training data and leak private information. Additionally, some works generate malicious prompts to induce models to output private data\cite{huang2022large,nakka2024pii}. In addition, \citet{SPT} enhances data leakage by training a set of soft prompt tokens and adding them before the prompt template.

\noindent{\textbf{Data leakage from embeddings.}}
Another line of research focuses on embedding inversion attacks, where the attacker aims to reconstruct text from embedding representations. \citet{song2020information} reconstructs 50\%-70\% of the input words from embedding models. However, word-level information alone is insufficient to fully reconstruct privacy. \citet{li2023sentence} proposes a generative embedding inversion attack that reconstructs sentences similar to the original input from embeddings. \citet{morris2023text} utilizes an iterative correction approach to reconstruct text information. \citet{huang2024transferable} investigates a black-box attack scenario, reducing the discrepancy between the surrogate model and the victim model through adversarial training. These studies assume that the victim model is fully trained and static, allowing the attacker to access the input sentence embeddings from the victim model, build a shadow dataset, and then train an attack model to reconstruct the original text. However, in decentralized training settings, the malicious stage only has access to a portion of the model, and thus cannot directly access the victim model.

\section{Preliminaries}

\subsection{Threat Model}
\label{sec:threat_model}

\noindent{\textbf{Attack scenario.}}
% We propose a decentralized training framework consisting of $K$ computation stages, where $M_i$ represents the sub-layers (e.g., decode layers in LLMs) of the $i$-th stage. During training iteration $t$, $M_i$ transmits activations $a_i^{(t)}$ to $M_{i+1}$ and gradients $g_i^{(t)}$ to $M_{i-1}$.
We consider a decentralized training scenario where the user intends to fine-tune a pre-trained model ${M}_\text{pre}$ using their private dataset $\mathcal{D}_\text{vic}$, resulting in a fine-tuned model ${M}_\text{fine}$. The framework consists of $K$ stages, where $M_i$ represents the sub-layers (e.g., decode layers in LLMs) of the $i$-th stage. During training iteration $t$, $M_i$ transmits activations $\bm a_i^{(t)}$ to $M_{i+1}$ and gradients $\bm g_i^{(t)}$ to $M_{i-1}$. However, an unmonitored decentralized training framework may introduce an honest-but-curious stage as an attacker.

\noindent{\textbf{Attacker's goals.}}
The attacker's objective is to reconstruct character-level training data $\bm d^{(t)}$ from $\mathcal{D}_\text{vic}$ during iteration $t$ in victim decentralized training. Additionally, the attacker seeks to conceal their malicious activities, executing the attack without disrupting the training process to avoid detection by the training initiator or other detection mechanisms.

\noindent{\textbf{Attacker's knowledge.}}
We assume the attacker, as the $i_\text{att}$-th stage, has access to all information related to its own stage, including the sub-layers $M_{i_\text{att}}$ and transmitted data $\bm a_{i_\text{att}}$ and $\bm g_{i_\text{att}}$. This enables the attacker to infer the architecture of ${M}_\text{fine}$ based on the structure of $M_{i_\text{att}}$. However, the attacker is assumed to have no access to other training-related information, such as transmitted data between benign stages or auxiliary information about the training data. This assumption is realistic, as it facilitates the deployment of this attack in real-world decentralized training environments.


\subsection{Motivation}
\label{sec:act_cos}

In Section \ref{sec:threat_model}, it is established that attackers can only reconstruct training data through the transmitted values during the victim model's training process, such as activations and gradients. This section discusses the challenges of using gradients to conduct such attacks and explores the feasibility of using activations to achieve similar objectives.

% In decentralized training, using gradients to steal training data faces a major challenge: The unknowability of global gradients. In decentralized training, the gradients received at each stage correspond to the current sub-layers, which significantly differs from previous studies on gradient-based privacy leakage that rely on global gradients.

In decentralized training, traditional deep gradient leakage attacks encounter a significant limitation: the unavailability of the global model and global gradients. Previous researches~\cite {zhu2019deep, gupta2022recovering, balunovic2022lamp} focus on training or searching for a set of texts that, through the victim model’s gradient, approximate the leaked gradient to reconstruct private data. However, in decentralized training, each stage only has access to a partial model and gradients, making it difficult to reconstruct data through gradients.

% In contrast, to further explore changes in activations for the same data before and after fine-tuning LLMs, we have conducted a preliminary experiment: As shown in Figure~\ref{fig:layer_idx_act_cos}, we fine-tune three common LLMs using four different datasets and record the cosine similarity of activations for the same data sample before and after fine-tuning (specific experimental settings can be found in Section 5.1). The results show that as the layer index increases, the changes in the decoder layers closer to the lm\_head layer become more pronounced. Nevertheless, the activation similarities in the final layers still exceed 50\%. During fine-tuning, we observed that the decoder layers in earlier stages show minimal variation, with activation similarities nearly reaching 100\%. This demonstrates that activations are highly correlated with the training data, making it feasible to use activations to steal training data.

In contrast, reconstructing data using the intermediate outputs of the victim model is much more straightforward, as these intermediate outputs can be directly used as inputs to train the attack model~\cite{pasquini2021unleashing,li2023sentence}. Inspired by this, we examine the cosine similarity between $\bm a_i^{(t)}$ for $\bm d^{(t)}$ in ${M}_\text{pre}$ and ${M}_\text{fine}$ across layer index $i$ (experimental details can be found in Section \ref{sec:experiment_setup}). As shown in Figure \ref{fig:layer_idx_act_cos}, activation similarity in early layers approaches 100\%, while similarity in later layers remains above 50\%. These results suggest that the activations of the same data exhibit minimal variation before and after fine-tuning, indicating a strong correlation between activations and the training data. This preliminary experiment provides key insights for our attacks in Section \ref{sec:AIA}.


% The most critical aspect of activation inversion attacks is ensuring that the features of the shadow activations closely resemble those of the victim activations. This similarity is essential for the trained attack model to have transferability. Figure \ref{fig:layer_idx_act_cos} illustrates the cosine similarity of activations across different layers before and after fine-tuning on the GPT2-XL, Bloom-7B1, and LLaMA3-8B pre-trained models using various datasets. It is observed that as the layer index increases, the changes in the decoder layers nearer to the lm\_head layer become more pronounced. Nevertheless, the activation similarities in the final layers still exceed 50\%. During the fine-tuning process, we note that the decoder layers in the earlier stages show minimal variation, with activation similarities nearly reaching 100\%. This indicates that directly using a pre-trained model as the shadow model without additional training can still produce shadow activations closely resembling the victim activations.
\begin{figure}[t]
  \includegraphics[width=\linewidth]{figures/layer_idx_cos.pdf} 
  \caption {Cosine similarity between activations for the same data in the pre-trained model and the fine-tuned model across layer index.}
  \label{fig:layer_idx_act_cos}
  \vspace{-1em}
\end{figure}


\section{AIA: \textit{Activation Inversion Attack}}
\label{sec:AIA}
\begin{figure*}[t]
  \includegraphics[width=\textwidth]{figures/Figure_2.pdf}
  % \caption{Overview of the activation inversion attack (AIA). In an honest-but-curious decentralized training system, the victim model $M_{\text{vic}}$ is fine-tuned using its private data $\mathcal{D}_{\text{vic}}$. At one stage of the pipeline, which is set to be "curious," intermediate activation values $\mathcal{D}_{\text{vic}}$ during training are recorded, and shadow activations $\mathcal{D}_{\text{sha}}$ are acquired from the shadow model $M_{\text{sha}}$. These activations are then used to train an attack model $M_{\text{att}}$, which attempts to invert the private data.}
  \caption{Overview of Activation Inversion Attack (AIA). In a decentralized training system, the victim model $M_{\text{vic}}$ undergoes fine-tuning using private data $\mathcal{D}_{\text{vic}}$, which may contain personally identifiable information values (highlighted in yellow). An honest-but-curious attacker controlling the $i_{\text{att}}$-th stage of the pipeline: (1) records intermediate activation values $\bm a_{i_\text{att}-1}^{(t)}$ captured during the training process, and (2) collects shadow activations $\mathcal{D}_{\text{sha}}$ from the shadow model $M_{\text{sha}}$ to train the attack model $M_{\text{att}}$. Finally, the attacker uses $M_{\text{att}}$ to reconstruct the private data $\mathcal{D}_{\text{vic}}$, with the red and purple text representing precisely recovered and mostly recovered PII data, respectively.}
  \label{fig:system}
  \vspace{-1em}
\end{figure*}


We introduce AIA, a framework for training data reconstruction through activations in decentralized training. During the victim model training, an attacker at the $i_\text{att}$-th stage has access to the activations $\bm a_{i_\text{att}-1}^{(t)}$ passed from $M_{i_\text{att}-1}$ during forward propagation. We denote the mapping function from the original training data $\bm d_\text{vic}^{(t)}$ to $\bm a_{i_\text{att}-1}^{(t)}$ as $f_{[1:i_\text{att}-1]}^{(t)}(\cdot)$. Therefore, we can conclude that: 
$$
\bm a_{i_\text{att}-1}^{(t)}=f_{[1:i_\text{att}-1]}^{(t)}(\bm d_\text{vic}^{(t)})
$$
The attacker's goal can thus be simplified to constructing a mapping function $\phi \approx  (f_{[1:i_\text{att}-1]}^{(t)})^{-1}(\cdot)$ that reconstructs $\bm d_\text{vic}^{(t)}$ from $\bm a_{i_\text{att}-1}^{(t)}$. AIA adopts a learning-based approach by training a generative model to perform this reconstruction. In simple terms, AIA consists of two steps: (1) \textbf{Shadow Dataset Construction}: The attacker first generates a shadow dataset containing text labels and corresponding activations leveraging a public dataset. (2) \textbf{Attack Model Training}: The attacker then uses $\mathcal{D}_\text{sha}$ to train a generative attack model ${M}_\text{att}$ that learns the mapping function $\phi$. Finally, the attacker inputs the actual activations transmitted during the victim model training into ${M}_\text{att}$ to reconstruct the training data. We provide a detailed description of these two steps in the following.



% \subsection{Pipeline Parallelism Training}
% Consider a pipeline training system in which a language model is partitioned into multiple segments, with each segment trained at a different stage. Except for the first and last stages, one intermediate stage is "curious." This curious stage does not directly interfere with the model's training process, as such actions would be easily detectable. Instead, it is primarily interested in the victim's private training data. When the victim fine-tunes the language model with their own data within this system, the curious stage covertly stores the intermediate activation values during training. These activations are subsequently used to invert and reconstruct the user’s original dataset. This process does not affect the model’s training performance, but it does require additional storage space, meaning that the user remains unaware of this activity.

% \subsection{Shadow Model}
% The primary goal of the shadow model is to generate shadow activations from the attacker's shadow dataset, ensuring that their distribution closely mirrors that of the victim activations. However, since the fine-tuning process of the victim model is dynamic, the victim activations change with each training iteration. This presents a significant challenge: maintaining the consistency of the shadow activations with the victim activations is nearly impossible. First, the attacker has no access to the victim's private data, making it difficult to align the shadow model’s activations with the victim’s. Second, fine-tuning the shadow model simultaneously with the fine-tuning of the victim model incurs substantial computational costs. The difficulty in achieving precise alignment between shadow and victim activations underscores the inherent challenges in conducting an effective activation inversion attack.

% Fortunately, we observe that pre-trained models released by various institutions and organizations have undergone thorough training, demonstrating impressive generalization capabilities. As a result, when users fine-tune these models on their own private data, the changes in activations are relatively small. This observation allows us to directly utilize the pre-trained weights of various models from Hugging Face as the shadow model. In other words, we do not need to invest any additional effort in fine-tuning the shadow model, significantly reducing the cost of the attack. The rationale behind using pre-trained model weights directly as the shadow model is further elaborated through experiments in Section \ref{sec:act_cos}.

\subsection{Step 1: Shadow Dataset Construction}

Since the attacker cannot access $\mathcal{D}_\text{vic}$, a straightforward approach is to construct a shadow dataset $\mathcal{D}_\text{sha}$ using a public dataset $\mathcal{D}_\text{pub}$. Specifically, we use the frozen pre-trained LLM $M_\text{pre}$ as the shadow model $M_\text{sha}$, with the same type of the victim model, to generate shadow activations $\bm a_\text{sha}$, i.e., 
$$
\bm a_\text{sha}=M_{\text{sha}[1:i_\text{att} -1]}(\bm d_\text{pub})
$$
where $\bm d_\text{pub} \in \mathcal{D}_\text{pub}$. The rationale for this approach is analyzed in Section \ref{sec:act_cos}: the generalizability of $M_\text{pre}$ ensures that the activations remain relatively stable when fine-tuning the victim model $M_\text{vic}$ on $\mathcal{D}_\text{vic}$, allowing us to directly leverage the pre-trained weights from HuggingFace as $M_\text{sha}$. In other words, no additional effort is required to train $M_\text{sha}$, significantly reducing the cost of AIA.

\subsection{Step 2: Attack Model Training}
Next, we focus on training ${M}_\text{att}$ using the shadow dataset $\mathcal{D}_\text{sha}=\{(\bm a_\text{sha}, \bm d_\text{pub})\}$. ${M}_\text{att}$ is designed to take activations as input and output the distribution probabilities of the generated text. It consists of a set of decoder layers and an \texttt{lm\_head} layer. Structurally, it differs from a standard language model by the absence of the initial embedding layer. 
% In terms of training methodology, the loss function is set for a text reconstruction task, rather than a text generation task commonly used in language models. 
% The training objective is defined as:
Similar to the recent work~\cite{li2023sentence}, the training objective is to minimize the standard language model loss using teacher forcing~\cite{williams1989learning}:
$$
  L = - \sum_{k=1}^{N} \log P(y_k | x_1, x_2, \dots, x_{k-1})
$$
where $y_k$ is the target word, and $x_i$ represent the input activations. 
% This approach aims to reconstruct the original text, rather than generate new content as in traditional language modeling tasks. 
Finally, we input the activations $\bm a_{i_\text{att}-1}^{(t)}$ to ${M}_\text{att}$ and obtain $\bm d_\text{vic}^{(t)}$.



% \section{Challenges}
% \subsection{Victim Activations vs Shadow Activations}
% \label{sec:act_cos}
% The most critical aspect of activation inversion attacks is ensuring that the features of the shadow activations closely resemble those of the victim activations. This similarity is essential for the trained attack model to have transferability. Figure \ref{fig:layer_idx_act_cos} illustrates the cosine similarity of activations across different layers before and after fine-tuning on the GPT2-XL, Bloom-7B1, and LLaMA3-8B pre-trained models using various datasets. It is observed that as the layer index increases, the changes in the decoder layers nearer to the lm\_head layer become more pronounced. Nevertheless, the activation similarities in the final layers still exceed 50\%. During the fine-tuning process, we note that the decoder layers in the earlier stages show minimal variation, with activation similarities nearly reaching 100\%. This indicates that directly using a pre-trained model as the shadow model without additional training can still produce shadow activations closely resembling the victim activations.
% \begin{figure}[t]
%   \includegraphics[width=\linewidth]{figures/layer_idx_cos.pdf} 
%   \caption {Cosine Similarity of Activations Before and After Fine-Tuning Across Layer Indices.}
%   \label{fig:layer_idx_act_cos}
% \end{figure}






\section{Experiments}
\subsection{Experimental Setup}
\noindent\textbf{Victim models.}
\label{sec:experiment_setup}
We conduct experiments on three models: GPT2-XL~\cite{gpt2}, Bloom-7B1~\cite{bloom}, and LLaMA3-8B~\cite{llama3}, which have 48, 30, and 32 decoder layers, respectively. We directly download the pre-trained models from HuggingFace and use them as $M_\text{sha}$ to collect $\mathcal{D}_\text{sha}$. 
To investigate the effects of AIA under extreme conditions, we fine-tune $M_\text{vic}$ for 5 epochs on the corresponding dataset to induce overfitting on the privacy data, thereby maximizing the feature gap between $\mathcal{D}_\text{vic}$ and $\mathcal{D}_\text{sha}$.
% To explore the effects of AIA under extreme conditions, we fine-tune $M_\text{vic}$ for 5 epochs on the corresponding dataset to induce overfitting on the clean data. 
The training process is divided into 6 stages, with the assumption that the third stage is malicious. 
% Therefore, unless otherwise specified, all experiments are conducted at the one-third point of each model’s layers. 
The architecture of the attack model is identical to that of the victim model, with all attack models set to 12 decoder layers. 
% During the training of the attack model, the sequence length is set to 160. For fine-tuning the victim models, the sequence length is set to 1600 for LLaMA3-8B and Bloom-7B1, and 800 for GPT2-XL. The AdamW optimizer is used for all training and fine-tuning processes, with learning rates set to 5e-5 for GPT2-XL and Bloom-7B1, and 7e-5 for LLaMA3-8B, along with an epsilon value of 1e-8.

\noindent\textbf{Datasets.}
%Large language models are typically trained for text generation tasks using an autoregressive approach. 
% We use WikiText~\cite{wikitext} as the attacker's known dataset $\mathcal{D}_\text{pub}$ to construct the shadow dataset $\mathcal{D}_\text{sha}$. The WikiText dataset is a collection of high-quality, clean, and large-scale English text extracted from Wikipedia articles. The victim datasets $\mathcal{D}_\text{vic}$ include ArXiv, OpenWebText~\cite{openwebtext}, The Pile~\cite{pile}, and a public PII dataset\footnote{https://github.com/zzzzsdaw/PII-dataset} containing sensitive information. The ArXiv dataset is a large-scale collection of scientific papers from arXiv. The OpenWebText dataset is a high-quality, large-scale corpus of English web content curated from URLs shared on Reddit with high karma. The Pile is a diverse, 800GB English text dataset designed for training large language models, combining content from 22 high-quality sources, including books, academic papers, code, and web text. The PII dataset consists of 1,000 instances of sensitive information and includes 10 personally identifiable information (PII) types, including \textit{phone numbers}, \textit{email addresses}, and \textit{home addresses}, in a structured format. Figure \ref{fig:PII_data_example} presents an example of a PII data item. These data are randomly generated using regular expressions and do not represent real private information.
We use the WikiText~\cite{wikitext} dataset as the attacker's known dataset $\mathcal{D}_\text{pub}$ to construct the shadow dataset $\mathcal{D}_\text{sha}$. The victim datasets $\mathcal{D}_\text{vic}$ include ArXiv, OpenWebText~\cite{openwebtext}, The Pile~\cite{pile}, and a public PII dataset\footnote{https://github.com/zzzzsdaw/PII-dataset}, which contains sensitive information.  An example of a PII data item is shown in Figure \ref{fig:PII_data_example}.

\begin{figure}[t]
  \includegraphics[width=\columnwidth]{figures/PII.pdf}
  \caption{An example of PII data and baseline attacks. The private data includes information such as names, phone numbers, and email addresses. The True-Prefix attack leverages other private attributes to prompt the model to generate the target private attribute, while the SPT attack employs a trained soft prompt added before the query template to extract private information.}
  \label{fig:PII_data_example}
  \vspace{-1em}
\end{figure}


\noindent\textbf{Baselines.}
In the privacy leakage experiments, we adopt the following two methods as baselines. The two methods do not apply to decentralized training, we use them solely for comparison to illustrate the potential risks of our attack. Their attack examples can be seen in Figure \ref{fig:PII_data_example}.

\begin{itemize}[nolistsep, leftmargin=*, topsep=0pt]

    \item \textit{True-Prefix Attack}~\cite{true-prefix} utilizes real prefixes from $\mathcal{D}_\text{vic}$ to prompt the model. In our experiments, we use real PII data of other types within each PII item as the prompt, attempting to induce the model to output the value of the target PII type.

    \item \textit{SPT Attack}~\cite{SPT} trains an additional set of prompt embeddings, which are appended to the original query template. We train the prompt embeddings using 64 PII data pairs, during which the victim model remains frozen and does not require gradient updates.
    
\end{itemize}



% \noindent\textit{True-Prefix Attack~\cite{true-prefix}.} The true prefix attack utilizes real prefixes from the fine-tuning dataset to prompt the model. In our experiments, we use real PII data of other types within each PII item as the prompt, attempting to induce the model to output the value of target PII type.

% \noindent\textit{SPT Attack~\cite{SPT}.} The SPT attack trains an additional set of prompt embeddings, which are appended to the original query template. We train the prompt embeddings using 64 PII data pairs, during which the victim model remains frozen and does not require gradient updates.


\noindent\textbf{Evaluation metrics.}
To evaluate the quality of text reconstruction, we employ the following four metrics. 


\begin{itemize}[nolistsep, leftmargin=*, topsep=0pt]

    \item \textit{Perplexity}~\cite{perplexity} assesses the model's capability by measuring the probability distribution of its outputs, with lower values indicating better performance.

    \item \textit{ROUGE}~\cite{rouge} measures the similarity between the generated text and reference text by comparing overlapping words or phrases.

    \item \textit{BLEU}~\cite{bleu} evaluates the similarity between generated text and reference text based on n-gram overlap and is commonly used in machine translation tasks.

    \item \textit{Embedding cosine similarity} calculates the semantic similarity between the generated text and reference text using the all-MiniLM-L6-v2 model\footnote{https://huggingface.co/sentence-transformers/all-MiniLM-L6-v2}~\cite{minilmv2}.
    
\end{itemize}

% \textbf{Perplexity}~\cite{perplexity} assesses the model's capability by measuring the probability distribution of its outputs, with lower values indicating better performance. \textbf{ROUGE}~\cite{rouge} measures the similarity between the generated text and reference text by comparing overlapping words or phrases. \textbf{BLEU}~\cite{bleu} evaluates the similarity between generated text and reference text based on n-gram overlap and is commonly used in machine translation tasks. \textbf{Embedding cosine similarity} calculates the semantic similarity between the generated text and reference text using the MiniLM model~\cite{minilmv2}.

% In the privacy leakage experiments, we evaluate the attack success rates of our method and two baselines in exactly matching the value of the target PII type. Precise matching is defined as the ability to output numbers and letters in the correct sequence while ignoring spaces and special characters, which do not affect the evaluation of precision.

In the privacy leakage experiments, we evaluate the \textit{attack success rate (ASR)} of our AIA method and two baselines in precisely recovering the values of the target PII types. Precise recovery is defined as correctly outputting the digits and letters in the correct order. During the matching process between the generated data and the original private data, spaces and special characters, such as '-', are ignored, as they do not affect the identification of private data values. The \textit{ASR} is calculated as the ratio of the number of precisely recovered data entries to the total amount of data.


% \section{Results}
\subsection{Text Reconstruction}
% tables.tex

\begin{table*}[ht]
\centering
\caption{Text reconstruction performance of GPT2-XL, Bloom-7B1, and LLaMA3-8B on four datasets. For all metrics except PPL, higher values indicate better performance.}
\label{tab:base_result}
\resizebox{0.94\textwidth}{!}{\begin{tabular}{ccc ccc ccc c}
\toprule[2pt]
\multirow{2}{*}{\textbf{Victim Model}} & \multirow{2}{*}{\textbf{Dataset}} & \multirow{2}{*}{\textbf{PPL}} & \multicolumn{3}{c}{\textbf{ROUGE}} & \multicolumn{3}{c}{\textbf{BLEU}} & \multirow{2}{*}{\textbf{COS}} \\ 
\cmidrule(lr){4-6} \cmidrule(lr){7-9}
                              &                          &                      & \textbf{ROUGE-1}   & \textbf{ROUGE-2}   & \textbf{ROUGE-L}    & \textbf{BLEU-1}    & \textbf{BLEU-2}    & \textbf{BLEU-4}    &                      \\ \hline
\multirow{4}{*}{GPT2-XL}      & PIIs                     & 3.73                 & 0.84     & 0.74     & 0.84      & 0.77     & 0.71     & 0.59     & 0.89                 \\
                              & openwebtext              & 3.09                 & 0.95     & 0.90     & 0.95      & 0.88     & 0.84     & 0.77     & 0.94                 \\
                              & arxiv                    & 5.43                 & 0.92     & 0.85     & 0.92      & 0.81     & 0.75     & 0.64     & 0.92                 \\
                              & pile                     & 1.65                 & 0.98     & 0.95     & 0.98      & 0.95     & 0.93     & 0.89     & 0.97                 \\ \hline
\multirow{4}{*}{Bloom-7B1}    & PIIs                     & 14.82                & 0.80     & 0.67     & 0.80      & 0.67     & 0.60     & 0.47     & 0.89                 \\
                              & openwebtext              & 4.64                 & 0.95     & 0.92     & 0.95      & 0.89     & 0.86     & 0.80     & 0.95                 \\
                              & arxiv                    & 15.45                & 0.91     & 0.83     & 0.90      & 0.77     & 0.70     & 0.56     & 0.90                 \\
                              & pile                     & 2.09                 & 0.97     & 0.95     & 0.97      & 0.95     & 0.93     & 0.90     & 0.95                 \\ \hline
\multirow{4}{*}{LLaMA3-8B}    & PIIs                     & 7.36                 & 0.80     & 0.67     & 0.79      & 0.73     & 0.66     & 0.54     & 0.77                 \\
                              & openwebtext              & 6.50                 & 0.93     & 0.88     & 0.93      & 0.88     & 0.84     & 0.77     & 0.88                 \\
                              & arxiv                    & 9.26                 & 0.88     & 0.78     & 0.88      & 0.80     & 0.73     & 0.60     & 0.83                 \\
                              & pile                     & 2.18                 & 0.96     & 0.93     & 0.96      & 0.94     & 0.92     & 0.89     & 0.92                 \\ \bottomrule[1.5pt]
\end{tabular}}
\vspace{-1em}
\end{table*}

% \begin{table*}[ht]
% \centering
% \caption{Text reconstruction performance of GPT-2-XL, Bloom-7B1, and LLaMA3-8B on four datasets. For all metrics except PPL, higher values indicate better performance.}
% \label{tab:base_result}
% \resizebox{\textwidth}{!}{
% \begin{tabular}{ccc|ccc|ccc|c}
% \toprule[2pt]
% % \hline
% \multirow{2}{*}{Victim Model} & \multirow{2}{*}{Dataset} & \multirow{2}{*}{PPL} & \multicolumn{3}{c|}{Rouge}      & \multicolumn{3}{c|}{Bleu}      & \multirow{2}{*}{Cos} \\ \cline{4-9}
%                               &                          &                      & Rouge1   & Rouge2   & RougeL    & Bleu1    & Bleu2    & Bleu4    &                      \\ \hline
% \multirow{4}{*}{Gpt2-xl}      & PIIs                     & 3.73                 & 0.84     & 0.74     & 0.84      & 0.77     & 0.71     & 0.59     & 0.89                 \\
%                               & openwebtext              & 3.09                 & 0.95     & 0.90     & 0.95      & 0.88     & 0.84     & 0.77     & 0.94                 \\
%                               & arxiv                    & 5.43                 & 0.92     & 0.85     & 0.92      & 0.81     & 0.75     & 0.64     & 0.92                 \\
%                               & pile                     & 1.65                 & 0.98     & 0.95     & 0.98      & 0.95     & 0.93     & 0.89     & 0.97                 \\ \hline
% \multirow{4}{*}{Bloom-7b1}    & PIIs                     & 14.82                & 0.80     & 0.67     & 0.80      & 0.67     & 0.60     & 0.47     & 0.89                 \\
%                               & openwebtext              & 4.64                 & 0.95     & 0.92     & 0.95      & 0.89     & 0.86     & 0.80     & 0.95                 \\
%                               & arxiv                    & 15.45                & 0.91     & 0.83     & 0.90      & 0.77     & 0.70     & 0.56     & 0.90                 \\
%                               & pile                     & 2.09                 & 0.97     & 0.95     & 0.97      & 0.95     & 0.93     & 0.90     & 0.95                 \\ \hline
% \multirow{4}{*}{Llama3-8b}    & PIIs                     & 7.36                 & 0.80     & 0.67     & 0.79      & 0.73     & 0.66     & 0.54     & 0.77                 \\
%                               & openwebtext              & 6.50                 & 0.93     & 0.88     & 0.93      & 0.88     & 0.84     & 0.77     & 0.88                 \\
%                               & arxiv                    & 9.26                 & 0.88     & 0.78     & 0.88      & 0.80     & 0.73     & 0.60     & 0.83                 \\
%                               & pile                     & 2.18                 & 0.96     & 0.93     & 0.96      & 0.94     & 0.92     & 0.89     & 0.92                 \\ \bottomrule[1.5pt]

% \end{tabular}
% }
% \end{table*}
Table~\ref{tab:base_result} presents the performance of AIA across different victim LLMs and datasets. The results indicate that the perplexity of the generated sentences remains below 20, with most values under 10, suggesting that the reconstructed text is relatively fluent and closely aligns with the original fine-tuning data. Both ROUGE-1 and BLEU-1 scores exceed 0.7, with the highest result reaching nearly 0.95, which confirms that the majority of words from the original fine-tuning data are accurately recovered. ROUGE-L scores are generally higher than ROUGE-2, indicating that the generated text maintains high global similarity while exhibiting slightly lower local continuity. However, this slight discontinuity in certain lexical elements has minimal impact on human readability. We further compute the cosine similarity between the embeddings of the generated text and the original text, with values ranging from 0.77 to 0.96, confirming a high level of semantic similarity. These results validate the effectiveness of AIA in reconstructing the original fine-tuning data.

\subsection{Privacy Leakage}
% \begin{table}[ht]
\centering
\caption{Comparison of the ASR between our method and baselines in stealing phone and email data.}
\label{tab:pii_compare}
\resizebox{0.8\columnwidth}{!}{
\begin{tabular}{cccc}
\toprule[2pt]
\multirow{2}{*}{Victim Model} & \multirow{2}{*}{method} & \multicolumn{2}{c}{ASR} \\
                              &                         & phone      & email      \\ \hline
\multirow{3}{*}{Gpt2-xl}      & True prefix             & 0          & 0.04       \\
                              & SPT                     & 0          & 0.02       \\
                              & ours                    & 0.25       & 0.55       \\ \hline
\multirow{3}{*}{Bloom-7b1}    & True prefix             & 0.01       & 0.18       \\
                              & SPT                     & 0          & 0.10       \\
                              & ours                    & 0.42       & 0.62       \\ \hline
\multirow{3}{*}{Llama3-8b}    & True prefix             & 0          & 0          \\
                              & SPT                     & 0          & 0          \\
                              & ours                    & 0.16       & 0.42       \\ \bottomrule[1.5pt]
\end{tabular}
}
\end{table}
\begin{table*}[ht]
    \centering
    \begin{minipage}{0.285\textwidth}
        \centering
        \makeatletter\def\@captype{table}\makeatother
        \caption{Comparison of the ASR between our AIA method and baselines in stealing phone and email data.}
\label{tab:pii_compare}
        \resizebox{\textwidth}{!}{
        \begin{tabular}{cccc}
\toprule[2pt]
\multirow{2}{*}{\textbf{Victim Model}} & \multirow{2}{*}{\textbf{Method}} & \multicolumn{2}{c}{\textbf{ASR}} \\
                              &                         & \textbf{phone}      & \textbf{email}      \\ \hline
\multirow{3}{*}{GPT2-XL}      & True-Prefix             & 0          & 0.04       \\
                              & SPT                     & 0          & 0.02       \\
                              & AIA(ours)                    & 0.25       & 0.55       \\ \hline
\multirow{3}{*}{Bloom-7B1}    & True-Prefix             & 0.01       & 0.18       \\
                              & SPT                     & 0          & 0.10       \\
                              & AIA(ours)                    & 0.42       & 0.62       \\ \hline
\multirow{3}{*}{LLaMA3-8B}    & True-Prefix             & 0          & 0          \\
                              & SPT                     & 0          & 0          \\
                              & AIA(ours)                    & 0.16       & 0.42       \\ \bottomrule[1.5pt]
\end{tabular}
}
    \end{minipage}%
    \hfill % 自动填充空间,避免表格重叠
    \begin{minipage}{0.665\textwidth}
        \setcounter{table}{4}
        \centering
        \makeatletter\def\@captype{table}\makeatother
        \caption{The impact of attack model architecture on the attack performance of AIA. Each attack model is configured with 6 decoder layers. The results are presented in terms of perplexity.}
\label{tab:attack_model}
        \resizebox{\textwidth}{!}{
        \begin{tabular}{ccc|cccc}
\toprule[2pt]
\multirow{2}{*}{\textbf{Victim Model}} & \multirow{2}{*}{\textbf{\begin{tabular}[c]{@{}c@{}}Attack Model\\ Architecture\end{tabular}}} & \textbf{Shadow Datasets} & \multicolumn{4}{c}{\textbf{Victim Datasets}}                          \\
                                       &                                                                                               & \textbf{wikitext}        & \textbf{PIIs} & \textbf{openwebtext} & \textbf{arxiv} & \textbf{pile} \\ \hline
\multirow{3}{*}{GPT2-XL}      & Mistral                                    & 1.53            & 117.45  & 44.14       & 109.31  & 24.54  \\
                              & Qwen2.5                                    & 1.71            & 410.47  & 115.35      & 301.26  & 68.74  \\
                              & GPT2                                       & 1.54            & 4.17    & 2.61        & 3.81    & 1.70   \\ \hline
\multirow{3}{*}{Bloom7B1}     & Mistral                                    & 1.54            & 7277.80 & 537.97      & 1203.97 & 445.71 \\
                              & Qwen2.5                                    & 1.48            & 7404.53 & 839.47      & 1947.76 & 651.55 \\
                              & Bloom                                      & 1.41            & 16.81   & 9.14        & 13.45   & 2.12   \\ \hline
\multirow{3}{*}{LLaMA3-8B}    & Mistral                                    & 2.60            & 2016.21 & 447.20      & 692.70  & 134.76 \\
                              & Qwen2.5                                    & 2.89            & 1810.44 & 549.28      & 1315.82 & 151.34 \\
                              & LLaMA                                      & 1.85            & 12.57   & 4.16        & 10.11   & 2.03   \\ \bottomrule[1.5pt]

\end{tabular}
}
    \end{minipage}
\end{table*}

\noindent{\textbf{Results compared with baselines.}}
We compare the ASR of AIA with the baselines on the PII types of email and phone, with the detailed results presented in Table~\ref{tab:pii_compare}. The findings indicate that our method performs effectively on both phone numbers and email addresses. For instance, the Bloom-7B1 model achieves precise recovery rates of 41\% for phone numbers and 61\% for email addresses. Even the relatively less effective LLaMA3-8B model accurately recovers 15\% of phone numbers and 41\% of email addresses. 

In contrast, the \textit{True-Prefix Attack} and \textit{SPT Attack} exhibit poor performance, showing minimal success in recovering phone numbers. On the Bloom-7B1 model, both baselines recover only a small portion of email addresses, with ASR of 18\% and 10\%, respectively. We hypothesize that this discrepancy arises from the structure of the PII dataset, where email prefixes consist of a person's name combined with random numbers, enhancing the model's memory of the email. The GPT2-XL model recovers only 2\% to 4\% of email addresses, significantly lower than Bloom-7B1, likely due to its smaller size and weaker capacity for data retention. Notably, neither baseline is able to recover any private data accurately on the LLaMA3-8B model. This may be attributed to the LLaMA3-8B model's alignment and data protection mechanisms implemented during pre-training, which results in the frequent generation of placeholders such as “[email protected]”.

% making it resistant to being induced to output private data. The frequent generation of placeholders such as “[email protected]” further supports this observation.

\begin{figure*}[t]
  \includegraphics[width=\textwidth]{figures/PII_attack_example2.pdf}
  \caption{Three comparative examples of generated texts versus original data. The yellow text represents the original PII data, while the red and purple texts represent precisely recovered and mostly recovered PII data, respectively. The text recovery performance improves from left to right.}
  \label{fig:PII_attack_example}
  \vspace{-1em}
\end{figure*}

\setcounter{table}{2}
\begin{table}[ht]
\centering
\caption{The ASR of AIA on all models in precisely recovering the seven PII types: fax, birthday, SSN, address, job, bitcoin, and UUID.}
\label{tab:pii_attack_result}
\resizebox{0.5\textwidth}{!}{
\begin{tabular}{cccccccc}
\toprule[2pt]
\multicolumn{1}{l}{}           & \textbf{fax}      & \textbf{birthday} & \textbf{SSN}      & \textbf{address}  & \textbf{job}      & \textbf{bitcoin}  & \textbf{UUID}     \\ \hline
\multicolumn{1}{c|}{GPT2-XL}   & 0.25     & 1.00     & 0.76     & 0.56     & 0.97     & 0.22     & 0.17     \\
\multicolumn{1}{c|}{Bloom-7B1} & 0.48     & 0.99     & 0.57     & 0.57     & 0.98     & 0.04     & 0.04     \\
\multicolumn{1}{c|}{LLaMA3-8B} & 0.20     & 0.95     & 0.38     & 0.41     & 0.89     & 0.03     & 0.10     \\ \bottomrule[1.5pt]
\end{tabular}
}
\vspace{-2em}
\end{table}

\noindent{\textbf{Results on various PII types.}}
Table~\ref{tab:pii_attack_result} presents the ASR of AIA in precisely recovering the seven PII types: fax, birthday, SSN, address, job, bitcoin, and UUID. Remarkably, the ASR for birthdays and jobs approaches 100\%. Birthdays, which are short and highly structured numerical sequences, likely benefit from the model's pre-training exposure to similar formats, resulting in minimal changes to their semantic encoding after fine-tuning. Jobs, typically consisting of one to three words, are relatively easier to recover compared to other PII types. This observation is further supported by the ROUGE-1 and BLEU-1 results on the PII dataset across different victim LLMs shown in Table~\ref{tab:base_result}.

All victim models exhibit strong recovery performance for PII types other than Bitcoin addresses and UUID, with recovery rates generally ranging from one-third to over half of the data. Owing to the inherent irregularity and extended length characteristics of Bitcoin addresses and UUIDs, precise reconstruction is significantly more challenging. Specifically, only the GPT2-XL model achieves a recovery rate of approximately 20\% for the two PII types, while the ASR for Bloom-7B1 and LLaMA3-8B remains below 10\%. Notably, even in cases of incomplete reconstruction, the generated outputs maintain substantial proximity to ground truth values, exhibiting only minor character-level discrepancies in alphanumeric sequences (e.g., single-letter substitutions or partial numeric mismatches).

Figure~\ref{fig:PII_attack_example} shows three comparison examples between the generated text and the original private data, with the quality of text reconstruction improving from left to right. The majority of common words and PII data can be precisely recovered, as indicated by the red highlights in the figure. However, the recovery of less frequent words (e.g., "Bitcoin") and special characters (e.g., "@") tends to be less successful. Additionally, the recovery of named entities may occasionally be imprecise. For long character sequences, such as phone numbers or UUIDs, over 80\% of the characters are typically recovered, although some minor errors in individual characters or capitalization issues may occur, as highlighted in purple in the figure.
% Achieving precise recovery of such machine-irrelevant character sequences remains a challenging task.

\subsection{Ablation Study}
To explore the factors influencing the attack performance of AIA, we conducted three sets of ablation experiments on the decoder layer index, model size, and attack model architecture. The conclusions are as follows: 
\begin{itemize}[nolistsep, leftmargin=*, topsep=0pt]
    \item As the layer index increases, the attack performance decreases; however, the original private data can still be recovered to some extent. 
    \item The attack performance is independent of model size and AIA performs well in all model sizes.
    \item The attack performance is highly sensitive to the architecture of the attack model, with different architectures leading to poorer attack results.
\end{itemize}

\subsubsection{Decoder Layer Index}
Figure~\ref{fig:layer_idx} illustrates the trend of PPL on GPT2-XL and Bloom-7B1 models as the attacker's decoder layer index varies. The results show that as the decoder layer index increases, i.e., as the data leakage layer moves closer to the output layers, the overall attack effectiveness declines. This observation aligns with the trend described in Section \ref{sec:act_cos}, where the cosine similarity of activations before and after fine-tuning decreases as the decoder layer index increases. The decline in attack performance can be attributed to the greater changes in the activations of the decoder layers that is closer to the output layer during fine-tuning. 
% These layers are heavily involved in generating the final output and, therefore, undergo more significant updates to adapt to the fine-tuning dataset. As a result, the original patterns in these layers that could be exploited for text inversion become less stable, leading to reduced attack effectiveness.

Interestingly, when the cosine similarity of activations before and after fine-tuning drops below 60\% for a particular decoder layer, the perplexity of the generated text remains below 40. This indicates that the generated sentences become less natural, with noticeable grammatical or contextual inconsistencies, which suggests a reduction in the fluency and coherence of the generated texts. However, despite these linguistic limitations, the attacker is still able to infer the original fine-tuning data to a certain extent. This highlights the robustness of AIA, even when the stage controlled by the attacker is positioned further back in the pipeline.

% \begin{tcolorbox}[colframe=black,colback=gray!10,arc=3mm,boxrule=0.5mm]
% \textbf{Takeaways:} As the layer index increases, attack performance decreases, though some private data can still be recovered.
% \end{tcolorbox}

\begin{figure}[t]
  % \includegraphics[width=0.48\linewidth]{figures/gpt2_layer_idx.pdf} \hfill
  % \includegraphics[width=0.48\linewidth]{figures/bloom_layer_idx.pdf}
  \includegraphics[width=\linewidth]{figures/layer_idx_ppl.pdf} 
  \caption {The attack performance of AIA on GPT2-XL and Bloom-7B1 models as the attacker's decoder layer index varies, with the attack performance generally decreasing as the layer index increases.}
  \label{fig:layer_idx}
  \vspace{-1em}
\end{figure}

\subsubsection{Model Size}
\begin{table}[!thp]
% \vspace{-1em}
\caption{\label{tab:model_size}
Performance comparison of \jola{} across different model sizes: Llama-3.2-1B-Instruct, Llama-3.2-3B-Instruct, and Llama-3.1-70B-Instruct.
}
\resizebox{\columnwidth}{!}{
\begin{tabular}{l|l|cccccc}
\toprule
                     &            & \multicolumn{2}{c}{\textbf{Reasoning}} & \multicolumn{2}{c}{\textbf{Understanding}}  & \multicolumn{2}{c}{\textbf{Generation}}  \\
\cmidrule(lr){3-4}\cmidrule(lr){5-6}\cmidrule(lr){7-8}
                     &            & SIQA      & WinoGrande & Law           & Physics & E2E\_NLG   & WEG\_NLG \\
\midrule
\multirow{3}{*}{1B}  & \textbf{zero\_shot} & 23.34     & 2.45       & 9.00          & 3.00    & 7.59       & 6.32     \\
                     & \textbf{LoRA}       & 31.92     & 29.13      & 12.00         & 12.00   & 16.95      & 16.52    \\
                     & \textbf{our}        & 37.82     & 49.33      & 16.00         & 19.00   & 26.89      & 27.72    \\
\midrule
\multirow{3}{*}{3B}  & \textbf{zero\_shot} & 27.33     & 5.05       & 14.00         & 8.00    & 7.48       & 22.42    \\
                     & \textbf{LoRA}       & 50.09     & 43.28      & 15.00         & 21.00   & 23.76      & 28.19    \\
                     & \textbf{our}        & 63.05     & 56.67      & 20.00         & 32.00   & 34.78      & 31.47    \\
\midrule
\multirow{3}{*}{70B} & \textbf{zero\_shot} & 51.28     & 29.83      & 37.00         & 32.00   & 10.26      & 24.15    \\
                     & \textbf{LoRA}       & 61.78     & 56.54      & 42.00         & 38.00   & 35.87      & 42.81    \\
                     & \textbf{our}        & 72.17     & 70.26      & 51.00         & 45.00   & 46.18      & 57.12    \\
\bottomrule
\end{tabular}
}
% \vspace{-1.2em}
\end{table}
Table~\ref{tab:model_size} systematically presents the experimental results for GPT2 and Bloom models with varying parameter scales. To ensure comprehensive experiments, we select three representative configurations for each model family: the GPT2 series includes 355M, 774M, and 1.5B parameter variants, while the Bloom series comprises 560M, 1.7B, and 7.1B parameter configurations. 
% Notably, in the attack model architecture design, we implemented an adaptive depth configuration strategy that dynamically sets the decoder layers to one-third of the target model's depth (e.g., employing an 8-layer decoder for the 24-layer GPT2-355M model). 
The experimental results demonstrate that the attack performance of AIA is highly dependent on the victim dataset, and it maintains stable performance across different model sizes, with most PPL consistently below 10, ROUGE-L scores exceeding 0.9, and BLEU-4 scores above 0.6 in most cases. 
% These quantitative findings underscore the robustness and generalizability of AIA in attacking models of varying sizes.

% \begin{tcolorbox}[colframe=black,colback=gray!10,arc=3mm,boxrule=0.5mm]
% \textbf{Takeaways:} AIA performs well across models of different sizes.
% \end{tcolorbox}

\subsubsection{Attack Model Architecture}
% \begin{table}[ht]
\centering
\caption{The Impact of Attack Model Architecture on Attack Performance. Each attack model is configured with 6 decoder layers. The results are presented in terms of PPL.}
\label{tab:attack_model}
\resizebox{\columnwidth}{!}{
\begin{tabular}{ccc|cccc}
\toprule[2pt]
\multirow{2}{*}{Victim Model} & \multirow{2}{*}{Attack Model Architecture} & Shadow Datasets & \multicolumn{4}{c}{Victim Datasets}      \\
                              &                                            & wikitext        & PIIs    & openwebtext & arxiv   & pile   \\ \hline
\multirow{3}{*}{GPT2-xl}      & Mistral                                    & 1.53            & 117.45  & 44.14       & 109.31  & 24.54  \\
                              & Qwen2.5                                    & 1.71            & 410.47  & 115.35      & 301.26  & 68.74  \\
                              & Gpt2                                       & 1.54            & 4.17    & 2.61        & 3.81    & 1.70   \\ \hline
\multirow{3}{*}{Bloom7b1}     & Mistral                                    & 1.54            & 7277.80 & 537.97      & 1203.97 & 445.71 \\
                              & Qwen2.5                                    & 1.48            & 7404.53 & 839.47      & 1947.76 & 651.55 \\
                              & Bloom                                      & 1.41            & 16.81   & 9.14        & 13.45   & 2.12   \\ \hline
\multirow{3}{*}{Llama3-8B}    & Mistral                                    & 2.60            & 2016.21 & 447.20      & 692.70  & 134.76 \\
                              & Qwen2.5                                    & 2.89            & 1810.44 & 549.28      & 1315.82 & 151.34 \\
                              & Llama                                      & 1.85            & 12.57   & 4.16        & 10.11   & 2.03   \\ \bottomrule[1.5pt]

\end{tabular}
}
\end{table}
To explore the impact of the attack model architecture on attack performance, we conduct experiments using Mistral~\cite{mistral} and Qwen2.5~\cite{qwen2.5} as attack model architectures and compare them to the victim model architecture. Each attack model is configured with six decoder layers. As shown in Table \ref{tab:attack_model}, while all attack models exhibit excellent performance when trained on the shadow dataset, their effectiveness significantly declines when transitioning to inverting the victim dataset after switching the attack model architecture. Notably, even the best-performing configuration on GPT2-XL still yields perplexity values ranging from 24 to 120. On the Bloom-7B1 and LLaMA3-8B models, the perplexity can even reach values above a thousand, rendering AIA almost completely ineffective.

% \begin{tcolorbox}[colframe=black,colback=gray!10,arc=3mm,boxrule=0.5mm]
% \textbf{Takeaways:} AIA is highly sensitive to the architecture of the attack model.
% \end{tcolorbox}

% Two main factors contribute to these results. First, the variation in attack performance is partially due to the differing distributions of the datasets used. Since the shadow dataset and victim dataset may have different underlying characteristics, it is expected that a model trained on one would struggle to generalize well to the other. Secondly, the intermediate activation values are highly intertwined with the model architecture itself. Each model operates within its own distinct semantic space, meaning that the activations in Mistral and Qwen2.5 differ substantially from those in the victim models. This architectural misalignment leads to overfitting in the attack model, as it fails to generalize effectively to the new data. Therefore, even though the attack models perform well during training on the shadow dataset, they struggle to perform similarly when applied to victim datasets with different architecture-specific nuances.





\section{Conclusion}
% In this paper, we propose an honest-but-curious pipeline training system and introduce a text inversion attack based on intermediate activations. Extensive experimental evaluations demonstrate that modern pretrained large language models exhibit strong generalization capabilities, such that fine-tuning with specific datasets leads to minimal updates to the model. We establish the feasibility of recovering unknown fine-tuning data from pretrained models. Furthermore, we use a set of PII datasets to evaluate the potential for privacy data leakage, showing that a significant portion of private data can be accurately recovered. Although text inversion attacks in pipeline training systems can have severe consequences, defenses against such attacks remain underexplored. We call on researchers to address this critical privacy risk and to develop effective, low-cost defense mechanisms to counteract text inversion attacks.
In this paper, we explore the privacy risks inherent in decentralized training, particularly in scenarios where an honest-but-curious attacker exists in the pipeline. Despite lacking access to the complete model weights, we demonstrate the feasibility of simulating the victim model using a pre-trained model and introduce Activation Inversion Attack (AIA). We conduct extensive experiments on various large language models and public datasets to emphasize the effectiveness of our attack. As the application of decentralized training continues to grow, we call for the development of effective defense measures to mitigate the risk of AIA.


\section*{Limitations}
Our method has a key limitation: the architecture of the attack model must be consistent with that of the clean model. While the attack model performs well on the shadow dataset when using different architectures, its effectiveness significantly decreases when applied to the clean dataset. This constraint limits the flexibility in choosing the attack model. Additionally, the generated text exhibits issues such as lack of fluency, inconsistencies in letter casing, errors with special characters, uncommon words, and difficulty in accurately recovering long sequences. These observations indicate that our method is influenced by the challenges of transferring to unknown data distributions and the variations introduced during model fine-tuning.


\section*{Ethics Statement}
We declare that all authors of this paper adhere to the ACM Code of Ethics and uphold its code of conduct. This paper investigates activation inversion attack in decentralized training. The objective of our work is to highlight the potential data leakage risks associated with decentralized training, aiming to encourage the community to give greater attention to privacy protection in such settings and to advocate for measures to prevent such information leaks. No real sensitive data is used in our experiments; all experiments are conducted with publicly available datasets. The data in the PII dataset we use is randomly generated and does not represent actual private information. All models employed in this study are open-source and thus do not pose any threat to proprietary models.



% Bibliography entries for the entire Anthology, followed by custom entries
%\bibliography{anthology,custom}
% Custom bibliography entries only
\documentclass{MITstyle}

%\usepackage[table]{xcolor}
\usepackage{chngcntr}
\usepackage{hyperref}
\usepackage{microtype}

\title{A Lightweight and Extensible Cell Segmentation and Classification Model for Whole Slide Images}

\author{Nikita Shvetsov~$^{1, }$\footnote{Correspondence e-mail: nikita.shvetsov@uit.no}, Thomas K. Kilvaer~$^{2, 3}$, Masoud Tafavvoghi~$^{4}$, Anders Sildnes~$^{1}$, \\ Kajsa Møllersen~$^{4}$, Lill-Tove Rasmussen Busund~$^{5, 6}$, Lars Ailo Bongo~$^{1}$ \\
%
\vspace{1em} % Space between authors and afilliations
%
\normalfont{\small $^{1}$Department of Computer Science, UiT The Arctic University of Norway}\\
\normalfont{\small $^{2}$Department of Oncology, University Hospital of North Norway}\\
\normalfont{\small $^{3}$Department of Clinical Medicine, UiT The Arctic University of Norway}\\
\normalfont{\small $^{4}$Department of Community Medicine, UiT The Arctic University of Norway}\\
\normalfont{\small $^{5}$Department of Medical Biology, UiT The Arctic University of Norway} \\
\normalfont{\small $^{6}$Department of Clinical Pathology, University Hospital of North Norway} %\vspace{2em}
}

\begin{document}
\maketitle

\section*{Abstract}

% \begin{abstract}
% Developing clinically useful cell-level analysis tools in digital pathology remains challenging due to limitations in dataset granularity, inconsistent annotations, computational demands of advanced models, and difficulties in integrating new technologies into clinical workflows. To address these challenges, we propose a multi-faceted solution that enhances data quality, model performance, and usability to create a lightweight and extensible cell segmentation and classification model.

% First, we update data labels by employing a cross-relabeling process that refines the labels of two existing datasets, PanNuke and MoNuSAC, to create a new unified dataset with enhanced granularity, encompassing seven distinct cell types. Second, we leverage the H-Optimus foundation model as a fixed encoder to improve feature representation for simultaneous cell segmentation and classification tasks. Third, to address the computational demands of foundation models, we employ knowledge distillation to reduce model size and complexity while maintaining comparable performance. Finally, to facilitate integration into clinical workflows, we integrate the distilled model into the QuPath software, a widely used open-source platform in digital pathology.

% Our results demonstrate improvements in cell segmentation and classification performance using the H‑Optimus-based model compared to a CNN-based model. Specifically, the average $R^2$ improved from 0.575 to 0.871, and the average $PQ$ score improved from 0.450 to 0.492, indicating better alignment with actual cell counts and enhanced segmentation and classification quality. Furthermore, the distilled student model maintains performance comparable to the larger foundation model while reducing the parameter count by a factor of 48.
% Overall, by reducing computational complexity and integrating it into existing workflows, the proposed approach may significantly impact diagnostic processes, reduce the workload of pathologists, and contribute to improved patient outcomes. Though our approach shows potential enhancements in efficiency and usability of cell segmentation and classification models in digital pathology, extensive validation is needed to deploy these models in clinical practice.
% \end{abstract}

%%% shortened abstract
\begin{abstract}
Developing clinically useful cell-level analysis tools in digital pathology remains challenging due to limitations in dataset granularity, inconsistent annotations, high computational demands, and difficulties integrating new technologies into workflows. To address these issues, we propose a solution that enhances data quality, model performance, and usability by creating a lightweight, extensible cell segmentation and classification model. 

First, we update data labels through cross-relabeling to refine annotations of PanNuke and MoNuSAC, producing a unified dataset with seven distinct cell types. Second, we leverage the H-Optimus foundation model as a fixed encoder to improve feature representation for simultaneous segmentation and classification tasks. Third, to address foundation models' computational demands, we distill knowledge to reduce model size and complexity while maintaining comparable performance. Finally, we integrate the distilled model into QuPath, a widely used open-source digital pathology platform. 

Results demonstrate improved segmentation and classification performance using the H-Optimus-based model compared to a CNN-based model. Specifically, average $R^2$ improved from 0.575 to 0.871, and average $PQ$ score improved from 0.450 to 0.492, indicating better alignment with actual cell counts and enhanced segmentation quality. The distilled model maintains comparable performance while reducing parameter count by a factor of 48. By reducing computational complexity and integrating into workflows, this approach may significantly impact diagnostics, reduce pathologist workload, and improve outcomes. Although the method shows promise, extensive validation is necessary prior to clinical deployment.
\end{abstract}
\clearpage

\section{Introduction}
In digital pathology, accurate segmentation and classification of cells are crucial for many diagnostic, prognostic, and predictive analyses \cite{Jaber_Beziaeva_etal._2019,Lin_Pan_etal._2022,Park_Ock_etal._2022,Shen_Choi_etal._2024}. Nowadays, developments in computational pathology offer multiple solutions \cite{H._Qu_P._Wu_etal._2020,Javed_Mahmood_etal._2020} to utilize cell-level datasets to train machine learning models that solve these problems. The quality and specificity of training datasets are critical for robust and accurate models. Adhering to the principle of "garbage in, garbage out", it is essential to ensure that these datasets are extensively and accurately labeled with distinct classes that reflect the diverse biological characteristics of different cell types. Unfortunately, the number of open-source datasets comprising such high-quality annotations is limited. Existing cell segmentation datasets \cite{Gamper_Koohbanani_etal._2019,Graham_Vu_etal._2019,Verma_Kumar_etal._2021} may offer extensive annotations for certain cell types while providing more general labels for others. For example, in PanNuke, which is one of the largest open-source datasets comprising labeled cells, various types of morphologically and functionally different inflammatory cells like macrophages and lymphocytes are clustered in a broad "inflammatory" class. Consequently, these classes are frequently omitted from analyses or aggregated into broader meta-classes \cite{Gamper_Koohbanani_etal._2020} and likely interfere with other cell classes included in the dataset. This and similar inconsistencies in annotation granularity limit the ability of machine learning models to learn the comprehensive and nuanced features necessary for accurate cell segmentation and classification. To address these challenges, methods for refining and standardizing dataset annotations are essential to enhance the quality of training data.

A complementary approach to mitigate the absence of high-quality training data is the use of foundation models. Foundation models as encoders are defined as large-scale, versatile networks pre-trained on vast, diverse datasets using self-supervised learning, contrasting with convolutional neural network (CNN) pre-trained encoders that rely on supervised learning with labeled data. In practice, foundation models leverage enormous amounts of weakly or unlabeled data from millions of whole slide images (WSIs) and employ self-attention mechanisms to capture long-range dependencies and global context \cite{Chen_Ding_etal._2024,Saillard_Jenatton_etal._2024,Vorontsov_Bozkurt_etal._2024,Xu_Usuyama_etal._2024}. As a consequence, foundation models are able to produce transferable feature representations across different cell types and tissue environments. The feature representations can be leveraged by decoder networks to produce segmentation masks and pixel-level classifications. Because foundation models have comprehensive feature representations, they can be effectively fine-tuned using much smaller amounts of cell-level data compared to the large datasets needed to train models from scratch. Furthermore, foundation models incorporate adversarial training elements or contrastive learning \cite{Chen_Ding_etal._2024,Xu_Usuyama_etal._2024}, enhancing their resilience and adaptability by exposing them to challenging and varied scenarios during training. This may result in more generalizable models, often making them well-suited for diverse and complex tasks in digital pathology.

Despite the inherent advantages of foundation models, their deployment for practical use faces its own obstacles. In particular, they require substantial computational power, financial investments and rigorous testing to ensure reliability and efficacy for a given task \cite{Akkus_Dangott_etal._2022,Dragomir_Cocuz_etal._2022,Go_2022,Jafri_Farooqui_etal._2024}. Moreover, while foundation models enhance feature representation and performance, they depend on the quality of available annotations for decoder fine-tuning and, like any other model, cannot resolve existing inconsistencies or ambiguities in data labels. Therefore, there remains a critical need for solutions that address both data quality and practical deployment considerations.
Further, integrating new technologies into existing clinical workflows often encounters resistance, as it necessitates adjustments to established diagnostic processes. So, there is a need to develop solutions that could be integrated into current practices, minimizing the burden on medical professionals to adopt new tools \cite{King_Williams_etal._2023}.

Existing solutions \cite{Goldsborough_Philps_etal._2024,Hörst_Rempe_etal._2024}, while addressing some aspects of these challenges, fall short in providing a comprehensive approach. To address the data quality and clinical deployment issues, we propose a multi-faceted solution that encompasses data refinement, model optimization, and integration with existing pathology tools (\hyperref[fig:fig1]{Figure 1}). The outcome is a lightweight cell segmentation and classification model that can be integrated into digital pathology workflows for practical clinical use.

\begin{figure}[h!]
    \centering
    \includegraphics[width=\textwidth, height=0.82\textheight, keepaspectratio]{images/Figure_1.pdf}
    \caption{Overview of the proposed solution, including 1) Data refinement using cross-relabeling, 2) Teacher model development and fine tuning, 3) Student model optimization with knowledge distillation and 4) Student model and QuPath integration}
    \label{fig:fig1}
\end{figure}
\clearpage

Our approach begins with preparing the data for the fine-tuning and training of the machine learning models. We create a refined dataset, acquired via cross-relabeling two cell-level datasets, enhancing annotation specificity and consistency of the labeled data. Subsequently, we create a cell segmentation and classification model based on the foundation model. We leverage the foundation model as a fixed encoder and fine-tune a decoder using the refined dataset to improve generalization across diverse tissue- and cell types.
To ensure that the model remains lightweight and deployable in a possibly resource-constrained environment, we employ knowledge distillation to approximate the functionality of the foundation model. Finally, to facilitate the practical application of our model in digital pathology workflows, we integrate it with the QuPath \cite{Bankhead_Loughrey_etal._2017} application. Each methodological component contributes to the overarching goal of enhancing model performance, generalizability, and usability in clinical settings.

The primary contributions of this paper are:
\begin{enumerate}
    \item \textit{Data labels refinement through cross-relabeling:}
    
    We propose a new method for refining labels of cell-level datasets through cross-relabeling. This method employs classification models to re-label broad and ambiguous instances, resulting in a more diverse dataset. Our evaluation demonstrates that these classification models achieve high accuracy on test subsets, indicating the reliability of the method for label refinement.

    \item \textit{Enhanced model performance via foundation models:}
    
    We employ a foundation model as a feature extractor for the cell segmentation and classification task. In comparison with training a CNN model from scratch, the foundation model backbone only needs fine-tuning, which significantly reduces training time, computational resources and data requirements. We show that using a foundation model encoder leads to better performance in cell segmentation and classification networks than using a CNN-based encoder. This improvement may enable the model to generalize more effectively across various tissue types and imaging methods.
    
    \item \textit{Model optimization through knowledge distillation:}
    
    We show that a smaller student model trained using knowledge distillation on the refined dataset obtained via our cross-relabeling approach from a foundation model achieves comparable performance in cell segmentation and quantification tasks. As a result, this model is more suitable for deployment in environments without high-performance computing resources.
    
    \item \textit{Integration with QuPath:}
    
    We integrate the distilled cell segmentation and classification model into QuPath, a widely used open-source digital pathology platform, to accelerate clinical adaptation by enabling pathologists to more easily incorporate advanced computational tools into their existing workflows.
\end{enumerate}

Through these methodological steps, we aim to bridge the gap between advanced machine learning techniques and practical clinical applications, making accurate and efficient digital pathology accessible in a broader range of healthcare settings.

\section{Refining Existing Datasets Using Cross-Relabeling}
To address the limitations of sparse and ambiguous labeling of cell-level datasets, we propose a generalizable cross-relabeling strategy that can be applied to any dataset containing broadly categorized or imprecisely labeled cell types. This approach involves training and subsequently leveraging classification models to refine broad categories into more specific or biologically relevant classes.
When applied to cell-level data, the methodology includes extracting individual cell images from the dataset patches, preprocessing these images to standardize the size and accommodate partial cells, and then training deep learning classifiers capable of distinguishing between the finer cell subtypes within the coarser categories. 
To illustrate our approach, we focus on the PanNuke \cite{Gamper_Koohbanani_etal._2020, Gamper_Koohbanani_etal._2019} and MoNuSAC \cite{Verma_Kumar_etal._2021} datasets that we have used to train models for cell quantification in our previous works \cite{Shvetsov_Grønnesby_etal._2022,Shvetsov_Sildnes_etal._2024}. We find that for better cell differentiation we have to introduce more granular labels. PanNuke includes a broad classification of "inflammatory" cells, encompassing lymphocytes, macrophages, and neutrophils. Each cell type differs significantly in structure, function, and clinical relevance. Conversely, MoNuSAC uses the label "epithelial" for a class that comprises both benign epithelial cells and malignant neoplastic cells. This practice makes it challenging to differentiate between benign and malignant epithelial cells in the dataset, which is a critical distinction when identifying tumor areas within tissue samples. To address these issues, we implement a cross-relabeling strategy as shown in \hyperref[fig:fig2]{Figure 2}. The key components are two classification models: one is trained on singular cell images from PanNuke data to classify the epithelial meta-class into epithelial and neoplastic classes. The other is trained on MoNuSAC to refine the inflammatory class into lymphocytes, neutrophils, and macrophages.

\begin{figure}[h!]
    \centering
    \includegraphics[width=\textwidth]{images/Figure_2.pdf}
    \caption{Refined dataset generation via cross relabeling}
    \label{fig:fig2}
\end{figure}

The refining approach consists of three consecutive steps. The first is the preprocessing step, in which we extract individual cells from both datasets (\hyperref[fig:fig3]{Figure 3}). The specifics of PanNuke and MoNuSAC patch preparation before cell preprocessing are provided in \hyperref[chap:S1]{Appendix S1}.

\begin{figure}[h!]
    \centering
    \includegraphics[width=\textwidth]{images/Figure_3.pdf}
    \caption{Cell instances preprocessing including (1) cell map extraction, (2) bounding box delineation, (3) adjusting cell boxes and (4) cropping and resizing of cell images}
    \label{fig:fig3}
\end{figure}

During preprocessing, we extract cell type maps from the ground truth label mask and calculate bounding boxes around each cell instance. To accommodate partial cells at patch borders, a common issue in cropped patch images, we employ mirror padding and extend the field of view of the cell label by 15 pixels to capture adjacent cells. We then crop and resize the identified regions to $64 \times 64$ pixels using bicubic interpolation.

The preprocessed PanNuke dataset comprises 68,031 neoplastic and 23,207 epithelial cell images, while MoNuSAC comprises  33,104 lymphocytes, 1,252 neutrophils, and 1,695 macrophages, which we subsequently use in training cell classification models and classifying the cell image data \hyperref[fig:S2]{Appendix Figure S2 (1)}. 

The next step is to train two distinct ResNet50-based classifiers tailored to address the specific labeling challenges inherent in each dataset. We use ResNet50 for classification models due to its proven effectiveness for image classification tasks in histopathology \cite{pan2022reviewmachinelearningapproaches}, and its compatibility with small images. For the PanNuke dataset, we design the classifier, trained on MoNuSAC data, to disaggregate the heterogeneous "inflammatory" cell category into distinct subtypes: lymphocytes, macrophages, and neutrophils. Similarly, for the MoNuSAC dataset, the classifier is trained on PanNuke data and distinguishes between benign and malignant epithelial cells within the overarching "epithelial" label. By applying these targeted classifiers to their respective datasets, we assign more specific labels to individual cell instances, thus enabling us to create a unified dataset.
To ensure a balanced representation of classes, we train both models on datasets that had been equalized to match the size of the least represented class. Thus, we obtain datasets comprising 23,207 samples per class for PanNuke and 1,252 samples per class for MoNuSAC data. Next, we partition both of them into training (70\%), validation (20\%), and testing (10\%) subsets. To mitigate the risk of overfitting, we use a single dropout layer with a rate of p=0.5 in both models and data augmentation using randomized color perturbations, rotation, and horizontal and vertical flipping. We employ AdamW optimizer and the cross-entropy loss function for the training criterion.

To evaluate the two trained models, we measure the classification accuracy on the respective test subsets. The accuracies on the test subset for both classifiers are presented in \hyperref[tab:1]{Table 1}. The PanNuke model achieves an average accuracy of 93.57\%, with higher accuracy for neoplastic cells (96.06\%) compared to epithelial cells (86.26\%). The confusion matrix in Figure A3.1 shows that the model predominantly distinguishes accurately between epithelial and neoplastic tissues, with a substantial number of correct classifications and relatively few misclassifications. The MoNuSAC model demonstrates an average accuracy of 98.92\%, excelling in classifying lymphocytes (99.67\%) and macrophages (94.12\%), with lower performance for neutrophils (85.71\%). The confusion matrix in Figure A3.2 shows that the model identifies lymphocytes and performs reasonably well with macrophages and neutrophils.

\begin{table}[h!]
\renewcommand{\arraystretch}{1.5}
  \centering
  \caption{Cell classification results for PanNuke and MoNuSAC trained models (CI 95\%).}
  \label{tab:1}
  \begin{tabular}{|l|c|c|}
   \hline
   %\rowcolor{gray!30}
    Accuracy               & PanNuke model              & MoNuSAC model              \\
    \hline
    Average      & 0.936 (0.931--0.941)         & 0.989 (0.986--0.993)        \\
    \hline
    Neoplastic   & 0.961 (0.956--0.965)         & -                          \\
    \hline
    Epithelial   & 0.863 (0.849--0.877)         & -                          \\
    \hline
    Lymphocytes  & -                          & 0.997 (0.995--0.999)        \\
    \hline
    Neutrophils  & -                          & 0.857 (0.796--0.918)        \\
    \hline
    Macrophages  & -                          & 0.941 (0.906--0.976)        \\
    \hline
  \end{tabular}
\end{table}

Finally, during the last step, we use the model trained on PanNuke data for epithelial cells in MoNuSAC and the model trained on MoNuSAC for the inflammatory cells class in PanNuke. Specifically, we use classifier models to relabel epithelial cells in MoNuSAC and inflammatory cells in PanNuke data. Then we combine cells with refined labels and the rest of the cells in both datasets to create a refined dataset (\hyperref[fig:S2]{Appendix Figure S2 (2)}). The process of relabeling cells and visualizing them on a patch is shown in \hyperref[fig:fig4]{Figure 4}. The cell counts in the refined dataset are provided in \hyperref[tab:S4]{Appendix Table S4}.

\begin{figure}[h!]
    \centering
    \includegraphics[width=\textwidth, height=0.42\textheight, keepaspectratio]{images/Figure_4.pdf}
    \caption{Cell relabeling procedure for epithelial and inflammatory cell classes}
    \label{fig:fig4}
\end{figure}

%\hfill

Relabeling and combining datasets have been explored in a prior study \cite{Parulekar_Kanwat_etal._2023}, where consecutive fine-tuning on multiple datasets was employed to account for hierarchical class label structures. While the method presented in \cite{Parulekar_Kanwat_etal._2023} is intuitive, it often lacks consistency and requires multiple fine-tuning runs, which can be cumbersome and time-consuming. 
In contrast, cross-relabeling simplifies this process by using specialized classification models tailored to each dataset's specific labeling challenges. This approach provides better transparency and produces a unified dataset encompassing seven distinct cell types across multiple tissue samples, enhancing data diversity for further model training or fine-tuning.

Despite these improvements, cross-relabeling does not entirely resolve issues related to poor labeling quality or the amount of labeled data. Specifically, our results show lower accuracies persist for underrepresented classes, such as macrophages, which may stem from a limited sample availability and intrinsic challenges in distinguishing these cells based solely on H\&E staining. Furthermore, while our method enhances label specificity, it relies on the initial quality of the broad labels; thus, any fundamental inaccuracies in the original annotations can propagate through the relabeling process. Addressing the overall problem of limited data labels may require integrating additional data sources or utilizing complementary immunohistochemical staining methods.
Although the reported performance metrics are obtained from evaluations on the native test sets of each dataset, it is important to note that the primary application of these classifiers is to perform cross-relabeling, where a model trained on one dataset (e.g., PanNuke) is applied to another (e.g., MoNuSAC) and vice versa. We acknowledge that a more systematic evaluation of cross-dataset generalization is needed and could be performed in future work.

Overall, the refined dataset produced by our approach can enhance the supervised training or fine-tuning of cell segmentation and classification models, especially those that utilize pre-trained foundation models to improve feature extraction robustness. In addition, these models can detect nuanced classes that enable researchers to conduct more detailed analyses of biological processes in computational pathology.

\section{Foundation models for robust cell segmentation and classification}

Accurate cell segmentation and classification in digital pathology are hindered by limited labeled data and the fact that conventional CNNs are unable to capture global contextual information due to their local receptive field constraints \cite{Gheflati_Rivaz_2022,Yang_Marcus_etal.}. Traditional approaches in cell quantification have predominantly relied on CNN encoders, such as ResNet50, given their proven effectiveness in semantic segmentation tasks \cite{Deshmane_2023,Graham_Vu_etal._2019,Mukasheva_Koishiyeva_etal._2024,Stringer_Wang_etal._2021}. However, approaches that include fine-tuning of pretrained CNNs, data augmentation, and stain normalization to partially increase data variability and address staining differences often fail to achieve the necessary generalization and robustness across diverse tissue types and staining conditions \cite{G._Wang_W._Li_etal._2018,Gao_Bagci_etal._2018,Karim_El_Khoury_Martin_Fockedey_etal._2021}.

To overcome these challenges, we leverage an encoder-decoder network that uses a foundation model as the encoder and a CNN upsampling decoder (\hyperref[fig:fig5]{Figure 5}) for simultaneous cell segmentation and classification in 2D patches extracted from WSIs. Foundation models with transformer-based architectures are viable alternatives to CNN-based encoders \cite{Shamshad_Khan_etal._2023,Sourget_2023}. They enable the creation of more advanced architectures that can decode or transform learned features more effectively \cite{Chen_Duan_etal._2023,Cheng_Misra_etal._2022,Xie_Wang_etal._2021}.

\begin{figure}[h!]
    \centering
    \includegraphics[width=\textwidth]{images/Figure_5.pdf}
    \caption{UNETR-like model with foundational model as backbone}
    \label{fig:fig5}
\end{figure}

By utilizing a transformer-based encoder, we incorporate global contextual information into the feature extraction process, which is a key advantage of such architectures \cite{Chen_Lu_etal._2021}. This foundation model integration facilitates accurate pixel-wise segmentation and classification without the need for extensive encoder training, thereby potentially improving generalization across varied cellular structures and tissue types.
In our implementation, we employ a modified UNETR \cite{Hatamizadeh_Tang_etal._2021} architecture that combines a vision transformer (ViT) \cite{Dosovitskiy_Beyer_etal._2021} encoder with a CNN-based decoder. The encoder utilizes the pretrained H-Optimus foundation model, which contains 1.1 billion parameters and is trained on over 500,000 H\&E stained WSIs \cite{Saillard_Jenatton_etal._2024}. We extract outputs from four evenly spaced transformer blocks $Z_i$, where $i \in [1, 14, 26, 38]$, to serve as residual connections for the CNN decoder. We select these blocks based on our observation that features from non-adjacent levels of the encoder lead to better overall performance on the test subset.

The CNN decoder upsamples the feature representations, acquired from the transformer blocks, to generate an intermediate vector that is handled by two task-specific layers that generate cell segmentation and classification masks. The first task-specific layer is the ‘Cellpose head’,  which is used to delineate cell instances. The layer generates horizontal and vertical gradient maps to form vector fields that are refined through gradient tracking in a post-processing step using the Cellpose algorithm \cite{Stringer_Wang_etal._2021}, known for its efficacy in cell segmentation tasks and generalizability across multiple domains \cite{Pachitariu_Stringer_2022,Stringer_Pachitariu_2024}. The second task-specific layer is the "Cell type head", which assigns labels to individual pixels. In the post-processing step, we determine the output classification label of each segmented cell instance by majority voting over the labeled pixels that comprise the cell in the segmentation map.

To evaluate model performance and measure the impact of adding a foundation model as backbone, we compare it to a ResNet50-based model. ResNet50 is a widely used solution for encoders in segmentation architectures in the medical domain \cite{Deshmane_2023,Graham_Vu_etal._2019,Mukasheva_Koishiyeva_etal._2024,Stringer_Wang_etal._2021}. For the H-Optimus-based model, we utilize frozen weights for the encoder and only fine-tune the decoder to take advantage of the extensive pre-training of the foundation model. For the ResNet50-based model we start with ImageNet \cite{Deng_Dong_etal.} weights and train both encoder and decoder parts. Hyperparameters for the training step are set to be identical, where possible, for comparable evaluation. 
For this evaluation, we deliberately use the PanNuke dataset to provide a standardized and controlled comparison between the H‑Optimus and ResNet50-based models (\hyperref[fig:S2]{Appendix Figure S2 (3)}). Specifically, we use two of the default PanNuke dataset splits (66\%) for training and validation, and reserve the third split (33\%) for testing.

To address the challenge of cell class imbalance in the PanNuke dataset, which is a common characteristic in most cell-level H\&E patch datasets, both models’ training processes employ a weighted loss function comprising cross-entropy and focal loss \cite{Lin_Goyal_etal._2018}. The focal loss component is adjusted with coefficients derived from each cell class' instance frequency, emphasizing learning from underrepresented classes and enhancing the model's sensitivity to rare but significant cellular patterns. The cross-entropy loss is augmented with spectral decoupling regularization \cite{Pezeshki_Kaba_etal._2021,Pohjonen_Stürenberg_etal._2022} and spatially varying label smoothing \cite{Islam_Glocker_2021}, which potentially stabilizes training and improves generalization in case of complex tissue morphologies. For optimization, we employ the \textit{AdamW} \cite{Loshchilov_Hutter_2019} to counter unbalanced class scenarios, with cosine annealing learning rate scheduler.

We utilize the scikit-learn library \cite{Van_der_Walt_Schönberger_etal._2014} and HoVer-Net \cite{Graham_Vu_etal._2019} implementations of $R^2$ (the coefficient of determination) and $PQ$ (panoptic quality) to evaluate our experiments. Complete mathematical formulations and detailed explanations of these metrics are provided in \hyperref[chap:S5]{Appendix S5}. To compute confidence intervals, we use nonparametric bootstrapping, where after calculating the metric on the full sample, we generated 1000 bootstrap replicates by resampling with replacement and then determined the 95\% confidence intervals as the 2.5th and 97.5th percentiles of the resulting empirical distribution.

%\hfill

The model comparisons are summarized in \hyperref[tab:2]{Table 2}. The H‑Optimus-based model achieves higher $R^2$ across all cell classes compared to the ResNet50-based model, which means that its predictions are more closely aligned with the PanNuke cell counts, indicating a stronger correlation with the observed data. Notably, the improvement of $R^2_{dead}$ may be an indicator of better global contextual representations provided by the foundation model backbone. In terms of segmentation and classification quality combined, measured by the PQ score, the H‑Optimus-based model demonstrates notable improvements across most cell classes. Overall, the average $R^2$ improved from 0.575 to 0.871, while the average $PQ$ score improved from 0.450 to 0.492, demonstrating better performance of the H-Optimus-based model.

\begin{table}[h!]
\renewcommand{\arraystretch}{1.5}
  \centering
  \caption{Cell quantification metrics for baseline and proposed models (CI 95\%).}
  \label{tab:2}
  \begin{tabular}{|l|c|c|}
    \hline
    %\rowcolor{gray!30}
    Metric             & Resnet50-based            & H-optimus-based              \\
    \hline
    $R^2_{neoplastic}$    & 0.681 (0.576--0.769)       & \textbf{0.941 (0.917--0.960)} \\
    \hline
    $R^2_{inflammatory}$  & 0.863 (0.778--0.903)       & \textbf{0.949 (0.918--0.966)} \\
    \hline
    $R^2_{connective}$    & 0.600 (0.488--0.698)       & 0.609 (0.436--0.772)          \\
    \hline
    $R^2_{dead}$          & 0.097 (-11.389--0.669)     & 0.925 (0.404--0.982)          \\
    \hline
    $R^2_{epithelial}$    & 0.635 (0.490--0.747)       & \textbf{0.930 (0.886--0.964)} \\
    \hline
    $PQ_{neoplastic}$       & 0.517 (0.499--0.535)       & \textbf{0.589 (0.575--0.604)} \\
    \hline
    $PQ_{inflammatory}$     & 0.455 (0.429--0.482)       & \textbf{0.528 (0.507--0.549)} \\
    \hline
    $PQ_{connective}$       & 0.416 (0.400--0.431)       & \textbf{0.451 (0.436--0.465)} \\
    \hline
    $PQ_{dead}$             & 0.374 (0.342--0.408)       & 0.292 (0.209--0.365)          \\
    \hline
    $PQ_{epithelial}$       & 0.488 (0.460--0.519)       & \textbf{0.599 (0.579--0.618)} \\
    \hline
  \end{tabular}
\end{table}

Our results  show that integrating the H‑Optimus foundation model within the UNETR architecture enhances the model's ability to segment and classify cells across diverse tissues from PanNuke data. The pretrained transformer encoder provides robust feature representations, resulting in higher average $R^2$ and $PQ$ scores compared to the CNN-based model. This leads to more reliable cell quantification and more accurate downstream analysis. Additionally, the streamlined fine-tuning process reduces computational overhead and training time, making the model more adaptable for new data.

Despite these advancements, the foundation model-based approach does not fully resolve all challenges related to cell segmentation and classification. We observe lower metric scores for underrepresented classes in the training data. Furthermore, foundation models typically encompass billions of parameters, resulting in substantial computational and memory requirements. It therefore poses challenges for deployment in resource-constrained environments, limiting their practical applicability in certain clinical settings.

\section{Model optimization via Knowledge Distillation}

To address the limitations posed by the extensive size of foundation models, we implement knowledge distillation — a model compression technique that leverages the teacher-student paradigm \cite{Hinton_Vinyals_etal._2015}. By training a smaller, more efficient student model to replicate the output of a larger, pre-trained teacher model, we retain performance while significantly reducing the model's complexity and resource requirements (\hyperref[fig:fig6]{Figure 6}).

\begin{figure}[h!]
    \centering
    \includegraphics[width=\textwidth, height=0.45\textheight, keepaspectratio]{images/Figure_6.pdf}
    \caption{Knowledge distillation framework for training a student model using a pre-trained teacher}
    \label{fig:fig6}
\end{figure}

We employ knowledge distillation to compress the H‑Optimus-based teacher model into a more efficient student model. The teacher model is the modified UNETR architecture with the H‑Optimus foundation model described in the previous chapter. The student model is based on a UNet architecture augmented with residual connections and incorporates a smaller ViT encoder with 9 million parameters \cite{Steiner_Kolesnikov_etal._2022,Wightman_2019}. 

First, we fine-tune the teacher model using the refined dataset from the cross-relabeling procedure (Section 2). Initially we train the decoder of the teacher model while keeping the encoder weights frozen. We split the refined dataset into train (70\%), validation (20\%) and test (10\%) subsets (\hyperref[fig:S2]{Appendix Figure S2 (4)}). During fine-tuning, we use the train and validation subsets, while leaving the test subset for model evaluation. We set the training procedure and model hyperparameters to be identical to those that were used to demonstrate the utility of foundation models for the simultaneous cell segmentation and classification task.

Next, we perform knowledge distillation from teacher to student using the refined dataset used to fine-tune the teacher model. The student model is trained to replicate the teacher model's outputs. We utilize a specialized loss function that aligns the student's predicted probability distribution with the teacher's, incorporating the teacher's class probability distribution derived from the output. Following the methodology of Hinton et al. \cite{Hinton_Vinyals_etal._2015}, we experiment with various hyperparameter settings for the temperature ($T$) and the balancing coefficients ($\alpha$ and $\beta$) in the loss function. We vary $T$ from 1 to 20 and adjust $\alpha$ and $\beta$ to balance the distillation and student losses. Through iterative tuning and evaluation, we identify that setting $T=14$, $\alpha=0.3$, and $\beta=0.7$ yields a configuration that converges and closely approximates the teacher model's performance during training.

Finally, we assess the performance of both models using the $R^2$ and $PQ$ (defined in \hyperref[chap:S5]{Appendix S5}) on the test set of the refined dataset (\hyperref[tab:3]{Table 3}). We observe that the 95\% confidence intervals overlap for most cell types, so we cannot claim statistically significant performance differences between the teacher and student models. One exception appears in the neoplastic class. The teacher model produces an $R^2$ of 0.919, while the student model shows an $R^2$ of 0.852. In addition, the student model achieves higher $PQ$ values for the neoplastic and connective classes, though the confidence intervals show overlap.

\begin{table}[h!]
\renewcommand{\arraystretch}{1.5}
  \centering
  \caption{Cell quantification metrics for teacher and distilled student models (CI 95\%).}
  \label{tab:3}
  \begin{tabular}{|l|c|c|}
    \hline
    %\rowcolor{gray!30}
    Metric & Teacher & Student \\
    \hline
    $R^2_{neoplastic}$    & \textbf{0.919} (0.898--0.939) & 0.852 (0.800--0.891) \\
    \hline
    $R^2_{lymphocyte}$    & 0.969 (0.956--0.977)         & 0.969 (0.956--0.978) \\
    \hline
    $R^2_{connective}$    & 0.694 (0.548--0.809)         & 0.618 (0.469--0.741) \\
    \hline
    $R^2_{dead}$          & 0.755 (0.400--0.908)         & 0.424 (0.100--0.731) \\
    \hline
    $R^2_{epithelial}$    & 0.922 (0.870--0.958)         & 0.843 (0.738--0.917) \\
    \hline
    $R^2_{macrophage}$    & 0.384 (-0.369--0.724)        & 0.704 (0.352--0.859) \\
    \hline
    $R^2_{neutrofil}$     & 0.854 (0.578--0.929)         & 0.833 (0.502--0.925) \\
    \hline
    $PQ_{neoplastic}$       & 0.581 (0.569--0.593)         & 0.601 (0.588--0.613) \\
    \hline
    $PQ_{lymphocyte}$       & 0.536 (0.520--0.553)         & 0.563 (0.544--0.579) \\
    \hline
    $PQ_{connective}$       & 0.436 (0.421--0.451)         & 0.457 (0.441--0.474) \\
    \hline
    $PQ_{dead}$             & 0.272 (0.235--0.315)         & 0.279 (0.201--0.369) \\
    \hline
    $PQ_{epithelial}$       & 0.522 (0.500--0.545)         & 0.530 (0.506--0.555) \\
    \hline
    $PQ_{macrophage}$       & 0.524 (0.459--0.588)         & 0.474 (0.405--0.543) \\
    \hline
    $PQ_{neutrofil}$        & 0.541 (0.490--0.592)         & 0.565 (0.522--0.607) \\
    \hline
  \end{tabular}
\end{table}


We further decompose the $PQ$ metric into its $SQ$ and $DQ$ components (\hyperref[tab:S6]{Appendix Table S6}). Both models produce nearly identical $SQ$ values, which indicates that they predict instance boundaries with similar precision. Although the student model shows some improvement in $DQ$ scores for certain classes, the confidence intervals overlap and do not confirm a statistically significant difference.

We observe that the student and teacher models yield comparable detection performance despite the student model using a much smaller and simpler architecture. A model with fewer parameters reduces the risk of overfitting when training data are scarce relative to the model’s complexity \cite{Farias_Ludermir_etal._2022}. The knowledge distillation process also encourages the student model to focus on the most generalizable detection features learned from the teacher. These factors enable the student model to achieve similar detection performance across different cell types.

Additionally, considering the model sizes reported in \hyperref[tab:4]{Table 4}, the distilled model achieves a significant reduction compared to the teacher model, with a 48-fold decrease in parameter count and a 5.5-fold reduction in on-disk size. In inference mode, the teacher model requires 16 GB of VRAM for a batch size of 32, while the distilled model only needs 3 GB of VRAM for the same batch size. These reductions make the distilled model significantly more practical for fine-tuning and deployment in resource-constrained environments.

\begin{table}[h!]
\renewcommand{\arraystretch}{1.5}
  \centering
  \caption{Parameter counts and size of teacher and distilled model}
  \label{tab:4}
  \adjustbox{max width=\textwidth}{%
  \begin{tabular}{|l|c|c|c|}
    \hline
    %\rowcolor{gray!30}
    Metric & H-optimus-based (Teacher) & mobileViT-based (Student) & Magnitude of difference \\
    \hline
    Parameters count       & 1,158,917,906   & \textbf{24,093,393}   & \textbf{48x}  \\
    \hline
    Estimated Total Size (MB) & 87,912       & \textbf{15,935}    & \textbf{5.5x} \\
    \hline
  \end{tabular}%
}
\end{table}

%\hfill

With recent advancements in complex network architectures and the use of pretrained encoders to achieve state-of-the-art performance \cite{Baumann_Dislich_etal._2024,Hörst_Rempe_etal._2024} in cell segmentation and classification tasks, model size, computational complexity, and processing times have increased. This limits the scalability and accessibility of these models. As we demonstrate, this may be mitigated using knowledge distillation. Studies in the field of natural language processing have demonstrated the efficacy of knowledge distillation in retaining the capabilities of the teacher model while achieving significant reductions in size and complexity \cite{Huangpu_Gao_2024,Sun_Yu_etal.}. 

We demonstrate the feasibility of knowledge distillation in digital pathology, specifically for cell segmentation and classification tasks. Moreover, we achieve this performance while also significantly reducing the parameter count. In addressing the challenge of knowledge transfer, we found that distillation from a transformer-based model to a smaller transformer is more straightforward than attempting to map transformer features to CNN blocks. In our experiments, using a CNN-based network as a student results in worse cell quantification performance due to the structural constraints of CNN feature space dimensions. 

Although our primary approach relies on a transformer-based student model that performs well, it can be further optimized to incorporate advantages from CNN architectures. For example, employing alternative techniques such as using ViT adapters \cite{Chen_Duan_etal._2023} or $1 \times 1$ convolutions to adjust feature map sizes may be beneficial for harnessing CNN advantages like enhanced local feature extraction. Moreover, if additional performance improvements are desired, the process can be further enhanced by applying supplementary knowledge distillation techniques, such as self-distillation \cite{Zhang_Song_etal._2019} or online distillation \cite{Houyon_Cioppa_etal._2023}.

Despite these promising results, further validation on independent datasets is necessary to fully understand the model's limitations. Underrepresented classes may pose challenges when addressing complex cases. Pathologists need to validate these models to adopt them in clinical settings. While the distilled models are smaller and more deployable, a technological gap persists because pathologists traditionally rely on established methods for inspecting WSIs and diagnosing diseases. Addressing the complexities involved in deploying models for inference and supporting pathologists in adopting new tools is essential for integrating these models into clinical workflows.

\section{Model integration with QuPath}
Digital pathology tools with graphical user interfaces are essential for visualizing and analyzing WSIs. To make our student model useful in clinical pathology workflows, it needs to be integrated into a tool that enables inspecting regions, creating annotations, and providing quantitative analyses of biomarkers. Therefore, we integrate the trained student model from the previous chapter into the QuPath open‑source platform \cite{Bankhead_Loughrey_etal._2017}. QuPath provides the required annotation, visualization, and analysis tools to interpret complex histological data, including workflows for cell segmentation, classification, and quantification (\hyperref[fig:fig7]{Figure 7}). 

\begin{figure}[h!]
    \centering
    \includegraphics[width=\textwidth]{images/Figure_7.pdf}
    \caption{Visualization of model-generated cell quantification annotations (left) and the corresponding unannotated slide (right) in QuPath}
    \label{fig:fig7}
\end{figure}

To identify the regions in a WSI critical for prognosticating tumor development, such as specific tumor areas or border regions without overlapping healthy tissue, the pathologist uses QuPath to outline these regions. Then, the pathologist initiates a cell segmentation and classification script through the QuPath interface for the selected regions. The resulting annotations and quantified cell information are then directly overlaid onto the WSI in the QuPath interface. Additional design and implementation details are in \hyperref[chap:S7]{Appendix S7}. 

Two common approaches for integrating deep learning models into QuPath are Java‑based native QuPath extensions \cite{Goldsborough_Philps_etal._2024} and the execution of RESTful API requests to a model server coupled with handling the response via an extension, as demonstrated in the application of cell segmentation models applied to immunofluorescence images \cite{Sugawara_2023}. While the community is actively working on these integration strategies, there is currently no universal solution that fully addresses all integration and performance requirements.

Extensions may offer better integration with QuPath, allowing slightly improved performance and more widespread usage of the built-in QuPath models, but they lack the flexibility to customize models and modify their behavior. For example, the newest version of QuPath includes models such as StarDist \cite{Weigert_Schmidt} and InstanSeg \cite{Goldsborough_Philps_etal._2024} that can perform cell segmentation. Both models pose limitations when applied to simultaneous cell segmentation and classification. StarDist performs well only on convex, round shapes by design, whereas some neoplastic, inflammatory, and connective cells exhibit complex and non-convex shapes. InstanSeg provides only semantic segmentation without assigning classes to the segmented cells.

%\hfill

In contrast, our approach offers an alternative integration strategy. It utilizes the paquo library to directly interact with QuPath’s internal application programming interface from within Python. This enables data exchange and processing without the need for intermediate conversion steps and provides greater control over model customization, retraining, and the incorporation of custom processing steps.

The integration of our custom model with QuPath underscores its potential to significantly enhance the diagnostic process by reducing the time burden on pathologists and enabling them to focus on more complex interpretative tasks using familiar software. Leveraging a tool that is already well-established among pathologists increases the likelihood of its adoption into daily clinical workflows. The quantitative data generated through the automated workflow is critical for both clinical decision-making and research, facilitating more accurate biomarker analysis, enabling robust statistical evaluations, and supporting hypothesis generation and testing. Additionally, by streamlining cell segmentation and classification, the tool enhances the scalability and reproducibility of pathological assessments, ultimately contributing to improved diagnostic accuracy and patient outcomes.

\section{Conclusion and future work}

In this study, we address critical challenges in digital pathology and tackle the usability and deployment issues of the developed models in standard computing environments without the need for high-performance computing systems. Our multi-faceted approach encompasses data refinement through cross-relabeling, leveraging foundation models for robust cell segmentation and classification, optimizing model performance via knowledge distillation, and integrating the optimized model into the QuPath software for practical application. This approach is used to construct a capable, versatile, and adjustable model for cell segmentation and classification, with enhanced performance and usability.

\begin{sloppypar}
While our approach shows potential in the field of computational pathology, certain limitations persist. 
For example, our implementation currently exhibits lower performance in detecting macrophages. 
This serves as an instance of the broader challenge of accurately identifying complex cell types. In order to address this issue, extending our approach to incorporate additional data sources, exploring alternative modeling approaches, and integrating other imaging modalities such as immunohistochemical staining may help improve detection accuracy. Moreover, although the distilled model reduces computational demands, integrating advanced deep learning models into clinical practice requires addressing technological gaps and potential resistance to adopting new tools within established diagnostic processes.
\end{sloppypar}

Future work could focus on several key areas to refine the proposed approach and facilitate its adoption in clinical environments. Enhancing the cell-relabeling process with additional datasets \cite{Graham_Jahanifar_etal._2021} could improve the representation of underrepresented cell types and enhance overall model performance. Also, incorporating additional data sources, such as multi-modal imaging or complementary staining methods, may address limitations related to cell type differentiation and class imbalance. Exploring other foundation models \cite{Vorontsov_Bozkurt_etal._2024,Zimmermann_Vorontsov_etal._2024} or introducing additional modalities \cite{Ding_Wagner_etal._2024,Vaidya_Zhang_etal._2025} may provide alternative architectures better suited to specific tasks or offer improved efficiency. Implementing more complex knowledge distillation techniques \cite{Houyon_Cioppa_etal._2023,Zhang_Song_etal._2019} could further optimize the model's performance and adaptability. Additionally, deeper integration with QuPath or other digital pathology software could provide pathologists more control over cell quantification analysis directly within the QuPath interface, thereby increasing accessibility and usability. Such enhancements would not only refine model performance but also ensure greater adaptability and scalability within various clinical environments. Finally, extensive validation of the model by pathologists and benchmarking against independent datasets are essential steps toward establishing the model's reliability and fostering confidence in its clinical utility.

\section*{Acknowledgments} 
This work was funded in part by the Research Council of Norway grant no. 309439 SFI Visual Intelligence, and the North Norwegian Health Authority grant no. HNF1521-20.

\bibliographystyle{IEEEtran}
\begin{sloppypar}
\begin{thebibliography}{99}

\bibitem{chaplot2020neural} Chaplot, Devendra Singh, et al. "Neural topological slam for visual navigation." Proceedings of the IEEE/CVF conference on computer vision and pattern recognition. 2020.

\bibitem{maksymets2021thda} Maksymets, Oleksandr, et al. "Thda: Treasure hunt data augmentation for semantic navigation." Proceedings of the IEEE/CVF International Conference on Computer Vision. 2021.

\bibitem{mezghan2022memory} Mezghan, Lina, et al. "Memory-augmented reinforcement learning for image-goal navigation." 2022 IEEE/RSJ International Conference on Intelligent Robots and Systems (IROS). IEEE, 2022.

\bibitem{al2022zero} Al-Halah, Ziad, Santhosh Kumar Ramakrishnan, and Kristen Grauman. "Zero experience required: Plug \& play modular transfer learning for semantic visual navigation." Proceedings of the IEEE/CVF Conference on Computer Vision and Pattern Recognition. 2022.

\bibitem{ye2021auxiliary} Ye, Joel, et al. "Auxiliary tasks and exploration enable objectgoal navigation." Proceedings of the IEEE/CVF international conference on computer vision. 2021.

\bibitem{chaplot2020object} Chaplot, Devendra Singh, et al. "Object goal navigation using goal-oriented semantic exploration." Advances in Neural Information Processing Systems 33 (2020)

\bibitem{ramakrishnan2022poni} Ramakrishnan, Santhosh Kumar, et al. "Poni: Potential functions for objectgoal navigation with interaction-free learning." Proceedings of the IEEE/CVF Conference on Computer Vision and Pattern Recognition. 2022.

\bibitem{ramrakhya2022habitat} Ramrakhya, Ram, et al. "Habitat-web: Learning embodied object-search strategies from human demonstrations at scale." Proceedings of the IEEE/CVF Conference on Computer Vision and Pattern Recognition. 2022.

\bibitem{mousavian2019visual} Mousavian, Arsalan, et al. "Visual representations for semantic target driven navigation." 2019 International Conference on Robotics and Automation (ICRA). IEEE, 2019.

\bibitem{dhariwal2021diffusion} Dhariwal, Prafulla, and Alexander Nichol. "Diffusion models beat gans on image synthesis." Advances in neural information processing systems 34 (2021)

\bibitem{ho2022classifier} Ho, Jonathan, and Tim Salimans. "Classifier-free diffusion guidance." arXiv preprint arXiv:2207.12598 (2022).

\bibitem{nichol2021glide} Nichol, Alex, et al. "Glide: Towards photorealistic image generation and editing with text-guided diffusion models." arXiv preprint arXiv:2112.10741 (2021)

\bibitem{brooks2023instructpix2pix} Brooks, Tim, Aleksander Holynski, and Alexei A. Efros. "Instructpix2pix: Learning to follow image editing instructions." Proceedings of the IEEE/CVF Conference on Computer Vision and Pattern Recognition. 2023.

\bibitem{fu2023guiding} Fu, Tsu-Jui, et al. "Guiding instruction-based image editing via multimodal large language models." arXiv preprint arXiv:2309.17102 (2023).

\bibitem{geng2024instructdiffusion} Geng, Zigang, et al. "Instructdiffusion: A generalist modeling interface for vision tasks." Proceedings of the IEEE/CVF Conference on Computer Vision and Pattern Recognition. 2024.

\bibitem{zhou2024minedreamer} Zhou, Enshen, et al. "Minedreamer: Learning to follow instructions via chain-of-imagination for simulated-world control." arXiv preprint arXiv:2403.12037 (2024).

\bibitem{zhou2023esc} Zhou, Kaiwen, et al. "Esc: Exploration with soft commonsense constraints for zero-shot object navigation." International Conference on Machine Learning. PMLR, 2023.

\bibitem{yu2023l3mvn} Yu, Bangguo, Hamidreza Kasaei, and Ming Cao. "L3mvn: Leveraging large language models for visual target navigation." 2023 IEEE/RSJ International Conference on Intelligent Robots and Systems (IROS). IEEE, 2023.

\bibitem{gadre2023cows} Gadre, Samir Yitzhak, et al. "Cows on pasture: Baselines and benchmarks for language-driven zero-shot object navigation." Proceedings of the IEEE/CVF Conference on Computer Vision and Pattern Recognition. 2023.

\bibitem{shah2023navigation} Shah, Dhruv, et al. "Navigation with large language models: Semantic guesswork as a heuristic for planning." Conference on Robot Learning. PMLR, 2023.

\bibitem{cai2024bridging} Cai, Wenzhe, et al. "Bridging zero-shot object navigation and foundation models through pixel-guided navigation skill." 2024 IEEE International Conference on Robotics and Automation (ICRA). IEEE, 2024.

\bibitem{yu2023co} Yu, Bangguo, Hamidreza Kasaei, and Ming Cao. "Co-NavGPT: Multi-robot cooperative visual semantic navigation using large language models." arXiv preprint arXiv:2310.07937 (2023).

\bibitem{wu2024voronav} Wu, Pengying, et al. "Voronav: Voronoi-based zero-shot object navigation with large language model." arXiv preprint arXiv:2401.02695 (2024).

\bibitem{qin2023mp5} Qin, Yiran, et al. "Mp5: A multi-modal open-ended embodied system in minecraft via active perception." arXiv preprint arXiv:2312.07472 (2023).

\bibitem{du2024learning} Du, Yilun, et al. "Learning universal policies via text-guided video generation." Advances in Neural Information Processing Systems 36 (2024).

\bibitem{ajay2024compositional} Ajay, Anurag, et al. "Compositional foundation models for hierarchical planning." Advances in Neural Information Processing Systems 36 (2024).

\bibitem{liang2024skilldiffuser} Liang, Zhixuan, et al. "Skilldiffuser: Interpretable hierarchical planning via skill abstractions in diffusion-based task execution." Proceedings of the IEEE/CVF Conference on Computer Vision and Pattern Recognition. 2024.

\bibitem{heusel2017gans} Heusel, Martin, et al. "Gans trained by a two time-scale update rule converge to a local nash equilibrium." Advances in neural information processing systems 30 (2017).

\bibitem{zhang2018unreasonable} Zhang, Richard, et al. "The unreasonable effectiveness of deep features as a perceptual metric." Proceedings of the IEEE conference on computer vision and pattern recognition. 2018.

\bibitem{brown2020language} Brown, Tom B. "Language models are few-shot learners." arXiv preprint arXiv:2005.14165 (2020).

\bibitem{podell2023sdxl} Podell, Dustin, et al. "Sdxl: Improving latent diffusion models for high-resolution image synthesis." arXiv preprint arXiv:2307.01952 (2023).

\bibitem{brohan2022rt} Brohan, Anthony, et al. "Rt-1: Robotics transformer for real-world control at scale." arXiv preprint arXiv:2212.06817 (2022).

\bibitem{brohan2023rt} Brohan, Anthony, et al. "Rt-2: Vision-language-action models transfer web knowledge to robotic control." arXiv preprint arXiv:2307.15818 (2023).

\bibitem{li2024manipllm} Li, Xiaoqi, et al. "Manipllm: Embodied multimodal large language model for object-centric robotic manipulation." Proceedings of the IEEE/CVF Conference on Computer Vision and Pattern Recognition. 2024.

\bibitem{shah2023vint} Shah, Dhruv, et al. "ViNT: A foundation model for visual navigation." arXiv preprint arXiv:2306.14846 (2023).

\bibitem{liu2024visual} Liu, Haotian, et al. "Visual instruction tuning." Advances in neural information processing systems 36 (2024).

\bibitem{hu2021lora} Hu, Edward J., et al. "Lora: Low-rank adaptation of large language models." arXiv preprint arXiv:2106.09685 (2021).

\bibitem{qin2023supfusion} Qin, Yiran, et al. "SupFusion: Supervised LiDAR-camera fusion for 3D object detection." Proceedings of the IEEE/CVF International Conference on Computer Vision. 2023.

\bibitem{qin2024worldsimbench} Qin, Yiran, et al. "Worldsimbench: Towards video generation models as world simulators." arXiv preprint arXiv:2410.18072 (2024).

\bibitem{yu2025gamefactory} Yu, Jiwen, et al. "GameFactory: Creating New Games with Generative Interactive Videos." arXiv preprint arXiv:2501.08325 (2025).

\bibitem{zhou2024code} Zhou, Enshen, et al. "Code-as-Monitor: Constraint-aware Visual Programming for Reactive and Proactive Robotic Failure Detection." arXiv preprint arXiv:2412.04455 (2024).

\bibitem{zhang2024ad} Zhang, Zaibin, et al. "AD-H: Autonomous Driving with Hierarchical Agents." arXiv preprint arXiv:2406.03474 (2024).

\bibitem{wang2024toward} Wang, Chaoqun, et al. "Toward Accurate Camera-based 3D Object Detection via Cascade Depth Estimation and Calibration." arXiv preprint arXiv:2402.04883 (2024).

\bibitem{huang2024story3d} Huang, Yuzhou, et al. "Story3d-agent: Exploring 3d storytelling visualization with large language models." arXiv preprint arXiv:2408.11801 (2024).

\bibitem{savinov2018semi} Savinov, Nikolay, Alexey Dosovitskiy, and Vladlen Koltun. "Semi-parametric topological memory for navigation." arXiv preprint arXiv:1803.00653 (2018).

\bibitem{majumdar2022zson} Majumdar, Arjun, et al. "Zson: Zero-shot object-goal navigation using multimodal goal embeddings." Advances in Neural Information Processing Systems 35 (2022): 32340-32352.

\bibitem{yadav2023offline} Yadav, Karmesh, et al. "Offline visual representation learning for embodied navigation." Workshop on Reincarnating Reinforcement Learning at ICLR 2023. 2023.

\bibitem{yadav2023ovrl} Yadav, Karmesh, et al. "Ovrl-v2: A simple state-of-art baseline for imagenav and objectnav." arXiv preprint arXiv:2303.07798 (2023).

\bibitem{sun2024fgprompt} Sun, Xinyu, et al. "FGPrompt: fine-grained goal prompting for image-goal navigation." Advances in Neural Information Processing Systems 36 (2024).

\bibitem{zhu2017target} Zhu, Yuke, et al. "Target-driven visual navigation in indoor scenes using deep reinforcement learning." 2017 IEEE international conference on robotics and automation (ICRA). IEEE, 2017.

\bibitem{koh2024generating} Koh, Jing Yu, Daniel Fried, and Russ R. Salakhutdinov. "Generating images with multimodal language models." Advances in Neural Information Processing Systems 36 (2024).

\bibitem{krantz2022instance} Krantz, Jacob, et al. "Instance-specific image goal navigation: Training embodied agents to find object instances." arXiv preprint arXiv:2211.15876 (2022).

\bibitem{schulman2017proximal} Schulman, John, et al. "Proximal policy optimization algorithms." arXiv preprint arXiv:1707.06347 (2017).

\bibitem{anderson2018evaluation} Anderson, Peter, et al. "On evaluation of embodied navigation agents." arXiv preprint arXiv:1807.06757 (2018).

\bibitem{lin2024navcot} Lin, Bingqian, et al. "NavCoT: Boosting LLM-Based Vision-and-Language Navigation via Learning Disentangled Reasoning." arXiv preprint arXiv:2403.07376 (2024).

\bibitem{NavGPT} Zhou, Gengze, Yicong Hong, and Qi Wu. "Navgpt: Explicit reasoning in vision-and-language navigation with large language models." Proceedings of the AAAI Conference on Artificial Intelligence.

\bibitem{hahn2021no} Hahn, Meera, et al. "No rl, no simulation: Learning to navigate without navigating." Advances in Neural Information Processing Systems 34 (2021): 26661-26673.

\bibitem{li2025t2isafety} Li, Lijun, et al. "T2ISafety: Benchmark for Assessing Fairness, Toxicity, and Privacy in Image Generation." arXiv preprint arXiv:2501.12612 (2025).

\bibitem{an2024agfsync} An, Jingkun, et al. "AGFSync: Leveraging AI-Generated Feedback for Preference Optimization in Text-to-Image Generation." arXiv preprint arXiv:2403.13352 (2024).


\end{thebibliography}
\end{sloppypar}

\clearpage
\beginsupplement
\section*{Appendix}
\renewcommand{\thesubsection}{S\arabic{subsection}}

\subsection{\label{chap:S1}PanNuke and MoNuSAC preprocessing}
The PanNuke dataset comprises a set of 7,901 RGB patches, each with dimensions of $256 \times 256$ pixels, which we set as the standard patch size for our analysis. In contrast, the MoNuSAC dataset encompasses 294 images of heterogeneous dimensions. To standardize the MoNuSAC images with our experiments, we implement a standardization protocol. Specifically, for images exceeding the dimensions of $256 \times 256$ pixels, we segment them into equal-sized patches and apply mirror padding to the remaining portions to avoid information loss at the peripherals. Patches with dimensions less than $128 \times 128$ pixels are excluded from the dataset due to the insufficient resolution to capture relevant cellular details. For patches where either dimension falls between 128 and 256 pixels, we employ upsampling to achieve the standard patch size. As a result, we obtain a total of 2,823 RGB patches derived from the MoNuSAC dataset for subsequent analysis. For additional details on the MoNuSAC data preparation process, refer to the source code \cite{Shvetsov_2025a}.
\clearpage

\subsection{\label{chap:S2}Data usage for the methodology}

\counterwithin{figure}{subsection}
\renewcommand{\thefigure}{S\arabic{subsection}}

\begin{figure}[h!]
    \centering
    \includegraphics[width=\textwidth, height=0.85\textheight, keepaspectratio]{images/A2.pdf}
    \caption{Overview of the methodology for cross-labeling, dataset refinement, and model comparison. (1) Cross-relabeling - training and testing cell classification models, (2) Cross-relabeling - using cell classification models to create refined dataset, (3) Fine-tuning and training models for comparison, (4) Student knowledge distillation with refined dataset}
    \label{fig:S2}
\end{figure}
\clearpage

\subsection{\label{chap:S3}Confusion matrices for classification models}
\counterwithin{figure}{subsection}
\renewcommand{\thefigure}{S\arabic{subsection}.\arabic{figure}}

\begin{figure}[h!]
    \centering
    \includegraphics[width=\textwidth, height=0.4\textheight, keepaspectratio]{images/A3_1.pdf}
    \caption{Confusion matrix for PanNuke trained model}
    \label{fig:S3.1}
\end{figure}

\begin{figure}[h!]
    \centering
    \includegraphics[width=\textwidth, height=0.4\textheight, keepaspectratio]{images/A3_2.pdf}
    \caption{Confusion matrix for MoNuSAC trained model}
    \label{fig:S3.2}
\end{figure}

\clearpage

\subsection{\label{chap:S4}Datasets cell counts}

\counterwithin{table}{subsection}
\renewcommand{\thetable}{S\arabic{subsection}}

\begin{table}[h!]
\renewcommand{\arraystretch}{2.0}
\centering
\caption{\label{tab:S4}Cell counts for PanNuke, MoNuSAC and refined datasets. Numbers in parentheses indicate preprocessed cell counts for cell classifier models training and testing.}
%\adjustbox{max width=\textwidth}{%
\begin{tabular}{|l|c|c|c|}
\hline
%\rowcolor{gray!30}
Cell type & PanNuke & MoNuSAC & Refined \\
\hline
Neoplastic & 77,403 (68,031) & - & 105,451 \\
\hline
Epithelial & 26,572 (23,207) & - & 29,926 \\
\hline
Epithelial (benign and malignant) & - & 31,402 & - \\
\hline
Inflammatory & 32,276 & - & - \\
\hline
Lymphocytes & - & 37,045 (33,104) & 65,275 \\
\hline
Neutrophils & - & 1,355 (1,252) & 3,833 \\
\hline
Macrophage & - & 1,842 (1,695) & 3,410 \\
\hline
Dead & 2,908 & - & 2,908 \\
\hline
Connective & 50,585 & - & 50,585 \\
\hline
\end{tabular}
%
%}
\end{table}



\clearpage

\subsection{\label{chap:S5}Definition of validation metrics}
\counterwithin{equation}{subsection}
\renewcommand{\theequation}{\arabic{equation}}

\subsubsection{\label{chap:S5.1}R\textsuperscript{2}}
The coefficient of determination, denoted as $R^2$, is a statistical measure that represents the proportion of variance in the dependent variable that is predictable from the independent variables. In the context of cell quantification in pathology, $R^2$ is used to assess how well the predicted quantities of different cell types in a patch align with the actual quantities observed in the ground truth data, with higher values representing more accurate quantification. $R^2$ is defined as
\begin{equation*}
R^2 = 1 - \frac{\sum_{i=1}^n (y_i - \hat{y}_i)^2}{\sum_{i=1}^n (y_i - \bar{y})^2},
\end{equation*}
where $y_i$ represents the actual number of cells of a specific type in the $i$-th image, $\hat{y}_i$ represents the predicted number of cells of that type in the $i$-th image, $\bar{y}$ is the mean of the actual numbers across all images, and $n$ is the total number of images in the dataset.

The $R^2$ metric has a range of $(-\infty, 1]$. An $R^2$ of 1 indicates perfect prediction, where all predicted values exactly match the actual values. An $R^2$ of 0 suggests that the model explains none of the variability of the response data around its mean. If $R^2$ is negative, it indicates that the model performs worse than a model that simply predicts the mean of the actual values for all observations.

\subsubsection{\label{chap:S5.2}PQ}
Panoptic Quality ($PQ$) is a comprehensive metric used to evaluate the performance of segmentation models in tasks that require both instance segmentation and classification. $PQ$ provides a single score that encapsulates both the detection accuracy (i.e., how many objects were correctly identified) and the segmentation quality (i.e., how accurately the objects' boundaries were delineated). This metric is particularly useful in multiclass scenarios where each pixel is classified into distinct categories, such as different cell types in pathology images.

$PQ$ is calculated as the product of two terms: Detection Quality ($DQ$) and Segmentation Quality ($SQ$). It can be expressed as
\begin{equation*}
PQ = DQ \cdot SQ,
\end{equation*}
where
\begin{equation*}
DQ = \frac{TP}{TP + 0.5\, FP + 0.5\, FN},
\end{equation*}
\begin{equation*}
SQ = \frac{\sum_{(p, g) \in \mathcal{M}} IoU(p, g)}{TP}.
\end{equation*}
In these formulas, $TP$ denotes the number of correctly matched instances between ground truth and prediction, $FP$ denotes the predicted instances that have no corresponding ground truth, $FN$ denotes the ground truth instances that were not detected, $IoU(p, g)$ is the Intersection over Union for a pair of matched instances $p$ (prediction) and $g$ (ground truth), and $\mathcal{M}$ is the set of matched pairs.

The $PQ$ metric is calculated for each class and is averaged across classes to provide a global performance measure.

The $PQ$ score has a range of $[0, 1.0]$, where a higher score indicates better performance in both detecting and segmenting the instances correctly. A $PQ$ of 1 signifies perfect identification and segmentation of all instances, whereas a $PQ$ of 0 indicates that no instances were correctly identified and segmented.

\clearpage

\subsection{\label{chap:S6}Segmentation and Detection quality metrics for teacher and student models}

\begin{table}[h!]
\renewcommand{\arraystretch}{2.0}
\centering
\caption{Segmentation and detection quality for student and teacher models (CI 95\%)}
\label{tab:S6}
%\adjustbox{max width=\textwidth}{%
\begin{tabular}{|l|c|c|}
\hline
%\rowcolor{gray!30}
Metric & Teacher & Student \\
\hline
$SQ_{neoplastic}$ & 0.819 (0.815--0.823) & 0.824 (0.819--0.828) \\
\hline
$SQ_{lymphocyte}$ & 0.795 (0.788--0.802) & 0.790 (0.783--0.796) \\
\hline
$SQ_{connective}$ & 0.770 (0.762--0.776) & 0.780 (0.772--0.786) \\
\hline
$SQ_{dead}$ & 0.659 (0.623--0.688) & 0.657 (0.624--0.695) \\
\hline
$SQ_{epithelial}$ & 0.780 (0.770--0.790) & 0.788 (0.779--0.797) \\
\hline
$SQ_{macrophage}$ & 0.788 (0.760--0.810) & 0.757 (0.730--0.783) \\
\hline
$SQ_{neutrofil}$ & 0.782 (0.761--0.801) & 0.775 (0.759--0.792) \\
\hline
$DQ_{neoplastic}$ & 0.706 (0.692--0.719) & 0.727 (0.712--0.741) \\
\hline
$DQ_{lymphocyte}$ & 0.675 (0.656--0.698) & 0.713 (0.691--0.734) \\
\hline
$DQ_{connective}$ & 0.566 (0.546--0.584) & 0.583 (0.565--0.602) \\
\hline
$DQ_{dead}$ & 0.410 (0.361--0.465) & 0.435 (0.306--0.561) \\
\hline
$DQ_{epithelial}$ & 0.668 (0.639--0.694) & 0.673 (0.644--0.702) \\
\hline
$DQ_{macrophage}$ & 0.657 (0.583--0.727) & 0.615 (0.531--0.703) \\
\hline
$DQ_{neutrofil}$ & 0.691 (0.625--0.753) & 0.729 (0.679--0.778) \\
\hline
\end{tabular}
%
%}
\end{table}

\clearpage

\subsection{\label{chap:S7}QuPath integration method}
We adopt an integration strategy leveraging the paquo \cite{Bayer_AG} library, a Python package that enables direct interaction with QuPath’s internal API, thereby facilitating seamless data exchange without intermediate conversion steps. The data processing pipeline (\hyperref[fig:S7]{Appendix Figure S7}) begins with the acquisition of WSIs and their associated annotations from QuPath, which are represented as Shapely \cite{Gillies_Wel_etal._2024} polygons. Utilizing paquo, we directly read, create, and modify these annotations and detections within a QuPath project in the Python environment. Images are then cropped using these polygons and processed by cell segmentation and classification models employing standard vision processing toolkits such as OpenCV, pyvips, and PyTorch. Additionally, QuPath employs Groovy scripts to initiate a Python process that starts the entire pipeline from QuPath graphical interface: fetching polygons, extracting images from them, and running deep learning model inference on the cropped images. 
The results are returned to QuPath, leveraging paquo's Python bindings to manipulate QuPath data while minimizing the computational overhead typically associated with cross-environment communication.

\counterwithin{figure}{subsection}
\renewcommand{\thefigure}{S\arabic{subsection}}

\begin{figure}[h!]
    \centering
    \includegraphics[width=\textwidth]{images/A7.pdf}
    \caption{QuPath integration workflow using Python environment}
    \label{fig:S7}
\end{figure}

Compared to traditional workflows that involve exporting annotations as GeoJSON, classifying them in Python, and reimporting them into QuPath, our approach offers several advantages. We eliminate the need to switch between programming languages, providing a cohesive and streamlined development process entirely within QuPath software and removing the necessity to use other tools. Meanwhile, we avoid storing annotations as intermediate JSON files unless required for external use or archiving. By conducting the entire inference and post-processing workflow within the Python environment, we leverage the power and flexibility of Python libraries for image processing and machine learning. This approach also enables adjustments to any set of labels and models, thereby improving its applicability.

%\hfill

The distilled model and QuPath integration code are packaged into a Docker container, enabling streamlined execution with the Docker engine. Detailed integration code and deployment instructions can be found in the GitHub repository \cite{Shvetsov_2025b}.

Despite these benefits, we acknowledge that the paquo library is a proof‑of‑concept project in its early development stage and has not been tested across all versions of QuPath.

\clearpage

\subsection{\label{chap:S8}Data and code availability statement}
All datasets, models, and code used in this study are publicly available and can be obtained from the repositories listed below. 
The PanNuke \cite{Gamper_Koohbanani_etal._2019} and MoNuSAC \cite{Verma_Kumar_etal._2021} datasets are publicly accessible, and download information along with detailed descriptions can be found in their respective articles. Preprocessing scripts for PanNuke and MoNuSAC data, as well as individual cell extraction scripts, are available on GitHub \cite{Shvetsov_2025a}. The H-Optimus foundation model used in our experiments can be downloaded from the HuggingFace repository \cite{hoptimus2024}, and model information is available on GitHub \cite{Saillard_Jenatton_etal._2024}. In addition, the integration code for QuPath and the distilled model packaged in a Docker container are provided in the repository \cite{Shvetsov_2025b}, and paquo Python library is available from the authors GitHub repository \cite{Bayer_AG}.
\clearpage

\end{document}

% \bibliography{main}

\newpage
\appendix

\section{Hyperparameters}
During the training of the attack model, the sequence length is set to 160. For fine-tuning the victim models, the sequence length is set to 1600 for LLaMA3-8B and Bloom-7B1, and 800 for GPT2-XL. The AdamW\cite{loshchilov2017decoupled} optimizer is used for all training and fine-tuning processes, with learning rates set to 5e-5 for GPT2-XL and Bloom-7B1, and 7e-5 for LLaMA3-8B, along with an epsilon value of 1e-8.

\section{Datasets}
The WikiText dataset serves as a high-quality, clean, and large-scale collection of English text extracted from Wikipedia articles, providing a solid foundation for creating the shadow dataset for the attacker's model. The ArXiv dataset is a large-scale collection of scientific papers from the arXiv repository. The OpenWebText dataset is a high-quality, large-scale corpus of English web content, curated from URLs shared on Reddit with high karma. The Pile is an 800GB, diverse English text dataset designed for training large language models, combining content from 22 high-quality sources, including books, academic papers, code, and web text. The PII dataset consists of 1,000 instances of sensitive information and includes 10 types of personally identifiable information (PII), such as phone numbers, email addresses, and home addresses, presented in a structured format. These data are randomly generated using regular expressions and do not represent real private information.

\section{Toolkits}
We use the NLTK package to measure the BLEU score, the rouge\_score library to calculate the ROUGE score, and scikit-learn to compute the cosine similarity.

\section{True-Prefix and SPT Attack Examples}
Figure \ref{fig:true_prefix_example} and Figure \ref{fig:spt_example} present two examples of True-Prefix~\cite{true-prefix} and SPT~\cite{SPT} attacks, respectively. In the True-Prefix attack, we insert real data of additional PII types, such as address or birthday, before the prompt templates, as shown in the blue sections in Figure \ref{fig:true_prefix_example}. In the SPT attack, we train on 64 PII data pairs for 5 epochs to obtain the soft prompt embeddings, which are set to a length of 10. The soft prompt embeddings are then concatenated before the prompt templates. During the training, the victim model remains frozen, with no gradient updates applied.

\begin{figure}[H]
  \includegraphics[width=\linewidth]{figures/True-Prefix_example.pdf} 
  \caption {Two True-Prefix attack examples. Blue text represents the real private data, while green and red text indicate successful and failed privacy theft, respectively.}
  \label{fig:true_prefix_example}
  \vspace{-1em}
\end{figure}

\begin{figure}[H]
  \includegraphics[width=\linewidth]{figures/SPT_example.pdf} 
  \caption {Two SPT attack examples. Orange text represents the soft prompt embeddings, with green and red text indicating successful and failed privacy theft, respectively.}
  \label{fig:spt_example}
  \vspace{-1em}
\end{figure}

\end{document}
