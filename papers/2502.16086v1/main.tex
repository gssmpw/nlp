% This must be in the first 5 lines to tell arXiv to use pdfLaTeX, which is strongly recommended.
\pdfoutput=1
% In particular, the hyperref package requires pdfLaTeX in order to break URLs across lines.

\documentclass[11pt]{article}

% Change "review" to "final" to generate the final (sometimes called camera-ready) version.
% Change to "preprint" to generate a non-anonymous version with page numbers.
% \usepackage[review]{acl}
\usepackage{acl}
% Standard package includes
\usepackage{times}
\usepackage{latexsym}
\usepackage{enumitem}
\usepackage{amsmath}
\usepackage{tcolorbox}
\usepackage{tabularx} % 支持自动调整列宽
\usepackage{lipsum}   % 仅用于生成示例文本
\usepackage{bm} % 导入bm宏包

% For proper rendering and hyphenation of words containing Latin characters (including in bib files)
\usepackage[T1]{fontenc}
% For Vietnamese characters
% \usepackage[T5]{fontenc}
% See https://www.latex-project.org/help/documentation/encguide.pdf for other character sets

% This assumes your files are encoded as UTF8
\usepackage[utf8]{inputenc}

% This is not strictly necessary, and may be commented out,
% but it will improve the layout of the manuscript,
% and will typically save some space.
\usepackage{microtype}

% This is also not strictly necessary, and may be commented out.
% However, it will improve the aesthetics of text in
% the typewriter font.
\usepackage{inconsolata}

%Including images in your LaTeX document requires adding
%additional package(s)
\usepackage{graphicx}
\usepackage{multirow}
\usepackage{booktabs}
\usepackage{array}
\usepackage{float}


% If the title and author information does not fit in the area allocated, uncomment the following
%
%\setlength\titlebox{<dim>}
%
% and set <dim> to something 5cm or larger.

\title{Stealing Training Data from Large Language Models \\ in Decentralized Training through Activation Inversion Attack}

% Author information can be set in various styles:
% For several authors from the same institution:
% \author{Author 1 \and ... \and Author n \\
%         Address line \\ ... \\ Address line}
% if the names do not fit well on one line use
%         Author 1 \\ {\bf Author 2} \\ ... \\ {\bf Author n} \\
% For authors from different institutions:
% \author{Author 1 \\ Address line \\  ... \\ Address line
%         \And  ... \And
%         Author n \\ Address line \\ ... \\ Address line}
% To start a separate ``row'' of authors use \AND, as in
% \author{Author 1 \\ Address line \\  ... \\ Address line
%         \AND
%         Author 2 \\ Address line \\ ... \\ Address line \And
%         Author 3 \\ Address line \\ ... \\ Address line}

\author{Chenxi Dai\thanks{Equal contribution} \and
        Lin Lu\footnotemark[1] \and 
        Pan Zhou\thanks{Corresponding author} \\
  Huazhong University of Science of Technology \\
  \{dcx001,loserlulin,panzhou\}@hust.edu.cn}

% \usepackage[T1]{fontenc}
% \usepackage[utf8]{inputenc}
% \usepackage{authblk}

% \author[a]{Chenxi Dai}
% \author[a]{Lin Lu}
% \author[a]{Pan Zhou C \thanks{Corresponding author: panzhou@hust.edu.cn}}
% \affil[a]{Huazhong University of Science of Technology}




%\author{
%  \textbf{First Author\textsuperscript{1}},
%  \textbf{Second Author\textsuperscript{1,2}},
%  \textbf{Third T. Author\textsuperscript{1}},
%  \textbf{Fourth Author\textsuperscript{1}},
%\\
%  \textbf{Fifth Author\textsuperscript{1,2}},
%  \textbf{Sixth Author\textsuperscript{1}},
%  \textbf{Seventh Author\textsuperscript{1}},
%  \textbf{Eighth Author \textsuperscript{1,2,3,4}},
%\\
%  \textbf{Ninth Author\textsuperscript{1}},
%  \textbf{Tenth Author\textsuperscript{1}},
%  \textbf{Eleventh E. Author\textsuperscript{1,2,3,4,5}},
%  \textbf{Twelfth Author\textsuperscript{1}},
%\\
%  \textbf{Thirteenth Author\textsuperscript{3}},
%  \textbf{Fourteenth F. Author\textsuperscript{2,4}},
%  \textbf{Fifteenth Author\textsuperscript{1}},
%  \textbf{Sixteenth Author\textsuperscript{1}},
%\\
%  \textbf{Seventeenth S. Author\textsuperscript{4,5}},
%  \textbf{Eighteenth Author\textsuperscript{3,4}},
%  \textbf{Nineteenth N. Author\textsuperscript{2,5}},
%  \textbf{Twentieth Author\textsuperscript{1}}
%\\
%\\
%  \textsuperscript{1}Affiliation 1,
%  \textsuperscript{2}Affiliation 2,
%  \textsuperscript{3}Affiliation 3,
%  \textsuperscript{4}Affiliation 4,
%  \textsuperscript{5}Affiliation 5
%\\
%  \small{
%    \textbf{Correspondence:} \href{mailto:email@domain}{email@domain}
%  }
%}

\begin{document}
\maketitle
\begin{abstract}
Decentralized training has become a resource-efficient framework to democratize the training of large language models (LLMs). However, the privacy risks associated with this framework, particularly due to the potential inclusion of sensitive data in training datasets, remain unexplored. This paper identifies a novel and realistic attack surface: the privacy leakage from training data in decentralized training, and proposes \textit{activation inversion attack} (AIA) for the first time. AIA first constructs a shadow dataset comprising text labels and corresponding activations using public datasets. Leveraging this dataset, an attack model can be trained to reconstruct the training data from activations in victim decentralized training. We conduct extensive experiments on various LLMs and publicly available datasets to demonstrate the susceptibility of decentralized training to AIA. These findings highlight the urgent need to enhance security measures in decentralized training to mitigate privacy risks in training LLMs.
\end{abstract}


% Previous research has focused on fully trained and fixed models, whereas we explore a part of the model that changes dynamically during pipeline fine-tuning. To reduce the cost of the attack, we directly use pre-trained models as shadow models, and experiments validate the rationality and effectiveness of this approach. 

% \section{Introduction}
% Deep neural networks (DNNs), particularly large language models (LLMs)~\cite{gpt3,chen2023extending,mistral,gemma2}, have achieved remarkable success and found widespread applications across various domains due to their exceptional performance~\cite{li2024ecomgpt,wu2024chateda,lu2024chameleon}. However, the performance of these models is closely tied to both their size and the scale of the datasets they are trained on~\cite{bahri2024explaining}, which in turn leads to increasingly demanding requirements in terms of device memory and computational time. For instance, the training of the DeepSeek-V3 model, with 671 billion parameters, on a dataset of 14.8 trillion tokens required 2.664 million H800 GPU hours~\cite{liu2024deepseek}. As a result, distributed training across multiple devices has become a common approach for training modern large-scale models.

% To address this challenge, existing DNN training systems often employ model parallelism~\cite{Lepikhin2020Gshard,narayanan2021efficient,zheng2022alpa}, which divides a DNN into multiple segments and places them on devices with sufficient memory. Pipeline parallelism~\cite{huang2019gpipe,narayanan2019pipedream,narayanan2021memory}, on the other hand, assigns different stages of the model to different devices in a sequential manner. Since each device can simultaneously process different stages of different data, this approach significantly enhances resource utilization. Compared to traditional data parallelism~\cite{li2014scaling,luo2020prague}, pipeline parallelism alleviates the issue of memory constraints and computation bottlenecks, making it a prevalent paradigm for training large models today.

% Deep neural networks (DNNs), particularly large language models (LLMs)~\cite{gpt3,chen2023extending,mistral,gemma2}, have achieved significant success. However, their computational demands have increased exponentially due to the massive scale of their parameters and the high requirements for training data~\cite{naveed2023comprehensive,raiaan2024review}. For instance, the DeepSeek-V3 model, which trained 671 billion parameters on 14.8 trillion tokens, consumed 2.664 million H800 GPU hours~\cite{liu2024deepseek}. This exponential growth necessitates distributed training across multiple devices, and decentralized training based on pipeline parallelism has become a key solution. Unlike traditional data parallelism~\cite{li2014scaling,luo2020prague}, pipeline parallelism~\cite{narayanan2019pipedream} strategically divides model layers across devices, enabling concurrent processing of different data batches in successive stages. This approach not only optimizes memory utilization but also alleviates computational bottlenecks. Represented by frameworks such as GPipe~\cite{huang2019gpipe} and Megatron-LM~\cite{narayanan2021efficient}, this distributed training paradigm effectively balances resource limitations with training efficiency, forming the foundation of modern LLM development.

% However, the training paradigm of pipeline parallelism differs from that of data parallelism or local training, and introduces several new risks. Existing research~\cite{thorpe2023bamboo,jang2023oobleck,duan2024parcae} predominantly focuses on addressing fault tolerance related to hardware crashes in pipeline parallelism, largely overlooking the impact of human threats. While the potential of poisoning attacks in pipeline parallelism has been explored, these studies~\cite{lu2024position} typically assume the presence of an adversary capable of randomly controlling a specific stage within the pipeline. Such an adversary could manipulate activations or gradient values at that stage, potentially delaying or completely preventing model convergence. Although these manipulations can indeed disrupt the training process, they are generally easy to detect through standard anomaly detection methods. As a result, these attacks are unlikely to cause significant, long-term damage to the training process or the model itself.

% Despite this, there are subtler forms of threats in pipeline parallelism that require further attention. Inspired by leakage from gradients in federated learning~\cite{zhu2019deep,zhao2020idlg} and embedding inversion attacks~\cite{li2023sentence,chen2024text}, this paper introduces the first activation inversion attack under decentralized training. Unlike previous work, this attack does not require modifying the transmitted values during decentralized training; instead, it only requires access to the transmitted values to achieve the attack goal. Specifically, we consider a system in which a pipeline fine-tunes a large language model, with one of the stages being honest-but-curious. During the training process, this stage performs forward propagation and gradient calculation as usual but attempts to reconstruct the original training data from intermediate data. Since this stage only possesses a portion of the model, it cannot directly obtain activation values from sentence tokens, nor can it easily reconstruct data via gradients. Our attack aims to achieve the following two goals: 1) \textbf{the shadow model} is used to mimic the behavior of the victim model, even as the victim model evolves during fine-tuning; 2) \textbf{the attack model} is trained using a dataset constructed from the shadow model and then attempts to attack the victim model, attempting to reverse-engineer the original text from the activation values.

% To achieve the first goal, we used existing pre-trained models as shadow models and demonstrated the validity of this approach through experimentation. For the second goal, we employed a variety of model architectures as attack models to explore the effectiveness of text reconstruction.

% To validate the effectiveness of our attack, we conducted extensive experiments on three popular models: GPT2-XL~\cite{gpt2}, Bloom-7B1~\cite{bloom}, and LLaMA3-8B~\cite{llama3}. The experimental results show that the perplexity of the reconstructed text is consistently around 10, indicating that the reconstructed text closely resembles the original text. To further investigate the potential harm of activation inversion attacks, we constructed a Personally Identifiable Information (PII) dataset for privacy item extraction experiments. The results demonstrate that a large number of private items can be accurately reconstructed, with near 100\% recovery rates for the PII type of birthday and job.

\section{Introduction}

Large language models (LLMs)~\cite{gpt3, chen2023extending, mistral, gemma2} have demonstrated remarkable efficacy across diverse domains~\cite{li2024ecomgpt, wu2024chateda, lu2024chameleon} due to their advanced capabilities in semantic understanding and text generation. However, their emergent abilities follow the scaling law~\cite{bahri2024explaining, naveed2023comprehensive, raiaan2024review}, which leads to state-of-the-art LLMs typically comprising billions of parameters. For instance, the DeepSeek-V3~\cite{liu2024deepseek} model, with its 671 billion parameters, requires 2,664 million H800 GPU hours for training. This resource-intensive training and fine-tuning process presents significant barriers to the democratization of LLMs. As a result, decentralized training~\cite{yuan2022decentralized, ryabinin2023swarm} is gaining increasing attention as a promising solution to mitigate these resource challenges.

Decentralized training is mainly based on parallel training (e.g., \textit{pipeline parallelism}~\cite{narayanan2019pipedream}), which distributes training computations across heterogeneous computing devices (typically GPUs) in a pipeline, with each device acting as a distinct stage. Unlike traditional federated learning (FL), which is based on data parallelism~\cite{li2014scaling, luo2020prague}, pipeline parallelism allocates model layers across devices, facilitating the concurrent processing of multiple data batches over successive stages. During decentralized training, each stage transmits activations during forward propagation and gradients during backward propagation to iteratively update model parameters. This approach enhances memory utilization and alleviates computational bottlenecks. Frameworks such as GPipe~\cite{huang2019gpipe} and Megatron-LM~\cite{narayanan2021efficient} effectively balance resource constraints with training efficiency, supporting the democratization of LLMs.

As research on the robustness of decentralized training progresses, the security vulnerabilities of this framework have become increasingly evident. However, most existing studies~\cite{thorpe2023bamboo, jang2023oobleck, duan2024parcae} primarily focus on addressing fault tolerance issues related to hardware failures in pipeline parallelism, often neglecting the impact of human threats. While some research~\cite{lu2024position} has examined the role of attackers, demonstrating that malicious stages in decentralized training can significantly disrupt training outcomes and hinder model convergence, this study typically assumes that attackers can control any stage of decentralized training. Such strong assumptions about the attackers' capabilities make the attack methods impractical in real-world training scenarios, where tampering with transmitted values is highly likely to be detected by the training initiator. Furthermore, the above studies fail to address privacy risks, which could lead to more severe consequences~\cite{bethany2024large}.

Motivated by this gap, we aim to investigate whether malicious stages in decentralized training can steal privacy without disrupting the training process. However, implementing this privacy reconstruction attack presents a significant challenge: decentralized training differs substantially from traditional training methods, such as localized training or FL. In traditional training, attackers may have access to a complete model copy~\cite{li2023sentence,morris2023text} or its inputs and corresponding outputs~\cite{huang2024transferable}. In contrast, within the decentralized training, malicious stages can only access the transmitted values between stages. This raises a critical research question: \textit{How to steal privacy, such as training data, solely through transmitted values in decentralized training?}


To address this critical research question, this paper first introduces the \textbf{\textit{\underline{A}ctivation \underline{I}nversion \underline{A}ttack}} (AIA) targeting decentralized training. Specifically, we demonstrate how a malicious stage in decentralized training can steal training data by exploiting activations through a two-step process. In the first step: \textbf{Shadow Dataset Construction}, the attacker creates a shadow dataset of text-activation pairs using a public dataset, aiming to align the data distribution of the shadow dataset with that of the actual training process. In the second step: \textbf{Attack Model Training}, the attacker trains a generative model using the shadow dataset to learn the mapping from activations to text labels. The attacker then reconstructs the corresponding training data from victim activations. In summary, the contributions of this paper are as follows:


\begin{itemize}[nolistsep, leftmargin=*, topsep=0pt]

    \item We identify a novel attack surface, marking the first attempt to steal private training data within decentralized training frameworks.

    \item We propose a two-step attack framework, AIA, that steals training data through activations in decentralized training without detection.

    \item We conduct a comprehensive evaluation of the effectiveness of AIA, demonstrating its character-level capability for training data reconstruction. Specifically, AIA achieves 62\% accuracy in stealing private emails when fine-tuning GPT2-XL.
    
\end{itemize}


\section{Related Work}

\subsection{Decentralized Training Safety}

\citet{yuan2022decentralized} initially explores decentralized training for LLMs. Several studies then examine decentralized training in slow networks~\cite{ryabinin2023swarm, wang2023cocktailsgd} and explore the development of geo-distributed training systems tailored for LLMs~\cite{gandhi2024improving, tang2024fusionllm}. While safety concerns in decentralized training have been identified in previous works~\cite{tang2023fusionai, borzunov2022training}, most existing research focuses mainly on ensuring seamless pipeline operations on preemptible devices, employing techniques such as model backup and redundant computation~\cite{thorpe2023bamboo, jang2023oobleck}. \citet{lu2024position} comprehensively evaluate the potential threats in decentralized training. However, the proposed \textit{forward attack} can be easily mitigated by detection methods, making it impractical in real-world scenarios.


% Decentralized training has emerged as the mainstream approach for training large models. \citet{yuan2022decentralized} investigates model parallelism-based training of large models in heterogeneous environments. Some research examines distributed training in the presence of slow network conditions~\cite{ryabinin2023swarm,wang2023cocktailsgd}. Other efforts focus on pipeline training on preemptible devices, employing techniques such as model backup and redundant computation~\cite{thorpe2023bamboo,jang2023oobleck}. Several works explore the development of geo-distributed training systems tailored for large language models~\cite{gandhi2024improving,tang2024fusionllm}. Furthermore, \citet{tang2023fusionai} utilizes consumer-grade GPUs for training large models.


\subsection{Data Leakage from Transmitted Values}

\noindent{\textbf{Data leakage from gradients.}}
In the context of FL, researchers such as \citet{zhu2019deep} have explored deep gradient leakage attacks on both visual and language models. 
\citet{balunovic2022lamp} uses auxiliary language models to model prior probabilities, reducing the loss through alternating continuous and discrete optimization. \citet{gupta2022recovering} first recovers a set of words from gradients, and then reconstructs the sentence from this set of words using beam search. \citet{fowl2022decepticons} and \citet{boenisch2023curious} propose a powerful threat model in which the server is malicious and can manipulate model weights, easily reconstructing the data.
\citet{wu2023learning} proposes a simple adaptive attack method that can bypass various defense mechanisms, including differential privacy and gradient compression, and successfully reconstruct the original text.
% \citet{li2022you} leverages the hidden states of conversational models to conduct attribute inference attacks. Several studies assume that attackers can access the language model's output tokens and logits, which they exploit to induce the model to inadvertently leak private data.

% Large language models perform well across a variety of tasks but also expose increasing privacy risks~\cite{das2024security,yan2024protecting}. In the context of federated learning, researchers such as \citet{zhu2019deep} have studied deep gradient leakage attacks on both visual and language models, while \citet{gupta2022recovering} and \citet{balunovic2022lamp} specifically investigate deep gradient leakage attacks in language models. \citet{li2022you} utilizes the hidden states of conversational models to conduct attribute inference attacks. Several studies assume that attackers can access the language model and obtain output tokens and logits, which they use to induce the model to leak private data. Research by \citet{carlini2021extracting} shows that large models retain training data and leak private information. Additionally, some works generate malicious prompts to induce models to output private data\cite{huang2022large,nakka2024pii}. In addition, \citet{SPT} enhances data leakage by training a set of soft prompt tokens and adding them before the prompt template.

\noindent{\textbf{Data leakage from embeddings.}}
Another line of research focuses on embedding inversion attacks, where the attacker aims to reconstruct text from embedding representations. \citet{song2020information} reconstructs 50\%-70\% of the input words from embedding models. However, word-level information alone is insufficient to fully reconstruct privacy. \citet{li2023sentence} proposes a generative embedding inversion attack that reconstructs sentences similar to the original input from embeddings. \citet{morris2023text} utilizes an iterative correction approach to reconstruct text information. \citet{huang2024transferable} investigates a black-box attack scenario, reducing the discrepancy between the surrogate model and the victim model through adversarial training. These studies assume that the victim model is fully trained and static, allowing the attacker to access the input sentence embeddings from the victim model, build a shadow dataset, and then train an attack model to reconstruct the original text. However, in decentralized training settings, the malicious stage only has access to a portion of the model, and thus cannot directly access the victim model.

\section{Preliminaries}

\subsection{Threat Model}
\label{sec:threat_model}

\noindent{\textbf{Attack scenario.}}
% We propose a decentralized training framework consisting of $K$ computation stages, where $M_i$ represents the sub-layers (e.g., decode layers in LLMs) of the $i$-th stage. During training iteration $t$, $M_i$ transmits activations $a_i^{(t)}$ to $M_{i+1}$ and gradients $g_i^{(t)}$ to $M_{i-1}$.
We consider a decentralized training scenario where the user intends to fine-tune a pre-trained model ${M}_\text{pre}$ using their private dataset $\mathcal{D}_\text{vic}$, resulting in a fine-tuned model ${M}_\text{fine}$. The framework consists of $K$ stages, where $M_i$ represents the sub-layers (e.g., decode layers in LLMs) of the $i$-th stage. During training iteration $t$, $M_i$ transmits activations $\bm a_i^{(t)}$ to $M_{i+1}$ and gradients $\bm g_i^{(t)}$ to $M_{i-1}$. However, an unmonitored decentralized training framework may introduce an honest-but-curious stage as an attacker.

\noindent{\textbf{Attacker's goals.}}
The attacker's objective is to reconstruct character-level training data $\bm d^{(t)}$ from $\mathcal{D}_\text{vic}$ during iteration $t$ in victim decentralized training. Additionally, the attacker seeks to conceal their malicious activities, executing the attack without disrupting the training process to avoid detection by the training initiator or other detection mechanisms.

\noindent{\textbf{Attacker's knowledge.}}
We assume the attacker, as the $i_\text{att}$-th stage, has access to all information related to its own stage, including the sub-layers $M_{i_\text{att}}$ and transmitted data $\bm a_{i_\text{att}}$ and $\bm g_{i_\text{att}}$. This enables the attacker to infer the architecture of ${M}_\text{fine}$ based on the structure of $M_{i_\text{att}}$. However, the attacker is assumed to have no access to other training-related information, such as transmitted data between benign stages or auxiliary information about the training data. This assumption is realistic, as it facilitates the deployment of this attack in real-world decentralized training environments.


\subsection{Motivation}
\label{sec:act_cos}

In Section \ref{sec:threat_model}, it is established that attackers can only reconstruct training data through the transmitted values during the victim model's training process, such as activations and gradients. This section discusses the challenges of using gradients to conduct such attacks and explores the feasibility of using activations to achieve similar objectives.

% In decentralized training, using gradients to steal training data faces a major challenge: The unknowability of global gradients. In decentralized training, the gradients received at each stage correspond to the current sub-layers, which significantly differs from previous studies on gradient-based privacy leakage that rely on global gradients.

In decentralized training, traditional deep gradient leakage attacks encounter a significant limitation: the unavailability of the global model and global gradients. Previous researches~\cite {zhu2019deep, gupta2022recovering, balunovic2022lamp} focus on training or searching for a set of texts that, through the victim model’s gradient, approximate the leaked gradient to reconstruct private data. However, in decentralized training, each stage only has access to a partial model and gradients, making it difficult to reconstruct data through gradients.

% In contrast, to further explore changes in activations for the same data before and after fine-tuning LLMs, we have conducted a preliminary experiment: As shown in Figure~\ref{fig:layer_idx_act_cos}, we fine-tune three common LLMs using four different datasets and record the cosine similarity of activations for the same data sample before and after fine-tuning (specific experimental settings can be found in Section 5.1). The results show that as the layer index increases, the changes in the decoder layers closer to the lm\_head layer become more pronounced. Nevertheless, the activation similarities in the final layers still exceed 50\%. During fine-tuning, we observed that the decoder layers in earlier stages show minimal variation, with activation similarities nearly reaching 100\%. This demonstrates that activations are highly correlated with the training data, making it feasible to use activations to steal training data.

In contrast, reconstructing data using the intermediate outputs of the victim model is much more straightforward, as these intermediate outputs can be directly used as inputs to train the attack model~\cite{pasquini2021unleashing,li2023sentence}. Inspired by this, we examine the cosine similarity between $\bm a_i^{(t)}$ for $\bm d^{(t)}$ in ${M}_\text{pre}$ and ${M}_\text{fine}$ across layer index $i$ (experimental details can be found in Section \ref{sec:experiment_setup}). As shown in Figure \ref{fig:layer_idx_act_cos}, activation similarity in early layers approaches 100\%, while similarity in later layers remains above 50\%. These results suggest that the activations of the same data exhibit minimal variation before and after fine-tuning, indicating a strong correlation between activations and the training data. This preliminary experiment provides key insights for our attacks in Section \ref{sec:AIA}.


% The most critical aspect of activation inversion attacks is ensuring that the features of the shadow activations closely resemble those of the victim activations. This similarity is essential for the trained attack model to have transferability. Figure \ref{fig:layer_idx_act_cos} illustrates the cosine similarity of activations across different layers before and after fine-tuning on the GPT2-XL, Bloom-7B1, and LLaMA3-8B pre-trained models using various datasets. It is observed that as the layer index increases, the changes in the decoder layers nearer to the lm\_head layer become more pronounced. Nevertheless, the activation similarities in the final layers still exceed 50\%. During the fine-tuning process, we note that the decoder layers in the earlier stages show minimal variation, with activation similarities nearly reaching 100\%. This indicates that directly using a pre-trained model as the shadow model without additional training can still produce shadow activations closely resembling the victim activations.
\begin{figure}[t]
  \includegraphics[width=\linewidth]{figures/layer_idx_cos.pdf} 
  \caption {Cosine similarity between activations for the same data in the pre-trained model and the fine-tuned model across layer index.}
  \label{fig:layer_idx_act_cos}
  \vspace{-1em}
\end{figure}


\section{AIA: \textit{Activation Inversion Attack}}
\label{sec:AIA}
\begin{figure*}[t]
  \includegraphics[width=\textwidth]{figures/Figure_2.pdf}
  % \caption{Overview of the activation inversion attack (AIA). In an honest-but-curious decentralized training system, the victim model $M_{\text{vic}}$ is fine-tuned using its private data $\mathcal{D}_{\text{vic}}$. At one stage of the pipeline, which is set to be "curious," intermediate activation values $\mathcal{D}_{\text{vic}}$ during training are recorded, and shadow activations $\mathcal{D}_{\text{sha}}$ are acquired from the shadow model $M_{\text{sha}}$. These activations are then used to train an attack model $M_{\text{att}}$, which attempts to invert the private data.}
  \caption{Overview of Activation Inversion Attack (AIA). In a decentralized training system, the victim model $M_{\text{vic}}$ undergoes fine-tuning using private data $\mathcal{D}_{\text{vic}}$, which may contain personally identifiable information values (highlighted in yellow). An honest-but-curious attacker controlling the $i_{\text{att}}$-th stage of the pipeline: (1) records intermediate activation values $\bm a_{i_\text{att}-1}^{(t)}$ captured during the training process, and (2) collects shadow activations $\mathcal{D}_{\text{sha}}$ from the shadow model $M_{\text{sha}}$ to train the attack model $M_{\text{att}}$. Finally, the attacker uses $M_{\text{att}}$ to reconstruct the private data $\mathcal{D}_{\text{vic}}$, with the red and purple text representing precisely recovered and mostly recovered PII data, respectively.}
  \label{fig:system}
  \vspace{-1em}
\end{figure*}


We introduce AIA, a framework for training data reconstruction through activations in decentralized training. During the victim model training, an attacker at the $i_\text{att}$-th stage has access to the activations $\bm a_{i_\text{att}-1}^{(t)}$ passed from $M_{i_\text{att}-1}$ during forward propagation. We denote the mapping function from the original training data $\bm d_\text{vic}^{(t)}$ to $\bm a_{i_\text{att}-1}^{(t)}$ as $f_{[1:i_\text{att}-1]}^{(t)}(\cdot)$. Therefore, we can conclude that: 
$$
\bm a_{i_\text{att}-1}^{(t)}=f_{[1:i_\text{att}-1]}^{(t)}(\bm d_\text{vic}^{(t)})
$$
The attacker's goal can thus be simplified to constructing a mapping function $\phi \approx  (f_{[1:i_\text{att}-1]}^{(t)})^{-1}(\cdot)$ that reconstructs $\bm d_\text{vic}^{(t)}$ from $\bm a_{i_\text{att}-1}^{(t)}$. AIA adopts a learning-based approach by training a generative model to perform this reconstruction. In simple terms, AIA consists of two steps: (1) \textbf{Shadow Dataset Construction}: The attacker first generates a shadow dataset containing text labels and corresponding activations leveraging a public dataset. (2) \textbf{Attack Model Training}: The attacker then uses $\mathcal{D}_\text{sha}$ to train a generative attack model ${M}_\text{att}$ that learns the mapping function $\phi$. Finally, the attacker inputs the actual activations transmitted during the victim model training into ${M}_\text{att}$ to reconstruct the training data. We provide a detailed description of these two steps in the following.



% \subsection{Pipeline Parallelism Training}
% Consider a pipeline training system in which a language model is partitioned into multiple segments, with each segment trained at a different stage. Except for the first and last stages, one intermediate stage is "curious." This curious stage does not directly interfere with the model's training process, as such actions would be easily detectable. Instead, it is primarily interested in the victim's private training data. When the victim fine-tunes the language model with their own data within this system, the curious stage covertly stores the intermediate activation values during training. These activations are subsequently used to invert and reconstruct the user’s original dataset. This process does not affect the model’s training performance, but it does require additional storage space, meaning that the user remains unaware of this activity.

% \subsection{Shadow Model}
% The primary goal of the shadow model is to generate shadow activations from the attacker's shadow dataset, ensuring that their distribution closely mirrors that of the victim activations. However, since the fine-tuning process of the victim model is dynamic, the victim activations change with each training iteration. This presents a significant challenge: maintaining the consistency of the shadow activations with the victim activations is nearly impossible. First, the attacker has no access to the victim's private data, making it difficult to align the shadow model’s activations with the victim’s. Second, fine-tuning the shadow model simultaneously with the fine-tuning of the victim model incurs substantial computational costs. The difficulty in achieving precise alignment between shadow and victim activations underscores the inherent challenges in conducting an effective activation inversion attack.

% Fortunately, we observe that pre-trained models released by various institutions and organizations have undergone thorough training, demonstrating impressive generalization capabilities. As a result, when users fine-tune these models on their own private data, the changes in activations are relatively small. This observation allows us to directly utilize the pre-trained weights of various models from Hugging Face as the shadow model. In other words, we do not need to invest any additional effort in fine-tuning the shadow model, significantly reducing the cost of the attack. The rationale behind using pre-trained model weights directly as the shadow model is further elaborated through experiments in Section \ref{sec:act_cos}.

\subsection{Step 1: Shadow Dataset Construction}

Since the attacker cannot access $\mathcal{D}_\text{vic}$, a straightforward approach is to construct a shadow dataset $\mathcal{D}_\text{sha}$ using a public dataset $\mathcal{D}_\text{pub}$. Specifically, we use the frozen pre-trained LLM $M_\text{pre}$ as the shadow model $M_\text{sha}$, with the same type of the victim model, to generate shadow activations $\bm a_\text{sha}$, i.e., 
$$
\bm a_\text{sha}=M_{\text{sha}[1:i_\text{att} -1]}(\bm d_\text{pub})
$$
where $\bm d_\text{pub} \in \mathcal{D}_\text{pub}$. The rationale for this approach is analyzed in Section \ref{sec:act_cos}: the generalizability of $M_\text{pre}$ ensures that the activations remain relatively stable when fine-tuning the victim model $M_\text{vic}$ on $\mathcal{D}_\text{vic}$, allowing us to directly leverage the pre-trained weights from HuggingFace as $M_\text{sha}$. In other words, no additional effort is required to train $M_\text{sha}$, significantly reducing the cost of AIA.

\subsection{Step 2: Attack Model Training}
Next, we focus on training ${M}_\text{att}$ using the shadow dataset $\mathcal{D}_\text{sha}=\{(\bm a_\text{sha}, \bm d_\text{pub})\}$. ${M}_\text{att}$ is designed to take activations as input and output the distribution probabilities of the generated text. It consists of a set of decoder layers and an \texttt{lm\_head} layer. Structurally, it differs from a standard language model by the absence of the initial embedding layer. 
% In terms of training methodology, the loss function is set for a text reconstruction task, rather than a text generation task commonly used in language models. 
% The training objective is defined as:
Similar to the recent work~\cite{li2023sentence}, the training objective is to minimize the standard language model loss using teacher forcing~\cite{williams1989learning}:
$$
  L = - \sum_{k=1}^{N} \log P(y_k | x_1, x_2, \dots, x_{k-1})
$$
where $y_k$ is the target word, and $x_i$ represent the input activations. 
% This approach aims to reconstruct the original text, rather than generate new content as in traditional language modeling tasks. 
Finally, we input the activations $\bm a_{i_\text{att}-1}^{(t)}$ to ${M}_\text{att}$ and obtain $\bm d_\text{vic}^{(t)}$.



% \section{Challenges}
% \subsection{Victim Activations vs Shadow Activations}
% \label{sec:act_cos}
% The most critical aspect of activation inversion attacks is ensuring that the features of the shadow activations closely resemble those of the victim activations. This similarity is essential for the trained attack model to have transferability. Figure \ref{fig:layer_idx_act_cos} illustrates the cosine similarity of activations across different layers before and after fine-tuning on the GPT2-XL, Bloom-7B1, and LLaMA3-8B pre-trained models using various datasets. It is observed that as the layer index increases, the changes in the decoder layers nearer to the lm\_head layer become more pronounced. Nevertheless, the activation similarities in the final layers still exceed 50\%. During the fine-tuning process, we note that the decoder layers in the earlier stages show minimal variation, with activation similarities nearly reaching 100\%. This indicates that directly using a pre-trained model as the shadow model without additional training can still produce shadow activations closely resembling the victim activations.
% \begin{figure}[t]
%   \includegraphics[width=\linewidth]{figures/layer_idx_cos.pdf} 
%   \caption {Cosine Similarity of Activations Before and After Fine-Tuning Across Layer Indices.}
%   \label{fig:layer_idx_act_cos}
% \end{figure}






\section{Experiments}
\subsection{Experimental Setup}
\noindent\textbf{Victim models.}
\label{sec:experiment_setup}
We conduct experiments on three models: GPT2-XL~\cite{gpt2}, Bloom-7B1~\cite{bloom}, and LLaMA3-8B~\cite{llama3}, which have 48, 30, and 32 decoder layers, respectively. We directly download the pre-trained models from HuggingFace and use them as $M_\text{sha}$ to collect $\mathcal{D}_\text{sha}$. 
To investigate the effects of AIA under extreme conditions, we fine-tune $M_\text{vic}$ for 5 epochs on the corresponding dataset to induce overfitting on the privacy data, thereby maximizing the feature gap between $\mathcal{D}_\text{vic}$ and $\mathcal{D}_\text{sha}$.
% To explore the effects of AIA under extreme conditions, we fine-tune $M_\text{vic}$ for 5 epochs on the corresponding dataset to induce overfitting on the clean data. 
The training process is divided into 6 stages, with the assumption that the third stage is malicious. 
% Therefore, unless otherwise specified, all experiments are conducted at the one-third point of each model’s layers. 
The architecture of the attack model is identical to that of the victim model, with all attack models set to 12 decoder layers. 
% During the training of the attack model, the sequence length is set to 160. For fine-tuning the victim models, the sequence length is set to 1600 for LLaMA3-8B and Bloom-7B1, and 800 for GPT2-XL. The AdamW optimizer is used for all training and fine-tuning processes, with learning rates set to 5e-5 for GPT2-XL and Bloom-7B1, and 7e-5 for LLaMA3-8B, along with an epsilon value of 1e-8.

\noindent\textbf{Datasets.}
%Large language models are typically trained for text generation tasks using an autoregressive approach. 
% We use WikiText~\cite{wikitext} as the attacker's known dataset $\mathcal{D}_\text{pub}$ to construct the shadow dataset $\mathcal{D}_\text{sha}$. The WikiText dataset is a collection of high-quality, clean, and large-scale English text extracted from Wikipedia articles. The victim datasets $\mathcal{D}_\text{vic}$ include ArXiv, OpenWebText~\cite{openwebtext}, The Pile~\cite{pile}, and a public PII dataset\footnote{https://github.com/zzzzsdaw/PII-dataset} containing sensitive information. The ArXiv dataset is a large-scale collection of scientific papers from arXiv. The OpenWebText dataset is a high-quality, large-scale corpus of English web content curated from URLs shared on Reddit with high karma. The Pile is a diverse, 800GB English text dataset designed for training large language models, combining content from 22 high-quality sources, including books, academic papers, code, and web text. The PII dataset consists of 1,000 instances of sensitive information and includes 10 personally identifiable information (PII) types, including \textit{phone numbers}, \textit{email addresses}, and \textit{home addresses}, in a structured format. Figure \ref{fig:PII_data_example} presents an example of a PII data item. These data are randomly generated using regular expressions and do not represent real private information.
We use the WikiText~\cite{wikitext} dataset as the attacker's known dataset $\mathcal{D}_\text{pub}$ to construct the shadow dataset $\mathcal{D}_\text{sha}$. The victim datasets $\mathcal{D}_\text{vic}$ include ArXiv, OpenWebText~\cite{openwebtext}, The Pile~\cite{pile}, and a public PII dataset\footnote{https://github.com/zzzzsdaw/PII-dataset}, which contains sensitive information.  An example of a PII data item is shown in Figure \ref{fig:PII_data_example}.

\begin{figure}[t]
  \includegraphics[width=\columnwidth]{figures/PII.pdf}
  \caption{An example of PII data and baseline attacks. The private data includes information such as names, phone numbers, and email addresses. The True-Prefix attack leverages other private attributes to prompt the model to generate the target private attribute, while the SPT attack employs a trained soft prompt added before the query template to extract private information.}
  \label{fig:PII_data_example}
  \vspace{-1em}
\end{figure}


\noindent\textbf{Baselines.}
In the privacy leakage experiments, we adopt the following two methods as baselines. The two methods do not apply to decentralized training, we use them solely for comparison to illustrate the potential risks of our attack. Their attack examples can be seen in Figure \ref{fig:PII_data_example}.

\begin{itemize}[nolistsep, leftmargin=*, topsep=0pt]

    \item \textit{True-Prefix Attack}~\cite{true-prefix} utilizes real prefixes from $\mathcal{D}_\text{vic}$ to prompt the model. In our experiments, we use real PII data of other types within each PII item as the prompt, attempting to induce the model to output the value of the target PII type.

    \item \textit{SPT Attack}~\cite{SPT} trains an additional set of prompt embeddings, which are appended to the original query template. We train the prompt embeddings using 64 PII data pairs, during which the victim model remains frozen and does not require gradient updates.
    
\end{itemize}



% \noindent\textit{True-Prefix Attack~\cite{true-prefix}.} The true prefix attack utilizes real prefixes from the fine-tuning dataset to prompt the model. In our experiments, we use real PII data of other types within each PII item as the prompt, attempting to induce the model to output the value of target PII type.

% \noindent\textit{SPT Attack~\cite{SPT}.} The SPT attack trains an additional set of prompt embeddings, which are appended to the original query template. We train the prompt embeddings using 64 PII data pairs, during which the victim model remains frozen and does not require gradient updates.


\noindent\textbf{Evaluation metrics.}
To evaluate the quality of text reconstruction, we employ the following four metrics. 


\begin{itemize}[nolistsep, leftmargin=*, topsep=0pt]

    \item \textit{Perplexity}~\cite{perplexity} assesses the model's capability by measuring the probability distribution of its outputs, with lower values indicating better performance.

    \item \textit{ROUGE}~\cite{rouge} measures the similarity between the generated text and reference text by comparing overlapping words or phrases.

    \item \textit{BLEU}~\cite{bleu} evaluates the similarity between generated text and reference text based on n-gram overlap and is commonly used in machine translation tasks.

    \item \textit{Embedding cosine similarity} calculates the semantic similarity between the generated text and reference text using the all-MiniLM-L6-v2 model\footnote{https://huggingface.co/sentence-transformers/all-MiniLM-L6-v2}~\cite{minilmv2}.
    
\end{itemize}

% \textbf{Perplexity}~\cite{perplexity} assesses the model's capability by measuring the probability distribution of its outputs, with lower values indicating better performance. \textbf{ROUGE}~\cite{rouge} measures the similarity between the generated text and reference text by comparing overlapping words or phrases. \textbf{BLEU}~\cite{bleu} evaluates the similarity between generated text and reference text based on n-gram overlap and is commonly used in machine translation tasks. \textbf{Embedding cosine similarity} calculates the semantic similarity between the generated text and reference text using the MiniLM model~\cite{minilmv2}.

% In the privacy leakage experiments, we evaluate the attack success rates of our method and two baselines in exactly matching the value of the target PII type. Precise matching is defined as the ability to output numbers and letters in the correct sequence while ignoring spaces and special characters, which do not affect the evaluation of precision.

In the privacy leakage experiments, we evaluate the \textit{attack success rate (ASR)} of our AIA method and two baselines in precisely recovering the values of the target PII types. Precise recovery is defined as correctly outputting the digits and letters in the correct order. During the matching process between the generated data and the original private data, spaces and special characters, such as '-', are ignored, as they do not affect the identification of private data values. The \textit{ASR} is calculated as the ratio of the number of precisely recovered data entries to the total amount of data.


% \section{Results}
\subsection{Text Reconstruction}
% tables.tex

\begin{table*}[ht]
\centering
\caption{Text reconstruction performance of GPT2-XL, Bloom-7B1, and LLaMA3-8B on four datasets. For all metrics except PPL, higher values indicate better performance.}
\label{tab:base_result}
\resizebox{0.94\textwidth}{!}{\begin{tabular}{ccc ccc ccc c}
\toprule[2pt]
\multirow{2}{*}{\textbf{Victim Model}} & \multirow{2}{*}{\textbf{Dataset}} & \multirow{2}{*}{\textbf{PPL}} & \multicolumn{3}{c}{\textbf{ROUGE}} & \multicolumn{3}{c}{\textbf{BLEU}} & \multirow{2}{*}{\textbf{COS}} \\ 
\cmidrule(lr){4-6} \cmidrule(lr){7-9}
                              &                          &                      & \textbf{ROUGE-1}   & \textbf{ROUGE-2}   & \textbf{ROUGE-L}    & \textbf{BLEU-1}    & \textbf{BLEU-2}    & \textbf{BLEU-4}    &                      \\ \hline
\multirow{4}{*}{GPT2-XL}      & PIIs                     & 3.73                 & 0.84     & 0.74     & 0.84      & 0.77     & 0.71     & 0.59     & 0.89                 \\
                              & openwebtext              & 3.09                 & 0.95     & 0.90     & 0.95      & 0.88     & 0.84     & 0.77     & 0.94                 \\
                              & arxiv                    & 5.43                 & 0.92     & 0.85     & 0.92      & 0.81     & 0.75     & 0.64     & 0.92                 \\
                              & pile                     & 1.65                 & 0.98     & 0.95     & 0.98      & 0.95     & 0.93     & 0.89     & 0.97                 \\ \hline
\multirow{4}{*}{Bloom-7B1}    & PIIs                     & 14.82                & 0.80     & 0.67     & 0.80      & 0.67     & 0.60     & 0.47     & 0.89                 \\
                              & openwebtext              & 4.64                 & 0.95     & 0.92     & 0.95      & 0.89     & 0.86     & 0.80     & 0.95                 \\
                              & arxiv                    & 15.45                & 0.91     & 0.83     & 0.90      & 0.77     & 0.70     & 0.56     & 0.90                 \\
                              & pile                     & 2.09                 & 0.97     & 0.95     & 0.97      & 0.95     & 0.93     & 0.90     & 0.95                 \\ \hline
\multirow{4}{*}{LLaMA3-8B}    & PIIs                     & 7.36                 & 0.80     & 0.67     & 0.79      & 0.73     & 0.66     & 0.54     & 0.77                 \\
                              & openwebtext              & 6.50                 & 0.93     & 0.88     & 0.93      & 0.88     & 0.84     & 0.77     & 0.88                 \\
                              & arxiv                    & 9.26                 & 0.88     & 0.78     & 0.88      & 0.80     & 0.73     & 0.60     & 0.83                 \\
                              & pile                     & 2.18                 & 0.96     & 0.93     & 0.96      & 0.94     & 0.92     & 0.89     & 0.92                 \\ \bottomrule[1.5pt]
\end{tabular}}
\vspace{-1em}
\end{table*}

% \begin{table*}[ht]
% \centering
% \caption{Text reconstruction performance of GPT-2-XL, Bloom-7B1, and LLaMA3-8B on four datasets. For all metrics except PPL, higher values indicate better performance.}
% \label{tab:base_result}
% \resizebox{\textwidth}{!}{
% \begin{tabular}{ccc|ccc|ccc|c}
% \toprule[2pt]
% % \hline
% \multirow{2}{*}{Victim Model} & \multirow{2}{*}{Dataset} & \multirow{2}{*}{PPL} & \multicolumn{3}{c|}{Rouge}      & \multicolumn{3}{c|}{Bleu}      & \multirow{2}{*}{Cos} \\ \cline{4-9}
%                               &                          &                      & Rouge1   & Rouge2   & RougeL    & Bleu1    & Bleu2    & Bleu4    &                      \\ \hline
% \multirow{4}{*}{Gpt2-xl}      & PIIs                     & 3.73                 & 0.84     & 0.74     & 0.84      & 0.77     & 0.71     & 0.59     & 0.89                 \\
%                               & openwebtext              & 3.09                 & 0.95     & 0.90     & 0.95      & 0.88     & 0.84     & 0.77     & 0.94                 \\
%                               & arxiv                    & 5.43                 & 0.92     & 0.85     & 0.92      & 0.81     & 0.75     & 0.64     & 0.92                 \\
%                               & pile                     & 1.65                 & 0.98     & 0.95     & 0.98      & 0.95     & 0.93     & 0.89     & 0.97                 \\ \hline
% \multirow{4}{*}{Bloom-7b1}    & PIIs                     & 14.82                & 0.80     & 0.67     & 0.80      & 0.67     & 0.60     & 0.47     & 0.89                 \\
%                               & openwebtext              & 4.64                 & 0.95     & 0.92     & 0.95      & 0.89     & 0.86     & 0.80     & 0.95                 \\
%                               & arxiv                    & 15.45                & 0.91     & 0.83     & 0.90      & 0.77     & 0.70     & 0.56     & 0.90                 \\
%                               & pile                     & 2.09                 & 0.97     & 0.95     & 0.97      & 0.95     & 0.93     & 0.90     & 0.95                 \\ \hline
% \multirow{4}{*}{Llama3-8b}    & PIIs                     & 7.36                 & 0.80     & 0.67     & 0.79      & 0.73     & 0.66     & 0.54     & 0.77                 \\
%                               & openwebtext              & 6.50                 & 0.93     & 0.88     & 0.93      & 0.88     & 0.84     & 0.77     & 0.88                 \\
%                               & arxiv                    & 9.26                 & 0.88     & 0.78     & 0.88      & 0.80     & 0.73     & 0.60     & 0.83                 \\
%                               & pile                     & 2.18                 & 0.96     & 0.93     & 0.96      & 0.94     & 0.92     & 0.89     & 0.92                 \\ \bottomrule[1.5pt]

% \end{tabular}
% }
% \end{table*}
Table~\ref{tab:base_result} presents the performance of AIA across different victim LLMs and datasets. The results indicate that the perplexity of the generated sentences remains below 20, with most values under 10, suggesting that the reconstructed text is relatively fluent and closely aligns with the original fine-tuning data. Both ROUGE-1 and BLEU-1 scores exceed 0.7, with the highest result reaching nearly 0.95, which confirms that the majority of words from the original fine-tuning data are accurately recovered. ROUGE-L scores are generally higher than ROUGE-2, indicating that the generated text maintains high global similarity while exhibiting slightly lower local continuity. However, this slight discontinuity in certain lexical elements has minimal impact on human readability. We further compute the cosine similarity between the embeddings of the generated text and the original text, with values ranging from 0.77 to 0.96, confirming a high level of semantic similarity. These results validate the effectiveness of AIA in reconstructing the original fine-tuning data.

\subsection{Privacy Leakage}
% \begin{table}[ht]
\centering
\caption{Comparison of the ASR between our method and baselines in stealing phone and email data.}
\label{tab:pii_compare}
\resizebox{0.8\columnwidth}{!}{
\begin{tabular}{cccc}
\toprule[2pt]
\multirow{2}{*}{Victim Model} & \multirow{2}{*}{method} & \multicolumn{2}{c}{ASR} \\
                              &                         & phone      & email      \\ \hline
\multirow{3}{*}{Gpt2-xl}      & True prefix             & 0          & 0.04       \\
                              & SPT                     & 0          & 0.02       \\
                              & ours                    & 0.25       & 0.55       \\ \hline
\multirow{3}{*}{Bloom-7b1}    & True prefix             & 0.01       & 0.18       \\
                              & SPT                     & 0          & 0.10       \\
                              & ours                    & 0.42       & 0.62       \\ \hline
\multirow{3}{*}{Llama3-8b}    & True prefix             & 0          & 0          \\
                              & SPT                     & 0          & 0          \\
                              & ours                    & 0.16       & 0.42       \\ \bottomrule[1.5pt]
\end{tabular}
}
\end{table}
\begin{table*}[ht]
    \centering
    \begin{minipage}{0.285\textwidth}
        \centering
        \makeatletter\def\@captype{table}\makeatother
        \caption{Comparison of the ASR between our AIA method and baselines in stealing phone and email data.}
\label{tab:pii_compare}
        \resizebox{\textwidth}{!}{
        \begin{tabular}{cccc}
\toprule[2pt]
\multirow{2}{*}{\textbf{Victim Model}} & \multirow{2}{*}{\textbf{Method}} & \multicolumn{2}{c}{\textbf{ASR}} \\
                              &                         & \textbf{phone}      & \textbf{email}      \\ \hline
\multirow{3}{*}{GPT2-XL}      & True-Prefix             & 0          & 0.04       \\
                              & SPT                     & 0          & 0.02       \\
                              & AIA(ours)                    & 0.25       & 0.55       \\ \hline
\multirow{3}{*}{Bloom-7B1}    & True-Prefix             & 0.01       & 0.18       \\
                              & SPT                     & 0          & 0.10       \\
                              & AIA(ours)                    & 0.42       & 0.62       \\ \hline
\multirow{3}{*}{LLaMA3-8B}    & True-Prefix             & 0          & 0          \\
                              & SPT                     & 0          & 0          \\
                              & AIA(ours)                    & 0.16       & 0.42       \\ \bottomrule[1.5pt]
\end{tabular}
}
    \end{minipage}%
    \hfill % 自动填充空间,避免表格重叠
    \begin{minipage}{0.665\textwidth}
        \setcounter{table}{4}
        \centering
        \makeatletter\def\@captype{table}\makeatother
        \caption{The impact of attack model architecture on the attack performance of AIA. Each attack model is configured with 6 decoder layers. The results are presented in terms of perplexity.}
\label{tab:attack_model}
        \resizebox{\textwidth}{!}{
        \begin{tabular}{ccc|cccc}
\toprule[2pt]
\multirow{2}{*}{\textbf{Victim Model}} & \multirow{2}{*}{\textbf{\begin{tabular}[c]{@{}c@{}}Attack Model\\ Architecture\end{tabular}}} & \textbf{Shadow Datasets} & \multicolumn{4}{c}{\textbf{Victim Datasets}}                          \\
                                       &                                                                                               & \textbf{wikitext}        & \textbf{PIIs} & \textbf{openwebtext} & \textbf{arxiv} & \textbf{pile} \\ \hline
\multirow{3}{*}{GPT2-XL}      & Mistral                                    & 1.53            & 117.45  & 44.14       & 109.31  & 24.54  \\
                              & Qwen2.5                                    & 1.71            & 410.47  & 115.35      & 301.26  & 68.74  \\
                              & GPT2                                       & 1.54            & 4.17    & 2.61        & 3.81    & 1.70   \\ \hline
\multirow{3}{*}{Bloom7B1}     & Mistral                                    & 1.54            & 7277.80 & 537.97      & 1203.97 & 445.71 \\
                              & Qwen2.5                                    & 1.48            & 7404.53 & 839.47      & 1947.76 & 651.55 \\
                              & Bloom                                      & 1.41            & 16.81   & 9.14        & 13.45   & 2.12   \\ \hline
\multirow{3}{*}{LLaMA3-8B}    & Mistral                                    & 2.60            & 2016.21 & 447.20      & 692.70  & 134.76 \\
                              & Qwen2.5                                    & 2.89            & 1810.44 & 549.28      & 1315.82 & 151.34 \\
                              & LLaMA                                      & 1.85            & 12.57   & 4.16        & 10.11   & 2.03   \\ \bottomrule[1.5pt]

\end{tabular}
}
    \end{minipage}
\end{table*}

\noindent{\textbf{Results compared with baselines.}}
We compare the ASR of AIA with the baselines on the PII types of email and phone, with the detailed results presented in Table~\ref{tab:pii_compare}. The findings indicate that our method performs effectively on both phone numbers and email addresses. For instance, the Bloom-7B1 model achieves precise recovery rates of 41\% for phone numbers and 61\% for email addresses. Even the relatively less effective LLaMA3-8B model accurately recovers 15\% of phone numbers and 41\% of email addresses. 

In contrast, the \textit{True-Prefix Attack} and \textit{SPT Attack} exhibit poor performance, showing minimal success in recovering phone numbers. On the Bloom-7B1 model, both baselines recover only a small portion of email addresses, with ASR of 18\% and 10\%, respectively. We hypothesize that this discrepancy arises from the structure of the PII dataset, where email prefixes consist of a person's name combined with random numbers, enhancing the model's memory of the email. The GPT2-XL model recovers only 2\% to 4\% of email addresses, significantly lower than Bloom-7B1, likely due to its smaller size and weaker capacity for data retention. Notably, neither baseline is able to recover any private data accurately on the LLaMA3-8B model. This may be attributed to the LLaMA3-8B model's alignment and data protection mechanisms implemented during pre-training, which results in the frequent generation of placeholders such as “[email protected]”.

% making it resistant to being induced to output private data. The frequent generation of placeholders such as “[email protected]” further supports this observation.

\begin{figure*}[t]
  \includegraphics[width=\textwidth]{figures/PII_attack_example2.pdf}
  \caption{Three comparative examples of generated texts versus original data. The yellow text represents the original PII data, while the red and purple texts represent precisely recovered and mostly recovered PII data, respectively. The text recovery performance improves from left to right.}
  \label{fig:PII_attack_example}
  \vspace{-1em}
\end{figure*}

\setcounter{table}{2}
\begin{table}[ht]
\centering
\caption{The ASR of AIA on all models in precisely recovering the seven PII types: fax, birthday, SSN, address, job, bitcoin, and UUID.}
\label{tab:pii_attack_result}
\resizebox{0.5\textwidth}{!}{
\begin{tabular}{cccccccc}
\toprule[2pt]
\multicolumn{1}{l}{}           & \textbf{fax}      & \textbf{birthday} & \textbf{SSN}      & \textbf{address}  & \textbf{job}      & \textbf{bitcoin}  & \textbf{UUID}     \\ \hline
\multicolumn{1}{c|}{GPT2-XL}   & 0.25     & 1.00     & 0.76     & 0.56     & 0.97     & 0.22     & 0.17     \\
\multicolumn{1}{c|}{Bloom-7B1} & 0.48     & 0.99     & 0.57     & 0.57     & 0.98     & 0.04     & 0.04     \\
\multicolumn{1}{c|}{LLaMA3-8B} & 0.20     & 0.95     & 0.38     & 0.41     & 0.89     & 0.03     & 0.10     \\ \bottomrule[1.5pt]
\end{tabular}
}
\vspace{-2em}
\end{table}

\noindent{\textbf{Results on various PII types.}}
Table~\ref{tab:pii_attack_result} presents the ASR of AIA in precisely recovering the seven PII types: fax, birthday, SSN, address, job, bitcoin, and UUID. Remarkably, the ASR for birthdays and jobs approaches 100\%. Birthdays, which are short and highly structured numerical sequences, likely benefit from the model's pre-training exposure to similar formats, resulting in minimal changes to their semantic encoding after fine-tuning. Jobs, typically consisting of one to three words, are relatively easier to recover compared to other PII types. This observation is further supported by the ROUGE-1 and BLEU-1 results on the PII dataset across different victim LLMs shown in Table~\ref{tab:base_result}.

All victim models exhibit strong recovery performance for PII types other than Bitcoin addresses and UUID, with recovery rates generally ranging from one-third to over half of the data. Owing to the inherent irregularity and extended length characteristics of Bitcoin addresses and UUIDs, precise reconstruction is significantly more challenging. Specifically, only the GPT2-XL model achieves a recovery rate of approximately 20\% for the two PII types, while the ASR for Bloom-7B1 and LLaMA3-8B remains below 10\%. Notably, even in cases of incomplete reconstruction, the generated outputs maintain substantial proximity to ground truth values, exhibiting only minor character-level discrepancies in alphanumeric sequences (e.g., single-letter substitutions or partial numeric mismatches).

Figure~\ref{fig:PII_attack_example} shows three comparison examples between the generated text and the original private data, with the quality of text reconstruction improving from left to right. The majority of common words and PII data can be precisely recovered, as indicated by the red highlights in the figure. However, the recovery of less frequent words (e.g., "Bitcoin") and special characters (e.g., "@") tends to be less successful. Additionally, the recovery of named entities may occasionally be imprecise. For long character sequences, such as phone numbers or UUIDs, over 80\% of the characters are typically recovered, although some minor errors in individual characters or capitalization issues may occur, as highlighted in purple in the figure.
% Achieving precise recovery of such machine-irrelevant character sequences remains a challenging task.

\subsection{Ablation Study}
To explore the factors influencing the attack performance of AIA, we conducted three sets of ablation experiments on the decoder layer index, model size, and attack model architecture. The conclusions are as follows: 
\begin{itemize}[nolistsep, leftmargin=*, topsep=0pt]
    \item As the layer index increases, the attack performance decreases; however, the original private data can still be recovered to some extent. 
    \item The attack performance is independent of model size and AIA performs well in all model sizes.
    \item The attack performance is highly sensitive to the architecture of the attack model, with different architectures leading to poorer attack results.
\end{itemize}

\subsubsection{Decoder Layer Index}
Figure~\ref{fig:layer_idx} illustrates the trend of PPL on GPT2-XL and Bloom-7B1 models as the attacker's decoder layer index varies. The results show that as the decoder layer index increases, i.e., as the data leakage layer moves closer to the output layers, the overall attack effectiveness declines. This observation aligns with the trend described in Section \ref{sec:act_cos}, where the cosine similarity of activations before and after fine-tuning decreases as the decoder layer index increases. The decline in attack performance can be attributed to the greater changes in the activations of the decoder layers that is closer to the output layer during fine-tuning. 
% These layers are heavily involved in generating the final output and, therefore, undergo more significant updates to adapt to the fine-tuning dataset. As a result, the original patterns in these layers that could be exploited for text inversion become less stable, leading to reduced attack effectiveness.

Interestingly, when the cosine similarity of activations before and after fine-tuning drops below 60\% for a particular decoder layer, the perplexity of the generated text remains below 40. This indicates that the generated sentences become less natural, with noticeable grammatical or contextual inconsistencies, which suggests a reduction in the fluency and coherence of the generated texts. However, despite these linguistic limitations, the attacker is still able to infer the original fine-tuning data to a certain extent. This highlights the robustness of AIA, even when the stage controlled by the attacker is positioned further back in the pipeline.

% \begin{tcolorbox}[colframe=black,colback=gray!10,arc=3mm,boxrule=0.5mm]
% \textbf{Takeaways:} As the layer index increases, attack performance decreases, though some private data can still be recovered.
% \end{tcolorbox}

\begin{figure}[t]
  % \includegraphics[width=0.48\linewidth]{figures/gpt2_layer_idx.pdf} \hfill
  % \includegraphics[width=0.48\linewidth]{figures/bloom_layer_idx.pdf}
  \includegraphics[width=\linewidth]{figures/layer_idx_ppl.pdf} 
  \caption {The attack performance of AIA on GPT2-XL and Bloom-7B1 models as the attacker's decoder layer index varies, with the attack performance generally decreasing as the layer index increases.}
  \label{fig:layer_idx}
  \vspace{-1em}
\end{figure}

\subsubsection{Model Size}
\begin{table}[htbp]
  \centering
  \caption{Model size comparison of ViT and \memt}
  \label{tab:model_size}

\resizebox{\linewidth}{!}{%
    \begin{tabular}{lcc}
    \toprule
    \toprule
    \multicolumn{1}{p{4.57em}}{\centering{\textbf{Model \newline{}Name}}}  & \multicolumn{1}{p{5em}}{\centering{\textbf{Model Size\newline{} (MB)}}} & \multicolumn{1}{p{14.215em}}{\centering{\textbf{Size Increase \newline{}from ViT-Base (\%)}}} \\
    \midrule
    ViT-Base & 86    & - \\
    \memt(Our) & 88    & 2.4 \\
    \bottomrule
    \bottomrule
    \end{tabular}%
}
\end{table}


Table~\ref{tab:model_size} systematically presents the experimental results for GPT2 and Bloom models with varying parameter scales. To ensure comprehensive experiments, we select three representative configurations for each model family: the GPT2 series includes 355M, 774M, and 1.5B parameter variants, while the Bloom series comprises 560M, 1.7B, and 7.1B parameter configurations. 
% Notably, in the attack model architecture design, we implemented an adaptive depth configuration strategy that dynamically sets the decoder layers to one-third of the target model's depth (e.g., employing an 8-layer decoder for the 24-layer GPT2-355M model). 
The experimental results demonstrate that the attack performance of AIA is highly dependent on the victim dataset, and it maintains stable performance across different model sizes, with most PPL consistently below 10, ROUGE-L scores exceeding 0.9, and BLEU-4 scores above 0.6 in most cases. 
% These quantitative findings underscore the robustness and generalizability of AIA in attacking models of varying sizes.

% \begin{tcolorbox}[colframe=black,colback=gray!10,arc=3mm,boxrule=0.5mm]
% \textbf{Takeaways:} AIA performs well across models of different sizes.
% \end{tcolorbox}

\subsubsection{Attack Model Architecture}
% \begin{table}[ht]
\centering
\caption{The Impact of Attack Model Architecture on Attack Performance. Each attack model is configured with 6 decoder layers. The results are presented in terms of PPL.}
\label{tab:attack_model}
\resizebox{\columnwidth}{!}{
\begin{tabular}{ccc|cccc}
\toprule[2pt]
\multirow{2}{*}{Victim Model} & \multirow{2}{*}{Attack Model Architecture} & Shadow Datasets & \multicolumn{4}{c}{Victim Datasets}      \\
                              &                                            & wikitext        & PIIs    & openwebtext & arxiv   & pile   \\ \hline
\multirow{3}{*}{GPT2-xl}      & Mistral                                    & 1.53            & 117.45  & 44.14       & 109.31  & 24.54  \\
                              & Qwen2.5                                    & 1.71            & 410.47  & 115.35      & 301.26  & 68.74  \\
                              & Gpt2                                       & 1.54            & 4.17    & 2.61        & 3.81    & 1.70   \\ \hline
\multirow{3}{*}{Bloom7b1}     & Mistral                                    & 1.54            & 7277.80 & 537.97      & 1203.97 & 445.71 \\
                              & Qwen2.5                                    & 1.48            & 7404.53 & 839.47      & 1947.76 & 651.55 \\
                              & Bloom                                      & 1.41            & 16.81   & 9.14        & 13.45   & 2.12   \\ \hline
\multirow{3}{*}{Llama3-8B}    & Mistral                                    & 2.60            & 2016.21 & 447.20      & 692.70  & 134.76 \\
                              & Qwen2.5                                    & 2.89            & 1810.44 & 549.28      & 1315.82 & 151.34 \\
                              & Llama                                      & 1.85            & 12.57   & 4.16        & 10.11   & 2.03   \\ \bottomrule[1.5pt]

\end{tabular}
}
\end{table}
To explore the impact of the attack model architecture on attack performance, we conduct experiments using Mistral~\cite{mistral} and Qwen2.5~\cite{qwen2.5} as attack model architectures and compare them to the victim model architecture. Each attack model is configured with six decoder layers. As shown in Table \ref{tab:attack_model}, while all attack models exhibit excellent performance when trained on the shadow dataset, their effectiveness significantly declines when transitioning to inverting the victim dataset after switching the attack model architecture. Notably, even the best-performing configuration on GPT2-XL still yields perplexity values ranging from 24 to 120. On the Bloom-7B1 and LLaMA3-8B models, the perplexity can even reach values above a thousand, rendering AIA almost completely ineffective.

% \begin{tcolorbox}[colframe=black,colback=gray!10,arc=3mm,boxrule=0.5mm]
% \textbf{Takeaways:} AIA is highly sensitive to the architecture of the attack model.
% \end{tcolorbox}

% Two main factors contribute to these results. First, the variation in attack performance is partially due to the differing distributions of the datasets used. Since the shadow dataset and victim dataset may have different underlying characteristics, it is expected that a model trained on one would struggle to generalize well to the other. Secondly, the intermediate activation values are highly intertwined with the model architecture itself. Each model operates within its own distinct semantic space, meaning that the activations in Mistral and Qwen2.5 differ substantially from those in the victim models. This architectural misalignment leads to overfitting in the attack model, as it fails to generalize effectively to the new data. Therefore, even though the attack models perform well during training on the shadow dataset, they struggle to perform similarly when applied to victim datasets with different architecture-specific nuances.





\section{Conclusion}
% In this paper, we propose an honest-but-curious pipeline training system and introduce a text inversion attack based on intermediate activations. Extensive experimental evaluations demonstrate that modern pretrained large language models exhibit strong generalization capabilities, such that fine-tuning with specific datasets leads to minimal updates to the model. We establish the feasibility of recovering unknown fine-tuning data from pretrained models. Furthermore, we use a set of PII datasets to evaluate the potential for privacy data leakage, showing that a significant portion of private data can be accurately recovered. Although text inversion attacks in pipeline training systems can have severe consequences, defenses against such attacks remain underexplored. We call on researchers to address this critical privacy risk and to develop effective, low-cost defense mechanisms to counteract text inversion attacks.
In this paper, we explore the privacy risks inherent in decentralized training, particularly in scenarios where an honest-but-curious attacker exists in the pipeline. Despite lacking access to the complete model weights, we demonstrate the feasibility of simulating the victim model using a pre-trained model and introduce Activation Inversion Attack (AIA). We conduct extensive experiments on various large language models and public datasets to emphasize the effectiveness of our attack. As the application of decentralized training continues to grow, we call for the development of effective defense measures to mitigate the risk of AIA.


\section*{Limitations}
Our method has a key limitation: the architecture of the attack model must be consistent with that of the clean model. While the attack model performs well on the shadow dataset when using different architectures, its effectiveness significantly decreases when applied to the clean dataset. This constraint limits the flexibility in choosing the attack model. Additionally, the generated text exhibits issues such as lack of fluency, inconsistencies in letter casing, errors with special characters, uncommon words, and difficulty in accurately recovering long sequences. These observations indicate that our method is influenced by the challenges of transferring to unknown data distributions and the variations introduced during model fine-tuning.


\section*{Ethics Statement}
We declare that all authors of this paper adhere to the ACM Code of Ethics and uphold its code of conduct. This paper investigates activation inversion attack in decentralized training. The objective of our work is to highlight the potential data leakage risks associated with decentralized training, aiming to encourage the community to give greater attention to privacy protection in such settings and to advocate for measures to prevent such information leaks. No real sensitive data is used in our experiments; all experiments are conducted with publicly available datasets. The data in the PII dataset we use is randomly generated and does not represent actual private information. All models employed in this study are open-source and thus do not pose any threat to proprietary models.



% Bibliography entries for the entire Anthology, followed by custom entries
%\bibliography{anthology,custom}
% Custom bibliography entries only
% This must be in the first 5 lines to tell arXiv to use pdfLaTeX, which is strongly recommended.
\pdfoutput=1
% In particular, the hyperref package requires pdfLaTeX in order to break URLs across lines.

\documentclass[11pt]{article}

% Change "review" to "final" to generate the final (sometimes called camera-ready) version.
% Change to "preprint" to generate a non-anonymous version with page numbers.
\usepackage{acl}

% Standard package includes
\usepackage{times}
\usepackage{latexsym}

% Draw tables
\usepackage{booktabs}
\usepackage{multirow}
\usepackage{xcolor}
\usepackage{colortbl}
\usepackage{array} 
\usepackage{amsmath}

\newcolumntype{C}{>{\centering\arraybackslash}p{0.07\textwidth}}
% For proper rendering and hyphenation of words containing Latin characters (including in bib files)
\usepackage[T1]{fontenc}
% For Vietnamese characters
% \usepackage[T5]{fontenc}
% See https://www.latex-project.org/help/documentation/encguide.pdf for other character sets
% This assumes your files are encoded as UTF8
\usepackage[utf8]{inputenc}

% This is not strictly necessary, and may be commented out,
% but it will improve the layout of the manuscript,
% and will typically save some space.
\usepackage{microtype}
\DeclareMathOperator*{\argmax}{arg\,max}
% This is also not strictly necessary, and may be commented out.
% However, it will improve the aesthetics of text in
% the typewriter font.
\usepackage{inconsolata}

%Including images in your LaTeX document requires adding
%additional package(s)
\usepackage{graphicx}
% If the title and author information does not fit in the area allocated, uncomment the following
%
%\setlength\titlebox{<dim>}
%
% and set <dim> to something 5cm or larger.

\title{Wi-Chat: Large Language Model Powered Wi-Fi Sensing}

% Author information can be set in various styles:
% For several authors from the same institution:
% \author{Author 1 \and ... \and Author n \\
%         Address line \\ ... \\ Address line}
% if the names do not fit well on one line use
%         Author 1 \\ {\bf Author 2} \\ ... \\ {\bf Author n} \\
% For authors from different institutions:
% \author{Author 1 \\ Address line \\  ... \\ Address line
%         \And  ... \And
%         Author n \\ Address line \\ ... \\ Address line}
% To start a separate ``row'' of authors use \AND, as in
% \author{Author 1 \\ Address line \\  ... \\ Address line
%         \AND
%         Author 2 \\ Address line \\ ... \\ Address line \And
%         Author 3 \\ Address line \\ ... \\ Address line}

% \author{First Author \\
%   Affiliation / Address line 1 \\
%   Affiliation / Address line 2 \\
%   Affiliation / Address line 3 \\
%   \texttt{email@domain} \\\And
%   Second Author \\
%   Affiliation / Address line 1 \\
%   Affiliation / Address line 2 \\
%   Affiliation / Address line 3 \\
%   \texttt{email@domain} \\}
% \author{Haohan Yuan \qquad Haopeng Zhang\thanks{corresponding author} \\ 
%   ALOHA Lab, University of Hawaii at Manoa \\
%   % Affiliation / Address line 2 \\
%   % Affiliation / Address line 3 \\
%   \texttt{\{haohany,haopengz\}@hawaii.edu}}
  
\author{
{Haopeng Zhang$\dag$\thanks{These authors contributed equally to this work.}, Yili Ren$\ddagger$\footnotemark[1], Haohan Yuan$\dag$, Jingzhe Zhang$\ddagger$, Yitong Shen$\ddagger$} \\
ALOHA Lab, University of Hawaii at Manoa$\dag$, University of South Florida$\ddagger$ \\
\{haopengz, haohany\}@hawaii.edu\\
\{yiliren, jingzhe, shen202\}@usf.edu\\}



  
%\author{
%  \textbf{First Author\textsuperscript{1}},
%  \textbf{Second Author\textsuperscript{1,2}},
%  \textbf{Third T. Author\textsuperscript{1}},
%  \textbf{Fourth Author\textsuperscript{1}},
%\\
%  \textbf{Fifth Author\textsuperscript{1,2}},
%  \textbf{Sixth Author\textsuperscript{1}},
%  \textbf{Seventh Author\textsuperscript{1}},
%  \textbf{Eighth Author \textsuperscript{1,2,3,4}},
%\\
%  \textbf{Ninth Author\textsuperscript{1}},
%  \textbf{Tenth Author\textsuperscript{1}},
%  \textbf{Eleventh E. Author\textsuperscript{1,2,3,4,5}},
%  \textbf{Twelfth Author\textsuperscript{1}},
%\\
%  \textbf{Thirteenth Author\textsuperscript{3}},
%  \textbf{Fourteenth F. Author\textsuperscript{2,4}},
%  \textbf{Fifteenth Author\textsuperscript{1}},
%  \textbf{Sixteenth Author\textsuperscript{1}},
%\\
%  \textbf{Seventeenth S. Author\textsuperscript{4,5}},
%  \textbf{Eighteenth Author\textsuperscript{3,4}},
%  \textbf{Nineteenth N. Author\textsuperscript{2,5}},
%  \textbf{Twentieth Author\textsuperscript{1}}
%\\
%\\
%  \textsuperscript{1}Affiliation 1,
%  \textsuperscript{2}Affiliation 2,
%  \textsuperscript{3}Affiliation 3,
%  \textsuperscript{4}Affiliation 4,
%  \textsuperscript{5}Affiliation 5
%\\
%  \small{
%    \textbf{Correspondence:} \href{mailto:email@domain}{email@domain}
%  }
%}

\begin{document}
\maketitle
\begin{abstract}
Recent advancements in Large Language Models (LLMs) have demonstrated remarkable capabilities across diverse tasks. However, their potential to integrate physical model knowledge for real-world signal interpretation remains largely unexplored. In this work, we introduce Wi-Chat, the first LLM-powered Wi-Fi-based human activity recognition system. We demonstrate that LLMs can process raw Wi-Fi signals and infer human activities by incorporating Wi-Fi sensing principles into prompts. Our approach leverages physical model insights to guide LLMs in interpreting Channel State Information (CSI) data without traditional signal processing techniques. Through experiments on real-world Wi-Fi datasets, we show that LLMs exhibit strong reasoning capabilities, achieving zero-shot activity recognition. These findings highlight a new paradigm for Wi-Fi sensing, expanding LLM applications beyond conventional language tasks and enhancing the accessibility of wireless sensing for real-world deployments.
\end{abstract}

\section{Introduction}

In today’s rapidly evolving digital landscape, the transformative power of web technologies has redefined not only how services are delivered but also how complex tasks are approached. Web-based systems have become increasingly prevalent in risk control across various domains. This widespread adoption is due their accessibility, scalability, and ability to remotely connect various types of users. For example, these systems are used for process safety management in industry~\cite{kannan2016web}, safety risk early warning in urban construction~\cite{ding2013development}, and safe monitoring of infrastructural systems~\cite{repetto2018web}. Within these web-based risk management systems, the source search problem presents a huge challenge. Source search refers to the task of identifying the origin of a risky event, such as a gas leak and the emission point of toxic substances. This source search capability is crucial for effective risk management and decision-making.

Traditional approaches to implementing source search capabilities into the web systems often rely on solely algorithmic solutions~\cite{ristic2016study}. These methods, while relatively straightforward to implement, often struggle to achieve acceptable performances due to algorithmic local optima and complex unknown environments~\cite{zhao2020searching}. More recently, web crowdsourcing has emerged as a promising alternative for tackling the source search problem by incorporating human efforts in these web systems on-the-fly~\cite{zhao2024user}. This approach outsources the task of addressing issues encountered during the source search process to human workers, leveraging their capabilities to enhance system performance.

These solutions often employ a human-AI collaborative way~\cite{zhao2023leveraging} where algorithms handle exploration-exploitation and report the encountered problems while human workers resolve complex decision-making bottlenecks to help the algorithms getting rid of local deadlocks~\cite{zhao2022crowd}. Although effective, this paradigm suffers from two inherent limitations: increased operational costs from continuous human intervention, and slow response times of human workers due to sequential decision-making. These challenges motivate our investigation into developing autonomous systems that preserve human-like reasoning capabilities while reducing dependency on massive crowdsourced labor.

Furthermore, recent advancements in large language models (LLMs)~\cite{chang2024survey} and multi-modal LLMs (MLLMs)~\cite{huang2023chatgpt} have unveiled promising avenues for addressing these challenges. One clear opportunity involves the seamless integration of visual understanding and linguistic reasoning for robust decision-making in search tasks. However, whether large models-assisted source search is really effective and efficient for improving the current source search algorithms~\cite{ji2022source} remains unknown. \textit{To address the research gap, we are particularly interested in answering the following two research questions in this work:}

\textbf{\textit{RQ1: }}How can source search capabilities be integrated into web-based systems to support decision-making in time-sensitive risk management scenarios? 
% \sq{I mention ``time-sensitive'' here because I feel like we shall say something about the response time -- LLM has to be faster than humans}

\textbf{\textit{RQ2: }}How can MLLMs and LLMs enhance the effectiveness and efficiency of existing source search algorithms? 

% \textit{\textbf{RQ2:}} To what extent does the performance of large models-assisted search align with or approach the effectiveness of human-AI collaborative search? 

To answer the research questions, we propose a novel framework called Auto-\
S$^2$earch (\textbf{Auto}nomous \textbf{S}ource \textbf{Search}) and implement a prototype system that leverages advanced web technologies to simulate real-world conditions for zero-shot source search. Unlike traditional methods that rely on pre-defined heuristics or extensive human intervention, AutoS$^2$earch employs a carefully designed prompt that encapsulates human rationales, thereby guiding the MLLM to generate coherent and accurate scene descriptions from visual inputs about four directional choices. Based on these language-based descriptions, the LLM is enabled to determine the optimal directional choice through chain-of-thought (CoT) reasoning. Comprehensive empirical validation demonstrates that AutoS$^2$-\ 
earch achieves a success rate of 95–98\%, closely approaching the performance of human-AI collaborative search across 20 benchmark scenarios~\cite{zhao2023leveraging}. 

Our work indicates that the role of humans in future web crowdsourcing tasks may evolve from executors to validators or supervisors. Furthermore, incorporating explanations of LLM decisions into web-based system interfaces has the potential to help humans enhance task performance in risk control.






\section{Related Work}
\label{sec:relatedworks}

% \begin{table*}[t]
% \centering 
% \renewcommand\arraystretch{0.98}
% \fontsize{8}{10}\selectfont \setlength{\tabcolsep}{0.4em}
% \begin{tabular}{@{}lc|cc|cc|cc@{}}
% \toprule
% \textbf{Methods}           & \begin{tabular}[c]{@{}c@{}}\textbf{Training}\\ \textbf{Paradigm}\end{tabular} & \begin{tabular}[c]{@{}c@{}}\textbf{$\#$ PT Data}\\ \textbf{(Tokens)}\end{tabular} & \begin{tabular}[c]{@{}c@{}}\textbf{$\#$ IFT Data}\\ \textbf{(Samples)}\end{tabular} & \textbf{Code}  & \begin{tabular}[c]{@{}c@{}}\textbf{Natural}\\ \textbf{Language}\end{tabular} & \begin{tabular}[c]{@{}c@{}}\textbf{Action}\\ \textbf{Trajectories}\end{tabular} & \begin{tabular}[c]{@{}c@{}}\textbf{API}\\ \textbf{Documentation}\end{tabular}\\ \midrule 
% NexusRaven~\citep{srinivasan2023nexusraven} & IFT & - & - & \textcolor{green}{\CheckmarkBold} & \textcolor{green}{\CheckmarkBold} &\textcolor{red}{\XSolidBrush}&\textcolor{red}{\XSolidBrush}\\
% AgentInstruct~\citep{zeng2023agenttuning} & IFT & - & 2k & \textcolor{green}{\CheckmarkBold} & \textcolor{green}{\CheckmarkBold} &\textcolor{red}{\XSolidBrush}&\textcolor{red}{\XSolidBrush} \\
% AgentEvol~\citep{xi2024agentgym} & IFT & - & 14.5k & \textcolor{green}{\CheckmarkBold} & \textcolor{green}{\CheckmarkBold} &\textcolor{green}{\CheckmarkBold}&\textcolor{red}{\XSolidBrush} \\
% Gorilla~\citep{patil2023gorilla}& IFT & - & 16k & \textcolor{green}{\CheckmarkBold} & \textcolor{green}{\CheckmarkBold} &\textcolor{red}{\XSolidBrush}&\textcolor{green}{\CheckmarkBold}\\
% OpenFunctions-v2~\citep{patil2023gorilla} & IFT & - & 65k & \textcolor{green}{\CheckmarkBold} & \textcolor{green}{\CheckmarkBold} &\textcolor{red}{\XSolidBrush}&\textcolor{green}{\CheckmarkBold}\\
% LAM~\citep{zhang2024agentohana} & IFT & - & 42.6k & \textcolor{green}{\CheckmarkBold} & \textcolor{green}{\CheckmarkBold} &\textcolor{green}{\CheckmarkBold}&\textcolor{red}{\XSolidBrush} \\
% xLAM~\citep{liu2024apigen} & IFT & - & 60k & \textcolor{green}{\CheckmarkBold} & \textcolor{green}{\CheckmarkBold} &\textcolor{green}{\CheckmarkBold}&\textcolor{red}{\XSolidBrush} \\\midrule
% LEMUR~\citep{xu2024lemur} & PT & 90B & 300k & \textcolor{green}{\CheckmarkBold} & \textcolor{green}{\CheckmarkBold} &\textcolor{green}{\CheckmarkBold}&\textcolor{red}{\XSolidBrush}\\
% \rowcolor{teal!12} \method & PT & 103B & 95k & \textcolor{green}{\CheckmarkBold} & \textcolor{green}{\CheckmarkBold} & \textcolor{green}{\CheckmarkBold} & \textcolor{green}{\CheckmarkBold} \\
% \bottomrule
% \end{tabular}
% \caption{Summary of existing tuning- and pretraining-based LLM agents with their training sample sizes. "PT" and "IFT" denote "Pre-Training" and "Instruction Fine-Tuning", respectively. }
% \label{tab:related}
% \end{table*}

\begin{table*}[ht]
\begin{threeparttable}
\centering 
\renewcommand\arraystretch{0.98}
\fontsize{7}{9}\selectfont \setlength{\tabcolsep}{0.2em}
\begin{tabular}{@{}l|c|c|ccc|cc|cc|cccc@{}}
\toprule
\textbf{Methods} & \textbf{Datasets}           & \begin{tabular}[c]{@{}c@{}}\textbf{Training}\\ \textbf{Paradigm}\end{tabular} & \begin{tabular}[c]{@{}c@{}}\textbf{\# PT Data}\\ \textbf{(Tokens)}\end{tabular} & \begin{tabular}[c]{@{}c@{}}\textbf{\# IFT Data}\\ \textbf{(Samples)}\end{tabular} & \textbf{\# APIs} & \textbf{Code}  & \begin{tabular}[c]{@{}c@{}}\textbf{Nat.}\\ \textbf{Lang.}\end{tabular} & \begin{tabular}[c]{@{}c@{}}\textbf{Action}\\ \textbf{Traj.}\end{tabular} & \begin{tabular}[c]{@{}c@{}}\textbf{API}\\ \textbf{Doc.}\end{tabular} & \begin{tabular}[c]{@{}c@{}}\textbf{Func.}\\ \textbf{Call}\end{tabular} & \begin{tabular}[c]{@{}c@{}}\textbf{Multi.}\\ \textbf{Step}\end{tabular}  & \begin{tabular}[c]{@{}c@{}}\textbf{Plan}\\ \textbf{Refine}\end{tabular}  & \begin{tabular}[c]{@{}c@{}}\textbf{Multi.}\\ \textbf{Turn}\end{tabular}\\ \midrule 
\multicolumn{13}{l}{\emph{Instruction Finetuning-based LLM Agents for Intrinsic Reasoning}}  \\ \midrule
FireAct~\cite{chen2023fireact} & FireAct & IFT & - & 2.1K & 10 & \textcolor{red}{\XSolidBrush} &\textcolor{green}{\CheckmarkBold} &\textcolor{green}{\CheckmarkBold}  & \textcolor{red}{\XSolidBrush} &\textcolor{green}{\CheckmarkBold} & \textcolor{red}{\XSolidBrush} &\textcolor{green}{\CheckmarkBold} & \textcolor{red}{\XSolidBrush} \\
ToolAlpaca~\cite{tang2023toolalpaca} & ToolAlpaca & IFT & - & 4.0K & 400 & \textcolor{red}{\XSolidBrush} &\textcolor{green}{\CheckmarkBold} &\textcolor{green}{\CheckmarkBold} & \textcolor{red}{\XSolidBrush} &\textcolor{green}{\CheckmarkBold} & \textcolor{red}{\XSolidBrush}  &\textcolor{green}{\CheckmarkBold} & \textcolor{red}{\XSolidBrush}  \\
ToolLLaMA~\cite{qin2023toolllm} & ToolBench & IFT & - & 12.7K & 16,464 & \textcolor{red}{\XSolidBrush} &\textcolor{green}{\CheckmarkBold} &\textcolor{green}{\CheckmarkBold} &\textcolor{red}{\XSolidBrush} &\textcolor{green}{\CheckmarkBold}&\textcolor{green}{\CheckmarkBold}&\textcolor{green}{\CheckmarkBold} &\textcolor{green}{\CheckmarkBold}\\
AgentEvol~\citep{xi2024agentgym} & AgentTraj-L & IFT & - & 14.5K & 24 &\textcolor{red}{\XSolidBrush} & \textcolor{green}{\CheckmarkBold} &\textcolor{green}{\CheckmarkBold}&\textcolor{red}{\XSolidBrush} &\textcolor{green}{\CheckmarkBold}&\textcolor{red}{\XSolidBrush} &\textcolor{red}{\XSolidBrush} &\textcolor{green}{\CheckmarkBold}\\
Lumos~\cite{yin2024agent} & Lumos & IFT  & - & 20.0K & 16 &\textcolor{red}{\XSolidBrush} & \textcolor{green}{\CheckmarkBold} & \textcolor{green}{\CheckmarkBold} &\textcolor{red}{\XSolidBrush} & \textcolor{green}{\CheckmarkBold} & \textcolor{green}{\CheckmarkBold} &\textcolor{red}{\XSolidBrush} & \textcolor{green}{\CheckmarkBold}\\
Agent-FLAN~\cite{chen2024agent} & Agent-FLAN & IFT & - & 24.7K & 20 &\textcolor{red}{\XSolidBrush} & \textcolor{green}{\CheckmarkBold} & \textcolor{green}{\CheckmarkBold} &\textcolor{red}{\XSolidBrush} & \textcolor{green}{\CheckmarkBold}& \textcolor{green}{\CheckmarkBold}&\textcolor{red}{\XSolidBrush} & \textcolor{green}{\CheckmarkBold}\\
AgentTuning~\citep{zeng2023agenttuning} & AgentInstruct & IFT & - & 35.0K & - &\textcolor{red}{\XSolidBrush} & \textcolor{green}{\CheckmarkBold} & \textcolor{green}{\CheckmarkBold} &\textcolor{red}{\XSolidBrush} & \textcolor{green}{\CheckmarkBold} &\textcolor{red}{\XSolidBrush} &\textcolor{red}{\XSolidBrush} & \textcolor{green}{\CheckmarkBold}\\\midrule
\multicolumn{13}{l}{\emph{Instruction Finetuning-based LLM Agents for Function Calling}} \\\midrule
NexusRaven~\citep{srinivasan2023nexusraven} & NexusRaven & IFT & - & - & 116 & \textcolor{green}{\CheckmarkBold} & \textcolor{green}{\CheckmarkBold}  & \textcolor{green}{\CheckmarkBold} &\textcolor{red}{\XSolidBrush} & \textcolor{green}{\CheckmarkBold} &\textcolor{red}{\XSolidBrush} &\textcolor{red}{\XSolidBrush}&\textcolor{red}{\XSolidBrush}\\
Gorilla~\citep{patil2023gorilla} & Gorilla & IFT & - & 16.0K & 1,645 & \textcolor{green}{\CheckmarkBold} &\textcolor{red}{\XSolidBrush} &\textcolor{red}{\XSolidBrush}&\textcolor{green}{\CheckmarkBold} &\textcolor{green}{\CheckmarkBold} &\textcolor{red}{\XSolidBrush} &\textcolor{red}{\XSolidBrush} &\textcolor{red}{\XSolidBrush}\\
OpenFunctions-v2~\citep{patil2023gorilla} & OpenFunctions-v2 & IFT & - & 65.0K & - & \textcolor{green}{\CheckmarkBold} & \textcolor{green}{\CheckmarkBold} &\textcolor{red}{\XSolidBrush} &\textcolor{green}{\CheckmarkBold} &\textcolor{green}{\CheckmarkBold} &\textcolor{red}{\XSolidBrush} &\textcolor{red}{\XSolidBrush} &\textcolor{red}{\XSolidBrush}\\
API Pack~\cite{guo2024api} & API Pack & IFT & - & 1.1M & 11,213 &\textcolor{green}{\CheckmarkBold} &\textcolor{red}{\XSolidBrush} &\textcolor{green}{\CheckmarkBold} &\textcolor{red}{\XSolidBrush} &\textcolor{green}{\CheckmarkBold} &\textcolor{red}{\XSolidBrush}&\textcolor{red}{\XSolidBrush}&\textcolor{red}{\XSolidBrush}\\ 
LAM~\citep{zhang2024agentohana} & AgentOhana & IFT & - & 42.6K & - & \textcolor{green}{\CheckmarkBold} & \textcolor{green}{\CheckmarkBold} &\textcolor{green}{\CheckmarkBold}&\textcolor{red}{\XSolidBrush} &\textcolor{green}{\CheckmarkBold}&\textcolor{red}{\XSolidBrush}&\textcolor{green}{\CheckmarkBold}&\textcolor{green}{\CheckmarkBold}\\
xLAM~\citep{liu2024apigen} & APIGen & IFT & - & 60.0K & 3,673 & \textcolor{green}{\CheckmarkBold} & \textcolor{green}{\CheckmarkBold} &\textcolor{green}{\CheckmarkBold}&\textcolor{red}{\XSolidBrush} &\textcolor{green}{\CheckmarkBold}&\textcolor{red}{\XSolidBrush}&\textcolor{green}{\CheckmarkBold}&\textcolor{green}{\CheckmarkBold}\\\midrule
\multicolumn{13}{l}{\emph{Pretraining-based LLM Agents}}  \\\midrule
% LEMUR~\citep{xu2024lemur} & PT & 90B & 300.0K & - & \textcolor{green}{\CheckmarkBold} & \textcolor{green}{\CheckmarkBold} &\textcolor{green}{\CheckmarkBold}&\textcolor{red}{\XSolidBrush} & \textcolor{red}{\XSolidBrush} &\textcolor{green}{\CheckmarkBold} &\textcolor{red}{\XSolidBrush}&\textcolor{red}{\XSolidBrush}\\
\rowcolor{teal!12} \method & \dataset & PT & 103B & 95.0K  & 76,537  & \textcolor{green}{\CheckmarkBold} & \textcolor{green}{\CheckmarkBold} & \textcolor{green}{\CheckmarkBold} & \textcolor{green}{\CheckmarkBold} & \textcolor{green}{\CheckmarkBold} & \textcolor{green}{\CheckmarkBold} & \textcolor{green}{\CheckmarkBold} & \textcolor{green}{\CheckmarkBold}\\
\bottomrule
\end{tabular}
% \begin{tablenotes}
%     \item $^*$ In addition, the StarCoder-API can offer 4.77M more APIs.
% \end{tablenotes}
\caption{Summary of existing instruction finetuning-based LLM agents for intrinsic reasoning and function calling, along with their training resources and sample sizes. "PT" and "IFT" denote "Pre-Training" and "Instruction Fine-Tuning", respectively.}
\vspace{-2ex}
\label{tab:related}
\end{threeparttable}
\end{table*}

\noindent \textbf{Prompting-based LLM Agents.} Due to the lack of agent-specific pre-training corpus, existing LLM agents rely on either prompt engineering~\cite{hsieh2023tool,lu2024chameleon,yao2022react,wang2023voyager} or instruction fine-tuning~\cite{chen2023fireact,zeng2023agenttuning} to understand human instructions, decompose high-level tasks, generate grounded plans, and execute multi-step actions. 
However, prompting-based methods mainly depend on the capabilities of backbone LLMs (usually commercial LLMs), failing to introduce new knowledge and struggling to generalize to unseen tasks~\cite{sun2024adaplanner,zhuang2023toolchain}. 

\noindent \textbf{Instruction Finetuning-based LLM Agents.} Considering the extensive diversity of APIs and the complexity of multi-tool instructions, tool learning inherently presents greater challenges than natural language tasks, such as text generation~\cite{qin2023toolllm}.
Post-training techniques focus more on instruction following and aligning output with specific formats~\cite{patil2023gorilla,hao2024toolkengpt,qin2023toolllm,schick2024toolformer}, rather than fundamentally improving model knowledge or capabilities. 
Moreover, heavy fine-tuning can hinder generalization or even degrade performance in non-agent use cases, potentially suppressing the original base model capabilities~\cite{ghosh2024a}.

\noindent \textbf{Pretraining-based LLM Agents.} While pre-training serves as an essential alternative, prior works~\cite{nijkamp2023codegen,roziere2023code,xu2024lemur,patil2023gorilla} have primarily focused on improving task-specific capabilities (\eg, code generation) instead of general-domain LLM agents, due to single-source, uni-type, small-scale, and poor-quality pre-training data. 
Existing tool documentation data for agent training either lacks diverse real-world APIs~\cite{patil2023gorilla, tang2023toolalpaca} or is constrained to single-tool or single-round tool execution. 
Furthermore, trajectory data mostly imitate expert behavior or follow function-calling rules with inferior planning and reasoning, failing to fully elicit LLMs' capabilities and handle complex instructions~\cite{qin2023toolllm}. 
Given a wide range of candidate API functions, each comprising various function names and parameters available at every planning step, identifying globally optimal solutions and generalizing across tasks remains highly challenging.



\section{Preliminaries}
\label{Preliminaries}
\begin{figure*}[t]
    \centering
    \includegraphics[width=0.95\linewidth]{fig/HealthGPT_Framework.png}
    \caption{The \ourmethod{} architecture integrates hierarchical visual perception and H-LoRA, employing a task-specific hard router to select visual features and H-LoRA plugins, ultimately generating outputs with an autoregressive manner.}
    \label{fig:architecture}
\end{figure*}
\noindent\textbf{Large Vision-Language Models.} 
The input to a LVLM typically consists of an image $x^{\text{img}}$ and a discrete text sequence $x^{\text{txt}}$. The visual encoder $\mathcal{E}^{\text{img}}$ converts the input image $x^{\text{img}}$ into a sequence of visual tokens $\mathcal{V} = [v_i]_{i=1}^{N_v}$, while the text sequence $x^{\text{txt}}$ is mapped into a sequence of text tokens $\mathcal{T} = [t_i]_{i=1}^{N_t}$ using an embedding function $\mathcal{E}^{\text{txt}}$. The LLM $\mathcal{M_\text{LLM}}(\cdot|\theta)$ models the joint probability of the token sequence $\mathcal{U} = \{\mathcal{V},\mathcal{T}\}$, which is expressed as:
\begin{equation}
    P_\theta(R | \mathcal{U}) = \prod_{i=1}^{N_r} P_\theta(r_i | \{\mathcal{U}, r_{<i}\}),
\end{equation}
where $R = [r_i]_{i=1}^{N_r}$ is the text response sequence. The LVLM iteratively generates the next token $r_i$ based on $r_{<i}$. The optimization objective is to minimize the cross-entropy loss of the response $\mathcal{R}$.
% \begin{equation}
%     \mathcal{L}_{\text{VLM}} = \mathbb{E}_{R|\mathcal{U}}\left[-\log P_\theta(R | \mathcal{U})\right]
% \end{equation}
It is worth noting that most LVLMs adopt a design paradigm based on ViT, alignment adapters, and pre-trained LLMs\cite{liu2023llava,liu2024improved}, enabling quick adaptation to downstream tasks.


\noindent\textbf{VQGAN.}
VQGAN~\cite{esser2021taming} employs latent space compression and indexing mechanisms to effectively learn a complete discrete representation of images. VQGAN first maps the input image $x^{\text{img}}$ to a latent representation $z = \mathcal{E}(x)$ through a encoder $\mathcal{E}$. Then, the latent representation is quantized using a codebook $\mathcal{Z} = \{z_k\}_{k=1}^K$, generating a discrete index sequence $\mathcal{I} = [i_m]_{m=1}^N$, where $i_m \in \mathcal{Z}$ represents the quantized code index:
\begin{equation}
    \mathcal{I} = \text{Quantize}(z|\mathcal{Z}) = \arg\min_{z_k \in \mathcal{Z}} \| z - z_k \|_2.
\end{equation}
In our approach, the discrete index sequence $\mathcal{I}$ serves as a supervisory signal for the generation task, enabling the model to predict the index sequence $\hat{\mathcal{I}}$ from input conditions such as text or other modality signals.  
Finally, the predicted index sequence $\hat{\mathcal{I}}$ is upsampled by the VQGAN decoder $G$, generating the high-quality image $\hat{x}^\text{img} = G(\hat{\mathcal{I}})$.



\noindent\textbf{Low Rank Adaptation.} 
LoRA\cite{hu2021lora} effectively captures the characteristics of downstream tasks by introducing low-rank adapters. The core idea is to decompose the bypass weight matrix $\Delta W\in\mathbb{R}^{d^{\text{in}} \times d^{\text{out}}}$ into two low-rank matrices $ \{A \in \mathbb{R}^{d^{\text{in}} \times r}, B \in \mathbb{R}^{r \times d^{\text{out}}} \}$, where $ r \ll \min\{d^{\text{in}}, d^{\text{out}}\} $, significantly reducing learnable parameters. The output with the LoRA adapter for the input $x$ is then given by:
\begin{equation}
    h = x W_0 + \alpha x \Delta W/r = x W_0 + \alpha xAB/r,
\end{equation}
where matrix $ A $ is initialized with a Gaussian distribution, while the matrix $ B $ is initialized as a zero matrix. The scaling factor $ \alpha/r $ controls the impact of $ \Delta W $ on the model.

\section{HealthGPT}
\label{Method}


\subsection{Unified Autoregressive Generation.}  
% As shown in Figure~\ref{fig:architecture}, 
\ourmethod{} (Figure~\ref{fig:architecture}) utilizes a discrete token representation that covers both text and visual outputs, unifying visual comprehension and generation as an autoregressive task. 
For comprehension, $\mathcal{M}_\text{llm}$ receives the input joint sequence $\mathcal{U}$ and outputs a series of text token $\mathcal{R} = [r_1, r_2, \dots, r_{N_r}]$, where $r_i \in \mathcal{V}_{\text{txt}}$, and $\mathcal{V}_{\text{txt}}$ represents the LLM's vocabulary:
\begin{equation}
    P_\theta(\mathcal{R} \mid \mathcal{U}) = \prod_{i=1}^{N_r} P_\theta(r_i \mid \mathcal{U}, r_{<i}).
\end{equation}
For generation, $\mathcal{M}_\text{llm}$ first receives a special start token $\langle \text{START\_IMG} \rangle$, then generates a series of tokens corresponding to the VQGAN indices $\mathcal{I} = [i_1, i_2, \dots, i_{N_i}]$, where $i_j \in \mathcal{V}_{\text{vq}}$, and $\mathcal{V}_{\text{vq}}$ represents the index range of VQGAN. Upon completion of generation, the LLM outputs an end token $\langle \text{END\_IMG} \rangle$:
\begin{equation}
    P_\theta(\mathcal{I} \mid \mathcal{U}) = \prod_{j=1}^{N_i} P_\theta(i_j \mid \mathcal{U}, i_{<j}).
\end{equation}
Finally, the generated index sequence $\mathcal{I}$ is fed into the decoder $G$, which reconstructs the target image $\hat{x}^{\text{img}} = G(\mathcal{I})$.

\subsection{Hierarchical Visual Perception}  
Given the differences in visual perception between comprehension and generation tasks—where the former focuses on abstract semantics and the latter emphasizes complete semantics—we employ ViT to compress the image into discrete visual tokens at multiple hierarchical levels.
Specifically, the image is converted into a series of features $\{f_1, f_2, \dots, f_L\}$ as it passes through $L$ ViT blocks.

To address the needs of various tasks, the hidden states are divided into two types: (i) \textit{Concrete-grained features} $\mathcal{F}^{\text{Con}} = \{f_1, f_2, \dots, f_k\}, k < L$, derived from the shallower layers of ViT, containing sufficient global features, suitable for generation tasks; 
(ii) \textit{Abstract-grained features} $\mathcal{F}^{\text{Abs}} = \{f_{k+1}, f_{k+2}, \dots, f_L\}$, derived from the deeper layers of ViT, which contain abstract semantic information closer to the text space, suitable for comprehension tasks.

The task type $T$ (comprehension or generation) determines which set of features is selected as the input for the downstream large language model:
\begin{equation}
    \mathcal{F}^{\text{img}}_T =
    \begin{cases}
        \mathcal{F}^{\text{Con}}, & \text{if } T = \text{generation task} \\
        \mathcal{F}^{\text{Abs}}, & \text{if } T = \text{comprehension task}
    \end{cases}
\end{equation}
We integrate the image features $\mathcal{F}^{\text{img}}_T$ and text features $\mathcal{T}$ into a joint sequence through simple concatenation, which is then fed into the LLM $\mathcal{M}_{\text{llm}}$ for autoregressive generation.
% :
% \begin{equation}
%     \mathcal{R} = \mathcal{M}_{\text{llm}}(\mathcal{U}|\theta), \quad \mathcal{U} = [\mathcal{F}^{\text{img}}_T; \mathcal{T}]
% \end{equation}
\subsection{Heterogeneous Knowledge Adaptation}
We devise H-LoRA, which stores heterogeneous knowledge from comprehension and generation tasks in separate modules and dynamically routes to extract task-relevant knowledge from these modules. 
At the task level, for each task type $ T $, we dynamically assign a dedicated H-LoRA submodule $ \theta^T $, which is expressed as:
\begin{equation}
    \mathcal{R} = \mathcal{M}_\text{LLM}(\mathcal{U}|\theta, \theta^T), \quad \theta^T = \{A^T, B^T, \mathcal{R}^T_\text{outer}\}.
\end{equation}
At the feature level for a single task, H-LoRA integrates the idea of Mixture of Experts (MoE)~\cite{masoudnia2014mixture} and designs an efficient matrix merging and routing weight allocation mechanism, thus avoiding the significant computational delay introduced by matrix splitting in existing MoELoRA~\cite{luo2024moelora}. Specifically, we first merge the low-rank matrices (rank = r) of $ k $ LoRA experts into a unified matrix:
\begin{equation}
    \mathbf{A}^{\text{merged}}, \mathbf{B}^{\text{merged}} = \text{Concat}(\{A_i\}_1^k), \text{Concat}(\{B_i\}_1^k),
\end{equation}
where $ \mathbf{A}^{\text{merged}} \in \mathbb{R}^{d^\text{in} \times rk} $ and $ \mathbf{B}^{\text{merged}} \in \mathbb{R}^{rk \times d^\text{out}} $. The $k$-dimension routing layer generates expert weights $ \mathcal{W} \in \mathbb{R}^{\text{token\_num} \times k} $ based on the input hidden state $ x $, and these are expanded to $ \mathbb{R}^{\text{token\_num} \times rk} $ as follows:
\begin{equation}
    \mathcal{W}^\text{expanded} = \alpha k \mathcal{W} / r \otimes \mathbf{1}_r,
\end{equation}
where $ \otimes $ denotes the replication operation.
The overall output of H-LoRA is computed as:
\begin{equation}
    \mathcal{O}^\text{H-LoRA} = (x \mathbf{A}^{\text{merged}} \odot \mathcal{W}^\text{expanded}) \mathbf{B}^{\text{merged}},
\end{equation}
where $ \odot $ represents element-wise multiplication. Finally, the output of H-LoRA is added to the frozen pre-trained weights to produce the final output:
\begin{equation}
    \mathcal{O} = x W_0 + \mathcal{O}^\text{H-LoRA}.
\end{equation}
% In summary, H-LoRA is a task-based dynamic PEFT method that achieves high efficiency in single-task fine-tuning.

\subsection{Training Pipeline}

\begin{figure}[t]
    \centering
    \hspace{-4mm}
    \includegraphics[width=0.94\linewidth]{fig/data.pdf}
    \caption{Data statistics of \texttt{VL-Health}. }
    \label{fig:data}
\end{figure}
\noindent \textbf{1st Stage: Multi-modal Alignment.} 
In the first stage, we design separate visual adapters and H-LoRA submodules for medical unified tasks. For the medical comprehension task, we train abstract-grained visual adapters using high-quality image-text pairs to align visual embeddings with textual embeddings, thereby enabling the model to accurately describe medical visual content. During this process, the pre-trained LLM and its corresponding H-LoRA submodules remain frozen. In contrast, the medical generation task requires training concrete-grained adapters and H-LoRA submodules while keeping the LLM frozen. Meanwhile, we extend the textual vocabulary to include multimodal tokens, enabling the support of additional VQGAN vector quantization indices. The model trains on image-VQ pairs, endowing the pre-trained LLM with the capability for image reconstruction. This design ensures pixel-level consistency of pre- and post-LVLM. The processes establish the initial alignment between the LLM’s outputs and the visual inputs.

\noindent \textbf{2nd Stage: Heterogeneous H-LoRA Plugin Adaptation.}  
The submodules of H-LoRA share the word embedding layer and output head but may encounter issues such as bias and scale inconsistencies during training across different tasks. To ensure that the multiple H-LoRA plugins seamlessly interface with the LLMs and form a unified base, we fine-tune the word embedding layer and output head using a small amount of mixed data to maintain consistency in the model weights. Specifically, during this stage, all H-LoRA submodules for different tasks are kept frozen, with only the word embedding layer and output head being optimized. Through this stage, the model accumulates foundational knowledge for unified tasks by adapting H-LoRA plugins.

\begin{table*}[!t]
\centering
\caption{Comparison of \ourmethod{} with other LVLMs and unified multi-modal models on medical visual comprehension tasks. \textbf{Bold} and \underline{underlined} text indicates the best performance and second-best performance, respectively.}
\resizebox{\textwidth}{!}{
\begin{tabular}{c|lcc|cccccccc|c}
\toprule
\rowcolor[HTML]{E9F3FE} &  &  &  & \multicolumn{2}{c}{\textbf{VQA-RAD \textuparrow}} & \multicolumn{2}{c}{\textbf{SLAKE \textuparrow}} & \multicolumn{2}{c}{\textbf{PathVQA \textuparrow}} &  &  &  \\ 
\cline{5-10}
\rowcolor[HTML]{E9F3FE}\multirow{-2}{*}{\textbf{Type}} & \multirow{-2}{*}{\textbf{Model}} & \multirow{-2}{*}{\textbf{\# Params}} & \multirow{-2}{*}{\makecell{\textbf{Medical} \\ \textbf{LVLM}}} & \textbf{close} & \textbf{all} & \textbf{close} & \textbf{all} & \textbf{close} & \textbf{all} & \multirow{-2}{*}{\makecell{\textbf{MMMU} \\ \textbf{-Med}}\textuparrow} & \multirow{-2}{*}{\textbf{OMVQA}\textuparrow} & \multirow{-2}{*}{\textbf{Avg. \textuparrow}} \\ 
\midrule \midrule
\multirow{9}{*}{\textbf{Comp. Only}} 
& Med-Flamingo & 8.3B & \Large \ding{51} & 58.6 & 43.0 & 47.0 & 25.5 & 61.9 & 31.3 & 28.7 & 34.9 & 41.4 \\
& LLaVA-Med & 7B & \Large \ding{51} & 60.2 & 48.1 & 58.4 & 44.8 & 62.3 & 35.7 & 30.0 & 41.3 & 47.6 \\
& HuatuoGPT-Vision & 7B & \Large \ding{51} & 66.9 & 53.0 & 59.8 & 49.1 & 52.9 & 32.0 & 42.0 & 50.0 & 50.7 \\
& BLIP-2 & 6.7B & \Large \ding{55} & 43.4 & 36.8 & 41.6 & 35.3 & 48.5 & 28.8 & 27.3 & 26.9 & 36.1 \\
& LLaVA-v1.5 & 7B & \Large \ding{55} & 51.8 & 42.8 & 37.1 & 37.7 & 53.5 & 31.4 & 32.7 & 44.7 & 41.5 \\
& InstructBLIP & 7B & \Large \ding{55} & 61.0 & 44.8 & 66.8 & 43.3 & 56.0 & 32.3 & 25.3 & 29.0 & 44.8 \\
& Yi-VL & 6B & \Large \ding{55} & 52.6 & 42.1 & 52.4 & 38.4 & 54.9 & 30.9 & 38.0 & 50.2 & 44.9 \\
& InternVL2 & 8B & \Large \ding{55} & 64.9 & 49.0 & 66.6 & 50.1 & 60.0 & 31.9 & \underline{43.3} & 54.5 & 52.5\\
& Llama-3.2 & 11B & \Large \ding{55} & 68.9 & 45.5 & 72.4 & 52.1 & 62.8 & 33.6 & 39.3 & 63.2 & 54.7 \\
\midrule
\multirow{5}{*}{\textbf{Comp. \& Gen.}} 
& Show-o & 1.3B & \Large \ding{55} & 50.6 & 33.9 & 31.5 & 17.9 & 52.9 & 28.2 & 22.7 & 45.7 & 42.6 \\
& Unified-IO 2 & 7B & \Large \ding{55} & 46.2 & 32.6 & 35.9 & 21.9 & 52.5 & 27.0 & 25.3 & 33.0 & 33.8 \\
& Janus & 1.3B & \Large \ding{55} & 70.9 & 52.8 & 34.7 & 26.9 & 51.9 & 27.9 & 30.0 & 26.8 & 33.5 \\
& \cellcolor[HTML]{DAE0FB}HealthGPT-M3 & \cellcolor[HTML]{DAE0FB}3.8B & \cellcolor[HTML]{DAE0FB}\Large \ding{51} & \cellcolor[HTML]{DAE0FB}\underline{73.7} & \cellcolor[HTML]{DAE0FB}\underline{55.9} & \cellcolor[HTML]{DAE0FB}\underline{74.6} & \cellcolor[HTML]{DAE0FB}\underline{56.4} & \cellcolor[HTML]{DAE0FB}\underline{78.7} & \cellcolor[HTML]{DAE0FB}\underline{39.7} & \cellcolor[HTML]{DAE0FB}\underline{43.3} & \cellcolor[HTML]{DAE0FB}\underline{68.5} & \cellcolor[HTML]{DAE0FB}\underline{61.3} \\
& \cellcolor[HTML]{DAE0FB}HealthGPT-L14 & \cellcolor[HTML]{DAE0FB}14B & \cellcolor[HTML]{DAE0FB}\Large \ding{51} & \cellcolor[HTML]{DAE0FB}\textbf{77.7} & \cellcolor[HTML]{DAE0FB}\textbf{58.3} & \cellcolor[HTML]{DAE0FB}\textbf{76.4} & \cellcolor[HTML]{DAE0FB}\textbf{64.5} & \cellcolor[HTML]{DAE0FB}\textbf{85.9} & \cellcolor[HTML]{DAE0FB}\textbf{44.4} & \cellcolor[HTML]{DAE0FB}\textbf{49.2} & \cellcolor[HTML]{DAE0FB}\textbf{74.4} & \cellcolor[HTML]{DAE0FB}\textbf{66.4} \\
\bottomrule
\end{tabular}
}
\label{tab:results}
\end{table*}
\begin{table*}[ht]
    \centering
    \caption{The experimental results for the four modality conversion tasks.}
    \resizebox{\textwidth}{!}{
    \begin{tabular}{l|ccc|ccc|ccc|ccc}
        \toprule
        \rowcolor[HTML]{E9F3FE} & \multicolumn{3}{c}{\textbf{CT to MRI (Brain)}} & \multicolumn{3}{c}{\textbf{CT to MRI (Pelvis)}} & \multicolumn{3}{c}{\textbf{MRI to CT (Brain)}} & \multicolumn{3}{c}{\textbf{MRI to CT (Pelvis)}} \\
        \cline{2-13}
        \rowcolor[HTML]{E9F3FE}\multirow{-2}{*}{\textbf{Model}}& \textbf{SSIM $\uparrow$} & \textbf{PSNR $\uparrow$} & \textbf{MSE $\downarrow$} & \textbf{SSIM $\uparrow$} & \textbf{PSNR $\uparrow$} & \textbf{MSE $\downarrow$} & \textbf{SSIM $\uparrow$} & \textbf{PSNR $\uparrow$} & \textbf{MSE $\downarrow$} & \textbf{SSIM $\uparrow$} & \textbf{PSNR $\uparrow$} & \textbf{MSE $\downarrow$} \\
        \midrule \midrule
        pix2pix & 71.09 & 32.65 & 36.85 & 59.17 & 31.02 & 51.91 & 78.79 & 33.85 & 28.33 & 72.31 & 32.98 & 36.19 \\
        CycleGAN & 54.76 & 32.23 & 40.56 & 54.54 & 30.77 & 55.00 & 63.75 & 31.02 & 52.78 & 50.54 & 29.89 & 67.78 \\
        BBDM & {71.69} & {32.91} & {34.44} & 57.37 & 31.37 & 48.06 & \textbf{86.40} & 34.12 & 26.61 & {79.26} & 33.15 & 33.60 \\
        Vmanba & 69.54 & 32.67 & 36.42 & {63.01} & {31.47} & {46.99} & 79.63 & 34.12 & 26.49 & 77.45 & 33.53 & 31.85 \\
        DiffMa & 71.47 & 32.74 & 35.77 & 62.56 & 31.43 & 47.38 & 79.00 & {34.13} & {26.45} & 78.53 & {33.68} & {30.51} \\
        \rowcolor[HTML]{DAE0FB}HealthGPT-M3 & \underline{79.38} & \underline{33.03} & \underline{33.48} & \underline{71.81} & \underline{31.83} & \underline{43.45} & {85.06} & \textbf{34.40} & \textbf{25.49} & \underline{84.23} & \textbf{34.29} & \textbf{27.99} \\
        \rowcolor[HTML]{DAE0FB}HealthGPT-L14 & \textbf{79.73} & \textbf{33.10} & \textbf{32.96} & \textbf{71.92} & \textbf{31.87} & \textbf{43.09} & \underline{85.31} & \underline{34.29} & \underline{26.20} & \textbf{84.96} & \underline{34.14} & \underline{28.13} \\
        \bottomrule
    \end{tabular}
    }
    \label{tab:conversion}
\end{table*}

\noindent \textbf{3rd Stage: Visual Instruction Fine-Tuning.}  
In the third stage, we introduce additional task-specific data to further optimize the model and enhance its adaptability to downstream tasks such as medical visual comprehension (e.g., medical QA, medical dialogues, and report generation) or generation tasks (e.g., super-resolution, denoising, and modality conversion). Notably, by this stage, the word embedding layer and output head have been fine-tuned, only the H-LoRA modules and adapter modules need to be trained. This strategy significantly improves the model's adaptability and flexibility across different tasks.


\section{Experiment}
\label{s:experiment}

\subsection{Data Description}
We evaluate our method on FI~\cite{you2016building}, Twitter\_LDL~\cite{yang2017learning} and Artphoto~\cite{machajdik2010affective}.
FI is a public dataset built from Flickr and Instagram, with 23,308 images and eight emotion categories, namely \textit{amusement}, \textit{anger}, \textit{awe},  \textit{contentment}, \textit{disgust}, \textit{excitement},  \textit{fear}, and \textit{sadness}. 
% Since images in FI are all copyrighted by law, some images are corrupted now, so we remove these samples and retain 21,828 images.
% T4SA contains images from Twitter, which are classified into three categories: \textit{positive}, \textit{neutral}, and \textit{negative}. In this paper, we adopt the base version of B-T4SA, which contains 470,586 images and provides text descriptions of the corresponding tweets.
Twitter\_LDL contains 10,045 images from Twitter, with the same eight categories as the FI dataset.
% 。
For these two datasets, they are randomly split into 80\%
training and 20\% testing set.
Artphoto contains 806 artistic photos from the DeviantArt website, which we use to further evaluate the zero-shot capability of our model.
% on the small-scale dataset.
% We construct and publicly release the first image sentiment analysis dataset containing metadata.
% 。

% Based on these datasets, we are the first to construct and publicly release metadata-enhanced image sentiment analysis datasets. These datasets include scenes, tags, descriptions, and corresponding confidence scores, and are available at this link for future research purposes.


% 
\begin{table}[t]
\centering
% \begin{center}
\caption{Overall performance of different models on FI and Twitter\_LDL datasets.}
\label{tab:cap1}
% \resizebox{\linewidth}{!}
{
\begin{tabular}{l|c|c|c|c}
\hline
\multirow{2}{*}{\textbf{Model}} & \multicolumn{2}{c|}{\textbf{FI}}  & \multicolumn{2}{c}{\textbf{Twitter\_LDL}} \\ \cline{2-5} 
  & \textbf{Accuracy} & \textbf{F1} & \textbf{Accuracy} & \textbf{F1}  \\ \hline
% (\rownumber)~AlexNet~\cite{krizhevsky2017imagenet}  & 58.13\% & 56.35\%  & 56.24\%& 55.02\%  \\ 
% (\rownumber)~VGG16~\cite{simonyan2014very}  & 63.75\%& 63.08\%  & 59.34\%& 59.02\%  \\ 
(\rownumber)~ResNet101~\cite{he2016deep} & 66.16\%& 65.56\%  & 62.02\% & 61.34\%  \\ 
(\rownumber)~CDA~\cite{han2023boosting} & 66.71\%& 65.37\%  & 64.14\% & 62.85\%  \\ 
(\rownumber)~CECCN~\cite{ruan2024color} & 67.96\%& 66.74\%  & 64.59\%& 64.72\% \\ 
(\rownumber)~EmoVIT~\cite{xie2024emovit} & 68.09\%& 67.45\%  & 63.12\% & 61.97\%  \\ 
(\rownumber)~ComLDL~\cite{zhang2022compound} & 68.83\%& 67.28\%  & 65.29\% & 63.12\%  \\ 
(\rownumber)~WSDEN~\cite{li2023weakly} & 69.78\%& 69.61\%  & 67.04\% & 65.49\% \\ 
(\rownumber)~ECWA~\cite{deng2021emotion} & 70.87\%& 69.08\%  & 67.81\% & 66.87\%  \\ 
(\rownumber)~EECon~\cite{yang2023exploiting} & 71.13\%& 68.34\%  & 64.27\%& 63.16\%  \\ 
(\rownumber)~MAM~\cite{zhang2024affective} & 71.44\%  & 70.83\% & 67.18\%  & 65.01\%\\ 
(\rownumber)~TGCA-PVT~\cite{chen2024tgca}   & 73.05\%  & 71.46\% & 69.87\%  & 68.32\% \\ 
(\rownumber)~OEAN~\cite{zhang2024object}   & 73.40\%  & 72.63\% & 70.52\%  & 69.47\% \\ \hline
(\rownumber)~\shortname  & \textbf{79.48\%} & \textbf{79.22\%} & \textbf{74.12\%} & \textbf{73.09\%} \\ \hline
\end{tabular}
}
\vspace{-6mm}
% \end{center}
\end{table}
% 

\subsection{Experiment Setting}
% \subsubsection{Model Setting.}
% 
\textbf{Model Setting:}
For feature representation, we set $k=10$ to select object tags, and adopt clip-vit-base-patch32 as the pre-trained model for unified feature representation.
Moreover, we empirically set $(d_e, d_h, d_k, d_s) = (512, 128, 16, 64)$, and set the classification class $L$ to 8.

% 

\textbf{Training Setting:}
To initialize the model, we set all weights such as $\boldsymbol{W}$ following the truncated normal distribution, and use AdamW optimizer with the learning rate of $1 \times 10^{-4}$.
% warmup scheduler of cosine, warmup steps of 2000.
Furthermore, we set the batch size to 32 and the epoch of the training process to 200.
During the implementation, we utilize \textit{PyTorch} to build our entire model.
% , and our project codes are publicly available at https://github.com/zzmyrep/MESN.
% Our project codes as well as data are all publicly available on GitHub\footnote{https://github.com/zzmyrep/KBCEN}.
% Code is available at \href{https://github.com/zzmyrep/KBCEN}{https://github.com/zzmyrep/KBCEN}.

\textbf{Evaluation Metrics:}
Following~\cite{zhang2024affective, chen2024tgca, zhang2024object}, we adopt \textit{accuracy} and \textit{F1} as our evaluation metrics to measure the performance of different methods for image sentiment analysis. 



\subsection{Experiment Result}
% We compare our model against the following baselines: AlexNet~\cite{krizhevsky2017imagenet}, VGG16~\cite{simonyan2014very}, ResNet101~\cite{he2016deep}, CECCN~\cite{ruan2024color}, EmoVIT~\cite{xie2024emovit}, WSCNet~\cite{yang2018weakly}, ECWA~\cite{deng2021emotion}, EECon~\cite{yang2023exploiting}, MAM~\cite{zhang2024affective} and TGCA-PVT~\cite{chen2024tgca}, and the overall results are summarized in Table~\ref{tab:cap1}.
We compare our model against several baselines, and the overall results are summarized in Table~\ref{tab:cap1}.
We observe that our model achieves the best performance in both accuracy and F1 metrics, significantly outperforming the previous models. 
This superior performance is mainly attributed to our effective utilization of metadata to enhance image sentiment analysis, as well as the exceptional capability of the unified sentiment transformer framework we developed. These results strongly demonstrate that our proposed method can bring encouraging performance for image sentiment analysis.

\setcounter{magicrownumbers}{0} 
\begin{table}[t]
\begin{center}
\caption{Ablation study of~\shortname~on FI dataset.} 
% \vspace{1mm}
\label{tab:cap2}
\resizebox{.9\linewidth}{!}
{
\begin{tabular}{lcc}
  \hline
  \textbf{Model} & \textbf{Accuracy} & \textbf{F1} \\
  \hline
  (\rownumber)~Ours (w/o vision) & 65.72\% & 64.54\% \\
  (\rownumber)~Ours (w/o text description) & 74.05\% & 72.58\% \\
  (\rownumber)~Ours (w/o object tag) & 77.45\% & 76.84\% \\
  (\rownumber)~Ours (w/o scene tag) & 78.47\% & 78.21\% \\
  \hline
  (\rownumber)~Ours (w/o unified embedding) & 76.41\% & 76.23\% \\
  (\rownumber)~Ours (w/o adaptive learning) & 76.83\% & 76.56\% \\
  (\rownumber)~Ours (w/o cross-modal fusion) & 76.85\% & 76.49\% \\
  \hline
  (\rownumber)~Ours  & \textbf{79.48\%} & \textbf{79.22\%} \\
  \hline
\end{tabular}
}
\end{center}
\vspace{-5mm}
\end{table}


\begin{figure}[t]
\centering
% \vspace{-2mm}
\includegraphics[width=0.42\textwidth]{fig/2dvisual-linux4-paper2.pdf}
\caption{Visualization of feature distribution on eight categories before (left) and after (right) model processing.}
% 
\label{fig:visualization}
\vspace{-5mm}
\end{figure}

\subsection{Ablation Performance}
In this subsection, we conduct an ablation study to examine which component is really important for performance improvement. The results are reported in Table~\ref{tab:cap2}.

For information utilization, we observe a significant decline in model performance when visual features are removed. Additionally, the performance of \shortname~decreases when different metadata are removed separately, which means that text description, object tag, and scene tag are all critical for image sentiment analysis.
Recalling the model architecture, we separately remove transformer layers of the unified representation module, the adaptive learning module, and the cross-modal fusion module, replacing them with MLPs of the same parameter scale.
In this way, we can observe varying degrees of decline in model performance, indicating that these modules are indispensable for our model to achieve better performance.

\subsection{Visualization}
% 


% % 开始使用minipage进行左右排列
% \begin{minipage}[t]{0.45\textwidth}  % 子图1宽度为45%
%     \centering
%     \includegraphics[width=\textwidth]{2dvisual.pdf}  % 插入图片
%     \captionof{figure}{Visualization of feature distribution.}  % 使用captionof添加图片标题
%     \label{fig:visualization}
% \end{minipage}


% \begin{figure}[t]
% \centering
% \vspace{-2mm}
% \includegraphics[width=0.45\textwidth]{fig/2dvisual.pdf}
% \caption{Visualization of feature distribution.}
% \label{fig:visualization}
% % \vspace{-4mm}
% \end{figure}

% \begin{figure}[t]
% \centering
% \vspace{-2mm}
% \includegraphics[width=0.45\textwidth]{fig/2dvisual-linux3-paper.pdf}
% \caption{Visualization of feature distribution.}
% \label{fig:visualization}
% % \vspace{-4mm}
% \end{figure}



\begin{figure}[tbp]   
\vspace{-4mm}
  \centering            
  \subfloat[Depth of adaptive learning layers]   
  {
    \label{fig:subfig1}\includegraphics[width=0.22\textwidth]{fig/fig_sensitivity-a5}
  }
  \subfloat[Depth of fusion layers]
  {
    % \label{fig:subfig2}\includegraphics[width=0.22\textwidth]{fig/fig_sensitivity-b2}
    \label{fig:subfig2}\includegraphics[width=0.22\textwidth]{fig/fig_sensitivity-b2-num.pdf}
  }
  \caption{Sensitivity study of \shortname~on different depth. }   
  \label{fig:fig_sensitivity}  
\vspace{-2mm}
\end{figure}

% \begin{figure}[htbp]
% \centerline{\includegraphics{2dvisual.pdf}}
% \caption{Visualization of feature distribution.}
% \label{fig:visualization}
% \end{figure}

% In Fig.~\ref{fig:visualization}, we use t-SNE~\cite{van2008visualizing} to reduce the dimension of data features for visualization, Figure in left represents the metadata features before model processing, the features are obtained by embedding through the CLIP model, and figure in right shows the features of the data after model processing, it can be observed that after the model processing, the data with different label categories fall in different regions in the space, therefore, we can conclude that the Therefore, we can conclude that the model can effectively utilize the information contained in the metadata and use it to guide the model for classification.

In Fig.~\ref{fig:visualization}, we use t-SNE~\cite{van2008visualizing} to reduce the dimension of data features for visualization.
The left figure shows metadata features before being processed by our model (\textit{i.e.}, embedded by CLIP), while the right shows the distribution of features after being processed by our model.
We can observe that after the model processing, data with the same label are closer to each other, while others are farther away.
Therefore, it shows that the model can effectively utilize the information contained in the metadata and use it to guide the classification process.

\subsection{Sensitivity Analysis}
% 
In this subsection, we conduct a sensitivity analysis to figure out the effect of different depth settings of adaptive learning layers and fusion layers. 
% In this subsection, we conduct a sensitivity analysis to figure out the effect of different depth settings on the model. 
% Fig.~\ref{fig:fig_sensitivity} presents the effect of different depth settings of adaptive learning layers and fusion layers. 
Taking Fig.~\ref{fig:fig_sensitivity} (a) as an example, the model performance improves with increasing depth, reaching the best performance at a depth of 4.
% Taking Fig.~\ref{fig:fig_sensitivity} (a) as an example, the performance of \shortname~improves with the increase of depth at first, reaching the best performance at a depth of 4.
When the depth continues to increase, the accuracy decreases to varying degrees.
Similar results can be observed in Fig.~\ref{fig:fig_sensitivity} (b).
Therefore, we set their depths to 4 and 6 respectively to achieve the best results.

% Through our experiments, we can observe that the effect of modifying these hyperparameters on the results of the experiments is very weak, and the surface model is not sensitive to the hyperparameters.


\subsection{Zero-shot Capability}
% 

% (1)~GCH~\cite{2010Analyzing} & 21.78\% & (5)~RA-DLNet~\cite{2020A} & 34.01\% \\ \hline
% (2)~WSCNet~\cite{2019WSCNet}  & 30.25\% & (6)~CECCN~\cite{ruan2024color} & 43.83\% \\ \hline
% (3)~PCNN~\cite{2015Robust} & 31.68\%  & (7)~EmoVIT~\cite{xie2024emovit} & 44.90\% \\ \hline
% (4)~AR~\cite{2018Visual} & 32.67\% & (8)~Ours (Zero-shot) & 47.83\% \\ \hline


\begin{table}[t]
\centering
\caption{Zero-shot capability of \shortname.}
\label{tab:cap3}
\resizebox{1\linewidth}{!}
{
\begin{tabular}{lc|lc}
\hline
\textbf{Model} & \textbf{Accuracy} & \textbf{Model} & \textbf{Accuracy} \\ \hline
(1)~WSCNet~\cite{2019WSCNet}  & 30.25\% & (5)~MAM~\cite{zhang2024affective} & 39.56\%  \\ \hline
(2)~AR~\cite{2018Visual} & 32.67\% & (6)~CECCN~\cite{ruan2024color} & 43.83\% \\ \hline
(3)~RA-DLNet~\cite{2020A} & 34.01\%  & (7)~EmoVIT~\cite{xie2024emovit} & 44.90\% \\ \hline
(4)~CDA~\cite{han2023boosting} & 38.64\% & (8)~Ours (Zero-shot) & 47.83\% \\ \hline
\end{tabular}
}
\vspace{-5mm}
\end{table}

% We use the model trained on the FI dataset to test on the artphoto dataset to verify the model's generalization ability as well as robustness to other distributed datasets.
% We can observe that the MESN model shows strong competitiveness in terms of accuracy when compared to other trained models, which suggests that the model has a good generalization ability in the OOD task.

To validate the model's generalization ability and robustness to other distributed datasets, we directly test the model trained on the FI dataset, without training on Artphoto. 
% As observed in Table 3, compared to other models trained on Artphoto, we achieve highly competitive zero-shot performance, indicating that the model has good generalization ability in out-of-distribution tasks.
From Table~\ref{tab:cap3}, we can observe that compared with other models trained on Artphoto, we achieve competitive zero-shot performance, which shows that the model has good generalization ability in out-of-distribution tasks.


%%%%%%%%%%%%
%  E2E     %
%%%%%%%%%%%%


\section{Conclusion}
In this paper, we introduced Wi-Chat, the first LLM-powered Wi-Fi-based human activity recognition system that integrates the reasoning capabilities of large language models with the sensing potential of wireless signals. Our experimental results on a self-collected Wi-Fi CSI dataset demonstrate the promising potential of LLMs in enabling zero-shot Wi-Fi sensing. These findings suggest a new paradigm for human activity recognition that does not rely on extensive labeled data. We hope future research will build upon this direction, further exploring the applications of LLMs in signal processing domains such as IoT, mobile sensing, and radar-based systems.

\section*{Limitations}
While our work represents the first attempt to leverage LLMs for processing Wi-Fi signals, it is a preliminary study focused on a relatively simple task: Wi-Fi-based human activity recognition. This choice allows us to explore the feasibility of LLMs in wireless sensing but also comes with certain limitations.

Our approach primarily evaluates zero-shot performance, which, while promising, may still lag behind traditional supervised learning methods in highly complex or fine-grained recognition tasks. Besides, our study is limited to a controlled environment with a self-collected dataset, and the generalizability of LLMs to diverse real-world scenarios with varying Wi-Fi conditions, environmental interference, and device heterogeneity remains an open question.

Additionally, we have yet to explore the full potential of LLMs in more advanced Wi-Fi sensing applications, such as fine-grained gesture recognition, occupancy detection, and passive health monitoring. Future work should investigate the scalability of LLM-based approaches, their robustness to domain shifts, and their integration with multimodal sensing techniques in broader IoT applications.


% Bibliography entries for the entire Anthology, followed by custom entries
%\bibliography{anthology,custom}
% Custom bibliography entries only
\bibliography{main}
\newpage
\appendix

\section{Experiment prompts}
\label{sec:prompt}
The prompts used in the LLM experiments are shown in the following Table~\ref{tab:prompts}.

\definecolor{titlecolor}{rgb}{0.9, 0.5, 0.1}
\definecolor{anscolor}{rgb}{0.2, 0.5, 0.8}
\definecolor{labelcolor}{HTML}{48a07e}
\begin{table*}[h]
	\centering
	
 % \vspace{-0.2cm}
	
	\begin{center}
		\begin{tikzpicture}[
				chatbox_inner/.style={rectangle, rounded corners, opacity=0, text opacity=1, font=\sffamily\scriptsize, text width=5in, text height=9pt, inner xsep=6pt, inner ysep=6pt},
				chatbox_prompt_inner/.style={chatbox_inner, align=flush left, xshift=0pt, text height=11pt},
				chatbox_user_inner/.style={chatbox_inner, align=flush left, xshift=0pt},
				chatbox_gpt_inner/.style={chatbox_inner, align=flush left, xshift=0pt},
				chatbox/.style={chatbox_inner, draw=black!25, fill=gray!7, opacity=1, text opacity=0},
				chatbox_prompt/.style={chatbox, align=flush left, fill=gray!1.5, draw=black!30, text height=10pt},
				chatbox_user/.style={chatbox, align=flush left},
				chatbox_gpt/.style={chatbox, align=flush left},
				chatbox2/.style={chatbox_gpt, fill=green!25},
				chatbox3/.style={chatbox_gpt, fill=red!20, draw=black!20},
				chatbox4/.style={chatbox_gpt, fill=yellow!30},
				labelbox/.style={rectangle, rounded corners, draw=black!50, font=\sffamily\scriptsize\bfseries, fill=gray!5, inner sep=3pt},
			]
											
			\node[chatbox_user] (q1) {
				\textbf{System prompt}
				\newline
				\newline
				You are a helpful and precise assistant for segmenting and labeling sentences. We would like to request your help on curating a dataset for entity-level hallucination detection.
				\newline \newline
                We will give you a machine generated biography and a list of checked facts about the biography. Each fact consists of a sentence and a label (True/False). Please do the following process. First, breaking down the biography into words. Second, by referring to the provided list of facts, merging some broken down words in the previous step to form meaningful entities. For example, ``strategic thinking'' should be one entity instead of two. Third, according to the labels in the list of facts, labeling each entity as True or False. Specifically, for facts that share a similar sentence structure (\eg, \textit{``He was born on Mach 9, 1941.''} (\texttt{True}) and \textit{``He was born in Ramos Mejia.''} (\texttt{False})), please first assign labels to entities that differ across atomic facts. For example, first labeling ``Mach 9, 1941'' (\texttt{True}) and ``Ramos Mejia'' (\texttt{False}) in the above case. For those entities that are the same across atomic facts (\eg, ``was born'') or are neutral (\eg, ``he,'' ``in,'' and ``on''), please label them as \texttt{True}. For the cases that there is no atomic fact that shares the same sentence structure, please identify the most informative entities in the sentence and label them with the same label as the atomic fact while treating the rest of the entities as \texttt{True}. In the end, output the entities and labels in the following format:
                \begin{itemize}[nosep]
                    \item Entity 1 (Label 1)
                    \item Entity 2 (Label 2)
                    \item ...
                    \item Entity N (Label N)
                \end{itemize}
                % \newline \newline
                Here are two examples:
                \newline\newline
                \textbf{[Example 1]}
                \newline
                [The start of the biography]
                \newline
                \textcolor{titlecolor}{Marianne McAndrew is an American actress and singer, born on November 21, 1942, in Cleveland, Ohio. She began her acting career in the late 1960s, appearing in various television shows and films.}
                \newline
                [The end of the biography]
                \newline \newline
                [The start of the list of checked facts]
                \newline
                \textcolor{anscolor}{[Marianne McAndrew is an American. (False); Marianne McAndrew is an actress. (True); Marianne McAndrew is a singer. (False); Marianne McAndrew was born on November 21, 1942. (False); Marianne McAndrew was born in Cleveland, Ohio. (False); She began her acting career in the late 1960s. (True); She has appeared in various television shows. (True); She has appeared in various films. (True)]}
                \newline
                [The end of the list of checked facts]
                \newline \newline
                [The start of the ideal output]
                \newline
                \textcolor{labelcolor}{[Marianne McAndrew (True); is (True); an (True); American (False); actress (True); and (True); singer (False); , (True); born (True); on (True); November 21, 1942 (False); , (True); in (True); Cleveland, Ohio (False); . (True); She (True); began (True); her (True); acting career (True); in (True); the late 1960s (True); , (True); appearing (True); in (True); various (True); television shows (True); and (True); films (True); . (True)]}
                \newline
                [The end of the ideal output]
				\newline \newline
                \textbf{[Example 2]}
                \newline
                [The start of the biography]
                \newline
                \textcolor{titlecolor}{Doug Sheehan is an American actor who was born on April 27, 1949, in Santa Monica, California. He is best known for his roles in soap operas, including his portrayal of Joe Kelly on ``General Hospital'' and Ben Gibson on ``Knots Landing.''}
                \newline
                [The end of the biography]
                \newline \newline
                [The start of the list of checked facts]
                \newline
                \textcolor{anscolor}{[Doug Sheehan is an American. (True); Doug Sheehan is an actor. (True); Doug Sheehan was born on April 27, 1949. (True); Doug Sheehan was born in Santa Monica, California. (False); He is best known for his roles in soap operas. (True); He portrayed Joe Kelly. (True); Joe Kelly was in General Hospital. (True); General Hospital is a soap opera. (True); He portrayed Ben Gibson. (True); Ben Gibson was in Knots Landing. (True); Knots Landing is a soap opera. (True)]}
                \newline
                [The end of the list of checked facts]
                \newline \newline
                [The start of the ideal output]
                \newline
                \textcolor{labelcolor}{[Doug Sheehan (True); is (True); an (True); American (True); actor (True); who (True); was born (True); on (True); April 27, 1949 (True); in (True); Santa Monica, California (False); . (True); He (True); is (True); best known (True); for (True); his roles in soap operas (True); , (True); including (True); in (True); his portrayal (True); of (True); Joe Kelly (True); on (True); ``General Hospital'' (True); and (True); Ben Gibson (True); on (True); ``Knots Landing.'' (True)]}
                \newline
                [The end of the ideal output]
				\newline \newline
				\textbf{User prompt}
				\newline
				\newline
				[The start of the biography]
				\newline
				\textcolor{magenta}{\texttt{\{BIOGRAPHY\}}}
				\newline
				[The ebd of the biography]
				\newline \newline
				[The start of the list of checked facts]
				\newline
				\textcolor{magenta}{\texttt{\{LIST OF CHECKED FACTS\}}}
				\newline
				[The end of the list of checked facts]
			};
			\node[chatbox_user_inner] (q1_text) at (q1) {
				\textbf{System prompt}
				\newline
				\newline
				You are a helpful and precise assistant for segmenting and labeling sentences. We would like to request your help on curating a dataset for entity-level hallucination detection.
				\newline \newline
                We will give you a machine generated biography and a list of checked facts about the biography. Each fact consists of a sentence and a label (True/False). Please do the following process. First, breaking down the biography into words. Second, by referring to the provided list of facts, merging some broken down words in the previous step to form meaningful entities. For example, ``strategic thinking'' should be one entity instead of two. Third, according to the labels in the list of facts, labeling each entity as True or False. Specifically, for facts that share a similar sentence structure (\eg, \textit{``He was born on Mach 9, 1941.''} (\texttt{True}) and \textit{``He was born in Ramos Mejia.''} (\texttt{False})), please first assign labels to entities that differ across atomic facts. For example, first labeling ``Mach 9, 1941'' (\texttt{True}) and ``Ramos Mejia'' (\texttt{False}) in the above case. For those entities that are the same across atomic facts (\eg, ``was born'') or are neutral (\eg, ``he,'' ``in,'' and ``on''), please label them as \texttt{True}. For the cases that there is no atomic fact that shares the same sentence structure, please identify the most informative entities in the sentence and label them with the same label as the atomic fact while treating the rest of the entities as \texttt{True}. In the end, output the entities and labels in the following format:
                \begin{itemize}[nosep]
                    \item Entity 1 (Label 1)
                    \item Entity 2 (Label 2)
                    \item ...
                    \item Entity N (Label N)
                \end{itemize}
                % \newline \newline
                Here are two examples:
                \newline\newline
                \textbf{[Example 1]}
                \newline
                [The start of the biography]
                \newline
                \textcolor{titlecolor}{Marianne McAndrew is an American actress and singer, born on November 21, 1942, in Cleveland, Ohio. She began her acting career in the late 1960s, appearing in various television shows and films.}
                \newline
                [The end of the biography]
                \newline \newline
                [The start of the list of checked facts]
                \newline
                \textcolor{anscolor}{[Marianne McAndrew is an American. (False); Marianne McAndrew is an actress. (True); Marianne McAndrew is a singer. (False); Marianne McAndrew was born on November 21, 1942. (False); Marianne McAndrew was born in Cleveland, Ohio. (False); She began her acting career in the late 1960s. (True); She has appeared in various television shows. (True); She has appeared in various films. (True)]}
                \newline
                [The end of the list of checked facts]
                \newline \newline
                [The start of the ideal output]
                \newline
                \textcolor{labelcolor}{[Marianne McAndrew (True); is (True); an (True); American (False); actress (True); and (True); singer (False); , (True); born (True); on (True); November 21, 1942 (False); , (True); in (True); Cleveland, Ohio (False); . (True); She (True); began (True); her (True); acting career (True); in (True); the late 1960s (True); , (True); appearing (True); in (True); various (True); television shows (True); and (True); films (True); . (True)]}
                \newline
                [The end of the ideal output]
				\newline \newline
                \textbf{[Example 2]}
                \newline
                [The start of the biography]
                \newline
                \textcolor{titlecolor}{Doug Sheehan is an American actor who was born on April 27, 1949, in Santa Monica, California. He is best known for his roles in soap operas, including his portrayal of Joe Kelly on ``General Hospital'' and Ben Gibson on ``Knots Landing.''}
                \newline
                [The end of the biography]
                \newline \newline
                [The start of the list of checked facts]
                \newline
                \textcolor{anscolor}{[Doug Sheehan is an American. (True); Doug Sheehan is an actor. (True); Doug Sheehan was born on April 27, 1949. (True); Doug Sheehan was born in Santa Monica, California. (False); He is best known for his roles in soap operas. (True); He portrayed Joe Kelly. (True); Joe Kelly was in General Hospital. (True); General Hospital is a soap opera. (True); He portrayed Ben Gibson. (True); Ben Gibson was in Knots Landing. (True); Knots Landing is a soap opera. (True)]}
                \newline
                [The end of the list of checked facts]
                \newline \newline
                [The start of the ideal output]
                \newline
                \textcolor{labelcolor}{[Doug Sheehan (True); is (True); an (True); American (True); actor (True); who (True); was born (True); on (True); April 27, 1949 (True); in (True); Santa Monica, California (False); . (True); He (True); is (True); best known (True); for (True); his roles in soap operas (True); , (True); including (True); in (True); his portrayal (True); of (True); Joe Kelly (True); on (True); ``General Hospital'' (True); and (True); Ben Gibson (True); on (True); ``Knots Landing.'' (True)]}
                \newline
                [The end of the ideal output]
				\newline \newline
				\textbf{User prompt}
				\newline
				\newline
				[The start of the biography]
				\newline
				\textcolor{magenta}{\texttt{\{BIOGRAPHY\}}}
				\newline
				[The ebd of the biography]
				\newline \newline
				[The start of the list of checked facts]
				\newline
				\textcolor{magenta}{\texttt{\{LIST OF CHECKED FACTS\}}}
				\newline
				[The end of the list of checked facts]
			};
		\end{tikzpicture}
        \caption{GPT-4o prompt for labeling hallucinated entities.}\label{tb:gpt-4-prompt}
	\end{center}
\vspace{-0cm}
\end{table*}
% \section{Full Experiment Results}
% \begin{table*}[th]
    \centering
    \small
    \caption{Classification Results}
    \begin{tabular}{lcccc}
        \toprule
        \textbf{Method} & \textbf{Accuracy} & \textbf{Precision} & \textbf{Recall} & \textbf{F1-score} \\
        \midrule
        \multicolumn{5}{c}{\textbf{Zero Shot}} \\
                Zero-shot E-eyes & 0.26 & 0.26 & 0.27 & 0.26 \\
        Zero-shot CARM & 0.24 & 0.24 & 0.24 & 0.24 \\
                Zero-shot SVM & 0.27 & 0.28 & 0.28 & 0.27 \\
        Zero-shot CNN & 0.23 & 0.24 & 0.23 & 0.23 \\
        Zero-shot RNN & 0.26 & 0.26 & 0.26 & 0.26 \\
DeepSeek-0shot & 0.54 & 0.61 & 0.54 & 0.52 \\
DeepSeek-0shot-COT & 0.33 & 0.24 & 0.33 & 0.23 \\
DeepSeek-0shot-Knowledge & 0.45 & 0.46 & 0.45 & 0.44 \\
Gemma2-0shot & 0.35 & 0.22 & 0.38 & 0.27 \\
Gemma2-0shot-COT & 0.36 & 0.22 & 0.36 & 0.27 \\
Gemma2-0shot-Knowledge & 0.32 & 0.18 & 0.34 & 0.20 \\
GPT-4o-mini-0shot & 0.48 & 0.53 & 0.48 & 0.41 \\
GPT-4o-mini-0shot-COT & 0.33 & 0.50 & 0.33 & 0.38 \\
GPT-4o-mini-0shot-Knowledge & 0.49 & 0.31 & 0.49 & 0.36 \\
GPT-4o-0shot & 0.62 & 0.62 & 0.47 & 0.42 \\
GPT-4o-0shot-COT & 0.29 & 0.45 & 0.29 & 0.21 \\
GPT-4o-0shot-Knowledge & 0.44 & 0.52 & 0.44 & 0.39 \\
LLaMA-0shot & 0.32 & 0.25 & 0.32 & 0.24 \\
LLaMA-0shot-COT & 0.12 & 0.25 & 0.12 & 0.09 \\
LLaMA-0shot-Knowledge & 0.32 & 0.25 & 0.32 & 0.28 \\
Mistral-0shot & 0.19 & 0.23 & 0.19 & 0.10 \\
Mistral-0shot-Knowledge & 0.21 & 0.40 & 0.21 & 0.11 \\
        \midrule
        \multicolumn{5}{c}{\textbf{4 Shot}} \\
GPT-4o-mini-4shot & 0.58 & 0.59 & 0.58 & 0.53 \\
GPT-4o-mini-4shot-COT & 0.57 & 0.53 & 0.57 & 0.50 \\
GPT-4o-mini-4shot-Knowledge & 0.56 & 0.51 & 0.56 & 0.47 \\
GPT-4o-4shot & 0.77 & 0.84 & 0.77 & 0.73 \\
GPT-4o-4shot-COT & 0.63 & 0.76 & 0.63 & 0.53 \\
GPT-4o-4shot-Knowledge & 0.72 & 0.82 & 0.71 & 0.66 \\
LLaMA-4shot & 0.29 & 0.24 & 0.29 & 0.21 \\
LLaMA-4shot-COT & 0.20 & 0.30 & 0.20 & 0.13 \\
LLaMA-4shot-Knowledge & 0.15 & 0.23 & 0.13 & 0.13 \\
Mistral-4shot & 0.02 & 0.02 & 0.02 & 0.02 \\
Mistral-4shot-Knowledge & 0.21 & 0.27 & 0.21 & 0.20 \\
        \midrule
        
        \multicolumn{5}{c}{\textbf{Suprevised}} \\
        SVM & 0.94 & 0.92 & 0.91 & 0.91 \\
        CNN & 0.98 & 0.98 & 0.97 & 0.97 \\
        RNN & 0.99 & 0.99 & 0.99 & 0.99 \\
        % \midrule
        % \multicolumn{5}{c}{\textbf{Conventional Wi-Fi-based Human Activity Recognition Systems}} \\
        E-eyes & 1.00 & 1.00 & 1.00 & 1.00 \\
        CARM & 0.98 & 0.98 & 0.98 & 0.98 \\
\midrule
 \multicolumn{5}{c}{\textbf{Vision Models}} \\
           Zero-shot SVM & 0.26 & 0.25 & 0.25 & 0.25 \\
        Zero-shot CNN & 0.26 & 0.25 & 0.26 & 0.26 \\
        Zero-shot RNN & 0.28 & 0.28 & 0.29 & 0.28 \\
        SVM & 0.99 & 0.99 & 0.99 & 0.99 \\
        CNN & 0.98 & 0.99 & 0.98 & 0.98 \\
        RNN & 0.98 & 0.99 & 0.98 & 0.98 \\
GPT-4o-mini-Vision & 0.84 & 0.85 & 0.84 & 0.84 \\
GPT-4o-mini-Vision-COT & 0.90 & 0.91 & 0.90 & 0.90 \\
GPT-4o-Vision & 0.74 & 0.82 & 0.74 & 0.73 \\
GPT-4o-Vision-COT & 0.70 & 0.83 & 0.70 & 0.68 \\
LLaMA-Vision & 0.20 & 0.23 & 0.20 & 0.09 \\
LLaMA-Vision-Knowledge & 0.22 & 0.05 & 0.22 & 0.08 \\

        \bottomrule
    \end{tabular}
    \label{full}
\end{table*}




\end{document}

% \bibliography{main}

\newpage
\appendix

\section{Hyperparameters}
During the training of the attack model, the sequence length is set to 160. For fine-tuning the victim models, the sequence length is set to 1600 for LLaMA3-8B and Bloom-7B1, and 800 for GPT2-XL. The AdamW\cite{loshchilov2017decoupled} optimizer is used for all training and fine-tuning processes, with learning rates set to 5e-5 for GPT2-XL and Bloom-7B1, and 7e-5 for LLaMA3-8B, along with an epsilon value of 1e-8.

\section{Datasets}
The WikiText dataset serves as a high-quality, clean, and large-scale collection of English text extracted from Wikipedia articles, providing a solid foundation for creating the shadow dataset for the attacker's model. The ArXiv dataset is a large-scale collection of scientific papers from the arXiv repository. The OpenWebText dataset is a high-quality, large-scale corpus of English web content, curated from URLs shared on Reddit with high karma. The Pile is an 800GB, diverse English text dataset designed for training large language models, combining content from 22 high-quality sources, including books, academic papers, code, and web text. The PII dataset consists of 1,000 instances of sensitive information and includes 10 types of personally identifiable information (PII), such as phone numbers, email addresses, and home addresses, presented in a structured format. These data are randomly generated using regular expressions and do not represent real private information.

\section{Toolkits}
We use the NLTK package to measure the BLEU score, the rouge\_score library to calculate the ROUGE score, and scikit-learn to compute the cosine similarity.

\section{True-Prefix and SPT Attack Examples}
Figure \ref{fig:true_prefix_example} and Figure \ref{fig:spt_example} present two examples of True-Prefix~\cite{true-prefix} and SPT~\cite{SPT} attacks, respectively. In the True-Prefix attack, we insert real data of additional PII types, such as address or birthday, before the prompt templates, as shown in the blue sections in Figure \ref{fig:true_prefix_example}. In the SPT attack, we train on 64 PII data pairs for 5 epochs to obtain the soft prompt embeddings, which are set to a length of 10. The soft prompt embeddings are then concatenated before the prompt templates. During the training, the victim model remains frozen, with no gradient updates applied.

\begin{figure}[H]
  \includegraphics[width=\linewidth]{figures/True-Prefix_example.pdf} 
  \caption {Two True-Prefix attack examples. Blue text represents the real private data, while green and red text indicate successful and failed privacy theft, respectively.}
  \label{fig:true_prefix_example}
  \vspace{-1em}
\end{figure}

\begin{figure}[H]
  \includegraphics[width=\linewidth]{figures/SPT_example.pdf} 
  \caption {Two SPT attack examples. Orange text represents the soft prompt embeddings, with green and red text indicating successful and failed privacy theft, respectively.}
  \label{fig:spt_example}
  \vspace{-1em}
\end{figure}

\end{document}
