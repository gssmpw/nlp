

\chapter*{Abstract}

Inflation prediction is essential for guiding decisions on interest rates, investments, and wages, as well as for enabling central banks to establish effective monetary policies to ensure economic stability. The complexity of predicting inflation arises from the interplay of numerous dynamic factors and the hierarchical structure of the Consumer Price Index (CPI), which organizes goods and services into categories and subcategories to capture their contributions to overall inflation. This hierarchical nature demands advanced modeling techniques to achieve accurate forecasts.

%Traditional approaches to predictive modeling with hierarchical data include training separate models for each series within the hierarchy or using a single model for the combined data. While the former can mitigate underfitting by focusing on specific data segments, it often leads to overfitting due to limited data for individual models. The latter leverages larger datasets but is computationally expensive and struggles to address the varying dynamics across hierarchical levels.

In this work, we introduce Bi-directional Hierarchical Recurrent Neural Network (BiHRNN) model, a novel modeling approach that strikes a balance between these extremes by leveraging the hierarchical structure of datasets like the CPI. BiHRNN facilitates bidirectional information flow between hierarchical levels, where higher-level nodes influence lower-level ones and vice versa. This is achieved using hierarchical informative constraints applied to the parameters of Recurrent Neural Networks (RNNs), enhancing predictive accuracy across all hierarchy levels. By integrating hierarchical relationships without the inefficiencies of a unified model, BiHRNN offers an effective solution for accurate and scalable inflation forecasting.

We implemented our BiHRNN model on three distinct inflation datasets from major markets: the United States, Canada, and Norway, all of which include a variety of economic indices. For each use case, we gathered the necessary data, trained and evaluated the BiHRNN model, and fine-tuned its hyper-parameters. Additionally, we experimented with various loss functions to enhance the model's performance.

The results show that the BiHRNN model significantly outperforms traditional RNN approaches in forecasting accuracy across most levels of the hierarchy. The unique bidirectional architecture of the model, which facilitates the flow of information across hierarchical levels, played a crucial role in achieving these improvements.

Looking ahead, we aim to expand the application of BiHRNN to additional hierarchical inflation markets, exploring different domains, and refining the model to address the specific challenges of these datasets. This study provides a strong foundation for employing BiHRNN in inflation forecasting and underscores its potential to surpass traditional methods.

The code for this Thesis is available at:  \href{https://github.com/mayavilenko/Maya-Thesis}{https://github.com/mayavilenko/Maya-Thesis}. 
