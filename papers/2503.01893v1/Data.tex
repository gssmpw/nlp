
\chapter{Dataset} 
This work is based on monthly CPI data released by the US Bureau of Labor and Statistics (BLS) \footnote{Taken from \url{www.bls.gov/cpi/overview.htm}.}, Statistics Canada \footnote{Taken from \url{www.statcan.gc.ca/en/start}.}, and Statistics Norway \footnote{Taken from \url{www.ssb.no/en}.}, covering different time periods: the US dataset spans from January 2012 to September 2023, the Norway dataset from January 2009 to May 2023, and the Canada dataset from January 2013 to February 2023. In the following section, we outline the dataset's characteristics and detail our pre-processing methodology. To ensure reproducibility, the final version of the processed data is included within our BiHRNN code.

\section{The US Consumer Price Index}

The CPI for each month is released by the BLS a few days into the following month. Price data is gathered from around 24,000 retail and service establishments across 75 urban areas in the US. Additionally, housing and rent information is collected from approximately 50,000 landlords and tenants nationwide.
The BLS provides two distinct CPI measurements based on urban demographics:
\begin{enumerate} \item
    The {\bf CPI-U}, or Consumer Price Index for All Urban Consumers, covers about 93\%  of the total population. The items included in the CPI, along with their relative weights, are determined based on the Consumer Expenditure Survey, which estimates household spending. These items and weights are updated annually each January.
    \item 
    The {\bf CPI-W}, or Consumer Price Index for Urban Wage Earners and Clerical Workers, covers around 29\% of the population. This index focuses on households where at least 50\% of income is earned through wage-paying or clerical jobs, with at least one household member employed for 70\% or more of the year. The CPI-W is commonly used to track changes in benefit costs and inform future contract obligations.
\end{enumerate}    

In this work, we focus on the CPI-U, as it is widely regarded as the most reliable indicator of the average cost of living in the United States. The CPI-U prices are seasonally adjusted, with updates released in February that reflect price movements from the previous calendar year. It is important to note that, over time, new indexes have been introduced while others have been discontinued, causing shifts in the hierarchical structure of the dataset, which adds complexity to our analysis.

\subsection{US CPI Hierarchy}
The CPI-U is structured as an nine-level hierarchy consisting of 350 distinct nodes (indexes). At the top, Level 0 represents the headline CPI, which is the aggregated index of all components. Each index in the hierarchy is assigned a weight between 0 and 100, reflecting its contribution to the headline CPI at Level 0. 

Level 1 contains the 8 (not including ``All items excluding X'' categories) main aggregate categories, or sectors: (1) ``Food and Beverages,'' (2) ``Housing,'' (3) ``Apparel,'' (4) ``Transportation,'' (5) ``Medical Care,'' (6) ``Recreation,'' (7) ``Education and Communication,'' and (8) ``Other Goods and Services.''

Mid-levels (2-5) include more specific groupings such as ``Milk'' and ``Motor Fuel.'' The lower levels (6-8) feature finer-grained indexes, such as ``Rice,'' ``Child Care and Nursery School,'' ``Legal Services,'' ``Funeral Expenses,'' and ``Parking Fees and Tolls,'' among others. 

\section{Canada Consumer Price Index}

The  CPI in Canada is measured on a monthly basis by Statistics Canada\footnote{\href{https://www.statcan.gc.ca/en/start}{https://www.statcan.gc.ca/en/start}}. %Statistics Canada conducts over 350 surveys and programs to measure the country's economy and society, providing essential data for researchers and policymakers. 
The target population for the CPI includes families and individuals living in urban and rural private households, excluding those in communal or institutional settings like prisons or long-term care. The price sample is gathered from various geographic areas, goods, services, and retail outlets to estimate price changes. Outlet selection focuses on high-revenue retailers, with prices mainly collected from retail stores and agencies. The relative importance of items in the CPI basket is derived using data from the Household Final Consumption Expenditure (HFCE) and the Survey of Household Spending (SHS).
It should be noted that in some years, category weights for the CPI were unavailable. Therefore, we employed Linear Regression to estimate the missing weights, with the CPI values of the child categories serving as independent variables and the parent category CPI as the dependent variable. We will expand on this further in Section \ref{subsec:data_aug}.

\subsection{Canada CPI Hierarchy}

The CPI is organized into a seven-level hierarchy with 293 distinct nodes (indexes). At the top, Level 0 represents the headline CPI, an aggregated index of all components. Each index in the hierarchy is assigned a weight between 0 and 1, indicating its contribution to its parent CPI.

Level 1 consists of 10 (not including ``All items excluding X'' categories) main aggregate categories, or sectors: (1) ``Food,'' (2) ``Shelter,'' (3) ``Household operations, furnishings and equipment,'' (4) ``Clothing and footwear,'' (5) ``Transportation,'' (6) ``Health and personal care,'' (7) ``Recreation, education and reading,'' (8) ``Alcoholic beverages, tobacco products and recreational cannabis,'' (9) ``Food and energy,'' and (10) ``Energy.''

Mid-levels (2-4) include more specific categories like ``Meat'' and ``Women's clothing.'' The lower levels (5-6) feature more detailed indexes such as ``Whole milk,'' ``Pasta mixes,'' ``Purchase of used passenger vehicles,'' and ``Roasted or ground coffee,'' alongside others.


\section{Norway Consumer Price Index}

The CPI in Norway is collected monthly by Statistics Norway\footnote{\href{https://www.ssb.no/en}{https://www.ssb.no/en}}, primarily through electronic questionnaires sent to outlets on the 10th of each month, with responses due by the first working day after the 15th. In addition, electronic scanner data from grocery stores, pharmacies, clothing, sports retailers, and petrol stations are received monthly. The sample includes around 650 goods and services, selected based on household budget surveys and industry information, covering approximately 2,000 firms. For rent surveys, 2,500 tenants are sampled from the Rental Market Survey. Firms are chosen from the Business Register, with larger firms having a higher probability of selection based on turnover, after stratifying by industry and region.


\subsection{Norway CPI Hierarchy}
The CPI is arranged in a three-level hierarchy with 52 distinct nodes (indexes). At the top, Level 0 represents the headline CPI, which aggregates all components. Each index in the hierarchy is given a weight between 0 and 1000, reflecting its contribution to the overall CPI.

Level 1 is made up of 12 main aggregate categories, or sectors: (1) ``Food and non-alcoholic beverages,'' (2) ``Alcoholic beverages and tobacco,'' (3) ``Clothing and footwear,'' (4) ``Housing, water, electricity, gas and other fuels,'' (5) ``Furnishings, household equipment and routine maintenance,'' (6) ``Health,'' (7) ``Transport,'' (8) ``Communications,'' (9) ``Recreation and culture,'' (10) ``Education,'' (11) ``Restaurants and hotels,'' and (12) ``Miscellaneous goods and services.''

Level 2 includes more specific categories like ``Footware'' and ``Clothing.'' 


The tables below hold data of the first three hierarchies of the US CPI (levels 0-2):

\setlength{\tabcolsep}{2pt}
\renewcommand{\arraystretch}{0.9}

\begin{table}[H]
\caption{Indexes Level 0 And 1 - US} \label{tab:indexes_level_0_1}

\begin{center}
{\small
\begin{tabularx}{\textwidth}{@{}Xll@{}}
\toprule
Level & Index & Parent \\ 
\midrule
0 & All items & - \\ 
\midrule
1 & All items less energy & All items \\ 
1 & All items less food & All items \\ 
1 & All items less food and energy & All items \\ 
1 & All items less food and shelter & All items \\ 
1 & All items less food, shelter, and energy & All items \\ 
1 & All items less food, shelter, energy, and used cars and trucks & All items \\ 
1 & All items less homeowners costs & All items \\ 
1 & All items less medical care & All items \\ 
1 & All items less shelter & All items \\ 
1 & Apparel & All items \\ 
1 & Apparel less footwear & All items \\ 
1 & Commodities & All items \\ 
1 & Commodities less food & All items \\ 
1 & Durables & All items \\ 
1 & Education and communication & All items \\ 
1 & Energy & All items \\ 
1 & Entertainment & All items \\ 
1 & Food & All items \\ 
1 & Food and beverages & All items \\ 
1 & Fuels and utilities & All items \\ 
1 & Household furnishings and operations & All items \\ 
1 & Housing & All items \\ 
1 & Medical care & All items \\ 
1 & Nondurables & All items \\ 
1 & Nondurables less food & All items \\ 
1 & Nondurables less food and apparel & All items \\ 
1 & Other goods and services & All items \\ 
1 & Other services & All items \\ 
1 & Recreation & All items \\ 
1 & Services & All items \\ 
1 & Services less medical care services & All items \\ 
1 & Services less rent of shelter & All items \\ 
1 & Transportation & All items \\ 
1 & Utilities and public transportation & All items \\ \bottomrule
\end{tabularx}}
\floatfoot*{\textit{Note}: Levels and Parents of Indexes might change through time}
\end{center}
\end{table}


\begin{table}[H]
\renewcommand{\arraystretch}{1}

\caption{Indexes Level 2 - US} \label{tab:indexes_level_2}
\begin{center}
{\footnotesize
\begin{tabularx}{\textwidth}{@{}Xll@{}}
\toprule
Level & Index & Parent \\ \midrule
2 & All items less food and energy & All items less energy \\ 
2 & Apparel commodities & Apparel \\ 
2 & Apparel services & Apparel \\ 
2 & Commodities less food & Commodities \\ 
2 & Commodities less food and beverages & Commodities \\ 
2 & Commodities less food and energy commodities & All items less food and energy \\ 
2 & Commodities less food, energy, and used cars and trucks & Commodities \\ 
2 & Communication & Education and communication \\ 
2 & Domestically produced farm food & Food and beverages \\ 
2 & Education & Education and communication \\ 
2 & Energy commodities & Energy \\ 
2 & Energy services & Energy \\ 
2 & Entertainment commodities & Entertainment \\ 
2 & Entertainment services & Entertainment \\ 
2 & Food & Food and beverages \\ 
2 & Food at home & Food \\ 
2 & Food away from home & Food \\ 
2 & Footwear & Apparel \\ 
2 & Fuels and utilities & Housing \\ 
2 & Homeowners costs & Housing \\ 
2 & Household energy & Fuels and utilities \\ 
2 & Household furnishings and operations & Housing \\ 
2 & Infants’ and toddlers’ apparel & Apparel \\ 
2 & Medical care commodities & Medical care \\ 
2 & Medical care services & Medical care \\ 
2 & Men’s and boys’ apparel & Apparel \\ 
2 & Nondurables less food & Nondurables \\ 
2 & Nondurables less food and apparel & Nondurables \\ 
2 & Nondurables less food and beverages & Nondurables \\ 
2 & Nondurables less food, beverages, and apparel & Nondurables \\ 
2 & Other services & Services \\ 
2 & Personal and educational expenses & Other goods and services \\ 
2 & Personal care & Other goods and services \\ 
2 & Pets, pet products and services & Recreation \\ 
2 & Photography & Recreation \\ 
2 & Private transportation & Transportation \\ 
2 & Public transportation & Transportation \\ 
2 & Rent of shelter & Services \\ 
2 & Services less energy services & All items less food and energy \\ 
2 & Services less medical care services & Services \\ 
2 & Services less rent of shelter & Services \\ 
2 & Shelter & Housing \\ 
2 & Tobacco and smoking products & Other goods and services \\ 
2 & Transportation services & Services \\ 
2 & Video and audio & Recreation \\ 
2 & Women’s and girls’ apparel & Apparel \\ \bottomrule
\end{tabularx}}
\floatfoot*{\textit{Note}: Levels and Parents of Indexes have changed over the years.}
\end{center}
\end{table}
\clearpage
%\begin{figure}[!htb]
%\centering
%\includegraphics[width=\textwidth]{figures/All_Products.png}
%\caption{Predicted vs Actual - All products}
%\end{figure}

%\input{appendix_actual_vs_predictions}

\section{Data Preparation}\label{subsec:data_prep}
%We used publicly available data for all the datasets explored, but each required different preprocessing methods. 
The hierarchical CPI data is provided as monthly index values. We transformed these CPI values into monthly logarithmic change rates. Let \(x_{t}\) represent the CPI value (of any node) at month \(t\). The logarithmic change rate at month \(t\), denoted as \(rate(t)\), is calculated as follows:
\begin{equation}
\label{eq:ratio}
rate(t) = 100 \times \log \left(\frac{x_{t}}{x_{t-1}}\right).
\end{equation}

Unless specified otherwise, the remainder of this paper focuses on monthly logarithmic change rates as defined in Equation~\eqref{eq:ratio}.

We divided the data into a \emph{training} set and a \emph{test} set as follows: for each time series, the first 75\% of the measurements (earliest in time) were assigned to the \emph{training} set, while the remaining 25\% were set aside for the \emph{test} set. The \emph{training} set was used to train the BiHRNN model and other baseline models. The \emph{test} set was used for evaluation. The results presented in Section~\ref{sec:Results} are based on this data split.

\subsection{General Data Statistics}
Table~\ref{tab:general-statistics} summarizes the number of data points and general statistics of the CPI time series after applying Equation~\eqref{eq:ratio}. A comparison between the headline CPI and the full hierarchy reveals that lower levels exhibit significantly higher standard deviations (STD) and wider ranges, indicating greater volatility. 

\setlength{\tabcolsep}{5pt}
\begin{table}[H]
\caption{Descriptive Statistics} 
\label{tab:general-statistics}
{\footnotesize
\begin{tabularx}{\textwidth}{@{}Xccccccc@{}}
\toprule[1.1pt]
{Data Set} & {\# Monthly} & {Mean} & {STD} & {Min} & {Max} & {\# of} & {Avg. Measurements} \\ 
 & Measurements & & & & & Indexes & per Index \\
 \cmidrule{2-8}
US - Headline Only     & 476 & 0.23 & 0.32 & -1.93 & 1.31 & 1 & 476 \\ 
US - Level 1           & 2064 & 0.19 & 0.85 & -18.61 & 11.32 & 26 & 79.38 \\ 
US - Level 2           & 7128 & 0.19 & 1.64 & -32.92 & 16.67 & 24 & 297 \\ 
US - Level 3           & 6476 & 0.17 & 1.75 & -34.24 & 24.81 & 25 & 259.04 \\ 
US - Level 4           & 9767 & 0.13 & 1.67 & -35.00 & 28.17 & 48 & 203.48 \\ 
US - Level 5           & 17656 & 0.11 & 2.39 & -23.89 & 242.50 & 110 & 160.51 \\ 
US - Level 6           & 10604 & 0.16 & 1.54 & -16.49 & 17.06 & 73 & 145.26 \\ 
US - Level 7           & 4968 & 0.21 & 1.64 & -11.89 & 17.84 & 36 & 138 \\ 
US - Level 8           & 980 & 0.21 & 1.70 & -8.65 & 7.80 & 7 & 140 \\ 
\bottomrule
\end{tabularx}\vspace{20pt}
\begin{tabularx}{\textwidth}{@{}Xccccccc@{}}
\toprule[1.1pt]
{Data Set} & {\# Monthly} & {Mean} & {STD} & {Min} & {Max} & {\# of} & {Avg. Measurements} \\ 
 & Measurements & & & & & Indexes & per Index \\
 \cmidrule{2-8}
Canada - Headline Only & 121 & 0.20 & 0.40 & -0.72 & 1.42 & 1 & 121 \\ 
Canada - Level 1       & 2057 & 0.19 & 1.11 & -10.78 & 8.17 & 17 & 121 \\ 
Canada - Level 2       & 2541 & 0.18 & 1.56 & -14.36 & 16.20 & 21 & 121 \\ 
Canada - Level 3       & 7381 & 0.20 & 1.92 & -20.69 & 27.92 & 61 & 121 \\ 
Canada - Level 4       & 10648 & 0.17 & 2.29 & -27.09 & 31.86 & 88 & 121 \\ 
Canada - Level 5       & 9680 & 0.19 & 2.93 & -34.03 & 35.43 & 80 & 121 \\ 
Canada - Level 6       & 3025 & 0.20 & 2.48 & -19.12 & 20.03 & 25 & 121 \\ 
\bottomrule
\end{tabularx}\vspace{20pt}
\begin{tabularx}{\textwidth}{@{}Xccccccc@{}}
\toprule[1.1pt]
{Data Set} & {\# Monthly} & {Mean} & {STD} & {Min} & {Max} & {\# of} & {Avg. Measurements} \\ 
 & Measurements & & & & & Indexes & per Index \\
 \cmidrule{2-8}
Norway - Headline Only & 172 & 0.22 & 0.45 & -0.93 & 1.36 & 1 & 172 \\ 
Norway - Level 1       & 2064 & 0.21 & 1.49 & -12.82 & 10.88 & 12 & 172 \\ 
Norway - Level 2       & 6708 & 0.21 & 2.15 & -20.79 & 27.31 & 39 & 172 \\ 
\bottomrule
\end{tabularx}
\begin{tablenotes}
\item {\footnotesize \textit{Notes:} General statistics of the headline CPI and CPI per each level in the hierarchy across Canada, Norway, and the US.}
\end{tablenotes}}
\end{table}




Figure~\ref{fig:boxplots_hierarchy_all} depicts box plots of the CPI change rate distributions at different levels. The boxes depict the median value and the upper 75'th and lower 25'th percentiles. The figure further emphasize that the change rates are more volatile as we go down the CPI hierarchy.
A high dynamic range, high standard deviation, and limited training data all signal increased difficulty in making predictions within the hierarchy. Given this, we can anticipate that predictions for disaggregated components within the hierarchy will be more challenging than those for the headline.

\begin{figure}[H]
    \centering
    
    \subfloat[Canada CPI by Hierarchy Level]{
        \includegraphics[width=0.6\textwidth]{figures/canada_boxplot_per_indent.pdf}
        \label{fig:boxplots_hierarchy_canada}
    }
    \hfill
    \subfloat[Norway CPI by Hierarchy Level]{
        \includegraphics[width=0.6\textwidth]{figures/norway_boxplot_per_indent.pdf}
        \label{fig:boxplots_hierarchy_norway}
    }
    \hfill
    \subfloat[US CPI by Hierarchy Level]{
        \includegraphics[width=0.6\textwidth]{figures/us_boxplot_per_indent.pdf}
        \label{fig:boxplots_hierarchy_us}
    }
    
    \caption{CPI Distributions by Hierarchy Level for Canada, Norway, and the US}
    \label{fig:boxplots_hierarchy_all}
\end{figure}

Figure~\ref{fig:boxplot_per_sector} presents a box plot showing the distribution of CPI change rates across various sectors. It is evident that certain sectors, like Apparel and Transportation, exhibit greater volatility than others across all markets. This higher volatility is likely to make predictions for these sectors more challenging, as anticipated.

\begin{figure}[H]
    \centering
    \includegraphics[width=1.1\textwidth]{figures/boxplot_per_sector.pdf} % Adjust width as needed
    \caption{Monthly Rate per Sector Per Country}
    \label{fig:boxplot_per_sector}
\end{figure}


\subsubsection{Data Augmentation}\label{subsec:data_aug}
As part of the pre-processing process we addressed the issue of missing weights in the Canadian dataset by employing a linear regression-based imputation method. In this approach, the prices of child categories were used as the independent variables (predictors), while the price of the parent category was treated as the dependent variable (target). This regression model allowed us to estimate the missing weights by leveraging the relationship between the prices of child and parent categories, thereby ensuring a consistent and reliable imputation strategy. Since the model's loss function incorporates the weights of each child category, this imputation step was a critical part of the preprocessing pipeline.