%\title{TAU_MSc_PhD_Thesis_Template}
%%
%% Copyright 2007, 2008, 2009 Elsevier Ltd
%%
%% This file is part of the 'Elsarticle Bundle'.
%% --------------------------------------------
%%
%% It may be distributed under the conditions of the LaTeX Project Public
%% License, either version 1.2 of this license or (at your option) any
%% later version.  The latest version of this license is in
%%    http://www.latex-project.org/lppl.txt
%% and version 1.2 or later is part of all distributions of LaTeX
%% version 1999/12/01 or later.
%%
%% The list of all files belonging to the 'Elsarticle Bundle' is
%% given in the file manifest.txt'.
%%

%% Template article for Elsevier's document class elsarticle'
%% with numbered style bibliographic references
%% SP 2008/03/01
%%
%%
%%
%% $Id: elsarticle-template-num.tex 4 2009-10-24 08:22:58Z rishi $
%%
%%


%% Use the option review to obtain double line spacing
%% \documentclass[preprint,review,12pt]{elsarticle}

%% Use the options 1p,twocolumn; 3p; 3p,twocolumn; 5p; or 5p,twocolumn
%% for a journal layout:
%% \documentclass[final,1p,times]{elsarticle}
%% \documentclass[final,1p,times,twocolumn]{elsarticle}
%% \documentclass[final,3p,times]{elsarticle}
%% \documentclass[final,3p,times,twocolumn]{elsarticle}
%% \documentclass[final,5p,times]{elsarticle}
%% \documentclass[final,5p,times,twocolumn]{elsarticle}

%% if you use PostScript figures in your article
%% use the graphics package for simple commands
%% \usepackage{graphics} 
%\usepackage{graphicx}
%% or use the epsfig package if you prefer to use the old commands
%% \usepackage{epsfig}

%% The amssymb package provides various useful mathematical symbols

%% The amsthm package provides extended theorem environments
%% \usepackage{amsthm}

%% The lineno packages adds line numbers. Start line numbering with
%% \begin{linenumbers}, end it with \end{linenumbers}. Or switch it on
%% for the whole article with \linenumbers after \end{frontmatter}.
%% \usepackage{lineno}

%% natbib.sty is loaded by default. However, natbib options can be
%% provided with \biboptions{...} command. Following options are
%% valid:
 
%%   round  -  round parentheses are used (default)
%%   square -  square brackets are used   [option]
%%   curly  -  curly braces are used      {option}
%%   angle  -  angle brackets are used    <option>
%%   semicolon  -  multiple citations separated by semi-colon
%%   colon  - same as semicolon, an earlier confusion 
%%   numbers-  selects numerical citations
%%   super  -  numerical citations as superscripts
%%   sort   -  sorts multiple citations according to order in ref. list
%%   sort&compress   -  like sort, but also compresses numerical citations
%%   compress - compresses without sorting
%%
%% \biboptions{comma,round}

% \biboptions{}
\documentclass{report}
\usepackage{mathptmx}
\renewcommand{\familydefault}{\rmdefault}
\usepackage[a4paper]{geometry}
\geometry{verbose,tmargin=2cm,bmargin=2cm,lmargin=2cm,rmargin=2cm,headheight=1cm,headsep=1cm,footskip=1cm}
\setcounter{secnumdepth}{3}
\setcounter{tocdepth}{3}
\setlength{\parskip}{\medskipamount}
\setlength{\parindent}{0pt}
\usepackage{verbatim}
\usepackage{pdfpages}
%\usepackage{graphicx}
\usepackage{setspace}
% \usepackage[numbers]{natbib}
\usepackage{nomencl}
\usepackage{blindtext} % needed for creating dummy text passages
%\usepackage{ngerman} % needed for German default language
\usepackage{amsmath} % needed for command eqref
\usepackage{amssymb} % needed for math fonts
%\usepackage[hyphenbreaks]{breakurl} 
\usepackage{array}
%\usepackage{cite} % needed for cite
%\usepackage[round,authoryear]{natbib} % needed for cite and abbrvnat bibliography style
%\usepackage[nottoc]{tocbibind} % needed for displaying bibliography and other in the table of contents

\usepackage{graphicx} % needed for \includegraphics 

\usepackage{longtable} % needed for long tables over pages
\usepackage{bigstrut} % needed for the command \bigstrut
\usepackage{enumerate} % needed for some options in enumerate
\usepackage{todonotes} % needed for todos
\usepackage{makeidx} % needed for creating an index
\makeindex

\usepackage{svg}
\usepackage{amsmath,amsthm,amssymb}
\usepackage{verbatim}
\usepackage{dsfont}
\usepackage[T1]{fontenc}
\usepackage[osf]{newpxtext}
\usepackage{textcomp} % required for special glyphs
\usepackage[varg,bigdelims,cmintegrals,cmbraces]{newpxmath}
\usepackage[scr=rsfso]{mathalfa}
\usepackage{bm}
\usepackage{xspace}
\linespread{1.05}% Give Palatino more leading (space between lines)
\usepackage{soul} %highlight text

%% Margins and spaces
\usepackage{geometry}
\geometry{left=1.2in,right=1.2in ,top=1.2in,bottom=1.2in}
\geometry{left=1in,right=1in ,top=1in,bottom=1in}
\usepackage{setspace}

%\usepackage[affil-sl]{authblk}
\usepackage{microtype}

%% Bibliography
\usepackage{natbib}
\setlength{\bibsep}{0pt plus 0.3ex}
\renewcommand\refname{References}

%% Paragraphs, tables figures and appendix
\usepackage{graphicx}
\usepackage{epstopdf}
\usepackage{epsfig}
\usepackage{color}
\usepackage{psfrag}
\usepackage{subfig}
\usepackage{tabulary}
\usepackage{tabularx}
\usepackage{booktabs}
\usepackage[flushleft]{threeparttable}
\usepackage{makecell}
%\usepackage{slashbox}
\setlength{\parskip}{0.1ex}
\usepackage[title,titletoc,toc]{appendix}
\usepackage{placeins}
\usepackage{ragged2e}
\usepackage{caption}
%\captionsetup[table]{format=plain,labelsep=period, justification=centering,singlelinecheck=false, labelfont = {small,bf}, textfont = {small,it}, skip=0pt}
\captionsetup[figure]{format=plain,labelsep=period, justification=centering,singlelinecheck=false, labelfont = {small,bf}, textfont = {small,it}, skip=0pt}
\captionsetup[subfigure]{format=plain,labelsep=space, labelformat=simple, justification=centering, position = top, labelfont = {small,bf}, textfont = {small,it}}
\renewcommand\thesubfigure{(\alph{subfigure})}

\definecolor{darkblue}{rgb}{0,0,0.5}
\definecolor{mauve}{rgb}{0.88,0.69,1}
\usepackage[pdftex, colorlinks=true, citecolor=darkblue, linkcolor=darkblue, urlcolor=darkblue]{hyperref}
\hypersetup{
    pdftitle={},
    pdfauthor={},
    pdfsubject={},
    pdfkeywords={},
    bookmarksnumbered=true,     
    bookmarksopen=true,         
    bookmarksopenlevel=1,       
    colorlinks=true,            
    pdfstartview={XYZ null null 0.75},        
    pdfpagemode=UseOutlines,    % this is the option you were lookin for
    pdfpagelayout=SinglePage
}
\usepackage{multirow}

\usepackage{titlesec}
%\titlelabel{\thetitle.\:\:}
\titleformat{\chapter}[display]
  {\normalfont\scshape\Large\centering}
  {\chaptertitlename\ \thechapter}{0pt}{\Huge}
%\titleformat{\section}
%  {\normalfont\sffamily\large\bfseries\centering}
%  {\thesection}{1em}{}
%\titleformat{\subsection}
%  {\normalfont\sffamily\normalsize\centering}
%  {\thesubsection}{1em}{}
\usepackage[bottom]{footmisc}
\usepackage[capposition=top]{floatrow}
\usepackage{lscape} 

\usepackage{rotating}

\usepackage{lineno,hyperref}
\modulolinenumbers[5]
\usepackage{amsmath}
\usepackage{amsfonts}
\usepackage{mathtools}
\usepackage{import}
\usepackage{graphicx}[width=\columnwidth] %,scale=1.5
\usepackage[export]{adjustbox}
\usepackage{rotating}
\usepackage{amssymb}
\usepackage[inline]{enumitem}
\usepackage{multimedia}
\usepackage{graphics}
\usepackage{bbm}
\usepackage{pstricks}
\usepackage{pgf}
\usepackage{wrapfig} 
\usepackage{rotfloat}
\usepackage{float}
% \usepackage{tabulary}
% \usepackage{tabularx}
\usepackage{booktabs}
\usepackage{multirow}
%\usepackage{subfig}
%\usepackage{graphicx} 
%\usepackage{caption}
\usepackage{pdflscape}



% the following is useful when we have the old nomencl.sty package
\providecommand{\printnomenclature}{\printglossary}
\providecommand{\makenomenclature}{\makeglossary}
\makenomenclature
\doublespacing

\makeatletter

%%%%%%%%%%%%%%%%%%%%%%%%%%%%%% LyX specific LaTeX commands.
\providecommand{\LyX}{L\kern-.1667em\lower.25em\hbox{Y}\kern-.125emX\@}
%% Because html converters don't know tabularnewline
\providecommand{\tabularnewline}{\\}
%% A simple dot to overcome graphicx limitations
\newcommand{\lyxdot}{.}


\DeclarePairedDelimiterX{\kldivX}[2]{(}{)}{%
	#1\;\delimsize\|\;#2%
}
\DeclarePairedDelimiter{\norm}{\lVert}{\rVert}
%%%%%%%%%%%%%%%%%%%%%%%%%%%%%% User specified LaTeX commands.
\usepackage{tauthesis}
\usepackage[font={small,bf}, labelfont={small,bf}, margin=1cm]{caption}
\usepackage{titlesec}
\newcommand{\hsp}{\hspace{20pt}}

\newcommand{\noam}[1]{\textcolor{red}{#1 ~(noam)}}

\titleformat{\chapter}[hang]{\Huge\bfseries}{\thechapter\hsp}{0pt}{\Huge\bfseries}


\Title{\textbf{BiHRNN: Bi-Directional Hierarchical Recurrent Neural Network for Inflation Forecasting}}
\Author{Maya Vilenko}
\Year{December 2024}
\Supervisor{Prof. Noam Koeninstein}
\Department{School of Industrial Engineering}
\Degree{Master of Science}
% \Degree{Doctor of Philosophy}

\makeatother

\usepackage[english]{babel}
\begin{document}

\prelimpages

\titlepage




Since 2020, GitGuardian has been detecting checked-in hard-coded secrets in GitHub repositories. During 2020-2023, GitGuardian has observed an upward annual trend and a four-fold increase in hard-coded secrets, with 12.8 million exposed in 2023. However, removing all the secrets from software artifacts is not feasible due to time constraints and technical challenges. Additionally, the security risks of the secrets are not equal, protecting assets ranging from obsolete databases to sensitive medical data. Thus, secret removal should be prioritized by security risk reduction, which existing secret detection tools do not support. \textit{The goal of this research is to aid software practitioners in prioritizing secrets removal efforts through our security risk-based tool}. We present RiskHarvester, a risk-based tool to compute a security risk score based on the value of the asset and ease of attack on a database. We calculated the value of asset by identifying the sensitive data categories present in a database from the database keywords in the source code. We utilized data flow analysis, SQL, and Object Relational Mapper (ORM) parsing to identify the database keywords. To calculate the ease of attack, we utilized passive network analysis to retrieve the database host information. To evaluate RiskHarvester, we curated RiskBench, a benchmark of 1,791 database secret-asset pairs with sensitive data categories and host information manually retrieved from 188 GitHub repositories. RiskHarvester demonstrates precision of (95\%) and recall (90\%) in detecting database keywords for the value of asset and precision of (96\%) and recall (94\%) in detecting valid hosts for ease of attack. Finally, we conducted a survey (52 respondents) to understand whether developers prioritize secret removal based on security risk score. We found that 86\% of the developers prioritized the secrets for removal with descending security risk scores.

\tableofcontents{}
%\footnote{Split the thesis into separate chapters. Use \textbackslash{}include mode to include the separate files.}

\acknowledgments{I would like to express my deepest gratitude to my supervisor, Dr. Noam Koeningstein, for his invaluable guidance, support, and encouragement throughout this research. His expertise and insights have been instrumental in shaping the direction and outcomes of this work.}

\textpages

\listoffigures


\listoftables


\printnomenclature{}

\section{Introduction}

Large language models (LLMs) have achieved remarkable success in automated math problem solving, particularly through code-generation capabilities integrated with proof assistants~\citep{lean,isabelle,POT,autoformalization,MATH}. Although LLMs excel at generating solution steps and correct answers in algebra and calculus~\citep{math_solving}, their unimodal nature limits performance in plane geometry, where solution depends on both diagram and text~\citep{math_solving}. 

Specialized vision-language models (VLMs) have accordingly been developed for plane geometry problem solving (PGPS)~\citep{geoqa,unigeo,intergps,pgps,GOLD,LANS,geox}. Yet, it remains unclear whether these models genuinely leverage diagrams or rely almost exclusively on textual features. This ambiguity arises because existing PGPS datasets typically embed sufficient geometric details within problem statements, potentially making the vision encoder unnecessary~\citep{GOLD}. \cref{fig:pgps_examples} illustrates example questions from GeoQA and PGPS9K, where solutions can be derived without referencing the diagrams.

\begin{figure}
    \centering
    \begin{subfigure}[t]{.49\linewidth}
        \centering
        \includegraphics[width=\linewidth]{latex/figures/images/geoqa_example.pdf}
        \caption{GeoQA}
        \label{fig:geoqa_example}
    \end{subfigure}
    \begin{subfigure}[t]{.48\linewidth}
        \centering
        \includegraphics[width=\linewidth]{latex/figures/images/pgps_example.pdf}
        \caption{PGPS9K}
        \label{fig:pgps9k_example}
    \end{subfigure}
    \caption{
    Examples of diagram-caption pairs and their solution steps written in formal languages from GeoQA and PGPS9k datasets. In the problem description, the visual geometric premises and numerical variables are highlighted in green and red, respectively. A significant difference in the style of the diagram and formal language can be observable. %, along with the differences in formal languages supported by the corresponding datasets.
    \label{fig:pgps_examples}
    }
\end{figure}



We propose a new benchmark created via a synthetic data engine, which systematically evaluates the ability of VLM vision encoders to recognize geometric premises. Our empirical findings reveal that previously suggested self-supervised learning (SSL) approaches, e.g., vector quantized variataional auto-encoder (VQ-VAE)~\citep{unimath} and masked auto-encoder (MAE)~\citep{scagps,geox}, and widely adopted encoders, e.g., OpenCLIP~\citep{clip} and DinoV2~\citep{dinov2}, struggle to detect geometric features such as perpendicularity and degrees. 

To this end, we propose \geoclip{}, a model pre-trained on a large corpus of synthetic diagram–caption pairs. By varying diagram styles (e.g., color, font size, resolution, line width), \geoclip{} learns robust geometric representations and outperforms prior SSL-based methods on our benchmark. Building on \geoclip{}, we introduce a few-shot domain adaptation technique that efficiently transfers the recognition ability to real-world diagrams. We further combine this domain-adapted GeoCLIP with an LLM, forming a domain-agnostic VLM for solving PGPS tasks in MathVerse~\citep{mathverse}. 
%To accommodate diverse diagram styles and solution formats, we unify the solution program languages across multiple PGPS datasets, ensuring comprehensive evaluation. 

In our experiments on MathVerse~\citep{mathverse}, which encompasses diverse plane geometry tasks and diagram styles, our VLM with a domain-adapted \geoclip{} consistently outperforms both task-specific PGPS models and generalist VLMs. 
% In particular, it achieves higher accuracy on tasks requiring geometric-feature recognition, even when critical numerical measurements are moved from text to diagrams. 
Ablation studies confirm the effectiveness of our domain adaptation strategy, showing improvements in optical character recognition (OCR)-based tasks and robust diagram embeddings across different styles. 
% By unifying the solution program languages of existing datasets and incorporating OCR capability, we enable a single VLM, named \geovlm{}, to handle a broad class of plane geometry problems.

% Contributions
We summarize the contributions as follows:
We propose a novel benchmark for systematically assessing how well vision encoders recognize geometric premises in plane geometry diagrams~(\cref{sec:visual_feature}); We introduce \geoclip{}, a vision encoder capable of accurately detecting visual geometric premises~(\cref{sec:geoclip}), and a few-shot domain adaptation technique that efficiently transfers this capability across different diagram styles (\cref{sec:domain_adaptation});
We show that our VLM, incorporating domain-adapted GeoCLIP, surpasses existing specialized PGPS VLMs and generalist VLMs on the MathVerse benchmark~(\cref{sec:experiments}) and effectively interprets diverse diagram styles~(\cref{sec:abl}).

\iffalse
\begin{itemize}
    \item We propose a novel benchmark for systematically assessing how well vision encoders recognize geometric premises, e.g., perpendicularity and angle measures, in plane geometry diagrams.
	\item We introduce \geoclip{}, a vision encoder capable of accurately detecting visual geometric premises, and a few-shot domain adaptation technique that efficiently transfers this capability across different diagram styles.
	\item We show that our final VLM, incorporating GeoCLIP-DA, effectively interprets diverse diagram styles and achieves state-of-the-art performance on the MathVerse benchmark, surpassing existing specialized PGPS models and generalist VLM models.
\end{itemize}
\fi

\iffalse

Large language models (LLMs) have made significant strides in automated math word problem solving. In particular, their code-generation capabilities combined with proof assistants~\citep{lean,isabelle} help minimize computational errors~\citep{POT}, improve solution precision~\citep{autoformalization}, and offer rigorous feedback and evaluation~\citep{MATH}. Although LLMs excel in generating solution steps and correct answers for algebra and calculus~\citep{math_solving}, their uni-modal nature limits performance in domains like plane geometry, where both diagrams and text are vital.

Plane geometry problem solving (PGPS) tasks typically include diagrams and textual descriptions, requiring solvers to interpret premises from both sources. To facilitate automated solutions for these problems, several studies have introduced formal languages tailored for plane geometry to represent solution steps as a program with training datasets composed of diagrams, textual descriptions, and solution programs~\citep{geoqa,unigeo,intergps,pgps}. Building on these datasets, a number of PGPS specialized vision-language models (VLMs) have been developed so far~\citep{GOLD, LANS, geox}.

Most existing VLMs, however, fail to use diagrams when solving geometry problems. Well-known PGPS datasets such as GeoQA~\citep{geoqa}, UniGeo~\citep{unigeo}, and PGPS9K~\citep{pgps}, can be solved without accessing diagrams, as their problem descriptions often contain all geometric information. \cref{fig:pgps_examples} shows an example from GeoQA and PGPS9K datasets, where one can deduce the solution steps without knowing the diagrams. 
As a result, models trained on these datasets rely almost exclusively on textual information, leaving the vision encoder under-utilized~\citep{GOLD}. 
Consequently, the VLMs trained on these datasets cannot solve the plane geometry problem when necessary geometric properties or relations are excluded from the problem statement.

Some studies seek to enhance the recognition of geometric premises from a diagram by directly predicting the premises from the diagram~\citep{GOLD, intergps} or as an auxiliary task for vision encoders~\citep{geoqa,geoqa-plus}. However, these approaches remain highly domain-specific because the labels for training are difficult to obtain, thus limiting generalization across different domains. While self-supervised learning (SSL) methods that depend exclusively on geometric diagrams, e.g., vector quantized variational auto-encoder (VQ-VAE)~\citep{unimath} and masked auto-encoder (MAE)~\citep{scagps,geox}, have also been explored, the effectiveness of the SSL approaches on recognizing geometric features has not been thoroughly investigated.

We introduce a benchmark constructed with a synthetic data engine to evaluate the effectiveness of SSL approaches in recognizing geometric premises from diagrams. Our empirical results with the proposed benchmark show that the vision encoders trained with SSL methods fail to capture visual \geofeat{}s such as perpendicularity between two lines and angle measure.
Furthermore, we find that the pre-trained vision encoders often used in general-purpose VLMs, e.g., OpenCLIP~\citep{clip} and DinoV2~\citep{dinov2}, fail to recognize geometric premises from diagrams.

To improve the vision encoder for PGPS, we propose \geoclip{}, a model trained with a massive amount of diagram-caption pairs.
Since the amount of diagram-caption pairs in existing benchmarks is often limited, we develop a plane diagram generator that can randomly sample plane geometry problems with the help of existing proof assistant~\citep{alphageometry}.
To make \geoclip{} robust against different styles, we vary the visual properties of diagrams, such as color, font size, resolution, and line width.
We show that \geoclip{} performs better than the other SSL approaches and commonly used vision encoders on the newly proposed benchmark.

Another major challenge in PGPS is developing a domain-agnostic VLM capable of handling multiple PGPS benchmarks. As shown in \cref{fig:pgps_examples}, the main difficulties arise from variations in diagram styles. 
To address the issue, we propose a few-shot domain adaptation technique for \geoclip{} which transfers its visual \geofeat{} perception from the synthetic diagrams to the real-world diagrams efficiently. 

We study the efficacy of the domain adapted \geoclip{} on PGPS when equipped with the language model. To be specific, we compare the VLM with the previous PGPS models on MathVerse~\citep{mathverse}, which is designed to evaluate both the PGPS and visual \geofeat{} perception performance on various domains.
While previous PGPS models are inapplicable to certain types of MathVerse problems, we modify the prediction target and unify the solution program languages of the existing PGPS training data to make our VLM applicable to all types of MathVerse problems.
Results on MathVerse demonstrate that our VLM more effectively integrates diagrammatic information and remains robust under conditions of various diagram styles.

\begin{itemize}
    \item We propose a benchmark to measure the visual \geofeat{} recognition performance of different vision encoders.
    % \item \sh{We introduce geometric CLIP (\geoclip{} and train the VLM equipped with \geoclip{} to predict both solution steps and the numerical measurements of the problem.}
    \item We introduce \geoclip{}, a vision encoder which can accurately recognize visual \geofeat{}s and a few-shot domain adaptation technique which can transfer such ability to different domains efficiently. 
    % \item \sh{We develop our final PGPS model, \geovlm{}, by adapting \geoclip{} to different domains and training with unified languages of solution program data.}
    % We develop a domain-agnostic VLM, namely \geovlm{}, by applying a simple yet effective domain adaptation method to \geoclip{} and training on the refined training data.
    \item We demonstrate our VLM equipped with GeoCLIP-DA effectively interprets diverse diagram styles, achieving superior performance on MathVerse compared to the existing PGPS models.
\end{itemize}

\fi 


\chapter{BiHRNN: Bi-Directional Hierarchical Recurrent Neural Network}

Hierarchical data refers to structured data organized into multiple levels, with each level representing a different degree of aggregation or detail. In the context of inflation, hierarchical data can range from the overall inflation index, such as the Consumer Price Index (CPI), down to disaggregated components like regional indices or individual product categories. This multi-level structure allows for analysis at varying levels of granularity, capturing relationships and dependencies across different layers. Such an approach is particularly valuable for inflation modeling, as changes at lower levels, such as specific goods or services, can propagate upward and influence aggregate measures, enabling more accurate and comprehensive forecasting.

\section{Hierarchical Recurrent Neural Networks}
Before delving into the specifics of the BiHRNN model, we provide a brief overview of its predecessor, the HRNN model \citep{BARKAN20231145}. The HRNN model is specifically designed to address the challenges of inflation forecasting in hierarchically structured datasets, where lower levels are often characterized by data sparsity and heightened volatility in change rates. To enhance predictions, the HRNN propagates information from parent categories to node categories within the hierarchy by employing hierarchical Gaussian priors. This approach connects each node’s parameters to its parent’s, allowing the model to share information across levels of aggregation. By leveraging parent-level information, the HRNN mitigates the effects of sparse or noisy data at finer levels and ensures consistency in forecasts across the hierarchy. Furthermore, it utilizes RNNs, specifically Gated Recurrent Units (GRUs) \citep{GRU}, which incorporate a feedback loop, enabling predictions to account for temporal dependencies. This combination of hierarchical priors and temporal modeling makes the HRNN particularly effective for capturing both cross-level interactions and dynamic patterns in inflation data \citep{RNNs_Book, RNN_evaluations}.

\subsection{Gated Recurrent Units (GRU)}
A Gated Recurrent Unit (GRU) is a type of Recurrent Neural Network (RNN) designed to capture long-term dependencies in sequential data by addressing the vanishing gradient problem common in traditional RNNs. GRUs use two gates—update and reset—to manage information flow. The update gate controls how much past information is passed along to future states, while the reset gate determines how much past information is forgotten, allowing GRUs to retain relevant information over longer sequences.

The following set of equations defines a GRU unit: 
\begin{equation}
\label{eq:gru}
\begin{split}
    z = & \sigma(x_tu^z + s_{t-1}w^z + b^z), \\
    r = & \sigma(x_tu^r + s_{t-1}w^r + b^r), \\
    v = & \tanh{(x_tu^v+(s_{t-1} \times r)w^v +b^v)}, \\
    s_t = & z \times v + (1-z)s_{t-1},
\end{split}
\end{equation}
where $u^z$, $w^z$ and $b^z$ are learned parameters governing the \emph{update gate} $z$, while $u^r$, $w^r$ and $b^r$ are the learned parameters for the \emph{reset gate} $r$. The candidate activation $v$ determined by the input $x_t$ and the previous output $s_{t-1}$, and is influenced by the learned parameters: $u^v$, $w^v$ and $b^v$. Finally, the output $s_t$ is a combination of the candidate activation $v$ and the previous state $s_{t-1}$ controlled by the \emph{update gate} $z$. Figure~\ref{fig:GRU} depicts an illustration of a GRU unit.

\begin{figure}[!htb]
    \centering
    %\subfigure
    {   \makebox[\textwidth]{
        \includegraphics[scale=0.6]{figures/GRU.pdf}}
    }
    \caption
    {An illustration of a GRU unit.}
    \label{fig:GRU}
     \floatfoot*{\textit{Each line represents a vector, connecting the output of one node to the inputs of others. Pink circles indicate point-wise operations, while yellow boxes represent learned neural network layers. When lines merge, it signifies concatenation; when a line forks, it means the vector is copied, with each copy directed to different destinations.}}
\end{figure}

GRUs form the foundational unit of the HRNN model, detailed in Section \ref{subsec:hrnn} as well as our BiHRNN model detailed in Section \ref{subsec:model}.

\subsection{HRNN}\label{subsec:hrnn}
Next, we proceed with a description of the HRNN model. 
The following notations apply to both HRNN and Bidirectional HRNN models:

Let \(\mathcal{I}=\{n\}_{n=1}^N\) represent dataset graph nodes, each associated with a parent \(\pi_n\). For node \(n\), \(x_t^n\) is its observed value at time \(t\), and \(X_t^n\) represents the sequence up to \(t\).
A parametric function \(g\) (a GRU node) predicts the next value in the sequence, learning parameters \(\theta_n\) to predict \(x_{t+1}^n\). Assuming Gaussian errors, the likelihood of the time series is modeled as a product of normal distributions, with \(\tau_n^{-1}\) as the error variance.
A hierarchical informative prior connects each node’s parameters to its parent’s, with the prior \( p(\theta_n|\theta_{\pi_n},\tau_{\theta_n}) \) using \(\tau_{\theta_n}\) as a precision parameter. Higher \(\tau_{\theta_n}\) values indicate stronger parameter connections between node \(n\) and its parent \(\pi_n\).
Instead of globally optimizing \(\tau_{\theta_n}\), the HRNN sets \(\tau_{\theta_n} = e^{\alpha + C_n}\), where \(\alpha\) is a hyperparameter and \(C_n\) is the Pearson correlation between node \(n\) and its parent \(\pi_n\). This ensures that node \(n\) stays close to \(\pi_n\) in parameter space, especially when correlation is high. For the root node, a non-informative Gaussian prior with zero mean and unit variance is used.

Let \(X=\{X_{T_n}^n\}_{n\in \mathcal{I}}\) represent all time series, \(\theta=\{\theta_n\}_{n\in \mathcal{I}}\) the GRU parameters, and \(\tau=\{\tau_n\}_{n\in \mathcal{I}}\) the precision parameters. Here, \(X\) is observed, \(\theta\) contains learned variables, and \(\tau\) is defined by \(\tau_{\theta_n}\).

Given the the likelihood of the observed time series and the priors aforementioned above, the posterior probability is then extracted and is formulated according to Equation~\eqref{eq:posterior}.
\begin{equation}
\label{eq:posterior}
\begin{split}
p(\theta|X,\tau) &= \frac{p(X|\theta,\tau)p(\theta)}{P(X)} \propto  
\prod_{n\in \mathcal{I}}\prod_{t=1}^{T_n}\mathcal{N}(x_t^n;g(\theta_n,X_{t-1}^n),\tau_n^{-1})\prod_{n\in \mathcal{I}}\mathcal{N}(\theta_n;\theta_{\pi_n},\tau_{\theta_n}^{-1}\mathbf{I}).
\end{split}
\end{equation}

HRNN optimization follows a \textit{Maximum A-Posteriori} (MAP) approach to find the optimal parameters \(\theta^*\) by maximizing the posterior probability.
\begin{equation}
\label{eq:obj}
\theta^*=\underset{ \theta}{\text{argmax}}\log p(\theta|X,\tau).
\end{equation}

The optimization is performed using stochastic gradient ascent on this objective. Figure~\ref{fig:HRNN} illustrates the HRNN architecture.

\begin{figure}[H]
    \centering
    \includegraphics[scale=0.25]{figures/Slide2.jpeg}
    \caption{Illustration of the HRNN Model.}
    \label{fig:HRNN}
    
\end{figure}


\section{Bidirectional HRNN Model}\label{subsec:model}

The Bidirectional HRNN (BiHRNN) builds upon the HRNN framework by enabling bidirectional information flow within hierarchical data structures, addressing a critical limitation of its predecessor. While the HRNN model effectively propagates information from parent categories to child categories, improving predictions for lower, more volatile levels, it does not leverage the potential benefits of propagating information in the reverse direction—from child categories back to their parents. Granular-level data often contains unique patterns or anomalies that can inform and refine higher-level predictions. By incorporating  bidirectional information flow, BiHRNN enhances the consistency and accuracy of predictions across all levels of the hierarchy.

The motivation for this extension lies in the interconnected nature of hierarchical data, particularly in the context of inflation forecasting. Economic indices at higher aggregation levels, such as national inflation rates, are directly influenced by fluctuations in disaggregated categories, such as specific goods, services, or regional indices. Without accounting for these influences, predictions at higher levels may miss critical insights embedded in lower-level data. By enabling information to flow upward, BiHRNN allows parent-level categories to benefit from the granularity and detail captured at the child level, thereby improving overall forecast accuracy and coherence across the entire hierarchy.

BiHRNN introduces a dual-constraint formulation, which integrates both top-down and bottom-up information; the constraints are applied during training to ensure structural consistency; and the enhanced loss function that governs its optimization, balancing accuracy across all hierarchical levels. These advancements make BiHRNN a more robust and versatile tool for forecasting in complex hierarchical datasets, particularly in domains like inflation modeling where cross-level interactions are critical.

\subsection{Bidirectional Information Flow}

BiHRNN is formulated as a risk minimization optimization problem. Instead of HRNN's informative prior, BiHRNN introduces two constraints on the models parameters. One ties the parameters of each time series to its parent's (similar to HRNN's prior) and the second ties the parameters of each of its' child series, with an appropriate weight:
\begin{itemize}
    \item \textbf{Parent-Node Constraint:} This constraint governs the relationship between the parameters of a node $n$ and its parent $\pi_n$. By aligning the parameters of node $n$ with those of its parent, this constraint ensures hierarchical consistency and allows top-down information to flow through the network.
    \item \textbf{Child-Node Constraint:} This constraint governs the relationship between node $n$ and its children $\eta_{i_n}$, where $\eta_{i_n}$ represents the $i$-th child of node $n$. This enables the node to aggregate information from its children, allowing bottom-up feedback to influence higher-level nodes.
\end{itemize}

By incorporating these two constraints, BiHRNN enables information to flow in both directions---\textit{downward} from parent to child and \textit{upward} from child to parent. This bidirectional approach significantly enhances the model’s ability to capture complex dependencies and interactions across hierarchical levels, leading to improved predictive performance.

\subsection{Customized Loss Function}

The information flow in BiHRNN is governed by a \textit{customized loss function} that balances prediction accuracy with hierarchical consistency. The loss function comprises three key components:

\begin{enumerate}
    \item \textbf{Mean Squared Error (MSE):}
    The primary objective of BiHRNN is to minimize prediction errors. This is achieved using the mean squared error:
    \begin{equation}
    \text{MSE} = \frac{1}{N} \sum \left( y - \hat{y} \right)^2
    \end{equation}
    where $y$ represents the observed values, $\hat{y}$ denotes the predicted values, and $N$ is the number of observations.
    
    \item \textbf{Parent Regularization ($l_{\text{parent}}$):}
    To ensure hierarchical coherence, the model penalizes the squared Euclidean distance between the parameters of a node $n$ and its parent $\pi_n$:
    \begin{equation}
    l_{\text{parent}} = \left( \theta_{\text{parent}} - \theta \right)^2
    \end{equation}
    This term enforces consistency between a node and its parent, ensuring that higher-level nodes influence lower-level nodes appropriately.
    
    \item \textbf{Child Regularization ($l_{\text{child}}$):}
    Similarly, the model incorporates the influence of child nodes through a weighted penalty:
    \begin{equation}
    l_{\text{child}} = \sum_{i \in \text{children}} w_i \left( \theta - \theta_{\text{child}} \right)^2
    \end{equation}
    where $w_i$ is a weight controlling the contribution of each child. This term allows the parameters of node $n$ to aggregate information from its children, enabling bottom-up information flow.
\end{enumerate}

The final loss function combines these components, weighted by hyperparameters $\lambda_1$ and $\lambda_2$ to control the relative importance of parent and child regularization:
\begin{equation}
\text{Loss}_{\text{BiHRNN}} = \frac{1}{N} \sum \left( y - \hat{y} \right)^2 
+ \lambda_1 \cdot l_{\text{parent}} + \lambda_2 \cdot l_{\text{child}}
\end{equation}

\subsection{Hyperparameter Tuning}

The hyperparameters $\lambda_1$ and $\lambda_2$ play a critical role in balancing the bidirectional information flow:
\begin{itemize}
    \item A higher $\lambda_1$ emphasizes top-down influence by prioritizing alignment with parent nodes.
    \item A higher $\lambda_2$ strengthens bottom-up feedback from child nodes.
\end{itemize}

Proper tuning of these hyperparameters is essential for optimizing the model’s performance while maintaining hierarchical consistency. To achieve this, Optuna\footnote{\url{https://optuna.org/}} was employed for hyperparameter tuning, utilizing its Tree-structured Parzen Estimator (TPE), a Bayesian optimization approach, to efficiently explore the hyperparameter space and ensure robust performance.

\subsection{Fixed Constraints During Training}

A key feature of the BiHRNN model is its use of \textit{fixed constraints} for the parent and child relationships throughout the training process. The procedure involves the following steps:
\begin{enumerate}
    \item \textbf{Pretraining the Base Model:} The HRNN (or a similar baseline model) is first trained independently for each category, and the learned weights are saved. These weights serve as the initial representations of the parent and child nodes.
    \item \textbf{Freezing the Weights:} Once the HRNN weights are trained, they are frozen and used as fixed constraints for the bidirectional model. This means that the weights representing parent and child relationships remain constant during the BiHRNN training process.
    \item \textbf{Stabilization and Regularization:} By leveraging frozen weights, the BiHRNN anchors predictions, ensuring that hierarchical relationships are preserved and reducing the risk of overfitting. This approach is particularly beneficial for datasets with limited samples or high variability, as it provides a stable foundation for the bidirectional model.
\end{enumerate}


\subsection{Summary}

The BiHRNN introduces a bidirectional approach to hierarchical modeling, leveraging dual constraints and a customized loss function to enable efficient information flow between nodes. By using fixed constraints and balancing top-down and bottom-up interactions, the model achieves superior forecasting accuracy and hierarchical coherence, establishing itself as a robust framework for hierarchical time series forecasting.

Figure ~\ref{fig:Bidirectional HRNN} below depicts the BiHRNN architechture.

\begin{figure}[H]
    \centering
    {   \makebox[\textwidth]{
        \includegraphics[scale=0.25]{figures/Slide1.jpeg}}
    }
        \caption
    {An illustration of our  BiHRNN Model.}
    \label{fig:Bidirectional HRNN}
\end{figure}



\subsection{Data \label{sec:data}}
Our analysis utilizes the BIOSCAN-5M dataset\footnote{The BIOSCAN-5M dataset contains 5.15\,M arthropod records, each with associated an image and DNA barcode sequence. Although the images are different for each record, the same barcode can occur across multiple records, hence there are fewer than 5\,M unique barcodes.
%However, since multiple images can map to the same DNA barcode sequence, the dataset contains approximately 2.4\,M unique DNA barcodes.
}, a comprehensive collection of 2.4\,M unique DNA barcodes organized into three distinct partitions: {\it (i) Pretrain}: Contains 2.28\,M unique DNA barcodes from unclassified specimens, used for self-supervised pretraining. {\it (ii) Seen}: Encompasses DNA barcodes with validated scientific species names, split into training (118\,k barcodes), validation (6.6\,k barcodes), and test (18.4\,k barcodes) subsets for closed-world evaluation tasks. {\it (iii) Unseen}: Contains novel species with reliable placeholder taxonomic labels, distributed across reference key (12.2\,k barcodes), validation (2.4\,k barcodes), and test (3.4\,k barcodes) subsets for open-world evaluation tasks.
For each sample in \textit{unseen}, its species does not appear in \textit{seen}, but its genus does appear.
%
% The dataset partitioning ensures species-level isolation between the {\it Seen} and the {\it Unseen} partitions, with the test sets also incorporating a flattened species distribution to mitigate taxonomic imbalance.
This structure enables the evaluation of both closed-world classification and open-world species identification capabilities.

\chapter{Evaluation and Results} 
%In the following section, we outline the evaluation process conducted for inflation prediction using Bidirectional HRNNs. 
We evaluate  the performance of the BiHRNN and compare it to well-established baselines for inflation prediction, as well as some additional machine learning approaches.
We use the following notation: Let $x_t$ be the CPI log-change rate at month $t$ .
Models for $\hat{x}_{t}$ are considered as an estimate for $x_t$ based on historical values.
Furthermore, we denote the estimation error at time $t$ by $\varepsilon_{t}$ .
%In all cases, the $h$-horizon forecasts were generated by recursively iterating the one-step forecasts forward. 
Hyper-parameters were set using Bayesian-inspired optimization procedures.

\section{Baseline Models} \label{baslines}
We compare Bidirectional HRNN with the following CPI prediction baselines:

\begin{enumerate}

\item{\bf Autoregression (AR) -} The AR($\rho$) model estimates $\hat{x}_{t}$ based on the previous $\rho$ timestamps using the following equation: $\hat{x}_{t}= \alpha_0 + \left(\sum_{i=1}^{\rho} \alpha_{i} x_{t-i} \right) + \varepsilon_{t}$, where $\{ \alpha_i \}_{i=0}^{\rho}$ represent the model's parameters.

\item{\bf Random Walk (RW) -} We consider the RW($\rho$) model from \cite{atkeson2002}. RW($\rho$) is a straightforward but powerful model that predicts the next timestamps by taking the average of the last $\rho$ timestamps, using the formula:  $\hat{y}_{t}=\frac{1}{\rho} \sum_{i=1}^{\rho} x_{t-i} + \varepsilon_{t}$.

\item{\bf Random Forests (RF) - } The RF($\rho$) model is an ensemble learning approach that constructs multiple decision trees~\citep{song2015decision} to reduce overfitting and enhance generalization~\citep{breiman2001random}. During prediction, the model returns the average of the predictions made by each individual tree. The inputs to the RF($\rho$) model are the last $\rho$ samples, and the output is the predicted value for the next timestamp.

\item{\bf Extreme Gradient Boost (XGBoost) - } The XGBoost($\rho$) model \citep{Chen_2016} is based on an ensemble of decision trees which are trained in a stage-wise fashion similar to other boosting models \citep{schapire1999brief}. Unlike RF($\rho$) which averages the prediction of multiple decision trees, the XGBoost($\rho$) trains each tree to minimize the remaining residual error of all previous trees. At prediction time, the sum of predictions of all the trees is returned.  The inputs to the XGBoost($\rho$) model are the last $\rho$ samples and the output is the predicted value for the next timestamps.

\item{\bf Fully Connected Neural Network (FC) -} The FC($\rho$) model is a fully connected neural network with one hidden layer of size 100 and a ReLU activation function~\citep{ActivationFunctions}. The output layer does not use any activation function to frame the task as a regression problem, optimized using a squared loss function. The inputs to the FC($\rho$) model consist of the last $\rho$ samples, and the output is the predicted value for the next timestamp.

\item{\bf Support Vector Regression (SVR) - } 
SVR($\rho$)  is a machine learning model based on Support Vector Machines (SVM), used for regression tasks ~\cite{NIPS1996_d3890178}.  SVR($\rho$) attempts to find a function that fits the data within a certain margin of tolerance, while minimizing the prediction error outside this margin. It is particularly effective for capturing complex relationships in data and is robust to outliers due to its focus on maximizing the margin around the prediction.  The kernel used for the prediction is "rbf" and the degree of the polynomial kernel function is three.
The inputs to the SVR($\rho$) model are the last $\rho$ samples and the output is the predicted value for the next timestamps.
\end{enumerate}


\section{Ablation Models}
To highlight the impact of the hierarchical component in the Bidirectional HRNN model, we performed an ablation study by comparing it to "simpler" alternatives, specifically GRU-based models that exclude the hierarchical component: 

\begin{enumerate}

\item{\bf Single GRU (S-GRU) -} The S-GRU($\rho$) is a single GRU unit that receives the last $\rho$ values as inputs in order to predict the next value. In GRU($\rho$), a single GRU is used for all the time series that comprise the CPI hierarchy. This baseline utilizes all the benefits of a GRU but assumes that the different components of the CPI behave similarly and a single unit is sufficient to model all the nodes.   %The model is given by the following formula: $\hat{y}_{t}=h_{t-1}+t \varepsilon_{t}$ where $h_{t-1}$ is the output value of a single scalar GRU layer with input shape of $\left[\begin{array}{ll}{d X} & {1}\end{array}\right]$  and $d$ is the time dimension. In other words the inputs are $x_{t-d}, x_{t-d+1} \ldots x_{t-1}$.

\item {\bf Independent GRUs (I-GRUs) -}
In I-GRUs($\rho$), we trained a different GRU($\rho$) unit for each CPI node. 
The S-GRU and I-GRU approaches represent two extremes: The first attempts to model all the CPI nodes with a single model, while the second treats each node separately. 

To emphasize the effect of bidirectionality, we also compared the Bidirectional HRNN to its predecessor, the Hierarchical Recurrent Neural Network (HRNN).

\item {\bf Hierarchical Recurrent Neural Network (HRNN) -}
In HRNN($\rho$), we trained a separate GRU($\rho$) unit for each CPI node, while incorporating the model weights of its parent category. This approach allows information to flow from parent to child categories, effectively leveraging the hierarchical structure of the data and enhancing prediction accuracy.
\end{enumerate}



\section{Evaluation Metrics}
Following \cite{faust2013forecasting} and \cite{AparicioBertolotto2020a}, we report our results using three evaluation metrics: 
\begin{enumerate}
    \item{\bf Root Mean Squared Error (RMSE) -} 
    The RMSE is calculated as: \begin{equation}
        RMSE=\sqrt{\frac{1}{T}\sum_{t=1}^T \left(x_t- \hat{x}_t \right)^2},
    \end{equation}
     where $x_t$ represents the actual monthly change rate for month $t$, and $\hat{x}_t$ denotes the corresponding predicted value.
    
    \item{\bf Pearson Correlation Coefficient -} The Pearson correlation coefficient $\phi$ is defined as:
    \begin{equation}
        \phi = \frac{COV(X_T,\hat{X}_T)}{\sigma_{X} \times \sigma_{\hat{X}}},    
    \end{equation}
        where $COV(X_T,\hat{X}_T)$ is the covariance between the actual values and predictions, and $\sigma_{X_T}$ and $\sigma_{\hat{X}_T}$ are the standard deviations of the actual values and the predictions, respectively.
    
\item{\bf Distance Correlation Coefficient -} 
Unlike the Pearson correlation, which only measures linear relationships, the distance correlation coefficient can detect both linear and nonlinear associations ~\citep{SzekelyRizzoBakirov2007a,distanceCorrelation}. 
%Distance correlation  is a measure of dependence between two paired random vectors of arbitrary, not necessarily equal dimension. 
%which can only detect linear association between two random variables and Spearman's rho or Kendall's tau which both measure monotonic relationship but cannot capture all types of non-linear dependency. 
The distance correlation coefficient $r_d$ is given by:
    \begin{equation}
    \label{eq:distance_corr}
        r_d= \frac{\operatorname{dCov}(X_T, \hat{X}_T)}{\sqrt{ \operatorname{dVar}(X_T) \times \operatorname{dVar}(\hat{X}_T)}}
    \end{equation}
where $\operatorname{dCov}(X_T, \hat{X}_T)$ is the distance covariance between the actual values and the predictions, and $\operatorname{dVar}(X_T)$ and $\operatorname{dVar}(\hat{X}_T)$ are the distance variances of the actual values and the predictions, respectively.

%\item{\bf Temporal Drift R\textsuperscript{2} -} 
%To establish a fairer baseline that avoids using information from the test data, we calculated the R\textsuperscript{2} score based on the mean and variance of the training data rather than the test data. This approach ensures that the model's performance is assessed without any advantage gained from knowing the test set distribution. Specifically, the total sum of squares (TSS) is computed using the training data as:

%\[
%TSS = \sum{(y_i - \bar{y}_{\text{train}})^2}
%\]

%where $\bar{y}_{\text{train}}$ is the mean of the target variable in the training data. 

%The residual sum of squares (RSS) is then calculated as:

%\[
%RSS = \sum{(y_i - \hat{y}_i)^2}
%\]

%where $\hat{y}_i$ represents the predicted values from the model. 

%Finally, the R\textsuperscript{2} is computed using the training data mean:
%\[
%R^2 = 1 - \frac{RSS}{TSS}
%\]
%This version of R\textsuperscript{2} reflects how well the model performs relative to the variance in the training data, ensuring a consistent and unbiased baseline that does not rely on knowledge of the test set's distribution. This method is more representative of real-world conditions, where the test data distribution is typically unknown, making it a more accurate measure of how the model will perform once it is deployed.

\end{enumerate}


\section{Results} 
\label{sec:Results}
The BiHRNN model stands out for its ability to leverage information flow both from higher levels to lower levels and from lower levels to higher levels within the hierarchy. The model leverages the inherent hierarchy of the CPI, enhancing predictions at both granular and broader, more significant levels, such as the CPI Headline.
Therefore, we will provide the headline results separately, along with the aggregated results across all categories.

The results are relative to the $AR(1)$ model and normalized according to: $\frac{RMSE_{Model}}{RMSE_{AR\left( 1\right) }}$.

\subsection{US CPI Results}
\setlength{\tabcolsep}{3pt}
\begin{table}[H]
\begin{threeparttable} 
\caption{Average Results on Disaggregated CPI Components - US} 
\label{tab:allCPIResults - US}
{\scriptsize  % Reduced font size to fit headers within the table width
\begin{tabularx}{\textwidth}{l>{\centering\arraybackslash}X>{\centering\arraybackslash}X>{\centering\arraybackslash}X>{\centering\arraybackslash}X>{\centering\arraybackslash}X>{\centering\arraybackslash}X>{\centering\arraybackslash}X>{\centering\arraybackslash}X}
\toprule[1.1pt]
\textbf{Model} & \textbf{\parbox[c]{1cm}{\centering Avg. \\ RMSE}}  & \textbf{\parbox[c]{1.2cm}{\centering Pearson \\ Corr.}} & \textbf{\parbox[c]{1.2cm}{\centering Dist. \\ Corr.}} & \textbf{\parbox[c]{1.2cm}{\centering Headline \\ RMSE}} & \textbf{\parbox[c]{1.4cm}{\centering Headline Pearson \\ Corr.}} & \textbf{\parbox[c]{1.4cm}{\centering Headline Dist. \\ Corr.}} \\ 
\midrule
I-GRU & 1.215 & 0.138 & 0.338 & 1.015 & 0.347 & 0.350 \\
AR\_1 & 1.000 & 0.176 & 0.513 & 1.000 & 0.327 & 0.459 \\
AR\_2 & 1.267 & 0.105 & 0.467 & 1.312 & 0.327 & 0.565 \\
AR\_3 & 1.487 & 0.082 & 0.437 & 1.560 & 0.349 & 0.510 \\
AR\_4 & 1.902 & 0.052 & 0.411 & 1.749 & 0.308 & 0.427 \\
FC\_p\_12 & 1.229 & -0.014 & 0.355 & 1.592 & 0.027 & 0.251 \\
RF\_p\_12 & 1.143 & 0.112 & 0.377 & 1.210 & 0.368 & 0.377 \\
RW\_p\_4 & 1.189 & -0.013 & 0.353 & 1.420 & -0.050 & 0.310 \\
SVR\_p\_12 & 1.115 & 0.067 & 0.363 & 1.280 & 0.473 & 0.529 \\
XGB\_p\_12 & 1.228 & 0.087 & 0.369 & 1.312 & 0.392 & 0.423 \\
HRNN & 1.028 & 0.158 & 0.346 & 1.015 & 0.347 & 0.350 \\
BiHRNN & 0.966 & 0.230 & 0.378 & 1.052 & 0.225 & 0.290 \\
\bottomrule[1.1pt]
\end{tabularx}
}
\end{threeparttable}
\end{table}


\subsection{Canada CPI Results}
\setlength{\tabcolsep}{3pt}
\begin{table}[H]
\begin{threeparttable} 
\caption{Average Results on Disaggregated CPI Components - Canada} 
\label{tab:allCPIResults - Canada}
{\scriptsize  % Reduced font size to fit headers within the table width
\begin{tabularx}{\textwidth}{l>{\centering\arraybackslash}X>{\centering\arraybackslash}X>{\centering\arraybackslash}X>{\centering\arraybackslash}X>{\centering\arraybackslash}X>{\centering\arraybackslash}X>{\centering\arraybackslash}X>{\centering\arraybackslash}X}
\toprule[1.1pt]
\textbf{Model} & \textbf{\parbox[c]{1cm}{\centering Avg. \\ RMSE}} & \textbf{\parbox[c]{1.2cm}{\centering Pearson \\ Corr.}} & \textbf{\parbox[c]{1.2cm}{\centering Dist. \\ Corr.}} & \textbf{\parbox[c]{1.2cm}{\centering Headline \\ RMSE}} & \textbf{\parbox[c]{1.4cm}{\centering Headline Pearson \\ Corr.}} & \textbf{\parbox[c]{1.4cm}{\centering Headline Dist. \\ Corr.}} \\ 
\midrule
I-GRU & 0.892 & 0.351 & 0.476 & 1.261 & 0.329 & 0.516 \\
AR\_1 & 1.000 & 0.128 & 0.633 & 1.000 & 0.628 & 0.693 \\
AR\_2 & 1.188 & -0.008 & 0.592 & 1.155 & 0.241 & 0.409 \\
AR\_3 & 1.305 & -0.004 & 0.561 & 0.970 & 0.646 & 0.671 \\
AR\_4 & 1.684 & -0.004 & 0.522 & 1.424 & 0.356 & 0.416 \\
FC\_p\_12 & 0.907 & 0.237 & 0.465 & 2.051 & 0.152 & 0.352 \\
RF\_p\_12 & 0.861 & 0.318 & 0.509 & 1.635 & 0.542 & 0.540 \\
RW\_p\_4 & 0.901 & 0.138 & 0.414 & 1.261 & 0.495 & 0.658 \\
SVR\_p\_12 & 0.852 & 0.308 & 0.517 & 1.721 & 0.449 & 0.580 \\
XGB\_p\_12 & 0.936 & 0.271 & 0.502 & 1.635 & 0.411 & 0.527 \\
HRNN & 0.824 & 0.358 & 0.490 & 1.261 & 0.329 & 0.516 \\
BiHRNN & 0.795 & 0.386 & 0.511 & 1.170 & 0.321 & 0.497 \\
\bottomrule[1.1pt]
\end{tabularx}
}
\end{threeparttable}
\end{table}


\subsection{Norway CPI Results}
\setlength{\tabcolsep}{3pt}
\begin{table}[H]
\begin{threeparttable} 
\caption{Average Results on Disaggregated CPI Components - Norway} 
\label{tab:allCPIResults - Norway}
{\scriptsize  % Reduced font size to fit headers within the table width
\begin{tabularx}{\textwidth}{l>{\centering\arraybackslash}X>{\centering\arraybackslash}X>{\centering\arraybackslash}X>{\centering\arraybackslash}X>{\centering\arraybackslash}X>{\centering\arraybackslash}X>{\centering\arraybackslash}X>{\centering\arraybackslash}X}
\toprule[1.1pt]
\textbf{Model} & \textbf{\parbox[c]{1cm}{\centering Avg. \\ RMSE}} & \textbf{\parbox[c]{1.2cm}{\centering Pearson \\ Corr.}} & \textbf{\parbox[c]{1.2cm}{\centering Dist. \\ Corr.}} & \textbf{\parbox[c]{1.2cm}{\centering Headline \\ RMSE}} & \textbf{\parbox[c]{1.4cm}{\centering Headline Pearson \\ Corr.}} & \textbf{\parbox[c]{1.4cm}{\centering Headline Dist. \\ Corr.}} \\ 
\midrule
I-GRU & 0.866 & 0.355 & 0.5063 & 0.724 & 0.208 & 0.413 \\
AR\_1 & 1.000 & 0.053 & 0.653 & 1.000 & -0.583 & 0.546 \\
AR\_2 & 1.251 & -0.003 & 0.617 & 1.132 & -0.743 & 0.695 \\
AR\_3 & 1.378 & 0.009 & 0.568 & 1.189 & -0.797 & 0.777 \\
AR\_4 & 1.727 &  0.011 & 0.535 & 1.296 & -0.536 & 0.574 \\
FC\_p\_12 & 0.974 & 0.226 & 0.454 & 0.924 & 0.242 & 0.422 \\
RF\_p\_12 & 0.849 & 0.349 & 0.539 & 0.672 & 0.613 & 0.721 \\
RW\_p\_4 & 0.973 & 0.187 & 0.405 & 0.788 & 0.165 & 0.337 \\
SVR\_p\_12 & 0.851 & 0.376 & 0.550 & 0.848 & 0.251 & 0.362 \\
XGB\_p\_12 & 0.904 & 0.303 & 0.530 & 0.669 & 0.644 & 0.725 \\
HRNN & 0.832 & 0.3677 & 0.5365 & 0.724 & 0.208 & 0.413 \\
BiHRNN & 0.767 & 0.478 & 0.567 & 0.655 & 0.390 & 0.477 \\
\bottomrule[1.1pt]
\end{tabularx}
}
\end{threeparttable}
\end{table}


The results in Tables ~\ref{tab:allCPIResults - Canada}, ~\ref{tab:allCPIResults - Norway}, and ~\ref{tab:allCPIResults - US} show that the BiHRNN consistently outperforms other models in terms of predictive accuracy and stability. With some of the lowest RMSE values across datasets, this model demonstrates its ability to reliably minimize error between various components of the CPI. In comparison, simpler models like AR(1) and AR(4) often exhibit higher RMSE and greater variability, indicating that the BiHRNN model offers a stronger, more stable fit for this complex data.

%The Temporal Drift R\textsuperscript{2}, further emphasizes the Bidirectional HRNN’s advantage. This model attains positive values, indicating a closer alignment with observed data over time. By contrast, many benchmark models (e.g., AR(3), FC(12)) report negative R\textsuperscript{2} values, signaling poor temporal alignment and a failure to capture time-based patterns effectively. 

Correlation metrics reinforce this model’s capacity to understand the underlying relationships in the data. The BiHRNN achieves high Pearson and Distance correlations indicating a strong alignment between model predictions and actual outcomes. Although a few models, like RF(12), show competitive correlations, their higher RMSE values demonstrate an inability to consistently maintain accuracy across metrics.

The BiHRNN model demonstrates top-tier performance in headline predictions, excelling in both RMSE and correlation metrics. However, our findings indicate that the Headline data alone is sufficient for accurate headline predictions and yields the best results. Attempts to incorporate additional regularization terms do not enhance prediction performance. Consequently, we recommend that future work on headline predictions focus exclusively on using the Headline data.

This balance across both disaggregated components and headline metrics highlights the model's robustness and adaptability, making it a preferable choice for forecasting CPI trends. Overall, the BiHRNN stands out as the most effective model, combining low error rates, strong fit, and high correlation, all of which contribute to a more accurate and reliable CPI prediction framework across categories.

Figure ~\ref{fig:disaggregated_index_predictions} below showcases examples of several disaggregated indexes from different hierarchy levels and sectors. The solid black line shows the actual CPI values, while the dashed lines depict predictions from the top-performing models—all variations of RNN models: BiHRNN, HRNN, and I-GRU in blue, green, and red, respectively. As shown in the graphs, the BiHRNN model demonstrates superior predictive accuracy, achieving lower RMSEs and more effectively capturing shifts in trends compared to its counterparts.

\begin{figure}[H]
    \centering
    
    \subfloat[Food]{
        \includegraphics[width=0.6\textwidth]{figures/Food_Canada.png}
        \label{fig:Food - Canada}
    }
    \hfill
    \subfloat[Housekeeping]{
        \includegraphics[width=0.6\textwidth]{figures/Housekeeping_Canada.png}
        \label{fig:Housekeeping - Canada}
    }
    
%    \vspace{0.5cm}  % Space between the two rows of images
    
    \subfloat[Footwear]{
        \includegraphics[width=0.6\textwidth]{figures/Footwear_Norway.png}
        \label{fig:Footwear Norway}
    }
    \hfill
    \subfloat[Out-patient Services]{
        \includegraphics[width=0.6\textwidth]{figures/Out-patient_services_Norway.png}
        \label{fig:Out-patient services - Norway}
    }
    
    \caption{Disaggregated Index CPI Predictions}
    \label{fig:disaggregated_index_predictions}
\end{figure}


We present RiskHarvester, a risk-based tool to compute a security risk score based on the value of the asset and ease of attack on a database. We calculated the value of asset by identifying the sensitive data categories present in a database from the database keywords. We utilized data flow analysis, SQL, and Object Relational Mapper (ORM) parsing to identify the database keywords. To calculate the ease of attack, we utilized passive network analysis to retrieve the database host information. To evaluate RiskHarvester, we curated RiskBench, a benchmark of 1,791 database secret-asset pairs with sensitive data categories and host information manually retrieved from 188 GitHub repositories. RiskHarvester demonstrates precision of (95\%) and recall (90\%) in detecting database keywords for the value of asset and precision of (96\%) and recall (94\%) in detecting valid hosts for ease of attack. Finally, we conducted an online survey to understand whether developers prioritize secret removal based on security risk score. We found that 86\% of the developers prioritized the secrets for removal with descending security risk scores.

\bibliographystyle{plainnat}
\bibliography{references}


%\appendix
%\subsection{Lloyd-Max Algorithm}
\label{subsec:Lloyd-Max}
For a given quantization bitwidth $B$ and an operand $\bm{X}$, the Lloyd-Max algorithm finds $2^B$ quantization levels $\{\hat{x}_i\}_{i=1}^{2^B}$ such that quantizing $\bm{X}$ by rounding each scalar in $\bm{X}$ to the nearest quantization level minimizes the quantization MSE. 

The algorithm starts with an initial guess of quantization levels and then iteratively computes quantization thresholds $\{\tau_i\}_{i=1}^{2^B-1}$ and updates quantization levels $\{\hat{x}_i\}_{i=1}^{2^B}$. Specifically, at iteration $n$, thresholds are set to the midpoints of the previous iteration's levels:
\begin{align*}
    \tau_i^{(n)}=\frac{\hat{x}_i^{(n-1)}+\hat{x}_{i+1}^{(n-1)}}2 \text{ for } i=1\ldots 2^B-1
\end{align*}
Subsequently, the quantization levels are re-computed as conditional means of the data regions defined by the new thresholds:
\begin{align*}
    \hat{x}_i^{(n)}=\mathbb{E}\left[ \bm{X} \big| \bm{X}\in [\tau_{i-1}^{(n)},\tau_i^{(n)}] \right] \text{ for } i=1\ldots 2^B
\end{align*}
where to satisfy boundary conditions we have $\tau_0=-\infty$ and $\tau_{2^B}=\infty$. The algorithm iterates the above steps until convergence.

Figure \ref{fig:lm_quant} compares the quantization levels of a $7$-bit floating point (E3M3) quantizer (left) to a $7$-bit Lloyd-Max quantizer (right) when quantizing a layer of weights from the GPT3-126M model at a per-tensor granularity. As shown, the Lloyd-Max quantizer achieves substantially lower quantization MSE. Further, Table \ref{tab:FP7_vs_LM7} shows the superior perplexity achieved by Lloyd-Max quantizers for bitwidths of $7$, $6$ and $5$. The difference between the quantizers is clear at 5 bits, where per-tensor FP quantization incurs a drastic and unacceptable increase in perplexity, while Lloyd-Max quantization incurs a much smaller increase. Nevertheless, we note that even the optimal Lloyd-Max quantizer incurs a notable ($\sim 1.5$) increase in perplexity due to the coarse granularity of quantization. 

\begin{figure}[h]
  \centering
  \includegraphics[width=0.7\linewidth]{sections/figures/LM7_FP7.pdf}
  \caption{\small Quantization levels and the corresponding quantization MSE of Floating Point (left) vs Lloyd-Max (right) Quantizers for a layer of weights in the GPT3-126M model.}
  \label{fig:lm_quant}
\end{figure}

\begin{table}[h]\scriptsize
\begin{center}
\caption{\label{tab:FP7_vs_LM7} \small Comparing perplexity (lower is better) achieved by floating point quantizers and Lloyd-Max quantizers on a GPT3-126M model for the Wikitext-103 dataset.}
\begin{tabular}{c|cc|c}
\hline
 \multirow{2}{*}{\textbf{Bitwidth}} & \multicolumn{2}{|c|}{\textbf{Floating-Point Quantizer}} & \textbf{Lloyd-Max Quantizer} \\
 & Best Format & Wikitext-103 Perplexity & Wikitext-103 Perplexity \\
\hline
7 & E3M3 & 18.32 & 18.27 \\
6 & E3M2 & 19.07 & 18.51 \\
5 & E4M0 & 43.89 & 19.71 \\
\hline
\end{tabular}
\end{center}
\end{table}

\subsection{Proof of Local Optimality of LO-BCQ}
\label{subsec:lobcq_opt_proof}
For a given block $\bm{b}_j$, the quantization MSE during LO-BCQ can be empirically evaluated as $\frac{1}{L_b}\lVert \bm{b}_j- \bm{\hat{b}}_j\rVert^2_2$ where $\bm{\hat{b}}_j$ is computed from equation (\ref{eq:clustered_quantization_definition}) as $C_{f(\bm{b}_j)}(\bm{b}_j)$. Further, for a given block cluster $\mathcal{B}_i$, we compute the quantization MSE as $\frac{1}{|\mathcal{B}_{i}|}\sum_{\bm{b} \in \mathcal{B}_{i}} \frac{1}{L_b}\lVert \bm{b}- C_i^{(n)}(\bm{b})\rVert^2_2$. Therefore, at the end of iteration $n$, we evaluate the overall quantization MSE $J^{(n)}$ for a given operand $\bm{X}$ composed of $N_c$ block clusters as:
\begin{align*}
    \label{eq:mse_iter_n}
    J^{(n)} = \frac{1}{N_c} \sum_{i=1}^{N_c} \frac{1}{|\mathcal{B}_{i}^{(n)}|}\sum_{\bm{v} \in \mathcal{B}_{i}^{(n)}} \frac{1}{L_b}\lVert \bm{b}- B_i^{(n)}(\bm{b})\rVert^2_2
\end{align*}

At the end of iteration $n$, the codebooks are updated from $\mathcal{C}^{(n-1)}$ to $\mathcal{C}^{(n)}$. However, the mapping of a given vector $\bm{b}_j$ to quantizers $\mathcal{C}^{(n)}$ remains as  $f^{(n)}(\bm{b}_j)$. At the next iteration, during the vector clustering step, $f^{(n+1)}(\bm{b}_j)$ finds new mapping of $\bm{b}_j$ to updated codebooks $\mathcal{C}^{(n)}$ such that the quantization MSE over the candidate codebooks is minimized. Therefore, we obtain the following result for $\bm{b}_j$:
\begin{align*}
\frac{1}{L_b}\lVert \bm{b}_j - C_{f^{(n+1)}(\bm{b}_j)}^{(n)}(\bm{b}_j)\rVert^2_2 \le \frac{1}{L_b}\lVert \bm{b}_j - C_{f^{(n)}(\bm{b}_j)}^{(n)}(\bm{b}_j)\rVert^2_2
\end{align*}

That is, quantizing $\bm{b}_j$ at the end of the block clustering step of iteration $n+1$ results in lower quantization MSE compared to quantizing at the end of iteration $n$. Since this is true for all $\bm{b} \in \bm{X}$, we assert the following:
\begin{equation}
\begin{split}
\label{eq:mse_ineq_1}
    \tilde{J}^{(n+1)} &= \frac{1}{N_c} \sum_{i=1}^{N_c} \frac{1}{|\mathcal{B}_{i}^{(n+1)}|}\sum_{\bm{b} \in \mathcal{B}_{i}^{(n+1)}} \frac{1}{L_b}\lVert \bm{b} - C_i^{(n)}(b)\rVert^2_2 \le J^{(n)}
\end{split}
\end{equation}
where $\tilde{J}^{(n+1)}$ is the the quantization MSE after the vector clustering step at iteration $n+1$.

Next, during the codebook update step (\ref{eq:quantizers_update}) at iteration $n+1$, the per-cluster codebooks $\mathcal{C}^{(n)}$ are updated to $\mathcal{C}^{(n+1)}$ by invoking the Lloyd-Max algorithm \citep{Lloyd}. We know that for any given value distribution, the Lloyd-Max algorithm minimizes the quantization MSE. Therefore, for a given vector cluster $\mathcal{B}_i$ we obtain the following result:

\begin{equation}
    \frac{1}{|\mathcal{B}_{i}^{(n+1)}|}\sum_{\bm{b} \in \mathcal{B}_{i}^{(n+1)}} \frac{1}{L_b}\lVert \bm{b}- C_i^{(n+1)}(\bm{b})\rVert^2_2 \le \frac{1}{|\mathcal{B}_{i}^{(n+1)}|}\sum_{\bm{b} \in \mathcal{B}_{i}^{(n+1)}} \frac{1}{L_b}\lVert \bm{b}- C_i^{(n)}(\bm{b})\rVert^2_2
\end{equation}

The above equation states that quantizing the given block cluster $\mathcal{B}_i$ after updating the associated codebook from $C_i^{(n)}$ to $C_i^{(n+1)}$ results in lower quantization MSE. Since this is true for all the block clusters, we derive the following result: 
\begin{equation}
\begin{split}
\label{eq:mse_ineq_2}
     J^{(n+1)} &= \frac{1}{N_c} \sum_{i=1}^{N_c} \frac{1}{|\mathcal{B}_{i}^{(n+1)}|}\sum_{\bm{b} \in \mathcal{B}_{i}^{(n+1)}} \frac{1}{L_b}\lVert \bm{b}- C_i^{(n+1)}(\bm{b})\rVert^2_2  \le \tilde{J}^{(n+1)}   
\end{split}
\end{equation}

Following (\ref{eq:mse_ineq_1}) and (\ref{eq:mse_ineq_2}), we find that the quantization MSE is non-increasing for each iteration, that is, $J^{(1)} \ge J^{(2)} \ge J^{(3)} \ge \ldots \ge J^{(M)}$ where $M$ is the maximum number of iterations. 
%Therefore, we can say that if the algorithm converges, then it must be that it has converged to a local minimum. 
\hfill $\blacksquare$


\begin{figure}
    \begin{center}
    \includegraphics[width=0.5\textwidth]{sections//figures/mse_vs_iter.pdf}
    \end{center}
    \caption{\small NMSE vs iterations during LO-BCQ compared to other block quantization proposals}
    \label{fig:nmse_vs_iter}
\end{figure}

Figure \ref{fig:nmse_vs_iter} shows the empirical convergence of LO-BCQ across several block lengths and number of codebooks. Also, the MSE achieved by LO-BCQ is compared to baselines such as MXFP and VSQ. As shown, LO-BCQ converges to a lower MSE than the baselines. Further, we achieve better convergence for larger number of codebooks ($N_c$) and for a smaller block length ($L_b$), both of which increase the bitwidth of BCQ (see Eq \ref{eq:bitwidth_bcq}).


\subsection{Additional Accuracy Results}
%Table \ref{tab:lobcq_config} lists the various LOBCQ configurations and their corresponding bitwidths.
\begin{table}
\setlength{\tabcolsep}{4.75pt}
\begin{center}
\caption{\label{tab:lobcq_config} Various LO-BCQ configurations and their bitwidths.}
\begin{tabular}{|c||c|c|c|c||c|c||c|} 
\hline
 & \multicolumn{4}{|c||}{$L_b=8$} & \multicolumn{2}{|c||}{$L_b=4$} & $L_b=2$ \\
 \hline
 \backslashbox{$L_A$\kern-1em}{\kern-1em$N_c$} & 2 & 4 & 8 & 16 & 2 & 4 & 2 \\
 \hline
 64 & 4.25 & 4.375 & 4.5 & 4.625 & 4.375 & 4.625 & 4.625\\
 \hline
 32 & 4.375 & 4.5 & 4.625& 4.75 & 4.5 & 4.75 & 4.75 \\
 \hline
 16 & 4.625 & 4.75& 4.875 & 5 & 4.75 & 5 & 5 \\
 \hline
\end{tabular}
\end{center}
\end{table}

%\subsection{Perplexity achieved by various LO-BCQ configurations on Wikitext-103 dataset}

\begin{table} \centering
\begin{tabular}{|c||c|c|c|c||c|c||c|} 
\hline
 $L_b \rightarrow$& \multicolumn{4}{c||}{8} & \multicolumn{2}{c||}{4} & 2\\
 \hline
 \backslashbox{$L_A$\kern-1em}{\kern-1em$N_c$} & 2 & 4 & 8 & 16 & 2 & 4 & 2  \\
 %$N_c \rightarrow$ & 2 & 4 & 8 & 16 & 2 & 4 & 2 \\
 \hline
 \hline
 \multicolumn{8}{c}{GPT3-1.3B (FP32 PPL = 9.98)} \\ 
 \hline
 \hline
 64 & 10.40 & 10.23 & 10.17 & 10.15 &  10.28 & 10.18 & 10.19 \\
 \hline
 32 & 10.25 & 10.20 & 10.15 & 10.12 &  10.23 & 10.17 & 10.17 \\
 \hline
 16 & 10.22 & 10.16 & 10.10 & 10.09 &  10.21 & 10.14 & 10.16 \\
 \hline
  \hline
 \multicolumn{8}{c}{GPT3-8B (FP32 PPL = 7.38)} \\ 
 \hline
 \hline
 64 & 7.61 & 7.52 & 7.48 &  7.47 &  7.55 &  7.49 & 7.50 \\
 \hline
 32 & 7.52 & 7.50 & 7.46 &  7.45 &  7.52 &  7.48 & 7.48  \\
 \hline
 16 & 7.51 & 7.48 & 7.44 &  7.44 &  7.51 &  7.49 & 7.47  \\
 \hline
\end{tabular}
\caption{\label{tab:ppl_gpt3_abalation} Wikitext-103 perplexity across GPT3-1.3B and 8B models.}
\end{table}

\begin{table} \centering
\begin{tabular}{|c||c|c|c|c||} 
\hline
 $L_b \rightarrow$& \multicolumn{4}{c||}{8}\\
 \hline
 \backslashbox{$L_A$\kern-1em}{\kern-1em$N_c$} & 2 & 4 & 8 & 16 \\
 %$N_c \rightarrow$ & 2 & 4 & 8 & 16 & 2 & 4 & 2 \\
 \hline
 \hline
 \multicolumn{5}{|c|}{Llama2-7B (FP32 PPL = 5.06)} \\ 
 \hline
 \hline
 64 & 5.31 & 5.26 & 5.19 & 5.18  \\
 \hline
 32 & 5.23 & 5.25 & 5.18 & 5.15  \\
 \hline
 16 & 5.23 & 5.19 & 5.16 & 5.14  \\
 \hline
 \multicolumn{5}{|c|}{Nemotron4-15B (FP32 PPL = 5.87)} \\ 
 \hline
 \hline
 64  & 6.3 & 6.20 & 6.13 & 6.08  \\
 \hline
 32  & 6.24 & 6.12 & 6.07 & 6.03  \\
 \hline
 16  & 6.12 & 6.14 & 6.04 & 6.02  \\
 \hline
 \multicolumn{5}{|c|}{Nemotron4-340B (FP32 PPL = 3.48)} \\ 
 \hline
 \hline
 64 & 3.67 & 3.62 & 3.60 & 3.59 \\
 \hline
 32 & 3.63 & 3.61 & 3.59 & 3.56 \\
 \hline
 16 & 3.61 & 3.58 & 3.57 & 3.55 \\
 \hline
\end{tabular}
\caption{\label{tab:ppl_llama7B_nemo15B} Wikitext-103 perplexity compared to FP32 baseline in Llama2-7B and Nemotron4-15B, 340B models}
\end{table}

%\subsection{Perplexity achieved by various LO-BCQ configurations on MMLU dataset}


\begin{table} \centering
\begin{tabular}{|c||c|c|c|c||c|c|c|c|} 
\hline
 $L_b \rightarrow$& \multicolumn{4}{c||}{8} & \multicolumn{4}{c||}{8}\\
 \hline
 \backslashbox{$L_A$\kern-1em}{\kern-1em$N_c$} & 2 & 4 & 8 & 16 & 2 & 4 & 8 & 16  \\
 %$N_c \rightarrow$ & 2 & 4 & 8 & 16 & 2 & 4 & 2 \\
 \hline
 \hline
 \multicolumn{5}{|c|}{Llama2-7B (FP32 Accuracy = 45.8\%)} & \multicolumn{4}{|c|}{Llama2-70B (FP32 Accuracy = 69.12\%)} \\ 
 \hline
 \hline
 64 & 43.9 & 43.4 & 43.9 & 44.9 & 68.07 & 68.27 & 68.17 & 68.75 \\
 \hline
 32 & 44.5 & 43.8 & 44.9 & 44.5 & 68.37 & 68.51 & 68.35 & 68.27  \\
 \hline
 16 & 43.9 & 42.7 & 44.9 & 45 & 68.12 & 68.77 & 68.31 & 68.59  \\
 \hline
 \hline
 \multicolumn{5}{|c|}{GPT3-22B (FP32 Accuracy = 38.75\%)} & \multicolumn{4}{|c|}{Nemotron4-15B (FP32 Accuracy = 64.3\%)} \\ 
 \hline
 \hline
 64 & 36.71 & 38.85 & 38.13 & 38.92 & 63.17 & 62.36 & 63.72 & 64.09 \\
 \hline
 32 & 37.95 & 38.69 & 39.45 & 38.34 & 64.05 & 62.30 & 63.8 & 64.33  \\
 \hline
 16 & 38.88 & 38.80 & 38.31 & 38.92 & 63.22 & 63.51 & 63.93 & 64.43  \\
 \hline
\end{tabular}
\caption{\label{tab:mmlu_abalation} Accuracy on MMLU dataset across GPT3-22B, Llama2-7B, 70B and Nemotron4-15B models.}
\end{table}


%\subsection{Perplexity achieved by various LO-BCQ configurations on LM evaluation harness}

\begin{table} \centering
\begin{tabular}{|c||c|c|c|c||c|c|c|c|} 
\hline
 $L_b \rightarrow$& \multicolumn{4}{c||}{8} & \multicolumn{4}{c||}{8}\\
 \hline
 \backslashbox{$L_A$\kern-1em}{\kern-1em$N_c$} & 2 & 4 & 8 & 16 & 2 & 4 & 8 & 16  \\
 %$N_c \rightarrow$ & 2 & 4 & 8 & 16 & 2 & 4 & 2 \\
 \hline
 \hline
 \multicolumn{5}{|c|}{Race (FP32 Accuracy = 37.51\%)} & \multicolumn{4}{|c|}{Boolq (FP32 Accuracy = 64.62\%)} \\ 
 \hline
 \hline
 64 & 36.94 & 37.13 & 36.27 & 37.13 & 63.73 & 62.26 & 63.49 & 63.36 \\
 \hline
 32 & 37.03 & 36.36 & 36.08 & 37.03 & 62.54 & 63.51 & 63.49 & 63.55  \\
 \hline
 16 & 37.03 & 37.03 & 36.46 & 37.03 & 61.1 & 63.79 & 63.58 & 63.33  \\
 \hline
 \hline
 \multicolumn{5}{|c|}{Winogrande (FP32 Accuracy = 58.01\%)} & \multicolumn{4}{|c|}{Piqa (FP32 Accuracy = 74.21\%)} \\ 
 \hline
 \hline
 64 & 58.17 & 57.22 & 57.85 & 58.33 & 73.01 & 73.07 & 73.07 & 72.80 \\
 \hline
 32 & 59.12 & 58.09 & 57.85 & 58.41 & 73.01 & 73.94 & 72.74 & 73.18  \\
 \hline
 16 & 57.93 & 58.88 & 57.93 & 58.56 & 73.94 & 72.80 & 73.01 & 73.94  \\
 \hline
\end{tabular}
\caption{\label{tab:mmlu_abalation} Accuracy on LM evaluation harness tasks on GPT3-1.3B model.}
\end{table}

\begin{table} \centering
\begin{tabular}{|c||c|c|c|c||c|c|c|c|} 
\hline
 $L_b \rightarrow$& \multicolumn{4}{c||}{8} & \multicolumn{4}{c||}{8}\\
 \hline
 \backslashbox{$L_A$\kern-1em}{\kern-1em$N_c$} & 2 & 4 & 8 & 16 & 2 & 4 & 8 & 16  \\
 %$N_c \rightarrow$ & 2 & 4 & 8 & 16 & 2 & 4 & 2 \\
 \hline
 \hline
 \multicolumn{5}{|c|}{Race (FP32 Accuracy = 41.34\%)} & \multicolumn{4}{|c|}{Boolq (FP32 Accuracy = 68.32\%)} \\ 
 \hline
 \hline
 64 & 40.48 & 40.10 & 39.43 & 39.90 & 69.20 & 68.41 & 69.45 & 68.56 \\
 \hline
 32 & 39.52 & 39.52 & 40.77 & 39.62 & 68.32 & 67.43 & 68.17 & 69.30  \\
 \hline
 16 & 39.81 & 39.71 & 39.90 & 40.38 & 68.10 & 66.33 & 69.51 & 69.42  \\
 \hline
 \hline
 \multicolumn{5}{|c|}{Winogrande (FP32 Accuracy = 67.88\%)} & \multicolumn{4}{|c|}{Piqa (FP32 Accuracy = 78.78\%)} \\ 
 \hline
 \hline
 64 & 66.85 & 66.61 & 67.72 & 67.88 & 77.31 & 77.42 & 77.75 & 77.64 \\
 \hline
 32 & 67.25 & 67.72 & 67.72 & 67.00 & 77.31 & 77.04 & 77.80 & 77.37  \\
 \hline
 16 & 68.11 & 68.90 & 67.88 & 67.48 & 77.37 & 78.13 & 78.13 & 77.69  \\
 \hline
\end{tabular}
\caption{\label{tab:mmlu_abalation} Accuracy on LM evaluation harness tasks on GPT3-8B model.}
\end{table}

\begin{table} \centering
\begin{tabular}{|c||c|c|c|c||c|c|c|c|} 
\hline
 $L_b \rightarrow$& \multicolumn{4}{c||}{8} & \multicolumn{4}{c||}{8}\\
 \hline
 \backslashbox{$L_A$\kern-1em}{\kern-1em$N_c$} & 2 & 4 & 8 & 16 & 2 & 4 & 8 & 16  \\
 %$N_c \rightarrow$ & 2 & 4 & 8 & 16 & 2 & 4 & 2 \\
 \hline
 \hline
 \multicolumn{5}{|c|}{Race (FP32 Accuracy = 40.67\%)} & \multicolumn{4}{|c|}{Boolq (FP32 Accuracy = 76.54\%)} \\ 
 \hline
 \hline
 64 & 40.48 & 40.10 & 39.43 & 39.90 & 75.41 & 75.11 & 77.09 & 75.66 \\
 \hline
 32 & 39.52 & 39.52 & 40.77 & 39.62 & 76.02 & 76.02 & 75.96 & 75.35  \\
 \hline
 16 & 39.81 & 39.71 & 39.90 & 40.38 & 75.05 & 73.82 & 75.72 & 76.09  \\
 \hline
 \hline
 \multicolumn{5}{|c|}{Winogrande (FP32 Accuracy = 70.64\%)} & \multicolumn{4}{|c|}{Piqa (FP32 Accuracy = 79.16\%)} \\ 
 \hline
 \hline
 64 & 69.14 & 70.17 & 70.17 & 70.56 & 78.24 & 79.00 & 78.62 & 78.73 \\
 \hline
 32 & 70.96 & 69.69 & 71.27 & 69.30 & 78.56 & 79.49 & 79.16 & 78.89  \\
 \hline
 16 & 71.03 & 69.53 & 69.69 & 70.40 & 78.13 & 79.16 & 79.00 & 79.00  \\
 \hline
\end{tabular}
\caption{\label{tab:mmlu_abalation} Accuracy on LM evaluation harness tasks on GPT3-22B model.}
\end{table}

\begin{table} \centering
\begin{tabular}{|c||c|c|c|c||c|c|c|c|} 
\hline
 $L_b \rightarrow$& \multicolumn{4}{c||}{8} & \multicolumn{4}{c||}{8}\\
 \hline
 \backslashbox{$L_A$\kern-1em}{\kern-1em$N_c$} & 2 & 4 & 8 & 16 & 2 & 4 & 8 & 16  \\
 %$N_c \rightarrow$ & 2 & 4 & 8 & 16 & 2 & 4 & 2 \\
 \hline
 \hline
 \multicolumn{5}{|c|}{Race (FP32 Accuracy = 44.4\%)} & \multicolumn{4}{|c|}{Boolq (FP32 Accuracy = 79.29\%)} \\ 
 \hline
 \hline
 64 & 42.49 & 42.51 & 42.58 & 43.45 & 77.58 & 77.37 & 77.43 & 78.1 \\
 \hline
 32 & 43.35 & 42.49 & 43.64 & 43.73 & 77.86 & 75.32 & 77.28 & 77.86  \\
 \hline
 16 & 44.21 & 44.21 & 43.64 & 42.97 & 78.65 & 77 & 76.94 & 77.98  \\
 \hline
 \hline
 \multicolumn{5}{|c|}{Winogrande (FP32 Accuracy = 69.38\%)} & \multicolumn{4}{|c|}{Piqa (FP32 Accuracy = 78.07\%)} \\ 
 \hline
 \hline
 64 & 68.9 & 68.43 & 69.77 & 68.19 & 77.09 & 76.82 & 77.09 & 77.86 \\
 \hline
 32 & 69.38 & 68.51 & 68.82 & 68.90 & 78.07 & 76.71 & 78.07 & 77.86  \\
 \hline
 16 & 69.53 & 67.09 & 69.38 & 68.90 & 77.37 & 77.8 & 77.91 & 77.69  \\
 \hline
\end{tabular}
\caption{\label{tab:mmlu_abalation} Accuracy on LM evaluation harness tasks on Llama2-7B model.}
\end{table}

\begin{table} \centering
\begin{tabular}{|c||c|c|c|c||c|c|c|c|} 
\hline
 $L_b \rightarrow$& \multicolumn{4}{c||}{8} & \multicolumn{4}{c||}{8}\\
 \hline
 \backslashbox{$L_A$\kern-1em}{\kern-1em$N_c$} & 2 & 4 & 8 & 16 & 2 & 4 & 8 & 16  \\
 %$N_c \rightarrow$ & 2 & 4 & 8 & 16 & 2 & 4 & 2 \\
 \hline
 \hline
 \multicolumn{5}{|c|}{Race (FP32 Accuracy = 48.8\%)} & \multicolumn{4}{|c|}{Boolq (FP32 Accuracy = 85.23\%)} \\ 
 \hline
 \hline
 64 & 49.00 & 49.00 & 49.28 & 48.71 & 82.82 & 84.28 & 84.03 & 84.25 \\
 \hline
 32 & 49.57 & 48.52 & 48.33 & 49.28 & 83.85 & 84.46 & 84.31 & 84.93  \\
 \hline
 16 & 49.85 & 49.09 & 49.28 & 48.99 & 85.11 & 84.46 & 84.61 & 83.94  \\
 \hline
 \hline
 \multicolumn{5}{|c|}{Winogrande (FP32 Accuracy = 79.95\%)} & \multicolumn{4}{|c|}{Piqa (FP32 Accuracy = 81.56\%)} \\ 
 \hline
 \hline
 64 & 78.77 & 78.45 & 78.37 & 79.16 & 81.45 & 80.69 & 81.45 & 81.5 \\
 \hline
 32 & 78.45 & 79.01 & 78.69 & 80.66 & 81.56 & 80.58 & 81.18 & 81.34  \\
 \hline
 16 & 79.95 & 79.56 & 79.79 & 79.72 & 81.28 & 81.66 & 81.28 & 80.96  \\
 \hline
\end{tabular}
\caption{\label{tab:mmlu_abalation} Accuracy on LM evaluation harness tasks on Llama2-70B model.}
\end{table}

%\section{MSE Studies}
%\textcolor{red}{TODO}


\subsection{Number Formats and Quantization Method}
\label{subsec:numFormats_quantMethod}
\subsubsection{Integer Format}
An $n$-bit signed integer (INT) is typically represented with a 2s-complement format \citep{yao2022zeroquant,xiao2023smoothquant,dai2021vsq}, where the most significant bit denotes the sign.

\subsubsection{Floating Point Format}
An $n$-bit signed floating point (FP) number $x$ comprises of a 1-bit sign ($x_{\mathrm{sign}}$), $B_m$-bit mantissa ($x_{\mathrm{mant}}$) and $B_e$-bit exponent ($x_{\mathrm{exp}}$) such that $B_m+B_e=n-1$. The associated constant exponent bias ($E_{\mathrm{bias}}$) is computed as $(2^{{B_e}-1}-1)$. We denote this format as $E_{B_e}M_{B_m}$.  

\subsubsection{Quantization Scheme}
\label{subsec:quant_method}
A quantization scheme dictates how a given unquantized tensor is converted to its quantized representation. We consider FP formats for the purpose of illustration. Given an unquantized tensor $\bm{X}$ and an FP format $E_{B_e}M_{B_m}$, we first, we compute the quantization scale factor $s_X$ that maps the maximum absolute value of $\bm{X}$ to the maximum quantization level of the $E_{B_e}M_{B_m}$ format as follows:
\begin{align}
\label{eq:sf}
    s_X = \frac{\mathrm{max}(|\bm{X}|)}{\mathrm{max}(E_{B_e}M_{B_m})}
\end{align}
In the above equation, $|\cdot|$ denotes the absolute value function.

Next, we scale $\bm{X}$ by $s_X$ and quantize it to $\hat{\bm{X}}$ by rounding it to the nearest quantization level of $E_{B_e}M_{B_m}$ as:

\begin{align}
\label{eq:tensor_quant}
    \hat{\bm{X}} = \text{round-to-nearest}\left(\frac{\bm{X}}{s_X}, E_{B_e}M_{B_m}\right)
\end{align}

We perform dynamic max-scaled quantization \citep{wu2020integer}, where the scale factor $s$ for activations is dynamically computed during runtime.

\subsection{Vector Scaled Quantization}
\begin{wrapfigure}{r}{0.35\linewidth}
  \centering
  \includegraphics[width=\linewidth]{sections/figures/vsquant.jpg}
  \caption{\small Vectorwise decomposition for per-vector scaled quantization (VSQ \citep{dai2021vsq}).}
  \label{fig:vsquant}
\end{wrapfigure}
During VSQ \citep{dai2021vsq}, the operand tensors are decomposed into 1D vectors in a hardware friendly manner as shown in Figure \ref{fig:vsquant}. Since the decomposed tensors are used as operands in matrix multiplications during inference, it is beneficial to perform this decomposition along the reduction dimension of the multiplication. The vectorwise quantization is performed similar to tensorwise quantization described in Equations \ref{eq:sf} and \ref{eq:tensor_quant}, where a scale factor $s_v$ is required for each vector $\bm{v}$ that maps the maximum absolute value of that vector to the maximum quantization level. While smaller vector lengths can lead to larger accuracy gains, the associated memory and computational overheads due to the per-vector scale factors increases. To alleviate these overheads, VSQ \citep{dai2021vsq} proposed a second level quantization of the per-vector scale factors to unsigned integers, while MX \citep{rouhani2023shared} quantizes them to integer powers of 2 (denoted as $2^{INT}$).

\subsubsection{MX Format}
The MX format proposed in \citep{rouhani2023microscaling} introduces the concept of sub-block shifting. For every two scalar elements of $b$-bits each, there is a shared exponent bit. The value of this exponent bit is determined through an empirical analysis that targets minimizing quantization MSE. We note that the FP format $E_{1}M_{b}$ is strictly better than MX from an accuracy perspective since it allocates a dedicated exponent bit to each scalar as opposed to sharing it across two scalars. Therefore, we conservatively bound the accuracy of a $b+2$-bit signed MX format with that of a $E_{1}M_{b}$ format in our comparisons. For instance, we use E1M2 format as a proxy for MX4.

\begin{figure}
    \centering
    \includegraphics[width=1\linewidth]{sections//figures/BlockFormats.pdf}
    \caption{\small Comparing LO-BCQ to MX format.}
    \label{fig:block_formats}
\end{figure}

Figure \ref{fig:block_formats} compares our $4$-bit LO-BCQ block format to MX \citep{rouhani2023microscaling}. As shown, both LO-BCQ and MX decompose a given operand tensor into block arrays and each block array into blocks. Similar to MX, we find that per-block quantization ($L_b < L_A$) leads to better accuracy due to increased flexibility. While MX achieves this through per-block $1$-bit micro-scales, we associate a dedicated codebook to each block through a per-block codebook selector. Further, MX quantizes the per-block array scale-factor to E8M0 format without per-tensor scaling. In contrast during LO-BCQ, we find that per-tensor scaling combined with quantization of per-block array scale-factor to E4M3 format results in superior inference accuracy across models. 


\newpage{}

\begin{comment}
It is possible to create the Hebrew part in \LyX{}, but this is less
of our concern. Any typesetting software like \LyX{} (or Word or OpenOffice)
is as good for this purpose. After creating the PDF file from the
Hebrew document, include it here using the Insert -> File -> External
material -> PDFpages (one of the options). See the example below. 
\end{comment}


\includepdf[pages=-]{hebrew_part_thesis}
\end{document}