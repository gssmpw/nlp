

\section{Results of the Paper}
In this section, we study how to use a novel explanatory dialogue model to determine and explain the acceptance of a formula $\phi$ wrt argumentation semantics.

Intuitively, a \textit{successful dialogue} for formula $\phi$ wrt argumentation semantics is a \textit{dialectical proof procedure} for $\phi$. To argue for the usefulness of the dialogue model, we will study \emph{winning conditions} ("conditions" for short) for a successful dialogue to be \textit{sound} and \textit{complete} wrt argumentation semantics. To do so, we use dialogue trees. When the agent decides what to utter or whether a terminated dialogue is \emph{successful}, it needs to consider the current dialogue tree and ensure that its new utterances will keep the tree fulfilling desired \textit{properties}. 
Thus, the dialogue tree drawn from a dialogue can be seen as \emph{commitment store}~\cite{prakken_2006} holding information disclosed and used in the dialogue. Successful dialogues, in this sense, can be regarded as explanations for the acceptance of a formula.
% Let us consider the \emph{conditions} for a successful dialogue. 

Before continuing, we present preliminary notions/results to prove the soundness and completeness results.

\subsection{Notions for Soundness and Completeness Results}
Let us introduce notions that will be useful in the next sections. 
These notions include: \textbf{potential argument} obtained from a dialogue tree, \textbf{collective attacks} against a potential argument in a dialogue tree, and \textbf{P-SAF} drawn from a dialogue tree. % Since $\mT(\delta)$ is drawn from $\delta$, we can say $\mAF_ \delta$ drawn from $\delta$ instead. 
%Due to limitation pages, we refer to Appendix~\ref{app:sec-dialog-tree} for the formal definitions.

A \emph{potential argument} is an argument obtained from a dialogue tree.

% \begin{definition} \label{def:arg-t} A \emph{potential argument} obtained from a dialogue tree $\mT(\delta)$ is a \emph{sub-tree} $\mT^{s}$ of  $\mT(\delta)$ such that:
% \begin{itemize}
%     \item all nodes in $\mT^{s}$ have the same label (either $\po$ or $\op$);
    % \item if there is an utterance $(\_ , \_ , \off(\Delta, \alpha), \id)$, where $\alpha \in \cn(\Delta)$, in $\delta$ and the node $(\alpha, [\nf, \tL, \_])$ is in $\mT^{s}$, then for every $\beta_j \in \Delta$, the nodes
    % $(\beta_1, [\_ , \tL , \id]), \ldots, (\beta_m, [\_ , \tL , \id])$ are in $\mT^{s}$;
    % \item there is no node $N$ in $\mT(\delta)$ such that $N$ is parent or child of some node $N_i$ in $\mT^{s}$, $N$ is not in $\mT^{s}$ and $N_i$, $N$ have the same label.
% \end{itemize}  

\begin{definition} \label{def:arg-t} A \emph{potential argument} obtained from a dialogue tree $\mT(\delta)$ is a \emph{sub-tree} $\mT^{s}$ of  $\mT(\delta)$ such that:
\begin{itemize}
    \item all nodes in $\mT^{s}$ have the same label (either $\po$ or $\op$);
    \item if there is an utterance $(\_ , \_ , \off(\Delta, \alpha), \id)$
        and a node $(\beta_i, [\_ , \tL , \id])$ in $\mT^{s}$ with $\beta_i \in \Delta$,
        then all the nodes
    $(\beta_1, [\_ , \tL , \id]), \ldots, (\beta_m, [\_ , \tL , \id])$ are in $\mT^{s}$
    \item for every node $(\alpha, [\nf, \tL, \_])$ in $\mT^{s}$,
        all its immediate children in $\mT^{s}$ have the same identifier (they belong to a single utterance).
\end{itemize}  
The formula $\phi$ in the root of $\mT^{s}$ is the \emph{conclusion}. The set of the formulas $H$ held by the descended nodes in $\mT^{s}$, i.e., $H = \{\beta \mid (\beta, [\f,\ \_,\ \_]) \text{ is a node in } \mT^{s}\}$, is the \emph{support} of $\mT^{s}$. %This argument can be written $H \rightarrow_{\mT^{s}} \psi$.
A potential argument obtained from a dialogue tree is a \emph{proponent (opponent) argument} if its nodes are labelled $\po$ ($\op$, respectively). 
\end{definition}

To shorten notation, we use the term "an argument for $\phi$" instead of the term "an argument with the conclusion $\phi$".

\begin{example} [Continue Example~\ref{ex:tab-dia}]
Figure~\ref{fig:comple-tree} shows two potential arguments obtained from $\mT(\delta)$.
\end{example}

Potential arguments correspond to the conventional P-SAF arguments.

\begin{lemma}
\label{lem:potential-arg}
A potential argument $\mT^{s}$ corresponds to an argument for $\phi$ supported by $H$ as in conventional P-SAF (in Definition~\ref{def:ab-arg}).
\end{lemma}

\begin{proof} This lemma is trivially true as a node in a potential argument can be mapped to a node in a conventional P-SAF argument (in Definition~\ref{def:ab-arg}) by dropping the tag $\tT$ and the identifier $\ID$.
\end{proof}

We introduce \emph{collective attacks} against a potential argument, or a sub-tree, in a dialogue tree. This states that a potential argument is \emph{attacked} when there exist nodes within the tree that are children of the argument. Formally:

\begin{definition}
Let $\mT(\delta)$ be a dialogue tree and $\mT^{s}$ be a potential argument obtained from $\mT(\delta)$. $\mT^{s}$ is \emph{attacked} iff there is a node $N = (\tL, [\tT, \_, \_])$ in $\mT^{s}$, with $\tL \in \{\po, \op\}$ and $\tT \in \{\f, \nf\}$, such that $N$ has children $M_1, \ldots, M_k$ labelled by $\tL^{\prime} \in \{\po, \op\}\setminus\{\tL\}$ in $\mT(\delta)$ and the children have the same identifier.
    
    We say that the sub-trees rooted at $M_j$ ($1 \leq j \leq k$) \emph{attacks} $\mT^{s}$.

\end{definition}

\begin{definition} (A) \emph{P-SAF drawn from} $\mT(\delta)$ is $ \mAF_ \delta = (\Arg_{\delta}, \Att_{\delta})$, where 
\begin{itemize}
    \item $\Arg_{\delta}$ is the set of potential arguments obtained from $\mT(\delta)$;
    \item $\Att_{\delta}$ contains the attacks between the potential arguments.
\end{itemize}
\end{definition}
Since $\mT(\delta)$ is drawn from $\delta$, we can say $\mAF_ \delta$ drawn from $\delta$ instead.


 As in~\cite{DUNG2006114}, two useful concepts that are used for our soundness result in the next sections are the \emph{defence set} and the \emph{culprits} of a dialogue tree. 
 \begin{definition}
 Given a dialogue tree $\mT(\delta)$, 
 \begin{itemize}
 \item The \emph{defence set} $\mDE(\mT(\delta))$ is the set of  all facts $\alpha$ in proponent nodes of the form $N = (\alpha,[ \f, \po, \_])$ such that $N$ is in a potential argument;

\item The \emph{culprits} $\mCU(\mT(\delta))$ is the set of facts $\beta$ in opponent nodes $N = (\beta, [\f, \op, \_])$ such that $N$ has the child node $N^{\prime} = (\_,[ \_, \po, \_])$ and $N$ and $N^{\prime}$ are in potential arguments.
\end{itemize}
\end{definition}

\begin{example}
Figure~\ref{fig:comple-tree} (Left) gives the focused dialogue tree drawn from the dialogue $D(\rese(\vi))$ in Example~\ref{ex:tab-dia}. The defence set is  $\{\te(\vi, \kr), \gc(\kr), \te(\vi, \kd)\}$; the culprits are $\{\teAs(\vi, \kd), \uc(\kd)\}$.
\end{example}


\begin{figure}
\centering
\begin{tikzpicture}
    \node (dt) at (0,0) {\includegraphics[scale=0.6]{Picture/dialogtree.pdf}};
    \node (d1) at (7,1.5) {\includegraphics[scale=0.65]{Picture/argument1.pdf}};
    \node (d2) at (7,-2) {\includegraphics[scale=0.65]{Picture/argument2.pdf}};
\end{tikzpicture}
\caption{
Left:
A focused dialogue tree $\mT(\delta)$ drawn from $D(\rese(\vi))$ in Table~\ref{tab:dia}.
Right: Some potential argument obtained from $\mT(\delta)$.
}
\label{fig:comple-tree}
\end{figure}


\subsection{Soundness Results}
\label{sec:soundness}

\subsubsection{Computing credulous acceptance}
\label{sec:credulously-success}

We present winning conditions for a \textit{credulously successful dialogue} to prove whether a formula is credulously accepted under admissible/ preferred/ stable semantics. 

Let us sketch the idea of a dialectical proof procedure for computing the credulous acceptance as follows:
Assume that a (dispute) dialogue between an agent $a_1$ and $a_2$ in which $a_1$ persuades $a_2$ about its belief "$\phi$ is accepted". Two agents take alternating turns in exchanging their arguments in the form of formulas. When the (dispute) dialogue progresses, we are increasingly building, starting from the root $\phi$, a dialogue tree. Each node of such tree, labelled with either $\po$ or $\op$, corresponds to an utterance played by the agent. The credulous acceptance of $\phi$ is proven if $\po$ can win the game by ending the dialogue in its favour according to a “\textit{last-word}” principle. 

To facilitate our idea, we introduce the properties of a dialogue tree:\textit{ patient, last-word, defensive and non-redundant.}

%We refer readers to Appendix~\ref{app:sec-credulous-semantics} for definitions of the properties.


Firstly, we restrict dialogue trees to be \emph{patient}. This means that agents wait until a potential argument has been fully constructed before beginning to attack it. Formally: A dialogue tree $\mT(\delta)$ is \emph{patient} iff for all nodes $N = (\_, [\f,\_,\_])$ in $\mT(\delta)$, $N$ is in (the support of) a potential argument obtained from $\mT(\delta)$.
Through this paper, the term "dialogue trees" refers to \emph{patient dialogue trees}.

% We observe that a formula $\phi$ can have many arguments leading to $\phi$. Thus, the dialogue tree $\mT(\delta)$ (drawn from a dialogue $D(\phi) = \delta$) with root $\phi$ corresponds to one, none, or multiple \emph{abstract dialogue trees} (as defined in~\cite{loanho_2024}) for each single potential argument for $\phi$. Intuitively, the dialogue $\delta$ can be understood as the collection of several independent \emph{focused sub-dialogues}
% \footnote{Given a dialogue $D(\phi) = \delta$, $\delta^{\prime}$ is a \emph{focused sub-dialogue} of $\delta$ iff it is a dialogue for $\phi$, and for all utterances $u \in \delta^{\prime}$, $u \in \delta$. We say that $\delta$ is the \emph{full-dialogue} of $\delta^{\prime}$.}
% $\delta_1, \ldots, \delta_n$, where each dialogue tree drawn from $\delta_i$ is a subtree of $\mT(\delta)$ and corresponds to the abstract one.    

% Note that each such subtree of $\mT(\delta)$ has the desired properties: (1) $\phi$ is supported by a single proponent argument; (2) An opponent argument is attacked by either a single proponent argument or a set of collective proponent arguments; (3) A proponent argument can be attacked by either many single opponent arguments or sets of collective opponent arguments. We call the tree with these properties the \emph{focused dialogue tree}.

% \begin{definition}
% \label{def:t-focused-patient}
% A dialogue tree $\mT(\delta)$ is \emph{focused} iff
% \begin{enumerate}
%     \item for all nodes of the form $(\beta_0,\ [ \nf, \po, \id])$ with children $(\beta_i,\ [\_, \po, \_]),$ $ \ldots, (\beta_m,\ [\_, \po, \_])$, there is an utterance in $\delta$ of the form
    
%     \[ (\_, \_, \off(\Delta,\ \beta_0), \id), \]
    
%     where $\Delta = \{\beta_1, \ldots , \beta_m \}$ and $\beta_0 \in \cn(\Delta)$;
    
%     \item for all potential arguments $A$ obtained from $\mT(\delta)$, if $A$ contains a node $(\eta, [\_, \op, \_])$, then there is at most one node $N$ of the form $(\eta, [\_, \op, \_])$ in $A$ such that $N$ has a single child or children of the form $(\nu_k, [\_, \po, \_])$, where $\bigwedge \nu_k \cup \{ \eta \}$ is a minimal conflict. 
% \end{enumerate}
% \end{definition}



We now present the "last-word" principle to specify a winning condition for the proponent. In a dialogue tree, $\po$ wins if either $\po$ finishes the dialogue tree with the un-attacked facts (Item 1), or any attacks used by $\op$ have been attacked with valid counter attacks (Item 2). Formally:

\begin{definition} A focused dialogue tree $\mT(\delta)$ is \emph{last-word} iff

    \begin{enumerate}
     \item for all leaf nodes $N$ in $\mT(\delta)$, $N$  is the form of $(\_, [\f, \po, \_])$, and
     
     %either $(\_, [\nf, \po, \_])$ or $(\_, [\f, \po, \_])$, 

     \item if a node $N$ is of the form $(\_, [\tT, \op, \_])$ with $\tT \in \{\f, \nf\}$, then $N$ is in a potential argument and $N$ is properly attacked.   
    \end{enumerate}
\end{definition}
In the above definition, we say that a node $N$ of a potential argument is attacked, meaning that $N$ has children labelled by $\po$ with the same identifier.

 The definition of "last-word" incorporates the requirement that a set of potential arguments $\mS$ (supported by the defence set) attacks every attack against $\mS$. However, it does not include the requirement that $\mS$ does not attack itself. This requirement is incorporated in the definition of \emph{defensive dialogue trees}. 

\begin{definition} 
\label{def:defensive-tree}
A focused dialogue tree $\mT(\delta)$ is \emph{defensive} iff it is
\begin{itemize}
    \item last-word, and
    %\item $S \cup \mDE(\mT(\delta))$ is consistent where $\mS = \mDE(\mT(\delta)) \cap \mCU(\mT(\delta))$.
    \item no formulas $\Delta$ in opponent nodes belong to $\mDE(\mT(\delta))$ such that $\Delta \cup \mDE(\mT(\delta))$ is inconsistent.
   % \item no formula $\alpha$ in an opponent node belongs to $\mDE(\mT(\delta))$ such that $\alpha$ is in a potential argument attacking any potential arguments supported by $\mDE(\mT(\delta))$.
\end{itemize}
\end{definition}

%Notice that it is not required that the opponent and the proponent have no arguments in common. This is because the opponent can use the proponent's arguments against the proponent. If the opponent can attack the proponent using only the proponent's arguments, then the proponent loses. To win, the proponent must identify and counter-attack each opponent's attack with some culprit not part of their defence.

In admissible dialogue trees, nodes labelled $\po$ and $\op$ within potential arguments can have common facts when considering potential arguments that attack or defend others.
However, potential arguments with nodes sharing common facts cannot attack proponent potential arguments whose facts are in the defence set.
Let us show this in the following example.


\begin{example}
\label{ex:a-succ}
Consider a query $q_4 = A(a) $ to a KB $\mK_4 = (\mR_4, \mC_4, \mF_4)$ where 
\begin{align*}
    \mR_4 = &\emptyset \\
    \mC_4 = & \{c_1 : A(x) \land \ B(x) \land C(x) \rightarrow \bot \} \\
    \mF_4 = & \{A(a), B(a) , C(a) \}
\end{align*}
Consider the focused dialogue tree $\mT(\delta_i)$ (see Figure~\ref{fig:tree-ex} (Left)) drawn from the focused sub-dialogue $\delta_i$ of a dialogue $D(A(a)) = \delta$. The defence set $\mDE(\mT(\delta_i)) = \{A(a), C(a)\}$; the culprits $\mCU(\mT(\delta_i)) = \{B(a), C(a)\}$.
We have $\mDE(\mT(\delta_i)) \cap \mCU(\mT(\delta_i)) = \{C(a)\}$. It can seen that $\{C(a)\} \cup \mDE(\mT(\delta_i))$ is inconsistent. In other words, there exists a potential argument, say $A$, such that $\{C(a)\}$ is the support of $A$, and $A$ cannot attack any proponent argument supported by $\mDE(\mT(\delta_i))$. Clearly, $\mDE(\mT(\delta_i))$ and $\mCU(\mT(\delta_i))$ have the common formula, but the set of arguments supported by $\mDE(\mT(\delta_i))$ does not attack itself.
\end{example}

\begin{figure}
\centering
\begin{tikzpicture}
    \node (dt) at (0,0) {\includegraphics[scale=0.6]{Picture/focused-tree-ex7.pdf}};
    \node (d1) at (5, 0) {\includegraphics[scale=0.55]{Picture/ex8.pdf}};
\end{tikzpicture}
\caption{
Left: A focused dialogue tree $\mT(\delta_i)$.
Right: An infinite dialogue tree.
}
\label{fig:tree-ex}
\end{figure}


%The following lemma says that if a dialogue is a-successful, then it has an outcome.

From the above observation, it follows immediately that.

\begin{lemma}
    Let $\mT(\delta)$ be a defensive dialogue tree. The set of proponent arguments (supported by $\mDE(\mT(\delta))$) does not attack itself in the P-SAF drawn from $\delta$.
\end{lemma}

% \begin{proof}
%     fsdfsadf
% \end{proof}

Consider the following dialogue to see why the "non-redundant" property is necessary.

\begin{example}
\label{ex:infinite-credulous}
Consider a query $q_5 = A(a)$ to a KB $\mK_5 = (\mR_5, \mC_5, \mF_5)$ where
\begin{align*}
    \mR_5 = &\emptyset \\
    \mC_5 = & \{ A(x) \land \ B(x) \rightarrow \bot \} \\
    \mF_5 = & \{A(a), B(a) \}
\end{align*}
Initially, an argument $A_1$ asserts that "$A(a)$ is accepted" where $A(a)$ is at the $\po$ node. $A_1$ is attacked by $A_2$ by using $B(a)$ that is at the $\op$ node. $A_1$ counter-attacks $A_2$ by using $A(a)$, then $A_2$ again attacks $A_1$ by using $B(a)$, ad infinitum (see Figure~\ref{fig:tree-ex} (Right)). Hence $\po$ cannot win.
%
%loan: rewrite this
%
Since the grounded extension is empty, $A(a)$ is not groundedly accepted in the P-SAF, thus $\po$ should not win under the grounded semantics. Since $A(a)$ is credulously accepted in the P-SAF, we expect that $\po$ can win in a terminated dialogue under the credulous semantics.
\end{example}




%\loan{Consider the following dialogue - \textbf{Example} for credulous semantics}

To ensure credulous acceptance, all possible opponent nodes must be accounted for. But if such a parent node is already in the dialogue tree, then deploying it will not help the opponent win the dialogues. To avoid this, we define a dialogue tree to be \emph{non-redundant}. 


\begin{definition}
\label{def:non-re}
     A focused dialogue tree $\mT(\delta)$ is \emph{non-redundant} iff for any two nodes $N_1 = (\beta, [\f, \tL, \id_1])$  and $N_2 = (\beta, [\f, \tL, \id_2])$ with $\tL \in \{\po, \op\}$ and $N_1 \neq N_2$, if $N_1$ is in a potential argument $\mT_1^{s}$ and $N_2$ is in a potential argument $\mT_2^{s}$, then $\mT_1^{s} \neq \mT_2^{s}$.
\end{definition}


In Definition~\ref{def:non-re}, when comparing two arguments, we compare their respective proof trees. Here, we only consider the formula and the tag of each node in the tree, disregarding the label and identifier of the node.


The following theorem establishes credulous soundness for admissible semantics.

\begin{restatable}{theorem} {thmcredulous} \label{thm:adm}
 Let $\delta$ be a dialogue for a formula $\phi \in \mL$. If there is a dialogue tree $\mT(\delta_i)$ drawn from a focused sub-dialogue $\delta_i$ of $\delta$ such that it is defensive and non-redundant, then 
  \begin{itemize}
      \item $\delta$ is admissible-successful; 
      \item $\phi$ is credulously accepted under $\adm$ in $\mAF_ \delta$ drawn from $\delta$ (supported by $\mDE(\mT(\delta_i))$.
\end{itemize}
\end{restatable}

The proof of this theorem is in Appendix~\ref{app:proof-soundness}.

We can define a notion of \emph{preferred-successful dialogue} with a formula accepted under $\prf$ in the P-SAF framework drawn from the dialogue. Since every admissible set (of arguments) is necessarily contained in a preferred set (see~\cite{Dung95,Nielsen2007}), and every preferred set is admissible by definition, trivially a dialogue is preferred-successful iff it is admissible-successful. The following theorem is analogous to Theorem~\ref{thm:adm} for $\prf$ semantics.

\begin{restatable} {theorem} {thmpreferred} 
\label{thm:prf-stb}
Let $\delta$ be a dialogue for a formula $\phi \in \mL$. If there is a dialogue tree $\mT(\delta_i)$ drawn from a focused sub-dialogue $\delta_i$ of $\delta$ such that it is defensive and non-redundant, then $\delta$ is preferred-successful and $\phi$ is credulously accepted under $\prf$ in $\mAF_ \delta$ drawn from $\delta$ (supported by $\mDE(\mT(\delta_i))$.    
\end{restatable}

\begin{proof} [Sketch]
The proof of this theory follows the fact that every preferred dialogue tree is an admissible dialogue tree. Thus, the proof of this theorem is analogous to those of Theorem~\ref{thm:adm}.
\end{proof}


\begin{remark}
    We can similarly define a notion of \emph{stable dialogue trees} for a formula accepted under $\stb$ in the P-SAF. Since stable and preferred sets coincide, trivially a dialogue tree is stable iff it is defensive and non-redundant. Thus we can use the result of Theorem~\ref{thm:prf-stb} for stable semantics.
\end{remark}
    


\subsubsection{Computing grounded acceptance}
We present winning conditions for a \textit{groundedly successful dialogue} to determine grounded acceptance of a given formula. %which are used to prove whether a formula is accepted under grounded semantics.
The conditions require that whenever $\op$ could advance any evidence, $\po$ still wins.
This requirement is incorporated in dialogue trees being defensive.
Note that credulously successful dialogues for computing credulous acceptance also require dialogue trees to be defensive (see in Theorem~\ref{thm:adm}).
However, the credulously successful dialogues cannot be used for computing the grounded acceptance, as shown by Example~\ref{ex:infinite-credulous}.
In Example~\ref{ex:infinite-credulous}, it would be incorrect to infer from the depicted credulously successful dialogue that $A(a)$ is groundedly accepted as the grounded extension is empty. Note that the dialogue tree for $A(a)$ is infinite.
From this observation, it follows that the credulously successful dialogues are not sound for computing grounded acceptance. Since all dialogue trees of a formula that is credulously accepted but not groundedly accepted can be infinite,
we could detect this situation by checking if constructed dialogue trees are infinite. This motivates us to consider "\textit{finite}" dialogue trees as a winning condition.

The following theorem establishes the soundness of grounded acceptance.
\begin{restatable} {theorem} {thmground}   
\label{thm:ground}
Let $\delta$ be a dialogue for a formula $\phi \in \mL$. If there is a dialogue tree $\mT(\delta_i)$ drawn from a focused sub-dialogue $\delta_i$ of $\delta$ such that it is defensive and finite, then
  \begin{itemize}
    \item $\delta$ is groundedly-successful;
      \item $\phi$ is groundedly accepted under grounded semantics in $\mAF_ \delta$ drawn from $\delta$ (supported by $\mDE(\mT(\delta_i))$.
\end{itemize}
\end{restatable}

The proof of this theorem is in Appendix~\ref{app:proof-soundness}.

\subsubsection{Computing sceptical acceptance}

Inspired by~\cite{DUNG2007642}, to determine the sceptically acceptance of an argument for $\phi$, we verify the following:
(1) There exists an admissible set of arguments $S$ that includes the argument for $\phi$;
(2) For each argument $A$ attacking $S$, there exists no admissible set of arguments containing $A$.
These steps can be interpreted through the following winning conditions for a \emph{sceptical successful dialogue} to compute the sceptical acceptance of $\phi$:
\begin{enumerate}
    \item $\po$ wins the game by ending the dialogue,
    \item none of $\op$ wins by the same line of reasoning.
\end{enumerate}
This perspective allows us to introduce a notion of \emph{ideal dialogue trees}.

\begin{definition}
\label{def:tree-ideal}
     A defensive and non-redundant dialogue tree $\mT(\delta)$ is \emph{ideal} iff none of the opponent arguments obtained from $\mT(\delta)$ belongs to an admissible set of potential arguments in $ \mAF_ \delta$ drawn from $\mT(\delta)$.
\end{definition}

The following result sanctions the soundness of sceptical acceptance.

\begin{restatable} {theorem}{thmsceptical}
\label{thm:scep}
Let $\delta$ be a dialogue for a formula $\phi \in \mL$. If there is a dialogue tree $\mT(\delta)$ drawn from $\delta$ such that it is ideal, then
\begin{itemize}
    \item $\delta$ is sceptically-successful;
    \item $\phi$ is sceptically accepted under $\sem$ in $\mAF_ \delta$ drawn from $\delta$ (supported by $\mDE(\mT(\delta))$, where $\sem \in \{\adm, \prf, \stb\}$.
\end{itemize} 
\end{restatable}
The proof of this theorem is in Appendix~\ref{app:proof-soundness}.

\subsection{Completeness Results}
\label{sec:completeness}
We now present completeness. 
In this work, dialogues viewed as dialectical proof procedures are sound but not always complete in general.
The reason is that the dialectical proof procedures might enter a non-terminating loop during the process of argument constructions, which leads to the incompleteness wrt the admissibility semantics.
To illustrate this, we refer to Example 1 using logic programming in~\cite{ThangDP22} for an explanation.
We also provide another example using \datalogPM.

\begin{example} Consider a query $q_6 = P(a)$ to a \datalogPM KB $\mK_6 = (\mR_6 , \mC_6 , \mF_6)$ where
\begin{align*}
    \mR_6 = & \{r_1: P(x) \rightarrow Q(x), r_2: Q(x) \rightarrow P(x) \} \\
    \mC_6 = & \{ P(x) \land R(x) \rightarrow \bot \} \\
    \mF_6 = & \{P(a) , R(a)\}
\end{align*}
The semantics of the corresponding P-SAF $\mAF_4$ are determined by the arguments illustrated in Figure~\ref{fig:infinite-loop}. The result should state that "$P(a)$ is a possible answer" as the argument $B_1$ for $P(a)$ is credulously accepted under the admissible sets $\{B_1\}$  and $\{B_2\}$ of $\mAF_4$. 
 But the dialectical proof procedures fail to deliver the admissible set $\{ B_1 \}$ wrt $\mAF_4$
as they could not overcome the non-termination of the process to construct an argument $B_1$ for $P(a)$ due to the “infinite loop”. 
    
\end{example}
\begin{figure}
    \centering
    \includegraphics[width=0.25\linewidth]{Picture/infinite-loop.pdf}
    \caption{Arguments of $\mAF_4$}
    \label{fig:infinite-loop}
\end{figure}

% \begin{example} [Example 1~\cite{ThangDP22}] Consider a logic program $P \subseteq \mL$ including  a set of literals and rules.

% $P = \{r :\ \neg \alpha \rightarrow p,\ r^{\prime} : f(0) \rightarrow \alpha ,\ r_n: f(n+1) \rightarrow f(n), n \geq 0,\ t: \rightarrow \beta \} $

% Consider a query $q = \alpha$. It is clear to see that the dialectical proof procedures are non-terminated when constructing an argument for $\alpha$ (However, the corresponding P-SAF admits a single preferred and stable set $\{A, B\}$).    
% \end{example}
%
%Consider a query $q = A(a)$ and a KB $\mK_4 = (\mR_4, \mC_4, \mF_4)$ where $\mR_4 = \{A(x) \rightarrow B(x),\ B(x) \rightarrow A(x)\}$, $\mC_4 = \emptyset$, $\mF_4 = \{B(a)\}$. The S-PAF  admits a single preferred and stable set including $\{B(a)\}$. Then, there is an admissible dialogue tree for $q = A(a)$ but a dialogue for $q = A(a)$ is infinite loops. 
%

Intuitively, since the dialogues as dialectical proof procedures (implicitly) incorporate the computation of arguments top-down, the process of argument construction should be finite (also known as finite tree-derivations in the sense of Definition~\ref{def:ab-arg}) to achieve the completeness results. Thus, we restrict the attention to decidable logic with cycle-restricted conditions that its corresponding P-SAF framework produces arguments to be computed finitely in a top-down fashion. For example, given a \datalogPM KB $\mK = (\mR, \mC, \mF)$, the \emph{dependency graph} of the KB  as defined in~\cite{HechamBC17}  consists of the vertices representing the atoms and the edges from an atom $u$ to $v$ iff $v$ is obtained from $u$ (possibly with other atoms) by the application of a rule in $\mR$. The intuition behind the use of the dependency graph is that no infinite tree-derivation exists if the dependency graph of KB is acyclic. By restricting such acyclic dependency graph condition, the process of argument construction in the corresponding  P-SAF of the KB $\mK$ will be finite, which leads to the completeness of the dialogues wrt argumentation semantics. The following theorems show the completeness of credulous acceptances wrt admissible semantics.



% The dependency graph of logic $(\mL, \cn)$ is a directed graph where:
% \begin{itemize}
%     \item the vertices are the formulas of $\mL$;
%     \item a (dirrected) arc from a node $N$ to a node $N^{\prime}$ is in the graph iff $\alpha \in \cn(\beta)$ where $\alpha$ and $\beta$ are formulas in the node $N$ and $N^{\prime}$.
% \end{itemize}
 
%Studying the completeness of dialogue models remains an open problem. The results in~\cite{ThangDP22} on dispute derivations for ABA provide a useful starting point. We will consider the idea for future work.
%  Here to obtain the completeness results, we only consider the following sufficient conditions: 

% 1. The language $\mL$ is finite.

% 2. not cyclic

% We obtain completeness results ithe n the case of p-acyclic framework with a finite language.

\begin{restatable} {theorem}{compadm}
\label{thm:com-adm}
Let $\delta$ be a dialogue for a formula $\phi \in \mL$. If $\phi$ is credulously accepted under $\adm$ in $\mAF_ \delta$ drawn from $\delta$ (supported by $\mDE(\mT(\delta))$)
and $\delta$ is admissible-successful, then there is a defensive and non-redundant dialogue tree $\mT(\delta_i)$ for $\phi$ drawn from a focused sub-dialogue $\delta_i$ of $\delta$.
\end{restatable}

The proof of this theorem is in Appendix~\ref{app:proof-completeness}.

The following theorem is analogous to Theorem~\ref{thm:com-adm} for preferred semantics.
\begin{restatable} {theorem}{comppreferred}
\label{thm:com-prf}
Let $\delta$ be a dialogue for a formula $\phi \in \mL$. If $\phi$ is credulously accepted under $\prf$ in $\mAF_ \delta$ drawn from $\delta$ (supported by $\mDE(\mT(\delta))$)
and $\delta$ is preferred-successful, then there is a defensive and non-redundant dialogue tree $\mT(\delta_i)$ for $\phi$ drawn from a focused sub-dialogue $\delta_i$ of $\delta$.
\end{restatable}

\begin{proof} [Sketch]
The proof of this theory follows the fact that every preferred-successful dialogue is an admissible-successful dialogue. Thus, the proof of this theorem is analogous to those of Theorem~\ref{thm:com-adm}.
\end{proof}
Theorem~\ref{thm:com-ground} presents the completeness of grounded acceptances.
\begin{restatable} {theorem}{compground}
\label{thm:com-ground}
Let $\delta$ be a dialogue for a formula $\phi \in \mL$. If $\phi$ is groundedly accepted under $\grd$ in $\mAF_ \delta$ drawn from $\delta$ (supported by $\mDE(\mT(\delta))$) and $\delta$ is groundedly-successful, then there is a defensive and finite dialogue tree $\mT(\delta_i)$ for $\phi$ drawn from a focused sub-dialogue $\delta_i$ of $\delta$.
\end{restatable}

The proof of this theorem is in Appendix~\ref{app:proof-completeness}.

Theorem~\ref{thm:com-scep} presents the completeness of sceptical acceptances.

\begin{restatable} {theorem}{compsceptical}
\label{thm:com-scep}
Let $\delta$ be a dialogue for a formula $\phi \in \mL$. If $\phi$ is sceptically accepted under $\sem$ in $\mAF_ \delta$ drawn from $\delta$ (supported by $\mDE(\mT(\delta))$), where $\sem \in \{\adm, \prf, \stb\}$, and $\delta$ is sceptically-successful, then there is an ideal dialogue tree $\mT(\delta)$ for $\phi$ drawn from $\delta$.
\end{restatable}

The proof of this theorem is in Appendix~\ref{app:proof-completeness}.

\subsection{Results for a Link between Inconsistency-Tolerant Reasoning and Dialogues}

In Section~\ref{sec:soundness} and~\ref{sec:completeness}, we demonstrated the use of dialogue trees to determine the acceptance of a formula in the P-SAF drawn from the dialogue tree. As a direct corollary of Theorem~\ref{thm:ab-link} -\ \ref{thm:com-scep}, we show how to determine and explain the entailment of a formula in KBs by using dialogue trees, which was the main goal of this paper.

\begin{corollary}
    Let $ \mK$ be a KB, $\phi$ be a formula in $\mL$. Then $\phi$ is entailed in
    \begin{itemize}
        \item some maximal consistent subset of $\mK$ iff there is a defensive and non-redundant dialogue tree $\mT(\delta)$ for $\phi$.
        
        \item the intersection of maximal consistent subsets of $\mK$ iff there is a defensive and finite dialogue tree $\mT(\delta)$ for $\phi$.
        
        \item all maximal consistent subsets of $\mK$ iff there is an ideal dialogue tree $\mT(\delta)$ for $\phi$.
    \end{itemize}   
\end{corollary}

