\section{Explanatory Dialogue Models}
\label{sec:model-exp-dia}

Inspired by~\cite{prakken_2006,Prakken05}, we develop a \textit{novel explanatory dialogue model} of P-SAF by examining the dispute process involving the exchange of arguments (represented as formulas in KBs) between two agents. The novel explanatory dialogue model can show how to determine and explain the acceptance of a formula wrt argumentation semantics.
%Successful dialogues can be regarded as explanations in this regard.

%which is incremental from that of G-SAF in~\cite{loanho_2024}.
% We introduce a \textit{novel dialogue model} for \datalogPM.
%This novel dialogue model differs from that of G-SAF in~\cite{loanho_2024} by considering the process of moving formulas in KBs, rather than the process of moving arguments and counter-arguments. Then, it can show how to determine the acceptance of a formula wrt argumentation semantics.

%Ours differs from those presented in the works~\cite{Prakken2002,DUNNE2003221,Cayrol2001,Arioua2016} in that it does not have a limited focus on persuasion. Instead, it enables agents to build 'shared' knowledge and play interchangeably. Thus, our dialogue model is generic and can support various types of dialogues, such as seeking information, persuasion, or inquiry dialogues.

\subsection{Basic Notions}
\textbf{Concepts} of a novel dialogue model for P-SAFs include \textbf{utterances, dialogues} and \textbf{concrete dialogue trees} ("\textbf{dialogue tree}" for short).
%
In this model, a topic language $\mL_{t}$ is abstract logic $(\mL, \cn)$; dialogues are sequences of utterances between two agents $a_1$ and $a_2$ sharing a common language $\mL_{c}$. Utterances are defined as follows:

\begin{definition} [Utterances]
An \emph{utterance} of agents $a_i,\ i \in \{1,2\}$ has the form $u = (a_i, \TG, \CO, \ID)$, where:
\begin{itemize}
   % \item $a_i$, $i \in \{1,2\}$ is the \emph{player} who played the utterance,
    \item $\ID \in \mathbb{N}$ is the \emph{identifier} of the utterance,

    \item $\TG$ is the \emph{target} of the utterance and we impose that $\TG < \ID$,
    \item $\CO \in \mL_c$ (the \emph{content}) is one of the following forms: Fix $\phi \in \mL$ and $\Delta \subseteq \mL$.
    
    \begin{itemize}
         \item $\cla(\phi)$: The agent asserts that $\phi$ is the case,
        
         \item $\off(\Delta, \phi)$: The agent advances \emph{grounds} $\Delta$ for $\phi$ uttered by the previously advanced utterances such that $\phi \in \cn(\Delta)$,
    
        \item $\cont(\Delta,\ \phi)$: The agent advances the formulas $\Delta$ that are contrary to $\phi$ uttered by the previously advanced utterance,
        \item $\cond(\phi)$: The agent gives up debating and admits that $\phi$ is the case,

         \item $\fa(\phi)$: The agent asserts that $\phi$ is a fact in $\mK$.

         \item $\kappa$: The agent does not have or wants to contribute information at that point in the dialogue.

    \end{itemize}  
\end{itemize}
We denote by $\mU$ the set of all utterances. 
\end{definition}

To determine which utterances agents can make to construct a dialogue, we define a notion of \emph{legal move}, similarly to communication protocols. For any two utterances $u_i,\ u_j \in \mU$, $u_i \neq u_j$, we say that:
\begin{itemize}
    \item $u_i$ is the \emph{target utterance} of $u_j$ iff the target of $u_j$ is the identifier of $u_i$, i.e., $u_i = (\_, \_, \CO_i, \ID)$ and $u_j = (\_, \ID, \CO_j, \_)$;

    \item $u_j$ is the \emph{legal move} after $u_i$ iff $u_i$ is the target utterance of $u_j$ and one of the following cases in Table~\ref{tab:legal-moves} holds.
    \end{itemize}

    \begin{table}\vspace{-6mm}
    \centering
        \caption{Locutions and responses}
   \label{tab:legal-moves}
    \begin{tabular}{|l|l|}
    \hline
    Locution $u_i$ &  Available responses $u_j$ \\
    \hline
    $\CO_i = \cla(\phi)$ & (1) $\CO_j = \off(\_ , \phi)$ if $\phi \in \cn(\{ \_ \})$, \\
                         & (2) $\CO_j =  \fa(\phi)$ if $\phi \in \mK$, \\
                         & (3) $\CO_j =  \cont(\_,\ \phi)$ where $\{\_, \phi \}$ is inconsistent; \\
    \hline
    $\CO_i = \fa(\phi)$ & $\CO_j = \cont(\_ , \phi)$ where $\{ \_, \phi \}$ is inconsistent; \\
    \hline
    $\CO_i = \off(\Delta, \phi)$ & (1) $\CO_j =  \cont(\_,\ \phi)$ where $\{\_, \phi \}$ is inconsistent, \\
     with $\phi \in \cn(\Delta)$ & (2) $\CO_j =  \cont(\_,\ \Delta)$ where $\{\_ \} \cup \Delta$ is inconsistent, \\
                                                         & (3) $\CO_j =  \off(\_, \beta_i)$ with $\beta_i \in \Delta$ and $\beta_j \in \cn(\{\_ \})$ \\
    \hline
    $\CO_i = \cont(\beta, \_)$ & (1) $\CO_j =  \cont(\_, \beta)$ where $\{ \_, \beta \}$ is inconsistent \\
                              & (2) $\CO_j =  \off(\_, \beta)$ with $\beta \in \cn(\{\_\})$. \\
    \hline
    \end{tabular}
\end{table}
     
An utterance is a legal move after another if any of the following cases happens: (1) it with content $\off$ contributes to expanding an argument; (2) it with content $\fa$ identifies a fact in support of an argument; (3) it with content $\cont$ starts the construction of a counter-argument. An utterance can be from the same agent or not. 





