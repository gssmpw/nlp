\section{Introduction}
This paper addresses the problem of explaining logical reasoning in (inconsistent) KBs. Several approaches have been proposed by~\cite{Lukasiewicz2020,Thomas2022Neg,Meghyn2019,Alrabbaa2022}, which mostly include \emph{set-based explanations} and \emph{proof-based explanations}.
Set-based explanations, which are responsible for the derived answer, are defined as \emph{minimal sets of facts} in the existential rules~\cite{Lukasiewicz2020,Thomas2022Neg} or as \emph{causes} in Description Logics (DLs)~\cite{Meghyn2019}.
%In the existential rules context, the prominent approach defines explanations as \emph{minimal sets of facts} (that are responsible for the derived answer)~\cite{Lukasiewicz2020,Thomas2022Neg}, and called \emph{causes} in Description Logics (DLs)~\cite{Meghyn2019}. 
 Additionally, the work in \cite{Meghyn2019} provides the notion of \textit{conflicts} that are \emph{minimal sets of assertions} responsible for a KB to be inconsistent. %Set-based explanations may not be intuitively straightforward to understand for users. 
Set-based explanations present the necessary premises of entailment and, as such, do not articulate the (often non-obvious) reasoning that connects those premises with the conclusion nor track conflicts. %These are all one-shot-based explanations where no interaction with the user is considered. 
Proof-based explanations provide graphical representations to allow users to understand the reasoning progress better~\cite{Alrabbaa2022}. Unfortunately, the research in this area generally focuses on reasoning in \emph{consistent} KBs.

The limit of the above approaches is that they lack the tracking of contradictions, whereas argumentation can address this issue.
Clearly, argumentation offers a potential solution to address inconsistencies. Those are divided into three approaches:
\begin{itemize}
       \item \emph{Argumentation approach based on Deductive logic}: Various works propose instantiations of abstract argumentation (AFs) for  \datalogPM~\cite{ARIOUA201776,Yun2020SetsOA}, Description Logic~\cite{ZhangL13} or Classical Logic~\cite{DAgostinoM18}, focusing on the translation of KBs to argumentation without considering explanations.
       In~\cite{LoanHo2022}, explanations can be viewed as dialectical trees defined abstractly, requiring a deep understanding of formal arguments and trees, making the work impractical for non-experts.
       In~\cite{Yun2020SetsOA,yun2018}, argumentation with collective attacks is proposed to capture non-binary conflicts in \datalogPM, i.e., assuming that every conflict has more than two formulas.

        \item \emph{Sequent-based argumentation}~\cite{ArieliS19} and its extension (\emph{Hypersequent-based argumentation})~\cite{BorgAS17}, using \textit{Propositional Logic}, provide non-monotonic extensions for Gentzen-style proof systems in terms of argumen-tation-based.
        Moreover, the authors conclude by wishing future work to include “the study of more expressive formalisms, like those that are based on first-order logics”~\cite{ArieliS19}.
       
       \item \emph{Rule-based argumentation}: 
       \emph{DeLP/DeLP} with \emph{collective attacks} are introduced for defeasible logic programming~\cite{Alejandro2014,AlsinetBG10}. However, in~\cite{Yun2020SetsOA}, the authors claim that they cannot instantiate DeLP for \datalogPM, since DeLP only considers ground rules.
       In \cite{Prakken2002,ModgilP14}, \emph{ASPIC/ASPIC+} is introduced for defeasible logic. Following~\cite{Amgoud12}, the logical formalism in ASPIC+ is ill-defined, i.e., the contrariness relation is not general enough to consider n-ary constraints. This issue is stated in ~\cite{Yun2020SetsOA} for \datalogPM, namely, the ASPIC+ cannot be directly instantiated with Datalog. The reasons behind this are that Datalog does not have the negation and the contrariness function of ASPIC+ is not general for this language.

        Notable works include assumption-based argumentation (ABA)~\cite{Dung2009} and ABA with collective attack~\cite{DimopoulosD0R0W24}, which are applied for default logic and logic programming. However, ABAs ignore cases of the inferred assumptions conflicting, which is allowed in the existential rules, Description Logic and Logic Programming with Negation as Failure in the Head. We call the ABAs "\emph{flat} ABAs".       
        In~\cite{SCHULZ_TONI_2016}, "flat" ABAs link to Answer Set Programming but only consider a single conflict for each assumption. In~\cite{Rapberger2024,Lehtonen2024},
        "\emph{Non-flat}" ABAs overcome the limits of "flat" ABAs, which allow the inferred assumptions to conflict. However, like ASPIC+, the non-flat ABAs ignore the n-ary constraints case.
        % Indeed, it is not immediately obvious how to represent more complicated logic programming languages in ASPIC+ and the non-flat ABAs, such as disjunctive logic programming.   
        \emph{Contrapositive} ABAs~\cite{HEYNINCK2020103} and its collective attack version~\cite{ArieliH24}, which use \emph{contrapositive propositional logic}, propose extended forms for  'flat' and 'non-flat' ABAs. While~\cite{ArieliH24} mainly focuses on representation (which can be simulated in our setting, see Section~\ref{subsec:relation-framework}), we extend our study to proof procedures in AFs with collective attacks.

\end{itemize}


Argumentation offers dialogue games to determine and explain the acceptance of propositions for classical logic~\cite{Castagna21}, for \datalogPM ~\cite{ARIOUA2017244,Arioua2014FE,Arioua2016,ARIOUA201776}, for logic programming/ default logic~\cite{ThangDH12,Xiuyi14}, and for defeasible logic~\cite{Prakken05}. However, the models have limitations.
In~\cite{ARIOUA2017244,Arioua2014FE}, the dialogue models take place between a domain expert but are only applied to a specific domain (agronomy). 
In~\cite{Arioua2016,ARIOUA201776,Prakken05,Castagna21}, persuasion dialogues (dialectic proof procedures) generate the abstract dispute trees defined abstractly that include arguments and attacks and ignore the internal structure of the argument. These works lack exhaustive explanations, making them insufficient for understanding inference steps and argument structures. The works in~\cite{DUNG2006114,DUNG2007642} provide dialectical proof procedures, while the works~\cite{Xiuyi14,ThangDH12} offer dialogue games (as a distributed mechanism) for "flat" ABAs to determine sentence acceptance under (grounded/ admissible/ ideal) semantics. Although the works in~~\cite{Xiuyi14,ThangDH12} are similar to our idea of using dialogue and tree, these approaches do not generalize to n-ary conflicts.



The existing studies are mostly restricted to (1) specific logic or have limitations in representation aspects, (2) AFs with binary conflicts, and (3) lack exhaustive explanations. 
This paper addresses the limitations by introducing a general framework that provides dialogue models as dialectical proof procedures for acceptance in structured argumentation.
The following is a simple illustration of how our approach works in a university example.
%a dialogical explanation for a university example.

\begin{example}
\label{ex:motivation-ex}
Consider inconsistent knowledge about a university domain, in which we know that: \emph{lecturers} $(\lect)$ and \emph{researchers} $(\rese)$ are \emph{employers} $(\emp)$; \emph{full professors} $(\fp)$ are researchers; everyone who is a teaching assistant $(\teAs)$ of an \emph{undergraduate course} $(\uc)$ is a teaching assistant $(\ta)$; everyone who teaches a course is a lecturer and everyone who teaches a \emph{graduate course} $(\gc)$ is a full professor. 
However, teaching assistants can be neither researchers nor lecturers, which leads to inconsistency. 
We also know that an individual \emph{Victor}  apparently is or was a teaching assistant of the KD course $(\teAs(\vi, \kd))$, and the KD course is an undergraduate course $(\uc(\kd))$. Additionally, \emph{Victor} teaches either the KD course $(\te(\vi, \kd))$ or the KR course $(\te(\vi, \kr))$, where the KR course is a graduate course $(\gc(\kr))$.
The KB $\mK_1$ is modelled as follows:
%
\begin{align*}
\mF_1 = & \{\teAs(\vi,\kd),\ \te(\vi, \kd),\ \uc(\kd),\ \te(\vi,\kr),\ \gc(\kr) \} \\
\mR_1 = & \{ r_1:\ \lect(x) \rightarrow \emp(x) ,\
  r_2:\ \rese(x) \rightarrow \emp(x),\ \\
& r_3:\ \fp(x) \rightarrow \rese(x) ,\
  r_4:\ \teAs(x,y) \land \uc(y)  \rightarrow \ta(x),\ \\
& r_5:\ \te(x,y) \rightarrow \lect(x),\
  r_6:\ \te(x,y) \land \gc(y)  \rightarrow \fp(x) \} \\
\mC_1 = & \{ c_1:\ \ta(x) \land \rese(x) \rightarrow \bot,\
  c_2:\ \lect(x) \land \ta(x) \rightarrow \bot\}
%&c_2: \lect(x), \fp(x) \rightarrow \bot\\
\end{align*}
%
%
When a user asks "\emph{Is Victor a researcher?}", the answer will be "\emph{Yes, but Victor is possibly a researcher}". 
The current method~\cite{Lukasiewicz2020,Thomas2022Neg,Meghyn2019} will provide a set-based explanation consisting of (1) the \emph{cause} $\{\te(\vi,\kr),\ \gc(\kr)\}$ entailing the answer (why the answer is accepted) and (2) the \emph{conflict} $\{\teAs(\vi,\kd), \uc(\kd) \}$ being inconsistent with every cause (why the answer cannot be \textit{accepted}).
The cause, though, does not show a series of reasoning steps to reach $\rese(\vi)$ from the justification $\{\te(\vi,\kr),\ \gc(\kr)\}$.
The conflict still lacks \emph{all relevant information} to explain this result.
Indeed, in the KB, the fact $\te(\vi, \kd)$ deducing $\lect(\vi)$ makes the conflict $\{\teAs(\vi,\kd), \uc(\kd) \}$ deducing $\ta(\vi)$ \emph{uncertain}, due to the constraint that lecturers cannot be teaching assistants. Thus, using the conflict in the explanation is \emph{insufficient} to assert the non-acceptance of the answer. 
It remains unclear why the answer is possible.
Instead, $\te(\vi, \kd)$ deducing $\lect(\vi)$ should be included in the explanation.
Without knowing the relevant information, it is impossible for the user - especially non-experts in logic - to understand why this is the case.

However, using the argumentation approach will provide a dialogical explanation that is more informative and intuitive.
The idea involves a dialogue between a proponent and opponent, where they exchange logical formulas to a dispute agree.
The proponent aims to prove that the argument in question is acceptable, while the opponent exhaustively challenges the proponent’s moves. The dialogue where the proponent wins represents a proof that the argument in question is accepted. The dialogue whose graphical representation is shown in Figure~\ref{fig:tree-user} proceeds as follows:

1. Suppose that the proponent wants to defend their claim $\rese(\vi)$. They can do so by putting forward an argument, say $A_1$, supported by facts $\te(\vi, \kr)$ and $\gc(\kr)$: 
\begin{align*}
     A_1 :\ & \rese(\vi) (\text{by } r_3) \\
          & \fp(\vi) (\text{by } r_6) \\
          & \te(\vi, \kr), \gc(\kr) (\text{by facts})\\
    \emph{Proponent: }  & \emph{I believe that } \vi \emph{ is a researcher because he is a full professor; and given the fact }\\
    & \emph{that he teaches the KR course and KR is a graduate course.}
\end{align*}

2. The opponent challenges the proponent’s argument by attacking the claim $\rese(\vi)$ with an argument $\ta(\vi)$, say $A_2$, supported by facts $\teAs(\vi, \kd)$ and $\uc(\kd)$:
%
\begin{align*}
    A_2:\ & \ta(\vi) (\text{by } r_4) \\
         & \teAs(\vi, \kd), \uc(\kd) (\text{by facts})\\
  \emph{Opponent: }    &  \vi \emph{ is not possibly a researcher because } \vi \emph{ is a TA given the fact that he is a TA }\\
    & \emph{of the KD course and  KD is an undergraduate course.}
\end{align*}

3. To argue that the opponent's attack is not possible - and to further defend the initial claim $\rese(\vi)$ - the proponent can counter the opponent's argument by providing additional evidence $\lect(\vi)$ supported by a fact $\te(\vi, \kd)$:
%
%with an argument, say $A_9$, with the statement $\lect(\vi)$ supported by a fact $\te(\vi, \kr)$:
%
\begin{align*}
     A_3 :\ & \lect(\vi) (\text{by } r_5) \\
           & \te(\vi, \kd)(\text{by facts})\\
     \emph{Proponent: } & \emph{$\vi$ is not possibly a TA because } \vi \emph{ is certainly a lecturer given the fact that he } \\
    & \emph{teaches the KD course.}
\end{align*}

4. The opponent concedes $\rese(\vi)$ since it has no argument to argue the proponent.
%
\begin{align*}
     \emph{Opponent: } & \emph{I concede  that $\vi$ is a researcher because I have no argument to argue that } \vi \emph{ is } \\
    & \emph{neither a lecturer nor researcher.}
\end{align*}
%
%
The proponent's belief $\rese(\vi)$ is defended successfully, namely, $\rese(\vi)$ that is justified by facts $\{\te(\vi, \kr)$, $\gc(\kr)\}$ that be extended to be the defending set $\{\te(\vi, \kr)$, $\gc(\kr)$, $\te(\vi, \kd)\}$ that can counter-attack every attack.

By the same line of reasoning, the opponent can similarly defend their belief in the contrary statement $\ta(\vi)$ based on the defending set~$\{\teAs(\vi, \kd), \uc(\kd)\}$.
Because different agents can hold contrary claims, the acceptance semantics of the answer can be considered \emph{credulous} rather than \emph{sceptical}. In other words, the answer is deemed \emph{possible} rather than \emph{plausible}. Thus, the derived system can conclude that $\vi$ \emph{is possibly a researcher}.
    
\end{example}


%To overcome the first and second ones, we consider abstract logic to generalize (monotonic and non-monotonic) logics involving reasoning with maximal consistent subsets in Section~\ref{sec:preliminary}, and we show how any such logic can be translated to our argumentation framework with collective attacks in Section~\ref{sec:proof-arg}. The third limit is addressed in Section~\ref{sec:model-exp-dia}, in which we present explanatory dialogue models as dialectical proof procedures to compute and explain the credulous, grounded and sceptical acceptances in the context of argumentation with collective attacks. The explanatory dialogues display either inference steps used to construct arguments or the attack relations between arguments.
%The dialogue models examine the dispute process involving the exchange of arguments (represented as formulas in KBs) between two agents.
%As our main theoretical result, we show soundness and completeness of dialogues wrt argumentation semantics in Section~\ref{sec:soundness} and~\ref{sec:completeness}.

%GIVE AN SIMPLE EXAMPLE OF EXPLANATORY DIALOGUE

The main contributions of this paper are the following:
\begin{itemize}
    \item We propose a \textbf{proof-oriented (logical) argumentation framework with collective attacks} (P-SAF), 
    in which we consider abstract logic to generalize monotonic and non-monotonic logics involving reasoning with maximal consistent subsets, and we show how any such logic can be translated to P-SAFs.
    We also conduct a detailed investigation of how existing argumentation frameworks in the literature can be instantiated as P-SAFs.
    Thus, we demonstrate that the P-SAF framework is sufficiently generic to encode n-ary conflicts and to enable logical reasoning with (inconsistent) KBs. 
    
    


    \item We introduce a \textbf{novel explanatory dialogue model} viewed as a dialectical proof procedure to compute and explain the \textit{credulous}, \textit{grounded} and \textit{sceptical} acceptances in P-SAFs. The dialogues, in this sense, can be regarded as explanations for the acceptances. As our main theoretical result, we prove the soundness and completeness of the dialogue model wrt argumentation semantics.
    
    %This explanatory dialogue model is used to compute and explain the acceptance of a given query wrt inconsistency-tolerant semantics. 
    
    \item This novel explanatory dialogue model provides \textbf{dialogical explanations} for the acceptance of a given query wrt inconsistency-tolerant semantics, and  \textbf{dialogue trees} as graphical representations of the dialogical explanations. Based on these dialogical explanations, our framework assists in understanding the intermediate steps of a reasoning process and enhancing human communication on logical reasoning with inconsistencies.
    \end{itemize}
    


    



    % \item \emph{Logical reasoning aspects}: Starting with abstract logic $(\mL, \cn)$, we propose a \textbf{general} framework for logical reasoning with inconsistencies.
    % In this setting, we provide \textbf{dialogue-based explanations} for inconsistency-tolerant semantics, and  \textbf{dialogue trees} as graphical representations of the explanations. %Our approach can be used as a basis for explaining why an answer is accepted wrt inconsistency-tolerant semantics.
    
    %Based on these explanations, our framework assists in understanding intermediate steps of a reasoning process and enhancing human communication on logical reasoning.
      
%to provide a very flexible environment for logical argumentation

    % \item \emph{Argumentation aspects}: We introduce 
    % \begin{itemize}    
    % \item a \textbf{proof-oriented (logical) argumentation framework} (P-SAF) that is generic enough to encode n-ary conflicts and to logical reason with inconsistencies. 

    
    
    % \item a \textbf{novel explanatory dialogue model} viewed as dialectical proof procedure for P-SAFs under the grounded, preferred (credulous and sceptical) semantics.
    % This explanatory dialogue model is used to compute and explain the acceptance of a given query wrt inconsistency-tolerant semantics.

    %This explanatory dialogue model is used to compute and explain the acceptance of a given query wrt inconsistency-tolerant semantics.
    
    %and for logical reasoning with inconsistencies.
    
    %\item a \textbf{link} between dialogues and argumentation semantics (grounded, credulous and sceptical acceptances) to   
%     \end{itemize}


%Due to space constraints, more details can be found in the companion paper~\url{https://drive.google.com/file/d/1ZvhJRQitkU0w-ZI0rbRoCfaQUoYP9Hvq/view?usp=sharing}.