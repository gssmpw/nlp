\appendix
% \section{Appendix for Section~\ref{sub:trans}}

% \subsection{Proof for Section~\ref{sub:trans}}
% \label{sub:appA}

% \subsubsection{Preliminaries}

% \begin{table}
%   \caption{Supports and conclusions of arguments}
%   \label{tab:arg}
%   \begin{tabular}{cll}
%     \toprule
%     \textbf{Argument} & \textbf{$\Sup(A_i)$} & \textbf{$\Con(A_i)$} \\
%     \midrule
%     $A_0$ & $\{\te(\vi, \kr)\}$ & $\te(\vi, \kr)$ \\
%     $A_1$ & $\{\gc(\kr)\}$ & $\gc(\kr)$ \\
%     $A_2$ & $\{ \gc(\kr), \te(\vi, \kr) \}$ & $\fp(\vi)$ \\
%     $A_3$ & $\{ \gc(\kr), \te(\vi, \kr) \}$ & $\rese(\vi)$\\
%     $A_4$ & $\{\te(\vi, \kd)\}$ & $\te(\vi, \kd)$ \\
%     $A_5$ & $\{\teAs(\vi, \kd)\}$ & $\teAs(\vi, \kd)$\\
%     $A_6$ & $\{ \uc(\kd) \}$ & $\uc(\kd)$\\
%     $A_7$ & $\{ \teAs(\vi, \kd),\ \uc(\kd) \}$ & $\ta(\vi)$\\
%     $A_9$ & $\{\te(\vi, \kr)\}$ & $\lect(\vi)$ \\
%     $A_{10}$ & $\{\te(\vi, \kd)\}$ & $\lect(\vi)$ \\
%     $A_{11}$ & $\{\te(\vi, \kr)\}$ & $\emp(\vi)$ \\
%     $A_{12}$ & $\{\te(\vi, \kd)\}$ & $\emp(\vi)$ \\
%      $A_{13}$ & $\{ \gc(\kr), \te(\vi, \kr)\}$ & $\emp(\vi)$ \\
%     \bottomrule
%   \end{tabular}
% \end{table}

% To prove Theorem~\ref{thm:da-link}, we show a link between repairs of $\mK$ and extensions of $\mAF_{\D}$ by the following propositions. %The propositions are inspired by Proposition~\ref{pro:ab-link} of G-LAFs.




% \section{Appendix of Section~\ref{sec:con-DT}}
% \subsection{Proof for Section~\ref{sec:con-DT}}

% \textbf{Lemma~\ref{lem:potential-arg}}. A potential argument $\mT^{s}$ corresponds to an argument for $\psi$ supported by $H$ as in conventional P-SAF (see Definition~\ref{def:ab-arg}).

% \begin{proof} This lemma is trivially true as a node in a potential argument can be mapped to a node in a conventional P-SAF argument (in Definition~\ref{def:ab-arg}) by dropping the tag $\tT$ and the identifier $\ID$.
% \end{proof}

% \subsection{Dialogue Tree Properties}
% \label{apx:tree-properties}
% We discuss \emph{properties of dialogue trees} mentioned in Section~\cite{sec:con-DT} to formally link dialogues with argumentation semantics, and show how successful dialogues can be constructed in Section~\ref{sec:soundness}.


% During dialogues, agents may choose to attack arguments while these arguments are still under construction. This is not allowed, i.e. arguments fully constructed before being attacked. Then we restrict dialogue trees to be \emph{patient}.

% \begin{definition} A dialogue tree $\mT(\delta)$ is \emph{patient} iff for all nodes $N = (\_, [\f,\_,\_])$ in $\mT(\delta)$ such that $N$ is in a potential argument drawn from $\mT(\delta)$.    
% \end{definition}

% CHECK THIS DEF

% In the case of abstract dialogue trees, the defence set of a dialogue tree may attack itself. To avoid this, we restrict dialogue trees to be \emph{conflict-free}.

% \begin{definition} We call $\mT(\delta)$ \emph{conflict-free} iff $\mDE(\mT(\delta)) \cap \mCU(\mT(\delta)) = \emptyset$.
% \end{definition}

% For short, we refer to \emph{patient dialogue trees}
% simply as \emph{dialogue trees}.

% \loan{Consider the following dialogue - \textbf{Example} for credulous semantics}

% To ensure credulous soundness, all possible nodes of the opponent must be accounted for. But if such a parent node is already in the dialogue tree, then deploying it will not help the opponent win the dialogues. To avoid this, we define a dialogue tree as being non-redundant. 


% \begin{definition}
% \label{def:non-re}
%     % A dialogue tree $\mT(\delta)$ is \emph{non-redundant} iff for any two nodes in $\mT(\delta)$: $N_1 = (\beta, [\f, \tL, id_1])$  and $N_2 = (\beta, [\f, \tL, id_2])$, $\tL \in \{\po, \op\}$, $N_1 \neq N_2$, iff $N_1$ has a child node $N^{\prime}_1$ in an argument drawn from $T_1(\delta)$ and $N_2$ has a child node $N^{\prime}_2$ in an argument drawn from $T_2(\delta)$, then $T_1(\delta) \neq T_2(\delta)$, where $T_1(\delta)$ and $T_2(\delta)$ are sub-trees of $\mT(\delta)$.
%      A dialogue tree $\mT(\delta)$ is \emph{non-redundant} iff for any two nodes $N_1 = (\beta, [\f, \tL, \id_1])$  and $N_2 = (\beta, [\f, \tL, \id_2])$ with $\tL \in \{\po, \op\}$ and $N_1 \neq N_2$, if $N_1$ is in a potential argument $\mT_1^{s}$ and $N_2$ is in a potential argument $\mT_2^{s}$, then $\mT_1^{s} \neq \mT_2^{s}$.
% \end{definition}

% In Definition~\ref{def:non-re}, when comparing two arguments, we compare their respective proof trees. Here, we only consider the formula and the tag of each node in the tree, disregarding the label and identifier of the node. 


% In a dialogue tree, the proponent wins if either the proponent finishes the dialogue with the rules and un-attacked facts (Item 1), or any attacks used by the opponent have been attacked with valid counter attacks (Item 2). We define \emph{admissible dialogue trees} as follows:

% \begin{definition} A dialogue tree $\mT(\delta)$ is \emph{admissible} iff

%     \begin{enumerate}
%      \item for all leaf nodes $N$ in $\mT(\delta)$, $N$  is $(\_, [\f, \po, \_])$, and% is either $(\_, [\nf, \po, \_])$ or $(\_, [\f, \po, \_])$, and 

%      \item if a leaf node $N$ is of the form $(\_, [\tT, \op, \_])$ with $\tT \in \{\f, \nf\}$, then $N$ is in a potential argument and $N$ is properly attacked.
     
     
%      %that contains one child node $N^{\prime}$ of the form $(\alpha, [\tT, \op, \_])$ with $\tT \in \{\f, \nf\}$, such that $N^{\prime} \neq N$ and $N^{\prime}$ is attacked.
      
%       %\item if a leaf node $N$ is of the form $(\_, [\_, \op, \_])$, then $N$ is in a potential argument that contains one child node $N^{\prime}$ of the form $(\alpha, [\tT, \op, \_])$ with $\tT \in \{\f, \nf\}$, such that $N^{\prime} \neq N$ and $N^{\prime}$ is attacked.
      
      
%       %there is another node $N^{\prime \prime}$ in $\mT(\delta)$ of the form $(\alpha, [\f, \op, \_])$, $N^{\prime \prime} \neq N$, and $N^{\prime \prime}$ is attacked.
      
%      %\item if a leaf node $N$ is of the form $(\_, [\_, \op, \_])$, then $N$ is in a potential argument that contains one child node $N^{\prime}$ of the form $(\alpha, [\f, \op, \_])$ such that there is another node $N^{\prime \prime}$ in $\mT(\delta)$ of the form $(\alpha, [\f, \op, \_])$, $N^{\prime \prime} \neq N$, and $N^{\prime \prime}$ is attacked.   
%     \end{enumerate}
    
% \end{definition}



% % 1. We determine the properties of a dialogue tree in the case of admissible semantics.

% % - Because one is an admissible extension, then it is the preferred extension. Thus we can specify the form of a dialogue tree in the case of preferred extension. This implies to conditions for credulous acceptance. 

% % % \begin{definition} \label{def: t-pre} A dialogue tree $\mT(\delta)$ is \emph{admissible} if it is admissible, conflict-free and non-redundant. %no utterance of the opponent is repeated in the dialogue tree 

% % % We say that $\mT(\delta)$ is preferred if $\mT(\delta)$ is admissible.
% % % \end{definition}


% % 2. Definition determines the properties of a dialogue tree in the case of grounded semantics

% % % \begin{definition} \label{def:t-ground} A dialogue tree $\mT(\delta)$ is \emph{grounded} if it is admissible and finite.

% % % \end{definition}

% % 3. The properties of a dialogue tree in the case of sceptical semantics

% % % \begin{definition} \label{def:t-scep} A dialogue tree $\mT(\delta)$ is sceptical iff for any node $N$ labelled $\op$ in the tree, there does not exist an admissible tree with root $N$.
    
% % % \end{definition}

% %Since every admissible set of arguments is necessarily contained in a preferred set~\cite{Dung95}, and every preferred set is admissible by definition, a dialogue tree is preferred-successful iff it is admissible-successful.

% % The following definition presents different types of dialogue trees w.r.t argumentation semantics.

% % \begin{definition} The dialogue tree $\mT(\delta)$ drawn from a dialogue $\delta$ is called
% %   \begin{itemize} 
% %       \item  \emph{admissible} iff it is admissible, conflict-free and non-redundant.

% %      \item  \emph{preferred} iff it is admissible.    

% %       \item  \emph{grounded} iff it admissible and conflict-free.
    
% %       \item  \emph{sceptical-accepted} iff it is admissible and none of the opponent arguments drawn from $\mT(\delta)$ belongs to an admissible set of arguments in $ \mAF_ \delta$ drawn from $\delta$.
% %  \end{itemize}
% % \end{definition}

\section{Preliminaries}
\label{app:pre}

To prove the soundness and complete results, we sketch out \textbf{a general strategy} as follows:

 \begin{enumerate}
    \item Our proof starts with the observation that a dialogue $\delta$ for a formula $\phi$ can be seen as a collection of several (independent) focused sub-dialogues $\delta_1, \ldots, \delta_n$. 
    The dialogue tree $\mT(\delta_i)$ drawn from $\delta_i$ is a subtree of $\mT(\delta)$ drawn from the sub-dialogue $\delta$, and it corresponds to the \emph{abstract dialogue tree} that has root an argument with conclusion $\phi$. 
    (The notion of abstract dialogue tree can be found in~\cite{loanho_2024}).
   Thus it is necessary to consider a \emph{correspondence principle} that links
    dialogue trees to abstract dialogue trees. The materials for this step can be found in Section~~\ref{app:partition} and~\ref{app:DT-AbstractDT}.
   
    \item The correspondence principle allows to utilize the results of abstract dialogue trees in ~[\cite{loanho_2024}, Corollary 1] to prove the soundness results. We extend Corollary 1 of~\cite{loanho_2024} to prove the completeness results.
\end{enumerate}

The proof of the soundness and completeness results depends on some notions and results that we describe next. 


\begin{notation} 
\label{not:arg}
Let $\mK$ be a KB, $\mX \subseteq \mK$ be a set of facts and $\mS \subseteq \Arg_{\mK}$ be a set of arguments induced from $\mK$. Then,
\begin{itemize}
    \item $\Args(\mX)  =  \{A \in \Arg_{\mK} \mid \Sup(A)\subseteq \mX \}$ are the set of \emph{arguments generated by} $\mX$,

   % \item $\Cons(\mS)  =  \underset{A \in \mS}{\bigcup} \Con(A)$ are the set of \emph{conclusions} of arguments in $\mS$,

    \item $\base(\mS)  =  \underset{A\in\mS}{\bigcup}\Sup(A)$ are the set of \emph{supports} of arguments in $\mS$,

    \item An argument $B$ is a \emph{subargument} of argument $A$ iff $\Sup(B)\subseteq \Sup(A)$. We denote the set of subarguments of $A$ as   $\subs(A)$.
\end{itemize}
\end{notation}

\subsection{Abstract Dialogue Trees and Abstract Dialogue Forests}
\label{def:abstract-dia-forest}
 We observe that a formula $\phi$ can have many arguments whose conclusion is $\phi$. 
 Thus a dialogue tree with root $\phi$ can correspond to one, none, or multiple \emph{abstract dialogue trees}, one for each argument for $\phi$.
 We call this set of abstract dialogue trees an \emph{abstract dialogue forest}.
 The following one presents a definition of abstract dialogue forests
 and reproduces a definition of abstract dialogue trees (analogous to Definition 8 in~\cite{loanho_2024}).
 Formally:
 

\begin{definition} [Abstract dialogue forests]
\label{def:abstract-forests}
Let $ \mAF_ \delta  =  (\Arg_ \delta, \Att_ \delta)$ be the P-SAF drawn from a dialogue $D(\phi)  =  \delta$. An \emph{abstract dialogue forest} (obtained from $ \mAF_ \delta$) for $\phi$ is a set of \emph{abstract dialogue trees}, written $\mF_{\G}(\phi)  =  \{\mT^{1}_{\G}, \ldots, \mT^{h}_{\G}\}$, such that: For each abstract dialogue tree $\mT^{j}_{\G}$ ($j = 1, \ldots, h$), 

\begin{itemize}
    \item the root of $\mT^{j}_{\G}$ is the proponent argument (in $\Arg_ \delta$) with the conclusion $\phi$,

    \item if a node $A$ in $\mT^{j}_{\G}$ is a proponent argument (in $\Arg_ \delta$), then all its children (possibly none) are opponent arguments (in $\Arg_ \delta$) that attack $A$
    
    % and attacked, then $A$ has children holding opponent arguments  that attack $A$. 

     \item if a node $A$ in $\mT^{j}_{\G}$ is an opponent argument (in $\Arg_ \delta$), then exactly one of the following is true: (1) $A$ has exactly one child, and this child is a proponent argument  (in $\Arg_ \delta$) that attacks $A$; (2) $A$ has more than one child, and all these children are proponent argument  (in $\Arg_ \delta$) that collectively attacks $A$.
     
     % and attacked, then $A$ has (1) either exactly one child holding a proponent argument (in $\Arg_ \delta$) that attacks $A$; (2) children holding proponent arguments (in $\Arg_ \delta$) that collectively attacks $A$.
\end{itemize}   
\end{definition}

\begin{remark} We call the abstract dialogue tree that has root an argument with conclusion $\phi$ an abstract dialogue tree for $\phi$.
\end{remark}

    %the argument labelling $\po$ node ($\op$ node, respectively) a proponent argument (an opponent argument);





 Fix an abstract dialogue forest $\mF_{\G}(\phi)  =  \{\mT^{1}_{\G}, \ldots, \mT^{h}_{\G}\}$. For such abstract dialogue tree $\mT^{j}_ {\G}$ ($i = 1 , \ldots, h$), we adopt the following conventions:
\begin{itemize}
    \item Let $\mB_1$ be the set of all proponent arguments in $\mT^{j}_{\G}$. $\mDE(\mT^{j}_{\G}) = \{\alpha \mid  \forall A \in \mB_1, \alpha \in \Sup(A) \} \subseteq \mF$ is the \emph{defence set} of $\mT^{j}_{\G}$, i.e. the set of facts in the support of the arguments in $\mB_1$.
   
    \item Similarly, let $\mB_2$ be the set of all opponent arguments in $\mT^{j}_{\G}$. $\mCU(\mT^{j}_{\G}) = \{\beta \mid  \forall B \in \mB_2, \beta \in \Sup(B) \} \subseteq \mF$ is the \emph{culprit set} of $\mT^{j}_{\G}$, i.e. the set of facts in support of the arguments in $\mB_2$. 

    %We write $\mDE(\mT^{a}_{i})$ ($\mCU(\mT^{a}_{i})$ resp.) for the defence set of $\mT^{a}_i$ (for the culprit set of $\mT^{a}_i$). This notation differs from that of Definition~\ref{} in that we define them in terms of arguments.

    % denoted by $\mDE(\mT^{a}_{i})$

    % denoted by $\mCU(\mT^{a}_{i})$
\end{itemize}

 We reproduce a definition of \emph{admissible abstract dialogue trees} given in~\cite{loanho_2024}. This notion will be needed for Section~\ref{app:DT-AbstractDT}.

\begin{definition} [Admissible abstract dialogue trees]
\label{def:adm-ab-dt}
An abstract dialogue tree for $\phi$ is said to \emph{admissible} iff the proponent wins and no argument labels both a proponent and an opponent node. 
\end{definition}

Intuitively, in an abstract dialogue tree, a proponent wins if either the tree ends with arguments labelled by proponent nodes or every argument labelling an opponent node has a child.









 % Following the notation used in Notation~\ref{not:arg}, then
 % \begin{itemize}
 %     \item  $\base(de(\mT^{a}_{i}))  =  \underset{A\in \mDE(\mT^{a}_{i})} {\bigcup}\Sup(A)$ are the set of supports of all arguments in $de(\mT^{a}_{i})$,

 %     \item $\base(cu(\mT^{a}_{i}))  =  \underset{A\in \mCU(\mT^{a}_{i})}{\bigcup}\Sup(A)$ are the set of the supports of all arguments in $cu(\mT^{a}_{i})$.
 % \end{itemize}

%If a dialogue tree whose claim is supported by only a single potential argument, it is said to be a \emph{focused}. In focused dialogue trees, no alternative ways to support or defend the claim are considered simultaneously; an opponent argument is attacked by either a single proponent argument or a set of collective proponent arguments; a proponent argument can be attacked by many opponent arguments.

%We introduce a notion of \emph{focused dialogue trees}. 

% \begin{definition}
% \label{def:t-focused}
% A dialogue tree $\mT(\delta)$ is \emph{focused} iff
% \begin{enumerate}
%     \item all the immediate children of the root node have the same identifier (that is, are part of a single utterance);
    
%     \item all the children labelled~$\po$ of each potential argument labelled~$\op$
%         have the same identifier (that is, are part of a single utterance)
% \end{enumerate}
% \end{definition}

%In the above definition, we call child of a potential argument a node that is child of any of the nodes of the potential argument.



% \begin{example} Consider a query $q = A(a) $ to a KB $\mK_3 = (\mR_3, \mC_3, \mF_3)$ where 
% \begin{align*}
%     \mR_3 = & \{r_1: C(x) \land B(x) \rightarrow A(x),\ r_2: D(x) \rightarrow A(x) \} \\
%     \mC_3 = & \{ D(x) \land C(x) \rightarrow \bot ,\ E(x) \land C(x) \rightarrow \bot \} \\
%     \mF_3 = & \{B(a) , C(a), D(a), E(a) \}
% \end{align*}
%   Figure~\ref{fig:non-foc-tree} (Left) shows a non-focused dialogue tree drawn for a dialogue $D(A(a)) = \delta$.  Figure~\ref{fig:non-foc-tree}(Right) shows a focused dialogue tree $\mT(\delta_1)$ drawn for a focused sub-dialogue $\delta_1$ of $\delta$. This tree is the sub-tree of $\mT(\delta)$.
% \end{example}

% \begin{figure}
% \centering
% \begin{tikzpicture}
%     \node (dt) at (0,0) {\includegraphics[scale=0.6]{Picture/non-foc-tree.pdf}};
%     \node (d1) at (6, 0) {\includegraphics[scale=0.55]{Picture/focused-dia-tree.pdf}};
% \end{tikzpicture}
% \caption{
% Left: A non-focused dialogue tree.
% Right: A focused dialogue tree $\mT(\delta_1)$.
% }
% \label{fig:non-foc-tree}
% \end{figure}

%From the above observation, each tree in the set of abstract dialogue trees drawn from a dialogue $\delta$ is a top-portion of the focused dialogue tree drawn from $\delta$. 
\subsection{Partitioning a Dialogue Tree into Focused Substrees}
\label{app:partition}
This section shows how to partition a dialogue tree $\mT(\delta)$ into focused subtrees of $\mT(\delta)$. We will need this result to prove soundness and completeness results.

We observe that a dialogue $D(\phi) = \delta$, from which the dialogue tree $\mT(\delta)$ is drawn, may contain one, none, or multiple \emph{focused sub-dialogues} $\delta_i$ of $\delta$. Each dialogue tree $\mT(\delta_i)$ drawn from the focused sub-dialogue $\delta_i$ is a subtree of $\mT(\delta)$ and focused. This is proven in the following lemma.

\begin{lemma}
     \label{lem:DT-subT} 
     Let $\mT(\delta)$ be a dialogue tree (with root $\phi$) drawn from a dialogue $D(\phi) = \delta$. 
     Every focused subtree of $\mT(\delta)$ with root~$\phi$ is the dialogue tree
     drawn from a focused sub-dialogue $\delta_i$ of $\delta$.
\end{lemma}

\begin{proof}
All the subtrees considered in this proof are assumed to have root $\phi$.
The proof proceeds as follows:
\begin{enumerate}  
    \item First, we construct the set of focused dialogue subtrees of $\mT(\delta)$
    \item Second, we show that each focused subtree $\mT(\delta_i)$ of $\mT(\delta)$ is drawn from a focused sub-dialogue $\delta_i$ of $\delta$.
\end{enumerate}

Let $\mT(\delta_1), \ldots, \mT(\delta_m)$ be all the focused subtrees contained
in $\mT(\delta)$ with root~$\phi$. Each focused subtree is obtained in the following way:
First, choose a single utterance at the root and discard the subtrees corresponding to the other utterances at the root. Then proceed (depth first) and for each potential argument
labelled $\op$ with children labelled $\po$, choose a single identifier and select
among them those (and only those) with that identifier; discard the other children labelled $\po$ (which have a different identifier) and the corresponding subtrees.
By Definition~\ref{def:t-focused}, every focused subtree can be obtained in this way.


2. We next prove (2) by showing the construction of the focused sub-dialogue $\delta_i$ that draws $\mT(\delta_i)$.

 Given $m$ dialogue trees $\mT(\delta_1), \ldots, \mT(\delta_m)$ constructed from $\mT(\delta)$, the focused sub-dialogue $\delta_i$ ($1 \leq i \leq m$) drawing the dialogue tree $\mT(\delta_i)$ is constructed as follows:

\begin{itemize}
    \item $\delta_i$ is initialised to empty;
    \item for each node $\psi, [ \_ , \_ , \id]) = N $ in $\mT(\delta_i)$,
    \begin{itemize}
        \item if $u_{id} = (\_ ,\ \tg,\ \_ ,\ \id )$ is in $\delta$ but not in $\delta_i$, then add $u_{id}$ to $\delta_i$;
        \item let $u_{\tg}$ be the utterance in $\delta$; if $u_{\tg}$ is the target utterance of $u_{\id}$, then add $u_{\tg}$ to $\delta_i$;
    \end{itemize}
    \item Sort $\delta_i$ in the order of utterances $ID$.
\end{itemize}

It is easy to see that each $\delta_i$ constructed as above is a focused sub-dialogue of $\delta$ (by the definition of focused sub-dialogues in Definition~\ref{def:dia-tree-DLAF}), and $\mT(\delta_i)$ is drawn from $\delta_i$. Thus (2) is proved.
\end{proof}


\subsection{Transformation from Dialogue Trees into Abstract Dialogue Trees}
\label{app:DT-AbstractDT}

% focused and last-word = admissible
The following \emph{correspondence principle} allows to
translate dialogue trees into abstract dialogue trees and vice versa.


\begin{remark}
Recall that a dialogue tree for $\phi$
has root $\phi$, while an abstract dialogue tree for $\phi$
has root an argument for $\phi$.
\end{remark}

\begin{theorem} [Correspondence principle] \label{thm:def-abs}
Let $\phi \in \mL$ be a formula. Then:

\begin{enumerate}
    \item  For every defensive and non-redudant dialogue tree $\mT(\delta)$ for $\phi$, there exists an admissible abstract dialogue tree $\mT_{\G}$ for $\phi$ such that $\mDE(\mT_{\G}) \subseteq  \mDE(\mT(\delta))$ and $ \mCU(\mT_{\G})  \subseteq  \mCU(\mT(\delta))$.

    \item For every admisslbe abstract dialogue tree $\mT_{\G}$ for $\phi$, there exists a defensive and non-redundant dialogue tree $\mT(\delta)$ for $\phi$ such that $\mDE(\mT(\delta)) \subseteq \mDE(\mT_{\G})$ and $\mCU(\mT(\delta)) \subseteq \mCU(\mT_{\G})$.
\end{enumerate}
\end{theorem}

\begin{proof}
We prove the theorem by transforming dialogue trees into abstract dialogue trees and vice versa.

\textbf{1. The transformation from dialogue trees into abstract dialogue trees}

Given a defensive and non-redundant dialogue tree $\mT(\delta)$ with root $\phi$, its equivalent abstract dialogue tree $\mT_{\G}$ for an argument for $\phi$ in $\mT_{\G}^{1}, \ldots, \mT_{\G}^{h}$ is constructed inductively as follows:

\begin{enumerate}
    \item Modify $\mT(\delta)$ by adding a new \emph{flag} ($0$ or $1$) to  nodes in $\mT(\delta)$ and initialise $0$ for all nodes; a node looks like $(\_, [\_, \_, \_])-0$. The obtained tree is $\mT^{\prime}(\delta)$.

    \item $\mT_{\G}$ is $\mT_{\G}^{h}$ in the sequence $\mT_{\G}^{1}, \ldots, \mT_{\G}^{h}$ constructed inductively as follows:
    \begin{enumerate} [a)] %[label=(\alph*)]
    
    \item Let $A$ be the potential argument drawn from $\mT(\delta)$ that contains root $\phi$. $\mT_{\G}^{1}$ contains exactly one node that holds $A$ and is labelled by $\po$. Set the nodes in $\mT^{\prime}(\delta)$ that are in $A$ to $1$. The obtained tree is $\mT^{\prime}_{1}(\delta)$.


    \item Let $\mT_{\G}^{k}$ be the $k$-th tree, with $1 \leq k \leq h$. $\mT^{k+1}_{\G}$ is expanded from $\mT^{k}_{\G}$ by adding nodes $(\tL :\ B_j)$ with $\tL \in \{\po, \op\}$.
    
    For each node $(\tL :\ B_j)$, $B_j$ is a potential argument drawn from $\mT^{\prime}_{k}(\delta)$, which is a child of $C$ - another potential argument drawn from $\mT^{\prime}_{k}(\delta)$, such that:

    \begin{itemize}
        \item there is at least one node in $B_j$ that is assigned $0$;
        \item the root of $B_j$ has a parent node $t$ in $\mT^{\prime}_{k}(\delta)$ such that the flag of $t$  is $1$ and $t$ is in $C$;
        \item if the root of $B_j$ is labelled by $\po$, then $\tL$ is $\po$. Otherwise, $\tL$ is $\op$.
        \item set all nodes in $\mT^{\prime}_{k}(\delta)$ that are also in $B_j$ to 1. The obtained tree is $\mT^{\prime}_{k+1}(\delta)$.
    \end{itemize}
    \end{enumerate}
    \item $h$ is the smallest index s.t there is no node in $\mT^{\prime}_{h}(\delta)$ where its flag is $0$.
\end{enumerate}

$\mT_{\G}$ is constructed as follows:
%that is constructed above is an abstract dialogue tree as follows:

\begin{itemize}
    \item Every node of $\mT_{\G} = \mT_{\G}^{h}$ includes a potential argument. For each potential argument, there is a unique node in $\mT_{\G}$. Each node is labelled $\po$ or $\op$ as potential arguments drawn from $\mT(\delta)$ are labelled either $\po$ or $\op$. 

    \item  The root of $\mT_{\G}$ includes the potential argument for $\phi$ of the dialogue and labelled $\po$ by constructing $\mT(\delta)$.

    \item  Since $\mT(\delta)$ is defensive, by Definition~\ref{def:defensive-tree}, it is focused and patient.
    Thus there is only one way of attacking a potential argument labelled by $\op$.  
    Since $\mT(\delta)$ is defensive, by Definition~\ref{def:defensive-tree}, it is last-word.
    Then there is no un-attacked (potential) argument labelled by $\op$.
    From the above, it follows that every $\op$ node has exactly one $\po$ node as its child.

    \item Since $\mT(\delta)$ is non-redundant, by Definition~\ref{def:non-re}, no potential argument labels $\op$ and $\po$.    
    %Since $\mT(\delta)$ is last-word, all leaves are nodes labelled by $\po$, namely $\po$ wins.
\end{itemize}

Recall that, in $\mT_{\G}$, the potential arguments labelling $\po$ ($\op$, respectively) are called proponent arguments (opponent arguments, respectively). 

It can be seen that $\mT_{\G}$ has the following properties:
\begin{itemize}
    \item the root is the proponent argument for $\phi$.
    \item the $\op$ node has either exactly one child holding one proponent argument that attacks it or children holding the proponent arguments that collectively attack it.
    \item all leaves are nodes labelled by $\po$, namely $\po$ wins and there is no node labelled by both $\po$ and $\op$.
\end{itemize}

It follows immediately that $\mT_{\G}$ is an admissible abstract dialogue tree (by the definition of abstract dialogue trees in Definition~\ref{def:abstract-forests} and Definition~\ref{def:adm-ab-dt}).


Since $\mT_{\G}$ contains the same potential arguments as $\mT(\delta)$ and the arguments have the same $\po / \op$ labelling in both $\mT_{\G}$ and $\mT(\delta)$, we have $\mDE(\mT_{\G})  =  \mDE(\mT(\delta))$ and $ \mCU(\mT_{\G})  =  \mCU(\mT(\delta))$.

\textbf{2. The transformation from abstract dialogue trees into dialogue trees}

We first need to introduce some new concepts that together constitute the dialogue tree.

\begin{definition} [Support trees]
A \emph{support tree} of a formula $\alpha$ is defined as follows:
\begin{enumerate}
    \item The root is a proponent node labelled by $\alpha$.
    \item Let $N$ be a proponent node labelled by $\sigma$. If $\sigma$ is a fact, then either $N$ has no children, or $N$ has children that are opponent nodes labelled by $\beta_k$, $k = 1, \ldots, n$ such that $\{ \beta_k \} \cup \{ \sigma \}$ is inconsistent.
     If $\sigma$ is a non-fact, then one of the following holds:
     \begin{itemize}
         \item either (1) $N$ has children that are proponent nodes labelling $\omega_l$, $l = 1, \dots, m$, such that $\sigma \in \cn(\{ \omega_l \})$,
         \item or (2) $N$ has children that are opponent nodes labelling $\beta_k$, $k = 1, \ldots, n$, such that $\{ \beta_k \} \cup \{ \sigma \}$ is inconsistent,
         \item or both (1) and (2) hold.
     \end{itemize}
\end{enumerate}   
\end{definition}
%It is easy to see that finite support trees of $\alpha$ correspond to derivable trees of $\alpha$.

\begin{definition} [Context trees]
A \emph{context tree} of a formula $\alpha$ is defined as follows:
\begin{enumerate}
    \item The root is an opponent node labelled by $\alpha$.
    \item Let $N$ be an opponent node labelled by $\sigma$. If $\sigma$ is a fact, then $N$ has children that are proponent nodes labelled by $\beta_k$, with $k = 1, \ldots, n$, such that $\{ \beta_k \} \cup \{ \sigma \}$ is inconsistent and the children have the same identify ~\footnote{The condition of "children having the same identify" ensure that a potential argument is attacked by exactly one potential argument if $k = 1$ or collectively attacked by one set of potential arguments if $k > 1$.}.
     If $\sigma$ is a non-fact, then one of the following holds:
     \begin{itemize}
         \item either (1) $N$ has children that are opponent nodes labelled by $\omega_l$, with $l = 1, \dots, m$, such that $\sigma \in \cn(\{ \omega_l \})$,
         \item or (2) $N$ has children that are proponent nodes labelling $\beta_k$, with $k = 1, \ldots, n$, such that $\{ \beta_k \} \cup \{ \sigma \}$ is inconsistent and the children have the same identify,
         %either $N$ has exactly one child that is a proponent node labelling $\beta$ such that $\{\beta \cup \sigma \}$ is inconsistent,
         %or 
         \item or both (1) and (2) hold.
     \end{itemize}
%For such a context tree, a \emph{context path} is an opponent path from a root labelling the opponent node to .
\end{enumerate}
   
\end{definition}
%Let $\alpha$ be a formula and $\mT$ be a context tree of $\alpha$. 
%Let $D = S_0, \ldots, S_k$ be a tree-derivation of $\alpha$ from a set of facts-premises $F \subseteq \mK$, in which ....

%We say that a context path $p = N_0, \ldots, N_k$ from the root $N_0$ of $\mT$ to $N_i$, with $0 \leq i \leq k$ corresponds to a prefix of $D$.
%It is easy to see that the following property holds:

% \begin{lemma}
% \begin{enumerate}
%     \item[] 
%     \item Let $A$ be and argument for $\alpha$ and $B$ be the set of facts in a context tree $\mT$ of $\alpha$. Then $B$ is a context set of the supports of $A$.

%     \item Let $B$ be a context set of supports of $A$. Then, there is a context tree $\mT$ of $\alpha$ such that the culprit set in the tree is a subset of $B$.
% \end{enumerate}
% \end{lemma}

% \begin{proof}

% 1. Let $F$ be a set of facts in support of an argument $A$ for $\alpha$. We want to prove that $F \cap B \neq \{ \}$. There is a tree-derivation $D  = S_0, \ldots, S_k$ of $\alpha$ such that $S_k \subseteq F$ and $S_0 = \{ \alpha \}$.
% Then, there exists a node $N$ in $\mT$ such that the context path from the root to the parent node of $N$ corresponds to a prefix of $D$ and $N$ is a proponent node. 
% It follows that the formula selected at the parent of $N$ belongs to $S_m$.
% That means $F \cap B \neq \{ \}$.

% 2. $\mT$ is defined as follows:

%  The root is an opponent node labelled by $\{ \alpha \}$.

%  Let $N$ be an opponent node labelled by $B$
% If $B \neq \{ \}$, select a sentence $\phi \in B$.
% If $\phi$ is a fact not belonging to B, then $N$ has children that are opponent nodes labelled by $B - \{\phi\}$.
% If $\phi$ is a fact belonging to $B$, then $N$ has exactly one child that is a proponent node labelled by $\beta$.
% If $\phi$ is not a fact and there exists no inference rule, then
% N is a terminal node. Otherwise, the children of $N$ are a set of opponent nodes labelled by the sets of sentences  where S  (There being one such child for each such inference rule.)
% \end{proof}

Now we prove the translation from abstract dialogue trees to dialogue trees.

Given an abstract dialogue tree $\mT_{\G}$ for $\phi$, its equivalent focused subtree $\mT(\delta_i)$ of a dialogue tree $\mT(\delta)$ for $\phi$ is constructed inductively as follows:

%Let $\mT_{\G}$ be an abstract dialogue tree for $\phi$.
Let $\mDE(\mT_{\G})$ and $\mCU(\mT_{\G})$ be the defence set and the set of culprits of $\mT_{\G}$ , respectively.

For each $\alpha \in \mCU(\mT_{\G})$, let $arg(\beta_k)$, with $k = 1, \ldots, n$, be a set of arguments for $\beta_k$ labelling nodes in $\mT_{\G}$ such that $\{ \beta_k \} \cup \{\alpha\}$ is inconsistent.
If $k = 1$, then there exists a single argument for $\beta$ that attacks an argument including $\alpha$. We say that an argument including $\alpha$ is an argument whose conclusion or support includes $\alpha$.
If $ k > 1$, then there exists a set of arguments for $\beta_k$ that collectively attacks an argument including $\alpha$.
%The former case represents a single argument attacking another, while the latter represents a set of arguments collectively attacking an argument.

For each $\alpha \in \mDE(\mT_{\G})$, let $B^{\alpha}_{k}$ be the set of facts in the support of arguments for $\beta_k$ in $\mT_{\G}$ such that the arguments for $\beta_k$ attack an argument including $\alpha$.
Clearly, there exists an argument for $\beta$ that attacks an argument including $\alpha$ if $k = 1$ and there exist arguments for $\beta_k$ that attack or collectively attack an argument including $\alpha$ if $ k > 1$.
% Clearly, $B_{\alpha}$ is a context set in the supports of each argument for $\beta_k$. 

We construct inductively the sequence of trees $\mT^{1}(\delta_i), \ldots, \mT^{h}(\delta_i)$ as follows:
%$\mT(\delta_i)$ is $\mT^{h}(\delta_i)$ in the sequence of $\mT^{1}(\delta_i), \ldots, \mT^{h}(\delta_i)$ constructed inductively as follows:
\begin{enumerate}
    \item $\mT^{1}(\delta_i)$ is a support tree of $\phi$ corresponding to the argument labelling the root of $\mT_{\G}$.

    \item  %EVEN
    Let $j = 2n$ such that the non-terminal nodes in the frontier of  $\mT^{j}(\delta_i)$ are opponent nodes labelled by a set of formulas $\beta_k$ where $\{\beta_k \} \cup \{ \alpha \}$ is inconsistent and  $\alpha \in \mDE(\mT_{\G})$.
    Expand each such node by a context tree of $\beta_k$ wrt $B^{\alpha}_{k}$. The obtained tree is $\mT^{j+1}(\delta_i)$. 

   
    \item  %ODD
    Let $j = 2n + 1$ such that the non-terminal nodes in the frontier of $\mT^{j}(\delta_i)$ are proponent nodes labelled by a set of formula $\beta_k$, where $\{\beta_k \} \cup \{ \alpha \}$ is inconsistent and $\alpha \in \mCU(\mT_{\G})$.
    Expand each such node by a support tree of $\beta_k$ corresponding to the argument for $\beta_k$. The obtained tree is $\mT^{j+1}(\delta_i)$.

    \item Define $\mT(\delta_i)$ to be the limit of $\mT^{j}(\delta_i)$.  
\end{enumerate}

It follows immediately that $\mT(\delta_i)$ is a dialogue tree for $\phi$ whose defence set is a subset of the defence set of $\mT_{\G}$ and whose the culprit set is a subset of the culprit set of $\mT_{\G}$.
Since $\mT_{\G}$ is admissible, by Defintion~\ref{def:adm-ab-dt}, the proponent wins, namely either the tree ends with arguments labelled by proponent nodes or every argument labelled by an opponent node has a child. By Definition~\ref{def:defensive-tree}, $\mT(\delta_i)$ is defensive.
Since $\mT_{\G}$ is admissible,  by Defintion~\ref{def:adm-ab-dt}, $\mT_{\G}$ has no argument labelling both a proponent and an opponent node. By Definition~\ref{def:non-re}, $\mT(\delta_i)$ is non-redundant.

\end{proof}

\subsection{Notions and Results of Acceptance of an Argument from Its Abstract Dialogue Trees}

For reader's convenience, we reproduce here definitions and results for abstract dialogues for $\phi$ that can also be found in~\cite{loanho_2024}. 
We use a similar argument for focused sub-dialogues of a dialogue $\delta$ for $\phi$. 
In fact, the definitions and the results we reproduce here are essentially the same with abstract dialogues replaced by focused sub-dialogues of $\delta$ for $\phi$. This replacement is because an abstract dialogue for $\phi$ can be seen as a focused sub-dialogue of a dialogue $\delta$ for $\phi$. This follows immediately from the results in Lemma~\ref{lem:DT-subT} 
(i.e., showing a dialogue tree drawn from a dialogue $\delta$ for $\phi$ can be divided into focused sub-trees drawn from focused sub-dialogues of $\delta$ for $\phi$)
and in Theorem~\ref{thm:def-abs} 
(i.e., showing each such focused subtrees corresponds with an abstract dialogue tree drawn from an abstract dialogue).

\begin{definition} [Analogous to Definition 9 in~\cite{loanho_2024}]
\label{def:analogous-Def9}
Let $\mT_{\G}$ be the abstract dialogue tree drawn from a focused sub-dialogue $\delta^ \prime$ of a dialogue for $\phi$. The focused sub-dialogue $\delta^ \prime$ is called
    \begin{itemize}
        \item \emph{admissible-successful} iff $\mT_{\G}$ is admissible;

        \item \emph{preferred-successful} iff it is admissible-successful;
        
        \item  \emph{grounded-successful} iff $\mT_{\G}$ is admissible and finite;

        \item  \emph{sceptical-successful} iff $\mT_{\G}$ is admissible and for no opponent node in it, there exists an admissible dialogue tree for the argument labelling an opponent node.     
    \end{itemize} 
\end{definition}

% \begin{theorem} [Analogous to Theorem 2 in~\cite{loanho_2024}]

% Let $\phi$ be a formula of $\mL$ and $\delta^ \prime$ a focused sub-dialogue (for $\phi$) of a dialogue $\delta$ for $\phi$. Then an argument for $\phi$ is
% \begin{itemize}
%     \item credulously accepted in some admissible extension if $\delta^ \prime$ is admissible-succesful;

%      \item credulously accepted in some preferred extension if $\delta^ \prime$ is preferred-successful;
     
%     \item  groundedly accepted  if $\delta^ \prime$ grounded-successful;
    
%     \item sceptical accepted in all preferred extensions if $\delta^ \prime$ is sceptical-successful.
% \end{itemize}
% \end{theorem}

\begin{corollary} [Analogous to Corollary 1 in~\cite{loanho_2024}]
\label{cor:analogous-Cor1}
Let $\mK$ be a KB, $\phi \in \mL$ a formula and $\mAF_{\mK}$ be the corresponding P-SAF of $\mK$. Then, $\phi$ is
\begin{itemize}
    \item   credulously accepted in some admissible/preferred extension of $\mAF_{\mK}$ if there is a focused sub-dialogue $\delta^{\prime}$ of a dialogue for $\phi$ such that $\delta^{\prime}$ is admissible/preferred-successful;
    
    \item groundedly accepted in a grounded extension of $\mAF_{\mK}$ if there is a focused sub-dialogue $\delta^{\prime}$ of a dialogue for $\phi$  such that $\delta^{\prime}$ is grounded-successful;

    \item  sceptically accepted in all preferred extensions of $\mAF_{\mK}$ if there is a focused sub-dialogue $\delta^{\prime}$ of a dialogue for $\phi$ such that $\delta^{\prime}$ is sceptical-successful.   
\end{itemize} 
\end{corollary}


\section{Proofs for Section~\ref{sec:soundness}} 
\label{app:proof-soundness}

We follow the general strategy to prove Theorem~\ref{thm:adm},~\ref{thm:prf-stb}, ~\ref{thm:ground} and~\ref{thm:scep}. In particular, we:

\begin{enumerate}
    \item partition a dialogue tree for $\phi$ drawn from a dialogue $\delta$ into subtrees for $\phi$ drawn from the focused sub-dialogue of $\delta$ (by Lemma~\ref{lem:DT-subT}),
    \item use the correspondence principle to transfer each subtree for $\phi$ into an abstract dialogue tree for $\phi$ (by Theorem~\ref{thm:def-abs}),
    \item apply Definition~\ref{def:analogous-Def9} (analogous to Definition 9 in~\cite{loanho_2024} and Corollary~\ref{cor:analogous-Cor1} (analogous to Corollary 1 in~\cite{loanho_2024}) for abstract dialogue trees to prove the soundness results. 
\end{enumerate}

%The proof of our soundness results depends on some notions and results in Theorem 2 of~\cite{loanho_2024} that we describe next. The material in this section is also needed for the results concerning the proof of completeness results.

% \begin{itemize}
%     \item We show the connection between a dialogue tree for a formula $\phi$ and a set of \emph{abstract dialogue trees} for the arguments for $\phi$ that are...
    
%     \item We utilise the results of the abstract dialogue tree in [~\cite{loanho_2024}, Theorem 2] to prove the soundness results.
% \end{itemize}



 % By showing the relation between a focused and last-word dialogue tree and an abstract dialogue forest, we obtain soundness results for the concrete case similar to those (Corollary 1) for the abstract case~\cite{loanho_2024}.
  
\subsection{Proof of Theorem~\ref{thm:adm}}

\thmcredulous*

  \begin{proof}
  
  
  Let $\mT(\delta)$ be a dialogue tree drawn from a dialogue $\delta$ for $\phi$. Let $\mT(\delta_1), \ldots , \mT(\delta_m)$ be sub-trees (with root $\phi$) constructed from $\mT(\delta)$. By Lemma~\ref{lem:DT-subT}, we know all sub-trees $\mT(\delta_i)$ of $\mT(\delta)$ are the dialogue trees drawn from focused sub-dialogues $\delta_i$, with $i = 1, \ldots, m$, of $\delta$ and each such sub-tree is focused. We assume that $\mT(\delta_i)$ is defensive and non-redundant.
%
%  Since $\mT(\delta_i)$ is defensive, 
  By the correspondence principle of Theorem~\ref{thm:def-abs},  there is an abstract dialogue tree $\mT_{\G}^{i}$ for $\phi$ such that $\mDE(\mT_{\G}^{i})  =  \mDE(\mT(\delta_i))$ and $\mCU(\mT_{\G}^{i})  =  \mCU(\mT(\delta_ i))$.
  
  For such abstract dialogue tree, let $\mB$ be a set of proponent arguments in $\mT_{\G}^{i}$, we prove that 
  \begin{enumerate}
      \item $\mB$ attacks every attack against it and $\mB$ does not attack itself;
      \item no argument in $\mT_{\G}^{i}$ that labels both $\po$ or $\op$.
  \end{enumerate}

 Since $\mT(\delta_i)$ is defensive and non-redundant, we get the following statements:
  
  \begin{enumerate} [a)]
        \item By Definition~\ref{def:defensive-tree}, it is focused. Then the root of $\mT_{\G}^{i}$ hold a proponent argument for $\phi$.
        
      \item By Definition~\ref{def:defensive-tree}, it is last-word. Then the proponent arguments in $\mT_{\G}^{i}$ are leaf nodes, i.e., $\po$ wins. Thus, $\mB$ attacks every attack against it.

      \item By Definition~\ref{def:defensive-tree}, $\mT(\delta_{i})$ has no formulas $\alpha_h$ in opponent nodes belong to $\mDE(\mT(\delta_i))$ such that $\{ \alpha_h \} \cup \mDE(\mT(\delta_i))$ is inconsistent. 
      $h = 0$ corresponds to the case that there is no potential argument, say $A$, such that $\{ \alpha \}$ is the support of $A$ and $A$ attacks any potential arguments supported by $\mDE(\mT(\delta_i))$. 
      Similarly, for $h > 0$, there are no potential arguments collectively attacking any potential arguments supported by $\mDE(\mT(\delta_i))$.
      Thus, $\mB$ does not attack itself.

      \item By Definition~\ref{def:non-re} of non-redundant trees, there is no argument in $\mT_{\G}^{i}$ that labels both $\po$ or $\op$.
  \end{enumerate}

  It can be seen that (b) and (c) prove (1), and (d) proves (2). 
  It follows that $\mT_{\G}^{i}$ is admissible. This result directly follows from the definition of admissible abstract dialogue trees in Definition~\ref{def:adm-ab-dt} (analogous to those in~\cite{loanho_2024}).  
  By Definition~\ref{def:analogous-Def9} (analogous to Definition 9 in~\cite{loanho_2024}), $\delta_i$ is admissible-successful in the P-SAF framework $\mAF_{\delta_{i}}$ drawn from $\delta_i$ (supported by $\mDE(\mT(\delta_i))$).

  Now, we need to show $\delta$ is admissible-successful. By Definition~\ref{def:abstract-dia-forest}, each tree in the abstract dialogue forest contains its own set of proponent arguments, namely, $\mDE(\mT_{\G}^{i}) \neq \mDE(\mT_{\G}^{l})$, with $1 \leq i,\ l \leq m,\ i \neq l$. Thus, arguments in other trees do not affect arguments in $\mAF_{\delta_{i}}$. It follows that $\delta$ is admissible-successful. 
  By Corollary~\ref{cor:analogous-Cor1} (analogous to Corollary 1 in~\cite{loanho_2024}), $\phi$ is credulously accepted under $\adm$ semantics in $\mAF_{\delta}$.
\end{proof}

% \thmpreferred*

% \begin{proof} [Sketch]
% The proof of this theory follows the fact that every preferred dialogue tree is an admissible dialogue tree. Thus, the proof of this theorem is analogous to those of Theorem~\ref{thm:adm}.
% \end{proof}
    

 \subsection{Proof of Theorem~\ref{thm:ground}}
The following lemma is used in the proof of Theorem~\ref{thm:ground}
\begin{lemma}
\label{lem:inf-groundedset}
   An abstract dialogue tree $\mT_{\G}$ is finite iff the set of arguments labelling $\po$ in $\mT_{\G}$ is a subset of the grounded set of arguments.
\end{lemma}

\begin{proof} Let $\mB$ is a set of arguments labbeling $\po$ in$\mT_{\G}$. We prove the lemma as follows:

If part: The height of a finite dialogue tree $\mT_{\G}$ is $2h$. We prove that $\mB$ is a subset of the grounded set by induction on $h$. Observer that (1) if $\mA_1$, $\mA_2$ are an admissible subset of the grounded set, then so is $\mA_1 \cup \mA_2$; (2) if an argument $A$ is accepted wrt $\mA_1$ then $\mA_1 \cup \{A\}$ is an admissible subset of the grounded set. 

$h = 0$ corresponds to dialogue trees containing a single node labelled by an argument that is not attacked. So $\mB$ containing only that argument is a subset of the grounded set. 

Assume the assertion holds for all finite dialogue tree of height smaller than $2h$. Let $A$ be an argument labelling $\po$ at the root of $\mT_{\G}$. For each argument $B$ attacking a child of $A$, let $\mT_{B}$ be the subtree of $\mT_{\G}$ rooted at $B$. Clearly, $\mT_{B}$ is a finite dispute tree with height smaller than $2h$. The union of sets of arguments labelling $\po$ of all $\mT_{B}$ is a subset of the grounded set and defends $A$. So $\mB$ is a subset of the grounded set.

Only if part: to construct a finite abstract dialogue tree for a groundedly accepted argument in a finite P-SAF framework, we need the following lemma: In a finite P-SAF framework, the grounded set equals $\emptyset \cup \mF(\emptyset) \cup \mF^{2}(\emptyset) \cup \cdots$. This lemma follows from two facts, proven in~~\cite{Dung95}, of the characteristic function $\mF$:
\begin{itemize}
    \item $\mF$ is monotonic w.r.t. set inclusion.
    \item if the argument framework is finite, then $\mF$ is $\omega-$ continuous.
\end{itemize}
For each argument $A$ in the grounded set, $A$ can be ranked by a natural number $r(A)$ such that $A \in \mF^{n(A)} (\emptyset) \setminus \mF^{r(A) -1} (\emptyset)$. So $r(A) = 1$, then $A$ belongs to the grounded set and is not attacked. $r(A) = 2$, the set of arguments (of the grounded set) defended by the set of arguments (of the grounded set) such that $r(A) \leq 1$ and so on. For each set of arguments $\mS$ collectively attacking the grounded set,  the rank of $\mS$ is $\texttt{min} \{r(A) \mid A \text{ is in the grounded set and attacks some argument in } \mS \}$. Clearly, $\mS$ does not attack any argument in the grounded set of rank smaller than the rank of $\mS$.

Given any argument $A$ in the grounded set, we can build an abstract dialogue tree $\mT_{\G}$ for $A$ as follows: The root of $\mT_{\G}$ is labelled by $A$. For each set of arguments $\mS$ attacking $A$, we select a set of arguments $\mC$ to counterattacks $\mS$ such that the rank of $\mC$ equals the rank of $\mS$, then for each set of arguments $\mE$ attacking some arguments of $\mC$, we select arguments $\mF$ to counterattacks $\mE$ such that the rank of $\mF$ equals to the rank of $\mE$, and so on. So for each branch of $\mT_{\G}$, the rank of a proponent node is equal to that of its opponent parent node, but the rank of an opponent node is smaller than that of its parent proponent node. Clearly, ranking decreases downwards. So all branches of $\mT_{\G}$ are of finite length. Since the P-SAF is finite, $\mT_{\G}$ is finite in breath. Thus  $\mT_{\G}$ is finite.




    
\end{proof}

% \textbf{Theorem~\ref{thm:ground}}
% \emph{Let $D(\phi) = \delta$ be a dialogue. If there is a dialogue tree $\mT(\delta_i)$ drawn from a focused sub-dialogue $\delta_i$ of $\delta$ such that it is admissible, then
%   \begin{itemize}
%     \item $\delta$ is \emph{groundedly-successful};
%       \item $\phi$ is groundedly accepted under grounded semantics in $\mAF_ \delta$ drawn from $\delta$ (supported by $\mDE(\mT(\delta_i))$).
% \end{itemize}}

\thmground*

\begin{proof}
Let $\mT(\delta)$ be a dialogue tree drawn from a dialogue $\delta$ for $\phi$ and $\mT(\delta_1), \ldots , \mT(\delta_m)$ (with root $\phi$) be subtrees of $\mT(\delta)$. By Lemma~\ref{lem:DT-subT}, all sub-trees of $\mT(\delta)$ are the dialogue trees drawn from focused sub-dialogues $\delta_i$, with $i = 1, \ldots, m$, of $\delta$ and each such subtree is focused. Assume that $\mT(\delta_i)$ is defensive and finite.
%
Since $\mT(\delta_i)$ is defensive,
by the correspondence principle of Theorem~\ref{thm:def-abs}, there is an abstract dialogue tree $\mT_{\G}^{i}$ for $\phi$ such that $\mDE(\mT_{\G}^{i})  =  \mDE(\mT(\delta_i))$ and $ \mCU(\mT_{\G}^{i})  =  \mCU(\mT(\delta_ i))$. 


For such abstract dialogue tree, let $\mB$ be the set of arguments labelling $\po$ in  $\mT_{\G}^{i}$. We need to show that 

\begin{enumerate}
    \item $\mT_{\G}^{i}$ is finite;
    \item $\mB$ attacks ever attract against it, and
    \item $\mB$ does not attack itself.
\end{enumerate}

 Similar to the proof of Theorem~\ref{thm:adm}, (2) and (3) directly follows from the fact that $\mT(\delta_i)$ is admissible.
 Trivially, every $\mT(\delta_i)$ is finite, then $\mT_{\G}^{i}$ is finite.
 As a direct consequence of Lemma~\ref{lem:inf-groundedset}, we obtain that $\mB$ is a subset of the grounded set of arguments in $\mT_{\G}^{i}$.
 By Definition~\ref{def:analogous-Def9} (analogous to Definition 9 in~\cite{loanho_2024}), $\delta_i$ is grounded-successful in $\mAF_{\delta_i}$ (drawn from $\delta_i$).


We next prove that $\delta$ is grounded-successful. Since $\delta_i$ is grounded-successful in $\mAF_{\delta_{i}}$ drawn from $\delta_i$, 
it follows that there are no arguments attacking the arguments in $\mB$ that have not been counter-attacked in the abstract dialogue forest $\mT_{\G}^{1}, \ldots, \mT_{\G}^{m}$ (obtained from the P-SAF drawn from $\delta$).
%
% then it is not the case that there are arguments, attacking the arguments in $\mB$, that have not been counter-attacked in the abstract dialogue forest $\mT_{\G}^{1}, \ldots, \mT_{\G}^{m}$ (obtained from the P-SAF drawn from $\delta$).
%
By Definition~\ref{def:abstract-dia-forest}, each tree in the abstract dialogue forest contains its own set of proponent arguments, namely, $\mDE(\mT_{\G}^{i}) \neq \mDE(\mT_{\G}^{l})$, with $1 \leq i,\ l, \leq m,\ i \neq l$. If the set of proponent arguments in $\mT_{\G}^{i}$ drawn from the focused sub-dialogue $\delta_i$ is grounded, it is also grounded in $\mAF_{\delta}$ drawn from $\delta$. Thus, $\delta$ is grounded successful. By Corollary~\ref{cor:analogous-Cor1} (analogous to Corollary 1 in~\cite{loanho_2024}), $\phi$ is groundedly accepted in $\mAF_{\delta}$ (supported by $\mDE(\mT(\delta_i))$).
\end{proof}
  
\subsection{Proof of Theorem~\ref{thm:scep}}

\thmsceptical*

 \begin{proof} We prove this theorem for the case of admissible semantics. The proof for preferred (stable) semantics is analogous.
 
  Let $\mT(\delta)$  be a dialogue tree drawn from a dialogue $\delta$ for $\phi$, and $\mT(\delta_1), \ldots , \mT(\delta_m)$ (with root $\phi$) be subtrees constructed from $\mT(\delta)$. By Lemma~\ref{lem:DT-subT}, we know all subtrees in $\mT(\delta)$ are dialogue trees drawn from focused sub-dialogues $\delta_i$, ($i = 1, \ldots , m$) of $\delta$ and each tree is focused. Assume that $\mT(\delta)$ is ideal.   
 % 
  Since $\mT(\delta)$ is ideal, $\mT(\delta_i)$ is ideal. By Definition~\ref{def:tree-ideal}, $\mT(\delta_i)$ is defensive.
  By the correspondence principle, there is an abstract dialogue tree $\mT_{\G}^{i}$ for $\phi$ such that $\mDE(\mT_{\G}^{i})  =  \mDE(\mT(\delta_i))$ and $ \mCU(\mT_{\G}^{i})  =  \mCU(\mT(\delta_ i))$.

  
  For such abstract dialogue tree $\mT_{\G}^{i}$, we need to show that:
  \begin{enumerate}
      \item $\mT_{\G}^{i}$ is admissible, and
      \item for no opponent node $\op$ in it there exists an admissible dialogue tree for the opponent argument.
  \end{enumerate} 
  
  Similar to the proof of Theorem~\ref{thm:adm}, (1) holds.
%
  Since $\mT(\delta)$ is ideal, by Definition~\ref{def:tree-ideal}, there is a dialogue tree $\mT(\delta_i)$ such that none of the opponent arguments drawn from $\mT(\delta_{i})$ belongs to an admissible set of arguments in $\mAF_{\delta}$ drawn from $\delta$. Then, (2) holds.
  
  We have shown that $\mT_{\G}^{i}$ is defensive and none of the opponent arguments belongs to an admissible set of arguments in $\mAF_{\delta}$ drawn from $\delta$. By Definition~\ref{def:analogous-Def9} (analogous to Definition 9 in~\cite{loanho_2024}), $\delta$ is sceptical-successful. By Corollary~\ref{cor:analogous-Cor1} (analogous to Corollary 1 in~\cite{loanho_2024}), $\phi$ is sceptically accepted in $\mAF_{\delta}$.
  \end{proof}


  
\section{Proofs for Section~\ref{sec:completeness}}
\label{app:proof-completeness}

To prove the completeness results, we

\begin{enumerate}
    \item partition a dialogue $\delta$ for $\phi$ into its focused sub-dialogues of $\delta$ (by Definition~\ref{def:focused-sub-dia}),

    \item apply Corollary~\ref{cor:extend-cor2} and Definition~\ref{def:analogous-Def9} (analogous to Definition 9 in~\cite{loanho_2024}) to obtain the existence of an abstract dialogue tree drawn from each focused sub-dialogue of $\delta$ wrt argumentation semantics,

    \item use the correspondence principle to transfer from each abstract dialogue tree for $\phi$ into a dialogue tree for $\phi$ (by Theorem~\ref{thm:def-abs}), thereby proving the completeness results.
    
\end{enumerate}

\subsection{Preliminaries}

Corollary~\ref{cor:analogous-Cor1} is used for the proof of the soundness results. To prove the completeness result, we need to extend Corollary~\ref{cor:analogous-Cor1} as follows:

\begin{corollary}
\label{cor:extend-cor2}

Let $\mK$ be a KB, $\phi \in \mL$ a formula and $\mAF_{\mK}$ be the corresponding P-SAF of $\mK$. We say that if $\phi$ is

\begin{itemize}
    \item   credulously accepted in some admissible/preferred extension of $\mAF_{\mK}$, then there is a focused sub-dialogue $\delta_i$, with $i = 1, \ldots, m$, of a dialogue for $\phi$ such that $\delta_i$ is admissible/preferred-successful;
    
    \item groundedly accepted in a grounded extension of $\mAF_{\mK}$, then there is a focused sub-dialogue $\delta_i$ of a dialogue for $\phi$  such that $\delta_i$ is grounded-successful;

    \item  sceptically accepted in all preferred extensions of $\mAF_{\mK}$, then there is a focused sub-dialogue $\delta_i$ of a dialogue for $\phi$ such that $\delta_i$ is sceptical-successful.   
\end{itemize}     
\end{corollary}

 \begin{proof} 
 Let $\mT(\delta)$ be a dialogue tree drawn from a dialogue $\delta$ for $\phi$ and $\mT(\delta_1), \ldots , \mT(\delta_m)$ (with root $\phi$) be subtrees of $\mT(\delta)$. By Lemma~\ref{lem:DT-subT}, all sub-trees of $\mT(\delta)$ are the dialogue trees drawn from focused sub-dialogues $\delta_i$, with $i = 1, \ldots, m$, of $\delta$ and each such subtree is focused.
%
 By the correspondence principle, there is an abstract dialogue tree $\mT_{\G}^{i}$ for an argument $A$ with conclusion $\phi$, or simply, an abstract dialogue $\mT_{\G}^{i}$ for $\phi$, which corresponds to the dialogue tree $\mT(\delta_i)$ for $\phi$.
 
 It is clear that there is a focused sub-dialogue $\delta_i$  such that it is admissible-successful if $\phi$ is credulously accepted in some admissible/preferred extension of $\mAF_ \mK$.
 The argument given in Definition 2 and Lemma 1 of~\cite{ThangDH09} for binary attacks generalizes to collective attacks, implying that $A$ is accepted in some admissible extension iff $\mT_{\G}^{i}$ is admissible which in turn holds iff $\po$ wins and no argument labels both a proponent and an opponent node.  By Definition~\ref{def:analogous-Def9}, $\delta_i$ is admissible-successful iff $\mT_{\G}^{i}$ is admissible. Thus, the statement is proved.
 
  $\delta_i$ is preferred-successful if $\delta_i$ is admissible-successful.
  This result directly follows from the results of~\cite{Dung95} that states that an extension is preferred if it is admissible. Thus, if $\phi$ is credulously accepted in some preferred extension, then $\delta_i$ is preferred-successful. 
  
  The other statement follows in a similar way as a straightforward generalization of Theorem 1 of~\cite{ThangDH09} for the "grounded-successful" semantic; Definition 3.3 and Theorem 3.4 of~\cite{DUNG2007642} for the "sceptical-successful" semantic.
\end{proof}



\subsection{Proof of Theorem~\ref{thm:com-adm}}

\compadm*

\begin{proof} 
Given $\delta$ be a dialogue for a formula $\phi \in \mL$,
we assume that $\phi$ is credulously accepted under $\adm$ in $\mAF_{\delta}$ drawn from $\delta$ and $\delta$ is admissible-successful.
We prove that there exists a dialogue tree drawn from the sub-dialogue of $\delta$ such that the dialogue tree is defensive and non-redundant.

% Let $\mT(\delta)$ be a dialogue tree drawn from the dialogue $\delta$ and $\mT(\delta_1), \ldots , \mT(\delta_m)$ with root $\phi$ be subtrees of $\mT(\delta)$. 
% By Lemma~\ref{lem:DT-subT}, all sub-trees $\mT(\delta_i)$, where $i = 1, \ldots, m$, of the dialogue tree $\mT(\delta)$ are dialogue trees drawn from focused sub-dialogues $\delta_i$ of $\delta$ and each such subtree is focused.

%For such dialogue tree $\mT(\delta_{i})$, we need to prove that $\mT(\delta_{i})$ is defensive and non-redundant. 


Let $\delta_1, \ldots, \delta_m$, where $i = 1, \ldots, m$, be sub-dialogues of $\delta$ and each sub-dialogue is focused.
Since $\phi$ is credulously accepted under $\adm$ in $\mAF_{\delta}$ drawn from the dialogue $\delta$, by Corollary~\ref{cor:extend-cor2}, the focused sub-dialogue $\delta_{i}$ is admissible-successful.
By Definition~\ref{def:analogous-Def9} (analogous to Definition 9 in~\cite{loanho_2024}), there is an admissible abstract dialogue tree $\mT^{i}_{\G}$ for $\phi$ drawn from the admissible-successful dialogue $\delta_i$.
%
By the correspondence principle, there exists a dialogue tree $\mT(\delta_i)$ for $\phi$ that corresponds to the abstract dialogue tree $\mT_{\G}^{i}$ for $\phi$ such that  $\mDE(\mT_{\G}^{i})  =  \mDE(\mT(\delta_i))$ and $ \mCU(\mT_{\G}^{i})  =  \mCU(\mT(\delta_ i))$.
Since $\mT_{\G}^{i}$ is admissible, $\mT(\delta_i)$ is defensive and non-redundant. Thus, the statement is proved.    
\end{proof}

\subsection{Proof of Theorem~\ref{thm:com-ground}}

\compground*

\begin{proof}
Given $\delta$ be a dialogue for a formula $\phi \in \mL$, we assume that $\phi$ is groundedly accepted under $\grd$ in $\mAF_{\delta}$ drawn from $\delta$ and $\delta$ is groundedly-successful. We prove that there exists a dialogue tree drawn from the sub-dialogue of $\delta$ such that the dialogue tree is defensive and finite.

Let $\delta_1, \ldots, \delta_m$, where $i = 1, \ldots, m$, be sub-dialogues of $\delta$ and each sub-dialogue is focused.
Since $\phi$ is groundedly accepted under $\grd$ in $\mAF_{\delta}$ drawn from the dialogue $\delta$, by Corollary~\ref{cor:extend-cor2}, the focused sub-dialogue $\delta_{i}$ is grounded-successful.
By Definition~\ref{def:analogous-Def9} (analogous to Definition 9 in~\cite{loanho_2024}), there is an admissible and finite abstract dialogue tree $\mT^{i}_{\G}$ for $\phi$ drawn from the grounded-successful dialogue $\delta_i$.
%
By the correspondence principle, there is a dialogue tree $\mT(\delta_i)$ for $\phi$ that corresponds to the abstract dialogue tree $\mT_{\G}^{i}$ for $\phi$ such that  $\mDE(\mT(\delta_i)) = \mDE(\mT_{\G}^{i})$ and $\mCU(\mT(\delta_ i)) = \mCU(\mT_{\G}^{i})$.
Since $\mT_{\G}^{i}$ is admissible, $\mT(\delta_i)$ is defensive. Since $\mT_{\G}^{i}$ is finite, $\mT(\delta_i)$ is finite.  Thus, the statement is proved.
\end{proof}

\subsection{Proof of Theorem~\ref{thm:com-scep}}

\compsceptical*

\begin{proof} We prove this theorem for the case of admissible semantics. The proof for preferred (stable) semantics is analogous.

Given $\delta$ be a dialogue for a formula $\phi \in \mL$, we assume that $\phi$ is credulously accepted under $\adm$ in $\mAF_{\delta}$ drawn from $\delta$ and $\delta$ is admissible-successful. 
We prove that there is an ideal dialogue tree $\mT(\delta)$ for $\phi$ drawn from the dialogue $\delta$.

Let $\delta_1, \ldots, \delta_m$, where $i = 1, \ldots, m$,  be focused sub-dialogues of $\delta$.
% From the correspondence between dialogue trees and dialogues, there is a dialogue tree $\mT(\delta)$ drawn from the dialogue $\delta$ and $\mT(\delta_1), \ldots , \mT(\delta_m)$ with root $\phi$ are subtrees of $\mT(\delta)$. 
% By Lemma~\ref{lem:DT-subT}, all sub-trees $\mT(\delta_i)$ of the dialogue tree $\mT(\delta)$ are dialogue trees drawn from focused sub-dialogues $\delta_i$ of $\delta$ and each such subtree is focused. 
%
Since $\phi$ is sceptically accepted under $\adm$ in $\mAF_{\delta}$ drawn from the dialogue $\delta$, by Corollary~\ref{cor:extend-cor2}, the focused sub-dialogue $\delta_{i}$ is sceptical-successful.
By Definition~\ref{def:analogous-Def9} (analogous to Definition 9 in~\cite{loanho_2024}), there is an abstract dialogue tree $\mT^{i}_{\G}$ for $\phi$ drawn from the sceptical-successful dialogue $\delta_i$ such that $\mT^{i}_{\G}$ is admissible and for no opponent node in it there exists an admissible abstract dialogue tree for the argument labelling an opponent node.
%
Since $\mT^{i}_{\G}$ is admissible, by the correspondence principle, there is a dialogue tree $\mT(\delta_i)$ for $\phi$ that corresponds to the abstract dialogue tree $\mT_{\G}^{i}$ for $\phi$ such that  $\mDE(\mT(\delta_i)) = \mDE(\mT_{\G}^{i})$ and $\mCU(\mT(\delta_ i)) = \mCU(\mT_{\G}^{i})$.

For such the dialogue tree $\mT(\delta_i)$, we need to show that:
\begin{enumerate}
    \item $\mT(\delta_i)$ is defensive and non-redundant,
    \item none of the opponent arguments obtained from $\mT(\delta_i)$ belongs to an admissible set of potential arguments in $\mAF_{\delta_i}$ drawn from $\mT(\delta_i)$.
\end{enumerate}


Since the abstract dialogue $\mT_{\G}^{i}$ is admissible, (1) holds. 
We have that, for no opponent node in $\mT_{\G}^{i}$, there exists an admissible abstract dialogue tree for the argument labelling by $\op$. From this, we obtain that (2) holds.
By Definition~\ref{def:tree-ideal}, $\mT(\delta_i)$ is ideal. Thus, $\mT(\delta)$ is ideal. Thus, the statement is proved.

%Each sub-tree of $\mT(\delta)$ contains its own set of proponent arguments, namely, $\mDE(\mT(\delta_i)) \neq \mDE(\mT(\delta_l))$, with $1 \leq i,\ l, \leq m,\ i \neq l$. If the set of proponent arguments in $\mDE(\mT(\delta_i)$ drawn from the focused sub-dialogue $\delta_i$ is ideal, it is also grounded in $\mAF_{\delta}$ drawn from $\delta$. Thus, $\delta$ is grounded successful.


   
\end{proof}

%\section{Figures}

%\begin{table} 
%\centering
%  \begin{tabular}{cll}
%    \toprule
%    \textbf{Argument} & \textbf{$\Sup(A_i)$} & \textbf{$\Con(A_i)$} \\
%    \midrule
%    $A_0$ & $\{\te(\vi, \kr)\}$ & $\te(\vi, \kr)$ \\
%    $A_1$ & $\{\gc(\kr)\}$ & $\gc(\kr)$ \\
%    $A_2$ & $\{ \gc(\kr), \te(\vi, \kr) \}$ & $\fp(\vi)$ \\
%    $A_3$ & $\{ \gc(\kr), \te(\vi, \kr) \}$ & $\rese(\vi)$\\
%    $A_4$ & $\{\te(\vi, \kd)\}$ & $\te(\vi, \kd)$ \\
%    $A_5$ & $\{\teAs(\vi, \kd)\}$ & $\teAs(\vi, \kd)$\\
%    $A_6$ & $\{ \uc(\kd) \}$ & $\uc(\kd)$\\
%    $A_7$ & $\{ \teAs(\vi, \kd),\ \uc(\kd) \}$ & $\ta(\vi)$\\
%    $A_9$ & $\{\te(\vi, \kr)\}$ & $\lect(\vi)$ \\
%    $A_{10}$ & $\{\te(\vi, \kd)\}$ & $\lect(\vi)$ \\
%    $A_{11}$ & $\{\te(\vi, \kr)\}$ & $\emp(\vi)$ \\
%    $A_{12}$ & $\{\te(\vi, \kd)\}$ & $\emp(\vi)$ \\
%    $A_{13}$ & $\{ \gc(\kr), \te(\vi, \kr)\}$ & $\emp(\vi)$ \\
%    \bottomrule
%  \end{tabular}
%  \caption{Supports and conclusions of all arguments induced by $\mK$}
%  \label{tab:full-args} 
%\end{table}


%

%


% Like "non-flat" ABA frameworks with collective attacks~\cite{ArieliH24}, our approach provides a very flexible environment for logical argumentation. Our framework, using abstract logic with no requirement on the language $\mL$ and $\cn$ (a function from $2^{\mL}$ to $2^{\mL}$), is as general as the framework of ~\cite{ArieliH24}. The latter uses underlying logic with \emph{explosive} and \emph{contrapositive} requirements and a consequence relation (a function from $2^{\mL}$ to $\mL$). While ~\cite{ArieliH24} mainly focuses on representation considerations, our work extends the study further. 





% Our work overcomes this by defining attacks based on inconsistency within the the \emph{closure} of formulas set.
%
%The work of~\cite{ArieliH24} considers the "non-flat" ABA framework. The idea is to use collective attacks where formulas in the attacking set entail the contrary of a formula in the attacked set, which is close to our notion of inconsistency. We believe that our framework, using abstract logic with no requirement on the language $\mL$ and $\cn$ (a function from $2^{\mL}$ to $2^{\mL}$), is as general as the framework of ~\cite{ArieliH24}. The latter uses underlying logic with \emph{explosive} and \emph{contrapositive} requirements and a consequence relation (a function from $2^{\mL}$ to $\mL$). Our approach provides a very flexible environment for logical argumentation like~\cite{ArieliH24} and examines dialogues (proof procedures) for various argumentation semantics in the context of collective attacks.
%

%To the best of our knowledge, there is no general framework unifying all of these approaches.%Thus we propose a unifying framework, considering arbitrary logic.


 %~\cite{Amgoud12} highlights the weaknesses of ASPIC+: the underlying logic of ASPIC+ cannot encode abstract logic: regardless of whether they use a notion of negation, etc.  %In contrast, our framework can address this by collective attacks. 
% 
  




