\section{Related Work}
\label{se:relate_work}

Artificial intelligence is making significant advances in tumor detection, with deep learning models, like EfficentNet~\cite{medina2023high}, achieving high accuracy rates with MRI images. Other standard models, such as VGG and MobileNet, have also shown outstanding results~\cite{reyes2024performance}. Furthermore, the use of Transformer models, such as the Vision and the Swin Transformers, are being explored to overcome the limitations of CNNs in medical image classification and segmentation. These models can capture global relationships in images, which is crucial for accuracy in medical diagnosis.

A recent work~\cite{khan2023recentsurveyvisiontransformers} highlights the importance of segmenting medical images based on Vision Transformers instead of convolutional networks. The latter are effective in capturing local correlations, although they are limited in capturing global relationships. 

\subsection{Brain tumors}
Brain tumor classification has received significant attention in the last few years, particularly through the analysis of MR images. Various studies have been conducted to enhance the performance of brain tumor classification using different methodologies. The work presented in ~\cite{cheng2015enhanced} focused on the classification of brain tumors based on T1-weighted contrast-enhanced MRI. They proposed a dataset that has been used in many subsequent works.  

Traditional techniques typically classify images based on two main steps: feature extraction and then classification. The features proposed in~\cite{gumaei2019hybrid} were based on PCA and GIST techniques and the classification was carried out through a regularized extreme learning machine. The last step can also involve the use of neural networks, like the work proposed in \cite{ismael2018brain}, which relies on the 2D Discrete Wavelet Transform (DWT) and 2D Gabor filters to extract statistical features from MRI and then feed into a neural network. The approach presented in~\cite{bahadure2018comparative} tackled the problem of segmentation and classification of MRI using genetic algorithms. The authors of~\cite{afshar2018brain} explored brain tumor classification using Capsule Networks. This type of network has several benefits over CNNs since they are robust to rotation and affine transformations and require less training data.

The system proposed in~\cite{sajjad2019multi} consists of three main steps: tumor segmentation, data augmentation, and deep feature extraction and classification. It relied on extensive data augmentation techniques and the fine-tuning of a VGG-19 network. Many works \cite{ayadi2021deep} have extensively used pure CNN models for brain tumor classification, obtaining high accuracy in different datasets. 

Several works~\cite{srinivas2022deep} have conducted a performance analysis of transfer learning in CNN models (such as VGG-16, ResNet-50, or Inception-v3) for automatic prediction of tumor cells in the brain. A recent work~\cite{reyes2024performance} reported a detailed performance assessment analysis of many convolutional architectures from different perspectives, drawing important conclusions about these techniques. Several models can attain high accuracy, even for networks with a relatively low number of parameters like EffientNet~\cite{medina2023high}. 

\subsection{Lung tumors}
    
Lung cancer remains one of the leading causes of cancer-related deaths worldwide. Early and accurate detection through imaging techniques like CT scans is crucial for improving patient outcomes. Deep learning models have shown significant promise in automating the classification of lung tumors from CT scans. 

Classification of lung tumors using machine learning techniques has been an active area of research. Numerous studies have explored various algorithms and methodologies to improve diagnostic accuracy and efficiency. 

Early work in lung tumor classification primarily utilized traditional machine learning algorithms such as Support Vector Machines (SVM), k-Nearest Neighbors (k-NN), and Decision Trees. For instance, the authors of~\cite{7158331} employed SVM for classifying lung nodules in CT images, achieving notable accuracy by optimizing hyperparameters and using feature selection techniques to reduce dimensionality. Similarly, \cite{sathishkumar2019detection} used k-NN combined with SVM, demonstrating improved classification performance on small datasets.

With the advent of deep learning, CNNs have become the preferred choice for image-based lung tumor classification. The work presented in \cite{shen15} developed a deep CNN model that outperformed traditional methods by automatically learning hierarchical features from raw CT images. Their approach significantly reduced the need for manual feature extraction, leading to higher accuracy and robustness. On the other hand, \cite{LIU2018262} further advanced this field by introducing a multi-view CNN that integrates information from multiple CT slices, enhancing the ability of the model to capture spatial dependencies and improve classification accuracy. This method demonstrated superior performance in distinguishing between benign and malignant nodules compared to single-view CNNs.

Ensemble methods, which combine the predictions of multiple models, have been explored to enhance classification performance. For example, the authors of \cite{zhang2019ensemble} utilized an ensemble of CNNs and several traditional machine-learning techniques to classify lung tumors, achieving improved accuracy and robustness by mitigating the weaknesses of individual models.

Transfer learning, which involves fine-tuning pre-trained models on specific datasets, has gained popularity due to its effectiveness in scenarios with limited labeled data. The model presented in \cite{8624570} employed transfer learning using pre-trained CNNs, achieving high classification accuracy with reduced training time. Their approach demonstrated that leveraging pre-trained models can enhance performance, especially when dealing with small medical datasets.

Recent studies have focused on improving the interpretability and explainability of machine learning models in lung tumor classification. The work in~\cite{WANG2024105646} introduced an explainable AI framework that combines a 3D U-Net with attention mechanisms to highlight the most relevant regions in CT images.


\subsection{Kidney tumors}

The application of machine learning to the classification of kidney tumors has seen significant advancements in recent years. Various methods and technologies have been developed to aid in the early detection and accurate segmentation of kidney tumors. For instance, the work in~\cite{kim04} utilized a computer-aided detection system for kidney tumors on abdominal CT scans, employing a gray-level threshold method for segmentation and texture analysis for tumor detection. Similarly, \cite{zhou2016atlas} presented a semi-automatic kidney tumor detection and segmentation method using atlas-based segmentation.

The classification of kidney tumors into subtypes such as benign, malignant, and histological categories (e.g., clear cell renal carcinoma, papillary renal cell carcinoma) is crucial for guiding treatment strategies. Early approaches often utilized traditional techniques, such as SVMs and Random Forests, relying on manually extracted features such as texture, shape, and intensity descriptors. For example, \cite{feng2018machine} explored handcrafted features combined with SVMs for detecting renal masses, achieving promising results in sensitivity and specificity. Similarly, \cite{erdim2020prediction} utilized radiomics features in combination with eight machine-learning techniques, like logistic regression, decision trees, and SVMs, to classify renal tumors.

Advancements in deep learning techniques have also played a crucial role in kidney tumor detection. For example, \cite{rathnayaka2019kidney} proposed a CNN-based U-Net architecture with an attention mechanism for kidney tumor detection. On the other hand, \cite{lin2021automated} developed a 3D U-Net-based deep convolutional neural network for automatically segmenting kidney and renal masses, achieving high accuracy in tumor segmentation. Furthermore, \cite{alzu2022kidney} and \cite{praveen2023resnet} focused on deep learning approaches for kidney tumor detection and classification, utilizing models such as 2D-CNN, ResNet, and ResNeXt to improve the detection accuracy. The work in~\cite{ozbay2024kidney} explored self-supervised learning for kidney tumor classification on CT images. In addition, \cite{bogomolov2017development} studied the development of an LED-based near-infrared sensor for human kidney tumor diagnostics, providing a simplified alternative to conventional NIR spectroscopic methods.


%-------------------------- Materials and Methods ---------------------------