\section{Related Works}
\label{Related_Works}

This section presents commonly used methods for offline global trajectory planning and online local trajectory planning methods in autonomous racing.

The offline global trajectory planning methods for autonomous racing typically utilize path-velocity decomposition (PVD)  framework\cite{heilmeier2019minimum}. Path planning methods can be categorized based on the optimization objective into two primary types: path length-optimal\cite{braghinRaceDriverModel2008} and path curvature-optimal\cite{heilmeier2019minimum}. Velocity planning methods usually take into account the curvature changes of the planned racing path and the physical constraints of the vehicle for optimal racing performance\cite{heilmeier2020minimum}. A particularly efficient approach involves generating an initial velocity profile, which is then iteratively optimized based on the curvature of the planned path and the longitudinal force constraints of the vehicle \cite{kapaniaSequentialTwoStepAlgorithm2016}. Vehicle drifting is a frequently used strategy in racing as well. The framework proposed in\cite{weng2024aggressive} uses a mode-switching  controller to implement aggressive cornering.
	
Among the online local trajectory planning methods, MPCC is a typical representative. By introducing the concept of the contour control in industrial applications, MPCC has modeled the racing trajectory planning problem as the task of moving fastest along the racetrack centerline. Because the MPCC method does not consider the curvature of the \textcolor{black}{racetrack} centerline, the method\cite{lyonsCurvatureAwareModelPredictive2023}  improves it by introducing the curvature state into the system dynamics. However, this method is only experimented with low velocity and not extended to racing scenarios. Similar to the MPCC method, \cite{sivashangaranNonlinearModelPredictive2022} proposes an advanced nonlinear model predictive control (NMPC) strategy that uses a cost function to achieve a time-optimal trajectory by minimizing the sum of times in the prediction time horizon while maximizing the velocity. This method also does not consider the constraints imposed by the racetrack curvature on the vehicle velocity. Graph search combined with smooth spline is also an effective approach\cite{stahl2019multilayer}.

In general, offline global trajectory planning methods typically focus on planning trajectories with minimal path curvature while considering the physical constraints of the vehicle. Although these methods can generate globally optimal trajectories using comprehensive racetrack information, they lack the ability to respond to real-time vehicle states during runtime. On the other hand, MPCC, as an online local trajectory planning method, excels at dynamically responding to runtime conditions. However, it ignores the curvature of the racetrack centerline, which is crucial for optimizing autonomous racing performance. The CiMPCC method proposed in this paper fills this gap. It considers the curvature of the racetrack centerline in the optimization problem, thereby achieving performance-oriented autonomous racing.


This paper consists of several sections. Section \ref{PRELIMINARIES} describes the traditional MPCC method. Section \ref{CiMPCC} explains CiMPCC in detail, including the modeling method of the racetrack centerline and its incorporation into the optimization problem. Section \ref{EXPERIMENTAL}  presents the results of applying CiMPCC to autonomous racing with the self-built 1:10 vehicle. Section \ref{Conclusion} serves as a conclusion that summarizes the findings and outlines directions for future research.

\begin{figure}[!t]
	\centering
	\includegraphics[scale=0.26]{./media/mpcc.pdf}
	\caption{Schematic representation of the approximation of the contour error and lag error of MPCC.}
	\label{fig:mpcc}
\end{figure}