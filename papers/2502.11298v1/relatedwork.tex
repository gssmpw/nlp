\section{Related Work}
Several strategies have been developed for SFC provisioning to optimize resource allocation while minimizing latency. Traditional optimization methods, such as Mixed Integer Linear Programming (MILP) \cite{pham020}, and heuristic approaches, such as Nearest Candidate Node Selection (NCNS)\cite{atinafu2024}, Fastest Candidate Node Selection (FCNS)\cite{atinafu2024}, and Priority-Based FCNS (PB-FCNS)\cite{atinafu2024}, utilize SDN and queueing models to efficiently assign computing nodes \cite{atinafu2024}. While these solutions reduce latency and increase service success rates, they rely on predefined decision criteria and are not adaptable to dynamic network conditions. 

RL can outperfrom heuristic approaches in SFC provisioning by continuously learning and optimizing from interactions with the environment rather than relying on predefined rules.
Thinh et al. \cite{tran2024} has proposed a DRL-based framework for SFC provisioning that optimizes VNF embedding and routing while meeting time and resource constraints. They have compared Deep Q-learning (DQL) and Advantage Actor-Critic (A2C), demonstrating that both achieve over 95\% service request acceptance along with high network throughput, performing similarly to optimization-based methods but with considerably shorter execution times. 
In another study  by Onsu et al." \cite{arda24}, a DRL-based SFC provisioning approach has been proposed that offers reconfigurability of networks without the need to retrain the model in different network setups. Moreover, the advanced DNN architecture for their DRL model includes an attention layer to optimize resource utilization \cite{arda24}. 
%%%Jaumard et al.\cite{jaumard2024} has proposed a DRL approach combined with Graph Neural Networks (GNNs), specifically Graph Convolutional Networks (GCN) and Graph Attention Networks (GAT), to improve SFC provisioning by efficiently routing requests using an ordered list of VNFs while meeting network, computational, delay, and bandwidth constraints.
%%%


Although DRL is effective for SFC provisioning, it has drawbacks such as  slow adaptation to unexpected changes, and high training costs while correcting an agent wrong decisions. Recently, a few studies have explored the use of LMs for SFC provisioning.
Van et al.\cite{VAN24} have utilized LMs for intent-based NFV configuration, translating natural language intents into JSON templates that include attributes such as VNF name, type (e.g., firewall, IDS), computational requirements (CPUs, memory, storage), and actions (create, update, destroy). The proposed system has been integrated with NI-testbed, an AI-driven NFV lifecycle management framework, to automate VNF and SFC deployment \cite{VAN24}. However, the proposed system lacks adaptive learning capabilities, meaning that decision-making does not improve over time. If an SFC request is suboptimal, the system does not learn from previous mistakes to make better decisions in the future. Instead, it automatically converts users requests to JSON without considering whether they are ideal for the present network conditions. While the intent is not deployable, it only warns the administrator of errors rather than optimizing or modifying configurations on its own.
Additionally, the research does not address whether a VNF is idle or  whether the system has enough resources to handle all requests. This information is crucial for ensuring optimal resource usage, preventing over- or under-provisioning, and preserving Quality of Service (QoS) while reducing operational costs. 



Li et al.\cite{Li2025} have proposed an LLM-assisted  SFC optimization that uses LLMs to dynamically create heuristic functions for solving optimization problems and a Non-dominated Sorting Genetic Algorithm (NSGA) to refine the heuristic functions generated by the LLMs. However, the LLM-generated heuristics are fixed until manually changed, which prevents automatic adjustments to unexpected network changes. While the proposed model optimizes SFC deployment over multiple iterations, it does not  store previous network states or adapt in real time. If conditions change after optimization, the system must be restarted from scratch, which limits continuous learning and real-time adaptability.
 \begin{figure*}[!t]
        \centering
        \includegraphics[width = 0.8\textwidth, trim=0cm 0cm 0cm 0cm,clip]{fig/system_model_BERT.png}
        \caption{System Architecture for LM-Assisted Network State Monitoring with DRL-Driven SFC Provisioning.}
        \label{fig:system} 
\end{figure*}
Our proposed methodology, based on our previous work \cite{arda24}, integrates  LMs to monitor the network's current state that can direct DRL to improve SFC provisioning and VNF placement. By leveraging DRL-learned patterns and following LM's instructions the system can dynamically optimize SFC deployment, ensuring efficient resource utilization and improved QoS. Additionally, this approach adjusts to changing network conditions without requiring manual heuristic updates or re-optimization from scratch. Furthermore, by using DRL-generated outputs, SFC's and DCs'  information as input for LM, our model improves system interpretability and enables a more detailed analysis of SFC configurations, resource allocation, and VNF state for future requests.