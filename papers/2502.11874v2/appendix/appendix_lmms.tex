\section{Linear Mixed Effects Models}\label{appendix:lmms}
Below we provide the details for the linear mixed effects models that we fit to our data.
All LMMs are fit using the \verb|lme4| package in R. 

\subsection{Human Data (\dataset)}
In \S\ref{sec:dataset}, we are interested in predicting human judgments from the main effects of quantifiers, object count, segmentation area and size norms, as well as the interaction between these predictors.
We include the participants and object categories as random effects. Put concretely, 

\begin{verbatim}
  judgment ~ quantifier * count 
           * segmentation * size_norm 
            + (1|participant) + (1|object)
\end{verbatim}

We scale judgments, count, segmentation area and size norm to make sure they all have a mean of 0 and a standard deviation of 1. For example,
\begin{verbatim}
  count <- scale(count,
                center=TRUE,
                scale=TRUE)
\end{verbatim}
\noindent
This way, we ensure that we can meaningfully interpret the relation between one unit of change in one variable with a change in another.
Additionally, we make the variables for quantifier and object category a \verb|factor| and relevel the quantifier to use the unquantified (\verb|base|) condition as the reference category.
\begin{verbatim}
  quantifier <- relevel(quantifier,
                        ref="base")
\end{verbatim}


\subsection{Model Data (Experiment 1)}
For the models, we follow the same steps taken as those for fitting an LMM to human data, but now we no longer have to account for different participants.
That is, 

\begin{verbatim}
  log_prob ~ quantifier * count 
            * segmentation * size_norm 
            + (1|object)
\end{verbatim}

