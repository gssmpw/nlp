%%%%%%%% ICML 2025 EXAMPLE LATEX SUBMISSION FILE %%%%%%%%%%%%%%%%%

\documentclass{article}

% Recommended, but optional, packages for figures and better typesetting:
\usepackage{microtype}
\usepackage{graphicx}
\usepackage{subfigure}
\usepackage{booktabs} % for professional tables

% hyperref makes hyperlinks in the resulting PDF.
% If your build breaks (sometimes temporarily if a hyperlink spans a page)
% please comment out the following usepackage line and replace
% \usepackage{icml2025} with \usepackage[nohyperref]{icml2025} above.
\usepackage{hyperref}


% Attempt to make hyperref and algorithmic work together better:
\newcommand{\theHalgorithm}{\arabic{algorithm}}

% Use the following line for the initial blind version submitted for review:
%\usepackage{icml2025}

% If accepted, instead use the following line for the camera-ready submission:
\usepackage[accepted]{icml2025}

% For theorems and such
\usepackage{amsmath}
\usepackage{amssymb}
\usepackage{mathtools}
\usepackage{amsthm}

% if you use cleveref..
\usepackage[capitalize,noabbrev]{cleveref}

%%%%%%%%%%%%%%%%%%%%%%%%%%%%%%%%
% THEOREMS
%%%%%%%%%%%%%%%%%%%%%%%%%%%%%%%%
\theoremstyle{plain}
\newtheorem{theorem}{Theorem}[section]
\newtheorem{proposition}[theorem]{Proposition}
\newtheorem{lemma}[theorem]{Lemma}
\newtheorem{corollary}[theorem]{Corollary}
\theoremstyle{definition}
\newtheorem{definition}[theorem]{Definition}
\newtheorem{assumption}[theorem]{Assumption}
\theoremstyle{remark}
\newtheorem{remark}[theorem]{Remark}

% Todonotes is useful during development; simply uncomment the next line
%    and comment out the line below the next line to turn off comments
%\usepackage[disable,textsize=tiny]{todonotes}
\usepackage[textsize=tiny]{todonotes}


%%%%%%%%%%%%%%%%%% imported libraries %%%%%%%%%%%%%%%%%%
\usepackage{multirow}
\usepackage{arydshln}
\usepackage{makecell}
\newcommand{\etal}{\textit{et al.}}
\newcommand{\eg}{\textit{e.g.}}
\newcommand{\ie}{\textit{i.e.}}

% The \icmltitle you define below is probably too long as a header.
% Therefore, a short form for the running title is supplied here:
%\icmltitlerunning{Submission and Formatting Instructions for ICML 2025}

\begin{document}

\twocolumn[
\icmltitle{Fine-Grained Video Captioning through Scene Graph Consolidation}

% It is OKAY to include author information, even for blind
% submissions: the style file will automatically remove it for you
% unless you've provided the [accepted] option to the icml2025
% package.

% List of affiliations: The first argument should be a (short)
% identifier you will use later to specify author affiliations
% Academic affiliations should list Department, University, City, Region, Country
% Industry affiliations should list Company, City, Region, Country

% You can specify symbols, otherwise they are numbered in order.
% Ideally, you should not use this facility. Affiliations will be numbered
% in order of appearance and this is the preferred way.

\begin{center}
    {\large
      Sanghyeok Chu$^{1}$\quad 
      Seonguk Seo$^{1}$\quad
      Bohyung Han$^{1,2}$ \\
    $^{1}$ECE \& $^{2}$IPAI, Seoul National University \\
    }
    \texttt{\{sanghyeok.chu, seonguk, bhhan\}@snu.ac.kr} 
\end{center}

% You may provide any keywords that you
% find helpful for describing your paper; these are used to populate
% the "keywords" metadata in the PDF but will not be shown in the document
\icmlkeywords{Machine Learning, ICML}

\vskip 0.3in
]


% This command actually creates the footnote in the first column
% listing the affiliations and the copyright notice.
% The command takes one argument, which is text to display at the start of the footnote.
% The \icmlEqualContribution command is standard text for equal contribution.
% Remove it (just {}) if you do not need this facility.

%\printAffiliationsAndNotice{}  % leave blank if no need to mention equal contribution
%\printAffiliationsAndNotice{\icmlEqualContribution} % otherwise use the standard text.

%-------------------------------------------------------------------------
%                                    Abstract
%-------------------------------------------------------------------------
\begin{abstract}


The choice of representation for geographic location significantly impacts the accuracy of models for a broad range of geospatial tasks, including fine-grained species classification, population density estimation, and biome classification. Recent works like SatCLIP and GeoCLIP learn such representations by contrastively aligning geolocation with co-located images. While these methods work exceptionally well, in this paper, we posit that the current training strategies fail to fully capture the important visual features. We provide an information theoretic perspective on why the resulting embeddings from these methods discard crucial visual information that is important for many downstream tasks. To solve this problem, we propose a novel retrieval-augmented strategy called RANGE. We build our method on the intuition that the visual features of a location can be estimated by combining the visual features from multiple similar-looking locations. We evaluate our method across a wide variety of tasks. Our results show that RANGE outperforms the existing state-of-the-art models with significant margins in most tasks. We show gains of up to 13.1\% on classification tasks and 0.145 $R^2$ on regression tasks. All our code and models will be made available at: \href{https://github.com/mvrl/RANGE}{https://github.com/mvrl/RANGE}.

\end{abstract}



%-------------------------------------------------------------------------
%%                                  Introduction
%-------------------------------------------------------------------------
\section{Introduction}

Video generation has garnered significant attention owing to its transformative potential across a wide range of applications, such media content creation~\citep{polyak2024movie}, advertising~\citep{zhang2024virbo,bacher2021advert}, video games~\citep{yang2024playable,valevski2024diffusion, oasis2024}, and world model simulators~\citep{ha2018world, videoworldsimulators2024, agarwal2025cosmos}. Benefiting from advanced generative algorithms~\citep{goodfellow2014generative, ho2020denoising, liu2023flow, lipman2023flow}, scalable model architectures~\citep{vaswani2017attention, peebles2023scalable}, vast amounts of internet-sourced data~\citep{chen2024panda, nan2024openvid, ju2024miradata}, and ongoing expansion of computing capabilities~\citep{nvidia2022h100, nvidia2023dgxgh200, nvidia2024h200nvl}, remarkable advancements have been achieved in the field of video generation~\citep{ho2022video, ho2022imagen, singer2023makeavideo, blattmann2023align, videoworldsimulators2024, kuaishou2024klingai, yang2024cogvideox, jin2024pyramidal, polyak2024movie, kong2024hunyuanvideo, ji2024prompt}.


In this work, we present \textbf{\ours}, a family of rectified flow~\citep{lipman2023flow, liu2023flow} transformer models designed for joint image and video generation, establishing a pathway toward industry-grade performance. This report centers on four key components: data curation, model architecture design, flow formulation, and training infrastructure optimization—each rigorously refined to meet the demands of high-quality, large-scale video generation.


\begin{figure}[ht]
    \centering
    \begin{subfigure}[b]{0.82\linewidth}
        \centering
        \includegraphics[width=\linewidth]{figures/t2i_1024.pdf}
        \caption{Text-to-Image Samples}\label{fig:main-demo-t2i}
    \end{subfigure}
    \vfill
    \begin{subfigure}[b]{0.82\linewidth}
        \centering
        \includegraphics[width=\linewidth]{figures/t2v_samples.pdf}
        \caption{Text-to-Video Samples}\label{fig:main-demo-t2v}
    \end{subfigure}
\caption{\textbf{Generated samples from \ours.} Key components are highlighted in \textcolor{red}{\textbf{RED}}.}\label{fig:main-demo}
\end{figure}


First, we present a comprehensive data processing pipeline designed to construct large-scale, high-quality image and video-text datasets. The pipeline integrates multiple advanced techniques, including video and image filtering based on aesthetic scores, OCR-driven content analysis, and subjective evaluations, to ensure exceptional visual and contextual quality. Furthermore, we employ multimodal large language models~(MLLMs)~\citep{yuan2025tarsier2} to generate dense and contextually aligned captions, which are subsequently refined using an additional large language model~(LLM)~\citep{yang2024qwen2} to enhance their accuracy, fluency, and descriptive richness. As a result, we have curated a robust training dataset comprising approximately 36M video-text pairs and 160M image-text pairs, which are proven sufficient for training industry-level generative models.

Secondly, we take a pioneering step by applying rectified flow formulation~\citep{lipman2023flow} for joint image and video generation, implemented through the \ours model family, which comprises Transformer architectures with 2B and 8B parameters. At its core, the \ours framework employs a 3D joint image-video variational autoencoder (VAE) to compress image and video inputs into a shared latent space, facilitating unified representation. This shared latent space is coupled with a full-attention~\citep{vaswani2017attention} mechanism, enabling seamless joint training of image and video. This architecture delivers high-quality, coherent outputs across both images and videos, establishing a unified framework for visual generation tasks.


Furthermore, to support the training of \ours at scale, we have developed a robust infrastructure tailored for large-scale model training. Our approach incorporates advanced parallelism strategies~\citep{jacobs2023deepspeed, pytorch_fsdp} to manage memory efficiently during long-context training. Additionally, we employ ByteCheckpoint~\citep{wan2024bytecheckpoint} for high-performance checkpointing and integrate fault-tolerant mechanisms from MegaScale~\citep{jiang2024megascale} to ensure stability and scalability across large GPU clusters. These optimizations enable \ours to handle the computational and data challenges of generative modeling with exceptional efficiency and reliability.


We evaluate \ours on both text-to-image and text-to-video benchmarks to highlight its competitive advantages. For text-to-image generation, \ours-T2I demonstrates strong performance across multiple benchmarks, including T2I-CompBench~\citep{huang2023t2i-compbench}, GenEval~\citep{ghosh2024geneval}, and DPG-Bench~\citep{hu2024ella_dbgbench}, excelling in both visual quality and text-image alignment. In text-to-video benchmarks, \ours-T2V achieves state-of-the-art performance on the UCF-101~\citep{ucf101} zero-shot generation task. Additionally, \ours-T2V attains an impressive score of \textbf{84.85} on VBench~\citep{huang2024vbench}, securing the top position on the leaderboard (as of 2025-01-25) and surpassing several leading commercial text-to-video models. Qualitative results, illustrated in \Cref{fig:main-demo}, further demonstrate the superior quality of the generated media samples. These findings underscore \ours's effectiveness in multi-modal generation and its potential as a high-performing solution for both research and commercial applications.

%-------------------------------------------------------------------------
%                                Related work
%-------------------------------------------------------------------------
% !TEX root = ../main.tex

\section{Related Works}
\label{sec:related}

This section reviews existing approaches to video captioning, including both standard supervised learning methods and zero-shot methods. 
We also discuss video paragraph captioning, a task focused on detailed descriptions of a video using multiple sentences.

\vspace{-2mm}
\paragraph{Video captioning}
Recent supervised approaches leverage large-scale models pretrained on vision-language data and advanced architectures for improved video representation. 
For example, ClipBERT~\cite{lei2021less} and OmniVL~\cite{wang2022omnivl} incorporate multi-modal transformers to directly process video frames and generate contextual captions without extensive pre-processing.
More recent models, such as Flamingo~\cite{alayrac2022flamingo} and VideoCoCa~\cite{yan2022videococa}, perform vision-language pretraining using diverse datasets, which allows the models to generalize better across a range of video domains and tasks, including video captioning.

\vspace{-2mm}
\paragraph{Zero-shot video captioning}
Researchers have explored video captioning methods that bypass the need for paired video-text annotations during training. 
One prominent direction involves text-only training, where pretrained text decoders are used in conjunction with image-text aligned encoders such as CLIP~\cite{radford2021learning} and ImageBind~\cite{girdhar2023imagebind}. 
These methods, including DeCap~\cite{lidecap} and C$^{3}$~\cite{zhang2024connect}, align visual and textual representations within a shared embedding space to facilitate caption generation.
%
Another approach focuses on refining language model outputs at test time to better incorporate visual context.
ZeroCap~\cite{tewel2021zero} and related methods~\cite{Tewel_2023_BMVC} use CLIP-guided gradient updates to adjust language model features, while MAGIC~\cite{su2022language} employs a CLIP-based scoring mechanism to ensure semantic relevance. 
Although these methods were initially developed for image captioning tasks, they have been extended to video captioning by averaging frame-level captions into a single video-level description.
%
Recent techniques leverage the general reasoning capabilities of LLMs. 
For example, VidIL~\cite{wang2022language} uses a hierarchical framework that integrates multi-level textual representations derived from image-language models. 
By combining these representations with few-shot in-context examples, VidIL enables LLMs to perform a wide range of video-to-text tasks without extensive video-centric training. 
Similarly, Video ChatCaptioner~\cite{chen2023video} adopts an interactive framework where an LLM queries an image VLM across frames and aggregates the results to generate enriched spatiotemporal captions.
%

\vspace{-2mm}
\paragraph{Video paragraph captioning}
This task extends standard video captioning by generating coherent, multi-sentence descriptions that capture the semantics of events observed throughout an entire video. 
Unlike single-sentence captioning approaches, which typically focus on salient events, video paragraph captioning produces comprehensive and coherent captions in multiple sentences, reflecting a range of activities, background elements, and scene changes across various frames. 
To this end, MFT~\cite{xiong2018move} and PDVC~\cite{wang2021end} incorporate mechanisms for long-range temporal dependency modeling and multi-stage captioning, enabling nuanced descriptions that evolve naturally with the video. 
Vid2Seq~\cite{yang2023vid2seq} builds on this approach with a hierarchical structure that first detects key events and then generates descriptive sentences for the remaining content, maintaining a logical narrative flow. 
In contrast, Streaming GIT~\cite{zhou2024streaming} uses multi-modal pretrained transformers to produce real-time captions, facilitating seamless transitions across scenes in a continuous narrative.


%-------------------------------------------------------------------------
%                                  Method
%-------------------------------------------------------------------------
\section{Background}
% \begin{tcolorbox}[simplebox]
% We first formally define the problem and highlight its challenge. 
% Then we present an EM approach to address this challenge. 
% \end{tcolorbox}
% \vspace{-0.3cm}
% \subsection{Problem Statement }\label{sec_ps}

% Here’s a polished and enriched version of your problem formulation section, with improved clarity, precision, and academic tone:

% ---
\begin{figure}[t]
    \centering % Center the figure
    \includegraphics[width=\linewidth]{figs/example.pdf} % Include the figure
    \caption{\small \textbf{Example of Autonomous Code Integration.} \small We aim to enable LLMs to determine tool-usage strategies
based on their own capability boundaries. In the example, the model write code to solve the problem that demand special tricks, strategically bypassing its inherent limitations.} 
    \label{fig_example}
    \vspace{-0.2cm}
\end{figure}
\textbf{Problem Statement.} Modern tool-augmented language models address mathematical problems \( x_q \in \mathcal{X}_Q \) by generating step-by-step solutions that interleave natural language reasoning with executable Python code (Fig.~\ref{fig_example}). Formally, given a problem \( x_q \), a model \( \mathcal{M}_\theta \) iteratively constructs a solution \( y_a = \{y_1, \dots, y_T\} \) by sampling components \( y_t \sim p(y_t | y_{<t}, x_q) \), where \( y_{<t} \) encompasses both prior reasoning steps, code snippets and execution results \( \mathbf{e}_t \) from a Python interpreter. The process terminates upon generating an end token, and the solution is evaluated via a binary reward \( r(y_a,x_q) = \mathbb{I}(y_a \equiv y^*) \) indicating equivalence to the ground truth \( y^* \). The learning objective is formulated as:
\[
\max_{\theta} \mathbb{E}_{x_q \sim \mathcal{X}_Q} \left[r(y_a, x_q) \right]
\]

\noindent\textbf{Challenge and Motivation.} Developing autonomous code integration (AutoCode) strategies poses unique challenges, as optimal tool-usage behaviors must dynamically adapt to a model's intrinsic capabilities and problem-solving contexts. While traditional supervised fine-tuning (SFT) relies on imitation learning from expert demonstrations, this paradigm fundamentally limits the emergence of self-directed tool-usage strategies. Unfortunately, current math LLMs predominantly employ SFT to orchestrate tool integration~\citep{mammoth, tora, dsmath, htl}, their rigid adherence to predefined reasoning templates therefore struggles with the dynamic interplay between a model’s evolving problem-solving competencies and the adaptive tool-usage strategies required for diverse mathematical contexts.

Reinforcement learning (RL) offers a promising alternative by enabling trial-and-error discovery of autonomous behaviors. Recent work like DeepSeek-R1~\citep{dsr1} demonstrates RL's potential to enhance reasoning without expert demonstrations. However, we observe that standard RL methods (e.g., PPO~\cite{ppo}) suffer from a critical inefficiency (see Sec.~\ref{sec_ablation}): Their tendency to exploit local policy neighborhoods leads to insufficient exploration of the vast combinatorial space of code-integrated reasoning paths, especially when only given a terminal reward in mathematical problem-solving.

To bridge this gap, we draw inspiration from human metacognition -- the iterative process where learners refine tool-use strategies through deliberate exploration, outcome analysis, and belief updates. A novice might initially attempt manual root-finding via algebraic methods, observe computational bottlenecks or inaccuracies, and therefore prompting the usage of calculators. Through systematic reflection on these experiences, they internalize the contextual efficacy of external tools, gradually forming stable heuristics that balance reasoning with judicious tool invocation. 


To this end, \emph{our focus diverges from standard agentic tool-use frameworks~\citep{agentr}}, which merely prioritize successful tool execution. Instead, \emph{we aim to instill \emph{human-like metacognition} in LLMs, enabling them to (1) determine tool-usage based on their own capability boundaries (see the analysis in Sec.~\ref{sec_ablation}), and (2) dynamically adapt tool-usage strategies as their reasoning abilities evolve (via our EM framework).}
% For instance, while an LLM might solve a combinatorics problem via CoT alone, it should autonomously invoke code for eigenvalue calculations in linear algebra where symbolic computations are error-prone. Achieving this requires models to \emph{jointly optimize} their reasoning and tool-integration policies in a mutually reinforcing manner.


% Mirroring this metacognitive cycle, we propose an Expectation-Maximization (EM) framework that allows LLMs to develop AutoCode strategies via guided exploration (the E-step) and self-refinement (the M-step).


% \vspace{-0.3cm}
\section{Methodology}

Inspired by human metacognitive processes, we introduce an Expectation-Maximization (EM) framework that trains LLMs for autonomous code integration (AutoCode) through alternations (Fig.~\ref{fig_overview}):

\begin{enumerate}[leftmargin=0.5cm,topsep=1pt,itemsep=0pt,parsep=0pt]
    \item \emph{Guided Exploration (E-step):} Identifies high-potential code-integrated solutions by systematically probing the model's inherent capabilities.
\item \emph{Self-Refinement (M-step):} Optimizes the model's tool-usage strategy and chain-of-thought reasoning using curated trajectories from the E-step.
\end{enumerate}


\begin{figure*}[t]
    \centering
    \includegraphics[width=\linewidth]{figs/overview.pdf}
    \caption{\small \textbf{Method Overview.} \small (Left) shows an overview for the EM framework, which alternates between finding a reference strategy for guided exploration (E-step) and off-policy RL (M-step). (Right) shows the data curation for guided exploration. We generate \(K\) rollouts, estimate values of code-triggering decisions and subsample the initial data with sampling weights per Eq.~\ref{eq_sampling}.}
    \label{fig_overview}
\end{figure*}

\subsection{The EM Framework for AutoCode}

A central challenge in AutoCode lies in the code triggering decisions, represented by the binary decision \(c \in \{0, 1\}\).  While supervised fine-tuning (SFT) suffers from missing ground truth for these decisions, standard reinforcement learning (RL) struggles with the combinatorial explosion of code-integrated reasoning paths. Our innovation bridges these approaches through systematic exploration of both code-enabled (\(c=1\)) and non-code (\(c=0\)) solution paths, constructing reference decisions for policy optimization.

We formalize this idea within a maximum likelihood estimation (MLE) framework. Let \( P (r=1 | x_q;\theta\) denote the probability of generating a correct response to query \( x_q \) under model \(\mathcal{M}_\theta\). Our objective becomes:
\begin{align}
    \mathcal{J}_{\mathrm{MLE}}(\theta) \doteq \log P(r=1 | x_q; \theta) \label{eq_mle}
\end{align}
This likelihood depends on two latent factors: (1) the code triggering decision \(\pi_\theta(c | x_q)\) and (2) the solution generation process \(\pi_\theta(y_a | x_q, c)\). Here, for notation-wise clarity, we consider  code-triggering decision at a solution's beginning (\( c\) following \(x_q\) immediately). We show generalization to mid-reasoning code integration in Sec.~\ref{sec_impl}.

The EM framework provides a principled way to optimize this MLE objective in the presence of latent variables~\cite{prml}. We derive the evidence lower bound (ELBO): \( \mathcal{J}_{\mathrm{ELBO}}(s, \theta) \doteq \)
\begin{align}
    % \mathcal{J}_{\mathrm{MLE}}(\theta) &
    % \ge 
    \mathbb{E}_{s(c | x_q)}\left[\log \frac{\pi_\theta(c | x_q) \cdot P(r=1 | c, x_q; \theta)}{s(c | x_q)}\right] 
    % \\
     \label{eq_elbo}
\end{align}
where \(s(c | x_q)\) serves as a surrogate distribution approximating optimal code triggering strategies. It is also considered as the reference decisions for code integration. 

\noindent\textbf{E-step: Guided Exploration}  computes the reference strategy \(s(c | x_q)\) by maximizing the ELBO, equivalent to minimizing the KL-divergence: \( \max_s \mathcal{J}_{\mathrm{ELBO}}(s, \theta) = \)
\begin{align}
     - \mathrm{D_{KL}}\left(s(c | x_q) \| P(r=1, c | x_q; \theta)\right) \label{eq_estep}
\end{align}

The reference strategy \(s(c | x_q)\) thus approximates the posterior distribution over code-triggering decisions \(c\) that maximize correctness, i.e., \(P(r=1, c | x_q; \theta)\).  Intuitively, it guides exploration by prioritizing decisions with high potential: if decision \(c\) is more likely to lead to correct solutions, the reference strategy assigns higher probability mass to it, providing guidance for the subsequent RL procedure.

\noindent\textbf{M-step: Self-Refinement } updates the model parameters \(\theta\) through a composite objective:
\begin{multline}
\max_\theta \mathcal{J}_{\mathrm{ELBO}}(s, \theta) =\mathbb{E}_{\substack{c \sim s(c|x_q) \\ y_a \sim \pi_\theta(y_a|x_q, c)}} \Big[ r(x_q, y_a) \Big] \\- \mathcal{CE}\Big(s(c|x_q) \,\|\, \pi_\theta(c|x_q)\Big)\label{eq_mstep}
\end{multline}
The first term implements reward-maximizing policy gradient updates for solution generation, while while the second aligns native code triggering with reference strategies through cross-entropy minimization (see Fig.~\ref{fig_overview} for an illustration of the optimization). This dual optimization jointly enhances both tool-usage policies and reasoning capabilities.



\subsection{Practical Implementation}\label{sec_impl}
In the above EM framework, we alternate between finding a reference strategy \( s \) for code-triggering decisions  in the E-step, and perform reinforcement learning under the guidance from \( s \) in the M-step. We implement this framework through an iterative process of offline data curation and off-policy RL.

\noindent\textbf{Offline Data Curation.} We implement the E-step through Monte Carlo rollouts and subsampling. For each problem \(x_q\), we estimate the reference strategy as an energy distribution: 
\begin{equation}
    s^\ast(c | x_q)  = \frac{\exp\left(\alpha\cdot \pi_\theta(c | x_q) Q(x_q,c;\theta)\right)}{Z(x_q)}.\label{eq_sampling}
\end{equation}
where \( Q(x_q,c;\theta)\) estimates the expected value through \( K \) rollouts per decision, \(\pi_\theta(c|x_q) \) represents the model's current prior and the \( Z(x_q) \) is the partition function to ensure normalization. Intuitively, the strategy will assign higher probability mass to the decision \( c \) that has higher expected value \( Q(x_q,c;\theta)\) meanwhile balancing its intrinsic preference \( \pi_\theta(c|x_q)\). 

Our curation pipeline proceeds through: 
\begin{itemize}[leftmargin=0.5cm,topsep=1pt,itemsep=0pt,parsep=0pt]
\item Generate \(K\) rollouts for \(c=0\) (pure reasoning) and \(c=1\) (code integration), creating candidate dataset \(\mathcal{D}\).  
\item Compute \(Q(x_q,c)\) as the expected success rate across rollouts for each pair \((x_q,c)\).  
\item Subsample \(\mathcal{D}_{\text{train}}\) from \(\mathcal{D}\) using importance weights according to Eq.~\ref{eq_sampling}.  
\end{itemize}

To explicitly probe code-integrated solutions, we employ prefix-guided generation -- e.g., prepending prompts like \texttt{``Let’s first analyze the problem, then consider if python code could help''} -- to bias generations toward free-form code-reasoning patterns.

 This pipeline enables guided exploration by focusing on high-potential code-integrated trajectories identified by the reference strategy, contrasting with standard RL’s reliance on local policy neighborhoods. As demonstrated in Sec.~\ref{sec_ablation}, this strategic data curation significantly improves training efficiency by shaping the exploration space.





\noindent\textbf{Off-Policy RL.}
To mitigate distributional shifts caused by mismatches between offline data and the policy, we optimize a clipped off-policy RL objective. The refined M-step (Eq.~\ref{eq_mstep}) becomes:
\begin{multline}
    % \max_\theta 
    \underset{(x_q,y_a)}{\mathbb{E}}\left[
\text{clip}\left(\frac{\pi_\theta(y_a|x_q)}{\pi_{\text{ref}}(y_a|x_q)},1-\epsilon,1+\epsilon\right)\cdot A\right]
\\-\mathbb{E}_{(x_q,c)}\Big[\log \pi_\theta(c|x_q) \Big]\label{eq_finalm}
\end{multline}
where  \( (x_q, c, y_a) \) is sampled from the dataset \( \mathcal{D}_{\text{train}} \). The importance weight \(\frac{\pi_\theta(y_a|x_q)}{\pi_{\text{ref}}(y_a|x_q)}\) accounts for off-policy correction with PPO-like clipping. The advantage function \(A(x_q,y_a)\) is computed via query-wise reward normalization~\cite{ppo}. 

\noindent\textbf{Generalizing to Mid-Reasoning Code Integration.} Our method extends to mid-reasoning code integration by initiating Monte Carlo rollouts from partial solutions \((x_q, y_{<t})\). Notably, we observe emergence of mid-reasoning code triggers after initial warm-up with prefix-probed solutions. Thus, our implementation requires only two initial probing strategies: explicit prefix prompting for code integration and vanilla generation for pure reasoning, which jointly seed diverse mid-reasoning code usage in later iterations.


%-------------------------------------------------------------------------
%                                  Experiments
%-------------------------------------------------------------------------
\section{RESULTS}

\begin{figure*}[t!]
    %\vspace{-0.5cm}
    \centering
    \includegraphics[width=1\linewidth]{images/SystemArchitecture_2.png}
    \caption{From a single user demonstration, the system extracts the desired task goal state with the help of user interaction to solve ambiguities. Using the created environment variation, the system computes a task execution plan to bring new environments into the goal state. It sends the plan to agents in the environment to execute.} \label{fig:system_architecture}
\end{figure*}

Figure \ref{fig:system_architecture} shows our proposed framework to define a task goal, i.e. an environment goals state, and to turn a given environment into this goal state. The system visually observes a task execution by a user and segments this \underline{single} demonstration into \skills. \actions\ and \skills\ are defined in \ref{ssec:actions_skills}. The demonstration changed one or several properties of entities in the environment; environment which is now in the goal state. This information and the differences in entity properties from the start and end environment states are used to represent the task goal state. More on that in \ref{ssec:exp_model_def}. To turn a new environment into the defined goal state, a planning problem must be solved. This entails computing the differences between the environment's current state and the goal state, finding \actions\ that solve these differences, instantiating \skills\ that implement the \actions\ in the environment, selecting the \skills\ to execute by minimizing a given metric, and finally, sending the \skills\ to the agents in the environment to execute. This process is detailed in \ref{ssec:exp_model_use}.

% To prove the usability of our model, we present experiments to create a new goal state and turn the current environment into an (already-defined) goal state.

\subsection{Actions and Skills}\label{ssec:actions_skills}
A change in the environment is modeled using \actions, i.e. \textbf{what} has happened, and \skills, i.e. \textbf{how} did the change happen \cite{conceptHierarchyGeriatronicsSummit24}. Like in STRIPS \cite{strips} and PDDL \cite{pddl}, we represent \actions\ by their effects on entity properties and \skills\ by their preconditions and effects. \actions\ do not need preconditions because they only describe the \textbf{what} part of a change, not which conditions must be satisfied to perform the change. Besides preconditions and effects, \skills\ have a list of checks that tell our system if the \skill\ is executed in the environment. These checks allow the creation of a \skill\ recognition program, like the one presented in \cite{conceptHierarchyGeriatronicsSummit24}.
%\todo{citation of Geriatronics summit paper or the journal/unsubmitted paper?}
Using the \skill\ recognition output, we capture the changes from a task demonstration.

A \skill\ is thus the physical enactment of an abstract \action\ in an environment. Hence, \skills\ are correlated with \actions\ via their effects. A \skill\ can have more effects than a corresponding \action. For example, the \skill\ of scooping jam from a jar with a spoon implements the \action\ of \textit{TransferringContents}, but it also \textit{Dirties} the spoon.

\subsection{How To Parameterize The Model}\label{ssec:exp_model_def}
Creating a new goal state should be easier than manually specifying all variations wanted from the goal state. Doing so requires programming knowledge, which should not be needed to define goal states. One can let the system, which knows how to represent goal states, question the user about the desired state of the environment. However, this tedious process requires many questions from the system, also leading to decreased system usability.

Therefore, our approach is to let the user turn a given environment into a desired goal state and analyze the differences between the initial and final environment state to create the goal state representation. This single demonstration highlights the entity property values that were not in the desired goal state before being changed by the user.

We capture the demonstration via an Intel Realsense 3D camera \cite{realsense}, analyze the human skeleton via the OpenPose human pose estimation method \cite{openpose}, and determine the 3d pose of objects with AprilTag markers \cite{aprilTag}.

One demonstration contains the initial environment, not in the task goal state, and the final environment, in the goal state. The final environment state alone is not enough to create the environment variation. Thus, additional questions, guided by the differences between the two environment values, are posed by the system to the user to determine the desired variation in the environment state.

In a demonstration in which the user pours milk into a bowl, as shown in the top of Figure \ref{fig:system_architecture}, the initial question posed to the user is which entities that have changed properties are relevant for the goal state. If the goal state is to have more milk in the bowl, the milk carton is irrelevant; it is a means to achieve the goal state but not relevant to the goal itself. The bowl is thus selected as a relevant entity. 

Next, the list of relevant modified properties must also be determined for each relevant entity. It could have happened that during pouring of the milk into the bowl, the bowl's location also changed, e.g. touched accidentally by the user. Thus, not all modified properties could be relevant to the task. After selecting the relevant properties, the system knows from the knowledge base \cite{conceptHierarchyGeriatronicsSummit24} their \textit{ValueDomain} and the list of implemented \textbf{variations} for that \textit{ValueDomain}. Thus, the user parametrizes a selected \textbf{variation} from the list: choosing either a fixed value, a \textit{ValueDomain}-specific \textbf{RangeVariation} that must be parametrized, a conjunction or disjunction of \textbf{RangeVariations}, or the whole \textit{ValueDomain}.

In the example above, the user chooses the \textit{contentLevel} property as relevant. The system knows this property's defined set of values: a non-negative real number, and the possible range variation types: an open interval, a closed interval, an open-closed or closed-open interval, an intersection or union of intervals, etc. The user chooses a closed interval of $[0.28, 0.32]$ around the final \textit{contentLevel} value of $0.3L$. The user also specifies a variation for the entity's concept. It is generalized from that specific bowl instance to a \textit{LiquidContainer}.

After each modified property of each entity has a represented \textbf{variation}, the system automatically collects the entities into a variation of type $A$, see \ref{ssec:variations}, which is the assigned \textbf{variation} for the collection of entities in the environment.

Thus, the environment variation is determined in $\mathcal{O}\left(n\times m \times p\right)$ questions to the user, where $n$ is the number of entities in the environment, $m$ is the maximal number of properties that an entity can have, and $p$ is the maximal number of parameters that a \textbf{RangeVariation} needs to be represented. In the example above, $10$ questions were necessary to determine the task goal state shown in Figure \ref{fig:system_architecture} of a \textit{LiquidContainer} with \textit{contentLevel} between $0.28$ and $0.32L$. Figure \ref{fig:task_goal_state} shows the internal JSON-like representation of the goal state as the environment variation.
% 1 question which entities are relevant -> just bowl
% 1 question which properties are relevant -> contentLevel and concept
% 1 question about concept values being the same; should create variation?
% 1 question which ConceptValue-variation to select -> ConceptValue in Environment
% 1 question: which generalized concept?
% 1 question -> add other range-variation
% 1 question which Number-variation to select -> Interval
% 1 question: min-bound?
% 1 question: max-bound?
% 1 question -> add other range-variation

\begin{figure}[t!]
    %\vspace{-0.5cm}
    \centering
    \includegraphics[width=1\linewidth]{images/TaskDefinition_7.png}
    \caption{The goal state is a \textbf{RangeVariation} of the environment, of type EnvironmentDataRangeEntityVariation, which contains a \textbf{variation} of entities. This sub-variation is a \textbf{RangeVariation} of type MapRangeInstanceSubset (\textbf{variation} of type $A$, see \ref{ssec:variations}) and contains one instance \textbf{RangeVariation} of type InstanceRangePropertiesVariation. It defines the instance's concept \textbf{RangeVariation}, a \textit{LiquidContainer} to be found in the environment, and the \textit{contentLevel} property \textbf{RangeVariation}, the closed interval $\left[0.28, 0.32\right]$.} \label{fig:task_goal_state}
\end{figure}

\subsection{How To Use The Model}\label{ssec:exp_model_use}
Assuming the representation of a task's goal state is given, i.e. an environment variation, we detail our procedure (see Figure \ref{fig:experiment_description}) to turn the current environment into the goal state.

First, a Comparison between the environment and the goal variation is computed. This leads, as described in \ref{ssec:comparisons}, to a list of reasons why the environment is not in the variation. These reasons, i.e. differences $\delta$ of concept properties $p$, must be fixed to turn the environment into the goal state.

% Computing the differences between an EnvironmentData and an EnvironmentData-Variation, that has a Collection-Variation of type $A$, see \ref{ssec:variations}, is done via a maximal matching algorithm, where an edge between an entity $e$ an an entity variation $v_e$ means $e \in v_e$. 
For an EnvironmentData-Variation $v_{env}$ that defines a Collection-RangeVariation of type $A$, see \ref{ssec:variations}, computing the Comparison between an EnvironmentData $env$ and this target $v_{env}$ leads to a list of reasons for each entity $e_{env}$ in the entity collection of $env$, why $e_{env} \not\in v, \forall v \in A$. This can be seen in Figure \ref{fig:experiment_description}, where for each entity of \textit{LiquidContainer} concept in the environment, there is a list of differences, i.e. Comparisons, created for why the respective entity does not match the defined variation on the top-right.

\begin{figure}[t!]
    %\vspace{-0.5cm}
    \centering
    \includegraphics[width=1\linewidth]{images/Experiment_DescriptionUsingVariations_2.png}
    \caption{The procedure to turn an environment into its goal state is divided into 5 steps: computing differences, finding abstract solutions (i.e. \actions), computing practical solutions for the abstract ones (i.e. \actions\ $\rightarrow$ \skills), selecting the best practical solution, and executing the solution.} \label{fig:experiment_description}
\end{figure}
The second step of the procedure is to turn the list of differences into a list of \actions\ that can fix them. In notation, \action\ $A_x$ solves a difference in the concept property $p_x$. The system knows which properties \actions\ modify by analyzing the definition of their effects. Thus, \actions\ are created (parametrized) to fix the differences in entity properties.

% Because multiple instances can fit the instance variation, the third step is to match instances with the variations. Our matching optimization criterion is to minimize the amount of \textit{Actions} needed to fix the instances' property differences. \todo{continue!}

In the third step, each \action\ $A_x$ is converted into an execution plan $P_x$ that implements solving the difference $\delta_{p_x}$ in the environment. It is also possible that there is no possibility to implement the \action\ $A_x$ in the environment; this is represented as an execution plan $P_x = \emptyset$. An execution plan $P_x$ is otherwise, in its simplest form, a set of \skill\ alternatives $\left\{S_y\right\}$, where the \skill\ $S_y$ implements the \action\ $A_x$. There is the case to consider that the \skill\ $S_y$ has preconditions that are not met. And so, before executing the skill $S_y$, a different execution plan $P_{S_y}$ has to be computed and executed to allow the \skill\ $S_y$ to solve the property difference $\delta_{p_x}$. It is also possible that one single \skill\  $S_y$ is not enough to implement the \action\ $A_x$. Consider the case where the environment contains three cups with $0.1L$ of water, and the goal is to have one cup with $0.3L$ of content. One single \textit{Pouring} \skill\ is not enough to fulfill the goal; two \textit{Pouring} \skills\ must be executed. Thus, in the most general form, an execution plan $P_x = \left[\left\{ S_{iy}, P_{S_{iy}} \right\}_i\right]$ is a list of skill alternatives $\left\{ S_{iy}, P_{S_{iy}} \right\}_i$, that possibly contain other execution plans $P_{S_{iy}}$ to solve the skill's preconditions.

Our procedure to parameterize the \skills\ $S_y$ that implement the \action\ $A_x$ is a custom solution for each property $p_x$. One could backtrack through all possible parameter values of all possible skills to create a general solution that works for all properties. Another idea is to invert \skill\ effects and thus guide the \skill\ parameter search from the target variation to the value. However, both approaches would be computationally intense and would not create execution plans in a reasonable time. 
% reinforcement learning with policy for each property

The procedure to solve an entity $e$'s \underline{contentLevel} property difference searches for other \textit{Container} object instances in the environment, sorts them according to their content volume, and iterates through them in ascending order if $e.contentLevel \le target.contentLevel$; otherwise, in descending order. If a \skill\ $S$ can be executed with the two objects, that reduces the difference between $e.contentLevel$ and $target.contentLevel$, the \skill\ is added to the execution plan. If, after checking all objects, $e.contentLevel \not\in target.contentLevel$, there is no solution to solve this property difference.

Thus, the result of the third step is an execution plan $P_x$ for each entity property difference.

Fourth, after having the execution plans $P_x$ per entity-variation and entity, a \underline{solution selector} scores all solutions according to defined metrics and then, via a maximal matching algorithm, selects the solutions to execute to satisfy all variations of the Collection-RangeVariation of type $A$. The edges in the maximal matching have the cost of the solution score. For this paper, the scoring metric by the \underline{solution selector} is the number of steps of the execution plan.

The fifth and final step is to pass the execution plan to the agent(s) to execute in the environment. Figure \ref{fig:data_flow} presents the flow of data through the five steps.
We have used the Franka Emika Panda robot in CoppeliaSim \cite{coppeliaSim} to perform the computed execution plan.
% Note that the approach is independent of the used robot; only when instantiating \skills\ must the robot's abilities, manipulability region, and workspace be considered. How the \skills\ are executed in the environment is separated from the modeling of what must be done.

\begin{figure}[t!]
    % \vspace{-0.2cm}
    \centering
    \includegraphics[width=1\linewidth]{images/Experiment_DescriptionUsingVariations_DataFlow_2.png}
    \caption{Data flow when transforming an environment into a given goal state. $\Delta$ are differences of entity properties $p$, $A$ are \actions, $P$ is an execution plan and $S$ are \skills.} \label{fig:data_flow}
\end{figure}

The experiments aim to compute solution plans for solving the difference of the \textbf{contentLevel} property of \textit{Container} objects. For this, we consider the following criteria. $C1$: \textbf{variation} type = $\left\{\text{fixed},\text{interval},\text{interval union}\right\}$. $C2$: target relative to content = \{$\left\{t < cL \le cV \right\}$, $\left\{cL < t < cV \right\}$, $\left\{cL < t \ni cV \right\}$, $\left\{cL \le cV < t \right\}$\}, where $t$ is the \textbf{variation} value and $cL$ and $cV$ are the \textit{contentLevel} and \textit{contentVolume} properties respectively. $C3$: achievable in environment $ = \left\{\text{yes}, \text{no}\right\}$. Figure \ref{fig:experiment_table} presents planning results for different environments and the criteria described above. The lower table shows cases where the computed solution does not match the actual solution. This only happens when multiple instance variations are defined. The reason is that the implemented procedure to turn the list of differences into an execution plan treats each difference independently. Thus, dependencies between two variations are not accurately solved.

In the upper table of Figure \ref{fig:experiment_table}, there are two solutions for $C1.3$, $C2.3$, $C3.1$: one with the bowl $B$ as the instance in the \textbf{variation} $V1$, the other with $M$. The solution when $B$ is the matched instance has three steps: 1) pouring $0.1L$ from $M$ into $B$, 2) pouring $0.1L$ from $C1$ into $B$, and, finally, 3) pouring  $0.02L$ from $C2$ into $B$. This plan is sent to the robot in simulation and is executed as shown in Figure \ref{fig:robot_plan_execution}.

% \begin{figure}[t!]
%     % \vspace{-0.2cm}
%     \centering
%     \includegraphics[width=1\linewidth]{images/Experiment_Table_1Variation_compressed.png}
%     \caption{$B$ is a bowl with $0.5L$ \textit{contentVolume}, $M$ is a milk carton with $1.0L$ \textit{contentVolume}, $C1$ and $C2$ are cups with $0.3L$ \textit{contentVolume} each. Times, in seconds, averaged across 10 runs. Criteria $C2.4$ and $C3.1$ are mutually exclusive (a solution does not exist to let a container have more \textit{contentLevel} than its \textit{contentVolume}); thus, they are not included in the table.} \label{fig:experiment_table}
% \end{figure}
\begin{figure}[t!]
    % \vspace{-0.2cm}
    \centering
    \includegraphics[width=1\linewidth]{images/Experiment_Table_Results.png}
    \caption{$B$ is a bowl with $0.5L$ \textit{contentVolume}, $M$ is a milk carton with $1.0L$ \textit{contentVolume}, $C1$ and $C2$ are cups with $0.3L$ \textit{contentVolume} each. Times, in seconds, averaged across 10 runs. Criteria $C2.4$ and $C3.1$ are mutually exclusive (a solution does not exist to let a container have more \textit{contentLevel} than its \textit{contentVolume}); thus, they are not included in the upper table. The lower table presents results for open intervals and multiple variations in the environment.} \label{fig:experiment_table}
\end{figure}

\begin{figure}[t!]
    %\vspace{-0.1cm}
    \centering
    \includegraphics[width=1\linewidth]{images/Robot_PouringInBowl_M_PC1_PC2.png}
    \caption{Robot executing plan to bring $B$, the bowl, into the goal state. Because no liquids were simulated, the pouring amount was associated with the pouring time via: $t_{pour} = 10 * amount_{pour}$.} \label{fig:robot_plan_execution}
\end{figure}

%-------------------------------------------------------------------------
%                                  Conclusion
%-------------------------------------------------------------------------
\section{Conclusion}
We introduce a novel approach, \algo, to reduce human feedback requirements in preference-based reinforcement learning by leveraging vision-language models. While VLMs encode rich world knowledge, their direct application as reward models is hindered by alignment issues and noisy predictions. To address this, we develop a synergistic framework where limited human feedback is used to adapt VLMs, improving their reliability in preference labeling. Further, we incorporate a selective sampling strategy to mitigate noise and prioritize informative human annotations.

Our experiments demonstrate that this method significantly improves feedback efficiency, achieving comparable or superior task performance with up to 50\% fewer human annotations. Moreover, we show that an adapted VLM can generalize across similar tasks, further reducing the need for new human feedback by 75\%. These results highlight the potential of integrating VLMs into preference-based RL, offering a scalable solution to reducing human supervision while maintaining high task success rates. 

\section*{Impact Statement}
This work advances embodied AI by significantly reducing the human feedback required for training agents. This reduction is particularly valuable in robotic applications where obtaining human demonstrations and feedback is challenging or impractical, such as assistive robotic arms for individuals with mobility impairments. By minimizing the feedback requirements, our approach enables users to more efficiently customize and teach new skills to robotic agents based on their specific needs and preferences. The broader impact of this work extends to healthcare, assistive technology, and human-robot interaction. One possible risk is that the bias from human feedback can propagate to the VLM and subsequently to the policy. This can be mitigated by personalization of agents in case of household application or standardization of feedback for industrial applications. 

%-------------------------------------------------------------------------
%                                  Impact statements
%-------------------------------------------------------------------------

\clearpage

%-------------------------------------------------------------------------
%                                  References
%-------------------------------------------------------------------------

% In the unusual situation where you want a paper to appear in the
% references without citing it in the main text, use \nocite
%\nocite{langley00}
%\clearpage
\bibliography{main}
\bibliographystyle{icml2025}


%%%%%%%%%%%%%%%%%%%%%%%%%%%%%%%%%%%%%%%%%%%%%%%%%%%%%%%%%%%%%%%%%%%%%%%%%%%%%%%
%%%%%%%%%%%%%%%%%%%%%%%%%%%%%%%%%%%%%%%%%%%%%%%%%%%%%%%%%%%%%%%%%%%%%%%%%%%%%%%
% APPENDIX
%%%%%%%%%%%%%%%%%%%%%%%%%%%%%%%%%%%%%%%%%%%%%%%%%%%%%%%%%%%%%%%%%%%%%%%%%%%%%%%
%%%%%%%%%%%%%%%%%%%%%%%%%%%%%%%%%%%%%%%%%%%%%%%%%%%%%%%%%%%%%%%%%%%%%%%%%%%%%%%
\newpage
\appendix
\onecolumn

%---------------------------------------------------------------------------------------------------------------------
%                                    Supplementary Materials   : Section A
%---------------------------------------------------------------------------------------------------------------------
% !TEX root = ../main.tex

\section{Illustration of the Overall Framework}
\label{appendix_sec:detailed_overview}
We provide illustrations of the end-to-end flow of our proposed zero-shot video captioning framework, along with additional example in Figures~\ref{appendix_fig:framework_detailed}. 
The framework includes frame captioning via image VLMs, scene graph parsing for individual frames, scene graph consolidation to produce a unified representation, and graph-to-text translation for generate video generation.

\begin{figure*}[h!]
    \centering
                \includegraphics[width=0.95\textwidth]{appendix/figures/overview_supple_0_compressed.pdf}
    \caption{Illustrations of the end-to-end flow of the proposed framework. The pipeline consists of: (1) frame captioning via image VLMs, (2) scene graph parsing for individual frames, (3) scene graph merging to produce a unified representation, and (4) graph-to-text transformation for final caption generation.}
    \label{appendix_fig:framework_detailed}
\end{figure*}

%---------------------------------------------------------------------------------------------------------------------
%                                    Supplementary Materials   : Section B
%---------------------------------------------------------------------------------------------------------------------
% !TEX root = ../main.tex

\section{Additional Qualitative Results}
\label{appendix_sec:qual}
We provide additional qualitative examples for video captioning on the test set of MSR-VTT~\cite{xu2016msr-vtt} dataset in Figure~\ref{fig:qual_vc_supple} and for video paragraph captioning on the \text{\textit{ae-val}} set of the ActivityNet~\cite{krishna2017dense} Captions dataset in Figure~\ref{appendix_fig:qual_vpc_supple}. 
We compare the zero-shot results of our framework with several existing approaches, including 1) Tewel~\etal~\cite{Tewel_2023_BMVC}, which employs test-time optimization via gradient manipulation with CLIP embeddings 2) text-only training methods, \ie DeCap-COCO~\cite{lidecap} and C$^{3}$~\cite{zhang2024connect}, and 3) LLM summarization using Mistral-7B-Instruct-v0.3, 4) Video ChatCaptioner, an LLM-based video understanding method. 
Our method generates detailed and contextually rich captions, while other zero-shot methods often produce captions that are overly generic, irrelevant to the visual content, or occasionally nonsensical.


% !TEX root = ../main.tex

\begin{figure*}[!t]
    \centering
%    \scalebox{1}{
        \begin{tabular}{@{}cc@{}}
            \includegraphics[width=0.48\textwidth, height=5.8cm]{figures/qual/q_vc_0_short.pdf} & 
            %\hspace{0.5em}
            \includegraphics[width=0.48\textwidth, height=5.8cm]{figures/qual/q_vc_1_short.pdf} \\
            \includegraphics[width=0.48\textwidth, height=5.8cm]{figures/qual/q_vc_2_short.pdf} &
            %\hspace{0.5em}
            \includegraphics[width=0.48\textwidth, height=5.8cm]{figures/qual/q_vc_3_short.pdf} \\

        \end{tabular}
    \vspace{-2mm}
    \caption{Example of zero-shot video captioning results on MSR-VTT test set. We compare our results with other comparisons, listed from top to bottom as 1) Tewel \etal: test-time optimization method, 2) Decap-COCO: text-only trained on COCO 3) C$^{3}$: text-only trained on MSRVTT, 4) LLM summarization using Mistral-7B-Instruct-v0.3, 5) Video ChatCaptioner: LLM-based video understanding method, and 6) SGVC (Ours).}
    \label{fig:qual_vc}
    \vspace{-2mm}
\end{figure*}

% !TEX root = ../main.tex

% sh: subfigure with \textwidth results error
\begin{figure*}[t]
  \centering
  \includegraphics[width=0.98\linewidth]{figures/qual/q_vpc_2.pdf}
  \vskip -1em 
  \includegraphics[width=0.98\linewidth]{figures/qual/q_vpc_1.pdf}
  \vspace{-5mm}
  \caption{Example of zero-shot video paragraph captioning results on the \textit{ae-val} set of the ActivityNet~\cite{krishna2017dense} dataset, comparing LLM summarization using Mistral-7B-Instruct-v0.3 with SGVC (Ours).}
  \label{fig:qual_vpc}
\end{figure*} 


%---------------------------------------------------------------------------------------------------------------------
%                                    Supplementary Materials   : Section C
%---------------------------------------------------------------------------------------------------------------------

% !TEX root = ../main.tex

\clearpage
\section{Prompt Instructions} 
\label{appendix_sec:prompt}
\paragraph{Frame caption generation} 
Table~\ref{appendix_tab:imageVLM_fc_prompt} lists the instructional prompts, generated using ChatGPT-4, which guide the image VLM to generate the frame captions.
These prompts are designed to keep captions grounded in the visible content of the image, avoiding factual inaccuracies, unsupported details, or fabricated information.
A prompt was randomly selected for each frame, allowing captions to reflect diverse aspects of a video.
For all experiments, we employed LLAVA-NEXT-7B~\cite{liu2024llavanext} as a backbone model for caption generation.
%
\begin{table*}[h]
	\centering
	\small
	\caption{The list of instructional prompts for frame caption generation using an imageVLM.}
	\vspace{3mm}
	\label{appendix_tab:imageVLM_fc_prompt}
	\scalebox{0.95}{
		\fbox{
		    \begin{minipage}{0.8\linewidth}
		    {\linespread{1.2}\selectfont
		    \textbullet\ “Please describe what is happening in the image using one simple sentence. Focus only on what is visible.'' \par
   		    \textbullet\ “Now, provide a single sentence caption that describes only what is explicitly shown in the image” \par
   		    \textbullet\ “In one sentence, describe what you see in the image without adding any extra details.” \par
		    \textbullet\ “Provide a concise one-sentence description of the image, focusing on only the visible elements.” \par
  		    \textbullet\ “Please give a one-sentence caption that includes only what is clearly shown in the image.” \par
   		    \textbullet\ “Describe what is happening in the image in one simple sentence, without any added information.” \par
  		    \textbullet\ “Please generate a single sentence caption that describes only what can be seen in the image.” \par
  		    \textbullet\ “Provide a one-sentence description of the image, focusing solely on what is shown.” \par
		    \textbullet\ “Now, give a brief, one-sentence caption based strictly on the visible content in the image.” \par
  		    \textbullet\ “In a single sentence, describe what the image shows, without including anything extra.” \par
		    }
		    \end{minipage}
		}
	}
\end{table*}
%
\paragraph{LLM summarization} 
To construct the LLM summarization baseline in our experiments, we designed the prompts by combining the instructional prompt and example frame captions as illustrated in Table~\ref{appendix_tab:llm_summ_prompt}.
This inputs guide the LLM to generate a concise and coherent video-level summary. 
We used Mistral-7B-Instruct-v0.3 for this summarization task.
%
\begin{table*}[h]
	\centering
	\small
	\caption{Illustration of the input construction for LLM summarization, consisting of the instructional prompt and frame captions. We show an example for the frame captions.}
	\vspace{3mm}
	\label{appendix_tab:llm_summ_prompt}
	\scalebox{0.95}{
		\fbox{
			\begin{minipage}{0.9\linewidth}
			{\linespread{1.2}\selectfont
			\textbf{Instructional prompt:} \par
			Below are captions generated from individual frames of a video, each describing specific moments. Please review these frame-by-frame captions and summarize them into a single, compact caption. \par
			\par
			\textbf{Frame captions:} \par
			[1 / 6] A woman in a blue jacket is sitting in front of a sports logo. \par
			[2 / 6] Woman in blue jacket standing outdoors. \par
			[3 / 6] A man in a military uniform is standing in front of a navy sign. \par
			[4 / 6] Man in military uniform standing in front of navy sign. \par
			[5 / 6] The image shows three women wearing sports uniforms and holding medals, smiling and posing for the camera.\par
			[6 / 6] Three women wearing blue and white uniforms, smiling and holding medals. \par
			}
			\end{minipage}
		}
	}
\end{table*}
%


\end{document}


% This document was modified from the file originally made available by
% Pat Langley and Andrea Danyluk for ICML-2K. This version was created
% by Iain Murray in 2018, and modified by Alexandre Bouchard in
% 2019 and 2021 and by Csaba Szepesvari, Gang Niu and Sivan Sabato in 2022.
% Modified again in 2023 and 2024 by Sivan Sabato and Jonathan Scarlett.
% Previous contributors include Dan Roy, Lise Getoor and Tobias
% Scheffer, which was slightly modified from the 2010 version by
% Thorsten Joachims & Johannes Fuernkranz, slightly modified from the
% 2009 version by Kiri Wagstaff and Sam Roweis's 2008 version, which is
% slightly modified from Prasad Tadepalli's 2007 version which is a
% lightly changed version of the previous year's version by Andrew
% Moore, which was in turn edited from those of Kristian Kersting and
% Codrina Lauth. Alex Smola contributed to the algorithmic style files.
