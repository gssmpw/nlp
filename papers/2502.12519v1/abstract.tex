We present an efficient algorithm for the min-max correlation clustering problem. The input is a complete graph where edges are labeled as either positive $(+)$ or negative $(-)$, and the objective is to find a clustering that minimizes the $\ell_{\infty}$-norm of the disagreement vector over all vertices.

We resolve this problem with an efficient $(3 + \epsilon)$-approximation algorithm that runs in nearly linear time, $\tilde{O}(|E^+|)$, where $|E^+|$ denotes the number of positive edges. This improves upon the previous best-known approximation guarantee of 4 by Heidrich, Irmai, and
Andres~\cite{heidrich20244}, whose algorithm runs in $O(|V|^2 + |V| D^2)$ time, where $|V|$ is the number of nodes and $D$ is the maximum degree in the graph.

Furthermore, we extend our algorithm to the massively parallel computation (MPC) model and the semi-streaming model. In the MPC model, our algorithm runs on machines with memory sublinear in the number of nodes and takes $O(1)$ rounds. In the streaming model, our algorithm requires only $\tilde{O}(|V|)$ space, where $|V|$ is the number of vertices in the graph.

Our algorithms are purely combinatorial. They are based on a novel structural observation about the optimal min-max instance, which enables the construction of a $(3 + \epsilon)$-approximation algorithm using $O(|E^+|)$ neighborhood similarity queries. By leveraging random projection, we further show these queries can be computed in nearly linear time.

% We introduce fast algorithms for correlation clustering with respect to the Min Max objective
% that provide constant factor approximations on complete graphs. Our algorithms are the first
% purely combinatorial approximation algorithms for this problem. We construct a novel semimetric on the set of vertices, which we call the correlation metric, that indicates to our clustering
% algorithms whether pairs of nodes should be in the same cluster. The paper demonstrates
% empirically that, compared to prior work, our algorithms sacrifice little in the objective quality
% to obtain significantly better run-time. Moreover, our algorithms scale to larger networks that
% are effectively intractable for known algorithms.