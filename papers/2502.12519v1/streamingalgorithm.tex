\tcbset{
    algbox/.style={
        colframe=black,
        colback=gray!10,
        boxrule=0.8pt,
        arc=3mm,
        left=0mm,
        right=0mm,
        top=0mm,
        bottom=0mm,
        boxsep=0pt,
        sharp corners,
    }
}

In this section, we will show how to adapt Algorithm~\ref{alg:lp} to the streaming model. 

\subsection{Key Challenges}
For the streaming algorithm, we no longer need to compute $\Esim$. To compute $\mathcal{L}$, we first sample $A_v$ for each node at the beginning. Then, for each edge $vx$, we update $A \cdot \vec{N}[x]$ by adding $A_v$ to both $x$ and $v$. We can verify whether $|N[u] \Delta N[v]| \leq_\eta 2\phi$ using Lemma~\ref{lem:test}. The partition for high-degree nodes can then be computed using $O(n \log n / \epsilon^2)$ space.


However, an additional challenge arises, specifically in Lines \ref{ln:R} of Algorithm~\ref{alg:lp}. In Algorithm~\ref{alg:lp}, we select a node $u_i$ and identify all its neighbors that have at most $2\phi$ differing neighbors. While it is straightforward to check whether $|N[w] \Delta N[u_i]| \leq 2\phi$ using the $A$ matrix, determining whether $w \in N(u_i)$ is challenging unless we store all edges incident to $w$.

Fortunately, Assadi and Wang~\cite{AW22} introduced a method that allows sampling each node with probability $\Theta(\log n / d(v))$ and storing all edges incident to the sampled vertices, as stated in the following lemma:

\begin{lemma}[Lemma 3.14 of \cite{AW22}]
\label{lemma:assadi2021}
In $O(n \log n)$ space, with high probability (in the streaming model), we can sample each vertex $v$ with probability $\Theta(\log n / d(v))$ and store all edges incident to the sampled vertices.
\end{lemma}

Using this lemma, we can store the neighbors of a high-degree node whenever there are sufficiently many high-degree nodes inside $L_i$. However, in some cases, there may be very few high-degree nodes in $C^*_i$, and we might only be able to sample a node whose degree is less than $(3 + \eta) \phi$.

\Cref{alg:streamingcp} describes the modifications needed to address this issue. Instead of considering the neighbors of $u_i$, we consider the neighbors of $y_i$, where $y_i$ is any node from $\mathcal{C}^*_{u_i}$. We then check whether these neighbors have at most $2\phi$ differing neighbors with respect to $u_i$ and $y_i$. 

To make the algorithm more general, we also define a candidate set $\cand(L_i)$, which contains all nodes such that the differing neighbors between this node and any node from $L_i$ are at most $2\phi$, while the differing neighbors between this node and any high-degree node not in $L_i$ are greater than $2\phi$. This follows from Theorem~\ref{thm:nostealing}, which implies that any node in $\mathcal{C}^*_{u_i}$ must have a large differing neighborhood with high-degree nodes outside $\mathcal{C}^*_{u_i}$. The modification is shown in Lines~\ref{ln:streamingstart}--\ref{ln:streamingend} of Algorithm~\ref{alg:streamingcp} (within the boxed section).


Note that for any cluster $L_i \in \mathcal{L}$, it contains at least $|L_i|$ nodes, and each node has a degree of at most $|L_i| + \phi \leq 2|L_i|$. Thus, if we sample each node with probability $\Theta(\log n / d(v))$, we can always ensure that we sample some node $y_i \in \mathcal{C}^*_{u_i}$. 
We now show that Algorithm~\ref{alg:streamingcp} remains correct if we modify Lines~\ref{ln:streamingstart}--\ref{ln:streamingend} accordingly. In Section~\ref{sec:streamingimplementation}, we describe how to adapt Algorithm~\ref{alg:streamingcp} to the streaming model.



\begin{algorithm}[t]
\caption{{\textsc{StreamingClusterPhi}}$(G^+=(V, E^+), \phi, \eta)$\label{alg:streaming}\\
\textbf{Input}: A graph $G^{+}$ and parameters $\phi$ and $0 \leq \eta < 1$. \\
\textbf{Output}: A clustering $\mathcal{C}$ with $\obj(\mathcal{C})\leq (3+\eta)\phi$ or ``$\OPT > \phi$''}
\label{alg:streamingcp}
\begin{algorithmic}[1]
\Function {StreamingClusterPhi}{$G^+=(V, E^+), \phi, \eta$}
\Statex {\small \textit{\underline{$\triangleright$ Initialization}}}

    \For{$u,v \in V$} \Comment{$uv$ not necessarily in $E^{+}$.}
        \If{$|N[u] \Delta N[v]| \leq_{\eta} 2 \phi$}
            \State $E' \leftarrow E' \cup \{uv\}$.
        \EndIf
    \EndFor

    \State $\Vlow \leftarrow \{w \mid \mbox{$\deg(w) \leq (3+\eta) \phi$} \}, \Vhigh \leftarrow V \setminus \Vlow$
    \State $V_1 = \Vlow$

    \State Let $\mathcal{L} = \{L_i\}_{i=1}^{|\mathcal{L}|}$ be the partition formed by the connected components in $(\Vhigh ,E')$. 
        
    
     \For{$i$ from $1$ to $|\mathcal{L}|$} \label{ln:streamingupper}
    % \vspace{-\baselineskip}  
    \State Choose any node $u_i \in L_i$.
    {
    \begin{tcolorbox}[standard, algbox]
    \State \label{ln:streamingstart} $\cand(L_i) = \{ w \in V_1 \mid |N(u) \Delta N(w) | \leq_{\eta/2} 2\phi \text{ for $u \in L_i$, and } |N(v) \Delta N(w) | \not\leq_{\eta/2} 2\phi \text{ for $v \in \Vhigh \setminus L_i$ }\}$

    \State Let $y_i$ be any vertex in $\mathcal{C}^{*}_{u_i}$.

    \State \label{ln:streamingend} $R(y_i) = \{ w \in V_i \cap \bm{N[y_i]} \mid |N[w] \Delta N[y_i]|  \leq_{\eta/2} 2\phi  \} \cap \cand(L_i)$
    \end{tcolorbox}
    }

% \State Choose any node $u_i \in L_i$.
        % \State \label{ln:streamingend} $R(u_i) = \{ w \in V_i \cap \bm{N[y_i]} \mid |N[w] \Delta N[u_i]|  \leq_{\eta/2} 2\phi  \mbox{ and }  \bm{|N[w] \Delta N[y_i]|  \leq_{\eta/2} 2\phi}  \mbox{ and } \nexists u \in \Vhigh \setminus L_i \mbox{ s.t.~}  |N[w] \Delta N[u]| \leq_{\eta/2} 2\phi \}$
 
    \State  \label{ln:streamingC} $C_i \leftarrow L_i \cup R(y_i)$                                       
    \State $V_{i + 1} \leftarrow V_i \setminus R(y_i)$ 

    \EndFor   \label{ln:streaminglower}
     \If{for some $i$ there exists $u \in C_i$ such that $\rho_{C_{i}}(u) > (3+\eta)\phi$}
    \State \Return ``$\OPT > \phi$'' 
    \Else
    \State \Return $\mathcal{C} = \{C_i\}_{i=1}^{|\mathcal{L}|} \cup \bigcup_{v \in V_{|\mathcal{L}|+1}}\{ \{v\} \}$
    \EndIf

\EndFunction
\end{algorithmic}
\end{algorithm}

\subsection{Approximate Ratio}

In this section, we show that the modifications to Lines~\ref{ln:streamingstart}--\ref{ln:streamingend} of Algorithm~\ref{alg:streamingcp} do not affect the approximation ratio. Since we introduce the concept of a candidate set, we first prove that the candidate sets considered in each round are disjoint, ensuring that in each iteration, we consider distinct sets.

\begin{lemma}
\label{lem:canddisjoint}
Assume that $\OPT \leq \phi$. For any $i \neq j$, the candidate sets are disjoint, i.e., $\cand(L_i) \cap \cand(L_j) = \emptyset$. Moreover, we have $\mathcal{C}^*_{u_i} \cap V_1 \subset \cand(L_i)$.
\end{lemma}

\begin{proof}
We proceed by contradiction. Suppose there exists a node $w \in \cand(L_i) \cap \cand(L_j)$. Since $w \in \cand(L_i)$, we know that for any $v \in L_j$, it holds that $|N(v) \Delta N(w)| \not\leq_{\eta / 2} 2\phi$, which contradicts the assumption that $w \in \cand(L_j)$. Hence, the candidate sets must be disjoint.

For any node $w \in \mathcal{C}^*_{u_i} \cap V_1$, Lemma~\ref{lem:highdegreeclustering} ensures that $|N(u) \Delta N(w)| \leq_{\eta/2} 2\phi$ for all $u \in \mathcal{C}^*_{u_i} \cap \Vhigh$. On the other hand, Theorem~\ref{thm:nostealing} states that $|N(w) \Delta N(v)| \not\leq_{\eta/2} 2\phi$ for any high-degree node not in $\mathcal{C}^*_{u_i}$. Thus, $\mathcal{C}^*_{u_i} \cap V_1 \subset \cand(L_i)$, completing the proof.
\end{proof}

We only add nodes from $\cand(L_i)$ as a condition for $w$ to be included in $R(y_i)$. Based on Lemma~\ref{lem:canddisjoint}, we conclude that $R(y_i)$ does not steal low-degree vertices from other clusters in the optimal clustering. 

Now, we are able to state our key lemma. The proof follows a similar structure to the approximation ratio proof for Algorithm~\ref{alg:lp}. The analogues of \Cref{lem:inclusion} and \Cref{lem:closeness} hold as follows.

\begin{lemma}\label{lem:inclusion2} 
Let $C^{*}_i$ be the cluster in $\mathcal{C}^{*}$ such that $L_i \cap \Vhigh = {C}^{*}_{i} \cap \Vhigh$. For any $i$, we have $C^{*}_i \cap \bm{N[y_i]} \subseteq C_i$. 
\end{lemma}

\begin{proof}
If $w \in C^{*}_i \cap \Vhigh$, then $w \in C_i$ because $C_i = L_i \cup R(u_i)$ and ${L}_i \cap \Vhigh = C^{*}_i \cap \Vhigh$ by assumption. 

Now, consider the case where $w \in C^{*}_i \cap \Vlow \cap N[y_i]$. To establish that $w \in C_i$, it suffices to show that $|N[w] \Delta N[y_i]| \leq_{\eta/2} 2\phi$ and $w \in \cand(L_i)$. 

Since $w, y_i \in C^{*}_i$, by \Cref{lem:samecluster}, we have $|N[w] \Delta N[y_i]| \leq 2\phi$, which implies that $|N[w] \Delta N[y_i]| \leq_{\eta/2} 2\phi$ must hold. Furthermore, by \Cref{lem:canddisjoint}, since $w \in C^*_i$, we conclude that $w \in C^*_i \cap \Vlow \subset \cand(L_i)$. 
\end{proof}


Similar to \Cref{lem:closeness}, we show that $N[y_i]$, $C_i$, and $C^*_i$ do not differ a lot.


\begin{lemma}
\label{lem:closeness2} 
Let $C^{*}_i$ be the cluster in $\mathcal{C}^{*}$ such that ${L}_i \cap \Vhigh = {C}^{*}_{i} \cap \Vhigh$. For any $i$, $|N[y_i] \Delta C_i| \leq \phi$ and $|C^{*}_i \Delta C_{i}| \leq \phi$. 
\end{lemma}


\begin{proof}
We first show that $|N[y_i] \Delta C_i| \leq \phi$.
\begin{align}
|\Vlow \cap (N[y_i] \Delta C_i)| &= |\Vlow \cap (N[y_i] \setminus C_i)| + |\Vlow \cap (C_i \setminus N[y_i])| \nonumber \\
&= |\Vlow \cap (N[y_i] \setminus C_i)|  && (\Vlow \cap C_i) \subseteq N[y_i]  \nonumber  \\
&\leq |\Vlow \cap (N[y_i] \setminus C^{*}_i)|  && \mbox{by \Cref{lem:inclusion2}}   \nonumber \\
&\leq |\Vlow \cap (N[y_i] \Delta C^{*}_i)|  \label{eqn:streamingdiffneighborCstar1}
\end{align}

Moreover, since $C^{*}_i \cap \Vhigh = C_i \cap \Vhigh$, it must be the case that 
\begin{align}
|(N[y_i] \Delta C_i) \cap \Vhigh| = |(N[y_i] \Delta C^{*}_i) \cap \Vhigh|  \label{eqn:streamingdiffneighborCstar2}
\end{align}

Therefore,
\begin{align*}
|N[y_i] \Delta C_i| &= |(N[y_i] \Delta C_i) \cap \Vlow| + |(N[y_i] \Delta C_i) \cap \Vhigh| \\
&\leq |(N[y_i] \Delta C^{*}_i) \cap \Vlow| + |(N[y_i] \Delta C^{*}_i) \cap \Vhigh| && \mbox{by (\ref{eqn:streamingdiffneighborCstar1}) and (\ref{eqn:streamingdiffneighborCstar2})} \\
& = |N[y_i] \Delta C^{*}_i| \leq \phi  && \mbox{$u_i \in C^{*}_i$ and $\rho_{C^{*}}(y_i) \leq \phi$}
\end{align*}

Now we show that $|C^{*}_i \Delta C_{i}| \leq \phi$. 

\begin{align*}
&|C^{*}_i \Delta C_{i}| \\
&= |\Vlow \cap (C^{*}_i \Delta C_{i})| && C^{*}_i \cap \Vhigh = C_i \cap \Vhigh \\
&= |\Vlow \cap (C^{*}_i \setminus C_{i})| + |\Vlow \cap (C_i \setminus C^{*}_{i})| \\
&\leq |\Vlow \cap (C^{*}_i \setminus N[y_i])| + |\Vlow \cap (C_i \setminus C^{*}_{i})| && C^{*}_i \cap N[y_i] \subseteq C_i \\
&\leq |\Vlow \cap (C^{*}_i \setminus N[y_i])| + |\Vlow \cap (N[y_i] \setminus C^{*}_{i})| && \Vlow \cap C_i \subseteq V_1 \cap N[y_i] \\
&= |\Vlow \cap (C^{*}_i \Delta N[y_i])| \leq |C^{*}_i \Delta N[y_i]| \leq \phi
\end{align*}


% \begin{align*}
% &|C^{*}_i \Delta C_{i}| \\
% &= |\Vlow \cap (C^{*}_i \Delta C_{i})| && C^{*}_i \cap \Vhigh = C_i \cap \Vhigh \\
% &= |\Vlow \cap (C^{*}_i \setminus C_{i})| + |\Vlow \cap (C_i \setminus C^{*}_{i})| \\
% &= |\Vlow \cap (C^{*}_i\cap N[y_i] \setminus C_{i})| + |\Vlow \cap (C^{*}_i \setminus N[y_i] \setminus C_{i})| \\
% & \hspace{50mm} + |\Vlow \cap (C_i \setminus C^{*}_{i})| \\
% &= |\Vlow \cap (C^{*}_i\cap N[y_i] \setminus C_{i})| + |\Vlow \cap (C^{*}_i \setminus N[y_i])| \\
% & \hspace{50mm} + |\Vlow \cap (C_i \setminus C^{*}_{i})| && \Vlow \cap C_i \subseteq \Vlow \cap N[y_i]\\
% &= |\Vlow \cap (C^{*}_i \setminus N[y_i])| + |\Vlow \cap (C_i \setminus C^{*}_{i})| && C^{*}_i\cap N[y_i] \subseteq C_i \\
% &\leq |\Vlow \cap (C^{*}_i \setminus N[y_i])| + |\Vlow \cap (N[y_i] \setminus C^{*}_{i})| && \Vlow \cap C_i \subseteq V_1 \cap N[u_i] \\
% &= |\Vlow \cap (C^{*}_i \Delta N[y_i])| \leq |C^{*}_i \Delta N[y_i]| \leq \phi
% \end{align*}
\end{proof}

Using \Cref{lem:closeness2}, we can show that the approximation ratio remains the same for Algorithm~\ref{alg:streamingcp}.

\begin{theorem} Suppose that $\OPT \leq \phi$, the above modification outputs a clustering $\mathcal{C}$ with $\obj(\mathcal{C})\leq (3+\eta) \phi$.  \end{theorem}
\begin{proof}
Consider a component $C_i$ of $\mathcal{C}$. If $C_i$ is a singleton $\{v\}$ with $\deg(v) \leq (3+\eta)\phi$, then obviously, $\rho_{\mathcal{C}}(v) \leq (3+\eta)\phi$. Otherwise, $C_i$ contains some vertex $x$ with $\deg(x) >  (3+\eta)\phi$. By \Cref{lem:highdegreeclustering}, there exists $C^{*}_i$ such that $C_i \cap \Vhigh = {C}^{*}_{i} \cap \Vhigh$.

Let $v$ be any vertex in $C_i$. We will show that $\rho_{\mathcal{C}}(v) \leq 3\phi$. Suppose that $v \in C^{*}_i \cap C_i$, we have: 
\begin{align*}
\rho_{\mathcal{C}}(v) &= |N[v] \Delta C_i|\\
&\leq |N[v] \Delta C^{*}_i | + |C^{*}_i \Delta C_i|  && \mbox{by \Cref{lem:triangleineq}}\\
&= \rho_{\mathcal{C^{*}}}(v) + |C^{*}_i \Delta C_i| && \mbox{by \Cref{lem:closeness2}}\\
&\leq \phi + \phi  = 2\phi
\end{align*}
Otherwise, if $v \notin C^{*}_i \cap C_i$ then it must be the case that $v \in C_i \setminus C^{*}_i$. In such a case, $v$ must be a vertex in $R(u_i)$. This implies $|N[v] \Delta N[y_i]| \leq_{\eta/2} 2\phi$ and thus $|N[v] \Delta N[y_i]| \leq (1+\eta/2)\cdot2\phi$. Therefore:
\begin{align*}
\rho_{\mathcal{C}}(v) = |N[v] \Delta C_i| 
&\leq |N[v] \Delta N[y_i]| + |N[y_i] \Delta C_i| && \mbox{by \Cref{lem:triangleineq} and \Cref{lem:closeness2}} \\
&\leq (1+\eta/2)\cdot 2\phi + \phi = (3+\eta)\phi
\end{align*}
\end{proof}

\subsection{Implementation in the streaming model}
\label{sec:streamingimplementation}

To implement Algorithm~\ref{alg:streamingcp} in the streaming model, we still have several challenges to address. First, the cluster $\mathcal{C}^*_{u_i}$ belongs to the optimal solution, meaning we cannot directly verify whether a sampled node $y_i$ is in $\mathcal{C}^*_{u_i}$. To overcome this, instead of selecting a node from $\mathcal{C}^*_{u_i}$, we consider every node $y$ from the sampled set $S$ that also belongs to the candidate set $\cand(L_i)$:

\[
    R(y) = \{ w \in V_i \cap \bm{N[y]} \mid |N[w] \Delta N[y]|  \leq_{\eta/2} 2\phi \} \cap \cand(L_i).
\]

If $y \in \mathcal{C}^*_{u_i}$, then from the previous section, we know that the final cluster $C_i$ satisfies $\max_{u \in C_i} \rho_{C_i}(u) \leq (3 + \eta) \phi$. If $y \not\in \mathcal{C}^*_{u_i}$, it is still possible to find a cluster $C_i = L_i \cup R(y)$ such that $\max_{u \in C_i} \rho_{C_i}(u) \leq (3 + \eta) \phi$. 

A key observation in this case is that we never steal any low-degree nodes from another cluster $C^*_j$ for $j > i$. This follows from the fact that every time we include $w$ into $R(y)$, we have $w \in \cand(L_i)$, and by Lemma~\ref{lem:canddisjoint}, the candidate sets are disjoint. 

Combining these observations, we conclude that even if we choose an incorrect node $y \notin \mathcal{C}^*_{u_i}$, it does not affect later iterations of the algorithm. Moreover, the order of iteration does not matter—that is, we can process different high-degree node clusters $L_i$ in any order, and the final clustering remains the same when $\OPT \leq \phi$. This property is crucial when using random projection for estimation.


\textbf{Algorithm Description} Now, we describe the streaming implementation, given in Algorithm~\ref{alg:streamingcpfinal}. The algorithm samples a matrix $A$ to check whether $|N[u] \Delta N[v]| \leq_{\eta} 2\phi$, which requires $O(n\log n/\epsilon^2)$ space. Then, using Lemma~\ref{lemma:assadi2021}, it samples each node with probability $\Theta(\log n / d(v))$ along with its neighboring edges. To execute Lines~\ref{ln:streamingfinalstart}--\ref{ln:streamingfinalend}, we iterate over every node in $S \cap \cand(L_i)$. 

We use random projection to compute $\cand(L_i)$, and checking the objective value of $C_i$ can also be done via random projection. At the end, after processing all high-degree clusters, we either output a clustering or report that the current $\OPT > \phi$. 

Throughout the process, we must be careful when using random projection for estimation. This is because dependencies exist both within the same round and across different rounds for $C_i$, making it difficult to bound the probability of correct estimation.

\begin{algorithm}[H]
\caption{{\textsc{StreamingClusterPhiFinal}}$(G^+=(V, E^+), \phi, \eta)$\label{alg:streamingfinal}\\
\textbf{Input}: A graph $G^{+}$ and parameters $\phi$ and $0 \leq \eta < 1$. \\
\textbf{Output}: A clustering $\mathcal{C}$ with $\obj(\mathcal{C})\leq (3+\eta)\phi$ or ``$\OPT > \phi$''}
\label{alg:streamingcpfinal}
\begin{algorithmic}[1]
\Function {StreamingClusterPhiFinal}{$G^+=(V, E^+), \phi, \eta$}
\Statex {\small \textit{\underline{$\triangleright$ Initialization}}}

    \For{$u,v \in V$} \Comment{$uv$ not necessarily in $E^{+}$.}
        \If{$|N[u] \Delta N[v]| \leq_{\eta} 2 \phi$}
            \State $E' \leftarrow E' \cup \{uv\}$.
        \EndIf
    \EndFor

    \State $\Vlow \leftarrow \{w \mid \mbox{$\deg(w) \leq (3+\eta) \phi$} \}, \Vhigh \leftarrow V \setminus \Vlow$
    \State $V_1 = \Vlow$

    \State Let $\mathcal{L} = \{L_i\}_{i=1}^{|\mathcal{L}|}$ be the partition formed by the connected components in $(\Vhigh ,E')$.
        
    \begin{tcolorbox}[standard, algbox]
    
    \State Sample each node $v \in V$ with probability $\Theta(\log n / d(v))$ independently.
    
    \State Let the set of sampled nodes be $S$.

    \end{tcolorbox}

    \For{$i$ from $1$ to $|\mathcal{L}|$} \label{ln:streamingfinalupper}
    \State \label{ln:streamingfinalstart} $\cand(L_i) = \{ w \in V_1 \mid |N(u) \Delta N(w) | \leq_{\eta/2} 2\phi \text{ for $u \in L_i$, and } |N(v) \Delta N(w) | \not\leq_{\eta/2} 2\phi \text{ for $v \in \Vhigh \setminus L_i$ }\}$
    \begin{tcolorbox}[standard, algbox]
    \For{$y \in S \cap \cand(L_i)$}   
    % \hsinhao{it looks it may be easier to set $y_i \in S \cap Cand(L_i)$}

    \State \label{ln:streamingfinalend} $R = \{ w \in V_i \cap \bm{N[y]} \mid |N[w] \Delta N[y]|  \leq_{\eta/2} 2\phi  \} \cap \cand(L_i)$
    \State  \label{ln:streamingfinalc}$C_i \leftarrow L_i \cup R$
    \If{$\max_{u \in C_i} \rho_{C_{i}}(u) \leq (3+\eta) \phi$}
        \State Break
    \EndIf
    \EndFor
    \If{$\max_{u \in C_i} \rho_{C_{i}}(u) \leq (3+\eta)\phi$}
        \State $V_{i + 1} \leftarrow V_i \setminus R$
    \Else
        \State \Return ``$\OPT > \phi$''
    \label{ln:streamingfinalend2} \EndIf
    \end{tcolorbox}
    
    \EndFor   \label{ln:streamingfinallower}
\State \Return $\mathcal{C} = \{C_i\}_{i=1}^{|\mathcal{L}|} \cup \bigcup_{v \in V_{|\mathcal{L}|+1}}\{ \{v\} \}$

\EndFunction
\end{algorithmic}
\end{algorithm}



\textbf{Remove dependency for random projection} One technical challenge is checking the value $\max_{u \in C_i} \rho_{C_i}(u)$. The matrix $A$ cannot be used for random projection in this step because it was originally used to compute $C_i$. If we treat $\vec{C_i} \in \{0,1\}^n$ as the characteristic vector of $C_i$ and use $|A \cdot \vec{C_i} - A \cdot \vec{N}[u]|$ to estimate $|C_i \Delta N[u]|$, there exists a dependency between $C_i$ and $A$. Moreover, there may be up to $2^n$ different $C_i$, and if $C_i$ is chosen in a way that maximizes the difference between $|A \cdot \vec{C_i} - A \cdot \vec{N}[u]|$ and $|C_i \Delta N[u]|$, we can no longer apply Lemma~\ref{lem:test} for estimation.

To resolve this, we introduce a second $k \times n$ matrix $B$. When estimating $|C_i \Delta N[u]|$, we compute $|B\cdot \vec{C_i} - B\cdot \vec{N}[u]|$. Since $C_i$ is independent of $B$, random projection yields a correct estimate with high probability.

Another issue arises due to potential dependencies between different rounds of Lines~\ref{ln:streamingfinalstart}--\ref{ln:streamingfinalend}. This dependency is not problematic for the $A$ matrix, as at most $O(n^2)$ neighborhood queries need to be evaluated. Since each query succeeds with high probability, a union bound ensures that all queries for a pair of nodes $u, v$ are answered correctly with high probability. However, arguing about the dependency between different rounds for the $B$ matrix is more delicate. 

Indeed, if we remove the candidate set, adding nodes in the $i$-th iteration may affect later iterations, influencing how $C_i$ is chosen. Fortunately, thanks to the candidate set, Lemma~\ref{lem:canddisjoint} ensures that the nodes considered in each round are disjoint. Thus, $C_i$ does not depend on any $C_j$ for $j \neq i$ and only depends on the randomness of the $A$ matrix and the graph itself. Consequently, when using the $B$ matrix to estimate $|C_i \Delta N[u]|$, it is valid to reuse $B$ across different rounds.

 


Lastly, we must show the existence of $y \in \mathcal{C}^*_{u_i}$ to guarantee a good cluster. This is formally captured in the following lemma:

\begin{lemma}
Consider Algorithm~\ref{alg:streamingcpfinal}. Assume that $\OPT \leq \phi$. For each $i$, let $C^*_i$ be the cluster in $\mathcal{C}^*$ such that $L_i \cap \Vhigh = C^*_i \cap \Vhigh$. Then, with high probability, $|C^*_i \cap \cand(L_i) \cap S | \geq 1$.
\end{lemma}

\begin{proof}
From Lemma~\ref{lem:canddisjoint}, we know that $C^*_i \cap V_i = C^*_i \cap V_1$, meaning no nodes are lost at iteration $i$ for $C^*_i$. Note that there exists a node $u_i \in C^*_i$ with $d(u_i) \geq 3\phi$. We also know that $d(u_i) - \phi \leq |C^*_i| \leq d(u_i) + \phi$, since otherwise we would have $\rho_{C^*_i}(u_i) > \phi$. Additionally, for any $v \in C^*_i$, we must have $d(v) \leq d(u_i) + 2\phi \leq d(u_i) + (2/3)d(u_i) \leq (5/3) d(u_i)$, as otherwise $\rho_{C^*_i}(v) > \phi$. As $d(v) = O(d(u_i))$, each $v$ is sampled with probability at least $O(\log n / d(u_i))$

Thus, at least $d(u_i) - \phi \geq d(u_i) / 2$ nodes in $C^{*}_i$ are sampled with probability at least $\Omega(\frac{\log n}{d(u_i)})$. By a standard Chernoff bound, we conclude that at least one node from $C^*_i$ is sampled with high probability.
\end{proof}

Combining all points together, we now present our main lemma for the streaming model.

\begin{lemma}[Item \ref{itm:main3} of \Cref{thm:thmmain}]
\label{lemma:mainresultstreaming}
Given a min-max correlation clustering instance $G = (V, E)$, for any $\epsilon > 0$, there exists a randomized streaming algorithm that, in one round, outputs a clustering that is a $(3 + \epsilon)$-approximation for min-max correlation clustering. This algorithm succeeds with high probability and uses a total space of $O(n \log n/\epsilon^2)$.
\end{lemma}

\begin{proof}
We start with $\phi = n$ and run Algorithm~\ref{alg:streamingcpfinal}, which either returns a clustering $\mathcal{C}$ such that $\obj(\mathcal{C}) \leq (3 + \eta) \phi$ or determines that $\OPT > \phi$. To refine $\phi$, we perform a binary search, which takes $O(\log n)$ rounds.

During this process, we need to store the sampled nodes and their neighbors, requiring $O(n\log n)$ space. Additionally, we store the vector $A\cdot \vec{N}[u]$ to answer differing neighbor queries. To compute $\rho_{C_i}(u)$, we use the matrix $B$, where we store $B \cdot \vec{N}[v]$ and compute $B\cdot C_i$ as $\sum_{u} B_u$. 

In summary, Algorithm~\ref{alg:streamingcpfinal} produces a $(3+\epsilon)$-approximate solution with a space complexity of $O(n \log n / \epsilon^2)$.
\end{proof}


\begin{comment}
\subsection{Measurements}
We need the following to show that we perform at most $O(\log n/ \epsilon)$ measurements per cluster. 
\begin{lemma}
Consider a cluster $C^{*}_i$ in the optimal clustering. Let $u_i \in C^{*}_i$ be a high-degree vertex whose degree is at least $(3+\eta)\phi$. Let $T = \{y \in V \setminus C^{*}_i \mid |N[y] \Delta N[u_i|| \leq (1+\eta/4)2\phi\}$. It must be the case that $|T| = O(\phi/\eta)$. 
\end{lemma}

\begin{proof}
Let $E(A,B)$ be the set of positive edges between $A$ and $B$. First, we show that $|E(C^{*}_i, T)| \geq |T| \cdot \phi \cdot \eta /2$. 

Let $y$ be a vertex in $T$. We will show that $E(C^{*}_i, \{y\}) \geq \phi\eta/2$. 
Using the fact $|A| \geq |A\cap B| + |A \cap C| - |A\cap B \cap C|$, we have:
\begin{align*}
|N[u_i]| &\geq |N[u_i] \cap C^{*}_i| + |N[u_i] \cap N[y_i]| - |N[u_i] \cap N[y_i] \cap C^{*}_i|  \\
&\geq (|N[u_i]| - \phi) + (|N[u_i]|- |N[u_i] \Delta N[y_i]) - |N[y_i] \cap C^{*}_i| \\
0&\geq -\phi + ((3+\eta)\phi - (1+\eta/4)\cdot 2\phi) - |N[y_i] \cap C^{*}_i| \\
|N[y_i] \cap C^{*}_i| &\geq \eta\phi/2 \\
E(C^{*}_i, \{y\}) &\geq \eta\phi/2
\end{align*}

As a result, $|E(C^{*}_i, T)| = \sum_{y \in T} E(C^{*}_i, \{y\}) \geq |T| \cdot \phi \cdot \eta /2$.

Now note that since $C^{*}_i$ contains a low-degree vertex $z$, $|C^{*}_i| \leq |N[z]| + \phi \leq 5\phi$. If $|T| > 10\phi/\eta$, then $|E(C^{*}_i, T)| > 5\phi^2$. This implies there exists a vertex in $C^{*}_i$ more than $5\phi^2 / |C^{*}_i| \geq 5\phi^2 / 5\phi = \phi$ edges connecting to $T$. As these edges contribute to the disagreements of that vertex, a contradiction occurs. Therefore, $|T| \leq 10\phi/\eta$. 
\end{proof}
\end{comment}