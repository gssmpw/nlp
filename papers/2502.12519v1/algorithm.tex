
% \begin{algorithm}[htbp]
% \caption{{\textsc{CONNECTEDCOMPONENT}}$(G^+=(V, E^+), \phi)$\label{alg:main}\\
% \textbf{Input}: A graph $G^{+}$ and a parameter $\phi$. \\
% \textbf{Output}: A clustering $\mathcal{C}$ with $\obj(\mathcal{C})\leq 3\phi$ or ``$\OPT > \phi$''}
% %, where $x_{uv}$ is the positive edge value and $z_e$ is the negative edge value. 
% %\\
% %\textbf{Auxiliary Information}: $E^-:={\binom{V}{2}}\setminus E^+$;. 
% %$c = \exp(\epsilon)$, where $e$ is the Euler's number.
% \label{alg:lp}
% \begin{algorithmic}[1]
% \Function {ClusterPhi}{$G^+=(V, E^+), \phi$}
% \Statex {\small \textit{\underline{$\triangleright$ Initialization}}}
%     % \State $l_e\gets 1, \cong_e \gets 0$ for all $e\in E^+\cup E^-$
%     % \State $\alpha \gets 3$

% %    \Comment{Construct partial cluster using node $u$ with $d(u) \geq 4\phi$.}
%     %\For{$u \in V$ such that $d(u) \geq 4 \phi$}
%     %    \For{$v \in V$ such $|N(u) \Delta N(v)| \leq 2 \phi$}
%     %        \State $L_1(u) \gets L_1(u) \cup \{ v \}$
%     %    \EndFor
%     %\EndFor

%     \For{$u,v \in V$} \Comment{$uv$ not necessarily in $E^{+}$.}
%         \If{$|N[u] \cap N[v]| > 2 \phi$}
%             \State $E' \leftarrow E' \cup \{uv\}$.
%         \EndIf
%     \EndFor

%     \State Output connected components in $(V, E')$ as $\mathcal{C}$.
    
%     % \State $V_1 \leftarrow \{w \mid \mbox{$\deg_{G}(w) \leq 3 \phi$} \}$
    
%     % \State Let $\mathcal{L} = \{L_i\}_{i=1}^{|\mathcal{L}|}$ be the partition formed by the connected components in $(V \setminus V_1 ,E')$.
    
% %    \State $L \gets \{ u \in V \mid 3\phi \leq d(u) < 4\phi \}$
% %    \State Partition $L$ such that for
% %    \For{$u$ such that $3\phi \leq d(u) < 4\phi$}
% %    \State Cluster it and $L(u) = \{ v : |N(v) \Delta N(u)| \le 2\phi, 3\phi \le d(v) < 4\phi \}$ be the clustering
% %    \EndFor

%     % \For{$i$ from $1$ to $|\mathcal{L}|$}
%     % \State Choose the node $u_i \in L_i$ with maximum degree.
%     % \State Compute $R(u_i) = \{ w \in V_i \cap N(u_i) \mid |N(w) \Delta N(u_i) | \le 2\phi  \}$
%     % \State $C_i \leftarrow L_i \cup R(u_i)$
%     % \State $V_{i + 1} \leftarrow V_i \setminus R(u_i)$
%     % \EndFor
%     % \If{for some $i$ there exists $u \in C_i$ such that $\rho_{C_{i}}(u) > 3\phi$}
%     % \State \Return ``$\OPT > \phi$'' 
%     % \Else
%     % \State \Return $\mathcal{C} = \{C_i\}_{i=1}^{|\mathcal{L}|} \cup \bigcup_{v \in V_{|\mathcal{L}|+1}}\{v\}$
%     % \EndIf
% %    \medskip
% \EndFunction
% \end{algorithmic}
% \end{algorithm}


\begin{definition} An $\eta$-similarity query $\Delta_{\eta}(u,v,t)$ returns: 
$$\Delta_{\eta}(u,v,t) =   \begin{cases}0, &\mbox{if $|N[u] \Delta N[v]| > (1+\eta)t$.} \\ \mbox{$0$ or $1$}, &\mbox{if $t <  |N[u] \Delta N[v]| \leq (1+\eta)t$.} \\ 1, &\mbox{if $|N[u] \Delta N[v]| \leq t$.} \end{cases}$$.
\end{definition}

\begin{definition} (Shorthand for $\eta$-similarity query) For readability, we use \( |N[u] \Delta N[v]| \leq_{\eta} t \) to denote that an \( \eta \)-similarity query \( \Delta_{\eta}(u,v,t) \) has been conducted and returned 1. We use \( |N[u] \Delta N[v]| \nleq_{\eta} t \) to denote that the query \( \Delta_{\eta}(u,v,t) \) returned 0.
%$A \leq_{\eta} B$ is a test such that if $A \leq B$ then $A \leq_{\eta} B$ always holds. In addition, if $A \leq_{\eta} B$ then it must be the case that $A \leq (1+\eta) B$ holds. We use $A \nleq_{\eta} B$ to denote that the test is false. 
\end{definition}

The parameter $\eta$ was introduced to accommodate the error induced by an approximate test of whether $|N[u] \Delta N[v]| \leq 2\phi$. For the sake of simplicity, we urge the readers to assume $\eta = 0$ when reading for the first time. When $\eta = 0$, $|N[u] \Delta N[v]| \leq_{\eta} 2\phi$ if and only if $|N[u] \Delta N[v]| \leq 2\phi$.

\begin{definition} Define $\Vlow \leftarrow \{w \mid \mbox{$\deg(w) \leq (3+\eta) \phi$} \}$ and $\Vhigh \leftarrow V \setminus \Vlow$. \end{definition}

%Here, we describe it using a sequential description instead of a parallel description in the introduction.

\paragraph{Algorithm Description} \Cref{alg:lp} takes two parameters $\phi$ and $0 \leq \eta < 1$, where $\phi$ is a guess on the upper bound of $\OPT$ and $\eta$ is an error control parameter.  The goal of the algorithm is to output a solution of value $(3+\eta)\phi$ provided $\OPT \leq \phi$. Note that we use $\eta$ instead of $\epsilon$ here to indicate that it can be set to zero (at the cost of computing the more expensive exact neighborhood similarity). 

%It will be the case that we will set $\eta = \epsilon$ for our $(3+\epsilon)$-approximation algorithms. However, we use $\eta$ here to indicate that it can be set to zero, if we are willing spend more time for testing $|N[u] \Delta N[v]| \leq 2\phi$ exactly. 

The algorithm works as follows: First, we form a clustering of high-degree vertices $\mathcal{L}$ based on the similarity between the neighborhood of vertices. If two high-degree vertices $u$ and $v$ have similar neighborhood (i.e.~$N[u] \Delta N[v]| \leq_{\eta} 2\phi$) then they will be placed in the same cluster. 

Once the high-degree clusters are formed, we go through each cluster $L_i \in \mathcal{L}$. For each $L_i$, we pick a pivot $u_i \in L_i$. For each neighbor $w$ of $u_i$ that remains unclustered (i.e.~in $V_i$), we include it in $R(u_i)$ if (i.e.~$N[w] \Delta N[u_i]| \leq_{\eta} 2\phi$). Then we set our cluster $C_i$ to be $L_i \cup R(u_i)$. Then we update the unclustered vertices to be $V_{i+1} = V_{i} \setminus R(u_i)$. 

Once we went through every $L_i$, there might be still some unclustered low-degree vertices (i.e.~those in $V_{|\mathcal{L}|+1}$). For each such vertex, we put it in a singleton cluster. Then, we check if the clustering we have obtained has an objective value at most $(3+\eta)\phi$ or not. If it does, then we are done. If not, we conclude that $\OPT > \phi$, so we would need to set our guess of $\phi$ larger.

 
\begin{algorithm}[t]
\caption{{\textsc{ClusterPhi}}$(G^+=(V, E^+), \phi, \eta)$\label{alg:main}\\
\textbf{Input}: A graph $G^{+}$ and parameters $\phi$ and $0 \leq \eta < 1$. \\
\textbf{Output}: A clustering $\mathcal{C}$ with $\obj(\mathcal{C})\leq (3+\eta)\phi$ or ``$\OPT > \phi$''}
%, where $x_{uv}$ is the positive edge value and $z_e$ is the negative edge value. 
%\\
%\textbf{Auxiliary Information}: $E^-:={\binom{V}{2}}\setminus E^+$;. 
%$c = \exp(\epsilon)$, where $e$ is the Euler's number.
\label{alg:lp}
\begin{algorithmic}[1]
\Function {ClusterPhi}{$G^+=(V, E^+), \phi, \eta$}
\Statex {\small \textit{\underline{$\triangleright$ Initialization}}}
    % \State $l_e\gets 1, \cong_e \gets 0$ for all $e\in E^+\cup E^-$
    % \State $\alpha \gets 3$

%    \Comment{Construct partial cluster using node $u$ with $d(u) \geq 4\phi$.}
    %\For{$u \in V$ such that $d(u) \geq 4 \phi$}
    %    \For{$v \in V$ such $|N(u) \Delta N(v)| \leq 2 \phi$}
    %        \State $L_1(u) \gets L_1(u) \cup \{ v \}$
    %    \EndFor
    %\EndFor

    \For{$u,v \in V$} \label{ln:highdegstart}\Comment{$uv$ not necessarily in $E^{+}$.}
        \If{$|N[u] \Delta N[v]| \leq_{\eta} 2 \phi$}
            \State $E' \leftarrow E' \cup \{uv\}$.
        \EndIf
     \EndFor
    
    \State $\Vlow \leftarrow \{w \mid \mbox{$\deg(w) \leq (3+\eta) \phi$} \}, \Vhigh \leftarrow V \setminus \Vlow$

    \State $V_1 = \Vlow$
    
    \State Let $\mathcal{L} = \{L_i\}_{i=1}^{|\mathcal{L}|}$ be the partition formed by the connected components in $(\Vhigh ,E')$. \label{ln:highdegend}
%    \State $L \gets \{ u \in V \mid 3\phi \leq d(u) < 4\phi \}$
%    \State Partition $L$ such that for
%    \For{$u$ such that $3\phi \leq d(u) < 4\phi$}
%    \State Cluster it and $L(u) = \{ v : |N(v) \Delta N(u)| \le 2\phi, 3\phi \le d(v) < 4\phi \}$ be the clustering
%    \EndFor

    \For{$i$ from $1$ to $|\mathcal{L}|$} \label{ln:upper} 
    \State Choose any node $u_i \in L_i$.
    \State \label{ln:R}Compute $R(u_i) = \{ w \in V_i \cap N(u_i) \mid |N[w] \Delta N[u_i]| \leq_{\eta/2} 2\phi   \}$
    %Compute $R(u_i) = \{ w \in V_i \cap N(u_i) \mid |N[w] \Delta N[u_i] | \le 2\phi   \}$   
    \State  \label{ln:C}$C_i \leftarrow L_i \cup R(u_i)$                                                  
    \State $V_{i + 1} \leftarrow V_i \setminus R(u_i)$
    \EndFor   \label{ln:lower} 
    \If{for some $i$ there exists $u \in C_i$ such that $\rho_{C_{i}}(u) > (3+\eta)\phi$}\label{ln:checking}
    \State \Return ``$\OPT > \phi$'' 
    \Else
    \State \Return $\mathcal{C} = \{C_i\}_{i=1}^{|\mathcal{L}|} \cup \bigcup_{v \in V_{|\mathcal{L}|+1}}\{ \{v\} \}$
    \EndIf
%    \medskip
\EndFunction
\end{algorithmic}
\end{algorithm}



\begin{theorem} Suppose that $\OPT \leq \phi$, \Cref{alg:main} outputs a clustering $\mathcal{C}$ with $\obj(\mathcal{C})\leq (3+\eta) \phi$. 
\end{theorem}
\begin{proof} Let $\mathcal{C}^{*}$ be an optimal solution so $\obj(\mathcal{C}^{*}) \leq \phi$. In the next subsections, we show the following:
\begin{enumerate}
\item \label{itm:first} (High-Degree Nodes Clustering). For any $u \in \Vhigh$, $\mathcal{L}_u \cap  \Vhigh = \mathcal{C}^{*}_{u} \cap \Vhigh$. 
\item (No Stealing on Low-Degree Nodes). For each $i$, let $C^{*}_i$ be the cluster in $\mathcal{C}^{*}$ such that $L_i \cap \Vhigh = {C}^{*}_{i} \cap  \Vhigh$. We have $C^{*}_i \cap V_i = C^{*}_i \cap \Vlow$. That is, those low-degree vertices in $C^{*}_i$ are not taken by other clusters $C_j$ for $j<i$.
%\nairen{maybe it's better to move $i$ at the beginning of this item.}
\item (Low-Degree Nodes Inclusion). For any $i$, $C^{*}_i \cap N(u_i) \subseteq C_i$. 
\item (Closeness). \label{itm:last} For any $i$, $|N[u_i] \Delta C_i| \leq \phi$ and $|C^{*}_i \Delta C_{i}| \leq \phi$.
%\item For any $w \in V_1$, for any $v \in V \setminus V_1$, if $|N(w) \Delta N(v)| \leq 2\phi$ then $v \in \mathcal{C}^{*}_{w}$. 
\end{enumerate}
Once we have shown the above, we can see that $\obj(\mathcal{C}) \leq (3+\eta) \phi$ as follows. Consider a component $C_i$ of $\mathcal{C}$. If $C_i$ is a singleton $\{v\}$ with $\deg(v) \leq (3+\eta)\phi$, then obviously, $\rho_{\mathcal{C}}(v) \leq (3+\eta)\phi$. Otherwise, $C_i$ contains some vertex $x$ with $\deg(x) >  (3+\eta)\phi$. By \Cref{itm:first}, there must exist $C^{*}_i$ such that $C_i \cap \Vhigh = {C}^{*}_{i} \cap \Vhigh$.

Let $v$ be any vertex in $C_i$. We will show that $\rho_{\mathcal{C}}(v) \leq 3\phi$. Suppose that $v \in C^{*}_i \cap C_i$, we have: 
\begin{align*}
\rho_{\mathcal{C}}(v) &= |N[v] \Delta C_i|
\leq |N[v] \Delta C^{*}_i | + |C^{*}_i \Delta C_i|  && \mbox{by \Cref{lem:triangleineq}}\\
&= \rho_{\mathcal{C^{*}}}(v) + |C^{*}_i \Delta C_i| \leq \phi + \phi  = 2\phi 
\end{align*}
Otherwise, if $v \notin C^{*}_i \cap C_i$ then it must be the case that $v \in C_i \setminus C^{*}_i$. In such a case, $v$ must be a vertex in $R(u_i)$ added to $C_i$ in Line \ref{ln:R} to Line \ref{ln:C}. This implies $|N[v] \Delta N[u_i]| \leq_{\eta/2} 2\phi$ and thus $|N[v] \Delta N[u_i]| \leq (1+\eta/2)\cdot2\phi$. Therefore:
\begin{align*}
\rho_{\mathcal{C}}(v) = |N[v] \Delta C_i| 
&\leq |N[v] \Delta N[u_i]| + |N[u_i] \Delta C_i| && \mbox{by \Cref{lem:triangleineq}} \\
&\leq (1+\eta/2)\cdot 2\phi + \phi = (3+\eta)\phi
\end{align*}
\end{proof}

\begin{remark} To get a $(3+\epsilon)$-approximation algorithm, $O(\log n)$ calls of  \Cref{alg:lp} are sufficient, as we can perform a binary search on $\phi$ in the range of $[0,n]$ with $\eta=\epsilon$ and take the solution output by the algorithm with the smallest $\phi$. For a $3$-approximation algorithm, it can also be achieved within $O(\log n)$ calls of \Cref{alg:lp} by setting $\eta = 0$. \end{remark}

We prove each of the four items in the proof in the following subsections. Note that the no-stealing property of low-degree vertices (\Cref{thm:nostealing} in \Cref{sec:nostealing}) is the main structural result, which not only leads to a guarantee on the approximation ratio, but it is also crucial in the design of efficient algorithms in the subsequent sections. %in \Cref{sec:efficientimplementation} and \Cref{sec:streaming}. 

\subsection{High-Degree Nodes Clustering}
Recall that $\phi$ is our guess of the upper bound of the optimal solution.  In this subsection, we show if $\phi$ is indeed such an upper bound, then there is a unique way to form clustering on nodes with degrees greater than $(3+\eta)\phi$ in the optimal solution. Moreover, in the algorithm, our construction of clustering on those nodes aligns with the way. The following two lemmas are observed by \cite{heidrich20244}.


\begin{lemma}\label{lem:diffcluster} Suppose that $|N[x] \cap N[y]| > t$. If there is a clustering $\mathcal{C}'$ such that $x$ and $y$ are in different clusters, then $\obj(\mathcal{C}') > t/2$. \end{lemma}
\begin{proof}
\begin{align*}
t < |N[x] \cap N[y]| \leq |N[x] \cap N[y] \setminus \mathcal{C'}_{y}| + |N[x] \cap N[y] \setminus \mathcal{C'}_{x}| \leq \rho_{\mathcal{C}'}(y) + \rho_{\mathcal{C}'}(x)
\end{align*}
Therefore, $\obj(\mathcal{C}') \geq \max(\rho_{\mathcal{C'}}(x),  \rho_{\mathcal{C'}}(y)) \geq (\rho_{\mathcal{C'}}(x) + \rho_{\mathcal{C'}}(y) )/2 > t/2$.
\end{proof}

\begin{lemma}\label{lem:samecluster} Suppose that $|N[x] \Delta N[y]| > t$. If there is a clustering $\mathcal{C}'$ such that $x$ and $y$ are in the same clusters, then $\obj(\mathcal{C}') > t/2$. \end{lemma}
\begin{proof}
Let $C$ be the cluster containing $x$ and $y$. By \Cref{lem:triangleineq}, we have: 
\begin{align*}
t &< |N[x] \Delta N[y]| \leq |N[x]\Delta C| + |N[y] \Delta C|= \rho_{\mathcal{C'}}(x) + \rho_{\mathcal{C'}}(y) 
\end{align*}
Therefore, $\obj(\mathcal{C}') \geq \max(\rho_{\mathcal{C'}}(x),  \rho_{\mathcal{C'}}(y)) \geq (\rho_{\mathcal{C'}}(x) + \rho_{\mathcal{C'}}(y) )/2 > t/2$.
\end{proof}


\begin{lemma}\label{lem:highdegreeclustering}
Let $\mathcal{C}^{*}$ be a clustering with $\obj(\mathcal{C}^{*}) \leq \phi$. For any $u \in \Vhigh$, $\mathcal{L}_u \cap \Vhigh = \mathcal{C}^{*}_{u} \cap \Vhigh$. 
\end{lemma}
\begin{proof}
Suppose on the contrary that $\mathcal{L}_u \cap \Vhigh \neq \mathcal{C}^{*}_{u} \cap \Vhigh$. Then there exists some node $x$ such that $x \in (\mathcal{L}_u \cap \Vhigh) \Delta (\mathcal{C}^{*}_{u} \cap \Vhigh)$.

Suppose that there exists a node $x$ such that $x \in \mathcal{L}_u \cap \Vhigh$ but  $x \notin \mathcal{C}^{*}_{u} \cap \Vhigh$. Then there must exist $y \in \mathcal{C}^{*}_{u}$ such that $(x,y) \in E'$. This implies $|N[x] \Delta N[y]| \leq_{\eta} 2\phi$ and so $|N[x] \Delta N[y]| \leq (1+\eta) 2\phi$. Therefore,
\begin{align*}
|N[x] \cap N[y]| &= (\deg(x) +1 + \deg(y)+ 1 - |N[x] \Delta N[y]|)/2 \\
&> ((3+\eta)\phi + 1 + (3+\eta)\phi + 1 - (1+\eta)2\phi ) / 2 \\
&> (6\phi + 2  - 2\phi)/2 = 2\phi + 1
\end{align*}
By \Cref{lem:samecluster}, $\obj(\mathcal{C}^{*}) > \phi$, a contradiction.

Otherwise, it must be the case that $x \in \mathcal{C}^{*}_{u} \cap \Vhigh$ but $x \notin \mathcal{L}_u \cap \Vhigh$. In this case, there exists $y \in \mathcal{C}^{*}_u$ such that $(x,y) \notin E'$. This implies that $|N[x] \Delta N[y]| \nleq_{\eta} 2\phi$, which in turns implies $|N[x] \Delta N[y]| > 2\phi$. By \Cref{lem:samecluster}, $\obj(\mathcal{C}^{*}) > \phi$, a contradiction.

\end{proof}

\subsection{No Stealing on Low-Degree Nodes}\label{sec:nostealing}
In the algorithm, low-degree nodes (i.e.~nodes with degree at most $(3+\eta)\phi$) are added to the clusters formed by high-degree nodes iteratively. In this subsection, we show that if a low-degree node degree node $y$ belongs to $C^{*}_i$ for some $i$, then it will not be included in $C_j$ for $j<i$. This implies that $y \in V_i$, the candidate set of vertices to be added $C_i$. Then in the next subsection, we show that it will be added to $C_i$. 

\begin{theorem}\label{thm:nostealing}
Let $\mathcal{C}^\ast$ be a clustering with $\text{obj}(\mathcal{C}^{*}) \le \phi.$ Let $u$ and $v$ be vertices of degree greater than $(3+\eta)\phi$ where $\mathcal{C}^{*}_u \neq \mathcal{C}^{*}_v$, and $w$ be a vertex with $\text{deg}(w) \leq (3+\eta)\phi$. If $w \in \mathcal{C}^{*}_u$ then $|N[v] \Delta N[w]| > (1+\eta)2\phi$ and so $|N[v] \Delta N[w]| \nleq_{\eta} 2\phi$.
\end{theorem}

A representative case that illustrates the intuition of why the theorem holds is when $|N[u]\cap N[v] \cap N[w]| \leq \phi$, i.e.~the three sets have small intersections. Since they have small intersection, together with the fact that $u$ and $v$ have degrees greater than $(3+\eta) \phi$, it cannot be the case that both the symmetric differences $N[u] \Delta N[w]$ and $N[v] \Delta N[w]$ are small, as illustrated in \Cref{fig:1}. We will refer the case that $|N[u] \cap N[v] \cap N[w]| \leq \phi$ as the {\it easy case}. 

%The main intuition behind the theorem is that if $w \in \mathcal{C}^{*}_v$ then $N(v)$ and $N(w)$ intersect a lot, as $\rho_{\mathcal{C}^{*}}(v)$ is upper bounded by $\phi$. Since $w$ is a low-degree vertex (i.e.~$\deg(w)\leq 3\phi$), it is impossible $N(w)$ and $N(u)$ will intersect a lot, {\it provided that $N(u)$ and $N(v)$ does not intersect a lot}. We refer to this condition as the {\it easy case}.

When the three sets have a large intersection, with a more sophisticated argument on the relations among $N[u], N[v]$, and $N[w]$, it can also be shown that $N[w]$ and $N[u]$ will not intersect a lot. We will refer to the condition that $|N[u] \cap N[v] \cap N[w]| > \phi$ as {\it the hard case}.
\begin{figure}
\centering
\includegraphics[scale=0.45]{figure1.png}
\caption{A pictorial illustration of the proof of \Cref{lem:nostealingeasy} when $\eta = 0$}\label{fig:1}
\end{figure}
\paragraph{The easy case.} We will first show the theorem for the easy case as follows. 
\begin{lemma}\label{lem:nostealingeasy}
Let $\mathcal{C}^\ast$ be a clustering with $\text{obj}(\mathcal{C}^{*}) \le \phi.$ Let $u$ and $v$ be vertices of degree greater than $(3+\eta)\phi$ where $\mathcal{C}^{*}_u \neq \mathcal{C}^{*}_v$, and $w$ be a vertex with $\text{deg}(w) \leq (3+\eta)\phi$. If $w \in \mathcal{C}^{*}_u$ and $|N[v] \cap N[u] \cap N[w]| \leq \phi$, then $|N[v] \Delta N[w]| > (1+\eta)2\phi$ and so $|N[v] \Delta N[w]| \nleq_{\eta} 2\phi$.
\end{lemma}
\begin{proof} We will show that $|N[v] \Delta N[w]| + |N[u] \Delta N[w]|$ is greater than $(1+\eta)4\phi$. \Cref{fig:1} gives a high level illustration of why this should be true. To show this formally, first note that: 
\begin{align*}
&|N[v] \cap N[w] \cap N[u]| + |N[w] \setminus N[u]| + |N[v] \setminus N[w]| \\
&\geq |N[v] \cap N[w] \cap N[u]| + |(N[v] \cap N[w]) \setminus N[u]| + |N[v] \setminus N[w]| \\
&= |N[v] \cap N[w] |+ |N[v] \setminus N[w]| = |N[v]|
\end{align*}
Re-arranging, we have:
\begin{align}
|N[w] \setminus N[u]| + |N[v] \setminus N[w]| &\geq |N[v]| - |N[v] \cap N[w] \cap N[u]| > (3+\eta)\phi - \phi = (2+\eta)\phi \label{eqn:dif1}
\end{align}
By the same reasoning, we have
\begin{align}
|N[w] \setminus N[v]| + |N[u] \setminus N[w]| > (2+\eta)\phi \label{eqn:dif2}
\end{align}
Now consider the following:
\begin{align*}
&|N[v] \Delta N[w]| + |N[u] \Delta N[w]| \\
&= |N[v] \setminus N[w]| + |N[w] \setminus N[v]| + |N[u] \setminus N[w]| + |N[w] \setminus N[u]| \\
&> (2+\eta)\phi + (2+\eta)\phi && \mbox{by (\ref{eqn:dif1}) and (\ref{eqn:dif2})} \\
&= (4+2\eta)\phi
\end{align*}
Therefore, assume to the contrary that if $|N[v] \Delta N[w]| \leq_{\eta} 2\phi$ and so $|N[v] \Delta N[w]| \leq (1+\eta)2\phi$ then it must be the case that $|N[u] \Delta N[w]| > (4+2\eta)\phi - |N[v] \Delta N[w]| \geq 2\phi$. By the fact that $w \in \mathcal{C}^{*}_v$ and \Cref{lem:samecluster}, $\obj(\mathcal{C}^{*}) > \phi$, a contradiction.
\end{proof}

\paragraph{The hard case} Now we consider the case where $|N[w] \cap N[u] \cap N[v]| > \phi$. To illustrate the idea of proof, suppose that $|N[w] \cap N[u] \cap N[v]| = \phi + t$ for some $t>0$. If we follow the same argument as \Cref{lem:nostealingeasy}, we would only be able to show that $|N[w] \Delta N[v]| + |N[w] \Delta N[u]| > (4+2\eta)\phi - 2t$ as opposed to $(4+2\eta)\phi$. 

However, if this is the case, we can show that $|N[w] \Delta N[u]| \leq 2(\phi-t)$, as stated in \Cref{lem:largeintersection}. This would make $|N[w] \Delta N[v]| > (1+\eta)2\phi$. 

The high-level idea of the proof of \Cref{lem:largeintersection} is illustrated in \Cref{fig:2}. First, we observe that at most $\phi$ vertices in $|N[u] \cap N[v] \cap N[w]|$ can be contained in $\mathcal{C}^{*}_u$; otherwise the vertex $v$ would have disagreements greater than $\phi$. This implies the contribution of disagreement from $N[u] \cap N[w]$ to $u$ (and to $w$) is already at least $t$. If the symmetric difference $|N[u] \Delta N[w]|$ is too large, i.e.~larger than $2(\phi -t)$, then it would force the disagreements of either $u$ or $w$ to be too large.

\begin{lemma}\label{lem:largeintersection}
Let $\mathcal{C}^\ast$ be a clustering with $\text{obj}(\mathcal{C}^{*}) \le \phi.$ Let $u$ and $v$ be vertices of degree greater than $(3+\eta)\phi$ where $\mathcal{C}^{*}_u \neq \mathcal{C}^{*}_v$, and $w \in \mathcal{C}^{*}_u$ be a vertex with $\text{deg}(w) \leq (3+\eta)\phi$. Suppose that $|N[w] \cap N[v] \cap N[u]| = \phi + t$ for some $t>0$. Then, $|N[w] \Delta N[u]| \le 2(\phi - t)$.
\end{lemma}
\begin{figure}
\centering
\includegraphics[scale=0.45]{figure2.png}
\caption{A pictorial illustration of the proof of \Cref{lem:largeintersection} when $\eta = 0$}\label{fig:2}
\end{figure}
\begin{proof}
Let $S = N[w] \cap N[v] \cap N[u]$. First we claim that at most $\phi$ vertices in $S$ can be in $\mathcal{C}^{*}_u$. This in turns implies that vertices in $S$ contributes at least $t$ disagreements to vertex $u$ and vertex $w$. To see why this holds, suppose to the contrary that $|S \cap \mathcal{C}^{*}_u| > \phi$. Since $S \subseteq N[v]$ and $\mathcal{C}^{*}_v \neq \mathcal{C}^{*}_u$, it must be the case that $\rho_{\mathcal{C}^{*}}(v) \geq |N[v] \setminus C^{*}_v| \geq |S \setminus C^{*}_v| \geq |S \cap \mathcal{C}^{*}_u| > \phi$, a contradiction. Therefore, we conclude that:
\begin{align}|S \setminus \mathcal{C}^{*}_u| \geq t\end{align}

Now note that the disagreements of $u$ and $w$ are at least:
\begin{align*}
\rho_{\mathcal{C}^{*}}(u) &\geq |S \setminus \mathcal{C}^{*}_u| + |(N[w] \setminus N[u]) \cap \mathcal{C}^{*}_u| + |(N[u] \setminus N[w]) \setminus \mathcal{C}^{*}_u| \\
\rho_{\mathcal{C}^{*}}(w) &\geq |S \setminus \mathcal{C}^{*}_u| + |(N[u] \setminus N[w]) \cap \mathcal{C}^{*}_u| + |(N[w] \setminus N[u]) \setminus \mathcal{C}^{*}_u|
\end{align*}

Using the fact $\rho_{\mathcal{C}^{*}}(u), \rho_{\mathcal{C}^{*}}(w) \leq \phi$ and adding the above two inequalities together, we have:
\begin{align*}
2\phi &\geq \rho_{\mathcal{C}^{*}}(u) + \rho_{\mathcal{C}^{*}}(w) \\
&\geq 2|S \setminus \mathcal{C}^{*}_u| + |(N[w] \setminus N[u]) \cap \mathcal{C}^{*}_u| + |(N[u] \setminus N[w]) \setminus \mathcal{C}^{*}_u| + \\
& \hspace{25mm} |(N[u] \setminus N[w]) \cap \mathcal{C}^{*}_u| + |(N[w] \setminus N[u]) \setminus \mathcal{C}^{*}_u| \\
&= 2|S \setminus \mathcal{C}^{*}_u| + |(N[w] \Delta N[u]) \cap \mathcal{C}^{*}_u | + |(N[u] \Delta N[w]) \setminus \mathcal{C}^{*}_u| \\
&= 2|S \setminus \mathcal{C}^{*}_u| + |(N[w] \Delta N[u])| \\
&\geq 2t + |(N[w] \Delta N[u])|
\end{align*}

Therefore, $|N[w] \Delta N[v]|\leq 2(\phi - t)$.
\end{proof}


\begin{lemma}\label{lem:nostealinghard}
Let $\mathcal{C}^\ast$ be a clustering with $\text{obj}(\mathcal{C}^{*}) \le \phi.$ Let $u$ and $v$ be vertices of degree greater than $(3+\eta)\phi$ where $\mathcal{C}^{*}_u \neq \mathcal{C}^{*}_v$, and $w$ be a vertex with $\text{deg}(w) \leq (3+\eta)\phi$. If $w \in \mathcal{C}^{*}_u$ and $|N[v] \cap N[u] \cap N[w]| > \phi$, then $|N[v] \Delta N[w]| > (1+\eta)2\phi$ and so $|N[v] \Delta N[w]| \nleq_{\eta} 2\phi$.
\end{lemma}
    
%    We know $N(v)$ shares $\phi + x$ vertices with each $N(w), N(u)$. If we have $|S \Delta C^\ast(v)| > \phi$, this implies $\rho_{C^\ast(v)} > \phi$, which is a contradiction. So it must be that $|S \cap C^\ast(u)| \le \phi$. \\
    
%    Now with the facts that $|S \cap C^\ast(u)| \le \phi$ and $w \in C^\ast(u)$, we have
%    \begin{align*}
%        &|S \backslash C^\ast(u)| \ge x\\
%    \end{align*}
    
%    Each of the vertices in $S \backslash C^\ast(u)$ that contribute to the disagreement values are in both $N(w)$ and $N(u)$. Thus, those vertices do not contribute to $|N(w) \Delta N(u)|$. \\
    
%    Recall that $\rho_{C^\ast(u)}(u) \le \phi,$ and we have $w \in C^\ast(u) \implies \rho_{C^\ast(u)}(w) \le \phi$, by definition of the optimal clustering for $u$. Thus, considering disagreements within $C^\ast(u)$, we must have $\rho_{[C^\ast(u)]}(u) \le \phi - x, \rho_{[C^\ast(u)]}(w) \le \phi - x$. \\
    
%    It follows $|N(u) \Delta N(w)| \le 2(\phi - x)$. If otherwise, then either $|N(u) \backslash N(w)| > \phi - x$ or $|N(w) \backslash N(u)| > \phi - x$. \\
    
%    In either case, we cannot have that both $\rho_{C^\ast(u)}(u) \le \phi$ and $\rho_{C^\ast(u)}(w) \le \phi$. 
%\end{proof}

%\begin{lemma}
%    Suppose that $|N(w) \cap N(v) \cap N(u) | = (\phi + x)$, then $|N(v) \cap N(w)| \le |N(w) \cap N(u)| - x $.
%\end{lemma}
%\begin{proof}
%\begin{align*}
%    &|N(u) \Delta N(w)| \le 2(\phi - x) \text{ by Lemma 1.7} \\
%    &\implies |N(w) \backslash N(u)| + |N(u) \backslash N(w)| \le 2(\phi - x) \\
%    &\implies |N(w) \backslash N(u)| + |N(u)| - |N(u) \cap N(w)| \le 2(\phi - x) \\
%    &\implies |N(w) \backslash N(u)| \le 2(\phi - x) - |N(u)| + |N(u) \cap N(w)| \\
%    &\implies|N(w) \backslash N(u)| \le -\phi - 2x + |N(u) \cap N(w)|  \text{$\quad$ (using $|N(u)| \ge 3\phi$)} 
%\end{align*}
%The end result is the important result that the next part uses. I refer to this as $(1)$. \\
%For $|N(v) \cap N(w)|$, we have:
%\begin{align*}
%   |N(v) \cap N(w)| &\le |N(v) \cap N(w) \cap N(u)| + |N(w) \backslash N(u)| \text{ by set inclusion (see below)} \\
%   &\le |N(v) \cap N(w) \cap N(u)| + (-\phi - 2x + |N(u) \cap N(w)|) \text{ (by using (1))} \\
%   &= (\phi + x) + (-\phi - 2x + |N(u) \cap N(w)|) \text{ (by assumption that $|N(v) \cap N(w) \cap N(u)|  = \phi+x$)} \\
%   &= |N(u) \cap N(w)| - x 
%\end{align*}
%Here is the set inclusion:
%Take any $x \in N(v) \cap N(w)$. Either $x \in N(u)$ or it is not. If so, then $x \in N(v) \cap N(w) \cap N(u)$. If not, then it is in $N(w)$ and not in $N(u), \text{ i.e. } x \in N(w) \backslash N(u)$. Thus, $N(v) \cap N(w) \subseteq (N(v) \cap N(w) \cap N(u)) \cup (N(w) \backslash N(u))$. 
%\end{proof}

\begin{proof}
By (\ref{eqn:dif1}) in the proof \Cref{lem:nostealingeasy}, we have:
\begin{align*}
|N[w] \setminus N[u]| + |N[v] \setminus N[w]| &\geq |N[v]| - |N[v] \cap N[w] \cap N[u]|  \\
& > (3+\eta) \phi - (\phi + t) = (2+\eta)\phi - t \\
|N[w] \setminus N[v]| + |N[u] \setminus N[w]| &\geq |N[u]| - |N[v] \cap N[w] \cap N[u]|  \\
&> (3+\eta) \phi - (\phi + t) = (2+\eta)\phi - t
\end{align*}
Therefore, 
\begin{align*}
|N[v] \Delta N[w]| + |N[u] \Delta N[w]| &= |N[v] \backslash N[w]| + |N[w] \backslash N[v]| + |N[u] \backslash N[w]| + |N[u] \backslash N[w]| \\
&> (4+2\eta)\phi - 2t
\end{align*}
By \Cref{lem:largeintersection}, we have $|N[u] \Delta N[w]| \geq 2(\phi - t) $, therefore:
$$|N[v] \Delta N[w]| > (4+2\eta)\phi - 2t - 2(\phi - t) = (1+\eta)2\phi$$
Hence, $|N[v] \Delta N[w]| \nleq_{\eta} 2\phi$.
\end{proof}
\Cref{lem:nostealingeasy} and \Cref{lem:nostealinghard} complete the proof of \Cref{thm:nostealing}. \Cref{thm:nostealing} immediately implies the following:
\begin{corollary}\label{cor:nostealing}
Let $C^{*}_i$ be the cluster in $\mathcal{C}^{*}$ such that $L_i \cap \Vhigh = {C}^{*}_{i} \cap  \Vhigh$. For any $i$, $C^{*}_i \cap V_i = C^{*}_i \cap \Vlow$.
\end{corollary}
\subsection{Low-Degree Nodes Inclusion}
Here, we show that all the vertices in $\mathcal{C}^{*}_i$ have similar neighborhoods with $u_i$. As a result, if they are neighbors of $u_i$, they will be added to $C_i$. 
\begin{lemma}\label{lem:inclusion} Let $C^{*}_i$ be the cluster in $\mathcal{C}^{*}$ such that $L_i \cap \Vhigh = {C}^{*}_{i} \cap \Vhigh$. For any $i$, $C^{*}_i \cap N(u_i) \subseteq C_i$. 
\end{lemma}

\begin{proof}
If $w \in C^{*}_i \cap \Vhigh$, then $w \in C_i$ because $C_i = L_i \cup R(u_i)$ and $L_i \cap \Vhigh = C^{*}_i \cap \Vhigh$ by assumption. Now it remains to consider the case $w \in C^{*}_i \cap \Vlow \cap N(u_i)$. It suffices for us to show that $w \in R(u_i)$, where $R(u_i) = \{ w \in V_i \cap N(u_i) \mid |N[w] \Delta N[u_i] | \leq_{\eta} 2\phi  \}$, as $R(u_i)$ will be added to $C_i$. 


By \Cref{cor:nostealing}, we have $w \in C^{*}_i \cap V_i \cap N(u_i)$, which means $w$ is not going to be stolen by other clusters. To show that $w \in C_i$, now it suffices to show that $|N[w] \Delta N[u_i]| \leq_{\eta} 2\phi$ always holds. This is indeed true, because:
\begin{align*}
|N[w] \Delta N[u_i]| &\leq |N[w] \Delta C^{*}_i| + |C^{*}_i \Delta N[u_i]| && \mbox{by \Cref{lem:triangleineq}} \\
&\leq \phi + \phi && \mbox{$u_i, w \in C^{*}_i,\rho_{\mathcal{C}^{*}}(w), \rho_{\mathcal{C}^{*}}(u_i) \leq \phi$} \\
&\leq 2\phi
\end{align*}
\end{proof}
\subsection{Closeness}
In the following, we will show that the cluster we constructed $C_i$ will be similar to $C^{*}_i$ and $N[u_i]$. Intuitively, this holds because the low-degree part of $C_i$ will be sandwiched between the low-degree part of $C^{*}_i \cap N[u_i]$ and $N[u_i]$, i.e.~$C^{*}_i \cap N[u_i] \cap \Vlow \subseteq C_i \cap \Vlow \subseteq N[u_i] \cap \Vlow$. Also, we know that $N[u_i]$ and $C^{*}_i$ are close (i.e.~$|N[u_i] \Delta C_i| \leq \phi$), so somehow $C_i$ cannot be too far away from $C^{*}_i$ and $N[u_i]$. Note that the high-degree parts of $C_i$ and $C^{*}_i$ coincide. 
\begin{restatable}{lemma}{closeness}\label{lem:closeness} Let $C^{*}_i$ be the cluster in $\mathcal{C}^{*}$ such that $L_i \cap \Vhigh = {C}^{*}_{i} \cap \Vhigh$. For any $i$, $|N[u_i] \Delta C_i| \leq \phi$ and $|C^{*}_i \Delta C_{i}| \leq \phi$. 
\end{restatable}

\begin{proof}
We first show that $|N[u_i] \Delta C_i| \leq \phi$.
\begin{align}
|\Vlow \cap (N[u_i] \Delta C_i)| &= |\Vlow \cap (N[u_i] \setminus C_i)| + |\Vlow \cap (C_i \setminus N[u_i])| \nonumber \\
&= |\Vlow \cap (N[u_i] \setminus C_i)|  && (\Vlow \cap C_i) \subseteq N[u_i]  \nonumber  \\
&\leq |\Vlow \cap (N[u_i] \setminus C^{*}_i)|  && \mbox{by \Cref{lem:inclusion}}   \nonumber \\
&\leq |\Vlow \cap (N[u_i] \Delta C^{*}_i)|  \label{eqn:diffneighborCstar1}
\end{align}

Moreover, since $C^{*}_i \cap \Vhigh = C_i \cap \Vhigh$, it must be the case that 
\begin{align}
|(N[u_i] \Delta C_i) \cap \Vhigh| = |(N[u_i] \Delta C^{*}_i) \cap \Vhigh|  \label{eqn:diffneighborCstar2}
\end{align}

Therefore,
\begin{align*}
|N[u_i] \Delta C_i| &= |(N[u_i] \Delta C_i) \cap \Vlow| + |(N[u_i] \Delta C_i) \cap \Vhigh| \\
&\leq |(N[u_i] \Delta C^{*}_i) \cap \Vlow| + |(N[u_i] \Delta C^{*}_i) \cap \Vhigh| && \mbox{by (\ref{eqn:diffneighborCstar1}) and (\ref{eqn:diffneighborCstar2})} \\
& = |N[u_i] \Delta C^{*}_i| \leq \phi  && \mbox{$u_i \in C^{*}_i$ and $\rho_{C^{*}}(u_i) \leq \phi$}
\end{align*}

Now we show that $|C^{*}_i \Delta C_{i}| \leq \phi$. 
\begin{align*}
&|C^{*}_i \Delta C_{i}| \\
&= |\Vlow \cap (C^{*}_i \Delta C_{i})| && C^{*}_i \cap \Vhigh = C_i \cap \Vhigh \\
&= |\Vlow \cap (C^{*}_i \setminus C_{i})| + |\Vlow \cap (C_i \setminus C^{*}_{i})| \\
&= |\Vlow \cap (C^{*}_i\cap N[u_i] \setminus C_{i})| + |\Vlow \cap (C^{*}_i \setminus N[u_i] \setminus C_{i})| \\
& \hspace{50mm} + |\Vlow \cap (C_i \setminus C^{*}_{i})| \\
&= |\Vlow \cap (C^{*}_i\cap N[u_i] \setminus C_{i})| + |\Vlow \cap (C^{*}_i \setminus N[u_i])| \\
& \hspace{50mm} + |\Vlow \cap (C_i \setminus C^{*}_{i})| && \Vlow \cap C_i \subseteq \Vlow \cap N[u_i]\\
&= |\Vlow \cap (C^{*}_i \setminus N[u_i])| + |\Vlow \cap (C_i \setminus C^{*}_{i})| && C^{*}_i\cap N[u_i] \subseteq C_i \\
&\leq |\Vlow \cap (C^{*}_i \setminus N[u_i])| + |\Vlow \cap (N[u_i] \setminus C^{*}_{i})| && \Vlow \cap C_i \subseteq \Vlow \cap N[u_i] \\
&= |\Vlow \cap (C^{*}_i \Delta N[u_i])| \leq |C^{*}_i \Delta N[u_i]| \leq \phi
\end{align*}

\end{proof}

%&&  V_1 \cap C_i \subseteq V_1 \cap N(u) 




%Assume that for some $x \geq 0$, $|N(w) \cap N(u) \cap N(v)| = \phi + x$. Now see that:
%\begin{align*}
%|N(v) \Delta N(w)| &= |N(v) \backslash N(w)| + |N(w) \backslash N(v)| \text{ by def. of sym. diff.}\\ 
%&= |N(v)| - |N(v) \cap N(w)| + |N(w) \backslash N(v)| \text{ by set equality} \\
%&\ge  |N(v)| - |N(v) \cap N(w)| + |N(w) \cap N(u)| - |N(w) \cap N(u) \cap N(v)| \text{ by set inclusion}\\
%&\ge |N(v)| - |N(v) \cap N(w)| + |N(w) \cap N(v)| + x - |N(w) \cap N(u) \cap N(v)| \text{ (by Lemma 1.11)}\\
%&= |N(v)| + x - |N(w) \cap N(u) \cap N(v)| \\
%&\ge 3\phi + x - (\phi + x) = 2\phi \text{ using the observation above}
%\end{align*}



%\begin{lemma}
%Let $C^\ast$ be a clustering with $\text{obj}(C^{*}) \le \phi.$ Let $u$ and $v$ be vertices of degree at least $3\phi$ and $w$ be a vertex with $\text{deg}(w) < 3\phi$.\\
%Let $C^{*}(u)$ and $C^{*}(v)$ be clusters containing $u$ and $v$ with $C^{*}(u) \neq C^{*}(v)$. \\
%If $|N(u) \Delta N(w)| \le 2\phi$, then $|N(v) \Delta N(w)| > 2\phi$, implying that $w \not \in R(v)$.  
%\end{lemma}
%\begin{proof}
%By Theorem 1, for all $w \in C^\ast(u)$, $|N(v) \Delta N(w)| \ge 2\phi$.
%\end{proof}

%\subsection{To Be Organized}


%Assume that at $i$-th round, we choose $u \in L_i$, then we have 
%\begin{lemma}
%\label{lem:keylemma0}
%$|N(u) \Delta C^\ast(u)| \le \phi$, when $\phi$ is the optimal solution.
%\end{lemma}
%\begin{proof}
%Assume that $|N(u) \Delta C^\ast(u)| > \phi$. By definition, $\rho_{C^\ast(u)} > \phi$, which %contradicts the definition of $C^\ast(u)$.
%\end{proof}

%Note that $C_i$ only contains nodes from $N(u)$, the next lemma show that we must add all nodes from $C^*(u) \cap N(u)$ to $R(u)$


%For following proofs, we assume that $\phi$ is the optimal solution. We will handle how to search for $\phi$ in section XXX.



%\begin{lemma}
%$C^\ast(u) \cap V_0 \cap N(u) \subseteq R(u)$.
%\end{lemma}

%\begin{proof}
%For each node $w \in C^\ast(u) \cap V_0 \cap N(u)$, we must have: 
%\begin{align*}
%    |N(w) \Delta C^*(u)| \leq \phi
%\end{align*}
%We also have, by Lemma 1.1 \label{lem:keylemma0}:
%\begin{align*}
%    |N(u) \Delta C^*(u)| \leq \phi
%\end{align*}
%So for every node $w \in C^\ast(u) \cap V_0 \cap N(u)$, we have using Lemma \ref{lem:keylemma1}:
%\begin{align*}
%    |N(u) \Delta N(w)| \le |N(w) \Delta C^\ast(u)| + |N(u) \Delta C^\ast(u)| \le 2\phi
%\end{align*}
%So all such nodes are added to $R(u)$ at the end as each node $w$ satisfies $w \in V_0 \cap N(u)$ and $|N(u) \Delta N(w)| \le 2\phi$.

%\end{proof}
% NOTE: add lemma about things in $C^\ast(u)$ not getting deleted due to the algo.
% Also, redo this lemma using the sheet from discussion w/ Su
% The next lemma give us relation between $C(u)$ and $C^*(u)$

%\begin{lemma}
%For each $u$ chosen in each round, $C(u) \subseteq C^\ast(u)$. 
%\end{lemma}
%\begin{proof}
%For every vertex $x$ with $d(x) \ge 3\phi$, we must have the following to be true for every $y %\in C^\ast(x)$:
%\begin{align*}
%|N(x) \Delta N(y)| \le 2\phi
%\end{align*}
%If it does not hold, then the number of disagreements for either $x$ or $y$ will exceed $\phi$ if they are clustered together. \\

%We now let $d(w) < 3\phi$ and $w \in N(u)$. The algorithm will add $w$ to the clustering if $|N(w) \Delta N(u)| \le 2\phi$. In this case, we add $w$ to $R(u)$. Lemma 1.3 says that if $C^\ast(u) \not = C^\ast(v)$, then $w \not \in R(v)$. Also, it says $|N(v) \Delta N(w)| > 2\phi$. Thus it suffices to show that for such vertices $w$, having $|N(u) \Delta N(w)| \le 2\phi$ implies $w \in C^\ast(u)$. \\

%So assume a vertex $w$ as described above. 

%\end{proof}
%\begin{lemma}
%$C(u) \Delta N(u) \subset C^*(u) \Delta N(u)$
%\end{lemma}
%\begin{proof}
%By Lemma 1.5, $C(u) \subseteq C^\ast(u)$. So, every $C(u) \cap N(u) \subseteq C^\ast(u) \cap N(u)$. Therefore, $C(u) \Delta N(u) \subset C^*(u) \Delta N(u)$.
%\end{proof}


%\begin{lemma}
%    For any node $w \in C(u)$ such that $d(w) \leq 3 \phi$, we have $|C(u) \Delta N(w)| \leq 3\phi$.
%\end{lemma}
%\begin{proof} We know that $|C(u) \Delta N(u)| \leq |C^*(u) \Delta N(u)| \leq \phi$. On the other hand, for any node $w \in C(u)$ such that $d(w) \leq 3 \phi$, we add it because $|N(u) \Delta N(w)| \leq 2\phi$, using triangle inequality, we have $|C(u) \Delta N(w)| \leq 3\phi$.
%\end{proof}

%\begin{lemma}
%    For any node $w \in C(u)$ such that $d(w) \in [3 \phi, 4 \phi)$, we have $|C(u) \Delta N(w)| \leq 3\phi$.
%\end{lemma}

%\begin{lemma}
%    Let $H$ contain all nodes with degree at least $4\phi$. In any clustering $C$ with obj($C$) $\le \phi$, $u$ and $v$ must be in the same clustering if $|N(u) \cap N(v)| > 2\phi$.
%\end{lemma}
%\begin{proof}
%    Let $|N(u) \cap N(v)| > 2\phi$ and assume we have a clustering $C$ with obj($C$) $\le \phi$. Put $v$ in cluster $V$, $u$ in cluster $U$, and have $V \neq U$. Vertices in $N(u) \cap N(v)$ can either be clustered in $V$, $U$, or in neither $V$ or $U$. If they are not placed in either $V$ or $U$, they contribute a disagreement to both $u$ and $v$. If they are clustered in $V$, they contribute a disagreement to $u$. If they are clustered in $U$, they contribute a disagreement to $v$. \\

%    In all cases, any choice of how to cluster a vertex in $N(u) \cap N(v)$ contributes at least one disagreement to either $u$ or $v$. There are greater than $2\phi$ such vertices, so either $u$ or $v$ will have greater than $\phi$ disagreements. This contradicts the assumption that we have a clustering $C$ with obj($C$) $\le \phi$. Thus, we must cluster $u$ and $v$ together whenever we have such a clustering.
%\end{proof}


%Show this works now for x negative
