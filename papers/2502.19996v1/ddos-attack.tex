\section{DDoS Attacks}\label{sec:ddos-attack}
In this section, we present a comprehensive review of the latest developments in DDoS attacks.
A successful DDoS attack include botnet formation, exploiting target selection, and malicious traffic generation.
Specifically, botnet formation involves botnet recruitment and coordination.
Attackers first create a network of compromised computers, known as a botnet, by exploiting vulnerabilities in devices to install malware.
Then the attacker uses command and control (C\&C) servers to manage the botnet for synchronized coordination of the attack.
With the botnet, the next step is to select exploiting targets.
Attackers may identify vulnerabilities in network protocols, such as HTTP, DNS, or TCP/IP, to exploit during the attack.
Specific features and weaknesses in the target system (e.g., content caching) can also be leveraged.
Finally, the attacker decides on the type of malicious traffic to generate (e.g., SYN request).
Attackers also design the traffic pattern to maximize disruption, potentially using slow and low attacks to evade detection or high-volume bursts to overwhelm the target quickly.
Adversarial tactics can be enforced during the attack (e.g., encrypted traffic), which bypass traditional detection efforts.

Following the DDoS attack workflow, our exploration begins with an examination of sophisticated methodologies for botnet recruitment and coordination, which serve as the primary mechanism for attackers to orchestrate DDoS campaigns (see Section~\ref{subsec:botnet-setup} and Section~\ref{subsec:botnet-coordination} for details).
We proceed to classify the spectrum of current DDoS threats from three innovative angles:
Firstly, attacks that exploit emerging network protocols are addressed in Section~\ref{subsec:ddos-targeting-protocol};
secondly, we discuss attacks that specifically target new and evolving systems in Section~\ref{subsec:ddos-targeting-system};
and thirdly, we delve into adversarial strategies designed to evade detection mechanisms, outlined in Section~\ref{subsec:adversarial-ddos-bypassing-detection}.
A visual representation of the DDoS attack taxonomy and a summarizing overview can be found in Figure~\ref{fig:ddos-attack-taxonomy}.
\begin{figure*}
    \centering
    \includegraphics[scale=.3]{figures/ddos-attack-taxonomy.pdf}
    \caption{Taxonomy of DDoS attacks.}
    \label{fig:ddos-attack-taxonomy}
\end{figure*}

\subsection{Botnet Recruitment}\label{subsec:botnet-setup}
To mount a formidable DDoS attack, attackers must first construct a botnet by amassing a collection of compromised devices known as bots. This section outlines three prevalent strategies for bot recruitment.

\textbf{Network services with weak authentication.}
The pivotal role of vulnerable network services as the primary channels for malware distribution has been consistently underscored by recent research.
Pa et al.~\cite{pa2016iotpot} innovatively created IoTPOT, a honeypot emulating the Telnet vulnerabilities specific to IoT devices.
This honeypot’s effectiveness was proven by the 481,521 malware download attempts it recorded, vividly illustrating the allure of weakly secured network services to potential attackers.
The study detailed the malware propagation process, beginning with attackers exploiting a list of common Telnet credentials to infiltrate devices.
Once access was gained, the attackers proceeded to download a malicious binary, setting the stage for monetization of the breach.
The compromised devices were predominantly used for DDoS attacks, demonstrating the severe consequences of inadequate authentication measures.
Supporting this, Choi et al.~\cite{choi2022understanding} provided an analysis of malware's traffic patterns, noting a substantial sharing of target IPs across varying malware sources—a testament to the widespread exploitation of network service vulnerabilities.
Their findings indicated a preferential exploitation of ports 80 (HTTP service) and 22 (TCP service), although attackers also strategically targeted lesser-known ports such as 111 and 123.

\textbf{Exposed system vulnerabilities.}
The strategic exploitation of device vulnerabilities for the dissemination of DDoS malware is a widely recognized tactic among attackers.
Al et al.~\cite{al2023bin} conducted a comprehensive analysis of 11,893 malware binaries, revealing that a significant number—2,629 binaries—specifically targeted known vulnerabilities documented in databases like CVE and NVD.
This starkly highlights how reported vulnerabilities can serve as a beacon for attackers to compromise devices and assimilate them into botnets.
The research further categorized the vulnerabilities, with remote code execution vulnerabilities being the most prevalent, followed by command injection.
This prioritization indicates a tactical selection by attackers, opting for vulnerabilities that provide the most control over compromised devices.
In their targeting strategy, attackers exhibited a preference for unpatched vulnerabilities or those with complex patching processes, revealing a calculated exploitation of slower mitigation responses.
Additionally, Al et al. pointed out that the availability of public proof-of-concept (PoC) exploits considerably increased the risk of a vulnerability being targeted.
These PoC exploits act as a double-edged sword: while they serve to inform security practitioners about potential vulnerabilities, they also provide attackers with a roadmap for exploitation.
Contrary to what might be expected, the study observed that the severity of a vulnerability did not necessarily correlate with its exploitation frequency. This insight suggests that attackers are opportunistic, focusing less on the potential impact of a vulnerability and more on the ease of exploitation and the effectiveness in spreading DDoS malware.

\textbf{Online integrated development environments (IDEs)}.
Srinivasa et al.~\cite{srinivasa2022bad} have unearthed a concerning trend wherein online Integrated Development Environments (IDEs) emerge as unsuspecting tools in the hands of attackers to propagate DDoS malware.
Their study reveals that the inherent vulnerabilities within unregulated online IDEs can be systematically exploited, turning the IDE servers into unwilling participants in DDoS attacks as part of a botnet.
The crux of the issue lies in the permissiveness of these online IDEs, which, with their unrestricted imports and non-sandboxed operational environments, allow attackers to effortlessly introduce malicious libraries.
In addition to the ease of infiltration, the potential for unbounded resource consumption within these platforms presents an opportunity for attackers to orchestrate attacks of substantial scale.
The research highlighted the alarming prevalence of such vulnerabilities, with 719 out of 2,269 online Python IDEs identified as uncontrolled.
The capability of these compromised online IDEs to generate massive amounts of network traffic serves as a testament to their potency in DDoS attacks.
The study indicates that a coordinated utilization of just 32 of these vulnerable IDEs is enough to unleash an average of 6 million requests per minute, capable of overwhelming systems and bringing down critical online services.

\subsection{Botnet Coordination}\label{subsec:botnet-coordination}
Once the bots are recruited, the attackers coordinate them for the targets.
While the size of the botnet and the target may vary, existing works point out some common coordination strategies.
Wang et al.~\cite{wang2018data} investigate 50,704 different Internet DDoS attacks across the globe in a seven-month period, and study how attackers scheduled their controlled bots.
The result shows that attackers deliberately schedule their controlled bots in a dynamic fashion.
For example, attackers do not uniformly sample bots across different countries. Instead, the distribution shifts over time, and such shifts can be statistically modeled.
Moreover, the author identifies similar shifting patterns over different botnets, e.g., Dirtjumper and Pandora.
This implies that different botnets may collaborate or share resources when launching DDoS, and a single bot may participate in attacks launched by different botnets.

Attackers may also validate the effectiveness of their coordination strategies by simulation.
For example, Obaidat et al.~\cite{obaidat2023creating} provide a simulation framework DDOSim.
It allows attacker to build attacker/victim nodes by loading docker containers with their malicious binaries and software.
Network stacks, malware, attack scripts, and softwarized defense systems can be contained in the container to support diverse evaluation tasks.
Moreover, attackers can customize the simulated network topology and configuration, such that he can evaluate the attack impacts under different network conditions.
Finally, with quantitative measurement of various metrics, e.g., server network throughput, DDOSim achieves real-time monitoring and evaluation of attack/defense progression.

\subsection{DDoS Targeting Protocols}\label{subsec:ddos-targeting-protocol}
As network protocols evolve to offer improved speed, reliability, and security, malicious actors adapt their strategies accordingly.
DDoS attacks are now progressively aimed at these advanced protocols, taking advantage of vulnerabilities that have yet to be addressed with security patches or protocol enhancements.
At the same time, the advanced features of these protocols (e.g., multiplexing) are being exploited to design advanced DDoS attacks.
This section will explore the protocols that are susceptible to such attacks, identify their specific vulnerabilities, and describe the methods by which attackers exploit these weaknesses to carry out DDoS campaigns.

\subsubsection{Transport-Layer Protocol}
\textbf{Transmission Control Protocol (TCP)}.
TCP serves as the backbone of the internet, facilitating reliable communication across its vast network.
However, the ubiquity of TCP also makes it a prime target for security exploits, with its inherent vulnerabilities becoming a focal point for researchers.
The most famous attacks targeting TCP are SYN flooding and Shrew DDoS attacks.
Specifically, SYN flooding attacks leverage the TCP three-way handshake by sending excessive SYN requests, which the server responds to with SYN/ACK packets, awaiting completion of the connection that never occurs~\cite{wang2002detecting}.
This results in an accumulation of half-open connections in the server's backlog queue—a memory structure with limited capacity.
Once this queue is full, legitimate connection requests are rejected, leading to a denial of service.
This attack method exploits the finite size of the backlog queue and highlights a critical vulnerability within the TCP session management. 

Luo et al.~\cite{luo2014mathematical} conducted a formal analysis of the Shrew DDoS attack, which manipulates the TCP's retransmission timeout (RTO) mechanism.
By sending high-rate packet bursts at a frequency that aligns with the RTO intervals, the Shrew attack induces repeated timeouts in legitimate TCP connections.
This results in severe congestion at the network bottleneck upon each recovery attempt, causing the throughput of legitimate traffic to plummet, potentially to near-zero levels.
Luo et al. developed a mathematical model to assess how varying attack patterns and network conditions influence the success of this attack, delineating the least resources required for a successful Shrew attack and its potential maximum impact.

Expanding upon this concept, Tang et al.~\cite{tang2013modeling} generalized the principles behind the Shrew attack to formulate a model for low-rate denial-of-service attacks, also known as Reduction of Quality (RoQ) attacks.
These attacks exploit the feedback control mechanisms within network protocols, e.g., TCP's dynamic congestion window adjustment, forcing the victim's system into a suboptimal state controlled by the attacker.
This strategy effectively diverges the system from its intended operational state.

It’s important to note that low-rate denial-of-service attacks are not exclusive to the TCP protocol.
Schuchard et al.~\cite{schuchard2010losing} extended the attack vector to the Border Gateway Protocol (BGP), introducing the Coordinated Cross Plane Session Termination (CXPST) attack.
This method disrupts BGP sessions by causing intermittent link congestion, leading to repeated session disconnections and reconnections between victim routers.
In the wireless domain, Chen et al.~\cite{chen2008feasibility} demonstrated the feasibility of such attacks on 802.11 networks by periodically interfering with TCP acknowledgment (ACK) packets, forcing the sender into unnecessary retransmissions and throttling the transmission rate.
Lastly, He et al.~\cite{he2009reduction} applied the low-rate DDoS concept to peer-to-peer (P2P) protocols, showing that attackers can destabilize these networks through timed patterns of joining and leaving, knocking the system off its stable equilibrium.

\textbf{Quick UDP Internet Connection (QUIC).}
QUIC, a modern transport protocol analogous to UDP, was created by Google and has been standardized by the Internet Engineering Task Force (IETF).
This protocol is designed to enhance transport layer security and privacy while simultaneously reducing connection establishment latency.
Despite these improvements, research has uncovered that QUIC can be susceptible to Distributed Denial of Service (DDoS) attacks.

Nawrocki et al.~\cite{nawrocki2021quicsand} explore two primary attack strategies that threaten the integrity of QUIC: state-overflow and reflective amplification attacks.
A state-overflow attack involves an attacker impersonating a QUIC client to inundate the server with a deluge of connection states.
The attacker initiates handshakes repeatedly, compelling the server to consume resources to track each supposed connection by issuing a unique Source Connection ID (SCID) and its corresponding Transport Layer Security (TLS) certificate.
These fraudulent requests impose a heavy cryptographic load and exhaust server resources dedicated to managing connection states.
The attacker exacerbates this situation by using spoofed IP addresses and port numbers, inflating the server's state management workload to the point where it may become incapable of serving legitimate requests, leading to service outages.

Reflective amplification attacks exploit the UDP foundation of QUIC, enabling IP spoofing.
In this scenario, attackers control numerous bots that send QUIC Initial packets with falsified source IP addresses—typically that of the intended victim—to a QUIC server.
The server then replies with QUIC Initial messages that include TLS handshakes, which, by including server certificates, are significantly larger than the incoming requests.
The QUIC standard restricts servers from sending more than three times the data they receive prior to client verification.
Attackers circumvent this by padding Initial packets with superfluous bytes, inflating the response size.
This tactic is deceptive, as it resembles a recommended practice for streamlining the handshake process, where large initial packets enable the server to transmit certificates in a single message, thus reducing delays.
Consequently, the victim's network is bombarded with an overwhelming response from the server.

\subsubsection{Network-Layer Protocol}
\textbf{Internet Protocol (IP).}
The integrity of the Internet Protocol (IP) is critical for the stable operation of networked systems.
However, research has brought to light substantial vulnerabilities within the IP protocol that can be weaponized to execute denial of service attacks.
A primary method of exploitation is the IP fragmentation attack, which targets inherent weaknesses in the IP fragmentation process.
Attackers can send meticulously crafted packets with overlapping fragment offsets, which confound the target system's ability to correctly reassemble the fragments.
This confusion can lead to improper packet reassembly or even buffer overflows, potentially resulting in crashes and service interruptions.

Gilad et al.~\cite{gilad2011fragmentation} have demonstrated that even attackers who lack a direct path to the communication stream can predict the IP identification values that packets will use.
These values are crucial for the reassembly process, as they indicate which fragments belong to which packets.
Attackers can then send fraudulent fragments bearing these anticipated identifiers to a victim.
When the legitimate fragments are received, they are incorrectly reassembled with the attacker's fragments, causing packet corruption and loss.
In the realm of IPv6, Atlasis et al.~\cite{atlasis2012attacking} reveal that the exploitation potential is more severe.
Unlike IPv4, IPv6 introduces a more complex system of extension headers.
Attackers can exploit this by deliberately crafting packets that split critical TCP/UDP header information across multiple fragments and bypass firewall detection.
The initial malicious fragment which contains the IPv6 header and possibly a fragment of the TCP/UDP header can pass through the firewall, since it typically inspects only the IPv6 header of the traffic. Subsequent fragments, which may contain the rest of the TCP/UDP headers, can slip through the firewall unchecked since many security systems do not perform full reassembly of packet fragments before inspection.

\subsubsection{Application-Layer Protocol}
\textbf{HTTP(/2)}.
DDoS attacks have evolved significantly with the advancement of web protocols.
Attacks that exploit the HTTP protocol, particularly HTTP/1.x and HTTP/2, have become a pressing concern.
Dantas et al.~\cite{dantas2014selective} have identified three distinct DDoS attack strategies that exploit HTTP/1.x: HTTP GET, HTTP PRAGMA, and HTTP POST.
Slow Write attacks, which make use of the HTTP GET and POST requests, operate by sending request fragments to a server at a deliberately slow pace.
This was notably employed in the Slowloris attack, which emerged after the 2009 Iranian Presidential elections.
By transmitting tiny fragments of a request and pausing until just before the server's timeout interval expires, attackers can keep connections open indefinitely.
This forces servers to maintain these malicious connections, eventually exhausting their capacity to accept legitimate requests.
The efficacy of these attacks is further enhanced by exploiting the HTTP PRAGMA header.
By including this header in requests, attackers can reset the server's timeout timer, allowing the malicious connection to persist even longer.
This tactic effectively monopolizes server resources, leaving fewer sockets available for genuine users.

The research by Beckett et al.\cite{beckett2017http} and Praseed et al.\cite{praseed2019multiplexed} reveals the susceptibility of HTTP/2 to DDoS attacks.
While HTTP/2 introduced advanced features like multiplexing (the ability to send multiple requests/responses in a single connection) and server push (the server pre-emptively sends resources to the client), they can be weaponized by attackers.
Beckett et al.~\cite{beckett2017http} demonstrate that HTTP/2's multiplexing capability amplifies the impact of HTTP floods.
Attackers can bundle numerous requests into a single packet, leading to a flood that is significantly more potent than one using HTTP/1.1.
The study showed that with the same rate of packet transmission, an attack could be magnified by up to 95 times when compared to HTTP/1.1.
Building on this, Praseed et al.~\cite{praseed2019multiplexed} introduce an enhanced multiplexing attack that selects high-workload requests to maximize the target server's CPU usage while remaining under the radar.
The server push feature exacerbates the situation; it prompts the server to handle not only the direct requests but also the associated inline requests, such as resources linked to a web page.
This can lead to the server's CPU usage spiking to 80\% with as few as four attacking bots, illustrating the severe impact of such attacks.

\textbf{SIP}.
The Session Initiation Protocol (SIP) is a cornerstone of Voice over Internet Protocol (VoIP) technologies, facilitating a wide array of communication services.
As an application-layer protocol, SIP relies heavily on the functionality of intermediate proxy servers to manage the signaling and control of voice sessions.
Despite its widespread adoption, SIP's reliance on these servers and its session management mechanisms introduce vulnerabilities ripe for exploitation.
Sisalem et al.~\cite{sisalem2006denial} and Tang et al.~\cite{tang2014sip} have conducted extensive research into the vulnerabilities inherent in SIP, identifying multiple avenues through which attackers can launch Denial of Service (DoS) attacks.
Their work categorizes three primary types of SIP flooding attacks that affect the stability and availability of SIP proxy servers—INVITE flooding, BYE flooding, and Multi-Attribute flooding.

The INVITE flooding attack is a method by which an attacker overwhelms the SIP proxy with an excessive number of INVITE requests.
These requests aim to initiate new SIP sessions, and the proxy server, in attempting to maintain state information for each session, eventually depletes its memory resources.
This form of attack targets the fundamental role of the proxy in establishing communication sessions, thereby crippling its ability to service legitimate users.
BYE flooding takes a different approach.
In this scenario, the attacker sends a large volume of spoofed BYE messages, which are protocol methods designed to terminate existing SIP sessions.
By generating these messages with brute-forced user addresses, the attacker can trick the proxy into prematurely ending a substantial number of active VoIP calls from benign users, causing widespread disruption.
Finally, the Multi-Attribute flooding attack combines various forms of SIP flooding, such as INVITE and BYE flooding, to create a more complex and damaging assault.
By varying the attack vectors, attackers can inflict compounded harm on the proxy server while simultaneously evading detection systems that typically rely on analyzing the proportion of different SIP methods used in the traffic flow.

\textbf{Domain Name System (DNS).}
The Domain Name System (DNS) is a critical component of the Internet's infrastructure, underpinning the resolution of domain names into IP addresses.
Its significance is paralleled by its attractiveness as a target for Distributed Denial of Service (DDoS) attacks, particularly those aiming for amplification.
DNS servers, especially those that are publicly accessible, are essential for handling name resolution requests from clients.
These servers are capable of querying multiple DNS zones—each containing a set of DNS records—in a single request.
This capability makes them prime targets for amplification attacks because they often return responses significantly larger than the incoming requests.
Moreover, since DNS protocol relies on UDP, which does not require a connection and is susceptible to IP address spoofing, it allows attackers to alter the source address of DNS queries, making the responses go to the victim's IP (i.e., reflection).

The DNS amplification attack, as described by Kim et al.~\cite{kim2017preventing}, exploits these characteristics.
Attackers forge the source IP in a DNS request to match that of their intended victim.
The DNS server, unaware of the spoofing, sends a response, which can be many times larger than the request, to the victim's IP address, thereby flooding it with unsolicited traffic.
The attack is highly efficient and difficult to trace for two reasons.
(1) Economical: It requires minimal effort from the attacker, who need only generate small query packets to elicit large responses, resulting in a significant amplification of traffic.
(2) The attack traffic appears to originate from legitimate DNS servers, not the attacker, making it challenging to identify the true source through traffic analysis.
To identify candidate DNS resolvers, Yazdani et al.~\cite{yazdani2022mirrors} explore the misuse of cloud-based DNS infrastructures.
Their findings indicate that a substantial number—around 12\%—of the 3 million DNS resolvers analyzed are hosted within cloud networks.
These cloud-based resolvers can be powerful instruments for attackers seeking to amplify their attacks. Furthermore, some cloud providers' lack of destination-side address validation exacerbates the vulnerability of their DNS resolvers to external attacks.

Griffioen et al.~\cite{griffioen2021scan} examined the procedures of amplification DDoS attacks through the deployment of 549 honeypots across five public cloud platforms.
The researchers implemented traffic shaping techniques to ensure that these honeypots did not contribute to the attacks while enabling the monitoring and analysis of the attacks' characteristics.
Throughout the duration of the study, approximately 13,000 attacks were recorded, leading to several noteworthy discoveries.
Firstly, the study found that attackers engage in preliminary testing of servers to evaluate their potential use in amplification attacks.
This testing involves sending bursts of requests to assess whether the servers' responses are consistent with expected protocol behaviors.
Moreover, the data revealed that attackers keep track of servers that have previously demonstrated a high amplification factor.
Evidence from honeypot records indicated that attackers would revisit IP addresses of servers that had been effective amplifiers in the past, even if those servers had since ceased responding.
Lastly, the research highlighted a tactical approach employed by sophisticated attackers: Pulsing their traffic instead of sending a constant stream.
This strategy is indicative of an effort to optimize the cost-efficiency of their attacks.

Note that in addition to DNS, a multitude of network protocols has been exploited to facilitate amplification attacks. In 2020, the Federal Bureau of Investigation (FBI) issued an alert regarding the exploitation of three specific network protocols: Apple Remote Management Services (ARMS), Web Services Dynamic Discovery (WS-DD), and the Constrained Application Protocol (CoAP)~\cite{fbi-amplification-attack}. These protocols were found to be vulnerable to misuse for amplifying malicious traffic in Distributed Denial of Service (DDoS) attacks.
In the following year, SECURELIST expanded on this list by identifying three additional protocols that had been abused: the Microsoft Remote Desktop Protocol (RDP), the Chameleon Protocol for Virtual Private Networks (VPNs), and the Datagram Transport Layer Security Protocol (DTLS)~\cite{securelist-amplification-attack}.
The discovery of such a diverse array of exploitable protocols for amplification underscores the attractiveness of these methods to attackers.
The potential for significant damage and the relative ease of orchestrating these attacks make them a persistent threat in the cyber landscape.

Beyond amplification DDoS, Yin et al.~\cite{yin2023waterpurifier} introduce the concept of the DNS water torture attack.
This method exploits the recursive nature of DNS resolution, where queries are forwarded between servers, rather than leveraging the size disparity between request and response.
In this attack, a botnet inundates a target domain, such as example.com, with requests for non-existent subdomains.
Due to the recursive lookup required to resolve these fabricated subdomains, the authoritative server for example.com becomes overwhelmed.
Consequently, legitimate DNS queries for the domain fail, leading to service disruption for the targeted domain.

Pan et al.~\cite{pan2024loopy} reveal that with spoofed packets, attackers can create loops between two servers.
Severely, attackers can create infinite communication loops between two servers with even single packet.
The root cause is the flaw design about error message handling.
An error message as input can create an error message as output for two DNS systems.
As a result, with a spoofed error message,  two DNS systems will keep
sending error messages back and forth indefinitely.
Such vulnerabilities can be easily exploited to create DDoS attacks.
For example, an attacker can create many loops with other loop
servers, all of which concentrate on a single target loop server.
As a result, the target server either exhausts its host
bandwidth or computational resources.

\textbf{IoT protocols.}
The Modbus protocol is designed to facilitate communication among IoT devices within industrial control systems, such as electricity and gas supply networks.
Due to their critical importance, these systems often become targets for attackers seeking to hijack and disable them, with the exploitation of the Modbus protocol being a primary focus.
Mohammed et al.~\cite{mohammed2023detection} introduced a novel field flooding attack that leverages the structure of Modbus packets to execute a DoS attack.
The adversary can craft malicious packets by modifying the ModbusTCP packet header, and these modifications aim to enlarge the allocated memory for the malicious packets.
For instance, the adversary can modify the length field in the write packet header with extremely large valuess.
Consequently, this triggers the target control units to allocate large memory and create an overflow of the memory bank, causing them to crash.

Besides IoT protocols used in the industrial system, researchers show that protocols for smart home devices are also vulnerable.
Wang et al.~\cite{wang2022zigbee} exposes a specific DoS vulnerability lurking in the device rejoin procedure of the Zigbee protocol.
The study demonstrates how an attacker can exploit this vulnerability by coordinating compromised Zigbee devices to send falsified rejoin requests to target routers.
A flaw in the aging-out process of the rejoin procedure is revealed, allowing these unauthorized connections to overwhelm the router’s capacity, with the router erroneously maintaining these connections.
This in turn prevents legitimate Zigbee devices from joining the network, effectively resulting in a denial of service for the intended users.


\subsection{DDoS Targeting Systems}\label{subsec:ddos-targeting-system}
The advancement of technology brings with it innovative systems aimed at enhancing efficiency and user experience.
However, alongside these developments, a worrying trend emerges in the landscape of cyber threats.
Recent patterns in denial-of-service attacks reveal a targeted interest in these modern systems.
Malicious actors are keenly searching for and exploiting vulnerabilities in these cutting-edge frameworks, particularly those still in early stages of deployment.
Within this section, we will delve into the systems impacted by these threats, examine their specific vulnerabilities, and explore the methods attackers employ to leverage these weaknesses, resulting in denial-of-service incidents.

\subsubsection{Networking Infrastructure}

\textbf{Routing system.}
The routing system, an intricate web of routers and connecting links, is pivotal in directing network traffic.
Its seamless operation is essential for maintaining the integrity and availability of network services.
However, recent research illustrates that this system is not impervious to attack; malefactors have developed methods to exploit it, potentially causing widespread denial of service that could cripple a district or even bring a nation's digital infrastructure to a standstill.

Studer et al.~\cite{studer2009coremelt} have uncovered novel attack strategies that specifically target and overwhelm crucial network links, severing connections to the intended victim host.
For instance, the Coremelt attack operates by utilizing a network of compromised machines that exchange high volumes of data, thereby inundating and incapacitating a vital link within the network.
This results in a denial of service for all servers dependent on the affected link.
The insidious nature of this attack lies in the fact that the compromised machines, being the recipients of the flooding traffic, enable the attacker to circumvent traditional filtering-based DoS defenses that are typically employed to protect the server.

The Coremelt attack presupposes a substantial botnet under the attacker's control, with these bots strategically positioned both upstream and downstream of the target link.
Recognizing the limitation of this assumption, further research by Kang et al. introduced the Crossfire attack~\cite{kang2013crossfire}, which aims to mitigate the dependency on a vast botnet.
This method involves coordinating the bots to send traffic to a series of decoy servers, strategically situated downstream of the critical link.
The malicious traffic, destined for the decoy servers, must traverse the targeted link, resulting in its congestion. Consequently, all servers within the region that rely on this link would suffer from a denial of service.
The Crossfire attack represents an evolution in denial of service techniques by reducing the reliance on the number and distribution of bots, and instead focusing on the strategic generation of traffic to exploit the routing system's vulnerabilities.
Through the aggregation of traffic at critical junctures, attackers can induce a significant impact with fewer resources, posing a grave threat to the robustness of modern network infrastructure.

\textbf{Cellular network.}
The cellular network plays a crucial role in mobile communication, with the Long Term Evolution (LTE) standard—developed by the 3rd Generation Partnership Project (3GPP)—serving as the backbone for current and emerging cellular technologies, including 4G and 5G.
Despite its advancements, LTE is vulnerable to Distributed Denial of Service (DDoS) attacks that pose significant challenges to network stability and user security.

Attackers, aiming to disrupt the LTE network, amass mobile malware to create a formidable botnet.
By exploiting LTE's architecture—which distinctly separates the control plane (responsible for signaling) from the data plane (responsible for user data)—these attackers can specifically target the control plane with a deluge of signaling traffic.
Research has identified critical vulnerabilities in LTE procedures, such as the user attach and handover processes, where attackers can induce signaling storms with minimal effort, creating an amplification effect that leads to service disruptions~\cite{henrydoss2014critical,silva2020repel}.
Another aspect of DDoS susceptibility in mobile communications is network slicing, a key technology in 5G networks that enables differentiated services through a shared infrastructure.
Literature suggests that due to the inherent design of physical resource sharing in network slicing, an attack on a single service can have cascading effects, disrupting multiple services across different slices and magnifying the attack's impact~\cite{olimid20205g,sattar2019towards,javadpour2023reinforcement}.

The emergency service supported by the cellular network can also be exploited by attackers.
In particular, the 911 emergency service system, a critical component of public safety, is not immune to these threats.
Mirsky et al. highlighted the vulnerability of 911 services to DDoS attacks perpetrated through mobile phone botnets~\cite{mirsky2020ddos}.
An attacker orchestrates this by infecting smartphones with malware to form a botnet, which is then directed to place continuous emergency calls using randomized IMSIs.
The Federal Communications Commission's (FCC) mandate to route unidentified emergency calls without blocking creates an exploitable loophole.
The botnet, leveraging this policy and the randomized IMSIs, can evade detection by the cellular network and flood the 911 service infrastructure, resulting in a critical service outage.

\textbf{Software-Defined Network (SDN).}
SDN has revolutionized network architecture by decoupling the control plane from the data plane, thereby introducing greater flexibility and programmability.
However, this paradigm shift also presents novel vulnerabilities, particularly to denial of service attacks.
Among these, DDoS poses a critical threat due to its capability to leverage multiple launch points and its potential to inflict severe service disruptions.

A pivotal study by Shin et al.~\cite{shin2013avant} elucidates the vulnerability inherent in the separation of the control and data planes, particularly to what is termed a control plane saturation attack.
In an SDN environment, when a switch encounters a packet from an unrecognized flow, it refers the packet to the centralized controller for further instructions.
This controller is integral to the SDN's operation as it manages flow requests and configures the network dynamically. However, it is also a singular point of failure.
An adversary can exploit this by coordinating a multitude of compromised devices, or bots, to generate an overwhelming number of unique flow requests.
This orchestrated effort can saturate the control plane, effectively paralyzing the network's ability to manage legitimate traffic.

In addition to the control plane, the data plane is also susceptible to a similar form of exploitation.
As demonstrated in another work by Shin et al.~\cite{shin2013attacking}, attackers are capable of initiating a data plane saturation attack by inundating the network with a vast array of unique flows.
This barrage of flow requests leads to the generation of numerous redundant flow rules, which the data plane must process and store.
The data plane, encumbered by this deluge of spurious rules, becomes less efficient or even incapable of handling legitimate network flows, severely degrading network performance.

Cao et al.~\cite{cao2019crosspath} have further identified that attackers could exploit the shared links between control and data traffic paths, thereby disrupting the SDN control channel.
The proposed CrossPath Attack involves first probing the SDN with data traffic bursts to identify shared links by observing control message delays.
Once identified, the attacker can employ a low-rate, TCP-targeted DoS attack to create data traffic pulses, inducing congestion on these critical links and impairing control message transmission.

\textbf{Named Data Networking (NDN}).
The Named Data Networking (NDN) paradigm represents a promising shift in network infrastructure, focusing on content-centric operations rather than the traditional location-centric approach characteristic of the IP protocol.
Unlike the IP architecture which relies on specific location addresses, NDN operates on a named resource basis, allowing users to request content by name without requiring knowledge of its physical location.
For instance, a news article from CNN could be requested with the name `/ndn/cnn/news/2012May20', which NDN routers can process to retrieve the content directly, bypassing the need to locate the CNN server.
While NDN's inherent features, such as in-network caching and the symmetry of interest and content paths, offer a degree of resistance against conventional DDoS attacks like bandwidth depletion and reflection attacks, they are not a panacea.
Research has shown that modified traditional DDoS attacks can still effectively exploit these features and compromise NDN's operations.

Interest Flooding~\cite{gasti2013and,mannes2019naming} is an attack that overwhelms NDN routers by exploiting their caching capability for unsatisfied Interest requests.
Attackers can coordinate botnets to generate excessive Interest requests, saturating the router's cache and obstructing the processing of legitimate interests.
Additionally, Content Poisoning attacks aim to corrupt the content caches within benign routers, obstructing the caching of legitimate content.
This attack involves using bots to issue a multitude of interest requests, followed by a compromised host responding with poisoned content, leading to the proliferation of tainted content across the network's cache.

\subsubsection{Distributed System}

\textbf{Internet of Things.}
In the burgeoning landscape of interconnected devices, the susceptibility of smart home devices to cyber-attacks poses a significant threat.
The study by Tushir et al.~\cite{tushir2020quantitative} quantitatively assesses the impact of DDoS on these devices.
The findings highlight a considerable variation in the minimum attack rate required to disrupt different smart home devices and cause power outrage.
The research underscores a critical vulnerability inherent in the dependency of these devices on WiFi connections; specifically, the process of group key updating in WiFi, which is shown to exacerbate the risk of DDoS attacks by precipitating faster disconnections of devices.
Furthermore, the study delineates several key factors that influence the energy consumption of victim devices during an attack, including the utilized communication protocols, the rate and size of the attack payloads, and the state of the device ports.

In the realm of smart grid systems, vulnerabilities to DDoS attacks have been identified~\cite{vukovic2014security}.
The research focuses on the distributed state estimation module, which is integral to the operation and supervision of the power system.
An attacker gaining control of a central control center can manipulate the state data communicated between this center and its adjacent centers.
This manipulation can compromise the reliability of the data used by neighboring control centers, which is crucial for their operational decision-making.
Consequently, the dissemination of false data can incapacitate these systems, leading to a denial of service and potentially catastrophic failures in power system management.

\textbf{Blockchain system.}
Blockchain technology has emerged as a groundbreaking innovation, yet it is not immune to the prevalent threat of denial of service attacks.
Vasek et al.~\cite{vasek2014empirical} have documented a significant number of DDoS attacks targeting the Bitcoin ecosystem, identifying 142 unique instances across 40 services.
Their research indicates that approximately 7\% of service operators have been subjected to such attacks, with currency exchanges and mining pools being the most frequently targeted.

Further exploring this avenue, studies by Johnson et al.~\cite{johnson2014game} and Wu et al.~\cite{wu2020survive} elaborate on the strategies used by malicious entities within the competitive landscape of mining.
Johnson et al.~\cite{johnson2014game} reveal the strategic trade-offs faced by resource-limited attackers: Either to allocate computing resources to their mining efforts or to engage in DDoS attacks to diminish the success rate of rival pools.
Through game-theoretical modeling, they identify optimal DDoS strategies, which include the selection of victim pools and the allocation of resources for the attack.
Wu et al.~\cite{wu2020survive} extend this analysis to a dynamic environment where miners frequently switch pools, leading to evolving pool sizes.
They model this interaction as a general-sum stochastic game and develop a Nash learning algorithm to deduce near-optimal attack strategies, thereby maximizing the attacker's rewards.

Li et al.~\cite{li2021deter} examine a different attack vector within the Ethereum network, focusing on the abuse of transaction handling mechanisms.
They identify how malicious actors can disrupt the network by sending malformed transactions with nonces that exceed the expected value or by initiating transactions that overdraft an account's balance.
These transactions can lead to the eviction of legitimate transactions from a victim's transaction pool and their replacement with invalid ones, effectively preventing the victim from disseminating valid transactions or including them in the blockchain.
Notably, these attacks incur minimal costs, as they consume negligible amounts of Ether.

Other research efforts, such as those by Heilman et al.~\cite{heilman2015eclipse} and Tran et al.~\cite{tran2020stealthier}, focus on connection manipulation attacks aimed at isolating nodes from the blockchain network.
The Eclipse attack~\cite{heilman2015eclipse} exploits Bitcoin's peer selection process, allowing attackers with numerous IP addresses to flood a victim's peer address database with malicious nodes, effectively segregating the victim from legitimate peers.
The EREBUS attack~\cite{tran2020stealthier} leverages an adversary's position as a man-in-the-middle Autonomous System to influence peering decisions over time, eventually replacing all of a victim's peers with spoofed ones, thus isolating them from the network.

Gervais et al.~\cite{gervais2015tampering} and Walck et al.~\cite{walck2019tendrilstaller} have researched methods to introduce delays into blockchain operations.
They exploit the data request protocols of Bitcoin, where nodes avoid redundant data requests from their peers.
Attackers can take advantage of this by advertising transactions to a victim node, causing it to wait indefinitely for data that the attacker never sends~\cite{gervais2015tampering}.
The TendrilStaller attack~\cite{walck2019tendrilstaller} delays block propagation to the victim with fewer attack resources.
The attack exploits a recent block propagation protocol which prescribes a Bitcoin node to select three neighbors that can send unsolicited blocks.
As a result, the attacker can induce the victim to select three attack nodes, which perform the delaying procedure to make the victim node stuck in the waiting process.

Finally, studies by Apostolaki et al.~\cite{apostolaki2017hijacking} and Li et al.~\cite{li2023bijack} highlight vulnerabilities in the underlying network protocols (BGP and TCP) used by blockchain systems.
These works illustrate how attackers can leverage AS-level BGP hijacks to intercept and disrupt Bitcoin traffic to and from victim nodes.
Specifically, Apostolaki et al.~\cite{apostolaki2017hijacking} demonstrate how adversaries with control over an Autonomous System can manipulate routing tables by executing BGP hijacks.
This is done by broadcasting spoofed BGP announcements that claim ownership of a victim node's IP prefix.
As a result, the malicious AS can intercept all traffic intended for the victim, selectively filter out Bitcoin traffic, and drop those packets.
The Bijack attack~\cite{li2023bijack} takes advantage of a specific flaw in the assignment method of the IPID field within the TCP protocol (Section~\ref{subsec:adversarial-ddos-bypassing-detection}).
The vulnerability enables attackers to deduce the active TCP connections of a victim node, including sensitive information such as the TCP sequence number, victim node’s port number, and the IP and port numbers of peers.
Armed with this information, the attacker can forge TCP RST packets to sever these connections, forcing the victim node to disconnect from the blockchain network.

\subsubsection{Computing Infrastructure}
\textbf{Remote direct memory access system (RDMA).}
Remote Direct Memory Access (RDMA) technology has seen a rapid adoption in a variety of settings, spanning from private data centers to multi-tenant cloud environments.
A notable example of its application is in distributed machine learning, where RDMA's ability to facilitate direct memory access from a client to a remote server's memory via an RDMA-enabled network interface card offers significant performance improvements.
This is chiefly because the data transfer operation circumvents the operating system and traditional network stack, leading to a more efficient communication process.

However, the introduction of RDMA has not come without its security implications.
In the work of Wang et al.~\cite{wanglordma}, it is highlighted that the congestion control mechanisms provided by the RDMA API, particularly Priority-based Flow Control (PFC) and Data Center Quantized Congestion Notification (DCQCN), inadvertently create new opportunities for denial of service attacks.
Attackers can exploit these mechanisms by initiating low-rate DDoS attacks with potentially severe consequences.
The adversaries direct bots to intermittently send bursts of traffic to a chosen egress port.
The pulsating nature of this traffic can manipulate the behavior of DCQCN, which employs an Additive-Increase/Multiplicative-Decrease (AIMD) algorithm similar to that used by TCP congestion control.
Consequently, legitimate traffic destined for the targeted port can be unfairly penalized and throttled, mimicking the effects of a TCP slow-rate DDoS attack.

The situation is exacerbated by the behavior of PFC during periods of congestion.
In an effort to prevent packet loss, PFC issues a PAUSE frame that travels in the opposite direction of the congested traffic, instructing all upstream switches to halt forwarding operations.
This has a cascading effect, as not only is the traffic heading towards the targeted egress port affected, but so too is any unrelated traffic that happens to traverse the impacted switches.
Such collateral damage extends the disruptive impact of the attack well beyond its intended target, illustrating the potential for widespread disruption within an RDMA-enabled network infrastructure.

\textbf{Serverless platform.}
The advent of serverless computing has introduced a paradigm where users can deploy and execute web applications through serverless functions without the overhead of managing servers.
This model only requires users to pay for the actual compute resources used during the execution of these functions.
A characteristic of serverless platforms is the assignment of platform-provided IP addresses, known as egress IPs, for outbound connectivity from these functions.
Notably, these egress IPs are shared among multiple serverless functions.

A study by Xiong et al.~\cite{xiong2021warmonger} exposes a vulnerability inherent to this architecture, wherein an attacker can orchestrate a DDoS attack by exploiting these shared egress IPs \cite{xiong2021warmonger}.
The attack is carried out by deploying several malicious serverless functions that generate a high volume of intrusive requests, such as HTTP floods, using the platform's egress IPs.
These requests are directed towards a targeted server, with the intention of overwhelming it.
Given that egress IPs are typically few in number and remain constant over time, a defensive action taken by the targeted server—such as blocking these IPs to mitigate the attack—can inadvertently lead to collateral damage.
Specifically, legitimate users of the serverless platform who share the blocked egress IPs find themselves inadvertently denied access to the targeted server.
This not only disrupts the services offered by the targeted server but also impacts the availability of services reliant on the serverless platform, illustrating a significant security concern within the serverless computing model.


\subsection{Adversarial Attack}\label{subsec:adversarial-ddos-bypassing-detection}
Recent trends in DDoS attacks have seen a shift towards the deployment of adversarial tactics.
In these sophisticated attacks, the perpetrator meticulously designs malicious traffic to mimic legitimate network flows.
This deceptive strategy is intended to evade current detection and mitigation systems, allowing the harmful data to pass unchecked and be received by the targeted victim.
This section delves into the recent advancements in understanding these covert attack methodologies, and we categorize these attacks according to their targeted security enforcement.

\subsubsection{Learning-based Detection System}
In recent years, the deployment of machine learning (ML) and deep learning (DL) techniques for intrusion detection has seen a significant rise, with systems being trained on substantial traffic data to distinguish between normal and malicious flows and filter out the latter in real-time.
However, as these detection systems become more sophisticated, so as the methods employed by attackers to circumvent them.

Fogla et al.~\cite{fogla2006evading} address how rule-based intrusion detection systems, where rules are usually inferred by learning techniques (e.g., decision tree), can be evaded through polymorphic blending attacks (PBAs).
These attacks cleverly disguise malicious packets to appear statistically similar to legitimate traffic, thus evading detection.
The core challenge lies in mutating malicious packets such that they conform to the regular grammar of the detection system (e.g., requirement on the payload size).
The authors demonstrate the computational complexity in finding an optimal PBA, highlighting its NP-complete nature.
To tackle this, they recommend the use of satisfiability (SAT) or integer linear programming (ILP) solvers to discover near-optimal PBAs.

Yan et al.~\cite{yan2023automatic} examine the vulnerabilities in ML-based intrusion detection systems, particularly their susceptibility to evasion attacks using adversarial examples.
In a simulated attack scenario, the adversary has limited feedback, e.g., only the binary result of detection success or failure.
By interacting with the system, the adversary discerns patterns of benign and malicious DDoS traffic.
With this knowledge, substitute models are trained using ensemble learning to approximate the decision boundaries of the target system.
This equips the adversary with a quasi-white-box view, facilitating the creation of adversarial traffic samples that are statistically representative of network traffic while still harboring malicious payloads.
These samples are then used to successfully bypass the actual IDS, exposing significant vulnerabilities in these systems.

Abusnaina et al.\cite{abusnaina2019examining} delve into adversarial learning attacks against deep-learning-based detection systems.
They observe that while general adversarial attacks (e.g., those documented by Papernot et al.\cite{papernot2016limitations} and Moosavi-Dezfooli et al.\cite{moosavi2016deepfool}) can induce misclassification in standard tasks like image recognition, they fail to generate adversarial network flows that require to maintain the characteristics of legitimate traffic.
To overcome this, they introduce a flow-merge technique, which merges attributes of benign flows with a mask flow using operations such as accumulation or averaging, thereby crafting adversarial flows that evade detection.
Similarly, Hashemi et al.\cite{hashemi2019towards} pinpoint manipulations, such as the splitting and injecting of packet payloads, that can modify network features perceived by a detection system without violating network protocol requirements.

Mustapha et al.~\cite{mustapha2023detecting} propose the use of Generative Adversarial Networks (GANs) for creating adversarial flows.
They utilize a Wasserstein GAN (WGAN), which includes a generator that creates malicious flow samples from random noise, aiming to mirror the distribution of benign traffic data.
Meanwhile, the discriminator, acting as a surrogate for the target detection system, aims to differentiate between genuine and synthetic samples.
The closed feedback loop between the generator and discriminator ensures a continuous refinement of the adversarial samples. 
The process iterates until the discriminator's accuracy plummets, at which point the generated flows can effectively evade the target system.

Finally, Matta et al.~\cite{matta2017ddos} conceptualize randomized DDoS attacks, a particularly stealthy form of DDoS.
The process begins with bots monitoring online activity to capture normal traffic patterns, which are then used to compile an emulation dictionary.
Bots mimic legitimate traffic by selecting packets from this dictionary at random.
To balance the trade-off between message innovation (uniqueness) and independence (redundancy), the bots employ randomization in their message selection and transmission rates.
This method generates malicious flows with high innovation rates, effectively fooling detection systems.

\subsubsection{Commercial DDoS protection}
DoS attacks continue to evolve, necessitating sophisticated defensive measures from Internet Service Providers (ISPs) and cloud service providers like CloudFlare, who offer traffic scrubbing services to their customers.
These services typically employ strategies such as IP hiding and address validation to distinguish and filter out malicious traffic.
Despite these measures, research indicates that attackers can still circumvent these defenses.

Jin et al.~\cite{jin2018your} examined the dependence of Denial of Service Protection Services (DPS) on concealing the server's true IP address and the effectiveness of traffic scrubbing techniques.
They found that DPS works by providing a false DNS record to mask the server's actual IP address.
Consequently, both legitimate and malicious traffic are directed to a scrubbing center, where the latter is intended to be filtered out.
However, this defense has vulnerabilities, especially when changes are made to the DPS configuration.
If a user discontinues their DPS service or switches providers, the original DPS may retain records of the server's actual IP address.
Attackers can exploit this by querying the name servers of the former DNS provider, thus unmasking the target server's IP and allowing them to bypass the DPS altogether.

Nosyk et al. \cite{nosyk2023closed} introduced a scanning algorithm designed to detect networks that forego Source Address Validation (SAV), which is a critical defense against amplification attacks.
SAV works by rejecting packets with spoofed source IP addresses at the network's edge.
The study demonstrated that attackers could easily discern whether a network has implemented SAV.
By sending spoofed DNS requests to each host within a target network and observing responses from an attacker-controlled authoritative name server, it can be determined whether SAV is absent.
If the network lacks SAV, it becomes a potential target for amplification attacks.
This scanning method revealed that a significant portion of networks, 49\% of IPv4 and 26\% of IPv6 Autonomous Systems (AS), do not implement SAV, leaving millions of DNS resolvers vulnerable to these attacks.
Similarly, the Spoofer project, maintained by CAIDA~\cite{spoofer}, measures a network's vulnerability to spoofing by sending packets with forged source addresses to a measurement server.
The success or failure of these transmissions reveals whether the network can be exploited for spoofing.
Lone et al.~\cite{lone2017using} propose another active inference technique using routing loops identified in traceroute data.
An attacker dispatches a traceroute packet with a fabricated source IP address to a customer network within an Internet Service Provider (ISP).
If the ISP fails to filter out the spoofed packet, it forwards the packet to the customer.
The absence of accurate routing for the fake address causes the packet to oscillate between the customer and ISP, indicating that the ISP is susceptible to source IP spoofing.

Further research highlighted by Wu et al.~\cite{wu2014software} and Shankesi et al.~\cite{shankesi2010resource} has shown that client puzzle schemes, designed to mitigate DDoS attacks, are not foolproof against determined adversaries.
Attackers can leverage computational resources such as GPUs or integrated CPU-GPU systems to solve puzzles more rapidly.
If the puzzle is parallelizable, an attacker might distribute the task across hundreds of GPU cores, significantly decreasing the time required to solve it.
Alternatively, if the puzzle function is non-parallelizable, the attacker might inundate the server with requests, assigning each GPU core to solve different puzzles independently.
This technique effectively reduces the time needed to solve these challenges, thereby increasing the potency of the attack.

\subsubsection{Protocol Security Enforcement}
Communication protocols are typically engineered with inherent security mechanisms, enabling the parties involved in the communication to validate the messages they receive.
Nonetheless, contemporary research has uncovered that these security provisions are not infallible; indeed, attackers have identified and exploited design flaws within these mechanisms to circumvent security checks..

Cao et al.~\cite{cao2016off} identified a critical vulnerability in TCP, wherein a mechanism designed to protect against DoS attacks inadvertently introduces a new attack surface.
The introduction of a global rate limit, as specified in RFC 5961, was meant to mitigate DDoS attacks.
However, it inadvertently introduced a new attack vector.
By sending spoofed packets, attackers can manipulate the global rate limit counter, a shared resource, and monitor its effects to deduce the existence of a TCP connection and its sequence number.
Consequently, this information enables them to disrupt the connection by transmitting a malicious RST packet with the correct sequence number, impersonating the victim.

Feng et al.~\cite{feng2020off} discovered a similar exploit within the Linux kernel's mixed IPID assignment method, which was originally implemented to counter TCP hijacking.
Attackers can observe changes in the IPID counter, induced by spoofed packets, to infer details about active TCP connections and hijack them using malicious RST packets.
Wang et al.~\cite{wang2024off} extended this attack vector to WiFi networks by demonstrating that the size of encrypted frames can be observed and used to infer TCP connection details, including sequence and acknowledgment numbers.

Further examining the ICMP protocol, Feng et al.~\cite{feng2022off} revealed a disconnection between the legitimacy check mechanism for ICMP redirect messages and a suite of stateless protocols such as UDP, ICMP, GRE, IPIP, and SIT.
This gap allows off-path attackers to craft evasive ICMP error messages that bypass the legitimacy checks, leading to the revival of ICMP redirect attacks.
These attackers can orchestrate stealthy DoS attacks, tricking public servers into redirecting their traffic into black holes with just one forged ICMP redirect message.

Feng et al.~\cite{feng2022pmtud} also investigated the interaction between IP fragmentation and TCP, challenging the assumption that IP is protected from fragmentation attacks by the default implementation of Path Maximum Transmission Unit Discovery (PMTUD).
They found that ICMP error messages could desynchronize the path MTU values between the IP and TCP layers.
This desynchronization can result in IP fragmentation, even when PMTUD is used, allowing an off-path attacker to trigger fragmentation and inject malicious packet fragments, causing legitimate packets to be lost.

\subsection{Summary}
We have summarized the attacks discussed previously in Table~\ref{tab:survey-attack-summary}.
To categorize these attacks, we examine them from six distinct perspectives, including the protocols and systems they target.
Among these perspectives, we emphasize two specific characteristics of malicious traffic: traffic type and traffic pattern.
Regarding traffic types, we distinguish attacks as either direct or indirect. Direct attacks involve sending malicious traffic straight to the victim, whereas indirect attacks route the malicious traffic through intermediaries, such as DNS resolvers.
In terms of traffic patterns, we categorize malicious traffic based on its volume and timing.
In terms of volume, attacks can be volumetric (i.e., large quantities of data), or low-rate (i.e., smaller amounts of data).
In terms of timing, attacks are classified by their duration and frequency, such as continuous flooding or periodic pulsing.
\begin{table*}
\centering
\caption{Summary of surveyed DDoS attacks}
\scalebox{0.75}{
\begin{tblr}{
  hlines,
  vlines,
}
\textbf{Research} & \textbf{Category} & \textbf{Target} & \textbf{Exploited Features} & \textbf{Traffic Type} & \textbf{Traffic Pattern} & \textbf{Impact} \\
       \begin{tabular}[c]{@{}l@{}}\cite{luo2014mathematical,tang2013modeling,schuchard2010losing},\\\cite{wang2002detecting}\end{tabular}        &              Transport-Layer Protocol                    &         TCP        &                  \begin{tabular}[c]{@{}l@{}}Session management,\\congestion control\end{tabular}                  &           Direct               &                 \begin{tabular}[c]{@{}l@{}}Volumetric flooding,\\Low-rate pulsing\end{tabular}                   &       Service down for TCP servers          \\
       \cite{nawrocki2021quicsand}        &              Transport-Layer Protocol                    &         QUIC        &                  \begin{tabular}[c]{@{}l@{}}Session management,\\flexible ID setting\end{tabular}                  &           (In)Direct              &                 Volumetric flooding                   &       Service down for QUIC servers          \\
       \cite{gilad2011fragmentation,atlasis2012attacking}        &              Network-Layer Protocol                    &         IP        &                  \begin{tabular}[c]{@{}l@{}}Header fragmentation\end{tabular}                  &           Direct              &                 Low-Rate flooding                   &       Service down for target servers          \\
       \cite{dantas2014selective}        &             Application-Layer Protocol                     &        HTTP         &              Request fragmentation                      &            Direct              &                 Low-rate pulsing                  &        Server socket exhaustion         \\
        \cite{beckett2017http,praseed2019multiplexed}       &                 Application-Layer Protocol                 &         HTTP/2        &          Multiplexing                          &      Direct                    &         Volumetric     flooding                      &            Server CPU exhaustion                \\
        \cite{sisalem2006denial,tang2014sip}       &                 Application-Layer Protocol                 &         SIP        &            \begin{tabular}[c]{@{}l@{}}Session management,\\flexible ID setting\end{tabular}                        &         Direct                 &         Volumetric    flooding                       &           \begin{tabular}[c]{@{}l@{}}Proxy socket exhaustion,\\stop benign VoIP calls\end{tabular}                \\
        \begin{tabular}[c]{@{}l@{}}\cite{yin2023waterpurifier,kim2017preventing,yazdani2022mirrors},\\\cite{griffioen2021scan}\end{tabular}       &                 Application-Layer Protocol                 &         DNS        &              \begin{tabular}[c]{@{}l@{}}Flexible ID setting,\\recursive resolution\end{tabular}                      &           (In)Direct               &              Low-rate flooding                    &        \begin{tabular}[c]{@{}l@{}}Service down for target servers,\\service down for DNS servers\end{tabular}               \\
        \cite{mohammed2023detection}       &                 IoT Protocol                 &         Modbus        &          Memory allocation                          &         Direct                 &               Low-rate flooding                     &           Service down for control units                \\
        \cite{wang2022zigbee}       &                 IoT Protocol                 &         Zigbee        &          Network rejoin procedure                          &         Direct                 &               Low-rate flooding                     &           Devices unable to join the network                \\
        \cite{studer2009coremelt,kang2013crossfire}       &                 Networking Infrastructure                 &         Routing system        &          Traffic concentration                          &         Direct                 &               Volumetric flooding                     &           Cut off connections to a region                \\
        \cite{shin2013avant,shin2013attacking,cao2019crosspath}       &                 Networking Infrastructure                 &         SDN system        &            \begin{tabular}[c]{@{}l@{}}Control-data plane separation,\\shared path for data and control\end{tabular}                        &         Direct                 &              Volumetric flooding                      &         Service down for switches/controllers                  \\
        \cite{gasti2013and,mannes2019naming}       &                 Networking Infrastructure                 &         NDN system       &              Content caching                      &           Direct               &              Volumetric flooding                    &        Service down for NDN routers               \\
        \begin{tabular}[c]{@{}l@{}}\cite{henrydoss2014critical,silva2020repel,olimid20205g},\\ \cite{sattar2019towards,javadpour2023reinforcement,mirsky2020ddos} \end{tabular}       &                 Networking Infrastructure                 &         Cellular network        &         \begin{tabular}[c]{@{}l@{}}LTE control-data separation,\\5G network slicing,\\unidentified emergency call\end{tabular}                           &           (In)Direct               &             Volumetric flooding                       &        \begin{tabular}[c]{@{}l@{}}Service down for stations,\\user equipment, and 911 center\end{tabular}                   \\
        \cite{tushir2020quantitative,vukovic2014security}       &                 Distributed System                 &         IoT system       &           \begin{tabular}[c]{@{}l@{}}Power support,\\device interaction\end{tabular}                         &   Direct                       &           \begin{tabular}[c]{@{}l@{}}Volumetric flooding\end{tabular}                         &       
        \begin{tabular}[c]{@{}l@{}}Service down for home devices\\and smart grid nodes\end{tabular}                    \\
        \begin{tabular}[c]{@{}l@{}}\cite{vasek2014empirical,johnson2014game,wu2020survive},\\\cite{li2021deter,heilman2015eclipse,tran2020stealthier},\\\cite{gervais2015tampering,walck2019tendrilstaller,apostolaki2017hijacking}\end{tabular}       &                 Distributed System                 &         Blockchain  system      &      \begin{tabular}[c]{@{}l@{}}Mining pool,\\transaction handling,\\peer selection\end{tabular}                           &            Direct              &       Volumetric flooding                             &        \begin{tabular}[c]{@{}l@{}}Service down for blockchain node\end{tabular}                   \\
        \cite{wanglordma}       &                 Computing Infrastructure                 &         RDMA  system      &            Congestion control                        &        (In)Direct                  &            Volumetric flooding                        &       \begin{tabular}[c]{@{}l@{}}Service down for direct and\\indirect RDMA API calls\end{tabular}                    \\

        \cite{xiong2021warmonger}       &                 Computing Infrastructure                 &         Serverless platform        &      Shared egress IP                           &            Indirect              &       Volumetric flooding                             &        Deployed web service down                  \\
        \begin{tabular}[c]{@{}l@{}}\cite{fogla2006evading,yan2023automatic,abusnaina2019examining},\\\cite{papernot2016limitations,moosavi2016deepfool,hashemi2019towards},\\\cite{mustapha2023detecting,matta2017ddos}\end{tabular}       &                 Adversarial DoS                 &         Learning-based IDS        &      Adversarial learning                           &            Direct              &       Volumetric flooding                             &        Bypass IDS detection                \\
        \begin{tabular}[c]{@{}l@{}}\cite{jin2018your,nosyk2023closed,wu2014software},\\\cite{shankesi2010resource}\end{tabular}       &                 Adversarial DDoS                 &         \begin{tabular}[c]{@{}l@{}}Commercial DoS\\protection\end{tabular}        &      \begin{tabular}[c]{@{}l@{}}IP hiding,\\address validation,\\client puzzle\end{tabular}                           &            Direct              &       Low-rate pulsing                             &        \begin{tabular}[c]{@{}l@{}}Bypass commercial DoS protection\end{tabular}                  \\
        \begin{tabular}[c]{@{}l@{}}\cite{cao2016off,feng2020off,wang2024off},\\\cite{feng2022off,feng2022pmtud}\end{tabular}       &                 Adversarial DDoS                 &         \begin{tabular}[c]{@{}l@{}}Protocol security\\enforcement\end{tabular}        &      \begin{tabular}[c]{@{}l@{}}TCP global rate limit,\\ICMP redirect,\\fragmentation protection\end{tabular}                           &            Direct              &       Low-rate pulsing                             &        \begin{tabular}[c]{@{}l@{}}Leverage security enforcement to\\construct DoS attacks\end{tabular}                  \\
\end{tblr}
}
\label{tab:survey-attack-summary}
\end{table*}

Our literature review reveals a concerning evolution: DoS attacks are becoming increasingly diverse, extending their reach to a broader spectrum of network protocols and capitalizing on weaknesses within newly established systems.
These advanced attacks are characterized by their nimbleness and intricacy, often bypassing traditional security measures with alarming facility.
In light of this context, it is essential to thoroughly examine the attributes and evolving patterns of these DoS attacks, enabling security professionals to more effectively identify new attack vectors as new protocols and systems are designed.
In this section, we dissect the current trends in DoS attacks.
In Section~\ref{sec:open-problem}, we offer strategic insights for conducting comprehensive investigations into the attack surfaces of new protocols and systems.

\textit{\textbf{DDoS attacks are exploiting advanced protocol features to increase the attack efficiency, stealthiness, and severity.}}
Traditional DDoS attacks often exploit familiar aspects of network protocols including congestion control, identity spoofing, and packet fragmentation.
However, as network protocols have evolved, they have developed sophisticated features that now present new opportunities for attackers.
Features such as HTTP/2 multiplexing, DNS recursive resolution, Modbus flexible header, and Zigbee network rejoin are now being leveraged to construct more potent and covert DDoS attacks.

Specifically, attackers are drawn to these advanced features for several reasons.
(1) \textit{Traffic amplification}.
Modern protocol features can significantly amplify traffic. This means attackers can use fewer resources to launch larger attacks.
For example, HTTP/2 multiplexing enables attackers to achieve up to 95 times the attack bandwidth compared to HTTP/1.0 traffic, under the same packet transmission rate.
(2) \textit{Increased stealthiness}.
These features often allow attackers to indirectly route malicious traffic towards the target, enhancing the stealthiness of the attack.
A notable tactic involves exploiting DNS recursive resolution, where attackers distribute malicious DNS requests across multiple resolvers.
These resolvers then unwittingly forward the flood of requests to the target authoritative server, complicating the tracing process and obscuring the origins of the attack.
(3) \textit{Resource diversity exploitation}.
By interacting with various system properties of the target host, these protocol features expand the attack surface.
This diversification enables attackers to manipulate different system properties and cause more severe disruptions.
For instance, attackers can use specially crafted Modbus packets to target a controller unit’s memory, or issue malicious Zigbee rejoin requests to overload the routing device's child table.

The foregoing discussion makes it clear that the evolution of network protocols, while enhancing efficiency and introducing new functionalities to modern computing environments, also creates new vulnerabilities in the realm of DDoS attacks.
As these protocols become more complex, they not only broaden the attack surface but also introduce subtle vulnerabilities that can be exploited in unexpected and innovative ways.

\textit{\textbf{DDoS attacks are increasingly targeting advanced systems and exploiting emerging vulnerabilities.}}
The landscape of DDoS attacks is undergoing a transformative shift, mirroring the rapid evolution of technological systems.
The focus of these attacks has broadened, moving beyond traditional web servers to encompass an array of sophisticated and emerging systems, such as SDN, cellular network, IoT, and blockchain systems.
Attackers are not only exploiting known issues including uncontrolled resource consumption (CWE-400) and Network amplification (CWE-406), but are also identifying and leveraging new vulnerabilities in these emerging systems.
In particular, our survey highlights three particularly concerning vulnerabilities.

(1) \textit{Separation of data and control planes}.
One significant change in network management is the separation of data and control planes, a technique used in popular systems such as SDN and cellular networks.
This architecture allows network administrators to manage, configure, and optimize network behavior from a centralized location.
However, this centralization also creates a critical vulnerability.
An attacker can target this central point by flooding the control-plane communication channel with fake control messages, disrupting the normal operations of the control plane.
Additionally, because the centralized controller relies on physical resources, e.g., CPU and memory, to manage the network, an attacker could deploy malicious bots to generate a massive number of suspicious network flows.
This tactic strains the controller's resources, making it difficult to maintain effective network monitoring and management.

(2) \textit{Vulnerable resource sharing mechanism}.
Resource sharing is a common trail for efficient resource utilization.
For example, functions running on a serverless platform may share the same egress IP addresses.
As another example, network slicing in 5G allows multiple virtual networks to operate on the same physical network infrastructure, optimizing CPU, memory, and bandwidth usage.
However, these shared environments can significantly increase the risk of widespread disruptions.
By targeting a single vulnerable service (e.g., a vulnerable serverless function), the attacker can impact the underlying infrastructure and paralyze all services co-located on the same infrastructure.
Even worse, if attackers have access to the infrastructure, they can initiate cyber attacks in representative of all co-located services, triggering others to blacklist them.
For instance, the attacker can rent a serverless platform and deploy malicious serverless functions to initiate cyber attacks on legitimate parties (e.g., DNS resolvers).
Consequently, all legitimate services sharing the same egress IP would also suffer from denial of service, even though they are not directly involved in the attack.

(3) \textit{Exploitable component interaction}.
Modern systems often feature complex inter-dependencies among their components.
For example, in a blockchain network, each node communicates with its nearest neighbors to stay updated about the entire network.
Similarly, in smart grid systems, nodes share their state estimates with each other, and each node bases its control decisions on the data received from its neighbors.
However, these intricate interactions can also introduce vulnerabilities.
Attackers can exploit these relationships to launch DDoS attacks.
For instance, an attacker might manipulate the peer selection process in Bitcoin to isolate a target node from the rest of the network.
Additionally, attackers can disseminate false information, such as incorrect state measurements, to disrupt the decision-making processes of benign nodes.
This can effectively paralyze their operational logic, leading to broader system disruptions.

The discussion underscores a critical development in the realm of cyber threats, highlighting how complexity in advanced systems can be both a driver of innovation and a magnet for vulnerabilities.
This dual nature presents significant challenges as we strive to advance technologically while securing the systems against increasingly sophisticated threats.

\textit{\textbf{The scale of traffic associated with DDoS attacks has significantly diminished: Even a single message can cause denial of service.}}
The evolution of DDoS attacks has seen a marked transition from the traditional volumetric flooding techniques to more insidious low-rate strategies.
These sophisticated attacks exploit specific design or implementation weaknesses inherent in communication protocols and systems, negating the need for attackers to generate large volumes of traffic.
For example, the TCP congestion control mechanism, IP fragmentation process, and DNS amplification can all be manipulated due to their protocol features.
One particularly concerning exploitation method involves the erroneous handling of DNS error messages~\cite{pan2024loopy}.
Attackers can leverage this flaw to send a single message that generates persistent, malicious traffic, thereby efficiently draining the targeted server's resources.
Similar exploitative tactics can be observed in systems with advanced features, such as Named Data Networking (NDN) content caching, interactions within smart grid centers, and the peer selection process in blockchain networks.
These examples underscore the adaptability of attackers in using complex system functionalities to their advantage.

\textit{\textbf{Adversarial DDoS tactics are constantly evolving, targeting various types of detection systems and exhibiting diverse levels of attack costs.}}
Adversarial DDoS tactics employ sophisticated and aggressive methods designed to disrupt the normal operations of targeted services and evade detection systems.
Our research identifies three types of adversarial attacks: Adversarial machine learning, bypassing commercial DDoS protection, and exploiting protocol security enforcement.
To assess these tactics, we here provide a detailed overview of their workflows, and analyze their advantages and disadvantages, focusing on factors such as attack cost and stealthiness.

(1) \textit{Adversarial machine learning}.
This tactic targets machine learning-based detection systems and typically unfolds in two phases:  Mimicking the decision boundary of the target detection system, and generating malicious samples that fall within these boundaries.
To understand the behavior of the target detection system, the adversary generates a large number of test samples and collects feedback (labels) from the system.
Then this feedback is used to refine the sample generator.
This iterative process continues over multiple rounds until the majority of the generated samples can bypass the detection system, effectively inferring its decision boundary.
Once the boundary is understood, the refined generator can then produce malicious samples tailored for subsequent attacks.


This tactic is applicable to a broad range of learning-based detection systems.
Moreover, it does not require extensive prior knowledge of the target system, making it a flexible approach for attackers.
However, the major drawback of this method is its high resource consumption.
The attacker must generate and test a potentially vast number of samples to accurately infer the decision boundary of the detection system.
Specifically, popular techniques like Generative Adversarial Networks (GANs) may require an extensive number of iterations to converge, and it is challenging to train the generator.
Moreover, the frequent submission of test samples can trigger system alerts, making it relatively easy for the detection system to recognize an ongoing attack.
This can lead to countermeasures such as blocking the source of the traffic (e.g., IP blocking), further complicating the attacker's efforts.

(2) \textit{Bypassing commercial DDoS protection}.
Commercial DDoS protection services employ several strategies to safeguard against denial of service attacks, e.g., IP hiding and source address validation.
However, our survey reveals that attackers have developed multiple techniques to bypass these defenses, exploiting inherent vulnerabilities or oversights of commercial protection systems.
By identifying and exploiting weaknesses (e.g., non-implemented SAV and retained DNS records) attackers can tailor their strategies to specific network vulnerabilities.
The cost of an attack and its stealthiness vary significantly based on the targeted vulnerabilities and the techniques employed.
For example, the process of scanning for SAV implementation is costly and lacks stealth due to the detectability of large-scale network scans.
Conversely, exploiting DNS records to uncover hidden IP addresses is generally low-cost and can be highly stealthy.
In this scenario, attackers can utilize publicly available datasets collected from various vantage points, making the method less conspicuous and more accessible.

(3) \textit{Exploiting protocol security enforcement}.
Recent research has revealed a variety of techniques that attackers use to exploit design flaws in protocol security mechanisms for conducting denial-of-service attacks.
These attackers cleverly use security mechanisms (e.g., the TCP global rate limit and the IPID counter) as unintended side channels.
This manipulation allows them to extract critical information about network communications (e.g., TCP session number) or to alter these communications detrimentally.
One of the primary challenges in addressing these attacks is their detection difficulty, as they utilize legitimate functions of communication protocols.
Once attackers gather critical data, such as TCP session numbers, they can initiate highly targeted attacks with minimal traffic (e.g., a single TCP RST packet).
This subtlety means that traditional DDoS metrics, which often focus on large volumes of traffic, fail to identify these attacks.
However, these techniques demand a sophisticated understanding of protocol dynamics and behaviors, and their effectiveness often hinges on the specific implementations of the protocol stack.
Furthermore, because protocol updates and patches can address these vulnerabilities, the lifespan of such attack methods may be limited.
Once a patch is applied, the methods that previously exploited these flaws can become obsolete.
