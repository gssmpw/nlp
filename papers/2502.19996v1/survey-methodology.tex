\section{Survey Methodology}
We start by introducing how we collect the papers from the literature and filter out most relevant papers. 
We aim to collect well-researched papers that span the last decade and are from the literature of DDoS attack and detection.
Specifically, we first leverage advanced searches to collect a number of papers from the conferences and transactions that are sponsored by IEEE, USENIX, ACM, and Elsevier.
We search papers with keywords "DDoS" and "distributed denial of service".
Moreover, we restrict the type to be the research article.
In this context, we acquire 3,348, 5,408, 666, and 31 papers from IEEE Xplore, ACM library, Elsevier ScienceDirect, and USENIX, respectively.
Then we filter out papers based on the ranking of their publication venue, retaining only papers from highly-ranked conferences and transactions to ensure quality.
In particular, we selects top-tier venues from Google Scholar Metrics, Conference Ranks, Core Conference Rankings, and China Computer Federation.
Specifically, we focused on subcategory of system, network, and security on these ranking sources, and pulled the top-20 ranking lists.
As a result, we totally selected 87 venues from these ranking sources, and examples of the selected venues are CCS, S\&P, USENIX Security, and TDSC.

Finally, considering that some papers do not focus on DDoS attack/detection but just occasionally mention the word "DDoS" somewhere in the paper, we extend the keyword dictionary by including "attack" and "detection".
For each paper, we further calculate a relevance degree by counting the frequency of keywords in the extended keyword dictionary, and sort these papers in descending order of their relevance degrees.
As a result, papers with low degrees or even zero degrees (e.g., false positives that are wrongly returned by the sponsor's search engine) are discarded.
Each preserved paper describes a concrete attack or detection technique about distributed denial of service.
Eventually, we select 184 papers for deep examination. 
