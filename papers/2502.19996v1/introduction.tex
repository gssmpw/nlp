\section{Introduction}\label{sec:introduction}
% Computer Society journal (but not conference!) papers do something unusual
% with the very first section heading (almost always called "Introduction").
% They place it ABOVE the main text! IEEEtran.cls does not automatically do
% this for you, but you can achieve this effect with the provided
% \IEEEraisesectionheading{} command. Note the need to keep any \label that
% is to refer to the section immediately after \section in the above as
% \IEEEraisesectionheading puts \section within a raised box.

Distributed Denial of Service (DDoS) attacks have persistently been one of the most prevalent threats to the stability and availability of online services and infrastructures.
According to CloudFlare's report~\cite{cloudflare-ddos-report}, a 117\% year-over-year increase in DDoS attacks has been observed.
Famous DDoS attacks like Mirai~\cite{antonakakis2017understanding} and Github paralyze~\cite{chadd2018ddos} demonstrates the severity of these attacks, where significant traffic volumes flood the critical network infrastructures or services.
The proliferation of Internet of Things (IoT) devices, combined with the increased availability of DDoS-as-a-Service platforms, further lowered the barrier to launching sophisticated attacks that can overwhelm even the most robust defensive mechanisms.

The landscape of DDoS attacks has not only expanded in scale but also advanced in complexity, demonstrating a two-pronged evolution.
The first aspect of this evolution is the diversification of targets and techniques.
Attackers are no longer limited to traditional transport-layer protocols. 
They are increasingly exploiting a variety of application-layer protocols for their attacks.
This trend also extends to the targets themselves, with sophisticated systems such as blockchain technologies and cellular network infrastructures coming under siege.
The second facet of this evolution pertains to the subtlety of the attacks.
There is a noticeable increase in the stealthiness with which these actions are executed.
Adversaries are crafting more sophisticated strategies designed to circumvent not only the current commercial detection systems but also the most advanced detection algorithms.
This advancement is indicative of a continuous arms race between attackers seeking invisibility and defenders aiming to maintain visibility and control.

Simultaneously, the advancement of sophisticated DDoS detection methodologies is advancing at a rapid pace.
These approaches meticulously model behavioral patterns and traffic characteristics to delineate the boundary between legitimate users and malevolent attackers.
Furthermore, the emergence of state-of-the-art network hardware, such as programmable switches, signifies a burgeoning domain ripe with opportunities to fortify network resilience.
Initiatives to amalgamate these cutting-edge technologies into comprehensive detection frameworks are currently in progress, marking a significant stride toward more robust defense mechanisms.

The preceding discussion underscores the necessity for a systematic literature review to meticulously examine the progression of DDoS attack trends.
Additionally, a critical evaluation of nascent detection technologies is crucial to yield insights that are fundamental to the architecture of a contemporary detection system.
%, one that is adept at counteracting the swift evolution of complex attack vectors.
Regrettably, existing surveys~\cite{agrawal2019defense,kumari2023comprehensive,chaudhary2023ddos,yan2015software,li2023comprehensive,zhang2024revealing} typically focus on specific scenarios (e.g., cloud computing~\cite{agrawal2019defense} and IoT~\cite{kumari2023comprehensive}).
This narrow focus often results in a lack of a comprehensive perspective that encompasses the full spectrum of DDoS attack characteristics and trends.
Consequently, these studies frequently overlook crucial aspects necessary for the development of modern DDoS detection systems, such as attack-agnostic detection capabilities and cross-domain data sharing.
Additionally, the potential benefits of emerging advanced network hardware in enhancing DDoS detection are rarely discussed.
As a result, we identify three critical questions that need to be explored to advance the understanding of DDoS attacks and their detection.
\begin{enumerate}
    \item What are the prevailing trends in emerging DDoS attacks, and what insights can be gleaned to inform the vulnerability analysis of nascent network protocols and systems (Section~\ref{sec:ddos-attack})?
    \item Given the diversity and increasing stealthiness of DDoS attacks, what guiding principles should inform the construction of modern detection systems (Section~\ref{ddos-detection})?
    \item With the rollout of advanced network hardware, such as programmable switches, which feature can be leveraged to augment DDoS detection (Section~\ref{sec:ddos-detection-system-deployment})?
\end{enumerate}
% (1) What are the prevailing trends in emerging DDoS attacks, and what insights can be gleaned to inform the vulnerability analysis of nascent network protocols and systems?
% (2) Given the diversity and increasing stealthiness of DDoS attacks, what guiding principles should inform the construction of modern detection systems?
% (3) With the rollout of advanced network hardware, such as programmable switches, which feature can be leveraged to augment DDoS detection, particularly in an environment of escalating traffic volumes and attackers engaging in sophisticated tactics to modify their attack patterns dynamically?

This survey is committed to tackling the questions previously delineated.
To answer the first question, we carefully unravel the progression of DDoS threats, providing actionable insights for identifying vulnerabilities within the DDoS landscape.
Instead of targeting single scenario, we studied emerging DDoS vulnerabilities lurking in nine popular communication protocols and eight advanced systems, revealing novel protocol features (e.g., DNS recursive resolution) and system weakness (e.g., vulnerable resource sharing mechanism) for attackers to orchestrate DDoS attacks.
Moreover, our study reveals three emerging types of adversarial DDoS tactics, which can efficiently bypass commercial and state-of-the-art DDoS detection techniques.
Our investigation shines a light on the frailties that adversaries target in emerging protocols and systems, while also assisting in the forecast of impending attack methodologies.

To answer the second and the third question, we present a detailed analysis of current detection systems to bridge the gap between present-day challenges and the novel solutions taking shape in the field of DDoS protection.
We also examine contemporary research that leverages emerging network primitives (SDN and programmable switches) to enhance DDoS detection, and summarize their benefits.
Finally, we highlight the unresolved challenges in the analysis of DDoS vulnerabilities and the development of contemporary detection systems.
We also spotlight promising methodologies and suggest future avenues for research to tackle these unresolved issues.

The roadmap of the survey is shown in Figure~\ref{fig:survey-taxonomy}.
\begin{figure*}
    \centering
    \includegraphics[scale=.3]{figures/survey-taxnomy.pdf}
    \caption{Survey roadmap}
    \label{fig:survey-taxonomy}
\end{figure*}
In Section 2, we present an overview of DDoS attacks, highlighting significant historical cases, and delving into the existing research in the field.
Section 3 outlines the methodology behind our survey.
Section 4 examines the sophisticated evolution of botnet recruitment and coordination techniques, as well as the progression of DDoS attacks that exploit new network protocols, systems, and incorporate adversarial tactics.
Advancing into Section 5, we explore a variety of proposed methods for DDoS attack detection, organizing them by their heuristic approaches and techniques employed.
The conversation further evolves in Section 6, where we assess the innovative deployment methods for DDoS defense systems, made possible by cutting-edge network primitives such as programmable switches.
Lastly, Section 7 contemplates the future of DDoS attack vulnerability analysis and modern detection systems, providing a roadmap for ongoing research in the domain.