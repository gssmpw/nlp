\section{DDoS Detection}
In this section, we report existing works which focus on DDoS detection.

\subsection{Flooding DDoS}
\textbf{Statistical-based approach.}
One of the effective approaches for detecting DDoS attacks is the statistical-based detection methodology.
These methods leverages statistical analysis to identify abnormal traffic patterns that could indicate an ongoing attack.
Popular statistical metrics are entropy~\cite{kalkan2018jess}.
Moreover, to ensure efficient statistics computation, estimation methods are also proposed including sketching.

\textbf{Some works highlight that the similarity of flooding packets is much higher than the similarity among legitimate packets, and propose to use the entropy to measure the randomness of packet attributes}. 
Specifically, Kalkan et al.~\cite{kalkan2018jess} propose JESS, an entropy-based detection method by applying joint entropy.
The joint entropy takes flow attributes (e.g., destination IP address) and transport-layer attributes (e.g., TCP flag) into consideration, such that different sets of attributes can be used to detect different types of flooding attacks.
Given a flow, the joint entropy of the attribute sets is calculated and compared with a threshold.
A low value of joint entropy indicates that the flow is likely to be a participant of a flooding attack. 

\textbf{Some works aim to improve the efficiency of flow monitoring by using sketching techniques.}
Sketching is one of the most popular monitoring technique~\cite{zhang2020poseidon,xu2022towards,liu2021jaqen}.
The main idea is to use resource-efficient data structures (called sketches) to approximate the needed statistics with well-known error bounds, and then set feature thresholds for detecting anomalous flows.

Specifically, Poseidon~\cite{zhang2020poseidon} applies Count-Min sketch to count flow sizes.
When a flow comes, the sketch uses a set of hash functions to compute several indexes for this flow, and then increments the corresponding flow counters in register arrays.
To retrieve the flow size of a flow, the counter values in register arrays are fetched, and the minimum value from these counters is taken as the estimate.
As a result, users can set thresholds to detect malicious flows with anomalous flow sizes.
Considering that Poseidon can only measure the flow size, Jaqen~\cite{liu2021jaqen} proposes universal sketches.
It uses Count Sketch to estimate a range of network statistics (e.g., source IP and source ports).
By doing so, users can set thresholds for specific network features and achieve fine-grained detection.
Finally, Xu et al.~\cite{xu2022towards} propose LiEffi-FM Sketch to achieve persistent and lightweight detection of DDoS attacks in Named Data Networking.
The observation is that the malicious bots usually incur a dramatic increase of the number of Interest packets that possess the same name prefix but request distinct data contents.
To continuously monitor the Interest packets at the NDN router with low resource consumption, the researcher proposes LiEffi-FM Sketch.
The sketch probabilistically counts the number of distinct data requests with a common name prefix.
Then the researcher applies Monte Carlo hypothesis testing to determine a counting threshold for classifying benign and malicious requests.

\textbf{Behavioral pattern.}
Scherrer et al.~\cite{scherrer2023albus} points out that existing threshold-based detection methods (e.g., network sketching) have difficulties determining the detection threshold when the malicious flows from individual bots are medium-rate and bursty.
To address the bursty flows generated by malicious bots, the researcher proposed ALBUS.
The key insight is that malicious flows have a pattern of consistently high data transfer rates over a short period, while normal flows may experience occasional but not persistent bursts.
Specifically, ALBUS uses the Leaky Bucket (LB) algorithm as an analogy to monitor the flow of data packets.
The LB algorithm is good for spotting when a flow exceeds a certain rate (i.e., flow burst).
Considering that the number of flows can be significant and monitoring all flows will incur significant memory consumption, ALBUS applies flow sampling: At any time, ALBUS only monitors a subset of flows.
To determine the flow to be monitored, ALBUS maps flows to checkpoints using a hashing function.
If a flow persistently show bursty patterns, the checkpoint will keep monitoring the flow, which increases the likelihood of the flow abnormality.
Otherwise, the flow will be removed from the checkpoint and regarded as a benign flow.

Tandon et al.~\cite{tandon2021defending} designs a system FRADE which leverages several heuristic rules to detect application-layer DDoS bots.
Specifically, the researcher detects bots by differentiating the behavior pattern between the bots and normal users.
The key insight is that while the generated requests from the bots and users may be identical, the dynamics (i.e., frequency of page visits) and semantics of web browsing (i.e., sequences of page visits) differs.
As a result, the author analyzes the web server access logs, estimating the rate of the client's interaction with the server and the transition probabilities between each pairs of pages.
If the rate is above the threshold or the transition probability is low, the client is suspicious and likely to be a bot.
Moreover, FRADE follows the idea of honeypots and deploy special objects (e.g., hyperlinks around pieces of text) which should only be found and accessible by bots.
If these objects are clicked and visited by some clients, these clients are likely to be bots, and their further requests will be blocked.

\subsection{Link Flooding DDoS}

Link flooding attacks (LFAs) as a new kind of DDoS attack target routing systems (Section~\ref{subsec:ddos-targeting-system}) and gain increasing attention to security analysts.
The bots send packets to publicly accessible decoy servers to flood indirectly a seemingly non-targeted node, while the under-the-radar” flows then cumulatively flood certain target links and affect the real target.
Considering its severity and stealthiness, various detection methods are proposed to handle LFAs.

% \textbf{Network measurement.}

\textbf{Statistical-based approach.}
Considering the difficulty of target link identification, LinkScope~\cite{xue2014towards} uses end-to-end and hop-by-hop network measurements to detect the links under attack.
Specifically, LinkScope first identifies potential target links which lead to a large number of downstream servers, and actively probes the involved path to gather end-to-end and hop-by-hop metrics (e.g., packet loss rate and round-trip time).
Then the cumulative sum (CUSUM) algorithm is employed to capture the abrupt changes of these metrics for the detection of congested paths.
Finally, LinkScope correlates end-to-end and hop-by-hop measurements to localize the target link.

RADAR~\cite{zheng2018realtime} performs correlation analysis on the flow information to detect the Crossfire attack.
It first collects traffic statistics in the SDN data plane and detects the attack by several heuristic rules, e.g., certain paths get congested regularly.
To filter malicious flows, RADAR adaptively analyze traffic in the SDN control plane to check if the statistics of certain flows match the pattern of attack (e.g., synchronized flow bursts and congestion events).
Ripple~\cite{xing2021ripple} observes the limitation of SDN (rely on a feedback loop between the switches and a central controller), and leverages the programming switch to achieve in-network detection.
It utilizes a set of distributed switches to maintain a "defense panorama", which is a synchronized and network-wide view of attack signals relevant to link-flooding defense.
With user-specified policies, Ripple periodically checks congested links larger than the congestion threshold and identifies suspicious hosts based on the flow statistics (e.g., the number of low-rate flows for each source and
destination IP address pair).
Zhou et al.~\cite{zhou2023mew} introduce Mew, a detection system designed for resource efficiency and runtime adaptability to counteract the shortcomings of Ripple, particularly its lack of scalability and difficulty in addressing dynamic Link-Flooding Attacks (LFAs) where attackers frequently change target links. Mew employs a lightweight protocol that distributes storage responsibilities across multiple switches, effectively balancing memory usage and enabling a more scalable approach. Additionally, Mew introduces a suite of APIs that support multi-granularity cooperation, allowing switches to collaboratively monitor flow statistics and respond to threats in a unified manner.

\textbf{Behavioral pattern.}
Liaskos et al.~\cite{liaskos2018network} relies on traffic engineering (re-routing) with reinforcement learning approaches to detect malicious flows participating in the LFA.
The key insight is that whenever the network topology changes, the bots will change their decoy servers and target new critical links to persistently affect the target.
As a result, network operators can change the network topology periodically and monitor the flows that often contribute to the congestion of critical links.
To do so, the researcher proposes to use reinforcement learning principle: A flow's probability of being malicious is reinforced each time it reappears in a congestion event.
Similarly, Gkounis et al.~\cite{gkounis2016interplay,liaskos2016novel} applies traffic rerouting in a manner that forces malevolent flows to constantly re-home to new destinations, accelerating their detection.
Kang et al.~\cite{kang2016spiffy} presents a re-routing scheme that forces malevolent flows to increase their traffic volume, leading to their exposure.
rezapour et al.~\cite{rezapour2021rl} base on traffic engineering and apply two reinforcement learning-based routing algorithms, the Efficient Routing Algorithm (ERA) and Detective Routing Algorithm (DRA), to expose LFA bots.
If a source IP is found to behave suspiciously several times, it is marked as a malicious flow and blocked out.

Kang et al.~\cite{kang2016spiffy} considers a cost-sensitive attacker who adopts an optimal and fixed strategy to send traffic. By intentionally increasing the bandwidth of a targeted bottleneck link and monitoring the subsequent traffic response, SPIFFY aims to identify malicious flows. In contrast to attack traffic, legitimate sources are expected to adjust their throughput in response to the newly available bandwidth. Attack flows, likely already operating at full capacity, will not significantly alter their rate, which allows the defense mechanism to detect them.

Ma et al.~\cite{ma2019randomized} propose a game-theoretic detection approach to identify Link Flooding Attacks (LFAs) within resource constraints and against adaptive adversaries. Their method models the confrontation as a Stackelberg security game, where the defender employs a randomized mixed-detection strategy to maximize detection effectiveness. This strategy is designed to vary the monitored links unpredictably, making it difficult for a well-informed adversary to avoid detection without revealing patterns indicative of malicious activity. By incorporating models of both rational and boundedly rational adversary behavior, the defender's strategy dynamically adapts to the adversary's possible actions, aiming to identify traffic anomalies that suggest an LFA is underway.


\subsection{Reflection DDoS}
\textbf{Source address validation.}
Considering that one important feature of reflection DDoS is IP spoofing, some works focus on the active detection of network spoofability and traffics with spoofed source addresses.
Specifically, the Spoofer project maintained by CAIDA~\cite{spoofer} measures spoofability within networks by actively sending packets with forged source addresses to a measurement server.
Based on the transmission result (success or failure), Spoofer can detect the spoofability within the network. 
%Their data reveals that spoofing was possible in more than 34\% of the 2.5K tested networks.
%Kührer et al. detect spoofed traffic utilizing DNS resolvers~\cite{kuhrer2014exit}.
%and found that more than 2.7K ASes do not perform egress filtering~\cite{kuhrer2014exit}.
Lone et al. present a method using routing loops appearing in traceroute data~\cite{lone2017using}.
Specifically, The attacker sends a traceroute packet with a random, fake source IP address (not belonging to the attacker) to a customer network within an ISP.
If the ISP allows spoofed packets, it does not discard the request but forward the request to the customer.
However, since the source address is randomly generated, neither the customer nor ISP has the valid routing information.
As a result, the request is transmitted back and forth between the customer and the ISP, and the presence of this loop indicates that the ISP allows source IP spoofing.

Some work exists that identifies spoofed traffic passively.
Dainotti et al. infers the legitimate address space within autonomous systems (AS)~\cite{dainotti2013estimating}.
They reconstruct bidirectional flows from NetFlow records and filter out flows with a large number of packets.
These flows are considered as non-spoofed, and their source IP addresses are used to infer legitimate IP address space and filter spoofed addresses. 
\cite{lichtblau2017detection} analyzes BGP routing data to construct the relationship among autonomous systems (AS), and infer the address space associated with each AS.
By doing so, legitimate source-destination IP records are identified.
Given a captured traffic, its source and destination addresses are extracted and compared with the legitimate record.
If the traffic's address information does not match any legitimate record, IP spoofing is detected and administrators (e.g., ISPs) can filter out these traffics.

\textbf{Request-response verification.}
Some works aim to couple queries and responses that belong to the same DNS transaction~\cite{di2011protecting,dai2024dampadf}.
As a result, all the responses that don’t have a coupled query are part of the reflection attack.
Specifically, Di Paola et al.~\cite{di2011protecting} uses the Bloom filter, a compact data structure for membership queries, to hash and store the request information (source IP, destination IP, and transaction IP).
The benefit of using Bloom filter is that it supports fast request lookup and stores memory spaces when the number of queries is large.
Once a response comes, its information will be extracted to form the key and check the existence of corresponding requests.
If no request is found, the response is detected as unsolicited responses.
Dai et al.~\cite{dai2024dampadf} further show that Bloom filter can incur high false positives when the number of queries increases, and propose a enhanced framework called DAmpADF.
Specifically, DAmpADF uses two Bloom filters to increase the storage space and alternately store DNS requests.
Moreover, DAmpADF identifies popular DNS servers which are frequently queried by users by applying a probabilistic method called exponential-weakening decay.
By doing so, requests to (as well as responses from) these servers can be directly passed without recording in the bloom filter.

\textbf{Honey-pot based detection.}
To effectively identify reflection-based Distributed Denial of Service (DDoS) attacks in the wild, security professionals often deploy specialized systems known as honeypots. These honeypots are designed to simulate vulnerable application protocols, e.g., DNS, which are commonly exploited for amplification in DDoS attacks.
Attackers inadvertently use amplification honeypots as reflectors, allowing operators to monitor and measure attacks. Honeypots differentiate attack traffic from benign activity by grouping packets into "flows" identified by common characteristics, such as source/destination addresses and ports~\cite{nawrocki2023sok}.

The Cambridge Cybercrime Center (CCC)~\cite{thomas20171000} runs a distributed honeypot network across 10 countries to monitor malicious internet traffic. CCC's detection method flags a flow as an attack if it targets a sensor with at least five packets within 900 seconds. A flow is identified by matching the source address, destination port, and the sensor's address. This system allows CCC to differentiate between regular traffic and potential attacks, enhancing their ability to analyze and understand cyber threats.
AmpPot~\cite{kramer2015amppot} is a global sensor network designed to detect amplification DDoS attacks by setting a threshold of 100 packets per flow within 3600 or 600 seconds. It operates in three modes and assigns flow IDs based on source address and destination port. AmpPotMod adjusts the idle timeout for closer attack analysis, but the effects of this change are not fully detailed. The system's approach is to minimize false positives by applying a conservative threshold based on typical scanner traffic behaviors.
The Honeypot Platform for Intrusion (HPI)~\cite{griffioen2021scan} is a global network of 549 honeypots that detects amplification DDoS attacks by monitoring six protocols across four modes. It uses a dual-sensor detection method with a 20-packet threshold and a one-minute idle timeout to identify attacks. The platform's approach is guided by research indicating a typical upper limit of 20 packets per source IP in such attacks.

The above honeypot platform widely uses threshold-based approaches to detect malicious flows. However, Wagner~\cite{wagner2021united} reveal that many amplification DDoS attacks go undetected by local mitigation platforms because the attacks do not exceed local traffic thresholds or have a multi-protocol profile that remains unseen at a single location.
As a result, the researchers suggest that leveraging multiple vantage points could lead to better attack detection.
They collaborate with 11 IXPs across three different regions, and develop a lightweight and easy-to-implement collaborative DDoS Information Exchange Point (DXP) that allows mitigation platforms to share information about ongoing amplification attacks.
The collaboration approach can detect 90\% more attack traffic locally when the DXP is operational.

Instead of grouping flows and counting the number of malicious requests, the BGPEEK-A-BOO method~\cite{krupp2021bgpeek}, as proposed by Krupp et al., leverages the constraint that attackers can spoof IP header information but are still dependent on their service provider's BGP routes. Through a technique called BGP Poisoning, the researchers manipulate BGP routes to isolate specific Autonomous Systems. The source of spoofed traffic can then be detected by observing changes such as TTL fluctuations or the cessation of attack traffic in response to these routing changes. To facilitate this process, amplification honeypots are deployed to attract and monitor attack traffic. By systematically excluding ASes and analyzing the changes in traffic, the origins of spoofed flows can be traced back to the responsible AS. Additionally, a BGP flow graph is constructed to model route propagation, significantly reducing the search space and expediting the traceback process. The effectiveness of this method is validated through simulation and real-world experiments, demonstrating its potential to identify the sources of malicious traffic without prior knowledge of the attacker or cooperation from external entities.

In summary, honeypot-based methods allow for accurate differentiation between normal traffic and potential attacks by setting specific thresholds for packet count and time duration.
They deploy sensors internationally (e.g., CCC across 10 countries, AmpPot and HPI globally), providing a wide coverage area for monitoring malicious internet traffic and the ability to detect attacks from various origins.
However, the effectiveness of these systems is highly dependent on their configuration. If attackers adapt their tactics to evade the predefined thresholds and detection criteria, the honeypots might not recognize the altered attack patterns.

\subsection{Low-Rate DDoS}
Rather than flooding the network with excessive traffic, low-rate DDoS attacks (e.g., Slowloris) generate low rates of network flows (e.g., application requests) that exploit some vulnerability and tie up a scarce key resource.
With the advantage of low attack cost, these attacks are on the rise, and it is necessary to defense them.
In this section, we discuss related works which focus on low-rate DDoS detection.

\textbf{Behavioral pattern.}
Tandon et al.~\cite{tandon2023leader} propose Leader, an attack-agnostic defense against low rates of malicious Web application requests.
Instead of mining patterns of specific low-rate attacks, Leader uses OS-level tracing to monitor fine-grained resource usage (CPU cycles, memory, processing function call sequence, etc) of requests from each connection.
With the resource usage pattern as the feature, Leader uses one-class SVM and elliptic envelope methods to build a model for legitimate traffic.
During the detection phase, Leader initiates anomaly detection by checking each incoming request.
Requests which significantly deviate from the inferred model will be regarded as anomalies (i.e., malicious requests).
Alcoz et al.~\cite{alcoz2022aggregate} propose ACC-Turbo as an enhanced detection mechanism for pulse-wave DDoS attacks, such as Shrew and RoQ attacks. ACC-Turbo builds upon the traditional Aggregate-based Congestion Control (ACC) mechanism, which targets high-bandwidth traffic aggregates that can overwhelm network links. Recognizing the limitations of ACC's offline inference, which struggles with rapid, pulsing traffic patterns and dynamic congestion, ACC-Turbo introduces online clustering to identify and manage suspicious traffic directly at line rate, enabling a more responsive defense against these types of attacks.

\subsection{Learning-based traffic analysis.}

\textbf{Machine learning}.
Some works utilize clustering approach to detect malicious flows~\cite{ahmed2018statistical,qin2015ddos}.
Specifically, Ahmed et al.~\cite{ahmed2018statistical} suggest using the clustering technique to build fingerprints of different web application.
They gather packet-level features (e.g., IP address) and stream-level features (e.g., number of bytes from client to server) to define application fingerprints.
Then the fingerprints are clustered using lightweight clustering algorithms (e.g., Mean-shift), and the output clusters are mapped to their respective applications (e.g., HTTP or SMTP) with assistance from the labeled training data.
Given a flow of client requests, its fingerprint (e.g., periodic burst or periodically small fragmented packets) is examined to determine the matched application.
If there is no matched application, the flow is regarded as suspicious.
Similarly, Qin et al.~\cite{qin2015ddos} use the entropy of flow features (e.g., packet size and flow duration) as the features and K-means as the clustering algorithm to cluster benign flows and model normal patterns of request behavior.
Online flows whose entropy vectors are far from the benign clusters are detected as malicious flows.
Bhatia et al.~\cite{bhatia2021mstream} clusters flows by identifying large volumes of suspiciously similar activities based on both categorical and numerical attributes.

Some works utilize classification approaches to detect malicious flows.
For example, MM et al.~\cite{mm2022efficient} applies Kernel Principal Component Analysis to optimize and select relevant flow features, and then train a Support Vector Machine-based Discrete Elephant Herding Optimization (SVM-DEHO) classifier to classify flow samples.
Panigrahi et al.~\cite{panigrahi2022intrusion} utilize Multi-Objective Evolutionary Feature Selection to select the most informative flow features, and then combines the Decision Table and Naive Bayes techniques to initiate traffic classification.


\textbf{Deep learning}.
The Kitsune framework~\cite{mirsky2018kitsune} utilizes an ensemble of autoencoders that operates online and unsupervised to distinguish between normal and anomalous traffic patterns. 
Network traffic instances are inputted to the autoencoder ensemble, where each autoencoder is tasked with reconstructing a subset of the traffic's features, subsequently computing the reconstruction error using the root mean squared error (RMSE) metric.
Aggregated RMSEs from all autoencoders are then evaluated by an output module, which compares the combined error against a pre-established threshold or decision boundary to determine if the traffic is benign or potentially malicious.
Similarly, Aktar et al.~\cite{aktar2023towards} propose a deep learning-based model using a contractive autoencoder to detect DDoS anomalies.
Diallo et al.~\cite{diallo2021adaptive} designs ACID~\cite{diallo2021adaptive} which uses a multi-kernel-based neural network for detection.
De et al.~\cite{de2021detection} strategically select three key traffic features—number of packets, entropy, and average inter-arrival time—to train a Multi-Layer Perceptron (MLP) neural network. When combined with Fuzzy Logic, this MLP network achieves high detection accuracy, particularly for identifying RoQ DDoS attacks.
Wang et al.~\cite{wang2020dynamic} further propose an MLP-based method for DDoS detection that employs sequential feature selection techniques to systematically identify optimal features during training, addressing the issue of feature redundancy and irrelevance. Moreover, they introduce a dynamic feedback mechanism to adapt the detector to evolving traffic patterns, aiming to enhance detection accuracy and maintain robustness over time.

Agiollo et al.~\cite{agiollo2023gnn4ifa} address the detection of Interest Flooding Attacks (IFAs) in Named Data Networking (NDN) by representing the network as a graph with routers as nodes. They aggregate router features to create node state vectors. Their methodology includes both Supervised Attack Detection (SAD), where a trained Graph Neural Network (GNN) classifies the state of the network, and Unsupervised Attack Detection (UAD), which employs self-supervision to enable the GNN to reconstruct masked graph segments and detect anomalies by evaluating the reconstruction error against a defined threshold.
To collect data for training detection models, Wichtlhuber et al.~\cite{wichtlhuber2022ixp} advocate for the collection of ISP blackholed traffic, typically comprising unwanted data, to amass malicious samples.

Aydin et al.~\cite{aydin2022long} developed a Long Short-Term Memory (LSTM)-based detection system, termed LSTM-CLOUD, designed to analyze and characterize network traffic in public cloud environments. The system leverages historical network traffic data to identify and potential DDoS attacks. By utilizing features extracted from past traffic patterns, the LSTM-CLOUD system aims to provide a robust defense mechanism against cyber-attacks targeting cloud computing services.
Xu et al.~\cite{xu2022xatu} observe that attackers often reuse tactics and exhibit predictable behavior before launching DDoS attacks. Utilizing auxiliary signals from prior incidents, such as traffic from spoofed or blocklisted sources, the team adopts a LSTM neural network. This network can learn from both long-term and short-term patterns in the data. The LSTM's multi-timescale design enables analysis over various time frames, enhancing detection even with weak or sporadic signals. Xatu employs survival analysis to predict attack onset, optimizing a loss function that seeks to detect attacks early while minimizing false positives and the cost of unnecessary traffic scrubbing. The proposed system demonstrates an ability to detect attacks significantly earlier than traditional methods, as shown through evaluations using real-world network data.

Duan et al.~\cite{duan2022application} recognize that current deep learning (DL) methods for DDoS detection fail to effectively capture the evolving interactions and topological information between IP pairs, which are crucial for anomaly detection. To address this, they propose a novel approach that represents network traffic as dynamic spatiotemporal graphs, thereby capturing both spatial and temporal characteristics of network communications. By employing a Dynamic Line Graph Neural Network (DLGNN), their system can extract spatial information from network interactions and track the temporal evolution of communication between IP pairs. When analyzing a specific flow, the model assesses the features in the context of the graph's structure and applies learned patterns to discern between benign and malicious activities.

Motivated by the challenge of false positives in learning-based flooding detection systems, Fu et al.~\cite{fu2023point} design pVoxel, which identifies false positives in the results of existing detection systems (e.g., Kitsune).
The key observation is that benign traffic features associated with false positives tend to be sparsely distributed in the feature space due to diverse benign user behaviors, while malicious traffic features are densely distributed due to the similarity of flows generated by attack tools.
As a result, pVoxel normalizes the feature vectors of the reported malicious flows, and construct high-dimensional cubes to group these feature vectors.
Isolated vectors with low density indicate that they are false positives of the detection result, and removal of them improves the detection accuracy. 
Zhao et al.~\cite{zhao2023ernn} find that existing learning-based detection systems often fail to account for network-induced phenomena (e.g., packet loss, retransmission, and out-of-order packets), which can lead to significant false positives.
To improve the detection accuracy, the researchers design ERNN, which trains a RNN model for detection.
The RNN model is trained with normal traffic and noisy traffic, which simulates four types of common network-induced phenomena.
A Mealy machine is applied to dynamically adjust the probability distribution of network-induced phenomena during the training process.
As a result, ERNN is robust against network-induced phenomena and improves the detection precision.

\textbf{Hybrid approach.}
Dong et al.~\cite{dong2023horuseye} propose a two-stage framework, HorusEye, which combines machine learning and deep learning to initiate detection.
HorusEye first applies an isolation forest model to filter out suspicious traffic at high throughput.
For traffic flagged as suspicious by the data plane, a deeper analysis is performed using an asymmetric autoencoder (AAE).
The encoder requires deeper layers for better representation extraction, while the decoder can reconstruct features without overly complex operations.
As a result, the Root Mean Squared Error Loss (RMSE) is calculated and used to determine the abnormality of the suspicious flow.
Long et al.~\cite{long2022hybrid} use a stacked sparse autoencoder coupled with a support vector machine (SSAE-SVM) for traffic detection.
The autoencoder aims to learn a good representation of the data in an unsupervised manner, while the SVM utilizes these learned features to accurately classify the traffic.
Mahadik et al.~\cite{mahadik2023edge} design a hybrid CNN-LSTM model, which combines the auto-feature extraction capability of CNN with the long-term sequence memory of LSTM, for the detection and classification of binary and multiclass DDoS attacks.


\subsection{Adversarial DoS}
Traditional detection methods falter against encrypted malicious traffic as attackers increasingly adopt encryption.
Fu et al.~\cite{fu2023detecting} address this by identifying that interaction patterns among multiple attack flows are distinct from legitimate traffic. HyperVision, their proposed system, constructs an interaction graph from network flows, aggregating short flows to lessen graph density. It segments the graph into connected components and clusters them based on high-level statistics, such as the count and size of flows. Components deviating significantly from cluster centers are flagged as abnormal. Within these, HyperVision clusters edges to pinpoint malicious flows. This approach enables the detection of encrypted malicious activities.
Cui et al.~\cite{cui2023cbseq} observe that while malware constantly evolves, the core intentions like launching DDoS attacks remain detectable through consistent network behaviors. Their proposed CBSeq method ignores encrypted content and instead analyzes traffic behavior—capturing features such as duration and flow count—to cluster similar traffic patterns. It then characterizes these patterns into behavior sequences indicative of malicious intent. By leveraging a Transformer-based model, MSFormer, CBSeq learns the nuances of these sequences to effectively distinguish between benign and malware traffic.

\textbf{Some works defend against adversarial attacks by adaptive training the detection model with adversarial samples.}
Mustapha et al.~\cite{mustapha2023detecting} propose a Long Short-Term Memory (LSTM) detection method, and show its inefficiency against various types of adversarial DDoS attacks generated by GAN.
Then the researchers enhance the LSTM-based detection scheme by using adversarial samples generated by a GAN framework to train the model, improving its ability to recognize such attacks.


Fu et al.~\cite{fu2021realtime} address the challenge that existing machine-learning-based detection methods for network traffic can be evaded by sophisticated attacks, particularly randomized DDoS attacks that blend benign packets with malicious traffic to bypass detection systems  (as discussed in Section~\ref{subsec:adversarial-ddos-bypassing-detection}).  They observe that frequency domain features of network traffic are more robust against such evasion tactics and can effectively represent the sequential information of traffic patterns with limited information loss. This leads to improved accuracy and throughput in detecting malicious activities.
To exploit these observations, the researchers designed Whisper, a system that applies Discrete Fourier Transformation (DFT) to convert traffic features from the time domain to the frequency domain, which reflects the repetition frequency of traffic patterns. The frequency-domain features are then used to train a classifier using clustering algorithms, enabling it to distinguish between benign and malicious run-time traffic effectively.
Similarly, Fouladi et al.~\cite{fouladi2022novel} demonstrate that the characteristics of both unique source IP addresses (USIP) and the normalized number of unique destination IP addresses (NUDIP) relative to the total number of packets undergo changes in both the time-domain and frequency-domain during DDoS attacks. They apply continuous wavelet transform (CWT) to convert the USIP and NUDIP statistics into two-dimensional features that encapsulate time and frequency information. These features are subsequently input into a Convolutional Neural Network (CNN) classifier, which is trained to distinguish between normal and malicious traffic.
Doriguzzi et al.~\cite{doriguzzi2020lucid} further perform kernel activation analysis to understand which features the CNN model deems significant when classifying traffic as DDoS, aiding in the interpretability and accuracy of the model's decisions.

Matta and Cirillo et al.~\cite{matta2017ddos,cirillo2021botnet} focus on the detection of randomized DDoS attacks.
The key insight is that the users belonging to a botnet are expected to exhibit a smaller degree of message innovation than normal users, which act by their own nature independently one each other.
To measure the innovation degree, Matta et al.~\cite{matta2017ddos} proposes a metric called Message Innovation Rate (MIR), which measures the the number of distinct messages per unit time transmitted by a given group of users.
The researcher also proposes a detection algorithm called BotBuster: By continuously monitoring and clustering clients with low MIR, the resulting clusters are reported as the botnets.
Cirillo et al.~\cite{cirillo2021botnet} further extend the setting to multiple emulation dictionaries: Instead of all bots using the same emulation dictionaries, bots form different groups, each of which uses different emulation dictionaries.
Moreover, the researcher proves that BotBuster is also applicable for handling the multi-group case.

Feng et al.~\cite{feng2020application} tackle randomized DDoS attacks by proposing a system that uses a Markov decision process to evaluate the context-dependent legitimacy of traffic, factoring in both environmental conditions (e.g., server resource usage) and client-server interaction history (e.g., client historical request and reputation). The system employs a reinforcement learning agent that assesses server metrics and user behavior to distinguish between legitimate and malicious requests. This agent adapts its response strategy—blocking or allowing traffic—based on real-time feedback, optimizing for minimal disruption to genuine users during normal conditions while effectively countering attacks when threats escalate.

Wang et al.~\cite{wang2023bars} observe that detection systems based on deep learning are vulnerable to evasion attacks due to their heterogeneous feature sets, diverse model designs, and operations in adversarial environments. Consequently, malicious traffic might evade detection. To address this, the researchers introduce a robustness certification framework named BARS (Boundary-Adaptive Randomized Smoothing), which aims to certify the robustness of these detection systems. BARS incorporates a distribution transformer that applies tailored noise distributions to different features, reflecting the understanding that various features may have different susceptibilities to adversarial attacks. This approach allows for varying robustness requirements across features. BARS then uses this noise-enhanced approach to generate adversarial examples that are utilized to certify and potentially improve the robustness of the detection system against evasion attacks.
Catillo et al.~\cite{catillo2023case} investigate the robustness of ML-based (e.g., autoencoder) and ML-based (e.g.,decision tree) intrusion detection systems against adversarial DDoS traffic generated by the virtual adversarial method.
The study, using the CICIDS2017 dataset, reveals that autoencoder-based systems demonstrate greater resilience to adversarial samples compared to decision trees, which exhibit significant vulnerability.

Yang et al.~\cite{yang2021cade} address concept drift in DDoS detection models that occurs when attackers alter their behavior, causing the testing data distribution to deviate from the training data, leading to detection failures. To combat this, they propose CADE, which refines the training process by mapping high-dimensional traffic features to a lower-dimensional latent space for clustering similar flows. CADE then employs contrastive learning to enhance the separation between these clusters. This method allows for the categorization of malicious samples into fine-grained sub-classes, unveiling diverse attack strategies and improving the model's training robustness against evolving threats.

\subsection{IoT Botnet}

\textbf{Behavioral pattern.}
The large and growing number of IoT devices, coupled with multiple security vulnerabilities, brings an increasing concern for launching DDoS.
As a result, instead of pinpointing malicious traffics and flows as shown in previous sections, fruitful research works focus on the detection of (infected) IoT devices and malicious device behaviors.

\textbf{Some works focuses on the identification of malware downloading to detect bots}~\cite{invernizzi2014nazca,kwon2015dropper}.
For example,  Nazca~\cite{invernizzi2014nazca} analyzed the download relationship graph to identify malware downloading.
The observation is that while individual malware downloads may not be conspicuous, when viewed collectively—as part of a larger malware distribution infrastructure—they exhibit distinct patterns that differentiate them from legitimate downloads.
As a result, Nazca first monitors web traffic, identifying HTTP requests and extracting metadata such as connection endpoints, URIs, and whether an executable program is being downloaded.
The metadata is then evaluated to identify suspicious web connections that download executable files and exhibit properties atypical of legitimate downloads (e.g., application of evasive techniques on executable and interaction with suspicious servers).
Nazca further aggregates suspicious connections to find related malicious activities, thus reducing potential false positives.
Kwon et al.~\cite{kwon2015dropper} examines the behavior of more complicated malware whose execution leads to additional malware download, e.g., trojans or droppers.
To detect these malware and affected hosts, the researchers propose the downloader graph which represents the download activity on end hosts.
These graphs are constructed by tracking executable files and their download relationships.
By utilizing patterns and properties that distinguish benign from malicious graphs (e.g., executable growth rate and diameters), the researchers develop a machine learning classifier based on a random forest algorithm to classify benign and malicious malware.

\textbf{Some works exploit the behavioral pattern of (infected) devices to initiate detection}~\cite{antonakakis2017understanding,herwig2019measurement,guo2020detecting,eshete2017dynaminer}.
Manos et al.~\cite{antonakakis2017understanding} detect existence of Mirai-infected IoT devices by watching for hosts doing Mirai-style scanning.
Herwig et al.~\cite{herwig2019measurement} focus on the malware Hajime and detect Hajime-infected devices by measuring the public distributed hash table (DHT), which are used by Hajime bots for C\&C communication.
Similarly, Tegeler et al.~\cite{tegeler2012botfinder} reveal that botnets belonging to a particular family exhibit similar patterns in their command-and-control (C\&C) communications (e.g.,  specific data upload formats and certain timing patterns for repeated connections to the C\&C infrastructure).
Leveraging these regularities in C\&C traffic, the researchers propose BOTFINDER, which utilizes five features extracted from bot flows (e.g., average duration of a connection) and initiates clustering to model the bot family.
The incoming flows are clustered to the bot family, and the clustering result determines the infection state of the host and the malware type.

Note that the above detection methods can only reveal existence of bots which are infected by specific malware (e.g., Mirai and Hajime).
Instead, Guo et al.~\cite{guo2020detecting} propose two detection algorithms for detecting IoT devices in the Internet.
These algorithms require the knowledge of what servers these devices talk to, which are usually run by IoT manufacturers.
With the knowledge, the first algorithm examines the destination IP address in the client-generated traffic and DNS queries sent by the client.
If the IP address or the queried domain name belongs to an IoT manufacturer, the client is classified as an IoT device.
The second algorithm detects IoT devices using HTTPS by active scanning for the TLS certificates used by the client.
If the certificate contains IoT manufacturer names, the client should be an IoT device.
With the detection method, the researcher identifies 254K IoT devices (mainly IP cameras) from 199 countries around the world.
DYNAMINER~\cite{eshete2017dynaminer} find that malware infections exhibit distinct behavioral dynamics before and after infection.
Therefore, DYNAMINER abstracts HTTP transactions into Web Conversation Graphs (WCG), which represents communication patterns and interactions between browsers, websites, and servers.
Since the temporal dynamics captured by WCG reflects the behavior shift of the infected hosts, DYNAMINER uses a set of graph properties (e.g., node degree) to train an ensemble random forest and classify benign and affected hosts.

6thSense~\cite{sikder20176thsense} show that benign user activities on IoT devices usually activate a specific set of sensors.
By understanding the context of sensor usage and learning the normal sensor data patterns associated with various user activities, it is possible to identify and differentiate between benign and malicious sensor activity.
Specifically, the researchers use machine learning models (e.g., Naive Bayes) to correlate sensor data with the activities.
With the trained classifier, 6thSense achieves context-aware intrusion detection for IoT devices.
Similarly, AEGIS~\cite{sikder2019aegis} focus on profiling the context in which user activities and sensor-device interactions occur.
It integrates phone app context (e.g., user clicking buttons to operate devices) to understand the relationship between smart home entities, and use Markov Chain-based machine learning technique to detect abnormal device behaviors.


\textbf{Some works identify (infected) IoT devices by their network fingerprint}~\cite{shodan,torabi2018inferring,durumeric2015search,mirian2016internet}.
Specifically, Shodan is a search engine that provides information about Internet-connected devices on public IP~\cite{shodan}.
It actively crawls all IPv4 addresses on a small set of ports to detect devices by matching texts (e.g., “IP camera”) with service banners and other device-specific information.
The identified devices are further analyzed by CAIDA to infer compromised IoT devices which send packets to allocated but unused IPs~\cite{torabi2018inferring}.
Censys is comparable to Shodan in its functionality~\cite{durumeric2015search}.
However, Censys offers an additional feature where community members can contribute rules that help identify the manufacturer and model of devices connected to the internet.
This is done by correlating specific text patterns found within device banners.
Finally, Mirian et al.~\cite{mirian2016internet} infer industrial control system (ICS) devices by scanning the IPv4 space with ICS-specific protocols and watching for positive responses.
The result reveals more than 60K publicly accessible systems that are potentially exploitable to launch DDoS attacks.

\textbf{Some works identify (infected) IoT devices by side channels}~\cite{khan2019idea}.
Khan et al.~\cite{khan2019idea} show that the electromagnetic (EM) signals have deviations when the IoT device is executing benign applications and DDoS attacks.
As a result, they propose IDEA, which uses electromagnetic as a side-channel signal to detect DDoS activities on IoT devices (i.e., bots).
IDEA first records EM emanations from an un-compromised device to establish a baseline of reference EM patterns.
Then it monitors the target device’s EM emanations.
When the observed EM emanations deviate from the reference patterns (in terms of signal reconstruction error), IDEA reports this as an anomalous or malicious activity.

Acar et al~\cite{acar2020peek} show that despite encryption, one can capture IoT protocol traffic (e.g., WiFi, Zigbee, and Bluetooth) and analyze communication metadata (e.g., packet lengths and traffic rates) to infer IoT device information and their activities.
For instance, a smart camera might send larger packets due to video data, while a temperature sensor transmits smaller packets.
With the sniffed encrypted wireless traffic and extracted metadata, various machine learning algorithms (e.g., kNN) are applied to train classifiers and classify device types, device states, and user behaviors.
The classification result can be used to check if devices behave abnormally (e.g., abused to send flooding traffic).

\textbf{Learning-based approach.}
Some works also use machine-learning and deep-learning based traffic analysis to detect IoT devices~\cite{meidan2017profiliot,le2019unearthing}.
Specifically, Meidan et al.~\cite{meidan2017profiliot} detect IoT devices from LAN-side measurement by identifying their traffic flow statistics with random forests and GBM.
They use a wide range of features (over 300) extracted from network, transport and application layers, such as number of bytes and number of HTTP GET requests.
Work from IBM~\cite{le2019unearthing} transforms DNS names into embeddings (the numeric representations that capture the semantics of DNS names), and classify devices as either IoT or non-IoT based on embeddings of their DNS queries using deep learning model (multi-layer perceptron).



\textbf{Superspreader detection.}
Superspreaders are hosts with a large number of distinct connections.
In the context of DDoS, they usually refer to the infected hosts that connect to many other hosts
for DDoS malware propagation.
%, or the servers overwhelmed by a botnet of zombie hosts under DDoS attacks.
As a result, the detection of superspreaders has significant amplifications for pinpointing DDoS attacks~\cite{plonka2000flowscan,kamiyama2007simple,guan2009new,liu2015detection,tang2022high}.

A naive method for finding superspreaders is to track all distinct destinations that each host contacts using a hash table, e.g., Flowscan~\cite{plonka2000flowscan}.
However, maintaining per-flow state requires large quantities of memory for operation, which is unfeasible for monitoring on high speed links.
To reduce the number of hosts to be monitored, Kamiyama et al.~\cite{kamiyama2007simple} proposed hash-based flow sampling to identify suspicious hosts.
The proposed method first samples packets based on hash of flow key.
Then it applies Bloom filter to check if sampled packet is new flow.
If a new flow is detected, the counter is incremented for host in host table, and the host with a counter larger than the threshold will be regarded as a superspreader.
While the sampling method can keep up with link speed in high-speed networks, it will lead to awkward performance in accuracy.

To enable efficient storage of flow information, some works utilize the sketching approach.
Specifically, Guan et al.~\cite{guan2009new} propose reversible sketches estimate the in/out degrees of the hased hosts.
However, its hash functions involve complex arithmetic operators, and they incur high computation cost when hashing IP addresses.
To reduce the complexity, Liu et al.~\cite{liu2015detection} design a new sketch called Vector Bloom Filter (VBF), which does not explicitly maintain any host identifier but can reconstruct the IDs of superpoints and estimate their cardinalities.
Recently, Tang et al.~\cite{tang2022high} designs a practical sketch data structure SpreadSketch for real-time superspreader detection under huge volumes of network traffic.
SpreadSketch maps each connection to a binary hash string that estimates the source's fan-out.
Moreover, multiple SpreadSketch instances can be merged to provide a network-wide view for recovering all superspreaders.

\subsection{Summary}
This section introduces the state-of-the-art DDoS detection methods.
The summary of these detection methods go here.
\begin{itemize}
    \item \textit{No uniform detection pattern.} Since the attack pattern varies, the detection pattern also varies.
    For example, the detection of LFA may exploit some network rerouting to make the bot expose themselves.
    \item \textit{From attack-specific detection to attack-agnostic detection.}
    \item \textit{Emerging hardware to assist in detection.}
    \item \textit{Reducing false positives.} See works~\cite{fu2023point}.
    \item \textit{Adversarial and robustness.} (1) Many adversarial attacks, but not enough robust methods to handle. (2) Even there are robust detection methods, usually focus on single type of DoS attacks, and lack of proof/evaluation to show their generality.
    Detection from traditional middleware to SDN to P4.
\end{itemize}