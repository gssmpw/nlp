\section{Background and Related Work}

\subsection{DDoS Overview}
DDoS attacks represent a formidable threat in the cyber landscape, designed to overwhelm and incapacitate an online service, rendering it inaccessible to its intended users~\cite{mirkovic2004taxonomy}.
These attacks are achieved by bombarding the target network or system with malicious traffic that far exceeds its processing capacity or triggers specific protocol/system vulnerabilities, leading to a breakdown in its services and functionalities.
This disruption can have significant repercussions, ranging from financial losses and damage to reputation~\cite{sachdeva2010ddos}, to broader impacts on internet infrastructure and service availability~\cite{sommese2022investigating,abhishta2019measuring}.
The anatomy of a DDoS attack is typically structured around a meticulously planned and executed workflow, generally encompassing three critical phases: Bot recruitment, malicious traffic generation, and traffic orchestration.

The bot recruitment stage lays the groundwork for the attack by establishing a network of compromised devices, known as a botnet~\cite{wang2018delving,rodriguez2013survey}.
Cyber attackers infiltrate these devices through malware or exploiting vulnerabilities to gain control over them without the owners' knowledge.
Each compromised device (called "bot") is then poised to contribute to the deluge of traffic directed at the target.
The size of the botnet can be a determining factor in the potential impact of the ensuing attack, with larger botnets capable of generating more traffic and causing more significant disruption.

Once an attacker has established a botnet, the next step is to coordinate the production of malicious traffic~\cite{wang2018data,de2018ddos}.
This traffic is not legitimate user data; instead, it is designed to mimic or disrupt normal traffic, thereby creating an overload condition.
Methods of generating this traffic can range from simple, such as flooding the target with superfluous requests, to complex, involving crafted packets that exploit specific vulnerabilities or weaknesses in the target's infrastructure.
The sophistication of this phase can vary, but the goal remains consistent: Generate enough traffic to exceed the target's handling capacity.

The final phase involves the strategic management and direction of the generated traffic towards the target system.
This step is akin to conducting an orchestra, with the attacker ensuring that the compromised devices in the botnet act in unison to deliver the attack traffic in a coordinated and timely manner~\cite{obaidat2023creating}.
Effective orchestration can amplify the impact of the attack, as it seeks to exploit choke points in the network or times of peak user activity to maximize disruption.
Moreover, the pattern and volume of the traffic generated by each bot can also be adjusted to evade detection and mitigation efforts~\cite{wang2018data}, making the attack more difficult to counter and resolve.

% \subsection{Notorious DDoS Attacks}
% In this section, we introduce a few notable DDoS attacks, which have had significant impacts on the internet and the entities targeted.

% \textit{2007 Estonia attack.}
% In April 2007, Estonia faced a series of cyber-attacks that lasted for weeks~\cite{ottis2008analysis}.
% The attackers used a botnet to flood Estonian websites with ICMP, TCP, and HTTP traffic.
% Targets included banks, media outlets, and government institutions.
% The sheer volume of traffic overwhelmed the websites, causing them to go offline.
    
% \textit{Slowloris.}
% Slowloris~\cite{slowloris} was reportedly used against Iranian government websites in 2009.
% It works by opening multiple connections to the targeted web server and keeping them open for as long as possible.
% It does this by continuously sending partial HTTP requests but never completing them.
% Periodically, it will send subsequent HTTP headers, but not the full request. The targeted server keeps each of these false connections open. This eventually overflows the maximum concurrent connection pool and leads to denial of additional connections from legitimate users.
    
% \textit{Mirai.}
% The Mirai botnet~\cite{antonakakis2017understanding} was responsible for a large-scale DDoS attack in October 2016, which affected websites such as Twitter, Reddit, and Netflix.
% The botnet primarily consisted of IoT devices such as digital cameras and DVR players.
% The attack on DNS provider Dyn with about 1.2 Tbps strength resulted in widespread outages for users trying to access impacted websites.
    
% \textit{2018 Github attack.}
% In February 2018, GitHub was hit with the largest DDoS attack recorded at the time, with a peak of 1.35 Tbps malicious traffic~\cite{chadd2s018ddos}.
% This attack utilized a method called "memcached reflection", which is a database caching system designed to speed up websites and networks.
% Attackers spoofed GitHub's IP address and queried memcached servers.
% As a result, the servers responded to Github by sending significant amounts of data to GitHub's servers, and the significant volumes of responses paralyzes the Github server.

\subsection{Related Work}\label{subsec:related-work}
Several survey papers have been published on DDoS attacks and detection.
The comparative analysis of the relevant survey papers is given in Table~\ref{tab:comparison-existing-survey}.
Specifically, the comparison is performed on the scope of DDoS attack and detection.
For DDoS attack, plenty of works focus on protocol exploitation.
These works~\cite{kumari2023comprehensive,li2023comprehensive,yan2015software} surveyed traditional protocol vulnerabilities, e.g., ICMP, TCP, and HTTP.
However, vulnerabilities for emerging protocols (e.g., HTTP/2 and IoT-specific protocols) are rarely discussed.
Moreover, existing surveys usually focus on specific systems.
For instance, Agrawal et al.~\cite{agrawal2019defense} present a survey that explores DDoS attacks within the context of cloud computing, and Kumari et al.~\cite{kumari2023comprehensive,chaudhary2023ddos} focuses on the IoT ecosystem instead.
The narrow scope of protocols and systems under investigation hinders the investigation of DDoS attack trends like low attack cost and common vulnerability patterns.
Additionally, the discussion of botnets and adversarial attack tactics is limited, while they become increasingly important in modern DDoS attacks.
In this survey, we studied DDoS attacks on nine network protocols and eight systems, from which we gained insight into the DDoS attack trends and the common vulnerability pattern.
We also surveyed related works which focus on botnet recruitment and coordination, and summarize the common exploits (e.g., weak authentication).
Finally, we summarized three types of adversarial tactics, revealing the emerging trend for adversarial DDoS. 
\begin{table*}
\centering
\caption{Comparisons with existing surveys on DDoS attacks and detection. The empty, half, and full circles mean ``Not mentioned", ``Partially mentioned", and ``Mentioned", respectively.}
\label{tab:comparison-existing-survey}
\scalebox{.8}{
\begin{tblr}{
  cells = {c},
  cell{1}{1} = {r=2}{},
  cell{1}{2} = {c=4}{},
  cell{1}{6} = {c=4}{},
  vlines,
  hline{1,3-4} = {-}{},
  hline{2} = {2-9}{},
}
\textbf{Authors} & \textbf{DDoS Attack}       &                          &                 &                                       & \textbf{DDoS Detection\textbf{}}                      &                                            &                                                         &                                          \\
                 & \textbf{\textbf{Protocol}} & \textbf{\textbf{System}} & \textbf{Botnet} & {\textbf{Adversarial}\\\textbf{DDoS}} & {\textbf{\textbf{Behavior}}\\\textbf{\textbf{Based}}} & {\textbf{Adversarial}\\\textbf{Detection}} & {\textbf{\textbf{Botnet}}\\\textbf{\textbf{Detection}}} & {\textbf{Innovative}\\\textbf{Hardware}} \\
Agrawal et al.~\cite{agrawal2019defense}                 &      \emptycirc                      &   \halfcirc                       &    \emptycirc             &   \emptycirc                                    &        \fullcirc                                               &     \emptycirc                                       &               \emptycirc                                          & \emptycirc \\
Li et al.~\cite{li2023comprehensive}                 &      \fullcirc                      &   \emptycirc                       &    \halfcirc             &   \emptycirc                                    &        \emptycirc                                               &     \emptycirc                                       &               \emptycirc                                          & \fullcirc \\
\hline
Kumari et al.~\cite{kumari2023comprehensive,chaudhary2023ddos}                 &      \fullcirc                      &   \halfcirc                       &    \fullcirc             &   \emptycirc                                    &        \emptycirc                                               &     \emptycirc                                       &               \emptycirc                                          & \emptycirc \\
\hline
Yan et al.\cite{yan2015software}                 &      \fullcirc                      &   \halfcirc                       &    \emptycirc             &   \emptycirc                                    &        \emptycirc                                               &     \emptycirc                                       &               \emptycirc                                          & \halfcirc \\
\hline
Zhang et al.\cite{zhang2024revealing}                 &      \halfcirc                      &   \emptycirc                       &    \emptycirc             &   \emptycirc                                    &        \halfcirc                                               &     \emptycirc                                       &               \emptycirc                                          & \halfcirc \\
\hline
Praseed et al.\cite{praseed2018ddos}                 &      \fullcirc                      &   \halfcirc                       &    \emptycirc             &   \emptycirc                                    &        \fullcirc                                               &     \emptycirc                                       &               \emptycirc                                          & \emptycirc \\
\hline
Our work                 &      \fullcirc                      &   \fullcirc                       &    \fullcirc             &   \fullcirc                                    &        \fullcirc                                               &     \fullcirc                                       &               \fullcirc                                          & \fullcirc \\
\hline
\end{tblr}
}
\end{table*}

For DDoS detection, a subset of works studied detection strategies based on the attack behavior~\cite{agrawal2019defense,praseed2018ddos,zhang2024revealing}.
For instance, Agrawal and Praseed et al.~\cite{agrawal2019defense,praseed2018ddos} discuss detection strategies for volumetric, low-rate, amplification attacks, e.t.c.
Zhang et al.~\cite{zhang2024revealing} mainly focus on the volumetric attack instead.
Besides the behavior-based detection strategy, our work includes a more comprehensive detection taxonomy, including behavior-based, statistics-based, learning-based, adversarial-based, and botnet detection methods.
Notably, the adversarial-based detection methods are rarely discussed in existing surveys, while our work covers it.
Some works also discuss innovative use of emerging hardware technologies for DDoS attack defense~\cite{yan2015software,li2023comprehensive}.
In particular, Yan et al.\cite{yan2015software} explore the implications of software-defined networking (SDN) in this domain, and Li et al.\cite{li2023comprehensive} consider the role of programmable switches.
These discussions focus on how such hardware can lower the costs associated with deploying DDoS attacks.
However, they fail to fully explore the unique capabilities of these technologies, such as line-speed packet processing, which could significantly enhance the efficiency and adaptability of DDoS attack detection systems.