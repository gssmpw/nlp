\section{Open Problems and Future Work}\label{sec:open-problem}

\subsection{Uncovering DDoS Vulnerabilities}
As discussed in Section~\ref{sec:ddos-attack}, DDoS attacks have emerged as a formidable weapon, constantly evolving to exploit the vulnerabilities of a myriad of protocols and systems.
With the dawn of new technological eras, we witness the birth of innovative protocols and intricate systems at a blistering pace~\cite{li2024survey,chen2024frequency}.
Yet, this rapid march of progress casts a shadow—a vast, unexplored territory where potential weaknesses against DDoS onslaughts lie hidden.
As a result, there is a critical need to forge a comprehensive analysis framework, one that is meticulously designed to dissect and scrutinize the DDoS attack vectors and vulnerabilities inherent within these nascent innovations.

Towards this end, we aim to summarize the critical features which are helpful to study the vulnerability of emerging protocols and features.
Specifically, we surprisingly find that despite the evolution of target protocols, certain patterns of vulnerability (e.g., session management) recur, suggesting a failure to learn from past mistakes.
This recurrence indicates a systemic issue within protocol design processes, wherein lessons from historical vulnerabilities are not adequately integrated into new developments.
In response to this challenge, we propose a comprehensive analysis of common protocol components that have historically and nowadays introduced vulnerabilities.
This analysis is intended to serve as a resource for security professionals and protocol designers, enabling them to more effectively evaluate the security posture of new protocols.
By identifying and understanding these commonalities in vulnerability patterns, it is possible to anticipate potential attack vectors, ensuring a higher degree of security evaluation in the early stages of protocol development.

\textbf{\textit{Improper session management.}}
    Improper session management is a central vulnerability that can lead to DDoS conditions in various protocols.
    The attacks on TCP, QUIC, and SIP protocols illustrate how exploitation of session management can overwhelm and incapacitate servers~\cite{nawrocki2021quicsand,sisalem2006denial,tang2014sip}.
    For instance, in TCP, the SYN flooding attack preys on the limit of the server’s backlog queue for half-open connections~\cite{luo2014mathematical,tang2013modeling,schuchard2010losing,wang2002detecting}.
    When examining new protocols, it is crucial to consider the robustness of session management mechanisms.
    Researchers and developers should validate the following aspects.
    First, the mechanisms for managing sessions should be scalable and resilient to unexpected surges in connection attempts.
    Second, protocols must incorporate aging-out strategies to quickly clean up sessions created by malicious session requests.

\textbf{\textit{Identity spoofing.}}
    Identity spoofing serves as a linchpin for DDoS attacks, including protocol-specific amplification and flooding attacks, e.g., QUIC, SIP, and DNS~\cite{yin2023waterpurifier,kim2017preventing,yazdani2022mirrors}.
    For example, attackers can exploit QUIC/DNS's UDP foundation by sending Initial packets with a spoofed source IP, which forces the server to respond with disproportionately large packets to the victim's address~\cite{nawrocki2021quicsand}.
    In VoIP environments, attackers can flood SIP routers with spoofed BYE messages, disrupting active calls~\cite{tang2014sip}.
    When evaluating new protocols, it is crucial to analyze their resilience to identity spoofing. 
    This involves a thorough examination of how the protocol handles source verification.
    Best practices for such evaluations should include stress testing under spoofed conditions, validating whether spoofed requests are filtered or rate-limiting measures are in place and effective.

\textbf{\textit{Packet fragmentation.}}
    Packet fragmentation has emerged as a salient vulnerability pattern that adversaries exploit to orchestrate DoS attacks, which affects IP and HTTP protocols~\cite{gilad2011fragmentation,atlasis2012attacking,dantas2014selective}.
    This pattern exploits the fundamental design of protocols where large packets must be broken down into smaller fragments for transmission and then reassembled at the destination.
    Attackers take advantage of the complexity in this reassembly process by injecting malicious, overlapping IP fragments and culminating in service disruptions or crashes.
    In examining new protocols for similar vulnerability patterns, it is imperative to scrutinize their packet handling and reassembly mechanisms under various edge cases and adversarial conditions.
    For instance, understanding how a protocol deals with fragmented packets and ensuring that it has robust checks against overlapping, missing, or redundant fragments is crucial~\cite{feng2022pmtud}.
    Moreover, mining for vulnerabilities should involve stress-testing the protocol with deliberately malformed packets to observe how it handles exceptional cases.


At the same time, it is of paramount importance to discern recurring vulnerability patterns embedded within emerging systems.
Such vulnerabilities are the Achilles' heel that could precipitate DoS threats. Understanding and addressing these weak links proactively is not just an exercise in threat mitigation but a fundamental strategy to fortify our digital ecosystem against the ever-evolving menace of service disruption.
Herein, we enumerate critical system components and features that are susceptible to DoS threats, offering guidance for scrutinizing new systems for potential vulnerabilities.

\textbf{\textit{System architecture.}}
    Analyzing system architecture is an essential step in identifying potential vulnerabilities to DoS attacks.
    A clear illustration of this can be found in the architecture of LTE networks, where the distinction between control and data planes can be a critical weakness~\cite{javadpour2023reinforcement,mirsky2020ddos,silva2020repel}.
    Similarly, SDN are susceptible to DDoS attacks due to the same principle, and this separation can be exploited by attackers who generate malicious traffic aimed at overwhelming the specific system plane~\cite{shin2013avant,shin2013attacking,cao2019crosspath}.
    
    To uncover comparable DoS vulnerabilities in new systems, security researchers may adopt a systematic approach that includes the following steps.
    (1) Develop a deep understanding of the system's architecture with a keen focus on the separation of planes and their interdependencies.
    It is critical to identify how communication between the control and data planes is managed and to determine potential choke points where traffic could be maliciously concentrated, e.g., the SDN path from the data plane to the control plane.
    (2) Identify and evaluate centralized components within the architecture—such as SDN controllers or management servers—that represent single points of failure.
    If these components are compromised or overwhelmed, it could result in a systemic failure, effectively crippling the network.
    (3)  Conduct controlled and ethical stress testing, simulating targeted attacks on independent planes.
    By doing so, it is possible to assess the system's resilience and response to high volumes of malicious traffic directed at specific planes.
    
\textbf{\textit{Resource sharing.}}
    Resource sharing is a vital component of contemporary networked systems, aimed at enhancing both efficiency and flexibility.
    However, the very act of sharing resources opens the door to potential security vulnerabilities, especially in the face of DDoS attacks.
    At the heart of the issue lies the principle of interdependence within shared resources, which can be exploited to carry out such attacks. Notable examples include network slicing in 5G technologies, the utilization of a common egress IP in serverless computing platforms~\cite{silva2020repel}, and the sharing of a single Software-Defined Networking (SDN) link for both data and control signal transmission~\cite{cao2019crosspath}.

    To identify and understand the possible DoS threats that this resource sharing might incur, it is crucial for security researchers to gain a comprehensive overview of the entire spectrum of shared resources in a given system. This encompasses not only the physical hardware but also the software components that might be utilized concurrently by different users or services.
    Equally important is the need to examine the patterns of resource consumption and to establish the network of interdependence among the system's clients. By creating an appropriate threat model — one that assumes the presence of malicious clients intent on manipulating shared resources — researchers can then investigate the impact of resource consumption on legitimate users, with a particular focus on assessing its effects on the reliability of the system. 

    
\textbf{\textit{Component dependency.}}
    Modern systems often involve complex inter-dependencies among components.
    These dependencies can create unforeseen DoS vulnerabilities, particularly when one component's performance is contingent upon availability or reliability of other components.
    For example, in a routing system, a large number of important server nodes usually rely on a few common critical links.
    Identification of these links can help to reveal potential vulnerabilities for launching link-flooding attacks~\cite{kang2013crossfire}.
    As another example, figuring out the state dependency among control centers in a smart grid system can help to pinpoint the critical position for injecting false measurement signals and paralyzing the whole grid with little cost~\cite{vukovic2014security}.
    Finally, considering that systems usually rely on underlying network protocols (e.g., the TCP and BGP protocol for blockchain system), the evaluation of underlying protocols' attack surface can bring more insight towards revealing DoS potentials of systems which are built on~\cite{heilman2015eclipse,tran2020stealthier}.

    For security researchers and practitioners looking to unearth similar vulnerabilities in new systems, the following strategies may apply.
    (1) Start by creating a detailed map of the system architecture, highlighting the dependencies between components. Tools like dependency graphs can be invaluable in visualizing and understanding complex interconnections.
    (2) Identify potential checkpoints where traffic or data converges, and assess the impact of their failure. This includes not only physical links but also critical software processes.
    (3) Study the underlying network protocols for known vulnerabilities and consider how they might be exploited in the context of the current system.

\subsection{Construction of Adversarial Attack and Detection Strategies}
As the arms race between cybersecurity defenses and adversarial attacks continues, the complexities of both are escalating.
The emergence of advanced adversarial learning techniques, the proliferation of commercial DDoS protection services, and the strengthening of protocol security mechanisms compel attackers to constantly innovate their strategies to circumvent state-of-the-art DDoS defense solutions.
This dynamic landscape opens up new research avenues in the study of adversarial DDoS attack and detection, which are shown as follows.

\textbf{\textit{Adversarial machine learning for malicious traffic generation.}}
    Attackers are leveraging machine learning (ML) and deep learning (DL) to generate malicious traffic that can elude detection systems~\cite{abusnaina2019examining,mustapha2023detecting,matta2017ddos}.
    By training models that can anticipate the behavior of learning-based detection systems, adversaries can craft traffic that blends with legitimate network activity, thus increasing the difficulty of detection.
    Techniques such as flow-merge and Generative Adversarial Networks (GANs) are particularly effective in refining malicious traffic to mimic benign characteristics, challenging the reliability of current detection methods.
    To counter the adversarial learning and detect the generated traffics, a promising direction is to enhance adversarial training for the detection model.
    Specifically, incorporating a wider array of adversarial examples during the training phase can prepare detection systems to handle unexpected attack vectors.
    Moreover, applying robustness tests across different types of models (e.g., decision trees, autoencoders) can help identify adversarial traffics specific to certain algorithms, leading to more resilient hybrid systems.

\textbf{\textit{Probing and circumventing commercial DoS protection services.}}
    With an increasing reliance on commercial DDoS protection services, assessing their efficacy has become vital~\cite{jin2018your,nosyk2023closed,wu2014software,shankesi2010resource}.
    The first stage in this process involves designing probing techniques to identify whether the victim is under protection.
    Active probing, which uses carefully crafted requests and subsequent response analysis, is the favored approach due to its precision.
    Once a protection service is identified, understanding its life-cycle (e.g., how it manages changes in client behavior and residual DNS records~\cite{jin2018your}) and dissecting its foundational protection mechanisms (e.g., client puzzles) are crucial to reveal vulnerabilities.
    This knowledge is instrumental in developing strategies to investigate the vulnerability of commercial services.

\textbf{\textit{Examination of protocol security mechanism.}}
    The protocol security mechanism is paramount for ensuring the authenticity, confidentiality, and integrity of communications.
    However, these mechanisms often interact with security credentials (e.g., TCP sequence numbers), which can be exploited to glean sensitive information~\cite{cao2016off,wang2024off,feng2022pmtud}.
    This potential side-channel can be weaponized to conduct DDoS attacks, for example, by sending malicious RST packets to prematurely close legitimate connections~\cite{feng2020off,feng2022off}.
    Therefore, rigorously evaluating the resilience of protocol security mechanism against such exploitation is necessary to reveal potential attacks.
    

\subsection{Privacy-Preserving DDoS Detection}
Various detection techniques necessitate differing amounts of data to discern the distinction between legitimate users and potential attackers.
For example, detection methods that rely on machine learning algorithms utilize traffic feature extraction from both user and attacker data streams to train their models~\cite{ahmed2018statistical,mirsky2018kitsune}.
However, certain techniques raise greater privacy concerns.
Behavior-based detection methods, in particular, often require access to detailed data such as users' browsing history or server resource utilization logs~\cite{tandon2021defending,tandon2023leader}.
These data are used to identify nuanced patterns of behavior, which can then aid in better differentiating between legitimate users and attackers.
Privacy considerations become increasingly complex when there is a need for cross-domain data sharing, where user information is exchanged between different autonomous systems (ASes).

As a result, it is critical to consider the equilibrium between fortifying security measures and upholding user privacy.
Towards this end, techniques like federated learning and data encryption shed lights on addressing the privacy issue.
For instance, Dimolianis et al.~\cite{dimolianis2022ddos} propose to use federated learning techniques, which collaborate multiple Autonomous Systems to train a shared model using their private data.
Each participant trains the model on their local data and computes an update to the model parameters.
As another example, Zhu et al.~\cite{zhu2018privacy} enforce perturbation encryption to encrypt the network traffic.
This encrypted data is then sent to the Computing Server (CS), which is responsible for performing traffic examination and DDoS detection.
This perturbation ensures that the actual traffic data cannot be directly read by the CS, thus preserving the privacy of the data.

However, existing privacy-preserving DDoS detection methods exhibit limitations that warrant further exploration.
The first issue is efficiency. The operations intended for privacy preservation, such as data encryption and federated communication, often introduce additional processing time, which can be significant when dealing with real-time network traffic
It is crucial to develop more advanced methods that can maintain privacy without a substantial impact on detection efficiency.
The second limitation involves the robustness of existing privacy-preserving detection methods, often tailored to defend against specific types of DDoS attacks.
To enhance the robustness of these systems, it is necessary to extend the models to be capable of recognizing a variety of attack vectors concurrently, thereby improving adaptability to different threats.
Lastly, certain methods, particularly those based on federated learning, are vulnerable to data poisoning. Adversaries can manipulate the model by injecting malicious traffic data, potentially compromising the model's integrity. Addressing this vulnerability requires the creation of more resilient models that can detect and mitigate the effects of data poisoning to safeguard the detection process.

\subsection{DDoS Detection Without Control Planes}
Current solutions (e.g., SDN) impose substantial burdens on the control plane.
It is tasked with aggregating traffic statistics, detecting anomalies, and deploying mitigation strategies onto the data plane.
However, the data plane is relegated to basic functions such as rudimentary traffic statistics computations (e.g., packet counts) and implementing detection rules (e.g., flow rules).
This configuration introduces several critical limitations.

\noindent\textit{(1) New attack vectors.} The control plane becomes a target for attackers.
A notable example is the SDN control plane saturation attack (Section \ref{sec:ddos-attack}), which can induce a single point of failure, effectively crippling the entire network.
    As a result, the system intended to safeguard against DDoS attacks can be exploited to disrupt network operations.
    
\noindent\textit{(2) Performance degradation.} Communication between the control and data planes significantly hampers performance.
The necessity for the data plane to send traffic statistics to the control plane, coupled with the control plane's periodic flow rule updates on the data plane, results in increased latency for standard traffic routing due to the additional round-trip communications.
    
\noindent\textit{(3) Resource constraints.} Developing frameworks for control and data plane interaction demands resources, such as memory, which are often scarce and valuable on the data plane.
    This resource allocation can strain the data plane's capabilities, leading to suboptimal performance.

To address the above limitation, a promising strategy is to focus defense mechanisms within the data plane.
This approach narrows potential attack surfaces while leveraging the inherent advantage of high-speed traffic processing, detection, and routing capabilities.
Programmable switches, equipped with advanced hardware primitives, are at the forefront of enabling this shift, with pioneering efforts already underway.
As elaborated in Section \ref{subsec:programmable-switch}, current research is converging on empowering programmable switches with sophisticated statistical computations (e.g., entropy measures) and complex detection methodologies (e.g., machine learning algorithms).
In pursuit of these advancements, certain areas merit further investigation:
\textit{Efficient data structures.}
    The necessity for compact and efficient data structures becomes paramount when dealing with high traffic volumes, particularly given the limited registers and memory available in programmable switches.
    Data sketching emerges as a potent technique to efficiently approximate traffic features under such constraints.
\textit{Complex algorithm deployment.}
    It is also intriguing to consider the deployment of more intricate detection algorithms within these switches.
    A compelling research question is how to effectively translate a fully trained neural network into a set of match-action rules that are compatible with the operational paradigms of switches.
