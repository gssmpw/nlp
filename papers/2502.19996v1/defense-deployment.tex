\section{Detection System Deployment}\label{sec:ddos-detection-system-deployment}

% \subsection{Edge Servers and Routers}
% Edge servers and edge routers are important positions for deploying detection primitives.
% For instance, some works deploy source address validation service on these edge entities.
% Beverly et al.~\cite{beverly2009understanding} show that 80\% of IP-spoofing packet filtering occurs at the first-hop IP router, with over 95\% of blocked packets being filtered within the edge router of the same AS as the source.
% Liu et al.~\cite{liu2011deployable} propose a scheme, DIA, which is deployed at AS borders.
% It support end-to-end verification, key sharing, and packet tagging techniques for inter-AS communications.
% The scheme allows ASes to collaborate for the spoofing IP detection, instead of detection within individual ASes.
% Similarly, Mutual Egress Filtering (MEF) is deployed between ASes with a more fine-grained management (e.g., a lightweight control system for inter-AS collaboration and a registry for ASes to register as deployers or become peers).

% However, deployment on edge devices have several limitations.
% \begin{itemize}
%     \item  Edge networks have many devices and interfaces, making update and management of detection primitives (e.g., address validation service) operationally challenging.
%     \item Since a single edge device with detection primitives (e.g., unfiltered ingress) point can undermine global protections, the benefits of individual deployments are minimal without large-scale cooperation.
% \end{itemize}

\subsection{SDN}
The advent of SDN represents a substantial advancement in network management and security.
SDN introduces innovative core concepts, such as the separation of the control plane from the data plane.
This separation provides a flexible framework for implementing DDoS defense strategies.
In this section, we will explore the benefits of SDN and examine contemporary research that leverages SDN capabilities to enhance DDoS defense mechanisms.

\subsubsection{Programmability}
SDN's programmability facilitates the rapid creation and deployment of software-based features to detect and counteract DoS attacks, eliminating the need for hardware upgrades.
Specifically, it enables rapid deployment of software-based detection targeting various types of DDoS attacks.
Tang et al.'s Performance and Features (P\&F) framework~\cite{tang2021performance} utilizes SDN's programmability to combat Low-rate Denial of Service (LDoS) attacks by incorporating detection and mitigation directly within the SDN controller, showcasing how software solutions can swiftly adapt to security challenges.
Similarly, the LFADefender system~\cite{wang2018detecting} leverages SDN to identify and thwart Link Flood Attacks (LFA) by analyzing network flows and adjusting switch rules in real-time, demonstrating SDN's capability to quickly respond to threats.
SDNShield~\cite{chen2021sdnshield} employs a three-stage defense against SYN flooding attacks, using SDN's programmable features for statistical analysis, authentication of TCP handshakes, and a recovery mechanism for legitimate traffic, emphasizing SDN's role in maintaining network security with agility.

SDN enables quick deployment of security measures and statistical-based detection methods through software. The Joint Entropy Statistical Scheme (JESS)~\cite{kalkan2018jess} leverages SDN's programmability for real-time entropy-based analysis, dynamically creating security rules to defend against DoS attacks, bypassing hardware modifications.
Additionally, the DOCUS framework~\cite{shalini2022docus} utilizes SDN's programmable controller to implement a Cumulative Sum (CUSUM) algorithm, which monitors connection patterns for early DDoS detection.
Finally, Long et al.~\cite{long2022hybrid} use SDN's flexibility to incorporate entropy-based anomaly detection and a hybrid machine learning model for real-time DoS attack classification. Through software updates to the SDN controller, the network autonomously adjusts flow tables to counter threats, showcasing SDN's ability to facilitate immediate, software-centric security responses without hardware upgrades.

SDN's programmability allows for the deployment of advanced learning-based detection systems without hardware upgrades.
Najar's study~\cite{najar2024cyber} demonstrates SDN's capacity to swiftly integrate deep learning techniques such as CNNs for traffic analysis, with preprocessing strategies like Balanced Random Sampling adapting to new attack patterns. This showcases SDN's adaptability in cyber defense.
Hnamte et al.~\cite{hnamte2024ddos} further illustrate SDN's capabilities by developing a DNN model for DDoS detection that can be quickly updated and redeployed in response to evolving threats, emphasizing the value of SDN's flexible infrastructure.
Ribeiro et al.~\cite{ribeiro2023detecting} present an SDN-based architecture that incorporates ML models for real-time traffic monitoring and malicious flow detection. The architecture also employs Moving Target Defense (MTD) to redirect attacks, leveraging SDN's dynamic reconfiguration to mitigate damage, thus exemplifying SDN's role in facilitating the swift implementation of effective software-defined security measures against DoS attacks.

Finally, SDN affords privacy and efficiency enhancement in DDoS attack detection. Zhu et al.~\cite{zhu2018privacy} leverages SDN's programmability to integrate an optimized k-Nearest Neighbors (kNN) algorithm into the SDN controller, enabling encrypted traffic analysis for privacy-preserving DDoS detection.
Rashidi et al.~\cite{rashidi2017collaborative} utilizes a game-theoretic model within the SDN control plane for dynamic resource allocation, illustrating SDN's capacity for efficient detection supports.

\subsubsection{Separation of Control and Data Planes}
Unlike traditional network architecture where the control logic is embedded within each network device (e.g., routers and switches), SDN decouples the control plane from the data plane enables centralized management of the network, allowing for more coordinated and efficient control decisions with a comprehensive overview.
The separation mechanism between control and data planes in SDN offers versatile deployment options for detection methods.
Rezapour et al.~\cite{rezapour2021rl} demonstrate the use of reinforcement learning within the context of SDN to combat link-flooding attacks.
The applied algorithms, ERA and DRA, benefit from the SDN's control-data plane separation by dynamically adapting routing decisions.
Yue et al.~\cite{yue2024ccs} leverage SDN's architecture for efficient LDoS attack defense, employing data plane anomaly detection to offload the controller and facilitate swift centralized responses.
Upon detecting irregularities, switches prompt a controller-led global analysis with Bayesian voting. Confirmed attacks trigger an immediate, optimized rerouting response, showcasing SDN's potential for coordinated, dynamic network defense.

SDN's centralized control management and unified network view enable rapid detection and coordinated mitigation.
Cao et al.~\cite{cao2021detecting} capitalize on the centralized view of the network provided by the SDN controller to maintain an accurate and up-to-date map of the network topology and monitor the state of switches.
This comprehensive visibility is crucial for early detection of link-flooding attacks, as the system can quickly notice and react to changes in traffic flow that may signal an attack.
Similarly, Najar et al.~\cite{najar2024cyber} show how SDN's unified network view helps the CNN models for traffic analysis, pinpointing anomalies across all switches.
Moreover, this centralized intelligence allows the SDN controller to swiftly command rate-limiting and flow removal responses.
Li et al.~\cite{li2023path} demonstrate that SDN's centralized control model is vital for strategic network management. Utilizing the CNNQ algorithm, the framework capitalizes on SDN's global network view to optimize NSF deployment paths, thereby enhancing the network's defense against DDoS attacks.


%\begin{itemize}
%    \item \textit{Single point of failure.} The centralized nature of SDN's control plane can be a single point of failure. If the controller is compromised or goes down, the entire network can be affected.
%    \item \textit{Communication latency.} The communication bandiwdth between SDN switches and controllers are still constrained, and they are still the bottleneck.
%\end{itemize}




\subsection{Programmable Switch}\label{subsec:programmable-switch}
The cybersecurity landscape is currently undergoing significant changes due to the emergence of programmable switches.
These switches offer impressive data-plane programmability that surpasses that of SDN, enabling the customization of packet parsing and processing at full network speed.
Consequently, there is a growing body of research dedicated to implementing DDoS defense techniques directly on programmable switches to take advantage of their rapid in-network processing capabilities.
However, the limited resources available on these switches, such as registers and memory, pose substantial challenges for the deployment of such techniques. As a result, recent studies have focused on developing sophisticated DDoS defense mechanisms that are not only effective but also resource-efficient. This section will explore the latest advancements in leveraging the potential of programmable switches to enhance DDoS defense mechanisms.

\subsubsection{Data-Plane Traffic Statistics Analysis}
Programmable switches have become a promising tool for traffic analysis, enabling the rapid collection of network statistics and performing statistic-based DDoS defense.
Given the inherent constraints of programmable switches, such as limited register and memory capacity, data sketching techniques are extensively employed.

Liu et al.~\cite{liu2016one} propose a traffic analysis method using universal sketching to optimize data collection within programmable switches.
The approach starts with packet sampling via flowkeys hashed by predefined functions.
These packets update sketch counters in the on-chip SRAM through the Count Sketch algorithm, while a dedicated list of heavy flowkeys is managed in a fixed-size TCAM for rapid updates.
A P4 program controls packet processing, allowing packets to sequentially move through sampling, sketching, and storage stages.
For DDoS detection, the technique tracks the number of unique flows to a host against a predefined threshold.
Programmable switches maintain counters for these flows using the data plane's universal sketch primitive, enabling real-time detection and mitigation of DDoS attacks with the current switch infrastructure.

Similarly, Jaqen~\cite{liu2021jaqen} applies universal sketching to programmable switches for DDoS attack detection.
The system architecture bifurcates into data and control planes.
The data plane integrates universal sketches with a signature detector to gauge attack-related metrics.
Efficiency is achieved by condensing short hashes into long ones to decrease hash calculations and by updating a single Count Sketch instance per packet to minimize memory accesses.
The control plane utilizes an API, featuring a Query function for metric retrieval and thresholds to sift through traffic for anomalies, thus identifying potential DDoS events.

Ding et al.~\cite{ding2021tracking} introduced two sketch-based methods, P4LogLog and P4NEntropy, which are designed to function within the constraints of programmable switches.
P4LogLog is designed to estimate the flow cardinality, which is the number of unique network flows. It operates by updating a register for each incoming packet. The update is based on the hash of packet attributes, allowing the system to maintain a count of distinct flows without storing each flow identifier.
P4NEntropy aims to estimate the normalized entropy of network traffic.  P4NEntropy leverages the flow cardinality estimated by P4LogLog and combines it with a sketch data structure, such as the Count Sketch, to estimate the counts of packets per flow. The switch then calculates the entropy using operations supported by the P4 language.


\subsubsection{Data-Plane Machine Learning}
There is a growing interest in the application of learning-based methods on programmable switches.
However, the practical implementation of machine learning models on such switches is challenging.
The primary obstacle is the limited set of operations that programmable switches support. For example, they typically lack the ability to perform multiplication or division, which are fundamental operations in many machine learning algorithms.
Additionally, the hardware's architecture, which revolves around match-action tables, is not naturally conducive to the complex computations required by these models.
Therefore, adapting machine learning models to function within these constraints is an active area of research.
The goal is to develop methods that can translate the sophisticated processes of learning-based models into operations compatible with the streamlined, efficient environment of programmable switches.


Barradas et al.~\cite{barradas2021flowlens} focus on using programmable switches for classification-based DDoS mitigation through the Flow Marker Accumulator (FMA), a data structure for efficient flow classification within the switch's data plane. The FMA captures flow markers (i.e., simplified encodings of packet distributions) using quantization to bin continuous features and truncation to fit the switch's memory constraints. Tailored for simplicity to match the switch's computational limitations, the FMA, written in P4 language, balances memory use and classification accuracy and is distributed across the switch pipeline using match-action units.

Alcoz et al.~\cite{alcoz2022aggregate} have implemented a DDoS detection system using online clustering techniques within programmable switches, functioning at line rate.
This system integrates a data plane-based online-clustering module for attack detection with a programmable-scheduling module that operates across the control and data planes to mitigate attacks.
The control plane routinely analyzes clusters to determine scheduling policies for the data plane, which processes incoming packets and manages traffic based on these policies.
The system accommodates the limitations of programmable switches by employing registers for ordinal feature clusters and bloom filters for nominal features, with a resubmission mechanism to update clusters for new traffic patterns.



Zhou et al.~\cite{zhou2023efficient} have developed a method to deploy decision tree-based ML models, specifically Random Forest (RF) and XGBoost (XGB), on programmable switches.
Their system, NetBeacon, features an innovative IDP (In-band Network Telemetry Data Plane) design that employs a sequential multi-phase model architecture.
It is tailored to process per-packet and flow-level features at line speed, including the computation of aggregate and summary statistics such as maximum, minimum, mean, and variance.
To circumvent the issue of match-action table bloat, NetBeacon introduces "range marking", a technique that maps numerical ranges to unique bit strings, enabling efficient representation within match/action tables.
Recognizing the hardware's limitations, NetBeacon applies alternative methods such as approximate calculations and bit shifting to estimate statistical measurements like variance and mean, avoiding operations like multiplication and division that are unsupported on programmable switches.

Similarly, Dong et al.~\cite{dong2023horuseye} introduce HorusEye, a system that implements an ensemble of Isolation Trees (iTrees) on programmable switches for anomaly detection.
HorusEye transforms the iForest model into a collection of whitelist matching rules, enabling the deployment on the switches' match-action tables.
It operates by first parsing incoming packets to attribute them to their respective flows, employing bi-hash algorithms for accurate flow identification.
Subsequently, HorusEye captures burst-level traffic features using a defined segmentation threshold and leverages bi-directional flow matching.
With the traffic features extracted, the system applies the derived iForest whitelist rules to detect anomalies.


Recent research has been focusing on how to efficiently compute network features, which are vital for ML detection, on programmable switches.
Romeiras et al.~\cite{romeiras2023poster} introduce Peregrine, which equips the data plane to calculate flow features in real-time as the switch processes packets, utilizing the PISA pipeline's capabilities for basic arithmetic and stateful memory operations.
It uses the switch's stateful memory to maintain counters for metrics such as packets and bytes per flow, updating these with each passing packet.
Peregrine computes a variety of statistics, including simple unidirectional metrics and more complex bidirectional ones, like mean, variance, standard deviation, magnitude, radius, and an approximate covariance. 
Similarly, Doriguzzi et al.~\cite{doriguzzi2024introducing} demonstrate how a programmable switch can incrementally calculate statistics, such as the average packet size per flow.


\subsubsection{Data-Plane Security Primitive}
In addition to offering direct defense solutions against specific DDoS attacks, several research efforts are dedicated to the creation of security primitives for programmable switches.
These primitives serve as fundamental building blocks that enable users to develop their own tailored defense strategies to counter DDoS threats effectively.

Xing et al.~\cite{xing2021ripple} introduce Ripple, a framework that incorporates a set of seven security primitives tailored for link-flooding DDoS (e.g., Crossfire and Coremelt attacks).
Ripple's standout feature is its ability to construct a defense panorama.
This is a comprehensive view that captures network-wide threat signals derived from local traffic patterns on each switch.
With the panorama, users can specify their detection and mitigation policies using Ripple's primitives, and these policies will be translated into a coordinated collection of P4 programs by the Ripple compiler.
By doing so, users can concentrate on designing their defense strategies at a high level, while Ripple takes care of the underlying complexity, generating the necessary P4 code to implement these strategies across the network's switches.

Mew~\cite{zhou2023mew} introduces a framework that enables programmable switches to synchronize information based on specific criteria.
It achieves multi-level cooperation through four APIs: \textit{Monitor} for state storage configuration, \textit{Sync} for state dissemination within a range, \textit{Request} for defining state request modes and intervals, and \textit{Trigger} for executing actions upon certain conditions.
These APIs simplify the design of detection and mitigation mechanisms.
To manage the limited resources on programmable switches, Mew implements a lightweight distribution protocol that balances storage by selecting the least-utilized switch with a greedy algorithm, adjusting the distribution of states over time.
Additionally, Mew incorporates a memory resizing mechanism that facilitates memory sharing and reallocation among defense functions, allowing for efficient utilization of switch memory and enabling more functions to run concurrently.

\subsection{Summary}
In this section, we summarize the advanced features brought by these advanced network hardware for efficient and effective DDoS detection.

\textbf{\textit{Emerging network hardware facilitates line-speed DDoS defense.}}
    The escalation of network traffic volumes presents a significant challenge to traditional defenses against DDoS attacks (e.g., middleware at edge routers and scrubbing centers).
    In this context, the advent of data-plane programmability, as seen in modern network hardware, offers a transformative approach to mitigating DDoS risks.
    By leveraging the flexibility of control plane decision-making along with the efficient execution capabilities of the data plane, networks can deploy more sophisticated and responsive defense mechanisms.
    
\textbf{\textit{Emerging network hardware facilitates the deployment of coordinated DDoS defenses.}}
    DDoS attacks (e.g., link flooding attacks) are being more sophisticated and difficult to detect and mitigate.
    Addressing these threats requires a dynamic and coordinated defense systems.
    SDN and programmable switches offer a beacon of hope in this cyber arms race.
    Two frameworks that epitomize this potential are Ripple and Mew.
    By constructing a network-wide defense panorama and enabling efficient information synchronization and memory management, these frameworks provide the tools necessary to counteract the stealth and complexity of modern DDoS threats.

\textbf{\textit{Techniques are advancing to develop sophisticated detection models that can operate on hardware with limited capabilities in a flexible manner.}}
    The advancements in DDoS detection techniques for hardware with limited capabilities, like programmable switches, showcase the ongoing innovation within the field of network security.
    One example of this is how complex machine learning models, such as decision trees, can be converted into formats like match-action rules that are compatible with the limited processing capabilities of programmable switches.
    This conversion process enables the implementation of advanced DDoS detection models on these powerful yet constrained devices.
    The development of security primitives also simplifies the deployment process of defense strategies.
    Additionally, the inherent programmability of these switches offers security analysts the flexibility to swiftly change and implement customized defense strategies without significant downtime.
    This adaptability is crucial for maintaining robust defense mechanisms against DDoS attacks in a dynamic threat landscape.

\textbf{\textit{Efficient resource management on these hardware is of paramount importance.}}
    The efficient management of advanced network hardware resources is a cornerstone in the battle against DDoS attacks.
    Techniques such as packet transformation~\cite{holland2021new}, data sketching~\cite{liu2016one}, and dynamic resource allocation~\cite{rashidi2017collaborative}, coupled with the programmability and cooperation enabled by frameworks like Mew, provide a robust foundation for network defense.
    These methods ensure that despite their limited resources, advanced network devices can effectively detect and mitigate DDoS attacks, preserving network reliability and service availability.


% Some hints for future works.

% Despite these advanded hardware, they still rely on the control plane, which incurs additional communication burdan and bring in additional attack vector. A promising direction is data-plane-only detection.

% These hardwares still cannot deploy complicated detection methods (e.g., deep learning methods). It will be useful if new transformation techniques are proposed to transform these methods to easy deployment mannars.