\section{Background}
\label{sec:background}

\begin{figure*}[htbp]
\centering
\includegraphics[width=\textwidth]{Fig_background.pdf}
\caption{Ciphertext side-channel examples and revisiting vulnerabilities from the perspective of compilation.}
\label{fig:background}
\end{figure*}

\subsection{Ciphertext Side-Channel Attacks}
\label{subsec:ciphertext}

The ciphertext side channel originates from the deterministic memory encryption implemented in AMD's TEE.
The encrypted memory is calculated by an XOR-Encrypt-XOR (XEX) mode, expressed as: $c = ENC(m \oplus T(P_{m})) \oplus T(P_{m})$, where the plaintext $m$ undergoes the XOR operations before and after AES-128 encryption with a tweak value $T(P_{m})$ that incorporates the physical address $P_{m}$.
Without freshness in the encryption process, the encryption of the same plaintext at a given physical address produces the identical ciphertext.
It is crucial to acknowledge that this vulnerability extends to other deterministic encryption-based TEE architectures as long as attackers have read accesses to ciphertext (via software access~\cite{li2021cipherleaks} or memory bus snooping~\cite{lee2020off}).

% \begin{figure}[htbp]
% \vspace{-5pt}
% \begin{minipage}[c]{0.5\linewidth}
%     \begin{subfigure}[b]{\linewidth}
%     \centering
%     \footnotesize
%     \begin{tabular}{l}
%         1: pbit $\leftarrow$ 1;\\
%         2: \textbf{for}\ i $\leftarrow$ cardinality\_bit - 1\ downto\ 0$\lbrace$\\
%         3: $\quad$ kbit $\leftarrow$ BN\_is\_bit\_set(k, i) $\wedge$ pbit;\\
%         4: $\quad$ EC\_POINT\_CSWAP(kbit, r, s, ...);\\
%         5: $\quad$ ...\\
%         6: $\quad$ pbit $\leftarrow$ pbit $\wedge$ kbit;$\rbrace$\\
%     \end{tabular}
%     \caption{ossl\_ec\_scalar\_mul\_ladder.}
%     \label{fig:channel1}
%     \end{subfigure}
% \end{minipage}
% \hspace{15pt}
% \begin{minipage}[c]{0.4\linewidth}
%     \begin{subfigure}[b]{0.9\linewidth}
%     \centering
%     \footnotesize
%     \begin{tabular}{l}
%         1: \textbf{for}\ i $\leftarrow$ 0\ to\ nwords - 1$\lbrace$\\
%         2: $\quad$ t $\leftarrow$ (a.d[i] $\wedge$ b.d[i])\\
%         3: $\quad \quad \quad$ \&\ condition;\\
%         4: $\quad$ a.d[i] $\leftarrow$ a.d[i] $\wedge$ t;\\
%         5: $\quad$ b.d[i] $\leftarrow$ b.d[i] $\wedge$ t;$\rbrace$\\
%     \end{tabular}
%     \caption{BN\_constant\_swap.}
%     \label{fig:channel2}
%     \end{subfigure}
% \end{minipage}
% \caption{Ciphertext side-channel examples.}%\yz{change font in figures.}
% \label{fig:channels}
% \vspace{-5pt}
% \end{figure}

Two attack schemes are introduced in~\cite{li2022systematic}.
The \textit{Dictionary} attack involves the continuous monitoring of the ciphertext at a fixed memory address to construct a dictionary containing mappings of ciphertext-plaintext pairs.
Consider the code snippet shown in \F~\ref{fig:background}(a), extracted from the ECDSA Montgomery ladder algorithm implemented in OpenSSL-3.0.2.
In each loop iteration, the \texttt{BN\_is\_bit\_set} function (denoted by $k_{i}$ in line 3) is utilized to obtain one bit of the secret $k$.
Following this, the $kbit$ variable is computed through an XOR operation with the value in $pbit$, which is then written back to $pbit$.
Given the dual XOR operations in lines 3 and 6, $pbit$ ultimately stores each bit of the secret $k$.
The attacker records consecutive ciphertext pairs ($pbit$-$kbit$) both before and after the \texttt{BN\_is\_bit\_set} function, aiming to deduce $k_{i}$ in each iteration based on the changes observed in ciphertext pairs.
In contrast, the \textit{Collision} attack focuses on identifying repetitions or alterations in certain ciphertexts to break the constant-time mechanism.
\F~\ref{fig:background}(b) shows the constant-time-swap function \texttt{BN\_constant\_swap}.
This function takes two variables $a$ and $b$, along with a decision $C$ (e.g., $kbit$ in line 4 of \F~\ref{fig:background}(a)).
If $C$ is set to 1, the values of $a$ and $b$ are exchanged, leading to observable changes in the ciphertext. Conversely, if $C$ is 0, the ciphertext remains unaltered.
In this way, the \textit{Collision} attack recovers the decision $C$, undermining the constant-time component.

Currently, many well-known cryptographic applications are vulnerable to this attack, including RSA and ECDSA (such as \textit{secp256k1} and \textit{secp384r1}) equipped with constant-time algorithms, ECDSA from WolfSSL-5.3.0, ECDSA and RSA from MbedTLS-3.1.0, as well as EdDSA (\textit{Ed25519}) from OpenSSH adopted by Ubuntu LTS 20.04~\cite{li2021cipherleaks, li2022systematic}.

\subsection{Countermeasures to Ciphertext Side-channels}
\label{subsec:countermeasures}

Hardware-based countermeasures provide stronger security by eliminating ciphertext side channels, but they require extensive validation before chip manufacturing. In contrast, we choose a software-based approach, enabling quicker implementation and deployment without modifying hardware.
Unfortunately, existing countermeasures for cache and timing side channels~\cite{percival2005cache, osvik2006cache, zhang2012cross, yarom2014flush, liu2015last, yarom2014recovering, ryan2019return, aranha2020ladderleak}, like constant-time cryptography, cannot mitigate ciphertext side channels. While constant-time cryptography avoids secret-dependent branches and memory accesses, it has been shown to be ineffective against ciphertext side-channel attacks~\cite{li2021cipherleaks, li2022systematic, deng2023cipherh}.

% Previous efforts adhering to this concept can be categorized into three classes. 
% 1) Researchers verify whether a cryptography program satisfies the constant-time criterion using various approaches, including the program counter model~\cite{agat2000transforming, molnar2005program, barthe2006preventing, kopf2007transformational, almeida2013certified, mantel2015transforming}, observation-equivalence-based noninterference~\cite{barthe2014system, almeida2016verifiable, rodrigues2016sparse, dehesa2017verifying}, and self-composition-based noninterference~\cite{almeida2013formal, almeida2016verifying, chen2017precise, antonopoulos2017decomposition, yang2018lazy, blazy2019verifying, daniel2020binsec}.
% 2) Conceptually, formally constructing high-assurance cryptography libraries shall fundamentally resolve the constant-time issues, leveraging formal languages like F$^{*}$~\cite{zinzindohoue2016verified}, HACL$^{*}$~\cite{zinzindohoue2017hacl}, Vale~\cite{bond2017vale}, Jasmin~\cite{almeida2017jasmin} and Fact~\cite{cauligi2019fact}.
% 3) Transforming existing programs into constant-time equivalents also significantly contributes to resisting side channels. For instance, some approaches~\cite{wu2018eliminating,soares2021memory} execute both real and decoy paths; Constantine~\cite{borrello2021constantine} leverages the linearization of control-flow and data-flow.

Without detailed implementation, AMD's whitepaper~\cite{amdmeasures} and Li et al.~\cite{li2022systematic} proposed countermeasures as follows, but no single software-based scheme is perfectly suited for both methodology and implementation. 
Therefore, exploring different mitigation approaches, particularly through compiler-level optimizations and combinations, offers valuable insights for improving defenses.

\begin{packed_itemize}
\item[1)] Preserving secret variables in registers instead of memory enhances security~\cite{li2022systematic}, but faces implementation challenges due to limited register availability.

\item[2)] Avoiding the reuse of fixed memory addresses ensures fresh ciphertexts~\cite{li2022systematic, amdmeasures}, but requires extra memory and precise runtime reference management, potentially leading to significant performance overhead.

\item[3)] Introducing a random nonce to the plaintext with each memory write increases ciphertext unpredictability~\cite{li2022systematic}. This includes masking and padding strategies~\cite{amdmeasures}, where padding requires extended data structures.
\end{packed_itemize}
