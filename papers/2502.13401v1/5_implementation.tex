\subsection{Implementation and Experiment Setup}
\label{sec:implementation}

\tool\ is built within the LLVM-9 framework, consisting of 4.6K lines of C++ code, but \textbf{it is not dependent on any LLVM-specific features}, making it portable to other compiler frameworks (discussed further in \S~\ref{sec:discussion}). 
% The source code is available at \url{https://anonymous.4open.science/r/CipherGuard}, 
The source code is available soon, with plans for migration to a newer LLVM version. 
Developers can easily use \tool\ by manually identifying secrets in cryptography programs and labeling them with the DFSan API, such as inserting \texttt{dfsan\_set\_label(label, key, sizeof(key))} record both the pointer and size of the secret.
Afterward, the taint analysis and mitigation processes are handled automatically without further intervention.

We evaluated 3 variants of \tool\ across several cryptography libraries, including libsodium-1.0.18, mbedTLS-3.3.0, OpenSSL-3.0.2, and WolfSSL-5.3.0. 
All experiments were performed on an AMD EPYC 7513 CPU with SEV-SNP enabled, running Ubuntu 20.04. 
For comparison with \ftool, the same library snapshots used in \ftool\ evaluations were tested. 
% To be specific, EdDSA (\textit{Ed25519}), ECDSA (\textit{secp256r1}), ECDH (\textit{X25519}) and RSA signature schemes are demonstrated to be vulnerable to ciphertext side channels in their implementations.
% Simultaneously, we incorporate symmetric primitives such as AES-GCM and ChaCha20-Poly1305 (AES and ChaCha20 for short, respectively), the hash function SHA512, and the decoding function Base64.
Default optimization levels were applied, with \texttt{-O3} for OpenSSL and \texttt{-O2} for the other libraries.
