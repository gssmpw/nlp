\section{Introduction}
\label{sec:introduction}

Data security in cloud computing has become a major concern, hindering data owners from fully leveraging Virtual Machines (VMs) offered by public service providers.
Accordingly, major processor vendors introduce a hardware-based technology known as TEE, which provides an isolated environment with memory encryption to fortify the integrity and confidentiality of VMs against privileged attackers, such as malicious hypervisors or host OS.
These TEE-enabled processors include AMD Secure Encrypted Virtualization (SEV)~\cite{kaplan2016amd} and its iterations: SEV Encrypted States (SEV-ES)~\cite{kaplan2017protecting} and SEV Secure Nested Paging (SEV-SNP)~\cite{kaplan2020amd}, Intel Software Guard Extensions (SGX)~\cite{intelsgx, johnson2021supporting}, Intel Trust Domain Extensions (TDX)~\cite{inteltdx} and ARM Confidential Compute Architecture (CCA)~\cite{armcca}.

Admittedly, existing TEEs still exhibit security weaknesses. For instance, unauthenticated encryption~\cite{buhren2017fault, du2017secure, wilke2020sevurity}, Nested Page Table (NPT) remapping~\cite{hetzelt2017security, morbitzer2019extracting, morbitzer2018severed}, unprotected I/O~\cite{li2019exploiting}, and unauthorized Address Space Identifier (ASID)~\cite{li2021crossline} exist in SEV. 
Providers release hardware upgrades such as SEV-SNP to resolve these known attacks.
However, recent works disclosed a new vulnerability in SEV-SNP, i.e., ciphertext side-channel attack~\cite{li2021cipherleaks, li2022systematic}.
Attackers leverage deterministic memory encryption, in which a consistent plaintext block at the same physical address is always encrypted into the same ciphertext block.
Hence, a one-to-one mapping between the plaintext and ciphertext can be constructed by continuously monitoring changes in the encrypted guest memory, ultimately, recovering the plaintext.
More seriously, this attack is not limited to SEV-SNP but can threaten any TEE processors utilizing deterministic memory encryption with the memory bus snooping technique~\cite{lee2020off}.

Several works are dedicated to detecting or mitigating this attack. 
\htool~\cite{deng2023cipherh} has successfully detected ciphertext side-channel vulnerabilities in many popular cryptography libraries.
It utilizes DataFlowSanitizer (DFSan)~\cite{dfsan} to efficiently mark sensitive memory writes. 
Within functions involving these marked instructions, \htool\ conducts symbolic execution and constructs formulas checked by a constraint solver to identify potential vulnerabilities. 
\ftool~\cite{wichelmann2023cipherfix} mitigates ciphertext side-channel leakage at the cost of an average performance overhead ranging from 2.4$\times$ to 16.8$\times$ in its most efficient variant.
With static binary instrumentation, \ftool\ transforms each tainted memory store into a copy with masking operations in a new section.
However, the instrumentation method for patching existing executable files has inherent limitations, such as inflexibility in layout adjustment, limited compilation resource usage, and potential challenges in maintaining code integrity and compatibility. 
\otool~\cite{wichelmann2024obelix}, the latest mitigation tool, leverages Oblivious RAM (ORAM) to execute and access enforced union code blocks, making execution flows indistinguishable from attackers. 
Nevertheless, \otool\ suffers from considerable performance overhead inherent to ORAM itself (hundreds of times).

Such being the case, we turn to a different methodology for enhancing security with higher efficiency: mitigating ciphertext side channels \textit{at the compilation stage}, which offers several advantages.
Firstly, it fixes the potential vulnerabilities directly in the code generation phase with insight and control over the compilation process.
Secondly, the compiler has full analysis capabilities over the entire program, while instrumentation is often constrained to specific locations where probes are inserted.
Thirdly, compilers can still utilize various resources and optimizations to generate efficient code during program repair. The instrumentation method, however, can only be applied to fixed executable files, facing constraints in layout adjustment and resource utilization.

To this end, we introduce \tool, an LLVM-based compiler-aided methodology for automated and efficient mitigation against ciphertext side channels. 
\tool\ captures all sensitive memory accesses and their precise memory references at the last optimized IR level by dynamic taint analysis.
At the Machine-IR level, \tool\ employs multiple variants such as software-based probabilistic encryption, which introduces a random nonce to the secret upon memory writing, and secret-aware register allocation that allows the retention of secrets in registers. 
The fixed Machine-IR is compiled into a secure executable that can run directly in existing TEEs. 
With efficient random nonce buffers management and flexible register allocation, \tool\ demonstrates a satisfactory performance overhead, averaging 1.76-3.10$\times$ across various cryptographic implementations.
This is a significant performance improvement over \ftool, highlighting the efficacy of the compiler-aided methodology in defeating ciphertext side channels.
In sum, this paper makes the following contributions:

\begin{packed_itemize}
\item For the first time, we introduce a compiler-aided methodology to address ciphertext side channels. Equipped with multiple variants, the methodology has been demonstrated to provide a more efficient and flexible solution in comparison to the binary instrumentation.

\item It presents \tool, an LLVM-based tool that harnesses dynamic taint analysis and deploys mitigation variants at the compilation stage. \tool\ excels in generating more efficient mitigated code through in-place code insertion, precise random nonce buffers management, and flexible register allocation.

\item It evaluates \tool\ in mitigating ciphertext side channels among real-world cryptographic software. \tool\ hardens all the sensitive memory accesses with superior performance compared to \ftool.
\end{packed_itemize}
