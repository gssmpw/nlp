\begin{abstract}
Cryptographic implementations bolster security against timing side-channel attacks by integrating constant-time components. 
However, the new ciphertext side channels resulting from the deterministic memory encryption in Trusted Execution Environments (TEEs), enable ciphertexts to manifest identifiable patterns when being sequentially written to the same memory address.
Attackers with read access to encrypted memory in TEEs can potentially deduce plaintexts by analyzing these changing ciphertext patterns. 

In this paper, we design \tool, a compiler-aided mitigation methodology to counteract ciphertext side channels with high efficiency and security.
\tool\ is based on the LLVM ecosystem, and encompasses multiple mitigation strategies, including software-based probabilistic encryption and secret-aware register allocation.
Through a comprehensive evaluation, we demonstrate that \tool\ can strengthen the security of various cryptographic implementations more efficiently than existing state-of-the-art defense mechanism, i.e., \ftool.
\end{abstract}

% Cryptographic implementations bolster security against timing side-channel attacks by integrating constant-time components. However, the new ciphertext side channels resulting from the deterministic memory encryption in Trusted Execution Environments (TEEs), enable ciphertexts to manifest identifiable patterns when being sequentially written to the same memory address. Attackers with read access to encrypted memory in TEEs can potentially deduce plaintexts by analyzing these changing ciphertext patterns. 

% In this paper, we design CipherGuard, a compiler-aided mitigation methodology to counteract ciphertext side channels with high efficiency and security. CipherGuard is based on the LLVM ecosystem, and encompasses multiple mitigation strategies, including software-based probabilistic encryption and secret-aware register allocation. Through a comprehensive evaluation, we demonstrate that CipherGuard can strengthen the security of various cryptographic implementations more efficiently than existing state-of-the-art defense mechanism, i.e., CipherFix.
