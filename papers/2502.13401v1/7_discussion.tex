\section{Discussion}
\label{sec:discussion}

\parh{Limitations.}
\tool\ uses dynamic taint analysis to identify potential sensitive memory access instructions and applies conservatively across multiple secrets and inputs, resulting in thorough taint detection. 
However, it can also lead to unnecessary protection of memory writes that don't actually cause ciphertext side-channel leakage, which adds runtime overhead. 
Reducing this overhead would require a precise, static whole-program taint analysis that can accurately identify true leakage points. 
Unfortunately, this remains a challenge, as current research only allows static analysis of a small portion of the program~\cite{wang2019identifying, brotzman2019casym}.

\parh{Potential Extensions.}
\tool\ provides extensive support for commonly used x86 instructions, covering data movement, arithmetic, logic operations, comparisons involving sensitive memory, and zero/flag extensions, as well as frequently used SSE/AVX vector instructions. 
This range is sufficient to mitigate ciphertext side-channel attacks in most cryptographic applications. 
Additionally, \tool\ offers a flexible interface for integrating support for instructions not yet covered, highlighting its extensibility for future security updates and architectural changes.

\parh{Compiler Compatibility.}
\tool\ is designed to be versatile and independent of specific compiler frameworks. 
For example, when using GCC, the LLVM-specific syntax can be replaced with GCC IR while keeping the same taint propagation rules. 
The tainted instructions identified during analysis are targeted for protection within the GCC IR. 
Additionally, integrating the mitigation process into GCC's register allocation phase is straightforward, reflecting \tool's design for easy adaptation to different compiler pipelines.
