\section{Evaluation}
\label{sec:evaluation}

\subsection{Implementation and Experiment Setup}
\label{sec:implementation}

\tool\ is built within the LLVM-9 framework, consisting of 4.6K lines of C++ code, but \textbf{it is not dependent on any LLVM-specific features}, making it portable to other compiler frameworks (discussed further in \S~\ref{sec:discussion}). 
% The source code is available at \url{https://anonymous.4open.science/r/CipherGuard}, 
The source code is available soon, with plans for migration to a newer LLVM version. 
Developers can easily use \tool\ by manually identifying secrets in cryptography programs and labeling them with the DFSan API, such as inserting \texttt{dfsan\_set\_label(label, key, sizeof(key))} record both the pointer and size of the secret.
Afterward, the taint analysis and mitigation processes are handled automatically without further intervention.

We evaluated 3 variants of \tool\ across several cryptography libraries, including libsodium-1.0.18, mbedTLS-3.3.0, OpenSSL-3.0.2, and WolfSSL-5.3.0. 
All experiments were performed on an AMD EPYC 7513 CPU with SEV-SNP enabled, running Ubuntu 20.04. 
For comparison with \ftool, the same library snapshots used in \ftool\ evaluations were tested. 
% To be specific, EdDSA (\textit{Ed25519}), ECDSA (\textit{secp256r1}), ECDH (\textit{X25519}) and RSA signature schemes are demonstrated to be vulnerable to ciphertext side channels in their implementations.
% Simultaneously, we incorporate symmetric primitives such as AES-GCM and ChaCha20-Poly1305 (AES and ChaCha20 for short, respectively), the hash function SHA512, and the decoding function Base64.
Default optimization levels were applied, with \texttt{-O3} for OpenSSL and \texttt{-O2} for the other libraries.


\begin{table*}[!htbp]
\centering
\caption{Performance statistics towards mitigated cryptography software with 3 variants of \tool. Results are obtained by measuring the average clock cycles using the \texttt{rdtsc} instruction. XS+ is short for XorShift128+. CC20 is short for ChaCha20.}
\label{tab:resultsoverview}
\footnotesize
\begin{tabular}{c|cc|cccc|cc|cc}
\hline
    \multirow{2}{*}{\textbf{Implementation-Orig}}
    &\multicolumn{2}{c|}{\textbf{Variant 1}}
    &\multicolumn{4}{c|}{\textbf{Variant 2}}
    &\multicolumn{2}{c|}{\textbf{Variant 3}}
    &\multicolumn{2}{c}{\textbf{RSV-SSE}}\\
    & \textbf{RDRAND} & \textbf{Factor}
    & \textbf{AES} & \textbf{Factor} & \textbf{XS+} & \textbf{Factor}
    & \textbf{REGISTER} & \textbf{Factor}
    & \textbf{RDRAND} & \textbf{Factor}\\
\hline
    libsodium-EdDSA-191618	&648861	&3.39 &	651930	&3.40 	&682159	&3.56 &	304067	&1.59 & 680723 & 3.55\\
    libsodium-SHA512-42482	&101088	&2.38 &	137298	&3.23 	&152227	&3.58 &	105061	&2.47 & 121020 & 2.85\\
\hline
    mbedTLS-AES-474926	&574695	&1.21 &	815542	&1.72 	&866192	&1.82 &	993255	&2.09 & 1135500 & 2.39\\
    mbedTLS-Base64-11220	&16877	&1.50 &	18360	&1.64 	&19642	&1.75 &	15493	&1.38 & 20820 & 1.86\\
    mbedTLS-CC20-597000	&752978	&1.26 &	1236573	&2.07 	&1385737	&2.32 &	1251150	&2.10 & 1329720 & 2.23\\
    mbedTLS-ECDH-4253887	&7680051	&1.81 &	8819828	&2.07 	&9626046	&2.26 &	6211978	&1.46 & 7981500 & 1.88\\
    mbedTLS-ECDSA-3942140	&7310694	&1.85 &	8133126	&2.06 	&8717678	&2.21 &	5448579	&1.38 & 10137960 & 2.57\\
    mbedTLS-RSA-1884568	&4314759	&2.29 &	4658550	&2.47 	&5504460	&2.92 &	2861580	&1.52 & 5459100 & 2.90\\
\hline
    OpenSSL-ECDH-986228	&6081644	&6.17 &	6132829	&6.22 	&6997656	&7.10 &	1519590	&1.54 & 6121680 & 6.21\\
    OpenSSL-ECDSA-16854915	&66816768	&3.96 &	65915307	&3.91 	&76290783	&4.53 &	34374077	&2.04 & 73640040 & 4.37\\
\hline
    WolfSSL-AES-639055	&1341064	&2.10 &	2126681	&3.33 	&2340311	&3.66 &	1897460	&2.97 & 1916760 & 3.00\\
    WolfSSL-CC20-481650	&526643	&1.09 &	507535	&1.05 	&517214	&1.07 &	537044	&1.12 & 614670 & 1.28\\
    WolfSSL-ECDH-348690	&737935	&2.12 &	731250	&2.10 	&761804	&2.18 &	522330	&1.50 & 849300 & 2.44\\
    WolfSSL-ECDSA-4226940	&20976600	&4.96 &	20141736	&4.77 	&25013569	&5.92 &	9132870	&2.16 & 21068010 & 4.98\\
    WolfSSL-EdDSA-419507	&1317104	&3.14 &	1296690	&3.09 	&1350810	&3.22 &	639840	&1.53 & 1428990 & 3.41\\
    WolfSSL-RSA-491970	&608190	&1.24 &	645980	&1.31 	&737520	&1.50 &	631806	&1.28 & 902452 & 1.83\\
\hline
    Average	&-	&2.53 &	-	&2.78 	&-	&3.10 &	-	&1.76 & - & 2.98\\
\hline
\end{tabular}
\end{table*}

\subsection{Comparison between Variants}
\label{subsec:variants}

\noindent \textbf{Performance Overhead.}
We employ the \texttt{rdtsc} instruction to accurately measure the execution cycles of both the original and patched cryptography libraries. 
Each library is tested over 200 iterations, and the average execution overhead is calculated. 
The performance statistics for the patched libraries using \tool\ are summarized in \T~\ref{tab:resultsoverview}.

Variant 1 incurs an average overhead of 2.53$\times$, ranging from 1.09$\times$ in the patched ChaCha20 of WolfSSL to 6.17$\times$ in OpenSSL's ECDH. 
This performance improvement compared to \ftool\ is primarily due to the use of a pre-generated mask buffer, particularly influenced by the impact of the \texttt{rdrand} instruction in patched cryptography software.
Variant 2 uses two methods for on-the-fly mask generation, resulting in average overheads of 2.78$\times$ and 3.10$\times$. 
These approaches slightly reduce performance compared to Variant 1's optimized \texttt{rdrand} scheme. 
The higher overhead in XorShift128+ is due to its multiple-instruction process, which takes longer than the single \texttt{vaesenc} instruction from AES-NI.
Variant 3 improves performance by keeping sensitive data in registers, achieving a lower average overhead of 1.76$\times$ compared to Variant 1's 2.53$\times$. 
However, this performance gain is not consistent across all cryptography libraries due to the trade-off in SSE register usage. While fewer mask instructions are needed, the reduced availability of SSE registers can impact other parts of the program.

\noindent \textbf{\texttt{rdrand} as the Bottleneck.}
Employing on-the-fly random number generation as nonces offers strong security but can lead to significant performance overhead, especially when relying on \texttt{rdrand}, as seen in \ftool-\textsc{Fast}, which incurs an average overhead of 16.8$\times$ and a peak of 40$\times$. 
To mitigate this, pre-generating a random nonce buffer is preferred, reducing the performance cost while maintaining robust protection.

To evaluate the performance benefits of Variant 1 of \tool\ compared to the on-the-fly method, we examined both asymmetric cryptography libraries (RSA and ECDSA) and symmetric cryptography libraries (AES and ChaCha20). 
The results show significant improvement for asymmetric libraries, averaging 6.79$\times$, while the improvement for symmetric libraries is more modest at 1.39$\times$. 
This demonstrates the effectiveness of pre-generating a nonce buffer in enhancing performance, especially for asymmetric cryptography.

% \begin{table}[htbp]
% \centering
% \caption{Performance improvement by Variant 1 over the on-the-fly \texttt{rdrand} method. CC20 is short for ChaCha20.}
% \label{tab:variant1benefit}
% \scriptsize
% \begin{tabular}{cccc}
% \hline
%     \multirow{2}{*}{\textbf{Implementation}}
%     &\multicolumn{1}{c}{\textbf{Variant 1}}
%     &\multicolumn{1}{c}{\textbf{On-the-fly \texttt{rdrand}}}
%     &\multirow{2}{*}{\textbf{Improvement}}\\
%     & \textbf{Factor} & \textbf{Factor} & \\
% \hline
%     ECDSA-mbedTLS & 1.85 & 10.87 & 5.87 \\
%     ECDSA-OpenSSL & 3.96 & 18.92 & 4.77 \\
%     ECDSA-WolfSSL & 4.96 & 18.60 & 3.75 \\
%     RSA-mbedTLS & 2.29 & 14.91 & 6.51 \\
%     RSA-WolfSSL & 1.24 & 16.21 & 13.07 \\
% \hline
%     AES-mbedTLS & 1.21 & 2.38 & 1.96 \\
%     AES-WolfSSL & 2.10 & 3.10 & 1.47 \\
%     CC20-mbedTLS & 1.26 & 1.30 & 1.03 \\
%     CC20-WolfSSL & 1.09 & 1.24 & 1.13 \\
% \hline
%     Average & 2.21 & 9.72 & 4.39 \\
% \hline
% \end{tabular}
% \end{table}

\noindent \textbf{Alternative Nonce.}
Unlike Variant 1, which selects and increments a random number for each sensitive memory write, Variant 2 generates a random nonce directly when an updated nonce is needed. 
AES-NI achieves this with a single \texttt{vaesenc} instruction, while XorShift128+ requires multiple instructions, resulting in slightly higher overhead. 
Additionally, Variant 2 requires reserving a portion of the SSE vector registers (two for AES-NI and three for XorShift128+), which can affect other parts of the cryptography library, adding more overhead compared to Variant 1, as reflected in performance results.

\noindent \textbf{The Profit of Registers.}
The primary benefit of using registers to store intermediate results is the reduced memory access overhead. 
However, the performance advantage of Variant 3 over Variant 1 is inconsistent across all cryptographic applications. 
We hypothesize that Variant 3's reliance on SSE registers may limit their availability in other program areas. 
To test this, we introduce a new strategy called ``RSV-SSE'', which combines the approach of Variant 1 with the reservation of 8 unused SSE registers. 
This strategy enables the compiler to either allocate these registers freely, as in Variant 1, or reserve them for storing secrets from sensitive memory access instructions, similar to Variant 3.

% \begin{table}[!htbp]
% \centering
% \caption{Profit analysis of variant 3.}
% \label{tab:variant3benefit}
% \scriptsize
% \begin{tabular}{ccccc}
% \hline
%     \multirow{2}{*}{\textbf{Implementation}}
%     &\multicolumn{1}{c}{\textbf{Variant 1}}
%     &\multicolumn{2}{c}{\textbf{RSV+RDRAND-OPT}}
%     &\multicolumn{1}{c}{\textbf{Variant 3}}\\
%     \cline{3-4}
%     & \textbf{Factor} & \textbf{Cycles} & \textbf{Factor} & \textbf{Factor} \\
% \hline
%     libsodium-EdDSA &3.39 &680723	&3.55 &1.59 \\
%     libsodium-SHA512 &2.38 &121020	&2.85 &2.47 \\
% \hline
%     mbedTLS-AES &1.21 &1135500	&2.39 &2.09 \\
%     mbedTLS-Base64 &1.50 &20820	&1.86 &1.38 \\
%     mbedTLS-CC20 &1.26 &1329720	&2.23 &2.10 \\
%     mbedTLS-ECDH &1.81 &7981500	&1.88 &1.46 \\
%     mbedTLS-ECDSA &1.85 &10137960	&2.57 &1.38 \\
%     mbedTLS-RSA &2.29 &5459100	&2.90 &1.52 \\
% \hline
%     OpenSSL-ECDH &6.17 &6121680	&6.21 &1.54 \\
%     OpenSSL-ECDSA &3.96 &73640040	&4.37 &2.04 \\
% \hline
%     WolfSSL-AES &2.10 &1916760	&3.00 &2.97 \\
%     WolfSSL-CC20 &1.09 &614670	&1.28 &1.12 \\
%     WolfSSL-ECDH &2.12 &849300	&2.44 &1.50 \\
%     WolfSSL-ECDSA &4.96 &21068010	&4.98 &2.16 \\
%     WolfSSL-EdDSA &3.14 &1428990	&3.41 &1.53 \\
%     WolfSSL-RSA &1.24 &902452	&1.83 &1.28 \\
% \hline
%     Average &2.53 &- &2.98 &1.76 \\
% \hline
% \end{tabular}
% \end{table}

\T~\ref{tab:resultsoverview} presents the results of the new ``RSV-SSE'' strategy, which demonstrates an average overhead of 2.98$\times$, higher than that of both Variant 1 and Variant 3. 
Certain cryptographic functions (SHA512 from libsodium, AES and ChaCha20 from mbedTLS, AES, ChaCha20 and RSA from WolfSSL) show decreased performance when SSE registers are used to safeguard sensitive memory accesses, negatively impacting other program components. 
Conversely, functions that benefit from this strategy display significant performance improvements, particularly ECDSA from OpenSSL, where Variant 3 shows exceptional profitability.

We further investigate the allocation of SSE registers for sensitive memory accesses in the ECDSA implementation from OpenSSL. By adhering to the secret-aware register allocation principles outlined in \S~\ref{subsec:registeralloc}, we analyze the range of MBBs in which these registers are assigned.
The findings indicate a significant advantage when registers are utilized to store stack memory within loops, leading to reduced cycle counts compared to using masking for protection, as seen in Variant 1. 
However, if protection extends to insufficient sensitive stack memory within a loop, such as allocating SSE registers for sequential sensitive memory accesses, the benefits may not outweigh the drawbacks of not using these registers in other areas of the cryptographic program. 
This underscores the critical role of context and algorithm specifics in evaluating the effectiveness of secret-aware register allocation.

\noindent \textbf{Variant Selection.}
All variants of \tool\ provide robust security guarantees (see \S~\ref{subsec:security}), and the choice of variant should focus on performance and usage needs. 
Variant 1 utilizes pre-generated random buffers, optimizing performance compared to on-the-fly random nonce generation. In contrast, Variant 2 is suitable for scenarios requiring dynamic nonce generation but involves the use of SSE registers, which may impact performance in cryptographic applications with frequent data movements. 
Variant 3 minimizes performance overhead but consumes more SSE registers than Variant 2. 
Thus, developers experienced in implementing cryptographic software in TEEs should consider Variants 2 and 3 for their specific contexts, while general users are recommended to opt for Variant 1.

\subsection{Comparison with \ftool}
\label{subsec:ftool}

\noindent \textbf{Dynamic Taint Analysis.}
Identifying sensitive memory access instructions in each cryptography library takes less than 20 minutes. 
Specifically, taint analysis for symmetric primitives like AES and ChaCha20 is completed in just 2 to 3 minutes. 
Although asymmetric primitives such as RSA and ECDSA, which use blinding and nonces, require more time, the taint analysis still finishes within approximately 10 minutes, which is considered a reasonable duration.

In the evaluation shown in \T~\ref{tab:comparisonftool}, we systematically document the number of functions and instructions identified by taint analysis for each cryptography library, as reflected in the fourth and fifth columns. 
The results reveal a wide range, with a minimum of 6 sub-functions in SHA512 from libsodium and a maximum of 796 sub-functions in ECDSA from OpenSSL, averaging 120 functions across the analyzed implementations. 
The tainted instructions within these functions vary significantly, from 275 in ChaCha20 of mbedTLS to 17,834 instructions in ECDSA of OpenSSL. 
In comparison to \ftool, as illustrated in the second and third columns of \T~\ref{tab:comparisonftool}, our approach demonstrates more comprehensive and detailed tracking of sensitive data.

\begin{table*}[!htbp]
\centering
\caption{Performance comparison with \ftool\ based on the same number of tainted functions. The replication of \ftool\ is conducted on its \textsc{Fast} version.}
\label{tab:comparisonftool}
\footnotesize
\resizebox{0.98\linewidth}{!}{
\begin{tabular}{ccccc|cccc|cccc}
\hline
    \multirow{2}{*}{\textbf{Implementation}}
    &\multicolumn{4}{c|}{\textbf{\ftool\quad \tool}}
    &\multicolumn{4}{c|}{\textbf{\ftool-\textsc{Fast}}}
    &\multicolumn{4}{c}{\textbf{\tool-Variant 2}}\\
    \cline{2-13}
    & \textbf{FUN} & \textbf{INS} & \textbf{FUN} & \textbf{INS}
    & \textbf{Orig} & \textbf{AES} & \textbf{Factor} & \textbf{Consume}
    & \textbf{INS} & \textbf{AES} & \textbf{Factor} & \textbf{Consume}\\
\hline
    libsodium-EdDSA &14 &616 &17 &1311 & 131790 &779580 &5.92 & 1052 & 630 &271557	&1.42	&127 \\
    libsodium-SHA512 &6 &155 &6 &586 & 60060 &103920 &1.73 & 283 & 211 &84139	&1.98	&197 \\
\hline
    mbedTLS-AES &19 &96 &17 &326 & 312990 &1165470 &3.72 & 8880 & 318 &780317	&1.64	&960 \\
    mbedTLS-Base64 &5 &25 &9 &386 & 10230 &16440 &1.61 & 248 & 26 &14897	&1.33	&141 \\
    mbedTLS-ChaCha20 &15 &234 &14 &275 & 397080 &922800 &2.32 & 2247 & 267 &1193730	&2.00	&2235 \\
    mbedTLS-ECDH &20 &65 &51 &1063 & 3314160 &10658700 &3.22 & 112993 & 171 &7760267	&1.82	&20505 \\
    mbedTLS-ECDSA &51 &448 &69 &1446 & 1385790 &10117620 &7.30 & 19491 & 1156 &7179808	&1.82	&2801 \\
    mbedTLS-RSA &35 &300 &44 &1238 & 2893260 &11347680 &3.92 & 28181 & 673 &3640590	&1.93	&2609 \\
\hline
    OpenSSL-ECDH &11 &117 &721 &12876 & 373800 &583680 &1.56 & 1794 & 123 &1280850	&1.30	&2395 \\
    OpenSSL-ECDSA &91 &653 &796 &17834 & 3236490 &6735210 &2.08 & 5358 & 991 &22131750	&1.31	&5325 \\
\hline
    WolfSSL-AES &5 &204 &6 &499 & 401550 &725040 &1.81 & 1586 & 192 &809066	&1.27	&885 \\
    WolfSSL-ChaCha20 &9 &182 &14 &404 & 706740 &1022880 &1.45 & 1737 & 343 &505800	&1.05	&70 \\
    WolfSSL-ECDH &10 &291 &20 &543 & 191280 &385440 &2.02 & 667 & 147 &701408	&2.01	&2399 \\
    WolfSSL-ECDSA &40 &328 &59 &932 & 3143010 &8558370 &2.72 & 16510 & 242 &8321686	&1.97	&16920 \\
    WolfSSL-EdDSA &31 &835 &28 &715 & 312360 &746250 &2.39 & 520 & 616 &893040	&2.13	&769 \\
    WolfSSL-RSA &48 &696 &43 &704 & 537750 &1200660 &2.23 & 952 & 529 &615510	&1.25	&234 \\
\hline
    Average &26 &328 &120 &2571 & - &- &2.87 &12656 & 415 &-	&1.64	&3661 \\
\hline
\end{tabular}}
\end{table*}

% \begin{table*}[!htbp]
% \centering
% \caption{Performance comparison with \ftool\ based on the same number of tainted functions.}
% \label{tab:comparisonftool}
% \footnotesize
% \begin{tabular}{ccccccccccccccc}
% \hline
%     \multirow{2}{*}{\textbf{Implementation}}
%     &\multicolumn{4}{c}{\textbf{\ftool\quad \tool}}
%     &\multirow{2}{*}{}
%     &\multicolumn{4}{c}{\textbf{\ftool}$\dagger$}
%     &\multirow{2}{*}{}
%     &\multicolumn{4}{c}{\textbf{\tool-Variant 2}$\ddagger$}\\
%     \cline{2-5}
%     \cline{7-10}
%     \cline{12-15}
%     & \textbf{FUN} & \textbf{INS} & \textbf{FUN} & \textbf{INS}
%     && \textbf{Orig} & \textbf{AES} & \textbf{Factor} & \textbf{Consume}
%     && \textbf{INS} & \textbf{AES} & \textbf{Factor} & \textbf{Consume}\\
% \hline
%     libsodium-EdDSA &14 &616 &17 &1311 && 131790 &779580 &5.92 & 1052 && 630 &271557	&1.42	&127 \\
%     libsodium-SHA512 &6 &155 &6 &586 && 60060 &103920 &1.73 & 283 && 211 &84139	&1.98	&197 \\
% \hline
%     mbedTLS-AES &19 &96 &17 &326 && 312990 &1165470 &3.72 & 8880 && 318 &780317	&1.64	&960 \\
%     mbedTLS-Base64 &5 &25 &9 &386 && 10230 &16440 &1.61 & 248 && 26 &14897	&1.33	&141 \\
%     mbedTLS-CC20 &15 &234 &14 &275 && 397080 &922800 &2.32 & 2247 && 267 &1193730	&2.00	&2235 \\
%     mbedTLS-ECDH &20 &65 &51 &1063 && 3314160 &10658700 &3.22 & 112993 && 171 &7760267	&1.82	&20505 \\
%     mbedTLS-ECDSA &51 &448 &69 &1446 && 1385790 &10117620 &7.30 & 19491 && 1156 &7179808	&1.82	&2801 \\
%     mbedTLS-RSA &35 &300 &44 &1238 && 2893260 &11347680 &3.92 & 28181 && 673 &3640590	&1.93	&2609 \\
% \hline
%     OpenSSL-ECDH &11 &117 &721 &12876 && 373800 &583680 &1.56 & 1794 && 123 &1280850	&1.30	&2395 \\
%     OpenSSL-ECDSA &91 &653 &796 &17834 && 3236490 &6735210 &2.08 & 5358 && 991 &22131750	&1.31	&5325 \\
% \hline
%     WolfSSL-AES &5 &204 &6 &499 && 401550 &725040 &1.81 & 1586 && 192 &809066	&1.27	&885 \\
%     WolfSSL-CC20 &9 &182 &14 &404 && 706740 &1022880 &1.45 & 1737 && 343 &505800	&1.05	&70 \\
%     WolfSSL-ECDH &10 &291 &20 &543 && 191280 &385440 &2.02 & 667 && 147 &701408	&2.01	&2399 \\
%     WolfSSL-ECDSA &40 &328 &59 &932 && 3143010 &8558370 &2.72 & 16510 && 242 &8321686	&1.97	&16920 \\
%     WolfSSL-EdDSA &31 &835 &28 &715 && 312360 &746250 &2.39 & 520 && 616 &893040	&2.13	&769 \\
%     WolfSSL-RSA &48 &696 &43 &704 && 537750 &1200660 &2.23 & 952 && 529 &615510	&1.25	&234 \\
% \hline
%     Average &26 &328 &120 &2571 && - &- &2.87 &12656 && 415 &-	&1.64	&3661 \\
% \hline
% \end{tabular}
% \vspace{-10pt}
% \end{table*}

\noindent \textbf{Mitigation Coverage Comparison.}
Accurate identification of tainted instructions is essential, as these instructions are key candidates for the subsequent mitigation process. 
Although \tool\ and \ftool\ use different compilers, Clang and GCC, both maintain consistent control flow when subjected to the same optimization settings. 
Capitalizing on this similarity, we extract binaries with instrumentation information from the taint analysis stage and analyze the basic blocks of tainted functions by constructing control-flow graphs (CFGs). 
For instance, \F~\ref{fig:cfg} compares the CFGs of the function \texttt{SHA512\_Transform} from libsodium-SHA512, with the left sub-figure representing the CFG from \tool\ and the right from \ftool. 
The CFGs exhibit notable structural similarities, and we manually identify three key nodes to further refine the control flow. 
A line-by-line comparison of the code within both binaries reveals only minor differences, averaging 2\%. 
Thus, we conclude that the variance in compilers does not significantly affect the discrepancy in the number of tainted instructions identified by \tool\ and \ftool\ as shown in \T~\ref{tab:comparisonftool}.

Our analysis reveals that the cryptography libraries listed in \T~\ref{tab:comparisonftool} are monitored for a similar number of tainted functions, with the exception of two implementations from the OpenSSL libraries. 
To investigate the variance in taint analysis among these libraries, we randomly select five tainted functions for comparison. 
We find that the additional instructions identified by \tool\ can be categorized into several types.
Notably, \tool\ conservatively tags the parameters of temporary variables used to store intermediate results, such as loop variables involved in sensitive calculations. 
In contrast, \ftool\ overlooks these intermediate and resultant variables, focusing primarily on directly involved secret keys.
Consequently, in the OpenSSL libraries, \tool\ identifies not only the parameters of temporary variables but also intermediate and resultant variables, resulting in more comprehensive identification of relevant functions and instructions compared to \ftool, as illustrated in \T~\ref{tab:comparisonftool}.
% We highlight that the comparison of taint coverage serves merely to demonstrate the superior performance of \tool\, which covers a significantly larger number of instructions than \ftool. The focus of this paper is not on the taint analysis itself. Hence, we refrain from delving into the detailed implementation of the taint analysis component of both \tool\ and \ftool.

\begin{figure}[!t]
\centering
\includegraphics[width=\linewidth]{Fig_cfg.pdf}
\caption{Function \texttt{SHA512\_Transform} from the libsodium-SHA512 serves as an example to illustrate the construction of CFGs and identify critical nodes.}
\label{fig:cfg}
\end{figure}

\noindent \textbf{Fair Performance Comparison.}
To ensure a fair performance comparison between \tool\ and \ftool, we evaluate both tools using as many common tainted instructions as possible within the same cryptography library. 
However, achieving this is challenging due to the variability in the number of tainted functions and instructions identified by each tool. For instance, protecting tainted instructions within loops can lead to increased execution cycles. 
To mitigate this issue, we manually select tainted functions that are common to both tools, allowing \tool\ to apply its mitigation process to functions flagged by \ftool. 
This approach helps align the execution flow between the two tools. Nevertheless, \tool\ still identifies more instructions than \ftool, although we strive to make the comparison as close as possible.

The details of selected tainted instructions are presented in the fourth-to-last columns of \T~\ref{tab:comparisonftool}. 
Both \tool\ and \ftool\ utilize the \texttt{vaesenc} instruction from AES-NI to insert masking instructions, thereby eliminating the influence of any optimization specific to \tool. 
The results indicate that \tool\ achieves an average overhead of 1.64$\times$, whereas \ftool\ experiences a higher overhead of 2.87$\times$. 
To quantify the additional average cycles incurred by the patched cryptography library, we introduce a metric labeled ``consume'', which shows an average consumption of 3,661 cycles for \tool\ compared to a significant 3.46$\times$ increase for \ftool.

The performance differences between \tool\ and \ftool\ are largely attributed to specific characteristics of \ftool\ that involve frequent jumps to instrumentation code, monitoring of \texttt{malloc} for heap memory allocation of nonce buffers, and the runtime initialization of the nonce cache. 
In contrast, \tool\ employs a compiler-aided approach that optimizes these time-intensive tasks by sequentially inserting mask instructions and efficiently managing nonce buffers in the \texttt{.bss} section. 
This design choice significantly enhances the mitigation process, reducing associated overhead and improving overall performance.

\subsection{Security Analysis}
\label{subsec:security}

\begin{figure}[htbp]
\centering
\includegraphics[width=\linewidth]{Fig_entropy.pdf}
\caption{Scatter distribution of masked $pbit$ under different variants. Each secret sequence comprises 512 values.}
\label{fig:entropy}
\end{figure}

\parh{Quality of Nonce.}
The effectiveness of our software-based probabilistic encryption scheme relies heavily on the quality of random number generation, as the reliability and randomness of nonces are crucial. 
Using the ECDSA Montgomery ladder algorithm from OpenSSL as an example (\F~\ref{fig:background}(a)), we examine how secrets are protected by masking operations in Variant 1 and Variant 2 of \tool\ with three different random number generation techniques.
Without protection, a vulnerability at the point where $pbit \leftarrow pbit \wedge kbit$ spills the secret $pbit$ into memory, allowing attackers to infer each bit of the secret by mapping ciphertext-plaintext pairs.
To assess the impact of our masking method, we disassemble the binary to locate the vulnerability, then use Pintool~\cite{luk2005pin} to observe the execution context at this point.
By running the cryptography library twice with different ECDSA private keys, we track the writing of $pbit$ 512 times within the loop. 
We then measure the entropy of the collected secret sequence as $\mathit{H(X)} = - \sum_{x \in \mathcal{X}} p(x) log_{2} p(x)$ to quantify the distribution changes and evaluate the scheme’s effectiveness.

The entropy analysis of six different unprotected secrets (0.99946, 0.99911, 0.99972, 0.99910, 0.99841, and 0.99814) shows values around 0.99 for each, indicating a significant leakage of information, as each bit can only be 0 or 1. 
This low entropy allows attackers to easily infer patterns in the secret sequences via ciphertext side-channel attacks. 
However, after applying Variant 1 and Variant 2 of \tool, the entropy of the secret sequences increases to approximately 9.0, close to the theoretical maximum for 512 random numbers. 
This higher entropy reflects that the secrets become random and unpredictable, making it extremely difficult for attackers to deduce any patterns. 
The scatter distribution of these masked secret sequences in \F~\ref{fig:entropy} further confirms the effectiveness of \tool's masking approach in preventing leakage.

% \begin{table}[htbp]
% \centering
% \caption{Entropy of the secret distribution under different variants. For each variant, we run the cryptography library twice to ensure that the original secrets are different.}
% \label{tab:entropy}
% \scriptsize
% \begin{tabular}{ccccc}
% \hline
%     \multirow{2}{*}{\textbf{Secret}}
%     &\multirow{2}{*}{\textbf{Orig}}
%     &\multirow{2}{*}{\textbf{RDRAND}}
%     &\multirow{2}{*}{\textbf{AES}}
%     &\multirow{2}{*}{\textbf{XS+}}\\
%     & & & &\\
% \hline
%     secret1 & 0.99946 & 9.0 & - & -\\
%     secret2 & 0.99911 & 9.0 & - & -\\
% \hline
%     secret3 & 0.99972 & - & 9.0 & -\\
%     secret4 & 0.99910 & - & 9.0 & -\\
% \hline
%     secret5 & 0.99841 & - & - & 9.0\\
%     secret6 & 0.99814 & - & - & 9.0\\
% \hline
% \end{tabular}
% \end{table}

\parh{Locations of Nonce Buffers.}
In \tool, the nonce buffers contain both pre-generated random numbers and currently used nonces, stored in the \texttt{.bss} section. 
The security of masking operations across all variants relies on the integrity and confidentiality of these nonce buffers to prevent leakage. 
While accessing these buffers may create execution traces that could be exploited, their security is protected by Address Space Layout Randomization (ASLR), which loads the \texttt{.bss} section into different memory locations each time. 
Attackers would need to exploit memory vulnerabilities in the software, which is not the main focus of \tool's protections. 
Thus, \tool\ provides similar security guarantees as \ftool\ for nonce buffers.

Furthermore, even if ASLR is bypassed~\cite{jang2016breaking}, the security of the nonce buffers remains robust due to SEV memory encryption, which ensures that these buffers are always encrypted.
The pre-generated nonce buffer stays consistent, leading to no variation in its ciphertext, while the currently used nonce is updated with each use during masking operations, resulting in unpredictable ciphertext.

\parh{Re-use of Nonces.}
In \tool, the nonce selection is based on the address of the target secret, leading to nonce reuse for secrets at the same address. When a function repeatedly calls the same callee, the stack frames share the same address space. 
However, \tool\ maintains security through two key mechanisms: varying parameters that keep ciphertext unpredictable under SEV and a masking process that safeguards the callee against information leakage. 
As a result, an attacker can only deduce that the same secret is being used by noticing identical ciphertext sequences.

\parh{Increment-by-Three in Nonces.}
Updating nonces in Variant 1 is essential for ensuring \tool's security. 
When consecutive keys differ significantly, masked secrets produce unpredictable ciphertexts. 
However, in cases where the key changes by only one bit, a weak nonce update (e.g., incrementing by one) can expose this change through identical masked plaintexts. 
For example, as illustrated in \F~\ref{fig:increment}, observing masked plaintexts starting from even and odd nonces can leak up to four key bits, weakening security.

\begin{figure}[htbp]
\centering
\includegraphics[width=\linewidth]{Fig_Increment.pdf}
\caption{A corner case arises when a one-bit change in the secret can be revealed by observing identical masked plaintext across multiple masking.}
\label{fig:increment}
\end{figure}

This weak nonce update scheme can be resolved by incrementing the nonce by three, ensuring that at least two bits of the masked plaintext differ between updates. 
This approach blocks identical ciphertexts from revealing bit changes while maintaining a similar performance.
