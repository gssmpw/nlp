\section{Related Work}
\label{sec:related}

Besides ciphertext side channels, timing and cache-based side-channel attacks also incur severe threats to cryptography applications~\cite{percival2005cache, osvik2006cache, zhang2012cross, yarom2014flush, liu2015last, yarom2014recovering, ryan2019return, aranha2020ladderleak}. 
To resist side-channel leakage, the constant-time was proposed.
Prior works abiding by this concept to defeat side-channel attacks are categorized into the following aspects.

\yz{not very relevant, please merge "related work" with "background"}

\noindent \textbf{Constant-time Verifying.}
A series of works~\cite{agat2000transforming, molnar2005program, barthe2006preventing, kopf2007transformational, almeida2013certified, mantel2015transforming} applied a policy called program counter, which balances the timing behavior of both branches.
Along with noninterference to verify the constant-time feature, VirtualCert~\cite{barthe2014system} and MEE-CBC~\cite{almeida2016verifiable} performed typing analysis in the CompCert~\cite{leroy2006formal}, enforcing the notion of noninterference that verifies the classic observation-equivalence.
Similarly, FlowTracker~\cite{rodrigues2016sparse} tracked the information-flow in the SSA form of LLVM programs to verify the noninterference.
Later, Dehesa-Azuara et al.~\cite{dehesa2017verifying} relaxed the noninterference into resource-aware metrics, where the resource can be execution time or cache behaviors.
In addition to verifying noninterference by equivalence, the reduced-based methods through the self-composition of programs or formulas were applied in~\cite{almeida2013formal, almeida2016verifying, chen2017precise, antonopoulos2017decomposition, yang2018lazy, blazy2019verifying, daniel2020binsec}.

\noindent \textbf{Constant-time Supplying.}
Conceptually, formally constructing high-assurance cryptography libraries shall fundamentally resolve the constant-time issues. 
Some work~\cite{zinzindohoue2016verified, zinzindohoue2017hacl} supplied verified cryptography libraries at the source level in F$^{*}$ language~\cite{swamy2016dependent}.
Differently, \cite{zinzindohoue2016verified} enforced a coding discipline for mitigating side channels by type-checking, while \cite{zinzindohoue2017hacl} proved freedom of timing side channels through manual pre- and post-conditions that at large rely on experts. 
Vale~\cite{bond2017vale} impelled the progress of constructing side-channel freedom of a cryptography program to an assembly-like code, i.e., Vale language. 
Further, Jasmin~\cite{almeida2017jasmin} and Fact~\cite{cauligi2019fact} were compiler-based formal frameworks that respectively transform Jasmin programs into assembly code, and timing-sensitive FaCT code into constant-time LLVM IR.

\noindent \textbf{Constant-time Repairing.}
Transforming existing programs into constant-time equivalents also plays a significant role in resisting side
channels. 
Prior works~\cite{agat2000transforming, kopf2007transformational, barthe2006preventing, molnar2005program, coppens2009practical} balanced the secret-dependent conditional branches by inserting dummy statements or paths.
Nevertheless, all these works only considered branches that are under the program counter security~\cite{molnar2005program}.
This means the branches of a balanced time property may still leak the secrets with more powerful attacks such as
cache-based attacks. 
Later, Barthe et al.~\cite{barthe2020formal} modified CompCert to enable it to capture constant-time during the compilation by proving the observational noninterference. 
Recently, SC-Eliminator~\cite{wu2018eliminating} removed secret-dependent branches by executing both real and decoy paths. 
Considering the unmodified addresses in the decoy path that may result in out-of-bounds memory accesses in~\cite{wu2018eliminating}, Soares and Pereira~\cite{soares2021memory} guaranteed memory-safe accesses in the decoy paths. 
Lastly, Borrello et al.~\cite{borrello2021constantine} proposed the linearization of control-flow and data-flow, which executes all possible code/data memory accesses.

Although the majority of cryptography applications are equipped with the constant-time feature, they cannot defeat the ciphertext side channels~\cite{li2021cipherleaks, li2022systematic, deng2023cipherh}.
Existing tools that fix cache-based side channels cannot be adapted to this scenario due to their distinct mitigation strategies.
Thus, defeating ciphertext side channels requires revisiting the root cause of this threat and developing new mitigation tools.
