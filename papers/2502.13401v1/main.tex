%%%%%%%%%%%%%%%%%%%%%%%%%%%%%%%%%%%%%%%%%%%%%%%%%%%%%%%%%%%%%%%%%%%%%%%%%%%%%%%%
% Template for ASPLOS papers.
%
% History:
% 
% ASPLOS originally used jpaper.cls for submission but required acmart.cls for the
% final camera-ready version. To avoid a change in format, starting ASPLOS 2024 Fall 
% cycle, both the submission and the camera-ready versions started using acmart.cls.
%
%%%%%%%%%%%%%%%%%%%%%%%%%%%%%%%%%%%%%%%%%%%%%%%%%%%%%%%%%%%%%%%%%%%%%%%%%%%%%%%%%%

% use the base acmart.cls version 1.92
% use the sigplan proceeding template with the default 10 pt fonts
% nonacm option removes ACM related text in the submission. 
\documentclass[nonacm,sigplan]{acmart}

% enable page numbers
\settopmatter{printfolios=true}

% make references clickable 
\usepackage{amsmath}
\usepackage{mathtools}
\usepackage{mathpartir}
\usepackage{caption}
\usepackage{subcaption}
\usepackage{listings}
\usepackage{multirow}
\usepackage[ruled,linesnumbered]{algorithm2e}
% \usepackage{enumitem}
\usepackage{threeparttable}
% \usepackage[compact]{titlesec}
\usepackage{pifont}
\usepackage{bbding}
\usepackage{pdfpages}

\usepackage{hyperref}
\hypersetup{
  colorlinks   = true,    % Colours links instead of ugly boxes
  urlcolor     = olive,    % Colour for external hyperlinks
  linkcolor    = red,    % Colour of internal links
  citecolor    = red      % Colour of citations
}

\newenvironment{packed_itemize}{
\begin{list}{\labelitemi}{\leftmargin=1em}
 \setlength{\labelwidth}{3em}
 \setlength{\parskip}{2pt}
 \setlength{\parsep}{2pt}
}{\end{list}}

%\newcommand{\parh}[1]{\smallskip\noindent\textbf{#1}}
\newcommand{\parh}[1]{\noindent\textbf{#1}}
%\newcommand{\sparh}[1]{\smallskip\noindent\underline{#1}}
\newcommand{\sparh}[1]{\noindent\underline{#1}}

\newcommand{\F}{Fig.}
\newcommand{\E}{Eq.}
\newcommand{\T}{Table}
\renewcommand{\S}{Sec.}
\newcommand{\A}{Alg.}

\newcommand{\tool}{\textsc{CipherGuard}}
\newcommand{\otool}{\textsc{Obelix}}
\newcommand{\ftool}{\textsc{CipherFix}}
\newcommand{\htool}{\textsc{CipherH}}

\newcommand{\fixme}[1]{{\color{blue}{#1}}} 
\newcommand{\sw}[1]{{\color{magenta}{SW:}\emph{\color{magenta}{#1}}}}
\newcommand{\yz}[1]{{\color{red}{YZ:}\emph{\color{red}{[#1]}}}}

\begin{document}

\title{CipherGuard: Compiler-aided Mitigation against Ciphertext Side-channel Attacks}

%\author{...} % removed for anonymity
\author{Ke Jiang}
\affiliation{
  \institution{Nanyang Technological University}
  \city{}
  \country{}}
\email{ke006@e.ntu.edu.sg}

\author{Sen Deng}
\affiliation{
  \institution{Hong Kong University of Science and Technology}
  \city{}
  \country{}}
\email{sdengan@cse.ust.hk}

\author{Yinshuai Li}
\affiliation{
  \institution{Southern University of Science and Technology}
  \city{}
  \country{}}
\email{12231139@mail.sustech.edu.cn}

\author{Shuai Wang}
\affiliation{
  \institution{Hong Kong University of Science and Technology}
  \city{}
  \country{}}
\email{shuaiw@cse.ust.hk}

\author{Tianwei Zhang}
\affiliation{
  \institution{Nanyang Technological University}
  \city{}
  \country{}}
\email{tianwei.zhang@ntu.edu.sg}

\author{Yinqian Zhang}
\affiliation{
  \institution{Southern University of Science and Technology}
  \city{}
  \country{}}
\email{yinqianz@acm.org}

\begin{abstract}

% Recent works to jointly reconstruct 3D human and object from a single RGB image, are mostly model-based, that fail to capture the fine details of the clothed human body and object surface. In this paper, we introduce ReCHOR, a novel, model-free, first-method to produce realistic clothed human-object reconstructions from a monocular view. This is extremely challenging due to human-object occlusions, diverse interactions and depth ambiguity, as it needs to infer both 3D spatial awareness and high resolution details. Our core idea is based on estimating neural implicit representations for human and object respectively by an attention-based neural implicit model that attends to pixel-aligned features from both the global human-object image for spatial awareness and  the local separate view of human and object images for high quality details. Additionally, the network is conditioned on semantic features from an initial estimated human-object pose prior and a generative diffusion model that inpaints occluded regions, thus enabling the retrieval of details from them.
% We also propose a synthetic dataset with rendered scenes of diverse, inter-occluded 3D human and object scans, to train our network. We evaluate our method on the synthetic and real world BEHAVE dataset. Our experiments show that our method outperforms the SOTA in achieving realistic clothed human-object reconstructions.
Recent approaches to jointly reconstruct 3D humans and objects from a single RGB image represent 3D shapes with template-based or coarse models, which fail to capture details of loose clothing on human bodies. In this paper, we introduce a novel implicit approach for jointly reconstructing realistic 3D clothed humans and objects from a monocular view. For the first time, we model both the human and the object with an implicit representation, allowing to capture more realistic details such as clothing. This task is extremely challenging due to human-object occlusions and the lack of 3D information in 2D images, often leading to poor detail reconstruction and depth ambiguity. To address these problems, we propose a novel attention-based neural implicit model that leverages image pixel alignment from both the input human-object image for a global understanding of the human-object scene and from local separate views of the human and object images to improve realism with, for example, clothing details. Additionally, the network is conditioned on semantic features derived from an estimated human-object pose prior, which provides 3D spatial information about the shared space of humans and objects. To handle human occlusion caused by objects, we use a generative diffusion model that inpaints the occluded regions, recovering otherwise lost details. For training and evaluation, we introduce a synthetic dataset featuring rendered scenes of inter-occluded 3D human scans and diverse objects. Extensive evaluation on both synthetic and real-world datasets demonstrates the superior quality of the proposed human-object reconstructions over competitive methods.
\end{abstract}

\maketitle % should come after the abstract
\pagestyle{plain} % should come right after \maketitle

% no keywords
\section{Introduction}\label{sec:intro}

In computational finance, Monte Carlo simulations are used extensively to estimate the expected value of financial payoffs based on the solution of stochastic differential equations (SDEs) which model the evolution of stock prices, interest rates, exchange rates and other quantities \cite{glasserman04}.  Monte Carlo methods are very general and flexible, but for high accuracy it requires generating a large number of costly SDE path approximations, which has motivated research into a number of variance reduction or, equivalently, cost reduction techniques. One such method is
Multilevel Monte Carlo (MLMC), which was proposed in \cite{GILES2008} and was adapted for various applications that are summarised in \cite{Giles_overview17} and successfully combined with other methods such as quasi-Monte Carlo methods. The main idea of MLMC is to approximate the payoff using different time stepping resolutions when numerically solving the underlying SDE and to generate an optimal number of samples on each level, such that the overall computational cost is minimised subject to the desired bound on the variance. %, such that the total computational cost is minimised. 
The computational savings come from the fact that most samples are computed on the coarser levels and hence are less expensive while only a few samples from the finest levels are required \cite{GILES2008}.


Among the directions in which the computational cost 
of MLMC methods could further be reduced, an important avenue is the use of lower precision calculations, especially for the first Monte Carlo levels where the targeted accuracy is relatively low. 
 An overview of the research on mixed precision for the standard Monte Carlo (MC) framework is provided in \cite{ChowMixedPrecisionStandardMC} but only a few references study the potential of low precision computation in the MLMC framework \cite{Rounding_error_oliver}. To the best of our knowledge, the only MLMC framework with customised precision in the literature is \cite{brugger2014mixed}, but they use a uniform precision for all operations on each Monte Carlo level instead of optimising 
 the precision of each intermediary variable to reduce as much as possible the cost of path generation.
 
An important motivation for an MLMC framework with variable precision would be performing the low precision computations on reconfigurable hardware devices such as Field Programmable Gate Arrays (FPGAs). FPGAs contain customizable logic blocks and connectors that make it easy to adapt the digital circuit architecture for a specific application, leading to a highly parallel and optimised implementation. Therefore they are successfully exploited in applications that require high speed and have high computational workload, such as signal processing \cite{woods2008fpga}, and real time applications like high frequency trading \cite{HFT1,HFT2}. That is why a number of previous works in hardware architecture design implemented the MLMC algorithm to price financial options using FPGAs as accelerators, which resulted in improved speed and power efficiency compared to full CPU architectures \cite{Schryver2013AMM}. The paper \cite{lindsey2016domain} also proposed 
a Domain Specific Language to automate the configuration of FPGAs for this specific application. However, only \cite{brugger2014mixed} proposed a heuristic to reduce the precision in calculations.

In addition, all aforementioned works considered that the random number generation (RNG) is performed in single or double precision. Yet in most cases an important portion of the workload in the overall MLMC simulation comes from the RNG and in \cite{brugger2014mixed} this limited the total computational savings.
To reduce the cost of MLMC simulations in particular those based on the Geometric Brownian Motion (GBM), \cite{approximateICDF_Oliver, NestedOliver} have proposed to use approximate random numbers that are generated by applying an approximation of the inverse CDF to uniform random numbers. In \cite{NestedOliver}, the authors proposed a way to integrate these lower precision random variables into a \textit{nested} MLMC framework and completed a numerical analysis to bound the resulting error at each MC level by a product of the time step and the error in the random number approximation. The same authors show in \cite{approximateICDF_Oliver} that using approximate random variables reduces the cost of path generation by a factor 7.


In this paper we propose a nested MLMC framework that combines the use of approximate random normal variables and lower precision calculations to reduce the computational cost of MLMC even further than \cite{brugger2014mixed,NestedOliver}. We illustrate the efficiency of our framework in Matlab, after making several assumptions on the cost of operations and size of the errors that we carefully justify. We focus on the case of GBM and use the approximate RNG methods presented in \cite{approximateICDF_Oliver} as well as a new slightly modified method that combines CDF inversion and the central limit theorem. To choose the precision of the variables in the low precision path generation, we introduce a novel method to optimise the bit-widths. This optimisation is performed before the main path generation loop is executed and is based on a linear model of the payoff error  
due to rounding when computing in low precision. The error model relies on algorithmic differentiation in a similar manner to \cite{unifying-bwoptim,bitwidth-AD,ADAPT}. The bit-width optimisation procedure can be performed off-line, so this stage can be excluded from the on-line time complexity of our framework. The user specified desired accuracy is then enforced by calculating on-line the number of samples that need to be generated.

In terms of hardware design, we suggest implementing the low precision path generation on FPGAs and the full-precision ones on a CPU or GPU. 
The FPGA offers enough flexibility to define a separate bit-width for every variable in the low precision path generation, and can be reconfigured periodically to update the bit-widths when the market parameters have changed considerably. 


The paper is organized as follows : \Cref{sec:MLMC} introduces MLMC and nested MLMC to make clear the estimator that is implemented in our framework. Then in \Cref{sec:RNG} we detail the methods that could be used to obtain approximate random normally distributed numbers very cheaply for the low precision path generation. In \Cref{sec:error_model} and \Cref{sec:costModel} we propose an error model and a cost model (resp.) that we then use to formulate the optimisation problem that is solved to obtain the optimal bit-widths of fixed point variables in \Cref{sec:optimisation}. Finally we summarise our results and future directions in \Cref{sec:conclusion}.



\section{Background}
\label{sec:background}

\begin{figure*}[htbp]
\centering
\includegraphics[width=\textwidth]{Fig_background.pdf}
\caption{Ciphertext side-channel examples and revisiting vulnerabilities from the perspective of compilation.}
\label{fig:background}
\end{figure*}

\subsection{Ciphertext Side-Channel Attacks}
\label{subsec:ciphertext}

The ciphertext side channel originates from the deterministic memory encryption implemented in AMD's TEE.
The encrypted memory is calculated by an XOR-Encrypt-XOR (XEX) mode, expressed as: $c = ENC(m \oplus T(P_{m})) \oplus T(P_{m})$, where the plaintext $m$ undergoes the XOR operations before and after AES-128 encryption with a tweak value $T(P_{m})$ that incorporates the physical address $P_{m}$.
Without freshness in the encryption process, the encryption of the same plaintext at a given physical address produces the identical ciphertext.
It is crucial to acknowledge that this vulnerability extends to other deterministic encryption-based TEE architectures as long as attackers have read accesses to ciphertext (via software access~\cite{li2021cipherleaks} or memory bus snooping~\cite{lee2020off}).

% \begin{figure}[htbp]
% \vspace{-5pt}
% \begin{minipage}[c]{0.5\linewidth}
%     \begin{subfigure}[b]{\linewidth}
%     \centering
%     \footnotesize
%     \begin{tabular}{l}
%         1: pbit $\leftarrow$ 1;\\
%         2: \textbf{for}\ i $\leftarrow$ cardinality\_bit - 1\ downto\ 0$\lbrace$\\
%         3: $\quad$ kbit $\leftarrow$ BN\_is\_bit\_set(k, i) $\wedge$ pbit;\\
%         4: $\quad$ EC\_POINT\_CSWAP(kbit, r, s, ...);\\
%         5: $\quad$ ...\\
%         6: $\quad$ pbit $\leftarrow$ pbit $\wedge$ kbit;$\rbrace$\\
%     \end{tabular}
%     \caption{ossl\_ec\_scalar\_mul\_ladder.}
%     \label{fig:channel1}
%     \end{subfigure}
% \end{minipage}
% \hspace{15pt}
% \begin{minipage}[c]{0.4\linewidth}
%     \begin{subfigure}[b]{0.9\linewidth}
%     \centering
%     \footnotesize
%     \begin{tabular}{l}
%         1: \textbf{for}\ i $\leftarrow$ 0\ to\ nwords - 1$\lbrace$\\
%         2: $\quad$ t $\leftarrow$ (a.d[i] $\wedge$ b.d[i])\\
%         3: $\quad \quad \quad$ \&\ condition;\\
%         4: $\quad$ a.d[i] $\leftarrow$ a.d[i] $\wedge$ t;\\
%         5: $\quad$ b.d[i] $\leftarrow$ b.d[i] $\wedge$ t;$\rbrace$\\
%     \end{tabular}
%     \caption{BN\_constant\_swap.}
%     \label{fig:channel2}
%     \end{subfigure}
% \end{minipage}
% \caption{Ciphertext side-channel examples.}%\yz{change font in figures.}
% \label{fig:channels}
% \vspace{-5pt}
% \end{figure}

Two attack schemes are introduced in~\cite{li2022systematic}.
The \textit{Dictionary} attack involves the continuous monitoring of the ciphertext at a fixed memory address to construct a dictionary containing mappings of ciphertext-plaintext pairs.
Consider the code snippet shown in \F~\ref{fig:background}(a), extracted from the ECDSA Montgomery ladder algorithm implemented in OpenSSL-3.0.2.
In each loop iteration, the \texttt{BN\_is\_bit\_set} function (denoted by $k_{i}$ in line 3) is utilized to obtain one bit of the secret $k$.
Following this, the $kbit$ variable is computed through an XOR operation with the value in $pbit$, which is then written back to $pbit$.
Given the dual XOR operations in lines 3 and 6, $pbit$ ultimately stores each bit of the secret $k$.
The attacker records consecutive ciphertext pairs ($pbit$-$kbit$) both before and after the \texttt{BN\_is\_bit\_set} function, aiming to deduce $k_{i}$ in each iteration based on the changes observed in ciphertext pairs.
In contrast, the \textit{Collision} attack focuses on identifying repetitions or alterations in certain ciphertexts to break the constant-time mechanism.
\F~\ref{fig:background}(b) shows the constant-time-swap function \texttt{BN\_constant\_swap}.
This function takes two variables $a$ and $b$, along with a decision $C$ (e.g., $kbit$ in line 4 of \F~\ref{fig:background}(a)).
If $C$ is set to 1, the values of $a$ and $b$ are exchanged, leading to observable changes in the ciphertext. Conversely, if $C$ is 0, the ciphertext remains unaltered.
In this way, the \textit{Collision} attack recovers the decision $C$, undermining the constant-time component.

Currently, many well-known cryptographic applications are vulnerable to this attack, including RSA and ECDSA (such as \textit{secp256k1} and \textit{secp384r1}) equipped with constant-time algorithms, ECDSA from WolfSSL-5.3.0, ECDSA and RSA from MbedTLS-3.1.0, as well as EdDSA (\textit{Ed25519}) from OpenSSH adopted by Ubuntu LTS 20.04~\cite{li2021cipherleaks, li2022systematic}.

\subsection{Countermeasures to Ciphertext Side-channels}
\label{subsec:countermeasures}

Hardware-based countermeasures provide stronger security by eliminating ciphertext side channels, but they require extensive validation before chip manufacturing. In contrast, we choose a software-based approach, enabling quicker implementation and deployment without modifying hardware.
Unfortunately, existing countermeasures for cache and timing side channels~\cite{percival2005cache, osvik2006cache, zhang2012cross, yarom2014flush, liu2015last, yarom2014recovering, ryan2019return, aranha2020ladderleak}, like constant-time cryptography, cannot mitigate ciphertext side channels. While constant-time cryptography avoids secret-dependent branches and memory accesses, it has been shown to be ineffective against ciphertext side-channel attacks~\cite{li2021cipherleaks, li2022systematic, deng2023cipherh}.

% Previous efforts adhering to this concept can be categorized into three classes. 
% 1) Researchers verify whether a cryptography program satisfies the constant-time criterion using various approaches, including the program counter model~\cite{agat2000transforming, molnar2005program, barthe2006preventing, kopf2007transformational, almeida2013certified, mantel2015transforming}, observation-equivalence-based noninterference~\cite{barthe2014system, almeida2016verifiable, rodrigues2016sparse, dehesa2017verifying}, and self-composition-based noninterference~\cite{almeida2013formal, almeida2016verifying, chen2017precise, antonopoulos2017decomposition, yang2018lazy, blazy2019verifying, daniel2020binsec}.
% 2) Conceptually, formally constructing high-assurance cryptography libraries shall fundamentally resolve the constant-time issues, leveraging formal languages like F$^{*}$~\cite{zinzindohoue2016verified}, HACL$^{*}$~\cite{zinzindohoue2017hacl}, Vale~\cite{bond2017vale}, Jasmin~\cite{almeida2017jasmin} and Fact~\cite{cauligi2019fact}.
% 3) Transforming existing programs into constant-time equivalents also significantly contributes to resisting side channels. For instance, some approaches~\cite{wu2018eliminating,soares2021memory} execute both real and decoy paths; Constantine~\cite{borrello2021constantine} leverages the linearization of control-flow and data-flow.

Without detailed implementation, AMD's whitepaper~\cite{amdmeasures} and Li et al.~\cite{li2022systematic} proposed countermeasures as follows, but no single software-based scheme is perfectly suited for both methodology and implementation. 
Therefore, exploring different mitigation approaches, particularly through compiler-level optimizations and combinations, offers valuable insights for improving defenses.

\begin{packed_itemize}
\item[1)] Preserving secret variables in registers instead of memory enhances security~\cite{li2022systematic}, but faces implementation challenges due to limited register availability.

\item[2)] Avoiding the reuse of fixed memory addresses ensures fresh ciphertexts~\cite{li2022systematic, amdmeasures}, but requires extra memory and precise runtime reference management, potentially leading to significant performance overhead.

\item[3)] Introducing a random nonce to the plaintext with each memory write increases ciphertext unpredictability~\cite{li2022systematic}. This includes masking and padding strategies~\cite{amdmeasures}, where padding requires extended data structures.
\end{packed_itemize}

\section{Methodology Overview}
\label{sec:overview}

\subsection{Threat Model}
\label{subsec:threat}

We adopt the same threat model as prior works~\cite{li2021cipherleaks,li2022systematic}, where a privileged attacker aims to steal secrets from cryptographic programs running in a confidential VM (e.g., AMD SEV-SNP). The attacker has full system privileges and can read the entire encrypted memory's ciphertext~\cite{li2022systematic}. 
We also account for single-step attack~\cite{wilke2023sev}, which aids in selecting optimal observation points in the control flow by controlling procedures within confidential VMs and pausing after each instruction.
While we focus on ciphertext side channels, we assume the confidential VM hardware and internal entities (i.e., guest OS, applications) are trusted. 
Although registers can leak during kernel context switches~\cite{li2022systematic}, a kernel patch can resolve this, so we consider registers are secure to protect secrets in this context.

\subsection{A Motivating Example}
\label{subsec:memorywrite}

We revisit the process of compiling a program using LLVM to illustrate how ciphertext side channels are identified.
\F~\ref{fig:background} presents the Machine Basic Blocks (MBBs) of the code snippet from the function \texttt{ossl\_ec\_scalar\_mul\_ladder}, handled by the register allocation pass in LLVM.
We simplify the representation of these MBBs by only showing the ciphertext side-channel relevant variables. 
Additionally, we format the first operand as the source and the second as the destination.

In $bb.73$ of \F~\ref{fig:background}(a), the register $eax$ is allocated to hold the variable $pbit$. 
Since $pbit$ is not used immediately or in succession, the compiler optimization inserts a memory store to write it back to $stack.28$. 
Subsequently, the compiler inserts a reload for the spilled variable $pbit$ once it is involved in operations, as shown in $bb.77$ where $pbit$ is reloaded.
This gives us the first source of ciphertext side-channel occurrence: the \textbf{spilling-reloading mechanism} during compilation generates additional memory writes, resulting in involuntary leakage.
A similar situation arises in $stack.29$, as demonstrated in $bb.75$ and $bb.76$, where the variable directly holds the secret returned from \texttt{BN\_is\_bit\_set}.

Next, the function \texttt{BN\_constant\_swap} in \F~\ref{fig:background}(b) demonstrates intermediate values being written back to allocated heap areas.
The compiler directly generates store instructions that point to heap memories, using $(rdi, rcx, 8)$ for $a\rightarrow d[i]$ and $(rdx, rcx, 8)$ for $b\rightarrow d[i]$.
Naturally, if these data are stored in registers during their lifecycles, the difficulty of the attack will be significantly increased as the attacker can only obtain the observation of final results. 
Unfortunately, even with the vector registers in Single Instruction Multiple Data (SIMD), big numbers in cryptography software cannot be continuously held. 
Thus, this gives us the second source of ciphertext side-channel occurrence: the compiler pervasively resorts to using stack and heap memory writes to store intermediate values due to \textbf{insufficient register resources}, thereby resulting in large attack surfaces for secret leakage.

\subsection{Architecture Overview of \tool}
\label{subsec:workflow}

\begin{figure}[t]
\centering
\includegraphics[width=\linewidth]{Fig_workflow.pdf}
\caption{Workflow of \tool.}
\label{fig:workflow}
\end{figure}

The compiler-aided methodology emerges as a natural and effective solution to defeat ciphertext side-channel attacks.
\F~\ref{fig:workflow} depicts the workflow of \tool, consisting of two phases: dynamic taint analysis and static rewriting.
The dynamic taint analysis tracks sensitive memory accesses on the last optimized IR (\S~\ref{subsec:tainting}), while the mitigation phase applies protection at the Machine-IR level using multiple variants.
Conducting mitigation at this level because: 
1) As analyzed in \S~\ref{subsec:memorywrite}, backend optimizations like register allocation may inadvertently undermine the efforts of mitigating ciphertext side channels.
2) The lower Machine-IR level could also support mitigation, although it would require additional efforts such as adapting taint analysis granularity, managing physical register usage, and manually allocating the mask buffer.
Therefore, ensuring the effectiveness of the mitigation requires fixing the program at \textbf{the LLVM Machine-IR level with the register allocation pass}.
During mitigation, precise buffer management for random nonces is carried out for all variants (\S~\ref{subsec:buffermanage}).
Lastly, the patched Machine-IR is subsequently compiled into a hardened binary for deployment.
Below are descriptions of the three variants.

\begin{packed_itemize}
    \item Variant 1 employs the \texttt{rdrand} instruction to generate random nonces when encountering sensitive memory stores. It optimizes the compiling process by utilizing a pre-generated nonce buffer (\S~\ref{subsec:datamasking}).
    
    \item Variant 2 employs the \texttt{vaesenc} instruction from AES-NI and a shift/rotate-based linear transformation from XorShift128+~\cite{vigna2017further} to fulfill the requirement of generating a random nonce on-the-fly (\S~\ref{subsec:datamasking}).
    
    \item Variant 3 seeks to safeguard secrets in sensitive stack areas by retaining them within vector registers, such as SSE registers, throughout their lifecycles (\S~\ref{subsec:registeralloc}).
\end{packed_itemize}

\noindent \textbf{Application Scope.}
The main audience for \tool\ is developers looking to deploy cryptographic software on modern TEEs. \tool\ offers security guarantees for existing cryptographic systems used in TEEs, as it relies on widely used hardware instructions for its three variants.

\subsection{Advantages of Compiler-aided Mitigation}
\label{subsec:advantages}

Opting for a compiler-aided methodology provides several advantages.

\noindent\textbf{In-place Code Insertion.}
The compiler-aided method repairs the program alongside the compilation process in an integrated and streamlined manner.
Thus, it ensures the fixed programs maintain their execution flow as much as possible, eliminating the need for frequent jumps to the instrumentation code that affect branch predictions, as seen in the binary instrumentation approach of \ftool.

\noindent\textbf{Smooth Management for Random Nonces.}
It is feasible to introduce and manage random nonces for the secrets with each memory writing during compilation (all Variants). We denote this method as \textit{software-based probabilistic encryption}. 
For example, the compiler directly allocates additional stack slots for nonces and links sensitive memory locations to their corresponding nonces through explicit virtual symbols. 
This approach contrasts with the binary instrumentation method, where allocating extra space for sensitive variables is challenging due to the fixed binary structure.

\noindent\textbf{Flexible Register Allocation.}
It is a significant advantage for the compiler to preserve secret variables within registers throughout their lifecycles, by leveraging register allocation pass before the final binary is generated (Variant 3).
We denote this method as \textit{secret-aware register allocation}.
For the binary instrumentation method, it is difficult to implement this mitigation approach.
There are some possible solutions, including performing liveness analysis on a fixed binary and preserving register values before allocating them to sensitive variables or reserving a certain number of registers at the beginning of compilation.
However, these solutions can inevitably lead to large performance degradation.

\noindent\textbf{Accurate Variable Length.}
Notably, the length of the stack and heap is crucial for subsequent vulnerability mitigation, as it aids in determining the duration for which memory needs to be protected. 
The compiler-aided approach enables the protection of each memory unit independently.
Oppositely, the binary instrumentation approach introduces unnecessary protection, focusing on protecting a slice of heap memory, primarily because it can only track explicit memory allocations and deallocations, such as \texttt{malloc}.

\noindent\textbf{Compensatory Dynamic Taint Analysis.}
Dynamic taint analysis quickly identifies secret variables but sacrifices coverage, as it only tracks sensitive memories in executed paths, leaving untouched paths unprotected. 
However, the virtual symbols in IR (as shown in \F~\ref{fig:background}) help compensate for this limitation. 
These virtual symbols, which represent sensitive memories, are initially tainted in executed paths and are recognized across all MBBs of a tainted function during the streamlined compilation process. 
This approach provides more comprehensive protection compared to binary instrumentation.

\subsection{Technical Challenges}
\label{subsec:challenges}

While the compiler-aided methodology facilitates the implementation of the three aforementioned variants, it faces several technical challenges. 
In the context of software-based probabilistic encryption, two random nonce buffers are necessary: one to provide the initial nonce for sensitive variables that require masking (utilized in Variants 1 and 3) and another to store the nonce currently in use (applicable across all variants). 
This requirement poses challenges in identifying suitable locations for these nonce buffers and establishing reliable references between sensitive variables and their corresponding nonces within these buffers. 
Furthermore, in the realm of secret-aware register allocation, a critical challenge lies in formulating a strategy for allocating limited registers to tainted stack slots. 
This includes determining which registers can temporarily accommodate sensitive variables without compromising performance and prioritizing access to these registers for tainted stack slots. 
We address these technical challenges in the following section.

\section{Detailed Design of \tool}
\label{sec:design}

\subsection{Tainting Secret Locations}
\label{subsec:tainting}

\noindent \textbf{IR Level.}
At present, binary-level taint analysis is employed to track secret propagation by logging execution traces, but it suffers from low speed~\cite{wang2017cached, jiang2022cache} and incomprehensive implicit information flow~\cite{weiser2018data}.
Tainting within compilation is another practice for detecting program vulnerabilities. 
Tools like DFSan have been utilized in prior works~\cite{borrello2021constantine, deng2023cipherh} to capture secret-related IR instructions.
Considering the subsequent program repair on Machine-IR and the need for comprehensive and readily available implicit information flow tracking, we also adopt the compiler-level taint analysis using DFSan.
This consideration involves aligning tainted results from the IR level with the Machine-IR level. 
Although IR is replaced by target instructions, the semantic difference is minimal since IR is transformed into Machine-IR through instruction selection and emission. 

\noindent \textbf{Detection Strategies.}
Mitigating only truly vulnerable points would minimize the performance impact on protected programs. 
However, accurately detecting ciphertext side-channel leaks is difficult, with tools like \htool\ still generating false positives. Hence, we opt to taint all sensitive memory accesses. 
Besides, we adopt the same strategy as \ftool\ to log execution traces multiple times for each cryptography program with varied secrets and inputs due to cryptography applications often following a relatively ``fixed'' execution flow as reported in \htool.

In \S~\ref{subsec:ciphertext}, we demonstrate how secrets are tracked through direct memory writes. 
In addition to the direct usage of secrets, it is necessary to consider data derived from the secrets as ``sensitive'' as well.
A secret may appear in control-flow branches as a condition or in memory accesses as the index. Then variables guarded by a tainted condition are tainted as secret-related; variables assigned through memory access are tainted once the index of the buffer is tainted.
In this manner, \tool\ comprehensively uncovers attack surfaces of the cryptography applications.

\subsection{Software-based Probabilistic Encryption}
\label{subsec:datamasking}

In software-based probabilistic encryption, the key is to introduce a random nonce into secret variables before they are written back into the memory, thereby achieving unpredictable ciphertexts. 

\begin{figure}[htbp]
    \centering
    \footnotesize
    \begin{tabular}{|ll|}
        \hline
        1:\ \ \textcolor{red}{load\ \ \ \ \ \ \ \ $MEM_{key}$\ \ \ \ $REG_{key}$} &//\ sensitive\ load\\
        2:\ \ load\ \ \ \ \ \ \ \ $MEM_{mask}$\ \ $REG_{mask}$ &//\ load\ mask\\
        3:\ \ save\ \ \ \ \ \ \ \ EFLAGS &\\
        4:\ \ xor\ \ \ \ \ \ \ \ \ $REG_{mask}$\ \ \ \ \textcolor{red}{$REG_{key}$} &\\
        5:\ \ restore\ \ \ \ \ EFLAGS &\\
        6:\ \ operate\ \ \ \ \textcolor{red}{$REG_{key}$} &\\
        7:\ \ save\ \ \ \ \ \ \ \ EFLAGS &\\
        8:\ \ update\ \ \ \ \ $REG_{mask}$ &//\ generate\ new\ mask\\
        9:\ \ xor\ \ \ \ \ \ \ \ \ $REG_{mask}$\ \ \ \ \textcolor{red}{$REG_{key}$} &\\
        10: restore\ \ \ \ \ EFLAGS &\\
        11: store\ \ \ \ \ \ \ $REG_{mask}$\ \ \ \ $MEM_{mask}$ &//\ store\ mask\\
        12: \textcolor{red}{store\ \ \ \ \ \ \ $REG_{key}$\ \ \ \ \ \ $MEM_{key}$} &//\ sensitive\ store \\
        \hline
    \end{tabular}
    \caption{In-place code insertion.}
    \label{fig:maskcode}
\end{figure}

\parh{Mitigation Code Insertion.}
\F~\ref{fig:maskcode} illustrates the general scenario of inserting mitigation code, where $MEM_{key}$ represents the memory cell of a sensitive memory access instruction.
The steps are: 1) loading the masked plaintext and corresponding random nonce (lines 1--2); 2) XORing the nonce with the masked plaintext to obtain true plaintext (lines 3--5); 3) performing calculations on the plaintext (line 6); 4) updating the random nonce and XORing it with the plaintext to obtain a new masked plaintext (lines 7--10); and 5) storing both the new masked plaintext and the new random nonce (lines 11--12).

We determine the memory cells holding the random nonces for stack and heap areas with different strategies.
In the stack, since the code is inserted during the register allocation phase in the LLVM backend, we can freely allocate a stack slot for the nonce. 
For example, a new stack slot, $MEM_{mask}$, is created to store the random nonce (line 11 of \F~\ref{fig:maskcode}) and is associated with the source memory $MEM_{key}$ (line 1). When loading the nonce, a load instruction is inserted to reference $MEM_{mask}$.
For heap memory, instead of intercepting memory allocation calls like \texttt{malloc}, \texttt{realloc}, \texttt{calloc}, and \texttt{free}, which would introduce significant overhead, we implement a more efficient hash-based mechanism. 
This scheme leverages the runtime heap address of the source memory $MEM_{key}$ to compute the index where the corresponding nonce is stored in the \texttt{.bss} section (see \S~\ref{subsec:buffermanage}). 
The random nonce is then stored at that index, allowing for efficient insertion of the same code for each tainted heap instruction without excessive system calls.

\parh{Random Nonce Generation.}
To enhance security, the random nonce is updated for each sensitive store instruction. 
In Variant 1, a 1K random nonce buffer is pre-generated using \texttt{rdrand} during the cryptography program's initialization, stored in the \texttt{.bss} section. 
For unmasked stack areas with secrets, Variant 1 selects a random nonce from this buffer as the initial nonce and increments it by three when storing new secrets at the same location (see \S~\ref{subsec:security} for security analysis). 
Variant 2, instead, generates a random nonce in real-time using AES-NI or XorShift128+ schemes.
AES-NI requires a single instruction and two 128-bit registers such as \texttt{xmm14} and \texttt{xmm15}, while XorShift128+ needs 11 instructions and three 128-bit registers, from \texttt{xmm13} to \texttt{xmm15}.

\subsection{Secret-aware Register Allocation}
\label{subsec:registeralloc}

Building upon Variant 1, the masking scheme for the stack area can be further enhanced through register allocation.

\parh{Feasibility Analysis.}
However, implementing this scheme is challenging due to limited register resources. To evaluate, we conducted a heuristic investigation. 
First, we identified the maximum number of stack slots involved in sensitive memory accesses among various cryptography programs from \T~\ref{tab:resultsoverview}, with libsodium's EdDSA implementation having the highest (583 slots). 
Accommodating this many vector registers in SIMD is difficult as discussed in \S~\ref{subsec:memorywrite}, prompting the exploration of a secret-aware register allocation approach. 
This involves tracking register liveness and timely deallocating frequently used stack slots in tainted functions. 
Next, we assessed the number of vector registers required. 
By analyzing disassembled cryptography programs, we found that the LLVM backend typically allocates the first 8 vector registers (\texttt{xmm0} - \texttt{xmm7}) for optimizing data movement, so we heuristically preserved the last 8 SSE registers (\texttt{xmm8} - \texttt{xmm15}) for sensitive stack slots. 
This approach resulted in an average performance impact of 7\%, ranging from 1\% in SHA512 (libsodium) to 20\% in AES (mbedTLS), demonstrating that it is a viable and practical solution.

% \begin{table}[htbp]
% \centering
% \caption{The maximum numbers of sensitive stack slots among tainted functions.}
% \label{tab:SSEimpact}
% \scriptsize
% \begin{tabular}{ccccc}
% \hline
%     \multirow{2}{*}{\textbf{Implementation}}
%     &\multirow{2}{*}{\textbf{Stack Slots}}
%     &\multirow{2}{*}{}
%     &\multicolumn{2}{c}{\textbf{Impact on performance}}\\
%     \cline{4-5}
%     & && \textbf{Cycles} & \textbf{Factor} \\
% \hline
%     libsodium-EdDSA &583 && 198331	&1.04 \\
%     libsodium-SHA512 &27 && 43043	&1.01 \\
% \hline
%     mbedTLS-AES &7 && 571968	&1.20 \\
%     mbedTLS-Base64 &25 && 11575	&1.03 \\
%     mbedTLS-ChaCha20 &4 && 609942	&1.02 \\
%     mbedTLS-ECDH &52 && 4586425	&1.08 \\
%     mbedTLS-ECDSA &52 && 4095496	&1.04 \\
%     mbedTLS-RSA &52 && 1943880	&1.03 \\
% \hline
%     OpenSSL-ECDH &157 && 1171024	&1.19 \\
%     OpenSSL-ECDSA &79 && 18180467	&1.08 \\
% \hline
%     WolfSSL-AES &43 && 689316	&1.08 \\
%     WolfSSL-ChaCha20 &5 && 512287	&1.06 \\
%     WolfSSL-ECDH &100 && 362967	&1.04 \\
%     WolfSSL-ECDSA &30 && 4559759	&1.08 \\
%     WolfSSL-EdDSA &159 && 426239	&1.02 \\
%     WolfSSL-RSA &28 && 543835	&1.11 \\
% \hline
%     Average &- && -	&1.07 \\
% \hline
% \end{tabular}
% \end{table}

\parh{Register Allocation.}
When analyzing tainted functions in cryptography programs, we prioritize frequently accessed stack slots. 
These stack slots are more susceptible to ciphertext side-channel attacks due to their sequential overwrites. Moreover, preserving these slots in registers minimizes the overhead of masking operations. 
To manage this, we create two structures in the LLVM backend: \texttt{StackUsage}, which maps all sensitive stack slots to their respective MBB locations in the tainted functions, and \texttt{StackOpt}, which selects mappings from \texttt{StackUsage} based on the number of MBB locations for each stack slot until the available vector register capacity is reached. 
These structures enable efficient secret-aware register allocation, focusing on optimizing register use and reducing masking overhead for frequently accessed stack slots.

$\bullet$ \textit{Allocation:} When encountering a sensitive stack store where its slot is mapped in \texttt{StackOpt}, Variant 3 assigns an available vector register to hold the variable. 
Simultaneously, the corresponding MBB location in \texttt{StackOpt} is removed to track the liveness of the allocated vector register.

$\bullet$ \textit{Deallocation:} Once all MBB locations for a sensitive stack slot in \texttt{StackOpt} are removed, the associated vector register is freed, marking the end of its liveness.
Subsequently, Variant 3 selects another stack slot from \texttt{StackUsage} based on the number of MBB locations and allocates available vector registers to the remaining stack slots in \texttt{StackOpt}.

\begin{lstlisting}[basicstyle=\footnotesize, frame=single, caption=Sensitive stack slots contained in MBBs., label=reglist, escapeinside=``]
42: entry,if.end19,if.end24,if.end30,...
38: entry,if.end24,if.end30,...
39: entry,while.body,if.end30,...
52: if.end30,...
49: if.end30,...
46: while.body,if.then10,if.end12,if.then18,
    if.end19,...
43: while.body,if.end19,if.then22,if.then28,
    if.end30,...
40: entry,while.cond,...
50: if.end30,...
44: while.body,if.end19,if.end24,if.end30,...
\end{lstlisting}

\begin{figure}[htbp]
\centering
\includegraphics[width=\linewidth]{Fig_register_liveness.pdf}
\caption{An example of register allocation from the function \texttt{bn\_mul\_add\_words} of OpenSSL-ECDSA. In the visualization, the white and shaded blocks represent the liveness of stacks, with shaded blocks containing numbers that denote registers holding the sensitive stack slots.}
\label{fig:liveness}
\end{figure}

We illustrate the register allocation process for sensitive stack slots in the function \texttt{bn\_mul\_add\_words} from OpenSSL-ECDSA. Listing~\ref{reglist} provides a sorted \texttt{StackOpt} structure, showing sensitive stack slots and their positions in MBBs.
The process is simulated up to the \texttt{if.end30} MBB and subsequent MBBs are omitted for brevity. 
As shown in \F~\ref{fig:liveness}, 16 stack slots, each 8 bytes in size, are allocated across vector registers \texttt{xmm8} to \texttt{xmm15}, with \texttt{H} and \texttt{L} representing the high and low 64 bits of each register. 
Registers such as \texttt{xmm13L}, \texttt{xmm13H}, and \texttt{xmm12L} are recycled and reallocated to stack slots 52, 49, and 50, respectively, demonstrating efficient register usage through liveness tracking.

\subsection{Managing Nonce Buffers}
\label{subsec:buffermanage}

In \tool, all inserted mitigation codes for the masking protection scheme must manage random nonces, including the \textit{initial nonce} for sensitive variables requiring masking and the \textit{nonce currently in use}.

\parh{Random Nonce Buffer.}
As introduced in \S~\ref{subsec:datamasking}, \tool pre-generates a buffer in the \texttt{.bss} section with 1K random nonces using \texttt{rdrand} during initialization. 
For unmasked stack or heap areas in Variants 1 and 3, \tool\ calculates an index based on the memory address (\textit{addr}) to select the corresponding random nonce from the buffer. 
The index is computed as $index = addr\ \&\ 0x3FF$, and the random nonce is retrieved from the \texttt{.bss} using $randomArray(, index, 8)$, where $randomArray$ is the starting address of the buffer.

\parh{Currently Used Nonces.}
Two strategies are adopted to store random nonces for sensitive data in the stack and heap areas, enabling their subsequent decoding.
For the stack, the compiler automatically allocates slots for nonces and associates them with the corresponding memory, accessed by applying an offset to the \texttt{rsp} register.
This method ensures the nonces are isolated from other stack locations by using the \texttt{rbp} register. 
Additionally, the nonce buffer, used to store active nonces, is initialized to zero and set in the entry block of each function.
Unlike \ftool, which risks overlap by storing nonces outside the runtime stack area, our approach avoids potential malfunctions.

For the heap, instead of applying a constant offset for the nonce buffer like in \ftool, we place the nonce buffer in the \texttt{.bss} section as well, pre-allocating sufficient space in advance. 
A heuristic hash function maps heap addresses to buffer indices for retrieving nonces: $((addr\ \&\ 0xFFFFF) * 648056) \gg 22$.
In this setting, 128K consecutive 8-byte entries are mapped independently into the 1M nonce buffer without collisions since all variants of \tool\ align masking operations to 8-byte addresses. 
However, the maximum index generated by the hash function is 162,012, which exceeds 128K. To accommodate this, a 256K-entry buffer is required, resulting in a 2M nonce buffer with around 50\% utilization in our configuration.
Through experimentation, we found that a nonce buffer space of 2MB is sufficient for cryptographic programs without collisions.

\tool\ offers key advantages in managing nonce buffers for heap areas compared to \ftool. 
First, it eliminates the overlap risk in \ftool, where a fixed distance between heap areas and nonce buffers can become insufficient with large dynamic allocations. 
\tool\ uses an indexing method to ensure dedicated nonce buffers for each heap area, preventing overlap. 
Second, \tool\ avoids the overhead caused by tracking dynamic memory calls (\texttt{malloc}, \texttt{realloc}, \texttt{calloc}, and \texttt{free}) in \ftool\ by directly calculating buffer indices, reducing system call dependency. 
Additionally, \tool\ simplifies initialization by zeroing out the \texttt{.bss} section during program linking.

To handle potential hash collisions in nonce buffers for heap areas, such as in server environments, we propose dynamically expanding the buffer by allocating multiple groups of entries for each index (10 groups per index, requiring a 20M nonce buffer). 
If two different heap addresses generate the same index, the first 8 bytes are used to match the heap address, allowing us to locate the corresponding nonce in the subsequent 8 bytes. 
We evaluated this method by simulating collisions across all 10$\times$128K entries. The results showed only a slight increase of time to initialize the expanded nonce buffer and handle each collision ($\approx 8 ms$), indicating minimal performance impact.

\section{Evaluation}

% \saidur{Working on it}




\begin{table*}[!t]
% \small
\centering
\caption{Summary of Results for EMBER Domain-IL Experiments.}
\vspace{-0.2cm}
\label{tab:ember_DIL}

\begin{tabular}{p{1.1cm}|l|c|c|c|c|c|c|c} 

% \toprule 

\multirow{2}{*}{\textbf{Group}} & \multirow{2}{*}{\textbf{Method}} & \multicolumn{7}{c}{\textbf{Budget}} \\ \cline{3-9}

&  & 1K & 10K & 50K & 100K & 200K & 300K & 400K \\ \midrule

\multirow{3}{*}{Baselines} 
& Joint  & \multicolumn{7}{c}{96.4$\pm$0.3} \\ 
& None   & \multicolumn{7}{c}{93.1$\pm$0.1} \\ 
& GRS    & 93.6$\pm$0.3 & 94.1$\pm$1.3 & 95.3$\pm$0.2 & 95.3$\pm$0.7 & 95.9$\pm$0.1 & 95.8$\pm$0.6 & 96.0$\pm$0.3 \\ 
\midrule

\multirow{4}{*}{\parbox{0.7cm}{Prior \\ Work}} 
& ER~\cite{er}     & 80.6$\pm$0.1 & 73.5$\pm$0.5 & 70.5$\pm$0.3 & 69.9$\pm$0.1 & 70.0$\pm$0.1 & 70.0$\pm$0.1 & 70.0$\pm$0.1 \\ 
& AGEM~\cite{agem}   & 80.5$\pm$0.1 & 73.6$\pm$0.2 & 70.4$\pm$0.3 & 70.0$\pm$0.1 & 70.0$\pm$0.2 & 70.0$\pm$0.1 & 70.0$\pm$0.1 \\ 
& GR~\cite{gr}     & \multicolumn{7}{c}{93.1$\pm$0.2} \\ 
& RtF~\cite{rtf}    & \multicolumn{7}{c}{93.2$\pm$0.2} \\ 
& BI-R~\cite{BIR}   & \multicolumn{7}{c}{93.4$\pm$0.1} \\ 
\midrule

\multirow{4}{*}{\system}      
& \system-R         & \textbf{93.7$\pm$0.1} & \textbf{94.7$\pm$0.1} & \textbf{95.4$\pm$0.1} & \textbf{95.3$\pm$0.6} & \textbf{96.0$\pm$0.1} & \textbf{96.1$\pm$0.1} & \textbf{96.1$\pm$0.1} \\ 
& \system-U         & \textbf{93.6$\pm$0.2} & 94.0$\pm$0.2 & 95.1$\pm$0.1 & \textbf{95.3$\pm$0.1} & 95.5$\pm$0.1 & 95.7$\pm$0.1 & 95.8$\pm$0.1 \\  \cline{2-9}
& MADAR$^{\theta}$-R & \textbf{93.6$\pm$0.1} & \textbf{94.4$\pm$0.3} & \textbf{95.3$\pm$0.2} & \textbf{95.8$\pm$0.1} & \textbf{96.1$\pm$0.1} & \textbf{96.1$\pm$0.1} & \textbf{96.1$\pm$0.1} \\ 
& MADAR$^{\theta}$-U & 93.5$\pm$0.2 & 94.1$\pm$0.2 & 94.9$\pm$0.1 & 95.2$\pm$0.2 & 95.6$\pm$0.1 & 95.7$\pm$0.1 & 95.7$\pm$0.1 \\ 

\bottomrule

\end{tabular}
\vspace{-0.2cm}
\end{table*}









\begin{figure}[!t]
    \centering
    \begin{subfigure}{0.485\linewidth}
        \centering
        \includegraphics[width=1.0\linewidth]{figures_TIFS/EMBER_IFS_DIL_RATIO.pdf}
        \label{fig:EMBER_DIL_IFS_R}
        \vspace{-0.4cm}
        \caption{MADAR Ratio}
    \end{subfigure}
    \hfill
    \begin{subfigure}{0.485\linewidth}
        \centering
        \includegraphics[width=1.0\linewidth]{figures_TIFS/EMBER_IFS_DIL_UNIFORM.pdf}
        \label{fig:EMBER_DIL_IFS_U}
        \vspace{-0.4cm}
        \caption{MADAR Uniform}
    \end{subfigure}
    \vfill
    \begin{subfigure}{0.485\linewidth}
        \centering
        \includegraphics[width=1.0\linewidth]{figures_TIFS/EMBER_AWS_DIL_RATIO.pdf}
        \label{fig:EMBER_DIL_AWS_R}
        \vspace{-0.4cm}
        \caption{MADAR$^\theta$ Ratio}
    \end{subfigure}
    \hfill
    \begin{subfigure}{0.485\linewidth}
        \centering
        \includegraphics[width=1.0\linewidth]{figures_TIFS/EMBER_AWS_DIL_UNIFORM.pdf}
        \label{fig:EMBER_DIL_AWS_U}
        \vspace{-0.4cm}
        \caption{MADAR$^\theta$ Uniform}
    \end{subfigure}

    \caption{EMBER Domain-IL: Comparison of the MADAR-R, MADAR-U, MADAR$^\theta$-R, and MADAR$^\theta$-U with Joint baseline.}
    \label{fig:ember_DIL}
    \vspace{-0.3cm}
\end{figure}





\begin{table*}[!t]
\centering
\caption{Summary of Results for AZ Domain-IL Experiments.}
\vspace{-0.3cm}
\label{tab:az_DIL}
\begin{tabular}{p{1.1cm}|l|c|c|c|c|c|c|c} 

% \toprule 

\multirow{2}{*}{\textbf{Group}} & \multirow{2}{*}{\textbf{Method}} & \multicolumn{7}{c}{\textbf{Budget}} \\ \cline{3-9}

&  & 1K & 10K & 50K & 100K & 200K & 300K & 400K \\ \midrule

\multirow{3}{*}{Baselines} 
& Joint  & \multicolumn{7}{c}{97.3$\pm$0.1} \\ 
& None   & \multicolumn{7}{c}{94.4$\pm$0.1} \\ 
& GRS    & 95.3$\pm$0.1 & 96.4$\pm$0.1 & 96.9$\pm$0.1 & 97.1$\pm$0.1 & 97.1$\pm$0.1 & 97.2$\pm$0.1 & 97.2$\pm$0.1 \\ 
\midrule

\multirow{4}{*}{\parbox{0.7cm}{Prior \\ Work}} 
& ER~\cite{er}     & 40.4$\pm$0.1 & 40.1$\pm$0.1 & 41.1$\pm$0.2 & 42.6$\pm$0.1 & 44.0$\pm$0.1 & 45.9$\pm$0.1 & 48.6$\pm$1.1 \\ 
& AGEM~\cite{agem}   & 45.4$\pm$0.1 & 47.4$\pm$0.2 & 49.2$\pm$0.2 & 53.7$\pm$0.6 & 54.2$\pm$0.3 & 54.8$\pm$0.4 & 56.7$\pm$0.3 \\ 
& GR~\cite{gr}     & \multicolumn{7}{c}{93.3$\pm$0.4} \\ 
& RtF~\cite{rtf}     & \multicolumn{7}{c}{93.4$\pm$0.2} \\ 
& BI-R~\cite{BIR}     & \multicolumn{7}{c}{93.5$\pm$0.1} \\ 
\midrule

\multirow{4}{*}{\system}      
& \system-R         & \textbf{95.8$\pm$0.1} & \textbf{96.6$\pm$0.1} & \textbf{96.9$\pm$0.1} & \textbf{97.0$\pm$0.1} & \textbf{97.0$\pm$0.1} & \textbf{97.0$\pm$0.1} & \textbf{97.0$\pm$0.1} \\ 
& \system-U         & \textbf{95.7$\pm$0.1} & 95.5$\pm$0.1 & 95.2$\pm$0.2 & 95.2$\pm$0.1 & 95.4$\pm$0.1 & 95.8$\pm$0.2 & 96.3$\pm$0.2 \\ \cline{2-9}
& MADAR$^{\theta}$-R & \textbf{95.8$\pm$0.2} & \textbf{96.6$\pm$0.1} & \textbf{96.9$\pm$0.1} & \textbf{96.9$\pm$0.1} & \textbf{97.1$\pm$0.1} & \textbf{97.1$\pm$0.1} & \textbf{97.2$\pm$0.1} \\ 
& MADAR$^{\theta}$-U & 95.6$\pm$0.1 & 96.1$\pm$0.1 & 96.6$\pm$0.1 & 96.8$\pm$0.1 & \textbf{97.0$\pm$0.1} & \textbf{97.1$\pm$0.1} & \textbf{97.1$\pm$0.1} \\ 

\bottomrule

\end{tabular}
\vspace{-0.3cm}
\end{table*}



\begin{figure}[!t]
    \centering
    \begin{subfigure}{0.485\linewidth}
        \centering
        \includegraphics[width=1.0\linewidth]{figures_TIFS/AZ_IFS_DIL_RATIO.pdf}
        \label{fig:AZ_DIL_IFS_R}
        \vspace{-0.4cm}
        \caption{MADAR Ratio}
    \end{subfigure}
    \hfill
    \begin{subfigure}{0.485\linewidth}
        \centering
        \includegraphics[width=1.0\linewidth]{figures_TIFS/AZ_IFS_DIL_UNIFORM.pdf}
        \label{fig:AZ_DIL_IFS_U}
        \vspace{-0.4cm}
        \caption{MADAR Uniform}
    \end{subfigure}
    \hfill
    \begin{subfigure}{0.485\linewidth}
        \centering
        \includegraphics[width=1.0\linewidth]{figures_TIFS/AZ_AWS_DIL_RATIO.pdf}
        \label{fig:AZ_DIL_AWS_R}
        \vspace{-0.4cm}
        \caption{MADAR$^\theta$ Ratio}
    \end{subfigure}
    \hfill
    \begin{subfigure}{0.485\linewidth}
        \centering
        \includegraphics[width=1.0\linewidth]{figures_TIFS/AZ_AWS_DIL_UNIFORM.pdf}
        \label{fig:AZ_DIL_AWS_U}
        \vspace{-0.4cm}
        \caption{MADAR$^\theta$ Uniform}
    \end{subfigure}

    \caption{AZ Domain-IL: Comparison of the MADAR-R, MADAR-U, MADAR$^\theta$-R, and MADAR$^\theta$-U with Joint baseline.}
    \label{fig:az_DIL}
    \vspace{-0.3cm}
\end{figure}






% \subsection{Experimental Setup, Datasets, and Baselines}


We present the results of our \system\ framework and MADAR$^\theta$ in the Domain-IL, Class-IL, and Task-IL scenarios using the EMBER and AZ datasets discussed in Section~\ref{sec:dataset}. To denote our techniques, we use the following abbreviations: {\bf \system-R} for \system-Ratio, {\bf \system-U} for \system-Uniform, {\bf MADAR$^\theta$-R} for MADAR$^\theta$-Ratio, and {\bf MADAR$^\theta$-U} for MADAR$^\theta$-Uniform.

For all three scenarios, we compare \system\ against widely studied replay-based continual learning (CL) techniques, including experience replay (ER)\cite{er}, average gradient episodic memory (AGEM)\cite{agem}, deep generative replay (GR)\cite{gr}, Replay-through-Feedback (RtF)\cite{rtf}, and Brain-inspired Replay (BI-R)\cite{BIR}. Additionally, we evaluate \system\ against iCaRL\cite{icarl}, a replay-based method specifically designed for Class-IL. For the Class-IL and Task-IL scenarios, we additionally compare \system\ with Task-specific Attention Modules in Lifelong Learning (TAMiL)\cite{tamil}. Furthermore, we benchmark MADAR against MalCL\cite{malcl}, a method specifically designed for Class-IL. Notably, most recent work focuses primarily on Class-IL and Task-IL scenarios, limiting direct comparisons in the Domain-IL scenario. In our results tables, the best-performing methods and those within the error margin of the top results are highlighted. 

%Finally, we built upon the codebase provided by \cite{continual-learning-malware} for implementation and evaluation.


% In this study, we utilize large-scale malware datasets, including the EMBER dataset~\cite{ember}, a widely recognized benchmark for Windows malware classification, and two Android malware datasets derived from AndroZoo~\cite{AndroZoo}, which were specifically curated for this research. Our approach is evaluated against two primary baselines:

% \begin{smitemize}
%     \item \textbf{None}: A baseline where the model is trained sequentially on each new task without employing any continual learning (CL) techniques, serving as an informal lower bound.
%     \item \textbf{Joint}: A baseline where the model is trained on both new and previously seen data at each step, representing an informal upper bound. While resource-intensive, the \textbf{Joint} baseline consistently achieves robust performance.
% \end{smitemize}

% Additionally, we introduce a third baseline: \textbf{Global Reservoir Sampling (GRS)}. This method is based on reservoir sampling~\cite{vitter1985random} and builds upon prior work by \cite{continual-learning-malware}. GRS provides an unbiased representation of class distributions and serves as a strong benchmark for comparing our diversity-aware approach.




% We now present the results of our \system framework for both \system and MADAR$^\theta$ in the Domain-IL, Class-IL, and Task-IL scenarios for EMBER and AZ datasets. We use the following four abbreviations to denote our techniques---{\bf \system-R} for \system-Ratio, {\bf ~\system-U} for \system-Uniform, {\bf MADAR$^\theta$-R} for MADAR$^\theta$-Ratio, and {\bf ~MADAR$^\theta$-U} for MADAR$^\theta$-Uniform.  For all three scenarios, we compare \system\ with the most widely studied replay-based CL techniques: experience replay (ER)~\cite{er}, average gradient episodic memory (AGEM)~\cite{agem}, deep generative replay (GR)~\cite{gr}, Replay-through-Feedback (RtF)~\cite{rtf}, and Brain-inspired Replay (BI-R)~\cite{BIR}. In addition, we compare \system\ with iCaRL~\cite{icarl}, a replay-based technique specifically designed for Class-IL. Furthermore, we compare \system with Task-specific Attention Modules in Lifelong learning (TAMiL)~\cite{bhat2023task} which is designed for Class-IL and Task-IL scenarios. In addition, we also compare MADAR with MalCL~\cite{malcl} specifically designed for Class-IL. We observe that recent works mostly focus on Class-IL and Task-IL scenarios which limits what we can compare with in the Domain-IL scenario. The results of the best-performing method, as well as those within the error range of the best results, are highlighted in the results tables. We built upon the code of the prior work by \cite{continual-learning-malware}.

% In this study, we use large-scale Windows and Android malware datasets: EMBER~\cite{ember}, a Windows malware dataset from prior work, recognized as a standard benchmark for malware classification, and two new Android malware datasets derived from AndroZoo~\cite{AndroZoo}, specifically assembled for this research.

% We adopt two baselines for comparison: {\em None} and {\em Joint}.  {\em None} sequentially trains the model on each new task without any CL techniques, serving as an informal minimum baseline. By contrast, {\em Joint} uses all new and prior data for training at each step, acting as an informal maximum baseline. Despite its resource demands, {\em Joint} ensures strong performance throughout the dataset. We also introduce an additional baseline -- Global Reservoir Sampling (GRS) built upon {\em reservoir sampling}~\cite{vitter1985random} and \cite{continual-learning-malware}. GRS provides an unbiased sampling of the underlying class distributions and serves as a strong point of comparison for our diversity-aware approach.

% In this study, we utilize large-scale malware datasets, including the EMBER dataset~\cite{ember}, a widely used benchmark for Windows malware classification, and two Android malware datasets derived from AndroZoo~\cite{AndroZoo}, specifically assembled for this research. We compare our approach against two baselines: {\em None}, where the model is trained sequentially on each new task without any CL techniques, serving as an informal lower bound; and {\em Joint}, which trains on both new and previous data at each step, representing an informal upper bound. Although resource-intensive, {\em Joint} ensures consistently strong results. Additionally, we introduce another baseline -- Global Reservoir Sampling (GRS), an approach based on {\em reservoir sampling}~\cite{vitter1985random} and \cite{continual-learning-malware}, which provides an unbiased representation of class distributions and serves as a strong point of comparison for our diversity-aware approach.


\subsection{Domain-IL}
\label{domainilexps}

%% #of training samples --> 674994
%As shown in Table~\ref{tab:combined_DIL}, a



In EMBER, we have 12 tasks, each representing the monthly data distribution spanning January--December 2018. Our results, detailed in Table~\ref{tab:ember_DIL}, provide a comprehensive view of each method's performance, reported as the average accuracy over all tasks $\mathbf{\overline{AP}}$. Additionally, Figure~\ref{fig:ember_DIL} illustrates the progression of average accuracy over time compared to the \textit{Joint} baseline. 

The informal lower and upper performance bounds for this configuration are approximated by the \textit{None} and \textit{Joint} methods, achieving $\mathbf{\overline{AP}}$ scores of 93.1\% and 96.4\%, respectively. Meanwhile, \textit{GRS} serves as a strong baseline, providing unbiased sampling without incorporating sample diversity awareness.

% In EMBER, we have 12 tasks, each representing the monthly data distribution spanning January--December 2018. Our results, detailed in Table~\ref{tab:ember_DIL}, present a nuanced view of each method's performance, reported as the average accuracy over all tasks $\mathbf{\overline{AP}}$. In addition, Figure~\ref{fig:ember_DIL} represents the progression of average accuracy as the task progresses compared with {joint} baseline. The informal lower and upper performance bounds for this configuration can be approximated by the {\em None} and {\em Joint} methods, which get $\mathbf{\overline{AP}}$ of 93.1\% and 96.4\%, respectively. Meanwhile, {\em GRS} represents a strong baseline for unbiased sampling without awareness of sample diversity.

At a lower budget of 1K, \system-R, \system-U, and MADAR$^\theta$-R exhibit competitive performance, all achieving $\mathbf{\overline{AP}}$ of over $93.6$\%, significantly outperforming prior approaches. This highlights their ability to effectively utilize limited resources. In particular, \system-R achieves the highest accuracy at this budget, with $\mathbf{\overline{AP}}$ of $93.7\%$.

As the memory budget increases, the performance of all \system\ and MADAR$^\theta$ variants improves steadily. At a budget of 200K, \system-R and MADAR$^\theta$-R achieve an impressive $\mathbf{\overline{AP}}$ of $96.0\%$ and $96.1\%$, respectively, closely approaching the $96.4\%$ achieved by the \textit{Joint} baseline, which utilizes over 670K samples. Uniform strategies, including \system-U and MADAR$^\theta$-U, are only slightly behind, with $\mathbf{\overline{AP}}$ values of $95.5\%$ and $95.6\%$, respectively.

% At lower budget of 1K, GRS, \system-R, and \system-U exhibit competitive performance, all significantly better than prior work with $\mathbf{\overline{AP}}$ above $93.6$\%, indicating their effective utilization of limited resources. ER and AGEM performed far below even the \emph{None} baseline, while GR could only match it. For higher budgets, GRS and \system\ methods all show excellent performance. At a 200K budget, \system-R yields $\mathbf{\overline{AP}}$ of $96.0$\%, close to the $96.4$\% reached by the Joint baseline that used over 670K samples. GRS is competitive, while Uniform strategies are only slightly behind.




\begin{table*}[!t]
\centering
\caption{Summary of Results for EMBER Class-IL Experiments.}
\vspace{-0.3cm}
\label{tab:ember_CIL}
\begin{tabular}{p{1.1cm}|l|c|c|c|c|c|c|c} 

% \toprule 

\multirow{2}{*}{\textbf{Group}} & \multirow{2}{*}{\textbf{Method}} & \multicolumn{7}{c}{\textbf{Budget}} \\ \cline{3-9}

&  & 100 & 500 & 1K & 5K & 10K & 15K & 20K \\ \midrule

\multirow{3}{*}{Baselines} 
& Joint  & \multicolumn{7}{c}{86.5$\pm$0.4} \\ 
& None   & \multicolumn{7}{c}{26.5$\pm$0.2} \\ 
& GRS    & 51.9$\pm$0.4 & 70.3$\pm$0.5 & 75.4$\pm$0.7 & 82.0$\pm$0.2 & 83.5$\pm$0.1 & 84.3$\pm$0.3 & 84.6$\pm$0.2 \\ \midrule

\multirow{6}{*}{\parbox{0.7cm}{Prior \\ Work}} 
& TAMiL~\cite{tamil}  & 32.2$\pm$0.3 & 33.1$\pm$0.2 & 35.3$\pm$0.2 & 36.7$\pm$0.1 & 38.2$\pm$0.3 & 37.2$\pm$0.2 & 38.8$\pm$0.2 \\ 
& iCaRL~\cite{icarl}  & 53.9$\pm$0.7 & 58.7$\pm$0.7 & 60.0$\pm$1.0 & 63.9$\pm$1.2 & 64.6$\pm$0.8 & 65.5$\pm$1.0 & 66.8$\pm$1.1 \\ 
& ER~\cite{er}     & 27.5$\pm$0.1 & 27.8$\pm$0.1 & 28.0$\pm$0.1 & 27.9$\pm$0.1 & 28.0$\pm$0.1 & 28.0$\pm$0.1 & 28.2$\pm$0.1 \\ 
& AGEM~\cite{agem}   & 27.3$\pm$0.1 & 27.4$\pm$0.1 & 27.7$\pm$0.1 & 28.5$\pm$0.1 & 28.2$\pm$0.1 & 28.3$\pm$0.1 & 28.2$\pm$0.1 \\ 
& GR~\cite{gr}     & \multicolumn{7}{c}{26.8$\pm$0.2} \\ 
& RtF~\cite{rtf}   & \multicolumn{7}{c}{26.5$\pm$0.1} \\ 
& BI-R~\cite{BIR}   & \multicolumn{7}{c}{26.9$\pm$0.1} \\ 
& MalCL~\cite{malcl}   & \multicolumn{7}{c}{54.5$\pm$0.3} \\ 
\midrule

\multirow{4}{*}{\system} 
& \system-R & \textbf{68.0$\pm$0.4} & 73.6$\pm$0.2 & 76.0$\pm$0.3 & 81.5$\pm$0.2 & 83.2$\pm$0.2 & 83.8$\pm$0.2 & 84.0$\pm$0.2 \\ 
& \system-U & 66.4$\pm$0.4 & \textbf{76.5$\pm$0.2} & \textbf{79.4$\pm$0.4} & \textbf{83.8$\pm$0.2} & \textbf{84.8$\pm$0.1} & \textbf{85.5$\pm$0.1} & \textbf{85.8$\pm$0.3} \\ \cline{2-9}
& MADAR$^{\theta}$-R & {\bf 67.9$\pm$0.3} & 72.7$\pm$0.5 & 72.7$\pm$0.5 & 81.7$\pm$0.2 & 83.2$\pm$0.1 & 83.9$\pm$0.1 & 84.5$\pm$0.2 \\ 
& MADAR$^{\theta}$-U & 67.5$\pm$0.3 & {\bf 76.4$\pm$0.4} & {\bf 78.5$\pm$0.4} & {\bf 84.1$\pm$0.1} & {\bf 85.3$\pm$0.1} & {\bf 85.8$\pm$0.2} & {\bf 86.2$\pm$0.2} \\ 

\bottomrule

\end{tabular}
\vspace{-0.2cm}
\end{table*}



\begin{figure}[!t]
    \centering
    \begin{subfigure}{0.485\linewidth}
        \centering
        \includegraphics[width=1.0\linewidth]{figures_TIFS/EMBER_CIL_IFS_RATIO.pdf}
        \label{fig:EMBER_CIL_IFS_R}
        \vspace{-0.4cm}
        \caption{MADAR Ratio}
    \end{subfigure}
    \hfill
    \begin{subfigure}{0.485\linewidth}
        \centering
        \includegraphics[width=1.0\linewidth]{figures_TIFS/EMBER_CIL_IFS_UNIFORM.pdf}
        \label{fig:EMBER_CIL_IFS_U}
        \vspace{-0.4cm}
        \caption{MADAR Uniform}
    \end{subfigure}
    \vfill
    \begin{subfigure}{0.485\linewidth}
        \centering
        \includegraphics[width=1.0\linewidth]{figures_TIFS/EMBER_CIL_AWS_RATIO.pdf}
        \label{fig:EMBER_CIL_AWS_R}
        \vspace{-0.4cm}
        \caption{MADAR$^\theta$ Ratio}
    \end{subfigure}
    \hfill
    \begin{subfigure}{0.485\linewidth}
        \centering
        \includegraphics[width=1.0\linewidth]{figures_TIFS/EMBER_CIL_AWS_UNIFORM.pdf}
        \label{fig:EMBER_CIL_AWS_U}
        \vspace{-0.4cm}
        \caption{MADAR$^\theta$ Uniform}
    \end{subfigure}

    \caption{EMBER Class-IL: Comparison of the MADAR-R, MADAR-U, MADAR$^\theta$-R, and MADAR$^\theta$-U with Joint baseline.}
    \label{fig:ember_CIL}
    \vspace{-0.3cm}
\end{figure}


For the experiments with AZ-Domain, we consider 9 tasks, each representing a yearly data distribution from 2008 to 2016. The performance of each method is presented in Table~\ref{tab:az_DIL} as $\mathbf{\overline{AP}}$ and compared to two baselines: \textit{None}, which achieves $94.4\%$, and \textit{Joint}, which reaches $97.3\%$. Additionally, Figure~\ref{fig:az_DIL} illustrates the progression of average accuracy across tasks, highlighting the comparison with the \textit{Joint} baseline.

Similar to the results observed with EMBER, our MADAR techniques consistently outperform prior methods such as ER, AGEM, GR, RtF, and BI-R across all budget levels. For lower budgets, such as 1K, \system-R achieves $\mathbf{\overline{AP}}$ of $95.8\%$ and coming within 1.5\% of the \textit{Joint} baseline.

At higher budgets, ranging from 100K to 400K, \system-R continues to demonstrate high $\mathbf{\overline{AP}}$ scores of up to $97.0\%$, closely matching GRS and only marginally below the \textit{Joint} baseline, which requires significantly more training samples (680K). Notably, MADAR$^\theta$-R exhibits comparable performance, reaching a peak $\mathbf{\overline{AP}}$ of $97.2\%$ at the highest budget level, further underscoring the efficacy of our diversity-aware approach.



% For the experiments with AZ-Domain, we have 9 tasks, each representing a year from 2008 to 2016. The performance of each method is shown in Table~\ref{tab:az_DIL} as $\mathbf{\overline{AP}}$ and compared with two baselines: {\em None} at $94.4\pm0.1$ and {\em Joint} at $97.3\pm0.1$. Additionally, Figure~\ref{fig:az_DIL} illustrates the progression of average accuracy as tasks progress, compared to the \textit{Joint} baseline. 

% As with EMBER, we find that our MADAR techniques greatly surpass previous methods like ER, AGEM, GR, RtF, and BI-R for every budget level. For lower budgets like 1K, \system-R slightly outperforms GRS and is within 1.5\% of {\em Joint}. For higher budgets (100K-400K), \system-R perform well -- in line with GRS and just slightly below {\em Joint}, which requires 680K training samples. 


% In summary, our results empirically depict the effectiveness of MADAR's diversity-aware sample selection in maximizing the efficiency and effectiveness of a malware classifier in Domain-IL. \system-R is either better or on par with GRS and significantly better than prior work.

In summary, these results empirically demonstrate the effectiveness of MADAR's diversity-aware sample selection in enhancing the efficiency and accuracy of malware classification in Domain-IL scenarios. \system-R and MADAR$^\theta$-R, in particular, consistently either yield on-par or outperform GRS while delivering significant improvements over prior methods.












\begin{table*}[!t]
\centering
\caption{Summary of Results for AZ Class-IL Experiments.}
\vspace{-0.3cm}
\label{tab:az_CIL}
\begin{tabular}{p{1.1cm}|l|c|c|c|c|c|c|c} 

% \toprule 

\multirow{2}{*}{\textbf{Group}} & \multirow{2}{*}{\textbf{Method}} & \multicolumn{7}{c}{\textbf{Budget}} \\ \cline{3-9}

&  & 100 & 500 & 1K & 5K & 10K & 15K & 20K \\ \midrule

\multirow{3}{*}{Baselines} 
& Joint  & \multicolumn{7}{c}{94.2$\pm$0.1} \\ 
& None   & \multicolumn{7}{c}{26.4$\pm$0.2} \\ 
& GRS    & 43.8$\pm$0.7 & 62.9$\pm$0.8 & 70.2$\pm$0.4 & 83.0$\pm$0.3 & 86.4$\pm$0.2 & 88.2$\pm$0.2 & 89.1$\pm$0.2 \\ \midrule

\multirow{6}{*}{\parbox{0.7cm}{Prior \\ Work}} 
& TAMiL~\cite{tamil}  & 53.4$\pm$0.3 & 55.2$\pm$0.3 & 57.6$\pm$0.3 & 60.8$\pm$0.2 & 63.5$\pm$0.1 & 65.3$\pm$0.5 & 67.7$\pm$0.3 \\ 
& iCaRL~\cite{icarl}  & 43.6$\pm$1.2 & 54.9$\pm$1.0 & 61.7$\pm$0.7 & 77.2$\pm$0.4 & 81.5$\pm$0.6 & 83.4$\pm$0.5 & 84.6$\pm$0.5 \\ 
& ER~\cite{er}     & 50.8$\pm$0.7 & 58.3$\pm$0.6 & 58.9$\pm$0.2 & 59.2$\pm$0.8 & 62.9$\pm$0.7 & 63.1$\pm$0.5 & 64.2$\pm$0.4 \\ 
& AGEM~\cite{agem}   & 27.3$\pm$0.7 & 28.0$\pm$1.4 & 27.1$\pm$0.3 & 28.0$\pm$0.6 & 28.2$\pm$1.0 & 29.8$\pm$2.6 & 28.0$\pm$0.8 \\ 
& GR~\cite{gr}     & \multicolumn{7}{c}{22.7$\pm$0.3} \\ 
& RtF~\cite{rtf}    & \multicolumn{7}{c}{22.9$\pm$0.3} \\ 
& BI-R~\cite{BIR}   & \multicolumn{7}{c}{23.4$\pm$0.2} \\ 
& MalCL~\cite{malcl}   & \multicolumn{7}{c}{59.8$\pm$0.4} \\ 
\midrule

\multirow{4}{*}{\system} 
& \system-R & \textbf{59.4$\pm$0.6} & 67.8$\pm$0.9 & 71.9$\pm$0.5 & 82.9$\pm$0.2 & 86.3$\pm$0.1 & 88.2$\pm$0.2 & 89.1$\pm$0.1 \\ 
& \system-U & 57.3$\pm$0.5 & \textbf{70.4$\pm$0.4} & \textbf{76.2$\pm$0.2} & \textbf{86.8$\pm$0.1} & \textbf{89.8$\pm$0.1} & \textbf{91.0$\pm$0.1} & \textbf{91.5$\pm$0.1} \\ \cline{2-9}
& MADAR$^{\theta}$-R & {\bf 58.8$\pm$0.3} & 66.2$\pm$0.7 & 71.0$\pm$0.7 & 81.2$\pm$0.3 & 85.1$\pm$0.2 & 86.9$\pm$0.2 & 88.1$\pm$0.1 \\ 
& MADAR$^{\theta}$-U & 58.5$\pm$0.7 & {\bf 70.1$\pm$0.2} & {\bf 74.7$\pm$0.2} & {\bf 85.5$\pm$0.1} & {\bf 88.7$\pm$0.1} & {\bf 90.3$\pm$0.2} & {\bf 90.7$\pm$0.1} \\ 

\bottomrule

\end{tabular}
\vspace{-0.2cm}
\end{table*}








\begin{figure}[!t]
    \centering
    \begin{subfigure}{0.485\linewidth}
        \centering
        \includegraphics[width=1.0\linewidth]{figures_TIFS/AZ_CIL_IFS_RATIO.pdf}
        \label{fig:AZ_CIL_IFS_R}
        \vspace{-0.4cm}
        \caption{MADAR Ratio}
    \end{subfigure}
    \hfill
    \begin{subfigure}{0.485\linewidth}
        \centering
        \includegraphics[width=1.0\linewidth]{figures_TIFS/AZ_CIL_IFS_UNIFORM.pdf}
        \label{fig:AZ_CIL_IFS_U}
        \vspace{-0.4cm}
        \caption{MADAR Uniform}
    \end{subfigure}
    \vfill
    \begin{subfigure}{0.485\linewidth}
        \centering
        \includegraphics[width=1.0\linewidth]{figures_TIFS/AZ_CIL_AWS_RATIO.pdf}
        \label{fig:AZ_CIL_AWS_R}
        \vspace{-0.4cm}
        \caption{MADAR$^\theta$ Ratio}
    \end{subfigure}
    \hfill
    \begin{subfigure}{0.485\linewidth}
        \centering
        \includegraphics[width=1.0\linewidth]{figures_TIFS/AZ_CIL_AWS_UNIFORM.pdf}
        \label{fig:AZ_CIL_AWS_U}
        \vspace{-0.4cm}
        \caption{MADAR$^\theta$ Uniform}
    \end{subfigure}

    \caption{AZ Class-IL: Comparison of the MADAR-R, MADAR-U, MADAR$^\theta$-R, and MADAR$^\theta$-U with Joint baseline.}
    \label{fig:az_CIL}
    \vspace{-0.3cm}
\end{figure}





\subsection{Class-IL}
\label{classilexps}



In this set of experiments with EMBER, we consider 11 tasks, starting with 50 classes (representing distinct malware families) in the initial task, and incrementally adding five new classes in each subsequent task. Table~\ref{tab:ember_CIL} presents the performance of each method, measured by average accuracy $\mathbf{\overline{AP}}$. The \textit{None} and \textit{Joint} baselines achieve $\mathbf{\overline{AP}}$ values of $26.5\%$ and $86.5\%$, respectively, providing informal lower and upper bounds. Figure~\ref{fig:ember_CIL} illustrates the progression of average accuracy across tasks, showing how the \system\ and MADAR$^\theta$ methods compare to the \textit{Joint} baseline.

At a very low budget of just 100 samples, \system-R achieves a notable $\mathbf{\overline{AP}}$ of $68.0\%$, outperforming GRS and prior methods by a significant margin. As the budget increases, \system-U emerges as the top performer, achieving $\mathbf{\overline{AP}}$ values of $76.5\%$ and $79.4\%$ at 1K and 10K budgets, respectively, surpassing all other methods, including GRS. 

%For example, at a 10K budget, \system-U outperforms GRS, which achieves $83.5\%$, with an $\mathbf{\overline{AP}}$ of $84.8\%$.

At higher budgets, \system-U and MADAR$^\theta$-U continue to excel, with MADAR$^\theta$-U achieving the best results overall. At a 20K budget, MADAR$^\theta$-U reaches an $\mathbf{\overline{AP}}$ of $86.2\%$, nearly equaling the \textit{Joint} baseline, which uses over {\bf 150 times} more training samples. These results clearly demonstrate the effectiveness of MADAR's diversity-aware sample selection and the effectiveness of \system-U and MADAR$^\theta$-U in leveraging limited resources.

In contrast, prior methods such as ER, AGEM, GR, RtF, and BI-R fail to exceed 30\% $\mathbf{\overline{AP}}$, while more advanced techniques like TAMiL and MalCL achieve only $38.2\%$ and $54.8\%$, respectively. Even iCaRL, designed specifically for Class-IL, achieves only $64.6\%$, further highlighting the significant advantage of our approaches in the malware domain.


% In this set of experiments with EMBER, we have 11 tasks, where the initial task starts with 50 classes---one for each of 50 malware families---and five classes are added in each subsequent task. The performance of these methods, detailed in Table~\ref{tab:az_CIL}, is measured by average accuracy $\mathbf{\overline{AP}}$ with {\em None} and {\em Joint} training baselines at an $\mathbf{\overline{AP}}$ of $26.5\pm0.2$ and $86.5\pm0.4$, respectively. Additionally, Figure~\ref{fig:ember_CIL} illustrates the progression of average accuracy across tasks, highlighting the comparison with the \textit{Joint} baseline. 

% For a very low budget of 100 samples, \system methods greatly outperform GRS, with \system-R getting 16\% higher $\mathbf{\overline{AP}}$. For more reasonable budgets, however, the uniform variant \system-U offers the best performance. For example, with a 10K budget, \system-U yields at least 84.8\% $\mathbf{\overline{AP}}$, which is better than GRS at 83.5\% $\mathbf{\overline{AP}}$. They also fare far better than all prior works, with ER, AGEM, GR, RtF, and BI-R below 30\%, TAMiL at 38.2\%, MalCL at 54.8\% and iCaRL at only 64.6\%. These poor results for the prior methods are in line with other findings in the malware domain~\cite{continual-learning-malware}. For a budget of 20K, \system-U reaches $85.8\pm0.3$, nearly as good as the Joint baseline that uses a maximum budget over 150 times larger.



In the Class-IL setting of AZ-Class, we consider 11 tasks. The summary results of all experiments are provided in Table~\ref{tab:az_CIL}, with comparisons against the \textit{None} and \textit{Joint} baselines, which achieve $\mathbf{\overline{AP}}$ scores of $26.4\%$ and $94.2\%$, respectively. Figure~\ref{fig:az_CIL} illustrates the progression of average accuracy across tasks, showing how each method performs relative to the \textit{Joint} baseline.

As shown in Table~\ref{tab:az_CIL}, among the prior methods, iCaRL performs best across most budget configurations, outperforming techniques such as MalCL, TAMiL, ER, AGEM, GR, RtF, and BI-R. Therefore, we focus on comparing MADAR's performance with iCaRL. At a low budget of 100 samples, iCaRL and GRS achieve less than $44\%$ $\mathbf{\overline{AP}}$, while all MADAR methods surpass $57\%$. In particular, \system-R and MADAR$^\theta$-R achieve $\mathbf{\overline{AP}}$ scores of $59.4\%$ and $58.8\%$, respectively, highlighting their efficiency at low-resource levels.

As the budget increases, all methods improve, but \system-U consistently delivers the best results. At a budget of 1K, \system-U achieves the highest $\mathbf{\overline{AP}}$ at $70.4\%$, followed closely by MADAR$^\theta$-U at $70.1\%$. This trend continues as budgets increase, with \system-U outperforming all other methods, achieving $\mathbf{\overline{AP}}$ scores of $89.8\%$ at 10K and $91.5\%$ at 20K. Compared to GRS, which achieves $90.1\%$ at 20K, and iCaRL, which trails at $84.6\%$, \system-U demonstrates clear superiority. MADAR$^\theta$-U also performs GRS reaching $90.7\%$ at 20K.



% We have 11 tasks for the Class-IL setting of AZ-Class. The summary results of all the experiments are shown in Table~\ref{tab:az_CIL} and benchmarked against {\em None} and {\em Joint} with $\mathbf{\overline{AP}}$ of $26.4\pm0.2$ and $94.2\pm0.1$, respectively. Figure~\ref{fig:az_CIL} illustrates the progression of average accuracy across tasks, highlighting the comparison with the \textit{Joint} baseline. 


% As we can from Table~\ref{tab:az_CIL} that, among TAMiL, iCaRL, ER, AGEM, GR, RtF, and BI-R, iCaRL outperforms in most of the budget configurations. Therefore, we discuss the results of MADAR in comparison with iCaRL. For a low budget of 100, iCaRL and GRS get less than 44\%, while all MADAR methods achieve over 57\%. As budgets increase, all methods improve, with \system-U offering the best results at every budget from 1K to 20K. At 20K, it reaches $91.5\pm0.1\%$, which is 1.4\% higher than GRS and 6.9\% higher than iCaRL.



In summary, our experiments demonstrate the effectiveness of \system's diversity-aware replay techniques in the Class-IL setting for both EMBER and AZ datasets. While GRS shows significant improvement with larger budgets, \system's uniform variants consistently outperform it across all budget levels. These results underscore \system's ability to mitigate catastrophic forgetting and enhance malware classification performance, even in resource-constrained environments.

% In summary, our experiments clearly demonstrate the effectiveness of \system's diversity-aware replay techniques in Class-IL for both EMBER and AZ datasets. Additionally, while GRS shows significant improvement with an increased budget, the uniform variants of \system  are more effective at every budget level. \system  significantly improves performance in malware classification by mitigating catastrophic forgetting, and they do so using fewer resources.












\begin{table*}[!t]
\centering
\caption{Summary of Results for EMBER Task-IL Experiments.}
\vspace{-0.3cm}
\label{tab:ember_TIL}
\begin{tabular}{p{1.1cm}|l|c|c|c|c|c|c|c} 

% \toprule 

\multirow{2}{*}{\textbf{Group}} & \multirow{2}{*}{\textbf{Method}} & \multicolumn{7}{c}{\textbf{Budget}} \\ \cline{3-9}

&  & 100 & 500 & 1K & 5K & 10K & 15K & 20K \\ \midrule

\multirow{3}{*}{Baselines} 
& Joint  & \multicolumn{7}{c}{97.0$\pm$0.3} \\ 
& None   & \multicolumn{7}{c}{74.6$\pm$0.7} \\ 
& GRS    & 86.9$\pm$0.3 & 87.4$\pm$0.3 & 93.6$\pm$0.3 & 94.4$\pm$0.2 & 94.7$\pm$0.3 & 94.9$\pm$0.1 & 95.0$\pm$0.1 \\ \midrule

\multirow{6}{*}{\parbox{0.7cm}{Prior \\ Work}} 
& TAMiL~\cite{tamil}  & 72.8$\pm$0.1 & 81.5$\pm$0.3 & 86.9$\pm$0.2 & 88.1$\pm$0.3 & 90.3$\pm$0.1 & 93.2$\pm$0.3 & 94.2$\pm$0.7 \\ 
& ER~\cite{er}     & 67.4$\pm$0.3 & 84.9$\pm$0.2 & 89.5$\pm$0.5 & 93.9$\pm$0.2 & 94.8$\pm$0.2 & 95.2$\pm$0.1 & 95.4$\pm$0.1 \\ 
& AGEM~\cite{agem}   & 79.6$\pm$0.2 & 81.7$\pm$0.2 & 83.8$\pm$0.4 & 84.9$\pm$0.2 & 86.1$\pm$0.2 & 88.9$\pm$0.2 & 89.3$\pm$0.1 \\ 
& GR~\cite{gr}     & \multicolumn{7}{c}{79.8$\pm$0.3} \\ 
& RtF~\cite{rtf}    & \multicolumn{7}{c}{77.8$\pm$0.2} \\ 
& BI-R~\cite{BIR}   & \multicolumn{7}{c}{87.2$\pm$0.3} \\ \midrule

\multirow{4}{*}{\system} 
& \system-R & 92.1$\pm$0.2 & 92.3$\pm$0.9 & 93.8$\pm$0.2 & 94.2$\pm$0.1 & 94.8$\pm$0.2 & {\bf 95.7$\pm$0.2} & {\bf 95.6$\pm$0.1} \\ 
& \system-U & {\bf 93.4$\pm$0.2} & {\bf 93.7$\pm$0.3} & {\bf 93.9$\pm$0.3} & {\bf 94.8$\pm$0.2} & {\bf 95.6$\pm$0.1} & {\bf 95.7$\pm$0.1} & {\bf 95.8$\pm$0.2} \\ \cline{2-9}
& MADAR$^{\theta}$-R & {\bf 93.1$\pm$0.2} & {\bf 93.3$\pm$0.1} & {\bf 93.6$\pm$0.1} & 94.3$\pm$0.1 & 94.6$\pm$0.2 & 94.8$\pm$0.2 & 94.7$\pm$0.3 \\ 
& MADAR$^{\theta}$-U & {\bf 93.2$\pm$0.1} & 93.1$\pm$0.2 & {\bf 93.8$\pm$0.2} & {\bf 94.4$\pm$0.1} & {\bf 94.8$\pm$0.1} & {\bf 95.3$\pm$0.2} & {\bf 95.5$\pm$0.3} \\ 

\bottomrule

\end{tabular}
\vspace{-0.3cm}
\end{table*}



\begin{figure}[!t]
    \centering
    \begin{subfigure}{0.485\linewidth}
        \centering
        \includegraphics[width=1.0\linewidth]{figures_TIFS/EMBER_TIL_IFS_RATIO.pdf}
        \label{fig:EMBER_TIL_IFS_R}
        \vspace{-0.4cm}
        \caption{MADAR Ratio}
    \end{subfigure}
    \hfill
    \begin{subfigure}{0.485\linewidth}
        \centering
        \includegraphics[width=1.0\linewidth]{figures_TIFS/EMBER_TIL_IFS_UNIFORM.pdf}
        \label{fig:EMBER_TIL_IFS_U}
        \vspace{-0.4cm}
        \caption{MADAR Uniform}
    \end{subfigure}
    \vfill
    \begin{subfigure}{0.485\linewidth}
        \centering
        \includegraphics[width=1.0\linewidth]{figures_TIFS/EMBER_TIL_AWS_RATIO.pdf}
        \label{fig:EMBER_TIL_AWS_R}
        \vspace{-0.4cm}
        \caption{MADAR$^\theta$ Ratio}
    \end{subfigure}
    \hfill
    \begin{subfigure}{0.485\linewidth}
        \centering
        \includegraphics[width=1.0\linewidth]{figures_TIFS/EMBER_TIL_AWS_UNIFORM.pdf}
        \label{fig:EMBER_TIL_AWS_U}
        \vspace{-0.4cm}
        \caption{MADAR$^\theta$ Uniform}
    \end{subfigure}

    \caption{EMBER Task-IL: Comparison of the MADAR-R, MADAR-U, MADAR$^\theta$-R, and MADAR$^\theta$-U with Joint baseline.}
    \label{fig:ember_TIL}
    \vspace{-0.3cm}
\end{figure}

























\subsection{Task-IL}
\label{taskilexps-ember}


In this set of experiments with EMBER, we consider 20 tasks, with 5 new classes added in each task. The summarized results are shown in Table~\ref{tab:ember_TIL}, where performance is reported as the average accuracy over all tasks ($\mathbf{\overline{AP}}$). It is worth noting that Task-IL is considered the easiest scenario in continual learning~\cite{van2022three, BIR}. The \textit{None} and \textit{Joint} methods serve as informal lower and upper bounds, achieving $\mathbf{\overline{AP}}$ scores of $74.6\%$ and $97\%$, respectively. Figure~\ref{fig:ember_TIL} visualizes the progression of average accuracy across tasks, highlighting comparisons with the \textit{Joint} baseline.

As shown in Table~\ref{tab:ember_TIL}, ER consistently outperforms TAMiL, A-GEM, GR, RtF, and BI-R across all budget configurations and even surpasses GRS in some cases. However, \system\ variants significantly outperform all prior methods, particularly under lower budget constraints (100–1K). For example, \system-U achieves the highest $\mathbf{\overline{AP}}$ of $93.4\%$ and $93.7\%$ at budgets of 100 and 1K, respectively, outperforming GRS and all other approaches. Similarly, MADAR$^\theta$-U performs competitively, with $\mathbf{\overline{AP}}$ of $93.2\%$ at a 100 budget and $93.8\%$ at 1K.

As the budget increases, the performance gap among \system, ER, and GRS narrows; however, \system\ variants continue to either outperform or match other techniques. Notably, the \system-U variant of MADAR achieves the best overall performance at a budget of 20K, attaining a $\mathbf{\overline{AP}}$ of $95.8\%$, which closely approaches the \textit{Joint} baseline. Similarly, \system-R yields $\mathbf{\overline{AP}}$ of $95.6\%$ at 20K.



% In this set of experiments with EMBER, we have 20 tasks with 5 new classes in each task. Table~\ref{tab:ember_TIL} shows a summarized view of this set of experiments, where the performances are presented as the average accuracy over all tasks ($\mathbf{\overline{AP}}$). Note that Task-IL is considered the easiest scenario of continual learning~\cite{van2022three, BIR}. The {\em None} and {\em Joint} methods, which are the informal lower and upper bounds of this configuration, attain $\overline{AP}$ of $74.6\%$ and $\overline{AP}$ of $97.03\%$, respectively. Figure~\ref{fig:ember_TIL} illustrates the progression of average accuracy across tasks, showing how each method performs relative to the \textit{Joint} baseline.

% As we can see from Table~\ref{tab:combined_TIL}, ER outperforms TAMiL, A-GEM, GR, RtF, and BI-R in all budget configurations and outperforms GRS for few configurations. \system, on the other hand, outperforms all the prior methods significantly in lower budget constraints ($100$–$1K$). For instance, \system-U reaches $\mathbf{\overline{AP}}$ of 93.9\% with only 1K replay samples, compared with 93.6\% for GRS. The performance gap among MADAR, ER, and GRS gets closer as the budget increases; however, \system  variants continue to either outperform or perform on par with other techniques. In particular, the \system-U variant of MADAR outperforms all the other techniques and attains $\mathbf{\overline{AP}}$ of 95.8\% with a 20K replay budget, which is close to joint level performance.


Task-IL for AZ consists of 20 tasks, each with 5 non-overlapping classes. The results are summarized in Table~\ref{tab:az_TIL} and benchmarked against the \textit{None} and \textit{Joint} baselines, which achieve $\mathbf{\overline{AP}}$ values of $74.5\%$ and $98.8\%$, respectively. Figure~\ref{fig:az_TIL} illustrates the progression of average accuracy across tasks, showing how each method performs relative to the \textit{Joint} baseline.

As seen in Table~\ref{tab:az_TIL}, ER consistently outperforms TAMiL, AGEM, GR, RtF, BI-R, and GRS across most budget configurations, making it a strong baseline for comparison. At a low budget of 100 samples, \system-U achieves $\mathbf{\overline{AP}}$ of $88.1\%$, which is 4.5\% higher than ER's performance. Similarly, MADAR$^\theta$-U demonstrates competitive performance, achieving $87.9\%$ at the same budget.

As the budget increases, \system-U continues to deliver the best performance, reaching $\mathbf{\overline{AP}}$ scores of $94.5\%$ at a 1K budget and $98.1\%$ at a 10K budget, outperforming all other methods, including ER and GRS. At the highest budget of 20K, \system-U achieves an $\mathbf{\overline{AP}}$ of $98.7\%$, surpassing ER by 1.2\% and nearly matching the \textit{Joint} baseline. Notably, MADAR$^\theta$-U also performs well, achieving $98.1\%$. In contrast, \system-R and MADAR$^\theta$-R perform slightly lower but remain competitive, with $\mathbf{\overline{AP}}$ values of $97.9\%$ and $96.9\%$ at a 20K budget, respectively. These results indicate that ratio-based methods, while effective, are slightly less robust than uniform sampling in this scenario.

In summary, \system-U and MADAR$^\theta$-U consistently demonstrate better performance across most of the budget levels, particularly excelling at low-resource settings and achieving near-optimal results at higher budgets. These findings underscore the effectiveness of \system\ framework in Task-IL scenarios and their ability to approach joint-level performance with significantly fewer resources.


% Task-IL for AZ contains 20 tasks, each with 5 non-overlapping classes. Our results are shown in Table~\ref{tab:az_TIL}, compared against the {\em None} and {\em Joint} benchmarks, with $\mathbf{\overline{AP}}$ of 74.5\% and 98.8\%, respectively. Figure~\ref{fig:az_TIL} illustrates the progression of average accuracy across tasks, showing how each method performs relative to the \textit{Joint} baseline. As with EMBER, ER outperforms TAMiL, AGEM, GR, RtF, BI-R, and GRS for most budgets, so we use it for comparison. For a low budget of 100, \system-U achieves an $\overline{AP}$ of 88.1\%, 4.5\% higher than that of ER. For a higher budget of 20K, \system-U attains an $\overline{AP}$ of 98.7\%, which is 1.2\% higher than that of ER and very close to the joint level performance of 98.8\%.


% Overall, mirroring the success seen with the EMBER dataset, our proposed techniques also surpass previous work in Task-IL in the context of the AZ-Class dataset. Additionally, while ER and GRS shows significant improvement with an increased budget, the uniform variant of IFS of MADAR is more effective at every budget level.








\begin{table*}[!t]
\centering
\caption{Summary of Results for AZ Task-IL Experiments.}
\vspace{-0.3cm}
\label{tab:az_TIL}
\begin{tabular}{p{1.1cm}|l|c|c|c|c|c|c|c} 

% \toprule 

\multirow{2}{*}{\textbf{Group}} & \multirow{2}{*}{\textbf{Method}} & \multicolumn{7}{c}{\textbf{Budget}} \\ \cline{3-9}

&  & 100 & 500 & 1K & 5K & 10K & 15K & 20K \\ \midrule

\multirow{3}{*}{Baselines} 
& Joint  & \multicolumn{7}{c}{98.8$\pm$0.2} \\ 
& None   & \multicolumn{7}{c}{74.5$\pm$0.2} \\ 
& GRS    & 85.2$\pm$0.1 & 89.2$\pm$0.2 & 90.8$\pm$0.1 & 91.6$\pm$0.2 & 93.5$\pm$0.1 & 93.9$\pm$0.1 & 95.2$\pm$0.1 \\ \midrule

\multirow{6}{*}{\parbox{0.7cm}{Prior \\ Work}} 
& TAMiL  & 80.5$\pm$0.4 & 85.3$\pm$0.6 & 91.5$\pm$0.2 & 92.1$\pm$0.1 & 93.5$\pm$0.1 & 94.0$\pm$0.2 & 94.8$\pm$0.2 \\ 
& ER     & 83.6$\pm$0.2 & 90.2$\pm$0.1 & 92.3$\pm$0.3 & 95.6$\pm$0.1 & 96.2$\pm$0.1 & 96.8$\pm$0.2 & 97.5$\pm$0.2 \\ 
& AGEM   & 76.7$\pm$0.5 & 82.8$\pm$0.2 & 85.3$\pm$0.1 & 85.6$\pm$0.2 & 86.7$\pm$0.2 & 88.9$\pm$0.2 & 91.3$\pm$0.3 \\ 
& GR     & \multicolumn{7}{c}{75.6$\pm$0.2} \\ 
& RtF    & \multicolumn{7}{c}{74.2$\pm$0.3} \\ 
& BI-R   & \multicolumn{7}{c}{85.4$\pm$0.2} \\ \midrule

\multirow{4}{*}{\system} 
& \system-R & 86.0$\pm$0.3 & 90.3$\pm$0.2 & 92.4$\pm$0.1 & 95.8$\pm$0.2 & 96.7$\pm$0.1 & 97.1$\pm$0.1 & 97.9$\pm$0.2 \\ 
& \system-U & {\bf 88.1$\pm$0.3} & {\bf 92.9$\pm$0.2} & {\bf 94.5$\pm$0.3} & {\bf 97.2$\pm$0.2} & {\bf 98.1$\pm$0.1} & {\bf 98.5$\pm$0.1} & {\bf 98.7$\pm$0.1} \\ \cline{2-9}
& MADAR$^{\theta}$-R & 87.3$\pm$0.3 & {\bf 90.6$\pm$0.2} & 93.2$\pm$0.2 & 95.7$\pm$0.2 & 95.9$\pm$0.1 & 96.6$\pm$0.1 & 96.9$\pm$0.1 \\ 
& MADAR$^{\theta}$-U & {\bf 87.9$\pm$0.2} & {\bf 90.8$\pm$0.2} & {\bf 93.6$\pm$0.1} & {\bf 96.2$\pm$0.3} & {\bf 97.2$\pm$0.2} & {\bf 97.5$\pm$0.2} & {\bf 98.1$\pm$0.1} \\ 

\bottomrule

\end{tabular}
\vspace{-0.3cm}
\end{table*}



\begin{figure}[!t]
    \centering
    \begin{subfigure}{0.45\linewidth}
        \centering
        \includegraphics[width=1.0\linewidth]{figures_TIFS/AZ_TIL_IFS_RATIO.pdf}
        \label{fig:AZ_TIL_IFS_R}
        \vspace{-0.4cm}
        \caption{MADAR Ratio}
    \end{subfigure}
    \hfill
    \begin{subfigure}{0.45\linewidth}
        \centering
        \includegraphics[width=1.0\linewidth]{figures_TIFS/AZ_TIL_IFS_UNIFORM.pdf}
        \label{fig:AZ_TIL_IFS_U}
        \vspace{-0.4cm}
        \caption{MADAR Uniform}
    \end{subfigure}
    \vfill
    \begin{subfigure}{0.45\linewidth}
        \centering
        \includegraphics[width=1.0\linewidth]{figures_TIFS/AZ_TIL_AWS_RATIO.pdf}
        \label{fig:AZ_TIL_AWS_R}
        \vspace{-0.4cm}
        \caption{MADAR$^\theta$ Ratio}
    \end{subfigure}
    \hfill
    \begin{subfigure}{0.45\linewidth}
        \centering
        \includegraphics[width=1.0\linewidth]{figures_TIFS/AZ_TIL_AWS_UNIFORM.pdf}
        \label{fig:AZ_TIL_AWS_U}
        \vspace{-0.4cm}
        \caption{MADAR$^\theta$ Uniform}
    \end{subfigure}

    \caption{AZ Task-IL: Comparison of the MADAR-R, MADAR-U, MADAR$^\theta$-R, and MADAR$^\theta$-U with Joint baseline.}
    \label{fig:az_TIL}
    \vspace{-0.3cm}
\end{figure}


\subsection{Analysis and Discussion}\label{diss}


Our results demonstrate that MADAR yields markedly better performances compared to previous methods for both the EMBER and AZ datasets across all CL settings. This clearly indicates that diversity-aware replay is effective in preserving the stability of a CL-based system for malware classification, while prior CL techniques largely fail to achieve acceptable performance.


\paragraphX{\bf MADAR in low-budget settings.} In Domain-IL, MADAR achieves competitive performance even with a 1K budget, surpassing prior work by over 3 percentage points in EMBER and AZ. At higher budgets, ratio-based selection (\system-R and MADAR$^{\theta}$-R) achieves near Joint baseline performance (96.4\% in EMBER and 97.3\% in AZ) while using significantly fewer resources. This demonstrates MADAR’s efficiency in leveraging limited samples to achieve robust classification.


\paragraphX{\bf MADAR is both effective and scalable.} Traditional CL methods, including ER and AGEM, experience significant performance degradation as tasks increase. In contrast, MADAR maintains high accuracy across 20 Task-IL tasks, with \system-U achieving 95.8\% in EMBER and 98.7\% in AZ at a 20K budget, nearly matching the {\em Joint} baseline.




\paragraphX{\bf Ratio vs. Uniform Budgeting.} A consistent trend across our experiments is that ratio-based selection performs best in Domain-IL, whereas uniform-based selection is superior in Class-IL and Task-IL. MADAR$^{\theta}$-U reaches 91.5\% in AZ at 20K, significantly outperforming iCaRL and TAMiL. Furthermore, in EMBER, \system-U achieves near {\em Joint} baseline performance at just a 5K budget, underscoring the effectiveness of uniform selection in class-incremental settings. Intuitively, this makes sense because ratio budgeting for binary classification in the Domain-IL setting naturally captures the contributions of each family to the overall malware distribution. Additionally, since there are many small families in the Domain-IL datasets, uniformly sampling from them consumes budget while offering little improvement in malware coverage. In contrast, our Class-IL and Task-IL experiments perform classification across families, which is better supported by Uniform budgeting to maintain class balance and ensure coverage over all families. Moreover, in most settings we can leverage efficient representations using MADAR$^\theta$ to scale the approach regardless of feature dimension without significant loss of performance.



\paragraphX{\bf GRS remains a strong baseline at high budgets.} While MADAR consistently outperforms GRS in low-resource settings, GRS performs comparably at higher budgets, particularly in Domain-IL. This suggests that diversity-aware replay is most impactful when the number of available samples per class is limited, whereas uniform selection provides sufficient representation at larger budgets.















\if 0
Our results demonstrate that MADAR yields markedly better performances compared to previous methods for both the EMBER and AZ datasets across all CL settings. This clearly indicates that diversity-aware replay is effective in preserving the stability of a CL-based system for malware classification, while prior CL techniques largely fail to achieve acceptable performance.


In the Domain-IL scenario, MADAR consistently achieves better performance than all other methods, particularly at lower budgets. For example, MADAR's uniform and ratio variants surpass other methods with $\mathbf{\overline{AP}}$ values exceeding $93.6\%$ in EMBER and $95.7\%$ in AZ at a 1K budget. As the memory budget increases, the ratio-based variants (\system-R and MADAR$^\theta$-R) excel, approaching the \textit{Joint} baselines of $96.4\%$ for EMBER and $97.3\%$ for AZ. Notably, these results are achieved with significantly fewer replay samples compared to the \textit{Joint} baseline, highlighting MADAR's efficiency in leveraging limited resources.


In the Class-IL scenario, MADAR achieves remarkable improvements over prior methods, including iCaRL and TAMiL, on both EMBER and AZ datasets. For EMBER, \system-U achieves near \textit{Joint} baseline performance with a budget as low as 5K, outperforming iCaRL  method with fewer resources. Similarly, in AZ, MADAR$^\theta$-U reaches an impressive $\mathbf{\overline{AP}}$ of $91.5\%$ at a 20K budget, significantly surpassing prior techniques. Across both datasets, uniform variants (\system-U and MADAR$^\theta$-U) consistently outperform other methods, demonstrating their effectiveness in managing resources and adapting to evolving class distributions.


In the Task-IL scenario, MADAR outperforms prior methods by a significant margin for both the EMBER and AZ datasets, confirming that diversity-aware replay is effective for this scenario. For EMBER, \system-U achieves $\mathbf{\overline{AP}}$ values of $95.8\%$ at a 20K budget, effectively matching \textit{Joint} performance with a fraction of the resources. For AZ, MADAR$^\theta$-U attains $98.7\%$ at 20K, further underscoring the efficacy of diversity-aware techniques in resource-constrained settings.These findings highlight that the MADAR framework, particularly the uniform variant, not only matches but often exceeds the effectiveness of existing techniques, confirming its robustness across various budget levels in Task-IL.


The Ratio variants worked better for Domain-IL experiments, while Uniform variants worked well in Class-IL and Task-IL. Intuitively, this makes sense because ratio budgeting for binary classification in the Domain-IL setting naturally captures the contributions of each family to the overall malware distribution. Additionally, since there are many small families in the Domain-IL datasets, uniformly sampling from them consumes budget while offering little improvement in malware coverage. In contrast, our Class-IL and Task-IL experiments perform classification across families, which is better supported by Uniform budgeting to maintain class balance and ensure coverage over all families. Moreover, in most settings we can leverage efficient representations using MADAR$^\theta$ to scale the approach regardless of feature dimension without significant loss of performance.


Our results show that GRS performs very well, in some cases closer to the performances of MADAR. Indeed, uniform random sampling should be expected to be a strong baseline, since it provides an unbiased estimate of the true underlying distribution. MADAR is particularly effective in Class-IL and Task-IL, and for lower budgets in Domain-IL, while GRS generally performs as well as MADAR in higher-budget Domain-IL settings. We hypothesize that MADAR's diversity-aware approach is more important when the number of samples per class is limited. In our Domain-IL experiments, larger budgets enable a sufficient representation of the distributions of both classes with uniform selection, making MADAR useful only at smaller budget sizes. 
\fi 

















This paper presents a planning approach for effective and efficient joint motion generation for manipulators to cover a surface, aiming to minimize specific joint space costs.

\textit{Limitations} -- Our work has several limitations that suggest potential directions for future research. First, our method uses a heuristic to accelerate the traditional Joint-GTSP approach. While we provide empirical evidence of its efficiency in producing high-quality solutions, we cannot guarantee consistent performance in all scenarios.
Second, our bi-level hierarchical method reduces the size of GTSP. Future research could extend it to multiple levels to further improve performance, though this may produce misleading guide paths.
Third, we observe that both Joint-GTSP and H-Joint-GTSP tend to generate paths with frequent turns, a pattern also observed in the motions of prior work \cite{kaljaca2020coverage, zhang2024jpmdp}.  Future work should explore strategies to balance joint movements with other objectives such as motion smoothness.

\footnotetext{Visualization tool: \url{https://github.com/uwgraphics/MotionComparator}}
\textit{Implications} -- The hierarchical approach presented in this work enables effective and efficient coverage path planning for robot manipulators. 
This approach is beneficial to applications that require dexterous surface coverage, such as sanding, polishing, wiping, and sensor scanning. 


% \section{Related Work}
\label{sec:related}

Besides ciphertext side channels, timing and cache-based side-channel attacks also incur severe threats to cryptography applications~\cite{percival2005cache, osvik2006cache, zhang2012cross, yarom2014flush, liu2015last, yarom2014recovering, ryan2019return, aranha2020ladderleak}. 
To resist side-channel leakage, the constant-time was proposed.
Prior works abiding by this concept to defeat side-channel attacks are categorized into the following aspects.

\yz{not very relevant, please merge "related work" with "background"}

\noindent \textbf{Constant-time Verifying.}
A series of works~\cite{agat2000transforming, molnar2005program, barthe2006preventing, kopf2007transformational, almeida2013certified, mantel2015transforming} applied a policy called program counter, which balances the timing behavior of both branches.
Along with noninterference to verify the constant-time feature, VirtualCert~\cite{barthe2014system} and MEE-CBC~\cite{almeida2016verifiable} performed typing analysis in the CompCert~\cite{leroy2006formal}, enforcing the notion of noninterference that verifies the classic observation-equivalence.
Similarly, FlowTracker~\cite{rodrigues2016sparse} tracked the information-flow in the SSA form of LLVM programs to verify the noninterference.
Later, Dehesa-Azuara et al.~\cite{dehesa2017verifying} relaxed the noninterference into resource-aware metrics, where the resource can be execution time or cache behaviors.
In addition to verifying noninterference by equivalence, the reduced-based methods through the self-composition of programs or formulas were applied in~\cite{almeida2013formal, almeida2016verifying, chen2017precise, antonopoulos2017decomposition, yang2018lazy, blazy2019verifying, daniel2020binsec}.

\noindent \textbf{Constant-time Supplying.}
Conceptually, formally constructing high-assurance cryptography libraries shall fundamentally resolve the constant-time issues. 
Some work~\cite{zinzindohoue2016verified, zinzindohoue2017hacl} supplied verified cryptography libraries at the source level in F$^{*}$ language~\cite{swamy2016dependent}.
Differently, \cite{zinzindohoue2016verified} enforced a coding discipline for mitigating side channels by type-checking, while \cite{zinzindohoue2017hacl} proved freedom of timing side channels through manual pre- and post-conditions that at large rely on experts. 
Vale~\cite{bond2017vale} impelled the progress of constructing side-channel freedom of a cryptography program to an assembly-like code, i.e., Vale language. 
Further, Jasmin~\cite{almeida2017jasmin} and Fact~\cite{cauligi2019fact} were compiler-based formal frameworks that respectively transform Jasmin programs into assembly code, and timing-sensitive FaCT code into constant-time LLVM IR.

\noindent \textbf{Constant-time Repairing.}
Transforming existing programs into constant-time equivalents also plays a significant role in resisting side
channels. 
Prior works~\cite{agat2000transforming, kopf2007transformational, barthe2006preventing, molnar2005program, coppens2009practical} balanced the secret-dependent conditional branches by inserting dummy statements or paths.
Nevertheless, all these works only considered branches that are under the program counter security~\cite{molnar2005program}.
This means the branches of a balanced time property may still leak the secrets with more powerful attacks such as
cache-based attacks. 
Later, Barthe et al.~\cite{barthe2020formal} modified CompCert to enable it to capture constant-time during the compilation by proving the observational noninterference. 
Recently, SC-Eliminator~\cite{wu2018eliminating} removed secret-dependent branches by executing both real and decoy paths. 
Considering the unmodified addresses in the decoy path that may result in out-of-bounds memory accesses in~\cite{wu2018eliminating}, Soares and Pereira~\cite{soares2021memory} guaranteed memory-safe accesses in the decoy paths. 
Lastly, Borrello et al.~\cite{borrello2021constantine} proposed the linearization of control-flow and data-flow, which executes all possible code/data memory accesses.

Although the majority of cryptography applications are equipped with the constant-time feature, they cannot defeat the ciphertext side channels~\cite{li2021cipherleaks, li2022systematic, deng2023cipherh}.
Existing tools that fix cache-based side channels cannot be adapted to this scenario due to their distinct mitigation strategies.
Thus, defeating ciphertext side channels requires revisiting the root cause of this threat and developing new mitigation tools.

\section{Conclusion}

In conclusion, we develop \ours, a tuning-free method for personalized alignment that adapts prior work on trial-and-error fine-tuning with scaling inference compute.   
Instead of fine-tuning a model with synthetic negative samples, \ours uses them to augment the prompt and generate explanations that provide further fine-grained guidance on how to align outputs towards a desirable style.
On personalized text generation tasks, \ours outperforms other competitive tuning-free baselines and the previous state-of-the-art.  
\ours provides an approach for leveraging black box models for personalization without any fine-tuning. 


\bibliographystyle{plain}
\bibliography{reference}

\end{document}
