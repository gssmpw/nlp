\section{Related Work}
Robotic simulators are often used in research____, e.g., for development of methods, and design on new robotic systems____ or for training learning-based methods____.
  General-purpose simulators allow users to simulate wide range of robots (e.g., CoppeliaSim____, Webots____ or Ignition Gazebo____), but they may lack particular features (e.g., support for hardware-in-the-loop the specific robots or high-fidelity rendering).
  In this section, we focus solely on simulators suitable for \acp{uav}.
  We refer to the comprehensive survey____ about physical simulators.
  Simulators of \acp{uav} should support system dynamics, various environments, and basic sensors, including cameras, possibly with more advanced features like weather simulation (e.g., wind, rain, fog)____.
  An overview of related simulators is shown in Tab.~\ref{tab:comparison}.

  Ignition Gazebo____ is a general-purpose physics-based simulator.
  Gazebo uses OpenGL for rendering but lacks photorealistic rendering.
  It offers cameras and \acp{lidar} and can be extended by plugins to simulate GPS and barometer____.
  Gazebo features a set of predefined robots (e.g., DJI Mavic 2 PRO).
  UAVs can be controlled using PX4 and Ardupilot SITL flight-controllers.
  Several other simulators are based on Gazebo,
  e.g., the Hector Quadrotor package____, RotorS____
  and CrazyS____.
  RotorS____ is a rudimentary \ac{uav} simulator  
  providing several multirotor UAV models (e.g., AscTec Hummingbird \& Firefly) and  
  provides \ac{imu} and 3D pose for the \ac{uav}.
  Besides, it provides a simulation of the VI-Sensor. 
  This package also contains some example controllers, basic worlds, and a joystick interface. 

  Isaac Sim____ is a photorealistic high-fidelity simulator for various robotic platforms (mobile, legged, manipulators).
  The above mentioned simulators____ either lack the high-fidelity rendering, or do not provide high-level autonomy capabilities.

\begin{table}
\centering
{\small
\setlength{\tabcolsep}{1pt}
\caption{\label{tab:comparison}
Key features of UAV simulators.
Sensors are: I: IMU, G: GPS, SS: semantic segmentation, L:\ac{lidar},
R+D: both RGB image + depth image, 
$\diamond$: \ac{lidar} + intensity \ac{lidar},
$\dagger:$ not open-source in the time of writing}
\vspace{-8pt}
  \begin{tabular}{l@{\hspace{0pt}}cccc@{\hspace{-1pt}}c}
\toprule
Simulator & Rendering & Physics & Sensors & Autonomy \\
 \midrule
\cite{Webots} Webots
 & WREN  
 & ODE% \cite{russel2008ODE} 
% & IMU, GPS, RGB, \ac{lidar}
& I,G,RGB,L
   & \NO              \\
\cite{isaacsim} Isaac Sim     
 & Omniverse                
 & PhysX %~\cite{physx}                
% & IMU, GPS, RGB, Depth, SS. 
& I,G,R+D,SS
  & \NO              \\
\cite{guerra2019flighgoggles} FlightGoggles
 & Unity %~\cite{juliani2020unity}               
 & Flexible             
 %& IMU, RGB                                             
& I,RGB
   & \NO              \\
\cite{madaan2020airsim} AirSim
 & UE 4          
 & PhysX %~\cite{physx}   
 %& IMU, RGB+D,  SS.
 & I,R+D,SS
 & \NO              \\
\cite{song2021flightmare} Flightmare
 & Unity %~\cite{juliani2020unity}                    
 & Flexible             
% & IMU, RGB+D,  SS.                         
& I,R+D,SS
  & \NO              \\
\cite{cui2024fastsim}  Fastsim & Unity & Flexible & I,R+D,SS,L & \YES $\dagger$ \\
 \midrule
 \textbf{FlightForge} & 
 \textbf{UE 5} & \textbf{Flexible}    
 & \textbf{I,G,R+D,SS,L$\diamond$} & \YES   \\
 \bottomrule
\end{tabular}
      
}
\vspace{-2em}
\end{table}



  The open-source AirSim____ supports research in AI, computer vision, and learning-based approaches for autonomous vehicles (cars and UAVs), for which it offers many sensors  
  and supports software-in-the-loop and hardware-in-the-loop simulation
  with flight controllers (e.g., PX4 and ArduPilot).
  AirSim is based on the \ac{ue} 4 providing photorealistic rendering.
  AirSim can also emulate weather effects (e.g., rain, snow, fog).

  Flightmare is an open-source flexible modular quadrotor simulator____ based on the Unity game engine.
  Its central principle is the decoupling of a rendering and physics engine, 
  so the users can decide which physics engine will be running.
  The simulator provides an RGB-D camera with ground-truth depth and semantic segmentation, rangefinder and supports collision detection between the UAVs and the environment.
  Cameras can be further modified, users can change the field of view, focal length, or even lens distortion.
  Flightmare can simulate up to several hundred of agents in parallel, which is useful, e.g., in learning-based research.

  FlightGoggles____ focuses on photorealistic
  rendering; it even generates simulated environments 
  using photogrammetry (i.e., by reconstructing 3D shapes (meshes) from many photos) to enable realistic simulation of exteroceptive sensors like RGB-D cameras.
  The simulator is based on Unity, and it adopts the modular architecture where other parts (e.g., sensor simulation or collision detection) are implemented as modules.
  The 3D meshes are generated with different levels of detail for fast rendering and collision detection.
  FlightGoggles nodes and the API can be used with either ROS____ or LCM____.
  FlightGoggles can have moving obstacles or light, which can be controlled in real-time.

  FastSim____ is a recent simulator for UAVs based on Unity.
  It supports IMU, RGB+D, segmentation, and event cameras.
  Moreover, it claims to provide high-level autonomy modules (control, motion planning and mapping modules).
  However, in the time of writing this manuscript, FastSim is still not published as an open-source.

  %%}

  %%{ Methodology