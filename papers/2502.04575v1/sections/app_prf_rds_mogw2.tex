\subsection{Proof of \cref{thm:mog_w2_action}}
\label{app:prf:mog_w2_action}
\begin{proof}
The claim of smoothness follows from \citet[Lem. 7]{guo2025provable}. Throughout this proof, let $\phi$ and $\Phi$ denote the p.d.f. and c.d.f. of the standard normal distribution $\n{0,1}$, respectively. Unless otherwise specified, the integration ranges are assumed to be $(-\infty,\infty)$.

Note that 
\begin{align*}
    \pi(x)\e^{-\frac{\lambda}{2}x^2}&\propto\ro{\e^{-\frac{x^2}{2}}+\e^{-\frac{(x-m)^2}{2}}}\e^{-\frac{\lambda}{2}x^2}\\
    &=\e^{-\frac{\lambda+1}{2}x^2}+\e^{-\frac{\lambda m^2}{2(\lambda+1)}}\e^{-\frac{\lambda+1}{2}\ro{x-\frac{m}{\lambda+1}}^2}\\
    &=\frac{1}{1+\e^{-\frac{\lambda m^2}{2(\lambda+1)}}}\n{x\left|0,\frac{1}{\lambda+1}\right.}+\frac{\e^{-\frac{\lambda m^2}{2(\lambda+1)}}}{1+\e^{-\frac{\lambda m^2}{2(\lambda+1)}}}\n{x\left|\frac{m}{\lambda+1},\frac{1}{\lambda+1}\right.}.
\end{align*}

Define $S(\theta):=\frac{1}{1+m^2(1-\theta)^r}$, and let 
\begin{align*}
    \piu_s(x):\propto\pi(x)\e^{-\frac{1/s-1}{2}x^2}=w(s)\n{x|0,s}+(1-w(s))\n{x|sm,s},    
\end{align*}
where
$$w(s)=\frac{1}{1+\e^{-(1-s)m^2/2}},\quad w'(s)=-\frac{\e^{-(1-s)m^2/2}m^2/2}{(1+\e^{-(1-s)m^2/2})^2}.$$
By definition, $\pi_\theta=\piu_{S(\theta)}$. The p.d.f. of $\piu_s$ is 
$$f_s(x)=\frac{w(s)}{\sqrt{s}}\phi\ro{\frac{x}{\sqrt{s}}}+\frac{1-w(s)}{\sqrt{s}}\phi\ro{\frac{x-sm}{\sqrt{s}}},$$
and the c.d.f. of $\piu_s$ is 
$$F_s(x)=w(s)\Phi\ro{\frac{x}{\sqrt{s}}}+(1-w(s))\Phi\ro{\frac{x-sm}{\sqrt{s}}}.$$

We now derive a formula for calculating the metric derivative. From \citet[Thm. 2.18]{villani2021topics}, $W_2^2(\mu,\nu)=\int_0^1(F_\mu^{-1}(y)-F_\nu^{-1}(y))^2\d y$, where $F_\mu,F_\nu$ stand for the c.d.f.s of $\mu,\nu$. Assuming regularity conditions hold, we have
$$\lim_{\delta\to0}\frac{W_2^2(\piu_s,\piu_{s+\delta})}{\delta^2}=\lim_{\delta\to0}\int_0^1\ro{\frac{F_{s+\delta}^{-1}(y)-F_s^{-1}(y)}{\delta}}^2\d y=\int_0^1(\partial_sF^{-1}_s(y))^2\d y.$$
Consider change of variable $y=F_s(x)$, then $\de{y}{x}=f_s(x)$. As $x=F_s^{-1}(y)$, $(F_s^{-1})'(y)=\de{x}{y}=\frac{1}{f_s(x)}$. Taking derivation of $s$ on both sides of the equation $x=F_s^{-1}(F_s(x))$ yields 
\begin{align*}
    0&=\partial_sF_s^{-1}(F_s(x))+(F_s^{-1})'(F_s(x))\partial_sF_s(x)\\
    &=\partial_sF_s^{-1}(y)+\frac{1}{f_s(x)}\partial_sF_s(x).
\end{align*}
Therefore,
\begin{align*}
    \int_0^1(\partial_sF^{-1}_s(y))^2\d y=\int\ro{\frac{\partial_sF_s(x)}{f_s(x)}}^2f_s(x)\d x=\int\frac{(\partial_sF_s(x))^2}{f_s(x)}\d x.
\end{align*}

Consider the interval $x\in\sq{\frac{m}{2}-0.1,\frac{m}{2}+0.1}$, and fix the range of $s$ to be $[0.9,0.99]$. We have
$$\left\{
\begin{array}{ll}
     1-w(s)=\frac{1}{1+\e^{(1-s)m^2/2}}\asymp\frac{1}{\e^{(1-s)m^2/2}},&\forall m\gtrsim1 \\
     -w'(s)=\frac{\e^{(1-s)m^2/2}m^2/2}{(1+\e^{(1-s)m^2/2})^2}\asymp\frac{m^2}{\e^{(1-s)m^2/2}},&\forall m\gtrsim1
\end{array}
\right.$$
First consider upper bounding $f_s(x)$. We have the following two bounds:
$$\frac{w(s)}{\sqrt{s}}\phi\ro{\frac{x}{\sqrt{s}}}\lesssim\e^{-\frac{x^2}{2s}}\le\e^{-\frac{(m/2-0.1)^2}{2\times0.99}}\le\e^{-\frac{m^2}{8}},~\forall m\gtrsim1,$$
$$\frac{1-w(s)}{\sqrt{s}}\phi\ro{\frac{x-sm}{\sqrt{s}}}\lesssim\frac{1}{\e^{(1-s)m^2/2}}\e^{-\frac{(sm-x)^2}{2s}}=\exp\ro{-\frac{1}{2}\sq{\frac{(sm-x)^2}{s}+(1-s)m^2}}.$$
The term in the square brackets above is
\begin{align*}
    \frac{(sm-x)^2}{s}+(1-s)m^2&\ge\frac{1}{s}\ro{sm-\frac{m}{2}-0.1}^2+(1-s)m^2\\
    &=\frac{m^2}{4s}-0.2\ro{1-\frac{1}{2s}}m+\frac{0.01}{s}\\
    &\ge\frac{m^2}{4\times0.99}-0.1m+0.1\ge\frac{m^2}{4},~\forall m\gtrsim1.
\end{align*}
Hence, we conclude that $f_s(x)\lesssim\e^{-\frac{m^2}{8}}$.

Next, we consider lower bounding the term $(\partial_sF_s(x))^2$. Note that
\begin{align*}
    -\partial_sF_s(x)&=-w'(s)\ro{\Phi\ro{\frac{x}{\sqrt{s}}}-\Phi\ro{\frac{x-sm}{\sqrt{s}}}}\\
    &+w(s)\phi\ro{\frac{x}{\sqrt{s}}}\frac{x}{2s^{\frac32}}+(1-w(s))\phi\ro{\frac{x-sm}{\sqrt{s}}}\ro{\frac{x}{2s^{\frac32}}+\frac{m}{2s^{\frac12}}}.    
\end{align*}
As $x\in\sq{\frac{m}{2}-0.1,\frac{m}{2}+0.1}$ and $s\in[0.9,0.99]$, all these three terms are positive. We only focus on the first term. Note the following two bounds:
$$\left\{
\begin{array}{ll}
\Phi\ro{\frac{x}{\sqrt{s}}}\ge\Phi\ro{\frac{m}{2}-0.1}\ge\frac{3}{4},&\forall m\gtrsim1,\\
\Phi\ro{\frac{x-sm}{\sqrt{s}}}\le\Phi\ro{\frac{m/2+0.1-sm}{\sqrt{s}}}\le\Phi(-0.4m+0.1)\le\frac{1}{4},&\forall m\gtrsim1.
\end{array}
\right.$$
Therefore, we have
$$-\partial_sF_s(x)\gtrsim\frac{m^2}{\e^{(1-s)m^2/2}}.$$

To summarize, we derive the following lower bound on the metric derivative:
\begin{align*}
    |\dot\piu|_s^2&=\int\frac{(\partial_sF_s(x))^2}{f_s(x)}\d x\ge\int_{\frac{m}{2}-0.1}^{\frac{m}{2}+0.1}\frac{(\partial_sF_s(x))^2}{f_s(x)}\d x\\
    &\gtrsim\int_{\frac{m}{2}-0.1}^{\frac{m}{2}+0.1}\frac{m^4\e^{-(1-s)m^2}}{\e^{-m^2/8}}\d x\\
    &\gtrsim m^4\e^{\ro{s-\frac{7}{8}}m^2}\ge m^4\e^{\frac{m^2}{40}},~\forall s\in[0.9,0.99].
\end{align*}

Finally, recall that $S(\theta):=\frac{1}{1+m^2(1-\theta)^r}$, and $\pi_\theta=\piu_{S(\theta)}$. Hence, by chain rule of derivative, $|\dot\pi|_\theta=|\dot\piu|_{S(\theta)}|S'(\theta)|$. Let 
$$\Theta:=\{\theta\in[0,1]:~S(\theta)\in[0.9,0.99]\}=\sq{1-\ro{\frac{1/0.9-1}{m^2}}^{\frac1r},1-\ro{\frac{1/0.99-1}{m^2}}^{\frac1r}}.$$
We have
\begin{align*}
    \cA_r&=\int_0^1|\dot\pi|_\theta^2\d\theta=\int_0^1|\dot\piu|_{S(\theta)}^2|S'(\theta)|^2\d\theta\ge\int_\Theta|\dot\piu|_{S(\theta)}^2|S'(\theta)|^2\d\theta\\
    &\ge\min_{\theta\in\Theta}|S'(\theta)|\cdot\int_\Theta|\dot\piu|_{S(\theta)}^2|S'(\theta)|\d\theta=\min_{\theta\in\Theta}|S'(\theta)|\cdot\int_{0.9}^{0.99}|\dot\piu|_s^2\d s.
\end{align*}
Since 
$$|S'(\theta)|=\frac{m^2r(1-\theta)^{r-1}}{(1+m^2(1-\theta)^r)^2}\ge\frac{m^2r\ro{\frac{1/0.99-1}{m^2}}^{1-1/r}}{\ro{1+m^2\ro{\frac{1/0.9-1}{m^2}}}^2}\gtrsim m^{2/r}\gtrsim1,\forall \theta\in\Theta,$$
the proof is complete.
\end{proof}

\begin{remark}
    In the above theorem, we established an exponential lower bound on the metric derivative of the $\text{W}_\text{2}$ distance, given by $\lim_{\delta\to0}\frac{W_2(\piu_s,\piu_{s+\delta})}{|\delta|}$. In OT, another useful distance, the \textbf{Wasserstein-1 ($\text{W}_\text{1}$) distance}, defined as $W_1(\mu,\nu)=\inf_{\gamma\in\Pi(\mu,\nu)}\int\|x-y\|\gamma(\d x,\d y)$, is a lower bound of the $\text{W}_\text{2}$ distance. Below, we present a surprising result regarding the metric derivative of $\text{W}_\text{1}$ distance on the same curve of probability distributions. This result reveals an exponentially large gap between the $\text{W}_\text{1}$ and $\text{W}_\text{2}$ metric derivatives on the same curve, which is of independent interest. 
\end{remark}

\begin{theorem}
    Define the probability distributions $\piu_s$ as in the proof of \cref{thm:mog_w2_action}, for some large enough $m\gtrsim1$. Then, for all $s\in[0.9,0.99]$, we have
    $$\lim_{\delta\to0}\frac{W_1(\piu_s,\piu_{s+\delta})}{|\delta|}\lesssim1.$$
    \label{thm:mog_w1_metder}
\end{theorem}

\begin{proof}
Since $W_1(\mu,\nu)=\int|F_\mu(x)-F_\nu(x)|\d x$ \cite[Thm. 2.18]{villani2021topics}, by assuming regularity conditions, we have 
\begin{align*}
    \lim_{\delta\to0}\frac{W_1(\piu_s,\piu_{s+\delta})}{|\delta|}&=\int|\partial_sF_s(x)|\d x\\
    &\le\int\abs{w'(s)\ro{\Phi\ro{\frac{x}{\sqrt{s}}}-\Phi\ro{\frac{x-sm}{\sqrt{s}}}}}\d x\\
    &+\int \abs{w(s)\phi\ro{\frac{x}{\sqrt{s}}}\frac{x}{2s^{\frac32}}}\d x\\
    &+\int\abs{(1-w(s))\phi\ro{\frac{x-sm}{\sqrt{s}}}\ro{\frac{x}{2s^{\frac32}}+\frac{m}{2s^{\frac12}}}}\d x.  
\end{align*}

To bound the first term, notice that for any $\lambda>0$,
\begin{equation*}
    \Phi\ro{\frac{x}{\sqrt{s}}}-\Phi\ro{\frac{x-sm}{\sqrt{s}}}\lesssim\begin{cases}
        \sqrt{s}m\e^{-\frac{(x-sm)^2}{2s}},&\frac{x-sm}{\sqrt{s}}\ge\lambda;\\
        \sqrt{s}m\e^{-\frac{x^2}{2s}},&\frac{x}{\sqrt{s}}\le-\lambda;\\
        1,&\text{otherwise}.
    \end{cases}
\end{equation*}
Therefore, using Gaussian tail bound $1-\Phi(\lambda)\le\frac{1}{2}\e^{-\frac{\lambda^2}{2}}$, the first term is bounded by
\begin{align*}
    &\lesssim\frac{m^2}{\e^{(1-s)m^2/2}}\sq{2\sqrt{s}\lambda+sm+sm(1-\Phi(\lambda))+sm\Phi(-\lambda)}\\
    &\lesssim\frac{m^2}{\e^{(1-s)m^2/2}}[\lambda+m+\e^{-\frac{\lambda^2}{2}}]\stackrel{\lambda\gets \Theta(m)}{\lesssim}\frac{m^3}{\e^{(1-s)m^2/2}}=o(1).
\end{align*}

The second term is bounded by
\begin{align*}
    \lesssim\int\phi\ro{\frac{x}{\sqrt{s}}}|x|\d x=s\int\phi(u)|u|\d u\lesssim1.
\end{align*}

Finally, the third term is bounded by
\begin{align*}
    &\lesssim\frac{1}{\e^{(1-s)m^2/2}}\int\phi\ro{\frac{x-sm}{\sqrt{s}}}(|x|+m)\d x\\
    &\lesssim\frac{1}{\e^{(1-s)m^2/2}}\int\phi(u)(|u|+m)\d u\lesssim\frac{m}{\e^{(1-s)m^2/2}}=o(1).
\end{align*}
\end{proof}