\section{Proofs for \cref{sec:jar}}
\subsection{A Complete Proof of \cref{thm:jar}}
\label{prf:thm:jar}

\begin{proof}
By Girsanov theorem (\cref{lem:rn_path_measure_contd}), we have
$$\log\de{\Pr}{\Pl}(\xi)=\log\frac{\pit_0(\xi_0)}{\pit_T(\xi_T)}+\frac{1}{2}\int_0^T(\inn{\nabla\log\pit_t(\xi_t),\d\xi_t}+\inn{\nabla\log\pit_t(\xi_t),*\d\xi_t}).$$
We first prove the following result \citep[Eq. (15)]{vargas2024transport}: if $\d x_t=a_t(x_t)\d t+\sqrt{2}\d B_t$, then 
$$\int_0^T\inn{a_t(x_t),*\d x_t}=\int_0^T\inn{a_t(x_t),\d x_t}+2\int_0^T\tr\nabla a_t(X_t)\d t.$$

\begin{proof}
    Due to \cref{eq:bsi_ito}, it suffices to calculate $\sq{a(X),X}_T$. By It\^o's formula, we have
    $$\d a_t(x_t)=(\partial_ta_t(x_t)+\inn{\nabla a_t(x_t),a_t(x_t)}+\Delta a_t(x_t))\d t+\sqrt{2}\nabla a_t\d B_t,$$
    and hence
    $$\sq{a(X),X}_T=\sq{\int_0^\cdot\sqrt{2}\nabla a_t(x_t)\d B_t,\int_0^\cdot\sqrt{2}\d B_t}_T=\tr\int_0^T2\nabla a_t(x_t)\d t.$$
\end{proof}

Therefore, for $X\sim\Pr$, we have
$$\log\de{\Pr}{\Pl}(X)=\log\frac{\pit_0(X_0)}{\pit_T(X_T)}+\int_0^T(\inn{\nabla\log\pit_t(X_t),\d X_t}+\Delta\log\pit_t(X_t)\d t).$$

On the other hand, by It\^o's formula, we have
$$\d\log\pit_t(X_t)=\partial_t\log\pit_t(X_t)+\inn{\nabla\log\pit_t(X_t),\d X_t}+\Delta\log\pit_t(X_t)\d t.$$
Taking the integral, we immediately obtain \cref{eq:jar_rn}, and the proof is complete.
\end{proof}

\subsection{Proof of \cref{thm:jar_complexity}}
\begin{proof}
\label{prf:thm:jar_complexity}
The proof builds on the techniques developed in \citet[Thm. 1]{guo2025provable}. We define $\P$ as the path measure of the following SDE:
\begin{equation}
    \d X_t=(\nabla\log\pit_t+v_t)(X_t)\d t+\sqrt{2}\d B_t,~t\in[0,T];~X_0\sim\pit_0,
    \label{eq:jar_p}
\end{equation}
where the vector field $(v_t)_{t\in[0,T]}$ is chosen such that $X_t\sim\pit_t$ under $\P$ for all $t\in[0,T]$. According to the Fokker-Planck equation\footnote{We assume the existence of a unique curve of probability measures solving the Fokker-Planck equation given the drift and diffusion terms, guaranteed under mild regularity conditions \citep{lebris2008existence}.}, $(v_t)_{t\in[0,T]}$ must satisfy the PDE
$$\partial_t\pit_t=-\nabla\cdot(\pit_t(\nabla\log\pit_t+v_t))+\Delta\pit_t=-\nabla\cdot(\pit_tv_t),~t\in[0,T],$$
which means that $(v_t)_{t\in[0,T]}$ generates $(\pit_t)_{t\in[0,T]}$. The Nelson's relation (\cref{lem:nelson}) implies an equivalent definition of $\P$ as the path measure of
$$\d X_t=(-\nabla\log\pit_t+v_t)(X_t)\d t+\sqrt{2}\d\Bl_t,~t\in[0,T];~X_T\sim\pit_T.$$

Now we bound the probability of $\varepsilon$ relative error:
\begin{align}
    \prob\ro{\abs{\frac{\Zh}{Z}-1}\ge\varepsilon} & =\Pr\ro{\abs{\frac{\e^{-W}}{\e^{-\Delta F}}-1}\ge\varepsilon} =\Pr\ro{\abs{\de{\Pl}{\Pr}-1}\ge\varepsilon}      \nonumber\\
                                                &\le\frac{1}{\varepsilon}\E_{\Pr}{\abs{\de{\Pl}{\Pr}-1}}=\frac{2}{\varepsilon}\tv(\Pl,\Pr)\nonumber\\
                                                &\le\frac{2}{\varepsilon}(\tv(\P,\Pr)+\tv(\P,\Pl))\nonumber\\
                                                &\le\frac{\sqrt{2}}{\varepsilon}\ro{\sqrt{\kl(\P\|\Pr)}+\sqrt{\kl(\P\|\Pl)}}.\label{eq:jar_acc_bound}
\end{align}
In the second line above, we apply Markov inequality along with an equivalent definition of the TV distance: $\tv(\mu,\nu)=\frac{1}{2}\int\abs{\de{\mu}{\lambda}-\de{\nu}{\lambda}}\d\lambda$, where $\lambda$ is a measure that dominates both $\mu$ and $\nu$. The third line follows from the triangle inequality for TV distance, while the final line is a consequence of Pinsker's inequality $\kl\ge2\tv^2$.

By Girsanov theorem (\cref{lem:rn_path_measure,lem:rn_path_measure_contd}), it is straightforward to see that
$$\kl(\P\|\Pl)=\kl(\P\|\Pr)=\frac{1}{4}\E_{\P}\int_{0}^{T}\|v_t(X_t)\|^2\d t=\frac{1}{4}\int_{0}^{T}\|v_t\|^2_{L^2(\pit_t)}\d t.$$
Leveraging the relation between metric derivative and continuity equation (\cref{lem:metric}), among all vector fields $(v_t)_{t\in[0,T]}$ that generate $(\pit_t)_{t\in[0,T]}$, we can choose the one that minimizes $\|v_t\|_{L^2(\pit_t)}$, thereby making $\|v_t\|_{L^2(\pit_t)}=|\dot\pit|_t$, the metric derivative. With the reparameterization $\pit_t=\pi_{t/T}$, we have the following relation by chain rule:
$$|\dot\pit|_t=\lim_{\delta\to0}\frac{W_2(\pit_{t+\delta},\pit_t)}{|\delta|}=\lim_{\delta\to0}\frac{W_2(\pi_{(t+\delta)/T},\pi_{t/T})}{T|\delta/T|}=\frac{1}{T}|\dot\pi|_{t/T}.$$
Employing the change-of-variable formula leads to
$$\kl(\P\|\Pl)=\kl(\P\|\Pr)=\frac{1}{4}\int_0^T|\dot\pit|_t^2\d t=\frac{1}{4T}\int_0^1|\dot\pi|_\theta^2\d\theta=\frac{\cA}{4T}.$$
Therefore, it suffices to choose $T=\frac{32\cA}{\varepsilon^2}$ to make the r.h.s. of \cref{eq:jar_acc_bound} less than $\frac{1}{4}$.
\end{proof}