\section{Analysis of the Jarzynski Equality}
\label{sec:jar}
To elucidate how annealing works in the task of normalizing constant estimation, we first consider \textbf{annealed Langevin diffusion (ALD)}, which runs LD with a dynamically changing target distribution. We introduce a reparameterized curve $(\pit_t=\pi_{\frac{t}{T}})_{t\in[0,T]}$ for some large $T$ to be determined later, and define the ALD as the following SDE:
\begin{align}
    \d X_t&=\nabla\log\pit_t(X_t)\d t+\sqrt{2}\d B_t,~t\in[0,T];~X_0\sim\pit_0.
    \label{eq:jar_pr}
\end{align}

The following Jarzynski equality provides a connection between the work functional and the free energy difference, which naturally yields a method for normalizing constant estimation.

\begin{theorem}[Jarzynski equality \citep{jarzynski1997nonequilibrium}]
    Let $\Pr$ be the path measure of \cref{eq:jar_pr}, and define the work functional $W$ and the free energy difference $\Delta F$ as
    $$W(X):=\frac{1}{T}\int_0^T\partial_\theta V_\theta|_{\theta=\frac{t}{T}}(X_t)\d t,\qquad\Delta F:=-\log\frac{Z_1}{Z_0}.$$
    Then we have the following relation:
    $$\E_{\Pr}\e^{-W}=\e^{-\Delta F}.$$
    \label{thm:jar}
\end{theorem}
\vspace{-2em}

Below, we sketch the proof from \citet[Prop. 3.3]{vargas2024transport}, which offers a crucial aspect for our analysis. The complete proof is detailed in \cref{prf:thm:jar}.

\begin{sketchofproof}
    Let $\Pl$ be the path measure of the following backward SDE:
    \begin{equation}
        \d X_t=-\nabla\log\pit_t(X_t)\d t+\sqrt{2}\d\Bl_t,~t\in[0,T];~X_T\sim\pit_T.
        \label{eq:jar_pl}        
    \end{equation}
    Leveraging Girsanov theorem (\cref{lem:rn_path_measure}) and It\^o's formula, one can establish the following identity of the RN derivative, known as the \emph{Crooks fluctuation theorem} \citep{crooks1998nonequilibrium,crooks1999entropy}:
    \begin{equation}
        \log\de{\Pr}{\Pl}(X)=-\int_0^T(\partial_t\log\pit_t)(X_t)\d t=W(X)-\Delta F,\quad\text{a.s.}~X\sim\Pr,
        \label{eq:jar_rn}
    \end{equation}
    which directly implies JE by the identity $\E_{\Pr}{\de{\Pl}{\Pr}}=1$. 
\end{sketchofproof}

Assume for the moment that (i) $Z_0$ is known, (ii) we can exactly simulate \cref{eq:jar_pr}, and (iii) we can calculate the work functional $W(X)$ given any continuous trajectory $X$. According to \cref{thm:jar}, $\Zh:=Z_0\e^{-W(X)}$ with $X\sim\Pr$ is an unbiased estimator of $Z=Z_0\e^{-\Delta F}$. We establish an upper bound on the time $T$ required to run the ALD in order to satisfy the accuracy criterion \cref{eq:acc_whp} in the following theorem, whose proof is detailed in \cref{prf:thm:jar_complexity}.

\begin{theorem}
    Under \cref{assu:AC}, it suffices to choose $T=\frac{32\cA}{\varepsilon^2}$ to obtain $\prob\ro{\abs{\frac{\Zh}{Z}-1}\le\varepsilon}\ge\frac{3}{4}$.
    \label{thm:jar_complexity}
\end{theorem}

To illustrate the proof idea of \cref{thm:jar_complexity}, note that while the ALD (\cref{eq:jar_pr}) targets the distribution $\pit_t$ at time $t$, there is always a lag between $\pit_t$ and the actual law of $X_t$. Similarly, the backward SDE (\cref{eq:jar_pl}) can also be seen as a time-reversed ALD which targets $\pit_t$ at time $t$, and the same lag exists. This lag turns out to be the source of the error in the estimator $\Zh$.

In practice, to alleviate the issue of high variance in estimating free energy differences, \cite{vaikuntanathan2008escorted} proposed adding a compensatory drift term $v_t(X_t)$ to the ALD (\cref{eq:jar_pr}). Ideally, the optimal choice would eliminate the lag entirely, ensuring $X_t\sim\pit_t$ for all $t\in[0,T]$. Inspired by this, we compare the path measure of ALD $\Pr$ to the SDE having the perfect compensatory drift term, whose path measure $\P$ has marginal distribution $\pit_t$ at time $t$. To make possible the perfect match, $v_t$ must satisfy the Fokker-Planck equation. The Girsanov theorem (\cref{lem:rn_path_measure}) enables the computation of $\kl(\P\|\Pr)$ and $\kl(\P\|\Pl)$, which are related to $\|v_t\|_{L^2(\pit_t)}^2$. Finally, among all admissible drift terms $v_t$, \cref{lem:metric} suggests the optimal choice of $v^*_t=\lim_{\delta\to0}\frac{T_{\pit_t\to\pit_{t+\delta}}-\id}{\delta}$ to minimize this norm, thereby leading to the metric derivative $|\dot\pit|_t$ and the action $\cA$. Through this approach, we derive a bound not explicitly relying on isoperimetric assumptions.

A similar connection between free energy and action integral was discovered in stochastic thermodynamics \citep{sekimoto2010stochastic,seifert2012stochastic}, one paradigm for non-equilibrium thermodynamics. By the second law of thermodynamics, the averaged dissipated work, defined as the averaged work minus the free energy difference, i.e., $\cW_\mathrm{diss}:=\cW-\Delta F:=\E_{\Pr}W-\Delta F$, is non-negative. When the underlying process is modeled by an overdamped LD, $\cW_\mathrm{diss}$ can be quantified by an action integral divided by the length of the process \citep{aurell2011optimal,chen2020stochastic}. This follows from the observation that $\cW_\mathrm{diss}=\kl(\Pr\|\Pl)$ and then a similar argument to that above. This connection provides a finer description of the second law of thermodynamics \citep{aurell2012refined} over a finite time horizon. Finally, we also observe that our bound aligns with the $O\ro{\frac1T}$ decay rate of the variance of the work in \cite{mazonka1999exactly} (see also \citet[Chap. 4.1.4]{lelievre2010free}), computed when the curve consists of Gaussian distributions with linearly varying means.