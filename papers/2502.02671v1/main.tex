%%%%%%%% ICML 2025 EXAMPLE LATEX SUBMISSION FILE %%%%%%%%%%%%%%%%%

\documentclass{article}

% Recommended, but optional, packages for figures and better typesetting:
\usepackage{microtype}
\usepackage{graphicx}
\usepackage{csquotes}
\usepackage{subfigure}
\usepackage{booktabs} % for professional tables

% hyperref makes hyperlinks in the resulting PDF.
% If your build breaks (sometimes temporarily if a hyperlink spans a page)
% please comment out the following usepackage line and replace
% \usepackage{icml2025} with \usepackage[nohyperref]{icml2025} above.
%\usepackage{hyperref}
\usepackage{notation}
\usepackage{tcolorbox}

\definecolor{colorbluefull}{rgb}{0.25882352941176473, 0.5215686274509804, 0.9568627450980393}
\colorlet{colorblue}{colorbluefull!30}



% Attempt to make hyperref and algorithmic work together better:
\newcommand{\theHalgorithm}{\arabic{algorithm}}

% Use the following line for the initial blind version submitted for review:
\usepackage[preprint]{icml2025}

% If accepted, instead use the following line for the camera-ready submission:
% \usepackage[accepted]{icml2025}

% For theorems and such
\usepackage{amsmath}
\usepackage{amssymb}
\usepackage{mathtools}
\usepackage{amsthm}

% if you use cleveref..
\usepackage[capitalize,noabbrev]{cleveref}

%%%%%%%%%%%%%%%%%%%%%%%%%%%%%%%%
% THEOREMS
%%%%%%%%%%%%%%%%%%%%%%%%%%%%%%%%
\theoremstyle{plain}
\newtheorem{theorem}{Theorem}
\newtheorem{proposition}{Proposition}
\newtheorem{lemma}{Lemma}
\newtheorem{corollary}{Corollary}
\theoremstyle{definition}
\newtheorem{definition}{Definition}
\newtheorem{assumption}{Assumption}
\theoremstyle{remark}
\newtheorem{remark}{Remark}

\usepackage[inline]{enumitem}


% Todonotes is useful during development; simply uncomment the next line
%    and comment out the line below the next line to turn off comments
%\usepackage[disable,textsize=tiny]{todonotes}
\usepackage[textsize=tiny]{todonotes}
\newcommand{\todoDT}[1]{\todo[inline,color=blue!30]{\textbf{Daniil}: #1}}

% The \icmltitle you define below is probably too long as a header.
% Therefore, a short form for the running title is supplied here:
\icmltitlerunning{On Teacher Hacking in Language Model Distillation}

\begin{document}

\twocolumn[
\icmltitle{On Teacher Hacking in Language Model Distillation}

% It is OKAY to include author information, even for blind
% submissions: the style file will automatically remove it for you
% unless you've provided the [accepted] option to the icml2025
% package.

% List of affiliations: The first argument should be a (short)
% identifier you will use later to specify author affiliations
% Academic affiliations should list Department, University, City, Region, Country
% Industry affiliations should list Company, City, Region, Country

% You can specify symbols, otherwise they are numbered in order.
% Ideally, you should not use this facility. Affiliations will be numbered
% in order of appearance and this is the preferred way.
%\icmlsetsymbol{equal}{*}

\begin{icmlauthorlist}
\icmlauthor{Daniil Tiapkin}{x}
\icmlauthor{Daniele Calandriello}{gdm}
\icmlauthor{Johan Ferret}{gdm}
\icmlauthor{Sarah Perrin}{gdm}
\icmlauthor{Nino Vieillard}{gdm}
\icmlauthor{Alexandre Ram{\'e}}{gdm}
\icmlauthor{Mathieu Blondel}{gdm}
\end{icmlauthorlist}

\icmlaffiliation{gdm}{Google DeepMind}
\icmlaffiliation{x}{CMAP, {\'E}cole Polytechnique, Palaiseau, France; Work done during an internship at Google DeepMind.}
%\icmlaffiliation{sch}{School of ZZZ, Institute of WWW, Location, Country}

\icmlcorrespondingauthor{Daniil Tiapkin}{daniil.tiapkin@polytechnique.edu}
%\icmlcorrespondingauthor{Firstname2 Lastname2}{first2.last2@www.uk}

% You may provide any keywords that you
% find helpful for describing your paper; these are used to populate
% the "keywords" metadata in the PDF but will not be shown in the document
\icmlkeywords{Machine Learning, ICML}

\vskip 0.3in
]

% this must go after the closing bracket ] following \twocolumn[ ...

% This command actually creates the footnote in the first column
% listing the affiliations and the copyright notice.
% The command takes one argument, which is text to display at the start of the footnote.
% The \icmlEqualContribution command is standard text for equal contribution.
% Remove it (just {}) if you do not need this facility.

%\printAffiliationsAndNotice{}  % leave blank if no need to mention equal contribution
\printAffiliationsAndNotice{\icmlEqualContribution} % otherwise use the standard text.

\begin{abstract}
Post-training of language models (LMs) increasingly relies on the following two stages: (i) knowledge distillation, where the LM is trained to imitate a larger teacher LM, and (ii) reinforcement learning from human feedback (RLHF), where the LM is aligned by optimizing a reward model.
% In the second RLHF stage, a well-known challenge is reward hacking, where the LM over-optimizes the reward model, leading to degraded performance on the true objective, in line with Goodhart's law.
In the second RLHF stage, a well-known challenge is reward hacking, where the LM over-optimizes the reward model. Such phenomenon is in line with Goodhart's law and can lead to degraded performance on the true objective.
In this paper, we investigate whether a similar phenomenon, that we call teacher hacking, can occur during knowledge distillation. This could arise because the teacher LM is itself an imperfect approximation of the true distribution. To study this, we propose a controlled experimental setup involving: (i) an oracle LM representing the ground-truth distribution, (ii) a teacher LM distilled from the oracle, and (iii) a student LM distilled from the teacher.
Our experiments reveal the following insights. When using a fixed offline dataset for distillation, teacher hacking occurs; moreover, we can detect it by observing when the optimization process deviates from polynomial convergence laws. In contrast, employing online data generation techniques effectively mitigates teacher hacking. More precisely, we identify data diversity as the key factor in preventing hacking.
Overall, our findings provide a deeper understanding of the benefits and limitations of distillation for building robust and efficient LMs.
\end{abstract}
\newpage
\section{Introduction}
\label{sec:intro}

\begin{figure*}[tb]
    \centering
    \includegraphics[width=0.848\linewidth]{figs/circuitnn.pdf} 
    \caption{Illustration of differentiable CircuitNN. CircuitNN is designed based on differentiable NAND gates. After DAS is guided by PI and PO pairs of the truth table, CircuitNN can get the precise circuit architecture logic equivalent to the truth table.}
    \label{fig:circuitnn}
\end{figure*}

% 1. Describe the importance of logic synthesis
% 2. Existing Problems
% (a) Neural Architecture Search: Unstable, Predefined Setting, etc.
% (b) Circuit Generation: Probabilistic Model, Logic Equivalence

With the rapid advancement of technology, the scale of integrated circuits (ICs) has expanded exponentially. 
This expansion has introduced significant challenges in chip manufacturing, particularly concerning power and area metrics.
A primary objective in IC design is achieving the same circuit function with fewer transistors, thereby reducing power usage and area occupancy.

Logic synthesis~\cite{hachtel2005logicsynth}, a critical step in electronic design automation (EDA), transforms behavioral-level circuit designs into optimized gate-level circuits, ultimately yielding the final IC layout. 
The primary goal of logic synthesis is to identify the physical implementation with the fewest gates for a given circuit function. 
This task constitutes a challenging NP-hard combinatorial optimization problem. 
Current logic synthesis tools~\cite{brayton2010abc, wolf2013yosys} rely on human-designed heuristics, often leading to sub-optimal outcomes.

Differentiable architecture search (DAS) techniques~\cite{liu2018darts, chu2020darts} offer novel perspectives on addressing challenges in this problem.
Circuit functions can be represented through truth tables, which map binary inputs to their corresponding outputs. 
Truth tables provide a precise representation of input-output relationships, ensuring the design of functionally equivalent circuits.
Inspired by this, researchers~\cite{deepmind2024ai4sys, wang2024tnet} have begun exploring the application of DAS to synthesize circuits directly from truth tables.
Specifically, \citet{deepmind2024ai4sys} proposed CircuitNN, a framework that learns differentiable connection structures with logic gates, enabling the automatic generation of logic circuits from truth tables.
This approach significantly reduces the complexity of traditional circuit generation. 
Building on this, \citet{wang2024tnet} introduced T-Net, a triangle-shaped variant of CircuitNN, incorporating regularization techniques to enhance the efficiency of DAS.

Despite these advancements, several challenges remain. 
The computational complexity of DAS grows quadratically with the number of gates, posing scalability issues.
Although triangle-shaped architecture~\cite{wang2024tnet} partially mitigates this problem, redundancy persists. 
%Additionally, DAS is susceptible to converging to local optima, limiting the ability to search architectures that satisfy the given truth tables~\cite{liu2018darts}. 
%Furthermore, hyperparameters (network depth and layer width) require extensive searches, introducing complexity and prolonging the synthesis process. 
Additionally, DAS is susceptible to converging to local optima~\cite{liu2018darts} and hyperparameters (network depth and layer width) require extensive searches. 
The challenges arise from the vast search space in DAS. 
% Even with predefined settings for CircuitNN, finding a configuration that meets the truth table requires extensive trial and error during the DAS process. 
Intuitively, limiting the search space through predefined parameters (network depth, gates per layer, and connection probabilities) can significantly reduce the complexity.

Recent advances~\cite{openai2023gpt4, abramson2024alphafold3, esser2024sd3, li2024mar} in conditional generative models have demonstrated remarkable performance across language, vision, and graph generation tasks. 
Motivated by these developments, we propose a novel approach to circuit generation that generates preliminary circuit structures to guide DAS in generating refined circuits matching specified truth tables. 
Firstly, we introduce CircuitVQ, a tokenizer with a discrete codebook for circuit tokenization. 
Built upon our Circuit AutoEncoder framework~\cite{hou2022graphmae,li2023maskgae,wu2025mgvga}, CircuitVQ is trained through a circuit reconstruction task. 
Specifically, the CircuitVQ encoder encodes input circuits into discrete tokens using a learnable codebook, while the decoder reconstructs the circuit adjacency matrix based on these tokens.
Subsequently, the CircuitVQ encoder serves as a circuit tokenizer for CircuitAR pretraining, which employs a masked autoregressive modeling paradigm~\cite{chang2022maskgit, li2023mage}. 
In this process, the discrete codes function as supervision signals. 
After training, CircuitAR can generate discrete tokens progressively, which can be decoded into initial circuit structures by the decoder of the CircuitVQ. 
These prior insights can guide DAS in producing refined circuits that match the target truth tables precisely.

Our key contributions can be summarized as follows:
\begin{itemize}
\item We introduce CircuitVQ, a circuit tokenizer that facilitates graph autoregressive modeling for circuit generation, based on our Circuit AutoEncoder framework;
\item Develop CircuitAR, a model trained using masked autoregressive modeling, which generates initial circuit structures conditioned on given truth tables;
\item Propose a refinement framework that integrates differentiable architecture search to produce functionally equivalent circuits guided by target truth tables;
\item Comprehensive experiments demonstrating the scalability and capability emergence of our CircuitAR and the superior performance of the proposed circuit generation approach.
\end{itemize}

% Motivation
% (a) Diffusion (Vision, Graph), Autoregressive (Language, Vision)
% (b) Circuit Generation for Predefined Setting
% (c) Neural Architecture Search for Strict Logic Equivalence

% Contribution
% (a) Circuit Tokenizer (new transformer arch, training strategy)
% (b) CircuitAR (train and gen strategies, post-ar strategy)
% (c) Extensive Evaluation including BitD (Bit Distance) for Scalability


\section{Preliminaries}\label{sec:preliminaries}



%We denote by $(\Ac(x_\Ac),\Bc(x_\Bc))(z)$ a random execution of $\pi$ with private inputs $(x_\Ac,y_\Ac)$, and common input $z$.

%\Jnote{Move to DP}
% At the end of such an execution, the protocol outputs a public transcript denoted by the random variable $\trans_\pi(x_\Ac,x_\Ac,z)$ we denotes the common as $\out(\trans_\pi(x_\Ac,x_\Ac,z)$, and each party $\Pc \in \set{\Ac,\Bc}$ obtains his view denoted $\view^\Pc_\pi(x_\Ac,x_\Bc,z)$, which may also contain a ``local output'' \Jnote{Local} $\out^\Pc(x_\Ac,x_\Bc,z)$ (if the protocol specifies such an output). \Jnote{Common output, and parties output}


\subsection{Distributions and Random Variables}\label{sec:prelim:dist}
The support of a distribution $P$ over a finite set $\cS$ is defined by $\Supp(P) \eqdef \set{x\in \cS: P(x)>0}$. For a distribution or a random variable $D$, let $d\from D$ denote that $d$ was sampled according to $D$. Similarly,  for a set $\cS$, let $x \from \cS$ denote that $x$ is drawn uniformly from $\cS$, and denote by $\cU_{\cS}$ the uniform distribution over $\cS$. For a finite set $\cX$ and a distribution $C_X$ over $\cX$, we use the capital letter $X$ to denote the random variable that takes values in $\cX$ and is sampled according to $C_X$. The {\sf statistical distance} (\aka {\sf~variation distance}) of two distributions $P$ and $Q$ over a discrete domain $\cX$ is defined by $\sdist{P}{Q} \eqdef \max_{\cS\subseteq \cX} \size{P(\cS)-Q(\cS)} = \frac{1}{2} \sum_{x \in \cS}\size{P(x)-Q(x)}$. 
For a vector $x = (x_1,\ldots,x_n)$ and index $i\in [n]$, we let $x_{-i} = (x_1,\ldots,x_{i-1},x_{i+1},\ldots,x_n)$ and $x^{(i)} = (x_1,\ldots,x_{i-1}, -x_i, x_{i+1},\ldots,x_n)$, for a set $\cS \subseteq [n]$ we let $x_{\cS} = (x_i)_{i \in \cS}$ and $x_{-\cS} = (x_i)_{i \in [n]\setminus \cS}$, and for a vector $r \in \zo^n$ we let $x_r = (x_i)_{\set{i \colon r_i = 1}}$ and $x_{-r} = (x_i)_{\set{i \colon r_i = 0}}$.

%For $n \in \N$ we let $U_n$ be the uniform distribution over $\oo^n$, and let $S_n$ be the distribution induces by the sum of $n$ i.i.d.\ random variables, each is distributed according to $U_1$. Let $\cN(0,1)$ be the standard normal distribution.
%For a distribution $\cD$ and a function $f$, we define by $f(\cD)$ the distribution that is induced by the output of $f(x)$ for $x \from \cD$. 





% \begin{theorem}[\cite{McGregorMPRTV10}]\label{thm:sv-extracotr}
% 	\Enote{Remove if not needed}
% 	There is a constant $c$ to make the following holds. Let $X$ be an $\alpha$-SV source on $\{0,1\}^n$, let $Y$ be a source on $\{0,1\}^n$ with min-entropy at least $\beta n$ (independent from $X$), and let $Z=\ip{X,Y}\mbox{mod m}$ for some $m\in\mathbb{N}$. Then for every $\delta\in[0,1]$, the random variable $(Y,Z)$ is $\delta$-close to $(Y,U)$ where $U$ is uniform on $\mathbb{Z}_m$ and independent of $Y$, provided that
% 	$$
% 	n\geq c\cdot\frac{m^2}{\alpha\beta}\cdot\log(\frac{m}{\beta})\cdot\log(\frac{m}{\delta}).
% 	$$
% \end{theorem}



\Enote{I removed the definition of DP since it already appears in the intro}
\remove{
\subsection{Differential Privacy}\label{sec:prelim:DP}
We use the following standard definition of (information theoretic) differential privacy, due to \citet{DMNS06}. For notational convenience, we focus on databases over $\oo$.
\begin{definition}[Differentially private mechanisms]\label{def:mech}
	A randomized function $f\colon\oo^n\mapsto \zs$ is an {\sf $n$-size, $(\eps,\delta)$-differentially private mechanism} (denoted $(\eps,\delta)$-\DP) if for every neighboring $w,w'\in \oo^n$ and every function $g\colon \zs\mapsto \zo$, it holds that 
	$$
	\pr{g(f(w))=1}\leq e^{\eps}\cdot \pr{g(f(w'))=1} +\delta.
	$$ 	
	If $\delta=0$, we omit it from the notation.
\end{definition}
}


\subsubsection{Computational Differential Privacy}
There are several ways for defining computational differential privacy (see \cref{sec:related-works}). We use the most relaxed version due to \cite{BNO08}. For notational convenience, we focus on databases over $\oo$.
\begin{definition}[Computational differentially private mechanisms]\label{def:ComMech}
	A randomized function ensemble $f=\set{f_\pk\colon\oo^{n(\pk)}\mapsto \zs}$ is an {\sf $n$-size, $(\eps,\delta)$-computationally differentially private} (denoted $(\eps,\delta)$-$\CDP$) if for every poly-size circuit family $\set{\Ac_\pk}_{\pk\in \N}$, the following holds for every large enough $\pk$ and every neighboring $w,w'\in\oo^{n(\pk)}$:
	$$
	\pr{\Ac_\pk(f_\pk(w))=1}\leq e^{\eps(\pk)}\cdot \pr{\Ac_\pk(f_\pk(w'))=1} +\delta(\pk).
	$$ 
	If $\delta(\pk) = \negl(\pk)$, we omit it from the notation. 
\end{definition}



\subsubsection{Two-Party Differential Privacy}\label{sec:DP}
In this section we formally define distributed differential privacy mechanism (\ie protocols). %For the ease of notation, we consider protocol with no common input.

\begin{definition}\label{def:DP}%\Nnote{fix security parameter}
	A two-party protocol $\Pi=(\Ac,\Bc)$ is {\sf $(\eps,\delta)$-differentially private}, denoted $(\eps,\delta)$-$\DP$, if the following holds for every algorithm $\Dc$: let $\V^\Pc(x,y)(\pk)$ be the view of party $\Pc$ in a random execution of $\Pi(x,y)(1^\pk)$. Then for every $\pk,n \in \N$, $x\in \oo^n$ and neighboring $y,y'\in\oo^n$:
	\begin{align*}
	\pr{\Dc(V^\Ac(x,y)(\pk))=1}\le e^{\eps(\pk)}\cdot \pr{\Dc(V^\Ac (x,y')(\pk))=1}+\delta(\pk),
	\end{align*} 
	and for every $y\in \oo^n$ and neighboring $x,x'\in\oo^{n}$:
	\begin{align*}
	\pr{\Dc(V^\Bc(x,y)(\pk))=1}\le e^{\eps(\pk)}\cdot \pr{\Dc(V^\Bc (x',y)(\pk))=1}+\delta(\pk).
	\end{align*} 	
	Protocol $\Pi$ is {\sf $(\eps,\delta)$-computational differentially private}, denoted $(\eps,\delta)$-$\CDP$, if the above inequalities only hold for a non-uniform \ppt $\Dc$ and large enough $\pk$. We omit $\delta = \negl(\pk)$ from the notation. \footnote{Note that define we give for two-party differentially private protocols is a semi-honest definition, in which we ask for the security to hold when the parties interact in an honest execution of the protocol. Since we are proving a lower bound, starting from this weaker guarantee (as opposed to security against malicious players), yields a stronger result.}
\end{definition}
%We omit $\delta$ from the notation if $\delta$ is a negligible function of $n$.

%\Enote{simulation-based}
\begin{remark}[The definition for computational differential privacy we use]\label{rem:comDPChannel} 
	An alternative, stronger definition of computational differential privacy, known as simulation-based computational differential privacy, requires that the distribution of each party’s view be computationally indistinguishable from a distribution that ensures privacy in an information-theoretic sense. \cref{def:DP} is a weaker notion in comparison. Consequently, establishing a lower bound for a protocol that satisfies this weaker guarantee (as we do in this work) yields a stronger result.%Actually, our lower bound only requires the privacy to hold against \emph{uniform} external observer.
	%\Nnote{Maybe add: When only interesting in \Dp against external observer, the two definitions can be achieve using key-agreement and (single-party) \Dp mechanism. }
\end{remark}




\subsection{Useful Claims}
\remove{
In this section, we state generic lemmas and propositions that we will use later in our proofs.

The following lemma which we prove in \cref{sec:missing-proofs:distance-I}, measures the distance between two uniform stings conditioned one a random index $i$ either being fixed to $0$ or to $1$.

\def\distanceILemma{
    Let $R \la \zo^n$. For any (randomized) function $f:\{0,1\}^n\rightarrow \{0,1\}$ and $\alpha > 0$, it holds that
    \begin{align}\label{eq:f-alpha}
        \ppr{i \la [n]}{\size{\:\ex{f(R) \mid R_i = 0}-\ex{f(R) \mid R_i = 1}\:}\geq \alpha} \leq \frac{2}{n \alpha^2},
    \end{align}
    where the expectations are taken over $R$ and the randomness of $f$.
}

\begin{lemma}\label{lem:distance-I}
    \distanceILemma
\end{lemma}
}

The following two propositions state that given the output of a differentially private function, it is not possible to predict well even a random index (even if all other indexes are leaked). The first proposition handles the information-theoretic case and the second handles the computation case. Both propositions are proven in \cref{sec:missing-proofs:hard-to-guess}. 

\def\propHardToGuessInf{
    Let $f\colon \oo^n \rightarrow \cY$ be an $(\eps,\delta)$-\DP function, let $g \colon [n] \times \oo^{n-1} \times \cY \rightarrow \set{-1,1,\bot}$ be a (randomized) function, and let $X = (X_1,\ldots,X_n) \la \oo^n$. Then the following holds for every $i \in [n]$ where $X_i^* = g(i,X_{-i},f(X_1,\ldots,X_n))$:
    \begin{align*}
        \pr{X_i^* = X_i} \leq e^{\eps}\cdot \pr{X_i^* = -X_i} + \delta.
    \end{align*}
}

\begin{proposition}\label{prop:hard-to-guess-inf}
    \propHardToGuessInf
\end{proposition}


\def\propHardToGuessComp{
    Let $f = \set{f_{\pk} \colon \oo^{n(\pk)} \rightarrow \zo^{m(\pk)}}_{\pk \in \bbN}$ be an $(\eps,\delta)$-\CDP function ensemble, and let $\set{g_{\pk}}_{\pk \in \bbN}$ be a poly-size circuit family. Then, for large enough $\pk$ and $X = (X_1,\ldots,X_{n(\pk)}) \la \oo^{n(\pk)}$, the following holds for every $i \in [n(\pk)]$ where $X_i^* = g_{\pk}(i,X_{-i},f_{\pk}(X_1,\ldots,X_n))$:
    \begin{align*}
        \pr{X_i^* = X_i} \leq e^{\eps(\pk)}\cdot \pr{X_i^* = -X_i} + \delta(\pk).
    \end{align*}
}

\begin{proposition}\label{prop:hard-to-guess-comp}
    \propHardToGuessComp
\end{proposition}





\remove{
\Enote{Chao's old statement:}
\begin{lemma}\label{lem:distance-I-old}
        Let $R \la \zo^n$. 
	For any function $f:\{0,1\}^n\rightarrow \{0,1\}$ and $\alpha<0.01$, it holds that
	$$
	\Pr_{i\la[n]}\left[\: \size{\:\mathbb{E}[f(R) \mid R_i = 0]-\mathbb{E}[f(R) \mid R_i = 1]\:}\geq \alpha\right]\leq \frac{2+2\log(\frac{1}{\alpha})}{n\alpha^2}.
	$$
\end{lemma}
\begin{proof}
	Define $S_1=\{r \in \zo^n \colon f(r)=1\}$. Then for any $i\in[n]$, we have
	$$
	\begin{array}{rl}
		\size{\mathbb{E}[f(R) \mid R_i = 0]-\mathbb{E}[f(R) \mid R_i = 1]}
		&=\size{\Pr[R\in S_1|R_i=0]-\Pr[R\in S_1|R_i=1]}\\
		&=\size{\frac{\Pr[R_i=0|R\in S_1]\cdot\Pr[R\in S_1]}{\Pr[R_i=0]}-\frac{\Pr[R_i=1|R\in S_1]\cdot\Pr[R\in S_1]}{\Pr[R_i=1]}}\\
		&=\frac{2\size{S_1}}{2^n}\size{\Pr[R_i=0|R\in S_1]-\Pr[R_i=1|R\in S_1]}
	\end{array}
	$$
	When $|S_1|\leq \alpha\cdot 2^{n-1}$, we have $\size{\mathbb{E}[f(R) \mid R_i = 0]-\mathbb{E}[f(R) \mid R_i = 1]}\leq\frac{2\size{S_1}}{2^n}\leq \alpha$ for any $i\in[n]$. Hence, in the following, we assume $|S_1|> \alpha\cdot 2^{n-1}$.

	%Define $I_{bad}=\{i|\size{\Pr[R_i=0|R\in S_1]-\Pr[R_i=1|R\in S_1]}>2\alpha\}$ and $k=\size{I_{bad}}$, then for any $i\notin I_{bad}$, we have 
    %$$
    %\begin{array}{rl}
    %    2\alpha&\geq \size{\Pr[R_i=0|R\in S_1]-\Pr[R_i=1|R\in S_1]}\\
    %    &=\size{\frac{\Pr[R\in S_1|R_i=0]\cdot\Pr[R_i=0]}{\Pr[R\in S_1]}-\frac{\Pr[R\in S_1|R_i=1]\cdot\Pr[R_i=1]}{\Pr[R\in S_1]}}\\
    %    &=\size{\Pr[R\in S_1|R_i=0]-\Pr[R\in S_1|R_i=1]}\cdot\frac{1}{2\Pr[R\in S_1]}\\
    %    &\geq \size{\mathbb{E}[f(R) \mid R_i = 0]-\mathbb{E}[f(R) \mid R_i = 1]}\cdot \frac{1}{2},
    %\end{array}
    %$$ 
    %where the last inequality is because $\Pr[R\in S_1]\leq 1$. So that $\size{\mathbb{E}}[f(R) \mid R_i = 0]-\mathbb{E}[f(R) \mid R_i = 1]\leq %4\alpha$.
    Define $I_{bad}=\{i \colon \size{\Pr[R_i=0|R\in S_1]-\Pr[R_i=1|R\in S_1]} \geq 2\alpha\}$ and $k=\size{I_{bad}}$, and denote $I_{bad}=\{i_1,\dots,i_k\}$. Define $(X_{i_1}, \ldots X_{i_k}) = (R_{i_1},\dots,R_{i_k})\mid_{R \in S_1}$. 
    Consider the min-entropy
	$$
	\begin{array}{rl}
		H_{min}(X_{i_1},\dots,X_{i_k})&\leq H(X_{i_1},\dots,X_{i_k})\\
		&\leq \sum_{j=1}^k H(X_{i_j})\\
		&\leq k\cdot \left(-(\frac{1}{2}+2\alpha)\cdot\log(\frac{1}{2}+2\alpha)-(\frac{1}{2}-2\alpha)\cdot\log(\frac{1}{2}-2\alpha)\right)\\
            &=k\cdot \left(-(\frac{1}{2}+2\alpha)\cdot(\log(1+4\alpha)-1)-(\frac{1}{2}-2\alpha)\cdot(\log(1-4\alpha)-1)\right)\\
            &=k\cdot \left(1-(\frac{1}{2}+2\alpha)\cdot\log(1+4\alpha)-(\frac{1}{2}-2\alpha)\cdot\log(1-4\alpha)\right),
		
	\end{array}
	$$
	where $H_{min}(Y)$ is the minimum entropy of $Y$ and $H(Y)$ is the Shannon entropy of $Y$.\Enote{add to preliminaries.}
        The third inequality holds since by the definition of $I_{bad}$, for every $j \in [k]$ it holds that $\size{\pr{X_{i_j} = 1}-\pr{X_{i_j} = 0}} > 2\alpha$, and therefore $H(X_{i_j}) \leq H(1/2 + 2\alpha)$\Enote{define}.
	
	Therefore, there exists $b_1,\dots,b_k\in\{0,1\}$, such that 
	
	\begin{align}\label{eq:min-entropy-result}
		\Pr\left[(R_{i_1},\ldots,R_{i_k}) = (b_1,\ldots,b_k) \mid R\in S_1\right]
		&= \pr{(X_{i_1},\ldots,X_{i_k}) = (b_1,\ldots,b_k)}\\
		&= 2^{-H_{min}(X_{i_1},\dots,X_{i_k})}\nonumber\\
		&\geq 2^{k\cdot \left(-1+(\frac{1}{2}+2\alpha)\cdot\log(1+4\alpha)+(\frac{1}{2}-2\alpha)\cdot\log(1-4\alpha)\right)}.\nonumber
	\end{align}
	
	Let $S_{bad}=\{r \in \zo^n  \colon \set{(r_{i_1},\ldots,r_{i_k}) = (b_1,\ldots,b_k)} \land \set{r\in S_1}\}$.
	It holds that
	\begin{align*}
		|S_{bad}|
		&= \size{S_1} \cdot \Pr\left[(R_{i_1},\ldots,R_{i_k}) = (b_1,\ldots,b_k) \mid R\in S_1\right]\\
		&\geq \alpha\cdot 2^{n-1}\cdot2^{k\cdot \left(-1+(\frac{1}{2}+2\alpha)\cdot\log(1+4\alpha)+(\frac{1}{2}-2\alpha)\cdot\log(1-4\alpha)\right)},
	\end{align*} 
	where the inequality holds by \cref{eq:min-entropy-result} and since $\size{S_1} \geq \alpha\cdot 2^{n-1}$.
	Notice that any string in $S_{bad}$ depends on at most $n-k$ bits. It implies that $|S_{bad}|\leq 2^{n-k}$. Therefore, we have
	$$
	\begin{array}{rl}
		&2^{n-k}\geq \alpha\cdot 2^{n-1}\cdot2^{k\cdot \left(-1+(\frac{1}{2}+2\alpha)\cdot\log(1+4\alpha)+(\frac{1}{2}-2\alpha)\cdot\log(1-4\alpha)\right)} \\
		\Rightarrow& n-k \geq \log \alpha+n-1+k\cdot \left(-1+(\frac{1}{2}+2\alpha)\cdot\log(1+4\alpha)+(\frac{1}{2}-2\alpha)\cdot\log(1-4\alpha)\right)\\
		\Rightarrow& 1-\log \alpha \geq k\cdot((\frac{1}{2}+2\alpha)\cdot\log(1+4\alpha)+(\frac{1}{2}-2\alpha)\cdot\log(1-4\alpha))\\
		\Rightarrow& 1-\log \alpha \geq k\cdot(4\alpha\cdot\log(1+4\alpha)+(\frac{1}{2}-2\alpha)\cdot\log(1-16\alpha^2))\\
        \Rightarrow& 1-\log\alpha \geq k\cdot(15.9\alpha^2-8\alpha^2+32\alpha^3)=k\cdot(7.9\alpha^2+32\alpha^3)>0.5k\alpha^2\\
		\Rightarrow& k\leq \frac{2-2\log \alpha}{\alpha^2} = \frac{2+2\log (1/\alpha)}{\alpha^2},
	\end{array}
	$$
	Where the third transition holds since 
	\begin{align*}
		\lefteqn{(\frac{1}{2}+2\alpha)\cdot\log(1+4\alpha)+(\frac{1}{2}-2\alpha)\cdot\log(1-4\alpha)}\\
		&= 4\alpha\cdot\log(1+4\alpha) + (\frac{1}{2}-2\alpha)\paren{\log(1+4\alpha)+\log(1-4\alpha)}\\
		&= 4\alpha\cdot\log(1+4\alpha)+(\frac{1}{2}-2\alpha)\cdot\log(1-16\alpha^2),
	\end{align*}
	and the forth transition holds since $4\alpha\cdot\log(1+4\alpha)+(\frac{1}{2}-2\alpha)\cdot\log(1-16\alpha^2) > 15.9\alpha^2-8\alpha^2+32\alpha^3$ for $\alpha < 0.01$.
	Thus, we conclude that 
	$$
	\Pr_{i\la[n]}\left[\size{\mathbb{E}[f(R) \mid R_i=0]-\mathbb{E}[f(R) \mid R_i = 1]}\geq \alpha\right]\leq \frac{k}{n}\leq \frac{2+2\log (1/\alpha)}{n\alpha^2}.
	$$
\end{proof}
}


\subsection{Channels and Two-Party Protocols}\label{sec:protocol}

\paragraph{Channels.}A channel is simply a distribution of a pair of tuples defined as follows. 
\begin{definition}[Channels]\label{def:channel} A {\sf channel} $C_{(X,U)(Y,V)}$ of size $\isize$ over alphabet $\Sigma$ is a probability distribution over $(\Sigma^\isize \times\zo^\ast) \times(\Sigma^\isize \times\zo^\ast)$. The ensemble $C_{(X,U)(Y,V)}= \set{C_{(X_\pk,U_\pk)(Y_\pk,V_\pk)}}_{\pk\in \N}$ is an $\isize$-size channel ensemble, if for every $\pk\in \N$, $C_{(X_\pk,U_\pk)(Y_\pk,V_\pk)}$ is an $\isize(\pk)$-size channel. %We denote a channel of size one by a \emph{single-bit} channel. 
We refer to $X$ and $Y$ as the {\sf local outputs}, and to $U$ and $V$ as the {\sf views}.	
\end{definition}

We view a  channel as the experiment in which there are two parties $\Ac$ and $\Bc$.  Party $\Ac$ receives ``output'' $X$ and ``view'' $U$, and party $\Bc$ receives ``output'' $Y$ and ``view'' $V$. Unless stated otherwise, the channels we consider are over the alphabet $\Sigma = \oo$. We naturally identify channels with the distribution that characterizes their output.








\subsubsection{Two-Party Protocols}

A two-party protocol $\Pi=(\Ac,\Bc)$ is \ppt if the running time of both parties is polynomial in their input length. We let $\Pi(x,y)(z)$ or $(\Ac(x),\Bc(y))(z)$ denote a random execution of $\Pi$ on a common input $z$, and private inputs $x,y$.%We assume \wlg that a protocol has a common output (part of its transcript).\Jnote{This is not really the case we consider in this paper..}

\begin{definition}[Oracle-aided protocols]\label{def:ChannelAidedProtocol}
	In a two-party protocol $\Pi$ with oracle access to a {\sf protocol} $\Psi$, denoted $\Pi^\Psi$, the parties make use of the \textit{next-message function} of $\Psi$.\footnote{The function that on a partial view of one of the parties, returns its next message.} In a two-party protocol $\Pi$ with oracle access to a {\sf channel} $C_{Z W}$, denoted $\Pi^C$, the parties can jointly invoke $C$ for several times. In each call, an independent pair $(z,w)$ is sampled according to $C_{Z W}$, one party gets $z$, the other gets $w$.
\end{definition}


\begin{definition}[The channel of a protocol]\label{def:ChannlOfProtocol}
	For a no-input two-party protocol $\Pi= (\Ac,\Bc)$, we associate the channel $C_\Pi$, defined by $\C_\Pi= C_{(X, U),(Y, V)}$, where $X$ and $Y$ are the local outputs of $\Ac$ and $\Bc$ (respectively) and
	$U$ and $V$ are the local views of $\Ac$ and $\Bc$ (respectively).
    
	For a two-party protocol $\Pi$ that gets a security parameter $1^\pk$ as its (only, common) input, we associate the channel ensemble $ \set{C_{\Pi(1^\pk)}}_{\pk\in \N}$. 
\end{definition}

\begin{definition}[$(\alpha,\gamma)$-Accurate channel]\label{def:accurate-func}
	A channel $C = C_{(X, U),(Y, V)}$ is {\sf $(\alpha,\gamma)$-accurate for the function $f$}, if $\ppr{C}{\size{\out(V)-f(X,Y)}\leq \alpha}\ge \gamma$, where $\out(V)$ is the designated output.
    A channel ensemble $C_{(X, U),(Y, V)}= \set{C_{(X_\pk, U_\pk),(Y_\pk, V_\pk)}}_{\pk\in \N}$ is  $(\alpha,\gamma)$-accurate for  $f$ if $C_{(X_\pk, U_\pk),(Y_\pk, V_\pk)}$ is $(\alpha(\pk),\gamma(\pk))$-accurate for $f$, for every $\pk \in \N$.
\end{definition}

\subsubsection{Differentially Private Channels}\label{sec:DPChannel}
Differentially private channels are naturally defined as follows:
\begin{definition}[Differentially private channels]\label{def:DPChannel}
	An $n$-size channel $C = C_{(X, U),(Y, V)}$ with $X, Y$ over $\oo^n$ 
	is {\sf$(\eps,\delta)$-differentially private} (denoted $(\eps,\delta)$-$\DP$) if for every $x \in \Supp(X)$ there exists an $n$-size $(\eps,\delta)$-$\DP$ mechanisms $\Mc_x$ such that $(X,Y,U) \equiv (X,Y,\Mc_X(Y))$, and for every $y \in \Supp(Y)$ there exists an $n$-size $(\eps,\delta)$-$\DP$ mechanisms $\Mc_y'$ such that $(X,Y,V) \equiv (X,Y,\Mc_Y'(X))$. In addition, we say that the channel is \emph{uniform} if $X$ and $Y$ are independent random variables uniformly distributed in $\oo^n$. 
\end{definition}

\begin{definition}[Computational differentially private channels]\label{def:CDPChannel}
	An $n$-size channel ensemble $C = \set{C_{(X_\pk, U_\pk),(Y_\pk, V_\pk)}}_{\pk\in\N}$ with $X_\pk, Y_\pk$ over $\oo^n$ 
	is {\sf$(\eps,\delta)$-computationally differentially private} (denoted $(\eps,\delta)$-$\CDP$) if for every ensemble $\set{x_\pk \in \Supp(X_\pk)}_{\pk\in\N}$ there exists an $n$-size $(\eps,\delta)$-\CDP mechanisms ensemble $\set{\Mc_{x_\pk}}_{\pk\in\N}$ such that $(X_\pk,Y_\pk,U_\pk) \equiv (X_\pk,Y_\pk,\Mc_{X_\pk}(Y_\pk))$, for every $\pk\in\N$, and for every ensemble $\set{y_\pk \in \Supp(Y_\pk)}_{\pk\in\N}$ there exists an $n$-size $(\eps,\delta)$-$\CDP$ mechanisms ensemble $\set{\Mc'_{y_\pk}}_{\pk\in\N}$ such that $(X_\pk,Y_\pk,V_\pk) \equiv (X_\pk,Y_\pk,\Mc_{Y_\pk}'(X_\pk))$ for every $\pk\in \N$. In addition, we say that the channel is \emph{uniform} if $X_\pk$ and $Y_\pk$ are independent random variables uniformly distributed in $\{\pm 1\}^n$ for all $\pk\in\N$.
\end{definition}




% \begin{lemma}~\label{lem:dp-sv-source}
% 	Let $P$ be an $\varepsilon$-DP randomized protocol. Let $X$ and $Y$ be independent random variables uniformly distributed in $\{\pm 1\}^n$ and let random variable $\Pi(X,Y)$ denote the transcript of running $P(X,y)$. Then for every $\pi\in Supp(\Pi)$, the random variables corresponding to the inputs conditioned on transcript $\pi$, $X_\pi$ and $Y_\pi$, are independent $e^{-\varepsilon}$-strong SV source.
% \end{lemma}





\subsubsection{Weak Erasure Channel (\WEC)}

\begin{definition}[\WEC]\label{def:WEC}
	A channel $((O_A,V_A), (O_B,V_B))$ with $O_A \in \set{0,1}$ and $O_B \in \set{0,1,\bot}$ is a {\sf weak erasure channel}, denoted $(\alpha,p,q)$-$\WEC$, if:
	\begin{itemize}
		%\item $O_A\in \set{-1,1}$ and $O_B\in \set{-1,1,\bot}$.
		\item Random erasure: $\pr{O_B = \perp} = 1/2$.
		
		\item Agreement: $\pr{O_A\ne O_B\mid O_B\ne \bot}\le \alpha$.
		
		\item Secrecy:
		
		\begin{enumerate}
			\item For every algorithm $\Dc$ it holds that\label{WEC:item:A}
			\begin{align*}
				%\size{\pr{\Ac(O_A,V_A) = 1 \mid O_B \neq \perp} - \pr{\Ac(O_A,V_A) = 1 \mid O_B = \perp}} \le p
				\size{\pr{\Dc(V_A) = 1 \mid O_B \neq \perp} - \pr{\Dc(V_A) = 1 \mid O_B = \perp}} \le p
			\end{align*}
			(Alice doesn't know if $O_B = \perp$.)
			
			\item For every algorithm $\Dc$ it holds that\label{WEC:item:B}
			\begin{align*}
				\pr{\Dc(V_B) = O_A \mid O_B=\bot} \leq \frac{1+q}{2}.
			\end{align*}
			(i.e., if $O_B=\bot$, Bob don't know what is the value of $O_A$).
			
			%\item $SD((O_A U|O_B=\bot),(O_A U|O_B\ne \bot))\le p$ (The sender don't know if $O_B=\bot$).
			
			%\item $SD(V O_A|O_B=\bot,V(-O_A)|O_B=\bot)\le q$ (If $O_B=\bot$, Bob don't know what the value of $O_A$).
		\end{enumerate}
	\end{itemize}
   We say that a channel ensemble $C=\set{C_\pk}_{\pk\in N}$ is a {\sf computational weak erasure channel}, denoted $(\alpha,p,q)$-\CompWEC, if for every \ppt algorithm $\Dc$ and every sufficiently large $\pk\in\N$, $C_\pk$ satisfies the properties stated in the items above, where the secrecy property holds with respect to a \ppt algorithm $\Dc$. A protocol $\Lambda$ is said to be $(\alpha,p,q)$-$\CompWEC$, if the ensemble induces by the protocol (that is, $C=\set{C_{\Lambda(\pk)}}_{\pk\in\N}$) is $(\alpha,p,q)$-$\CompWEC$.  
\end{definition}



\subsubsection{Approximate Weak Erasure Channel (\AWEC)}\label{sec:AWEC}

\begin{definition}[\AWEC]\label{def:AWEC}
	A channel $C = ((O_A,V_A), (O_B,V_B))$ over $([-n,n] \times \zo^*) \times (([-n,n] \cup \bot)  \times \zo^*)$ is an {\sf approximate weak erasure channel}, denoted $(\ell,\alpha,p,q)$-\AWEC if:
	\begin{itemize}
		
		\item Random erasure: $\pr{O_B = \perp} = 1/2$.
		
		\item Accuracy: $\pr{\size{O_A - O_B} > \ell \mid O_B \ne \bot}\le \alpha$.
		
		\item Secrecy:
		
		\begin{enumerate}
			\item For every algorithm $\Dc$ it holds that\label{AWEC:item:A}
			\begin{align*}
				%\size{\pr{\Ac(O_A,V_A) = 1 \mid O_B \neq \perp} - \pr{\Ac(O_A,V_A) = 1 \mid O_B = \perp}} \le p
				\size{\pr{\Dc(V_A) = 1 \mid O_B \neq \perp} - \pr{\Dc(V_A) = 1 \mid O_B = \perp}} \le p
			\end{align*}
			(Alice doesn't know if $O_B=\bot$).
			
			\item For every algorithm $\Dc$ it holds that\label{AWEC:item:B}
			\begin{align*}
				\pr{\size{\Dc(V_B) - O_A} \leq 1000 \ell \mid O_B=\bot} \leq q.
			\end{align*}
			(i.e., if $O_B=\bot$, Bob can't estimate the value of $O_A$ with error $\leq 1000 \ell$).
		\end{enumerate}
	\end{itemize}
     We say that a channel ensemble $C=\set{C_\pk}_{\pk\in N}$ is a {\sf computational approximate weak erasure channel}, denoted $(\ell,\alpha,p,q)$-\CompAWEC, if for every \ppt algorithm $\Dc$ and every sufficiently large $\pk\in\N$, $C_\pk$ satisfies the properties stated in the items above. A protocol $\Gamma$ is said to be $(\ell,\alpha,p,q)$-$\CompAWEC$, if the ensemble induced by the protocol (that is, $C=\set{C_{\Gamma(\pk)}}_{\pk\in\N}$) is $(\ell,\alpha,p,q)$-$\CompAWEC$.  
\end{definition}

We will make use of the following lemma, which shows that for some choices of the parameters, \AWEC implies \WEC. The lemma is proven in \cref{sec:AWEC-to-WEC}.

\begin{lemma}\label{lemma:AWEC-to-WEC}
	For every $\ell> 0$, there exists a \ppt protocol $\Lambda = (\Pc_1,\Pc_2)$ such that given an oracle access to an $(\ell,\alpha,p,q)$-\AWEC $C$, the channel $\tilde{C}$ induced by $\Lambda^C$ is $(\alpha'=\alpha+0.001,\: p' = p ,\:  q' = 1/2 + 2(q+0.01))$-\WEC.
	Furthermore, the proof is constructive in a black-box manner:
	\begin{enumerate}
		\item There exists an oracle-aided \ppt algorithm $\Ec_1$ such that for every channel $C = ((\OA,\VA), (\OB,\VB))$ and algorithm $\Dc$ violating the \WEC secrecy property~\ref{WEC:item:A} of $\tilde{C}$, algorithm $\Ec_1^{\Dc}$ violates the \AWEC secrecy property~\ref{AWEC:item:A} of $C$.
		
		\item There exists an oracle-aided \ppt algorithm $\Ec_2$ such that for every channel $C = ((\OA,\VA), (\OB,\VB))$ and algorithm $\Dc$ violating the \WEC secrecy property~\ref{WEC:item:B} of $\tilde{C}$, algorithm $\Ec_2^{\Dc}$ violates the \AWEC secrecy property~\ref{AWEC:item:B} of $C$.
	\end{enumerate}
\end{lemma}

Since \cref{lemma:AWEC-to-WEC} is constructive, the following is an immediate corollary.
\begin{corollary}\label{cor:CompAWEC to CompWEC}
There exists an oracle aided \ppt protocol $\Lambda$, such that given a protocol $\Gamma$ that induces $(\ell,\alpha,p,q)$-\CompAWEC, it holds that $\Lambda^\Gamma$ is $(\alpha'=\alpha+0.001,\: p' = p ,\:  q' = 1/2 + 2(q+0.01))$-\CompWEC.  
\end{corollary}
\begin{proof}[Proof of \ref{cor:CompAWEC to CompWEC}]
Let $\Lambda$ be the \ppt algorithm guaranteed  by Lemma \ref{lemma:AWEC-to-WEC}. Given an $(\ell,\alpha,p,q)$-\CompAWEC protocol $\Gamma$, we define $\Lambda(\pk)={\Lambda^{\Gamma(\pk)}(\pk)}$. Assume towards a contradiction that $\Lambda$ is not a $(\alpha',p',q')$-\CompWEC. It follows that there exists a \ppt $\Dc$ that for infinity many $\pk\in\N$ contradicts one of the \WEC secrecy properties of channel ensemble $\set{C_{\Lambda(\pk)}}_{\pk\in\N}$. Fix $\pk\in\N$ for which this holds. By Lemma \ref{lemma:AWEC-to-WEC}, there exists a \ppt $\Ec^\Dc$ that for every such $\pk$  contradicts one of the secrecy properties of the channel $C_{\Gamma(\pk)}$. This implies that for infinity many $\pk\in\N$, $\Ec^\Dc$  contradict the secrecy of the channel ensemble $\set{C_{\Gamma(\pk)}}_{\pk\in\N}$, which is a contradiction since this would means that $\Gamma$ is not a $(\ell,\alpha,p,q)$-\CompAWEC.       
\end{proof}



\subsection{Oblivious Transfer (\OT)}

\paragraph{Secure Computation.}
We use the standard notion of securely computing a functionality, \cf  \cite{Goldreich04}.
\begin{definition}[Secure computation]\label{def:SFE}
	A two-party protocol {\sf securely computes a functionality $f$}, if it does so according to the real/ideal paradigm.   We add the term perfectly/statistically/computationally/non-uniform computationally, if the simulator's output is  perfect/statistical/computationally indistinguishable/  non-uniformly indistinguishable from  the real distribution.  The protocol have the above notions of security {\sf against semi-honest  adversaries}, if its security only  guaranteed to holds against an adversary that follows the prescribed protocol.   Finally, for the case of perfectly secure computation, we naturally apply the above notion also to the non-asymptotic case: the protocol with no security parameter perfectly  compute a functionality $f$.
	
	A two-party protocol {\sf securely computes a functionality ensemble $f$ with oracle to a channel $C$}, if it does so according to the above definition when the parties have access to a trusted party computing $C$. All the above adjectives naturally extend to this setting.
\end{definition}

\paragraph{Oblivious Transfer.}
The (one-out-of-two) oblivious transfer functionality is defined as follows.
\begin{definition}[oblivious transfer functionality $f_{\OT}$]\label{def:OTfunc}
	The oblivious transfer functionality over $\zo \times (\zs)^2$ is defined by  $f_{\OT} (i,(\sigma_0,\sigma_1)) = (\perp,\sigma_i)$.
\end{definition}
A protocol is $\ast$ secure OT,   for \\$\ast\in \set{\text{semi-honest statistically/computationally/computationally non-uniform}}$, if it  compute the $f_{\OT}$  functionality with $\ast$ security.





% \begin{definition}[Computational oblivious transfer, semi-honest model]
% A protocol $\Pi=(\Ac,\Bc)$ is a semi-honest 1-out-of-2 computational oblivious transfer (comp-OT) protocol if the following holds. Given a common input $1^{\pk}$, the parties $\Ac$ and $\Bc$ run the protocol $\Pi(1^\pk)$ (in an honest manner) and    
% $\Ac$ outputs $X=(m_1,m_2)\in \zo\times\zo$ and has a view $U$ and $\Bc$ outputs $Y=(i,\hat{m})\in\zo\times\zo$ and has a view $V$, and the following properties are satisfied:
% \begin{enumerate}
%     \item \textbf{Correctness:} 
%     $\pr{\hat{m}\neq m_i}<\negl(\pk).$ 
    
%     \item \textbf{A's Privacy:} For every \ppt $\Dc$ and every sufficiently large $\pk$:
%     $\pr{\Dc(V)=m_{i-1}}<(1+\negl(\pk))/2$
    
%     \item \textbf{B's Privacy:} For every \ppt $\Dc$ and every sufficiently large $\pk$:
%     $\pr{\Dc(U)=i}<(1+\negl(\pk))/2$  
% \end{enumerate}
% \end{definition}

We make use of the following useful results by Wullschleger on oblivious transfer amplification from weak channels.
\begin{theorem}[\cite{Wullschleger09}, from \WEC to statistically secure \OT]\label{thm:WEC TO OT IT}
    There exists an oracle aided protocol $\Pi$ such that the following holds: Given a $(\alpha,p,q)$-\WEC $C$, if $44(\alpha+p)\le 1-q$ then $\Pi^{C}(1^\pk)$ is a semi-honest statistically secure \OT.
\end{theorem}

The following computational version of \cref{thm:WEC TO OT IT} is implicit in \cite{Wullschleger09} and is based on the computational proof explicitly stated in \cite{Wul07} (see Section 6 in \cite{Wullschleger09} for discussion).   

\begin{theorem}[\cite{Wullschleger09,   Wul07}, from \CompWEC to computinally secure \OT]\label{thm:WEC TO OT Comp}
    There exists an oracle aided protocol $\Pi$ such that the following holds: Given a $(\alpha,p,q)$-\CompWEC protocol $\Lambda$, if $44(\alpha+p)\le 1-q$ then $\Pi^{\Lambda}$ is a semi-honest computational secure \OT.
\end{theorem}



% \begin{definition}[Computational 1-out-of-2 Oblivious Transfer, semi-honest model]
% A protocol $\Pi=(\Ac,\Bc)$ is a semi-honest 1-out-of-2 $(\eps,\alpha,\beta)$-oblivious transfer (OT) protocol if the following holds. 

% The parties $\Ac$ and $\Bc$ run the protocol (in an honest manner) and    
% $\Ac$ outputs $X=(m_1,m_2)\in \zo\times\zo$ and has a view $U$ and $\Bc$ outputs $Y=(i,\hat{m})\in\zo\times\zo$ and has a view $V$, and following properties are satisfied:
% \begin{enumerate}
%     \item \textbf{Correctness:} 
%     $\pr{\hat{m}\neq m_i}<\eps.$ 
    
%     \item \textbf{A's Privacy:} For every adversary $\Dc$:
%     $\pr{\Dc(V)=m_{i-1}}<(1+\alpha)/2$
    
%     \item \textbf{B's Privacy:} For every adversary $\Dc$: $\pr{\Dc(U)=i}<(1+\beta)/2$  
% \end{enumerate}
% \end{definition}

\section{\label{sec:method}Methodology}

Each SIEM system uses its own RDL to define threat detection rules, and each RDL has its own schema.
For example, the Splunk SIEM uses the SPL to define its threat detection rules.
The task of understanding threat detection rules and recommending relevant MITRE ATT\&CK techniques (or sub-techniques) requires complex reasoning skills.
In the case of LLMs, this can be achieved with a technique called prompt chaining in which each task is divided into multiple sub-tasks in order to understand the complex reasoning behind the task.
Therefore, we employ a multi-phase architecture based on prompt chaining that leverages the power of LLMs to take a SIEM rule defined in any RDL and map it to relevant MITRE ATT\&CK techniques using the power of LLMs.
Our approach is based on the following intuitions:
\begin{itemize}[nosep,leftmargin=*]
    \item \textit{LLMs' implicit knowledge}: LLMs possess deep understanding of diverse RDLs. This enables them to interpret any rule, regardless of the RDL it is defined in, and convert it into comprehensible natural language text.
    \item \textit{LLMs' similarity comparison capability}: LLMs are adept at analyzing and comparing textual descriptions. 
    They can intelligently assess the similarity between two textual inputs to establish a meaningful connection.
\end{itemize}

\methodName has two main phases: (1) the rule to text translation phase, and (2) the MITRE ATT\&CK techniques recommendation phase.
These two phases in the pipeline include six key steps to determine relevant TTPs, as illustrated in Figure~\ref{fig:r2t}.

Although LLMs excel at translating SIEM rules into natural language, they often lack critical domain-specific contextual information related to IoCs in the rules.
To overcome this limitation, the \textit{rule to text translation} phase includes three steps: IoC extraction, contextual information retrieval, and natural language translation.

The workflow begins with the extraction of IoCs from the rules (for example, processes, log source, event codes, and file names) that the rule searches for in the logs (step (1)).In the next sstep a web search agent performs the task of obtaining additional contextual information about the IoCs discovered ((step 2)).
By incorporating this additional domain-specific information, the pipeline enhances the language translation, resulting in a more accurate and meaningful interpretation of SIEM rules.
The rule itself and the IoCs' contextual additional information from the previous stage are then used to translate the rule from RDL to natural language (step (3)).

The \textit{MITRE ATT\&CK techniques} recommendation phase of the pipeline includes the following three steps.
The rule in processed in data source identification step in which the probable origin of the data is identified (step (4)).
The description of the rule is then used to determine probable MITRE ATT\&CK techniques based on the implicit knowledge of the LLM (step (5)).
Finally, using chain-of-thought~\cite{wei2022chain} prompting, the most relevant techniques are extracted from the list of probable techniques (step (6)).
Each step of our method is further described in detail below.


% [bb=0 0 1440 900,width=1.43\linewidth,height=0.9\textwidth]
\begin{figure*}[htbp]
   \includegraphics[width=\textwidth]{Images/stages.jpg}
    
   \caption{An illustration of the different steps in \methodName.}
   \label{fig:stages}
\end{figure*} 

\subsection{IoC Extraction}
The context associated with a SIEM detection rule is crucial for its accurate interpretation and effective application. 
Obtaining this contextual understanding requires comprehensive analysis of the embedded IoCs in the SIEM rule.
In the first step, \methodName systematically identifies and extracts all IoCs, identifying the types of IoCs and their corresponding values that form the foundational elements of the detection rules. 
Leveraging the LLM's inherent understanding of rule structures and IoCs, we employ a zero-shot prompting approach for this task. 
Zero-shot prompting enables the direct extraction of IoCs from the rules without requiring extensive pre-training on specific datasets.

\noindent The result of this stage is a dictionary structure, where:
\begin{itemize}[nosep,leftmargin=*]
    \item Keys represent types of IoC, such as processes, files, IP addresses, and log sources.
    \item Values are lists containing specific IoC details, such as process names, file names, IP addresses, and log source identifiers.
\end{itemize}

In the example depicted in Figure~\ref{fig:stages}(a), the pipeline processes a rule for which relevant MITRE ATT\&CK techniques need to be recommended. 
The IoC extractor LLM produces a dictionary structure as output, organizing the IoCs in a structured format to support subsequent stages in the analysis pipeline. 



\subsection{Contextual Information Retrieval}
In this step, an LLM agent is employed to retrieve relevant information pertaining to the IoCs extracted from the rule.
A REACT agent~\cite{react} was used in this case to generate both reasoning traces and task-specific actions in an interleaved manner.
REACT agents interact with external tools to retrieve additional information that leads to more factual and reliable responses.
The LLM agent conducts a systematic search across web resources to gather additional contextual information for each IoC value present in the rule. 
This step addresses LLMS' lack of up-to-date knowledge or specialized domain expertise (which is critical to understanding the role and significance of the IoCs in the rule), without the need for retraining or fine-tuning.
Figure~\ref{fig:stages}(b) presents an example in which the rule includes the process name \texttt{soaphound.exe} as an IoC.
As can be seen, the web search results indicate that \texttt{soaphound.exe} is being used for active directory (AD) enumeration, which is important for the understanding of the attack. 

\subsection{Natural Language Translation}

The translation of detection rules into natural language textual descriptions fulfills three key objectives:
\begin{enumerate}[nosep,leftmargin=*]
    \item \textbf{Ensures that \methodName is format-agnostic}: It converts rules defined in various RDL formats into a generic, unstructured text format, ensuring compatibility with different SIEM systems, regardless of the specific rule format.
    \item \textbf{Provides contextual explanation}: It includes all relevant contextual information to produce a concise and comprehensible explanation of the rule.
    \item \textbf{Enhances the comprehension for LLMs}: It enables LLMs to more effectively compare the translated rule with descriptions in the MITRE ATT\&CK framework by providing a unified textual representation.
\end{enumerate}
To achieve these objectives, a zero-shot prompting technique is employed. 
The input to the LLM comprises two components:
\begin{itemize}
    \item \textbf{Syntactical information}: The rule itself, providing the structural and operational details.
    \item \textbf{Contextual information}: Details of the IoCs extracted from the rule, providing semantic insights into the rule's intent and function.
\end{itemize}
The LLM utilizes these inputs to generate a natural language textual description of the rule. 
This transformation not only ensures a more interpretable representation but also facilitates further steps of analysis and comparison, particularly in aligning the rule with MITRE ATT\&CK techniques and sub-techniques.



\subsection{Data Source or Mitigation Identification}
Identifying the most relevant data component or mitigation associated with the rule description in this step is critical for filtering out irrelevant MITRE ATT\&CK techniques (or sub-techniques) in subsequent steps of the pipeline.
In the MITRE ATT\&CK framework, data sources represent various categories of information that can be gathered from sensors or logs. 
These data sources include \textit{data components}, which are specific attributes or properties within a data source that are directly relevant to detecting a particular technique or sub-technique~. 
For example, in the context of the rule described in Figure~\ref{fig:stages}(a), the term \texttt{Endpoint.Processes} indicates that the activity is happening on an endpoint. 
Presence of the terms such as, \texttt{soaphound.exe}, \texttt{--buildcache}, \texttt{--certdump} and etc. indicate that the rule searches for command line execution of an executable named \texttt{soaphound.exe} with specific parameters. 
Therefore, the appropriate data source in this example is \textit{Command}, with the corresponding data component being \textit{Command Execution}.
Additionally, \textit{mitigations} are defined as categories of technologies or strategies that can prevent or reduce the impact of specific techniques or sub-techniques. 
The MITRE ATT\&CK framework explicitly establishes relationships between data components, mitigations, and techniques (or sub-techniques), enabling a systematic approach for identifying relevant elements.

To identify the most relevant data component or mitigation associated with a given rule description, we utilize agentic retrieval augmented generation (RAG), which incorporates an AI Agent-based implementation of the RAG framework.
Data from the MITRE ATT\&CK framework, specifically related to data components and mitigations, is stored in a vector database (e.g., ChromaDB). 
The process begins with the rule description from the previous stage, which serves as the input to the AI Agent. 
The LLM-powered agent automatically generates a search query tailored to retrieve relevant information from the RAG database.

For each query, the system retrieves the five most similar documents from the database, each containing contextual information about data components or mitigations. 
These documents are then utilized by the LLM agent to contextualize the rule description. 
By comparing the content of these retrieved documents with the rule description, the LLM agent determines and outputs the most relevant data component or mitigation along with a chain-of-thought as to why the data component or mitigation is related to the rule.


\subsection{Probable Technique Recommendation}

In this step, an LM agent is utilized to propose probable MITRE ATT\&CK techniques (and sub-techniques) that may be relevant to the description of the provided rule.
We used a REACT agent in this step as well to utilize both implicit and explicit knowledge during reasoning.
For explicit knowledge, the agent searches the MITRE ATT\&CK framework to obtain the list of probable techniques (and sub-techniques).
The natural language description of the rule from the previous step serves as input to the LLM agent.
The output of this stage consists of a list of JSON objects, each containing the MITRE technique ID, technique name, and technique description as seen in Figure~\ref{fig:stages}(c).

Throughout our experiments, we observed that as the number of recommendations ($k$) increases, both the framework's average recall and precision initially improve, however beyond a certain threshold of $k$, the %average 
precision begins to decline.
Based on these observations(please refer Table~\ref{tab:results3}), we selected a $k$-value of 11 to ensure a high recall.



\subsection{Relevant Technique Extraction}
In this step, \methodName refines the set of probable MITRE ATT\&CK techniques identified in the previous stage by eliminating irrelevant entries.
This step in the pipeline serves two primary purposes: (1) to enhance precision while maintaining recall achieved in previous step, and (2) to provide a clear rationale for the selection of the labels, ensuring transparency and interpretability of the mapping process.
This refinement process is grounded in the assumption that LLMs are effective for text similarity matching tasks.

The process comprises two key steps:
\begin{itemize}
    \item \textit{Rule-technique comparison}: The description of each technique in the set of probable techniques is compared with the rule description. 
    A chain-of-thought technique is then applied to elucidate the reasoning behind the association of each technique with the rule.
    \item \textit{Confidence calculation}: The generated chain-of-thought rationale for each technique (or sub-technique) is compared with the rule description to compute a relevance (or confidence) score, as done in prior work~\cite{freitas2024ai}.
    % \item \textbf{Reasoning}: \new{Add here the reasoning that it provides - explaining in NLP why it was selected...}
\end{itemize}

Techniques with higher confidence scores are deemed more relevant to the rule. 
Conversely, techniques with scores falling below a predefined threshold are excluded.
The techniques retained after this filtering step represent the most relevant techniques corresponding to the given rule's description. 


The chain-of-thought (CoT) rationale generated during the comparison of each rule to its probable technique is also provided as an output in this step.
This rationale offers a detailed natural language explanation, articulating why a particular technique is relevant to the given rule. 
Such explanations are highly valuable for security analysts, as they provide clear and transparent reasoning behind the mapping, enabling analysts to better understand and validate the association between the rule and the technique.
Other classification models proposed in previous works within this domain also suffer from the limitation of being black-box models, which lack the ability to provide clear reasoning or explanations. 
Unlike \methodName, these models fail to generate transparent, CoT rationales that explain why a particular rule is mapped to a specific technique, making them less interpretable and less useful for security analysts.
\begin{table}[ht!]
\centering
\caption{\textbf{Super Resolution Performance Results.} Our proposed WGAN EEG Spatial Upsampling method significantly outperforms a baseline of Bicubic Interpolation commonly used in EEG upsampling pipelines.}
\label{tab:results}
\resizebox{0.8\linewidth}{!}{%
\begin{tabular}{@{}cccccc@{}}
\toprule
\multirow{2}{*}{\textbf{Dataset}} & \multirow{2}{*}{\textbf{Scale}} & \multicolumn{2}{c}{\textbf{Bicubic}} & \multicolumn{2}{c}{\textbf{WGAN}} \\ \cmidrule(l){3-6} 
                      &   & \textbf{MSE} & \textbf{MAE} & \textbf{MSE}    & \textbf{MAE}   \\
\toprule
\multirow{2}{*}{Val}  & 2 & 3.71E7       & 3.89E3       & \textbf{2.01E3} & \textbf{24.38} \\
                      & 4 & 7.23E7       & 6.42E3       & \textbf{8.53E3} & \textbf{63.83} \\
\midrule
\multirow{2}{*}{Test} & 2 & 3.75E7       & 3.91E3       & \textbf{2.06E3} & \textbf{24.66} \\
                      & 4 & 7.30E7       & 6.45E3       & \textbf{8.68E3} & \textbf{64.39} \\
\bottomrule
\end{tabular}%
}
\end{table}
\section{Related Work}
% \subsection{Vision Language Model}
% 시각장애인에서 상황을 설명할 DB가 없으니 만들었다. 그리고 이를 VLM에 튜닝했다.
\subsection{Technical approaches for assisting the visually-impaired}


\subsection{Datasets for visual instruction tuning}

\section*{Conclusion}
This paper aims to enhance our understanding of the computational complexity of computing various Shapley value variants. We found that for various ML models --- including decision trees, regression tree ensembles, weighted automata, and linear regression --- both local and global interventional and baseline SHAP can be computed in polynomial time under HMM modeled distributions. This extends popular algorithms, such as TreeSHAP, beyond their empirical distributional scope. We also establish strict complexity gaps between the various SHAP variants (baseline, interventional, and conditional) and prove the intractability of computing SHAP for tree ensembles and neural networks in simplified scenarios. Overall, we present SHAP as a versatile framework whose complexity depends on four key factors: \begin{inparaenum}[(i)] \item model type, \item SHAP variant, \item distribution modeling approach, \item and local vs. global explanations\end{inparaenum}. We believe this perspective provides deeper insight into the computational complexity of SHAP, paving the way for future work.




%We believe that our framework provides a more intricate understanding of SHAP computation complexity across different models, distributions, and variants, paving the way for further research.

Our work opens promising directions for future research. First, expanding our computational analysis to other SHAP-related metrics, such as asymmetric SHAP~\citep{frye20} and SAGE~\citep{covert2020understanding}, would be valuable. Additionally, we aim to explore more expressive distribution classes and relaxed assumptions beyond those in Section \ref{sec:tractable} while maintaining tractable SHAP computation. Finally, when exact computation is intractable (Section \ref{sec:intractable}), investigating the approximability of SHAP metrics through approximation and parameterized complexity theory~\citep{downey2012parameterized} is an important direction.

%Our work opens several promising avenues for future research on the computational properties of explainable AI methods, with a particular focus on SHAP. First, it would be interesting to broaden the computational analysis conducted in this work to include other popular SHAP-related metrics in the literature, such as asymmetric SHAP \cite{frye20} and SAGE \cite{covert2020understanding}. Also, in the future, we aim to explore more expressive distribution classes and relaxed distributional assumptions—extending beyond those examined in Section \ref{sec:tractable} —that still yield tractable SHAP computation. Finally, when exact computation proves intractable (Section \ref{sec:intractable}), it is worthwhile to theoretically investigate the question of the approximability of computing the SHAP metrics across various configurations, through the lens of approximation and parametrized complexity theory \cite{arora2009computational}.

%This paper aims to deepen our understanding of the computational complexity involved in obtaining different Shapley value variants. We found that for a variety of ML models, including decision trees, tree ensembles for regression, weighted automata, and linear regression models — computing both local and global interventional and baseline SHAP can be done in polynomial time when distributions are modeled by HMMs. This extends the distributional scope of popular algorithms like TreeSHAP, which is limited to empirical distributions. Additionally, we demonstrate a strict complexity gap between SHAP variants, showing that interventional and baseline SHAP can be strictly easier to compute than conditional SHAP. Despite these positive results, we uncovered intractability for various SHAP variants in neural networks and tree ensembles. Finally, we provided generalized complexity relations across SHAP variants. We believe that our framework offers a deeper understanding of the complexity involved in computing SHAP across various variants, models, distributions, as well as in both local and global computations, laying the groundwork for future research.

\section*{Impact statement}




This paper presents work on language model distillation, which is actively used in the training of many modern language models. We identify a possible shortcoming of existing distillation procedures, called teacher hacking, that can lead to the transfer of unsafe behaviors from teacher to student. Additionally, we proposed several strategies to reduce the effect of this phenomenon. We believe that understanding and identifying such issues have positive societal consequences and allow the development of more reliable and safe language models.



\bibliography{reference}
\bibliographystyle{icml2025}


%%%%%%%%%%%%%%%%%%%%%%%%%%%%%%%%%%%%%%%%%%%%%%%%%%%%%%%%%%%%%%%%%%%%%%%%%%%%%%%
%%%%%%%%%%%%%%%%%%%%%%%%%%%%%%%%%%%%%%%%%%%%%%%%%%%%%%%%%%%%%%%%%%%%%%%%%%%%%%%
% APPENDIX
%%%%%%%%%%%%%%%%%%%%%%%%%%%%%%%%%%%%%%%%%%%%%%%%%%%%%%%%%%%%%%%%%%%%%%%%%%%%%%%
%%%%%%%%%%%%%%%%%%%%%%%%%%%%%%%%%%%%%%%%%%%%%%%%%%%%%%%%%%%%%%%%%%%%%%%%%%%%%%%
\newpage
\appendix
\onecolumn

\documentclass[preprint,12pt]{elsarticle}

%% Use the option review to obtain double line spacing
%% \documentclass[authoryear,preprint,review,12pt]{elsarticle}
\usepackage[english]{babel}
\usepackage[strings]{underscore}
\usepackage{microtype}
\usepackage{graphicx}
\usepackage{subcaption}
 \graphicspath{{./img/}}
 \DeclareGraphicsExtensions{.pdf}
\usepackage{booktabs}
\usepackage{multirow}
\usepackage{amsmath}
\usepackage[export]{adjustbox}
\usepackage{algcompatible}
\usepackage{lscape}
\usepackage[noend]{algpseudocode}
\usepackage{algorithm}
\usepackage[table]{xcolor}
\usepackage{xcolor}

\usepackage{xr}
\makeatletter
\newcommand*{\addFileDependency}[1]{
  \typeout{(#1)}
  \@addtofilelist{#1}
  \IfFileExists{#1}{}{\typeout{No file #1.}}
}
\makeatother

\newcommand*{\myexternaldocumenta}[1]{
    \externaldocument{#1}
    \addFileDependency{#1.tex}
    \addFileDependency{#1.aux}
}
%%% END HELPER CODE

% put all the external documents here!
\myexternaldocumenta{./JPDC2018}

% just to see what's happening
\listfiles


\begin{document}

\appendix
%%%%%%%%%%%%%%%%O
\section{Extended Simulation Results}
\label{results_appendix}

Figure \ref{fig:RLFT_HS10} and \ref{fig:RLFT_HS25} show the simulation results when 10\% and 25\% of the end-nodes generate congested traffic addressed to a single end-node (i.e., traffic pattern HS10-1 and HS25-1). 

As we can see in these figures, the 1-VC configuration cannot deal with congestion just applying restricted adaptive routing, due to the HoL blocking (see Figures \ref{fig_RLFT_HS10_1q},  \ref{fig_RLFT_HS10_1q-voq}, \ref{fig_RLFT_HS_1q} and \ref{fig_RLFT_HS_1q-voq}).


Note that when no VOQs are used, the adaptive routing using \emph{triggering thresholds} is able to raise a bit the performance, since this restriction delays the adaptivity decisions. The results show that better performance is achieved in a more congested scenario. 
This strange effect is due to the congestion tree. In fact, a higher incast congestion scenario creates a larger congestion tree that reaches the level 1 switches in the routing upward phase. Note that the performance raise in Figure \ref{fig_RLFT_HS10_1q} and \ref{fig_RLFT_HS_1q} when they reach 70\% and 50\% of the generated traffic. 
As a consequence, the performance of adaptive algorithms, restricted only to the second stage,  drops to 0.
By contrast, when VOQs are used, 1VC performance is close to 0\% as congestion trees grow into the VOQs, regardless the used routing algorithm.

DBBM (see Figures \ref{fig_RLFT_HS10_dbbm3}, \ref{fig_RLFT_HS10_dbbm3-voq}, \ref{fig_RLFT_HS_dbbm3} and \ref{fig_RLFT_HS_dbbm3-voq}) outperforms 1VC regardless the routing configurations.
When no VOQs are used,the results are homogeneous until the percentage of injected traffic exceeds 60\% and they show that K/$\Delta$ and 2S restrictions do not work as good as the other routing configurations.
The 2S restriction does not work well due to routing configuration uses $D$-mod-$K$ in the first stage of the Fat Tree, so that DBBM maps all the destinations to the same VC (see \figurename~\ref{fig_RLFT_3_2_destro}). Something similar happens with K/$\Delta$ restriction. This restriction uses a module to decide which upward port is chosen in the routing function \ref{algorithmAdaptive}. Since the module for choosing the port and mapping in the queue matches, then all adapted packets are mapped in the same queue. That's why the worst results are obtained when packets are adapted only in the second stage without any restrictions.
In other words, DBBM improves the deterministic routing when it adapts the routes in the first stage and manages to balance the use of queues.
However, when VOQs are used, low order HoL-blocking disappears and the results converge regardless the used routing algorithm.

When we use vFtree (see Figures \ref{fig_RLFT_HS10_vftree3}, \ref{fig_RLFT_HS10_vftree3-voq}, \ref{fig_RLFT_HS_vftree3-voq} and \ref{fig_RLFT_HS_vftree3}) the network efficiency drops significantly when adaptive routing is applied in the second stage (see the top-right switch \figurename~\ref{fig_RLFT_3_2_adapt} and downward stages in red), since all the destinations are mapped to all the VCs so that the HoL blocking probability increases.
In this case, the K/$\Delta$ restrictions avoid this defect and achieve productivity even when packets are adapted just in the second stage.
Note that *S-K configurations are able to raise the performance when the injection traffic is over 70\% and 60\%.  
This effect is more noticeable in the VOQ architecture because of its higher productivity.

When we use Flow2SL without VOQs (see Figures \ref{fig_RLFT_HS10_flow2sl3}, \ref{fig_RLFT_HS_flow2sl3}), we can see that better results are obtained when adaptive routing is applied in the first stage (FS and S*)  with triggering restrictions.
On the other hand, when switches implement VOQs (see Figures \ref{fig_RLFT_HS10_flow2sl3-voq}, \ref{fig_RLFT_HS_flow2sl3-voq}), the obtained results are similar.
The reason is that VOQs spread the congestion throughout all the VCs in the same buffer, since flow-control is performed at VC level, but not at VOQ level.

Figures \ref{fig:RLFT_HS104} and \ref{fig:RLFT_HS254} show experiment results for the same scenarios described before when we generate traffic creating four congestion trees  (i.e., traffic HS10-4 and HS25-4 depicted in \figurename~\ref{fig_traffic_HS4}).


This is a very strong congestion situation, since we generate four congestion trees whose branches affect to different VCs at the same time and thus reduces the effectiveness of the static queuing scheme.
Therefore, this situation makes complex, even for restricted routing combined with queuing schemes, to reduce the HoL blocking.
When no VOQs are used, the SQS show similar behaviour as described above but they yield less performance.
By contrast, when VOQs are used, the only queuing scheme that works is vFtree (see Figures \ref{fig_RLFT_HS_vftree3} and \ref{fig_RLFT_HS_vftree3}) due to its mapping properties. It works well with deterministic routing but a little more productivity can be achieved when we adapt packets using K/$\Delta$ restriction even when there are four strong congestion trees.

Figure \ref{fig:RLFT_IHS} shows simulation results for traffic generating hot-spots in intermediate stages of the topology (see \figurename~\ref{fig_traffic_HSW}).
In particular, hot-spots (i.e., congestion tree roots) are generated in the output ports of some switches placed at the second stage of the RLFT.
As we can see in the figures, this traffic pattern generates a congestion situation where $D$-mod-$K$, oblivious and fully-adaptive routing algorithms (both with and without restrictions) can deal with the HoL blocking appearing in the network.
For switches without VOQs, vFtree queuing scheme achieves the best results combined with adaptive routing using combined restrictions, such as ADAP-NoTH-SS-K, ADAP-TH-AS-K or ADAP-2TH-AS-$K/3$ outperform 1VC, DBBM and Flow2SL. 
This behavior is the same for vFtree when using switches with VOQs.
Note that switches with VOQs using the configuration ADAP-NoTH-SS-K achieve excellent results regardless the use of queuing schemes (see Figures  \ref{fig_RLFT_IHS_1q-voq}, \ref{fig_RLFT_IHS_dbbm3-voq},  \ref{fig_RLFT_IHS_vftree3-voq}, \ref{fig_RLFT_IHS_flow2sl3-voq}).
Although the use of restricted routing and queuing schemes are not necessary in this traffic scenario when using VOQs, note that their use preserves the performance gains shown in the previous traffic scenarios when using restricted adaptive routing and queuing schemes.
Therefore, the use of restricted adaptive routing combined with queuing schemes significantly increases the network performance under congested scenarios, compared to when we use deterministic and oblivious routing.

Figures \ref{fig:RLFT_histo_1q}, \ref{fig:RLFT_histo_dbbm}, \ref{fig:RLFT_histo_vftree} and \ref{fig:RLFT_histo_flow2sl} show in histograms the same results as the Tables \ref{tab:NOVOQ} and \ref{tab:VOQ}. This allows the comparison of different switch architectures for a given queuing scheme


% JESÜS: a partir de aquí está el texto anterior


%We have renamed the non-restricted adaptive routing as ``fully-adaptive'' routing, since it is the way it is called in the literature.
%More precisely, Figure \ref{fig:RLFT_HS10_vTime} shows the situation when 10\% of the end-nodes generate congesting packets addressed to a single destination (i.e., traffic pattern HS10-1), and \ref{fig:RLFT_HS25_vTime} shows the simulation results when 25\% of the end-nodes generate congested traffic addressed to a single end-node (i.e., traffic pattern HS25-1).


%As we can see in these figures, the 1-VC configuration (with and without VOQs) cannot deal with congestion just applying restricted and non-restricting adaptive routing, due to the HoL blocking (see Figures \ref{fig_RLFT_HS10_1q},  \ref{fig_RLFT_HS10_1q-voq}, \ref{fig_RLFT_HS_1q} and \ref{fig_RLFT_HS_1q-voq}).
%Note that when no VOQs are used, the adaptive routing using \emph{triggering thresholds} (i.e., ADAP-TH-*S-K and ADAP-2TH-*S-K) is able to raise a bit the performance (up to 30\% approximately), since this restriction delays the adaptivity decisions.
%By contrast, when VOQs are used, 1VC performance is close to 0\% as congestion trees grow into the VOQs, regardless the used routing algorithm.
%Figures \ref{fig_time_RLFT_HS10_1q} and  \ref{fig_time_RLFT_HS10_1q-voq} show similar results when simulations are run during $3$ms and we generate 100\% of traffic load.
%The hot-spot is generated by 10\% of the nodes.
%Similar results are shown in Figures \ref{fig_time_RLFT_HS_1q} and \ref{fig_time_RLFT_HS_1q-voq}, when 25\% of end-nodes generate congested traffic.

%When queuing schemes are used without VOQs, the network efficiency results raise from 30\% achieved by 1VC using the ADAP-2TH-AS-K configuration up to 55\% for DBBM, 64\% for vFtree and 73\%.
%DBBM (see Figures \ref{fig_RLFT_HS10_dbbm3}, \ref{fig_time_RLFT_HS10_dbbm3}, \ref{fig_RLFT_HS_dbbm3} and \ref{fig_time_RLFT_HS_dbbm3}) outperforms 1VC regardless the routing configurations.
%The best routing configuration for DBBM is 2TH-*S-H in these scenarios of moderate congestion.
%This routing configuration uses $D$-mod-$K$ in the first stage of the Fat Tree, so that DBBM maps all the destinations to the same VC (see \figurename~\ref{fig_RLFT_3_2_destro}).
%In the second stage, adaptive routing is used, but the path diversity is reduced to the destinations reaching the second stage of the Fat Tree (as $D$-mod-$K$ balances traffic flows in the first stage).
%Then, DBBM shares more destinations in the same VC and the queuing scheme effectiveness is reduced.
%For this reason, DBBM combined with the adaptive routing configurations using the restriction in the first stage (FS) obtains good performance results, regardless the traffic pattern (i.e., HS10-1 and HS25-1).
%Note also that $D$-mod-$K$ routing achieves 40\% of network efficiency (less than 55\% achieved by routing configurations with restrictions).
%Moreover, oblivious and fully-adaptive routing take advantage of the DBBM mapping policy, since they achieve almost 50\% of the network efficiency.

%By contrast, when we use vFtree (see Figures \ref{fig_RLFT_HS10_vftree3}, \ref{fig_time_RLFT_HS10_vftree3}, \ref{fig_RLFT_HS_vftree3} and \ref{fig_time_RLFT_HS_vftree3}) the network efficiency drops significantly when adaptive routing is applied in the second stage (SS) even through vFtree is using 3VCs (see the top-right switch \figurename~\ref{fig_RLFT_3_2_adapt} and downward stages in red), since all the destinations are mapped to all the VCs so that the HoL blocking probability increases.
%In the case of HS10-1 traffic, when the triggering restrictions are applied (TH or 2TH), the network efficiency raises after generation rate reaches 70\%, while, in the case of HS25-1 traffic it raisers when the generated traffic load is close to 50\%.
%Note that the congestion tree generated by the traffic HS25-1 is more intense than that generated by HS10-1.
%In the case of HS10-1, the congestion appears first in switches near the hot-spot destination end-node, and then it propagates backwards so that it reaches slower the switches in the first stage of the Fat Tree (upward-path), compared to the traffic pattern HS25-1.
%In the case of HS25-1, the congestion appears first in the first stages as more flows gather, then it propagates downwards.
%As we can see, vFtree requires more restrictions to the adaptive routing than DBBM, in order to perform properly.
%Note that vFtree with routing restrictions reaches 60\% of network efficiency, while $D$-mod-$K$, which can be considered as the most restricted routing, achieves 51\% of network effficiency.
%Indeed, the routing restrictions to vFtree also involve the number of used port counts ($K$).
%When this value is $K/3$ then vFtree performs better, although it requires to restrict the adaptivity per stages (FS or SS) and using the triggering thresholds (TH or 2TH).
%Note that the oblivious and fully-adaptive routing configurations results drop near 0\% due to the specific mapping of vFtree, which needs to reduce adaptivity mainly in the second stage, as we have described before.

%When we use Flow2SL (see Figures \ref{fig_RLFT_HS10_flow2sl3}, \ref{fig_time_RLFT_HS10_flow2sl3},  \ref{fig_RLFT_HS_flow2sl3} and \ref{fig_time_RLFT_HS_flow2sl3}), we can see that better results are obtained when adaptive routing is applied in the first stage (FS) and in all the stages (AS), compared to deterministic ($D$-mod-$K$) or oblivious routing configurations.
%Note that Flow2SL achieves the maximum throughput (around 70\% in Figures \ref{fig_RLFT_HS10_flow2sl3} and \ref{fig_RLFT_HS_flow2sl3}) with certain configurations of restrictions in the routing, such as ADAPT-2TH-AS-$K/3$.

%On the other hand, when switches implement VOQs, the obtained results are better, in general, than those obtained for switches without VOQs for DBBM, vFtree and Flow2SL.
%The reason is that VOQs spread the congestion throughout all the VCs in the same buffer, since flow-control is performed at VC level, but not at VOQ level.
%Hence, applying restrictions to the adaptive routing is unnecessary, since an efficient queuing scheme achieves good performance (as it happens for DBBM and Flow2SL). By contrast, vFtree needs to apply some of these restrictions when using VOQ-based switches, since the certain adaptivity degrees (i.e., mostly in the second stage) cause that the  vFtree mapping assigns many destinations to many VCs, thus increasing HoL blocking probability.
%As a consequence, another interesting observation is the homogeneous throughput obtained by VOQ-based switches.
%VOQ-based buffer organization switches show homogeneous results when generation rate achieves 60\% for different routing configurations while non-based show an unstable throughput, as in \figurename~\ref{fig_RLFT_HS10_1q-voq}, \ref{fig_RLFT_HS10_dbbm3-voq} ,\ref{fig_RLFT_HS10_vftree3-voq}, and \ref{fig_RLFT_HS10_flow2sl3-voq}.
%
%It is worth mentioning the additional throughput achieved in VOQ-based buffer organization configurations when the generation rates are lower than 40\% when adaptive routing is applied in the second stage (SS), as in \figurename~\ref{fig_RLFT_HS10_1q-voq}, \ref{fig_RLFT_HS10_vftree3-voq}, and \ref{fig_RLFT_HS10_flow2sl3-voq}. 
%Actually, the best results are usually obtained using path restrictions (K/3).

%Figures \ref{fig:RLFT_HS104}, \ref{fig:RLFT_HS104_vTime} \ref{fig:RLFT_HS254} and \ref{fig:RLFT_HS254_vTime} show experiment results for the same scenarios described before when we generate traffic creating four congestion trees  (i.e., traffic HS10-4 and HS25-4 depicted in \figurename~\ref{fig_traffic_HS4}).
%Note that, the four hot-spots in these scenarios are generated, respectively, by 10\% and 25\% of the source end-nodes generating traffic addressed to end-nodes $600$, $3400$, $5200$ and $9500$.
%This is a very strong congestion situation, since we generate four congestion trees whose branches  affect to different VCs at the same time.
%It is even possible that four different branches belonging to four different congestion trees grow within the same VC, producing HoL blocking in all the VCs. 
%Therefore, this situation makes complex, even for restricted routing combined with queuing schemes, to reduce the HoL blocking.


%As in the previous scenario, 1-VC with and without VOQs obtains the worst results, since the HoL blocking dramatically degrades the network performance.
%On the other hand, as the mapping of Flow2SL and DBBM is unfortunate, they achieve worse results than vFtree, since its mapping suits better the traffic situation.
%Note that $D$-mod-$K$ and oblivious routing achieve a performance near to 0\% for DBBM and Flow2SL queuing schemes, regardless the generated traffic load and the use of VOQs in the switches.
%Another reason for this bad performance is that this traffic pattern (see \figurename~\ref{fig_traffic_HS4}) produces very strong congestion situations in the first stage of the upward path and the adaptive and oblivious routing algorithms spread the congestion through the VCs and VOQs in the first stage of the RLFT.

%In the case of DBBM, it maps all the hot-spot destinations to all the VCs, since destination end-node $600$ is mapped to VC0, destinations $3400$ and $5200$ are mapped to VC1, and destination $9500$ is mapped to VC2.
%Note that the best results obtained for DBBM are those achieved by adaptive routing configurations using the triggering restrictions (i.e., TH and 2TH) and not restricting adaptivity to a particular stage (i.e., AS).
%By contrast, Flow2SL maps hot-spot destinations to all the groups in the topology, since destination $600$ is mapped to group $0$ (so to VC0), destinations $3400$ and $5200$ are mapped to group $1$ (so to VC1) and destination $9500$ is mapped to group $2$ (so to VC2).
%Hence, it is highly possible that all VCs suffer from HoL blocking, then degrading the Flow2SL efficiency.
%By contrast, vFtree maps destinations $600$ and $5200$ to VC0, while destinations $3400$ and $9500$ are mapped to VC2.
%Hence, the flows mapped to VC1 do not interact with congested flows in VC0 and VC1.
%We can see this effect in the results achieved by  $D$-mod-$K$.
%As for DBBM and Flow2SL, vFtree also takes advantage of the configurations using combined restrictions, such as ADAPT-2TH-AS-$K/3$.
%When VOQs are used, we can observe the unfortunate mapping effects for DBBM and Flow2SL, described in Section \ref{sec:problem}.

%Figures \ref{fig:RLFT_IHS} and \ref{fig:RLFT_IHS_vTime} show simulation results for traffic generating hot-spots in intermediate stages of the topology (see \figurename~\ref{fig_traffic_HSW}).
%In particular, hot-spots (i.e., congestion tree roots) are generated in the output ports of some switches placed at the second stage of the RLFT.
%As we can see in the figures, this traffic pattern generates a congestion situation where $D$-mod-$K$, oblivious and fully-adaptive routing algorithms (both with and without restrictions) can deal with the HoL blocking appearing in the network.
%For switches without VOQs, vFtree queuing scheme achieves the best results combined with adaptive routing using combined restrictions, such as ADAP-NoTH-SS-K, ADAP-TH-AS-K or ADAP-2TH-AS-$K/3$ outperform 1VC, DBBM and Flow2SL. 
%This behavior is the same for vFtree when using switches with VOQs.
%Note that switches with VOQs using the configuration ADAP-NoTH-SS-K achieve excellent results regardless the use of queuing schemes (see Figures  \ref{fig_RLFT_IHS_1q-voq}, \ref{fig_RLFT_IHS_dbbm3-voq},  \ref{fig_RLFT_IHS_vftree3-voq}, \ref{fig_RLFT_IHS_flow2sl3-voq}).
%Although the use of restricted routing and queuing schemes are not necessary in this traffic scenario when using VOQs, note that their use preserves the performance gains shown in the previous traffic scenarios when using restricted adaptive routing and queuing schemes.
%Therefore, the use of restricted adaptive routing combined with queuing schemes significantly increases the network performance under congested scenarios, compared to when we use deterministic and oblivious routing.

%Figures \ref{fig:RLFT_HS10} and \ref{fig:RLFT_HS25} show simulation results for the queuing schemes described before combined with deterministic ($D$-mod-$K$), oblivious and adaptive routing (restricted and non-restricted).
%We have re-named the non-restricted adaptive routing as ``fully-adaptive'' routing, since it is the way it is called in the literature.
%Also, we have added zooms to some of the figures in order to better reflect the differences among the data series.

\begin{figure*}[!htb]

\begin{subfigure}[!th]{\textwidth}
 \centering 
\includegraphics[width=1.0\textwidth]{leyenda3.pdf}
\end{subfigure}

 \begin{subfigure}[!th]{0.47\textwidth}
 \centering 
\includegraphics[width=0.82\textwidth]{./1q/Graphics/1q_synthetic_hotspot10_throughput_load.pdf}
\caption{1VC.}
\label{fig_RLFT_HS10_1q}
\end{subfigure}
 \begin{subfigure}[!th]{0.47\textwidth}
 \centering 
\includegraphics[width=0.82\textwidth]
{./1q-voq/Graphics/1q-voq_synthetic_hotspot10_throughput_load.pdf}
\caption{1VC and VOQs.}
\label{fig_RLFT_HS10_1q-voq}
\end{subfigure}

 \begin{subfigure}[!th]{0.47\textwidth}
 \centering 
\includegraphics[width=0.82\textwidth]
{./dbbm3/Graphics/dbbm3_synthetic_hotspot10_throughput_load.pdf}
\caption{DBBM.}
\label{fig_RLFT_HS10_dbbm3}
\end{subfigure}
 \begin{subfigure}[!th]{0.47\textwidth}
 \centering 
\includegraphics[width=0.82\textwidth]
{./dbbm3-voq/Graphics/dbbm3-voq_synthetic_hotspot10_throughput_load.pdf}
\caption{DBBM and VOQs.}
\label{fig_RLFT_HS10_dbbm3-voq}
\end{subfigure}

 \begin{subfigure}[!th]{0.47\textwidth}
 \centering 
\includegraphics[width=0.82\textwidth]
{./vftree3/Graphics/vftree3_synthetic_hotspot10_throughput_load.pdf}
\caption{vFtree.}
\label{fig_RLFT_HS10_vftree3}
\end{subfigure}
 \begin{subfigure}[!th]{0.47\textwidth}
 \centering 
\includegraphics[width=0.82\textwidth]
{./vftree3-voq/Graphics/vftree3-voq_synthetic_hotspot10_throughput_load.pdf}
\caption{vFtree and VOQs.}
\label{fig_RLFT_HS10_vftree3-voq}
\end{subfigure}

 \begin{subfigure}[!th]{0.47\textwidth}
 \centering 
\includegraphics[width=0.82\textwidth]
{./flow2sl3/Graphics/flow2sl3_synthetic_hotspot10_throughput_load.pdf}
\caption{Flow2SL.}
\label{fig_RLFT_HS10_flow2sl3}
\end{subfigure}
 \begin{subfigure}[!th]{0.47\textwidth}
 \centering 
\includegraphics[width=0.82\textwidth]
{./flow2sl3-voq/Graphics/flow2sl3-voq_synthetic_hotspot10_throughput_load.pdf}
\caption{Flow2SL and VOQs.}
\label{fig_RLFT_HS10_flow2sl3-voq}
\end{subfigure}

\caption{Normalized Throughput versus Generated Traffic Load in a $11664$-node RLFT under HS10-1 synthetic traffic pattern.}
\label{fig:RLFT_HS10}
\end{figure*}


%%%%%%%%%%%%%%%%
\begin{figure*}[!htb]
\vspace{-.5cm}
\begin{subfigure}[!th]{1\textwidth}
 \centering 
\includegraphics[width=1.0\textwidth]
{leyenda3.pdf}
\end{subfigure}

 \begin{subfigure}[!th]{0.47\textwidth}
 \centering 
\includegraphics[width=0.82\textwidth]
{./1q/Graphics/1q_synthetic_hotspot_throughput_load.pdf}
\caption{1VC.}
\label{fig_RLFT_HS_1q}
\end{subfigure}
 \begin{subfigure}[!th]{0.47\textwidth}
 \centering 
\includegraphics[width=0.82\textwidth]
{./1q-voq/Graphics/1q-voq_synthetic_hotspot_throughput_load.pdf}
\caption{1VC and VOQs.}
\label{fig_RLFT_HS_1q-voq}
\end{subfigure}

 \begin{subfigure}[!th]{0.47\textwidth}
 \centering 
\includegraphics[width=0.82\textwidth]
{./dbbm3/Graphics/dbbm3_synthetic_hotspot_throughput_load.pdf}
\caption{DBBM.}
\label{fig_RLFT_HS_dbbm3}
\end{subfigure}
 \begin{subfigure}[!th]{0.47\textwidth}
 \centering 
\includegraphics[width=0.82\textwidth]
{./dbbm3-voq/Graphics/dbbm3-voq_synthetic_hotspot_throughput_load.pdf}
\caption{DBBM and VOQs.}
\label{fig_RLFT_HS_dbbm3-voq}
\end{subfigure}

 \begin{subfigure}[!th]{0.47\textwidth}
 \centering 
\includegraphics[width=0.82\textwidth]
{./vftree3/Graphics/vftree3_synthetic_hotspot_throughput_load.pdf}
\caption{vFtree.}
\label{fig_RLFT_HS_vftree3}
\end{subfigure}
 \begin{subfigure}[!th]{0.47\textwidth}
 \centering 
\includegraphics[width=0.82\textwidth]
{./vftree3-voq/Graphics/vftree3-voq_synthetic_hotspot_throughput_load.pdf}
\caption{vFtree and VOQs.}
\label{fig_RLFT_HS_vftree3-voq}
\end{subfigure}

 \begin{subfigure}[!th]{0.47\textwidth}
 \centering 
\includegraphics[width=0.82\textwidth]
{./flow2sl3/Graphics/flow2sl3_synthetic_hotspot_throughput_load.pdf}
\caption{Flow2SL.}
\label{fig_RLFT_HS_flow2sl3}
\end{subfigure}
 \begin{subfigure}[!th]{0.47\textwidth}
 \centering 
\includegraphics[width=0.82\textwidth]
{./flow2sl3-voq/Graphics/flow2sl3-voq_synthetic_hotspot_throughput_load.pdf}
\caption{Flow2SL and VOQs.}
\label{fig_RLFT_HS_flow2sl3-voq}
\end{subfigure}

\caption{Normalized Throughput versus Generated Traffic Load in a $11664$-node RLFT under HS25-1 synthetic traffic pattern.}
\label{fig:RLFT_HS25}
\end{figure*}


%%%%%%%%%%%%%%%%
\begin{figure*}[!htb]
\vspace{-.5cm}
\begin{subfigure}[!th]{1\textwidth}
 \centering 
\includegraphics[width=1.0\textwidth]
{leyenda3.pdf}
\end{subfigure}

 \begin{subfigure}[!th]{0.47\textwidth}
 \centering 
\includegraphics[width=0.82\textwidth]
{./1q/Graphics/1q_synthetic_hotspot104_throughput_load.pdf}
\caption{1VC.}
\label{fig_RLFT_HS104_1q}
\end{subfigure}
 \begin{subfigure}[!th]{0.47\textwidth}
 \centering 
\includegraphics[width=0.82\textwidth]
{./1q-voq/Graphics/1q-voq_synthetic_hotspot104_throughput_load.pdf}
\caption{1VC and VOQs.}
\label{fig_RLFT_HS104_1q-voq}
\end{subfigure}

 \begin{subfigure}[!th]{0.47\textwidth}
 \centering 
\includegraphics[width=0.82\textwidth]
{./dbbm3/Graphics/dbbm3_synthetic_hotspot104_throughput_load.pdf}
\caption{DBBM.}
\label{fig_RLFT_HS104_dbbm3}
\end{subfigure}
 \begin{subfigure}[!th]{0.47\textwidth}
 \centering 
\includegraphics[width=0.82\textwidth]
{./dbbm3-voq/Graphics/dbbm3-voq_synthetic_hotspot104_throughput_load.pdf}
\caption{DBBM and VOQs.}
\label{fig_RLFT_HS104_dbbm3-voq}
\end{subfigure}

 \begin{subfigure}[!th]{0.47\textwidth}
 \centering 
\includegraphics[width=0.82\textwidth]
{./vftree3/Graphics/vftree3_synthetic_hotspot104_throughput_load.pdf}
\caption{vFtree.}
\label{fig_RLFT_HS104_vftree3}
\end{subfigure}
 \begin{subfigure}[!th]{0.47\textwidth}
 \centering 
\includegraphics[width=0.82\textwidth]
{./vftree3-voq/Graphics/vftree3-voq_synthetic_hotspot104_throughput_load.pdf}
\caption{vFtree and VOQs.}
\label{fig_RLFT_HS104_vftree3-voq}
\end{subfigure}

 \begin{subfigure}[!th]{0.47\textwidth}
 \centering 
\includegraphics[width=0.82\textwidth]
{./flow2sl3/Graphics/flow2sl3_synthetic_hotspot104_throughput_load.pdf}
\caption{Flow2SL.}
\label{fig_RLFT_HS104_flow2sl3}
\end{subfigure}
 \begin{subfigure}[!th]{0.47\textwidth}
 \centering 
\includegraphics[width=0.82\textwidth]
{./flow2sl3-voq/Graphics/flow2sl3-voq_synthetic_hotspot104_throughput_load.pdf}
\caption{Flow2SL and VOQs.}
\label{fig_RLFT_HS104_flow2sl3-voq}
\end{subfigure}

\caption{Normalized Throughput versus Generated Traffic Load in a $11664$-node RLFT under HS10-4 synthetic traffic pattern.}
\label{fig:RLFT_HS104}
\end{figure*}

%%%%%%%%%%%%%%%%
\begin{figure*}[!htb]
\vspace{-.5cm}
\begin{subfigure}[!th]{1\textwidth}
 \centering 
\includegraphics[width=1.0\textwidth]
{leyenda3.pdf}
\end{subfigure}

 \begin{subfigure}[!th]{0.47\textwidth}
 \centering 
\includegraphics[width=0.82\textwidth]
{./1q/Graphics/1q_synthetic_hotspot254_throughput_load.pdf}
\caption{1VC.}
\label{fig_RLFT_HS254_1q}
\end{subfigure}
 \begin{subfigure}[!th]{0.47\textwidth}
 \centering 
\includegraphics[width=0.82\textwidth]
{./1q-voq/Graphics/1q-voq_synthetic_hotspot254_throughput_load.pdf}
\caption{1VC and VOQs.}
\label{fig_RLFT_HS254_1q-voq}
\end{subfigure}

 \begin{subfigure}[!th]{0.47\textwidth}
 \centering 
\includegraphics[width=0.82\textwidth]
{./dbbm3/Graphics/dbbm3_synthetic_hotspot254_throughput_load.pdf}
\caption{DBBM.}
\label{fig_RLFT_HS254_dbbm3}
\end{subfigure}
 \begin{subfigure}[!th]{0.47\textwidth}
 \centering 
\includegraphics[width=0.82\textwidth]
{./dbbm3-voq/Graphics/dbbm3-voq_synthetic_hotspot254_throughput_load.pdf}
\caption{DBBM and VOQs.}
\label{fig_RLFT_HS254_dbbm3-voq}
\end{subfigure}

 \begin{subfigure}[!th]{0.47\textwidth}
 \centering 
\includegraphics[width=0.82\textwidth]
{./vftree3/Graphics/vftree3_synthetic_hotspot254_throughput_load.pdf}
\caption{vFtree.}
\label{fig_RLFT_HS254_vftree3}
\end{subfigure}
 \begin{subfigure}[!th]{0.47\textwidth}
 \centering 
\includegraphics[width=0.82\textwidth]
{./vftree3-voq/Graphics/vftree3-voq_synthetic_hotspot254_throughput_load.pdf}
\caption{vFtree and VOQs.}
\label{fig_RLFT_HS254_vftree3-voq}
\end{subfigure}

 \begin{subfigure}[!th]{0.47\textwidth}
 \centering 
\includegraphics[width=0.82\textwidth]
{./flow2sl3/Graphics/flow2sl3_synthetic_hotspot254_throughput_load.pdf}
\caption{Flow2SL.}
\label{fig_RLFT_HS254_flow2sl3}
\end{subfigure}
 \begin{subfigure}[!th]{0.47\textwidth}
 \centering 
\includegraphics[width=0.82\textwidth]
{./flow2sl3-voq/Graphics/flow2sl3-voq_synthetic_hotspot254_throughput_load.pdf}
\caption{Flow2SL and VOQs.}
\label{fig_RLFT_HS254_flow2sl3-voq}
\end{subfigure}

\caption{Normalized Throughput versus Generated Traffic Load in a $11664$-node RLFT under HS25-4 synthetic traffic pattern.}
\label{fig:RLFT_HS254}
\end{figure*}



%%%%%%%%%%%%%%%%
\begin{figure*}[!htb]
\vspace{-.5cm}
\begin{subfigure}[!th]{1\textwidth}
 \centering 
\includegraphics[width=1.0\textwidth]
{leyenda3.pdf}
\end{subfigure}

 \begin{subfigure}[!th]{0.47\textwidth}
 \centering 
\includegraphics[width=0.82\textwidth]
{./1q/Graphics/1q_synthetic_interhotspot_throughput_load.pdf}
\caption{1VC.}
\label{fig_RLFT_IHS_1q}
\end{subfigure}
 \begin{subfigure}[!th]{0.47\textwidth}
 \centering 
\includegraphics[width=0.82\textwidth]
{./1q-voq/Graphics/1q-voq_synthetic_interhotspot_throughput_load.pdf}
\caption{1VC and VOQs.}
\label{fig_RLFT_IHS_1q-voq}
\end{subfigure}

 \begin{subfigure}[!th]{0.47\textwidth}
 \centering 
\includegraphics[width=0.82\textwidth]
{./dbbm3/Graphics/dbbm3_synthetic_interhotspot_throughput_load.pdf}
\caption{DBBM.}
\label{fig_RLFT_IHS_dbbm3}
\end{subfigure}
 \begin{subfigure}[!th]{0.47\textwidth}
 \centering 
\includegraphics[width=0.82\textwidth]
{./dbbm3-voq/Graphics/dbbm3-voq_synthetic_interhotspot_throughput_load.pdf}
\caption{DBBM and VOQs.}
\label{fig_RLFT_IHS_dbbm3-voq}
\end{subfigure}

 \begin{subfigure}[!th]{0.47\textwidth}
 \centering 
\includegraphics[width=0.82\textwidth]
{./vftree3/Graphics/vftree3_synthetic_interhotspot_throughput_load.pdf}
\caption{vFtree.}
\label{fig_RLFT_IHS_vftree3}
\end{subfigure}
 \begin{subfigure}[!th]{0.47\textwidth}
 \centering 
\includegraphics[width=0.82\textwidth]
{./vftree3-voq/Graphics/vftree3-voq_synthetic_interhotspot_throughput_load.pdf}
\caption{vFtree and VOQs.}
\label{fig_RLFT_IHS_vftree3-voq}
\end{subfigure}

 \begin{subfigure}[!th]{0.47\textwidth}
 \centering 
\includegraphics[width=0.82\textwidth]
{./flow2sl3/Graphics/flow2sl3_synthetic_interhotspot_throughput_load.pdf}
\caption{Flow2SL.}
\label{fig_RLFT_IHS_flow2sl3}
\end{subfigure}
 \begin{subfigure}[!th]{0.47\textwidth}
 \centering 
\includegraphics[width=0.82\textwidth]
{./flow2sl3-voq/Graphics/flow2sl3-voq_synthetic_interhotspot_throughput_load.pdf}
\caption{Flow2SL and VOQs.}
\label{fig_RLFT_IHS_flow2sl3-voq}
\end{subfigure}

\caption{Normalized Throughput versus Generated Traffic Load in a $11664$-node RLFT under IHS synthetic traffic pattern.}
\label{fig:RLFT_IHS}
\end{figure*}



\begin{figure*}[!htb]
\vspace{-.5cm}
\begin{subfigure}[!th]{1\textwidth}
 \centering 
\includegraphics[width=1.0\textwidth]
{leyenda3.pdf}
\end{subfigure}

 \begin{subfigure}[!th]{1\textwidth}
 \centering 
\includegraphics[width=0.8\textwidth]
{./img/1q.pdf}
\caption{1VC.}
\label{fig_histo_1q}
\end{subfigure}
 \begin{subfigure}[!th]{1\textwidth}
 \centering 
\includegraphics[width=0.8\textwidth]
{./img/1q-voq.pdf}
\caption{1VC and VOQs.}
\label{fig_histo_1q-voq}
\end{subfigure}
\caption{Normalized Throughput after warmup period in a $11664$-node RLFT using 1Q and 1Q-VOQ.}
\label{fig:RLFT_histo_1q}
\end{figure*}

\begin{figure*}[!htb]
\vspace{-.5cm}
\begin{subfigure}[!th]{1\textwidth}
 \centering 
\includegraphics[width=1.0\textwidth]
{leyenda3.pdf}
\end{subfigure}

 \begin{subfigure}[!th]{1\textwidth}
 \centering 
\includegraphics[width=0.8\textwidth]
{./img/dbbm3.pdf}
\caption{DBBM.}
\label{fig_histo_dbbm}
\end{subfigure}
 \begin{subfigure}[!th]{1\textwidth}
 \centering 
\includegraphics[width=0.8\textwidth]
{./img/dbbm3-voq.pdf}
\caption{DBBM and VOQs.}
\label{fig_histo_dbbm-voq}
\end{subfigure}
\caption{Normalized Throughput after warmup period in a $11664$-node RLFT using DBBM and DBBM-VOQ.}
\label{fig:RLFT_histo_dbbm}
\end{figure*}

\begin{figure*}[!ht]
\vspace{-.5cm}
\begin{subfigure}[!th]{1\textwidth}
 \centering 
\includegraphics[width=1.0\textwidth]
{leyenda3.pdf}
\end{subfigure}

 \begin{subfigure}[!th]{1\textwidth}
 \centering 
\includegraphics[width=0.8\textwidth]
{./img/vftree3.pdf}
\caption{vFtree.}
\label{fig_histo_vftree}
\end{subfigure}
 \begin{subfigure}[!th]{1\textwidth}
 \centering 
\includegraphics[width=0.8\textwidth]
{./img/vftree3-voq.pdf}
\caption{vFtree and VOQs.}
\label{fig_histo_vftree-voq}
\end{subfigure}
\caption{Normalized Throughput after warmup period in a $11664$-node RLFT using vFtree and vFtree-VOQ.}
\label{fig:RLFT_histo_vftree}
\end{figure*}


\begin{figure*}[!ht]
\vspace{-.5cm}
\begin{subfigure}[!th]{1\textwidth}
 \centering 
\includegraphics[width=1.0\textwidth]
{leyenda3.pdf}
\end{subfigure}

 \begin{subfigure}[!th]{1\textwidth}
 \centering 
\includegraphics[width=0.8\textwidth]
{./img/flow2sl3.pdf}
\caption{Flow2SL.}
\label{fig_histo_flow2sl}
\end{subfigure}
 \begin{subfigure}[!th]{1\textwidth}
 \centering 
\includegraphics[width=0.8\textwidth]
{./img/flow2sl3-voq.pdf}
\caption{Flow2SL and VOQs.}
\label{fig_histo_flow2sl-voq}
\end{subfigure}
\caption{Normalized Throughput after warmup period in a $11664$-node RLFT using Flow2SL and Flow2SL-VOQ.}
\label{fig:RLFT_histo_flow2sl}
\end{figure*}

\end{document}

\endinput
\section{Additional details}\label{app:misc}

In this section, we provide an exact formulation for sequence-level divergences
\begin{equation}\label{eq:kl_divergence_lms}
    \KL_{\seq}(\pi,  \pi') \triangleq \E_{x \sim d(\cdot), y \sim p_{\pi}(\cdot | x)}\left[ \sum_{i=1}^{|y|}\KL(\pi(\cdot | x, y_{:i})), \pi'(\cdot | x, y_{:i})) \right]\,.
\end{equation}
In particular, to estimate this KL-divergence, we need to sample prompts $x$ and generate responses using a language model $\pi$. Additionally, we introduce a sequence-based Jensen-Shannon divergence, defined as
\begin{equation}\label{eq:js_divergence_lms}
    \begin{split}
    \JS_{\seq}(\pi, \pi') &\triangleq \E_{x \sim d(\cdot), y \sim p_{\pi}(\cdot | x), y' \sim p_{\pi'}(\cdot | x)}\bigg[ \frac{1}{2}\sum_{i=1}^{|y|}\KL(\pi(\cdot | x, y_{:i}),  m(\cdot | x, y_{:i}))  + \frac{1}{2}\sum_{i=1}^{|y'|}\KL(\pi'(\cdot | x, y'_{:i}), m(\cdot | x, y'_{:i})) \bigg]\,,
    \end{split}
\end{equation}
where $m(\omega | x, y) \triangleq 0.5 \cdot \pi(\omega| x,y) + 0.5 \cdot \pi'(\omega|x,y)$ is a mixture of two language models. To estimate this divergence, we need to use samples from both models $\pi$ and $\pi'$ and compute an average of two token-level KL-divergences. Notice that computation of a true Jensen-Shannon divergence between $p_{\pi}$ and $p_{\pi'}$ is computationally infeasible since the log-probabilities of the mixture $0.5 \cdot p_{\pi}(y|x) + 0.5 \cdot p_{\pi'}(y|x)$ does not satisfy a chain rule in terms of $\pi$ and $\pi'$.
\begin{table}[h!]
    \caption{Hyperparameters}
    \label{tab:TrainingParams}
    \centering
    \begin{tabular}{l|l}
        \textbf{Parameters} & \textbf{Tuning} \\% & \textbf{Estimation}  \\
        \hline
        Sampling time                   & $0.05$  \\   %& $0.05$     \\
        Reward discount factor $\gamma$ & $0.99$  \\   %& $0.99$     \\
        Learning rate for actor         & $10^{-3}$ \\% & $10^{-3}$  \\
        Learning rate for critic        & $10^{-3}$ \\% & $10^{-3}$  \\
        $L_2$ Regularization factor     & $10^{-5}$ \\% & $10^{-4}$  \\
        Optimizer parameter $\epsilon$  & $10^{-8}$ \\% & $10^{-8}$  \\
        Minimum batch size              & $1024$  \\%   & $64$     \\
        Experience buffer length        & $10^{6}$  \\% & $10^{6}$   \\
    \end{tabular}
\end{table}
%%%%%%%%%%%%%%%%%%%%%%%%%%%%%%%%%%%%%%%%%%%%%%%%%%%%%%%%%%%%%%%%%%%%%%%%%%%%%%%
%%%%%%%%%%%%%%%%%%%%%%%%%%%%%%%%%%%%%%%%%%%%%%%%%%%%%%%%%%%%%%%%%%%%%%%%%%%%%%%


\end{document}


% This document was modified from the file originally made available by
% Pat Langley and Andrea Danyluk for ICML-2K. This version was created
% by Iain Murray in 2018, and modified by Alexandre Bouchard in
% 2019 and 2021 and by Csaba Szepesvari, Gang Niu and Sivan Sabato in 2022.
% Modified again in 2023 and 2024 by Sivan Sabato and Jonathan Scarlett.
% Previous contributors include Dan Roy, Lise Getoor and Tobias
% Scheffer, which was slightly modified from the 2010 version by
% Thorsten Joachims & Johannes Fuernkranz, slightly modified from the
% 2009 version by Kiri Wagstaff and Sam Roweis's 2008 version, which is
% slightly modified from Prasad Tadepalli's 2007 version which is a
% lightly changed version of the previous year's version by Andrew
% Moore, which was in turn edited from those of Kristian Kersting and
% Codrina Lauth. Alex Smola contributed to the algorithmic style files.
