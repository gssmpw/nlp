%%%%%%%% ICML 2025 EXAMPLE LATEX SUBMISSION FILE %%%%%%%%%%%%%%%%%

\documentclass{article}

% Recommended, but optional, packages for figures and better typesetting:
\usepackage{microtype}
\usepackage{graphicx}
\usepackage{csquotes}
\usepackage{subfigure}
\usepackage{booktabs} % for professional tables

% hyperref makes hyperlinks in the resulting PDF.
% If your build breaks (sometimes temporarily if a hyperlink spans a page)
% please comment out the following usepackage line and replace
% \usepackage{icml2025} with \usepackage[nohyperref]{icml2025} above.
%\usepackage{hyperref}
\usepackage{notation}
\usepackage{tcolorbox}

\definecolor{colorbluefull}{rgb}{0.25882352941176473, 0.5215686274509804, 0.9568627450980393}
\colorlet{colorblue}{colorbluefull!30}



% Attempt to make hyperref and algorithmic work together better:
\newcommand{\theHalgorithm}{\arabic{algorithm}}

% Use the following line for the initial blind version submitted for review:
\usepackage[preprint]{icml2025}

% If accepted, instead use the following line for the camera-ready submission:
% \usepackage[accepted]{icml2025}

% For theorems and such
\usepackage{amsmath}
\usepackage{amssymb}
\usepackage{mathtools}
\usepackage{amsthm}

% if you use cleveref..
\usepackage[capitalize,noabbrev]{cleveref}

%%%%%%%%%%%%%%%%%%%%%%%%%%%%%%%%
% THEOREMS
%%%%%%%%%%%%%%%%%%%%%%%%%%%%%%%%
\theoremstyle{plain}
\newtheorem{theorem}{Theorem}
\newtheorem{proposition}{Proposition}
\newtheorem{lemma}{Lemma}
\newtheorem{corollary}{Corollary}
\theoremstyle{definition}
\newtheorem{definition}{Definition}
\newtheorem{assumption}{Assumption}
\theoremstyle{remark}
\newtheorem{remark}{Remark}

\usepackage[inline]{enumitem}


% Todonotes is useful during development; simply uncomment the next line
%    and comment out the line below the next line to turn off comments
%\usepackage[disable,textsize=tiny]{todonotes}
\usepackage[textsize=tiny]{todonotes}
\newcommand{\todoDT}[1]{\todo[inline,color=blue!30]{\textbf{Daniil}: #1}}

% The \icmltitle you define below is probably too long as a header.
% Therefore, a short form for the running title is supplied here:
\icmltitlerunning{On Teacher Hacking in Language Model Distillation}

\begin{document}

\twocolumn[
\icmltitle{On Teacher Hacking in Language Model Distillation}

% It is OKAY to include author information, even for blind
% submissions: the style file will automatically remove it for you
% unless you've provided the [accepted] option to the icml2025
% package.

% List of affiliations: The first argument should be a (short)
% identifier you will use later to specify author affiliations
% Academic affiliations should list Department, University, City, Region, Country
% Industry affiliations should list Company, City, Region, Country

% You can specify symbols, otherwise they are numbered in order.
% Ideally, you should not use this facility. Affiliations will be numbered
% in order of appearance and this is the preferred way.
%\icmlsetsymbol{equal}{*}

\begin{icmlauthorlist}
\icmlauthor{Daniil Tiapkin}{x}
\icmlauthor{Daniele Calandriello}{gdm}
\icmlauthor{Johan Ferret}{gdm}
\icmlauthor{Sarah Perrin}{gdm}
\icmlauthor{Nino Vieillard}{gdm}
\icmlauthor{Alexandre Ram{\'e}}{gdm}
\icmlauthor{Mathieu Blondel}{gdm}
\end{icmlauthorlist}

\icmlaffiliation{gdm}{Google DeepMind}
\icmlaffiliation{x}{CMAP, {\'E}cole Polytechnique, Palaiseau, France; Work done during an internship at Google DeepMind.}
%\icmlaffiliation{sch}{School of ZZZ, Institute of WWW, Location, Country}

\icmlcorrespondingauthor{Daniil Tiapkin}{daniil.tiapkin@polytechnique.edu}
%\icmlcorrespondingauthor{Firstname2 Lastname2}{first2.last2@www.uk}

% You may provide any keywords that you
% find helpful for describing your paper; these are used to populate
% the "keywords" metadata in the PDF but will not be shown in the document
\icmlkeywords{Machine Learning, ICML}

\vskip 0.3in
]

% this must go after the closing bracket ] following \twocolumn[ ...

% This command actually creates the footnote in the first column
% listing the affiliations and the copyright notice.
% The command takes one argument, which is text to display at the start of the footnote.
% The \icmlEqualContribution command is standard text for equal contribution.
% Remove it (just {}) if you do not need this facility.

%\printAffiliationsAndNotice{}  % leave blank if no need to mention equal contribution
\printAffiliationsAndNotice{\icmlEqualContribution} % otherwise use the standard text.

\begin{abstract}
Post-training of language models (LMs) increasingly relies on the following two stages: (i) knowledge distillation, where the LM is trained to imitate a larger teacher LM, and (ii) reinforcement learning from human feedback (RLHF), where the LM is aligned by optimizing a reward model.
% In the second RLHF stage, a well-known challenge is reward hacking, where the LM over-optimizes the reward model, leading to degraded performance on the true objective, in line with Goodhart's law.
In the second RLHF stage, a well-known challenge is reward hacking, where the LM over-optimizes the reward model. Such phenomenon is in line with Goodhart's law and can lead to degraded performance on the true objective.
In this paper, we investigate whether a similar phenomenon, that we call teacher hacking, can occur during knowledge distillation. This could arise because the teacher LM is itself an imperfect approximation of the true distribution. To study this, we propose a controlled experimental setup involving: (i) an oracle LM representing the ground-truth distribution, (ii) a teacher LM distilled from the oracle, and (iii) a student LM distilled from the teacher.
Our experiments reveal the following insights. When using a fixed offline dataset for distillation, teacher hacking occurs; moreover, we can detect it by observing when the optimization process deviates from polynomial convergence laws. In contrast, employing online data generation techniques effectively mitigates teacher hacking. More precisely, we identify data diversity as the key factor in preventing hacking.
Overall, our findings provide a deeper understanding of the benefits and limitations of distillation for building robust and efficient LMs.
\end{abstract}
\newpage
\section{Introduction}


\begin{figure}[t]
\centering
\includegraphics[width=0.6\columnwidth]{figures/evaluation_desiderata_V5.pdf}
\vspace{-0.5cm}
\caption{\systemName is a platform for conducting realistic evaluations of code LLMs, collecting human preferences of coding models with real users, real tasks, and in realistic environments, aimed at addressing the limitations of existing evaluations.
}
\label{fig:motivation}
\end{figure}

\begin{figure*}[t]
\centering
\includegraphics[width=\textwidth]{figures/system_design_v2.png}
\caption{We introduce \systemName, a VSCode extension to collect human preferences of code directly in a developer's IDE. \systemName enables developers to use code completions from various models. The system comprises a) the interface in the user's IDE which presents paired completions to users (left), b) a sampling strategy that picks model pairs to reduce latency (right, top), and c) a prompting scheme that allows diverse LLMs to perform code completions with high fidelity.
Users can select between the top completion (green box) using \texttt{tab} or the bottom completion (blue box) using \texttt{shift+tab}.}
\label{fig:overview}
\end{figure*}

As model capabilities improve, large language models (LLMs) are increasingly integrated into user environments and workflows.
For example, software developers code with AI in integrated developer environments (IDEs)~\citep{peng2023impact}, doctors rely on notes generated through ambient listening~\citep{oberst2024science}, and lawyers consider case evidence identified by electronic discovery systems~\citep{yang2024beyond}.
Increasing deployment of models in productivity tools demands evaluation that more closely reflects real-world circumstances~\citep{hutchinson2022evaluation, saxon2024benchmarks, kapoor2024ai}.
While newer benchmarks and live platforms incorporate human feedback to capture real-world usage, they almost exclusively focus on evaluating LLMs in chat conversations~\citep{zheng2023judging,dubois2023alpacafarm,chiang2024chatbot, kirk2024the}.
Model evaluation must move beyond chat-based interactions and into specialized user environments.



 

In this work, we focus on evaluating LLM-based coding assistants. 
Despite the popularity of these tools---millions of developers use Github Copilot~\citep{Copilot}---existing
evaluations of the coding capabilities of new models exhibit multiple limitations (Figure~\ref{fig:motivation}, bottom).
Traditional ML benchmarks evaluate LLM capabilities by measuring how well a model can complete static, interview-style coding tasks~\citep{chen2021evaluating,austin2021program,jain2024livecodebench, white2024livebench} and lack \emph{real users}. 
User studies recruit real users to evaluate the effectiveness of LLMs as coding assistants, but are often limited to simple programming tasks as opposed to \emph{real tasks}~\citep{vaithilingam2022expectation,ross2023programmer, mozannar2024realhumaneval}.
Recent efforts to collect human feedback such as Chatbot Arena~\citep{chiang2024chatbot} are still removed from a \emph{realistic environment}, resulting in users and data that deviate from typical software development processes.
We introduce \systemName to address these limitations (Figure~\ref{fig:motivation}, top), and we describe our three main contributions below.


\textbf{We deploy \systemName in-the-wild to collect human preferences on code.} 
\systemName is a Visual Studio Code extension, collecting preferences directly in a developer's IDE within their actual workflow (Figure~\ref{fig:overview}).
\systemName provides developers with code completions, akin to the type of support provided by Github Copilot~\citep{Copilot}. 
Over the past 3 months, \systemName has served over~\completions suggestions from 10 state-of-the-art LLMs, 
gathering \sampleCount~votes from \userCount~users.
To collect user preferences,
\systemName presents a novel interface that shows users paired code completions from two different LLMs, which are determined based on a sampling strategy that aims to 
mitigate latency while preserving coverage across model comparisons.
Additionally, we devise a prompting scheme that allows a diverse set of models to perform code completions with high fidelity.
See Section~\ref{sec:system} and Section~\ref{sec:deployment} for details about system design and deployment respectively.



\textbf{We construct a leaderboard of user preferences and find notable differences from existing static benchmarks and human preference leaderboards.}
In general, we observe that smaller models seem to overperform in static benchmarks compared to our leaderboard, while performance among larger models is mixed (Section~\ref{sec:leaderboard_calculation}).
We attribute these differences to the fact that \systemName is exposed to users and tasks that differ drastically from code evaluations in the past. 
Our data spans 103 programming languages and 24 natural languages as well as a variety of real-world applications and code structures, while static benchmarks tend to focus on a specific programming and natural language and task (e.g. coding competition problems).
Additionally, while all of \systemName interactions contain code contexts and the majority involve infilling tasks, a much smaller fraction of Chatbot Arena's coding tasks contain code context, with infilling tasks appearing even more rarely. 
We analyze our data in depth in Section~\ref{subsec:comparison}.



\textbf{We derive new insights into user preferences of code by analyzing \systemName's diverse and distinct data distribution.}
We compare user preferences across different stratifications of input data (e.g., common versus rare languages) and observe which affect observed preferences most (Section~\ref{sec:analysis}).
For example, while user preferences stay relatively consistent across various programming languages, they differ drastically between different task categories (e.g. frontend/backend versus algorithm design).
We also observe variations in user preference due to different features related to code structure 
(e.g., context length and completion patterns).
We open-source \systemName and release a curated subset of code contexts.
Altogether, our results highlight the necessity of model evaluation in realistic and domain-specific settings.





% !TEX root =  ../main.tex
\section{Background on causality and abstraction}\label{sec:preliminaries}

This section provides the notation and key concepts related to causal modeling and abstraction theory.

\spara{Notation.} The set of integers from $1$ to $n$ is $[n]$.
The vectors of zeros and ones of size $n$ are $\zeros_n$ and $\ones_n$.
The identity matrix of size $n \times n$ is $\identity_n$. The Frobenius norm is $\frob{\mathbf{A}}$.
The set of positive definite matrices over $\reall^{n\times n}$ is $\pd^n$. The Hadamard product is $\odot$.
Function composition is $\circ$.
The domain of a function is $\dom{\cdot}$ and its kernel $\ker$.
Let $\mathcal{M}(\mathcal{X}^n)$ be the set of Borel measures over $\mathcal{X}^n \subseteq \reall^n$. Given a measure $\mu^n \in \mathcal{M}(\mathcal{X}^n)$ and a measurable map $\varphi^{\V}$, $\mathcal{X}^n \ni \mathbf{x} \overset{\varphi^{\V}}{\longmapsto} \V^\top \mathbf{x} \in \mathcal{X}^m$, we denote by $\varphi^{\V}_{\#}(\mu^n) \coloneqq \mu^n(\varphi^{\V^{-1}}(\mathbf{x}))$ the pushforward measure $\mu^m \in \mathcal{M}(\mathcal{X}^m)$. 


We now present the standard definition of SCM.

\begin{definition}[SCM, \citealp{pearl2009causality}]\label{def:SCM}
A (Markovian) structural causal model (SCM) $\scm^n$ is a tuple $\langle \myendogenous, \myexogenous, \myfunctional, \zeta^\myexogenous \rangle$, where \emph{(i)} $\myendogenous = \{X_1, \ldots, X_n\}$ is a set of $n$ endogenous random variables; \emph{(ii)} $\myexogenous =\{Z_1,\ldots,Z_n\}$ is a set of $n$ exogenous variables; \emph{(iii)} $\myfunctional$ is a set of $n$ functional assignments such that $X_i=f_i(\parents_i, Z_i)$, $\forall \; i \in [n]$, with $ \parents_i \subseteq \myendogenous \setminus \{ X_i\}$; \emph{(iv)} $\zeta^\myexogenous$ is a product probability measure over independent exogenous variables $\zeta^\myexogenous=\prod_{i \in [n]} \zeta^i$, where $\zeta^i=P(Z_i)$. 
\end{definition}
A Markovian SCM induces a directed acyclic graph (DAG) $\mathcal{G}_{\scm^n}$ where the nodes represent the variables $\myendogenous$ and the edges are determined by the structural functions $\myfunctional$; $ \parents_i$ constitutes then the parent set for $X_i$. Furthermore, we can recursively rewrite the set of structural function $\myfunctional$ as a set of mixing functions $\mymixing$ dependent only on the exogenous variables (cf. \cref{app:CA}). A key feature for studying causality is the possibility of defining interventions on the model:
\begin{definition}[Hard intervention, \citealp{pearl2009causality}]\label{def:intervention}
Given SCM $\scm^n = \langle \myendogenous, \myexogenous, \myfunctional, \zeta^\myexogenous \rangle$, a (hard) intervention $\iota = \operatorname{do}(\myendogenous^{\iota} = \mathbf{x}^{\iota})$, $\myendogenous^{\iota}\subseteq \myendogenous$,
is an operator that generates a new post-intervention SCM $\scm^n_\iota = \langle \myendogenous, \myexogenous, \myfunctional_\iota, \zeta^\myexogenous \rangle$ by replacing each function $f_i$ for $X_i\in\myendogenous^{\iota}$ with the constant $x_i^\iota\in \mathbf{x}^\iota$. 
Graphically, an intervention mutilates $\mathcal{G}_{\mathsf{M}^n}$ by removing all the incoming edges of the variables in $\myendogenous^{\iota}$.
\end{definition}

Given multiple SCMs describing the same system at different levels of granularity, CA provides the definition of an $\alpha$-abstraction map to relate these SCMs:
\begin{definition}[$\abst$-abstraction, \citealp{rischel2020category}]\label{def:abstraction}
Given low-level $\mathsf{M}^\ell$ and high-level $\mathsf{M}^h$ SCMs, an $\abst$-abstraction is a triple $\abst = \langle \Rset, \amap, \alphamap{} \rangle$, where \emph{(i)} $\Rset \subseteq \datalow$ is a subset of relevant variables in $\mathsf{M}^\ell$; \emph{(ii)} $\amap: \Rset \rightarrow \datahigh$ is a surjective function between the relevant variables of $\mathsf{M}^\ell$ and the endogenous variables of $\mathsf{M}^h$; \emph{(iii)} $\alphamap{}: \dom{\Rset} \rightarrow \dom{\datahigh}$ is a modular function $\alphamap{} = \bigotimes_{i\in[n]} \alphamap{X^h_i}$ made up by surjective functions $\alphamap{X^h_i}: \dom{\amap^{-1}(X^h_i)} \rightarrow \dom{X^h_i}$ from the outcome of low-level variables $\amap^{-1}(X^h_i) \in \datalow$ onto outcomes of the high-level variables $X^h_i \in \datahigh$.
\end{definition}
Notice that an $\abst$-abstraction simultaneously maps variables via the function $\amap$ and values through the function $\alphamap{}$. The definition itself does not place any constraint on these functions, although a common requirement in the literature is for the abstraction to satisfy \emph{interventional consistency} \cite{rubenstein2017causal,rischel2020category,beckers2019abstracting}. An important class of such well-behaved abstractions is \emph{constructive linear abstraction}, for which the following properties hold. By constructivity, \emph{(i)} $\abst$ is interventionally consistent; \emph{(ii)} all low-level variables are relevant $\Rset=\datalow$; \emph{(iii)} in addition to the map $\alphamap{}$ between endogenous variables, there exists a map ${\alphamap{}}_U$ between exogenous variables satisfying interventional consistency \cite{beckers2019abstracting,schooltink2024aligning}. By linearity, $\alphamap{} = \V^\top \in \reall^{h \times \ell}$ \cite{massidda2024learningcausalabstractionslinear}. \cref{app:CA} provides formal definitions for interventional consistency, linear and constructive abstraction.
\section{Research Methodology}~\label{sec:Methodology}

In this section, we discuss the process of conducting our systematic review, e.g., our search strategy for data extraction of relevant studies, based on the guidelines of Kitchenham et al.~\cite{kitchenham2022segress} to conduct SLRs and Petersen et al.~\cite{PETERSEN20151} to conduct systematic mapping studies (SMSs) in Software Engineering. In this systematic review, we divide our work into a four-stage procedure, including planning, conducting, building a taxonomy, and reporting the review, illustrated in Fig.~\ref{fig:search}. The four stages are as follows: (1) the \emph{planning} stage involved identifying research questions (RQs) and specifying the detailed research plan for the study; (2) the \emph{conducting} stage involved analyzing and synthesizing the existing primary studies to answer the research questions; (3) the \emph{taxonomy} stage was introduced to optimize the data extraction results and consolidate a taxonomy schema for REDAST methodology; (4) the \emph{reporting} stage involved the reviewing, concluding and reporting the final result of our study.

\begin{figure}[!t]
    \centering
    \includegraphics[width=1\linewidth]{fig/methodology/searching-process.drawio.pdf}
    \caption{Systematic Literature Review Process}
    \label{fig:search}
\end{figure}

\subsection{Research Questions}
In this study, we developed five research questions (RQs) to identify the input and output, analyze technologies, evaluate metrics, identify challenges, and identify potential opportunities. 

\textbf{RQ1. What are the input configurations, formats, and notations used in the requirements in requirements-driven
automated software testing?} In requirements-driven testing, the input is some form of requirements specification -- which can vary significantly. RQ1 maps the input for REDAST and reports on the comparison among different formats for requirements specification.

\textbf{RQ2. What are the frameworks, tools, processing methods, and transformation techniques used in requirements-driven automated software testing studies?} RQ2 explores the technical solutions from requirements to generated artifacts, e.g., rule-based transformation applying natural language processing (NLP) pipelines and deep learning (DL) techniques, where we additionally discuss the potential intermediate representation and additional input for the transformation process.

\textbf{RQ3. What are the test formats and coverage criteria used in the requirements-driven automated software
testing process?} RQ3 focuses on identifying the formulation of generated artifacts (i.e., the final output). We map the adopted test formats and analyze their characteristics in the REDAST process.

\textbf{RQ4. How do existing studies evaluate the generated test artifacts in the requirements-driven automated software testing process?} RQ4 identifies the evaluation datasets, metrics, and case study methodologies in the selected papers. This aims to understand how researchers assess the effectiveness, accuracy, and practical applicability of the generated test artifacts.

\textbf{RQ5. What are the limitations and challenges of existing requirements-driven automated software testing methods in the current era?} RQ5 addresses the limitations and challenges of existing studies while exploring future directions in the current era of technology development. %It particularly highlights the potential benefits of advanced LLMs and examines their capacity to meet the high expectations placed on these cutting-edge language modeling technologies. %\textcolor{blue}{CA: Do we really need to focus on LLMs? TBD.} \textcolor{orange}{FW: About LLMs, I removed the direct emphase in RQ5 but kept the discussion in RQ5 and the solution section. I think that would be more appropriate.}

\subsection{Searching Strategy}

The overview of the search process is exhibited in Fig. \ref{fig:papers}, which includes all the details of our search steps.
\begin{table}[!ht]
\caption{List of Search Terms}
\label{table:search_term}
\begin{tabularx}{\textwidth}{lX}
\hline
\textbf{Terms Group} & \textbf{Terms} \\ \hline
Test Group & test* \\
Requirement Group & requirement* OR use case* OR user stor* OR specification* \\
Software Group & software* OR system* \\
Method Group & generat* OR deriv* OR map* OR creat* OR extract* OR design* OR priorit* OR construct* OR transform* \\ \hline
\end{tabularx}
\end{table}

\begin{figure}
    \centering
    \includegraphics[width=1\linewidth]{fig/methodology/search-papers.drawio.pdf}
    \caption{Study Search Process}
    \label{fig:papers}
\end{figure}

\subsubsection{Search String Formulation}
Our research questions (RQs) guided the identification of the main search terms. We designed our search string with generic keywords to avoid missing out on any related papers, where four groups of search terms are included, namely ``test group'', ``requirement group'', ``software group'', and ``method group''. In order to capture all the expressions of the search terms, we use wildcards to match the appendix of the word, e.g., ``test*'' can capture ``testing'', ``tests'' and so on. The search terms are listed in Table~\ref{table:search_term}, decided after iterative discussion and refinement among all the authors. As a result, we finally formed the search string as follows:


\hangindent=1.5em
 \textbf{ON ABSTRACT} ((``test*'') \textbf{AND} (``requirement*'' \textbf{OR} ``use case*'' \textbf{OR} ``user stor*'' \textbf{OR} ``specifications'') \textbf{AND} (``software*'' \textbf{OR} ``system*'') \textbf{AND} (``generat*'' \textbf{OR} ``deriv*'' \textbf{OR} ``map*'' \textbf{OR} ``creat*'' \textbf{OR} ``extract*'' \textbf{OR} ``design*'' \textbf{OR} ``priorit*'' \textbf{OR} ``construct*'' \textbf{OR} ``transform*''))

The search process was conducted in September 2024, and therefore, the search results reflect studies available up to that date. We conducted the search process on six online databases: IEEE Xplore, ACM Digital Library, Wiley, Scopus, Web of Science, and Science Direct. However, some databases were incompatible with our default search string in the following situations: (1) unsupported for searching within abstract, such as Scopus, and (2) limited search terms, such as ScienceDirect. Here, for (1) situation, we searched within the title, keyword, and abstract, and for (2) situation, we separately executed the search and removed the duplicate papers in the merging process. 

\subsubsection{Automated Searching and Duplicate Removal}
We used advanced search to execute our search string within our selected databases, following our designed selection criteria in Table \ref{table:selection}. The first search returned 27,333 papers. Specifically for the duplicate removal, we used a Python script to remove (1) overlapped search results among multiple databases and (2) conference or workshop papers, also found with the same title and authors in the other journals. After duplicate removal, we obtained 21,652 papers for further filtering.

\begin{table*}[]
\caption{Selection Criteria}
\label{table:selection}
\begin{tabularx}{\textwidth}{lX}
\hline
\textbf{Criterion ID} & \textbf{Criterion Description} \\ \hline
S01          & Papers written in English. \\
S02-1        & Papers in the subjects of "Computer Science" or "Software Engineering". \\
S02-2        & Papers published on software testing-related issues. \\
S03          & Papers published from 1991 to the present. \\ 
S04          & Papers with accessible full text. \\ \hline
\end{tabularx}
\end{table*}

\begin{table*}[]
\small
\caption{Inclusion and Exclusion Criteria}
\label{table:criteria}
\begin{tabularx}{\textwidth}{lX}
\hline
\textbf{ID}  & \textbf{Description} \\ \hline
\multicolumn{2}{l}{\textbf{Inclusion Criteria}} \\ \hline
I01 & Papers about requirements-driven automated system testing or acceptance testing generation, or studies that generate system-testing-related artifacts. \\
I02 & Peer-reviewed studies that have been used in academia with references from literature. \\ \hline
\multicolumn{2}{l}{\textbf{Exclusion Criteria}} \\ \hline
E01 & Studies that only support automated code generation, but not test-artifact generation. \\
E02 & Studies that do not use requirements-related information as an input. \\
E03 & Papers with fewer than 5 pages (1-4 pages). \\
E04 & Non-primary studies (secondary or tertiary studies). \\
E05 & Vision papers and grey literature (unpublished work), books (chapters), posters, discussions, opinions, keynotes, magazine articles, experience, and comparison papers. \\ \hline
\end{tabularx}
\end{table*}

\subsubsection{Filtering Process}

In this step, we filtered a total of 21,652 papers using the inclusion and exclusion criteria outlined in Table \ref{table:criteria}. This process was primarily carried out by the first and second authors. Our criteria are structured at different levels, facilitating a multi-step filtering process. This approach involves applying various criteria in three distinct phases. We employed a cross-verification method involving (1) the first and second authors and (2) the other authors. Initially, the filtering was conducted separately by the first and second authors. After cross-verifying their results, the results were then reviewed and discussed further by the other authors for final decision-making. We widely adopted this verification strategy within the filtering stages. During the filtering process, we managed our paper list using a BibTeX file and categorized the papers with color-coding through BibTeX management software\footnote{\url{https://bibdesk.sourceforge.io/}}, i.e., “red” for irrelevant papers, “yellow” for potentially relevant papers, and “blue” for relevant papers. This color-coding system facilitated the organization and review of papers according to their relevance.

The screening process is shown below,
\begin{itemize}
    \item \textbf{1st-round Filtering} was based on the title and abstract, using the criteria I01 and E01. At this stage, the number of papers was reduced from 21,652 to 9,071.
    \item \textbf{2nd-round Filtering}. We attempted to include requirements-related papers based on E02 on the title and abstract level, which resulted from 9,071 to 4,071 papers. We excluded all the papers that did not focus on requirements-related information as an input or only mentioned the term ``requirements'' but did not refer to the requirements specification.
    \item \textbf{3rd-round Filtering}. We selectively reviewed the content of papers identified as potentially relevant to requirements-driven automated test generation. This process resulted in 162 papers for further analysis.
\end{itemize}
Note that, especially for third-round filtering, we aimed to include as many relevant papers as possible, even borderline cases, according to our criteria. The results were then discussed iteratively among all the authors to reach a consensus.

\subsubsection{Snowballing}

Snowballing is necessary for identifying papers that may have been missed during the automated search. Following the guidelines by Wohlin~\cite{wohlin2014guidelines}, we conducted both forward and backward snowballing. As a result, we identified 24 additional papers through this process.

\subsubsection{Data Extraction}

Based on the formulated research questions (RQs), we designed 38 data extraction questions\footnote{\url{https://drive.google.com/file/d/1yjy-59Juu9L3WHaOPu-XQo-j-HHGTbx_/view?usp=sharing}} and created a Google Form to collect the required information from the relevant papers. The questions included 30 short-answer questions, six checkbox questions, and two selection questions. The data extraction was organized into five sections: (1) basic information: fundamental details such as title, author, venue, etc.; (2) open information: insights on motivation, limitations, challenges, etc.; (3) requirements: requirements format, notation, and related aspects; (4) methodology: details, including immediate representation and technique support; (5) test-related information: test format(s), coverage, and related elements. Similar to the filtering process, the first and second authors conducted the data extraction and then forwarded the results to the other authors to initiate the review meeting.

\subsubsection{Quality Assessment}

During the data extraction process, we encountered papers with insufficient information. To address this, we conducted a quality assessment in parallel to ensure the relevance of the papers to our objectives. This approach, also adopted in previous secondary studies~\cite{shamsujjoha2021developing, naveed2024model}, involved designing a set of assessment questions based on guidelines by Kitchenham et al.~\cite{kitchenham2022segress}. The quality assessment questions in our study are shown below:
\begin{itemize}
    \item \textbf{QA1}. Does this study clearly state \emph{how} requirements drive automated test generation?
    \item \textbf{QA2}. Does this study clearly state the \emph{aim} of REDAST?
    \item \textbf{QA3}. Does this study enable \emph{automation} in test generation?
    \item \textbf{QA4}. Does this study demonstrate the usability of the method from the perspective of methodology explanation, discussion, case examples, and experiments?
\end{itemize}
QA4 originates from an open perspective in the review process, where we focused on evaluation, discussion, and explanation. Our review also examined the study’s overall structure, including the methodology description, case studies, experiments, and analyses. The detailed results of the quality assessment are provided in the Appendix. Following this assessment, the final data extraction was based on 156 papers.

% \begin{table}[]
% \begin{tabular}{ll}
% \hline
% QA ID & QA Questions                                             \\ \hline
% Q01   & Does this study clearly state its aims?                  \\
% Q02   & Does this study clearly describe its methodology?        \\
% Q03   & Does this study involve automated test generation?       \\
% Q04   & Does this study include a promising evaluation?          \\
% Q05   & Does this study demonstrate the usability of the method? \\ \hline
% \end{tabular}%
% \caption{Questions for Quality Assessment}
% \label{table:qa}
% \end{table}

% automated quality assessment

% \textcolor{blue}{CA: Our search strategy focused on identifying requirements types first. We covered several sources, e.g., ~\cite{Pohl:11,wagner2019status} to identify different formats and notations of specifying requirements. However, this came out to be a long list, e.g., free-form NL requirements, semi-formal UML models, free-from textual use case models, UML class diagrams, UML activity diagrams, and so on. In this paper, we attempted to primarily focus on requirements-related aspects and not design-level information. Hence, we generalised our search string to include generic keywords, e.g., requirement*, use case*, and user stor*. We did so to avoid missing out on any papers, bringing too restrictive in our search strategy, and not creating a too-generic search string with all the aforementioned formats to avoid getting results beyond our review's scope.}


%% Use \subsection commands to start a subsection.



%\subsection{Study Selection}

% In this step, we further looked into the content of searched papers using our search strategy and applied our inclusion and exclusion criteria. Our filtering strategy aimed to pinpoint studies focused on requirements-driven system-level testing. Recognizing the presence of irrelevant papers in our search results, we established detailed selection criteria for preliminary inclusion and exclusion, as shown in Table \ref{table: criteria}. Specifically, we further developed the taxonomy schema to exclude two types of studies that did not meet the requirements for system-level testing: (1) studies supporting specification-driven test generation, such as UML-driven test generation, rather than requirements-driven testing, and (2) studies focusing on code-based test generation, such as requirement-driven code generation for unit testing.





\begin{table*}[t]
\centering
\fontsize{11pt}{11pt}\selectfont
\begin{tabular}{lllllllllllll}
\toprule
\multicolumn{1}{c}{\textbf{task}} & \multicolumn{2}{c}{\textbf{Mir}} & \multicolumn{2}{c}{\textbf{Lai}} & \multicolumn{2}{c}{\textbf{Ziegen.}} & \multicolumn{2}{c}{\textbf{Cao}} & \multicolumn{2}{c}{\textbf{Alva-Man.}} & \multicolumn{1}{c}{\textbf{avg.}} & \textbf{\begin{tabular}[c]{@{}l@{}}avg.\\ rank\end{tabular}} \\
\multicolumn{1}{c}{\textbf{metrics}} & \multicolumn{1}{c}{\textbf{cor.}} & \multicolumn{1}{c}{\textbf{p-v.}} & \multicolumn{1}{c}{\textbf{cor.}} & \multicolumn{1}{c}{\textbf{p-v.}} & \multicolumn{1}{c}{\textbf{cor.}} & \multicolumn{1}{c}{\textbf{p-v.}} & \multicolumn{1}{c}{\textbf{cor.}} & \multicolumn{1}{c}{\textbf{p-v.}} & \multicolumn{1}{c}{\textbf{cor.}} & \multicolumn{1}{c}{\textbf{p-v.}} &  &  \\ \midrule
\textbf{S-Bleu} & 0.50 & 0.0 & 0.47 & 0.0 & 0.59 & 0.0 & 0.58 & 0.0 & 0.68 & 0.0 & 0.57 & 5.8 \\
\textbf{R-Bleu} & -- & -- & 0.27 & 0.0 & 0.30 & 0.0 & -- & -- & -- & -- & - &  \\
\textbf{S-Meteor} & 0.49 & 0.0 & 0.48 & 0.0 & 0.61 & 0.0 & 0.57 & 0.0 & 0.64 & 0.0 & 0.56 & 6.1 \\
\textbf{R-Meteor} & -- & -- & 0.34 & 0.0 & 0.26 & 0.0 & -- & -- & -- & -- & - &  \\
\textbf{S-Bertscore} & \textbf{0.53} & 0.0 & {\ul 0.80} & 0.0 & \textbf{0.70} & 0.0 & {\ul 0.66} & 0.0 & {\ul0.78} & 0.0 & \textbf{0.69} & \textbf{1.7} \\
\textbf{R-Bertscore} & -- & -- & 0.51 & 0.0 & 0.38 & 0.0 & -- & -- & -- & -- & - &  \\
\textbf{S-Bleurt} & {\ul 0.52} & 0.0 & {\ul 0.80} & 0.0 & 0.60 & 0.0 & \textbf{0.70} & 0.0 & \textbf{0.80} & 0.0 & {\ul 0.68} & {\ul 2.3} \\
\textbf{R-Bleurt} & -- & -- & 0.59 & 0.0 & -0.05 & 0.13 & -- & -- & -- & -- & - &  \\
\textbf{S-Cosine} & 0.51 & 0.0 & 0.69 & 0.0 & {\ul 0.62} & 0.0 & 0.61 & 0.0 & 0.65 & 0.0 & 0.62 & 4.4 \\
\textbf{R-Cosine} & -- & -- & 0.40 & 0.0 & 0.29 & 0.0 & -- & -- & -- & -- & - & \\ \midrule
\textbf{QuestEval} & 0.23 & 0.0 & 0.25 & 0.0 & 0.49 & 0.0 & 0.47 & 0.0 & 0.62 & 0.0 & 0.41 & 9.0 \\
\textbf{LLaMa3} & 0.36 & 0.0 & \textbf{0.84} & 0.0 & {\ul{0.62}} & 0.0 & 0.61 & 0.0 &  0.76 & 0.0 & 0.64 & 3.6 \\
\textbf{our (3b)} & 0.49 & 0.0 & 0.73 & 0.0 & 0.54 & 0.0 & 0.53 & 0.0 & 0.7 & 0.0 & 0.60 & 5.8 \\
\textbf{our (8b)} & 0.48 & 0.0 & 0.73 & 0.0 & 0.52 & 0.0 & 0.53 & 0.0 & 0.7 & 0.0 & 0.59 & 6.3 \\  \bottomrule
\end{tabular}
\caption{Pearson correlation on human evaluation on system output. `R-': reference-based. `S-': source-based.}
\label{tab:sys}
\end{table*}



\begin{table}%[]
\centering
\fontsize{11pt}{11pt}\selectfont
\begin{tabular}{llllll}
\toprule
\multicolumn{1}{c}{\textbf{task}} & \multicolumn{1}{c}{\textbf{Lai}} & \multicolumn{1}{c}{\textbf{Zei.}} & \multicolumn{1}{c}{\textbf{Scia.}} & \textbf{} & \textbf{} \\ 
\multicolumn{1}{c}{\textbf{metrics}} & \multicolumn{1}{c}{\textbf{cor.}} & \multicolumn{1}{c}{\textbf{cor.}} & \multicolumn{1}{c}{\textbf{cor.}} & \textbf{avg.} & \textbf{\begin{tabular}[c]{@{}l@{}}avg.\\ rank\end{tabular}} \\ \midrule
\textbf{S-Bleu} & 0.40 & 0.40 & 0.19* & 0.33 & 7.67 \\
\textbf{S-Meteor} & 0.41 & 0.42 & 0.16* & 0.33 & 7.33 \\
\textbf{S-BertS.} & {\ul0.58} & 0.47 & 0.31 & 0.45 & 3.67 \\
\textbf{S-Bleurt} & 0.45 & {\ul 0.54} & {\ul 0.37} & 0.45 & {\ul 3.33} \\
\textbf{S-Cosine} & 0.56 & 0.52 & 0.3 & {\ul 0.46} & {\ul 3.33} \\ \midrule
\textbf{QuestE.} & 0.27 & 0.35 & 0.06* & 0.23 & 9.00 \\
\textbf{LlaMA3} & \textbf{0.6} & \textbf{0.67} & \textbf{0.51} & \textbf{0.59} & \textbf{1.0} \\
\textbf{Our (3b)} & 0.51 & 0.49 & 0.23* & 0.39 & 4.83 \\
\textbf{Our (8b)} & 0.52 & 0.49 & 0.22* & 0.43 & 4.83 \\ \bottomrule
\end{tabular}
\caption{Pearson correlation on human ratings on reference output. *not significant; we cannot reject the null hypothesis of zero correlation}
\label{tab:ref}
\end{table}


\begin{table*}%[]
\centering
\fontsize{11pt}{11pt}\selectfont
\begin{tabular}{lllllllll}
\toprule
\textbf{task} & \multicolumn{1}{c}{\textbf{ALL}} & \multicolumn{1}{c}{\textbf{sentiment}} & \multicolumn{1}{c}{\textbf{detoxify}} & \multicolumn{1}{c}{\textbf{catchy}} & \multicolumn{1}{c}{\textbf{polite}} & \multicolumn{1}{c}{\textbf{persuasive}} & \multicolumn{1}{c}{\textbf{formal}} & \textbf{\begin{tabular}[c]{@{}l@{}}avg. \\ rank\end{tabular}} \\
\textbf{metrics} & \multicolumn{1}{c}{\textbf{cor.}} & \multicolumn{1}{c}{\textbf{cor.}} & \multicolumn{1}{c}{\textbf{cor.}} & \multicolumn{1}{c}{\textbf{cor.}} & \multicolumn{1}{c}{\textbf{cor.}} & \multicolumn{1}{c}{\textbf{cor.}} & \multicolumn{1}{c}{\textbf{cor.}} &  \\ \midrule
\textbf{S-Bleu} & -0.17 & -0.82 & -0.45 & -0.12* & -0.1* & -0.05 & -0.21 & 8.42 \\
\textbf{R-Bleu} & - & -0.5 & -0.45 &  &  &  &  &  \\
\textbf{S-Meteor} & -0.07* & -0.55 & -0.4 & -0.01* & 0.1* & -0.16 & -0.04* & 7.67 \\
\textbf{R-Meteor} & - & -0.17* & -0.39 & - & - & - & - & - \\
\textbf{S-BertScore} & 0.11 & -0.38 & -0.07* & -0.17* & 0.28 & 0.12 & 0.25 & 6.0 \\
\textbf{R-BertScore} & - & -0.02* & -0.21* & - & - & - & - & - \\
\textbf{S-Bleurt} & 0.29 & 0.05* & 0.45 & 0.06* & 0.29 & 0.23 & 0.46 & 4.2 \\
\textbf{R-Bleurt} & - &  0.21 & 0.38 & - & - & - & - & - \\
\textbf{S-Cosine} & 0.01* & -0.5 & -0.13* & -0.19* & 0.05* & -0.05* & 0.15* & 7.42 \\
\textbf{R-Cosine} & - & -0.11* & -0.16* & - & - & - & - & - \\ \midrule
\textbf{QuestEval} & 0.21 & {\ul{0.29}} & 0.23 & 0.37 & 0.19* & 0.35 & 0.14* & 4.67 \\
\textbf{LlaMA3} & \textbf{0.82} & \textbf{0.80} & \textbf{0.72} & \textbf{0.84} & \textbf{0.84} & \textbf{0.90} & \textbf{0.88} & \textbf{1.00} \\
\textbf{Our (3b)} & 0.47 & -0.11* & 0.37 & 0.61 & 0.53 & 0.54 & 0.66 & 3.5 \\
\textbf{Our (8b)} & {\ul{0.57}} & 0.09* & {\ul 0.49} & {\ul 0.72} & {\ul 0.64} & {\ul 0.62} & {\ul 0.67} & {\ul 2.17} \\ \bottomrule
\end{tabular}
\caption{Pearson correlation on human ratings on our constructed test set. 'R-': reference-based. 'S-': source-based. *not significant; we cannot reject the null hypothesis of zero correlation}
\label{tab:con}
\end{table*}

\section{Results}
We benchmark the different metrics on the different datasets using correlation to human judgement. For content preservation, we show results split on data with system output, reference output and our constructed test set: we show that the data source for evaluation leads to different conclusions on the metrics. In addition, we examine whether the metrics can rank style transfer systems similar to humans. On style strength, we likewise show correlations between human judgment and zero-shot evaluation approaches. When applicable, we summarize results by reporting the average correlation. And the average ranking of the metric per dataset (by ranking which metric obtains the highest correlation to human judgement per dataset). 

\subsection{Content preservation}
\paragraph{How do data sources affect the conclusion on best metric?}
The conclusions about the metrics' performance change radically depending on whether we use system output data, reference output, or our constructed test set. Ideally, a good metric correlates highly with humans on any data source. Ideally, for meta-evaluation, a metric should correlate consistently across all data sources, but the following shows that the correlations indicate different things, and the conclusion on the best metric should be drawn carefully.

Looking at the metrics correlations with humans on the data source with system output (Table~\ref{tab:sys}), we see a relatively high correlation for many of the metrics on many tasks. The overall best metrics are S-BertScore and S-BLEURT (avg+avg rank). We see no notable difference in our method of using the 3B or 8B model as the backbone.

Examining the average correlations based on data with reference output (Table~\ref{tab:ref}), now the zero-shoot prompting with LlaMA3 70B is the best-performing approach ($0.59$ avg). Tied for second place are source-based cosine embedding ($0.46$ avg), BLEURT ($0.45$ avg) and BertScore ($0.45$ avg). Our method follows on a 5. place: here, the 8b version (($0.43$ avg)) shows a bit stronger results than 3b ($0.39$ avg). The fact that the conclusions change, whether looking at reference or system output, confirms the observations made by \citet{scialom-etal-2021-questeval} on simplicity transfer.   

Now consider the results on our test set (Table~\ref{tab:con}): Several metrics show low or no correlation; we even see a significantly negative correlation for some metrics on ALL (BLEU) and for specific subparts of our test set for BLEU, Meteor, BertScore, Cosine. On the other end, LlaMA3 70B is again performing best, showing strong results ($0.82$ in ALL). The runner-up is now our 8B method, with a gap to the 3B version ($0.57$ vs $0.47$ in ALL). Note our method still shows zero correlation for the sentiment task. After, ranks BLEURT ($0.29$), QuestEval ($0.21$), BertScore ($0.11$), Cosine ($0.01$).  

On our test set, we find that some metrics that correlate relatively well on the other datasets, now exhibit low correlation. Hence, with our test set, we can now support the logical reasoning with data evidence: Evaluation of content preservation for style transfer needs to take the style shift into account. This conclusion could not be drawn using the existing data sources: We hypothesise that for the data with system-based output, successful output happens to be very similar to the source sentence and vice versa, and reference-based output might not contain server mistakes as they are gold references. Thus, none of the existing data sources tests the limits of the metrics.  


\paragraph{How do reference-based metrics compare to source-based ones?} Reference-based metrics show a lower correlation than the source-based counterpart for all metrics on both datasets with ratings on references (Table~\ref{tab:sys}). As discussed previously, reference-based metrics for style transfer have the drawback that many different good solutions on a rewrite might exist and not only one similar to a reference.


\paragraph{How well can the metrics rank the performance of style transfer methods?}
We compare the metrics' ability to judge the best style transfer methods w.r.t. the human annotations: Several of the data sources contain samples from different style transfer systems. In order to use metrics to assess the quality of the style transfer system, metrics should correctly find the best-performing system. Hence, we evaluate whether the metrics for content preservation provide the same system ranking as human evaluators. We take the mean of the score for every output on each system and the mean of the human annotations; we compare the systems using the Kendall's Tau correlation. 

We find only the evaluation using the dataset Mir, Lai, and Ziegen to result in significant correlations, probably because of sparsity in a number of system tests (App.~\ref{app:dataset}). Our method (8b) is the only metric providing a perfect ranking of the style transfer system on the Lai data, and Llama3 70B the only one on the Ziegen data. Results in App.~\ref{app:results}. 


\subsection{Style strength results}
%Evaluating style strengths is a challenging task. 
Llama3 70B shows better overall results than our method. However, our method scores higher than Llama3 70B on 2 out of 6 datasets, but it also exhibits zero correlation on one task (Table~\ref{tab:styleresults}).%More work i s needed on evaluating style strengths. 
 
\begin{table}%[]
\fontsize{11pt}{11pt}\selectfont
\begin{tabular}{lccc}
\toprule
\multicolumn{1}{c}{\textbf{}} & \textbf{LlaMA3} & \textbf{Our (3b)} & \textbf{Our (8b)} \\ \midrule
\textbf{Mir} & 0.46 & 0.54 & \textbf{0.57} \\
\textbf{Lai} & \textbf{0.57} & 0.18 & 0.19 \\
\textbf{Ziegen.} & 0.25 & 0.27 & \textbf{0.32} \\
\textbf{Alva-M.} & \textbf{0.59} & 0.03* & 0.02* \\
\textbf{Scialom} & \textbf{0.62} & 0.45 & 0.44 \\
\textbf{\begin{tabular}[c]{@{}l@{}}Our Test\end{tabular}} & \textbf{0.63} & 0.46 & 0.48 \\ \bottomrule
\end{tabular}
\caption{Style strength: Pearson correlation to human ratings. *not significant; we cannot reject the null hypothesis of zero corelation}
\label{tab:styleresults}
\end{table}

\subsection{Ablation}
We conduct several runs of the methods using LLMs with variations in instructions/prompts (App.~\ref{app:method}). We observe that the lower the correlation on a task, the higher the variation between the different runs. For our method, we only observe low variance between the runs.
None of the variations leads to different conclusions of the meta-evaluation. Results in App.~\ref{app:results}.
\section{RELATED WORK}
\label{sec:relatedwork}
In this section, we describe the previous works related to our proposal, which are divided into two parts. In Section~\ref{sec:relatedwork_exoplanet}, we present a review of approaches based on machine learning techniques for the detection of planetary transit signals. Section~\ref{sec:relatedwork_attention} provides an account of the approaches based on attention mechanisms applied in Astronomy.\par

\subsection{Exoplanet detection}
\label{sec:relatedwork_exoplanet}
Machine learning methods have achieved great performance for the automatic selection of exoplanet transit signals. One of the earliest applications of machine learning is a model named Autovetter \citep{MCcauliff}, which is a random forest (RF) model based on characteristics derived from Kepler pipeline statistics to classify exoplanet and false positive signals. Then, other studies emerged that also used supervised learning. \cite{mislis2016sidra} also used a RF, but unlike the work by \citet{MCcauliff}, they used simulated light curves and a box least square \citep[BLS;][]{kovacs2002box}-based periodogram to search for transiting exoplanets. \citet{thompson2015machine} proposed a k-nearest neighbors model for Kepler data to determine if a given signal has similarity to known transits. Unsupervised learning techniques were also applied, such as self-organizing maps (SOM), proposed \citet{armstrong2016transit}; which implements an architecture to segment similar light curves. In the same way, \citet{armstrong2018automatic} developed a combination of supervised and unsupervised learning, including RF and SOM models. In general, these approaches require a previous phase of feature engineering for each light curve. \par

%DL is a modern data-driven technology that automatically extracts characteristics, and that has been successful in classification problems from a variety of application domains. The architecture relies on several layers of NNs of simple interconnected units and uses layers to build increasingly complex and useful features by means of linear and non-linear transformation. This family of models is capable of generating increasingly high-level representations \citep{lecun2015deep}.

The application of DL for exoplanetary signal detection has evolved rapidly in recent years and has become very popular in planetary science.  \citet{pearson2018} and \citet{zucker2018shallow} developed CNN-based algorithms that learn from synthetic data to search for exoplanets. Perhaps one of the most successful applications of the DL models in transit detection was that of \citet{Shallue_2018}; who, in collaboration with Google, proposed a CNN named AstroNet that recognizes exoplanet signals in real data from Kepler. AstroNet uses the training set of labelled TCEs from the Autovetter planet candidate catalog of Q1–Q17 data release 24 (DR24) of the Kepler mission \citep{catanzarite2015autovetter}. AstroNet analyses the data in two views: a ``global view'', and ``local view'' \citep{Shallue_2018}. \par


% The global view shows the characteristics of the light curve over an orbital period, and a local view shows the moment at occurring the transit in detail

%different = space-based

Based on AstroNet, researchers have modified the original AstroNet model to rank candidates from different surveys, specifically for Kepler and TESS missions. \citet{ansdell2018scientific} developed a CNN trained on Kepler data, and included for the first time the information on the centroids, showing that the model improves performance considerably. Then, \citet{osborn2020rapid} and \citet{yu2019identifying} also included the centroids information, but in addition, \citet{osborn2020rapid} included information of the stellar and transit parameters. Finally, \citet{rao2021nigraha} proposed a pipeline that includes a new ``half-phase'' view of the transit signal. This half-phase view represents a transit view with a different time and phase. The purpose of this view is to recover any possible secondary eclipse (the object hiding behind the disk of the primary star).


%last pipeline applies a procedure after the prediction of the model to obtain new candidates, this process is carried out through a series of steps that include the evaluation with Discovery and Validation of Exoplanets (DAVE) \citet{kostov2019discovery} that was adapted for the TESS telescope.\par
%



\subsection{Attention mechanisms in astronomy}
\label{sec:relatedwork_attention}
Despite the remarkable success of attention mechanisms in sequential data, few papers have exploited their advantages in astronomy. In particular, there are no models based on attention mechanisms for detecting planets. Below we present a summary of the main applications of this modeling approach to astronomy, based on two points of view; performance and interpretability of the model.\par
%Attention mechanisms have not yet been explored in all sub-areas of astronomy. However, recent works show a successful application of the mechanism.
%performance

The application of attention mechanisms has shown improvements in the performance of some regression and classification tasks compared to previous approaches. One of the first implementations of the attention mechanism was to find gravitational lenses proposed by \citet{thuruthipilly2021finding}. They designed 21 self-attention-based encoder models, where each model was trained separately with 18,000 simulated images, demonstrating that the model based on the Transformer has a better performance and uses fewer trainable parameters compared to CNN. A novel application was proposed by \citet{lin2021galaxy} for the morphological classification of galaxies, who used an architecture derived from the Transformer, named Vision Transformer (VIT) \citep{dosovitskiy2020image}. \citet{lin2021galaxy} demonstrated competitive results compared to CNNs. Another application with successful results was proposed by \citet{zerveas2021transformer}; which first proposed a transformer-based framework for learning unsupervised representations of multivariate time series. Their methodology takes advantage of unlabeled data to train an encoder and extract dense vector representations of time series. Subsequently, they evaluate the model for regression and classification tasks, demonstrating better performance than other state-of-the-art supervised methods, even with data sets with limited samples.

%interpretation
Regarding the interpretability of the model, a recent contribution that analyses the attention maps was presented by \citet{bowles20212}, which explored the use of group-equivariant self-attention for radio astronomy classification. Compared to other approaches, this model analysed the attention maps of the predictions and showed that the mechanism extracts the brightest spots and jets of the radio source more clearly. This indicates that attention maps for prediction interpretation could help experts see patterns that the human eye often misses. \par

In the field of variable stars, \citet{allam2021paying} employed the mechanism for classifying multivariate time series in variable stars. And additionally, \citet{allam2021paying} showed that the activation weights are accommodated according to the variation in brightness of the star, achieving a more interpretable model. And finally, related to the TESS telescope, \citet{morvan2022don} proposed a model that removes the noise from the light curves through the distribution of attention weights. \citet{morvan2022don} showed that the use of the attention mechanism is excellent for removing noise and outliers in time series datasets compared with other approaches. In addition, the use of attention maps allowed them to show the representations learned from the model. \par

Recent attention mechanism approaches in astronomy demonstrate comparable results with earlier approaches, such as CNNs. At the same time, they offer interpretability of their results, which allows a post-prediction analysis. \par


\section{Conclusion}
In this work, we propose a simple yet effective approach, called SMILE, for graph few-shot learning with fewer tasks. Specifically, we introduce a novel dual-level mixup strategy, including within-task and across-task mixup, for enriching the diversity of nodes within each task and the diversity of tasks. Also, we incorporate the degree-based prior information to learn expressive node embeddings. Theoretically, we prove that SMILE effectively enhances the model's generalization performance. Empirically, we conduct extensive experiments on multiple benchmarks and the results suggest that SMILE significantly outperforms other baselines, including both in-domain and cross-domain few-shot settings.

\section*{Impact statement}




This paper presents work on language model distillation, which is actively used in the training of many modern language models. We identify a possible shortcoming of existing distillation procedures, called teacher hacking, that can lead to the transfer of unsafe behaviors from teacher to student. Additionally, we proposed several strategies to reduce the effect of this phenomenon. We believe that understanding and identifying such issues have positive societal consequences and allow the development of more reliable and safe language models.



\bibliography{reference}
\bibliographystyle{icml2025}


%%%%%%%%%%%%%%%%%%%%%%%%%%%%%%%%%%%%%%%%%%%%%%%%%%%%%%%%%%%%%%%%%%%%%%%%%%%%%%%
%%%%%%%%%%%%%%%%%%%%%%%%%%%%%%%%%%%%%%%%%%%%%%%%%%%%%%%%%%%%%%%%%%%%%%%%%%%%%%%
% APPENDIX
%%%%%%%%%%%%%%%%%%%%%%%%%%%%%%%%%%%%%%%%%%%%%%%%%%%%%%%%%%%%%%%%%%%%%%%%%%%%%%%
%%%%%%%%%%%%%%%%%%%%%%%%%%%%%%%%%%%%%%%%%%%%%%%%%%%%%%%%%%%%%%%%%%%%%%%%%%%%%%%
\newpage
\appendix
\onecolumn

\documentclass[preprint,12pt]{elsarticle}

%% Use the option review to obtain double line spacing
%% \documentclass[authoryear,preprint,review,12pt]{elsarticle}
\usepackage[english]{babel}
\usepackage[strings]{underscore}
\usepackage{microtype}
\usepackage{graphicx}
\usepackage{subcaption}
 \graphicspath{{./img/}}
 \DeclareGraphicsExtensions{.pdf}
\usepackage{booktabs}
\usepackage{multirow}
\usepackage{amsmath}
\usepackage[export]{adjustbox}
\usepackage{algcompatible}
\usepackage{lscape}
\usepackage[noend]{algpseudocode}
\usepackage{algorithm}
\usepackage[table]{xcolor}
\usepackage{xcolor}

\usepackage{xr}
\makeatletter
\newcommand*{\addFileDependency}[1]{
  \typeout{(#1)}
  \@addtofilelist{#1}
  \IfFileExists{#1}{}{\typeout{No file #1.}}
}
\makeatother

\newcommand*{\myexternaldocumenta}[1]{
    \externaldocument{#1}
    \addFileDependency{#1.tex}
    \addFileDependency{#1.aux}
}
%%% END HELPER CODE

% put all the external documents here!
\myexternaldocumenta{./JPDC2018}

% just to see what's happening
\listfiles


\begin{document}

\appendix
%%%%%%%%%%%%%%%%O
\section{Extended Simulation Results}
\label{results_appendix}

Figure \ref{fig:RLFT_HS10} and \ref{fig:RLFT_HS25} show the simulation results when 10\% and 25\% of the end-nodes generate congested traffic addressed to a single end-node (i.e., traffic pattern HS10-1 and HS25-1). 

As we can see in these figures, the 1-VC configuration cannot deal with congestion just applying restricted adaptive routing, due to the HoL blocking (see Figures \ref{fig_RLFT_HS10_1q},  \ref{fig_RLFT_HS10_1q-voq}, \ref{fig_RLFT_HS_1q} and \ref{fig_RLFT_HS_1q-voq}).


Note that when no VOQs are used, the adaptive routing using \emph{triggering thresholds} is able to raise a bit the performance, since this restriction delays the adaptivity decisions. The results show that better performance is achieved in a more congested scenario. 
This strange effect is due to the congestion tree. In fact, a higher incast congestion scenario creates a larger congestion tree that reaches the level 1 switches in the routing upward phase. Note that the performance raise in Figure \ref{fig_RLFT_HS10_1q} and \ref{fig_RLFT_HS_1q} when they reach 70\% and 50\% of the generated traffic. 
As a consequence, the performance of adaptive algorithms, restricted only to the second stage,  drops to 0.
By contrast, when VOQs are used, 1VC performance is close to 0\% as congestion trees grow into the VOQs, regardless the used routing algorithm.

DBBM (see Figures \ref{fig_RLFT_HS10_dbbm3}, \ref{fig_RLFT_HS10_dbbm3-voq}, \ref{fig_RLFT_HS_dbbm3} and \ref{fig_RLFT_HS_dbbm3-voq}) outperforms 1VC regardless the routing configurations.
When no VOQs are used,the results are homogeneous until the percentage of injected traffic exceeds 60\% and they show that K/$\Delta$ and 2S restrictions do not work as good as the other routing configurations.
The 2S restriction does not work well due to routing configuration uses $D$-mod-$K$ in the first stage of the Fat Tree, so that DBBM maps all the destinations to the same VC (see \figurename~\ref{fig_RLFT_3_2_destro}). Something similar happens with K/$\Delta$ restriction. This restriction uses a module to decide which upward port is chosen in the routing function \ref{algorithmAdaptive}. Since the module for choosing the port and mapping in the queue matches, then all adapted packets are mapped in the same queue. That's why the worst results are obtained when packets are adapted only in the second stage without any restrictions.
In other words, DBBM improves the deterministic routing when it adapts the routes in the first stage and manages to balance the use of queues.
However, when VOQs are used, low order HoL-blocking disappears and the results converge regardless the used routing algorithm.

When we use vFtree (see Figures \ref{fig_RLFT_HS10_vftree3}, \ref{fig_RLFT_HS10_vftree3-voq}, \ref{fig_RLFT_HS_vftree3-voq} and \ref{fig_RLFT_HS_vftree3}) the network efficiency drops significantly when adaptive routing is applied in the second stage (see the top-right switch \figurename~\ref{fig_RLFT_3_2_adapt} and downward stages in red), since all the destinations are mapped to all the VCs so that the HoL blocking probability increases.
In this case, the K/$\Delta$ restrictions avoid this defect and achieve productivity even when packets are adapted just in the second stage.
Note that *S-K configurations are able to raise the performance when the injection traffic is over 70\% and 60\%.  
This effect is more noticeable in the VOQ architecture because of its higher productivity.

When we use Flow2SL without VOQs (see Figures \ref{fig_RLFT_HS10_flow2sl3}, \ref{fig_RLFT_HS_flow2sl3}), we can see that better results are obtained when adaptive routing is applied in the first stage (FS and S*)  with triggering restrictions.
On the other hand, when switches implement VOQs (see Figures \ref{fig_RLFT_HS10_flow2sl3-voq}, \ref{fig_RLFT_HS_flow2sl3-voq}), the obtained results are similar.
The reason is that VOQs spread the congestion throughout all the VCs in the same buffer, since flow-control is performed at VC level, but not at VOQ level.

Figures \ref{fig:RLFT_HS104} and \ref{fig:RLFT_HS254} show experiment results for the same scenarios described before when we generate traffic creating four congestion trees  (i.e., traffic HS10-4 and HS25-4 depicted in \figurename~\ref{fig_traffic_HS4}).


This is a very strong congestion situation, since we generate four congestion trees whose branches affect to different VCs at the same time and thus reduces the effectiveness of the static queuing scheme.
Therefore, this situation makes complex, even for restricted routing combined with queuing schemes, to reduce the HoL blocking.
When no VOQs are used, the SQS show similar behaviour as described above but they yield less performance.
By contrast, when VOQs are used, the only queuing scheme that works is vFtree (see Figures \ref{fig_RLFT_HS_vftree3} and \ref{fig_RLFT_HS_vftree3}) due to its mapping properties. It works well with deterministic routing but a little more productivity can be achieved when we adapt packets using K/$\Delta$ restriction even when there are four strong congestion trees.

Figure \ref{fig:RLFT_IHS} shows simulation results for traffic generating hot-spots in intermediate stages of the topology (see \figurename~\ref{fig_traffic_HSW}).
In particular, hot-spots (i.e., congestion tree roots) are generated in the output ports of some switches placed at the second stage of the RLFT.
As we can see in the figures, this traffic pattern generates a congestion situation where $D$-mod-$K$, oblivious and fully-adaptive routing algorithms (both with and without restrictions) can deal with the HoL blocking appearing in the network.
For switches without VOQs, vFtree queuing scheme achieves the best results combined with adaptive routing using combined restrictions, such as ADAP-NoTH-SS-K, ADAP-TH-AS-K or ADAP-2TH-AS-$K/3$ outperform 1VC, DBBM and Flow2SL. 
This behavior is the same for vFtree when using switches with VOQs.
Note that switches with VOQs using the configuration ADAP-NoTH-SS-K achieve excellent results regardless the use of queuing schemes (see Figures  \ref{fig_RLFT_IHS_1q-voq}, \ref{fig_RLFT_IHS_dbbm3-voq},  \ref{fig_RLFT_IHS_vftree3-voq}, \ref{fig_RLFT_IHS_flow2sl3-voq}).
Although the use of restricted routing and queuing schemes are not necessary in this traffic scenario when using VOQs, note that their use preserves the performance gains shown in the previous traffic scenarios when using restricted adaptive routing and queuing schemes.
Therefore, the use of restricted adaptive routing combined with queuing schemes significantly increases the network performance under congested scenarios, compared to when we use deterministic and oblivious routing.

Figures \ref{fig:RLFT_histo_1q}, \ref{fig:RLFT_histo_dbbm}, \ref{fig:RLFT_histo_vftree} and \ref{fig:RLFT_histo_flow2sl} show in histograms the same results as the Tables \ref{tab:NOVOQ} and \ref{tab:VOQ}. This allows the comparison of different switch architectures for a given queuing scheme


% JESÜS: a partir de aquí está el texto anterior


%We have renamed the non-restricted adaptive routing as ``fully-adaptive'' routing, since it is the way it is called in the literature.
%More precisely, Figure \ref{fig:RLFT_HS10_vTime} shows the situation when 10\% of the end-nodes generate congesting packets addressed to a single destination (i.e., traffic pattern HS10-1), and \ref{fig:RLFT_HS25_vTime} shows the simulation results when 25\% of the end-nodes generate congested traffic addressed to a single end-node (i.e., traffic pattern HS25-1).


%As we can see in these figures, the 1-VC configuration (with and without VOQs) cannot deal with congestion just applying restricted and non-restricting adaptive routing, due to the HoL blocking (see Figures \ref{fig_RLFT_HS10_1q},  \ref{fig_RLFT_HS10_1q-voq}, \ref{fig_RLFT_HS_1q} and \ref{fig_RLFT_HS_1q-voq}).
%Note that when no VOQs are used, the adaptive routing using \emph{triggering thresholds} (i.e., ADAP-TH-*S-K and ADAP-2TH-*S-K) is able to raise a bit the performance (up to 30\% approximately), since this restriction delays the adaptivity decisions.
%By contrast, when VOQs are used, 1VC performance is close to 0\% as congestion trees grow into the VOQs, regardless the used routing algorithm.
%Figures \ref{fig_time_RLFT_HS10_1q} and  \ref{fig_time_RLFT_HS10_1q-voq} show similar results when simulations are run during $3$ms and we generate 100\% of traffic load.
%The hot-spot is generated by 10\% of the nodes.
%Similar results are shown in Figures \ref{fig_time_RLFT_HS_1q} and \ref{fig_time_RLFT_HS_1q-voq}, when 25\% of end-nodes generate congested traffic.

%When queuing schemes are used without VOQs, the network efficiency results raise from 30\% achieved by 1VC using the ADAP-2TH-AS-K configuration up to 55\% for DBBM, 64\% for vFtree and 73\%.
%DBBM (see Figures \ref{fig_RLFT_HS10_dbbm3}, \ref{fig_time_RLFT_HS10_dbbm3}, \ref{fig_RLFT_HS_dbbm3} and \ref{fig_time_RLFT_HS_dbbm3}) outperforms 1VC regardless the routing configurations.
%The best routing configuration for DBBM is 2TH-*S-H in these scenarios of moderate congestion.
%This routing configuration uses $D$-mod-$K$ in the first stage of the Fat Tree, so that DBBM maps all the destinations to the same VC (see \figurename~\ref{fig_RLFT_3_2_destro}).
%In the second stage, adaptive routing is used, but the path diversity is reduced to the destinations reaching the second stage of the Fat Tree (as $D$-mod-$K$ balances traffic flows in the first stage).
%Then, DBBM shares more destinations in the same VC and the queuing scheme effectiveness is reduced.
%For this reason, DBBM combined with the adaptive routing configurations using the restriction in the first stage (FS) obtains good performance results, regardless the traffic pattern (i.e., HS10-1 and HS25-1).
%Note also that $D$-mod-$K$ routing achieves 40\% of network efficiency (less than 55\% achieved by routing configurations with restrictions).
%Moreover, oblivious and fully-adaptive routing take advantage of the DBBM mapping policy, since they achieve almost 50\% of the network efficiency.

%By contrast, when we use vFtree (see Figures \ref{fig_RLFT_HS10_vftree3}, \ref{fig_time_RLFT_HS10_vftree3}, \ref{fig_RLFT_HS_vftree3} and \ref{fig_time_RLFT_HS_vftree3}) the network efficiency drops significantly when adaptive routing is applied in the second stage (SS) even through vFtree is using 3VCs (see the top-right switch \figurename~\ref{fig_RLFT_3_2_adapt} and downward stages in red), since all the destinations are mapped to all the VCs so that the HoL blocking probability increases.
%In the case of HS10-1 traffic, when the triggering restrictions are applied (TH or 2TH), the network efficiency raises after generation rate reaches 70\%, while, in the case of HS25-1 traffic it raisers when the generated traffic load is close to 50\%.
%Note that the congestion tree generated by the traffic HS25-1 is more intense than that generated by HS10-1.
%In the case of HS10-1, the congestion appears first in switches near the hot-spot destination end-node, and then it propagates backwards so that it reaches slower the switches in the first stage of the Fat Tree (upward-path), compared to the traffic pattern HS25-1.
%In the case of HS25-1, the congestion appears first in the first stages as more flows gather, then it propagates downwards.
%As we can see, vFtree requires more restrictions to the adaptive routing than DBBM, in order to perform properly.
%Note that vFtree with routing restrictions reaches 60\% of network efficiency, while $D$-mod-$K$, which can be considered as the most restricted routing, achieves 51\% of network effficiency.
%Indeed, the routing restrictions to vFtree also involve the number of used port counts ($K$).
%When this value is $K/3$ then vFtree performs better, although it requires to restrict the adaptivity per stages (FS or SS) and using the triggering thresholds (TH or 2TH).
%Note that the oblivious and fully-adaptive routing configurations results drop near 0\% due to the specific mapping of vFtree, which needs to reduce adaptivity mainly in the second stage, as we have described before.

%When we use Flow2SL (see Figures \ref{fig_RLFT_HS10_flow2sl3}, \ref{fig_time_RLFT_HS10_flow2sl3},  \ref{fig_RLFT_HS_flow2sl3} and \ref{fig_time_RLFT_HS_flow2sl3}), we can see that better results are obtained when adaptive routing is applied in the first stage (FS) and in all the stages (AS), compared to deterministic ($D$-mod-$K$) or oblivious routing configurations.
%Note that Flow2SL achieves the maximum throughput (around 70\% in Figures \ref{fig_RLFT_HS10_flow2sl3} and \ref{fig_RLFT_HS_flow2sl3}) with certain configurations of restrictions in the routing, such as ADAPT-2TH-AS-$K/3$.

%On the other hand, when switches implement VOQs, the obtained results are better, in general, than those obtained for switches without VOQs for DBBM, vFtree and Flow2SL.
%The reason is that VOQs spread the congestion throughout all the VCs in the same buffer, since flow-control is performed at VC level, but not at VOQ level.
%Hence, applying restrictions to the adaptive routing is unnecessary, since an efficient queuing scheme achieves good performance (as it happens for DBBM and Flow2SL). By contrast, vFtree needs to apply some of these restrictions when using VOQ-based switches, since the certain adaptivity degrees (i.e., mostly in the second stage) cause that the  vFtree mapping assigns many destinations to many VCs, thus increasing HoL blocking probability.
%As a consequence, another interesting observation is the homogeneous throughput obtained by VOQ-based switches.
%VOQ-based buffer organization switches show homogeneous results when generation rate achieves 60\% for different routing configurations while non-based show an unstable throughput, as in \figurename~\ref{fig_RLFT_HS10_1q-voq}, \ref{fig_RLFT_HS10_dbbm3-voq} ,\ref{fig_RLFT_HS10_vftree3-voq}, and \ref{fig_RLFT_HS10_flow2sl3-voq}.
%
%It is worth mentioning the additional throughput achieved in VOQ-based buffer organization configurations when the generation rates are lower than 40\% when adaptive routing is applied in the second stage (SS), as in \figurename~\ref{fig_RLFT_HS10_1q-voq}, \ref{fig_RLFT_HS10_vftree3-voq}, and \ref{fig_RLFT_HS10_flow2sl3-voq}. 
%Actually, the best results are usually obtained using path restrictions (K/3).

%Figures \ref{fig:RLFT_HS104}, \ref{fig:RLFT_HS104_vTime} \ref{fig:RLFT_HS254} and \ref{fig:RLFT_HS254_vTime} show experiment results for the same scenarios described before when we generate traffic creating four congestion trees  (i.e., traffic HS10-4 and HS25-4 depicted in \figurename~\ref{fig_traffic_HS4}).
%Note that, the four hot-spots in these scenarios are generated, respectively, by 10\% and 25\% of the source end-nodes generating traffic addressed to end-nodes $600$, $3400$, $5200$ and $9500$.
%This is a very strong congestion situation, since we generate four congestion trees whose branches  affect to different VCs at the same time.
%It is even possible that four different branches belonging to four different congestion trees grow within the same VC, producing HoL blocking in all the VCs. 
%Therefore, this situation makes complex, even for restricted routing combined with queuing schemes, to reduce the HoL blocking.


%As in the previous scenario, 1-VC with and without VOQs obtains the worst results, since the HoL blocking dramatically degrades the network performance.
%On the other hand, as the mapping of Flow2SL and DBBM is unfortunate, they achieve worse results than vFtree, since its mapping suits better the traffic situation.
%Note that $D$-mod-$K$ and oblivious routing achieve a performance near to 0\% for DBBM and Flow2SL queuing schemes, regardless the generated traffic load and the use of VOQs in the switches.
%Another reason for this bad performance is that this traffic pattern (see \figurename~\ref{fig_traffic_HS4}) produces very strong congestion situations in the first stage of the upward path and the adaptive and oblivious routing algorithms spread the congestion through the VCs and VOQs in the first stage of the RLFT.

%In the case of DBBM, it maps all the hot-spot destinations to all the VCs, since destination end-node $600$ is mapped to VC0, destinations $3400$ and $5200$ are mapped to VC1, and destination $9500$ is mapped to VC2.
%Note that the best results obtained for DBBM are those achieved by adaptive routing configurations using the triggering restrictions (i.e., TH and 2TH) and not restricting adaptivity to a particular stage (i.e., AS).
%By contrast, Flow2SL maps hot-spot destinations to all the groups in the topology, since destination $600$ is mapped to group $0$ (so to VC0), destinations $3400$ and $5200$ are mapped to group $1$ (so to VC1) and destination $9500$ is mapped to group $2$ (so to VC2).
%Hence, it is highly possible that all VCs suffer from HoL blocking, then degrading the Flow2SL efficiency.
%By contrast, vFtree maps destinations $600$ and $5200$ to VC0, while destinations $3400$ and $9500$ are mapped to VC2.
%Hence, the flows mapped to VC1 do not interact with congested flows in VC0 and VC1.
%We can see this effect in the results achieved by  $D$-mod-$K$.
%As for DBBM and Flow2SL, vFtree also takes advantage of the configurations using combined restrictions, such as ADAPT-2TH-AS-$K/3$.
%When VOQs are used, we can observe the unfortunate mapping effects for DBBM and Flow2SL, described in Section \ref{sec:problem}.

%Figures \ref{fig:RLFT_IHS} and \ref{fig:RLFT_IHS_vTime} show simulation results for traffic generating hot-spots in intermediate stages of the topology (see \figurename~\ref{fig_traffic_HSW}).
%In particular, hot-spots (i.e., congestion tree roots) are generated in the output ports of some switches placed at the second stage of the RLFT.
%As we can see in the figures, this traffic pattern generates a congestion situation where $D$-mod-$K$, oblivious and fully-adaptive routing algorithms (both with and without restrictions) can deal with the HoL blocking appearing in the network.
%For switches without VOQs, vFtree queuing scheme achieves the best results combined with adaptive routing using combined restrictions, such as ADAP-NoTH-SS-K, ADAP-TH-AS-K or ADAP-2TH-AS-$K/3$ outperform 1VC, DBBM and Flow2SL. 
%This behavior is the same for vFtree when using switches with VOQs.
%Note that switches with VOQs using the configuration ADAP-NoTH-SS-K achieve excellent results regardless the use of queuing schemes (see Figures  \ref{fig_RLFT_IHS_1q-voq}, \ref{fig_RLFT_IHS_dbbm3-voq},  \ref{fig_RLFT_IHS_vftree3-voq}, \ref{fig_RLFT_IHS_flow2sl3-voq}).
%Although the use of restricted routing and queuing schemes are not necessary in this traffic scenario when using VOQs, note that their use preserves the performance gains shown in the previous traffic scenarios when using restricted adaptive routing and queuing schemes.
%Therefore, the use of restricted adaptive routing combined with queuing schemes significantly increases the network performance under congested scenarios, compared to when we use deterministic and oblivious routing.

%Figures \ref{fig:RLFT_HS10} and \ref{fig:RLFT_HS25} show simulation results for the queuing schemes described before combined with deterministic ($D$-mod-$K$), oblivious and adaptive routing (restricted and non-restricted).
%We have re-named the non-restricted adaptive routing as ``fully-adaptive'' routing, since it is the way it is called in the literature.
%Also, we have added zooms to some of the figures in order to better reflect the differences among the data series.

\begin{figure*}[!htb]

\begin{subfigure}[!th]{\textwidth}
 \centering 
\includegraphics[width=1.0\textwidth]{leyenda3.pdf}
\end{subfigure}

 \begin{subfigure}[!th]{0.47\textwidth}
 \centering 
\includegraphics[width=0.82\textwidth]{./1q/Graphics/1q_synthetic_hotspot10_throughput_load.pdf}
\caption{1VC.}
\label{fig_RLFT_HS10_1q}
\end{subfigure}
 \begin{subfigure}[!th]{0.47\textwidth}
 \centering 
\includegraphics[width=0.82\textwidth]
{./1q-voq/Graphics/1q-voq_synthetic_hotspot10_throughput_load.pdf}
\caption{1VC and VOQs.}
\label{fig_RLFT_HS10_1q-voq}
\end{subfigure}

 \begin{subfigure}[!th]{0.47\textwidth}
 \centering 
\includegraphics[width=0.82\textwidth]
{./dbbm3/Graphics/dbbm3_synthetic_hotspot10_throughput_load.pdf}
\caption{DBBM.}
\label{fig_RLFT_HS10_dbbm3}
\end{subfigure}
 \begin{subfigure}[!th]{0.47\textwidth}
 \centering 
\includegraphics[width=0.82\textwidth]
{./dbbm3-voq/Graphics/dbbm3-voq_synthetic_hotspot10_throughput_load.pdf}
\caption{DBBM and VOQs.}
\label{fig_RLFT_HS10_dbbm3-voq}
\end{subfigure}

 \begin{subfigure}[!th]{0.47\textwidth}
 \centering 
\includegraphics[width=0.82\textwidth]
{./vftree3/Graphics/vftree3_synthetic_hotspot10_throughput_load.pdf}
\caption{vFtree.}
\label{fig_RLFT_HS10_vftree3}
\end{subfigure}
 \begin{subfigure}[!th]{0.47\textwidth}
 \centering 
\includegraphics[width=0.82\textwidth]
{./vftree3-voq/Graphics/vftree3-voq_synthetic_hotspot10_throughput_load.pdf}
\caption{vFtree and VOQs.}
\label{fig_RLFT_HS10_vftree3-voq}
\end{subfigure}

 \begin{subfigure}[!th]{0.47\textwidth}
 \centering 
\includegraphics[width=0.82\textwidth]
{./flow2sl3/Graphics/flow2sl3_synthetic_hotspot10_throughput_load.pdf}
\caption{Flow2SL.}
\label{fig_RLFT_HS10_flow2sl3}
\end{subfigure}
 \begin{subfigure}[!th]{0.47\textwidth}
 \centering 
\includegraphics[width=0.82\textwidth]
{./flow2sl3-voq/Graphics/flow2sl3-voq_synthetic_hotspot10_throughput_load.pdf}
\caption{Flow2SL and VOQs.}
\label{fig_RLFT_HS10_flow2sl3-voq}
\end{subfigure}

\caption{Normalized Throughput versus Generated Traffic Load in a $11664$-node RLFT under HS10-1 synthetic traffic pattern.}
\label{fig:RLFT_HS10}
\end{figure*}


%%%%%%%%%%%%%%%%
\begin{figure*}[!htb]
\vspace{-.5cm}
\begin{subfigure}[!th]{1\textwidth}
 \centering 
\includegraphics[width=1.0\textwidth]
{leyenda3.pdf}
\end{subfigure}

 \begin{subfigure}[!th]{0.47\textwidth}
 \centering 
\includegraphics[width=0.82\textwidth]
{./1q/Graphics/1q_synthetic_hotspot_throughput_load.pdf}
\caption{1VC.}
\label{fig_RLFT_HS_1q}
\end{subfigure}
 \begin{subfigure}[!th]{0.47\textwidth}
 \centering 
\includegraphics[width=0.82\textwidth]
{./1q-voq/Graphics/1q-voq_synthetic_hotspot_throughput_load.pdf}
\caption{1VC and VOQs.}
\label{fig_RLFT_HS_1q-voq}
\end{subfigure}

 \begin{subfigure}[!th]{0.47\textwidth}
 \centering 
\includegraphics[width=0.82\textwidth]
{./dbbm3/Graphics/dbbm3_synthetic_hotspot_throughput_load.pdf}
\caption{DBBM.}
\label{fig_RLFT_HS_dbbm3}
\end{subfigure}
 \begin{subfigure}[!th]{0.47\textwidth}
 \centering 
\includegraphics[width=0.82\textwidth]
{./dbbm3-voq/Graphics/dbbm3-voq_synthetic_hotspot_throughput_load.pdf}
\caption{DBBM and VOQs.}
\label{fig_RLFT_HS_dbbm3-voq}
\end{subfigure}

 \begin{subfigure}[!th]{0.47\textwidth}
 \centering 
\includegraphics[width=0.82\textwidth]
{./vftree3/Graphics/vftree3_synthetic_hotspot_throughput_load.pdf}
\caption{vFtree.}
\label{fig_RLFT_HS_vftree3}
\end{subfigure}
 \begin{subfigure}[!th]{0.47\textwidth}
 \centering 
\includegraphics[width=0.82\textwidth]
{./vftree3-voq/Graphics/vftree3-voq_synthetic_hotspot_throughput_load.pdf}
\caption{vFtree and VOQs.}
\label{fig_RLFT_HS_vftree3-voq}
\end{subfigure}

 \begin{subfigure}[!th]{0.47\textwidth}
 \centering 
\includegraphics[width=0.82\textwidth]
{./flow2sl3/Graphics/flow2sl3_synthetic_hotspot_throughput_load.pdf}
\caption{Flow2SL.}
\label{fig_RLFT_HS_flow2sl3}
\end{subfigure}
 \begin{subfigure}[!th]{0.47\textwidth}
 \centering 
\includegraphics[width=0.82\textwidth]
{./flow2sl3-voq/Graphics/flow2sl3-voq_synthetic_hotspot_throughput_load.pdf}
\caption{Flow2SL and VOQs.}
\label{fig_RLFT_HS_flow2sl3-voq}
\end{subfigure}

\caption{Normalized Throughput versus Generated Traffic Load in a $11664$-node RLFT under HS25-1 synthetic traffic pattern.}
\label{fig:RLFT_HS25}
\end{figure*}


%%%%%%%%%%%%%%%%
\begin{figure*}[!htb]
\vspace{-.5cm}
\begin{subfigure}[!th]{1\textwidth}
 \centering 
\includegraphics[width=1.0\textwidth]
{leyenda3.pdf}
\end{subfigure}

 \begin{subfigure}[!th]{0.47\textwidth}
 \centering 
\includegraphics[width=0.82\textwidth]
{./1q/Graphics/1q_synthetic_hotspot104_throughput_load.pdf}
\caption{1VC.}
\label{fig_RLFT_HS104_1q}
\end{subfigure}
 \begin{subfigure}[!th]{0.47\textwidth}
 \centering 
\includegraphics[width=0.82\textwidth]
{./1q-voq/Graphics/1q-voq_synthetic_hotspot104_throughput_load.pdf}
\caption{1VC and VOQs.}
\label{fig_RLFT_HS104_1q-voq}
\end{subfigure}

 \begin{subfigure}[!th]{0.47\textwidth}
 \centering 
\includegraphics[width=0.82\textwidth]
{./dbbm3/Graphics/dbbm3_synthetic_hotspot104_throughput_load.pdf}
\caption{DBBM.}
\label{fig_RLFT_HS104_dbbm3}
\end{subfigure}
 \begin{subfigure}[!th]{0.47\textwidth}
 \centering 
\includegraphics[width=0.82\textwidth]
{./dbbm3-voq/Graphics/dbbm3-voq_synthetic_hotspot104_throughput_load.pdf}
\caption{DBBM and VOQs.}
\label{fig_RLFT_HS104_dbbm3-voq}
\end{subfigure}

 \begin{subfigure}[!th]{0.47\textwidth}
 \centering 
\includegraphics[width=0.82\textwidth]
{./vftree3/Graphics/vftree3_synthetic_hotspot104_throughput_load.pdf}
\caption{vFtree.}
\label{fig_RLFT_HS104_vftree3}
\end{subfigure}
 \begin{subfigure}[!th]{0.47\textwidth}
 \centering 
\includegraphics[width=0.82\textwidth]
{./vftree3-voq/Graphics/vftree3-voq_synthetic_hotspot104_throughput_load.pdf}
\caption{vFtree and VOQs.}
\label{fig_RLFT_HS104_vftree3-voq}
\end{subfigure}

 \begin{subfigure}[!th]{0.47\textwidth}
 \centering 
\includegraphics[width=0.82\textwidth]
{./flow2sl3/Graphics/flow2sl3_synthetic_hotspot104_throughput_load.pdf}
\caption{Flow2SL.}
\label{fig_RLFT_HS104_flow2sl3}
\end{subfigure}
 \begin{subfigure}[!th]{0.47\textwidth}
 \centering 
\includegraphics[width=0.82\textwidth]
{./flow2sl3-voq/Graphics/flow2sl3-voq_synthetic_hotspot104_throughput_load.pdf}
\caption{Flow2SL and VOQs.}
\label{fig_RLFT_HS104_flow2sl3-voq}
\end{subfigure}

\caption{Normalized Throughput versus Generated Traffic Load in a $11664$-node RLFT under HS10-4 synthetic traffic pattern.}
\label{fig:RLFT_HS104}
\end{figure*}

%%%%%%%%%%%%%%%%
\begin{figure*}[!htb]
\vspace{-.5cm}
\begin{subfigure}[!th]{1\textwidth}
 \centering 
\includegraphics[width=1.0\textwidth]
{leyenda3.pdf}
\end{subfigure}

 \begin{subfigure}[!th]{0.47\textwidth}
 \centering 
\includegraphics[width=0.82\textwidth]
{./1q/Graphics/1q_synthetic_hotspot254_throughput_load.pdf}
\caption{1VC.}
\label{fig_RLFT_HS254_1q}
\end{subfigure}
 \begin{subfigure}[!th]{0.47\textwidth}
 \centering 
\includegraphics[width=0.82\textwidth]
{./1q-voq/Graphics/1q-voq_synthetic_hotspot254_throughput_load.pdf}
\caption{1VC and VOQs.}
\label{fig_RLFT_HS254_1q-voq}
\end{subfigure}

 \begin{subfigure}[!th]{0.47\textwidth}
 \centering 
\includegraphics[width=0.82\textwidth]
{./dbbm3/Graphics/dbbm3_synthetic_hotspot254_throughput_load.pdf}
\caption{DBBM.}
\label{fig_RLFT_HS254_dbbm3}
\end{subfigure}
 \begin{subfigure}[!th]{0.47\textwidth}
 \centering 
\includegraphics[width=0.82\textwidth]
{./dbbm3-voq/Graphics/dbbm3-voq_synthetic_hotspot254_throughput_load.pdf}
\caption{DBBM and VOQs.}
\label{fig_RLFT_HS254_dbbm3-voq}
\end{subfigure}

 \begin{subfigure}[!th]{0.47\textwidth}
 \centering 
\includegraphics[width=0.82\textwidth]
{./vftree3/Graphics/vftree3_synthetic_hotspot254_throughput_load.pdf}
\caption{vFtree.}
\label{fig_RLFT_HS254_vftree3}
\end{subfigure}
 \begin{subfigure}[!th]{0.47\textwidth}
 \centering 
\includegraphics[width=0.82\textwidth]
{./vftree3-voq/Graphics/vftree3-voq_synthetic_hotspot254_throughput_load.pdf}
\caption{vFtree and VOQs.}
\label{fig_RLFT_HS254_vftree3-voq}
\end{subfigure}

 \begin{subfigure}[!th]{0.47\textwidth}
 \centering 
\includegraphics[width=0.82\textwidth]
{./flow2sl3/Graphics/flow2sl3_synthetic_hotspot254_throughput_load.pdf}
\caption{Flow2SL.}
\label{fig_RLFT_HS254_flow2sl3}
\end{subfigure}
 \begin{subfigure}[!th]{0.47\textwidth}
 \centering 
\includegraphics[width=0.82\textwidth]
{./flow2sl3-voq/Graphics/flow2sl3-voq_synthetic_hotspot254_throughput_load.pdf}
\caption{Flow2SL and VOQs.}
\label{fig_RLFT_HS254_flow2sl3-voq}
\end{subfigure}

\caption{Normalized Throughput versus Generated Traffic Load in a $11664$-node RLFT under HS25-4 synthetic traffic pattern.}
\label{fig:RLFT_HS254}
\end{figure*}



%%%%%%%%%%%%%%%%
\begin{figure*}[!htb]
\vspace{-.5cm}
\begin{subfigure}[!th]{1\textwidth}
 \centering 
\includegraphics[width=1.0\textwidth]
{leyenda3.pdf}
\end{subfigure}

 \begin{subfigure}[!th]{0.47\textwidth}
 \centering 
\includegraphics[width=0.82\textwidth]
{./1q/Graphics/1q_synthetic_interhotspot_throughput_load.pdf}
\caption{1VC.}
\label{fig_RLFT_IHS_1q}
\end{subfigure}
 \begin{subfigure}[!th]{0.47\textwidth}
 \centering 
\includegraphics[width=0.82\textwidth]
{./1q-voq/Graphics/1q-voq_synthetic_interhotspot_throughput_load.pdf}
\caption{1VC and VOQs.}
\label{fig_RLFT_IHS_1q-voq}
\end{subfigure}

 \begin{subfigure}[!th]{0.47\textwidth}
 \centering 
\includegraphics[width=0.82\textwidth]
{./dbbm3/Graphics/dbbm3_synthetic_interhotspot_throughput_load.pdf}
\caption{DBBM.}
\label{fig_RLFT_IHS_dbbm3}
\end{subfigure}
 \begin{subfigure}[!th]{0.47\textwidth}
 \centering 
\includegraphics[width=0.82\textwidth]
{./dbbm3-voq/Graphics/dbbm3-voq_synthetic_interhotspot_throughput_load.pdf}
\caption{DBBM and VOQs.}
\label{fig_RLFT_IHS_dbbm3-voq}
\end{subfigure}

 \begin{subfigure}[!th]{0.47\textwidth}
 \centering 
\includegraphics[width=0.82\textwidth]
{./vftree3/Graphics/vftree3_synthetic_interhotspot_throughput_load.pdf}
\caption{vFtree.}
\label{fig_RLFT_IHS_vftree3}
\end{subfigure}
 \begin{subfigure}[!th]{0.47\textwidth}
 \centering 
\includegraphics[width=0.82\textwidth]
{./vftree3-voq/Graphics/vftree3-voq_synthetic_interhotspot_throughput_load.pdf}
\caption{vFtree and VOQs.}
\label{fig_RLFT_IHS_vftree3-voq}
\end{subfigure}

 \begin{subfigure}[!th]{0.47\textwidth}
 \centering 
\includegraphics[width=0.82\textwidth]
{./flow2sl3/Graphics/flow2sl3_synthetic_interhotspot_throughput_load.pdf}
\caption{Flow2SL.}
\label{fig_RLFT_IHS_flow2sl3}
\end{subfigure}
 \begin{subfigure}[!th]{0.47\textwidth}
 \centering 
\includegraphics[width=0.82\textwidth]
{./flow2sl3-voq/Graphics/flow2sl3-voq_synthetic_interhotspot_throughput_load.pdf}
\caption{Flow2SL and VOQs.}
\label{fig_RLFT_IHS_flow2sl3-voq}
\end{subfigure}

\caption{Normalized Throughput versus Generated Traffic Load in a $11664$-node RLFT under IHS synthetic traffic pattern.}
\label{fig:RLFT_IHS}
\end{figure*}



\begin{figure*}[!htb]
\vspace{-.5cm}
\begin{subfigure}[!th]{1\textwidth}
 \centering 
\includegraphics[width=1.0\textwidth]
{leyenda3.pdf}
\end{subfigure}

 \begin{subfigure}[!th]{1\textwidth}
 \centering 
\includegraphics[width=0.8\textwidth]
{./img/1q.pdf}
\caption{1VC.}
\label{fig_histo_1q}
\end{subfigure}
 \begin{subfigure}[!th]{1\textwidth}
 \centering 
\includegraphics[width=0.8\textwidth]
{./img/1q-voq.pdf}
\caption{1VC and VOQs.}
\label{fig_histo_1q-voq}
\end{subfigure}
\caption{Normalized Throughput after warmup period in a $11664$-node RLFT using 1Q and 1Q-VOQ.}
\label{fig:RLFT_histo_1q}
\end{figure*}

\begin{figure*}[!htb]
\vspace{-.5cm}
\begin{subfigure}[!th]{1\textwidth}
 \centering 
\includegraphics[width=1.0\textwidth]
{leyenda3.pdf}
\end{subfigure}

 \begin{subfigure}[!th]{1\textwidth}
 \centering 
\includegraphics[width=0.8\textwidth]
{./img/dbbm3.pdf}
\caption{DBBM.}
\label{fig_histo_dbbm}
\end{subfigure}
 \begin{subfigure}[!th]{1\textwidth}
 \centering 
\includegraphics[width=0.8\textwidth]
{./img/dbbm3-voq.pdf}
\caption{DBBM and VOQs.}
\label{fig_histo_dbbm-voq}
\end{subfigure}
\caption{Normalized Throughput after warmup period in a $11664$-node RLFT using DBBM and DBBM-VOQ.}
\label{fig:RLFT_histo_dbbm}
\end{figure*}

\begin{figure*}[!ht]
\vspace{-.5cm}
\begin{subfigure}[!th]{1\textwidth}
 \centering 
\includegraphics[width=1.0\textwidth]
{leyenda3.pdf}
\end{subfigure}

 \begin{subfigure}[!th]{1\textwidth}
 \centering 
\includegraphics[width=0.8\textwidth]
{./img/vftree3.pdf}
\caption{vFtree.}
\label{fig_histo_vftree}
\end{subfigure}
 \begin{subfigure}[!th]{1\textwidth}
 \centering 
\includegraphics[width=0.8\textwidth]
{./img/vftree3-voq.pdf}
\caption{vFtree and VOQs.}
\label{fig_histo_vftree-voq}
\end{subfigure}
\caption{Normalized Throughput after warmup period in a $11664$-node RLFT using vFtree and vFtree-VOQ.}
\label{fig:RLFT_histo_vftree}
\end{figure*}


\begin{figure*}[!ht]
\vspace{-.5cm}
\begin{subfigure}[!th]{1\textwidth}
 \centering 
\includegraphics[width=1.0\textwidth]
{leyenda3.pdf}
\end{subfigure}

 \begin{subfigure}[!th]{1\textwidth}
 \centering 
\includegraphics[width=0.8\textwidth]
{./img/flow2sl3.pdf}
\caption{Flow2SL.}
\label{fig_histo_flow2sl}
\end{subfigure}
 \begin{subfigure}[!th]{1\textwidth}
 \centering 
\includegraphics[width=0.8\textwidth]
{./img/flow2sl3-voq.pdf}
\caption{Flow2SL and VOQs.}
\label{fig_histo_flow2sl-voq}
\end{subfigure}
\caption{Normalized Throughput after warmup period in a $11664$-node RLFT using Flow2SL and Flow2SL-VOQ.}
\label{fig:RLFT_histo_flow2sl}
\end{figure*}

\end{document}

\endinput
As machine learning systems handle increasingly sensitive data, the potential for privacy violations grows. Li et al.    ~\cite{liu2021machine} categorize these privacy challenges into two primary areas: privacy attacks and privacy-preserving techniques. Privacy attacks have emerged as a critical concern in ML due to the growing realization that models can act as unintended leak vectors. These attacks can broadly be classified into different types, such as model inversion attacks, model extraction attacks, MIA. Model inversion attacks~\cite{fredrikson2015model} reconstruct input data from model outputs, while model extraction attacks ~\cite{juuti2019prada} aim to replicate a model’s functionality without direct access to its architecture or parameters. Among these, MIAs have gained significant attention due to their ability to infer whether a specific data point was used in training a model. According to the survey conducted by ~\cite{hu2022membership}, MIAs were first proposed in the context of genomic data by Homer et al.~\cite{homer2008resolving} where an attacker could infer if an individual's genome was part of a dataset based on summary statistics. Later, Shokri et al.\cite{shokri} introduced the first systematic MIA framework, showing how adversaries could use shadow models to infer training data membership. Salem et al. ~\cite{salem2018ml} reduced the complexity by demonstrating that a single shadow model can perform well compared to using multiple models, and they introduced metric-based attacks that rely on confidence scores and entropy without the need for identical data distribution between shadow and target models. Nasr et al.~\cite{nasr2019comprehensive} further expanded MIA into white-box settings, demonstrating that attackers with access to internal model parameters can perform even more effective MIAs. Melis et al.~\cite{melis2019exploiting} extended MIA to federated learning, showing vulnerabilities even in distributed learning settings, where multiple parties collaboratively train a model. Song and Mittal~\cite{song2021systematic} highlighted the increased privacy risks in generative models such as GANs, where membership inference attacks could be carried out on synthetic data generators. Recent work by Ilyas et al. introduced LiRA (Likelihood Ratio Attack) ~\cite{carlini2022membership}, a method that further improves the accuracy of MIAs by leveraging confidence scores more effectively to distinguish between training and non-training data points.In 2024, Zarifzadeh et al.~\cite{zarifzadeh2024low} introduced RMIA, a high-power membership inference attack that outperforms prior methods like LiRA and Attack-R, demonstrating superior robustness, particularly at low FPRs, using likelihood ratio tests.
While much of the data privacy research has centered around ANNs, expanding these investigations to SNNs is necessary. SNNs not only offer performance levels comparable to ANNs but also exhibit superior energy efficiency and hardware integration capabilities, positioning them as promising candidates for exploring inherent privacy features. Although privacy attacks on neuromorphic architectures remain underexplored, existing studies have yet to confirm SNNs' potential resistance to such threats. However, significant strides have been made in privacy-preserving techniques within the neuromorphic domain. For instance, recent efforts by Han et al. ~\cite{han2023towards} focus on developing privacy-preserving methods for SNNs, particularly utilizing FL and DP to address both computational efficiency and privacy challenges. Li et al.\cite{li2023efficient} introduced a framework that combines Fully Homomorphic Encryption (FHE) with SNNs, enabling encrypted inference while preserving SNNs' energy efficiency and computational advantages. Similarly, Nikfam et al.\cite{nikfam2023homomorphic} developed an HE framework tailored for SNNs, offering enhanced accuracy over DNNs under encryption, while carefully balancing computational efficiency. Additionally, Safronov et al.\cite{kim2022privatesnn} proposed PrivateSNN, a privacy-preserving framework for SNNs that employs differential privacy to mitigate membership inference attacks, maintaining the energy-efficient nature of SNNs.


The escalating computational demands of deep neural networks (DNNs), characterized by increased parameter counts and complex operations, have propelled quantization as a leading approach for reducing model size and computational load. Quantization compresses neural networks by reducing the precision of parameters, such as weights and activations, which lowers the memory footprint and enables efficient deployment on resource-constrained devices without substantial performance loss \cite{yang2019quantization, nagel2021white}. Early work on quantization focused on inference, using methods such as post-training quantization (PTQ) and quantization-aware training (QAT), which apply low-bit representations across entire network layers \cite{jacob2018quantization}. Recent advancements, like adaptive quantization frameworks, have further enhanced efficiency by dynamically adjusting bit-widths per layer, achieving higher compression rates while preserving accuracy \cite{tran2024privacy, zhou2021balanced}.

Quantization has emerged not only as an efficiency tool but also as a subtle defense mechanism in privacy-sensitive applications. This line of inquiry is particularly important given the vulnerability of DNNs to membership inference attacks (MIA), where adversaries exploit model confidence to infer whether specific data was part of the training set \cite{kowalski2022towards}. Quantized models, by introducing noise and reducing overfitting, have shown inherent resilience against such attacks, suggesting a potential dual role for quantization in enhancing privacy \cite{famili2023deep}. Experimental results have demonstrated that quantized models yield lower precision in MIA success rates compared to full-precision models, reducing the adversary’s ability to extract sensitive information without compromising model accuracy \cite{wei2024q, hu2021quantized}.

In recent years, these quantization techniques have been adapted for Spiking Neural Networks (SNNs), a neuromorphic approach inspired by biological neurons, which operate efficiently in edge computing environments due to their low energy consumption and sparse activity patterns. The development of quantization frameworks tailored to SNNs, such as Q-SpiNN and QFFS, has addressed unique challenges in quantizing synaptic weights, membrane potentials, and spike dynamics, crucial for maintaining the model’s accuracy and efficiency on low-power devices \cite{putra2021q, li2022quantization}. These frameworks use methods like integer quantization and fixed-point representation to optimize computational loads while preserving the distinctive temporal dynamics of SNNs, thus achieving a balance between efficiency and accuracy in neuromorphic hardware.

The privacy implications of quantized SNNs, however, remain an under-explored frontier. Given the inherent sparsity and quantized nature of SNNs, these networks hold promise for robust privacy preservation against attacks like MIA. By constraining information representation and introducing noise, quantized SNNs may naturally reduce the leakage pathways adversaries exploit. This potential resilience provides strong motivation for further investigation into quantized SNNs as a privacy-preserving framework, paving the way for more secure neuromorphic models in sensitive applications. This work aims to build on these foundational insights, exploring the intersection of quantization, efficiency, and privacy within SNNs \cite{schaefer2020quantizing, stock2019and, kowalski2022towards}.



% \begin{figure*}[ht]
%     \centering
    
%     % First row - MNIST and F-MNIST
%     \begin{subfigure}[b]{0.24\textwidth}
%         \centering
%         \includegraphics[width=\textwidth]{Figure/activation/ann_mnist_acc.png}
%         \caption*{ANN}
%         \label{fig:mnist_ann}
%     \end{subfigure}
%     \hfill
%     \begin{subfigure}[b]{0.24\textwidth}
%         \centering
%         \includegraphics[width=\textwidth]{Figure/activation/snn_mnist_acc.png}
%         \caption*{SNN}
%         \label{fig:mnist_snn}
%     \end{subfigure}
%     \hfill
%     \begin{subfigure}[b]{0.24\textwidth}
%         \centering
%         \includegraphics[width=\textwidth]{Figure/activation/ann_fmnist_acc.png}
%         \caption*{ANN}
%         \label{fig:fmnist_ann}
%     \end{subfigure}
%     \hfill
%     \begin{subfigure}[b]{0.24\textwidth}
%         \centering
%         \includegraphics[width=\textwidth]{Figure/activation/snn_fmnist_acc.png}
%         \caption*{SNN}
%         \label{fig:fmnist_snn}
%     \end{subfigure}
%     \caption*{(a) MNIST \hspace{8cm} (b) F-MNIST}
    
%     % Second row - CIFAR-10 and CIFAR-100
%     \vspace{1em}
%     \begin{subfigure}[b]{0.24\textwidth}
%         \centering
%         \includegraphics[width=\textwidth]{Figure/activation/ann_cifar10_acc.png}
%         \caption*{ANN}
%         \label{fig:cifar10_ann}
%     \end{subfigure}
%     \hfill
%     \begin{subfigure}[b]{0.24\textwidth}
%         \centering
%         \includegraphics[width=\textwidth]{Figure/activation/snn_cifar10_acc.png}
%         \caption*{SNN}
%         \label{fig:cifar10_snn}
%     \end{subfigure}
%     \hfill
%     \begin{subfigure}[b]{0.24\textwidth}
%         \centering
%         \includegraphics[width=\textwidth]{Figure/activation/ann_bc_acc.png}
%         \caption*{ANN}
%         \label{fig:cifar100_ann}
%     \end{subfigure}
%     \hfill
%     \begin{subfigure}[b]{0.24\textwidth}
%         \centering
%         \includegraphics[width=\textwidth]{Figure/activation/snn_bc_acc.png}
%         \caption*{SNN}
%         \label{fig:cifar100_snn}
%     \end{subfigure}
%     \caption*{(c) CIFAR-10 \hspace{8cm} (d) Breast Cancer}
    
%     % Final caption for the entire figure
%     \caption{Activation Quantization impact on Model Accuracy in (a) MNIST, (b) F-MNIST, (c) CIFAR-10, and (d) Breast Cancer.}
%     \label{fig:act_acc}
% \end{figure*}


% \begin{figure*}[ht]
%     \centering
    
%     % First row - MNIST and F-MNIST
%     \begin{subfigure}[b]{0.24\textwidth}
%         \centering
%         \includegraphics[width=\textwidth]{Figure/activation/ann_mnist_auc.png}
%         \caption*{ANN}
%         \label{fig:mnist_ann}
%     \end{subfigure}
%     \hfill
%     \begin{subfigure}[b]{0.24\textwidth}
%         \centering
%         \includegraphics[width=\textwidth]{Figure/activation/snn_mnist_auc.png}
%         \caption*{SNN}
%         \label{fig:mnist_snn}
%     \end{subfigure}
%     \hfill
%     \begin{subfigure}[b]{0.24\textwidth}
%         \centering
%         \includegraphics[width=\textwidth]{Figure/activation/ann_fmnist_auc.png}
%         \caption*{ANN}
%         \label{fig:fmnist_ann}
%     \end{subfigure}
%     \hfill
%     \begin{subfigure}[b]{0.24\textwidth}
%         \centering
%         \includegraphics[width=\textwidth]{Figure/activation/snn_fmnist_auc.png}
%         \caption*{SNN}
%         \label{fig:fmnist_snn}
%     \end{subfigure}
%     \caption*{(a) MNIST \hspace{8cm} (b) F-MNIST}
    
%     % Second row - CIFAR-10 and CIFAR-100
%     \vspace{1em}
%     \begin{subfigure}[b]{0.24\textwidth}
%         \centering
%         \includegraphics[width=\textwidth]{Figure/activation/ann_cifar10_auc.png}
%         \caption*{ANN}
%         \label{fig:cifar10_ann}
%     \end{subfigure}
%     \hfill
%     \begin{subfigure}[b]{0.24\textwidth}
%         \centering
%         \includegraphics[width=\textwidth]{Figure/activation/snn_cifar10_auc.png}
%         \caption*{SNN}
%         \label{fig:cifar10_snn}
%     \end{subfigure}
%     \hfill
%     \begin{subfigure}[b]{0.24\textwidth}
%         \centering
%         \includegraphics[width=\textwidth]{Figure/activation/ann_bc_auc.png}
%         \caption*{ANN}
%         \label{fig:cifar100_ann}
%     \end{subfigure}
%     \hfill
%     \begin{subfigure}[b]{0.24\textwidth}
%         \centering
%         \includegraphics[width=\textwidth]{Figure/activation/snn_bc_auc.png}
%         \caption*{SNN}
%         \label{fig:cifar100_snn}
%     \end{subfigure}
%     \caption*{(c) CIFAR-10 \hspace{8cm} (d) Breast Cancer}
    
%     % Final caption for the entire figure
%     \caption{Activation Quantization impact on MIA AUC score in (a) MNIST, (b) F-MNIST, (c) CIFAR-10, and (d) Breast Cancer.}
%     \label{fig:act_acc}
% \end{figure*}



% \begin{figure*}[ht!]
%   \centering
%   % First row - MNIST and F-MNIST
%   \begin{subfigure}[b]{0.24\textwidth}
%     \centering
%     \includegraphics[width=\textwidth]{Figure/weight/ann_mnist_w_quant.png}
%     \caption*{ANN}
%     \label{fig:mnist_ann}
%   \end{subfigure}%
%   \hfill
%   \begin{subfigure}[b]{0.24\textwidth}
%     \centering
%     \includegraphics[width=\textwidth]{Figure/weight/snn_mnist_w_quant.png}
%     \caption*{SNN}
%     \label{fig:mnist_snn}
%   \end{subfigure}%
%   \hfill
%   \begin{subfigure}[b]{0.24\textwidth}
%     \centering
%     \includegraphics[width=\textwidth]{Figure/weight/ann_fmnist_w_quant.png}
%     \caption*{ANN}
%     \label{fig:fmnist_ann}
%   \end{subfigure}%
%   \hfill
%   \begin{subfigure}[b]{0.24\textwidth}
%     \centering
%     \includegraphics[width=\textwidth]{Figure/weight/snn_fmnist_w_quant.png}
%     \caption*{SNN}
%     \label{fig:fmnist_snn}
%   \end{subfigure}%
%   \caption*{(a) MNIST \hspace{8cm} (b) F-MNIST}
  
%   % Second row - CIFAR-10 and Iris
%   \vspace{1em}
%   \begin{subfigure}[b]{0.24\textwidth}
%     \centering
%     \includegraphics[width=\textwidth]{Figure/weight/ann_cifar10_w_quant.png}
%     \caption*{ANN}
%     \label{fig:cifar10_ann}
%   \end{subfigure}%
%   \hfill
%   \begin{subfigure}[b]{0.24\textwidth}
%     \centering
%     \includegraphics[width=\textwidth]{Figure/weight/snn_cifar10_w_quant.png}
%     \caption*{SNN}
%     \label{fig:cifar10_snn}
%   \end{subfigure}%
%   \hfill
%   \begin{subfigure}[b]{0.24\textwidth}
%     \centering
%     \includegraphics[width=\textwidth]{Figure/weight/ann_iris_w_quant.png}
%     \caption*{ANN}
%     \label{fig:iris_ann}
%   \end{subfigure}%
%   \hfill
%   \begin{subfigure}[b]{0.24\textwidth}
%     \centering
%     \includegraphics[width=\textwidth]{Figure/weight/snn_iris_w_quant.png}
%     \caption*{SNN}
%     \label{fig:iris_snn}
%   \end{subfigure}%
%   \caption*{(c) CIFAR-10 \hspace{8cm} (d) Iris}

%   % Third row - Breast Cancer only, centered
%   \vspace{1em}
%   \hspace{\fill} % Centering space
%   \begin{subfigure}[b]{0.24\textwidth}
%     \centering
%     \includegraphics[width=\textwidth]{Figure/weight/ann_bc_w_quant.png}
%     \caption*{ANN}
%     \label{fig:bc_ann}
%   \end{subfigure}%
%   \hspace{0.5cm} 
%   \begin{subfigure}[b]{0.24\textwidth}
%     \centering
%     \includegraphics[width=\textwidth]{Figure/weight/snn_bc_w_quant.png}
%     \caption*{SNN}
%     \label{fig:bc_snn}
%   \end{subfigure}%
%   \hspace{\fill} % Centering space
%   \caption*{(e) Breast Cancer}

%   % Final caption for the entire figure
%   \caption{Impact of weight quantization on ANN and SNN models across different datasets: (a) MNIST, (b) F-MNIST, (c) CIFAR-10, (d) Iris, and (e) Breast Cancer.}
%   \label{fig:dp_mia}
% \end{figure*}
\section{Hyperparameter Search}\label{app:hype}
\normalsize
We exclusively conduct hyperparameter search on fold 0. 
For \textbf{GraFITi}~\citep{Yalavarthi2024.GraFITi} the hyperparameters for the search are as follows:
\begin{itemize}
    \item The number of layers, with possible values [1, 2, 3, 4].
    \item The number of attention heads, with possible values [1, 2, 4].
    \item The latent dimension, with possible values [16, 32, 64, 128, 256].
\end{itemize}

For the \textbf{LinODEnet} model~\citep{Scholz2022.Latenta} we search the hyperparameters from:
\begin{itemize}
    \item The hidden dimension, with possible values [16, 32, 64, 128].
    \item The latent dimension, with possible values [64, 128, 192, 256].
\end{itemize}

For \textbf{GRU-ODE-Bayes}~\citep{DeBrouwer2019.GRUODEBayesd} we tune the hidden size from [16, 32, 64, 128, 256]

For \textbf{Neural Flows}~\citep{Bilos2021.Neurald} we define the hyperparameter spaces for the search are as follows:
\begin{itemize}
    \item The number of flow layers, with possible values [1, 2, 4].
    \item The hidden dimension, with possible values [16, 32, 64, 128, 256].
    \item The flow model type, with possible values [GRU, ResNet].
\end{itemize}

For the \textbf{CRU}~\citep{Schirmer2022.Modelingb} the hyperparameter space is as follows:
\begin{itemize}
    \item The latent state dimension, with possible values [10, 20, 30].
    \item The number of basis functions, with possible values [10, 20].
    \item The bandwidth with possible values [3, 10].
\end{itemize}

%%%%%%%%%%%%%%%%%%%%%%%%%%%%%%%%%%%%%%%%%%%%%%%%%%%%%%%%%%%%%%%%%%%%%%%%%%%%%%%
%%%%%%%%%%%%%%%%%%%%%%%%%%%%%%%%%%%%%%%%%%%%%%%%%%%%%%%%%%%%%%%%%%%%%%%%%%%%%%%


\end{document}


% This document was modified from the file originally made available by
% Pat Langley and Andrea Danyluk for ICML-2K. This version was created
% by Iain Murray in 2018, and modified by Alexandre Bouchard in
% 2019 and 2021 and by Csaba Szepesvari, Gang Niu and Sivan Sabato in 2022.
% Modified again in 2023 and 2024 by Sivan Sabato and Jonathan Scarlett.
% Previous contributors include Dan Roy, Lise Getoor and Tobias
% Scheffer, which was slightly modified from the 2010 version by
% Thorsten Joachims & Johannes Fuernkranz, slightly modified from the
% 2009 version by Kiri Wagstaff and Sam Roweis's 2008 version, which is
% slightly modified from Prasad Tadepalli's 2007 version which is a
% lightly changed version of the previous year's version by Andrew
% Moore, which was in turn edited from those of Kristian Kersting and
% Codrina Lauth. Alex Smola contributed to the algorithmic style files.
