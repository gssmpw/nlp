\begin{figure*}[ht]
\centering
\begin{tcolorbox}[colback=white!95!blue, colframe=blue!40!, title=Prompt for Tool Selection Module $\mathcal{M}_{TS}$, width=\textwidth]
\small
\texttt{Your task is to help the user select appropriate APIs from the provided toolbox and complete the function template to solve the problem accurately.}\\

\texttt{\#\#\# Instructions}\\
\texttt{1. Input Provided:}\\
\parbox{2pt}{\hspace{8em}} \texttt{- Problem Description: A specific question or request from the user (e.g., "give me the number of movies directed by Sofia Coppola").}\\
\parbox{2pt}{\hspace{8em}} \texttt{- Pseudo-Code Task Template: A function template with the following components:}\\
\parbox{2pt}{\hspace{8em}} \quad \texttt{- Function signature, parameters, return type, and a functional description.}\\
\parbox{2pt}{\hspace{8em}} \quad \texttt{- Structured subtask comments directly embedded in the pseudo-code template, outlining each planned subtask step-by-step.}\\

\texttt{2. Expected Output:}\\
\parbox{2pt}{\hspace{8em}} \texttt{- Based on the problem description and the structured action plan provided, complete the pseudo-code by selecting and integrating suitable API calls from the provided candidate toolbox.}\\
\parbox{2pt}{\hspace{8em}} \texttt{- Use the placeholder 'call\_api(api\_path, params)' for each API call, where:}\\
\parbox{2pt}{\hspace{8em}} \quad \texttt{- 'api\_path' strictly matches a valid, existing API path from the provided candidate toolbox. Do not create or assume any non-existent API paths. No 'api\_path' outside this toolbox should be used or inferred.}\\
\parbox{2pt}{\hspace{8em}} \quad \texttt{- Place all request parameters within the 'params' dictionary; do not format 'api\_path' using string interpolation or formatted strings.}\\
\parbox{2pt}{\hspace{8em}} \quad \texttt{- Example: use 'call\_api(api\_path="\//3\//movie\//\{movie\_id\}\//credits", params=\{"movie\_id": movie\_id\})' instead of 'call\_api(api\_path=f"\//3\//movie\//{movie\_id}\//credits", params=\{\})'.}\\

\texttt{3. Guidelines for API Selection:}\\
\parbox{2pt}{\hspace{8em}} \texttt{- Do not provide a direct answer to the problem. Instead, fill in the pseudo-code template to enable successful execution.}\\
\parbox{2pt}{\hspace{8em}} \texttt{- Follow these steps for each 'call\_api' integration:}\\
\parbox{2pt}{\hspace{8em}} \quad \texttt{- Identify any necessary preliminary calls (e.g., fetching an entity ID).}\\
\parbox{2pt}{\hspace{8em}} \quad \texttt{- Ensure each step logically contributes to the function’s purpose, using helper variables and conditional checks where needed.}\\
\parbox{2pt}{\hspace{8em}} \quad \texttt{- Keep pseudo-code succinct, adding comments that clarify the role of each API call.}\\

\texttt{\#\#\# Provided API toolbox:}\\
\texttt{\{toolbox\}}\\

\texttt{Let's begin!}\\
\texttt{Question: \{question\}}\\
\texttt{Pseudo-Code Task: \{pseudo\_code\_task\}}\\
\texttt{Please complete the pseudo-code solution:}
\end{tcolorbox}
\caption{The prompt used for implementing the tool selection module $\mathcal{M}_{TS}$ in the RestBench-TMDB dataset during our experiments.}
\label{fig:tool_selection_prompt}
\end{figure*}