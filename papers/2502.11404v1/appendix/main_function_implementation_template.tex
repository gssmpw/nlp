\begin{figure*}[ht]
\centering
\begin{tcolorbox}[colback=white!95!blue, colframe=blue!40!, title=Prompt for Code Generation Module $\mathcal{M}_{CG}$, width=\textwidth]
\small
\texttt{You are a coding assistant specializing in Python and API integration. Your task is to translate code snippets with 'call\_api' placeholders into executable Python code. Each 'call\_api' placeholder represents an API call that should be implemented as a separate function.}\\

\texttt{\#\#\# Input:}\\
\texttt{1. A Python code snippet containing one or more 'call\_api' placeholders.}\\
\texttt{2. The OpenAPI documentation describing the APIs, including:}\\
\texttt{\parbox{2pt}{\hspace{8em}} - Endpoints ('path'),}\\
\texttt{\parbox{2pt}{\hspace{8em}} - Expected input parameters ('parameters'),}\\
\texttt{\parbox{2pt}{\hspace{8em}} - Response structure ('schema').}\\

\texttt{\#\#\# Requirements:}\\
\texttt{1. Function Creation: For each API in the OpenAPI specification:}\\
\texttt{\parbox{2pt}{\hspace{8em}} - First, check if the API's dictionary in the documentation contains a 'reusable\_code' field. If present:}\\
\texttt{\parbox{2pt}{\hspace{8em}} \quad - Extract the function code from the 'reusable\_code' field and reuse it to implement the corresponding API functionality in 'call\_api'.}\\
\texttt{\parbox{2pt}{\hspace{8em}} \quad - Make minimal modifications if necessary to integrate the function with the current codebase, while preserving its original logic.}\\
\texttt{\parbox{2pt}{\hspace{8em}} - If the 'reusable\_code' field is not present:}\\
\texttt{\parbox{2pt}{\hspace{8em}} \quad - Implement a Python function using the 'requests' library.}\\
\texttt{\parbox{2pt}{\hspace{8em}} \quad - Use 'https://api.themoviedb.org' as the 'base\_url'. Construct the full URL by appending the 'api\_path' to this 'base\_url'.}\\
\texttt{\parbox{2pt}{\hspace{8em}} \quad - Use the 'GET' method for all requests.}\\
\texttt{\parbox{2pt}{\hspace{8em}} \quad - Include two parameters in the function signature:}\\
\texttt{\parbox{2pt}{\hspace{8em}} \quad \quad - 'params': An optional dictionary for query parameters.}\\
\texttt{\parbox{2pt}{\hspace{8em}} \quad \quad - 'headers': A mandatory dictionary containing the following: headers = \{"Authorization": "Bearer \{API\_KEY\}"\}}\\
\texttt{\parbox{2pt}{\hspace{8em}} \quad \quad - Ensure the function returns parsed response data in a usable format.}\\
\texttt{2. Integration: Replace each 'call\_api' placeholder in the original code with calls to the corresponding functions, ensuring that the replacement maintains the original logic.}\\
\texttt{3. Error Handling: Add error handling to manage API failures (e.g., HTTP errors). Use 'try/except' blocks to log errors or raise exceptions.}\\
\texttt{4. Code Quality:}\\
\texttt{\parbox{2pt}{\hspace{8em}}- Write clear and descriptive docstrings for each function.}\\
\texttt{\parbox{2pt}{\hspace{8em}}- Follow Python best practices for readability and maintainability.}\\

\texttt{\#\#\# Additional Notes:}\\
\texttt{\parbox{2pt}{\hspace{8em}} - Validate input parameters where applicable.}\\
\texttt{\parbox{2pt}{\hspace{8em}} - Ensure all requests include the 'headers' defined above.}\\
\texttt{\parbox{2pt}{\hspace{8em}} - Clearly document any assumptions made during implementation.}\\
\texttt{\parbox{2pt}{\hspace{8em}} - If reusing functions from the 'reusable\_code' field, ensure their integration complies with Python conventions and the overall project architecture.}\\

\texttt{\#\#\# Output:}\\
\texttt{The complete, executable Python code with:}\\
\texttt{\parbox{2pt}{\hspace{8em}} - Defined functions for each API,}\\
\texttt{\parbox{2pt}{\hspace{8em}} - Updated logic that calls these functions.}\\

\texttt{Now, it's your turn!}\\
\texttt{\#\#\# Question: \{question\}}\\
\texttt{\#\#\# Input:
\texttt{\{code\_solution\}}}\\
\texttt{\#\#\# OpenAPI Documents as well as a reusable code snippet (optional):}\\
\texttt{\{api\_doc\}}\\
\texttt{\#\#\# Output:}\\
\end{tcolorbox}
\caption{The prompt utilized for implementing the code generation module $\mathcal{M}_{CG}$ in the RestBench-TMDB dataset for our experiments.}
\label{fig:main_function_template}
\end{figure*}