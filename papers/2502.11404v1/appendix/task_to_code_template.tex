\begin{figure*}[ht]
\centering
\begin{tcolorbox}[colback=white!95!blue, colframe=blue!40!, title=Prompt for Task to Code Module $\mathcal{M}_{T2C}$, width=\textwidth]
\small
\texttt{You are a coding assistant specialized in converting task requirements into precise Python function definitions. Your role is to define a function that fully meets the task's objectives by specifying a clear function name, parameters, return type, and docstring.}\\
\\
\texttt{\#\#\# Instructions:}\\
\texttt{For each given task, carefully follow these steps to create a well-structured function definition:}\\
\texttt{1. Function Name: Choose a concise, meaningful function name that accurately reflects the task's purpose.}\\
\texttt{2. Parameters: Identify and define the essential input(s) the function requires. For each parameter:}\\
\texttt{\parbox{2pt}{\hspace{8em}} - Use a descriptive name and include a type annotation.}\\
\texttt{3. Return Type: Specify an appropriate return type (e.g., 'int', 'str', 'dict', 'list') that best represents the function’s output.}\\
\texttt{4. Docstring: Write a detailed, clear docstring that includes:}\\
\texttt{\parbox{2pt}{\hspace{8em}} - A brief description of the function’s purpose.}\\
\texttt{\parbox{2pt}{\hspace{8em}} - Descriptions for each parameter, detailing its type and role.}\\
\texttt{\parbox{2pt}{\hspace{8em}} - Explanation of the return value, with type and a brief on what it represents.}\\
\texttt{\parbox{2pt}{\hspace{8em}} - Explicit Notes on using 'call\_api', with the following rules:}\\
\texttt{\parbox{2pt}{\hspace{8em}} \quad - Each 'call\_api' invocation must use an 'api\_path' from the provided toolbox; do not invent or alter paths.}\\
\texttt{\parbox{2pt}{\hspace{8em}} \quad - Place all request parameters within the 'params' dictionary; do not format 'api\_path' using string interpolation or formatted strings.}\\
\texttt{\parbox{2pt}{\hspace{8em}} \quad - Example: use 'call\_api(api\_path="\//3\//movie\//\{movie\_id\}\//credits", params=\{"movie\_id": movie\_id\})' instead of 'call\_api(api\_path=f"\//3\//movie\//{movie\_id}\//credits", params=\{\})'.}\\
\texttt{\parbox{2pt}{\hspace{8em}} \quad - Refer to the example functions below for guidance on structuring this.}\\
\texttt{\parbox{2pt}{\hspace{8em}}  - Use triple single quotes for the docstring, as shown below.}\\
\texttt{5. Function Body: Leave the function body empty. Only include the function definition, parameters, and docstring in your output.}\\

\texttt{Now, let's begin!}\\
\texttt{You are given a question: \{question\}}\\
\texttt{Python Function:}
\end{tcolorbox}
\caption{The prompt used for implementing the task-to-code function module $\mathcal{M}_{T2C}$ in the RestBench-TMDB dataset during our experiments.}
\label{fig:task_to_code_prompt}
\end{figure*}