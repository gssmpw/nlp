\section{Conclusion and Future Work}

We proposed \methodName, an LLM-based approach for mapping SIEM threat detection rules to MITRE ATT\&CK techniques. 
While LLMs possess implicit knowledge derived from publicly available data, their direct application in cybersecurity contexts is often limited due to domain-specific challenges. 
Our experiments showed that \methodName's performance significantly improves when additional contextual information is integrated.

We identified two primary strategies for incorporating such information: (i) explicitly supplying contextual data in real time through LLM agents, and (ii) fine-tuning the LLM with domain-specific information. 
In this study, we adopted the first approach, enriching the pipeline in real time with publicly available contextual data sourced from the Internet as fine-tuning an LLM requires labelled dataset. 
As part of future work, we plan to enhance \methodName by incorporating organization-specific contextual information, which can further tailor the model to specific operational environments.
Additionally, we aim to explore fine-tuning LLMs with organization specific contextual data as an alternative approach to further improve the prediction accuracy of \methodName. 
Future research will also investigate optimization techniques such as hyperparameter tuning and ensemble methods to further enhance the performance of the proposed method. 
 With these enhancements \methodName will serve as a reliable and adaptable solution for mapping SIEM rules to MITRE ATT\&CK techniques in dynamic and complex cybersecurity landscapes.

The adoption of \methodName will contribute greatly to the automation of the cybersecurity incident response pipeline.
It also provides a roadmap for integrating advanced LLMs into the defensive strategies of organizations worldwide.
% 