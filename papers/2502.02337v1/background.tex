\section{\label{sec:background}Background}

LLMs, due to their advanced NLP and generation capabilities, are well-suited to analyze structured data by understanding its syntactical meaning, as well as for providing human understandable reasoning. 
However, effectively mapping structured data, such as SIEM rules, to the MITRE ATT\&CK framework~\cite{al2024mitre} requires not only syntactical understanding but also a deep comprehension of the semantic meaning of the rules. 
LLMs must discern the semantic content of a rule and identify its alignment with the techniques and sub-techniques. % defined in the MITRE ATT\&CK framework. 
Relying solely on the implicit knowledge of LLMs is insufficient for this complex task.

To address this challenge, various prompt engineering techniques~\cite{sahoo2024systematic} were implemented to enhance the LLMs' ability to understand the semantic nuances of SIEM rules and propose relevant MITRE ATT\&CK techniques and sub-techniques. 
These prompt engineering techniques are discussed in detail in this section.
Additionally, to provide context for readers unfamiliar with the MITRE ATT\&CK framework, a brief overview of its structure and purpose is also included in this section.

\subsection{MITRE ATT\&CK Framework}
The MITRE ATT\&CK framework~\cite{al2024mitre} is a comprehensive, globally recognized knowledge base that provides detailed insights into the tactics, techniques, and procedures (TTPs) used by adversaries in cyberattacks. 
It is designed to aid cybersecurity professionals in understanding adversarial behaviors, identifying attack patterns, and strengthening defensive strategies. 
By systematically categorizing adversarial actions, the framework helps organizations enhance their detection, response, and mitigation capabilities.

The framework is structured around three primary components: tactics, techniques, and procedures. 
\textit{Tactics} represent the high-level objectives that adversaries aim to achieve during an attack. 
These objectives outline the “why” behind an adversary’s actions and are categorized into stages of an attack lifecycle, such as Initial Access, Execution, Persistence, Privilege Escalation, and Exfiltration. 
\textit{Techniques} %, on the other hand, 
define the specific methods used to accomplish these tactical goals. 
These represent the “how” of an attack and include actions like phishing (to gain initial access), command and scripting interpreter usage (for execution), or credential dumping (to gain access to sensitive credentials). 
Complementing these, \textit{procedures} provide real-world examples of how techniques are operationalized, offering context on specific tools, scripts, or strategies used in documented adversarial campaigns.

In addition to TTPs, the MITRE ATT\&CK framework introduces two critical concepts: data sources and mitigations. 
\textit{Data sources} refer to the various types of telemetry and system-generated data that can be collected and analyzed to detect adversarial techniques.
\textit{Mitigations} describe the preventive or corrective actions that can be implemented to neutralize threats or limit their impact.

\subsection{Prompt Engineering}

Prompt engineering~\cite{sahoo2024systematic} is the practice of crafting precise and effective input prompts to optimize the performance and output of LLMs.
It involves designing queries or instructions that guide the model's understanding and execution of tasks. 
This technique has become pivotal in ensuring that LLMs generate accurate, contextually appropriate, and reliable results, especially in tasks requiring nuanced reasoning or complex problem-solving.

\subsubsection{Prompt Chaining}

Prompt chaining~\cite{wu2022promptchainer} is a technique that decomposes complex tasks into a sequence of smaller, logically ordered steps. 
In this approach, the output from one step serves as the input for the next, enabling a modular and iterative resolution of intricate problems. 
For instance, when generating a report, the initial prompt could request an outline, subsequent prompts could expand individual sections, and a final prompt could synthesize the results into a cohesive document. 
This method improves the clarity and manageability of multifaceted tasks.

\subsubsection{Chain-of-thought Prompting}
Another notable technique is chain-of-thought (CoT) prompting~\cite{wei2022chain}, which implements a step-by-step reasoning.
By explicitly including intermediate reasoning steps within the prompt or instructing the model to generate these steps, CoT prompting enhances the AI's capability to address tasks that require logical inference or multi-step computation.

\subsection{LLM Agents and React Framework}
LLM agents, or AI Agents~\cite{llmagents}, leverage the capabilities of large language models to perform tasks autonomously by reasoning, planning, and acting based on input instructions. 
They are typically used in applications like chatbots, decision-making systems, or task automation. 
These agents operate by combining LLMs with external tools, APIs, or environments to handle complex tasks that require more than natural language generation.

The REACT (REasoning \& ACTing) framework~\cite{yao2023reactsynergizingreasoningacting} enhances the functionality of LLM agents by combining logical reasoning with actionable operations in a unified system. 
REACT agents dynamically integrate high-level reasoning and decision-making with the execution of actions in an iterative feedback loop. 
This enables the agent to automatically analyze tasks, plan appropriate steps, execute actions, and adapt to new information or evolving contexts.
The REACT framework operates through a structured workflow: (1) The LLM interprets the input query or instruction, performs logical analysis, and generates a plan of action, (2) The agent executes the planned actions, such as querying an API or controlling an external system, (3) The results of the actions are analyzed, allowing the agent to refine its reasoning and plan subsequent steps, and (4) The final output integrates the outcomes of reasoning and acting, delivering a comprehensive response to the user.