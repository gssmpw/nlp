\section{Related Work}
\label{section2}
This section provides a comprehensive review and analysis of related work, including network telemetry and traffic prediction, which will serve as a foundation for future research.

\subsection{Network Telemetry}
In recent years, the development of Software-Defined Networking (SDN) and Programmable Data Planes (PDP) has significantly enhanced network telemetry capabilities. INT, by embedding real-time network status information into data packets, provides fine-grained, low-latency network monitoring, enabling more accurate network performance management. However, traditional INT methods are often limited in their ability to adapt to rapid network topology changes and high-frequency telemetry requirements. As network complexity increases, particularly with the introduction of high-load switches and dynamic traffic patterns, traditional INT systems struggle to offer both high precision and efficiency \cite{5}. INT uses data packets within the network to collect network information, which can be further classified into passive network telemetry (PNT) and ANT.

PNT does not actively inject probes into the network. Instead, it relies on existing packets within the network to collect network information. Compared to traditional network measurement techniques, 
Typical PNT systems, such as Sel-INT \cite{33}, PINT \cite{10}, and INT-label \cite{11}, offer finer-grained measurements, real-time capabilities, flexibility, scalability, and data consistency. These systems provide network managers with precise and comprehensive network state information, enabling more effective network management and optimization.

However, a significant problem of PNT lies in its inability to obtain comprehensive network-wide information due to the uncertainty of existing packet forwarding paths. To solve this problem, ANT actively sends probes to collect network information along user-specified forwarding paths. Typical ANT systems, such as INT-path \cite{12}, IntOpt \cite{8}, and NetView \cite{13}, can achieve full network coverage and enhanced flexibility. 
Nevertheless, existing ANT systems still face challenges in balancing telemetry overhead and accuracy. High-precision network telemetry often leads to significant telemetry overhead, potentially impacting network performance \cite{9}. 
These network telemetry systems can not adapt well to dynamic network environments and diversified telemetry requirements.

To address this issue, AdapINT is proposed as a DRL-based in-network telemetry system \cite{AdapINT}. Facing dynamic network environments, AdapINT brings flexible and adaptive solutions for probe path planning \cite{20}. The strong generalization capability of DRL enables it to quickly adapt and make wise decisions in new or unknown network environments, ensuring the continuous effectiveness of probe path planning \cite{23}. However, DRL introduces other pressing issues to AdapINT. Firstly, in response to sudden traffic changes in the network, AdapINT cannot pre-adjust the probe paths. This significantly reduces its ability to cope with sudden traffic changes. Secondly, for large-scale networks, the action space in the DRL model becomes extremely large, which causes the model training time to grow exponentially and severely affects the system's operation.


\subsection{Network Traffic Prediction}


Network traffic prediction technology is important in a dynamic network environment. Due to the constant changes in network traffic \cite{16}, traditional telemetry strategies based on the current state of the network can quickly become obsolete, affecting the real-time and accuracy of data collection \cite{15}. Network traffic prediction technology can use historical data to predict future network traffic \cite{17}, accurately identify high-load switches, and optimize data collection frequency, significantly improving the efficiency of telemetry systems.


Traditional traffic prediction technologies primarily use statistical characteristics derived from historical data, such as the Autoregressive Integrated Moving Average (ARIMA) model \cite{35}. However, with the advent of machine learning, there has been a significant evolution in traffic prediction methods. These advancements include the utilization of SVM \cite{36}, LSTM \cite{37}, GNN \cite{38}, and so on. These modern methods have demonstrated remarkable proficiency in capturing intricate patterns and dynamics within traffic data, thereby enhancing prediction accuracy.


Integrating traffic prediction with network telemetry can mitigate the challenges posed by dynamic network environments. However, existing prediction methods typically rely on historical data and fail to adapt to rapid network changes, such as topology shifts or traffic surges \cite{40}. The key challenge is effectively integrating telemetry technology to capture more accurate and comprehensive traffic data \cite{39}. Moreover, selecting the right prediction model for different network scenarios remains an open issue \cite{27}.

Several studies have explored combining network telemetry with traffic prediction to enhance monitoring efficiency. For example, integrating real-time telemetry data from INT with machine learning techniques like SVM or LSTM has improved prediction accuracy. However, these methods still struggle to adapt to sudden network changes quickly. The Multi-Temporal Graph Neural Network (MTGNN) applied in this paper addresses these issues by leveraging both real-time INT data and temporal traffic dependencies, offering more accurate and timely predictions, especially in dynamic environments.