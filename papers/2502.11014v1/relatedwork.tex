\section{Related Work}
In recent times, there has been a surge of comprehensive reviews and analytical studies in the field of text classification within NLP. These reviews span a gamut of methodologies, from conventional linguistic approaches to the latest deep learning techniques. For instance, Roy et al. [10] have effectively applied LSTM and CNN models to the challenge of filtering spam in SMS messages, although their models were primarily optimized for texts in English. Another study [11] advocates for the adoption of various supervised machine-learning models for differentiating between spam and legitimate messages. In this study, a comparative analysis among Naive Bayes, maximum entropy, and SVM classifiers revealed that SVMs outperformed the others with an accuracy of 97.4\% on datasets in real-time environments, albeit at the cost of higher memory usage.

\begin{table*}
\centering
\caption{Summary of Recent Research on Spam Detection and Related Areas}
\label{tab:research_summary}
\begin{tabular}{|p{3cm}|p{1cm}|p{5cm}|p{3.5cm}|p{4cm}|}
\hline
\small
\textbf{Author} & \textbf{Year} & \textbf{Aim} & \textbf{Method} & \textbf{Result} \\
\hline
Shen et al. [45] & 2025 & Leverage BERT and GCN to improve SMS spam detection. & BERT-G3CN & Achieved high accuracy (99.28\% and 93.78\%) in benchmark datasets. \\ \hline
Haider Rizvi et al. [46] & 2025 & Provide a comprehensive review of GCN-based text classification methods. & GCN-based Text Classification & Summarized key strengths, challenges, and research gaps in GCN approaches. \\ \hline
Tusher et al. [47] & 2025 & Analyze deep learning methods for filtering email spam. & Deep Learning & Outlined various deep learning methods, their effectiveness, and future research needs. \\ \hline
Gong et al. [48] & 2025 & Survey existing research on language models for code optimization. & Systematic Literature Review & Identified research gaps and challenges in LM-based code optimization. \\ \hline
Liu et al. [49] & 2024 & Review applications of PLMs in cybersecurity. & PLM Analysis & Explored various cybersecurity applications of PLMs and future research directions. \\ \hline
Chen et al. [50] & 2024 & Investigate how LLMs enhance threat detection in cybersecurity. & LLM-based Threat Detection & Reviewed use cases, limitations, and potential improvements for LLM-based threat detection. \\ \hline
Alshatnawi et al. [51] & 2024 & Utilize contextual word embeddings to refine social media spam detection. & BERT, ELMo & Achieved superior accuracy using ELMo embeddings in detecting social media spam. \\ \hline
Chataut et al. [52] & 2024 & Compare the effectiveness of ML models and LLMs in spam detection. & Traditional ML vs. LLMs & LLMs demonstrated better performance in spam detection compared to traditional ML models. \\ \hline
Raja Abdul et al. [58] & 2024 & Identify smishing attacks using ML and NLP techniques. & SMSecure & Showed improved accuracy in identifying smishing attacks. \\ \hline
Raja Abdul Samad et al. [54] & 2024 & Propose an advanced model integrating phonetic and textual embeddings for detecting Chinese spam. & BiGRU-TextCNN & Enhanced Chinese spam detection by incorporating phonetic embeddings. \\ \hline
Shrestha [55] & 2023 & Introduce a refined approach to detecting smishing attacks with machine learning. & Random Forest, Extreme Gradient Boosting & Machine learning models exhibited high accuracy in detecting smishing attempts. \\ \hline
Yao et al. [56] & 2022 & Develop a high-performance spam email detection model. & XLNet & XLNet-based model outperformed existing methods with high precision and recall. \\ \hline
\end{tabular}
\end{table*}



Further developments in the domain have been marked by [12, 13], who introduced new approaches for spam SMS filtering utilizing LSTMs and RNNs within the frameworks of Keras and TensorFlow. This resulted in an impressive accuracy rate of 98\% when tested on the UCI dataset, yet the complexity of these models was considerably high due to their intricate architectures.  Lee and Kang [14] developed an SMS spam filtering method that utilizes a feed-forward neural network in combination with CBOW word embeddings. Their study revealed that simply increasing the number of hidden layers—such as from 27 onward—did not significantly enhance the accuracy of the model. This emphasizes the importance of layer quality over sheer quantity. The research landscape has been further advanced by computer scientists concentrating on machine learning and deep learning-based models. For instance, Xu et al. [15] introduced a novel technique to extract textual features from email image attributes and employed SVM for classification to optimize spam filtering. However, this approach was limited to image-based spam detection. Moreover, Almeida et al. [16] investigated the integration of lexicographical resources, semantic dictionaries, and diverse semantic analysis techniques to refine and improve text messages, ultimately strengthening the effectiveness of the classification process.

This evolution in text classification reflects the dynamic and ever-expanding scope of NLP as it continues to explore new horizons in both theory and application.The realm of text classification, particularly in identifying spam content, has witnessed notable advances, with several researchers contributing novel techniques and methodologies that showcase the progress in machine learning and NLP. Almeida et al. [17] provided evidence that SVM classifiers had an edge over others within a new public SMS spam collection, reinforcing the relevance of SVM in text classification tasks. The traditional machine learning approaches they discussed necessitated procedures for dimensionality reduction, detailed feature engineering, and an extensive exploratory data analysis before application. Jain and Gupta [18] introduced a feature-based methodology aimed at identifying smishing attacks, focusing on distinguishing between genuine and malicious communications. Their approach underscores the evolving sophistication in detecting fraudulent content.
Ghourabi et al. [19] managed to achieve an impressive 98.37\% accuracy by harnessing the combined strengths of CNN and LSTM models, surpassing the performance of previous models. This demonstrates the effectiveness of hybrid deep learning models in handling complex pattern recognition tasks in text. Meanwhile, Bassiouni et al. [20] conducted experiments with different classifiers for the purpose of email filtering, testing against the Spambase dataset from UCI. The Random Forest (RF) algorithm stood out with a 95.45\% accuracy, while other classifiers showed competitive, albeit slightly lower, performance metrics.Abbasalizadeh et al. [57] developed PriLink, a secure link scheduling framework designed to enhance privacy in wireless networks by restricting unnecessary topology exposure while ensuring efficient resource allocation. Their approach demonstrated superior privacy preservation and faster execution compared to conventional scheduling methods. 
Saeed et al. [21] discussed the application of supervised machine learning algorithms such as J48 and KNN to segregate spam from non-spam messages, indicating that traditional classifiers still hold value in the ever-evolving landscape of spam detection. Srinivasarao et al. [22] broke new ground by developing a hybrid classifier that combines KNN and SVM. They innovated further by integrating Word2vec for feature extraction and Rat Swarm Optimization for tuning the model parameters, aiming to create a robust system for SMS spam classification. Lastly, Dharani et al. [23] proposed a model leveraging Naive Bayes coupled with TF-IDF vectorization, achieving a notable 95\% accuracy and a perfect precision score. This result exemplifies the continued relevance of Naive Bayes in text classification when paired with powerful feature extraction techniques. These varied studies and their results underscore the dynamic nature of spam detection research, where both traditional algorithms and newer, more complex models find their niches. They reflect ongoing efforts to improve accuracy, precision, and overall efficiency in text classification within the broader context of NLP.
Rayavaram et al. [55] introduced CryptoEL, a cryptography education tool for K-12 students that employs interactive simulations, AI-driven dialogues, and coding exercises to reinforce learning. Their study reported high student engagement and improved comprehension of core cryptographic concepts. The referenced studies underscore the wide-ranging applications of machine learning in categorizing spam and non-spam messages. Gautam [24] carried out a comprehensive evaluation of various machine learning classifiers for spam detection, emphasizing the distinction between spam and legitimate messages. The examined classifiers included K-Nearest Neighbors (KNN), Linear Support Vector Machine (SVM), Radial Basis Function-SVM (RBF-SVM), Random Forest, and Decision Trees, with performance metrics such as accuracy, precision, recall, and F1-score utilized for comparison. The study revealed that classifiers generally perform better on balanced datasets, as they prevent bias toward the majority class. However, KNN was found to be an exception to this trend. Notably, the Linear SVM classifier demonstrated the highest accuracy across both balanced and imbalanced datasets, highlighting its robustness and effectiveness in text classification.

Prasad et al. [36], in a 2022 study, sought to improve the accuracy of spam detection by leveraging machine learning methods, with a specific emphasis on the Multinomial Naive Bayes and Decision Tree classifiers. To ensure statistical validity, they employed GPower to determine an appropriate sample size, maintaining a power of 0.8 and an alpha threshold of 0.05. Their results revealed that the Multinomial Naive Bayes classifier attained a superior accuracy of 96.50\% compared to the Decision Tree model. Moreover, their analysis indicated no statistically significant difference in performance and loss rate between the two models, as demonstrated by a p-value exceeding 0.536. This suggests that the observed performance enhancement did not lead to additional computational burdens or overfitting