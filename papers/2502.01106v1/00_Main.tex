\documentclass[mnsc,nonblindrev,hidelinks]{informs3_homepage}

\OneAndAHalfSpacedXI

% Natbib setup for author-year style
\usepackage{natbib}
 \bibpunct[, ]{(}{)}{,}{a}{}{,}%
 \def\bibfont{\small}%
 \def\bibsep{\smallskipamount}%
 \def\bibhang{24pt}%
 \def\newblock{\ }%
 \def\BIBand{and}%

\TheoremsNumberedBySection  % (Theorem 1.1, Lema 1.1, Theorem 1.2)
\ECRepeatTheorems

\EquationsNumberedBySection % (1.1), (1.2), ...

\usepackage{float}
\usepackage[caption=false]{subfig}
\usepackage{amsmath}
\usepackage{algorithm}
\usepackage{algpseudocode}
\usepackage{ifthen}
\usepackage{enumitem}
\usepackage{cases}
\usepackage{mathtools}
\usepackage{amssymb}
\usepackage{graphicx}
\usepackage{empheq}
\usepackage{tikz}
\usetikzlibrary{shapes.geometric}
\usetikzlibrary{shapes}
\usetikzlibrary{arrows}
\usetikzlibrary{calc,positioning}
\usepackage{bm}
\usepackage{dsfont}
\usepackage[backref=page]{hyperref}
\usepackage{macros}
\usepackage{mathrsfs}
\usepackage{booktabs}


% Redefine \Require and \Ensure to use "Data" and "Result"
\renewcommand{\algorithmicrequire}{\textbf{Data:}}
\renewcommand{\algorithmicensure}{\textbf{Result:}}

%%%%%%%%%%%%%%%%
\begin{document}
%%%%%%%%%%%%%%%%

\RUNTITLE{Can We Validate Counterfactual Estimations in the Presence of Unknown Network Interference?}

\TITLE{Can We Validate Counterfactual Estimations in the Presence of General Network Interference?}
  
\ARTICLEAUTHORS{%
\AUTHOR{Sadegh Shirani$^1$, Yuwei Luo$^1$, William Overman$^1$, Ruoxuan Xiong$^2$, and Mohsen Bayati$^1$}
\AFF{$^1$Stanford University, $^2$Emory University} % 
}
\ABSTRACT{%
In experimental settings with network interference, a unit's treatment can influence outcomes of other units, challenging both causal effect estimation and its validation. Classic validation approaches fail as outcomes are only observable under one treatment scenario and exhibit complex correlation patterns due to interference. To address these challenges, we introduce a new framework enabling cross-validation for counterfactual estimation. At its core is our \batching\ method—a theoretically-grounded approach inspired by approximate message passing. This method creates multiple subpopulations while preserving the underlying distribution of network effects. We extend recent causal message-passing developments by incorporating heterogeneous unit-level characteristics and varying local interactions, ensuring reliable finite-sample performance through non-asymptotic analysis. We also develop and publicly release a comprehensive benchmark toolbox with diverse experimental environments, from networks of interacting AI agents to opinion formation in real-world communities and ride-sharing applications. These environments provide known ground truth values while maintaining realistic complexities, enabling systematic examination of causal inference methods. Extensive evaluation across these environments demonstrates our method's robustness to diverse forms of network interference. Our work provides researchers with both a practical estimation framework and a standardized platform for testing future methodological developments.
}%


\KEYWORDS{Network interference, message-passing model, experimental design, network data, high-dimensional data} 
% \HISTORY{}

\maketitle
%%%%%%%%%%%%%%%%%%%%%%%%%%%%%%%%%%%%%%%%%%%%%%%%%%%%%%%%%%%%%%%%%%%%%%


\section{Introduction}

\begin{figure*}
    \centering
    \includegraphics[width=\textwidth]{figures/Introduction.pdf}
    \caption{Showing the novel problem statement applied to traffic prediction use case. Multiple unstructured observations from the past are used to reconstruct a hidden traffic state from which a full traffic state is forecast with a set of query locations. }
    \label{fig:intro}
\end{figure*}

% Was sagen denn die anderen warum Traffic Prediction gut ist? 
Forecasting the traffic in the near future is an important task for city management.
Data from the near past is used to predict future traffic states with spatio-temporal Graph Neural Networks \cite{bui22}.
Accurate prediction provides the opportunity to optimize traffic flow, reduce traffic jams and increase air quality \cite{Po19}.

% Wieso ist Sparsity in allen Dimensionen wichtig.
While traffic prediction relies on the availability of data from traffic sensors, there exists a plethora of reasons why sensors may stop working temporarily, such as simple errors, energy saving, or overloaded communication systems.
Considering small- or medium-sized cities, the coverage of sensors may be low because the sensors are too expensive or not available.
Also, the sensors are typically static and do not adapt to changes in the traffic flow (e.g. caused by a construction site), which motivates moving sensors that for example could be mounted on cars. 
However, both missing and moving sensors introduce sparsity, since measurements may not be available for all locations at all times.
This sparsity must be explicitly addressed in traffic prediction for a realistic application scenario, which is illustrated in figure \ref{fig:intro}.
From one hour of data on Sunday morning, only few observations of the traffic state are available at each timestep.
The number of observations may differ throughout the observed time and the observation itself can be distributed arbitrarily in the city. 
We assume a relatively low number of sensors to account for resource saving and sensor failure in our proposed framework SUSTeR.
The task is to predict the dense traffic state one timestep after the observations at all possible sensor locations.
We study this problem on the traffic dataset Metr-LA and PEMS-BAY to test our assumption that only a fraction of the sensor values would be enough for good predictions.
By modifying an existing traffic dataset, we are able to compare our results from very sparse observations to the bottom line with all information available.
A successful study will provide insights in how sensors in new cities can be reduced before installing them and further mobile sensors would save more resources and are able to adapt to new traffic situations.
We argue that in order to be adaptable to other cities and changes in traffic flows, prior information like the road network should be neglected and just the sparse observations considered.
This comes with the added benefit of making our solution applicable in regions where no openly available road network is maintained or pathways change frequently (e.g. flood areas, animal observations). 


The aforementioned problem is novel and more challenging than the commonly considered traffic prediction problem, since there exist very few observations in each input sample.
Current works for the traffic prediction problem do not consider any missing values. \cite{Li2021, Shao22}
A common method among state of the art approaches is the usage of Graph Neural Networks on graphs that model the sensor network \cite{bui22}.
The values of a sensor are applied to the same graph node for each timestep which prohibits any non-stationary sensors . 
With fixed sensor locations, the resulting sensor network is highly correlated with the road network.
Streets connecting two intersections with sensors should be also an interesting point for correlations in the sensor network.
However, variable observations and high temporal sparsity rates can not be modeled adequately in a static network.
We show in our experiments that the road network has only a small influence on the traffic predictions.

Besides the traffic prediction for future timesteps, some works explore the field of traffic speed imputation \cite{Cini22, Cuza22} where missing sensor values are predicted.
But the amount of missing values is assumed to be at most 80\%, which on average are still over 40 given sensors in each timestep in the Metr-LA dataset with a total of 207 sensors.
We consider up to 99.9\% missing values which are on average 2.4 observations in each timestep that are used as input.
Such high sparsity rates drastically decrease the chance that multiple values are present in one input sample from the same sensor location, which makes it challenging to recognize and learn temporal correlations for each location on its own.

High sparsity rates (>95\%) result in few sensor values, but if a reconstruction of the traffic state would be possible, we question if spatio-temporal graphs require nodes for each sensor.
In SUSTeR we utilize only a small amount of graph nodes for the encoding of information and do not relate such nodes to the sensor network.
We call this the hidden graph (see figure \ref{fig:intro}), which is still able to reconstruct the complete traffic state.
Due to the reduced number of nodes SUSTeR achieves faster runtimes, as shown in the experiments.
This hidden graph is not embedded directly in the spatial domain, which is why the assignment of observations, as well as the querying of the future traffic, is done with an encoder and a decoder, implemented as neural networks.
The decoding from the hidden graph to future values depends on a set of query locations.
Figure \ref{fig:intro} shows the query locations as given from outside and in combination with the reconstructed traffic state the future values are predicted.

To construct the hidden graph we encode observations from each timestep into from multiple graphs, one for each timestep. 
The graphs are created in a residual style and information is added to the node embeddings from the previous timesteps.
We choose this method to incorporate all timesteps equally into the hidden state because the redundant information along the past is non-existing for high sparsity rates.
From the sequence of graphs where our framework inserted the observations step by step we apply STGCN \cite{Yu18}, an algorithm for traffic prediction to find and learn the spatio-temporal correlations on our small number of graph nodes.
The first future timestep of the STGCN is our hidden graph in which the traffic state is reconstructed. 

% Recent work has an implicit embedding of the graph nodes into the spatial domain as the assignment from the sensor to graph node is fixed one by one.
% Because the graph has the same structure as the road network spatio-temporal correlations can be learned between those sensors.
% We reduce the number of nodes and use a non-linear assignment learned data-driven from the observations.

We find in the experiments that SUSTeR outperforms the plain STGCN and modern traffic prediction frameworks like D2STGNN for high sparsity rates $(\geq 99\%)$.
This is equivalent to only $0.2$ to $2.4$ observation for each timestep on average.
SUSTeR uses fewer parameters than the baselines and can train faster and with less training data.
Our main contributions can be summarized as follows:
\begin{itemize}
    \item We introduce a sparse and unstructured variant of the traffic prediction problem with sparsity in all dimensions. The sensors report only a fraction of their values and are arbitrarily distributed in the spatial domain.
    \item We propose SUSTeR, a framework around the STGCN architecture, which maps sparse observations onto a dense hidden graph to reconstruct the complete traffic state.
    Our code is available at github.\footnote{https://github.com/ywoelker/SUSTeR}
    \item We conducts experiments that show that SUSTeR outperforms the baselines in very sparse situations ($\geq 95\%$) and has a competitive performance in low sparsity rates.
    % \item SUSTeR trains a third faster than the next competitor.
\end{itemize}

\section{General Counterfactual Estimation}
\label{sec:Causal_Estimands}
% 
Estimating counterfactual evolution $\left\{ \CFE{\OMtreatment{}{}'}{}{t} \right\}_{t=0}^T$ based on the observed outcomes $\Moutcome{}{}{}(\Mtreatment{}{} = \OMtreatment{}{})$ enables addressing a broad range of causal questions, such as, “\emph{What if we had delivered the treatments according to $\OMtreatment{}{}'$ instead of $\OMtreatment{}{}$}?" For example, if outcomes are observed after treating 20\% of potential voters, campaign organizers may want to explore how the population would have responded if 40\% had received the campaign materials. In addition, by estimating the dynamics of the counterfactuals over time, we gain insights into how the treatment effect may strengthen or weaken as time progresses.

The total treatment effect (TTE) provides a formal way to compare any two counterfactual scenarios. For two treatment allocations $\OMtreatment{}{}'$ and $\OMtreatment{}{}''$, the TTE measures the difference in population average outcomes:
% 
\begin{align}
    \label{eq:TTE_def}
    \TTE{t}{\OMtreatment{}{}'',\OMtreatment{}{}'} = \CFE{\OMtreatment{}{}''}{}{t} - \CFE{\OMtreatment{}{}'}{}{t},
    \quad\quad\quad
    t = 0,1, \ldots, T\,.
\end{align}
%
The literature typically focuses on a special case: when all entries of $\OMtreatment{}{}''$ equal one and all entries of $\OMtreatment{}{}'$ equal zero, with all outcomes at equilibrium \citep{yu2022estimating,candogan2023correlated,ni2023design,ugander2023randomized}. However, this extreme scenario of treating the entire population versus treating no one is often impractical. As we demonstrate in subsequent sections, our framework enables the estimation of TTEs between any two treatment allocations, providing decision-makers with more realistic comparisons \citep{muralidharan2017experimentation,egger2022general}.


\subsection{Experimental Design Framework}
\label{sec:Experimental_Design}
% 
We proceed by generalizing the experimental design. Specifically, we allow treatment assignments $\treatment{i}{t}$ to vary over time following a Bernoulli distribution with mean $\expr_t$, and define the experimental design as the product distribution $\expd = \expr_0 \times \ldots \times \expr_T$. In this context, we use $\Vtreatment{}{t}{} := (\treatment{1}{t}, \ldots, \treatment{N}{t})^\top$ to represent the vector containing the treatment assignments for all experimental units at time~$t$. This formulation encompasses a broad range of experimental designs, including staggered roll-out design \citep{xiong2024optimal}, micro-randomized trials \citep{li2022network}, and switchback experiments \citep{bojinov2023design}.

\begin{remark}
In Appendix~\ref{sec:Technical_Results}, we further generalize this framework by allowing $\treatment{i}{t}$ to follow more general probability distributions~$\pi_t$ and defining the experimental design as $\expd = \pi_0 \times \ldots \times \pi_T$. The random variable $\treatment{i}{t}$ can take values in either an integer set (e.g., different treatment types) or a continuous set (e.g., varying treatment doses). This generalization distinguishes our work from recent literature on network interference and longitudinal data, which primarily focuses on binary treatments \citep{arkhangelsky2023causal}.
\end{remark}

For a fixed integer $\dcovar$, we can also consider covariates (might be observed or not) in the form of a $\dcovar$ by $N$ matrix $\covar$, where each column (denoted by $\Vcovar{i} := (\Ecovar{1}{i}, \ldots,  \Ecovar{\dcovar}{i})^\top$) represents characteristics of unit $i$ (e.g., age and gender). The experimental data thus consists of treatment assignments~$\OMtreatment{}{}$, observed outcomes 
%
\[
\Moutcome{}{}{}(\OMtreatment{}{}) := \Moutcome{}{}{}(\Mtreatment{}{}=\OMtreatment{}{})\,,
\]
%
and covariates $\covar$. Therefore, we observe outcomes only under one specific treatment allocation $\OMtreatment{}{}$—a single realization among $2^{\UN (\TH+1)}$ distinct potential outcomes, where this number grows exponentially with population size and time horizon. Consequently, estimating causal effects under general interference is impossible due to non-identifiability \citep{karwa2018systematic}. To address this challenge, we propose a tractable outcome specification that aligns with and extends the causal message passing framework \citep{shirani2024causal}.


\subsection{Potential Outcome Specification}
\label{sec:Outcome_Specification}
% 
Let $\Mtreatment{t}{} := [\Vtreatment{}{0}{}| \ldots| \Vtreatment{}{t}{}]$ denote the treatment assignments up to time $t$; we represent by $\VoutcomeD{}{}{t}(\Mtreatment{t}{}) = \big(\outcomeD{}{1}{t}(\Mtreatment{t}{}), \ldots, \outcomeD{}{N}{t}(\Mtreatment{t}{})\big)^\top$ the potential outcome vector at time $t$. Consider unknown functions $\outcomeg{t}{}$ and $\outcomeh{t}{}$ which operate component-wise, and the expressions $\outcomeg{t}{}\big(\VoutcomeD{}{}{t}(\Mtreatment{t}{}), \Vtreatment{}{t+1}{}, \covar\big)$ and $\outcomeh{t}{}\big(\VoutcomeD{}{}{t}(\Mtreatment{t}{}), \Vtreatment{}{t+1}{}, \covar \big)$ represent the corresponding column vectors. Given $\VoutcomeD{}{}{0}(\Mtreatment{0}{})$ as the initial outcome vector, we consider the following specification for potential outcomes:
% 
\begin{align}
    \label{eq:outcome_function_matrix}
    \VoutcomeD{}{}{t+1}(\Mtreatment{t+1}{}) =
    \big(\IM+\IMatT{t}\big)\outcomeg{t}{}\left(\VoutcomeD{}{}{t}(\Mtreatment{t}{}) ,\Vtreatment{}{t+1}{}, \covar\right)
    +
    \outcomeh{t}{}\left(\VoutcomeD{}{}{t}(\Mtreatment{t}{}) ,\Vtreatment{}{t+1}{}, \covar\right)
    +
    \Vnoise{}{t},
    \;\;
    t=0,1,\ldots,T-1,
\end{align}
% 
where $\IM$ and $\IMatT{t}$ are $N\times N$ unknown matrices capturing the fixed and time-dependent interference effects, respectively. Additionally, $\Vnoise{}{t} = \big(\noise{1}{t},\ldots,\noise{N}{t}\big)^\top$ is the zero-mean noise term accounting for observation noise. Denoting by $\IMatl{ij}$ and $\IMatTl{ij}{t}$ the element in the $i^{th}$ row and $j^{th}$ column of $\IM$ and $\IMatT{t}$, respectively, the value $\IMatl{ij}+\IMatTl{ij}{t}$ quantifies the impact of unit $j$ on unit $i$ at time $t$.

The specification in Eq. \eqref{eq:outcome_function_matrix} captures several aspects of experimental data. It accommodates various types of interference, including treatment spillover effects, carryover effects, peer effects, and autocorrelation \citep{shirani2024causal}. Notably, this specification accounts for outcome dynamics by acknowledging the temporal interrelation of units' outcomes. This contrasts with existing approaches to panel data analysis, which assume that time labels can be shuffled without affecting causal effects \citep{arkhangelsky2023causal}.

%
\begin{remark}
In Appendix~\ref{sec:Technical_Results}, we further generalize the outcome specification in Eq.~\eqref{eq:outcome_function_matrix} to incorporate additional lag terms (e.g., $\VoutcomeD{}{}{t-1}(\Mtreatment{t-1}{})$) in the functions $\outcomeg{t}{}$ and $\outcomeh{t}{}$, the complete treatment matrix at any time point (allowing for anticipation effects), and time-dependent covariates.
\end{remark}
%

Next, we analyze the state evolution of the experimental population, characterizing the asymptotic dynamics of unit outcomes. These theoretical foundations establish the basis for developing robust counterfactual estimators in subsequent sections.


\subsection{Experimental State Evolution}
\label{sec:ESE}
% 
In this section, we analyze the distribution of unit outcomes over time. This analysis requires the following two assumptions about the interference matrices.
%
\begin{assumption}[Gaussian interference structure]
    \label{asmp:Gaussian Interference Matrice}
    For all $i,j$, the element $\IMatl{ij}$ in the $i^{th}$ row and $j^{th}$ column of $\IM$ is an independent Gaussian random variable with mean $\mu^{ij}/N$ and variance $\sigma^2/N$. Similarly, $\IMatTl{ij}{t}$, the element in the $i^{th}$ row and $j^{th}$ column of $\IMatT{t}$, is an independent Gaussian random variable with mean $\mu^{ij}_t/N$ and variance $\sigma^2_t/N$.
\end{assumption}
%
%
\begin{assumption}[Convergent interference pattern - informal statement]
    \label{asmp:Stable Interference Pattern}
    For all unit $i$ and any time $t$, the elements of vector $(\mu^{i1}, \ldots, \mu^{iN})$ admit a weak limit\footnote{This means that the empirical distribution of ${\mu^{i1}, \ldots, \mu^{iN}}$ converges to a probability distribution as $N$ increases.} and, separately, the elements of vector $(\mu^{i1}_t, \ldots, \mu^{iN}_t)$ admit a weak limit, where both limits are invariant in $i$.
\end{assumption}
%
According to Assumptions~\ref{asmp:Gaussian Interference Matrice} and \ref{asmp:Stable Interference Pattern}, the impact of unit $j$ on unit $i$ is captured by $\mu^{ij}$, $\mu^{ij}_t$, and two centered Gaussian random variables. This generalizes the model of \cite{shirani2024causal}, which assumes i.i.d. interference matrix elements across all units. This extension accommodates more heterogeneous local interactions and varying levels of uncertainty about the interference structure. Specifically, a fully known interference network has exact values for $\mu^{ij}$ and $\mu^{ij}_t$ with $\sigma^2 = 0$ and $\sigma^2_t = 0$, while a completely unknown interference means no knowledge of these quantities. Importantly, our estimation method is designed to handle the cases that we have \emph{no} knowledge of these underlying quantities for implementation.
% 
\begin{example}
    Consider the case where $\mu^{ij}$ and $\mu^{ij}_t$ take values of $0$ and $1$, generating time-dependent adjacency matrices of the graph representing the interference structure. If we have high confidence that interactions occur only through these adjacency matrices, we can imagine negligible values for $\sigma^2$ and $\sigma^2_t$. Conversely, significant uncertainty about potential interactions would be reflected in larger values of $\sigma^2$ and $\sigma^2_t$. %Importantly, our estimation method requires \emph{no} knowledge of these underlying quantities for implementation.
\end{example}
% 
\begin{remark}
    The formal version of Assumption~\ref{asmp:Stable Interference Pattern} appears in Assumption~\ref{asmp:weak_limits} in Appendix~\ref{apndx:batch_state_evolution}, where we generalize the condition by allowing greater variation across units. This interference model accommodates complex interaction patterns and admits further generalizations, including extensions to non-Gaussian interference matrices. For detailed discussions, see the Appendix of \cite{shirani2024causal}.
\end{remark}
% 

We then establish the following informal result for the large-sample regime that characterizes how outcome distributions evolve between consecutive time points, with its formal statement presented in Appendix~\ref{apndx:batch_state_evolution}.
% 
\begin{theorem}[State evolution - informal statement]
    \label{thm:SE_informal}
    % 
    Let $N \to \infty$ and suppose Assumptions~\ref{asmp:Gaussian Interference Matrice} and \ref{asmp:Stable Interference Pattern} hold. There exist mappings $\outcomef{t},\; t=0, \ldots, T-1$, such that the distribution of outcomes at time $t+1$ is determined by:
    % 
    \begin{equation}
        \label{eq:SE_informal}
        \law(
        \outcomeD{}{}{t+1})
        =
        \outcomef{t}\left(
        \law(
        \outcomeD{}{}{t}
        ),
        \law(
        \treatment{}{t+1}{}
        )
        \right),
    \end{equation}
    %
    where $\outcomeD{}{}{t}$ denotes the weak limit of unit outcomes $\outcomeD{}{1}{t}(\Mtreatment{t}{}), \ldots, \outcomeD{}{N}{t}(\Mtreatment{t}{})$ at time $t$, $\treatment{}{t+1}{}$ follows a Bernoulli distribution with mean $\expr_{t+1}$, and $\law(\cdot)$ refers to law or distribution of its argument.
\end{theorem}
% 
The mapping $\outcomef{t}$ in Eq. \eqref{eq:SE_informal} characterizes how the distribution of outcomes at time t+1 emerges from the previous distribution at time $t$ through the functions $\outcomeg{t}{}$ and $\outcomeh{t}{}$, while accounting for both direct treatment effects and indirect effects from the interference matrices and unit covariates. Specifically, in the large-sample limit, even though each unit's outcome depends on a complex network of interactions, the population-level distribution of outcomes follows a simpler evolution that depends only on the previous distribution and treatment assignment distribution. Building on \citep{shirani2024causal}, while extending their results to a broader setting, we refer to the relationship in \eqref{eq:SE_informal} as the experimental state evolution (SE) equation.


\subsection{A First Look at Estimation Through Theory and Practical Constraints}
\label{sec:estimation_theory_informal}


The state evolution equation~\eqref{eq:SE_informal} suggests a natural estimation strategy, motivated by the observation that in many settings, $\outcomef{t}$ remains relatively stable over time. Specifically, we can often decompose $\outcomef{t}$ into a substantial time-invariant component $\outcomef{}$ and a time-varying component. 

This decomposition, combined with the fact that empirical distributions from experimental data with many units should approximate the distributions in equation~\eqref{eq:SE_informal}, enables a supervised learning approach to estimate $\outcomef{}$. When strong time trends are present, one can first detrend the data to isolate the time-invariant component before applying these learning techniques. Once these mappings are estimated, we can generate any desired counterfactual evolution $\left\{ \CFE{\OMtreatment{}{}'}{}{t} \right\}_{t=0}^T$ through recursive application of Eq.~\eqref{eq:SE_informal} (see Algorithm~\ref{alg:CF estimation} in Appendix~\ref{sec:estimation_theory}). For the sake of building intuition, we will explain a simple version of this, adapted from \cite{shirani2024causal}.

\subsubsection{An Illustrative Example} 
Consider a special case of outcome specification \eqref{eq:outcome_function_matrix} where functions $\outcomeh{t}{}$ are equal to zero, and functions $\outcomeg{t}{}$ satisfy,
%
\begin{equation}
\label{eq:simple_function_structure}
    \begin{aligned}
        \outcomeg{t}{} \big(
        \outcomeD{}{i}{t}(\Mtreatment{t}{}),
        \treatment{i}{t+1}, \Vcovar{i}
        \big)
        = \ABE
        +
        \ACE \outcomeD{}{i}{t}(\Mtreatment{t}{})
        +
        \ADE \treatment{i}{t+1}
        +
        \APE \outcomeD{}{i}{t}(\Mtreatment{t}{}) \treatment{i}{t+1}\,.    
    \end{aligned}
\end{equation}
% 
In this case, if we denote the outcomes sample mean at time $t$ by $\AVO{}{}{t}$, the state evolution implies that as $N \rightarrow \infty$,
%
\begin{align}
    \label{eq:simple_SE_function}
    \AVO{}{}{t+1}
    =\outcomef{}(\AVO{}{}{t},\expr_{t+1})\,,
\end{align}
%
where $\outcomef{}$ has the form $\outcomef{}(a,b)=\ABE + \ACE a + \ADE b +
\APE ab$, and plase the role of the aforementioned time-invariant function. Subsequently, given observations $\Moutcome{}{}{}(\OMtreatment{}{})$ and $\OMtreatment{}{}$, Algorithm~\ref{alg:example_algorithm} enables consistent estimation of CFE for any target treatment allocation $\OMtreatment{}{}'$, provided that $\OMtreatment{}{}$ and $\OMtreatment{}{}'$ share identical columns for treatment assignment at time~$t=0$ and the design set $\{\expr_0, \ldots, \expr_T\}$ contains at least two distinct values\footnote{A detailed analysis of a more general setting appears in Appendix~\ref{sec:application_to_BRD}.}.

% 
\begin{algorithm}
\caption{Causal message passing counterfactual estimator (simple case)}
\label{alg:example_algorithm}
% 
\begin{algorithmic}
% 
% 
\Require $\Moutcome{}{}{}(\OMtreatment{}{}), \OMtreatment{}{} = [\Otreatment{i}{t}{}]_{i,t}$, and $\OMtreatment{}{}' = [\Otreatment{'i}{t}{}]_{i,t}$
% 

\State \hspace{-1.3em} \textbf{Step 1: Data processing}
% 
\For{$t = 0, \ldots, T$}
\State $\HAVO{}{}{t} \gets \frac{1}{N} \sum_{i=1}^N \outcomeD{}{i}{t} (\OMtreatment{t}{})$, \quad\quad $\Oexpr{}{}{t} \gets \frac{1}{N} \sum_{i=1}^N \Otreatment{i}{t}{}$, \quad\quad $\Dexpr{}{}{t} \gets \frac{1}{N} \sum_{i=1}^N \Otreatment{'i}{t}{}$
% 
\EndFor
% 

\State \hspace{-1.3em} \textbf{Step 2: Parameters estimation}
% 
\State $(\EABE, \EACE, \EADE, \EAPE)$ $\gets$ Estimation of $(\ABE, \ACE, \ADE, \APE)$ using OLS in $\HAVO{}{}{t+1} = \ABE + \ACE \HAVO{}{}{t} + \ADE \Oexpr{}{}{t+1} + \APE \HAVO{}{}{t} \Oexpr{}{}{t+1}$


\State \hspace{-1.3em} \textbf{Step 3: Counterfactual estimation}

\State $\ECF{}{0}{\OMtreatment{}{}'} \gets \HAVO{}{}{0}$

\For{$t = 1, \ldots, T$}
%
    \State $\ECF{}{t}{\OMtreatment{}{}'} \gets \EABE + \EACE \ECF{}{t-1}{\OMtreatment{}{}'} + \EADE \Dexpr{}{}{t} + \EAPE \ECF{}{t-1}{\OMtreatment{}{}'} \Dexpr{}{}{t}$
% 
\EndFor

\Ensure Estimated counterfactual evolution: $\{\ECF{}{t}{\OMtreatment{}{}'}\}_{t=0}^T$.
% 
\end{algorithmic}
\end{algorithm}
% 

\subsubsection{General Estimation Theory} 

The above example demonstrates that our counterfactual estimation problem reduces to the consistent estimation of $\outcomef{}$. This relationship can be formalized for our general outcome specification~\eqref{eq:outcome_function_matrix}. We provide an informal version of this result below, while the formal statement and complete proof appear in Theorem~\ref{thm:consistency} in Appendix~\ref{sec:estimation_theory}.

\begin{theorem}[Consistency - informal statement]
\label{thm:consistency_informal}
Under certain regularity conditions, given consistent estimation of the mappings $\outcomef{t}$ in the SE equation~\eqref{eq:SE_informal}, any desired counterfactual evolution 
$\left\{ \CFE{\OMtreatment{}{}'}{}{t} \right\}_{t=0}^T$ 
can be consistently estimated.
\end{theorem}

\subsubsection{Practical limitations} 

The effectiveness of approaches like Algorithm~\ref{alg:example_algorithm} in identifying treatment effects across various settings has been established by \cite{shirani2024causal}. However, two significant constraints require consideration. First, the estimation procedure in the second step of Algorithm~\ref{alg:example_algorithm} faces sample size limitations due to the experiment horizon. Second, as demonstrated by \cite{bayati2024higher}, certain settings require accounting for more complex structures, necessitating a richer class of functions $\outcomef{}$. To capture these more complex settings, one approach is to use higher moments of the outcome and treatment distributions in state evolution~\eqref{eq:SE_informal}.

For example, if we denote the variance of outcomes at time $t$ by $\VVO{}{}{t}$, and if functions $\outcomeg{}{}$ and $\outcomeh{}{}$ contain second-order terms, the state evolution becomes:
\begin{align}
    \label{eq:more_complex_SE_function}
    \Big(\AVO{}{}{t+1},(\VVO{}{}{t+1})^2\Big)
    =\outcomef{}\Big(\AVO{}{}{t},(\VVO{}{}{t})^2,\expr_{t+1},\expr_{t+1}^2\Big)\,.
\end{align}
%
An alternative approach to capturing complex structure is to consider a richer family of functions $\outcomef{}$ during the estimation phase. 

These considerations raise two key questions: First, how can we expand the sample size to improve estimation accuracy? Second, how can we determine the appropriate specification for the underlying experiment, similar to model selection in supervised learning? The following two sections address these questions systematically.
\section{Distribution-preserving Network Bootstrap}
\label{sec:DPNB}
% 
To enable the use of more powerful supervised learning techniques, we introduce a theoretically supported resampling methodology that generates additional samples from the state evolution equation.

The methodology centers on batching experimental units into subpopulations, denoted by $\batch$. Specifically, $\batch$ represents a subset of units from $\{1, \ldots, N\}$ where membership is determined exclusively by the treatment allocation $\Mtreatment{}{}$. Then, under the randomized treatment assumption—which specifies that $\Mtreatment{}{}$ is random and independent of all other variables—each $\batch$ constitutes a random sample from the experimental population. In this section, we demonstrate that each distinct selection of $\batch$ yields a new observation of the state evolution equation, provided the empirical treatment distributions exhibit sufficient variation across subpopulations.

Our analysis proceeds in two stages. First, we extend the state evolution concept to derive subpopulation-specific state evolutions. We establish that our batching technique generates statistically efficient samples of the experimental state evolution while preserving its fundamental structure—hence the term \batching{} (\batchingAcronym{}). Second, we derive a decomposition rule that characterizes the finite-sample behavior of potential outcomes. This decomposition enables the identification of endogenous noise within the experimental data and elucidates how \batchingAcronym{} addresses this challenge.

\begin{remark}
The theoretical frameworks presented here extend, with minimal or no modification, to arbitrary subpopulations. While our analysis focuses specifically on treatment-allocation-based subpopulations in this section, the appendices provide complete theoretical statements applicable to general subpopulation structures.
\end{remark}


\subsection{Subpopulation-specific State Evolution}
\label{sec:BSE}
%
We begin by extending the state evolution analysis to subpopulations of experimental units.
% 
\begin{theorem}[informal statement]
    \label{thm:BSE_informal}
Under the conditions of Theorem~\ref{thm:SE_informal}, consider a subpopulation $\batch$ whose size $\cardinality{\batch}$ grows to infinity as $N \to \infty$. There exist mappings $\outcomefb{t},\; t=0, \ldots, T-1$, such that:
    % 
    \begin{equation}
        \label{eq:BSE_informal}
        \law(
        \outcomeD{}{\batch}{t+1})
        =
        \outcomefb{t}
        \left(
        \law(
        \outcomeD{}{}{t}
        ),
        \law(
        \treatment{}{t+1}{}
        ),
        \law(
        \outcomeD{}{\batch}{t}
        ),
        \law(
        \treatment{\batch}{t+1}{}
        )
        \right),
    \end{equation}
    % 
    where $\outcomeD{}{\batch}{t}$ and $\treatment{\batch}{t+1}{}$ represent the weak limits of outcomes and treatments for units in subpopulation~$\batch$, while $\outcomeD{}{}{t}$ and $\treatment{}{t+1}{}$ are their corresponding population-level analogs as defined in Theorem~\ref{thm:SE_informal}.
\end{theorem}
% 
According to Equation \eqref{eq:BSE_informal}, each subpopulation's outcome distribution follows a distinct state evolution equation governed by mappings $\outcomefb{t},\; t=0, \ldots, T-1$. The outcomes for units within $\batch$ evolve through a dynamic interplay between population-level and subpopulation-specific outcomes and treatments. The formal statement of Theorem~\ref{thm:BSE_informal} appears in Theorem~\ref{thm:Batch_SE} of Appendix~\ref{apndx:batch_state_evolution}, where \eqref{eq:state evolution} explicitly defines $\outcomefb{t}$ through mappings $\outcomeg{t}{}$ and $\outcomeh{t}{}$.

To demonstrate the efficacy of \batchingAcronym{}, consider two subpopulations $\batch_1$ and $\batch_2$ such that $\treatment{\batch_1}{t+1}{} \neq \treatment{\batch_2}{t+1}{}$ in \eqref{eq:BSE_informal}. Then, two sequences $\outcomeD{}{\batch_1}{0}, \ldots, \outcomeD{}{\batch_1}{T}$ and $\outcomeD{}{\batch_2}{0}, \ldots, \outcomeD{}{\batch_2}{T}$ yield distinct observations of the state evolution described by \eqref{eq:BSE_informal}. Furthermore, for each time step $t$, the mapping $\outcomefb{t}$ exhibits fundamental similarities to $\outcomef{t}$, with the experimental state evolution equation in \eqref{eq:SE_informal} emerging as a special case of \eqref{eq:BSE_informal} when applied to the entire population. Accordingly, by strategically selecting subpopulations with distinct treatment allocations, we obtain multiple state evolution observations, enabling the estimation of $\outcomefb{t}$ and, consequently, $\outcomef{t}$.
% 
\begin{example}
    \label{exmpl:BernoulliD_subpopulation}
    Consider two distinct subpopulations: treated units ($\batch=\Tc$) and control units ($\batch=\Cc$). From \eqref{eq:BSE_informal}, we have:
    % 
    \begin{equation}
    \label{eq:BD_SP_SEs}
        \begin{aligned}
        \law(
        \outcomeD{}{\Tc}{t+1})
        =
        \outcomefb{t}\left(
        \law(
        \outcomeD{}{}{t}
        ),
        \law(
        \treatment{}{t+1}{}
        ),
        \law(
        \outcomeD{}{\Tc}{t}
        ),
        1
        \right),
        \\
        \law(
        \outcomeD{}{\Cc}{t+1})
        =
        \outcomefb{t}\left(
        \law(
        \outcomeD{}{}{t}
        ),
        \law(
        \treatment{}{t+1}{}
        ),
        \law(
        \outcomeD{}{\Cc}{t}
        ),
        0
        \right).
        \end{aligned}
    \end{equation}
    % 
    When the direct treatment effect is non-zero, the outcome distributions for the treated group $\outcomeD{}{\Tc}{t}$ and the control group $\outcomeD{}{\Cc}{t}$ are distinct, providing two different observations from the state evolution.
\end{example}
%


\subsection{Outcomes Decomposition Rule}
\label{sec:decomposition_rule}
% 
Building on the finite-sample analysis of AMP algorithms \citep{li2022non}, we obtain the following decomposition rule for the outcome specification in Eq.~\eqref{eq:outcome_function_matrix}.
% 
\begin{theorem}[Unit-level decomposition rule]
    \label{thm:outcome_decomposition}
    Suppose Assumption~\ref{asmp:Gaussian Interference Matrice} holds and let $\Vec{Z}_0, \Vec{Z}_1, \ldots, \Vec{Z}_{T-1}$ denote i.i.d. random vectors in $\R^N$ following $\Nc(0,\frac{1}{N}\I_N)$ distribution. For any unit $i$ and fixed $t \in \{0, \ldots, T-1\}$, we have
    % 
    \begin{equation}
        \label{eq:outcome_decomposition_rule}
        \begin{aligned}
        \outcomeD{}{i}{t+1}(\Mtreatment{t+1}{}) =
        \;&\frac{1}{N} \sum_{j=1}^N \left(\mu^{ij}+\mu_t^{ij}\right) \outcomeg{t}{}\left(\outcomeD{}{j}{t}(\Mtreatment{t}{}), \treatment{j}{t+1}{}, \Vcovar{j} \right) +
        \outcomeh{t}{}\left(\outcomeD{}{i}{t}(\Mtreatment{t}{}), \treatment{i}{t+1}{}, \Vcovar{i} \right)
        \\
        &+
        \sqrt{\sigma^2+\sigma_t^2} \norm{\outcomeg{t}{}\left(\VoutcomeD{}{}{t}(\Mtreatment{t}{}) ,\Vtreatment{}{t+1}{}, \covar\right)} \sumvec{i}{t}  + \noise{i}{t},
        \end{aligned}
    \end{equation}
    % 
    where $\sumvec{i}{t}$ represents a random variable such that $\Sumvec{}{t} := \left(\sumvec{1}{t}, \ldots, \sumvec{N}{t}\right)^\top = \sum_{i=0}^{t} \NPC{i}{t} \Vec{Z}_i$, with $\VNPC{}{t} = (\NPC{0}{t}, \ldots, \NPC{t}{t}, 0, \ldots, 0)^\top \in \R^N$ denoting a random vector that is correlated with $\Vec{Z}_0, \Vec{Z}_1, \ldots, \Vec{Z}_{t}$ and satisfies $\normWO{\VNPC{}{t}} = 1$. Furthermore,
    % 
    \begin{equation*}
        \Wc_1\left(\law\left(\Sumvec{}{t}\right),\Nc\left(0,\frac{1}{N} \I\right)\right) \leq c \sqrt{\frac{t \log N}{N}},
    \end{equation*}
    % 
    where $c$ is a constant independent of $N$ and $t$, $\law(\Sumvec{}{t})$ denotes the probability distribution of~$\Sumvec{}{t}$, and $\Wc_1$ is Wasserstein-1 distance.
\end{theorem}
% 
Equation~\eqref{eq:outcome_decomposition_rule} provides a decomposition of unit outcomes: the first term captures the interference effect, the second term reflects the unit-specific effect, the third term arises from uncertainties in the network structure, and the fourth term accounts for observation noise. We can then adjust the first term in Eq.~\eqref{eq:outcome_decomposition_rule} to reflect the available knowledge about the interference network.
% 
\begin{example}
    When it is known that no interference exists, Eq.~\eqref{eq:outcome_decomposition_rule} simplifies to $\outcomeD{}{i}{t+1}(\Mtreatment{t+1}{}) = \outcomeh{t}{}\big(\outcomeD{}{i}{t}(\Mtreatment{t}{}), \treatment{i}{t+1}{}, \Vcovar{i} \big) + \noise{i}{t}$, representing a general specification for the potential outcomes.
\end{example}
% 
% 
\begin{example}
    Consider a scenario where the interference network is partially known and exhibits a clustered structure with clusters $C^1, \ldots, C^K$ (e.g., the experimental population consists of individuals from different cities). In this context, Eq.~\eqref{eq:outcome_decomposition_rule} can be rewritten as follows:
    % 
    \begin{equation*}
        \begin{aligned}
        \outcomeD{}{i}{t+1}(\Mtreatment{t+1}{}) =
        \;&\frac{1}{N} \sum_{k=1}^K 
        \Bigg[
        \sum_{j \in C^k, i \notin C^k} \left(\mu^{ij}+\mu_t^{ij}\right) \outcomeg{t}{}\left(\outcomeD{}{j}{t}(\Mtreatment{t}{}), \treatment{j}{t+1}{}, \Vcovar{j} \right)
        \\
        \;&+
        \sum_{i,j \in C^k} \left(\mu^{ij}+\mu_t^{ij}\right) \outcomeg{t}{}\left(\outcomeD{}{j}{t}(\Mtreatment{t}{}), \treatment{j}{t+1}{}, \Vcovar{j} \right)
        \Bigg]
        \\
        \;&+
        \outcomeh{t}{}\left(\outcomeD{}{i}{t}(\Mtreatment{t}{}), \treatment{i}{t+1}{}, \Vcovar{i} \right)
        +
        \sqrt{\sigma^2+\sigma_t^2} \norm{\outcomeg{t}{}\left(\VoutcomeD{}{}{t}(\Mtreatment{t}{}) ,\Vtreatment{}{t+1}{}, \covar\right)} \sumvec{i}{t}  + \noise{i}{t},
        \end{aligned}
    \end{equation*}
    % 
    where we expect the magnitude of $\left(\mu^{ij} + \mu_t^{ij}\right)$ to be relatively larger whenever $i, j \in C^k$, indicating that units within the same cluster exhibit stronger ties.
\end{example}
% 


% 
The outcome decomposition rule in Eq.~\eqref{eq:outcome_decomposition_rule} leads to the following corollary, which characterizes the decomposition of the outcome sample mean for units belonging to subpopulation $\batch$.
% 
\begin{corollary}[Subpopulation-level decomposition rule]
    \label{crl:SampleMean_decomposition}
    Under the assumptions of Theorem~\ref{thm:outcome_decomposition}, we have
    %
    \begin{equation}
        \label{eq:SampleMean_decomposition}
        \begin{aligned}
        \frac{1}{\cardinality{\batch}} \sum_{i \in \batch} \outcomeD{}{i}{t+1}(\Mtreatment{t+1}{})
        =
        \;&\frac{1}{N \cardinality{\batch}}
        \sum_{i \in \batch}
        \sum_{j=1}^N \left(\mu^{ij}+\mu_t^{ij}\right) \outcomeg{t}{}\left(\outcomeD{}{j}{t}(\Mtreatment{t}{}), \treatment{j}{t+1}{}, \Vcovar{j} \right)
        \\
        \;&+
        \frac{1}{\cardinality{\batch}} \sum_{i \in \batch}
        \outcomeh{t}{}\left(\outcomeD{}{i}{t}(\Mtreatment{t}{}), \treatment{i}{t+1}{}, \Vcovar{i} \right)
        \\
        \;&+
        \sqrt{ \frac{\sigma^2+\sigma_t^2}{{\cardinality{\batch}}}} \norm{\outcomeg{t}{}(\VoutcomeD{}{}{t}(\Mtreatment{t}{}) ,\Vtreatment{}{t+1}{}, \covar)} \avesumvec{}{t}
        +
        \frac{1}{\cardinality{\batch}}
        \sum_{i \in \batch} \noise{i}{t}.
        \end{aligned}
    \end{equation}
% 
    where $\cardinality{\batch}$ denotes the size of the subpopulation $\batch$ and $\avesumvec{}{t}$ is a random variable satisfying
% 
    \begin{equation*}
        \Wc_1\left(\law\left(\avesumvec{}{t}\right),\Nc\left(0,\frac{1}{N} \right)\right) \leq c \sqrt{\frac{t \log N}{N}}.
    \end{equation*}
\end{corollary}
% 
While Theorem~\ref{thm:BSE_informal} establishes that distinct subpopulations yield different state evolution observations, Corollary~\ref{crl:SampleMean_decomposition} reveals two crucial implications for finite-sample analysis. First, the outcome distribution within each subpopulation incorporates unit-level heterogeneities, as demonstrated by the first two terms in Eq.\eqref{eq:SampleMean_decomposition}. Second, the outcome uncertainty terms—represented by the final two expressions in Eq.\eqref{eq:SampleMean_decomposition}—decay at a rate of $\sqrt{\cardinality{\batch}}$. The subsequent sections examine these implications in detail, showing how \batchingAcronym{} addresses such observational complexities while generating additional samples.


\subsection{Unit-level Heterogeneities}
% 
Experimental units exhibit distinct covariates, such as age and gender. Network interference also emerges through diverse local patterns, jointly inducing heterogeneous characteristics. These heterogeneities are captured in \eqref{eq:SampleMean_decomposition} through variations in $\mu^{ij}$, $\mu_t^{ij}$, and $\Vcovar{i}$. This foundation motivates our strategy of selecting members for each subpopulation $\batch$ solely based on treatment allocation. Specifically, this approach enables us to treat $\batch$ as a random sample, supporting the assumption that each subpopulation represents the original experimental population. Indeed, Theorem~\ref{thm:BSE_informal} demonstrates the effectiveness of this batching strategy, evidenced by the invariance of $\outcomefb{t}$ across subpopulations. Meanwhile, the result of Corollary~\ref{crl:SampleMean_decomposition} suggests maximizing subpopulation sizes to leverage stronger smoothing effects through averaging over larger groups of units.


\subsection{Endogenous Observation Noise}
% 
The noise terms in the outcome decomposition rule exhibit complex correlation structures. According to Theorem~\ref{thm:outcome_decomposition}, the random vectors $\Sumvec{}{0}, \ldots, \Sumvec{}{T-1}$ in Eq.~\eqref{eq:outcome_decomposition_rule} asymptotically follow a centered Gaussian distribution as population size $N$ increases, thus functioning as noise terms. These vectors, however, demonstrate a complex endogenous dependency structure through several mechanisms.

Initially, for each $t$, the random vector $\Sumvec{}{t}$ derives its randomness from $\Vec{Z}_0, \Vec{Z}_1, \ldots, \Vec{Z}_{t}$. Consequently, the random vectors $\Sumvec{}{0}, \ldots, \Sumvec{}{T-1}$ exhibit correlation, generating temporal correlation in the observational noise.

Additionally, in finite populations where $N$ is bounded, the elements of $\sumvec{i}{t}$ demonstrate cross-unit correlation. The magnitude of this correlation is governed by the constant $c$ in Theorem~\ref{thm:outcome_decomposition}, which varies across settings and remains unidentified.

Furthermore, the potential outcome of unit $i$ at time $t$ denoted by $\outcomeD{}{i}{t}(\Mtreatment{t}{})$ incorporates $\sumvec{i}{t-1}$, which correlates with elements of $\Sumvec{}{t}$; this introduces correlation between noise terms and unit outcomes. The strength of this correlation depends on the magnitudes of $\sigma$ and $\sigma_t$, which remain unknown and can vary substantially across different settings. 

These three factors collectively define a complex endogenous dependency structure, indicating how uncertainty in the interference structure can compromise outcome measurements and introduce substantial bias when overlooked. The following example examines an autoregressive model with unanticipated unit interactions, demonstrating how even minimal second-order interference induces endogeneity and generates biased estimates.
% 
\begin{example}
    \label{exmp:bias_of_correlation}
    Consider a simplified version of the outcome specification in Eq.~\eqref{eq:outcome_function_matrix} where $\outcomeg{t}{}(\outcomeD{}{i}{t}(\Mtreatment{t}{}), \treatment{i}{t+1}{}, \Vcovar{i}) = \outcomeD{}{i}{t}(\Mtreatment{t}{})$ and $\outcomeh{t}{}(\outcomeD{}{i}{t}(\Mtreatment{t}{}), \treatment{i}{t+1}{}, \Vcovar{i}) = \CE^i \outcomeD{}{i}{t}(\Mtreatment{t}{})$, with $\mu^{ij} + \mu_t^{ij} = 0$ for all $i$, $j$, and $t$. These conditions reflect the available partial information confirming negligible interference between units. Also, assume $\noise{i}{t}$ is independent from all other variables. We can write:
    % 
    \begin{equation*}
        \begin{aligned}
        \outcomeD{}{i}{t+1}(\OMtreatment{t+1}{}) =
        \CE^i \outcomeD{}{i}{t}(\OMtreatment{t}{})
        +
        \sqrt{\sigma^2+\sigma_t^2} \norm{\VoutcomeD{}{}{t}(\OMtreatment{t}{})} \sumvec{i}{t}  + \noise{i}{t}.
        \end{aligned}
    \end{equation*}
    % 
    To estimate $\CE^i$ for a specific unit $i$, we employ ordinary least squares regression, regressing outcomes $\outcomeD{}{i}{1}(\OMtreatment{1}{}), \ldots, \outcomeD{}{i}{T}(\OMtreatment{T}{})$ on their corresponding lagged values $\outcomeD{}{i}{0}(\OMtreatment{0}{}), \ldots, \outcomeD{}{i}{T-1}(\OMtreatment{T-1}{})$. Under the asymptotic regime where $N \rightarrow \infty$, we characterize the bias of estimator $\hat{\Theta}^i$ as follows:
    % 
    \begin{align*}
        \E\left[
        \hat{\Theta}^i
        \Big|
        \Moutcome{}{}{}(\OMtreatment{}{})
        \right]
        -
        \CE^i
        &=
        \frac{1}{\sum_{t=0}^{T-1} \outcomeD{}{i}{t}(\OMtreatment{t}{})^2}
        \sum_{t=0}^{T-1}
        \sqrt{\sigma^2+\sigma_t^2} 
        \norm{\VoutcomeD{}{}{t}(\OMtreatment{t}{})}
        \outcomeD{}{i}{t}(\OMtreatment{t}{})
        \E\left[
        \sumvec{i}{t}
        \big|
        \Moutcome{}{}{}(\OMtreatment{}{})
        \right].
    \end{align*}
    % 
\end{example}
% 

Based on the results of Corollary~\ref{crl:SampleMean_decomposition}, the magnitude of this complex noise diminishes as the subpopulation size $\cardinality{\batch}$ increases, supporting the selection of larger subpopulations. Therefore, in the extreme case, when $\batch$ encompasses the entire population ($\cardinality{\batch} = N$), the noise magnitude decays at a rate of $\sqrt{N}$.


\subsection{How to use \batchingAcronym{} to generate efficient samples?}
\label{sec:batching}
%
Both unit-level heterogeneities and endogenous observation noise suggest maximizing subpopulation sizes, which would ideally lead to selecting the entire experimental population. 
This approach aligns with the proposal in Section \ref{sec:estimation_theory_informal}, where we observe a single instance of state evolution. 
Therefore, this presents a nuanced trade-off: larger subpopulations increase overlap between groups, reducing variation across subpopulations and thereby decreasing the number of effective samples. Given that the magnitude of endogenous noise varies across different settings, we aim to develop a data-driven methodology to address this trade-off optimally. This leads us to propose a cross-validation method, which we examine in detail in the subsequent section.


\section{Counterfactual Cross Validation}
\label{sec:C-CV}
%
The precision of estimated counterfactuals, as established in Theorem~\ref{thm:consistency_informal}, depends on accurately estimating the state evolution mappings $\outcomef{t}$. This estimation process faces two main challenges that require careful validation:

\subsubsection*{Function approximation and model selection} 

The true specification of potential outcomes is often unknown, requiring us to approximate the state evolution mappings $\outcomef{t}$. This approximation involves two related aspects. First, we must choose appropriate summary statistics of the joint distribution of outcomes and treatments to serve as inputs to our model - an approach analogous to feature engineering in supervised learning. For example, \cite{shirani2024causal} used sample means of outcomes, treatments, and their products. While domain knowledge can guide this selection, we need a systematic way to validate these choices.
Second, depending on the complexity of experimental settings, we may need to employ a range of estimation techniques, from simple linear regression to more sophisticated methods such as neural networks. The choice of technique and its specific implementation (e.g., architecture, hyperparameters) must be validated to prevent issues like model instability and estimation bias. These two aspects are linked, as both contribute to how well we can approximate the true mapping $\outcomef{t}$. A data-driven validation methodology helps us jointly optimize these choices while reducing misspecification risks.


\subsubsection*{Optimal batch configuration}
% 
As established in \S\ref{sec:batching}, \batchingAcronym{} offers an effective strategy for addressing unit-level heterogeneities and controlling endogenous noise through batching. However, the choice of batch size presents a bias-variance trade-off. Larger batches better average out heterogeneities and noise, reducing estimation bias, but they also decrease the number of distinct batches available for learning, increasing estimation variance. Conversely, smaller batches provide more samples for learning, reducing variance, but may retain more heterogeneity and noise, potentially introducing bias.
Beyond batch size, the number of batches creates a similar trade-off that interacts with the batch size selection. Therefore, data-driven validation helps find the optimal combination of batch size and number of batches that balances these competing effects in a given experimental context.




\subsection{Counterfactual Cross Validation Algorithm}
\label{sec:CCV_Algorithm}
% 
To address these challenges, we propose a counterfactual cross-validation algorithm that divides the time horizon into a list of time blocks $\tblockList$, which serve as natural cross-validation folds. For each time block, we use all other blocks as training data and the held-out block as validation data. The training process involves generating batches as discussed above, with their size and number serving as tuning parameters. For validation, we use a fixed set of pre-specified validation batches $\batch_1^v, \ldots, \batch_{b_v}^v$ constructed as follows. We begin by computing each unit's average treatment exposure across the experimental horizon (e.g., a unit that is treated in 60\% of the time periods receives a treatment exposure of 0.6). We then rank the units in descending order based on their treatment exposure values and partition them into $b_v$ equal-sized groups, ensuring validation batches cover the full spectrum of treatment exposure. For each fold, the validation metric is averaged across these validation batches to provide a robust performance measure.

The parameters being selected through this cross-validation include both the choice of estimator from a candidate set $\estimatorList$ and the batching configuration from a candidate set $\batchList$. Each estimator $\estimator \in \estimatorList$ represents a specific combination of feature generation (what summary statistics to use) and estimation method (e.g., linear regression or neural network). The batching parameters $(\batchSize,\batchCount)\in\batchList$ specify the size $\batchSize$ and number $\batchCount$ of training batches. For each held-out time block, we train models using all possible combinations of these parameters on the remaining blocks and evaluate their performance on the validation batches obtained from the held-out block. 

\subsubsection*{Step 1: Reference counterfactual construction}
% 
After the experiment is completed, we obtain the following data. The treatment matrix $\OMtreatment{}{}$ and observed outcomes matrix, $\Moutcome{}{}{}(\OMtreatment{}{})$. 
We first compute the sample mean of outcomes over time as ground truth (counterfactual) for evaluating estimations in subsequent steps. More precisely, for each time period $t\in\{0,1,\ldots,T\}$, and each validation batch $\batch_{j}^v$ with $j\in \{1,\ldots,b_v\}$, we calculate average of outcomes at time $t$ across units in $\batch_{j}^v$ and denote that by $\CFETest{t}{\batch_{j}^v}$.

\subsubsection*{Step 2: Leave-one-out and counterfactual estimation}
% 
For each time block $\tblock\in\tblockList$, we construct training datasets $\outcomeTrain^{-\tblock}$ and $\treatmentTrain^{-\tblock}$ as submatrices of $\Moutcome{}{}{}(\OMtreatment{}{})$ and $\OMtreatment{}{}$, respectively, excluding columns within $\tblock$. We define $\treatmentTest^{\tblock}$ as the submatrix of $\OMtreatment{}{}$ containing only columns in $\tblock$. Each estimator $\estimator$ is then trained using $\outcomeTrain^{-\tblock}$ and $\treatmentTrain^{-\tblock}$, and used to estimate counterfactuals for treatment allocation $\treatmentTest^{\tblock}$ during the held-out time block $\tblock$.

\subsubsection*{Step 3: Optimal Estimator Selection}
% 
Following the estimation across all time blocks, we identify the optimal estimator and batch parameters by comparing the results with observed counterfactuals using a predetermined loss function. For instance, using mean square error:
% 
\begin{align*}
    \MSE_{\estimator,\batchSize,\batchCount}
    =
    \frac{1}{b_v (T+1)}
    \sum_{j=1}^{b_v}
    \sum_{t=0}^T
    \Big[\CFETest{t}{\batch_{j}^v} - 
    \ECFTrain{t}{\batch_{j}^v}{\estimator,\batchSize,\batchCount}
    \Big]^2.
\end{align*}
%

\begin{algorithm}
\caption{Counterfactual cross-validation}\label{alg:C-CV}
\begin{algorithmic}
% 
\Require Treatment allocation $\OMtreatment{}{}$, observed outcomes $\Moutcome{}{}{}(\OMtreatment{}{})$, validation batches $\{\batch_j^v\}_{j=1}^{b_v}$, time blocks $\tblockList$, loss function $\criteria:\R^{(T+1) b_v}\times\R^{(T+1)\times b_v}\to\R_+$, and candidate estimators $\estimatorList$ and batch parameters $\batchList$
%%%
\State \hspace{-1.3em} \textbf{Step 1: Reference Counterfactual Construction}
% 
\State $\big\{ \CFETest{t}{\batch_{j}^v} \big\}_{t = 0}^T \gets$ mean of $\Moutcome{}{}{}(\OMtreatment{}{})$ for units belonging to $\batch_{j}^v,\; j=1, \ldots, b_v$
%%%
\State \hspace{-1.3em} \textbf{Step 2: Leave-one-out and Counterfactual Estimation}
% 
\For{$\estimator,\batchSize,\batchCount \in \estimatorList\times\batchList$}
    % 
    % 
    \For{$\tblock \in \tblockList$}
        % 
        \State $\outcomeTrain^{-\tblock} \gets$ columns of $\Moutcome{}{}{}(\OMtreatment{}{})$ outside of $\tblock$
        % 
        \State $\treatmentTrain^{-\tblock} \gets$ columns of $\OMtreatment{}{}$ outside of $\tblock$
        % 
        \State $\treatmentTest^{\tblock} \gets$ columns of $\OMtreatment{}{}$ in $\tblock$
        %
        \State Train $\estimator$ with batching parameters  $(\batchSize,\batchCount)$ on data $\outcomeTrain^{-\tblock}$ and $\treatmentTrain^{-\tblock}$ 
        %%%
        \For{$j \in \{1,\ldots,b_v\}$}
        %
            \State $\big\{ \ECFTrain{t}{\batch_{j}^v}{\estimator,\batchSize,\batchCount} \big\}_{t \in \tblock} \gets$ estimate CFE for $\treatmentTest^{\tblock}$ via the trained model, for units in $\batch_j^v$
        \EndFor 
    \EndFor
    % 
\EndFor
%%%
\State \hspace{-1.3em} \textbf{Step 3: Optimal Estimator Selection}
% 
\State $\estimator^{*},\batchSize^{*},\batchCount^{*} \gets \arg\min_{(\estimator,\batchSize,\batchCount)\in\estimatorList\times\batchList} \criteria \left(\Big\{\big\{ \CFETest{t}{\batch_{j}^v} \big\}_{t = 0}^T\Big\}_{j=1}^{b_v}, \Big\{\big\{ \ECFTrain{t}{\batch_{j}^v}{\estimator,\batchSize,\batchCount} \big\}_{t = 0}^T\Big\}_{j=1}^{b_v}\right)$
% 
\end{algorithmic}
\end{algorithm}

\begin{remark}
    \label{rem:estimators}
    Algorithms~\ref{alg:FO-recursive} and~\ref{alg:FO_with_preprocessing}, presented in Appendices~\ref{sec:estimation_theory} and~\ref{sec:preprocessing} respectively, provide examples of candidate estimators based on linear regression.
\end{remark}

\begin{remark}
    \label{rem:proprocessing}
    In experimental settings with strong temporal patterns such as seasonality, we first \emph{\preprocess} the observed outcomes, e.g., see Algorithm~\ref{alg:FO_with_preprocessing} in Appendix~\ref{sec:preprocessing}. This procedure augments Step 2 of Algorithm~\ref{alg:C-CV} as follows. For each estimator $\estimator$, we utilize the complete observed dataset to estimate a baseline counterfactual—defined as the counterfactual outcome for all units under control—using an estimation model optimized for temporal pattern detection. We then generate filtered data by subtracting this baseline counterfactual from observed outcomes. This filtered data serves as input for the main estimator, which focuses specifically on treatment effect identification. The final estimates are obtained by combining the baseline counterfactual with the estimated treatment effects. The consistency of this estimation approach relies on a structural assumption of weak additive separability between baseline outcomes and treatment effects.
\end{remark}
\section{Benchmark Toolbox}
\label{sec:Benchmark_Toolbox}
% 
We evaluate our framework using six semi-synthetic experiments that combine simulated environments with real-world data. This approach offers two key advantages: it maintains realistic data characteristics while allowing us to compute ground truth values for our estimands. Unlike real experimental settings where outcomes are observed under a single scenario, these settings provide ground truth values for any desired scenario. This enables rigorous evaluation of estimation methods under realistic conditions. In the following sections, we specify the treatment allocation, outcomes, and network structure for each experimental setting.


\subsection{LLM-based Social Network}
\label{sec:LLM}
% 
This environment simulates a social media platform like Facebook and Quora where user interactions occur through content feeds, designed to study the effects of feed ranking algorithms on user engagement. The environment employs Large Language Model (LLM) agents to represent users, with demographically realistic personas derived from US Census data \citep{uscensus2023} following \cite{chang2024llms}'s methodology. Each agent possesses two interests selected from a predefined set of \emph{keywords}: Technology, Sports, Politics, Entertainment, Science, Health, and Fashion.

The treatment variable is the \emph{feed ranking algorithm}, which determines content ordering for each user. In the control condition, users receive randomly ordered content, while the treatment condition presents content weighted by friend engagement. The outcomes of interest are \emph{user engagement metrics}, measured as the sum of likes or replies generated by each user within a given time period, tracked in a comprehensive panel dataset.

The underlying directed follower-following relationships network is constructed based on a preferential attachment model \citep{barabasi1999emergence}. The dynamics of the environment center on content generation and user interactions. Content originates from an LLM-generated bank focused on a specified topic, with cross-keyword variations (e.g., climate change intersecting with technology or politics). The system generates feeds by combining interest-based content (matching user interests), trending content (high engagement), and random content according to specified proportions. Users interact with their feeds through an LLM-driven decision process that considers content relevance, friend engagement, feed position, and demographic characteristics.

The simulation maintains comprehensive state information across multiple dimensions: engagement metrics, user interaction histories, network relationships, content visibility, and conversation threads. At each time step, agents have a 1\% probability of generating new content, contributing to the platform's organic content evolution. The interaction process follows a structured decision framework, where each agent evaluates content through a detailed prompt incorporating the agent's persona attributes, content characteristics, social signals (including friend engagement), and feed positioning. This framework ensures that user behaviors and social influence patterns evolve naturally through the network while maintaining computational tractability.


\subsection{Belief Adoption Model}
\label{sec:Belief_Adoption}
%
This environment models the diffusion of competing opinions within interconnected communities, implementing the cascade model of \cite{montanari2010spread}. The system examines how opinions spread through social networks when individuals make decisions through coordination games with their neighbors. Through this framework, we evaluate the effectiveness of promotional campaigns in influencing opinion adoption patterns.

The environment considers two competing stances: Opinion~$\OPA$ (e.g., voting in an election) and Opinion~$\OPB$ (e.g., declining to vote). The treatment represents a \emph{campaign aimed at increasing Opinion~$\OPA$ adoption}. The outcome for each unit in each period is binary: \emph{1 if they adopt Opinion~$\OPA$, and 0 otherwise}.

The opinion evolution follows a network-based coordination game where each individual $i$ assigns payoff values $\payoff^i_\OPA$ and $\payoff^i_\OPB$ to both opinions. The probability of Opinion~$\OPA$ adoption in the next period depends on the neighbor configuration and relative payoffs:
% 
\begin{align*}
    \P\big(\text{adopting Opinion~$\OPA$} \big| \neighbor^i_\OPA,\neighbor^i_\OPB \big) =
    \frac{e^{\beta \left( \neighbor^i h^i + \neighbor^i_\OPA - \neighbor^i_\OPB  \right)}}{e^{\beta \left(\neighbor^i h^i + \neighbor^i_\OPA - \neighbor^i_\OPB  \right)} + e^{-\beta  \left(\neighbor^i h^i + \neighbor^i_\OPA - \neighbor^i_\OPB  \right)}}\,,
\end{align*}
% 
where $h^i = \frac{\payoff^i_\OPA - \payoff^i_\OPB}{\payoff^i_\OPA + \payoff^i_\OPB}$ and $\neighbor^i_\OPA$ represents the number of neighbors holding Opinion~$\OPA$ out of $\neighbor^i$ total neighbors in the current period, and $\beta$ is a predetermined constant. The underlying network in our simulator is derived from the \emph{Pokec social network} dataset \citep{takac2012data,snapnets}, focusing on three regional networks: Krupina (3,366 users), Topolcany (18,246 users), and Zilina (42,971 users).

The environment also utilizes detailed profile data from the Pokec social network dataset to characterize each user. We extract three demographic variables from these profiles: age, profile activity, and gender. For base payoffs, we assign higher values for Opinion~$\OPA$ to users aged 25-55 years and those maintaining active profiles. The treatment effectiveness follows a Gaussian distribution, reaching its maximum at age 35. Its impact scales directly with the user's profile completion percentage, as measured in their Pokec data. These profile-driven calculations create distinct patterns of payoffs and treatment responses across the network, as visualized in Figures~\ref{fig:BAM-krupina}-\ref{fig:BAM-zilina}.

\begin{figure}
    \centering
    \includegraphics[width=\linewidth]{plots/krupina.pdf}
    \caption{Distribution of base payoffs, treatment effects, and node degrees for individuals in Krupina.}
    \label{fig:BAM-krupina}
\end{figure}

\begin{figure}
    \centering
    \includegraphics[width=\linewidth]{plots/topolcany.pdf}
    \caption{Distribution of base payoffs, treatment effects, and node degrees for individuals in Topolcany.}
    \label{fig:BAM-topolcany}
\end{figure}

\begin{figure}
    \centering
    \includegraphics[width=\linewidth]{plots/zilina.pdf}
    \caption{Distribution of base payoffs, treatment effects, and node degrees for individuals in Zilina.}
    \label{fig:BAM-zilina}
\end{figure}


\subsection{Ascending Auction Model}
\label{sec:Auction}
% 
This environment simulates a competitive market where multiple bidders participate in an ascending auction for objects, following the model of \cite{bertsekas1990auction}. The auction mechanism creates a dynamic pricing system where bidder interactions generate complex patterns of market influence, even without direct object-to-object relationships.

The system operates with $N$ objects and $N$ bidders. Each object represents an experimental unit, with its \emph{final value} in each round serving as the outcome variable. The treatment consists of \emph{promotional interventions that increase bidder valuations} by $\tau$\% for randomly selected objects. This treatment affects all bidders interested in the selected objects.

The market evolution follows a structured bidding process. In each round, bidders evaluate objects based on their private valuations and current market prices. They submit bids for their preferred objects, with objects being assigned to the highest bidders. These assignments establish new price levels, which influence subsequent bidding behavior. As prices increase through competitive bidding, objects become progressively less attractive to competing bidders.

The environment demonstrates a unique form of interference: \emph{while objects do not directly influence each other, treatment effects propagate through the market via bidders' strategic responses to price changes.} This creates a network of indirect treatment effects, as promotional interventions for certain objects can influence market outcomes for others through shifts in bidder behavior and price dynamics.


\subsection{New York City Taxi Routes}
\label{sec:LiM}
% 
This environment models ride-sharing dynamics across New York City taxi zones using real-world TLC Trip Record Data \citep{nyc_tlc_trip_data}. The framework adapts the established linear-in-mean outcome model \citep{eckles2016design,cai2015social,leung2022causal} to represent how passengers utilize ride-sharing services throughout the city. By incorporating actual travel data, passenger density metrics, and inter-zone relationships, the simulation effectively captures the complex network of interactions between taxi routes across the city.

In this setting, the experimental units are defined as routes (origin-destination pairs) between the city's 263 taxi zones, with time segmented into 6-hour periods. The outcome variable measures the \emph{number of trips} along each route during each time period. The treatment represents a program implemented on randomly selected routes and the goal is to evaluate travelers response patterns.

Given baseline outcomes $[\OoutcomeD{}{i}{t}]_{i,t}$, the system's evolution follows Equation \eqref{eq:linear-in-mean}, where outcomes depend on baseline patterns, network effects, and treatment status:
% 
\begin{equation}
\label{eq:linear-in-mean}
\begin{aligned}
    \outcomeD{}{i}{t+1}
    =
    \OoutcomeD{}{i}{t+1}
    +
    \ACE
    \sum_{j=1}^N \adjMe^{ij}
    (\outcomeD{}{j}{t} - \OoutcomeD{}{i}{t})
    +
    \ADE_{\pl} \sum_{j=1}^N \adjMe^{ij}\treatment{j}{t+1}
    +
    \ADE^i_{\ul} \treatment{i}{t+1},
    \quad
    t \geq 1,
\end{aligned}
\end{equation}
% 
where we initiate the recursion by setting $\outcomeD{}{i}{0} = \OoutcomeD{}{i}{0}$ and $\treatment{i}{0} = 0$ for all $i$.
Here, $\adjM = [\adjMe^{ij}]_{i,j}$ represents the normalized route adjacency matrix, $\ADE_{\ul}^i$ represents route-specific direct treatment effects, and parameters $(\ACE,\ADE_{\pl}) = (0.4,0.2)$ control autocorrelation and spillover effects.
% 
\begin{figure}
    \centering
    \includegraphics[width=\linewidth]{plots/NYC_Data.pdf}
    \caption{Mean and variance of the number of trips in each route during each period, revealing a strong daily and weekly seasonality pattern.}
    \label{fig:NYC_taxi_panel}
\end{figure}
% 

The environment incorporates real-world data through three components: First, it uses ``High Volume For-Hire Vehicle Trip Records" (January-March 2024, 57,974,677 trips) to construct baseline outcomes (denoted by $[\OoutcomeD{}{i}{t}]_{i,t}$), focusing on 18,768 active routes across 366 periods. As shown in Figure~\ref{fig:NYC_taxi_panel}, the trips have a strong seasonality pattern, which is common in the ride-hailing application \citep{xiong2024data}. Second, it employs the ``Taxi Zone Lookup Table" and Claude (Model 3.5 Sonnet, Anthropic, 2024) to generate passenger density scores that determine route-specific treatment effects (Figure~\ref{fig:NYC_analysis} left). Third, it creates a route adjacency network based on geographic proximity, transit connections, shared roads, and functional relationships, yielding an average node degree of 8.32 (Figure~\ref{fig:NYC_analysis} right). This data-driven approach ensures the simulation reproduces key real-world characteristics: temporal patterns, heterogeneous treatment effects, and localized network interactions.
% 
\begin{figure}[ht]
    \centering
    \includegraphics[width=0.45\linewidth]{plots/NYC_Taxi_DTE.pdf}
    \includegraphics[width=0.45\linewidth]{plots/NYC_node_degrees_histogram.pdf}
    \caption{The left panel displays the distribution of route-specific direct treatment effects, illustrating the heterogeneity in treatment responses. The right panel shows the histogram of node degrees in the route interference network.}
    \label{fig:NYC_analysis}
\end{figure}


\subsection{Exercise Encouragement Program}
\label{sec:BOM}
% 
This environment simulates an exercise intervention program that combines individual characteristics from the 1994 Census Bureau database \citep{kohavi1994data} with social network effects. Drawing inspiration from mobile health intervention studies \citep{liao2016sample,klasnja2015microrandomized,klasnja2019efficacy}, the environment models how digital encouragement messages influence exercise decisions within a social network context.

The experimental units are individuals, with binary outcomes representing their \emph{exercise decisions in each period} (1 for exercise, 0 for no exercise). The treatment consists of \emph{digital intervention messages} designed to encourage physical activity. Inspired by \cite{li2022network}, we let the outcomes follow a Bernoulli distribution defined as follows:
% 
\begin{align}
    \label{eq:MRT}
    \outcomeD{}{i}{t+1} \sim \text{Bernoulli}\left(
    \frac{1}
    {
    1
    +
    \exp{-(
    \ABE_t^i
    +
    \ADE_t^i
    \treatment{i}{t+1}
    +
    \ACE
    \outcomeD{}{i}{t} Z^i_t
    +
    \APE \treatment{i}{t+1} \outcomeD{}{i}{t} Z^i_t
    )}}\right),
\end{align}
% 
where $Z^i_t = \sum_{j=1}^N \adjMe^{ij}\outcomeD{}{j}{t}$ represents the count of neighboring individuals who exercised in the previous period. Here, $\adjMe^{ij}$'s are the elements of the adjacency matrix from \emph{Twitter social circles} data \citep{leskovec2012learning}, with an average of 21.74 connections per individual (Figure~\ref{fig:twitter_network_hist}). 

In \eqref{eq:MRT}, each component captures specific aspects of exercise behavior. The baseline probability ($\ABE_t^i$) represents an individual's inherent tendency to exercise, derived from Census Bureau demographic data. This probability incorporates age (with higher values for younger individuals), working hours (showing an inverse relationship with exercise likelihood), and occupation type (assigning higher probabilities to active or professional occupations). These baseline probabilities exhibit weekly patterns, showing peak values on weekends due to increased free time, elevated rates on Mondays from new week motivation, stable mid-week patterns, and slightly lower values on Fridays reflecting end-of-week fatigue (Figure~\ref{fig:exercise_probability}).

The intervention effectiveness ($\ADE_t^i$) quantifies how individuals respond to exercise encouragement messages. This response varies based on multiple demographic factors from the Census Bureau database. Younger individuals show higher responsiveness to digital interventions, while education level correlates positively with intervention effectiveness. Job-related factors, including occupation type and working hours, influence response rates by indicating flexibility and availability to act on interventions. The impact of messages follows weekly cycles, demonstrating maximum effectiveness during weekends and Mondays, with gradually decreasing impact through mid-week (Figure~\ref{fig:message_impact}). The model also sets parameters $(\ACE, \APE) = (0.0.4,0.01)$ to capture peer influence and the interaction between treatment and peer effects.
% 
\begin{figure}
    \centering
    \includegraphics[width=1\linewidth]{plots/exercise_probability.pdf}
    \caption{Distribution of baseline exercise probabilities across the population.}
    \label{fig:exercise_probability}
\end{figure}
% 
% 
\begin{figure}
    \centering
    \includegraphics[width=1\linewidth]{plots/message_impact.pdf}
    \caption{Distribution of intervention message effects across the population}
    \label{fig:message_impact}
\end{figure}
% 

% 
\begin{figure}
    \centering
    \includegraphics[width=0.9\linewidth]{plots/twitter_network_hist.pdf}
    \caption{Distribution of node degrees in the Twitter social network.}
    \label{fig:twitter_network_hist}
\end{figure}
% 


\subsection{Data Center Server Utilization}
\label{sec:servers}
% 
This environment simulates a server farm, and the goal is to evaluate interventions for improving server utilization. Given the increasing demand for cloud computing resources, optimizing data center utilization has become critical for addressing global sustainability concerns \citep{zhang2023global,saxena2023sustainable}. Within this system, servers influence each other's performance through the system's physical characteristics, particularly via the join-the-shortest-queue routing policy \citep{gupta2007analysis}. This operational dynamic creates an implicit interference pattern in the absence of a pre-specified network structure.

The experimental units are individual servers within a parallel processing system of N servers. The outcome variable $\outcomeD{}{i}{t}$ represents \emph{server i's utilization} during the interval [t,t+1), measured as the proportion of time the server remains busy. The treatment consists of interventions that \emph{enhance the processing power} of selected servers.

The system evolution follows a structured routing mechanism. When tasks arrive, the system identifies capable servers for each task type and selects a random sample among them. Following the join-the-shortest-queue policy, tasks are assigned to servers with minimal queue lengths within this sample, using random assignment to resolve ties. This \emph{routing approach naturally creates interference effects}, as performance improvements in treated servers influence task distribution across the entire system.

The environment incorporates realistic workload patterns through a time-dependent Poisson arrival process. The demand model captures multiple temporal patterns: daily variations (night-time lows, morning increases, midday peaks, and evening declines), weekly cycles (heightened weekday activity), and stochastic elements (random fluctuations and event-driven spikes). Each server processes tasks with exponentially distributed service times, which can be modified by interventions. Through this design, the system replicates key characteristics of real-world data centers while enabling controlled experimentation.
% 
\begin{figure}
    \centering
    \includegraphics[width=0.9\linewidth]{plots/Data_center_time_trend.pdf}
    \caption{Average demand for the data center over time shows a strong seasonality.}
    \label{fig:Data_center_time_trend}
\end{figure}
%
\section{Results}
We identified key contexts of parental involvement, perceptions of AI-generated content, preferences for AI-assisted content creation, and collaborative patterns in shared interactions with a robot. We present the findings based on our research questions as follows.

%We identified four key areas that inform the design of an AI-assisted educational robot to support parental involvement in young children's learning activities. We present our results based on the study phases as follows: (1) \textit{Phase 1: Parent Contextual Needs and Scenarios}: understanding the real-life contexts and challenges parents face; (2) \textit{Phase 2.1: Parent Perspectives on AI-Generated Learning Content}: capturing parents' attitudes and concerns regarding the use of AI in generating educational content; (3) \textit{Phase 2.2: Parent Use of the LLM-Assisted Content Supervision Mechanism}: examining how parents review, edit, and supervise AI-generated content; and (4) \textit{Parent Use of the Robot Involvement Adjustment Mechanism}: exploring how parents delegate roles between themselves and the robot during learning activities.

\subsection{(RQ1) What contexts do parents encounter when involving in young children's learning activities?}

Each participant provided examples for the eight scenarios representing unique combinations of the three two-dimensional factors using the \texttt{SET-scenario cards}. We present P2's scenarios as an example in Table~\ref{tab:scneario-example}. The full set of scenario examples from all participants is documented as supplementary materials.\footnote{SET Scenarios: \url{https://osf.io/zfksg/?view_only=b59bd41287f543ce82ab85950aaf004f}} Beyond capturing the contexts shared by each parent, we analyzed these examples to identify and summarize key contextual patterns for the three factors of parental involvement: \textit{skills}, \textit{energy}, and \textit{time}.

\subsubsection{\textbf{Skill:} Parents face challenges in pedagogical skills, particularly with advanced or unfamiliar concepts.}
Parents mentioned several \textit{skills} in supporting their child's (1) intellectual, (2) pedagogical, and (3) social-emotional development, highlighting key challenges across these areas. While some parents (7/20) reported low confidence in \textit{intellectual} activities, especially advanced STEM topics (P1, P7, P12, P14, P17–19), many (16/20) felt confident in literacy (\textit{e.g.,} reading, spelling; P1–6, P8, P9, P11, P13, P14, P19, P20) and basic STEM (P6, P8–10, P12–15). Confidence often stemmed from personal expertise or interests, consistent with the Hoover-Dempsey and Sandler (HDS) framework \cite{green2007parents}. For example, P15, a physicist, felt confident teaching physics-related activities. The majority of parents (14/20) struggled with \textit{pedagogical} skills, such as explaining concepts (P7, P8, P13, P17, P18), answering or formulating questions (P3, P4, P6, P7, P9, P13), identifying developmental benchmarks (P4, P6, P10, P11), and allowing their child to learn from mistakes (P2, P12, P19). A smaller group of parents (7/20) expressed confidence in these areas, particularly explaining concepts (P14, P18, P19) and answering questions (P4, P5, P10, P14). \textit{Social-emotional} skills presented additional challenges. Some parents (6/20) struggled with teaching emotion regulation (P2, P17), behavioral management (P5, P15, P20), and interpersonal conflict resolution (P3, P15). Others (5/20) lacked confidence in encouraging participation in learning activities (P5, P11) or maintaining patience during learning support (P4, P10, P18, P19). Conversely, several parents (10/20) felt confident teaching emotion regulation (P1, P3, P10–13, P16, P18–20) and norms of polite communication (P7, P12, P16, P20).

\subsubsection{\textbf{Energy:} Parents' motivation depends on their physical and emotional status as well as the child's willingness to learn.}
%\paragraph{Parents' motivation depends on their physical and emotional status, time, and the child's willingness to learn.}
Parents suggested that their motivation to facilitate learning activities was affected by (1) physical status, (2) emotional status, and (3) time. Commenting on their \textit{physical status}, most parents (16/20) indicated low motivation when they need rest due to feeling ``\textit{hungry},'' ``\textit{sick},'' or ``\textit{tired}'' (P1--11, P14--17, P19), and many parents (9/20) reported being highly motivated when they are ``\textit{well rested}'' or after having ``\textit{a really good meal}'' (P1--4, P6, P7, P9, P11, P16). Regarding \textit{emotional state}, many parents (9/20) lacked motivation when they needed a mental break or ``\textit{me time}'' if they felt emotionally exhausted (P3, P7, P11--14, P17--19) or after spending time with their child (P5, P8, P20). In addition, some parents (7/20) lost motivation if their child appeared to be disinterested (P11, P12, P14) or poorly behaved (P4, P15, P17, P20). In contrast, many parents (16/20) were motivated when their child needed support (P12, P15), expressed interest and invited the parent to participate (P1, P4, P7, P8, P10, P12–14, P16–19), or is well behaved and ready to learn (P2, P5, P13, P18, P19, P20). Some parents (6/20) were highly motivated when they wanted to connect with their child (P3, P14, P15) or when they were personally interested in the activity (P5, P10, P12, P14).


\subsubsection{\textbf{Time:} Parents' availability depended on work, chores, other family members.}
Parents discussed (1) work and commitment, (2) household chores, and (3) family needs as factors that determined whether they had time, \textit{i.e.,} availability and presence, to facilitate learning activities. Most parents (19/20) were not available when they needed to be at \textit{work} (P1--4, P6, P7, P10--14, P16--19) and had other personal or professional engagements (P4, P5, P9, P11, P15, p20). Many parents (17/20) stated that \textit{household chores}, such as laundry, meal preparation, and cleaning, also determined their availability to be with their child (P1--6, P9--17, P20). Although some parents involved their child in chores (P1, P2, P7, P10, P15, P17), not all chores were seen as being appropriate or safe for children. Parents' availability also depended on the ability of other family members to provide support (P2--4, P6--8, P16--20), \textit{e.g.,} when a spouse helped with chores or an older child watches a younger sibling. Parents had less time if other family members needed them (P3, P7, P8, P12, P13, P16--18, P20), \textit{e.g.,} when a younger child is crying or a family member is sick. Finally, parents described their availability using specific time frames, \textit{e.g.,} ``\textit{weekday mornings} (P5, P8, P19),'' ``\textit{weekdays after dinner and before bedtime} (P5, P10, P18),'' ``\textit{anytime on weekends} (P1, P3, P4, P6, P8, P10, P16),'' or ``\textit{unstructured time} (P2, P11, P14, P16, P17, P20).'' They often structured their time and consider themselves available when they are physically present with their child (P1, P3--7, P14--16), such as during grocery shopping, car rides, or trips to the park together.





\subsection{(RQ2) How do parents perceive AI-generated content for young children?}\label{sec-result-2}

Parents showed mixed attitudes toward AI-generated learning content for young children. They discussed their perceived benefits and risks and envisioned ways to mitigate their concerns.

\subsubsection{Mixed Attitude towards AI-generated content}
Parents expressed a range of attitudes towards allowing AI to generate content for young children, ranging from skepticism and concern (P2--6, P10, P13--15) to open-minded caution (P9, P8, P16, P19, P20), acceptance (P6, P7, P11, P17, P18) and, in some cases, neutral (P1, P12). Parents who were \textit{\textbf{skeptical and concerned}} questioned whether AI-generated content met quality and safety standards, \textit{e.g.,} P3 questioned, ``\textit{Who's generating the content? Where is it getting the content from? Is it good? Is it safe?}'' On the other hand, parents who hold an attitude of \textit{\textbf{open-minded caution}} recognize the risks of using AI-generated content but feel open to use it under specific conditions. P16 highlighted model training, stating, ``\textit{I wouldn't be against it if the people training it were proficient in what the AI is teaching.}'' Similarly, P20 emphasized personal oversight, explaining, ``\textit{I can do my own evaluation to determine whether or not I think the content is good regardless of who it came from.}'' Furthermore, parents who have an attitude of \textit{\textbf{acceptance}} assume people who created the system have already ensure the appropriateness for children, \textit{e.g.,} P7 stated, ``\textit{I'm assuming because it's AI, there would be more research behind it.  So I would be okay with it.}'' Finally, parents who hold a \textit{\textbf{neutral}} attitude typically don't have much experience with AI and therefore feel unsure about their attitude for AI-generated content, \textit{e.g.,} P1 had ``\textit{not even thought about it until before this study.}''

\subsubsection{Perceived Benefits and Risks}
Parents identified several benefits of AI-generated content for young children. Some parents (P2, P4, P8, P10, P11, P16) highlighted AI's potential in \textit{\textbf{adaptability}} to adjust learning content to their child's evolving developmental needs, \textit{e.g.,} P11 expected AI to help ``\textit{adjust content as the child grows.}'' In addition, parents (P2, P3, P4, P16, P19) discussed \textit{\textbf{customization}}, illustrating that ``\textit{one of the big benefits would be to create material that are related to his[child's] interests and things that would be motivating to him[child]} (P4).'' Parents (P6, P7, P8, P10, P12, P17) also emphasized \textit{\textbf{efficiency}} of AI, explaining ``\textit{because it[AI] can access a huge amount of information very fast} (P12),'' enabling a ``\textit{quicker way to learn or to see something} (P7).'' Moreoever, a few parents (P1, P11, P18, P20) noted AI's potential to foster \textit{\textbf{affordability}}, suggesting that AI-generated content could enhance the scalability and accessibility of learning resources, making ``\textit{more learning materials available, more variety available} (P1),'' and making things ``\textit{cheaper and more accessible for people} (P20).'' Finally, a few parents (P14, P15) expected easier \textit{\textbf{pedagogical integration}} with AI, enabling parents to ``\textit{teach children things that sometimes parents don't know because not all parents know everything} (P14).''

Meanwhile, parents described their perceived risks of AI-generated content for children. Most parents (P1--3, P5, P11--15, P17, P19) were concerned about \textit{\textbf{age-inappropriateness}} of the content, which could be ``\textit{violent and don’t match family values} (P1),'' ``\textit{physically harmful and sexually inappropriate} (P3),'' and ``\textit{stuff about body image and certain people being better than other people} (P5).'' In addition many parents (P2, P3, P9, P14, P16, P17, P18) expressed concerns about the \textit{\textbf{inaccuracy}} of the information presented through AI-generated content, worrying that AI could provide ``\textit{factually inaccurate}'' learning materials or content that might imply theories that are ``\textit{misframed or misconstructed} (P2).'' Moreover, parents (P2--4, P14, P15) raised concerns about the \textit{\textbf{training data quality}} for AI models. P2 emphasized transparency stating, ``\textit{I'd want to know a lot more about where that training data came from or who supervised that learning process}.'' A few parents (P6, P7) expressed concerns about children's \textit{\textbf{over dependence}} on AI instead of developing their own cognitive abilities. P6 worried that constant use of AI could discourage critical thinking, stating, ``\textit{if they have a question, instead of thinking through the question, they just ask AI, not using their own brain}.'' Finally, two parents shared concerns over \textit{\textbf{message dilution}}, where AI oversimplifies complex ideas and diminishes their original intent. P15 worried that AI might dilute sociopolitical issues, such as racial diversity and gender identity. Similarly, P20 emphasized concern about whether the core message being conveyed to the child aligns with parental values, stating ``\textit{I'm more concerned about the message the book is trying to impart on the child}.

\subsubsection{Envisioned Risk Mitigation Methods}
Parents described what methods they envisioned to address their concerns. First, some parents (P3, P5, P11, P12, P14, P16) stressed the need to enable \textit{\textbf{parental review and verification}}. For example, P5 stated ``\textit{I would read it to make sure that it was actually something I wanted to read with her}.'' In addition, a few parents (P2, P17, P19) expressed that \textit{\textbf{social and public validation}} could also enhance their trust in AI-generated content, \textit{e.g.,} P2 described that ``\textit{if a thousand people used it...and endorsed this model, that would give me more confidence in it}.'' Moreover, some parents (P2, P9, P15) discussed \textit{\textbf{model and data transparency}}, emphasizing the need to understand how AI models are trained. As explained by P9, ``\textit{being able to know exactly what's going on or how it works...would make me feel more secure about what my child is learning}.'' Lastly, a few parents (P1, P15, P18) highlighted the importance of \textbf{expert involvement} in creating AI-generated content. For instance, P1 emphasized the need for oversight by ``\textit{people with a background in human development.}''

\subsection{(RQ3) How would parents prefer to collaborate with LLM on supervising content creation under different contexts?}

\begin{figure*}[b]
\includegraphics[width=\textwidth]{figures/figure-result-03-hho.pdf}
   \vspace{-6pt}
  \caption{Summary of parent's use of LLM-assisted content supervision mechanism: (1) content evaluation criteria, (2) use pattern, (3) perceived value.}
  \label{fig:result-03}
   \vspace{-6pt}
\end{figure*}

We found three main themes for parent-AI collaboration on content creation using the \textit{editor interface}: (1) \textit{Content evaluation and criteria}, referring to what parents pay attention to when reviewing and revising LLM-generated content. (2) \textit{Contextual usage patterns}, describing how parents envision using the LLM-powered interface in various contexts. (3) \textit{Perceived value and benefits}, covering what values parents believe LLM brings.

\subsubsection{Theme 1: Content evaluation and criteria}
Parents focus on balancing \textit{difficulty} and \textit{variety} of concepts as well as ensuring the \textit{quality} of questions when reviewing, regenerating, and revising LLM-generated content for young children.

\textit{\textbf{Parents aim to give their children the right level of challenge while reinforcing skills they can confidently accomplish.}} Many (8/20) avoided overly easy questions to prevent boredom but strategically included them at the start or after difficult questions to build confidence. As P2 explained, ``\textit{I want her to get the answers and then have it get increasingly difficult as she goes so she doesn't get discouraged at the beginning}.'' Meanwhile, most parents (11/20) valued challenges that stretch their child's abilities without overwhelming them. P12 concerned that ``\textit{underestimating her would be damaging for her},'' while P20 expressed interest in seeing how his child would handle harder concepts, saying, ``\textit{I'm actually really interested to see if she can answer.}'' Finally, parents (7/20) were also cautious of content that might be too advanced, \textit{e.g.,}``\textit{she[child] doesn't know uppercase or lowercase yet, so that doesn't mean anything to her} (P20).'' Additionally, \textit{\textbf{parents aim to maintain engagement by introducing diverse concepts and question types throughout the activity.}} Many parents (11/20) expressed concerns over repetitive content and preferred diverse topics to challenge their child differently. For instance, P20 changed the concept of a question to ``addition'', explaining, ``\textit{I just made the last question a `how many,' so this one I want a different concept}.'' Finally, \textit{\textbf{parents evaluate the quality of LLM-generated learning content based on standards} such as question clarity and coherence (9/20), wording precision (6/20), visual clarity (5/20), and cognitive load (P12, P20).} P12 raised issues with wording, stating, ``\textit{I don't think she's going to fully understand front legs versus back legs when it's a front view},'' while P6 expressed concerns about visual clarity: ``\textit{from the pictures, you can't really tell how many bugs with black bodies are flying in the air}.'' P20 also reflected on cognitive load, saying, ``\textit{I think it's just too long, too much information for her to process}.'' 

\subsubsection{Theme 2: Contextual usage patterns}
We discussed parents' preferences and behaviors when collaborating with LLM under two main contexts: (1) when parents have limited time or energy and (2) when they have sufficient time and energy.

\textit{When parents have limited time or energy}, most were still \textit{willing} to invest minimal effort (P4, P6--9, P11, P12, P15, P18), often opting to \textit{\textbf{skim through the LLM-generated content with minor self-editing}}. For example, P9 shared, ``\textit{I might skip quickly, skim through it, make sure there isn't anything that I feel is not appropriate}.'' This approach allows involvement with minimal time commitment. However, some parents (P4, P11, P15) emphasized that the \textit{\textbf{LLM output must be high-quality enough to require minimal editing}}, otherwise they may not use it at all. P11 explained, ``\textit{The more that stuff can be in really good shape before it gets to parents, the more we can minimize how much work we have to do ahead of time}.''

Some parents were \textit{unwilling} to invest effort when time or energy was limited. They preferred to either \textit{\textbf{reuse previously reviewed activities}} (P8, P11, P12) or directly \textit{\textbf{use LLM-generated content without review}} (P5, P6, P9, P10, P18, P20). As P8 explained, ``\textit{if I don't have time, I would have to be using something he's already done before, so I don't have to supervise it},'' while P10 noted, ``\textit{if the AI-generated questions were enough to keep him engaged, then it would be worth it}.'' A few parents (P1, P7) preferred to \textit{\textbf{avoid using the system entirely}}, as they feel uncomfortable leaving their child engaged with the content without supervision. As P1 explained, ``\textit{if I'm either physically or mentally not present. It's just not happening}.''

\textit{When parents have sufficient time and energy}, most of them (P6–P10, P12, P18, P20) choose to \textit{\textbf{review and edit the content in detail, even customizing questions}} to better supervise and personalize learning for their child. P9 shared, ``\textit{If I had more time and motivation, I would take the time to do it myself. I enjoy writing, so I'd probably spend time customizing the content}.'' Similarly, P12 noted, ``\textit{If I had all the time, I would go through and be picky with the wording and content of the questions}.''

In contrast, some parents (P1, P2, P7, P11, P12, P18) still prefer to \textit{\textbf{skim through the content with minor editing}}, as they found the detailed process too effortful even when time allowed, but they cannot fully trust LLM or themselves to come up with good questions. For example, P11 shared, ``\textit{I would probably scroll through and try to do as little editing as possible},'' while P7 expressed doubt, stating, ``\textit{I don't know that I would come up with better questions than this one from AI}.'' A few parents opted to \textit{\textbf{avoid using the system entirely}}, preferring to spend their time on other activities (P15) or relying on their ability to engage their child without the system (P4, P5). For instance, P15 shared, ``\textit{I would rather spend that time playing an imaginative game with her than spending time designing this,}'' P5 similarly expressed confidence, saying, ``\textit{I think I can and do ask her questions about stuff we read}.''

\subsubsection{Theme 3: Perceived value and benefits}
We found that parents perceive the value of the system to include not only \textit{content supervision}, but also \textit{content co-creation with LLM} and \textit{parent empowerment through pedagogical insights}.

First, and unsurprisingly, most parents suggested that the system allows them to \textit{\textbf{supervise the learning content generated by LLM}}. For example, P2, while feeling skeptical about trusting AI, noted, ``\textit{I don't know what AI model was used, still, I can confirm everything myself},'' reflecting the value parents place on maintaining oversight of the content presented to their children. Second, some parents (P6–10, P12, P18, P20) appreciated that the system allows them to \textit{\textbf{co-create personalized learning content with the LLM}} for their child without having to start from scratch. For example, P18 appreciated the ability to adapt the content to their child's needs, saying, ``\textit{tailoring it to her difficulty levels and seeing the ability to modify the content alleviates some concerns}.'' P10 highlighted how the LLM creates a draft to work from, stating, ``\textit{I do appreciate the concepts and the kinds of questions that it [LLM] provides, and how it has that template there}.'' This flexibility allowed parents to easily modify content while leveraging the assistance from LLM. Third, some parents found that the system \textit{\textbf{empowered parents with pedagogical insights}}. As many parents do not possess formal pedagogical knowledge--such as understanding how to effectively teach their child--they often struggle with determining what questions to ask or which concepts are age-appropriate. Since parents brought up the same value after interacting with the robot as well, we discuss this value more in-depth in Section \ref{sec-6.4.2}.

\subsection{(RQ4) How would parents prefer to collaborate with an AI-assisted robot to engage in learning activities with their children under different contexts?}

\begin{figure*}[b!]
\includegraphics[width=\textwidth]{figures/figure-result-04-hho.pdf}
   \vspace{-6pt}
  \caption{Summary of parent's use of robot involvement adjustment mechanisms: (1) usage pattern, (2) parenting education.}
  \label{fig:result-03}
   \vspace{-6pt}
\end{figure*}

We identified two major themes in the use of parent-robot collaboration mechanisms (\textit{i.e.,} \textit{mode-switching} and \textit{role-delegation}) within the \textit{activity interface}: (1) \textit{Contextual mode utilization}, referring to how parents adjust their involvement based on varying time and energy levels, and (2) \textit{Perceived educational impact on parenting}, highlighting how parents value the process for enhancing their skills and knowledge in parenting.

\subsubsection{Theme 1: Contextual mode utilization}

We discussed parents' preferences when collaborating with the AI-assisted robot across four contexts: (1) sufficient energy and time, (2) sufficient time but low energy, (3) sufficient energy but limited time, and (4) low energy and time. The impact of parental skill is discussed in specific cases.

(1) \textit{Parents have sufficient energy and time}: many parents (P2, P5–7, P9, P15, P18) preferred the \textit{\textbf{parent takeover mode}}, where they facilitate activities themselves while using LLM-generated content as a resource. For example, P18 shared, ``\textit{if I'm feeling motivated, I'd probably take over, but still look at some AI-generated questions to prompt me or remind me of things to ask or do with her}.'' Similarly, P15 noted, ``\textit{with full energy and time, I would use the parent-only mode because I want to interact with her and give her all my attention}.'' Parents valued the ability to take full control while using LLM-generated content for supplemental support when they have sufficient energy and time.

In addition, some parents (P2, P8–12, P15, P20) envisioned using \textit{\textbf{collaboration modes}}--where both the parent and the robot share responsibilities (\textit{i.e.,} parent-led or robot-led mode)--with the parents' \textit{skills} in specific areas relative to the robot playing a critical role in determining the pattern of role delegation. Parents often chose to involve the robot when they felt it could enhance their child's engagement especially in high-stakes tasks like quizzes. For example, P15 noted, ``\textit{I would use the robot for quizzes as a playful element to keep her engaged}.'' Similarly, P2 highlighted the objectivity of the robot in quizzing: ``\textit{I like the idea of reading her the book and then a neutral third party gets to test her on it}.'' On the other hand, parents took on specific roles when they believed their involvement would benefit their child more. P11 shared, ``\textit{I would let the robot read and ask questions but step in if he wasn't understanding or needed guidance},'' while P12 emphasized the emotional aspect of teaching: ``\textit{I can explain in a way that she understands, whereas the robot might come across as too harsh}.''

(2) \textit{Parents have sufficient time but lack energy}: some parents (P3, P7, P10--12, P15, P18, P20) opted for \textit{\textbf{collaboration modes}}, with their involvement influenced by their motivation levels and partially by their \textit{skill} relative to the robot. For example, P10 noted, ``\textit{when I'm not motivated, having the robot do the quiz takes some heat off me}.'' Similarly, P9 mentioned, ``\textit{I'd probably read the book, but have the robot do everything else}.'' Additionally, some parents (P1, P2, P4, P6–8, P15) chose to use \textit{\textbf{robot takeover mode}}--where the robot facilitates everything--while they remained nearby to supervise. For instance, P15 noted, ``\textit{I'd be around, but I wouldn't physically do much because I'm not feeling well}.'' Similarly, P2 noted, ``\textit{If I'm not motivated, I could see myself handing it all over to the robot}.''

(3) \textit{Parents have sufficient energy but lack time}: many parents (P2, P3, P5, P7--11, P15) opted for the \textit{\textbf{robot takeover mode}}--where the robot facilitates everything--while adjusting their usage based on \textit{how much they trust LLM}. Parents with higher trust allowed their child to use the content directly without review (P3, P5, P7, P9, P10). For example, P7 mentioned, ``\textit{If I'm trying to take a walk, I might do the robot takeover, then I can physically be gone}.'' In contrast, parents with less trust preferred to supervise while multitasking (P2, P7, P8), review content beforehand using the editor (P2, P8), or use the system only if the LLM model met high-quality standards (P11, P15). For example, P2 shared, ``\textit{If I'm not there, I wouldn't want them to do it, unless I had used the editor to review},'' while P7 described a multitasking scenario: ``\textit{I could be working from home while the robot takes over, and I'm nearby to supervise}.'' Moreover, a few parents (P2, P12) chose to \textit{\textbf{avoid using the system entirely}} due to their lack of trust in using LLM-generated content directly and insufficient time to review it, or because the system design did not support independent use for young children (P4, P12, P20). For instance, P2 noted, ``\textit{If I'm absent, I don't know if I'd want them to do any of this},'' while P12 stated, ``\textit{I know my daughter is sensitive, and if [the questions are too hard and] the robot keeps telling her she's wrong, she might take it personally and give up}.''

(4) \textit{Parents lack both energy and time}: Some parents chose \textit{\textbf{robot takeover mode}}, adjusting their usage based on their trust in LLM-generated content. Others \textit{\textbf{avoided using the system entirely}} due to low trust and insufficient time to review (P2, P12), or because the system design did not support independent use by young children (P4, P12, P20). Refer to the previous case—parents with sufficient energy but lacking time—as the usage patterns and contextual reasons are very similar.

\begin{figure*}[!t]
  \includegraphics[width=\textwidth]{figures/figure-quant-hho.pdf}
  \caption{Parent perception on child's math and literacy ability before and after the reading session. The result suggested that parents adjusted their perception after observing their child doing the activity and they tend to underestimate them, especially for advanced math concepts and phonological awareness concept for literacy. The horizontal lines represents significance from the Wilcoxon Signed-Ranked Test: $p < .01^{**}$, $p < .05^{*}$.}
  \label{fig:quant-result}
   \vspace*{-10pt}
\end{figure*}

\subsubsection{Theme 2: Perceived Educational Impact on Parenting} \label{sec-6.4.2}

Supported by mixed-method data, many parents thought \texttt{PAiREd} has value in parenting education, providing them with pedagogical strategies and giving them opportunities to observe their child's proficiency level systematically through observation. If the system provides a comprehensive framework and ample ideas, parents may not decrease their involvement in an activity just because they don't have the pedagogical skill; in fact, they may even increase their involvement. In addition, parents will be able to observe and adjust their understanding about what their child can do or cannot do, instead of under- or over- estimate their child's ability.

Several parents (P1, P4, P6, P7, P10--12, P18) appreciated that the system \textit{\textbf{offered ideas they might not have considered on their own}}, providing new topics to explore with their child. For instance, P1 emphasized, ``\textit{I hadn't even thought of all the different types of concepts},'' and P10 highlighted that the system ``\textit{gives more of a structure…even the drop down list of concepts is insightful, offering lenses I wouldn't normally consider when reading}.'' Others (P2, P6--9, P11, P12) valued that the LLM \textit{\textbf{generated example questions for each concept}}, allowing them to start with ready-made content without worrying whether their own questions reflected the intended learning goals. For example, P8 mentioned, ``\textit{What's nice about the AI-generated ones is that you can specifically choose a variety of concepts, whereas creating them on your own, you don't always know what the concepts are}.'' Additionally, some parents noted that the system allowed them to systematically select questions they were unsure their child could answer, which \textit{\textbf{provided a structured way to observe and assess their child's proficiency level}}. For example, P7 remarked, ``\textit{it gives her the opportunity to show me things she knows that I otherwise wouldn't have asked},'' while P8 shared, ``\textit{I was curious to see how he does if he doesn't know how to answer this, rather than just setting him up to succeed}.''

Before and after parent-child pairs engage in the activity, we asked parents to rate their perception on their child's math and literacy abilities. Our quantitative results suggest that parents adjusted their understanding of their child's proficiency in certain concepts after reading together. Specifically, \textit{\textbf{parents tended to underestimate what their child can or cannot do, especially with more advanced math concepts}} (Math-L3: $p < .01^{**}$, Math-L4: $p < .01^{**}$) and the phonological awareness concept in literacy ($p < .05^{*}$). Figure \ref{fig:quant-result} summarizes the significance results from the Wilcoxon Signed-Rank Test.

%A One-Way ANOVA revealed significant differences between levels (L1-L4), necessitating separate analyses. Subsequent Repeated Measures ANOVAs for each level showed significant improvement in Math L4 post-intervention. Post-hoc pairwise tests, using both parametric (Paired T-Tests) and non-parametric (Wilcoxon Signed-Rank) methods to ensure robustness, confirmed significance in Literacy L2, Math L3, and Math L4. This multi-layered approach—combining One-Way ANOVA, Repeated Measures ANOVA, and diverse post-hoc tests—ensured level independence, accounted for within-subject variability, and provided a comprehensive, robust understanding of level-specific improvements while minimising misinterpretation risks.





Software development is increasingly conceived as a collaboration activity between developers and AIs. Indeed, IDEs already implement features to enable interactive development, with AI suggesting implementations that are reused by developers.

Although multiple studies show this interaction can be successful, there is still limited understanding of how the models must be configured and used in the context of code generation tasks. This study addresses this gap, systematically investigating the impact of several key parameters, including the repeated submission of a prompt to accommodate for the non-deterministic nature of the models.

Our study reveals several key findings about the usage of ChatGPT. In particular, we discovered how creativity, although up to a limited extent, is useful to increase the range of methods whose code can be generated correctly. A major role is played by parameter top-p, which is commonly underrated, and instead has a major impact on the correctness of the results, with lower values producing better results. Finally, prompts should be submitted multiple times, with $5$ repetitions combined with a temperature of $1.2$ resulting in an effective configuration in our experiments.  

Future work concerns two main research directions. One is about replicating this experiment with other AI assistants, to validate our findings in multiple contexts. The second research direction concerns finding strategies to deal with the need to submit the same prompt multiple times to obtain a useful result, and thus developing approaches able to select or merge multiple responses automatically. 

\bibliography{mypaper}
\bibliographystyle{apalike}

\begin{APPENDICES}
\renewcommand{\theHsection}{A\arabic{section}}
\section*{Appendices Organization}
\label{appendices}
% 
These appendices provide detailed statements and proofs of theoretical results that support the main body of the paper.

The first appendix presents the technical results, beginning with essential notation and a reformulation of the outcome specification. We then establish the outcome decomposition rules through a sequence of proofs. This begins with Lemmas~\ref{lm:outcome_decomposition}-\ref{lm:apndx_outcome_decomposition}, culminating in the proof of Theorem~\ref{thm:outcome_decomposition} and Corollary~\ref{crl:SampleMean_decomposition}. The discussion then progresses to a rigorous statement of batch-level state evolution in \S\ref{apndx:batch_state_evolution}, which was informally introduced in Theorem~\ref{thm:BSE_informal}. This is followed by a brief overview of the conditioning technique in \S\ref{apndx:Conditioning_Technique}, which is essential for proving the state evolution equation, as detailed in \S\ref{apndx:big_lemma}.

Appendix~\ref{sec:estimation_theory} demonstrates how consistent estimation of state evolution parameters enables consistent estimation of desired counterfactuals. We present the necessary assumptions, provide and prove the main theoretical results, and detail the corresponding algorithms. We then analyze the application of these results to Bernoulli randomized designs, concluding in presenting two families of estimators in \S\ref{apndx:semi-recursive_estimators} and \S\ref{apndx:recursive_estimators}.

Building on these results, \S\ref{sec:preprocessing} presents an extension to the causal message-passing methodology that addresses strong time-trends, enabling general counterfactual estimation even in the presence of seasonality or temporal patterns. The appendices conclude with \S\ref{apndx:auxiliary_results}, which presents auxiliary theorems necessary for our main proofs.


\section{Technical Results}
\label{sec:Technical_Results}
% 
In this section, we delve into the analysis of the outcome specification in Eq.~\eqref{eq:outcome_function_matrix}. Our analysis draws upon various results from the literature on approximate message-passing algorithms \citep{donoho2009message,bayati2011dynamics, rush2018finite,li2022non}.

In the following, we first introduce several notations necessary for the rigorous presentation of our theoretical results. We then rewrite the potential outcome specification, facilitating the ensuing discussions. Next, we provide a rigorous proof for the outcome decomposition rule. This is rooted in the non-asymptotic results for the analysis of AMP algorithms \citep{li2022non} and can be seen as the finite sample analysis of the causal message-passing framework \citep{shirani2024causal}, within a broader class of outcome specifications.

\subsection{Notations}
\label{apndx:notations}
% 
For any vector $\Vec{v} \in \R^n$, we denote its Euclidean norm as $\norm{\Vec{v}}$. For a fixed $k \geq 1$, we define $\poly{k}$ as the class of functions $f: \R^n \rightarrow \R$ that are continuous and exhibit polynomial growth of order $k$. In other words, there exists a constant $c$ such that $|f(\Vec{v})| \leq c(1+\norm{\Vec{v}}^k)$. Moreover, we consider a probability space $(\Omega, \F, \P)$, where $\Omega$ represents the sample space, $\F$ is the sigma-algebra of events, and $\P$ is the probability measure. We denote the expectation with respect to $\P$ as $\E$. Additionally, for any other probability measure~$p$, we use $\E_p$ to denote the expectation with respect to $p$. 

For any set $S$, the indicator function $\1_S(\omega)$ evaluates to $1$ if $\omega$ belongs to $S$, and $0$ otherwise. We define $\R^{n\times m}$ as the set of matrices with $n$ rows and $m$ columns. Given a matrix $\bm{M}$, we denote its transpose as~$\bm{M}^\top$. Additionally, we represent a matrix of ones with dimensions $n\times m$ as $\ones{n\times m} \in \R^{n\times m}$. The symbol $\eqd$ is used to denote equality in distribution, while $\eqas$ is used for equalities that hold almost surely with respect to the reference probability measure $\P$.

\subsection{Preliminaries}
\label{apndx:Preliminaries}
% 
We initiate by rewriting the outcome model in Eq.~\eqref{eq:outcome_function_matrix}. Specifically, we consider the following more general specification:
% 
\begin{equation}
    \label{eq:apndx_outcome_function}
    \begin{aligned}
        \VoutcomeD{}{}{t+1}(\Mtreatment{}{}) =
        \Voutcome{}{}{t+1} +
        \IMatMean{t} \outcomeg{t}{}\left(\VoutcomeD{}{}{0}(\Mtreatment{}{}), \ldots, \VoutcomeD{}{}{t}(\Mtreatment{}{}), \Mtreatment{}{}, \covar\right) +
        \outcomeh{t}{}\left(\VoutcomeD{}{}{0}(\Mtreatment{}{}), \ldots, \VoutcomeD{}{}{t}(\Mtreatment{}{}), \Mtreatment{}{}, \covar \right)
        + \Vnoise{}{t},
    \end{aligned}
\end{equation}
% 
where
% 
\begin{align}
    \label{eq:apndx_outcome_function_WOD}
    \Voutcome{}{}{t+1} = \big( \IM + \IMatT{t} \big) \outcomeg{t}{}\left(\VoutcomeD{}{}{0}(\Mtreatment{}{}), \ldots, \VoutcomeD{}{}{t}(\Mtreatment{}{}), \Mtreatment{}{}, \covar\right).
\end{align}
% 
Above, $\IMatMean{t}$ is the matrix with the element $(\mu^{ij} + \mu_t^{ij})/N$ in the $i^{th}$ row and $j^{th}$ column. With a slight abuse of notation, we consider the entries of matrices $\IM$ and $\IMatT{t}$ to have zero mean. Formally, we state the following assumption regarding the interference matrices.
%
\begin{assumption}
    \label{asmp:apndx_Gaussian Interference Matrice}
    Entries of $\IM$ are i.i.d. Gaussian variables with zero mean and variance $\sigma^2/N$. Similarly, for any $t$, entries of $\IMatT{t}$ are i.i.d. Gaussian variables with zero mean and variance $\sigma^2_t/N$, independent of other components of the model.
\end{assumption}
%

Compared to \eqref{eq:outcome_function_matrix}, the specification in \eqref{eq:apndx_outcome_function} and \eqref{eq:apndx_outcome_function_WOD} is more comprehensive, incorporating both the complete outcome history and the full treatment allocation matrix. This generalization enables us to capture more complex temporal dynamics, including additional lag terms and potential anticipation effects of treatments. The functions $\outcomeg{t}{}$ and $\outcomeh{t}{}$ can also be specified to include only a finite number $l \in \N$ of historical lag terms. While all subsequent results hold for the general model in \eqref{eq:apndx_outcome_function} and \eqref{eq:apndx_outcome_function_WOD}, for notational simplicity, we present the proofs using only one lag term $\VoutcomeD{}{}{t}(\Mtreatment{}{})$. Furthermore, when the context is clear, we omit the treatment matrix notation $\Mtreatment{}{}$ and write simply $\VoutcomeD{}{}{t}$ as the vector of outcomes at time $t$.



\subsection{Outcome Decomposition Rule}
\label{apndx:OD_Rule}
% 
Fixing $N$, we proceed by setting more notations. Given $\VoutcomeD{}{}{0}$ as the vector of initial outcomes,  for $1\leq t < N$, define:
% 
\begin{equation}
    \label{eq:apndx_orthonormal_outcomes_vectors}
    \VOCO{}{}{0} :=
    \frac{\outcomeg{0}{}\left(\VoutcomeD{}{}{0} ,\Mtreatment{}{}{}, \covar\right)}{\norm{\outcomeg{0}{}\left(\VoutcomeD{}{}{0} ,\Mtreatment{}{}{}, \covar\right)}},
    \quad\quad\quad\quad
    \VOCO{}{}{t} :=
    \frac{\left(\I_{N} - \MOCO{}{}{t-1}\MOCO{}{\top}{t-1}\right) \outcomeg{t}{}\left(\VoutcomeD{}{}{t} ,\Mtreatment{}{}{}, \covar\right)}{\norm{\left(\I_{N} - \MOCO{}{}{t-1}\MOCO{}{\top}{t-1}\right) \outcomeg{t}{}\left(\VoutcomeD{}{}{t} ,\Mtreatment{}{}{}, \covar\right)}},
\end{equation}
% 
where, $\I_{N}$ is $N \x N$ identity matrix, and
% 
\begin{equation}
    \label{eq:apndx_orthonormal_outcomes_matrix}
    \MOCO{}{}{t-1} = \left[\VOCO{}{}{0}\Big|\ldots\Big|\VOCO{}{}{t-1}\right].
\end{equation}
% 
Note that $\I_{N} - \MOCO{}{}{t-1}\MOCO{}{\top}{t-1}$ functions as a projection onto the subspace that is orthogonal to the column space of $\MOCO{}{}{t-1}$. As a result, the vectors $\{\VOCO{}{}{0}, \ldots, \VOCO{}{}{N-1}\}$ constitute an orthonormal basis by definition. Therefore, we can represent the vector $\outcomeg{t}{}\left(\VoutcomeD{}{}{t} ,\Mtreatment{}{}{}, \covar\right)$ with respect to this basis as $\VPC{}{t} := (\PC{0}{t},\ldots,\PC{t}{t}, 0,\ldots,0)^\top \in \R^{N}$; that is:
% 
\begin{equation}
    \label{eq:apndx_representation_in_the_new_basis}
    \outcomeg{t}{}\left(\VoutcomeD{}{}{t} ,\Mtreatment{}{}{}, \covar\right) =
    \sum_{j=0}^t \PC{j}{t} \VOCO{}{}{j},
    \quad\quad\quad\quad
    \PC{j}{t} = \pdot{\outcomeg{t}{}\left(\VoutcomeD{}{}{t} ,\Mtreatment{}{}{}, \covar\right)}{\VOCO{}{}{j}}.
\end{equation}
% 
Then, it is immediate to get $\norm{\VPC{}{t}} = \normWO{\outcomeg{t}{}(\VoutcomeD{}{}{t} ,\Mtreatment{}{}{}, \covar)}$. In addition, we let $\MOCO{}{\perp}{t-1} \in \R^{N\times (N-t)}$ denote the orthogonal complement of $\MOCO{}{}{t-1}$ such that $(\MOCO{}{\perp}{t-1})^\top \MOCO{}{\perp}{t-1} = \I_{N-t}$.

We also define the following sequence of matrices based on the fixed interference matrix $\IM$:
% 
\begin{equation}
    \label{eq:apndx_IM_recursion}
    \IMat{0} := \IM,
    \quad\quad\quad\quad
    \IMat{t} := \IMat{t-1} \left(\I_{N} - \VOCO{}{}{t-1}\VOCO{}{\top}{t-1}\right).
\end{equation}
% 
Then,
Eq.~\eqref{eq:apndx_IM_recursion} enables us to write:
% 
\begin{equation}
    \label{eq:apndx_representation_of_IM}
    \IMat{0} = \IMat{t} + \sum_{j=0}^{t-1} \left(\IMat{j}-\IMat{j+1}\right) = \IMat{t} + \sum_{j=0}^{t-1} \IMat{j} \VOCO{}{}{j}\VOCO{}{\top}{j}.
\end{equation}
% 
Further, for $t=1, \ldots,N$, we define the following sequence of matrices:
% 
\begin{equation}
    \label{eq:apndx_new_IM_matrix} 
    \IMatnew{t} := \IMat{t} \MOCO{}{\perp}{t-1}
    = \IMat{t-1} \left(\I_{N} - \VOCO{}{}{t-1}\VOCO{}{\top}{t-1}\right) \MOCO{}{\perp}{t-1}
    =
    \IMat{t-1} \MOCO{}{\perp}{t-1}
    = \ldots = \IM \MOCO{}{\perp}{t-1}
    \in \R^{N \x (N-t)},
\end{equation}
% 
where we used the fact that $\VOCO{}{\top}{s-1} \MOCO{}{\perp}{t-1} = \Vec{0} \in \R^{N-t}$, for any $s \leq t$. Also, by \eqref{eq:apndx_IM_recursion}, we can write
% 
\begin{equation}
    \label{eq:apndx_new_IM_matrix_2}
    \IMat{t} = \IMat{t-1} \left(\I_{N} - \VOCO{}{}{t-1}\VOCO{}{\top}{t-1}\right) = \ldots = \IM \left(\I_{N} - \MOCO{}{}{t-1}\MOCO{}{\top}{t-1}\right) = \IM \MOCO{}{\perp}{t-1} (\MOCO{}{\perp}{t-1})^\top = \IMatnew{t} (\MOCO{}{\perp}{t-1})^\top.
\end{equation}
% 

We prove the outcome decomposition rule in multiple steps, beginning with the following lemma.
% 
\begin{lemma}
    \label{lm:outcome_decomposition}
    For any $t = 0, \ldots, N-1$, we have
    % 
    \begin{equation}
        \label{eq:apndx_outcome_decomposition}
        \VoutcomeD{}{}{t+1} =
        \IMatMean{t} \outcomeg{t}{}\left(\VoutcomeD{}{}{t}, \Mtreatment{}{}{}, \covar{} \right)
        +
        \sum_{j=0}^{t} \PC{j}{t} \left(\IMat{j} + \IMatT{t}\right) \VOCO{}{}{j} 
        + \outcomeh{t}{}\left(\VoutcomeD{}{}{t}, \Mtreatment{}{}{}, \covar{} \right)
        + \Vnoise{}{t}.
    \end{equation}
    % 
\end{lemma}
% 
\textbf{Proof.} By outcome model given in \eqref{eq:apndx_outcome_function} and \eqref{eq:apndx_representation_of_IM}, we can write:
% 
\begin{equation*}
    \begin{aligned}
    \VoutcomeD{}{}{t+1} 
    &=
    \IMat{t} \outcomeg{t}{}\left(\VoutcomeD{}{}{t}, \Mtreatment{}{}{}, \covar\right)
    + \sum_{j=0}^{t-1} \IMat{j} \VOCO{}{}{j}\VOCO{}{\top}{j} \outcomeg{t}{}\left(\VoutcomeD{}{}{t}, \Mtreatment{}{}{}, \covar\right)
    + \big(\IMatMean{t} + \IMatT{t}\big) \outcomeg{t}{}\left(\VoutcomeD{}{}{t}, \Mtreatment{}{}{}, \covar\right)
    + \outcomeh{t}{}\left(\VoutcomeD{}{}{t}, \Mtreatment{}{}{}, \covar{} \right)
    + \Vnoise{}{t}
    \\
    &=
    \IMat{t} \outcomeg{t}{}\left(\VoutcomeD{}{}{t}, \Mtreatment{}{}{}, \covar\right)
    + \sum_{j=0}^{t-1} \PC{j}{t} \IMat{j} \VOCO{}{}{j}
    + \sum_{j=0}^{t} \PC{j}{t} \IMatT{t} \VOCO{}{}{j}
    + \IMatMean{t} \outcomeg{t}{}\left(\VoutcomeD{}{}{t}, \Mtreatment{}{}{}, \covar{} \right)
    + \outcomeh{t}{}\left(\VoutcomeD{}{}{t}, \Mtreatment{}{}{}, \covar{} \right)
    + \Vnoise{}{t}
    \\
    &=
    \IMatMean{t} \outcomeg{t}{}\left(\VoutcomeD{}{}{t}, \Mtreatment{}{}{}, \covar{} \right)
    +
    \sum_{j=0}^{t} \PC{j}{t} (\IMat{j}+\IMatT{t}) \VOCO{}{}{j}
    + \outcomeh{t}{}\left(\VoutcomeD{}{}{t}, \Mtreatment{}{}{}, \covar{} \right)
    + \Vnoise{}{t},
    \end{aligned}
\end{equation*}
% 
where in the third line, we used \eqref{eq:apndx_representation_in_the_new_basis} and the fact that the vectors $\{\VOCO{}{}{0}, \ldots, \VOCO{}{}{t}\}$ constitute an orthonormal set. Also, in the last line, we utilized the following fact
% 
\begin{equation*}
    \IMat{t} \outcomeg{t}{}\left(\VoutcomeD{}{}{t}, \Mtreatment{}{}{}, \covar\right)
    = \IMat{t} \left(\I_{N} - \MOCO{}{}{t-1}\MOCO{}{\top}{t-1}\right) \outcomeg{t}{}\left(\VoutcomeD{}{}{t}, \Mtreatment{}{}{}, \covar\right)
    = \PC{t}{t} \IMat{t} \VOCO{}{}{t},
\end{equation*}
% 
that holds true because $\IMat{t} \MOCO{}{}{t-1}\MOCO{}{\top}{t-1} = 0$ by \eqref{eq:apndx_IM_recursion}; additionally, the term $\left(\I_{N} - \MOCO{}{}{t-1}\MOCO{}{\top}{t-1}\right) \outcomeg{t}{}\left(\VoutcomeD{}{}{t}, \Mtreatment{}{}{}, \covar\right)$ is equal to the projection of $\outcomeg{t}{}\left(\VoutcomeD{}{}{t}, \Mtreatment{}{}{}, \covar\right)$ on the subspace perpendicular to the column space of $\MOCO{}{}{t-1}$, which is $\PC{t}{t} \VOCO{}{}{t}$. It completes the proof. \ep

The next lemma characterizes the distribution of $\IMatnew{t}$, defined in \eqref{eq:apndx_new_IM_matrix}, based on the rotational invariance property of Gaussian matrices. 
% 
\begin{lemma}
    \label{lm:new_IM_distribution}
    Fix $1 \leq t < N$. Conditional on $\VoutcomeD{}{}{0}$, $\VoutcomeD{}{}{1}, \ldots, \VoutcomeD{}{}{t-1}$, $\Mtreatment{}{}{}$, and $\covar$, entries of the matrix $\IMatnew{t} \in \R^{N \times (N-t)}$ are i.i.d. with distribution $\Nc(0,{\sigma^2}/ {N})$, and the matrix $\IMatnew{t}$ (and so $\IMat{t}$) is independent of $\VoutcomeD{}{}{t},\VOCO{}{}{t},$ as well as $\IMat{0}\VOCO{}{}{0}, \ldots, \IMat{t-1}\VOCO{}{}{t-1}$. 
\end{lemma}
% 
\textbf{Proof.} First, note that given $\Mtreatment{}{}{}$ and $\covar$, by \eqref{eq:apndx_orthonormal_outcomes_vectors}, conditioning on $\VoutcomeD{}{}{0}$, $\VoutcomeD{}{}{1}, \ldots, \VoutcomeD{}{}{t-1}$, is equivalent to conditioning on $\VOCO{}{}{0}$, $\VOCO{}{}{1}, \ldots, \VOCO{}{}{t-1}$. Now, we use an induction on $t$ to prove the result.

\textbf{Step 1.} Let $t=1$. By \eqref{eq:apndx_new_IM_matrix} and the rotational invariance property of Gaussian matrices, we have
% 
\begin{equation}
    \label{eq:apndx_proof_new_IM_distribution_1}
    \IMatnew{1} = \IM \MOCO{}{\perp}{0} \eqd \IM \sbasis{\perp}{1},
\end{equation}
% 
where $\sbasis{\perp}{1}$ denotes the orthogonal complement of $\sbasis{}{1}$, which is the first standard basis vector. Letting $\sbasis{\perp}{1} = [\sbasis{}{2}|\ldots|\sbasis{}{N}]$, by Assumption~\ref{asmp:apndx_Gaussian Interference Matrice}, we get that the entries of $\IMatnew{1} \in \R^{N \times (N-1)}$ are i.i.d. with distribution $\Nc(0,{\sigma^2}/ {N})$. Furthermore, considering that $\MOCO{}{}{0} = \VOCO{}{}{0}$, the matrix $\IMatnew{1}$ is independent from $\IMat{0}\VOCO{}{}{0}$. Then, conditional on $\VoutcomeD{}{}{0}$, the outcome model in \eqref{eq:apndx_outcome_function} implies that $\IMatnew{1}$ is independent of $\VoutcomeD{}{}{1}$ and $\VOCO{}{}{1}$ (note that the randomness of $\VoutcomeD{}{}{1}$ comes from $\IMat{0}\VOCO{}{}{0}$, $\IMatT{0}$, and $\Vnoise{}{0}$). Finally, considering \eqref{eq:apndx_new_IM_matrix_2}, the same results about the independency hold true for the matrix $\IMat{1}$.

\textbf{Step 2.} Suppose that the result is true for $t=1, \ldots, s-1$. Given $\VoutcomeD{}{}{0}, \ldots, \VoutcomeD{}{}{s-1}$, we show the result also holds for $t=s$. Note that by the induction hypothesis, conditional on $\VoutcomeD{}{}{0}, \ldots, \VoutcomeD{}{}{s-2}$, the matrix $\IMat{s-1}$ is independent of $\VoutcomeD{}{}{s-1}$ and $\IMat{0}\VOCO{}{}{0}, \ldots, \IMat{s-2}\VOCO{}{}{s-2}$. Thus, $\IMat{s-1}$ and $\VoutcomeD{}{}{s-1}$ are conditionally independent. This implies that conditional on $\VoutcomeD{}{}{0}, \ldots, \VoutcomeD{}{}{s-1}$ (we added $\VoutcomeD{}{}{s-1}$), the matrix $\IMat{s-1}$ (and so $\IMat{s} := \IMat{s-1} (\I_N - \VOCO{}{}{s-1}\VOCO{}{\top}{s-1})$, see \eqref{eq:apndx_IM_recursion}, as well as $\IMatnew{s}$) is still independent of $\IMat{0}\VOCO{}{}{0}, \ldots, \IMat{s-2}\VOCO{}{}{s-2}$.

Next, we show that $\IMat{s}$ is also independent from $\IMat{s-1}\VOCO{}{}{s-1}$. By \eqref{eq:apndx_new_IM_matrix} and the rotational invariance property of Gaussian matrices, we can write
% 
\begin{equation*}
    \IMatnew{s} = \IM \MOCO{}{\perp}{s-1} \eqd \IM [\sbasis{}{1}| \ldots| \sbasis{}{s}]^\perp.
\end{equation*}
% 
Here, $[\sbasis{}{1}| \ldots| \sbasis{}{s}]^\perp$ represents the orthogonal complement of the first $s$ standard basis vectors. Then, a similar argument to the one in Step 1 implies that the matrix $\IMatnew{s} \in \R^{N\times (N-s)}$ has i.i.d. entries with a distribution of $\Nc(0,{\sigma^2}/ {N})$. Furthermore, it yields that $\IMatnew{s}$ (and consequently $\IMat{s}$, see Eq.~\eqref{eq:apndx_new_IM_matrix_2}) is independent of the vector $\IMat{s-1}\VOCO{}{}{s-1} = \IM (\I_N - \MOCO{}{}{s-2} \MOCO{}{\top}{s-2}) \VOCO{}{}{s-1} = \IM \VOCO{}{}{s-1}$; this holds true because of Eq.~\eqref{eq:apndx_new_IM_matrix_2} and the fact that $\VOCO{}{\top}{j} \VOCO{}{}{s-1} =0$, for $j = 0, \ldots, s-2$. 

Now, by Lemma~\ref{lm:outcome_decomposition}, we have
% 
\begin{equation*}
    \VoutcomeD{}{}{s} =
    \IMatMean{s-1} \outcomeg{s-1}{}\left(\VoutcomeD{}{}{s-1}, \Mtreatment{}{}{}, \covar{} \right)
    \sum_{j=0}^{s-1} \PC{j}{s-1} \left(\IMat{j} + \IMatT{s-1}\right) \VOCO{}{}{j}
    + \outcomeh{s-1}{}\left(\VoutcomeD{}{}{s-1}, \Mtreatment{}{}{}, \covar{} \right)
    + \Vnoise{}{s-1}.
\end{equation*}
% 
As a result, $\IMat{s}$ and $\IMatnew{s}$ are also independent of $\VoutcomeD{}{}{s}$ and $\VOCO{}{}{s}$. This concludes the proof. \ep

By combining the results of Lemmas~\ref{lm:outcome_decomposition} and~\ref{lm:new_IM_distribution}, we arrive at the conclusion of Lemma~\ref{lm:apndx_outcome_decomposition}.
% 
\begin{lemma}
    \label{lm:apndx_outcome_decomposition}
    For any $t = 0, \ldots, N-1$, we have
    % 
    \begin{equation}
        \label{eq:apndx_outcome_decomposition_distribtuion}
        \VoutcomeD{}{}{t+1} =
        \IMatMean{t} \outcomeg{t}{}\left(\VoutcomeD{}{}{t}, \Mtreatment{}{}{}, \covar{} \right) +
        \outcomeh{t}{}\left(\VoutcomeD{}{}{t}, \Mtreatment{}{}{}, \covar{} \right) +
        \sqrt{\sigma^2+\sigma_t^2} \norm{\outcomeg{t}{}(\VoutcomeD{}{}{t} ,\Mtreatment{}{}{}, \covar)} \sum_{j=0}^{t} \NPC{j}{t} \Vec{Z}_j  + \Vnoise{}{t},
    \end{equation}
    % 
    where $\Vec{Z}_0, \Vec{Z}_1, \ldots, \Vec{Z}_t$ are i.i.d. random vectors in $\R^N$ following $\Nc(0,\frac{1}{N}\I_N)$ distribution. Additionally, $\NPC{j}{t} := \PC{j}{t}/\normWO{\outcomeg{t}{}(\VoutcomeD{}{}{t}, \Mtreatment{}{}, \covar)}$, making $\VNPC{}{t} = (\NPC{0}{t}, \ldots, \NPC{t}{t}, 0, \ldots, 0)^\top \in \R^N$ a unit random vector (i.e., $\normWO{\VNPC{}{t}} = 1$). Note that $\NPC{j}{t}$ and $\Vec{Z}_j$ are not independent.
\end{lemma}
% 
\textbf{Proof.}
% 
Fixing $t$, we prove the result in two steps. First, we demonstrate that $\IMat{j} \VOCO{}{}{j}$, for $j=0, \ldots, t$, follows a Gaussian distribution with specified mean and variance. Then, we show that $\IMat{0} \VOCO{}{}{0}, \ldots, \IMat{t} \VOCO{}{}{t}$ are independent. 

Conditional on $\VoutcomeD{}{}{0}$, $\VoutcomeD{}{}{1}, \ldots, \VoutcomeD{}{}{j-1}$, $\Mtreatment{}{}{}$, and $\covar$, Lemma~\ref{lm:new_IM_distribution} implies that the matrix $\IMat{j}$ and the vector $\VOCO{}{}{j}$ are independent. Also, by  \eqref{eq:apndx_new_IM_matrix_2}, we can write
% 
\begin{equation*}
    \IMat{j} \VOCO{}{}{j} =
    \IM \MOCO{}{\perp}{j-1} (\MOCO{}{\perp}{j-1})^\top  \VOCO{}{}{j} = \IM \VOCO{}{}{j},
\end{equation*}
% 
where we used the fact that the vector $\VOCO{}{}{j}$ is perpendicular to the column space of the matrix $\MOCO{}{}{j-1}$. Thus, conditional on the value of $\VOCO{}{}{j}$, as well as $\VoutcomeD{}{}{0}$, $\VoutcomeD{}{}{1}, \ldots, \VoutcomeD{}{}{j-1}$, $\Mtreatment{}{}{}$, and $\covar$, by the rotational invariance property of Gaussian matrices, the elements of $\IMat{j} \VOCO{}{}{j}$ are i.i.d. random variables with distribution $\Nc(0,\frac{\sigma^2}{N})$. Furthermore, note that this conditional distribution of $\IMat{j} \VOCO{}{}{j}$ remains the same regardless of the value of $\VOCO{}{}{j}$. As a result, we can conclude that the elements of $\IMat{j} \VOCO{}{}{j}$ are i.i.d. Gaussian, even without conditioning on $\VOCO{}{}{j}$ as well as $\VoutcomeD{}{}{0}$, $\VoutcomeD{}{}{1}, \ldots, \VoutcomeD{}{}{j-1}$, $\Mtreatment{}{}{}$, and $\covar$. Precisely, for a deterministic vector $\Vec{v}$, we can write:
% 
\begin{align*}
    \P
    \left(
    \IMat{j} \VOCO{}{}{j} \leq \Vec{v}
    \right)
    =
    \E
    \left[
    \E
    \left[
    \P
    \left(
    \IMat{j} \VOCO{}{}{j} \leq \Vec{v}
    \right)
    \Big|
    \VOCO{}{}{j}
    \right]
    \bigg|
    \VoutcomeD{}{}{0}, \ldots, \VoutcomeD{}{}{j-1}, \Mtreatment{}{}{}, \covar
    \right]
    =
    \E
    \left[
    \E
    \left[
    \Phi(\Vec{v})
    \Big|
    \VOCO{}{}{j}
    \right]
    \bigg|
    \VoutcomeD{}{}{0}, \ldots, \VoutcomeD{}{}{j-1}, \Mtreatment{}{}{}, \covar
    \right]
    =
    \Phi(\Vec{v}),
\end{align*}
% 
where $\Phi$ denotes the CDF of a vector whose entries follow a normal distribution $\Nc(0, \frac{\sigma^2}{N})$.

We proceed by establishing the independence property. Note that by Lemma~\ref{lm:new_IM_distribution}, conditional on the values of $\VoutcomeD{}{}{0}, \ldots, \VoutcomeD{}{}{t}, \Mtreatment{}{}{}, \covar$ (and so on the values of $\VOCO{}{}{0}, \ldots, \VOCO{}{}{t-1}, \VOCO{}{}{t}$), it follows that $\IMat{t}$ (and so $\IMat{t} \VOCO{}{}{t}$) is independent of $\IMat{0}\VOCO{}{}{0}, \ldots, \IMat{t-1}\VOCO{}{}{t-1}$. Importantly, our previous demonstration confirmed that the distribution of $\IMat{t} \VOCO{}{}{t}$ remains unchanged across different values of $\VoutcomeD{}{}{0}, \ldots, \VoutcomeD{}{}{t}, \Mtreatment{}{}{}, \covar$. Consequently, we can assert that the random vector $\IMat{t}\VOCO{}{}{t}$ is independent of $\IMat{0}\VOCO{}{}{0}, \ldots, \IMat{t-1}\VOCO{}{}{t-1}$. More precisely, we can repeat this argument multiple times and, for deterministic vectors $\Vec{v}_0, \ldots, \Vec{v_t}$, show that
% 
\begin{align*}
    \P
    \left(
    \IMat{0} \VOCO{}{}{0} \leq \Vec{v}_0,
    \ldots,
    \IMat{t} \VOCO{}{}{t} \leq \Vec{v}_t
    \right)
    &=
    \E
    \left[
    \P
    \left(
    \IMat{0} \VOCO{}{}{0} \leq \Vec{v}_0,
    \ldots,
    \IMat{t} \VOCO{}{}{t} \leq \Vec{v}_t
    \right)
    \Big|
    \VoutcomeD{}{}{0}, \ldots, \VoutcomeD{}{}{t}, \Mtreatment{}{}{}, \covar
    \right]
    \\
    &=
    \E
    \left[
    \P
    \left(
    \IMat{0} \VOCO{}{}{0} \leq \Vec{v}_0,
    \ldots,
    \IMat{t-1} \VOCO{}{}{t-1} \leq \Vec{v}_{t-1}
    \right)
    \Phi(\Vec{v}_t)
    \Big|
    \VoutcomeD{}{}{0}, \ldots, \VoutcomeD{}{}{t}, \Mtreatment{}{}{}, \covar
    \right]
    \\
    &=
    \Phi(\Vec{v}_t)
    \E
    \left[
    \P
    \left(
    \IMat{0} \VOCO{}{}{0} \leq \Vec{v}_0,
    \ldots,
    \IMat{t-1} \VOCO{}{}{t-1} \leq \Vec{v}_{t-1}
    \right)
    \Big|
    \VoutcomeD{}{}{0}, \ldots, \VoutcomeD{}{}{t-1}, \Mtreatment{}{}{}, \covar
    \right]
    \\
    &=
    \ldots
    \\
    &=
    \Phi(\Vec{v}_0)
    \Phi(\Vec{v}_1)
    \ldots
    \Phi(\Vec{v}_t).
\end{align*}
% 

Based on the fact that the time-dependent interference matrix $\IMatT{t}$ and the noise vector $\Vnoise{}{t}$ are independent of everything else in the model, the proof is complete and we obtain the desired result in \eqref{eq:apndx_outcome_decomposition_distribtuion}. \ep


The dependence between $\NPC{j}{t}$ and $\Vec{Z}_j$ in Lemma~\ref{lm:apndx_outcome_decomposition} implies that the Gaussianity of the elements in $\VoutcomeD{}{}{t+1}$ cannot be inferred directly from the result of this lemma alone.


\noindent
\textbf{Proof of Theorem~\ref{thm:outcome_decomposition}.}
To obtain the desired result, we apply Lemma~\ref{lm:apndx_outcome_decomposition} and Lemma~\ref{lm:Gap_with_Gaussian_vector} together. To be more specific, in view of Lemma~\ref{lm:apndx_outcome_decomposition}, we know that
% 
\begin{equation*}
    \VoutcomeD{}{}{t+1} =
    \IMatMean{t} \outcomeg{t}{}\left(\VoutcomeD{}{}{t}, \Mtreatment{}{}{}, \covar{} \right) +
    \outcomeh{t}{}\left(\VoutcomeD{}{}{t}, \Mtreatment{}{}{}, \covar{} \right) +
    \sqrt{\sigma^2+\sigma_t^2} \norm{\outcomeg{t}{}(\VoutcomeD{}{}{t} ,\Mtreatment{}{}{}, \covar)}  \Sumvec{}{t} + \Vnoise{}{t},
\end{equation*}
%
where $\Sumvec{}{t} = \sum_{i=0}^{t} \NPC{i}{t} \Vec{Z}_i$. But, by Lemma~\ref{lm:Gap_with_Gaussian_vector}, we have
\begin{equation*}
    W_1\left(\law\left(\Sumvec{}{t}\right),\Nc\left(0,\frac{1}{N}\I_N\right)\right) \leq c \sqrt{\frac{t \log N}{N}},
\end{equation*}
which concludes the proof. \ep


\noindent
\textbf{Proof of Corollary~\ref{crl:SampleMean_decomposition}.}
% 
By the result of Theorem~\ref{thm:outcome_decomposition}, we can write
% 
\begin{align*}
    \frac{1}{\cardinality{\batch}} \sum_{i \in \batch} \outcomeD{}{i}{t+1}
    =
    \;&\frac{1}{N \cardinality{\batch}} \sum_{i \in \batch} \sum_{j=1}^N \left(\mu^{ij}+\mu_t^{ij}\right) \outcomeg{t}{}\left(\outcomeD{}{j}{t}, \Vtreatment{j}{}{}, \Vcovar{j} \right)
    +
    \frac{1}{\cardinality{\batch}} \sum_{i \in \batch} \outcomeh{t}{}\left(\outcomeD{}{i}{t}, \Vtreatment{i}{}{}, \Vcovar{i} \right)
    \\ \;&+
    \sqrt{\frac{\sigma^2+\sigma_t^2}{\cardinality{\batch}}} \norm{\outcomeg{t}{}(\VoutcomeD{}{}{t} ,\Mtreatment{}{}{}, \covar)} \frac{1}{\sqrt{\cardinality{\batch}}} \sum_{i \in \batch} \sumvec{i}{t}
    +
    \frac{1}{\cardinality{\batch}} \sum_{i \in \batch} \noise{i}{t}.
\end{align*}
% 
Letting $\avesumvec{}{t} := \frac{1}{\sqrt{\cardinality{\batch}}} \sum_{i \in \batch} \sumvec{i}{t}$ and applying Lemma~\ref{lm:Gap_with_Gaussian} for $\Vec{\Phi} = \Sumvec{}{t}$, we get the result. \ep




\subsection{Batch-level State Evolution}
\label{apndx:batch_state_evolution}
% 
Next, we analyze the large-sample behavior of the outcomes for a subpopulation of units. Specifically, let $\batch \subset [N]$ represent an arbitrary subpopulation, with its size $\cardinality{\batch}$ increasing indefinitely as the population size $N$ grows large to infinity. We investigate the asymptotic behavior of the elements in $\Moutcome{}{}{}$ as $N$ approaches infinity, This provides valuable insights into the evolution of outcomes within the subpopulation.
% 
\begin{assumption}
    \label{asmp:BL}
    Fixing $T \in \N$ and $k \geq 2$, we assume that
    \begin{enumerate}[label=(\roman*)]
        \item \label{asmp:BL-pl functions} For all $t\in[T]$, the function $\outcomeg{t}{}:\R^{1+T+M}\rightarrow \R$ is a $\poly{\frac{k}{2}}$ function.

        \item \label{asmp:BL-pl h-functions} For all $t\in[T]$, the function $\outcomeh{t}{}:\R^{1+T+M}\rightarrow \R$ is a $\poly{1}$ function.

        \item \label{asmp:BL-bound on initials} The sequence of initial outcome vectors $\VoutcomeD{}{}{0}$, the treatment allocation matrices~$\Mtreatment{}{}$, the covariates $\covar$, and the function $\outcomeg{0}{}$ are such that for a deterministic value $\MVVO{}{}{1}>0$, we have
        \begin{align*}
            (\MVVO{}{}{1})^2
            &=
            \lim_{N\rightarrow \infty}
            \frac{\sigma^2+\sigma_0^2}{N} \sum_{i=1}^N
            \outcomeg{0}{}\big(
            \outcomeD{}{i}{0},\Vtreatment{i}{},\Vcovar{i}
            \big)^2 < \infty.
        \end{align*}
    \end{enumerate}
\end{assumption}
% 
Assumption~\ref{asmp:BL} comprises a collection of regularity conditions on the model attributes. The first two parts ensure that the functions $\outcomeg{t}{}$ and $\outcomeh{t}{}$ do not demonstrate fast explosive behavior, thereby ensuring the well-posedness of the large system asymptotic. The final part pertains to the initial observation of the network and corroborates that the function $\outcomeg{0}{}$ is non-degenerate. This ensures that initial observations provide meaningful information and contribute to the evolution of outcomes.
% 
\begin{assumption}
    \label{asmp:weak_limits}
    % 
    Fix $T \in \N$, $k \geq 2$, and a subpopulation $\batch \subset [N]$, where the size $\cardinality{\batch}$ grows to infinity as $N \rightarrow \infty$. We assume the following statements hold:
    % 
    \begin{enumerate}[label=(\roman*)]
        \item $p_{y}^{\batch}(N)$ denotes the empirical distribution of the \textbf{initial outcomes} $\outcomeD{N}{i}{0}$ with $i \in \batch$, and converges weakly to a probability measure $p_{y}^{\batch}$ such that $\E_{p_{y}^{\batch}} \big[\normWO{\outcomeD{}{}{0}}^k\big]< \infty$,

        \item $p_{x}^{\batch}(N)$ denotes the empirical distribution of the \textbf{covariate} vectors $\Vcovar{i}(N)$ with $i \in \batch$, and converges weakly to a probability measure $p_{x}^{\batch}$ such that $\E_{p_{x}^{\batch}} \big[\normWO{\Vcovar{}}^k\big]< \infty$,

        \item $p_{w}^{\batch}(N)$ denotes the empirical distribution of the \textbf{treatment} vectors $\Vtreatment{i}{N}$ with $i \in \batch$, and converges weakly to a probability measure $p_{w}^{\batch}$ such that $\E_{p_{w}^{\batch}} \big[\normWO{\Vtreatment{}{}}^k\big]< \infty$.
        
        \item \label{asmp:interference_element_convergence} For all $i$ and any $t\in [T]_0$, let $p_{\mu^i}(N)$ and $p_{\mu_t^i}(N)$ be, respectively, the empirical distribution of the elements of the vectors $\Vec{\mu}^{\;i\cdot} := (\mu^{i1}, \ldots, \mu^{iN})^\top$ and $\Vec{\mu}^{\;i\cdot}_t := (\mu^{i1}_t, \ldots, \mu^{iN}_t)^\top$. Then, $p_{\mu^i}(N)$ converges weakly to $p_{\mu^i}$ and $p_{\mu_t^i}(N)$ converges weakly to $p_{\mu_t^i}$. Also, $\E_{p_{\mu^i}} \big[\normWO{\MIM{}{i}}^k\big]< \infty$ and $\E_{p_{\mu^i_t}} \big[\normWO{\MIM{t}{i}}^k\big]< \infty$.
        
        In addition, the limit distributions ($p_{\mu^i}$ and $p_{\mu^i_t}$) are independent of other randomnesses in the model and if $\bar{\mu}^i$ denotes the mean of a random variable under probability measure $p_{\mu^i}$ (i.e., $\bar{\mu}^i = \E_{p_{\mu^i}}[\MIM{}{i}]$), the empirical distributions of $\bar{\mu}^i,\; i \in \batch$, denoted by $p_{\mu}^{\batch}(N)$, converges weakly to a probability measure $p_{\mu}^{\batch}$. Likewise, we let $p_{\mu_t}^{\batch}$ denote the weak limit of the empirical distribution of the means under probability measures $p_{\mu^i_t}$. Finally, $\E_{p_{\mu}^{\batch}} \big[\normWO{\MIM{}{\batch}}^k\big]< \infty$ as well as $\E_{p_{\mu_t}^{\batch}} \big[\normWO{\MIM{t}{\batch}}^k\big]< \infty$.

        \item \label{asmp:k_moment_convergence} For all $i$ and $t\in [T]_0$, as $N \rightarrow \infty$, we have
        %
        \begin{align*}
            &\;\E_{p_{y}^{\batch}(N) \times p_{x}^{\batch}(N) \times p_{w}^{\batch}(N) \times p_{\mu^i}(N) \times p_{\mu_t^i}(N) \times p_{\mu}^{\batch}(N) \times p_{\mu_t}^{\batch}(N)} \big[\normWO{\outcomeD{}{}{0},\Vcovar{}, \Vtreatment{}{}, \MIM{}{i}, \MIM{t}{i}, \MIM{}{\batch}, \MIM{t}{\batch}}^k\big]
            \\
            \rightarrow
            &\;\E_{p_{y}^{\batch} \times p_{x}^{\batch} \times p_{w}^{\batch} \times p_{\mu^i} \times p_{\mu^i_t} \times p_{\mu}^{\batch} \times p_{\mu_t}^{\batch}}\big[\normWO{\outcomeD{}{}{0},\Vcovar{}, \Vtreatment{}{}, \MIM{}{i}, \MIM{t}{i}, \MIM{}{\batch}, \MIM{t}{\batch}}^k\big].
        \end{align*}
    \end{enumerate}
\end{assumption}
% 
\begin{remark}
    We can replace the second part of Assumption \ref{asmp:weak_limits}-\ref{asmp:interference_element_convergence} by assuming that $p_{\mu^i}$ is the same across all units, and similarly, $p_{\mu^i_t}$ is identical for all units.
\end{remark}
% 
\begin{remark}
    We can drop Assumptions \ref{asmp:BL} and \ref{asmp:weak_limits}-\ref{asmp:k_moment_convergence} by confining the functions $\outcomeg{t}{}$ and $\outcomeh{t}{}$ to be bounded and continuous.
\end{remark}
% 
\begin{remark}[Notation convention]
    In Assumption~\ref{asmp:weak_limits}, when $\batch$ represents the entire experimental population, we omit the superscript $\batch$ from all notations.
\end{remark}
% 
\begin{remark}[Time-dependent covariates]
    For each time $t = 0, 1, \ldots, T$, let $\Xc_t \in \R^{\dcovar_t \times N}$ denote the time-dependent covariate matrix, where the $i^{th}$ column corresponds to the covariates for unit $i$ at time $t$. We extend the covariate matrix $\covar$ by incorporating these time-dependent covariates, resulting in a new $(\dcovar + \dcovar_0 + \ldots + \dcovar_T) \times N$ covariate matrix in our model. The functions $\outcomeg{t}{}$ and $\outcomeh{t}{}$ are then modified accordingly to reference the appropriate portion of this extended covariate matrix at each time period. Along this, we exclude the noise vectors (i.e., $\Vnoise{}{t}$) from subsequent discussions on the potential outcome specification Eq.~\eqref{eq:apndx_outcome_function}.
\end{remark}
% 
Assumption~\ref{asmp:weak_limits} is a standard assumption in statistical theory, ensuring that the empirical distributions of system attributes remain stable and do not diverge as the sample size increases. This assumption holds, for example, when units' attributes $\left\{(\outcomeD{}{i}{0}, \Vcovar{i}, \Vtreatment{i}{}, \Vec{\mu}^{\;i\cdot}, \Vec{\mu}^{\;i\cdot}_t)\right\}_{i}$ follow an i.i.d. distribution with finite moments of order $k$. Moreover, a wide range of treatment assignments satisfies the conditions of Assumption~\ref{asmp:weak_limits}, including cases where the support of $\pi$ is bounded, such as the Bernoulli design. By imposing such conditions, we ensure the reliability of estimation by keeping the experimental design moments finite and manageable.


To state the main theoretical results, we need to define the \textbf{Batch State Evolution (BSE)} equations as follows:
% 
\begin{equation}
    \label{eq:state evolution}
    \begin{aligned}
        \MAVO{}{\batch}{1} &=
        \E\left[ (\MIM{}{\batch}+\MIM{0}{\batch})
        \outcomeg{0}{}\big(\outcomeD{}{}{0}, \Vtreatment{}{},\Vcovar{}\big)\right],
        \quad%%%
        \MVVO{}{}{1} =
        (\sigma+\sigma_0)
        \E\left[
        \outcomeg{0}{}\big(\outcomeD{}{}{0}, \Vtreatment{}{},\Vcovar{}\big)^2\right],
        \quad%%%
        \Houtcome{\batch}{}{0} = \outcomeh{0}{}\big(\outcomeD{}{\batch}{0}, \Vtreatment{\batch}{}, \Vcovar{\batch}\big),
        \\
        \Houtcome{\batch}{}{t} &= \outcomeh{t}{}\big(\MAVO{}{\batch}{t} + \MVVO{}{}{t} Z_t + \Houtcome{\batch}{}{t-1}, \Vtreatment{\batch}{}, \Vcovar{\batch}\big),
        \\
        \MAVO{}{\batch}{t+1} &=
        \E\left[ (\MIM{}{\batch}+\MIM{t}{\batch})
        \outcomeg{t}{}\big(\MAVO{}{}{t} + \MVVO{}{}{t} Z_t + \Houtcome{}{}{t-1}, \Vtreatment{}{}, \Vcovar{}\big)
        \right],
        \\
        (\MVVO{}{}{t+1})^2 &=
        (\sigma^2+\sigma_t^2) \E\left[
        \outcomeg{t}{} \big(\MAVO{}{}{t} + \MVVO{}{}{t} Z_t + \Houtcome{}{}{t-1}, \Vtreatment{}{},\Vcovar{}\big)^2
        \right],
        \\
        \AVO{}{\batch}{t+1} &= \MAVO{}{\batch}{t+1} + \E\left[\Houtcome{\batch}{}{t}\right],
        \\
        (\VVO{}{}{t+1})^2 &= (\MVVO{}{}{t+1})^2 + \Var\left[\Houtcome{}{}{t}\right].
    \end{aligned}
\end{equation}
% 
where 
% 
\begin{itemize}
    \item $\outcomeD{}{}{0} \sim p_y$ and $\outcomeD{}{\batch}{0} \sim p_y^\batch$ represent the weak limits of the population and subpopulation initial outcomes;

    \item $\Vtreatment{}{} \sim p_w$ and $\Vtreatment{\batch}{} \sim p_w^\batch$ are the weak limits of the population and subpopulation treatment assignments;

    \item $\Vcovar{} \sim p_x$ and $\Vcovar{\batch} \sim p_x^\batch$ represent the weak limits of the population and subpopulation covariates;

    \item $\MIM{}{} \sim p_{\MIM{}{}}$ and $\MIM{t}{} \sim p_{\MIM{t}{}}$ represent the weak limits of interference elements at the population level, as specified in Assumption~\ref{asmp:weak_limits}. Similarly, the corresponding subpopulation quantities are denoted by $\MIM{}{\batch} \sim p_{\MIM{}{}}^\batch$ and $\MIM{t}{\batch} \sim p_{\MIM{t}{}}^\batch$;

    \item and independent from all of them, $Z_t$ follows a standard Gaussian distribution.
\end{itemize}
% 

The following theorem characterizes the distribution of units' outcomes $\outcomeD{}{i}{1}, \ldots, \outcomeD{}{i}{t+1}$ within the large sample asymptotic, based on the BSE equations outlined in Eq.~\eqref{eq:state evolution}.
% 
\begin{theorem}
    \label{thm:Batch_SE}
    % 
    Fixing $k\geq 2$, consider the sequence of units' attributes $\left\{(\outcomeD{}{i}{0}, \Vcovar{i}, \Vtreatment{i}{}, \Vec{\mu}^{\;i\cdot}, \Vec{\mu}^{\;i\cdot}_t)\right\}_{i,t}$ and suppose that Assumptions~\ref{asmp:Gaussian Interference Matrice}-\ref{asmp:weak_limits} hold. Then, in view of the BSE equations given in Eq.~\eqref{eq:state evolution}, for any function $\psi \in \poly{k}$, we have,
    % 
    \begin{equation}
        \begin{aligned}
            \label{eq:BSE-limit}
            &\;\lim_{N \rightarrow \infty}
            \frac{1}{\cardinality{\batch}} \sum_{i \in \batch}
            \psi\big(
            \outcomeD{}{i}{0},
            \outcomeD{}{i}{1},
            \ldots,
            \outcomeD{}{i}{T},
            \Vtreatment{i}{}, \Vcovar{i}
            \big)
            \\
            \eqas
            \;&\E
            \Big[
            \psi\big(
            \outcomeD{}{\batch}{0},
            \MAVO{}{\batch}{1} + \MVVO{}{}{1} Z_1 + \Houtcome{\batch}{}{0},
            \ldots,
            \MAVO{}{\batch}{T} + \MVVO{}{}{T} Z_{T} + \Houtcome{\batch}{}{T-1},
            \Vtreatment{\batch}{}, \Vcovar{\batch}
            \big)
            \Big],
        \end{aligned}
    \end{equation}
    % 
    where $Z_t \sim \Nc(0,1),\; t= 1,\ldots,T,$ independent of $\Vtreatment{\batch}{} \sim p_w^\batch$ and $\Vcovar{\batch} \sim p_x^\batch$. In addition, for any $t = 1, \ldots, T$, the random variables $Z_t$ and $\Houtcome{}{}{t-1}$ are independent.
\end{theorem}
%
We prove the result of Theorem~\ref{thm:Batch_SE} by extending the theoretical results of \cite{shirani2024causal}, which in turn build on the AMP framework developed by \cite{bayati2011dynamics}. The proof mainly relies on a \emph{conditioning technique} introduced by \cite{bolthausen2014iterative}. Below, we first present a version of the conditioning technique adapted to our specific setting, followed by a detailed proof of the theorem.

\begin{remark}
    To derive the result in the second part of Theorem~\ref{thm:BSE_informal} from Theorem~\ref{thm:Batch_SE}, we note that when $\batch$ depends solely on the treatment allocation, which is independently distributed from all other variables, $\batch$ effectively functions as a random sample from the experimental population. As a result, the distributions of $\outcomeD{}{\batch}{0}$, $\Vcovar{\batch}$, $\MIM{}{\batch}$, and $\MIM{t}{\batch}$ are equivalent to those of $\outcomeD{}{}{0}$, $\Vcovar{}$, $\MIM{}{}$, and $\MIM{t}{}$, respectively. This equivalence follows from the fact that the empirical distribution of quantities in a random sample converges to the empirical distribution of the entire population.
\end{remark}



\subsection{Conditioning Technique}
\label{apndx:Conditioning_Technique}
% 
Recalling \eqref{eq:apndx_outcome_function_WOD} and
letting $\VUoutcome{}{}{t} = \outcomeg{t}{}\big(\VoutcomeD{}{}{t},\Mtreatment{}{},\covar\big)$ (and $\Uoutcome{i}{}{t} = \outcomeg{t}{}\big(\outcomeD{}{i}{t},\Vtreatment{i}{},\Vcovar{i}\big)$), we denote
% 
\begin{equation}
\label{eq:Q and R}
\begin{aligned}
    \bm{Q}_t
    :=
    \left[
    \VUoutcome{}{}{0}
    \Big|
    \VUoutcome{}{}{1}
    \Big|
    \ldots
    \Big|
    \VUoutcome{}{}{t-1}
    \right],
    \quad\quad
    \bm{R}_t
    :=
    \left[
    \Voutcome{}{}{1} -
    \IMatT{0} \VUoutcome{}{}{0}
    \Big|
    \ldots
    \Big|
    \Voutcome{}{}{t} -
    \IMatT{t-1} \VUoutcome{}{}{t-1}
    \right].
\end{aligned}
\end{equation}
% 
According to Eq.~\eqref{eq:Q and R}, $\bm{Q}_t$ and $\bm{R}_t$ are matrices with columns of $\VUoutcome{}{}{s-1}$ and $\Voutcome{}{}{s} - \IMatT{s-1} \VUoutcome{}{}{s-1}$, when $s=1,\ldots,t$, respectively. Then, we denote by $\VUoutcome{\parallel}{}{t}$ the projection of $\VUoutcome{}{}{t}$ onto the space generated by the columns of $\bm{Q}_t$ and define $\VUoutcome{\perp}{}{t} = \VUoutcome{}{}{t} - \VUoutcome{\parallel}{}{t}$. We also define $\VAPC{}{t} = (\APC{0}{t}, \APC{1}{t}, \ldots, \APC{t-1}{t})^\top$ such that
% 
\begin{align}
    \label{eq:projection sum}
    \VUoutcome{\parallel}{}{t}
    =
    \sum_{s=0}^{t-1} \APC{s}{t} \VUoutcome{}{}{s}
    =
    \sum_{s=0}^{t-1} \APC{s}{t}
    \outcomeg{s}{}
    \big(
    \VoutcomeD{}{}{s}, \Mtreatment{}{}, \covar
    \big),
\end{align}
% 
where
% 
\begin{align}
    \label{eq:projection coefficients}
    \VAPC{}{t}
    =
    \left(
    \bm{Q}_t^\top \bm{Q}_t
    \right)^{-1}
    \bm{Q}_t^\top \VUoutcome{}{}{t}.
\end{align}
% 
Now, note that the available observation at any time $t$ implicitly reveals information about the fixed interference matrix $\IM$. To manage this intricate randomness, we define $\Gc_t$ as the $\sigma$-algebra generated by $\VoutcomeD{}{}{0}, \VoutcomeD{}{}{1}, \ldots, \VoutcomeD{}{}{t}$, $\Voutcome{}{}{1}, \ldots, \Voutcome{}{}{t}$, $\IMatMean{t}$, $\IMatT{0}, \ldots, \IMatT{t-1}$, $\Mtreatment{}{}$, and $\covar$. We then compute the conditional distribution of $\IM$ given $\Gc_t$. In this framework, conditioning on $\Gc_t$ is equivalent to conditioning on the event $\IM \bm{Q}_t = \bm{R}_t$. When conditioned on $\Gc_t$, the entries of both $\bm{Q}_t$ and $\bm{R}_t$ become deterministic values, leading to the following lemma.


\begin{lemma}
    \label{lm:conditional dist of IM}
    Fix $t$ and assume that $\bm{Q}_t$ is a full-row rank matrix. Then, for the conditional distribution of the fixed interference matrix $\IM$ given $\IM \bm{Q}_t=\bm{R}_t$, we have
    % 
        \begin{align}
        \label{eq:conditional dist of IM}
        \IM|_{\IM \bm{Q}_t=\bm{R}_t}
        \eqd
        \bm{R}_t
        \left(
        \bm{Q}_t^\top \bm{Q}_t
        \right)^{-1}
        \bm{Q}_t^\top
        +
        \widetilde{\IM} P^\perp.
        \end{align}
        % 
    where $\widetilde{\IM} \eqd \IM$ independent of $\IM$ and $P^\perp = (\I-P)$ that P denotes the orthogonal projector onto the column space of $\bm{Q}_t$.
\end{lemma}

The proof of Lemma~\ref{lm:conditional dist of IM} relies on the rotational invariance of the Gaussian distribution and utilizes Lemma~11 from \cite{bayati2011dynamics}. The proofs of this lemma and the subsequent lemma— which describes the distribution of $\Voutcome{}{}{t+1}$ given the event $\IM \bm{Q}_t = \bm{R}_t$— follow a similar approach to the proofs of Lemmas~2 and 3 in \cite{shirani2024causal}, and we refer readers to that work for detailed derivations.

 
\begin{lemma}
    \label{lm:conditional dist of outcome}
    Fix $t$ and assume that $\bm{Q}_t$ is a full-row rank matrix. The following holds for the conditional distribution of the vector $\Voutcome{}{}{t+1}$:
    % 
            \begin{align}
                \label{eq:conditional dist of outcome_nonsym}
                \Voutcome{}{}{t+1}\big|_{\Gc_t}
                \eqd
                &\;
                \widetilde{\IM} 
                \VUoutcome{\perp}{}{t}
                + \bm{R}_t \VAPC{}{t} + \IMatT{t} \VUoutcome{}{}{t},
            \end{align}
            % 
    where the matrix $\widetilde{\IM}$ is independent of $\IM$ and has the same distribution.
\end{lemma}

\subsection{Detailed Proof of Theorem~\ref{thm:Batch_SE}}
\label{apndx:big_lemma}
% 
Here, we first state Lemma~\ref{lm:Big lemma} which is an expanded version of Theorem~\ref{thm:Batch_SE}. To this end, we need some new notations. Specifically, for vectors $\Vec{u},\Vec{v} \in \R^m$, we define the scalar product $\pdot{\Vec{u}}{\Vec{v}}:= \frac{1}{m} \sum_{i=1}^m u_i v_i$. Also, considering \eqref{eq:state evolution}, for $t\geq 1$, we define
% 
\begin{equation}
    \label{eq:state evolution_fixed part}
    \begin{aligned}
        (\BVVO{}{}{1})^2
            &=
            \lim_{N\rightarrow \infty}
            \frac{\sigma^2}{N} \sum_{i=1}^N
            \outcomeg{0}{}\big(
            \outcomeD{}{i}{0},\Vtreatment{i}{},\Vcovar{i}
            \big)^2 < \infty
        \\
        (\BVVO{}{}{t+1})^2 &=
        \sigma^2 \E\left[
        \outcomeg{t}{} \big(\MAVO{}{}{t} + \MVVO{}{}{t} Z_t + \Houtcome{}{}{t-1}, \Vtreatment{}{},\Vcovar{}\big)^2
        \right].
    \end{aligned}
\end{equation}
% 


\begin{lemma}
    \label{lm:Big lemma}
    For a fixed $k\geq 2$ and a specified subpopulation of experimental units $\batch$, consider the following conditions. Given that Assumption~\ref{asmp:BL} holds, and both the subpopulation $\batch$ and the complete experimental population satisfy Assumption~\ref{asmp:weak_limits}, the following statements are valid for all time steps $t$ under the BSE equations defined in \eqref{eq:state evolution}:
    % 
    \begin{enumerate}[label=(\alph*)]
        \item \label{part:BL-a} For any function $\psi \in \poly{k}$, we have
        % 
        \begin{equation}
            \label{eq:BL-a}
            \begin{aligned}
                \lim_{N \rightarrow \infty}
                \frac{1}{\cardinality{\batch}} \sum_{i \in \batch}
                \psi\big(
                \outcomeD{}{i}{0},
                \outcomeD{}{i}{1},
                \ldots,
                \outcomeD{}{i}{t+1},
                \Vtreatment{i}{},\Vcovar{i}
                \big)
                \\
                \eqas
                \E
                \Big[
                \psi
                \big(
                \outcomeD{}{\batch}{0},
                \MAVO{}{\batch}{1}
                + \MVVO{}{}{1} Z_1
                + \Houtcome{\batch}{}{0},
                \ldots,
                \MAVO{}{\batch}{t+1}
                + \MVVO{}{}{t+1} Z_t
                + \Houtcome{\batch}{}{t},
                \Vtreatment{\batch}{}, \Vcovar{\batch}
                \big)
                \Big],
            \end{aligned}
        \end{equation}
        % 
        where $Z_1, \ldots, Z_t$ are standard Gaussian random variables.

        \item \label{part:BL-b} For all $0 \leq r\neq s \leq t$, the following equations hold and all limits exist, are bounded, and have degenerate distribution (i.e. they are constant random variables)
        % 
        \begin{subequations}
            \label{eq:BL-b}
            \begin{align}
                \label{eq:BL-b-1}
                \lim_{N \rightarrow \infty} 
                \frac{1}{N}
                \sum_{i=1}^N
                (\outcome{}{i}{r+1})^2
                &\eqas
                (\MVVO{}{}{r+1})^2
                \eqas
                \lim_{N \rightarrow \infty}
                \frac{\sigma^2+\sigma^2_{r}}{N} \sum_{i=1}^N (\Uoutcome{i}{}{r})^2
                \\
                \label{eq:BL-b-2}
                \lim_{N \rightarrow \infty} 
                \frac{1}{N}
                \sum_{i=1}^N
                \outcome{}{i}{r+1}\outcome{}{i}{s+1}
                &\eqas
                \lim_{N \rightarrow \infty}
                \frac{\sigma^2}{N} \sum_{i=1}^N \Uoutcome{i}{}{r}\Uoutcome{i}{}{s},
                \\
                \label{eq:BL-b-3}
                \lim_{N \rightarrow \infty}
                \frac{1}{N}
                \sum_{i=1}^N
                \big(\outcome{}{i}{r+1}
                - \IMatTv{i \cdot}{r} \VUoutcome{}{}{r}
                \big)^2
                &\eqas
                (\BVVO{}{}{r+1})^2
                \eqas
                \lim_{N \rightarrow \infty}
                \frac{\sigma^2}{N}
                \sum_{i=1}^N
                \left(\Uoutcome{i}{}{r}\right)^2,
                \\
                \label{eq:BL-b-4}
                \lim_{N \rightarrow \infty}
                \frac{1}{N}
                \sum_{i=1}^N
                \big(\outcome{}{i}{r+1}
                - \IMatTv{i \cdot}{r} \VUoutcome{}{}{r}
                \big)
                \big(\outcome{}{i}{s+1}
                - \IMatTv{i \cdot}{s} \VUoutcome{}{}{s}
                \big)
                &\eqas
                \lim_{N \rightarrow \infty}
                \frac{\sigma^2}{N}
                \sum_{i=1}^N
                \Uoutcome{i}{}{r} \Uoutcome{i}{}{s},
                \\
                \label{eq:BL-b-5}
                \lim_{N \rightarrow \infty}
                \frac{1}{N}
                \sum_{i=1}^N
                \outcome{}{i}{r+1}
                \big(\outcome{}{i}{s+1}
                - \IMatTv{i \cdot}{s} \VUoutcome{}{}{s}
                \big)
                &\eqas
                \lim_{N \rightarrow \infty}
                \frac{\sigma^2}{N}
                \sum_{i=1}^N
                \Uoutcome{i}{}{r} \Uoutcome{i}{}{s}.
            \end{align}
        \end{subequations}
        % 

        \item \label{part:BL-c} Letting $\MOG{t} = [\Voutcome{}{}{1}|\ldots|\Voutcome{}{}{t}]$, the following matrices are positive definite almost surely:
        % 
        \begin{align}
            \label{eq:BL-lower bound for perps}
            \lim_{N \rightarrow \infty} \frac{\bm{Q}_{t}^\top \bm{Q}_{t}}{N} \succ 0,
            \quad\quad\quad
            \lim_{N \rightarrow \infty} \frac{\MOG{t}^\top \MOG{t}}{N} 
            \succ 0.
        \end{align}
        % 
    \end{enumerate}
\end{lemma}
In the following section, we will provide a comprehensive explanation of the conditioning technique, which will be employed to establish the results presented in Lemma~\ref{lm:Big lemma}.

\noindent
\textbf{Proof.}
% 
For all $t$, we assume, without loss of generality, that the mapping $y \mapsto \outcomeg{t}{}(y, \Vtreatment{}{}, \Vcovar{})$ is a non-constant function with positive probability with respect to the randomness of $\Vtreatment{}{}$ and $\Vcovar{}$; otherwise, the result is trivial and does not require further analysis. We prove the result by induction on $t$.

\textbf{Step 1.} Let $t = 0$. By definition, the matrices $\bm{Q}_0$ and $\bm{R}_0$ are empty, and the $\sigma$-algebra $\Gc_0$ is generated by $\VoutcomeD{}{}{0}$, $\Mtreatment{}{}$, and $\covar$. As the induction base case, we establish Parts \ref{part:BL-a} and \ref{part:BL-b} for $t=0$ and Part~\ref{part:BL-c} for $t = 1$.
% 
\begin{enumerate}[label=(\alph*)]
    \item \label{item:BL-average limit} Conditioning on the values of $\VoutcomeD{}{}{0}$, $\Mtreatment{}{}$, $\covar$, and so on the value of $\VUoutcome{}{}{0} = \outcomeg{0}{}\big(\VoutcomeD{}{}{0},\Mtreatment{}{},\covar\big)$, the elements of $\Voutcome{}{}{1}$ are i.i.d. Gaussian random variables with zero mean and variance~$(\MVVO{}{}{1N})^2$:
    % 
    \begin{equation}
        \label{eq:BL-a0-Y1 stat}
        \begin{aligned}
            (\MVVO{}{}{1N})^2
            &:=
            \Var
            \left[
            \outcome{}{i}{1}
            \Big|
            \VUoutcome{}{}{0}
            \right]
            =
            \frac{\sigma^2 + \sigma^2_0}{N}
            \sum_{i=1}^N
            \outcomeg{0}{} \left(\outcomeD{}{i}{0},\Vtreatment{i}{},\Vcovar{i}
            \right)^2.
        \end{aligned}
    \end{equation}
    % 
    Then, Assumption~\ref{asmp:BL}-\ref{asmp:BL-bound on initials} implies that the value of $(\MVVO{}{}{1N})^2$ is bounded independent from $N$, and
    % 
    \begin{equation}
        \label{eq:BL-a0-Y1 stats limits}
        \begin{aligned}
            (\MVVO{}{}{1})^2 :=
            \lim_{N\rightarrow \infty} (\MVVO{}{}{1N})^2.
        \end{aligned}
    \end{equation}
    % 
    
    Now, let $Z$ denote a standard Gaussian random variable. Fixing $l \geq 1$, it is straightforward to show that
    % 
    \begin{equation}
    \begin{aligned}
    \label{eq:BL-proof-a0-1}
        \E
        \left[
        \big|\outcome{}{i}{1}\big|^l
        \big|
        \VUoutcome{}{}{0}
        \right]
        &=
        \E
        \left[
        \big|\MVVO{}{}{1N} Z\big|^l
        \big|
        \VUoutcome{}{}{0}
        \right] \leq c,
    \end{aligned}
    \end{equation}
    % 
    where $c$ is a constant independent of $N$ and might alter in different lines.

    We next focus on the second term on the right-hand side of Eq.~\eqref{eq:apndx_outcome_function} and define
    % 
    \begin{equation}
        \label{eq:BL-a0-Y1 stat-2}
        \begin{aligned}
            \MAVO{}{i}{1N}
            &:=
            \frac{1}{N}
            \sum_{j=1}^N
            (\mu^{ij} + \mu_0^{ij})
            \outcomeg{0}{} \big(\outcomeD{}{j}{0},\Vtreatment{j}{},\Vcovar{j}\big),
            \quad\quad\quad
            i \in [N].
        \end{aligned}
    \end{equation}
    In view of Assumption~\ref{asmp:weak_limits}, we can apply Theorem~\ref{thm:SLLN-2}. Consequently, for all $i$, we can write
    % 
    \begin{equation}
        \label{eq:BL-a0-Y1 stat-2_limit}
        \begin{aligned}
            \lim_{N \rightarrow \infty} \MAVO{}{i}{1N}
            \eqas
            \E \left[ (\MIM{}{i}+\MIM{0}{i}) \outcomeg{0}{} \big(\outcomeD{}{}{0},\Vtreatment{}{},\Vcovar{}\big) \right] = (\bar{\mu}^i + \bar{\mu}^i_0) \E \left[ \outcomeg{0}{} \big(\outcomeD{}{}{0},\Vtreatment{}{},\Vcovar{}\big) \right] = \MAVO{}{i}{1} < \infty.
        \end{aligned}
    \end{equation}
    % 
    Above, we let $\bar{\mu}^i := \E[\MIM{}{i}]$ and $\bar{\mu}^i_0 := \E[\MIM{0}{i}]$, where $\MIM{}{i} \sim p_{\mu^i}$ and $\MIM{0}{i} \sim p_{\mu^i_0}$. Additionally, $\outcomeD{}{}{0}$, $\Vtreatment{}{}$, and $\Vcovar{}$ represent the weak limits of the initial outcomes, treatment allocations, and covariates for the entire population as outlined in Assumption~\ref{asmp:weak_limits}, respectively. In this context, $\bar{\mu}^i+\bar{\mu}^i_0$ determines the average interaction level of unit $i$ at time $0$.
    % 
    
    Also, note that \eqref{eq:BL-a0-Y1 stat-2_limit} yields the boundedness of $\MAVO{}{i}{1N}$ for all $i$ independent of $N$.
    % 
    Now, using Assmption~\ref{asmp:weak_limits} and the fact that $\psi \in \poly{k}$, for $\kappa > 0$, we have the following, 
    % 
    \begin{align*}
        \frac{1}{\cardinality{\batch}} \sum_{i \in \batch}
        \E
        \left[
        \Big|
        \psi\big(
        \outcomeD{}{i}{0},
        \outcome{}{i}{1},
        \Vtreatment{i}{}, \Vcovar{i}, \MAVO{}{i}{1N}
        \big)
        -
        \E_{\IM,\IMatT{0}}
        \left[
        \psi\big(
        \outcomeD{}{i}{0},
        \outcome{}{i}{1},
        \Vtreatment{i}{}, \Vcovar{i}, \MAVO{}{i}{1N}
        \big)
        \right]
        \Big|^{2+\kappa}
        \right]
        \leq
        c \cardinality{\batch}^{\kappa/2},
    \end{align*}
    % 
    where $\E_{\IM,\IMatT{0}}$ is the expectation with respect to the randomness of the interference matrices $\IM,\IMatT{0}$ and $c$ is a constant independent of $N$. Then, applying the Strong Law of Large Numbers (SLLN) for triangular arrays in Theorem~\ref{thm:SLLN}, we obtain the following result:
    % 
    \begin{align}
        \label{eq:BL-a0-SLLN}
        \lim_{N \rightarrow \infty}
        \frac{1}{\cardinality{\batch}} \sum_{i \in \batch}
        \Big(
        \psi\big(
        \outcomeD{}{i}{0},
        \outcome{}{i}{1},
        \Vtreatment{i}{}, \Vcovar{i}, \MAVO{}{i}{1N}
        \big)
        -
        \E_{\IM,\IMatT{0}}
        \left[
        \psi\big(
        \outcomeD{}{i}{0},
        \outcome{}{i}{1},
        \Vtreatment{i}{}, \Vcovar{i}, \MAVO{}{i}{1N}
        \big)
        \right]
        \Big)
        \eqas
        0.
    \end{align}
    %
    On the other hand, employing the dominated convergence theorem, e.g., Theorem 16.4 in \cite{billingsley2008probability}, allows us to interchange the limit and the expectation in view of the fact that $\psi \in \poly{k}$. We also utilize the continuous mapping theorem, e.g., Theorem 2.3 in \cite{van2000asymptotic}, to pass the limit through the function. As a result, considering $\outcome{}{i}{1} \eqd \MVVO{}{}{1N} Z$, we get
    % 
    \begin{align}
        \label{eq:BL-a0-limit to the function}
        \lim_{N \rightarrow \infty}
        \frac{1}{\cardinality{\batch}} \sum_{i \in \batch}
        \E_{\IM,\IMatT{0}}
        \left[
        \psi\big(
        \outcomeD{}{i}{0},
        \outcome{}{i}{1},
        \Vtreatment{i}{}, \Vcovar{i}, \MAVO{}{i}{1N}
        \big)
        \right]
        \eqas
        \lim_{N \rightarrow \infty}
        \frac{1}{\cardinality{\batch}} \sum_{i \in \batch}
        \E_{Z}
        \left[
        \psi\big(
        \outcomeD{}{i}{0},
        \MVVO{}{}{1} Z,
        \Vtreatment{i}{}, \Vcovar{i}, \MAVO{}{i}{1}
        \big)
        \right].
    \end{align}
    % 
    Then, applying Theorem~\ref{thm:SLLN-2} for the function $f(\outcomeD{}{i}{0}, \Vtreatment{i}{}, \Vcovar{i}, \MAVO{}{i}{1}) = \E_{Z}\left[\psi\big(\outcomeD{}{i}{0}, \MVVO{}{}{1} Z, \Vtreatment{i}{}, \Vcovar{i}, \MAVO{}{i}{1} \big)\right]$, we can write
    % 
    \begin{equation}
        \label{eq::BL-proof-a0-dynamics}
        \begin{aligned}
            \lim_{N \rightarrow \infty}
            \frac{1}{\cardinality{\batch}} \sum_{i \in \batch}
            \psi\big(
            \outcomeD{}{i}{0},
            \outcome{}{i}{1},
            \Vtreatment{i}{}, \Vcovar{i}, \MAVO{}{i}{1N}
            \big)
            \eqas
            \E
            \Big[
            \psi
            \big(
            \outcomeD{}{\batch}{0},
            \MVVO{}{}{1} Z,
            \Vtreatment{\batch}{}, \Vcovar{\batch}, \MAVO{}{\batch}{1}
            \big)
            \Big],
        \end{aligned}
    \end{equation}
    % 
    where $\outcomeD{}{\batch}{0},\;\Vtreatment{\batch}{},\;\Vcovar{\batch},\; \MAVO{}{\batch}{1}$ represent the weak limits of $\outcomeD{}{i}{0},\;\Vtreatment{i}{},\;\Vcovar{i},\; \MAVO{}{i}{1}$ over the subpopulation units, as specified in Assumption~\ref{asmp:weak_limits} and
    % 
    \begin{equation}
        \label{eq:BL-a0-Y1 stat-2_limit_without_n}
        \begin{aligned}
            \MAVO{}{\batch}{1}
            \eqas
            \E \left[ (\MIM{}{\batch}+\MIM{0}{\batch}) \outcomeg{0}{} \big(\outcomeD{}{}{0},\Vtreatment{}{},\Vcovar{}\big) \right],
        \end{aligned}
    \end{equation}
    %
    where $\MIM{}{\batch} + \MIM{0}{\batch}$ captures the weak limit of the average interference level for units in the subpopulation.
    
    In Eq.~\eqref{eq::BL-proof-a0-dynamics}, the function $f$ is within $\poly{k}$, since $\psi \in \poly{k}$ and expectation is a linear operator. It is important to note that above, $Z$ is independent of $\outcomeD{}{\batch}{0},\; \Vtreatment{\batch}{},\; \Vcovar{\batch}, \MIM{}{\batch}$, and $\MIM{0}{\batch}$.
    This is true because the randomness of $Z$ arises from the interference matrices which are assumed to be independent of everything in the model, see Assumption~\ref{asmp:apndx_Gaussian Interference Matrice}.

    Now, we use Eq.~\eqref{eq::BL-proof-a0-dynamics} to derive the main result. Fix an arbitrary function $\psi \in \poly{k}$ and based on Eqs.~\eqref{eq:apndx_outcome_function} and \eqref{eq:apndx_outcome_function_WOD}, define the function $\widetilde\psi$ such that
    % 
    \begin{align}
        \label{eq:BL-a0-function_re_def}
        \psi\big(\outcomeD{}{i}{0},
        \outcomeD{}{i}{1},
        \Vtreatment{i}{}, \Vcovar{i}
        \big)
        =
        \psi\left(
        \outcomeD{}{i}{0},
        \outcome{}{i}{1} +
        \MAVO{}{i}{1N} +
        \outcomeh{0}{} \big(\outcomeD{}{i}{0},\Vtreatment{i}{},\Vcovar{i}\big),
        \Vtreatment{i}{}, \Vcovar{i}
        \right)
        =
        \widetilde{\psi}\left(
        \outcomeD{}{i}{0},
        \outcome{}{i}{1}, 
        \Vtreatment{i}{}, \Vcovar{i}, \MAVO{}{i}{1N}
        \right).
    \end{align}
    % 
    The function $\widetilde\psi$ is within $\poly{k}$ by Assumption~\ref{asmp:BL}. Then, applying \eqref{eq::BL-proof-a0-dynamics} for the function $\widetilde{\psi}$, we obtain,
    % 
    \begin{equation}
        \label{eq::BL-proof-a0-dynamics2}
        \begin{aligned}
            \lim_{N \rightarrow \infty}
            \frac{1}{\cardinality{\batch}} \sum_{i \in \batch}
            \psi\big(\outcomeD{}{i}{0},
            \outcomeD{}{i}{1},
            \Vtreatment{i}{}, \Vcovar{i}
            \big)
            \eqas
            \E \left[
            \psi\left(
            \outcomeD{}{\batch}{0},
            \MAVO{}{\batch}{1}
            +
            \MVVO{}{}{1} Z + 
            \Houtcome{\batch}{}{0}
            ,
            \Vtreatment{\batch}{}, \Vcovar{\batch}
            \right)
            \right]
        \end{aligned}
    \end{equation}
    % 
    where $\Houtcome{\batch}{}{0} = \outcomeh{0}{} \big(\outcomeD{}{\batch}{0},\Vtreatment{\batch}{},\Vcovar{\batch}\big)$ is a random variable independent of $Z$. 
    
    In the second step of the induction, we also require the following results. Note that the result in \eqref{eq::BL-proof-a0-dynamics2} represents a specific instance of the more comprehensive result \eqref{eq:BL-proof-a0-dynamics-with-eps}. These results can be derived by following the same procedure outlined above.
    % 
    \begin{equation}
        \label{eq:BL-proof-a0-dynamics-with-eps0}
        \begin{aligned}
            \lim_{N \rightarrow \infty}
            \frac{1}{N} \sum_{i = 1}^N
            &\psi\big(
            \outcomeD{}{i}{0},
            \outcomeD{}{i}{1},
            \outcome{}{i}{1},
            \outcome{}{i}{1}
            - \IMatTv{i \cdot}{0} \VUoutcome{}{}{0},
            \Vtreatment{i}{},\Vcovar{i},
            \mu^{ji}, \mu^{ji}_r
            \big)
            \\
            \eqas
            \E
            \Big[
            &\psi
            \big(
            \outcomeD{}{}{0},
            \MAVO{}{}{1}
            + \MVVO{}{}{1} Z
            + \Houtcome{}{}{0},
            \MVVO{}{}{1} Z,
            % \BAVO{}{}{1}
            % +
            \BVVO{}{}{1} Z',
            \Vtreatment{}{},\Vcovar{},
            \MIM{}{j}, \MIM{r}{j}
            \big)
            \Big],
        \end{aligned}
    \end{equation}
    % 
    as well as
    % 
    \begin{equation}
        \label{eq:BL-proof-a0-dynamics-with-eps}
        \begin{aligned}
            \lim_{N \rightarrow \infty}
            \frac{1}{\cardinality{\batch}} \sum_{i \in \batch}
            &\psi\big(
            \outcomeD{}{i}{0},
            \outcomeD{}{i}{1},
            \outcome{}{i}{1},
            \outcome{}{i}{1}
            - \IMatTv{i \cdot}{0} \VUoutcome{}{}{0},
            \Vtreatment{i}{},\Vcovar{i},
            \bar{\mu}^i, \bar{\mu}^i_r
            \big)
            \\
            \eqas
            \E
            \Big[
            &\psi
            \big(
            \outcomeD{}{\batch}{0},
            \MAVO{}{\batch}{1}
            + \MVVO{}{}{1} Z
            + \Houtcome{\batch}{}{0},
            \MVVO{}{}{1} Z,
            % \BAVO{}{}{1}
            % +
            \BVVO{}{}{1} Z',
            \Vtreatment{\batch}{},\Vcovar{\batch},
            \MIM{}{\batch}, \MIM{r}{\batch}
            \big)
            \Big],
        \end{aligned}
    \end{equation}
    % % 
    for any $j$ and all $r \in [T]_0$; here, $Z'$ is a standard Gaussian random variable and $\IMatTv{i \cdot}{0}$ detnoes the $i^{th}$ row of the time-dependent interference matrix $\IMatT{0}$.    


    \item By \eqref{eq:BL-a0-Y1 stats limits} and \eqref{eq:BL-proof-a0-dynamics-with-eps0}, we get
    % 
    \begin{equation}
        \label{eq:BL-proof-b0-1}
        \begin{aligned}
            \lim_{N \rightarrow \infty}
            \frac{1}{N}
            \sum_{i=1}^N
            \big(\outcome{}{i}{1}
            \big)^2
            &\eqas
            (\MVVO{}{}{1})^2
            =
            \lim_{N \rightarrow \infty}
            \frac{\sigma^2+\sigma_0^2}{N}
            \sum_{i=1}^N
            \left(\Uoutcome{i}{}{0}\right)^2.
        \end{aligned}
    \end{equation}
    % 
    Similarly, by \eqref{eq:state evolution_fixed part} and \eqref{eq:BL-proof-a0-dynamics-with-eps0}, we can write % 
    \begin{equation}
        \label{eq:BL-proof-b0-2}
        \begin{aligned}
            \lim_{N \rightarrow \infty}
            \frac{1}{N}
            \sum_{i=1}^N
            \big(\outcome{}{i}{1}
            - \IMatTv{i \cdot}{0} \VUoutcome{}{}{0}
            \big)^2
            &\eqas
            (\BVVO{}{}{1})^2
            =
            \lim_{N \rightarrow \infty}
            \frac{\sigma^2}{N}
            \sum_{i=1}^N
            \left(\Uoutcome{i}{}{0}\right)^2.
        \end{aligned}
    \end{equation}
    % 
    Finally, given $\VUoutcome{}{}{0}$, we know that $\IMatv{i \cdot} \VUoutcome{}{}{0}$ and $\IMatTv{i \cdot}{0} \VUoutcome{}{}{0}$ are statistically independent. Then, applying \eqref{eq:BL-proof-a0-dynamics-with-eps0} together with \eqref{eq:BL-proof-b0-2}, we obtain the following result:
    % 
    \begin{equation*}
        % \label{eq:BL-proof-b0-3}
        \begin{aligned}
            \lim_{N \rightarrow \infty}
            \frac{1}{N}
            \sum_{i=1}^N
            \outcome{}{i}{1}
            \big(\outcome{}{i}{1}
            - \IMatTv{i \cdot}{0} \VUoutcome{}{}{0}
            \big)
            =
            \lim_{N \rightarrow \infty}
            \frac{1}{N}
            \sum_{i=1}^N
            \big(\IMatv{i \cdot} \VUoutcome{}{}{0}
            +
            \IMatTv{i \cdot}{0} \VUoutcome{}{}{0}
            \big)
            \big(\IMatv{i \cdot} \VUoutcome{}{}{0}
            \big)
            &\eqas
            (\BVVO{}{}{1})^2
            =
            \lim_{N \rightarrow \infty}
            \frac{\sigma^2}{N}
            \sum_{n=1}^N
            \left(\Uoutcome{n}{}{0}\right)^2.
        \end{aligned}
    \end{equation*}
    % 
    where $\IMatv{i \cdot}$ and $\IMatTv{i \cdot}{0}$ denote the $i^{th}$ row of the fixed interference matrix $\IMat{}$ and time-dependent interference matrix $\IMatT{}$.
    

    \item For $t=1$, the matrix $\bm{Q}_1$ is equal to the vector $\VUoutcome{}{}{0}$ and $\bm{V}_1$ is equal to the vector $\VoutcomeD{}{}{1}$. By Assumption~\ref{asmp:BL}-\ref{asmp:BL-bound on initials} and \eqref{eq:BL-proof-b0-1}, we have
    \begin{align*}
        \lim_{N \rightarrow \infty} \frac{\bm{Q}_1^\top \bm{Q}_1}{N}
        =
        \lim_{N \rightarrow \infty}
        \frac{1}{N}
        \sum_{i=1}^N \left(\Uoutcome{i}{}{0}\right)^2 > 0,
        \quad
        \lim_{N \rightarrow \infty} \frac{\MOG{1}^\top \MOG{1}}{N} 
        =
        \lim_{N \rightarrow \infty} \pdot{\Voutcome{}{}{1}}{\Voutcome{}{}{1}}
        > 0.
    \end{align*}
    % 
    \end{enumerate}


    \textbf{Induction Hypothesis (IH).} Now, we assume that the following results hold true:
    % 
    \begin{equation}
        \label{eq:BL-proof-IH-a}
        \tag{IH-1}
        \begin{aligned}
            \lim_{N \rightarrow \infty}
            \frac{1}{N} \sum_{i = 1}^N
            &\psi\big(
            \outcomeD{}{i}{0},
            \outcomeD{}{i}{1},
            \ldots,
            \outcomeD{}{i}{t},
            \outcome{}{i}{1},
            \ldots,
            \outcome{}{i}{t},
            \outcome{}{i}{1}
            - \IMatTv{i \cdot}{0} \VUoutcome{}{}{0},
            \ldots,
            \outcome{}{i}{t}
            - \IMatTv{i \cdot}{t-1} \VUoutcome{}{}{t},
            \Vtreatment{i}{},\Vcovar{i},
            \mu^{ji}, \mu^{ji}_r
            \big)
            \\
            \eqas
            \E
            \Big[
            &\psi
            \big(
            \outcomeD{}{}{0},
            \MAVO{}{}{1}
            + \MVVO{}{}{1} Z_1
            + \Houtcome{}{}{0},
            \ldots,
            \MAVO{}{}{t}
            + \MVVO{}{}{t} Z_t
            + \Houtcome{}{}{t-1},
            \MVVO{}{}{1} Z_1,
            \ldots,
            \MVVO{}{}{t} Z_t,
            \BVVO{}{}{1} Z'_1,
            \ldots,
            \BVVO{}{}{t} Z'_t,
            \Vtreatment{}{},\Vcovar{},
            \MIM{}{j}, \MIM{r}{j}
            \big)
            \Big],
        \end{aligned}
    \end{equation}
    % 
    and
    %
    \begin{equation}
        \label{eq:BL-proof-IH-a1}
        \tag{IH-2}
        \begin{aligned}
            \lim_{N \rightarrow \infty}
            \frac{1}{\cardinality{\batch}} \sum_{i \in \batch}
            &\psi\big(
            \outcomeD{}{i}{0},
            \outcomeD{}{i}{1},
            \ldots,
            \outcomeD{}{i}{t},
            \outcome{}{i}{1},
            \ldots,
            \outcome{}{i}{t},
            \outcome{}{i}{1}
            - \IMatTv{i \cdot}{0} \VUoutcome{}{}{0},
            \ldots,
            \outcome{}{i}{t}
            - \IMatTv{i \cdot}{t-1} \VUoutcome{}{}{t},
            \Vtreatment{i}{},\Vcovar{i},
            \bar{\mu}^i, \bar{\mu}^i_r
            \big)
            \\
            \eqas
            \E
            \Big[
            &\psi
            \big(
            \outcomeD{}{\batch}{0},
            \MAVO{}{\batch}{1}
            + \MVVO{}{}{1} Z_1
            + \Houtcome{\batch}{}{0},
            \ldots,
            \MAVO{}{\batch}{t}
            + \MVVO{}{}{t} Z_t
            + \Houtcome{\batch}{}{t-1},
            \MVVO{}{}{1} Z_1,
            \ldots,
            \MVVO{}{}{t} Z_t,
            \BVVO{}{}{1} Z'_1,
            \ldots,
            \BVVO{}{}{t} Z'_t,
            \Vtreatment{\batch}{},\Vcovar{\batch},
            \MIM{}{\batch}, \MIM{r}{\batch}
            \big)
            \Big].
        \end{aligned}
    \end{equation}
    % 
    Also, for $0 \leq s \neq r \leq t-1$, we have
    %
    \begin{align}
            \label{eq:BL-proof-IH-b-1}
            \tag{IH-3}
            \lim_{N \rightarrow \infty} 
            \frac{1}{N}
            \sum_{i=1}^N
            (\outcome{}{i}{r+1})^2
            &\eqas
            (\MVVO{}{}{r+1})^2
            \eqas
            \lim_{N \rightarrow \infty}
            \frac{\sigma^2+\sigma^2_{r}}{N} \sum_{i=1}^N (\Uoutcome{i}{}{r})^2
            \\
            \label{eq:BL-proof-IH-b-2}
            \tag{IH-4}
            \lim_{N \rightarrow \infty} 
            \frac{1}{N}
            \sum_{i=1}^N
            \outcome{}{i}{r+1}\outcome{}{i}{s+1}
            &\eqas
            \lim_{N \rightarrow \infty}
            \frac{\sigma^2}{N} \sum_{i=1}^N \Uoutcome{i}{}{r}\Uoutcome{i}{}{s},
            \\
            \label{eq:BL-proof-IH-b-3}
            \tag{IH-5}
            \lim_{N \rightarrow \infty}
            \frac{1}{N}
            \sum_{i=1}^N
            \big(\outcome{}{i}{r+1}
            - \IMatTv{i \cdot}{r} \VUoutcome{}{}{r}
            \big)^2
            &\eqas
            (\BVVO{}{}{r+1})^2
            \eqas
            \lim_{N \rightarrow \infty}
            \frac{\sigma^2}{N}
            \sum_{i=1}^N
            \left(\Uoutcome{i}{}{r}\right)^2,
            \\
            \label{eq:BL-proof-IH-b-4}
            \tag{IH-6}
            \lim_{N \rightarrow \infty}
            \frac{1}{N}
            \sum_{i=1}^N
            \big(\outcome{}{i}{r+1}
            - \IMatTv{i \cdot}{r} \VUoutcome{}{}{r}
            \big)
            \big(\outcome{}{i}{s+1}
            - \IMatTv{i \cdot}{s} \VUoutcome{}{}{s}
            \big)
            &\eqas
            \lim_{N \rightarrow \infty}
            \frac{\sigma^2}{N}
            \sum_{i=1}^N
            \Uoutcome{i}{}{r} \Uoutcome{i}{}{s},
            \\
            \label{eq:BL-proof-IH-b-5}
            \tag{IH-7}
            \lim_{N \rightarrow \infty}
            \frac{1}{N}
            \sum_{i=1}^N
            \outcome{}{i}{r+1}
            \big(\outcome{}{i}{s+1}
            - \IMatTv{i \cdot}{s} \VUoutcome{}{}{s}
            \big)
            &\eqas
            \lim_{N \rightarrow \infty}
            \frac{\sigma^2}{N}
            \sum_{i=1}^N
            \Uoutcome{i}{}{r} \Uoutcome{i}{}{s}.
    \end{align}
    % 
    Finally, the following condition holds almost surely:
    % 
    \begin{align}
            \label{eq:BL-proof-IH-c}
            \tag{IH-8}
            \lim_{N \rightarrow \infty} \frac{\MOG{t-1}^\top \MOG{t-1}}{N} 
            \succ 0.
    \end{align}

    \textbf{Step 2.} To establish the second step of the induction, we prove the assertions in reverse order, starting with Part (c), followed by Part (b), and concluding with Part (a).
    
    
    \begin{enumerate}[label=(\alph*)]
        \item[(c)] We begin the second step by applying \eqref{eq:BL-proof-IH-a} to the function $g_s\big(\outcomeD{}{i}{s},\Vtreatment{i}{},\Vcovar{i}\big) g_r\big(\outcomeD{}{i}{r},\Vtreatment{i}{},\Vcovar{i}\big)$, for $1 \leq r,s \leq t$. Precisely, by Assumption~\ref{asmp:BL}-\ref{asmp:BL-bound on initials} as well as \eqref{eq:BL-proof-IH-a}, we get
        % 
        \begin{equation}
            \label{eq:BL-proof-ct-1}
            \begin{aligned}
            \lim_{N \rightarrow \infty}
            \frac{1}{N}
            \sum_{i=1}^N
            (\Uoutcome{i}{}{0})^2
            &\eqas
            \E\left[
            \outcomeg{0}{}(\outcomeD{}{}{0},\Vtreatment{}{},\Vcovar{})^2\big)
            \right]
            =
            \frac{(\MVVO{}{}{1})^2}{\sigma^2 + \sigma_0^2} > 0,
            \\
            \lim_{N \rightarrow \infty}
            \frac{1}{N}
            \sum_{i=1}^N
            \Uoutcome{i}{}{0} \Uoutcome{i}{}{s}
            &\eqas
            \E\left[
            \outcomeg{0}{}(\outcomeD{}{}{0},\Vtreatment{}{},\Vcovar{}) \outcomeg{s}{}\big(\MAVO{}{}{s}
            +
            \MVVO{}{}{s} Z_{s} + 
            \Houtcome{}{}{s-1},
            \Vtreatment{}{},\Vcovar{}\big)
            \right],
            \\
            \lim_{N \rightarrow \infty}
            \frac{1}{N}
            \sum_{i=1}^N
            \Uoutcome{i}{}{s} \Uoutcome{i}{}{r}
            &\eqas
            \E\Big[
            \outcomeg{s}{}\big(\MAVO{}{}{s}
            +
            \MVVO{}{}{s} Z_{s} + 
            \Houtcome{}{}{s-1},
            \Vtreatment{}{},\Vcovar{}\big) \outcomeg{r}{}\big(\MAVO{}{}{r}+
            \MVVO{}{}{r} Z_{r} + 
            \Houtcome{}{}{r-1},
            \Vtreatment{}{},\Vcovar{}\big)
            \Big].
            \end{aligned}
        \end{equation}
        % 
        Now, let $\Vec{u} = (u_1,\ldots,u_t)^\top \in \R^t$ be a non-zero vector. By Assumption~\ref{asmp:BL}-\ref{asmp:BL-bound on initials} and \eqref{eq:BL-proof-ct-1}, we have
        % 
        \begin{equation}
            \label{eq:BL-proof-ct-2}
            \begin{aligned}
                \Vec{u}^\top \left(\lim_{N \rightarrow \infty} \frac{\bm{Q}_t^\top \bm{Q}_t}{N}\right) \Vec{u}
                &=
                \lim_{N \rightarrow \infty} \Vec{u}^\top \frac{\bm{Q}_t^\top \bm{Q}_t}{N} \Vec{u}
                \\
                &\eqas
                \E\left[
                \left(
                u_1
                \outcomeg{0}{}(\outcomeD{}{}{0},\Vtreatment{}{},\Vcovar{})
                +
                \sum_{s=1}^{t-1}
                u_{s+1} 
                \outcomeg{s}{}\big(\MAVO{}{}{s}
                +
                \MVVO{}{}{s} Z_{s} + 
                \Houtcome{}{}{s-1},
                \Vtreatment{}{},\Vcovar{}\big)
                \right)^2
                \right] \geq 0.
            \end{aligned}
        \end{equation}
        % 
        We show that the inequality in \eqref{eq:BL-proof-ct-2} is strict. To this end, note that $\Vec{u}$ is a non-zero vector, and there exists some $1\leq i \leq t$ such that $u_i \neq 0$. Whenever $u_1 \neq 0 = u_2 = \ldots = u_t$, the result is immediate. Otherwise, recall that $y \mapsto \outcomeg{s}{}(y,\Vtreatment{}{},\Vcovar{})$ is a non-constant function with a positive probability with respect to $(\Vtreatment{}{},\Vcovar{})$; consequently, the mapping $(y_0,\ldots,y_{t-1}) \mapsto \sum_{s=0}^{t-1} u_s \outcomeg{s}{}\big(y_s,\Vtreatment{}{},\Vcovar{}\big)$ is a non-constant function as well. Considering $\Houtcome{}{}{s} = \outcomeh{s}{} \big(\MAVO{}{}{s} + \MVVO{}{}{s} Z_{s} + \Houtcome{}{}{s-1},\Vtreatment{}{},\Vcovar{}\big)$ and $\Houtcome{}{}{0} = \outcomeh{0}{} \big(\outcomeD{}{}{0},\Vtreatment{}{},\Vcovar{}\big)$, the randomness of $u_1\outcomeg{0}{}(\outcomeD{}{}{0},\Vtreatment{}{},\Vcovar{}) + \sum_{s=1}^{t-1} u_{s+1} \outcomeg{s}{}\big(\MAVO{}{}{s} + \MVVO{}{}{s} Z_{s} + \Houtcome{}{}{s-1}, \Vtreatment{}{},\Vcovar{}\big)$ comes solely from $\outcomeD{}{}{0}, Z_1, \ldots, Z_{t-1}, \Vtreatment{}{}, \Vcovar{}$; as a result, there exists a measurable continuous function $\phi$ (which depends on $\Vec{u}, \outcomeg{0}{}, \ldots, \outcomeg{t-1}{}, \outcomeh{0}{}, \ldots, \outcomeh{t-1}{}$) such that we can rewrite the right-hand side of \eqref{eq:BL-proof-ct-2} as follows, 
        % 
        \begin{equation*}
            \begin{aligned}
                &\;\E\left[
                \left(
                u_1
                \outcomeg{0}{}(\outcomeD{}{}{0},\Vtreatment{}{},\Vcovar{})
                +
                \sum_{s=1}^{t-1}
                u_{s+1} 
                \outcomeg{s}{}\big(\MAVO{}{}{s}
                +
                \MVVO{}{}{s} Z_{s} + 
                \Houtcome{}{}{s-1},
                \Vtreatment{}{},\Vcovar{}\big)
                \right)^2
                \right]
                \\
                = 
                &\;\E\left[
                \phi\left(
                \outcomeD{}{}{0}, \MAVO{}{}{1}, \ldots, \MAVO{}{}{t-1}, \MVVO{}{}{1} Z_1, \ldots, \MVVO{}{}{t-1} Z_{t-1}, \Vtreatment{}{}, \Vcovar{}
                \right)^2
                \right]
                \\
                =
                &\;\E\left[
                \E\left[
                \phi\left(
                y_0, \MAVO{}{}{1}, \ldots, \MAVO{}{}{t-1}, \MVVO{}{}{1} Z_1, \ldots, \MVVO{}{}{t-1} Z_{t-1}, \Vtreatment{}{}, \Vcovar{}
                \right)^2
                \Bigg|
                \outcomeD{}{}{0} = y_0
                \right]
                \right],
            \end{aligned}
        \end{equation*}
        % 
        where in the last equality we used the tower property of conditional expectations. Then, it suffices to show that the random variable $\phi\left( y_0, \MAVO{}{}{1}, \ldots, \MAVO{}{}{t-1}, \MVVO{}{}{1} Z_1, \ldots, \MVVO{}{}{t-1} Z_{t-1}, \Vtreatment{}{}, \Vcovar{} \right)$ has a non-degenerate distribution. To obtain that, by \eqref{eq:BL-proof-IH-a}, it is straightforward to obtain the following, 
        % 
        \begin{equation*}
            \begin{aligned}
                \Cov\left[\left(
                \MVVO{}{}{1} Z_{1},
                \ldots, \MVVO{}{}{t-1} Z_{t-1} \right)\right]
                \eqas
                \lim_{N \rightarrow \infty} \frac{\MOG{t-1}^\top \MOG{t-1}}{N},
            \end{aligned}
        \end{equation*}
        % 
        which is positive definite by \eqref{eq:BL-proof-IH-c}, and the proof of the first claim is complete.
        
        To proceed to the proof of the second claim, for $1\leq r,s \leq t$, let us denote
        % 
        \begin{align*}
            v_{r,s}
            :=
            \left[
            \frac{\MOG{t}^\top \MOG{t}}{N} 
            \right]^{r,s}
            =
            \frac{\Voutcome{}{\top}{r} \Voutcome{}{}{s}}{N}. 
        \end{align*}
        % 
        By \eqref{eq:BL-proof-IH-b-2}, whenever $r\neq s$, we can write
        % 
        \begin{align*}
            \lim_{N \rightarrow \infty}
            v_{r,s}
            \eqas
            \lim_{N \rightarrow \infty}
            \frac{\sigma^2}{N} \sum_{i=1}^N \Uoutcome{i}{}{r-1}\Uoutcome{i}{}{s-1},
        \end{align*}
        % 
        and if $r=s$, by \eqref{eq:BL-proof-IH-b-1}, we have
        % 
        \begin{align*}
            \lim_{N \rightarrow \infty}
            v_{r,r}
            \eqas
            \lim_{N \rightarrow \infty}
            \frac{\sigma^2+\sigma_r^2}{N} \sum_{i=1}^N (\Uoutcome{i}{}{r-1})^2.
        \end{align*}
        % 
        Then, the result follows directly, as we have just established the almost sure positive definiteness of $\bm{Q}_t$.
        % 
        \begin{corollary}
            \label{cl:alpha is bounded}
            The vector $\VAPC{}{t}$ defined in \eqref{eq:projection coefficients} has a finite limit as $N \rightarrow \infty$.
        \end{corollary}
        % 
        Proof. By \eqref{eq:projection coefficients}, we can write
        \begin{align}
            \label{eq:cl-alpha is finite}
            \lim_{N \rightarrow \infty} \VAPC{}{t}
            =
            \lim_{N \rightarrow \infty} \left(
            \bm{Q}_t^\top \bm{Q}_t
            \right)^{-1}
            \bm{Q}_t^\top \VUoutcome{}{}{t}
            =
            \lim_{N \rightarrow \infty} \left(
            \frac{\bm{Q}_t^\top \bm{Q}_t}{N}
            \right)^{-1}
            \lim_{N \rightarrow \infty}
            \frac{\bm{Q}_t^\top \VUoutcome{}{}{t}}{N}.
        \end{align}    
        Using the result of part~\ref{part:BL-c}, for large values of $N$, the matrix $\frac{\bm{Q}_t^\top \bm{Q}_t}{N}$ is positive definite (this is true because the eigenvalues of a matrix vary continuously with respect to its entries). Then, note that the mapping $\bm{G} \mapsto \bm{G}^{-1}$ is continuous for any invertible matrix $\bm{G}$. As a result, we get
        \begin{align*}
            \lim_{N \rightarrow \infty} \left(\frac{\bm{Q}_t^\top \bm{Q}_t}{N}\right)^{-1}
            =
            \left(\lim_{N \rightarrow \infty}\frac{\bm{Q}_t^\top \bm{Q}_t}{N}\right)^{-1}.
        \end{align*}
        Since the matrix $\lim_{N \rightarrow \infty} \frac{\bm{Q}_t^\top \bm{Q}_t}{N}$ is positive definite, the left term in \eqref{eq:cl-alpha is finite} is well-defined and finite. The finiteness of the right term is the consequence of \eqref{eq:BL-proof-ct-1}. \ep

        \item[(b)] We first derive several intermediate results and then utilize them to demonstrate that \eqref{eq:BL-b} holds true for $0 \leq r, s \leq t$. In this process, we apply the Strong Law of Large Numbers (SLLN) from Theorem~\ref{thm:SLLN} multiple times, without explicitly verifying the conditions, as they are straightforward.

        By Lemma~\ref{lm:conditional dist of outcome}, conditioning on $\Gc_t$, the terms $\IMatvnew{i\cdot} \VUoutcome{\perp}{}{t}$ for $i \in [N]$ are i.i.d. Gaussian random variables. Similarly, the terms $\IMatTv{i \cdot}{t} \VUoutcome{}{}{t}$ for $i \in [N]$ are also i.i.d. Gaussian random variables:
        % 
        \begin{equation}
            \label{eq:BL-proof-bt-simple-1}
            \begin{aligned}
                \IMatvnew{i\cdot} \VUoutcome{\perp}{}{t} \sim \Nc\left(0, \frac{\sigma^2}{N} \sum_{j=1}^N (\Uoutcome{\perp,j}{}{t})^2\right),
            \quad\quad
                \IMatTv{i \cdot}{t} \VUoutcome{}{}{t} \sim \Nc\left(0, \frac{\sigma_t^2}{N} \sum_{j=1}^N (\Uoutcome{j}{}{t})^2\right).
            \end{aligned}
        \end{equation}
        % 
        Now, applying Theorem~\ref{thm:SLLN}, we get the following results:
        % 
        \begin{equation}
            \label{eq:BL-proof-bt-simple-4}
            \begin{aligned}
                \lim_{N \rightarrow \infty}
                \frac{1}{N}
                \sum_{i=1}^N
                \left(
                \IMatvnew{i\cdot} 
                \VUoutcome{\perp}{}{t}
                \right)^2
                &\eqas
                \lim_{N \rightarrow \infty} \frac{\sigma^2}{N}
                \sum_{i=1}^N
                \left(\Uoutcome{\perp,i}{}{t}\right)^2,
            \end{aligned}
        \end{equation}
        % 
        as well as
        % 
        \begin{equation}
            \label{eq:BL-proof-bt-simple-5}
            \begin{aligned}
                \lim_{N \rightarrow \infty}
                \frac{1}{N}
                \sum_{i=1}^N
                \left(
                \IMatTv{i \cdot}{t}
                \VUoutcome{}{}{t}
                \right)^2
                &\eqas
                \lim_{N \rightarrow \infty} \frac{\sigma_t^2}{N}
                \sum_{i=1}^N
                \left(\Uoutcome{i}{}{r}\right)^2.
            \end{aligned}
        \end{equation}
        % 
        Also, considering \eqref{eq:Q and R} and applying \eqref{eq:BL-proof-IH-b-4}, we obtain
        % 
        \begin{equation}
        \label{eq:BL-proof-bt-simple-7}
            \begin{aligned}
                &\lim_{N \rightarrow \infty}
                \frac{1}{N}
                \sum_{i=1}^N
                \left(
                \left[
                \bm{R}_t
                \VAPC{}{t}
                \right]^i
                \right)^2
                \\
                =
                &\lim_{N \rightarrow \infty}
                \frac{1}{N}
                \sum_{i=1}^N
                \left(
                \sum_{s=0}^{t-1}
                \APC{t}{s}
                \big(
                \outcome{}{i}{s+1}
                - \IMatTv{i \cdot}{s} \VUoutcome{}{}{s}
                \big)
                \right)^2
                \\
                =
                &\lim_{N \rightarrow \infty}
                \frac{1}{N}
                \sum_{i=1}^N
                \sum_{0\leq s,r < t}
                \APC{t}{s}
                \APC{t}{r}
                \big(
                \outcome{}{i}{s+1} - \IMatTv{i \cdot}{s} \VUoutcome{}{}{s}
                \big)
                \big(
                \outcome{}{i}{r+1} - \IMatTv{i \cdot}{r} \VUoutcome{}{}{r}
                \big)
                \\
                =
                &\sum_{0\leq s,r < t}
                \APC{t}{s}
                \APC{t}{r}
                \left(
                \lim_{N \rightarrow \infty}
                \frac{1}{N}
                \sum_{i=1}^N
                \big(
                \outcome{}{i}{s+1} - \IMatTv{i \cdot}{s} \VUoutcome{}{}{s}
                \big)
                \big(
                \outcome{}{i}{r+1} - \IMatTv{i \cdot}{r} \VUoutcome{}{}{r}
                \big)\right)
                \\
                \eqas
                & \lim_{N \rightarrow \infty}
                \frac{\sigma^2}{N}
                \sum_{i=1}^N
                \sum_{0\leq s,r < t}
                \APC{t}{s}
                \APC{t}{r}
                \Uoutcome{i}{}{s} \Uoutcome{i}{}{r}
                \\
                =
                &\lim_{N \rightarrow \infty}
                \frac{\sigma^2}{N}
                \sum_{i=1}^N
                \left(\Uoutcome{\parallel,i}{}{t}\right)^2,
            \end{aligned}
        \end{equation}
        % 
        where in the last line we used \eqref{eq:projection sum}.

        Now, we first obtain \eqref{eq:BL-b-1} for $r=t$. By \eqref{eq:conditional dist of outcome_nonsym}, \eqref{eq:BL-proof-bt-simple-4}, \eqref{eq:BL-proof-bt-simple-5}, and \eqref{eq:BL-proof-bt-simple-7}, we can write
        % 
        \begin{equation}
            \label{eq:BL-proof-bt-2-1}
            \begin{aligned}
                \lim_{N \rightarrow \infty}
                \frac{1}{N}
                \sum_{i=1}^N
                \big(
                \outcome{}{i}{t+1}
                \big)^2
                &\eqas
                \lim_{N \rightarrow \infty}
                \frac{1}{N}
                \sum_{i=1}^N
                \left(
                \IMatvnew{i\cdot} 
                \VUoutcome{\perp}{}{t}
                +
                \left[
                \bm{R}_t
                \VAPC{}{t}
                \right]^i
                +
                \IMatTv{i \cdot}{t} \VUoutcome{}{}{t}
                \right)^2
                \\
                &\eqas
                \lim_{N \rightarrow \infty} \frac{\sigma^2}{N}
                \sum_{i=1}^N
                \left(\Uoutcome{\perp,i}{}{t}\right)^2
                +
                \lim_{N \rightarrow \infty}
                \frac{\sigma^2}{N}
                \sum_{i=1}^N
                \left(\Uoutcome{\parallel,i}{}{t}\right)^2
                +
                \lim_{N \rightarrow \infty} \frac{\sigma_t^2}{N}
                \sum_{i=1}^N
                \left(\Uoutcome{i}{}{t}\right)^2
                \\
                &\quad
                +
                \lim_{N \rightarrow \infty}
                \frac{2}{N}
                \sum_{i=1}^N
                \left(\IMatvnew{i\cdot} 
                \VUoutcome{\perp}{}{t} \left[
                \bm{R}_t
                \VAPC{}{t}
                \right]^i \right)
                +
                \lim_{N \rightarrow \infty}
                \frac{2}{N}
                \sum_{i=1}^N
                \left(\IMatvnew{i\cdot} 
                \VUoutcome{\perp}{}{t} \IMatTv{i \cdot}{t} \VUoutcome{}{}{t} \right)
                \\
                &\quad
                +
                \lim_{N \rightarrow \infty}
                \frac{2}{N}
                \sum_{i=1}^N
                \left( \left[
                \bm{R}_t
                \VAPC{}{t}
                \right]^i \IMatTv{i \cdot}{t}\VUoutcome{}{}{t} \right).
            \end{aligned}
        \end{equation}
        % 
        Note that the only random elements in the right-hand side of \eqref{eq:BL-proof-bt-2-1} are $\IMatvnew{i \cdot}$ and $\IMatTv{i \cdot}{t}$. Thus, by \eqref{eq:BL-proof-bt-simple-1} and applying Theorem~\ref{thm:SLLN}, we can demonstrate that the last three terms vanish, resulting in the following:
        % 
        \begin{align}
            \label{eq:BL-proof-bt-2-result}
            \lim_{N \rightarrow \infty}
            \frac{1}{N}
            \sum_{i=1}^N
            \big(
            \outcome{}{i}{t+1}
            \big)^2
            &\eqas
            \lim_{N \rightarrow \infty} \frac{\sigma^2+\sigma_t^2}{N}
            \sum_{i=1}^N
            \left(\Uoutcome{i}{}{t}\right)^2
            \eqas
            (\MVVO{}{}{t+1})^2,
        \end{align}
        % 
        where the last equality is immediate by \eqref{eq:BL-proof-IH-a}.
        
        Next, we derive \eqref{eq:BL-b-2} for $r=t$ and $0\leq s \leq t-1$. Considering \eqref{eq:conditional dist of outcome_nonsym}, we can write
        % 
        \begin{equation}
        \label{eq:BL-proof-bt-3-1}
            \begin{aligned}
                &\lim_{N \rightarrow \infty}
                \frac{1}{N}
                \sum_{i=1}^N
                \outcome{}{i}{s+1}
                \outcome{}{i}{t+1}
                \\
                \eqas
                &\lim_{N \rightarrow \infty}
                \frac{1}{N}
                \sum_{i=1}^N
                \outcome{}{i}{s+1}
                \left(
                \IMatvnew{i\cdot} 
                \VUoutcome{\perp}{}{t}
                +
                \left[
                \bm{R}_t
                \VAPC{}{t}
                \right]^i
                + \IMatTv{i \cdot}{t} \VUoutcome{}{}{t}
                \right)
                \\
                \eqas
                &\lim_{N \rightarrow \infty}
                \frac{1}{N}
                \sum_{i=1}^N
                \left(
                \IMatvnew{i\cdot} 
                \VUoutcome{\perp}{}{t}
                \outcome{}{i}{s+1}
                +
                \left[
                \bm{R}_t
                \VAPC{}{t}
                \right]^i
                \outcome{}{i}{s+1}
                +
                \IMatTv{i \cdot}{t} 
                \VUoutcome{}{}{t}
                \outcome{}{i}{s+1}
                \right).
            \end{aligned}
        \end{equation}
        % 
        Note that by conditioning on $\Gc_t$, the value of $\outcome{}{n}{s+1}$ is deterministic. Then, applying Theorem~\ref{thm:SLLN} and considering \eqref{eq:BL-proof-bt-simple-1}, \eqref{eq:Q and R}, and by \eqref{eq:BL-proof-IH-b-5}, we get the desired result:
        % 
        \begin{equation}
            \label{eq:BL-proof-bt-3-4}
            \begin{aligned}
                \lim_{N \rightarrow \infty}
                \frac{1}{N}
                \sum_{i=1}^N
                \outcome{}{i}{s+1}
                \outcome{}{i}{t+1}
                &\eqas
                \lim_{N \rightarrow \infty}
                \frac{1}{N}
                \sum_{i=1}^N
                \left(
                \left[
                \bm{R}_t
                \VAPC{}{t}
                \right]^i
                \right)\outcome{}{i}{s+1}
                \\
                &=
                \lim_{N \rightarrow \infty}
                \frac{1}{N}
                \sum_{i=1}^N
                \left(
                \sum_{r=0}^{t-1}
                \APC{r}{t}
                \big(
                \outcome{}{i}{r+1}
                - \IMatTv{i \cdot}{r} \VUoutcome{}{}{r}
                \big)
                \outcome{}{i}{s+1}
                \right)
                \\
                &\eqas
                \sum_{r=0}^{t-1}
                \APC{r}{t} 
                \left(
                \lim_{N \rightarrow \infty}
                \frac{\sigma^2}{N} \sum_{i=1}^N \Uoutcome{i}{}{r}\Uoutcome{i}{}{s}
                \right)
                \\
                &\eqas
                \lim_{N \rightarrow \infty}\hspace{-1mm}
                \left(
                \frac{\sigma^2}{N} \sum_{i=1}^N \sum_{r=0}^{t-1}
                \APC{r}{t}
                \Uoutcome{i}{}{r}\Uoutcome{i}{}{s}\right)
                \\
                &=
                \lim_{N \rightarrow \infty}
                \frac{\sigma^2}{N}
                \sum_{i=1}^N
                \left(
                \Uoutcome{\parallel,i}{}{t}
                \Uoutcome{i}{}{s}
                \right)
                \\
                &=
                \lim_{N \rightarrow \infty}
                \frac{\sigma^2}{N}
                \sum_{i=1}^N
                \left(
                \Uoutcome{i}{}{t}
                \Uoutcome{i}{}{s}
                \right),
            \end{aligned}
        \end{equation}
        % 
        where in the last line, we used the fact that $\pdot{\VUoutcome{}{}{t}}{\VUoutcome{}{}{s}} = \pdot{\VUoutcome{\parallel}{}{t}}{\VUoutcome{}{}{s}}$ as $\VUoutcome{\perp}{}{t} \perp \VUoutcome{}{}{s}$.

        The derivations for \eqref{eq:BL-b-3} and \eqref{eq:BL-b-4} follow a similar procedure, which we omit here for brevity. We then apply a similar approach to obtain \eqref{eq:BL-b-5}. Specifically, fixing $0 \leq r \leq t-1$ and setting $s=t$, we can write the following using \eqref{eq:conditional dist of outcome_nonsym} and \eqref{eq:BL-proof-bt-3-4}:
        % 
        \begin{equation*}
            \begin{aligned}
                \lim_{N \rightarrow \infty}
                \frac{1}{N}
                \sum_{i=1}^N
                \big(
                \outcome{}{i}{t+1}
                - \IMatTv{i \cdot}{t} \VUoutcome{}{}{t}
                \big)
                \outcome{}{i}{r+1}
                \eqas
                \lim_{N \rightarrow \infty}
                \frac{1}{N}
                \sum_{i=1}^N
                \left(
                \IMatvnew{i\cdot} 
                \VUoutcome{\perp}{}{t}
                +
                \left[
                \bm{R}_t
                \VAPC{}{t}
                \right]^i
                \right)\outcome{}{i}{r+1}
                \eqas
                \lim_{N \rightarrow \infty}
                \frac{\sigma^2}{N}
                \sum_{i=1}^N
                \left(
                \Uoutcome{i}{}{t}
                \Uoutcome{i}{}{r}
                \right).
            \end{aligned}
        \end{equation*}
        % 
        Likewise, we can show the result for the case that $r=t$ and $0\leq s \leq t-1$:
        % 
        \begin{equation}
            \label{eq:BL-proof-bt-4-1}
            \begin{aligned}
                &\lim_{N \rightarrow \infty}
                \frac{1}{N}
                \sum_{i=1}^N
                \big(
                \outcome{}{i}{s+1}
                - \IMatTv{i \cdot}{s} \VUoutcome{}{}{s}
                \big)
                \outcome{}{i}{t+1}
                \\
                \eqas
                &\lim_{N \rightarrow \infty}
                \frac{1}{N}
                \sum_{i=1}^N
                \big(
                \outcome{}{i}{s+1}
                - \IMatTv{i \cdot}{s} \VUoutcome{}{}{s}
                \big)
                \left(
                \IMatvnew{i\cdot} 
                \VUoutcome{\perp}{}{t}
                +
                \left[
                \bm{R}_t
                \VAPC{}{t}
                \right]^i
                + \IMatTv{i \cdot}{t} \VUoutcome{}{}{t}
                \right)
                \\
                =
                &\lim_{N \rightarrow \infty}
                \frac{1}{N}
                \sum_{i=1}^N
                \left(
                \IMatvnew{i\cdot} 
                \VUoutcome{\perp}{}{t}
                \right)
                \big(
                \outcome{}{i}{s+1}
                - \IMatTv{i \cdot}{s} \VUoutcome{}{}{s}
                \big)
                \\
                &+
                \lim_{N \rightarrow \infty}
                \frac{1}{N}
                \sum_{i=1}^N
                \left(
                \IMatTv{i \cdot}{t} \VUoutcome{}{}{t}
                \right)
                \big(
                \outcome{}{i}{s+1}
                - \IMatTv{i \cdot}{s} \VUoutcome{}{}{s}
                \big),
                \\
                &+
                \lim_{N \rightarrow \infty}
                \frac{1}{N}
                \sum_{i=1}^N
                \left(
                \left[
                \bm{R}_t
                \VAPC{}{t}
                \right]^i
                \right)
                \big(
                \outcome{}{i}{s+1}
                - \IMatTv{i \cdot}{s} \VUoutcome{}{}{s}
                \big).
            \end{aligned}
        \end{equation}
        % 
        Then, by Theorem~\ref{thm:SLLN}, the first and second terms on the right-hand side are zero. Additionally,
        \eqref{eq:Q and R}, \eqref{eq:projection sum}, and \eqref{eq:BL-proof-IH-b-4} imply that
        % 
        \begin{equation}
        \label{eq:BL-proof-bt-4-4}
            \begin{aligned}
                \lim_{N \rightarrow \infty}
                \frac{1}{N}
                \sum_{i=1}^N
                \left(
                \left[
                \bm{R}_t
                \VAPC{}{t}
                \right]^i
                \right)
                \big(
                \outcome{}{i}{s+1}
                - \IMatTv{i \cdot}{s} \VUoutcome{}{}{s}
                \big)
                &=
                \lim_{N \rightarrow \infty}
                \frac{1}{N}
                \sum_{i=1}^N
                \sum_{r=0}^{t-1}
                \APC{r}{t}
                \big(
                \outcome{}{i}{r+1}
                - \IMatTv{i \cdot}{r} \VUoutcome{}{}{r}
                \big)
                \big(
                \outcome{}{i}{s+1}
                - \IMatTv{i \cdot}{s} \VUoutcome{}{}{s}
                \big)
                \\
                &\eqas
                \lim_{N \rightarrow \infty}
                \frac{\sigma^2}{N}
                \sum_{i=1}^N
                \Uoutcome{i}{}{s} \Uoutcome{i}{}{r},
            \end{aligned}
        \end{equation}
        % 
        where in the last line, we used $\pdot{\VUoutcome{}{}{t}}{\VUoutcome{}{}{s}} = \pdot{\VUoutcome{\parallel}{}{t}}{\VUoutcome{}{}{s}}$. The desired result follows by aggregating \eqref{eq:BL-proof-bt-4-1}-\eqref{eq:BL-proof-bt-4-4}.

        
        \item We use induction hypotheses to establish the following result:
        % 
        \begin{equation}
            \label{eq:BL-a_(t)}
            \begin{aligned}
                \lim_{N \rightarrow \infty}
                \frac{1}{\cardinality{\batch}} \sum_{i \in \batch}
                \psi\big(
                \outcomeD{}{i}{0},
                \outcomeD{}{i}{1},
                \ldots,
                \outcomeD{}{i}{t+1},
                \Vtreatment{i}{},\Vcovar{i}
                \big)
                \\
                \eqas
                \E
                \Big[
                \psi
                \big(
                \outcomeD{}{\batch}{0},
                \MAVO{}{\batch}{1}
                + \MVVO{}{}{1} Z_1
                + \Houtcome{\batch}{}{0},
                \ldots,
                \MAVO{}{\batch}{t+1}
                + \MVVO{}{}{t+1} Z_t
                + \Houtcome{\batch}{}{t},
                \Vtreatment{\batch}{}, \Vcovar{\batch}
                \big)
                \Big].
            \end{aligned}
        \end{equation}
        % 
        More general results related to the extension of \eqref{eq:BL-proof-a0-dynamics-with-eps0} and \eqref{eq:BL-proof-a0-dynamics-with-eps} follow a similar procedure and are omitted for brevity.
        
        We proceed by introducing a new notation. Fixing $i$ as an arbitrary unit, we define
        % 
        \begin{equation*}
            \begin{aligned}
                \Psi^{i}(N)
                &:=
                \psi\big(
                \outcomeD{}{i}{0},
                \outcomeD{}{i}{1}, \ldots,
                \outcomeD{}{i}{t},
                \outcome{}{i}{t+1},
                \Vtreatment{i}{},
                \Vcovar{i},
                \MAVO{}{i}{(t+1)N}
                \big),
            \end{aligned}
        \end{equation*}
        % 
        where
        % 
        \begin{equation*}
        \begin{aligned}
            \MAVO{}{i}{(t+1)N} &=
            \frac{1}{N} \sum_{k=1}^N (\mu^{ik} + \mu_t^{ik}) \outcomeg{t}{} \big(\outcomeD{}{k}{t}, \Vtreatment{k}{}, \Vcovar{k}\big).
        \end{aligned}
        \end{equation*}
        %
        Using \eqref{eq:BL-proof-IH-a}, this implies that
        %
        \begin{equation}
        \label{eq:BL-at-stat-2_limit}
        \begin{aligned}
            \lim_{N \rightarrow \infty} \MAVO{}{i}{(t+1)N}
            &\eqas
            \E \left[ (\MIM{}{i}+\MIM{t}{i}) \outcomeg{t}{} \big(\MAVO{}{}{t} + \MVVO{}{}{t} Z + \Houtcome{}{}{t-1},\Vtreatment{}{},\Vcovar{}\big) \right]
            \\ &=
            (\bar{\mu}^i+\bar{\mu}^i_t) \E \left[\outcomeg{t}{} \big(\MAVO{}{}{t} + \MVVO{}{}{t} Z + \Houtcome{}{}{t-1},\Vtreatment{}{},\Vcovar{}\big) \right]
            =
            \MAVO{}{i}{t+1} < \infty.
        \end{aligned}
        \end{equation}
        % 
        Now, by \eqref{eq:conditional dist of outcome_nonsym}, we can write
        % 
        \begin{equation*}
            \begin{aligned}
                \Psi^{i}(N)
                \Big|_{\Gc_t}
                \eqd
                \psi\left(
                \outcomeD{}{i}{0},
                \outcomeD{}{i}{1}, \ldots,
                \outcomeD{}{i}{t},
                % \outcome{}{n}{1}, \ldots,
                % \outcome{}{n}{t},
                \left[
                \widetilde{\IM}
                \VUoutcome{\perp}{}{t}
                + \bm{R}_t 
                \VAPC{}{t}
                +
                \IMatT{t} \VUoutcome{}{}{t}
                \right]^i,
                \Vtreatment{i}{},
                \Vcovar{i},
                \MAVO{}{i}{(t+1)N}
                \right),
            \end{aligned}
        \end{equation*}
        % 
        where $\left[ \widetilde{\IM} \VUoutcome{\perp}{}{t} + \bm{R}_t  \VAPC{}{t} + \IMatT{t} \VUoutcome{}{}{t} \right]^i$ represent the $i^{th}$ element in the vector $ \widetilde{\IM} \VUoutcome{\perp}{}{t} + \bm{R}_t  \VAPC{}{t} + \IMatT{t} \VUoutcome{}{}{t}$. We also let
        % 
        \begin{align*}
            \widetilde{\Psi}^{i}(N) = \Psi^{i}(N) - \E_{\IM,\IMatT{t}}[\Psi^{i}(N)].
        \end{align*}
        % 
        where $\E_{\IM,\IMatT{t}}$ denotes the expectation with respect to the randomness of the interference matrices $\IM$ and $\IMatT{t}$. We follow the same approach as Step~1-\ref{item:BL-average limit}. Note that given $\Gc_t$, the elements of $\widetilde{\IM} \VUoutcome{\perp}{}{t} + \IMatT{t} \VUoutcome{}{}{t}$ are i.i.d. Gaussian random variables with a zero mean and variance~$(\hat{\rho}_{tN})^2$:
        % 
        \begin{equation}
            \label{eq:BL-bt-Yt stat}
            \begin{aligned}
                (\hat{\rho}_{tN})^2
                &:=
                \Var
                \left[
                [
                \widetilde{\IM} \VUoutcome{\perp}{}{t} + \IMatT{t} \VUoutcome{}{}{t}
                ]^i
                \Big|
                \VUoutcome{}{}{t}
                \right]
                =
                \frac{\sigma^2}{N}
                \sum_{j=1}^N
                \left(\Uoutcome{\perp,j}{}{t}\right)^2
                +
                \frac{\sigma_t^2}{N}
                \sum_{j=1}^N
                \left(\Uoutcome{j}{}{t}\right)^2,
            \end{aligned}
        \end{equation}
        % 
        where $\Uoutcome{j}{}{t} = \outcomeg{t}{}\big(\outcomeD{}{j}{t}, \Vtreatment{j}{}, \Vcovar{j}\big)$ is the $j^{th}$ element of the column vector $\VUoutcome{}{}{t}$, and similarly, $\Uoutcome{\perp,j}{}{t}$ is the $j^{th}$ element of the column vector $\VUoutcome{\perp}{}{t}$. Letting
        % 
        \begin{align}
            \label{eq:BL-bt-(-1)}
            (\hat{\rho}_t)^2 = \lim_{N\rightarrow \infty } (\hat{\rho}_{tN})^2,
        \end{align}
        % 
        we have the following lemma regarding the finiteness of $\hat{\rho}_t$.
        % 
        \begin{lemma}
            \label{lm:finite variance}
            $\hat{\rho}_t$ is almost surely finite.
        \end{lemma}
        % 
        \textbf{Proof of Lemma~\ref{lm:finite variance}.} We show that the following relations hold with a probability of 1:
        % 
        \begin{align}
            \label{eq:BL-bt-1}
            \lim_{N \rightarrow \infty}
            \frac{1}{N}
            \sum_{j=1}^N
            \left(\Uoutcome{\perp,j}{}{t}\right)^2 < \infty,
            \;\;\;
            \lim_{N \rightarrow \infty}
            \frac{1}{N}
            \sum_{j=1}^N
            (\Uoutcome{j}{}{t})^2 < \infty.
        \end{align}
        % 
        By definition, we can write
        % 
        \begin{equation}
        \label{eq:BL-bt-2}
        \begin{aligned}
            \frac{1}{N}
            \sum_{j=1}^N
            \left(\Uoutcome{\perp,j}{}{t}\right)^2
            =
            \pdot{\VUoutcome{\perp}{}{t}}{\VUoutcome{\perp}{}{t}}
            &=
            \pdot{\VUoutcome{}{}{t}}{\VUoutcome{}{}{t}}
            -
            \pdot{\VUoutcome{\parallel}{}{t}}{\VUoutcome{\parallel}{}{t}}
            =
            \frac{1}{N}
            \sum_{j=1}^N
            \left(\Uoutcome{j}{}{t}\right)^2
            -
            \frac{1}{N}
            \sum_{j=1}^N
            \left(\Uoutcome{\parallel,j}{}{t}\right)^2.
        \end{aligned}
        \end{equation}
        % 
        Then, by \eqref{eq:BL-proof-IH-a} for the function $\outcomeg{t}{}\big(\outcomeD{}{j}{t}, \Vtreatment{j}{}, \Vcovar{j}\big)^2$, we get
        % 
        \begin{equation}
        \label{eq:BL-bt-3}
        \begin{aligned}
            \lim_{N \rightarrow \infty}
            \frac{1}{N}
            \sum_{j=1}^N
            \left(\Uoutcome{j}{}{t}\right)^2
            &=
            \lim_{N \rightarrow \infty}
            \frac{1}{N}
            \sum_{j=1}^N
            \outcomeg{t}{} \big(\outcomeD{}{j}{t}, \Vtreatment{j}{}, \Vcovar{j}\big)^2
            \eqas
            \E\left[
            \outcomeg{t}{} \big(\MAVO{}{}{t} + \MVVO{}{}{t} Z + \Houtcome{}{}{t-1}, \Vtreatment{}{}, \Vcovar{}\big)^2
            \right]  < \infty,
        \end{aligned}
        \end{equation}
        % 
        where $Z \sim \Nc(0,1)$. Further, by \eqref{eq:projection sum}, we have
        % 
        \begin{align*}
            \frac{1}{N}
            \sum_{j=1}^N
            \Uoutcome{\parallel,j}{}{t}
            &=
            \frac{1}{N}
            \sum_{j=1}^N
            \sum_{s=0}^{t-1} \APC{s}{t} \Uoutcome{j}{}{s}
            =
            \sum_{s=0}^{t-1}
            \frac{\APC{s}{t}}{N}
            \sum_{j=1}^N
            \Uoutcome{j}{}{s}
            \\
            \frac{1}{N}
            \sum_{j=1}^N
            \left(\Uoutcome{\parallel,j}{}{t}\right)^2,
            &=
            \frac{1}{N}
            \sum_{j=1}^N
            \left(
            \sum_{s=0}^{t-1} \APC{s}{t} \Uoutcome{j}{}{s}
            \right)^2
            =
            \sum_{r,s=0}^{t-1} \APC{r}{t} \APC{s}{t} \pdot{\VUoutcome{}{}{r}}{\VUoutcome{}{}{s}}.
        \end{align*}
        % 
        Considering Corollary~\ref{cl:alpha is bounded}, the vector $\VAPC{}{t}$ has a finite limit as $N\rightarrow \infty$. Similar to \eqref{eq:BL-bt-3}, the induction hypothesis for the function $\psi = \outcomeg{r}{}\big(\outcomeD{}{j}{r},\Vtreatment{j}{},\Vcovar{j}\big) \outcomeg{s}{}\big(\outcomeD{}{j}{s},\Vtreatment{j}{},\Vcovar{j}\big)$ implies that almost surely
        % 
        \begin{align}
            \label{eq:BL-bt-4}
            \quad\quad
            \lim_{N\rightarrow \infty}
            \frac{1}{N}
            \sum_{j=1}^N
            \left(\Uoutcome{\parallel,j}{}{t}\right)^2
            =
            \lim_{N\rightarrow \infty}
            \sum_{r,s=0}^{t-1} \APC{r}{t} \APC{s}{t} \pdot{\VUoutcome{}{}{r}}{\VUoutcome{}{}{s}}
            < \infty.
        \end{align}
        % 
        Consequently, by \eqref{eq:BL-bt-2}-\eqref{eq:BL-bt-4}, we get the result in \eqref{eq:BL-bt-1} and the proof is complete. \ep

        An immediate corollary of the result of Lemma~\ref{lm:finite variance} is that $\hat{\rho}_{tN}$, in~\eqref{eq:BL-bt-Yt stat}, is almost surely bounded independent of $N$. Then, for $l\geq 1$, it is straightforward to show that,
        % 
        \begin{align}
            \label{eq:BL-proof-t-2}
            \E
            \left[
            \left|
                \big[
                \widetilde{\IM} \VUoutcome{\perp}{}{t} + \IMatT{t} \VUoutcome{}{}{t}
                \big]^i
                + 
                \left[\bm{R}_t 
                \VAPC{}{t}
                \right]^i
            \right|^{l}
            \right]
            \leq
            2^{l-1}
            \E
            \left[
            \left|
                \big[
                \widetilde{\IM} \VUoutcome{\perp}{}{t} + \IMatT{t} \VUoutcome{}{}{t}
                \big]^i
                \right|^{l}
                +
                \left|
                \left[\bm{R}_t 
                \VAPC{}{t}
                \right]^i
            \right|^{l}
            \right]
            \leq c,
        \end{align}
        % 
        where $c$ is a constant independent of $N$ and we used the inequality $(v_1+v_2)^l \leq 2^{l-1} (v_1^l+v_2^l),\; v_1,v_2 \geq 0$. Note that in \eqref{eq:BL-proof-t-2}, given $\Gc_t$, the term $\bm{R}_t  \VAPC{}{t}$ is deterministic and bounded by Corollary~\ref{cl:alpha is bounded}. Then, fixing $0 < \kappa < 1$ and using the fact that $\psi \in \poly{k}$ and so $|\psi(\Vec{\omega})|\leq c (1 + \norm{\Vec{\omega}}^k)$, by Assumption~\ref{asmp:weak_limits}, we get,
        % 
        \begin{equation}
            \label{eq:BL-proof-t-1}
            \begin{aligned}
                \frac{1}{\cardinality{\batch}}
                \sum_{i \in \batch}
                \E\left[\left| \widetilde{\Psi}^{i}(N)\right|^{2+\kappa} \right]
                \leq c \cardinality{\batch}^{\kappa/2}.
            \end{aligned}
        \end{equation}
        % 
        Therefore, we can apply the SLLN for triangular arrays in Theorem~\ref{thm:SLLN} to obtain the following result:
        % 
        \begin{equation}
        \label{eq:BL-proof-t-3}
        \begin{aligned}
            \lim_{N \rightarrow \infty}
            \frac{1}{\cardinality{\batch}}
            \sum_{i\in \batch}
            &\psi\big(
                \outcomeD{}{i}{0},
                \outcomeD{}{i}{1}, \ldots,
                \outcomeD{}{i}{t},
                \outcome{}{i}{t+1},
                \Vtreatment{i}{},
                \Vcovar{i},
                \MAVO{}{i}{(t+1)N}
            \big)
            \\
            \eqas
            \;\lim_{N \rightarrow \infty}
            \frac{1}{\cardinality{\batch}}
            \sum_{i\in \batch}
            \E_{\IM,\IMatT{t}}
            \Big[
            \psi&\Big(
                \outcomeD{}{i}{0},
                \outcomeD{}{i}{1}, \ldots,
                \outcomeD{}{i}{t},
                \left[
                \widetilde{\IM}
                \VUoutcome{\perp}{}{t}
                + \bm{R}_t 
                \VAPC{}{t}
                +
                \IMatT{t} \VUoutcome{}{}{t}
                \right]^i,
                \Vtreatment{i}{}, \Vcovar{i},
                \HAVO{}{i}{(t+1)N}
                \Big)
            \Big].
        \end{aligned}
        \end{equation}
        % 
        Now, considering \eqref{eq:BL-at-stat-2_limit} and \eqref{eq:BL-bt-(-1)}, we can write
        % 
        \begin{equation}
        \label{eq:BL-proof-t-t}
        \begin{aligned}
            &\lim_{N \rightarrow \infty}
            \frac{1}{\cardinality{\batch}}
            \sum_{i\in \batch}
            \E_{\IM,\IMatT{t}}
            \Big[
            \psi\Big(
                \outcomeD{}{i}{0},
                \outcomeD{}{i}{1}, \ldots,
                \outcomeD{}{i}{t},
                \left[
                \widetilde{\IM}
                \VUoutcome{\perp}{}{t}
                + \bm{R}_t 
                \VAPC{}{t}
                +
                \IMatT{t} \VUoutcome{}{}{t}
                \right]^i,
                \Vtreatment{i}{}, \Vcovar{i},
                \MAVO{}{i}{(t+1)N}
                \Big)
            \Big]
            \\
            \eqas
            &\lim_{N \rightarrow \infty}
            \frac{1}{\cardinality{\batch}}
            \sum_{i \in \batch}
            \E_{Z}
            \Big[
            \psi\Big(
                \outcomeD{}{i}{0},
                \outcomeD{}{i}{1}, \ldots,
                \outcomeD{}{i}{t},
                \hat{\rho}_t Z
                +
                \left[
                \bm{R}_t 
                \VAPC{}{t}
                \right]^i,
                \Vtreatment{i}{}, \Vcovar{i},
                \MAVO{}{i}{t+1}
                \Big)
            \Big],
        \end{aligned}
        \end{equation}
        %
        where similar to \eqref{eq:BL-a0-limit to the function} in the first step, we utilized the dominated convergence theorem and the continuous mapping theorem to pass the limit through the expectation and the function, respectively.
        
        From Eq.~\eqref{eq:BL-at-stat-2_limit}, recall that $\MAVO{}{i}{t+1} = (\bar{\mu}^i+\bar{\mu}^i_t) \E \left[\outcomeg{t}{} \big(\MAVO{}{}{t} + \MVVO{}{}{t} Z + \Houtcome{}{}{t-1},\Vtreatment{}{},\Vcovar{}\big) \right]$; accordingly, we define
        % 
        \begin{equation*}
        \begin{aligned}
            \widehat{\psi}&
            \big(
            \outcomeD{}{i}{0},
            \outcomeD{}{i}{1},
            \ldots,
            \outcomeD{}{i}{t},
            \outcome{}{i}{1} - \IMatTv{i \cdot}{0}\VUoutcome{}{}{0}
            , \ldots,
            \outcome{}{i}{t} - \IMatTv{i\cdot}{t-1}\VUoutcome{}{}{t-1},
            \Vtreatment{i}{},
            \Vcovar{i},
            \bar{\mu}^i, \bar{\mu}^i_t
            \big)
            \\
            :=
            \E_{Z}
            \Big[
            \psi&\Big(
                \outcomeD{}{i}{0},
                \outcomeD{}{i}{1},
                \ldots,
                \outcomeD{}{i}{t},
                \hat{\rho}_{t} Z
                +
                \sum_{s=0}^{t-1}
                \APC{t}{s}
                \big(
                \outcome{}{i}{s+1}
                - \IMatTv{i \cdot}{s}\VUoutcome{}{}{s}
                \big),
                \Vtreatment{i}{},
                \Vcovar{i},
                \MAVO{}{i}{t+1}
                \Big)
            \Big].
        \end{aligned}
        \end{equation*}
        % 
        Considering \eqref{eq:BL-proof-IH-a1} as well as \eqref{eq:BL-proof-t-3}-\eqref{eq:BL-proof-t-t}, for the function $\widehat{\psi}$, we have
        % 
        \begin{equation}
        \label{eq:BL-proof-t-new function}
        \begin{aligned}
            &\lim_{N \rightarrow \infty}
            \frac{1}{\cardinality{\batch}}
            \sum_{i\in \batch}
            \psi\big(
                \outcomeD{}{i}{0},
                \outcomeD{}{i}{1}, \ldots,
                \outcomeD{}{i}{t},
                \outcome{}{i}{t+1},
                \Vtreatment{i}{},
                \Vcovar{i},
                \MAVO{}{i}{(t+1)N}
            \big)
            \\
            \eqas
            &\;
            \E
            \bigg[
            \widehat{\psi}\bigg(
                \outcomeD{}{\batch}{0},
                \MAVO{}{\batch}{1}
                +
                \MVVO{}{}{1} Z_1
                +
                \Houtcome{\batch}{}{0},
                \ldots,
                \MAVO{}{\batch}{t}
                +
                \MVVO{}{}{t} Z_t
                +
                \Houtcome{\batch}{}{t-1},
                \BVVO{}{}{1}
                Z'_{1},
                \ldots,
                \BVVO{}{}{t}
                Z'_{t},
                \Vtreatment{\batch}{}, \Vcovar{\batch},
                \MIM{}{\batch}, \MIM{t}{\batch}
                \bigg)
            \bigg]
            \\
            \eqas
            &\;
            \E
            \bigg[
            \psi\bigg(
                \outcomeD{}{\batch}{0},
                \MAVO{}{\batch}{1}
                +
                \MVVO{}{}{1} Z_1
                +
                \Houtcome{\batch}{}{0},
                \ldots,
                \MAVO{}{\batch}{t}
                +
                \MVVO{}{}{t} Z_t
                +
                \Houtcome{\batch}{}{t-1},
                \hat{\rho}_{t} Z
                +
                \sum_{s=0}^{t-1}
                \APC{t}{s}
                \BVVO{}{}{s+1}
                Z'_{s+1},
                \Vtreatment{\batch}{}, \Vcovar{\batch},
                \MAVO{}{\batch}{t+1}
                \bigg)
            \bigg],
        \end{aligned}
        \end{equation}
        % 
        where $Z$ is a standard Gaussian random variable, independent of all other variables, as the inherent randomness arises from $\widetilde{\IM}$ and $\IMatT{t}$. Additionally, $\widehat{\psi}$ belongs to $\poly{k}$, since $\MAVO{}{i}{t+1}$ in the calculations of \eqref{eq:BL-proof-t-new function} can be viewed as a linear function depending solely on $\bar{\mu}^i$ and $\bar{\mu}^i_t$.
        
        Now, we need to show that
        % 
        \begin{equation}
            \label{eq:BL-proof-bt-dynamics-1}
            \begin{aligned}
                \Var
                \left[
                \hat{\rho}_{t} Z
                +
                \sum_{s=0}^{t-1}
                \APC{t}{s}
                \big(
                \BVVO{}{}{s+1}
                Z'_{s+1}
                \big)
                \right]
                &=
                (\MVVO{}{}{t+1})^2.
            \end{aligned}
        \end{equation}
        % 
        Considering that $(Z'_1,\ldots,Z'_t)$ follows a joint Normal distribution independent of $Z$, the random variable $\hat{\rho}_{t} Z + \sum_{s=0}^{t-1} \APC{t}{s} \BVVO{}{}{s+1} Z'_{s+1}$ is Gaussian. To obtain \eqref{eq:BL-proof-bt-dynamics-1}, we let $\psi = (\outcome{}{i}{t+1})^2$ in \eqref{eq:BL-proof-t-new function}. This yields
        % 
        \begin{equation}
            \label{eq:BL-proof-bt-dynamics-2}
            \begin{aligned}
                \lim_{N \rightarrow \infty}
                \frac{1}{N}
                \sum_{i=1}^N
                \big(
                \outcome{}{i}{t+1}\big)^2
                &=
                \E
                \left[
                \left(
                \hat{\rho}_{t} Z
                +
                \sum_{s=0}^{t-1}
                \APC{t}{s}
                \BVVO{}{}{s+1}
                Z'_{s+1}
                \right)^2
                \right].
            \end{aligned}
        \end{equation}
        % 
        Meanwhile, by \eqref{eq:BL-proof-bt-2-result}, we have
        % 
        \begin{align}
        \label{eq:BL-proof-bt-dynamics-3}
            \lim_{N \rightarrow \infty}
            \frac{1}{N}
            \sum_{i=1}^N
            (\outcome{}{i}{t+1})^2
            \eqas
            (\MVVO{}{}{t+1})^2.
        \end{align}
        % 
        Combining \eqref{eq:BL-proof-bt-dynamics-2} and \eqref{eq:BL-proof-bt-dynamics-3}, we derive the desired result as stated in \eqref{eq:BL-proof-bt-dynamics-1}.

        Finally, similar to \eqref{eq:BL-a0-function_re_def} and based on outcome representation in \eqref{eq:apndx_outcome_function}, we define the function $\widetilde{\psi}$ such that
        % 
        \begin{align*}
            \psi\big(
                \outcomeD{}{i}{0},
                \outcomeD{}{i}{1}, \ldots,
                \outcomeD{}{i}{t},
                \outcomeD{}{i}{t+1},
                \Vtreatment{i}{},
                \Vcovar{i}
            \big)
            &=
            \psi\big(
                \outcomeD{}{i}{0},
                \outcomeD{}{i}{1}, \ldots,
                \outcomeD{}{i}{t},
                \outcome{}{i}{t+1} + \MAVO{}{i}{(t+1)N} + \outcomeh{t}{} \big(\outcomeD{}{i}{t}, \Vtreatment{i}{}, \Vcovar{i}\big),
                \Vtreatment{i}{},
                \Vcovar{i}
            \big)
            \\
            &=
            \widetilde\psi \big(
                \outcomeD{}{i}{0},
                \outcomeD{}{i}{1}, \ldots,
                \outcomeD{}{i}{t},
                \outcome{}{i}{t+1},
                \Vtreatment{i}{},
                \Vcovar{i}, \MAVO{}{i}{(t+1)N}
            \big).
        \end{align*}
        %
        Whenever $\psi \in \poly{k}$, by Assumption~\ref{asmp:BL}, we get that $\widetilde\psi \in \poly{k}$, and applying the result in Eq.~\eqref{eq:BL-proof-t-new function} for the function $\widetilde\psi$ yields the desired result. \ep
        
        
    \end{enumerate}




\section{Estimation of Counterfactual Evolutions}
\label{sec:estimation_theory}
% 
In this section, we present a general framework for counterfactual estimation based on the outcome specification in Eq.~\eqref{eq:outcome_function_matrix}. Specifically, given a desired treatment allocation matrix $\OMtreatment{u}{}$, and observing $\Moutcome{}{}{}(\Mtreatment{}{}=\OMtreatment{o}{})$, $\OMtreatment{o}{}$, and $\covar$, we aim to estimate the counterfactual evolution denoted by $\CFE{\OMtreatment{u}{}}{}{t}$ as defined in \eqref{eq:sample_mean_outcomes}.

To state the main theoretical result of this section, we parameterize the unknown functions in the state evolution equations. Specifically, let $\outcomeg{t}{}\big(\cdot; \param{\outcomeg{t}{}}\big)$ and $\outcomeh{t}{}\big(\cdot; \param{\outcomeh{t}{}}\big)$ represent the parameterized forms of $\outcomeg{t}{}(\cdot)$ and $\outcomeh{t}{}(\cdot)$ for $t = 0, \ldots, T-1$, respectively. Here, $\Eparam{\outcomeg{t}{}}$ and $\Eparam{\outcomeh{t}{}}$ are vectors of appropriate dimensions, denoting the parameters of the respective functions. 
% 
\begin{assumption}
    \label{asmp:independent_interference_mean}
    % 
    Considering \eqref{eq:state evolution}, for all $t$, we assume there exists a modification of the function $\outcomeg{t}{}$, which, with a slight abuse of notation, is also denoted by $\outcomeg{t}{}$ such that:
    % 
    \begin{align}
    \label{eq:independent_interference_mean}
        \MAVO{}{}{t+1} =
        \E\left[
        \outcomeg{t}{}\big(\MAVO{}{}{t} + \MVVO{}{}{t} Z_t + \Houtcome{}{}{t-1}, \Vtreatment{}{}, \Vcovar{}\big)
        \right].
    \end{align}
    % 
\end{assumption}
% 
A simple example where Assumption~\ref{asmp:independent_interference_mean} holds is when the random variable $\MIM{}{} + \MIM{t}{}$ is independent of all other sources of randomness. In this case, we can normalize the mean to 1 by adjusting the function $\outcomeg{t}{}$, scaling it by an appropriate constant factor.

For further simplicity in notation, we also define $\sigma_t := \sigma + \sigma_t$. Then, we can collect the unknown parameters in the state evolution equations in \eqref{eq:state evolution} as follows:
%
\begin{align}
    \label{eq:apndx_unknowns}
    \Uc := \left(\{\sigma_t\}, \{\param{\outcomeg{t}{}} \}, \{\param{\outcomeh{t}{}}\}\right).
\end{align}
%
We denote an estimation of $\Uc$ as $\widehat{\Uc} := \left(\left\{\hat{\sigma}_t\right\}, \{\Eparam{\outcomeg{t}{}}\}, \{\Eparam{\outcomeh{t}{}}\}\right)$. We can then employ Algorithm~\ref{alg:CF estimation} to compute the desired counterfactual.
% 
\begin{algorithm}
\caption{General counterfactual estimation}
\label{alg:CF estimation}
% 
\begin{algorithmic}
% 
\Require $\Moutcome{}{}{}(\Mtreatment{}{}=\OMtreatment{o}{}), \OMtreatment{o}{}, \covar$, $\OMtreatment{u}{}$, and $Z_i \sim \Nc(0,1)$ for $i=1, \ldots, N$

\State \hspace{-1.3em} \textbf{Step 1: Parameters Estimation}

\State Estimate the set of unknown parameters $\Uc$ by $\widehat{\Uc}$.


\State \hspace{-1.3em} \textbf{Step 2: Counterfactual Estimation}



\State $\EAVO{}{}{1} \gets \frac{1}{N} \sum_{i=1}^N \outcomeg{0}{}\big(\outcomeD{}{i}{0}, \OVtreatment{i}{u}, \Vcovar{i}; \Eparam{\outcomeg{0}{}} \big)$ 

\State $\EVVO{}{}{1} \gets \frac{\hat{\sigma}_0}{N} \sum_{i=1}^N \outcomeg{0}{}\big(\outcomeD{}{i}{0}, \OVtreatment{i}{u}, \Vcovar{i}; \Eparam{\outcomeg{0}{}}\big)^2$

\State $\EHoutcome{i}{}{0} \gets \outcomeh{0}{}\big(\outcomeD{}{i}{0}, \OVtreatment{i}{u}, \Vcovar{i}; \Eparam{\outcomeh{0}{}}\big),\; i = 1, \ldots, N$ 

\State $\ECF{}{1}{\OMtreatment{u}{}} \gets \EAVO{}{}{1} + \frac{1}{N} \sum_{i=1}^N \EHoutcome{i}{}{0}$

\For{$t = 1, \ldots, T-1$}

\State $\EHoutcome{i}{}{t} \gets \outcomeh{t}{}\big(\EAVO{}{}{t} + \EVVO{}{}{t} Z^i + \EHoutcome{i}{}{t-1}, \OVtreatment{i}{u}, \Vcovar{i}; \Eparam{\outcomeh{t}{}}\big),\; i = 1, \ldots, N$

\State $\EAVO{}{}{t+1} \gets \frac{1}{N} \sum_{i=1}^N \outcomeg{t}{}\big(\EAVO{}{}{t} + \EVVO{}{}{t} Z^i + \EHoutcome{i}{}{t-1}, \OVtreatment{i}{u}, \Vcovar{i}; \Eparam{\outcomeg{t}{}}\big)$

\State $\EVVO{}{}{t+1} \gets \frac{\hat{\sigma}_t}{N} \sum_{i=1}^N \outcomeg{t}{} \big(\EAVO{}{}{t} + \EVVO{}{}{t} Z^i + \EHoutcome{i}{}{t-1}, \OVtreatment{i}{u}, \Vcovar{i}; \Eparam{\outcomeg{t}{}}\big)^2$

\State $\ECF{}{t+1}{\OMtreatment{u}{}} \gets \EAVO{}{}{t+1} +  \frac{1}{N} \sum_{i=1}^N \EHoutcome{i}{}{t}$

\EndFor

\Ensure $\ECF{}{t}{\OMtreatment{u}{}}$, $t = 1, \ldots, T$.
% 
\end{algorithmic}
\end{algorithm}
% 

% 
\begin{assumption}[Continuous parameterization]
    \label{asmp:continuous_parameterization}
    For all $t$, the mappings $\param{\outcomeg{t}{}} \mapsto \outcomeg{t}{} \big(\cdot; \param{\outcomeg{t}{}} \big)$ and $\param{\outcomeh{t}{}} \mapsto \outcomeh{t}{} \big(\cdot; \param{\outcomeh{t}{}} \big)$ are continuous functions.
\end{assumption}
% 
\begin{assumption}[Consistent parameters estimation]
    \label{asmp:consistent_parameter_estimation}
    $\widehat{\Uc}$ is a consistent estimator of $\Uc$; that is, $\widehat{\Uc} \xrightarrow{P} \Uc$ as $N \rightarrow \infty$.
\end{assumption}
% 
\begin{assumption}[All control initialization]
    \label{asmp:all_control_initialization}
    There is no treatment at time $t=0$; that means $\treatment{i}{0}=0$, for all $i\in[N]$, and no treatment is anticipated.
\end{assumption}
% 

In the following, we demonstrate the consistency of the results from Algorithm~\ref{alg:CF estimation} under above assumptions.
% 
\begin{theorem}[Consistency]
    \label{thm:consistency}
    Let the conditions of Lemma~\ref{lm:Big lemma} and Assumptions~\ref{asmp:independent_interference_mean}-\ref{asmp:all_control_initialization} hold. In particular, Assumption~\ref{asmp:weak_limits} holds for both observed and desired treatment allocations $\OMtreatment{o}{}$ and $\OMtreatment{u}{}$. Then, for any $t$, $\ECF{}{t}{\OMtreatment{u}{}}$ provides a consistent estimator for $\AVOW{\OMtreatment{u}{}}{}{t}$.
\end{theorem}

\begin{remark}
    If the consistency in Assumption~\ref{asmp:consistent_parameter_estimation} is strong, i.e., $\widehat{\Uc} \xrightarrow{a.s.} \Uc$ as $N \rightarrow \infty$, then the consistency in Theorem~\ref{thm:consistency} also holds strongly.
\end{remark}
% 
\begin{remark}
    We can generalize Algorithm~\ref{alg:CF estimation} in several ways. First, when considering outcome specifications with $l \geq 1$ lag terms, we can modify the algorithm accordingly, and extend Assumption~\ref{asmp:all_control_initialization} to require $l$ historical observations: $\treatment{i}{t}=0$ for all $t \leq l$ and $i\in[N]$. We can also relax Assumption~\ref{asmp:all_control_initialization} by beginning from any arbitrary state, provided we enforce the initial conditions to the desired counterfactual scenario--- specifically, requiring that $\OMtreatment{o}{}$ and $\OMtreatment{u}{}$ match for the first $l$ time periods.
\end{remark}
\noindent
\textbf{Proof.}
% 
We use an induction argument on $t \geq 1$ to prove the following more general statement. As $N \rightarrow \infty$, we show that
% 
\begin{equation}
    \label{eq:detailed_consistency}
    \begin{aligned}
        \EAVO{}{}{t} \xrightarrow{P} \MAVO{}{}{t},
        \quad\quad\quad
        \EVVO{}{}{t} \xrightarrow{P} \MVVO{}{}{t},
        \quad\quad\quad
        \frac{1}{N} \sum_{i=1}^N \EHoutcome{i}{}{t-1}\xrightarrow{P} \E\left[ \Houtcome{}{}{t-1} \right],
    \end{aligned}
\end{equation}
% 
Then, it is straightforward to see that $\ECF{}{t}{\OMtreatment{u}{}}$ also converges to $\AVOW{\OMtreatment{u}{}}{}{t} = \MAVO{}{}{t} + \E\left[\Houtcome{}{}{t-1}\right]$ in probability, whenever $N \rightarrow \infty$.

\textbf{Step 1.} Let $t = 1$. We begin by proving the first result in \eqref{eq:detailed_consistency}. To this end, we add the notation $(N)$ to the quantities associated with the system containing $N$ experimental units. We have
% 
\begin{equation}
    \label{eq:consistency_proof_1}
    \begin{aligned}
        \EAVO{N}{}{1}
        &=
        \frac{1}{N} \sum_{i=1}^N \outcomeg{0}{}\big(\outcomeD{}{i}{0}, \OVtreatment{i}{u}, \Vcovar{i}; \Eparam{\outcomeg{0}{}}(N) \big)
    \end{aligned}
\end{equation}
% 
Note that, by Assumption~\ref{asmp:continuous_parameterization}, the right-hand side of \eqref{eq:consistency_proof_1} can be seen as a continuous function of $\Eparam{\outcomeg{0}{}}(N)$. Therefore, applying the continuous mapping theorem, e.g., Theorem 2.3 in \cite{van2000asymptotic}, implies that
% 
\begin{equation}
    \label{eq:consistency_proof_2}
    \begin{aligned}
        \lim_{N \rightarrow \infty} \EAVO{N}{}{1}
        =
        \lim_{N \rightarrow \infty} \frac{1}{N} \sum_{i=1}^N \outcomeg{0}{}\big(\outcomeD{}{i}{0}, \OVtreatment{i}{u}, \Vcovar{i}; \Eparam{\outcomeg{0}{}}(N) \big)
        \eqp
        \lim_{N \rightarrow \infty} \frac{1}{N} \sum_{i=1}^N \outcomeg{0}{}\big(\outcomeD{}{i}{0}, \OVtreatment{i}{u}, \Vcovar{i}; \param{\outcomeg{0}{}} \big)
        \eqas
        \MAVO{}{}{1}.
    \end{aligned}
\end{equation}
% 
In the last equality, we used the result of Theorem~\ref{thm:Batch_SE}. A similar argument, combined with Assumption~\ref{asmp:BL}-\ref{asmp:BL-bound on initials}, yields the second result in \eqref{eq:detailed_consistency} for $t=1$. The third result also follows immediately.

Fixing an arbitrary function $\psi \in \poly{k}$, we also need the following intermediary result:
% 
\begin{align}
    \label{eq:consistency_proof_intermediary}
    \lim_{N \rightarrow \infty} \frac{1}{N} \sum_{i=1}^N \psi\big(Z^i, \EHoutcome{i}{N}{0}, \OVtreatment{i}{u}, \Vcovar{i}\big)
    \eqp
    \E \left[
    \psi\big(Z, \Houtcome{}{}{0}, \Vtreatment{}{u}, \Vcovar{}\big)
    \right],
\end{align}
% 
where $\Vtreatment{}{u}$ represents the weak limit of $\OVtreatment{i}{u}$'s and $Z \sim \Nc(0,1)$. To obtain this result, let $\widetilde{\psi}$ be the function such that
% 
\begin{align*}
    \lim_{N \rightarrow \infty} \frac{1}{N} \sum_{i=1}^N \psi\big(Z^i, \EHoutcome{i}{N}{0}, \OVtreatment{i}{u}, \Vcovar{i}\big)
    &=
    \lim_{N \rightarrow \infty} \frac{1}{N} \sum_{i=1}^N \psi\big(Z^i, \outcomeh{0}{}\big(\outcomeD{}{i}{0}, \OVtreatment{i}{u}, \Vcovar{i}; \Eparam{\outcomeh{0}{}}(N)\big), \OVtreatment{i}{u}, \Vcovar{i}\big)
    \\
    &=
    \lim_{N \rightarrow \infty} \frac{1}{N} \sum_{i=1}^N \widetilde\psi\big(Z^i, \outcomeD{}{i}{0}, \OVtreatment{i}{u}, \Vcovar{i}, \Eparam{\outcomeh{0}{}}(N)\big)
    \\
    &\eqp
    \lim_{N \rightarrow \infty} \frac{1}{N} \sum_{i=1}^N \widetilde\psi\big(Z^i, \outcomeD{}{i}{0}, \OVtreatment{i}{u}, \Vcovar{i}, \param{\outcomeh{0}{}}\big)
    \\
    &\eqas
    \E \left[ \widetilde\psi\big(Z, \outcomeD{}{}{0}, \Vtreatment{}{u}, \Vcovar{}, \param{\outcomeh{0}{}}\big)
    \right].
\end{align*}
% 
Above, we used the continuous mapping theorem and Theorem~\ref{thm:SLLN-2} in view of Assumption~\ref{asmp:weak_limits}. Note that $\widetilde{\psi} \in \poly{k}$ because of Assumption~\ref{asmp:BL}-\ref{asmp:BL-pl h-functions}.


\textbf{Induction Hypothesis (IH).} Suppose that the limits in \eqref{eq:detailed_consistency} hold true for $t$ and also
% 
\begin{align}
    \label{eq:consistency_proof_intermediaryP_IH}
    \lim_{N \rightarrow \infty} \frac{1}{N} \sum_{i=1}^N \psi\big(Z^i, \EHoutcome{i}{N}{t-1}, \OVtreatment{i}{u}, \Vcovar{i}\big)
    \eqp
    \E \left[
    \psi\big(Z, \Houtcome{}{}{t-1}, \Vtreatment{}{u}, \Vcovar{}\big)
    \right].
\end{align}
% 

\textbf{Step 2.} We show that $\EAVO{N}{}{t+1} \xrightarrow{P} \MAVO{}{}{t+1}$. By the induction hypothesis and reusing the continuous mapping theorem, we have
% 
\begin{equation}
    \label{eq:consistency_proof_4}
    \begin{aligned}
        \lim_{N \rightarrow \infty} \EAVO{N}{}{t+1} 
        &=
        \lim_{N \rightarrow \infty} \frac{1}{N} \sum_{i=1}^N \outcomeg{t}{}\big(\EAVO{N}{}{t} + \EVVO{N}{}{t} Z^i + \EHoutcome{i}{N}{t-1}, \OVtreatment{i}{u}, \Vcovar{i}; \Eparam{\outcomeg{t}{}}(N)\big)
        \\
        &\eqp
        \lim_{N \rightarrow \infty} \frac{1}{N} \sum_{i=1}^N \outcomeg{t}{}\big(\MAVO{}{}{t} + \MVVO{}{}{t} Z^i + \EHoutcome{i}{N}{t-1}, \OVtreatment{i}{u}, \Vcovar{i}; \param{\outcomeg{t}{}}\big)
        \\
        &\eqp
        \MAVO{}{}{t+1}.
    \end{aligned}
\end{equation}
% 
Similarly, one can establish $\EVVO{N}{}{t+1} \xrightarrow{P} \MVVO{}{}{t+1}$ as well as $\frac{1}{N} \sum_{i=1}^N \EHoutcome{i}{}{t}\xrightarrow{P} \E\left[ \Houtcome{}{}{t} \right]$. We therefore conclude the proof by demonstrating the following result:
% 
\begin{align}
    \label{eq:consistency_proof_intermediaryP_step2}
    \lim_{N \rightarrow \infty} \frac{1}{N} \sum_{i=1}^N \psi\big(Z^i, \EHoutcome{i}{N}{t}, \OVtreatment{i}{u}, \Vcovar{i}\big)
    \eqp
    \E \left[
    \psi\big(Z, \Houtcome{}{}{t}, \Vtreatment{}{u}, \Vcovar{}\big)
    \right].
\end{align}
% 
For this purpose, considering $\EHoutcome{i}{N}{t} = \outcomeh{t}{}\big(\EAVO{N}{}{t} + \EVVO{N}{}{t} Z^i + \EHoutcome{i}{N}{t-1}, \OVtreatment{i}{u}, \Vcovar{i}; \Eparam{\outcomeh{t}{}}(N)\big)$, for a proper choice of the functions $\widetilde{\psi}$ and $\widetilde{\psi}'$, we can write:
% 
\begin{align*}
    \lim_{N \rightarrow \infty} \frac{1}{N} \sum_{i=1}^N \psi\big(Z^i, \EHoutcome{i}{N}{t}, \OVtreatment{i}{u}, \Vcovar{i}\big)
    &=
    \lim_{N \rightarrow \infty} \frac{1}{N} \sum_{i=1}^N \widetilde\psi\big(Z^i, \EAVO{N}{}{t}, \EVVO{N}{}{t}, \EHoutcome{i}{N}{t-1}, \OVtreatment{i}{u}, \Vcovar{i}, \Eparam{\outcomeh{t}{}}(N)\big)
    \\
    &\eqp
    \lim_{N \rightarrow \infty} \frac{1}{N} \sum_{i=1}^N \widetilde\psi\big(Z^i, \MAVO{}{}{t}, \MVVO{}{}{t}, \EHoutcome{i}{N}{t-1}, \OVtreatment{i}{u}, \Vcovar{i}, \param{\outcomeh{t}{}}\big)
    \\
    &=
    \lim_{N \rightarrow \infty} \frac{1}{N} \sum_{i=1}^N \widetilde\psi'\big(Z^i, \EHoutcome{i}{N}{t-1}, \OVtreatment{i}{u}, \Vcovar{i}\big)
    \\
    &\eqp
    \E \left[
    \widetilde\psi' \big(Z, \Houtcome{}{}{t-1}, \Vtreatment{}{u}, \Vcovar{}\big)
    \right],
\end{align*}
% 
where in the last line we used the induction hypothesis. Considering the definition of functions $\widetilde{\psi}$ and $\widetilde{\psi}'$, the proof is complete. \ep 


\subsection{Application to Bernoulli Randomized Design}
\label{sec:application_to_BRD}
%
We showcase the applicability of our framework by considering a Bernoulli randomized design, where each unit $i$ at time $t$ receives treatment with a probability denoted by $\expr_t$. Therefore, $\treatment{i}{t} \sim Bernoulli(\expr_t)$, and the $\treatment{i}{t}$'s are independent across experimental units.

We consider a first-order yet non-linear approximation of functions $\outcomeg{t}{}$ and $\outcomeh{t}{}$ in the outcome specification in \eqref{eq:outcome_function_matrix}. Specifically, we let $\Vcovar{i} = (\BE_{g}^i, \CE_{gl}^i, \ldots, \CE_{g1}^i, \DE_g^i, \PE_g^i, \BE_{h}^i, \CE_{hl}^i, \ldots, \CE_{h1}^i, \DE_h^i, \PE_h^i)^\top$ and define
% 
\begin{equation}
    \label{eq:function_structure}
    \begin{aligned}
        \outcomeg{t}{} \left(
        \outcomeD{}{i}{t-l+1}, \ldots, \outcomeD{}{i}{t},
        \Vtreatment{i}{}, \Vcovar{i}
        \right)
        &= \BE_{g}^i + \CE_{gl}^i \outcomeD{}{i}{t-l+1} + \ldots + \CE_{g1}^i \outcomeD{}{i}{t} + 
        \DE_g^i \treatment{i}{t+1} + \PE_g^i \outcomeD{}{i}{t} \treatment{i}{t+1}
        \\
        \outcomeh{t}{} \left(
        \outcomeD{}{i}{t-l+1}, \ldots, \outcomeD{}{i}{t},
        \Vtreatment{i}{}, \Vcovar{i}
        \right)
        &= \BE_{h}^i + \CE_{hl}^i \outcomeD{}{i}{t-l+1} + \ldots + \CE_{h1}^i \outcomeD{}{i}{t} + 
        \DE_h^i \treatment{i}{t+1} + \PE_h^i \outcomeD{}{i}{t} \treatment{i}{t+1}.
    \end{aligned}
\end{equation}
%
\begin{remark}
    In \eqref{eq:function_structure}, we allow the parameters of functions $\outcomeg{t}{}$ and $\outcomeh{t}{}$ to vary across units. These parameters can be viewed as unobserved unit-specific covariates that characterize how each unit responds to interventions. Additional unit-specific characteristics can be incorporated as observed covariates in the vectors $\Vcovar{i},\; i \in [N]$. We omit these details for brevity.
\end{remark}


We continue by considering a subpopulation of experimental units denoted by $\batch$. We assume that the sampling rule determining $\batch$ depends \emph{only} on the treatment allocations. Because the treatment allocation is independent of other variables, we can assume in the state evolution equations outlined in Eq.~\eqref{eq:state evolution} that $\MIM{\cdot}{\batch}$ and $\Vcovar{\batch}$ are equal to their global counterparts $\MIM{\cdot}{}$ and $\Vcovar{}$, respectively. This reflects the idea that sampling based on treatment allocation is equivalent to random sampling from the experimental population. Thus, by the state evolution equations, for $t = 0, 1, \ldots, l-1$, we can write,
% 
\begin{equation}
    \label{eq:SE_BRD_1}
    \begin{aligned}
        \AVO{}{}{t}
        &=
        \lim_{N \rightarrow \infty}
        \frac{1}{N}
        \sum_{i=1}^N
        \outcomeD{}{i}{t},
        \quad
        \AVO{}{\batch}{t}
        =
        \lim_{N \rightarrow \infty}
        \frac{1}{\cardinality{\batch}}
        \sum_{i \in \batch}
        \outcomeD{}{i}{t},
    \end{aligned}
\end{equation}
% 
and for $t \geq l-1$,
% 
\begin{equation}
    \label{eq:SE_BRD_2}
    \begin{aligned}
        \MAVO{}{}{t+1}
        &=
        \ABE_{g} +
        \ACE_{gl} \AVO{}{}{t-l+1} + \ldots + \ACE_{g1} \AVO{}{}{t} +  
        \ADE_g \expr_{t+1} +
        \APE_g \expr_{t+1} \AVO{}{}{t}
        \\
        \AVO{}{\batch}{t+1}
        &=
        \MAVO{}{}{t+1} +
        \ABE_{h} +
        \ACE_{hl} \AVO{}{\batch}{t-l+1} + \ldots + \ACE_{h1} \AVO{}{\batch}{t} +  
        \ADE_h \expr_{t+1}^\batch +
        \APE_h \expr_{t+1}^\batch \AVO{}{\batch}{t},
    \end{aligned}
\end{equation}
% 
where
% 
\begin{equation}
\label{eq:apndx_BRD_parameters_mean}
\begin{aligned}
    &(\ABE_{g}, \ACE_{gl}, \ldots, \ACE_{g1}, \ADE_g, \APE_g, \ABE_{h}, \ACE_{hl}, \ldots, \ACE_{h1}, \ADE_h, \APE_h)^\top
    \\
    :=
    &\E \left[ 
    \Vcovar{} =
    (\BE_{g}, \CE_{gl}, \ldots, \CE_{g1}, \DE_g, \PE_g, \BE_{h}, \CE_{hl}, \ldots, \CE_{h1}, \DE_h, \PE_h)^\top
    \right].
\end{aligned}
\end{equation}
% 
Here, $\Vcovar{}$ represents the weak limit of $\Vcovar{1}, \ldots, \Vcovar{N}$ when $N \rightarrow \infty$, as specified by Assumption~\ref{asmp:weak_limits}. Then, Equation~\eqref{eq:SE_BRD_2} follows from Assumption~\ref{asmp:independent_interference_mean} and the additional assumption about the elements of $\Vcovar{}$. For example, \eqref{eq:SE_BRD_2} holds when the elements of $\Vcovar{}$ are random variables independent of all other sources of randomness in the model.

To proceed with counterfactual estimation, we consider $b$ distinct subpopulations, denoted by $\batch_1, \ldots, \batch_b$, each determined solely by treatment allocations. With convention $\ABE := \ABE_{g} + \ABE_{h}$ in \eqref{eq:SE_BRD_2}, we can write the following linear regression model:
% 
\begin{equation}
    \label{eq:apndx_BRD_model}
    \begin{aligned}
        \begin{bmatrix}
            \AVO{}{\batch_1}{l}
            \\
            \AVO{}{\batch_1}{l+1}
            \\
            \vdots
            \\
            \AVO{}{\batch_1}{T}
            \\
            \\
            \vdots
            \\
            \\
            \AVO{}{\batch_b}{l}
            \\
            \AVO{}{\batch_b}{l+1}
            \\
            \vdots
            \\
            \AVO{}{\batch_b}{T}
        \end{bmatrix}
        =
        \ABE
        \begin{bmatrix}
            1
            \\
            1
            \\
            \vdots
            \\
            1
            \\
            \\
            \vdots
            \\
            \\
            1
            \\
            1
            \\
            \vdots
            \\
            1
        \end{bmatrix}
        &+ \ACE_{gl}
        \begin{bmatrix}
            \AVO{}{}{0}
            \\
            \AVO{}{}{1}
            \\
            \vdots
            \\
            \AVO{}{}{T-l}
            \\
            \\
            \vdots
            \\
            \\
            \AVO{}{}{0}
            \\
            \AVO{}{}{1}
            \\
            \vdots
            \\
            \AVO{}{}{T-l}
        \end{bmatrix}
        +
        \ldots
        + \ACE_{g1}
        \begin{bmatrix}
            \AVO{}{}{l-1}
            \\
            \AVO{}{}{l}
            \\
            \vdots
            \\
            \AVO{}{}{T-1}
            \\
            \\
            \vdots
            \\
            \\
            \AVO{}{}{l-1}
            \\
            \AVO{}{}{l}
            \\
            \vdots
            \\
            \AVO{}{}{T-1}
        \end{bmatrix}
        + \ADE_g 
        \begin{bmatrix}
            \expr_{l}
            \\
            \expr_{l+1}
            \\
            \vdots
            \\
            \expr_{T}
            \\
            \\
            \vdots
            \\
            \\
            \expr_{l}
            \\
            \expr_{l+1}
            \\
            \vdots
            \\
            \expr_{T}
        \end{bmatrix}
        + \APE_g  
        \begin{bmatrix}
            \expr_{l}\AVO{}{}{l-1}
            \\
            \expr_{l+1}\AVO{}{}{l}
            \\
            \vdots
            \\
            \expr_{T}\AVO{}{}{T-1}
            \\
            \\
            \vdots
            \\
            \\
            \expr_{l}\AVO{}{}{l-1}
            \\
            \expr_{l+1}\AVO{}{}{l}
            \\
            \vdots
            \\
            \expr_{T}\AVO{}{}{T-1}
        \end{bmatrix}
        \\
        &+ \ACE_{hl}
        \begin{bmatrix}
            \AVO{}{\batch_1}{0}
            \\
            \AVO{}{\batch_1}{1}
            \\
            \vdots
            \\
            \AVO{}{\batch_1}{T-l}
            \\
            \\
            \vdots
            \\
            \\
            \AVO{}{\batch_b}{0}
            \\
            \AVO{}{\batch_b}{1}
            \\
            \vdots
            \\
            \AVO{}{\batch_b}{T-l}
        \end{bmatrix}
        +
        \ldots
        + \ACE_{h1}
        \begin{bmatrix}
            \AVO{}{\batch_1}{l-1}
            \\
            \AVO{}{\batch_1}{l}
            \\
            \vdots
            \\
            \AVO{}{\batch_1}{T-1}
            \\
            \\
            \vdots
            \\
            \\
            \AVO{}{\batch_b}{l-1}
            \\
            \AVO{}{\batch_b}{l}
            \\
            \vdots
            \\
            \AVO{}{\batch_b}{T-1}
        \end{bmatrix}
        + \ADE_h 
        \begin{bmatrix}
            \expr_{l}^{\batch_1}
            \\
            \expr_{l+1}^{\batch_1}
            \\
            \vdots
            \\
            \expr_{T}^{\batch_1}
            \\
            \\
            \vdots
            \\
            \\
            \expr_{l}^{\batch_b}
            \\
            \expr_{l+1}^{\batch_b}
            \\
            \vdots
            \\
            \expr_{T}^{\batch_b}
        \end{bmatrix}
        + \APE_h
        \begin{bmatrix}
            \expr_{l}^{\batch_1} \AVO{}{\batch_1}{l-1}
            \\
            \expr_{l+1}^{\batch_1} \AVO{}{\batch_1}{l}
            \\
            \vdots
            \\
            \expr_{T}^{\batch_1} \AVO{}{\batch_1}{T-1}
            \\
            \\
            \vdots
            \\
            \\
            \expr_{l}^{\batch_b} \AVO{}{\batch_b}{l-1}
            \\
            \expr_{l+1}^{\batch_b} \AVO{}{\batch_b}{l}
            \\
            \vdots
            \\
            \expr_{T}^{\batch_b} \AVO{}{\batch_b}{T-1}
        \end{bmatrix},
    \end{aligned}
\end{equation}
%
or equivalently in a matrix form
% 
\begin{align*}
    \Vec{\Yc} = \Xc (\ABE, \ACE_{gl}, \ldots, \ACE_{g1}, \ADE_g, \APE_g, \ACE_{hl}, \ldots, \ACE_{h1}, \ADE_h, \APE_h)^\top,
\end{align*}
% 
where $\Vec{\Yc}$ represents the vector on the left-hand side of \eqref{eq:apndx_BRD_model}, while $\Xc$ denotes the matrix formed by the columns corresponding to the vectors on the right-hand side of \eqref{eq:apndx_BRD_model}.

For the regression model outlined in \eqref{eq:apndx_BRD_model}, we now show that the least squares estimator provides a strongly consistent estimate for the unknown coefficients. To this end, upon observing $\OMtreatment{}{}$ and $\Moutcome{}{}{}(\OMtreatment{}{})$ within a system of $N$ experimental units, we define the following:
%
\begin{align}
    \label{eq:apndx_BRD_reg_model}
    \Vec{\Yc}(N) \sim \Xc(N) \Vec{\Bc}(N),
\end{align}
%
where $\Vec{\Yc}(N)$ denotes the sample mean of observed outcomes over time and across various subpopulations, corresponding to the vector on the left-hand side of \eqref{eq:apndx_BRD_model}. The vector $\Vec{\Bc}(N)$ represents the unknown coefficients on the right-hand side of \eqref{eq:apndx_BRD_model}, while $\Xc(N)$ denotes a $b(T-l+1)$ by $1+l+2+l+2$ matrix. This matrix is aligned with the vectors on the right-hand side of \eqref{eq:apndx_BRD_model}, but with the corresponding sample means of observed outcomes and treatments replacing the asymptotic terms, see \eqref{eq:reg_cover_matrix}.
% 
\begin{proposition}
    \label{prp:BRD_consistency}
    % 
    Suppose that $\Xc(N)^\top \Xc(N)$ is invertible. Let $\Vec{\widehat{\Bc}}(N) := \big( \Xc(N)^\top \Xc(N) \big)^{-1} \Xc(N)^\top \Vec{\Yc}(N)$ be the least squares estimator. Then, $\Vec{\widehat{\Bc}}(N)$ provides a strongly consistent estimator for the coefficient vector $(\ABE, \ACE_{gl}, \ldots, \ACE_{g1}, \ADE_g, \APE_g, \ACE_{hl}, \ldots, \ACE_{h1}, \ADE_h, \APE_h)^\top$.
    % 
\end{proposition}
% 
\textbf{Proof.}
% 
Because the matrix $\Xc(N)^\top \Xc(N)$ is invertible, the estimator $\Vec{\widehat{\Bc}}(N)$ is a continuous function of the input data. Therefore, we can pass the limit through the estimator function as:
% 
\begin{align*}
    \lim_{N \rightarrow \infty} \Vec{\widehat{\Bc}}(N)
    =
    \left( \Big(\lim_{N \rightarrow \infty} \Xc(N)\Big)^\top \Big(\lim_{N \rightarrow \infty} \Xc(N)\Big) \right)^{-1} \Big(\lim_{N \rightarrow \infty} \Xc(N)\Big)^\top \Big(\lim_{N \rightarrow \infty} \Vec{\Yc}(N)\Big)
    \eqas
    ( \Xc^\top \Xc )^{-1} \Xc^\top \Vec{\Yc},
\end{align*}
% 
where we used the result of Theorem~\ref{thm:Batch_SE}. Now, note that \eqref{eq:apndx_BRD_model} defines a deterministic regression model, and $( \Xc^\top \Xc )^{-1} \Xc^\top \Vec{\Yc} = (\ABE, \ACE_{gl}, \ldots, \ACE_{g1}, \ADE_g, \APE_g, \ACE_{hl}, \ldots, \ACE_{h1}, \ADE_h, \APE_h)^\top$. This concludes the proof. \ep

To ensure the invertibility condition of the matrix $\Xc(N)^\top \Xc(N)$, we need to set some simple conditions on the experimental design. Basically, conducting the experiment in more than one stage (equivalent to having two distinct values for elements of the vector $\big( \hat\expr_l, \ldots, \hat\expr_T \big)^\top$ should suffice. This is needed to ensure that the first column of $\Xc(N)$ (i.e., $\Vec{1}_{b(T-l+1)}$) is linearly independent of the column $\big(\hat\expr_l, \ldots, \hat\expr_T, \ldots, \hat\expr_l, \ldots, \hat\expr_T\big)^\top$. 

Assuming a non-zero treatment effect (i.e., $\ADE_h \neq 0$), we can choose the batches to ensure enough variation across different batches. Therefore, upon a careful batching, columns of $\Xc(N)$ are linearly independent as each has its own specific variation patterns over time and/or subpopulations. 

Although the exact value of $\ADE$ is unknown, we suppose that contextual information suggests the presence of a non-zero direct treatment effect. In the case where $\ADE = 0$, according to contextual information, both the subpopulation sample mean $\AVO{}{\batch}{t}$ and the population sample mean $\AVO{}{}{t}$ are equal in \eqref{eq:SE_BRD_1} and  \eqref{eq:SE_BRD_2}, allowing for further simplification of the underlying model.
% 
\setcounter{MaxMatrixCols}{11}
\begin{equation}
\label{eq:reg_cover_matrix}
    \begin{aligned}
        \Xc(N)
        =
        \begin{bmatrix}
        1
        &\HAVO{}{}{0}
        &\ldots
        &\HAVO{}{}{l-1}
        &\Oexpr{}{}{l}
        &\Oexpr{}{}{l} \HAVO{}{}{l-1} 
        &\HAVO{}{\batch_1}{0}
        &\ldots
        &\HAVO{}{\batch_1}{l-1}
        &\Oexpr{}{\batch_1}{l}
        &\Oexpr{}{\batch_1}{l} \HAVO{}{\batch_1}{l-1} 
        % 
        \\
        % 
        1
        &\HAVO{}{}{1}
        &\ldots
        &\HAVO{}{}{l}
        &\Oexpr{}{}{l+1}
        &\Oexpr{}{}{l+1} \HAVO{}{}{l} 
        &\HAVO{}{\batch_1}{1}
        &\ldots
        &\HAVO{}{\batch_1}{l}
        &\Oexpr{}{\batch_1}{l+1}
        &\Oexpr{}{\batch_1}{l+1} \HAVO{}{\batch_1}{l}
        % 
        \\
        % 
        \vdots
        &\vdots
        &\vdots
        &\vdots
        &\vdots
        &\vdots
        &\vdots
        &\vdots
        &\vdots
        &\vdots
        &\vdots
        % 
        \\
        % 
        1
        &\HAVO{}{}{T-l}
        &\ldots
        &\HAVO{}{}{T-1}
        &\Oexpr{}{}{T}
        &\Oexpr{}{}{T} \HAVO{}{}{T-1} 
        &\HAVO{}{\batch_1}{T-l}
        &\ldots
        &\HAVO{}{\batch_1}{T-1}
        &\Oexpr{}{\batch_1}{T}
        &\Oexpr{}{\batch_1}{T} \HAVO{}{\batch_1}{T-1}
        % 
        \\
        \\
        % 
        \vdots
        &\vdots
        &\vdots
        &\vdots
        &\vdots
        &\vdots
        &\vdots
        &\vdots
        &\vdots
        &\vdots
        &\vdots
        % 
        \\
        \\
        % 
        1
        &\HAVO{}{}{0}
        &\ldots
        &\HAVO{}{}{l-1}
        &\Oexpr{}{}{l}
        &\Oexpr{}{}{l} \HAVO{}{}{l-1} 
        &\HAVO{}{\batch_b}{0}
        &\ldots
        &\HAVO{}{\batch_b}{l-1}
        &\Oexpr{}{\batch_b}{l}
        &\Oexpr{}{\batch_b}{l} \HAVO{}{\batch_b}{l-1} 
        % 
        \\
        % 
        1
        &\HAVO{}{}{1}
        &\ldots
        &\HAVO{}{}{l}
        &\Oexpr{}{}{l+1}
        &\Oexpr{}{}{l+1} \HAVO{}{}{l} 
        &\HAVO{}{\batch_b}{1}
        &\ldots
        &\HAVO{}{\batch_b}{l}
        &\Oexpr{}{\batch_b}{l+1}
        &\Oexpr{}{\batch_b}{l+1} \HAVO{}{\batch_b}{l}
        % 
        \\
        % 
        \vdots
        &\vdots
        &\vdots
        &\vdots
        &\vdots
        &\vdots
        &\vdots
        &\vdots
        &\vdots
        &\vdots
        &\vdots
        % 
        \\
        % 
        1
        &\HAVO{}{}{T-l}
        &\ldots
        &\HAVO{}{}{T-1}
        &\Oexpr{}{}{T}
        &\Oexpr{}{}{T} \HAVO{}{}{T-1} 
        &\HAVO{}{\batch_b}{T-l}
        &\ldots
        &\HAVO{}{\batch_b}{T-1}
        &\Oexpr{}{\batch_b}{T}
        &\Oexpr{}{\batch_b}{T} \HAVO{}{\batch_b}{T-1}
        \end{bmatrix}.
    \end{aligned}
\end{equation}


\subsection{First-order Estimators}
\label{apndx:estimators}
% 
Given the observed outcomes $\Moutcome{}{}{}(\OMtreatment{o}{})$, we can leverage Theorem~\ref{thm:consistency} and Proposition~\ref{prp:BRD_consistency} to consistently estimate counterfactuals under a desired treatment allocation $\OMtreatment{u}{}$. To this end, we propose two closely related families of estimators, which we detail below. Both family of estimators require that the delivered treatment allocation $\OMtreatment{o}{}$ and the desired treatment allocation $\OMtreatment{u}{}$ match during the first $l$ periods. These initial $l$ periods serve as the common foundation from which counterfactual trajectories are constructed.

Given subpopulation $\batch$, Algorithms~\ref{alg:FO-semi-recursive} and \ref{alg:FO-recursive} aim to estimate counterfactuals under the desired treatment allocation over $\batch$ using $b$ distinct subpopulations $\batch_1, \ldots, \batch_b$ as the estimation samples. Both algorithms share their first two steps. In the first step, they compute sample means of observed outcomes for both the entire population and each subpopulation, along with sample means of delivered and desired treatment allocations. The second step estimates unknown parameters in the state evolution equation \eqref{eq:SE_BRD_2} using least squares estimation as detailed in Proposition~\ref{prp:BRD_consistency}. The algorithms then diverge in their third step, applying these results through two distinct approaches detailed below. The consistency proofs for both algorithms follow directly from earlier results and are omitted for brevity.

\subsubsection{Semi-recursive Estimation Method}
\label{apndx:semi-recursive_estimators}
% 
This estimator, outlined in Algorithm~\ref{alg:FO-semi-recursive}, builds on the observed sample means in its third step. It uses the parameter estimates from the second step and the state evolution equation \eqref{eq:SE_BRD_2} to modify the observed sample means by adjusting the treatment level to the desired one. This approach transfers the original complexities of the observed outcomes to the estimated counterfactual, making it particularly suitable for scenarios with strong time trends or complex baselines. This estimator generalizes the algorithm proposed in \cite{shirani2024causal} by accommodating broader model classes and providing more general estimands.

% 
\begin{algorithm}
\caption{First-order semi-recursive counterfactual estimator}
\label{alg:FO-semi-recursive}
% 
\begin{algorithmic}
% 
\Require $\Moutcome{}{}{}(\OMtreatment{o}{}), \OMtreatment{o}{}$, $\OMtreatment{u}{}$, estimation batch $\batch$, sample batches $\batch_1, \ldots, \batch_b$, and $l$
% 

\State \hspace{-1.3em} \textbf{Step 1: Data processing}
% 
\For{$t = 0, \ldots, T$}
    % 
    \State $\HAVO{}{}{t}
        \gets
        \frac{1}{N} \sum_{i=1}^N \outcomeD{}{i}{t} (\OMtreatment{o}{})$
    % 
    \State $\HAVO{}{\batch}{t}
        \gets
        \frac{1}{\cardinality{\batch}} \sum_{i \in  \batch} \outcomeD{}{i}{t} (\OMtreatment{o}{})$
    % 
    \State $\Oexpr{}{}{t}
        \gets
        \frac{1}{N} \sum_{i=1}^N \Otreatment{i}{o,t}{}$
     % 
    \State $\Oexpr{}{\batch}{t}
        \gets
        \frac{1}{\cardinality{\batch}} \sum_{i \in  \batch} \Otreatment{i}{o,t}{}$
    % 
    \State $\Dexpr{}{}{t}
        \gets
        \frac{1}{N} \sum_{i=1}^N \Otreatment{i}{u,t}{}$
    % 
    \State $\Dexpr{}{\batch}{t}
        \gets
        \frac{1}{\cardinality{\batch}} \sum_{i \in  \batch} \Otreatment{i}{u,t}{}$
    % 
    \For{$j = 1, \ldots, b$}
        \State $\HAVO{}{\batch_j}{t}
        \gets
        \frac{1}{\cardinality{\batch_j}} \sum_{i \in  \batch_j} \outcomeD{}{i}{t} (\OMtreatment{o}{})$
        % 
        \State $\Oexpr{}{\batch_j}{t}
        \gets
        \frac{1}{\cardinality{\batch_j}} \sum_{i \in  \batch_j} \Otreatment{i}{o,t}{}$
    \EndFor    
% 
\EndFor
% 

\State \hspace{-1.3em} \textbf{Step 2: Parameters estimation}
% 
\State $(\EBE, \ECE_{gl}, \ldots, \ECE_{g1}, \EDE_g, \EPE_g, \ECE_{hl}, \ldots, \ECE_{h1}, \EDE_h, \EPE_h)^\top \gets \big( \hat{\Xc}^\top \hat{\Xc} \big)^{-1} \hat{\Xc}^\top \Vec{\hat{\Yc}}$


\State \hspace{-1.3em} \textbf{Step 3: Counterfactual estimation}

\State $\Big(\ECF{}{0}{\OMtreatment{u}{}}, \ldots, \ECF{}{l-1}{\OMtreatment{u}{}} \Big) \gets \left(\HAVO{}{}{0}, \ldots, \HAVO{}{}{l-1} \right)$

\State $\Big(\ECF{\batch}{0}{\OMtreatment{u}{}}, \ldots, \ECF{\batch}{l-1}{\OMtreatment{u}{}} \Big) \gets \left(\HAVO{}{\batch}{0}, \ldots, \HAVO{}{\batch}{l-1} \right)$

\For{$t = l, \ldots, T$}
% 
    \State $\Rc_g \gets \sum_{j=1}^l \ECE_{gj} (\ECF{}{t-j}{\OMtreatment{u}{}} - \HAVO{}{}{t-j}) + \EDE_g (\Dexpr{}{}{t}-\Oexpr{}{}{t}) + \EPE_g (\Dexpr{}{}{t} \ECF{}{t-1}{\OMtreatment{u}{}} - \Oexpr{}{}{t} \HAVO{}{}{t-1})$
    % 
    \State $\Rc_h \gets \sum_{j=1}^l \ECE_{hj} (\ECF{}{t-j}{\OMtreatment{u}{}} - \HAVO{}{}{t-j}) + \EDE_h (\Dexpr{}{}{t}-\Oexpr{}{}{t}) + \EPE_h (\Dexpr{}{}{t} \ECF{}{t-1}{\OMtreatment{u}{}} - \Oexpr{}{}{t} \HAVO{}{}{t-1})$
    % 
    \State $\Rc_h^\batch \gets \sum_{j=1}^l \ECE_{hj} (\ECF{\batch}{t-j}{\OMtreatment{u}{}} - \HAVO{}{\batch}{t-j}) + \EDE_h (\Dexpr{}{\batch}{t}-\Oexpr{}{\batch}{t}) + \EPE_h (\Dexpr{}{\batch}{t} \ECF{\batch}{t-1}{\OMtreatment{u}{}} - \Oexpr{}{\batch}{t} \HAVO{}{\batch}{t-1})$
    % 
    \State $\ECF{}{t}{\OMtreatment{u}{}} \gets \HAVO{}{}{t} + \Rc_g + \Rc_h$
    % 
    \State $\ECF{\batch}{t}{\OMtreatment{u}{}} \gets \HAVO{}{\batch}{t} + \Rc_g + \Rc_h^\batch$
% 
\EndFor

\Ensure $\ECF{}{t}{\OMtreatment{u}{}}$ and $\ECF{\batch}{t}{\OMtreatment{u}{}}$, for $t = 0, \ldots, T$.
% 
\end{algorithmic}
\end{algorithm}
% 

\subsubsection{Recursive Estimation Method}
\label{apndx:recursive_estimators}
% 
This estimator, outlined in Algorithm~\ref{alg:FO-recursive}, directly leverages the state evolution equation and parameter estimates from the second step. Specifically, it estimates counterfactuals recursively, using only the past $l$ terms and desired treatment levels. Unlike Algorithm~\ref{alg:FO-semi-recursive}, this algorithm can estimate counterfactuals even for time blocks where no outcome data were collected. For example, having observed data until December, it can predict counterfactual outcomes for January without requiring any observations during this month.



% 
\begin{algorithm}
\caption{First-order recursive counterfactual estimator}
\label{alg:FO-recursive}
% 
\begin{algorithmic}
% 
% 
\Require $\Moutcome{}{}{}(\OMtreatment{o}{}), \OMtreatment{o}{}$, $\OMtreatment{u}{}$, estimation batch $\batch$, sample batches $\batch_1, \ldots, \batch_b$, and $l$
% 

\State \hspace{-1.3em} \textbf{Step 1: Data processing}
% 
\For{$t = 0, \ldots, T$}
    % 
    \State $\HAVO{}{}{t}
        \gets
        \frac{1}{N} \sum_{i=1}^N \outcomeD{}{i}{t} (\OMtreatment{o}{})$
    % 
    \State $\HAVO{}{\batch}{t}
        \gets
        \frac{1}{\cardinality{\batch}} \sum_{i \in  \batch} \outcomeD{}{i}{t} (\OMtreatment{o}{})$
    % 
    \State $\Oexpr{}{}{t}
        \gets
        \frac{1}{N} \sum_{i=1}^N \Otreatment{i}{o,t}{}$
     % 
    \State $\Oexpr{}{\batch}{t}
        \gets
        \frac{1}{\cardinality{\batch}} \sum_{i \in  \batch} \Otreatment{i}{o,t}{}$
    % 
    \State $\Dexpr{}{}{t}
        \gets
        \frac{1}{N} \sum_{i=1}^N \Otreatment{i}{u,t}{}$
    % 
    \State $\Dexpr{}{\batch}{t}
        \gets
        \frac{1}{\cardinality{\batch}} \sum_{i \in  \batch} \Otreatment{i}{u,t}{}$
    % 
    \For{$j = 1, \ldots, b$}
        \State $\HAVO{}{\batch_j}{t}
        \gets
        \frac{1}{\cardinality{\batch_j}} \sum_{i \in  \batch_j} \outcomeD{}{i}{t} (\OMtreatment{o}{})$
        % 
        \State $\Oexpr{}{\batch_j}{t}
        \gets
        \frac{1}{\cardinality{\batch_j}} \sum_{i \in  \batch_j} \Otreatment{i}{o,t}{}$
    \EndFor    
% 
\EndFor
% 

\State \hspace{-1.3em} \textbf{Step 2: Parameters estimation}
% 
\State $(\EBE, \ECE_{gl}, \ldots, \ECE_{g1}, \EDE_g, \EPE_g, \ECE_{hl}, \ldots, \ECE_{h1}, \EDE_h, \EPE_h)^\top \gets \big( \hat{\Xc}^\top \hat{\Xc} \big)^{-1} \hat{\Xc}^\top \Vec{\hat{\Yc}}$


\State \hspace{-1.3em} \textbf{Step 3: Counterfactual estimation}

\State $\Big(\ECF{}{0}{\OMtreatment{u}{}}, \ldots, \ECF{}{l-1}{\OMtreatment{u}{}} \Big) \gets \left(\HAVO{}{}{0}, \ldots, \HAVO{}{}{l-1} \right)$

\State $\Big(\ECF{\batch}{0}{\OMtreatment{u}{}}, \ldots, \ECF{\batch}{l-1}{\OMtreatment{u}{}} \Big) \gets \left(\HAVO{}{\batch}{0}, \ldots, \HAVO{}{\batch}{l-1} \right)$

\For{$t = l, \ldots, T$}
% 
    \State $\Rc_g \gets \sum_{j=1}^l \ECE_{gj} \ECF{}{t-j}{\OMtreatment{u}{}} + \EDE_g \Dexpr{}{}{t} + \EPE_g \Dexpr{}{}{t} \ECF{}{t-1}{\OMtreatment{u}{}}$
    % 
    \State $\Rc_h \gets \sum_{j=1}^l \ECE_{hj} \ECF{}{t-j}{\OMtreatment{u}{}} + \EDE_h \Dexpr{}{}{t} + \EPE_h \Dexpr{}{}{t} \ECF{}{t-1}{\OMtreatment{u}{}}$
    % 
    \State $\Rc_h^\batch \gets \sum_{j=1}^l \ECE_{hj} \ECF{\batch}{t-j}{\OMtreatment{u}{}} + \EDE_h \Dexpr{}{\batch}{t} + \EPE_h \Dexpr{}{\batch}{t} \ECF{\batch}{t-1}{\OMtreatment{u}{}}$
    % 
    \State $\ECF{}{t}{\OMtreatment{u}{}} \gets \Rc_g + \Rc_h$
    % 
    \State $\ECF{\batch}{t}{\OMtreatment{u}{}} \gets \Rc_g + \Rc_h^\batch$
% 
\EndFor

\Ensure $\ECF{}{t}{\OMtreatment{u}{}}$ and $\ECF{\batch}{t}{\OMtreatment{u}{}}$, for $t = 0, \ldots, T$.
% 
\end{algorithmic}
\end{algorithm}
% 


% Then, by Theorem~\ref{thm:consistency}, applying Algorithm~\ref{alg:CF estimation} yields strongly consistent counterfactual estimates for any desired treatment allocation.

\subsection{Higher-order Recursive Estimators}
\label{apndx:HO_recursive_estimators}
% 
In light of Theorem~\ref{thm:consistency}, we can extend our approach to utilize higher-order approximations of $\outcomeg{t}{}$ and $\outcomeh{t}{}$. The estimator, outlined in Algorithm~\ref{alg:HO-recursive} for $l=1$ lag terms, incorporates up to order $m \geq 2$ moments of the unit outcomes. Precisely, we introduce two families of feature functions: $\Vec\phi = (\phi_1, \ldots, \phi_{n_1})^\top$ for population-level moments and $\Vec\psi = (\psi_1, \ldots, \psi_{n_2})^\top$ for subpopulation-level moments. The approach employs linear regression to estimate weights for a linear combination of these features. This can be realized as a generalization of \cite{bayati2024higher}'s method and enables capturing more complex patterns in counterfactuals by leveraging richer information about the outcomes' distributions over time. When the feature functions $\Vec\phi$ and $\Vec\psi$ are continuous, consistency follows from Theorem~\ref{thm:consistency}, though we defer rigorous treatment to future work.


% 
\begin{algorithm}
\caption{Higher-order recursive counterfactual estimator}
\label{alg:HO-recursive}
% 
\small
\begin{algorithmic}
% 
% 
\Require $\Moutcome{}{}{}(\OMtreatment{o}{}), \OMtreatment{o}{}$, $\OMtreatment{u}{}$, $\batch_1, \ldots, \batch_b$, $m \geq 2$, $\Vec\phi = (\phi_1, \ldots, \phi_{n_1})^\top$, and $\Vec\psi = (\psi_1, \ldots, \psi_{n_2})^\top$
% 

\State \hspace{-1.3em} \textbf{Step 1: Data processing}
% 
    \For{$t = 0, \ldots, T$}
    % 
    \State $\HAVO{}{}{t}
        \gets
        \frac{1}{N} \sum_{i=1}^N \outcomeD{}{i}{t} (\OMtreatment{o}{})$
    % 
    \State $\HAVO{}{\batch}{t}
        \gets
        \frac{1}{\cardinality{\batch}} \sum_{i \in \batch} \outcomeD{}{i}{t} (\OMtreatment{o}{})$
    % 
    \For{$k = 2, \ldots, m$}
    % 
        \State $\HVVO{}{(k)}{t}
        \gets
        \frac{1}{N} \sum_{i=1}^N 
        \left(
        \outcomeD{}{i}{t} (\OMtreatment{o}{})
        -
        \HAVO{}{}{t}
        \right)^k$
        % 
        \State $\HVVO{}{\batch,(k)}{t}
        \gets
        \frac{1}{\cardinality{\batch}} \sum_{i \in \batch}
        \left(
        \outcomeD{}{i}{t} (\OMtreatment{o}{})
        -
        \HAVO{}{\batch}{t}
        \right)^k$
    \EndFor
    % 
    \State $\Oexpr{}{}{t}
        \gets
        \frac{1}{N} \sum_{i=1}^N \Otreatment{i}{o,t}{}$
    % 
    \State $\Oexpr{}{\batch}{t}
        \gets
        \frac{1}{\cardinality{\batch}} \sum_{i \in \batch} \Otreatment{i}{o,t}{}$
    % 
    \State $\Dexpr{}{}{t}
        \gets
        \frac{1}{N} \sum_{i=1}^N \Otreatment{i}{u,t}{}$
    % 
    \State $\Dexpr{}{\batch}{t}
        \gets
        \frac{1}{\cardinality{\batch}} \sum_{i \in \batch} \Otreatment{i}{u,t}{}$
    % 
    \For{$j = 1, \ldots, b$}
        \State $\HAVO{}{\batch_j}{t}
        \gets
        \frac{1}{\cardinality{\batch_j}} \sum_{i \in  \batch_j} \outcomeD{}{i}{t} (\OMtreatment{o}{})$
        % 
        \For{$k = 2, \ldots, m$}
        % 
            \State $\HVVO{}{\batch_j,(k)}{t}
            \gets
            \frac{1}{\cardinality{\batch_j}} \sum_{i \in  \batch_j} 
            \left(
            \outcomeD{}{i}{t} (\OMtreatment{o}{})
            -
            \HAVO{}{\batch_j}{t}
            \right)^k$
        \EndFor
        % 
        \State $\Oexpr{}{\batch_j}{t}
        \gets
        \frac{1}{\cardinality{\batch_j}} \sum_{i \in  \batch_j} \Otreatment{i}{o,t}{}$
    \EndFor    
% 
\EndFor
% 

\State \hspace{-1.3em} \textbf{Step 2: Parameters estimation}
% 
\State Estimate $\bm{\Theta}_g \in \R^{m \times n_1}$ and $\bm{\Theta}_h  \in \R^{m \times n_2}$ as $\widehat{\bm{\Theta}}_g$ and $\widehat{\bm{\Theta}}_h$: 
% 
\begin{equation*}
    % \label{eq:HO_parameter_estimation}
    \begin{aligned}
        (\HAVO{}{\batch_j}{t+1}, \HVVO{}{\batch_j,(2)}{t+1}, \ldots, \HVVO{}{\batch_j,(m)}{t+1})^\top
        =
        \bm{\Theta}_g \Vec{\phi}(\HAVO{}{}{t}, \HVVO{}{(2)}{t}, \ldots, \HVVO{}{(m)}{t}, \Oexpr{}{}{t+1})
        +
        \bm{\Theta}_h \Vec{\psi}(\HAVO{}{\batch_j}{t}, \HVVO{}{\batch_j,(2)}{t}, \ldots, \HVVO{}{\batch_j,(m)}{t}, \Oexpr{}{\batch_j}{t+1}),
    \end{aligned}
\end{equation*}
% 
where $j = 1, \ldots, b$ and $t = 0, \ldots, T-1$.

\State \hspace{-1.3em} \textbf{Step 3: Counterfactual estimation}

\State $\ECF{}{0}{\OMtreatment{u}{}} \gets \HAVO{}{}{0}$ and $(\DHVVO{}{(2)}{0}, \ldots, \DHVVO{}{(m)}{0}) \gets (\HVVO{}{(2)}{0}, \ldots, \HVVO{}{(m)}{0})$
% 
\State $\ECF{\batch}{0}{\OMtreatment{u}{}} \gets \HAVO{}{\batch}{0}$ and $(\DHVVO{}{\batch,(2)}{0}, \ldots, \DHVVO{}{\batch,(m)}{0}) \gets (\HVVO{}{\batch,(2)}{0}, \ldots, \HVVO{}{\batch,(m)}{0})$
% 
\For{$t = 1, \ldots, T$}
% 
    \State $\Vec\Rc_g \gets
    \widehat{\bm{\Theta}}_g \Vec{\phi}(\ECF{}{t-1}{\OMtreatment{u}{}}, \DHVVO{}{(2)}{t-1}, \ldots, \DHVVO{}{(m)}{t-1}, , \Dexpr{}{}{t})$
    % 
    \State $\Vec\Rc_h \gets
    \widehat{\bm{\Theta}}_h \Vec{\psi}(\ECF{}{t-1}{\OMtreatment{u}{}}, \DHVVO{}{(2)}{t-1}, \ldots, \DHVVO{}{(m)}{t-1}, \Dexpr{}{}{t})$
    % 
    \State $\Vec\Rc_h^\batch \gets
    \widehat{\bm{\Theta}}_h \Vec{\psi}(\ECF{\batch}{t-1}{\OMtreatment{u}{}}, \DHVVO{}{\batch,(2)}{t-1}, \ldots, \DHVVO{}{\batch,(m)}{t-1}, \Dexpr{}{\batch}{t})$
    % 
    \State $(\ECF{}{t}{\OMtreatment{u}{}}, \DHVVO{}{(2)}{t}, \ldots, \DHVVO{}{(m)}{t})^\top \gets
    \Vec\Rc_g + \Vec\Rc_h$
    % 
    \State $(\ECF{\batch}{t}{\OMtreatment{u}{}}, \DHVVO{}{\batch,(2)}{t}, \ldots, \DHVVO{}{\batch,(m)}{t})^\top \gets
    \Vec\Rc_g + \Vec\Rc_h^\batch$
% 
\EndFor

\Ensure $\ECF{}{t}{\OMtreatment{u}{}}$ and $\ECF{\batch}{t}{\OMtreatment{u}{}}$, for $t = 0, \ldots, T$.
% 
\end{algorithmic}
\end{algorithm}
% 


\section{Detrending for Temporal Patterns}
\label{sec:preprocessing}
% 
While semi-recursive estimators (see \S\ref{apndx:semi-recursive_estimators}) can capture complex temporal patterns in unit outcomes, they face two major limitations. First, they cannot estimate out-of-sample counterfactuals because their architecture relies directly on observed sample means. Second, unlike recursive estimators (see Algorithm~\ref{alg:HO-recursive}), they cannot accommodate higher-order approximations of the outcome functions ($\outcomeg{t}{}$ and $\outcomeh{t}{}$). This limitation stems from their dependence on the closed-form of state evolution equation (as outlined in \eqref{eq:SE_BRD_2}), which may not exist for more complex characterizations of the outcome functions.

This section develops a two-stage estimation method that combines the advantages of both semi-recursive and recursive estimators. Although it requires an additional structural assumption on the outcome specification, this approach can handle complex temporal patterns while enabling both out-of-sample counterfactual estimation and higher-order approximations of the outcome functions. The method proceeds as follows: first, we employ a semi-recursive estimator with sufficient lag terms to accurately estimate temporal patterns. We use this to estimate the baseline outcome means (the counterfactual for all control units: $\CFE{\mathbf{0}}{}{t},\; t =0, 1, \ldots, T$). Next, we detrend the observed outcomes by subtracting the estimated baseline from them. We then apply a recursive estimator focused specifically on estimating treatment effects in the absence of temporal patterns. Finally, we add back the subtracted baseline to obtain the desired estimand (see Algorithm~\ref{alg:FO_with_preprocessing}). The following sections provide more detailed expositions of the algorithm.


\subsection{Baseline Outcome Estimation}
\label{sec:Y0_estimation}
% 
Letting $\VOoutcomeD{}{}{t} := \VoutcomeD{}{}{t}(\Mtreatment{}{}=\mathbf{0})$ denote the vector of baseline outcomes at time $t= 0, 1, \ldots, T$ under no treatment, we can write from \eqref{eq:outcome_function_matrix}:
% 
\begin{equation}
\label{eq:outcome_function_NoTreatment}
\begin{aligned}
    \VOoutcomeD{}{}{t+1}
    =
    \VoutcomeD{}{}{t+1}(\Mtreatment{}{}=\mathbf{0})
    &=
    \big(\IM+\IMatT{t}\big)\outcomeg{t}{}\left(\VOoutcomeD{}{}{t}, \mathbf{0}, \covar\right)
    +
    \outcomeh{t}{}\left(\VOoutcomeD{}{}{t}, \mathbf{0}, \covar\right).
\end{aligned}
\end{equation}
% 
Thus, the matrix $\MOoutcomeD{}{}{} = [\VOoutcomeD{}{}{0}| \ldots | \VOoutcomeD{}{}{T}]$ represents the panel data of baseline outcomes that would be observed in the absence of any intervention.

In the first step of Algorithm~\ref{alg:FO_with_preprocessing}, we employ a semi-recursive algorithm to estimate the sample means of the columns of $\MOoutcomeD{}{}{}$, denoted by $\ECF{}{t}{\mathbf{0}}$. The consistency of this estimation follows directly from the consistency of Algorithm~\ref{alg:FO-semi-recursive}.

\subsection{Augmented Causal Message-Passing Model}
\label{sec:augmented_CMP}
% 
The next two steps of Algorithm~\ref{alg:FO_with_preprocessing} require the following assumption.
% 
\begin{assumption}
    \label{asmp:additive_baseline}
    For $t=0,1,\ldots,T-1$, we assume there exist families of functions $\Toutcomeg{t}$ and $\Toutcomeh{t}$ such that the potential outcomes $\VoutcomeD{}{}{t}(\Mtreatment{}{})$ satisfy:
    % 
    \begin{equation}
        \label{eq:aug_outcome_function}
    \begin{aligned}    
        \VoutcomeD{}{}{t+1}(\Mtreatment{}{})
        =
        \VOoutcomeD{}{}{t+1} +
        \big(\IM+\IMatT{t}\big)\Toutcomeg{t}\left(\VoutcomeD{}{}{t}(\Mtreatment{}{}) - \VOoutcomeD{}{}{t} ,\Mtreatment{}{}, \covar\right)
        +
        \Toutcomeh{t} \left(\VoutcomeD{}{}{t}(\Mtreatment{}{}) - \VOoutcomeD{}{}{t} ,\Mtreatment{}{}, \covar\right).
    \end{aligned}
    \end{equation}
    % 
    Additionally, $\Toutcomeg{t}\left(\Vec{0} , \mathbf{0}, \covar\right) = \Toutcomeh{t}\left(\Vec{0} , \mathbf{0}, \covar\right) = \Vec{0}$, and functions $\Toutcomeg{t}$ and $\Toutcomeh{t}$ satisfy the conditions of Assumption~\ref{asmp:BL}.
\end{assumption}
% 
Note that enforcing conditions $\Toutcomeg{t}\left(\Vec{0} , \mathbf{0}, \covar\right) = \Toutcomeh{t}\left(\Vec{0} , \mathbf{0}, \covar\right) = \Vec{0}$ ensures that $\VoutcomeD{}{}{t+1}(\mathbf{0}) = \VOoutcomeD{}{}{t+1}$, aligning with \eqref{eq:outcome_function_NoTreatment}. 
% 
Letting $\PPVoutcomeD{}{}{t}(\Mtreatment{}{}) := \VoutcomeD{}{}{t}(\Mtreatment{}{}) - \VOoutcomeD{}{}{t}$, we can then rewrite the augmented model \eqref{eq:aug_outcome_function} to match the dynamics of the original outcome specification:
% 
\begin{equation}
\label{eq:preprocessed_outcome_function}
\begin{aligned}    
    \PPVoutcomeD{}{}{t+1}(\Mtreatment{}{})
    &=
    \big(\IM+\IMatT{t}\big)\Toutcomeg{t}\left(\PPVoutcomeD{}{}{t}(\Mtreatment{}{}) ,\Mtreatment{}{}, \covar\right)
    +
    \Toutcomeh{t}\left(\PPVoutcomeD{}{}{t}(\Mtreatment{}{}) ,\Mtreatment{}{}, \covar\right).
\end{aligned}
\end{equation}
% 
We emphasize that Equations \eqref{eq:outcome_function_NoTreatment} and \eqref{eq:preprocessed_outcome_function} provide distinct characterizations of the outcomes, and Algorithm~\ref{alg:FO_with_preprocessing} requires both to hold simultaneously. Specifically, assuming the conditions of \S\ref{apndx:batch_state_evolution} hold in both settings, the baseline outcomes $\VOoutcomeD{}{}{t}$ satisfy the state evolution equation corresponding to \eqref{eq:outcome_function_NoTreatment}. However, in the context of \eqref{eq:preprocessed_outcome_function}, state evolution becomes relevant only when treatment is delivered (i.e., $\Mtreatment{}{} \neq \mathbf{0}$). Indeed, under Assumption~\ref{asmp:additive_baseline}, the third condition in Assumption~\ref{asmp:BL} indicates that state evolution can be derived only when there exists a non-zero treatment effect.

Then, the consistency of the second step estimation in Algorithm~\ref{alg:FO_with_preprocessing} holds in the context of the outcome specification \eqref{eq:aug_outcome_function}. Finally, the consistency of the ultimate estimator follows from both the consistency of individual steps and the fact that Algorithm~\ref{alg:FO_with_preprocessing} can be viewed as a combination of continuous functions.



% 
\begin{algorithm}
\caption{First-order counterfactual estimator with preprocessing}
\label{alg:FO_with_preprocessing}
% 
\begin{algorithmic}
% 
\Require $\Moutcome{}{}{}(\OMtreatment{o}{}), \OMtreatment{o}{}$, $\OMtreatment{u}{}$, estimation batch $\batch$, sample batches $\batch_1, \ldots, \batch_b$, and $l$

\State \hspace{-1.3em} \textbf{Step 1: Detrending}
% 
\State Use Algorithm~\ref{alg:FO-semi-recursive} with $\OMtreatment{u}{} = \mathbf{0}$ to obtain $\ECF{}{t}{\mathbf{0}}$, $t = 0, \ldots, T$. 
% 
\For{$t = 0, \ldots, T$}
    %
    \State $\VoutcomeD{}{'}{t} \gets \VoutcomeD{}{}{t}(\Mtreatment{}{}=\OMtreatment{o}{}) - \ECF{}{t}{\mathbf{0}}$
    % 
\EndFor

\State \hspace{-1.3em} \textbf{Step 2: Counterfactual estimation with proprocessed data}
% 
\State Use Algorithm~\ref{alg:FO-recursive} with $\Moutcome{}{'}{}$ to obtain $\ECF{'\batch}{t}{\OMtreatment{u}{}}$, $t = 0, \ldots, T$.


\State \hspace{-1.3em} \textbf{Step 3: Post-processing}
% 
\For{$t = 0, \ldots, T$}
    %
    \State $\ECF{\batch}{t}{\OMtreatment{u}{}} \gets \ECF{'\batch}{t}{\OMtreatment{u}{}} + \ECF{}{t}{\mathbf{0}}$
    % 
\EndFor

\Ensure $\ECF{\batch}{t}{\OMtreatment{u}{}}$, $t = 0, \ldots, T$.
% 
\end{algorithmic}
\end{algorithm}
% 





\section{Auxiliary Results}
\label{apndx:auxiliary_results}
% 
We need the following strong law of large numbers (SLLN) for triangular arrays of independent but not identically distributed random variables. The form stated below is Theorem~3 in \cite{bayati2011dynamics} that is adapted from Theorem~2.1 in \cite{hu1997strong}.
\begin{theorem}[SLLN]
    \label{thm:SLLN}
    Let $\left\{X_{n,i}:1\leq i \leq n,\; n \geq 1\right\}$ be a triangular array of random variables such that $(X_{n,1},\ldots,X_{n,n})$ are mutually independent with a mean equal to zero for each $n$ and $\frac{1}{n} \sum_{i=1}^n E\left[|X_{n,i}|^{2+\kappa}\right] \leq c n^{\kappa/2}$ for some $0 < \kappa < 1$ and $c < \infty$. Then, we have
    \begin{align}
        \label{eq:SLLN}
        \lim_{n \rightarrow \infty}
        \frac{1}{n} \sum_{i=1}^n X_{n,i} \eqas 0.
    \end{align}
\end{theorem}
We also need the following form of the law of large numbers which is an extension of Lemma~4 in \cite{bayati2011dynamics}.
\begin{theorem}
    \label{thm:SLLN-2}
    Fix $k\geq 2$ and an integer $l$ and let $\left\{\bm v(N)\right\}_{N \geq 1}$ be a sequence of vectors that $\bm v(N) \in \R^{N\x l}$. That means, $\bm v(N)$ is a matrix with $N$ rows and $l$ columns. Assume that the empirical distribution of $\bm v(N)$, denoted by $\hat{p}_{N}$, converges weakly to a probability measure $p_v$ on $\R^l$ such that $\E_{p_v}\left[\norm{\Vec{V}}^k\right] < \infty$ and $\E_{\hat{p}_{N}}\left[\norm{\Vec{V}}^k\right] \rightarrow \E_{p_v}\left[\norm{\Vec{V}}^k\right]$ as $N \rightarrow \infty$. Then, for any continuous function $f:\R^l \mapsto \R$ with at most polynomial growth of order $k$, we have
    \begin{align}
        \label{eq:SLLN-2}
        \lim_{N \rightarrow \infty}
        \frac{1}{N} \sum_{n=1}^N f\big(\bm v_n(N)\big)
        \eqas \E_{p_v} [f(\Vec{V})].
    \end{align}
\end{theorem}
% 
Next, we present Lemma 9 from \cite{li2022non}, and then employ it to prove Lemma~\ref{lm:Gap_with_Gaussian}.
% 
\begin{lemma}
    \label{lm:Gap_with_Gaussian_vector}
    Fixing $N$ and $t < N$, consider a set of i.i.d. random vectors $\Vec{\phi_i} \sim \Nc(0,\frac{1}{N}\I_N),\; i=1, \ldots, t$, and any unit vector $\VNPC{}{} = (\NPC{1}{}, \ldots, \NPC{t}{})^\top$ that might be statistically dependent on $\{\Vec{\phi_i}\}_{i=1}^t$. Then, the 1-Wasserstein distance between the distribution of $\sum_{i=1}^t \NPC{i}{} \Vec{\phi_i}$, denoted by $\law(\sum_{i=1}^t \NPC{i}{} \Vec{\phi_i})$, and $\Nc(0,\frac{1}{N}\I_N)$ obeys
    \begin{equation*}
        W_1\left(\law\left(\sum_{i=1}^t \NPC{i}{} \Vec{\phi_i}\right),\Nc\left(0,\frac{1}{N}\I_N\right)\right) \leq c \sqrt{\frac{t \log N}{N}},
    \end{equation*}
    for some constant $c$ that does not depend on $N$.
\end{lemma}


\begin{lemma}
    \label{lm:Gap_with_Gaussian}
    Fixing $N$ and $t < N$, consider a set of i.i.d. random vectors $\Vec{\phi_i} \sim \Nc(0,\frac{1}{N}\I_N),\; i=1, \ldots, t$, and any unit vector $\VNPC{}{} = (\NPC{1}{}, \ldots, \NPC{t}{})^\top$ that might be statistically dependent on $\{\Vec{\phi_i}\}_{i=1}^t$. Let $\Vec{\Phi} := \sum_{i=1}^t \NPC{i}{} \Vec{\phi_i}$ such that $\Vec{\Phi} = (\Phi^1, \ldots, \Phi^N)^\top$ and $\batch \subset [N]$ be subset of the indices. Then, the 1-Wasserstein distance between the distribution of $\frac{1}{\sqrt{\cardinality{\batch}}}\sum_{n \in \batch} \Phi^n$, denoted by $\law(\frac{1}{\sqrt{\cardinality{\batch}}}\sum_{n \in \batch} \Phi^n)$, and $\Nc(0,{1}/ {N})$ satisfies
    \begin{equation*}
        W_1\left(\law\left(\frac{1}{\sqrt{\cardinality{\batch}}}\sum_{n \in \batch} \Phi^n\right),\Nc\left(0,\frac{1}{N}\right)\right) \leq c \sqrt{\frac{t \log N}{N}},
    \end{equation*}
    for some constant $c$ that does not depend on $N$ and $\batch$.
\end{lemma}
% 
\textbf{Proof}. Considering Kantorovich-Rubinstein duality, we use the dual representation of the 1-Wasserstein distance:
% 
\begin{equation}
    \label{eq:apndx_proof_Gap_with_Gaussian_1}
    W_1\left(\law\left(\frac{1}{\sqrt{\cardinality{\batch}}}\sum_{n \in \batch} \Phi^n\right),\Nc\left(0,\frac{1}{N}\right)\right)
    = \sup\left\{\E\left[f\left(\frac{1}{\sqrt{\cardinality{\batch}}}\sum_{n \in \batch} \Phi^n\right)\right] - \E\left[f\left(\frac{Z}{\sqrt{N}}\right)\right]\;\Bigg|\; f \text{ is 1-Lipschitz}\right\},
\end{equation}
% 
where $Z \sim \Nc(0,1)$. To proceed, fix the 1-Lipschitz function $f$ arbitrarily. Further, define the function $\psi: \R^N \mapsto \R$ such that for any vector $\Avec{}{} = (\avec{1}{}, \ldots, \avec{N}{})^\top$, we have $\psi(\Avec{}{}) = \frac{1}{\sqrt{\cardinality{\batch}}} \sum_{n \in \batch} \avec{n}{}$. Then, we define $\tilde{f} = f\circ \psi : \R^N \mapsto \R$. Note that both $f$ and $\psi$ are continuous functions; as a result,
the function $\tilde{f}$ is measurable. We show it is also $1$-Lipschitz. To this end, for vectors $\Avec{}{1}, \Avec{}{2} \in \R^N$, we write:
% 
\begin{equation*}
    \begin{aligned}
        \left|\tilde{f}\left(\Avec{}{2}\right) - \tilde{f}\left(\Avec{}{1}\right)\right|
        = \left|f\left(\psi(\Avec{}{2})\right) - f\left(\psi(\Avec{}{1})\right)\right|
        \leq \left|\psi(\Avec{}{2}) - \psi(\Avec{}{1})\right|
        \leq
        \frac{\sum_{n \in \batch} \left|\avec{n}{2} - \avec{n}{1} \right|}{\sqrt{\cardinality{\batch}}} 
        \leq
        \sqrt{\sum_{n \in \batch} \left|\avec{n}{2} - \avec{n}{1} \right|^2}
        \leq
        \norm{\Avec{}{2} - \Avec{}{1}},
    \end{aligned}
\end{equation*}
% 
where we used the fact that $f$ is 1-Lipschitz and the Cauchy–Schwarz inequality. Therefore, the function $\tilde{f}$ is 1-Lipschitz and by the result of Lemma~\ref{lm:Gap_with_Gaussian_vector}, we get
% 
\begin{equation*}
    \E\left[f\left(\frac{1}{\sqrt{N}}\sum_{n=1}^N \Phi^n\right)\right] - \E\left[f(\frac{Z}{\sqrt{N}})\right]
    = \E\left[\tilde{f}\left(\Vec{\Phi}\right)\right] - \E\left[\tilde{f}\left(\frac{1}{\sqrt{N}}\Vec{Z}\right)\right]
    \leq 
    c\sqrt{\frac{t \log N}{N}},
\end{equation*}
% 
where $\Vec{Z} \sim \Nc(0,\I)$. Because $f$ is chosen arbitrarily, by Eq.~\eqref{eq:apndx_proof_Gap_with_Gaussian_1}, we obtain the desired result. \ep

\end{APPENDICES}

\end{document}
%%%%%%%%%%%%%%%%%