\section{Benchmark Toolbox}
\label{sec:Benchmark_Toolbox}
% 
We evaluate our framework using six semi-synthetic experiments that combine simulated environments with real-world data. This approach offers two key advantages: it maintains realistic data characteristics while allowing us to compute ground truth values for our estimands. Unlike real experimental settings where outcomes are observed under a single scenario, these settings provide ground truth values for any desired scenario. This enables rigorous evaluation of estimation methods under realistic conditions. In the following sections, we specify the treatment allocation, outcomes, and network structure for each experimental setting.


\subsection{LLM-based Social Network}
\label{sec:LLM}
% 
This environment simulates a social media platform like Facebook and Quora where user interactions occur through content feeds, designed to study the effects of feed ranking algorithms on user engagement. The environment employs Large Language Model (LLM) agents to represent users, with demographically realistic personas derived from US Census data \citep{uscensus2023} following \cite{chang2024llms}'s methodology. Each agent possesses two interests selected from a predefined set of \emph{keywords}: Technology, Sports, Politics, Entertainment, Science, Health, and Fashion.

The treatment variable is the \emph{feed ranking algorithm}, which determines content ordering for each user. In the control condition, users receive randomly ordered content, while the treatment condition presents content weighted by friend engagement. The outcomes of interest are \emph{user engagement metrics}, measured as the sum of likes or replies generated by each user within a given time period, tracked in a comprehensive panel dataset.

The underlying directed follower-following relationships network is constructed based on a preferential attachment model \citep{barabasi1999emergence}. The dynamics of the environment center on content generation and user interactions. Content originates from an LLM-generated bank focused on a specified topic, with cross-keyword variations (e.g., climate change intersecting with technology or politics). The system generates feeds by combining interest-based content (matching user interests), trending content (high engagement), and random content according to specified proportions. Users interact with their feeds through an LLM-driven decision process that considers content relevance, friend engagement, feed position, and demographic characteristics.

The simulation maintains comprehensive state information across multiple dimensions: engagement metrics, user interaction histories, network relationships, content visibility, and conversation threads. At each time step, agents have a 1\% probability of generating new content, contributing to the platform's organic content evolution. The interaction process follows a structured decision framework, where each agent evaluates content through a detailed prompt incorporating the agent's persona attributes, content characteristics, social signals (including friend engagement), and feed positioning. This framework ensures that user behaviors and social influence patterns evolve naturally through the network while maintaining computational tractability.


\subsection{Belief Adoption Model}
\label{sec:Belief_Adoption}
%
This environment models the diffusion of competing opinions within interconnected communities, implementing the cascade model of \cite{montanari2010spread}. The system examines how opinions spread through social networks when individuals make decisions through coordination games with their neighbors. Through this framework, we evaluate the effectiveness of promotional campaigns in influencing opinion adoption patterns.

The environment considers two competing stances: Opinion~$\OPA$ (e.g., voting in an election) and Opinion~$\OPB$ (e.g., declining to vote). The treatment represents a \emph{campaign aimed at increasing Opinion~$\OPA$ adoption}. The outcome for each unit in each period is binary: \emph{1 if they adopt Opinion~$\OPA$, and 0 otherwise}.

The opinion evolution follows a network-based coordination game where each individual $i$ assigns payoff values $\payoff^i_\OPA$ and $\payoff^i_\OPB$ to both opinions. The probability of Opinion~$\OPA$ adoption in the next period depends on the neighbor configuration and relative payoffs:
% 
\begin{align*}
    \P\big(\text{adopting Opinion~$\OPA$} \big| \neighbor^i_\OPA,\neighbor^i_\OPB \big) =
    \frac{e^{\beta \left( \neighbor^i h^i + \neighbor^i_\OPA - \neighbor^i_\OPB  \right)}}{e^{\beta \left(\neighbor^i h^i + \neighbor^i_\OPA - \neighbor^i_\OPB  \right)} + e^{-\beta  \left(\neighbor^i h^i + \neighbor^i_\OPA - \neighbor^i_\OPB  \right)}}\,,
\end{align*}
% 
where $h^i = \frac{\payoff^i_\OPA - \payoff^i_\OPB}{\payoff^i_\OPA + \payoff^i_\OPB}$ and $\neighbor^i_\OPA$ represents the number of neighbors holding Opinion~$\OPA$ out of $\neighbor^i$ total neighbors in the current period, and $\beta$ is a predetermined constant. The underlying network in our simulator is derived from the \emph{Pokec social network} dataset \citep{takac2012data,snapnets}, focusing on three regional networks: Krupina (3,366 users), Topolcany (18,246 users), and Zilina (42,971 users).

The environment also utilizes detailed profile data from the Pokec social network dataset to characterize each user. We extract three demographic variables from these profiles: age, profile activity, and gender. For base payoffs, we assign higher values for Opinion~$\OPA$ to users aged 25-55 years and those maintaining active profiles. The treatment effectiveness follows a Gaussian distribution, reaching its maximum at age 35. Its impact scales directly with the user's profile completion percentage, as measured in their Pokec data. These profile-driven calculations create distinct patterns of payoffs and treatment responses across the network, as visualized in Figures~\ref{fig:BAM-krupina}-\ref{fig:BAM-zilina}.

\begin{figure}
    \centering
    \includegraphics[width=\linewidth]{plots/krupina.pdf}
    \caption{Distribution of base payoffs, treatment effects, and node degrees for individuals in Krupina.}
    \label{fig:BAM-krupina}
\end{figure}

\begin{figure}
    \centering
    \includegraphics[width=\linewidth]{plots/topolcany.pdf}
    \caption{Distribution of base payoffs, treatment effects, and node degrees for individuals in Topolcany.}
    \label{fig:BAM-topolcany}
\end{figure}

\begin{figure}
    \centering
    \includegraphics[width=\linewidth]{plots/zilina.pdf}
    \caption{Distribution of base payoffs, treatment effects, and node degrees for individuals in Zilina.}
    \label{fig:BAM-zilina}
\end{figure}


\subsection{Ascending Auction Model}
\label{sec:Auction}
% 
This environment simulates a competitive market where multiple bidders participate in an ascending auction for objects, following the model of \cite{bertsekas1990auction}. The auction mechanism creates a dynamic pricing system where bidder interactions generate complex patterns of market influence, even without direct object-to-object relationships.

The system operates with $N$ objects and $N$ bidders. Each object represents an experimental unit, with its \emph{final value} in each round serving as the outcome variable. The treatment consists of \emph{promotional interventions that increase bidder valuations} by $\tau$\% for randomly selected objects. This treatment affects all bidders interested in the selected objects.

The market evolution follows a structured bidding process. In each round, bidders evaluate objects based on their private valuations and current market prices. They submit bids for their preferred objects, with objects being assigned to the highest bidders. These assignments establish new price levels, which influence subsequent bidding behavior. As prices increase through competitive bidding, objects become progressively less attractive to competing bidders.

The environment demonstrates a unique form of interference: \emph{while objects do not directly influence each other, treatment effects propagate through the market via bidders' strategic responses to price changes.} This creates a network of indirect treatment effects, as promotional interventions for certain objects can influence market outcomes for others through shifts in bidder behavior and price dynamics.


\subsection{New York City Taxi Routes}
\label{sec:LiM}
% 
This environment models ride-sharing dynamics across New York City taxi zones using real-world TLC Trip Record Data \citep{nyc_tlc_trip_data}. The framework adapts the established linear-in-mean outcome model \citep{eckles2016design,cai2015social,leung2022causal} to represent how passengers utilize ride-sharing services throughout the city. By incorporating actual travel data, passenger density metrics, and inter-zone relationships, the simulation effectively captures the complex network of interactions between taxi routes across the city.

In this setting, the experimental units are defined as routes (origin-destination pairs) between the city's 263 taxi zones, with time segmented into 6-hour periods. The outcome variable measures the \emph{number of trips} along each route during each time period. The treatment represents a program implemented on randomly selected routes and the goal is to evaluate travelers response patterns.

Given baseline outcomes $[\OoutcomeD{}{i}{t}]_{i,t}$, the system's evolution follows Equation \eqref{eq:linear-in-mean}, where outcomes depend on baseline patterns, network effects, and treatment status:
% 
\begin{equation}
\label{eq:linear-in-mean}
\begin{aligned}
    \outcomeD{}{i}{t+1}
    =
    \OoutcomeD{}{i}{t+1}
    +
    \ACE
    \sum_{j=1}^N \adjMe^{ij}
    (\outcomeD{}{j}{t} - \OoutcomeD{}{i}{t})
    +
    \ADE_{\pl} \sum_{j=1}^N \adjMe^{ij}\treatment{j}{t+1}
    +
    \ADE^i_{\ul} \treatment{i}{t+1},
    \quad
    t \geq 1,
\end{aligned}
\end{equation}
% 
where we initiate the recursion by setting $\outcomeD{}{i}{0} = \OoutcomeD{}{i}{0}$ and $\treatment{i}{0} = 0$ for all $i$.
Here, $\adjM = [\adjMe^{ij}]_{i,j}$ represents the normalized route adjacency matrix, $\ADE_{\ul}^i$ represents route-specific direct treatment effects, and parameters $(\ACE,\ADE_{\pl}) = (0.4,0.2)$ control autocorrelation and spillover effects.
% 
\begin{figure}
    \centering
    \includegraphics[width=\linewidth]{plots/NYC_Data.pdf}
    \caption{Mean and variance of the number of trips in each route during each period, revealing a strong daily and weekly seasonality pattern.}
    \label{fig:NYC_taxi_panel}
\end{figure}
% 

The environment incorporates real-world data through three components: First, it uses ``High Volume For-Hire Vehicle Trip Records" (January-March 2024, 57,974,677 trips) to construct baseline outcomes (denoted by $[\OoutcomeD{}{i}{t}]_{i,t}$), focusing on 18,768 active routes across 366 periods. As shown in Figure~\ref{fig:NYC_taxi_panel}, the trips have a strong seasonality pattern, which is common in the ride-hailing application \citep{xiong2024data}. Second, it employs the ``Taxi Zone Lookup Table" and Claude (Model 3.5 Sonnet, Anthropic, 2024) to generate passenger density scores that determine route-specific treatment effects (Figure~\ref{fig:NYC_analysis} left). Third, it creates a route adjacency network based on geographic proximity, transit connections, shared roads, and functional relationships, yielding an average node degree of 8.32 (Figure~\ref{fig:NYC_analysis} right). This data-driven approach ensures the simulation reproduces key real-world characteristics: temporal patterns, heterogeneous treatment effects, and localized network interactions.
% 
\begin{figure}[ht]
    \centering
    \includegraphics[width=0.45\linewidth]{plots/NYC_Taxi_DTE.pdf}
    \includegraphics[width=0.45\linewidth]{plots/NYC_node_degrees_histogram.pdf}
    \caption{The left panel displays the distribution of route-specific direct treatment effects, illustrating the heterogeneity in treatment responses. The right panel shows the histogram of node degrees in the route interference network.}
    \label{fig:NYC_analysis}
\end{figure}


\subsection{Exercise Encouragement Program}
\label{sec:BOM}
% 
This environment simulates an exercise intervention program that combines individual characteristics from the 1994 Census Bureau database \citep{kohavi1994data} with social network effects. Drawing inspiration from mobile health intervention studies \citep{liao2016sample,klasnja2015microrandomized,klasnja2019efficacy}, the environment models how digital encouragement messages influence exercise decisions within a social network context.

The experimental units are individuals, with binary outcomes representing their \emph{exercise decisions in each period} (1 for exercise, 0 for no exercise). The treatment consists of \emph{digital intervention messages} designed to encourage physical activity. Inspired by \cite{li2022network}, we let the outcomes follow a Bernoulli distribution defined as follows:
% 
\begin{align}
    \label{eq:MRT}
    \outcomeD{}{i}{t+1} \sim \text{Bernoulli}\left(
    \frac{1}
    {
    1
    +
    \exp{-(
    \ABE_t^i
    +
    \ADE_t^i
    \treatment{i}{t+1}
    +
    \ACE
    \outcomeD{}{i}{t} Z^i_t
    +
    \APE \treatment{i}{t+1} \outcomeD{}{i}{t} Z^i_t
    )}}\right),
\end{align}
% 
where $Z^i_t = \sum_{j=1}^N \adjMe^{ij}\outcomeD{}{j}{t}$ represents the count of neighboring individuals who exercised in the previous period. Here, $\adjMe^{ij}$'s are the elements of the adjacency matrix from \emph{Twitter social circles} data \citep{leskovec2012learning}, with an average of 21.74 connections per individual (Figure~\ref{fig:twitter_network_hist}). 

In \eqref{eq:MRT}, each component captures specific aspects of exercise behavior. The baseline probability ($\ABE_t^i$) represents an individual's inherent tendency to exercise, derived from Census Bureau demographic data. This probability incorporates age (with higher values for younger individuals), working hours (showing an inverse relationship with exercise likelihood), and occupation type (assigning higher probabilities to active or professional occupations). These baseline probabilities exhibit weekly patterns, showing peak values on weekends due to increased free time, elevated rates on Mondays from new week motivation, stable mid-week patterns, and slightly lower values on Fridays reflecting end-of-week fatigue (Figure~\ref{fig:exercise_probability}).

The intervention effectiveness ($\ADE_t^i$) quantifies how individuals respond to exercise encouragement messages. This response varies based on multiple demographic factors from the Census Bureau database. Younger individuals show higher responsiveness to digital interventions, while education level correlates positively with intervention effectiveness. Job-related factors, including occupation type and working hours, influence response rates by indicating flexibility and availability to act on interventions. The impact of messages follows weekly cycles, demonstrating maximum effectiveness during weekends and Mondays, with gradually decreasing impact through mid-week (Figure~\ref{fig:message_impact}). The model also sets parameters $(\ACE, \APE) = (0.0.4,0.01)$ to capture peer influence and the interaction between treatment and peer effects.
% 
\begin{figure}
    \centering
    \includegraphics[width=1\linewidth]{plots/exercise_probability.pdf}
    \caption{Distribution of baseline exercise probabilities across the population.}
    \label{fig:exercise_probability}
\end{figure}
% 
% 
\begin{figure}
    \centering
    \includegraphics[width=1\linewidth]{plots/message_impact.pdf}
    \caption{Distribution of intervention message effects across the population}
    \label{fig:message_impact}
\end{figure}
% 

% 
\begin{figure}
    \centering
    \includegraphics[width=0.9\linewidth]{plots/twitter_network_hist.pdf}
    \caption{Distribution of node degrees in the Twitter social network.}
    \label{fig:twitter_network_hist}
\end{figure}
% 


\subsection{Data Center Server Utilization}
\label{sec:servers}
% 
This environment simulates a server farm, and the goal is to evaluate interventions for improving server utilization. Given the increasing demand for cloud computing resources, optimizing data center utilization has become critical for addressing global sustainability concerns \citep{zhang2023global,saxena2023sustainable}. Within this system, servers influence each other's performance through the system's physical characteristics, particularly via the join-the-shortest-queue routing policy \citep{gupta2007analysis}. This operational dynamic creates an implicit interference pattern in the absence of a pre-specified network structure.

The experimental units are individual servers within a parallel processing system of N servers. The outcome variable $\outcomeD{}{i}{t}$ represents \emph{server i's utilization} during the interval [t,t+1), measured as the proportion of time the server remains busy. The treatment consists of interventions that \emph{enhance the processing power} of selected servers.

The system evolution follows a structured routing mechanism. When tasks arrive, the system identifies capable servers for each task type and selects a random sample among them. Following the join-the-shortest-queue policy, tasks are assigned to servers with minimal queue lengths within this sample, using random assignment to resolve ties. This \emph{routing approach naturally creates interference effects}, as performance improvements in treated servers influence task distribution across the entire system.

The environment incorporates realistic workload patterns through a time-dependent Poisson arrival process. The demand model captures multiple temporal patterns: daily variations (night-time lows, morning increases, midday peaks, and evening declines), weekly cycles (heightened weekday activity), and stochastic elements (random fluctuations and event-driven spikes). Each server processes tasks with exponentially distributed service times, which can be modified by interventions. Through this design, the system replicates key characteristics of real-world data centers while enabling controlled experimentation.
% 
\begin{figure}
    \centering
    \includegraphics[width=0.9\linewidth]{plots/Data_center_time_trend.pdf}
    \caption{Average demand for the data center over time shows a strong seasonality.}
    \label{fig:Data_center_time_trend}
\end{figure}
%