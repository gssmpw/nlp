\newtcolorbox[auto counter, number within=section]{promptbox}[2][]{
    coltitle=white,
    fonttitle= \bfseries\ttfamily,
    sharp corners=south,
    title=Prompt~\thetcbcounter: #2,
    fontupper=\ttfamily,
    breakable
}


\appendix

\section{Examples}
\label{sec:appendix_examples}

This section presents an example of our data synthesis framework using \textsc{Reference-Level Feedback}. First, we select a reference sample:

\begin{promptbox}{Reference Sample}
\textbf{Instruction}: I have a string representing a unix timestamp (i.e.\ ``1284101485'') in Python, and I'd like to convert it to a readable date. When I use \texttt{time.strftime}, I get a \texttt{TypeError}:

>>>import time\\
>>>print time.strftime("\%B \%d \%Y", "1284101485")\\
Traceback (most recent call last):\\
  File "", line 1, in \\
TypeError: argument must be 9-item sequence, not str \\

\textbf{Response:}
There are two parts here:

\begin{itemize}
\item Convert the unix timestamp (``seconds since epoch'') to the local time
\item Display the local time in the desired format.
\end{itemize}

A portable way to get the local time that works even if the local time zone had a different utc offset in the past and python has no access to the tz database is to use a \texttt{pytz} timezone:\\

\#!/usr/bin/env python\\
from datetime import datetime\\
import tzlocal  \# \$ pip install tzlocal\\
unix\_timestamp = float("1284101485")\\
local\_timezone = tzlocal.get\_localzone() \# get pytz timezone\\
local\_time = datetime.fromtimestamp(unix\_timestamp, local\_timezone)\\

To display it, you could use any time format that is supported by your system e.g.:\\

print(local\_time.strftime("\%Y-\%m-\%d \%H:\%M:\%S.\%f\%z (\%Z)"))\\
print(local\_time.strftime("\%B \%d \%Y"))  \# print date in your format\\

If you do not need a local time, to get a readable UTC time instead:\\

utc\_time = datetime.utcfromtimestamp (unix\_timestamp)\\
print(utc\_time.strftime("\%Y-\%m-\%d \%H:\%M:\%S.\%f+00:00 (UTC)"))\\ 

If you don't care about the timezone issues that might affect what date is returned or if python has access to the tz database on your system:\\

local\_time = datetime.fromtimestamp (unix\_timestamp)\\
print(local\_time.strftime("\%Y-\%m-\%d \%H:\%M:\%S.\%f"))\\

On Python 3, you could get a timezone-aware datetime using only stdlib (the UTC offset may be wrong if python has no access to the tz database on your system e.g., on Windows):\\

\#!/usr/bin/env python3\\
from datetime import datetime, timezone\\
utc\_time = datetime.fromtimestamp (unix\_timestamp, timezone.utc)\\
local\_time = utc\_time.astimezone()\\
print(local\_time.strftime("\%Y-\%m-\%d \%H:\%M:\%S.\%f\%z (\%Z)"))\\

Functions from the \texttt{time} module are thin wrappers around the corresponding C API and therefore they may be less portable than the corresponding \texttt{datetime} methods otherwise you could use them too:\\

\#!/usr/bin/env python\\
import time\\
unix\_timestamp  = int("1284101485")\\
utc\_time = time.gmtime(unix\_timestamp)\\
local\_time = time.localtime(unix\_timestamp)\\
print(time.strftime("\%Y-\%m-\%d \%H:\%M:\%S", local\_time))\\
print(time.strftime("\%Y-\%m-\%d \%H:\%M:\%S+00:00 (UTC)", utc\_time))



\end{promptbox}

Then, we collect \textsc{Reference-Level Feedback} for that reference sample:
\begin{promptbox}{\textsc{Reference-Level Feedback}}
    \textbf{Instruction Subject Feedback:} Programming, specifically Python programming with a focus on date and time manipulation using Unix timestamps. \\

    \textbf{Instruction Skill Feedback:} Understanding of Unix timestamps, Python's datetime and time modules, exception handling in Python, formatting dates and times, knowledge of time zones and portability of code across different systems. \\

    \textbf{Response Feedback:} The reference response is effective in addressing the instruction for multiple reasons. Firstly, it accurately identifies the issue with the TypeError during the use of `time.strftime` and provides the correct method of converting a unix timestamp into a readable date format using the `datetime` module, which is more appropriate for this task. The response recognizes the importance of local time and considers timezone issues, which adds depth to the explanation.
    
    Additionally, the structure of the response is clear and organized into distinct sections that guide the user step-by-step through the conversion process. It provides multiple options (using different libraries and methods) for handling the conversion, catering to various user needs, which enhances comprehensiveness.
    
    However, there is room for improvement. 
    1. **Clarity**: While the response provides various methods, it could improve clarity by explicitly stating under what circumstances each method should be used (e.g., when to use `tzlocal`, when UTC is sufficient, etc.). 
    2. **Comprehensiveness**: The response could briefly explain what a Unix timestamp is for those unfamiliar with it and its relevance in this context. 
    3. **Engagement**: Incorporating a more conversational tone or additional commentary about best practices when dealing with date and time conversions could make the response feel more engaging. 
    4. **Potential errors**: It might be worth noting that `pytz` needs to be installed and that some users might run into issues if they don't set up their environment beforehand. 
    5. **Code snippets**: Ensure that code snippets are correctly formatted for clarity, especially in online platforms.
    
    Overall, the response effectively meets the instruction but could enhance user understanding and engagement with minor adjustments.
    
\end{promptbox}


Then, we begin the data synthesis. This process involves generating 10 new instructions for the instruction subject feedback and 10 for the instruction skill feedback. For each instruction, we generate a response and improve the response with the reference response feedback.

Below, we show an example of a new instruction that was generated based on the instruction skill feedback, as well as it's generated response and response improved with the reference response feedback.

\begin{promptbox}

\textbf{Instruction:} 
I am working with a Python script that reads timestamps in milliseconds from a file, but when I try to convert them to a readable format using \texttt{datetime.utcfromtimestamp}, I face a \texttt{TypeError}. My code looks like this:\\

import datetime\\
timestamps = [1640995200000, 1641081600000]\\
for ts in timestamps:\\
    print(datetime.utcfromtimestamp(ts)) \\

\textbf{Response:}
The issue you are encountering is due to the fact that \texttt{datetime.utcfromtimestamp} expects the timestamp to be in seconds, not milliseconds. Since your timestamps are in milliseconds, you need to convert them to seconds by dividing each timestamp by 1000 before passing it to \texttt{utcfromtimestamp}. Here's how you can modify your code to work correctly:\\

import datetime\\
timestamps = [1640995200000, 1641081600000]\\
for ts in timestamps:\\
    \# Convert milliseconds to seconds\\
    seconds = ts / 1000\\
    \# Get the UTC datetime from the seconds\\
    print(datetime. utcfromtimestamp(seconds))\\

In this code:
\begin{itemize}
\item We divide each timestamp by 1000 to convert from milliseconds to seconds.
\item Then, we call \texttt{datetime.utcfromtimestamp} with the converted value.
\end{itemize}

When you run this modified code, you should see the output in a readable format without encountering a \texttt{TypeError}:\\

2022-01-01 00:00:00\\
2022-01-02 00:00:00\\

This output represents the UTC datetime for January 1, 2022, and January 2, 2022, respectively. If you need to format the output differently, you can also use \texttt{strftime} method to customize the output format. For example:\\

for ts in timestamps:\\
    seconds = ts / 1000\\
    utc\_time = datetime.utcfromtimestamp(seconds)\\
    formatted\_time = utc\_time.strftime('\%Y-\%m-\%d \%H:\%M:\%S')\\
    print(formatted\_time) \\

\textbf{Improved Response: }
The issue you are facing with \texttt{TypeError} is due to the fact that the \texttt{datetime.utcfromtimestamp} function expects timestamps in seconds, not milliseconds. A Unix timestamp represents the number of seconds that have elapsed since January 1, 1970 (known as the Unix Epoch). In your case, the timestamps are in milliseconds, so we need to convert them to seconds by dividing each timestamp by 1000 before passing it to \texttt{utcfromtimestamp}. Here's how you can adjust your code to work correctly:\\

import datetime\\
timestamps = [1640995200000, 1641081600000]\\
for ts in timestamps:\\
    \# Convert milliseconds to seconds\\
    seconds = ts / 1000\\
    \# Get the UTC datetime from the seconds\\
    print(datetime. utcfromtimestamp(seconds))\\

In this modified code:
\begin{itemize}
\item Each timestamp is divided by 1000 to convert from milliseconds to seconds.
\item Then, we use \texttt{datetime.utcfromtimestamp} to get the UTC time.
\end{itemize}

When you run this code, you should see the output in a readable format without encountering any \texttt{TypeError}:\\

2022-01-01 00:00:00\\
2022-01-02 00:00:00\\

This output represents the UTC datetime for January 1, 2022, and January 2, 2022, respectively. If you require the date in a specific format, you can customize it using the \texttt{strftime} method. For instance:\\

for ts in timestamps:\\
    seconds = ts / 1000\\
    utc\_time = datetime.utcfromtimestamp(seconds)\\
    formatted\_time = utc\_time.strftime('\%Y-\%m-\%d \%H:\%M:\%S')\\
    print(formatted\_time)\\

Keep in mind that if you're working across different time zones and need to account for local time, you may want to explore libraries like \texttt{pytz}. Just remember that you will need to install it first and ensure your environment is set up correctly. Handling date and time conversions can often be tricky, so it's good practice to decide upfront whether you need UTC or local times based on the context of your application.

\end{promptbox}

After considering the relevant aspects from the reference response feedback, the improved response enhances understanding and engagement. Some changes include: adding an explanation of unix timestamps, explaining when to use UTC and when to use local time, adopting a more conversational tone, and including details of how to setup the environment.





\section{Prompt Templates}
\label{sec:appendix_prompt_templates}

This section presents the prompt templates we use throughout our pipeline.

\subsection{Prompts Used For Feedback Collection}

\begin{promptbox}{Reference-Level Instruction Feedback}
\# Task \\
Analyze the following instruction to extract key features that make it effective for instruction tuning. This analysis will help create high-quality instruction-response pairs for training language models to better follow instructions. \\

\# Context \\
The analysis will be used to create training data for instruction tuning language models. Focus on qualities that help language models learn to: \\
- Better understand user intentions \\
- Recognize instruction patterns \\
- Generate appropriate responses \\

\# Input \\
\{\{
  "instruction": \textcolor{blue}{\{instruction\}}, \\
  "reference\_response": \textcolor{blue}{\{reference\_response\}}, \\
\}\}

\# Output Format \\
\{\{
  "subject\_areas": str, \# This should be a description of the relevant subject areas and domains the instruction covers
  "relevant\_skills": str \# This should be a description of the relevant skills required to provide a good response to the instruction
\}\} \\

\# Analysis Guidelines \\
- Consider what makes this instruction clear and actionable \\
- Identify all relevant domains and skills \\
- Note structural elements that enhance instruction clarity \\

Output only a JSON object, in the format specified
\end{promptbox}

\begin{promptbox}{Response Feedback}
\# Task \\
Analyze the instruction-response pair and provide detailed feedback on how well it addresses the instruction. The feedback should: \\
- Highlight the specific qualities that make the response effective \\
- Provide actionable feedback for improvement \\

\# Input \\
\{\{
  "instruction": \textcolor{blue}{\{instruction\}}, \\
  "reference\_response": \textcolor{blue}{\{reference\_response\}}, \\
\}\} \\

\# Evaluation Criteria \\
\#\# Content Quality \\
- Accuracy and factual correctness \\
- Quality and depth of coverage \\

\#\# Communication \\
- Clarity and comprehensiveness \\
- Logical flow, organization, and structure \\
- Appropriate quality and depth \\
- Engagement and tone \\

\#\# Instruction Alignment \\
- How will it addresses the instruction \\
- Appropriate scope and focus \\
- Match with implied user needs \\

\# Output Format \\
\{\{
  "response\_feedback" : str \# Feedback describing strengths of the response and how it can be improved
\}\} \\

Output only a JSON object, in the format specified.\end{promptbox}

\subsection{Prompts Used For Data Synthesis}

\begin{promptbox}{Instruction Synthesis}
\# Task \\
Generate 10 new instructions based on the provided instruction feature and sample. Each instruction should: \\
- Be of similar complexity and length to the sample instruction \\
- Be practical and reasonable to answer \\
- Be diverse and high-quality \\

\# Sample Instruction: \\
\textcolor{blue}{\{instruction\}}

\# Instruction Features: \\
\textcolor{blue}{\{feature\}} \\

\# Output Format \\
\{\{
  "instructions": list \# List of 10 distinct instructions. Each instruction should be a single string.
\}\} \\

Output only a JSON object, in the format specified.
\end{promptbox}

\begin{promptbox}{Response Synthesis}
\# Task \\
I will provide an instruction. Generate a high-quality, helpful response to the instruction. The response should demonstrate expertise, clear reasoning, and natural language use. \\

\# Response Requirements \\
- Directly address all aspects of the instruction \\
- Response should demonstrate clear reasoning and expertise \\
- Use clear, natural language \\
- Include examples or evidence when relevant \\
- Show step-by-step reasoning where appropriate \\
- Maintain appropriate length and detail level \\
- Use proper formatting (lists, paragraphs) as needed \\

Here is an example of a response to an instruction: \\
\# Sample Input Instruction:
\textcolor{blue}{\{sample\_instruction\}} \\
\# Sample Response: \\
\textcolor{blue}{\{reference\_response\}} \\

\# Output Format \\
\{\{
  "response": "The complete response text here"
\}\}

\# Input \\
\{\{
  "instruction": \textcolor{blue}{\{instruction\}},
\}\} \\

Generate a properly formatted JSON response, as specified by the Output Format, that addresses this instruction.
\end{promptbox}

\begin{promptbox}{Response Refinement with Reference-Level Feedback}
\# Task \\
Given an instruction-response pair and feedback, generate an improved version of the response by applying the feedback. The feedback was given for a similar but different instruction-response pair. Not all aspects of the feedback may be directly applicable, so make sure to only apply relevant aspects of the feedback. \\

\# Input \\
\{\{
  "instruction": \textcolor{blue}{\{instruction\}},
  "original\_response": \textcolor{blue}{\{response\}},
  "feedback": \textcolor{blue}{\{response\_feedback\}}
\}\} \\


\# Quality Assessment Process \\
1. Analyze Original Response \\
- Core strengths and effective elements \\
- Structure and organization \\
- Depth and comprehensiveness \\
- Alignment with instruction \\

2. Evaluate Feedback \\
- Identify feedback points that are relevant to improving this response, and ignore points that are not relevant \\
- Identify actionable improvement suggestions \\
- Assess potential impact of each change \\
- Check alignment with original instruction \\
- Validate that suggested changes maintain or enhance quality \\

3. Improvement Strategy \\
- Prioritize changes with highest impact \\
- Preserve effective elements of the original response \\
- Ensure feedback applied enhance the response and do not remove valuable elements \\

\# Output Format  \\
\{\{
    "analysis": \{\{ \\
        "original\_strengths": ["list of key effective elements to preserve"],
        "improvement\_opportunities": ["list of specific areas that will benefit from enhancement"],
        "relevant\_feedback": ["list of feedback points that are relevant and beneficial"]
    \}\},
    "implementation\_strategy": \{\{ \\
        "planned\_changes": ["identify what feedback will be applied"],
        "rationale": "explain how this feedback will improve the original response"
    \}\},
    "improved\_response": "The revised and improved response"
\}\} \\

Output only a JSON object, in the format specified.
\end{promptbox}

\begin{promptbox}{Response Refinement with Sample-Level Feedback}
\# Task \\
Given an instruction-response pair and feedback, generate an improved version of the response by applying the feedback. \\

\# Input \\
\{\{
  "instruction": \textcolor{blue}{\{instruction\}},
  "original\_response": \textcolor{blue}{\{response\}},
  "feedback": \textcolor{blue}{\{self\_reflection\}}
\}\} \\

\# Quality Assessment Process \\
1. Analyze Original Response \\
- Core strengths and effective elements \\
- Structure and organization \\
- Depth and comprehensiveness \\
- Alignment with instruction \\

2. Evaluate Feedback \\
- Identify actionable improvement suggestions \\
- Assess potential impact of each change \\
- Check alignment with original instruction \\
- Validate that suggested changes maintain or enhance quality \\

3. Improvement Strategy \\
- Prioritize changes with highest impact \\
- Preserve effective elements of the original response \\
- Ensure feedback applied enhance the response and do not remove valuable elements \\

\# Output Format \\
\{\{
    "analysis": \{\{ \\
        "original\_strengths": ["list of key effective elements to preserve"],
        "improvement\_opportunities": ["list of specific areas that will benefit from enhancement"]
    \}\},
    "implementation\_strategy": \{\{ \\
        "planned\_changes": ["identify what feedback will be applied"],
        "rationale": "explain how this feedback will improve the original response"
    \}\},
    "improved\_response": "The revised and improved response"
\}\} \\

Output only a JSON object, in the format specified.
\end{promptbox}

\section{License}
Our use of existing artifact(s) is consistent with their intended use. The LIMA dataset follows the CC BY-NC-SA license, or a stricter license if the source data follows the same.