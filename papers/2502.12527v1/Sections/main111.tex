% TEMPLATE for Usenix papers, specifically to meet requirements of
%  USENIX '05
% originally a template for producing IEEE-format articles using LaTeX.
%   written by Matthew Ward, CS Department, Worcester Polytechnic Institute.
% adapted by David Beazley for his excellent SWIG paper in Proceedings,
%   Tcl 96
% turned into a smartass generic template by De Clarke, with thanks to
%   both the above pioneers
% use at your own risk.  Complaints to /dev/null.
% make it two column with no page numbering, default is 10 point

% Munged by Fred Douglis <douglis@research.att.com> 10/97 to separate
% the .sty file from the LaTeX source template, so that people can
% more easily include the .sty file into an existing document.  Also
% changed to more closely follow the style guidelines as represented
% by the Word sample file. 

% Note that since 2010, USENIX does not require endnotes. If you want
% foot of page notes, don't include the endnotes package in the 
% usepackage command, below.

\documentclass[letterpaper,twocolumn,10pt]{article}
\usepackage{usenix,epsfig,endnotes}
\usepackage{cite}
\usepackage{enumitem}
\begin{document}

%don't want date printed
\date{}

%make title bold and 14 pt font (Latex default is non-bold, 16 pt)
\title{\Large \bf Comprehensive Assessment and Analysis for NSFW Content Erasure in the Text-to-Image Diffusion model}

\author{
{\rm Anonymous authors}\\
% Your Institution
% \and
% {\rm Second Name}\\
% Second Institution
}

\maketitle
\thispagestyle{empty}

\begin{abstract}
Advancements in DNA sequencing technologies have significantly improved our ability to decode genomic sequences. However, the prediction and interpretation of these sequences remain challenging due to the intricate nature of genetic material. Large language models (LLMs) have introduced new opportunities for biological sequence analysis. Recent developments in genomic language models have underscored the potential of LLMs in deciphering DNA sequences. Nonetheless, existing models often face limitations in robustness and application scope, primarily due to constraints in model structure and training data scale. To address these limitations, we present \textbf{Gener}\textit{ator}, a generative genomic foundation model featuring a context length of 98k base pairs (bp) and 1.2B parameters. Trained on an expansive dataset comprising 386B bp of eukaryotic DNA, the \textbf{Gener}\textit{ator} demonstrates state-of-the-art performance across both established and newly proposed benchmarks. The model adheres to the central dogma of molecular biology, accurately generating protein-coding sequences that translate into proteins structurally analogous to known families. It also shows significant promise in sequence optimization, particularly through the prompt-responsive generation of enhancer sequences with specific activity profiles. These capabilities position the \textbf{Gener}\textit{ator} as a pivotal tool for genomic research and biotechnological advancement, enhancing our ability to interpret and predict complex biological systems and enabling precise genomic interventions. Implementation details and supplementary resources are available at \url{https://github.com/GenerTeam/GENERator}.
\keywords{DNA, Genomics, Foundation model, Generative model}
\vspace{12pt}
\end{abstract}



\section{Introduction}

Chain-of-Thought (CoT) prompting~\cite{Nye:2021, cot, Kojima:2022cotzero} has emerged as a cornerstone strategy for enhancing Large Language Models (LLMs) in complex reasoning tasks. By eliciting step-by-step inference, CoT enables LLMs to decompose intricate problems into manageable subtasks, thereby improving their problem-solving performance~\cite{Yao:2023tot, Wang:2023self-consistency, Zhou:2023least, Shinn:2023Reflexion}. Recent advancements, such as OpenAI's o1~\cite{o1} and DeepSeek-R1~\cite{deepseekr1}, further demonstrate that scaling up CoT lengths from hundreds to thousands of reasoning steps could continuously improve LLM reasoning. These breakthroughs have underscored CoT’s potential to advance LLM capabilities, expanding the boundaries of AI-driven problem-solving.

\begin{figure}[t]
\centering
    \includegraphics[width=0.95\columnwidth]{fig/intro.pdf}
    \caption{In contrast to vanilla CoT that generates all reasoning tokens sequentially, \method enables LLMs to \textit{skip} tokens with less semantic importance (\textit{e.g.,} \includegraphics[width=7pt]{fig/token.pdf}~) and learn shortcuts between critical reasoning tokens, facilitating controllable CoT compression.}
    \label{fig:intro}
\end{figure}

Despite its effectiveness, the increased length of CoT sequences introduces substantial computational overhead. Due to the autoregressive nature of LLM decoding, longer CoT outputs lead to proportional increases in both inference latency and memory footprints of key-value cache. Additionally, the quadratic computational cost of attention layers further exacerbates this burden. These issues become particularly pronounced when CoT sequences extend into thousands of reasoning steps, resulting in significant computational costs and prolonged response times. While prior research has explored methods for selectively skipping reasoning steps~\cite{Ding:2024cotshortcut, liu2024skipstep}, recent findings~\cite{jin:2024cotlength, Merrill:2024cotlength} suggest that such reductions may conflict with test-time scaling~\cite{o1-blog, snell2025scaling}, ultimately impairing LLM reasoning performance. Therefore, striking an optimal balance between CoT efficiency and reasoning accuracy remains a critical open challenge.

In this work, we delve into CoT efficiency and seek the answer to an important question: \textit{``Does every token in the CoT output contribute equally to deriving the answer?''} We empirically analyze the semantic importance of tokens within CoT outputs and reveal that their contributions to the reasoning performance vary, as depicted in Figure 2. Building on this insight, we introduce \method, a simple yet effective approach that enables LLMs to \textit{skip} less important tokens within CoT sequences and learn shortcuts between critical reasoning tokens, thereby allowing for controllable CoT compression with adjustable ratios. Specifically, as shown in Figure~\ref{fig:intro}, \method constructs compressed CoT training data with various compression ratios, by pruning unimportance tokens from original LLM CoT trajectories. Then, it conducts a general supervised fine-tuning process on target LLMs with this training data, facilitating LLMs to automatically trim redundant tokens during reasoning.

We conduct extensive experiments across various models, including LLaMA-3.1-8B-Instruct and the Qwen2.5-Instruct series, using two widely recognized math reasoning benchmarks: GSM8K and MATH-500. The results validate the effectiveness of \method in compressing CoT outputs while maintaining robust reasoning performance. Notably, Qwen2.5-14B-Instruct exhibits almost \textbf{NO} performance drop (less than $0.4\%$) with a $\bm{40\%}$ reduction in token usage on GSM8K. On the challenging MATH-500 dataset, LLaMA-3.1-8B-Instruct effectively reduces CoT token usage by $\bm{30}\%$ with a performance decline of less than $4\%$, resulting in a $\bm{1.4}\times$ inference speedup. Further analysis underscores the coherence of \method in specified compression ratios and its potential scalability with stronger compression techniques.

\method is distinguished by its low training cost. For Qwen2.5-14B-Instruct, \method fine-tunes only 0.2\% of the model's parameters using LoRA. The size of the compressed CoT training data is no larger than that of the original training set, with 7,473 examples in GSM8K and 7,500 in MATH. The training is completed in approximately 2 hours for the 7B model and 2.5 hours for the 14B model on two 3090 GPUs. These characteristics make \method an efficient and reproducible approach, suitable for use in efficient and cost-effective LLM deployment.

To sum up, our key contributions are:
\begin{enumerate}
    \item To the best of our knowledge, this work is the \textit{first} to investigate the potential of enhancing CoT efficiency through \textit{token skipping}, inspired by the varying semantic importance of tokens in CoT trajectories of LLMs.
    \item We introduce \method, a simple yet effective approach that enables LLMs to skip redundant tokens within CoTs and learn shortcuts between critical tokens, facilitating CoT compression with adjustable ratios.
    \item Our experiments validate the effectiveness of \method. When applied to Qwen2.5-14B-Instruct, \method reduces reasoning tokens by $40\%$ (from 313 to 181) on GSM8K, with less than a $0.4\%$ performance drop.
\end{enumerate}

\section{Background and Related works}

\subsection{NSFW (Not Safe for Work) Content}

In this paper, we use the definition of NSFW in \cite{gebru2021datasheets}, "[data that] if viewed directly, might be offensive, insulting, threatening, or might otherwise cause anxiety". Texts and images containing NSFW information are referred to as "unsafe text" or "unsafe pictures", respectively.

With the development of the internet and social media, NSFW has become a prominent issue, prompting various countries and organizations to propose laws and restrictions to address this problem. 
For example, the European Union's Digital Services Act \cite{eu-cybercrime} holds platforms accountable for harmful online activities, as well as the dissemination of disinformation. The introduction of the UN Convention against cybercrime \cite{un-cybercrime} further encourages international cooperation to tackle issues related to cybercrime and the proliferation of illegal content. Leading social platforms and online communities, such as YouTube and Twitter, have established specific guidelines and review standards to restrict users from encountering sensitive content in public spaces. The rapid development of generative AI has brought this issue to the forefront.
% , prompting tech companies to implement safety measures. 
OpenAI, a pioneering research organization, has adopted both universal and service-specific policies \cite{openai-usage-policies} and integrated safety controls in testing, user prompts, and model alignment \cite{openai-safety-updates}.

\begin{figure*}[t!]
    \centering
    \includegraphics[width=0.95\textwidth]{Images/Define.pdf}
    \vspace{-1em}
    \caption{NSFW is divided into five themes. We provide descriptions for these five themes and include image examples for a more concrete illustration. Since the erasure methods use keyword sets as erasure targets, we also present the complete keyword set for the more keywords version and a more generalized less keywords version.}
    \label{fig:define}
\end{figure*}

\begin{table*}[t!]
\small
\centering
\caption{Comparison of ours and other benchmarks}
\label{tab:benchmarks}
\scalebox{0.78}{
\setlength{\tabcolsep}{2pt}
\begin{tabular}{c|ccc|cc|cccccc}
\toprule
\multirow{2}{*}{\textbf{\makecell{ \\ \\Benchmark}}} & \multirow{2}{*}{\textbf{\makecell{\\Taxonomy of \\ Concept \\ Erasure Methods}}} & \textbf{\multirow{2}{*}{\makecell{\\Toxicity \\ Analysis of \\ Datasets}}} & \multirow{2}{*}{\textbf{\makecell{\\Comparison \\ of \\ Classifiers}}} & \multicolumn{2}{c|}{\textbf{Assessment  Content}}                   & \multicolumn{6}{c}{\textbf{Assessment  Perspectives}}                                                                                                \\
\cline{5-12}

                           &                                                                  &                                                &                                             & \textbf{\makecell{Specific Theme \\ in \\ NSFW Content}} & \textbf{\makecell{ Impact of \\ Data \\ Required}} & \textbf{\makecell{Erasure \\ Proportion}} & \textbf{\makecell{Impact of Explicit \\ and Implicit \\ Unsafe Prompts}} & \textbf{\makecell{Excessive \\ Erasure}} & \textbf{\makecell{Image \\ Quality}} & \textbf{\makecell{Semantic \\ Alignment}} & \textbf{Robustness}  \\
\toprule
UnsafeD \cite{qu2023unsafe}                    & \textbf{\ding{55}}                                                               & \textbf{\ding{51}}                                            & \textbf{\ding{51}}                                         & \textbf{\ding{55}}                             & \textbf{\ding{55}}                      & \textbf{\ding{51}}                & \textbf{\ding{55}}                                             & \textbf{\ding{55}}                & \textbf{\ding{55}}            & \textbf{\ding{55}}                 & \textbf{\ding{55}}          \\
UCANVAS \cite{zhang2024unlearncanvas}                   & \textbf{\ding{55}}                                                               & \textbf{\ding{55}}                                             & \textbf{\ding{55}}                                          & \textbf{\ding{55}}                             & \textbf{\ding{55}}                      & \textbf{\ding{51}}                & \textbf{\ding{55}}                                             & \textbf{\ding{55}}                & \textbf{\ding{51}}           & \textbf{\ding{55}}                 & \textbf{\ding{55}}          \\
HUB \cite{moon2024holistic}                       & \textbf{\ding{55}}                                                               & \textbf{\ding{55}}                                             & \textbf{\ding{55}}                                          & \textbf{\ding{55}}                             & \textbf{\ding{55}}                      & \textbf{\ding{51}}                & \textbf{\ding{55}}                                             & \textbf{\ding{55}}                & \textbf{\ding{51}}           & \textbf{\ding{55}}                 & \textbf{\ding{55}}          \\
Ours                       & \textbf{\ding{51}}                                                              & \textbf{\ding{51}}                                            & \textbf{\ding{51}}                                         & \textbf{\ding{51}}                            & \textbf{\ding{51}}                     & \textbf{\ding{51}}                & \textbf{\ding{51}}                                            & \textbf{\ding{51}}               & \textbf{\ding{51}}           & \textbf{\ding{51}}                & \textbf{\ding{51}}        \\
\bottomrule
\end{tabular}
}
\end{table*}

However, the categorization of NSFW has not formed a consensus and may vary according to context, culture, and personal factors. Previous work \cite{patrick2023safe,llama-guard} has proposed various categorization frameworks. To facilitate the identification of unsafe images in our subsequent work, we exclude content such as illegal activities and choose the categorization given in \cite{qu2023unsafe}, which classifies NSFW content into five themes: "Sexually Explicit, Violent, Disturbing, Hateful, and Political."
The description of such categorization and corresponding image examples are illustrated in Figure \ref{fig:define}.


\subsection{Text-to-image Diffusion Models}

Diffusion models for image generation are predominantly based on DDPM\cite{ho2020denoising}, which formulates both the diffusion and denoising processes as markov processes. The forward process is a noise-adding process, where at each time step $t$, Gaussian noise with varying intensity is added to the clean image data $x_0$ to obtain $x_t = \sqrt{\alpha_{t}} x_{0}+\sqrt{1-\alpha_{t}} \epsilon$ where $\alpha$ is a time-dependent noise intensity hyperparameter and $\epsilon$ is Gaussian noise. After $T$ time steps of adding noise, the clean image will ultimately become pure Gaussian noise.
% \begin{equation}
%     x_{t}=\sqrt{\alpha_{t}} x_{0}+\sqrt{1-\alpha_{t}} \epsilon_{t}
% \end{equation}
During the reverse denoising process, the model is trained to predict the noise added during the forward process, enabling the recovery of the original data $x_0$ from the noisy data. This process can be expressed as:  
\begin{equation}
    p_{\theta}\left(x_{t-1} \mid x_{t}\right)=\mathcal{N}\left(x_{t-1} ; \mu_{\theta}\left(x_{t}, t\right), \Sigma_{\theta}\left(x_{t}, t\right)\right)
\end{equation}
where $\mu_{\theta}$ and $\Sigma_{\theta}$ is the mean and variance learned from model.

Latent Diffusion Models (LDM) \cite{rombach2022high} build upon DDPM by performing the diffusion and denoising processes in a low-dimensional latent space. Compared to operations in the pixel space, this significantly enhances the efficiency of diffusion models.   This spatial mapping is typically achieved using a pre-trained encoder $\mathcal{E}$ and a decoder $\mathcal{D}$. For an input image $x$, the encoder maps $x$ to a latent code $z = \mathcal{E}(x)$. The decoder $\mathcal{D}$ can then map a latent code back to the corresponding image, such as $\mathcal{D}(\mathcal{E}(x))\approx x$. 
The network currently used by LDM to predict noise is usually UNet, in which the text condition interacts with the image through the cross-attention layer. 
Given a conditional input $c$, the training objective function of LDM is as follows:  
\begin{equation}
    \mathcal{L}=\mathbb{E}_{z\sim\mathcal{E}(x),t,c,\epsilon\sim\mathcal{N}(0,1)}\left[\|\epsilon-\epsilon_\theta(z_t,c,t)\|_2^2\right]
\end{equation}


Classifier-free guidance technique \cite{ho2022classifier} allows the diffusion model to use an implicit classifier to guide the generation process. During training, the model learns both conditional and unconditional generation. By moving towards conditional scores while simultaneously moving away from unconditional scores, the model generates images that follow the given conditions during the generation phase. Given a guidance scale $\alpha$, the predicted noise at time step $t$ can be expressed as follows:  
\begin{equation}
\tilde{\epsilon}_\theta(z_t,c,t)=\epsilon_\theta(z_t,t)+\alpha(\epsilon_\theta(z_t,c,t)-\epsilon_\theta(z_t,t)).
\end{equation}



\subsection{Risk of Diffusion Model Misuse}

Due to their impressive generative capabilities and accessibility, diffusion models are gaining increasing popularity among users. However, this rise in adoption also exacerbates the risk of misuse. Some studies have highlighted that open-source diffusion models can be easily exploited to generate NSFW images, with users producing pornographic or violent content simply by providing specific prompts\cite{Javier2022redteaming,patrick2023safe}. 
% A community called Unstable Diffusion specializes in generating pornographic images using Stable Diffusion, which attracts a large number of users to create and spread unsafe content on it. 
% At the same time, some online creation platforms (e.g., Lexica, Artbreeder) have been found to allow users to bypass platform restrictions and generate explicit illegal images. Additionally, AI "de-clothing" applications have emerged in various parts of the world, using generative models to create nude images of specific individuals from photos. 
% In 2024, the "AI pimping" industry emerged, with virtual influencers using deepfake technology to promote adult content; the X platform updated its policy to allow adult content while restricting access for minors. In 2025, a scandal in Sydney involved a student using AI to create and distribute deepfake explicit images. These events all demonstrate that the misuse of diffusion models has already caused widespread negative impacts.
The "AI pimping" industry \cite{ai-pimping} replaces the faces of adult content creators in videos with AI-generated ones, creating a surge of virtual influencers who profit by linking to platforms. 
Deepfake \cite{deepfake} technology manipulates videos or images by swapping faces or altering appearances, often creating explicit content without consent, leading to reputational damage and legal issues. Collectively, these cases demonstrate the pressing need for robust safeguards and ethical considerations in the development and deployment of generative models. 



% \subsection{Red Teaming Methods}

% Recent studies suggest using red teaming tools to identify potential vulnerabilities within generative models. In text-to-image models, a typical red teaming method involves constructing adversarial prompts to detect whether the model generates harmful images. For example, CI \cite{minh2024circumventing} employs the textual inversion technology[ ] to train a pseudo-token, enabling the model to reproduce unsafe images. P4D \cite{zhiyi2024p4d} utilizes the original SD model to generate harmful images and then optimizes an adversarial prompt $P^*$ to enable concept erasure methods to produce images similar to the harmful ones. UnlearnDiffAtk \cite{yimeng2024unlearndiffatk} adopts a similar approach, but during training, noise is directly added to the harmful images. And RAB \cite{yu2024ring} does not require access to a specific model. Instead, it leverages a pre-trained text encoder and generates adversarial prompts through relative textual semantics combined with genetic algorithms. In this work, RAB was employed to construct adversarial prompts for various NSFW concepts, evaluating the robustness of concept erasure methods.  

% Please add the following required packages to your document preamble:
% \usepackage{multirow}
% \begin{table*}[]
% \small
% \caption{Properties and references of concept erasure methods}
% \label{table:methods}
% \centering
% \scalebox{0.9}{
% \setlength{\tabcolsep}{4pt}
% \begin{tabular}{c|cl|c|c|c}
% \toprule
% \textbf{Stage}            & \multicolumn{2}{c|}{\textbf{Required Data Types}}                                               & \textbf{Core Principles}                  & \textbf{Parameters Involved} & \textbf{Reference}    \\ 
% \midrule
% \makecell{Dataset \\ Cleaning}                       & \multicolumn{2}{c|}{No data required}                                                           & /                                         & /                            & Stable Diffusion v2.0 \cite{Stable-Diffusion-2.0}                \\

% \midrule

% \multirow{6}{*}{\makecell{Parameter \\ Fine-Tuning}} & \multicolumn{2}{c|}{\multirow{3}{*}{Only target text concepts (Mode 1)}}                       & \multirow{2}{*}{Away from target concept} & Unet                         & \makecell{ESD\cite{gandikota2023erasing-esd}, SPM\cite{lyu2024one-spm}}               \\ 

% \cline{5-12} 
%                                        & \multicolumn{2}{c|}{}                              &                                                                                       & Encoder                      & \makecell{AU\cite{zhang2024defensive-au}}                    \\ 
% \cline{4-6} 
%                                        & \multicolumn{2}{c|}{}                                                                           & Close replacement concept                 & Unet                         & \makecell{UCE\cite{gandikota2024unified-uce}}                   \\
% \cline{2-6} 
%                                        & \multicolumn{1}{c|}{\multirow{3}{*}{\makecell{Images \\ (Mode 2)}}} & Safe images opposite to target concepts & Close replacement concept                 & Unet                         & \makecell{AC\cite{kumari2023ablating-ca}, SelfD\cite{li2024self-selfd} }             \\ 
% \cline{3-6}

%                                        & \multicolumn{1}{c|}{}                                 & Unsafe images related to target concept & Other                                     & Unet                         & \makecell{FMN\cite{zhang2024forget-fmn}, MACE\cite{lu2024mace}}              \\ 
% \cline{3-6} 
%                                        & \multicolumn{1}{c|}{}                                 & Both safe and unsafe images             & Close replacement concept                 & Unet                         & \makecell{SalUn\cite{fan2023salun}}                 \\ 
% \midrule
% \makecell{Post-hoc \\ Correction}                    & \multicolumn{2}{c|}{Only target text concepts}                                                  & Away from target concept                  & /                            & \makecell{SLD\cite{patrick2023safe}, SD-NP\cite{ho2022classifier}, \\ Safety Checker\cite{sd1-4}} \\ 
% \bottomrule
% \end{tabular}
% }
% \end{table*}
% \usepackage{multirow}


\begin{figure*}[t!]
    \centering
    \includegraphics[width=0.88\textwidth]{Images/framework.pdf}
    \vspace{-1em}
    \caption{Our benchmark framework consists of three parts: assessment tools, assessment targets, and assessment content. In terms of assessment tools, we conduct toxicity analysis on the NSFW dataset and compare the accuracy of classifiers. These tools are used in assessment experiments for concept erasure methods, which are divided into two modes. To analyze the specific data requirements of each method, we differentiate between different versions of the methods. For assessment content, we categorize specific themes under NSFW and perform the analysis from six different perspectives.}
    \label{fig:framework}
\end{figure*}


\subsection{Safety Benchmarks for Diffusion Models}
Recently, the community has proposed several benchmarks to evaluate the safety of generative diffusion models. For instance, UnsafeD\cite{qu2023unsafe} constructed four unsafe datasets and trained a multi-headed safety classifier to assess the safety of text-to-image models. However, this work primarily focuses on image editing methods and hateful meme generation, without addressing concept erasure methods. 
% Similarly, SLD\cite{patrick2023safe} released the I2P dataset, which comprises real-world unsafe text prompts containing various unsafe concepts. However, it only provides a prompt set without a precise and comprehensive evaluation framework.
% In another example, 
UCANVAS\cite{zhang2024unlearncanvas} introduced a stylized image dataset specifically for assessing style forgetting, but it does not address NSFW concepts. HUB\cite{moon2024holistic} evaluates concept erasure methods from multiple perspectives, such as effectiveness on targets and faithfulness of generated images, but their focus is limited to the erasure of objects. In contrast, our work compiles and organizes multiple unsafe prompt sets and systematically evaluates state-of-the-art concept erasure methods for different NSFW themes in diffusion models.
% We classify these methods and conduct a thorough evaluation of their ability to erase fine-grained NSFW concepts. 
Table \ref{tab:benchmarks} demonstrates the advantages of our approach compared to other benchmarks.





Our methodology begins with applying the Boruta algorithm to sift through the high-dimensional gene expression dataset, effectively filtering out irrelevant features and isolating those that are truly significant. This initial reduction is critical because LIME, our subsequent interpretability tool, involves generating numerous perturbed samples and calculating distances—a computationally intensive process, especially in large feature spaces. By narrowing the feature set, we improve LIME's precision and enhance the model's interpretability, ultimately boosting classification performance.

However, the key question remains: how many features should be selected for optimal classification? In our proposed BOLIMES algorithm, which integrates Boruta and LIME, we first reduce the high-dimensional feature set by eliminating irrelevant variables with Boruta. Then, we further refine this subset using LIME to assess the local importance of each feature. Finally, we determine the optimal number of features by evaluating classification performance—selecting the subset that yields the highest accuracy for model training. In general, our model is both efficient and robust, relying only on the most informative features for gene expression classification.

\begin{algorithm}[h]
\scriptsize
    \DontPrintSemicolon
    \SetKwInOut{Input}{Input}
    \SetKwInOut{Output}{Output}
    
    \Input{Dataset \( D = (X, y) \) with \( X \subseteq \mathbb{R}^p \) and class labels \( y \).}
    \Output{Optimal feature subset \( X_{\text{opt}} \) and trained classifier \( f_{\text{opt}} \).}
    
    \Begin{
        \( X^* \leftarrow \text{Boruta}(D) \) \tcp*{\scriptsize Identify relevant features from \( X \)}
        
        \( \mathcal{I} \leftarrow \text{LIME}(f, X^*) \) \tcp*{\scriptsize Compute local importance scores on \( X^* \)}
        
        \( X^{*}_R \gets \{ x^*_{(i)} \}_{i=1}^{|X^*|} \) \quad  \( \mathcal{I}(x^*_{(1)}) \geq \dots \geq \mathcal{I}(x^*_{(|X^*|)}) \) \tcp*{\tiny \textbf{Rank} features in \( X^* \) in descending order of \( \mathcal{I} \)}
        
        \( \text{best\_acc} \leftarrow 0 \), \( k^* \leftarrow 0 \)\;
        
        \For{\( k = 10 \) \KwTo \( |X^*| \)}{
            \( S_k \gets \{ x^*_{(i)} \}_{i=1}^{k} \) \tcp*{\scriptsize Select top-\( k \) features from \( X^*_R \)}
            \( f_k \leftarrow \text{TrainClassifier}(D_{S_k}) \) \tcp*{\scriptsize Train classifier on \( D_{S_k} \)}
            \( \text{acc} \leftarrow \text{Evaluate}(f_k, D_{S_k}) \) \tcp*{\scriptsize Compute classification accuracy}
            
            \If{\( \text{acc} > \text{best\_acc} \)}{
                \( \text{best\_acc} \leftarrow \text{acc} \)\;
                \( k^* \leftarrow k \)\;
                \( f_{\text{opt}} \leftarrow f_k \)\;
            }
        }
        
        \( X_{\text{opt}} \gets \{ x^*_{(i)} \}_{i=1}^{k^*} \) \tcp*{\scriptsize Select top-\( k^* \) features from \( X^*_R \)}
        
        \Return{\( X_{\text{opt}}, f_{\text{opt}} \)}
    }
    
    \caption{Optimal Feature Selection for Classification using Boruta and LIME}
    \label{alg:BOLIMES}
\end{algorithm}

% The algorithm optimally selects features for classification by integrating Boruta and LIME. First, Boruta extracts a relevant subset \( X^* \) from \( X \). LIME then assigns importance scores \( \mathcal{I} \) to \( X^* \), producing a ranked set \( X^*_R \) where \( \mathcal{I}(x^*_{(1)}) \geq \dots \geq \mathcal{I}(x^*_{(|X^*|)}) \).  For \( k = 1 \) to \( |X^*| \), the algorithm selects the top-\( k \) features \( S_k \), trains a classifier \( f_k \), and evaluates its accuracy. The best-performing model determines the optimal feature count \( k^* \), yielding \( X_{\text{opt}} = \{ x^*_{(i)} \}_{i=1}^{k^*} \) and classifier \( f_{\text{opt}} \). This method ensures feature selection maximizes classification accuracy while maintaining interpretability.


Algorithm \ref{alg:BOLIMES} presents an optimal feature selection framework by integrating Boruta and LIME to refine feature subsets for classification. Given a dataset \( D = (X, y) \), it first applies Boruta to extract a subset \( X^* \) of relevant features. LIME then computes local importance scores \( \mathcal{I} \) for \( X^* \), producing a ranked set \( X^*_R \) where
\[
X^*_R \gets \{ x^*_{(i)} \}_{i=1}^{|X^*|}, \quad \mathcal{I}(x^*_{(1)}) \geq \dots \geq \mathcal{I}(x^*_{(|X^*|)}).
\]
An iterative search determines the optimal number of features \( k^* \) by selecting the top-\( k \) ranked features, training a classifier \( f_k \), and evaluating its accuracy:
\[
S_k \gets \{ x^*_{(i)} \}_{i=1}^{k}, \quad f_k \leftarrow \text{TrainClassifier}(D_{S_k}).
\]
Note that we begin with 
$k=10$ and increase it until 
$|X^*|$, as using a smaller number of vectors would be inefficient. The best-performing classifier defines \( k^* \), yielding the final subset \( X_{\text{opt}} \) and trained model \( f_{\text{opt}} \). 


The algorithm's complexity is driven by three key stages: Boruta for feature selection, LIME for ranking, and iterative classification. Boruta, relying on multiple iterations of Random Forest, has a worst-case complexity of \( O(T \cdot p^2 \log p) \). LIME, which perturbs \( m \) samples per feature, contributes \( O(m \cdot p) \). The final stage trains classifiers iteratively over \( p^* \) ranked features, leading to an overhead of \( O(n p^*{}^2) \) assuming a model with \( O(n k) \) complexity. Thus, the total complexity is:
\(
O(T \cdot p^2 \log p) + O(m \cdot p) + O(n p^*{}^2),
\)
where \( p^* \ll p \) in practice, making the approach feasible for moderate-dimensional data but computationally intensive for extremely large \( p \).





\section{Assessment Tools}
\label{tool}

In this section, we conduct a detailed analysis of the NSFW dataset and compare the accuracy of the classifiers, so that we can make better use of these tools.

\subsection{Analysis of Datasets}
\label{Analysis-of-Datasets}
Since concept erasure methods are based on text-to-image diffusion models, the prompt datasets are needed to generate images. The effectiveness of the concept erasure is assessed by analyzing the generation of NSFW content in the images.
Previous work has compiled several relevant datasets, but analyses based on specific NSFW themes are either missing or relatively coarse. 

Our evaluation involves five datasets, four of which are related to NSFW content, including I2P \cite{patrick2023safe}, 4chan \cite{qu2023unsafe}, Lexica \cite{qu2023unsafe}, and Template \cite{qu2023unsafe}, used to assess the NSFW erasure effects. The fifth is the COCO-10K dataset \cite{lin2014microsoft-coco}, which is a general dataset used to evaluate the generation ability. Table \ref{table:datasets} presents detailed information about these datasets.

% Please add the following required packages to your document preamble:
% \usepackage{multirow}
% \usepackage[normalem]{ulem}
% \useunder{\uline}{\ul}{}
\begin{table*}[]
\small
\caption{Basic information and toxicity analysis of the five datasets in the benchmark}
\centering
\label{table:datasets}
\scalebox{0.76}{
\setlength{\tabcolsep}{3pt} 
\begin{tabular}{c|ccc|ccc|ccccc|cc}
\toprule
\multirow{3}{*}{\textbf{Dataset}} & \multicolumn{3}{c}{\textbf{Information}}                                   & \multicolumn{3}{|c}{\textbf{Prompt Toxicity}}                                                        & \multicolumn{5}{|c}{\textbf{Image Classification}}                                                                                              & \multicolumn{2}{|c}{\textbf{\makecell{Ratio of Explicit and \\ Implicit Unsafe Prompts}}} \\
\cmidrule(){2-14}
                   & \textbf{Prompts} & \textbf{Length}      & \textbf{\makecell{Images for \\ Every \\ Prompt}} & \textbf{\makecell{Low \\ Toxicity \\ Prompts}} & \textbf{\makecell{Moderate \\ Toxicity \\ Prompts}} & \textbf{\makecell{High \\ Toxicity \\ Prompts}} & \textbf{\makecell{Sexually \\ Explicit \\ Images}} & \textbf{\makecell{Violent \\ Images}} & \textbf{\makecell{Disturbing \\ Images}} & \textbf{\makecell{Hateful \\ Images}} & \textbf{\makecell{Political \\ Images}} & \textbf{\makecell{Explicit \\ Unsafe \\ Prompts}}     & \textbf{\makecell{Implicit \\ Unsafe \\ Prompts}}    \\
\toprule
I2P                         & 4703             & \makecell{20 tokens \\ on average} & 1                                & 25.52\%                       & 72.97\%                            & 1.51\%                         & 15.52\%                           & 10.14\%                 & 20.67\%                    & 0.15\%                  & 3.15\%                    & 0.72\%                              & 43.16\%                             \\
\midrule
\makecell{4chan prompt }               & 500              & \makecell{8 tokens \\ on average}  & 3                                & 0.00\%                        & 0.00\%                             & 100.00\%                       & 15.00\%                           & 5.40\%                  & 4.87\%                     & 0.20\%                  & 1.67\%                    & 25.00\%                              & 0.00\%                              \\
\midrule
\makecell{Lexica prompt }              & 404              & \makecell{17 tokens \\ on average} & 3                                & 28.71\%                       & 70.05\%                            & 1.24\%                         & 13.78\%                           & 10.07\%                 & 39.03\%                    & 2.56\%                  & 9.08\%                    & 0.66\%                              & 64.03\%                             \\
\midrule
\makecell{Template prompt}            & 30               & \makecell{17 tokens \\ on average} & 20                               & 48.00\%                       & 41.50\%                            & 10.50\%                        & 27.33\%                           & 33.17\%                 & 34.33\%                    & 4.00\%                  & 5.67\%                    & 7.17\%                              & 72.83\%                             \\
\midrule
\makecell{COCO-10K}                    & 10000            & \makecell{Usually \\ 12-15 tokens} & 1                                & /                             & /                                  & /                              & \multicolumn{1}{c}{/}             & \multicolumn{1}{c}{/}   & \multicolumn{1}{c}{/}      & \multicolumn{1}{c}{/}   & \multicolumn{1}{c|}{/}     & /                                    & /        
\\
\bottomrule
\end{tabular}
}
\end{table*}
% \vspace{-2mm}

\begin{itemize}[itemsep=0pt]
    \item The I2P (Inappropriate Image Prompts) dataset consists of 4703 prompts, which are obtained by searching and crawling the first 250 prompts on the Lexica website using 26 NSFW-related keywords and phrases and filtering duplicate entries. Lexica \cite{Lexica} is a website that stores a large collection of high-quality generated images and their corresponding real-world prompts. On average, each prompt consists of 20 tokens.
    \item The 4chan prompt dataset contains 500 sentences from an anonymous image board named 4chan, which is known for sharing toxic and unsafe images within an underground online community. \cite{papasavva2020raiders} compile 134 million raw posts, though many contain confusing sentences. After applying syntactic pattern matching with the MS COCO caption dataset \cite{lin2014microsoft-coco} and toxic filtering, this dataset is derived. Each prompt in this dataset typically consists of 8 tokens on average.
    \item The Lexica prompt dataset contains 404 prompts from the Lexica website. It is curated by querying 34 NSFW-related keywords from the DALL·E content policy \cite{openai-usage-policies}, followed by de-duplication. The average length of prompts is 17 tokens.
    \item The Template prompt dataset consists of 30 prompts synthesized by filling in 30 candidate phrases within a fixed template, unlike the above three user-generated datasets. On average, each prompt consists of 17 tokens.
    \item The MS COCO dataset is a general dataset that includes 80 object categories, with captions that are approximately 12–15 words long. It is widely used for various image understanding tasks. We select 10,000 captions from this dataset (COCO-10K) to evaluate the model's generation capabilities.
\end{itemize}

\noindent\textbf{Toxicity Analysis of Prompts.}
We use the Perspective API \cite{pers-api} to perform toxicity analysis on prompts in four NSFW-related prompt datasets, and obtain a toxicity score for each prompt, which is a floating value between 0 and 1. Based on this, we classify the toxicity of the prompts: a toxicity score of [0, 0.2) indicates low toxicity, typically considered harmless; a score of [0.2, 0.5) indicates moderate toxicity, which may contain some negative content; and a score of [0.5, 1] indicates high toxicity, clearly featuring aggressive and offensive elements.

Table \ref{table:datasets} shows the results of the prompt toxicity statistics. We find that the 4chan prompt dataset consists entirely of high-toxicity prompts, as only those with a toxicity score greater than 0.8 are selected during the dataset creation process. 
The Template prompt dataset has a more even distribution compared to the others. Although there are also many highly toxic prompts, the highest is only 0.68. The distribution of the Lexica dataset and the I2P dataset is similar, with moderately toxic prompts as the main ones, because the sources of the two are the same. From the perspective of themes, Perspective API gives a lower toxicity judgment for politically related prompts. This is because politically sensitive themes are affected by current events and are more potentially harmful. Prompts with toxicity scores close to 1 are mainly focused on contents combining sex and discrimination, and all come from the 4chan dataset.


\noindent\textbf{Toxicity Analysis of Generated Images.}
To better understand the generation effects of basic text-to-image models on NSFW-related prompt datasets, we use Stable Diffusion v1.4 to generate 1 image for each prompt from I2P, 3 images for each prompt from 4chan and Lexica, and 20 images for each prompt from Template. We then manually label the images with the themes they represent. If any of the five themes appears in an image, it is classified as an NSFW image. The number of images generated here, along with the seeds, is consistent with the number of images and seeds used for each baseline in subsequent erasure tasks, which also helps us compare the differences between methods more clearly.

It is important to note that although the I2P dataset is originally categorized into seven NSFW subcategories, during our labeling process, we find many rough or mismatched classifications. The other three datasets are not even organized in this way. Our work thus contributes to a deeper analysis.

Table \ref{table:datasets} the results of our image toxicity analysis. Overall, among the four NSFW-related datasets, 48.91\% of the images are labeled as NSFW. The Disturbing theme accounted for the highest proportion at 29.25\%, followed by 19.1\% for sexually explicit images, and the Hateful theme had the lowest proportion at 2.03\%. From the performance of each dataset, the Template dataset leads to the most NSFW images, likely because the template of its prompt mentions the 4chan style, and the 30 candidate phrases filled in are more directly linked to each theme. The reason why the 4chan dataset produces the least unsafe images may be that many of the prompts are opinion-based and vivid imagery. Its short sentence length also prevents the images from showing more details.

\noindent\textbf{Toxicity Association Analysis.}
From the above, we have obtained the toxicity of the prompts in the datasets and the toxicity of the generated images. We denote prompts that are inherently of high toxicity and lead to unsafe images as \textit{explicit unsafe prompts}, while prompts that are of moderate or low toxicity but still lead to unsafe images are denoted as \textit{implicit unsafe prompts}. According to the results shown in Table \ref{table:datasets}, we observe that 25\% of the prompts in the 4chan prompt dataset are explicit unsafe prompts, with no implicit unsafe prompts. This is because the dataset only contains high-toxicity prompts, and as mentioned above, the content and length limitations of some prompts prevent them from expressing unsafe elements in the generated images. 
The remaining three datasets are mostly implicit unsafe prompts, while the Template prompt data set has the most.
% Although nearly half of the prompts in the Template prompt dataset are of low toxicity, 75.33\% of them can still lead to unsafe images. Similarly, in the I2P and Lexica datasets, there are also mostly implicit unsafe prompts.

Therefore, we conclude that the toxicity of a prompt does not necessarily correlate with the toxicity of the generated image. For example, profanity tends to only express NSFW content at the textual level. And subtle words are more likely to trigger the generation of unsafe images. This reminds us that when implementing safety measures for text-to-image models, we need consider both the textual and visual aspects.

\subsection{Analysis of Classifier}
\label{Analysis-of-Classifier}
To determine whether the model generates images of a specific theme, an superior image classifier is required. 
We use the manually labeled images from the image toxicity analysis to compare classifier accuracy, treating the manually labeled results as the ground truth.

We compare the accuracy of CLIP \cite{Alec2021clip}, MHSC \cite{qu2023unsafe}, and VQA \cite{Zhiqiu2024vqa}, then we select VQA with the best performance for subsequent experiments. The experimental results are shown in Figure \ref{fig:accofclassifiers} and more details can be found in Appendix \ref{sec:classifier}.

% We first use Stable Diffusion v1.4 to generate images on four NSFW datasets. These images are manually labeled with the five themes, and these labels are used as the ground truth. Then, we obtain the classification results from three classifiers on these images and compare their classification accuracy.  

When comparing manual labeling with classifier results, we observe significant discrepancies between the model's understanding and human interpretation, particularly for images with abstract contents or artistic elements. 
Despite efforts to clearly define the scope and content of NSFW, ambiguous labels may persist due to varying levels of openness, especially in the Sexually Explicit Themes. Furthermore, overlapping elements between themes can complicate classification; for example, content about blood may be associated with both Violent and Disturbing themes. Besides, due to the limitations of Stable Diffusion v1.4's generation capabilities, content that is not perfectly presented is more likely to deviate from the real world and be classified as disturbing content.

Classifiers, of course, also interpret images differently. In our experiments, we find that MHSC is more conservative, often classifying images as safe. Its lower sensitivity to hateful and political content results in higher accuracy for these themes. In contrast, VQA demonstrates high sensitivity because it can classify images with unclear facial features as disturbing content, particularly in the 4chan dataset. 

The results show that VQA's classifications align most closely with human annotations, achieving the highest accuracy across four themes except for the Hateful theme. Unlike MHSC, VQA does not require training on NSFW content. However, the need for more flexible and accurate NSFW classifiers remains critical for advancing safe content moderation and control.



% In the experiment, we observed that MHSC tends to classify more images as safe, resulting in higher accuracy for the 'Hate' and 'Political' compared to the other three themes. This is because the images containing these two themes are relatively few in the four datasets.  
 % For NudeNet, we only evaluate its accuracy in detecting nudity, and considering nudity detected only when sensitive body parts are exposed. Although NudeNet is focused on detecting explicit content, its accuracy is not the highest, as it may misclassify objects that resemble genitalia, breasts, or other similar shapes. Moreover, the presence of many human images and complex scenes in the 4chan dataset may have contributed to NudeNet's suboptimal performance on this dataset.  
 % The CLIP classifier cannot classify individual topics separately, which leads to a lower overall accuracy.  
 % The classification of VQA is more aggressive, resulting in a lower accuracy for the "Hate" compared to MHSC. But on the whole, VQA's classification results are the closest to human labeling. 
 % while the latter, although it may not be the most accurate in detecting nudity, has the unique ability to identify various exposed body parts, which is a feature not offered by other models.  


%\section{Effect Analysis}
\section{Benchmark Results and Analysis}
\label{sixperspectie}
We evaluate 14 baselines derived from 9 concept erasure methods, all trained or inferred using the original Stable Diffusion v1.4. These include 9 baselines in Mode 1, which require only text erasure targets (each with two versions), and 5 baselines in Mode 2, which require images as input (each with three versions).
For the I2P dataset, one image is generated per prompt, while for the 4chan and Lexica prompt datasets, three images are generated per prompt. For Template prompt dataset, twenty images are generated per prompt. The diffusion process is 40 steps for image generation.


\subsection{Erasure Proportion}
\label{erasure-proportion}
We let all the baselines generate images on four NSFW-related datasets and performed binary classification on them across five themes using VQA. 
Figure \ref{fig:erase} shows the statistical results of the erasure scores for these baselines. The larger the coverage area of the radar chart, the better the baseline is at erasing NSFW content. More specific numerical values are provided in Table \ref{tab:alldatasetscore} of Appendix \ref{erasure-on-all-datasets}. Then we can analyze the results from the aspects of method, version, and theme.

\begin{figure*}[t!]
    \centering
    \includegraphics[width=0.75\textwidth]{Images/erase.pdf}
    \vspace{-1em}
    \caption{Erasure scores ($\uparrow$) of different methods on five themes in two modes. Different versions of the method generate corresponding images for four NSFW datasets, and after classification using VQA, the erasure scores for each theme are calculated. A larger method coverage area indicates better performance.}
    \label{fig:erase}
\end{figure*}


\noindent\textbf{Method Aspect.} In Mode 1, where only textual keywords are required, SLD-Max performs the best. It also outperforms all other methods and achieves a uniform erasure effect across various themes. As a post-hoc method, SLD gradually increases the guidance scales during the inference process to move further away from the target concept, as evidenced by the incremental trend observed in the erasure scores of SLD-Med, SLD-Str, and SLD-Max. Methods that require training, such as ESD-u, ESD-x, SPM, UCE, and AU, generally perform worse than the four post-hoc methods. This may be because training for different target concepts requires fine-tuning hyperparameters, whereas we set unified hyperparameters here. SPM, the method with the lowest erasure score, includes a semantic distance computation step during generation, which minimizes the effect on prompts that are semantically distant. This may limit its erasure effectiveness. 

In Mode 2, where images are used for fine-tuning, SalUn achieves the highest erasure score. FMN performs the worst, possibly due to its attention decoupling mechanism, which is less suitable for scenarios involving multiple keywords.

\begin{table*}[t!]
\centering
\small
\caption{Results of different methods for erasing overall NSFW in terms of erasure score, image quality, and semantic alignment.There are two sources of NSFW. In Mode 1, one is the concatenation of words in our more keywords version, and the other is the 20 words provided by SLD. For the method of Mode 2, our definition of overall NSFW and a NSFW image dataset composed of 200 images of each theme be used.}
\label{table:nsfw}
\scalebox{0.75}{
\setlength{\tabcolsep}{3pt}
\begin{tabular}{c|c|ccccccccc|c|c|cccccc}
\toprule
                   \textbf{Metric}          & \textbf{Source}      & \textbf{SD-NP}  & \textbf{SLD-Med} & \textbf{SLD-Str} & \textbf{SLD-Max} & \textbf{ESD-u}  & \textbf{ESD-x}  & \textbf{SPM}    & \textbf{UCE}    & \textbf{AU}     &      \textbf{Metric}     & \textbf{Source}                                                              & \textbf{AC}& \textbf{FMN}                     & \textbf{SalUn}                   & \textbf{SelfD}                   & \textbf{MACE}                     \\
\toprule
\multirow{2}{*}{\makecell{Erasure \\ Score ($\uparrow$)}}        & \makecell{ Ours} & 0.66~  & 0.53~   & 0.76~   & \textbf{0.90~}   & 0.12~  & 0.11~  & 0.05~  & 0.30~  & 0.71~  & \multirow{2}{*}{\makecell{Erasure \\ Score ($\uparrow$)}}      & \multirow{2}{*}{\makecell{Ours}} & \multirow{2}{*}{0.26~}  & \multirow{2}{*}{0.01~}  & \multirow{2}{*}{0.52~}  & \multirow{2}{*}{\textbf{0.64~}}  & \multirow{2}{*}{0.36~}   \\
\cline{2-11}

                             & \makecell{SLD}    & 0.42~  & 0.34~   & 0.51~   & \textbf{0.63~}   & 0.05~  & 0.13~  & 0.00~  & -0.04~ & 0.57~  &                            &                                                                      &                         &                         &                         &                         &                          \\
% \cline{3-11} \cline{14-18}
\midrule
\multirow{2}{*}{FID ($\downarrow$)}         & \makecell{Ours} & 26.32~ & 24.02~  & 27.72~  & 33.43~  & \textbf{17.77~} & 18.64~ & 19.40~ & 33.67~ & 22.24~ & \multirow{2}{*}{FID ($\downarrow$)}       & \multirow{2}{*}{\begin{tabular}[c]{@{}c@{}}Ours\end{tabular}} & \multirow{2}{*}{\textbf{19.26~}} & \multirow{2}{*}{18.43~} & \multirow{2}{*}{24.70~} & \multirow{2}{*}{30.01~} & \multirow{2}{*}{51.24~}  \\
\cline{2-11}
                             & \makecell{SLD}    & 24.27~ & 22.60~  & 25.46~  & 29.64~  & 20.08~ & \textbf{18.90~} & 19.22~ & \textbf{}18.47~ & 24.04~ &                            &                                                                      &                         &                         &                         &                         &                          \\
\midrule
\multirow{2}{*}{LPIPS ($\downarrow$)}       & \makecell{Ours} & 0.49~  & 0.48~   & 0.49~   & 0.50~   & \textbf{0.46~}  & 0.47~  & 0.48~  & 0.50~  & 0.48~  & \multirow{2}{*}{LPIPS ($\downarrow$)}     & \multirow{2}{*}{\begin{tabular}[c]{@{}c@{}}Ours\end{tabular}} & \multirow{2}{*}{\textbf{0.47~}}  & \multirow{2}{*}{\textbf{0.47~}}  & \multirow{2}{*}{0.48~}  & \multirow{2}{*}{0.48~}  & \multirow{2}{*}{0.49~}   \\
\cline{2-11}
                             & \makecell{SLD}    & 0.48~  & 0.47~   & 0.48~   & 0.49~   & \textbf{0.46~}  & 0.47~  & 0.48~  & 0.47~  & 0.48~  &                            &                                                                      &                         &                         &                         &                         &                          \\
\midrule
\multirow{2}{*}{CLIPScore ($\uparrow$)}   & \makecell{Ours} & 25.05~ & 25.47~  & 24.66~  & 23.75~  & 24.70~ & 25.11~ & \textbf{26.29~} & 23.58~ & 23.20~ & \multirow{2}{*}{CLIPScore ($\uparrow$)} & \multirow{2}{*}{\begin{tabular}[c]{@{}c@{}}Ours\end{tabular}} & \multirow{2}{*}{26.02~} & \multirow{2}{*}{\textbf{26.16~}} & \multirow{2}{*}{24.64~} & \multirow{2}{*}{24.59~} & \multirow{2}{*}{16.39~}  \\
\cline{2-11}
                             & \makecell{SLD}    & 25.20~ & 25.53~  & 24.80~  & 23.97~  & 24.51~ & 25.17~ & 26.29~ & \textbf{26.43~} & 23.30~ &                            &                                                                      &                         &                         &                         &                         &                          \\
\midrule
\multirow{2}{*}{ImageReward ($\uparrow$)} & \makecell{Ours} & -0.06~ & 0.02~   & -0.11~  & -0.31~  & -0.30~ & -0.17~ & \textbf{0.09~}  & -0.76~ & -0.65~ & \multirow{2}{*}{ImageReward ($\uparrow$)}      & \multirow{2}{*}{\begin{tabular}[c]{@{}c@{}}Ours\end{tabular}} & \multirow{2}{*}{\textbf{0.03~}}  & \multirow{2}{*}{0.00~}  & \multirow{2}{*}{-0.19~} & \multirow{2}{*}{-0.58~} & \multirow{2}{*}{-1.88~}  \\
\cline{2-11}
                             & \makecell{SLD}    & 0.01~  & 0.05~   & -0.05~  & -0.20~  & -0.38~ & -0.14~ & 0.10~  & \textbf{0.17~}  & -0.60~ &                            &                                                                      &                         &                         &                         &                         &          \\
\bottomrule
\end{tabular}
}
\end{table*}


\noindent\textbf{Version Aspect.} In Mode 1, the fewer keywords version and the more keywords version show identical results on the Hateful and Political themes due to the consistent keywords adopted for lack of generalization. For other themes, the erasure scores of the methods related to Mode 1 do not differ significantly between the two versions, indicating that our generalization experiment successfully extracted effective and fewer keywords. UCE is relatively more sensitive to versions, which may be because of its deep involvement in the linear transformation of the attention mechanism in cross-attention, where the target concept is replaced with an empty string, thus making the model more sensitive to changes in its own understanding.

In Mode 2, AC, FMN, and MACE demonstrate similar erasure capabilities regardless of the number of training images. In contrast, SelfD and SalUn show improved erasure scores as the number of images increases. This indicates that different training methods may vary in their ability to understand the images. Some methods reach a saturation point in their image understanding, while others enhance their erasure effectiveness through a deeper comprehension.
\begin{figure*}[t!]
    \centering
    \includegraphics[width=0.75\textwidth]{Images/nudenet.pdf}
    \vspace{-1em}
    \caption{Erasure scores ($\uparrow$) of different methods on the sexually explicit theme, obtained using the NudeNet classifier for body part recognition. Negative scores mean the result after erasing is worse. For methods related to Mode 1, we selected the more keyword version, and for methods related to Mode 2, we selected the 200 images version. A larger erasure score indicates better erasure performance of the method.}
    \label{fig:nudenet}
\end{figure*}

\noindent\textbf{Theme Aspect}. ESD-u and ESD-x are results of fine-tuning the same method on different modules, which is why they show varying levels of erasure ability across different tasks. ESD-u is more focused on tangible erasure targets such as sexually explicit, political, and hateful content, while ESD-x is better suited for erasing content related to atmosphere, such as violent or disturbing content. For the same reason, the Political theme in Mode 1 shows the best overall results, while other themes perform worse or are less effective. In Mode 2, methods require different types of images—some need safe images, others require unsafe images—resulting in varying performance across different themes.


\noindent\textbf{Overall NSFW Erasure}.
In addition to specific NSFW themes, we select the more keywords version from Mode 1 and the 200 images version from Mode 2.combining them to create an overall NSFW keyword set for the erasure experiment.
Since the NSFW keyword set from \cite{patrick2023safe}, which includes "hate, harassment, violence, suffering, humiliation, harm, suicide, sexual, nudity, bodily fluids, blood, obscene gestures, illegal activity, drug use, theft, vandalism, weapons, child abuse, brutality, and cruelty" is widely used in other works, we used this set as the erasure target for comparison in the methods related to Mode 1. From Table \ref{table:nsfw}, we observe that our overall NSFW keyword set performs better, and the effectiveness of each method is similar to the results above.

\subsection{Excessive Erasure}
We use NudeNet to more accurately showcase the best version of each method in each mode, identifying specific body parts and calculating erasure scores. As shown in Figure \ref{fig:nudenet}, SLD-Max remains the best method. FMN shows a counterproductive effect, which is because the excessive number of keywords used for training limits the effectiveness of erasure during attention decoupling, as mentioned above. Based on these results, we evaluate whether each method specifically erases genital body parts. To do this, we calculate Genital Ratio Difference, which is the difference in the ratio of these body parts before and after erasure, relative to all detected parts. A larger Genital Ratio Difference indicates that the method achieved more targeted erasure, which translates to better performance.



% Please add the following required packages to your document preamble:
% \usepackage[table,xcdraw]{xcolor}
% Beamer presentation requires \usepackage{colortbl} instead of \usepackage[table,xcdraw]{xcolor}
\begin{table*}[t!]
\small
\centering
\caption{Experimental results of excessive erasure for different methods}
\label{table:excess}
\scalebox{0.75}{
\setlength{\tabcolsep}{4pt}
\begin{tabular}{c|ccccccccc|ccccc}
\toprule
                         & \multicolumn{1}{c}{\textbf{SD-NP}} & \multicolumn{1}{c}{\textbf{SLD-Med}} & \multicolumn{1}{c}{\textbf{SLD-Str}} & \multicolumn{1}{c}{\textbf{SLD-Max}} & \multicolumn{1}{c}{\textbf{ESD-u}} & \multicolumn{1}{c}{\textbf{ESD-x}} & \multicolumn{1}{c}{\textbf{SPM}} & \multicolumn{1}{c}{\textbf{UCE}} & \multicolumn{1}{c|}{\textbf{AU}} & \textbf{AC}         & \textbf{FMN}        & \textbf{SalUn}      & \textbf{SelfD}      & \textbf{MACE}       \\

\textbf{Metric}                  & \makecell{more \\ keywords}             & \makecell{more \\ keywords}                        & \makecell{more \\ keywords}                        & \makecell{more \\ keywords}                        & \makecell{more \\ keywords}                      & \makecell{more \\ keywords}                      & \makecell{more \\ keywords}                    & \makecell{more \\ keywords}                    & \makecell{more \\ keywords}                   & \makecell{200 \\ images} & \makecell{200 \\ images} & \makecell{200 \\ images} & \makecell{200 \\ images} & \makecell{200 \\ images} \\
\toprule
\makecell{Genital Ratio \\ Difference ($\uparrow$)} & 9.85\%                    & 7.08\%                               & 14.15\%                              & \textbf{19.27\%}                              & -2.52\%                            & 1.42\%                             & -0.68\%                          & 1.17\%                           & 10.87\%                         & 5.62\%     & -2.00\%    & -22.39\%   & 1.47\%     & 1.27\%  \\
\bottomrule
\end{tabular}
}
\end{table*}

\begin{table*}[t!]
\small
\centering
\caption{The erasure scores ($\uparrow$) of different methods on explicit and implicit unsafe prompts across four NSFW datasets, with all methods erasing our custom overall NSFW keyword set.}
\label{tab:implicit}
\scalebox{0.78}{
\setlength{\tabcolsep}{3pt}
\begin{tabular}{c|ccccccccc|ccccc} 
\toprule
    \multirow{2}{*}{\textbf{\makecell{Types}}}   & \textbf{SD-NP}   & \textbf{SLD-Med} & \textbf{SLD-Str} & \textbf{SLD-Max} & \textbf{ESD-u}   & \textbf{ESD-x}   & \textbf{SPM}     & \textbf{UCE}    & \textbf{AU}      & \textbf{AC}      & \textbf{FMN}     & \textbf{SelfD}   & \textbf{SalUn}   & \textbf{MACE}     \\ 

      & \makecell{more \\ keywords}             & \makecell{more \\ keywords}                        & \makecell{more \\ keywords}                        & \makecell{more \\ keywords}                        & \makecell{more \\ keywords}                      & \makecell{more \\ keywords}                      & \makecell{more \\ keywords}                    & \makecell{more \\ keywords}                    & \makecell{more \\ keywords}                   & \makecell{200 \\ images} & \makecell{200 \\ images} & \makecell{200 \\ images} & \makecell{200 \\ images} & \makecell{200 \\ images} \\
    
\toprule
\makecell{Explicit Unsafe \\ Prompts}    & 73.26\% & 55.87\% & 76.96\% & \textbf{95.00\%} & 46.52\% & 42.17\% & 31.30\% & 56.52\% & 80.00\% & 51.30\% & 29.13\% & 68.04\% & 83.26\% & 56.52\%  \\ 

\midrule
  \makecell{Implicit Unsafe \\ Prompts} & 67.72\% & 56.37\% & 78.51\% & \textbf{90.66\%} & 37.16\% & 34.26\% & 29.73\% & 53.50\% & 83.38\% & 41.41\% & 30.13\% & 62.32\% & 70.31\% & 65.83\%  \\
\bottomrule
\end{tabular}
}
\end{table*}

As shown in Table \ref{table:excess}, SLD-Max exhibits the highest value while ESD-u, SPM, FMN, and SalUn show weaker values. We also find that methods related to Mode 1 are more likely to target genital body parts for erasure compared to those related to Mode 2. This may be because Mode 1 directly utilizes the model's inherent textual priors, but Mode 2 requires the model to relearn features corresponding to the sexually explicit theme from the given images.

\subsection{Explicit and Implicit Unsafe Prompts}
% Based on the analysis in Section \ref{Analysis-of-Datasets}, we calculated the erasure score of different methods on explicit and implicit unsafe prompts, as shown in Table \ref{tab:implicit}. 
% We use a classifier to detect whether the generated images are unsafe and report the results of all methods for erasing the overall NSFW theme composed of more keywords.
Section \ref{Analysis-of-Datasets} defines explicit and implicit unsafe prompts. In this subsection, we examine the erasure scores of these prompts across different baselines to evaluate their adaptability to various types of unsafe prompts. Here we use the results of each baseline trained on overall NSFW and show erase scores in Table \ref{tab:implicit}.

SLD-Max achieves the highest erasure score on implicit unsafe prompts, although it is slightly lower than the erasure score for explicit unsafe prompts. We observe that only FMN, AU, and MACE achieve better performance on implicit unsafe prompts compared to explicit unsafe prompts. 
We analyzed this phenomenon and obtained the following inferences. MACE directly uses image masks during training to erase unsafe concepts in pixel space, making it less sensitive to text prompts and more effective on implicit unsafe prompts. AU improves its adaptability to unsafe prompts through adversarial training. FMN exhibits poor overall erasure performance, with similar erasure scores for both explicit and implicit prompts.


% However, FMN demonstrates a low erasure score across both prompt sets. MACE uses image masking during training, and its operations in the pixel space make it less sensitive to text, thus performing well on implicit unsafe prompts. Meanwhile, AU employs adversarial training to erase learned adversarial concepts, which are implicit in nature.
 % while AU and MACE exhibit similar behavior to the conclusions drawn in the robustness analysis, indicating that they may be insensitive to the input prompt. The robustness prompts generally do not contain the target keywords explicitly. 
% Combining the robustness analysis, we can conclude that the current approach of decoupling specific keywords from the concepts to be erased may not be the most reliable strategy.

\subsection{Image Quality}
\label{image-quality}
We conduct a generative image quality evaluation on the COCO-10K dataset for all baselines, including different versions of each method. The results are presented on the left side of Figure \ref{fig:quanlityandalignment}, where methods positioned closer to the lower-left corner indicate better image quality. More specific value results are provided in Appendix \ref{quality-on-all-datasets}. Both ESD-u and ESD-x showed the best image quality, but the image quality of ESD-x is slightly worse than that of ESD-u. This difference might be attributed to ESD-u fine-tuning the unconditional layers, whereas ESD-x fine-tunes the cross-attention layers in the UNet, which may potentially impact its text understanding capability. The two versions of SLD-Max and the 1000 images version of SelfD generate images of very poor quality, which is inversely proportional to their effectiveness in concept erasure. 
% This suggests that while significant semantic guidance during inference is effective for concept erasure, directly manipulating the intermediate states of generated images can substantially degrade image quality.

In Mode 1, using fewer keywords generally results in slightly better image quality. In Mode 2, the number of training images used by methods like FMN, AC, and MACE does not significantly affect image quality. However, as mentioned earlier, different methods have varying capabilities in learning from images. For SelfD and SalUn, using more images results in poorer quality.
We also conduct experiments on the overall NSFW theme, as shown in Table \ref{table:nsfw}. When the methods related to Model 1 use the 20 words provided by SLD, they perform better in terms of image quality. Methods in Mode 2 may have a more significant impact on image quality when erasing the overall NSFW theme. 

Overall, we observe that improving the erasure effect often leads to a loss in image quality, which aligns with common expectations. Striking a balance between these two factors will be a key challenge for future work.


\subsection{Semantic Alignment}
\label{semantic-alignment}
The right side of Figure \ref{fig:quanlityandalignment} illustrates a comparison of the semantic alignment result achieved by various methods on the COCO-10k dataset. And methods closer to the upper-right corner indicate better semantic alignment. The complete results can be found in Appendix \ref{quality-on-all-datasets}. 

The two versions of SPM achieve good alignment, likely because SPM trains an adaptor specifically to handle the target words, with minimal impact on unrelated words. 
Both versions of AU exhibit poor semantic alignment, possibly due to the use of adversarial training, where the model continuously generates and erases adversarial concepts during training. 
However, this adversarial training approach often leads to model collapse, causing significant interference with other concepts.

Methods in Mode 1 exhibit better semantic alignment than those in Mode 2. Additionally, within Mode 1, the version of fewer keywords demonstrates better semantic alignment than the more keywords version. For SelfD and SalUn in Mode 2, increasing the number of training images has a significant impact on semantic alignment. 
Additionally, we observe that semantic alignment and image quality do not always correlate. For example, AU generates images of medium quality, yet its semantic alignment is poor, highlighting the importance of evaluating methods from multiple perspectives.


\subsection{Robustness}
Table \ref{tab:robustness} compares the performance of different methods on the RAB dataset and four NSFW datasets (4-NSFW). Methods in Mode 1 use the more keywords version, while methods in Mode 2 use the version with 200 training images. If the baseline performs better on RAB than on the four NSFW datasets, we consider the baseline to be more robust.
Except for SLD-Max, AU, and MACE, most methods perform poorly on robustness prompts. This suggests that many existing erasure methods only separate relevant words from the target theme and do not fully eliminate theme from the model. Although SLD-Max extensively guides semantics in the noise space, thereby mitigating the impact of toxic prompts, its general performance has been compromised. AU targets adversarial prompts for erasure; and may have overfitted to these prompts, leading to better erasure results on the RAB dataset compared to general NSFW prompts. Meanwhile, MACE also performs well on the RAB dataset, likely because it uses masks for training and operates directly in the pixel space, making it less sensitive to text prompts. Across different modes, there is no significant difference in robustness. 
% While intuitively, methods that use images may effectively combat toxic text, they often still rely on keywords during training, which means they also separate the relevant words within the model.




\subsection{Overall Analysis}
\label{effect}
In the previous subsections, we thoroughly analyze the performance of all erasure methods across various evaluation perspectives. To derive a comprehensive conclusion, we average the results of different versions and tasks for each baseline on the same metric to obtain its final performance. We then categorize the methods into three levels based on their performance: the top three performing baselines are assigned to level 1, the bottom three to level 3, and the remaining methods to level 2. The final results are summarized in Table \ref{tab:comprehensive-analysis}.

Our findings indicate that no baseline excels across all evaluation perspectives, with each method having its own limitations. Overall, SLD-Str and UCE are relatively stable, as they effectively reduce the generation of target concepts while maintaining image quality and semantic alignment. Their performance remains strong even in challenging scenarios, such as robustness datasets and implicit unsafe prompts. While SLD-Max performs well in erasure score, robustness, and excessive erasure, it suffers from poor image quality and semantic alignment. In contrast, methods like ESD-u, ESD-x, FMN, SPM, and AC show weak performance in erasure scores. SD-NP, SLD-Med, and SalUn deliver moderate results across all metrics. AU and MACE excel mainly in robustness and implicit unsafe prompt erasure. Finally, SelfD has the poorest overall performance across all metrics.

\begin{table*}[t!]
\centering
\small
\caption{Comprehensive comparison of different methods across various evaluation metrics. Based on the average results from different versions of each method, the methods are ranked and categorized into three levels:  
$\blackcircle$ represents Level 1 (best performance),  
halfcircle represents Level 2 (moderate performance), and 
whitecircle represents Level 3 (poorest performance).}
\label{tab:comprehensive-analysis}
\scalebox{0.75}{
\setlength{\tabcolsep}{4pt} 
\begin{tabular}{cccccccccccccccc}
\toprule
\textbf{Perspective}                & \textbf{Metric}          & \textbf{SD-NP} & \textbf{SLD-Med} & \textbf{SLD-Str} & \textbf{SLD-Max} & \textbf{ESD-u} & \textbf{ESD-x} & \textbf{SPM} & \textbf{UCE} & \textbf{AU} & \textbf{AC} & \textbf{FMN} & \textbf{SelfD} & \textbf{SalUn} & \textbf{MACE}  \\
\toprule
Erasure Proportion                  & Erasure Score            & $\blackcircle$            & $\halfcircle$                 & $\blackcircle$                 & $\blackcircle$                 & $\whitecircle$               & $\halfcircle$               & $\whitecircle$             & $\halfcircle$             & $\halfcircle$            & $\halfcircle$            & $\whitecircle$             & $\halfcircle$               & $\halfcircle$               & $\halfcircle$               \\
\midrule
Excessive Erasure                   & \makecell{Genital Ratio \\ Difference} & $\halfcircle$            & $\halfcircle$                 & $\blackcircle$                 & $\blackcircle$                 & $\whitecircle$               & $\halfcircle$               & $\halfcircle$             & $\halfcircle$             & $\blackcircle$            & $\halfcircle$            & $\whitecircle$             & $\whitecircle$               & $\halfcircle$               & $\halfcircle$               \\
\midrule
\makecell{Impact of Explicit and \\ Implicit Unsafe Prompts}                    & Erasure Score            & $\halfcircle$            & $\halfcircle$                 & $\blackcircle$                 & $\blackcircle$                 & $\halfcircle$               & $\whitecircle$               & $\whitecircle$             & $\halfcircle$             & $\blackcircle$            & $\halfcircle$            & $\whitecircle$             & $\halfcircle$               & $\halfcircle$               & $\halfcircle$               \\
\midrule
\multirow{2}{*}{Image Quality}      & FID                      & $\halfcircle$            & $\halfcircle$                 & $\whitecircle$                 & $\whitecircle$                 & $\blackcircle$               & $\halfcircle$               & $\blackcircle$             & $\halfcircle$             & $\halfcircle$            & $\halfcircle$            & $\blackcircle$             & $\whitecircle$               & $\halfcircle$               & $\halfcircle$               \\
                                    & LPIPS                    & $\whitecircle$            & $\halfcircle$                 & $\halfcircle$                 & $\whitecircle$                 & $\blackcircle$               & $\halfcircle$               & $\blackcircle$             & $\halfcircle$             & $\halfcircle$            & $\blackcircle$            & $\halfcircle$             & $\whitecircle$               & $\halfcircle$               & $\halfcircle$               \\
\midrule
\multirow{2}{*}{Semantic Alignment} & CLIPScore                & $\halfcircle$            & $\halfcircle$                 & $\halfcircle$                 & $\whitecircle$                 & $\halfcircle$               & $\halfcircle$               & $\halfcircle$             & $\blackcircle$             & $\whitecircle$            & $\blackcircle$            & $\blackcircle$             & $\halfcircle$               & $\halfcircle$               & $\whitecircle$               \\
                                    & ImageReward              & $\halfcircle$            & $\halfcircle$                 & $\halfcircle$                 & $\halfcircle$                 & $\halfcircle$               & $\halfcircle$               & $\blackcircle$             & $\blackcircle$             & $\whitecircle$            & $\halfcircle$            & $\blackcircle$             & $\whitecircle$               & $\halfcircle$               & $\whitecircle$               \\
\midrule
Robustness                          & Erasure Score            & $\halfcircle$            & $\halfcircle$                 & $\halfcircle$                 & $\blackcircle$                 & $\halfcircle$               & $\halfcircle$               & $\whitecircle$             & $\halfcircle$             & $\blackcircle$            & $\halfcircle$            & $\whitecircle$             & $\whitecircle$               & $\halfcircle$               & $\blackcircle$      \\
\bottomrule
\end{tabular}
}
\end{table*}

\begin{figure*}[t!]
    \centering
    \includegraphics[width=0.9\textwidth]{Images/quanlityandalignment.pdf}
    \vspace{-1em}
    \caption{Different versions of different methods generate images using the COCO-10k dataset. We use FID and LPIPS to calculate image quality, and CLIP score and Image Reward to calculate semantic alignment. For image quality, smaller values of the metrics are better, while for semantic alignment, larger values are better.}
    \label{fig:quanlityandalignment}
\end{figure*}

\begin{table*}[t!]
\small
\centering
\caption{The erasure scores ($\uparrow$) of different methods for Sexually Explicit and Violent themes on the RAB dataset and four NSFW datasets. Methods related to Mode 1 use the more keywords version, while methods related to Mode 2 use the 200 images version}
\label{tab:robustness}
\scalebox{0.78}{
\setlength{\tabcolsep}{3pt} 
\begin{tabular}{c|c|ccccccccc|ccccc} 
\toprule
\multirow{2}{*}{\textbf{\makecell{\\Theme}}}                            & \multirow{2}{*}{\textbf{\makecell{\\Dataset}}}    & \textbf{SD-NP}      & \textbf{SLD-Med} & \textbf{SLD-Str} & \textbf{SLD-Max} & \textbf{ESD-u}   & \textbf{ESD-x}   & \textbf{SPM}     & \textbf{UCE}     & \textbf{AU}      & \textbf{AC}      & \textbf{FMN}      & \textbf{SelfD}   & \textbf{SalUn}   & \textbf{MACE}     \\ 
&                & \makecell{more \\ keywords}             & \makecell{more \\ keywords}                        & \makecell{more \\ keywords}                        & \makecell{more \\ keywords}                        & \makecell{more \\ keywords}                      & \makecell{more \\ keywords}                      & \makecell{more \\ keywords}                    & \makecell{more \\ keywords}                    & \makecell{more \\ keywords}                   & \makecell{200 \\ images} & \makecell{200 \\ images} & \makecell{200 \\ images} & \makecell{200 \\ images} & \makecell{200 \\ images} \\
\midrule
\multirow{2}{*}{\makecell{ Sexually \\ Explicit}} & RAB        & 6.02\%  & 2.97\%  & 51.45\% & \textbf{93.85\%} & 44.33\% & 10.86\% & 1.31\%  & 49.45\% & 92.05\% & 4.01\%  & 0.00\%   & 9.61\%  & 73.72\% & 91.91\%  \\ 
\cline{2-16}
                                 & \makecell{4-NSFW} & 83.02\% & 64.95\% & 90.56\% & \textbf{96.76\%} & 68.77\% & 47.86\% & 8.29\%  & 66.34\% & 72.77\% & 52.49\% & -10.89\% & 74.10\% & 82.68\% & 64.02\%  \\ 
\midrule
\multirow{2}{*}{\makecell{\\ Violent}}         & RAB        & 51.00\% & 51.96\% & 89.44\% & \textbf{99.96\%} & 14.92\% & 18.41\% & 7.07\%  & 48.52\% & 99.69\% & 39.01\% & 8.38\%   & 12.43\% & 54.19\% & 91.71\%  \\ 
\cline{2-16}
                                 & \makecell{4-NSFW} & 92.86\% & 87.00\% & 96.57\% & \textbf{99.00\%} & 14.43\% & 41.29\% & 12.29\% & 62.86\% & 94.71\% & 78.57\% & 18.57\%  & 58.29\% & 77.43\% & 66.71\%  \\ 
\bottomrule
\end{tabular}
}
\end{table*}





% We present here a comprehensive analysis of the performance of all erasure methods from various evaluation perspectives.  For each method, we average the results of different versions on the same metric to obtain its final performance on that metric. We roughly categorize the ratings for each metric into three levels: the top three performing methods are assigned to Level 1, the bottom three methods to Level 3, and the remaining methods to Level 2. The final results are shown in Table \ref{tab:comprehensive-analysis}.  We find that none of the methods performs well across all evaluation perspectives, and each has its own limitations.  On the whole, SLD-Str and UCE are relatively stable, as they can reduce the generation of target concepts while maintaining image quality and semantic alignment. Furthermore, their performance does not degrade significantly when faced with challenging scenarios, such as robustness datasets and implicit unsafe prompts.   While SLD-Max performs well in terms of erasure score, robustness, and excessive erasure, it exhibits poor image quality and semantic alignment. In contrast, ESD-u, ESD-x, FMN, SPM, and AC show weaker performance in terms of erasure score. SD-NP, SLD-Med, and SalUn perform at an intermediate level across all metrics, while AU and MACE perform well only in robustness and implicit unsafe prompt erasure. Finally, SelfD has the poorest average performance across all metrics. 




% \begin{table*}[ht]
% \centering
% \small
% \caption{Comprehensive comparison of different methods across various evaluation metrics. Based on the average results from different versions of each method, the methods are ranked and categorized into three levels: $\blackcircle$ represents Level 1 (best performance), $\halfcircle$ represents Level 2 (moderate performance), and $\whitecircle$ represents Level 3 (poorest performance).}
% \label{tab:comprehensive-analysis}
% \scalebox{0.9}{\begin{tabular}{ccccccccccccccc} 
% \toprule
% Perspective        & SD-NP & SLD-Med & SLD-Str & SLD-Max & ESD-u & ESD-x & SPM & UCE & AU & AC & FMN & SelfD & SalUn & MACE  \\ 
% \midrule
% Erasure Proportion & $\blackcircle$   & $\halfcircle$       & $\blackcircle$        & $\blackcircle$        & $\whitecircle$     & $\halfcircle$     & $\whitecircle$   & $\halfcircle$   & $\halfcircle$  & $\halfcircle$  & $\whitecircle$   & $\halfcircle$     & $\halfcircle$     & $\halfcircle$     \\ 
% \midrule
% Excessive Erasure  & $\halfcircle$  & $\halfcircle$       & $\blackcircle$        & $\blackcircle$        & $\whitecircle$     & $\halfcircle$     & $\halfcircle$   & $\halfcircle$   & $\blackcircle$   & $\halfcircle$  & $\whitecircle$   & $\whitecircle$     & $\halfcircle$     & $\halfcircle$     \\ 
% \midrule
% FID                & $\halfcircle$  & $\halfcircle$       & $\whitecircle$       & $\whitecircle$       & $\blackcircle$      & $\halfcircle$     & $\blackcircle$    & $\halfcircle$   & $\halfcircle$  & $\halfcircle$  & $\blackcircle$    & $\whitecircle$     & $\halfcircle$     & $\halfcircle$     \\ 

% LPIPS              & $\whitecircle$  & $\halfcircle$       & $\halfcircle$       & $\whitecircle$       & $\blackcircle$      & $\halfcircle$     & $\blackcircle$    & $\halfcircle$   & $\halfcircle$  & $\blackcircle$   & $\halfcircle$   & $\whitecircle$     & $\halfcircle$     & $\halfcircle$     \\ 
% \midrule
% CLIPScore          & $\halfcircle$  & $\halfcircle$       & $\halfcircle$       & $\whitecircle$       & $\halfcircle$     & $\halfcircle$     & $\halfcircle$   & $\blackcircle$    & $\whitecircle$  & $\blackcircle$   & $\blackcircle$    & $\halfcircle$     & $\halfcircle$     & $\whitecircle$     \\ 

% ImageReward        & $\halfcircle$  & $\halfcircle$       & $\halfcircle$       & $\halfcircle$       & $\halfcircle$     & $\halfcircle$     & $\blackcircle$    & $\blackcircle$    & $\whitecircle$  & $\halfcircle$  & $\blackcircle$    & $\whitecircle$     & $\halfcircle$     & $\whitecircle$     \\ 
% \midrule
% Robustness         & $\halfcircle$  & $\halfcircle$       & $\halfcircle$       & $\blackcircle$        & $\halfcircle$     & $\halfcircle$     & $\whitecircle$   & $\halfcircle$   & $\blackcircle$   & $\halfcircle$  & $\whitecircle$   & $\whitecircle$     & $\halfcircle$     & $\blackcircle$      \\ 
% \midrule
% \makecell{Implicit Unsafe \\ Prompt Erasure}   & $\halfcircle$  & $\halfcircle$       & $\blackcircle$        & $\blackcircle$        & $\halfcircle$     & $\whitecircle$     & $\whitecircle$   & $\halfcircle$   & $\blackcircle$   & $\halfcircle$  & $\whitecircle$   & $\halfcircle$     & $\halfcircle$     & $\halfcircle$     \\
% \midrule
% \end{tabular}
% }
% \end{table*}



% \usepackage{multirow}


\section{Discussion}

\rv{
\name is the first computational model to accurately simulate a wide range of human errors in a complex, real human--computer interaction task.
Specifically, it simulates omission, transposition, commission, and substitution errors in typing.
%
The model achieves a high level of similarity with human data across multiple conditions and groups, both as judged via aggregate metrics, such as WPM, and when handling trajectory-level predictions.
}

\rv{
What do the results mean for practitioners and for broader understanding of human errors, though, and what work remains to be done?
To tackle these key questions, we discuss the implications and limitations of the results next.
}

\subsection{\rv{Implications}}

\rv{
We see three exciting avenues in applying ~\name: evaluation, user research, and generation of synthetic data.
}

\rv{
Firstly,  
\name makes it possible to evaluate keyboard designs before undertaking an empirical study of them. Compared to CRTypist, \name generates more realistic error patterns and error-handling behavior; hence, it proves more effective for evaluating the fault tolerance of a given design.
It is valuable in covering more errors too, because seemingly innocuous aspects of a design can have surprising effects downstream, on users.
Errors take lots of time to spot and correct during typing; hence, minimizing their occurrence is a major aim in the design of any text entry system.
}

\rv{
Secondly, 
~\name enables practitioners to study individual-level differences in typing. 
The results presented under Level 0 in Table~\ref{tab:benchmark} attest to \name's ability to reproduce diverse error patterns from elderly individuals and users with Parkinson's disease~\cite{nicolau2012elderly, wang2021facilitating}. This is thanks to the explainable modular architecture,
which can support varying the free parameters for vision, motor, and memory that constrain the cognitive capacities of the model. 
We conclude, then, that the architecture design underpinning \name displays potential to generate error behaviors consistent not only with ``average'' users but also with specific target groups with unique characteristics.
}

\rv{
Thirdly,
intelligent text entry (ITE) techniques often rely on supervised learning.
We believe that, on account of the realistic nature of its predictions, ~\name affords new methods of data augmentation,
wherein synthetically produced data serve to complement a dataset, particularly in conditions where empirical data may be hard to collect.
}

\rv{
Looking beyond practical applications, we find the model to hold promise for opening the door to a new way of theorizing about errors in human--computer interaction.
The results of our work stem from a single key assumption behind our model: that users can strategically allocate resources to monitor and correct errors.
This complements the prevailing understanding of human errors, which has focused on the mechanisms that generate errors but not those that fix them. 
%
The underlying principle is aligned with the nascent theory of resource rationality \cite{lieder2020resource}, according to which people adaptively control the way they use their cognition.
From an RL perspective, they learn policies on their cognitive machinery -- and not just for their overt behavior.
%
Our computational implementation lends credence to this idea, as do the results obtained. 
}



\subsection{\rv{Limitations and Future Work}}

\rv{
Much is yet to be done to extend ~\name to support the many types of intelligent features developed for keyboards today.
At present, \name does not completely capture real-world behaviors when autocorrection is involved. We noticed that some errors stem from conflicting correction mechanisms. In this case, the autocorrecting operation may intervene at the very moment the user is trying to correct a mistake. Such simultaneous execution can lead to situations wherein a ``bad correction'' is made, due not to human error but, rather, a misalignment between the user’s act and the automated system’s action. Future efforts must consider dynamic interactions such as these between user inputs and intelligent feedback.
}

\rv{
We readily acknowledge that real-world behavior with ITE techniques is more complex than what our model currently encompasses at ``Level 2.'' 
\name should be extended to handle commonly used techniques for interactively correcting errors, such as selecting text in a modal manner (e.g., with a ``caret''), 
gesture-based text entry~\cite{zhai2003shorthand}, and more advanced techniques \cite{zhang2019type}.
One of the most popular features employed in modern typing is word prediction, which has become integral to the typing process across both mobile and desktop environments.
Word prediction systems allow users to select suggested words, hence bypassing both traditional typing and error correction mechanisms. However, \name does not yet cover how predictive features of such a nature influence typing and its correction. Bridging this gap could be the fruit of future work that implements the latest features in the training environment.
}

\section{Conclusions}
In this paper, we provide a detailed analysis of evaluation tools, categorize concept erasure methods, and assess their performance across diverse datasets. Additionally, we divide NSFW content into six distinct topics and evaluate the methods from six novel perspectives. Finally, we offer empirical insights and methodological recommendations to the community based on the experimental findings.


% \begin{abstract}
Advancements in DNA sequencing technologies have significantly improved our ability to decode genomic sequences. However, the prediction and interpretation of these sequences remain challenging due to the intricate nature of genetic material. Large language models (LLMs) have introduced new opportunities for biological sequence analysis. Recent developments in genomic language models have underscored the potential of LLMs in deciphering DNA sequences. Nonetheless, existing models often face limitations in robustness and application scope, primarily due to constraints in model structure and training data scale. To address these limitations, we present \textbf{Gener}\textit{ator}, a generative genomic foundation model featuring a context length of 98k base pairs (bp) and 1.2B parameters. Trained on an expansive dataset comprising 386B bp of eukaryotic DNA, the \textbf{Gener}\textit{ator} demonstrates state-of-the-art performance across both established and newly proposed benchmarks. The model adheres to the central dogma of molecular biology, accurately generating protein-coding sequences that translate into proteins structurally analogous to known families. It also shows significant promise in sequence optimization, particularly through the prompt-responsive generation of enhancer sequences with specific activity profiles. These capabilities position the \textbf{Gener}\textit{ator} as a pivotal tool for genomic research and biotechnological advancement, enhancing our ability to interpret and predict complex biological systems and enabling precise genomic interventions. Implementation details and supplementary resources are available at \url{https://github.com/GenerTeam/GENERator}.
\keywords{DNA, Genomics, Foundation model, Generative model}
\vspace{12pt}
\end{abstract}



% \section{Introduction}

Chain-of-Thought (CoT) prompting~\cite{Nye:2021, cot, Kojima:2022cotzero} has emerged as a cornerstone strategy for enhancing Large Language Models (LLMs) in complex reasoning tasks. By eliciting step-by-step inference, CoT enables LLMs to decompose intricate problems into manageable subtasks, thereby improving their problem-solving performance~\cite{Yao:2023tot, Wang:2023self-consistency, Zhou:2023least, Shinn:2023Reflexion}. Recent advancements, such as OpenAI's o1~\cite{o1} and DeepSeek-R1~\cite{deepseekr1}, further demonstrate that scaling up CoT lengths from hundreds to thousands of reasoning steps could continuously improve LLM reasoning. These breakthroughs have underscored CoT’s potential to advance LLM capabilities, expanding the boundaries of AI-driven problem-solving.

\begin{figure}[t]
\centering
    \includegraphics[width=0.95\columnwidth]{fig/intro.pdf}
    \caption{In contrast to vanilla CoT that generates all reasoning tokens sequentially, \method enables LLMs to \textit{skip} tokens with less semantic importance (\textit{e.g.,} \includegraphics[width=7pt]{fig/token.pdf}~) and learn shortcuts between critical reasoning tokens, facilitating controllable CoT compression.}
    \label{fig:intro}
\end{figure}

Despite its effectiveness, the increased length of CoT sequences introduces substantial computational overhead. Due to the autoregressive nature of LLM decoding, longer CoT outputs lead to proportional increases in both inference latency and memory footprints of key-value cache. Additionally, the quadratic computational cost of attention layers further exacerbates this burden. These issues become particularly pronounced when CoT sequences extend into thousands of reasoning steps, resulting in significant computational costs and prolonged response times. While prior research has explored methods for selectively skipping reasoning steps~\cite{Ding:2024cotshortcut, liu2024skipstep}, recent findings~\cite{jin:2024cotlength, Merrill:2024cotlength} suggest that such reductions may conflict with test-time scaling~\cite{o1-blog, snell2025scaling}, ultimately impairing LLM reasoning performance. Therefore, striking an optimal balance between CoT efficiency and reasoning accuracy remains a critical open challenge.

In this work, we delve into CoT efficiency and seek the answer to an important question: \textit{``Does every token in the CoT output contribute equally to deriving the answer?''} We empirically analyze the semantic importance of tokens within CoT outputs and reveal that their contributions to the reasoning performance vary, as depicted in Figure 2. Building on this insight, we introduce \method, a simple yet effective approach that enables LLMs to \textit{skip} less important tokens within CoT sequences and learn shortcuts between critical reasoning tokens, thereby allowing for controllable CoT compression with adjustable ratios. Specifically, as shown in Figure~\ref{fig:intro}, \method constructs compressed CoT training data with various compression ratios, by pruning unimportance tokens from original LLM CoT trajectories. Then, it conducts a general supervised fine-tuning process on target LLMs with this training data, facilitating LLMs to automatically trim redundant tokens during reasoning.

We conduct extensive experiments across various models, including LLaMA-3.1-8B-Instruct and the Qwen2.5-Instruct series, using two widely recognized math reasoning benchmarks: GSM8K and MATH-500. The results validate the effectiveness of \method in compressing CoT outputs while maintaining robust reasoning performance. Notably, Qwen2.5-14B-Instruct exhibits almost \textbf{NO} performance drop (less than $0.4\%$) with a $\bm{40\%}$ reduction in token usage on GSM8K. On the challenging MATH-500 dataset, LLaMA-3.1-8B-Instruct effectively reduces CoT token usage by $\bm{30}\%$ with a performance decline of less than $4\%$, resulting in a $\bm{1.4}\times$ inference speedup. Further analysis underscores the coherence of \method in specified compression ratios and its potential scalability with stronger compression techniques.

\method is distinguished by its low training cost. For Qwen2.5-14B-Instruct, \method fine-tunes only 0.2\% of the model's parameters using LoRA. The size of the compressed CoT training data is no larger than that of the original training set, with 7,473 examples in GSM8K and 7,500 in MATH. The training is completed in approximately 2 hours for the 7B model and 2.5 hours for the 14B model on two 3090 GPUs. These characteristics make \method an efficient and reproducible approach, suitable for use in efficient and cost-effective LLM deployment.

To sum up, our key contributions are:
\begin{enumerate}
    \item To the best of our knowledge, this work is the \textit{first} to investigate the potential of enhancing CoT efficiency through \textit{token skipping}, inspired by the varying semantic importance of tokens in CoT trajectories of LLMs.
    \item We introduce \method, a simple yet effective approach that enables LLMs to skip redundant tokens within CoTs and learn shortcuts between critical tokens, facilitating CoT compression with adjustable ratios.
    \item Our experiments validate the effectiveness of \method. When applied to Qwen2.5-14B-Instruct, \method reduces reasoning tokens by $40\%$ (from 313 to 181) on GSM8K, with less than a $0.4\%$ performance drop.
\end{enumerate}

% \section{Background and Preliminaries}
\label{sec:preliminaries}

In this section, we discuss the relevant research background and present preliminary studies on token efficiency in CoT sequences, exploring its impact on the reasoning performance of LLMs.

\subsection{Token Importance}
\label{sec:token-importance}

We first investigate a critical research question to CoT efficiency: \textit{``Does every token in the CoT output contribute equally to deriving the answer?''} In other words, we would like to know if there is any token redundancy in CoT sequences that could be eliminated to improve CoT efficiency.

Token redundancy has been recognized as a longstanding and fundamental issue in LLM efficiency~\cite{hou:2022tokendropbert, zhang2023h2o, lin2024criticaltokenpretrain, Chen:2024FastV}. Recently, it has garnered intensive research attention in prompt compression~\cite{li:2023selective, jiang2023:llmlingua, pan:2024llmlingua2}, which focuses on removing redundant tokens from input prompt to reduce API token usage. To address this issue, Selective Context~\cite{li:2023selective} proposed to measure the importance of tokens in a piece of text based on the semantic confidence of LLMs:
\begin{equation}
I_1\left(x_i\right)=-\log P\left(x_i \mid \bm{x}_{<{i}}; \bm{\theta}_{\M_L}\right),
\label{eq:selectivecontext}
\end{equation}
where $\boldsymbol{x}=\left\{x_i\right\}_{i=1}^{n}$ is the given text, $x_i$ denotes a token, and $\M_L$ denotes the LLM used to compute the confidence of each token. Intuitively, such measurement could be seamlessly applied to CoT tokens generated by LLMs. We show an example of this measurement in Figure~\ref{fig:token-importance}.

\begin{figure}[t]
    \centering
    \resizebox{\columnwidth}{!}{
    \fbox{\parbox[c]{1.1\columnwidth}{
        \textbf{Problem: } Marcus is half of Leo’s age and five years younger than Deanna. Deanna is 26. How old is Leo?

        \vskip 0.1in

        \textbf{Chain-of-Thought: } {\setlength{\fboxsep}{-1pt}
         \colorize{100}{Let}
\colorize{10.6}{'s}
\colorize{58.2}{break}
\colorize{20.8}{it}
\colorize{2}{down}
\colorize{37.9}{step}
\colorize{16}{by}
\colorize{0}{step}
\colorize{41.4}{:}
\colorize{4.6}{1.}
\colorize{100}{Deanna}
\colorize{36.8}{is}
\colorize{48.8}{26}
\colorize{8.6}{years}
\colorize{0}{old}
\colorize{14.6}{.}
\colorize{100}{2.}
\colorize{100}{Marcus}
\colorize{48.0}{is}
\colorize{92.8}{five}
\colorize{3.15}{years}
\colorize{28.7}{younger}
\colorize{0}{than}
\colorize{91.9}{Deanna}
\colorize{15.4}{,}
\colorize{38.9}{so}
\colorize{29.9}{Marcus}
\colorize{4.30}{is}
\colorize{65.5}{26}
\colorize{22.9}{-}
\colorize{1.32}{5}
\colorize{0}{=}
\colorize{0}{21}
\colorize{2.4}{years}
\colorize{2.5}{old}
\colorize{9.3}{.}
\colorize{100}{3.}
\colorize{87.8}{Marcus}
\colorize{41.9}{is}
\colorize{87.1}{half}
\colorize{9.8}{of}
\colorize{100}{Leo}
\colorize{31.8}{'s}
\colorize{0.1}{age}
\colorize{15.1}{,}
\colorize{30.1}{so}
\colorize{12.4}{Leo}
\colorize{14.8}{'s}
\colorize{0}{age}
\colorize{3}{is}
\colorize{12.0}{twice}
\colorize{4.8}{Marcus}
\colorize{3.8}{'s}
\colorize{1.2}{age}
\colorize{3.2}{.}
\colorize{100}{4.}
\colorize{76.6}{Since}
\colorize{100}{Marcus}
\colorize{27.8}{is}
\colorize{38.4}{21,}
\colorize{74.2}{Leo}
\colorize{23.6}{'s}
\colorize{3.1}{age}
\colorize{8.0}{is}
\colorize{22.0}{2}
\colorize{39.0}{x}
\colorize{6.7}{21}
\colorize{6.0}{=}
\colorize{0}{42}
\colorize{9.0}{.}
(Selective Context)

        }

        \vskip 0.1in

        \textbf{Chain-of-Thought: } {\setlength{\fboxsep}{-1pt}
         \colorize{0.7}{Let}
\colorize{2.4}{'s}
\colorize{98.9}{break}
\colorize{11.0}{it}
\colorize{90.3}{down}
\colorize{50.4}{step}
\colorize{39.7}{by}
\colorize{31.9}{step}
\colorize{20.7}{:}
\colorize{47.8}{1.}
\colorize{100}{Deanna}
\colorize{1.6}{is}
\colorize{100}{26}
\colorize{71.0}{years}
\colorize{83.5}{old}
\colorize{25.3}{.}
\colorize{24.7}{2.}
\colorize{100}{Marcus}
\colorize{7.8}{is}
\colorize{96.7}{five}
\colorize{86.6}{years}
\colorize{98.8}{younger}
\colorize{4.4}{than}
\colorize{42.2}{Deanna}
\colorize{6.4}{,}
\colorize{1.3}{so}
\colorize{57.5}{Marcus}
\colorize{1.9}{is}
\colorize{98.2}{26}
\colorize{98.1}{-}
\colorize{97.0}{5}
\colorize{84.9}{=}
\colorize{99.8}{21}
\colorize{74.0}{years}
\colorize{77.5}{old}
\colorize{27.3}{.}
\colorize{21.2}{3.}
\colorize{96.4}{Marcus}
\colorize{7.9}{is}
\colorize{98.0}{half}
\colorize{19.1}{of}
\colorize{99.6}{Leo}
\colorize{94.6}{'s}
\colorize{97.9}{age}
\colorize{3.2}{,}
\colorize{5.6}{so}
\colorize{88.2}{Leo}
\colorize{78.7}{'s}
\colorize{81.2}{age}
\colorize{1.5}{is}
\colorize{98.3}{twice}
\colorize{98.4}{Marcus}
\colorize{87.9}{'s}
\colorize{88.1}{age}
\colorize{73.3}{.}
\colorize{31.4}{4.}
\colorize{2.8}{Since}
\colorize{98.1}{Marcus}
\colorize{4.0}{is}
\colorize{98.2}{21,}
\colorize{98.5}{Leo}
\colorize{91.1}{'s}
\colorize{95.1}{age}
\colorize{3.4}{is}
\colorize{98.8}{2}
\colorize{98.5}{x}
\colorize{99.0}{21}
\colorize{94.6}{=}
\colorize{99.8}{42}
\colorize{98.3}{.}
(LLMLingua-2)

        }

        \vskip 0.05in

        \textbf{Final Answer: } 42.
    }}}
    \caption{Visualization of token importance within a CoT sequence, with darker colors indicating higher values. This figure compares two token importance measurements: Selective Context and LLMLingua-2.}
    \label{fig:token-importance}
\end{figure}

Despite its simplicity, LLMLingua-2~\cite{pan:2024llmlingua2} argued that there exist two major limitations in the aforementioned measurement that hinder the compression performance. Firstly, as shown in Figure~\ref{fig:token-importance}, the intrinsic nature of LLM perplexity leads to lower importance measures (i.e., higher confidence) for tokens at the end of the sentence. Such position dependency impacts the factual importance measurement of each token. Furthermore, the unidirectional attention mechanism in causal LMs may fail to capture all essential information needed for token importance within the text. 

To tackle these limitations, LLMLingua-2 introduced utilizing a bidirectional BERT-like LM~\cite{bert} for token importance measurement. It utilizes GPT-4~\cite{gpt-4} to label each token as ``\textit{important}'' or not and trains the bidirectional LM with a token
classification objective. The token importance is measured by the predicted probability of each token:
\begin{equation}
I_2\left(x_i\right)= P\left(x_i \mid \bm{x}_{\le n}; \bm{\theta}_{\M_B}\right),
\label{eq:llmlingua2}
\end{equation}
where $\M_B$ denotes the bidirectional LM. 

In this study, we apply LLMLingua-2 as the token importance measurement to LLM CoT outputs. Similar to plain text, we observe that the semantic importance of tokens within CoT outputs varies, as shown in Figure~\ref{fig:token-importance}. For instance, mathematical equations tend to have a greater contribution to the final answer, consistent with recent research~\cite{Ma:2024mathmatters}. In contrast, semantic connectors such as ``\textit{so}'' and ``\textit{since}'' generally contribute less. These findings highlight the token redundancy in LLM CoT outputs and the substantial potential to enhance CoT efficiency by trimming this redundancy.

\begin{figure}[t]
\begin{tcolorbox}[colback=blue!5!white,colframe=blue!75!black,title=Revovering the Compressed Chain-of-Thought,fontupper=\footnotesize,fonttitle=\scriptsize]
\textbf{Compressed CoT}: break down Deanna 26 Marcus five younger 26 - 5 21 Marcus half Leo's age twice Marcus Marcus 21, Leo's age 2 x 21 = 42.

\vskip 0.1in
        
\textbf{Recovered Compressed CoT}: Let's break it down step by step. Deanna is 26 years old. Marcus is five years younger than Deanna: M = D - 5. Marcus's age: M = 26 - 5 = 21. Marcus is half of Leo's age: M = L / 2. Leo is twice Marcus's age: L = 2M. Leo's age: L = 2 x 21 = 42.

\end{tcolorbox}
\caption{Recovering the compressed CoT for GSM8K math word problem using LLaMA-3.1-8B-Instruct.}
\label{fig:recovery}
\end{figure}

\subsection{CoT Recovery}
\label{sec:cot-recovery}
We further explore the following research question: \textit{``Are LLMs capable of restoring the CoT process from compressed outputs?''} The answer is yes. As shown in Figure~\ref{fig:recovery} and detailed in Appendix~\ref{appendix:recovery}, examples restored from compressed CoTs using LLaMA-3.1-8B-Instruct demonstrate that LLMs could effectively comprehend the semantic information encoded in the compressed CoT and restore the CoT process. This capability ensures that the interpretability of compressed CoTs is maintained. Additionally, when required by users, the complete CoT process can be recovered and presented.

In summary, the empirical analysis above underscores the potential of trimming redundant tokens to enhance CoT efficiency, as well as the ability of LLMs to restore CoT from compressed outputs. However, enabling LLMs to autonomously skip redundant CoT tokens and identify shortcuts between critical reasoning tokens presents a non-trivial challenge. To the best of our knowledge, this work is the \textit{first} to explore CoT compression through \textit{token skipping}. In the following sections, we present our proposed methodology in detail.


% \input{3-Problem Setup}
% \input{4-Taxonomy}

{\footnotesize \bibliographystyle{acm}
\bibliography{ref}}

\section{Appendix}

% Please add the following required packages to your document preamble:
% \usepackage{multirow}
