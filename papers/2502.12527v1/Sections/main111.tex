% TEMPLATE for Usenix papers, specifically to meet requirements of
%  USENIX '05
% originally a template for producing IEEE-format articles using LaTeX.
%   written by Matthew Ward, CS Department, Worcester Polytechnic Institute.
% adapted by David Beazley for his excellent SWIG paper in Proceedings,
%   Tcl 96
% turned into a smartass generic template by De Clarke, with thanks to
%   both the above pioneers
% use at your own risk.  Complaints to /dev/null.
% make it two column with no page numbering, default is 10 point

% Munged by Fred Douglis <douglis@research.att.com> 10/97 to separate
% the .sty file from the LaTeX source template, so that people can
% more easily include the .sty file into an existing document.  Also
% changed to more closely follow the style guidelines as represented
% by the Word sample file. 

% Note that since 2010, USENIX does not require endnotes. If you want
% foot of page notes, don't include the endnotes package in the 
% usepackage command, below.

\documentclass[letterpaper,twocolumn,10pt]{article}
\usepackage{usenix,epsfig,endnotes}
\usepackage{cite}
\usepackage{enumitem}
\begin{document}

%don't want date printed
\date{}

%make title bold and 14 pt font (Latex default is non-bold, 16 pt)
\title{\Large \bf Comprehensive Assessment and Analysis for NSFW Content Erasure in the Text-to-Image Diffusion model}

\author{
{\rm Anonymous authors}\\
% Your Institution
% \and
% {\rm Second Name}\\
% Second Institution
}

\maketitle
\thispagestyle{empty}

\begin{abstract}
Recent advancements in 3D multi-object tracking (3D MOT) have predominantly relied on tracking-by-detection pipelines. However, these approaches often neglect potential enhancements in 3D detection processes, leading to high false positives (FP), missed detections (FN), and identity switches (IDS), particularly in challenging scenarios such as crowded scenes, small-object configurations, and adverse weather conditions. Furthermore, limitations in data preprocessing, association mechanisms, motion modeling, and life-cycle management hinder overall tracking robustness. To address these issues, we present \textbf{Easy-Poly}, a real-time, filter-based 3D MOT framework for multiple object categories. Our contributions include: (1) An \textit{Augmented Proposal Generator} utilizing multi-modal data augmentation and refined SpConv operations, significantly improving mAP and NDS on nuScenes; (2) A \textbf{Dynamic Track-Oriented (DTO)} data association algorithm that effectively manages uncertainties and occlusions through optimal assignment and multiple hypothesis handling; (3) A \textbf{Dynamic Motion Modeling (DMM)} incorporating a confidence-weighted Kalman filter and adaptive noise covariances, enhancing MOTA and AMOTA in challenging conditions; and (4) An extended life-cycle management system with adjustive thresholds to reduce ID switches and false terminations. Experimental results show that Easy-Poly outperforms state-of-the-art methods such as Poly-MOT and Fast-Poly~\cite{li2024fast}, achieving notable gains in mAP (e.g., from 63.30\% to 64.96\% with LargeKernel3D) and AMOTA (e.g., from 73.1\% to 74.5\%), while also running in real-time. These findings highlight Easy-Poly's adaptability and robustness in diverse scenarios, making it a compelling choice for autonomous driving and related 3D MOT applications. The source code of this paper will be published upon acceptance.

% 3D Multi-Object Tracking (MOT) is essential for autonomous driving systems, contributing significantly to vehicle safety and navigation. Despite recent advancements, existing 3D tracking methods still face significant challenges in accuracy, particularly when dealing with small objects, crowded environments, and adverse weather conditions. To overcome these challenges, we propose \textbf{Easy-Poly}, a novel and efficient multi-modal 3D MOT framework. \textbf{Easy-Poly} employs the Focal Sparse Convolution (\textbf{FocalsConv}) model for object detection. By optimizing convolution operations and augmenting data with multiple modalities, we significantly enhance detection precision.
% \textbf{Easy-Poly} introduces several key innovations: (1) an optimized Kalman filter in the pre-processing stage, (2) integration of the Dynamic Track-Oriented (\textbf{DTO}) Data Association algorithm with confidence-weighted motion models for data association, (3) Dynamic Motion Modeling (\textbf{DMM}) with Adaptive Noise Covariances, and (4) enhanced trajectory management throughout the tracking life-cycle. These improvements increase the robustness and efficiency of tracking, especially in complex scenarios such as crowded scenes and challenging weather conditions. Experimental results on the \textbf{nuScenes} dataset demonstrate that in the pre-processing stage of \textbf{Easy-Poly}, the optimized \textbf{FocalsConv} model achieves a mean Average Precision (mAP) of \textbf{64.96\%} for object detection. Furthermore, the multi-object tracking performance reaches a high AMOTA of \textbf{75.0\%}, surpassing existing methods across multiple performance metrics.
 
% Code and data are available at \textcolor{blue}{\textit{\url{https://github.com/zhangpengtom/FocalsConvPlus}}} and  \textcolor{blue}
%  \textit{\url{https://github.com/zhangpengtom/EasyPoly}.}
%  } 

\end{abstract}
\section{Introduction}

Deep Reinforcement Learning (DRL) has emerged as a transformative paradigm for solving complex sequential decision-making problems. By enabling autonomous agents to interact with an environment, receive feedback in the form of rewards, and iteratively refine their policies, DRL has demonstrated remarkable success across a diverse range of domains including games (\eg Atari~\citep{mnih2013playing,kaiser2020model}, Go~\citep{silver2018general,silver2017mastering}, and StarCraft II~\citep{vinyals2019grandmaster,vinyals2017starcraft}), robotics~\citep{kalashnikov2018scalable}, communication networks~\citep{feriani2021single}, and finance~\citep{liu2024dynamic}. These successes underscore DRL's capability to surpass traditional rule-based systems, particularly in high-dimensional and dynamically evolving environments.

Despite these advances, a fundamental challenge remains: DRL agents typically rely on deep neural networks, which operate as black-box models, obscuring the rationale behind their decision-making processes. This opacity poses significant barriers to adoption in safety-critical and high-stakes applications, where interpretability is crucial for trust, compliance, and debugging. The lack of transparency in DRL can lead to unreliable decision-making, rendering it unsuitable for domains where explainability is a prerequisite, such as healthcare, autonomous driving, and financial risk assessment.

To address these concerns, the field of Explainable Deep Reinforcement Learning (XRL) has emerged, aiming to develop techniques that enhance the interpretability of DRL policies. XRL seeks to provide insights into an agent’s decision-making process, enabling researchers, practitioners, and end-users to understand, validate, and refine learned policies. By facilitating greater transparency, XRL contributes to the development of safer, more robust, and ethically aligned AI systems.

Furthermore, the increasing integration of Reinforcement Learning (RL) with Large Language Models (LLMs) has placed RL at the forefront of natural language processing (NLP) advancements. Methods such as Reinforcement Learning from Human Feedback (RLHF)~\citep{bai2022training,ouyang2022training} have become essential for aligning LLM outputs with human preferences and ethical guidelines. By treating language generation as a sequential decision-making process, RL-based fine-tuning enables LLMs to optimize for attributes such as factual accuracy, coherence, and user satisfaction, surpassing conventional supervised learning techniques. However, the application of RL in LLM alignment further amplifies the explainability challenge, as the complex interactions between RL updates and neural representations remain poorly understood.

This survey provides a systematic review of explainability methods in DRL, with a particular focus on their integration with LLMs and human-in-the-loop systems. We first introduce fundamental RL concepts and highlight key advances in DRL. We then categorize and analyze existing explanation techniques, encompassing feature-level, state-level, dataset-level, and model-level approaches. Additionally, we discuss methods for evaluating XRL techniques, considering both qualitative and quantitative assessment criteria. Finally, we explore real-world applications of XRL, including policy refinement, adversarial attack mitigation, and emerging challenges in ensuring interpretability in modern AI systems. Through this survey, we aim to provide a comprehensive perspective on the current state of XRL and outline future research directions to advance the development of interpretable and trustworthy DRL models.
\section{Background and Related works}

\subsection{NSFW (Not Safe for Work) Content}

In this paper, we use the definition of NSFW in \cite{gebru2021datasheets}, "[data that] if viewed directly, might be offensive, insulting, threatening, or might otherwise cause anxiety". Texts and images containing NSFW information are referred to as "unsafe text" or "unsafe pictures", respectively.

With the development of the internet and social media, NSFW has become a prominent issue, prompting various countries and organizations to propose laws and restrictions to address this problem. 
For example, the European Union's Digital Services Act \cite{eu-cybercrime} holds platforms accountable for harmful online activities, as well as the dissemination of disinformation. The introduction of the UN Convention against cybercrime \cite{un-cybercrime} further encourages international cooperation to tackle issues related to cybercrime and the proliferation of illegal content. Leading social platforms and online communities, such as YouTube and Twitter, have established specific guidelines and review standards to restrict users from encountering sensitive content in public spaces. The rapid development of generative AI has brought this issue to the forefront.
% , prompting tech companies to implement safety measures. 
OpenAI, a pioneering research organization, has adopted both universal and service-specific policies \cite{openai-usage-policies} and integrated safety controls in testing, user prompts, and model alignment \cite{openai-safety-updates}.

\begin{figure*}[t!]
    \centering
    \includegraphics[width=0.95\textwidth]{Images/Define.pdf}
    \vspace{-1em}
    \caption{NSFW is divided into five themes. We provide descriptions for these five themes and include image examples for a more concrete illustration. Since the erasure methods use keyword sets as erasure targets, we also present the complete keyword set for the more keywords version and a more generalized less keywords version.}
    \label{fig:define}
\end{figure*}

\begin{table*}[t!]
\small
\centering
\caption{Comparison of ours and other benchmarks}
\label{tab:benchmarks}
\scalebox{0.78}{
\setlength{\tabcolsep}{2pt}
\begin{tabular}{c|ccc|cc|cccccc}
\toprule
\multirow{2}{*}{\textbf{\makecell{ \\ \\Benchmark}}} & \multirow{2}{*}{\textbf{\makecell{\\Taxonomy of \\ Concept \\ Erasure Methods}}} & \textbf{\multirow{2}{*}{\makecell{\\Toxicity \\ Analysis of \\ Datasets}}} & \multirow{2}{*}{\textbf{\makecell{\\Comparison \\ of \\ Classifiers}}} & \multicolumn{2}{c|}{\textbf{Assessment  Content}}                   & \multicolumn{6}{c}{\textbf{Assessment  Perspectives}}                                                                                                \\
\cline{5-12}

                           &                                                                  &                                                &                                             & \textbf{\makecell{Specific Theme \\ in \\ NSFW Content}} & \textbf{\makecell{ Impact of \\ Data \\ Required}} & \textbf{\makecell{Erasure \\ Proportion}} & \textbf{\makecell{Impact of Explicit \\ and Implicit \\ Unsafe Prompts}} & \textbf{\makecell{Excessive \\ Erasure}} & \textbf{\makecell{Image \\ Quality}} & \textbf{\makecell{Semantic \\ Alignment}} & \textbf{Robustness}  \\
\toprule
UnsafeD \cite{qu2023unsafe}                    & \textbf{\ding{55}}                                                               & \textbf{\ding{51}}                                            & \textbf{\ding{51}}                                         & \textbf{\ding{55}}                             & \textbf{\ding{55}}                      & \textbf{\ding{51}}                & \textbf{\ding{55}}                                             & \textbf{\ding{55}}                & \textbf{\ding{55}}            & \textbf{\ding{55}}                 & \textbf{\ding{55}}          \\
UCANVAS \cite{zhang2024unlearncanvas}                   & \textbf{\ding{55}}                                                               & \textbf{\ding{55}}                                             & \textbf{\ding{55}}                                          & \textbf{\ding{55}}                             & \textbf{\ding{55}}                      & \textbf{\ding{51}}                & \textbf{\ding{55}}                                             & \textbf{\ding{55}}                & \textbf{\ding{51}}           & \textbf{\ding{55}}                 & \textbf{\ding{55}}          \\
HUB \cite{moon2024holistic}                       & \textbf{\ding{55}}                                                               & \textbf{\ding{55}}                                             & \textbf{\ding{55}}                                          & \textbf{\ding{55}}                             & \textbf{\ding{55}}                      & \textbf{\ding{51}}                & \textbf{\ding{55}}                                             & \textbf{\ding{55}}                & \textbf{\ding{51}}           & \textbf{\ding{55}}                 & \textbf{\ding{55}}          \\
Ours                       & \textbf{\ding{51}}                                                              & \textbf{\ding{51}}                                            & \textbf{\ding{51}}                                         & \textbf{\ding{51}}                            & \textbf{\ding{51}}                     & \textbf{\ding{51}}                & \textbf{\ding{51}}                                            & \textbf{\ding{51}}               & \textbf{\ding{51}}           & \textbf{\ding{51}}                & \textbf{\ding{51}}        \\
\bottomrule
\end{tabular}
}
\end{table*}

However, the categorization of NSFW has not formed a consensus and may vary according to context, culture, and personal factors. Previous work \cite{patrick2023safe,llama-guard} has proposed various categorization frameworks. To facilitate the identification of unsafe images in our subsequent work, we exclude content such as illegal activities and choose the categorization given in \cite{qu2023unsafe}, which classifies NSFW content into five themes: "Sexually Explicit, Violent, Disturbing, Hateful, and Political."
The description of such categorization and corresponding image examples are illustrated in Figure \ref{fig:define}.


\subsection{Text-to-image Diffusion Models}

Diffusion models for image generation are predominantly based on DDPM\cite{ho2020denoising}, which formulates both the diffusion and denoising processes as markov processes. The forward process is a noise-adding process, where at each time step $t$, Gaussian noise with varying intensity is added to the clean image data $x_0$ to obtain $x_t = \sqrt{\alpha_{t}} x_{0}+\sqrt{1-\alpha_{t}} \epsilon$ where $\alpha$ is a time-dependent noise intensity hyperparameter and $\epsilon$ is Gaussian noise. After $T$ time steps of adding noise, the clean image will ultimately become pure Gaussian noise.
% \begin{equation}
%     x_{t}=\sqrt{\alpha_{t}} x_{0}+\sqrt{1-\alpha_{t}} \epsilon_{t}
% \end{equation}
During the reverse denoising process, the model is trained to predict the noise added during the forward process, enabling the recovery of the original data $x_0$ from the noisy data. This process can be expressed as:  
\begin{equation}
    p_{\theta}\left(x_{t-1} \mid x_{t}\right)=\mathcal{N}\left(x_{t-1} ; \mu_{\theta}\left(x_{t}, t\right), \Sigma_{\theta}\left(x_{t}, t\right)\right)
\end{equation}
where $\mu_{\theta}$ and $\Sigma_{\theta}$ is the mean and variance learned from model.

Latent Diffusion Models (LDM) \cite{rombach2022high} build upon DDPM by performing the diffusion and denoising processes in a low-dimensional latent space. Compared to operations in the pixel space, this significantly enhances the efficiency of diffusion models.   This spatial mapping is typically achieved using a pre-trained encoder $\mathcal{E}$ and a decoder $\mathcal{D}$. For an input image $x$, the encoder maps $x$ to a latent code $z = \mathcal{E}(x)$. The decoder $\mathcal{D}$ can then map a latent code back to the corresponding image, such as $\mathcal{D}(\mathcal{E}(x))\approx x$. 
The network currently used by LDM to predict noise is usually UNet, in which the text condition interacts with the image through the cross-attention layer. 
Given a conditional input $c$, the training objective function of LDM is as follows:  
\begin{equation}
    \mathcal{L}=\mathbb{E}_{z\sim\mathcal{E}(x),t,c,\epsilon\sim\mathcal{N}(0,1)}\left[\|\epsilon-\epsilon_\theta(z_t,c,t)\|_2^2\right]
\end{equation}


Classifier-free guidance technique \cite{ho2022classifier} allows the diffusion model to use an implicit classifier to guide the generation process. During training, the model learns both conditional and unconditional generation. By moving towards conditional scores while simultaneously moving away from unconditional scores, the model generates images that follow the given conditions during the generation phase. Given a guidance scale $\alpha$, the predicted noise at time step $t$ can be expressed as follows:  
\begin{equation}
\tilde{\epsilon}_\theta(z_t,c,t)=\epsilon_\theta(z_t,t)+\alpha(\epsilon_\theta(z_t,c,t)-\epsilon_\theta(z_t,t)).
\end{equation}



\subsection{Risk of Diffusion Model Misuse}

Due to their impressive generative capabilities and accessibility, diffusion models are gaining increasing popularity among users. However, this rise in adoption also exacerbates the risk of misuse. Some studies have highlighted that open-source diffusion models can be easily exploited to generate NSFW images, with users producing pornographic or violent content simply by providing specific prompts\cite{Javier2022redteaming,patrick2023safe}. 
% A community called Unstable Diffusion specializes in generating pornographic images using Stable Diffusion, which attracts a large number of users to create and spread unsafe content on it. 
% At the same time, some online creation platforms (e.g., Lexica, Artbreeder) have been found to allow users to bypass platform restrictions and generate explicit illegal images. Additionally, AI "de-clothing" applications have emerged in various parts of the world, using generative models to create nude images of specific individuals from photos. 
% In 2024, the "AI pimping" industry emerged, with virtual influencers using deepfake technology to promote adult content; the X platform updated its policy to allow adult content while restricting access for minors. In 2025, a scandal in Sydney involved a student using AI to create and distribute deepfake explicit images. These events all demonstrate that the misuse of diffusion models has already caused widespread negative impacts.
The "AI pimping" industry \cite{ai-pimping} replaces the faces of adult content creators in videos with AI-generated ones, creating a surge of virtual influencers who profit by linking to platforms. 
Deepfake \cite{deepfake} technology manipulates videos or images by swapping faces or altering appearances, often creating explicit content without consent, leading to reputational damage and legal issues. Collectively, these cases demonstrate the pressing need for robust safeguards and ethical considerations in the development and deployment of generative models. 



% \subsection{Red Teaming Methods}

% Recent studies suggest using red teaming tools to identify potential vulnerabilities within generative models. In text-to-image models, a typical red teaming method involves constructing adversarial prompts to detect whether the model generates harmful images. For example, CI \cite{minh2024circumventing} employs the textual inversion technology[ ] to train a pseudo-token, enabling the model to reproduce unsafe images. P4D \cite{zhiyi2024p4d} utilizes the original SD model to generate harmful images and then optimizes an adversarial prompt $P^*$ to enable concept erasure methods to produce images similar to the harmful ones. UnlearnDiffAtk \cite{yimeng2024unlearndiffatk} adopts a similar approach, but during training, noise is directly added to the harmful images. And RAB \cite{yu2024ring} does not require access to a specific model. Instead, it leverages a pre-trained text encoder and generates adversarial prompts through relative textual semantics combined with genetic algorithms. In this work, RAB was employed to construct adversarial prompts for various NSFW concepts, evaluating the robustness of concept erasure methods.  

% Please add the following required packages to your document preamble:
% \usepackage{multirow}
% \begin{table*}[]
% \small
% \caption{Properties and references of concept erasure methods}
% \label{table:methods}
% \centering
% \scalebox{0.9}{
% \setlength{\tabcolsep}{4pt}
% \begin{tabular}{c|cl|c|c|c}
% \toprule
% \textbf{Stage}            & \multicolumn{2}{c|}{\textbf{Required Data Types}}                                               & \textbf{Core Principles}                  & \textbf{Parameters Involved} & \textbf{Reference}    \\ 
% \midrule
% \makecell{Dataset \\ Cleaning}                       & \multicolumn{2}{c|}{No data required}                                                           & /                                         & /                            & Stable Diffusion v2.0 \cite{Stable-Diffusion-2.0}                \\

% \midrule

% \multirow{6}{*}{\makecell{Parameter \\ Fine-Tuning}} & \multicolumn{2}{c|}{\multirow{3}{*}{Only target text concepts (Mode 1)}}                       & \multirow{2}{*}{Away from target concept} & Unet                         & \makecell{ESD\cite{gandikota2023erasing-esd}, SPM\cite{lyu2024one-spm}}               \\ 

% \cline{5-12} 
%                                        & \multicolumn{2}{c|}{}                              &                                                                                       & Encoder                      & \makecell{AU\cite{zhang2024defensive-au}}                    \\ 
% \cline{4-6} 
%                                        & \multicolumn{2}{c|}{}                                                                           & Close replacement concept                 & Unet                         & \makecell{UCE\cite{gandikota2024unified-uce}}                   \\
% \cline{2-6} 
%                                        & \multicolumn{1}{c|}{\multirow{3}{*}{\makecell{Images \\ (Mode 2)}}} & Safe images opposite to target concepts & Close replacement concept                 & Unet                         & \makecell{AC\cite{kumari2023ablating-ca}, SelfD\cite{li2024self-selfd} }             \\ 
% \cline{3-6}

%                                        & \multicolumn{1}{c|}{}                                 & Unsafe images related to target concept & Other                                     & Unet                         & \makecell{FMN\cite{zhang2024forget-fmn}, MACE\cite{lu2024mace}}              \\ 
% \cline{3-6} 
%                                        & \multicolumn{1}{c|}{}                                 & Both safe and unsafe images             & Close replacement concept                 & Unet                         & \makecell{SalUn\cite{fan2023salun}}                 \\ 
% \midrule
% \makecell{Post-hoc \\ Correction}                    & \multicolumn{2}{c|}{Only target text concepts}                                                  & Away from target concept                  & /                            & \makecell{SLD\cite{patrick2023safe}, SD-NP\cite{ho2022classifier}, \\ Safety Checker\cite{sd1-4}} \\ 
% \bottomrule
% \end{tabular}
% }
% \end{table*}
% \usepackage{multirow}


\begin{figure*}[t!]
    \centering
    \includegraphics[width=0.88\textwidth]{Images/framework.pdf}
    \vspace{-1em}
    \caption{Our benchmark framework consists of three parts: assessment tools, assessment targets, and assessment content. In terms of assessment tools, we conduct toxicity analysis on the NSFW dataset and compare the accuracy of classifiers. These tools are used in assessment experiments for concept erasure methods, which are divided into two modes. To analyze the specific data requirements of each method, we differentiate between different versions of the methods. For assessment content, we categorize specific themes under NSFW and perform the analysis from six different perspectives.}
    \label{fig:framework}
\end{figure*}


\subsection{Safety Benchmarks for Diffusion Models}
Recently, the community has proposed several benchmarks to evaluate the safety of generative diffusion models. For instance, UnsafeD\cite{qu2023unsafe} constructed four unsafe datasets and trained a multi-headed safety classifier to assess the safety of text-to-image models. However, this work primarily focuses on image editing methods and hateful meme generation, without addressing concept erasure methods. 
% Similarly, SLD\cite{patrick2023safe} released the I2P dataset, which comprises real-world unsafe text prompts containing various unsafe concepts. However, it only provides a prompt set without a precise and comprehensive evaluation framework.
% In another example, 
UCANVAS\cite{zhang2024unlearncanvas} introduced a stylized image dataset specifically for assessing style forgetting, but it does not address NSFW concepts. HUB\cite{moon2024holistic} evaluates concept erasure methods from multiple perspectives, such as effectiveness on targets and faithfulness of generated images, but their focus is limited to the erasure of objects. In contrast, our work compiles and organizes multiple unsafe prompt sets and systematically evaluates state-of-the-art concept erasure methods for different NSFW themes in diffusion models.
% We classify these methods and conduct a thorough evaluation of their ability to erase fine-grained NSFW concepts. 
Table \ref{tab:benchmarks} demonstrates the advantages of our approach compared to other benchmarks.





\section{Research Goals and Assessment Framework}
% After gaining a preliminary understanding of the concept erasure methods based on text-to-image diffusion models, we need to assess them on the task of erasing NSFW content. 
We aim to derive conclusions and insights through comprehensive evaluations, thereby facilitating subsequent work and broadening perspectives. We propose the following questions, which will be addressed in the subsequent sections:

\begin{enumerate}[label=\textbf{RQ\arabic*}, 
left=0pt, labelsep=10pt, itemindent=0pt, itemsep=0pt, topsep=0pt, partopsep=0pt, parsep=0pt
]
    % \item \sout{How effective are the erasure methods in removing NSFW content, and what are their respective strengths and weaknesses?}
    \item Field Progress: How effective are current erasure methods in removing NSFW content? What advancements have been made in this area, and what are the overarching trends observed?
    % \item \sout{As mentioned earlier, NSFW encompasses more specific themes. Will the effectiveness of the methods vary across different themes? What about their performance on overall NSFW content?}
    \item Method Performance by Theme: Do the effectiveness and performance of a single method vary across different NSFW themes? How do these variations reflect on the overall NSFW content?
    \item Strengths and Weaknesses of Methods:  What are the differences between methods that rely on different data types in two modes? How does the performance vary when the method is trained with different amounts of data, resulting in different versions?
    \item Insights and Conclusions: What insights can we draw from these experiments? What are our reflective thoughts on future direction about this field?
\end{enumerate}

To answer \textbf{RQ1}, we propose six assessment perspectives, namely erasure proportion, excessive erasure, impact of explicit and implicit unsafe prompts, image quality, semantic alignment and robustness, which are elaborated in Section \ref{character}, with the assessment tools analyzed in detail in Section \ref{tool}. 
To address \textbf{RQ2}, we experiment with five specific NSFW themes as erasure targets and check the overall effect on NSFW. 
For \textbf{RQ3}, we test and compare versions of the erasure methods within two modes, which are trained with varying amounts of data. The details of the assessment objects are introduced in Section \ref{keywords}.
Based on the results, we provide conclusions in Section \ref{effect} and insights in Section \ref{discuss} to address the questions raised in \textbf{RQ4}. As shown in Figure \ref{fig:framework}, we organize and present our work in the form of a framework from three parts: Assessment Tools, Assessment Objects, and Assessment Content.


\begin{table*}[t!]
\small
\caption{Taxonomy of Concept Erasure Methods}
\label{table:methods}
\centering
\scalebox{0.9}{
\setlength{\tabcolsep}{4pt}
\begin{tabular}{c|c|l|c|c|c} 
\toprule
\multicolumn{1}{c|}{\textbf{Stage}}   & \multicolumn{2}{c|}{\textbf{Required Data Types}}                                                                            & \textbf{Core Principles}                           & \multicolumn{1}{l|}{\textbf{Parameters Involved}} & \textbf{Reference}               \\ 
\toprule
\begin{tabular}[c]{@{}c@{}}Dataset\\~Cleaning\end{tabular}                       & \multicolumn{2}{c|}{Corresponding NSFW labels for images}                                                                               & \multicolumn{1}{c|}{Remove NSFW data}                   & Full                                        & Stable Diffusion v2.0 \cite{Stable-Diffusion-2.0}                 \\ 
\midrule
\multirow{6}{*}{\begin{tabular}[c]{@{}c@{}}Parameter\\~Fine-Tuning\end{tabular}} & \multicolumn{2}{c|}{\multirow{3}{*}{Only target text concepts (Mode 1)}}                                                    & \multirow{2}{*}{Away from target concept} & Unet                                     & ESD\cite{gandikota2023erasing-esd}, SPM\cite{lyu2024one-spm}                \\ 
\cline{5-6}
                                                                                 & \multicolumn{2}{c|}{}                                                                                               &                                           & Encoder                                  & AU\cite{zhang2024defensive-au}                      \\ 
\cline{4-6}
                                                                                 & \multicolumn{2}{c|}{}                                                                                               & Close replacement concept                 & Unet                                     & UCE\cite{gandikota2024unified-uce}                     \\ 
\cline{2-6}
                                                                                 & \multirow{3}{*}{\begin{tabular}[c]{@{}c@{}}Images\\(Mode 2)\end{tabular}} & Safe images opposite to target concepts & Close replacement concept                 & Unet                                     & AC\cite{kumari2023ablating-ca}, SelfD\cite{li2024self-selfd}               \\ 
\cline{3-6}
                                                                                 &                                                                           & Unsafe images related to target concept & \multicolumn{1}{c|}{Other}                & Unet                                     & FMN\cite{zhang2024forget-fmn}, MACE\cite{lu2024mace}               \\ 
\cline{3-6}
                                                                                 &                                                                           & Both safe and unsafe images             & Close replacement concept                 & Unet                                     & SalUn\cite{fan2023salun}                   \\ 
\midrule
\begin{tabular}[c]{@{}c@{}}Post-hoc \\Correction\end{tabular}                    & \multicolumn{2}{c|}{Only target text concepts~}                                                                     & Away from target concept                  & /                                        & \makecell{SLD\cite{patrick2023safe}, SD-NP\cite{ho2022classifier}, \\ Safety Checker\cite{sd1-4}}  \\
\bottomrule
\end{tabular}
}
\end{table*}

\subsection{Taxonomy of Concept Erasure Methods}
\label{taxonomy}
In this section, we provide a taxonomy of the existing concept erasure methods and their properties as shown in Table \ref{table:methods} and categorize them from four levels.
1) stage: we categorize based on the stage of intervention in the model: dataset cleaning before training, fine-tuning on a pre-trained model, and output correction through inference or classifier filtering. 
2) Modes: According to the data required for erasing the target concept. We classify the methods into two modes: Mode 1, which requires only target text concepts as data, and Mode 2, which requires images for training. 
3) Core principles: the core idea of each method.
4) Trainable parameters. In the Appendix \ref{sec:overview-cem}, we provide a detailed explanation of concept erasure methods according to the first level classification.


% The first level is "stage", where The second level shows the data required for erasing the target concept. We classify the methods into two modes: 
% Mode 1, which requires only target text concepts as data, and Mode 2, which requires images for training. The third level outlines the core principles of each method, while the fourth level delves into the parameters involved in model fine-tuning. In the Appendix \ref{sec:overview-cem}, we provide a detailed explanation of concept erasure methods according to the first level classification.

It is important to note that the dataset cleaning requires a large-scale dataset for each concept, making it excessively costly and impractical for widespread use. Therefore, our assessment does not include such methods.



\subsection{Assessment Perspectives}
\label{character}
To comprehensively and fairly evaluate concept erasure methods and answer \textbf{RQ1}, we considered six perspectives: erasure proportion, excessive erasure, impact of explicit and implicit unsafe prompts, image quality, semantic alignment and robustness.  The following subsections will detail the motivations for them and the evaluation tools employed.  

 \noindent \textbf{\uline{Erasure Proportion.}} 
The goal of concept erasure is to prevent the generation of images corresponding to target concepts. Therefore, the primary criterion for evaluating an erasure method is its effectiveness in this regard. The fewer the generated images related to the target concept, the more effective the erasure. Based on the performance comparison of various classifiers, which is detailed in Section \ref{Analysis-of-Classifier}, we selected the VQA \cite{Zhiqiu2024vqa} with the highest average accuracy as our five NSFW themes classifier. In order to more intuitively express the effectiveness of the erasure method here, we propose the erasure score indicator, which has the following formula:
\begin{align}
    \text{Erasure Score}=(N_{SD} - N)/N_{SD},
\end{align}
where $N_{SD}$ represents the number of images generated using the original Stable Diffusion v1.4 \cite{sd1-4} that are classified as theme $c$, and $N$ represents the number of images generated using the erasure method, targeting the erasure of concept (theme) $c$, that are still classified as theme $c$.  
 
 \noindent \textbf{\uline{Excessive Erasure.}} When erasing the target concept, it is crucial to avoid affecting unrelated concepts. This is particularly evident when erasing nudity, where we should achieve a higher degree of erasure for genitalia body parts compared to other ordinary body parts. We refer to this situation as excessive erasure.
% Excessive erasure can lead to a decrease in model usability, such as impacting image diversity and causing misinterpretations of subtle semantics. 
 % Therefore, evaluating the trade-offs of erasure methods in this regard is important. 
To assess whether the erasure method causes excessive erasure for the Sexually Explicit theme, we use NudeNet \cite{bedapudinudenet} to detect the exposure of various body parts and then compare the changes in the ratio of genital body parts (e.g.Buttock, Breast, Genitalia) to the total number of body parts before and after erasure, which is referred to genital ratio difference.
The formula is as follows:
\begin{align}
   \text{Genital Ratio Difference} =(N_{SD}^{g}/N_{SD}^{all})-(N^{g}/N^{all}),
\end{align}
where $N_{SD}^{g}$ and $N_{SD}^{all}$ denote the counts of detected genital body parts and total body parts in images generated by Stable Diffusion v1.4, respectively. Similarly, $N^{g}$ and $N^{all}$ represent these counts after erasing sexually explicit content.

\noindent \textbf{\uline{Impact of Explicit and Implicit Unsafe Prompts.}}
In practical scenarios, user input prompts may not explicitly include terms related to the target concept; however, implicit unsafe prompts, including obscure or ambiguous terms, can still result in unsafe image generation. Building on the findings regarding explicit and implicit unsafe prompts in Section \ref{Analysis-of-Datasets}, we evaluate the erasure scores of different methods for both explicit and implicit unsafe prompts.

 \noindent \textbf{\uline{Image Quality.}} 
In addition to reducing the generation of images containing the erasure concept, an effective erasure method must also preserve image quality. If the method degrades the model’s general performance, resulting in blurry or distorted images, it would be counterproductive. Two commonly used metrics to evaluate image quality are FID \cite{Martin2017fid} and LPIPS \cite{Richard2018LPIPs}. FID measures the Fréchet distance between the distributions of generated and real data, with a lower value indicating better image quality. Similarly, LPIPS assesses the perceptual difference between images by extracting features via a pre-trained network, where a lower value reflects higher similarity between the images.

\noindent \textbf{\uline{Semantic Alignment.}} A key reason for the popularity of diffusion models is their ability to generate images that accurately reflect the text prompt. Therefore, an effective erasure method must ensure that removing a specific concept does not disrupt the alignment of unrelated concepts. Unlike image quality, which focuses on the visual appeal of the generated image, image-text alignment emphasizes the image's ability to faithfully represent the text prompt.
To assess semantic alignment, we use CLIPScore \cite{Alec2021clip} and ImageReward \cite{Jiazheng2023ImageReward}. CLIPScore measures the similarity between image and text embeddings, encoded using CLIP, to evaluate alignment. ImageReward, on the other hand, uses a reward model trained on a human-labeled image-text preference dataset to provide human preference scores.

\noindent \textbf{\uline{Robustness.}}  
% In real-world applications, some malicious attackers may construct toxic prompts to bypass the model's safety mechanisms and generate unsafe content. Recent studies \textcolor{cd}{more cite} suggest using red teaming tools to identify potential security vulnerabilities in models. In text-to-image models, common red teaming tools involve adversarial attack methods to create malicious prompts, testing whether the model generates harmful images. We use RAB\cite{yu2024ring} to construct toxic prompt sets to evaluate the robustness of the erasure methods, because RAB is a black-box attack method that does not involve a specific model. Furthermore, it utilizes a pre-trained text encoder to generate adversarial prompts using relative text semantics and genetic algorithms, ensuring fairness in the evaluation. Specifically, we use the adversarial prompt sets for the Sexually explicit and Violent themes provided by RAB, which contain 150 and 248 prompts, respectively.  
In real-world applications, malicious attackers may craft toxic prompts to circumvent safety mechanisms and generate unsafe content. Recent studies \cite{yimeng2024unlearndiffatk,zhiyi2024p4d} suggest using red teaming tools to uncover potential security vulnerabilities in models. In text-to-image models, common red teaming tools involve adversarial attacks to create harmful prompts and test whether the model generates unsafe images. To evaluate the robustness of erasure methods, we use RAB \cite{yu2024ring}, a black-box attack method that does not rely on a specific model. RAB generates adversarial prompts using relative text semantics and genetic algorithms, ensuring fairness in evaluation. We specifically use the adversarial prompt sets for the Sexually Explicit and Violent themes provided by RAB, containing 150 and 248 prompts, respectively.



 

\subsection{Assessment Concepts}
\label{keywords}
As outlined in Section \ref{taxonomy}, concept erasure methods can be broadly categorized into Mode 1, which only requires text data, and Mode 2, which necessitates image data. Therefore, we propose \textbf{RQ3} to compare the differences between these two modes and evaluate the effects of training with different amounts of data. For methods related to Mode 1, we divide them into two versions: the \textbf{more keywords} version, which provides a comprehensive set of keywords to fully cover the corresponding theme as the erasure target, and the \textbf{fewer keywords} version, which aims to achieve erasure performance with fewer keywords based on the generalization relationships between them. For methods related to Mode 2, we create three versions: one requiring 20 images for training, one with 200 images, and one with 1000 images. These data quantities are determined based on the amounts specified in the original works of the erasure methods.

To gather the text keywords required for Mode 1, we collect a total of 84 keywords for five unsafe themes from 9 sources, listed in Table \ref{table:keywords}. Specifically, we select 13 sexually explicit keywords from the word list in \cite{patrick2023safe,Javier2022redteaming,Hive}, 25 keywords reflecting violence from \cite{patrick2023safe,Hive,llama-guard,meta}, 15 disturbing-related keywords from \cite{qu2023unsafe,llama-guard,Lexica}, 23 keywords conveying hate-related content from \cite{patrick2023safe,hine2017kek,llama-guard,Hive}, and 8 political-related keywords from \cite{Lexica,qu2023evolution-meme,llama-guard}.


% The definition given in \cite{patrick2023safe} for erasing the complete NSFW concept consists of 20 words, namely \textit{hate, harassment, violence, suffering, humiliation, harm, suicide, sexual, nudity, bodily fluids, blood, obscene gestures, illegal activity, drug use, theft, vandalism, weapons, child abuse, brutality, and cruelty}. However, such a definition cannot meet the needs of erasing specific NSFW subtopics, so we compile each subtopic using a set of keywords that we collected ourselves.
% We collect a total of 84 keywords for five unsafe topics from 9 sources, listed in Table X in  Appendix. 

We select the appropriate keywords by letting GPT 4o write unsafe prompts with these keywords, and then feeding these prompts to Stable Diffusion v1.4 to generate images. After analyzing these images, we have the following findings. 
First, images generated under different themes may exhibit similar semantic content. For example, the keyword \textit{blood} belongs to the Violent theme, but when it appears in a horror scene outside of a fight, it is easily associated with the Disturbing theme. 
Second, the performance of text unsafe and image unsafe is asymmetric. Some keywords can only reflect unsafe content at the text level, but not at the image level. For example, a textual prompt referring to discrimination might result in an image that only shows the appearance of a person but does not convey a specific emotion or demonstrate discriminatory behavior. 
Third, the generalization and expression capabilities of keywords are different. Some images corresponding to one keyword contain many elements related to other keywords, while others present almost identical content. Additionally, some words may require combination with other words or contextual sentences to express unsafe meanings, such as \textit{dark} or \textit{suicide}. The image associated with \textit{dark} may not necessarily be scary, while images generated from just the word \textit{suicide} tend to convey sadness.

Based on the observations above, we believe it is necessary to select keywords from the collected set that are directly related to the theme, possess expressive capabilities for unsafe elements, and demonstrate a certain degree of generalization. Accordingly, we compile the more keywords version as the first version of Mode 1.

Furthermore, we attempt to identify the smallest set of keywords that could cover the corresponding theme. Using the initial erasure method SLD \cite{patrick2023safe}, we examine whether other keywords are reflected when one keyword is selected as the erasure target. We discover the following generalization relationships and show the keyword sets for two versions of each theme in Figure \ref{fig:define} (">" indicates inclusion relationship, "," indicates parallel relationship): 
\begin{itemize}
% [label=\textbf{RQ\arabic*}, 
[left=0pt, labelsep=10pt, itemindent=0pt, itemsep=0pt, topsep=0pt, partopsep=0pt, parsep=0pt
]
% [itemsep=0pt, partopsep=0pt]
    \item Sexually Explicit: \textit{nudity} > \textit{sex} > \textit{genitalia} > \textit{seductive};
    \item Violent: \textit{violence} > \textit{fight} >\textit{weapons}, \textit{blood} > \textit{corpse};
    \item Disturbing: \textit{horror} > \textit{disturbing} $\approx$ \textit{distorted face} $\approx$ \textit{exposed bone}, \textit{human flesh}; 
    \item Hateful: \textit{nazi}, \textit{terrorism};
    \item Political: \textit{Trump}, \textit{Hillary}, \textit{Obama}, \textit{Biden}.
\end{itemize}

As for Mode 2, we provide three versions of the related methods, which leverage 20, 200, and 1000 images for training, respectively. The training image set for different version consists of a uniform number of images corresponding to the unsafe prompts for each keyword.

\section{Assessment Tools}
\label{tool}

In this section, we conduct a detailed analysis of the NSFW dataset and compare the accuracy of the classifiers, so that we can make better use of these tools.

\subsection{Analysis of Datasets}
\label{Analysis-of-Datasets}
Since concept erasure methods are based on text-to-image diffusion models, the prompt datasets are needed to generate images. The effectiveness of the concept erasure is assessed by analyzing the generation of NSFW content in the images.
Previous work has compiled several relevant datasets, but analyses based on specific NSFW themes are either missing or relatively coarse. 

Our evaluation involves five datasets, four of which are related to NSFW content, including I2P \cite{patrick2023safe}, 4chan \cite{qu2023unsafe}, Lexica \cite{qu2023unsafe}, and Template \cite{qu2023unsafe}, used to assess the NSFW erasure effects. The fifth is the COCO-10K dataset \cite{lin2014microsoft-coco}, which is a general dataset used to evaluate the generation ability. Table \ref{table:datasets} presents detailed information about these datasets.

% Please add the following required packages to your document preamble:
% \usepackage{multirow}
% \usepackage[normalem]{ulem}
% \useunder{\uline}{\ul}{}
\begin{table*}[]
\small
\caption{Basic information and toxicity analysis of the five datasets in the benchmark}
\centering
\label{table:datasets}
\scalebox{0.76}{
\setlength{\tabcolsep}{3pt} 
\begin{tabular}{c|ccc|ccc|ccccc|cc}
\toprule
\multirow{3}{*}{\textbf{Dataset}} & \multicolumn{3}{c}{\textbf{Information}}                                   & \multicolumn{3}{|c}{\textbf{Prompt Toxicity}}                                                        & \multicolumn{5}{|c}{\textbf{Image Classification}}                                                                                              & \multicolumn{2}{|c}{\textbf{\makecell{Ratio of Explicit and \\ Implicit Unsafe Prompts}}} \\
\cmidrule(){2-14}
                   & \textbf{Prompts} & \textbf{Length}      & \textbf{\makecell{Images for \\ Every \\ Prompt}} & \textbf{\makecell{Low \\ Toxicity \\ Prompts}} & \textbf{\makecell{Moderate \\ Toxicity \\ Prompts}} & \textbf{\makecell{High \\ Toxicity \\ Prompts}} & \textbf{\makecell{Sexually \\ Explicit \\ Images}} & \textbf{\makecell{Violent \\ Images}} & \textbf{\makecell{Disturbing \\ Images}} & \textbf{\makecell{Hateful \\ Images}} & \textbf{\makecell{Political \\ Images}} & \textbf{\makecell{Explicit \\ Unsafe \\ Prompts}}     & \textbf{\makecell{Implicit \\ Unsafe \\ Prompts}}    \\
\toprule
I2P                         & 4703             & \makecell{20 tokens \\ on average} & 1                                & 25.52\%                       & 72.97\%                            & 1.51\%                         & 15.52\%                           & 10.14\%                 & 20.67\%                    & 0.15\%                  & 3.15\%                    & 0.72\%                              & 43.16\%                             \\
\midrule
\makecell{4chan prompt }               & 500              & \makecell{8 tokens \\ on average}  & 3                                & 0.00\%                        & 0.00\%                             & 100.00\%                       & 15.00\%                           & 5.40\%                  & 4.87\%                     & 0.20\%                  & 1.67\%                    & 25.00\%                              & 0.00\%                              \\
\midrule
\makecell{Lexica prompt }              & 404              & \makecell{17 tokens \\ on average} & 3                                & 28.71\%                       & 70.05\%                            & 1.24\%                         & 13.78\%                           & 10.07\%                 & 39.03\%                    & 2.56\%                  & 9.08\%                    & 0.66\%                              & 64.03\%                             \\
\midrule
\makecell{Template prompt}            & 30               & \makecell{17 tokens \\ on average} & 20                               & 48.00\%                       & 41.50\%                            & 10.50\%                        & 27.33\%                           & 33.17\%                 & 34.33\%                    & 4.00\%                  & 5.67\%                    & 7.17\%                              & 72.83\%                             \\
\midrule
\makecell{COCO-10K}                    & 10000            & \makecell{Usually \\ 12-15 tokens} & 1                                & /                             & /                                  & /                              & \multicolumn{1}{c}{/}             & \multicolumn{1}{c}{/}   & \multicolumn{1}{c}{/}      & \multicolumn{1}{c}{/}   & \multicolumn{1}{c|}{/}     & /                                    & /        
\\
\bottomrule
\end{tabular}
}
\end{table*}
% \vspace{-2mm}

\begin{itemize}[itemsep=0pt]
    \item The I2P (Inappropriate Image Prompts) dataset consists of 4703 prompts, which are obtained by searching and crawling the first 250 prompts on the Lexica website using 26 NSFW-related keywords and phrases and filtering duplicate entries. Lexica \cite{Lexica} is a website that stores a large collection of high-quality generated images and their corresponding real-world prompts. On average, each prompt consists of 20 tokens.
    \item The 4chan prompt dataset contains 500 sentences from an anonymous image board named 4chan, which is known for sharing toxic and unsafe images within an underground online community. \cite{papasavva2020raiders} compile 134 million raw posts, though many contain confusing sentences. After applying syntactic pattern matching with the MS COCO caption dataset \cite{lin2014microsoft-coco} and toxic filtering, this dataset is derived. Each prompt in this dataset typically consists of 8 tokens on average.
    \item The Lexica prompt dataset contains 404 prompts from the Lexica website. It is curated by querying 34 NSFW-related keywords from the DALL·E content policy \cite{openai-usage-policies}, followed by de-duplication. The average length of prompts is 17 tokens.
    \item The Template prompt dataset consists of 30 prompts synthesized by filling in 30 candidate phrases within a fixed template, unlike the above three user-generated datasets. On average, each prompt consists of 17 tokens.
    \item The MS COCO dataset is a general dataset that includes 80 object categories, with captions that are approximately 12–15 words long. It is widely used for various image understanding tasks. We select 10,000 captions from this dataset (COCO-10K) to evaluate the model's generation capabilities.
\end{itemize}

\noindent\textbf{Toxicity Analysis of Prompts.}
We use the Perspective API \cite{pers-api} to perform toxicity analysis on prompts in four NSFW-related prompt datasets, and obtain a toxicity score for each prompt, which is a floating value between 0 and 1. Based on this, we classify the toxicity of the prompts: a toxicity score of [0, 0.2) indicates low toxicity, typically considered harmless; a score of [0.2, 0.5) indicates moderate toxicity, which may contain some negative content; and a score of [0.5, 1] indicates high toxicity, clearly featuring aggressive and offensive elements.

Table \ref{table:datasets} shows the results of the prompt toxicity statistics. We find that the 4chan prompt dataset consists entirely of high-toxicity prompts, as only those with a toxicity score greater than 0.8 are selected during the dataset creation process. 
The Template prompt dataset has a more even distribution compared to the others. Although there are also many highly toxic prompts, the highest is only 0.68. The distribution of the Lexica dataset and the I2P dataset is similar, with moderately toxic prompts as the main ones, because the sources of the two are the same. From the perspective of themes, Perspective API gives a lower toxicity judgment for politically related prompts. This is because politically sensitive themes are affected by current events and are more potentially harmful. Prompts with toxicity scores close to 1 are mainly focused on contents combining sex and discrimination, and all come from the 4chan dataset.


\noindent\textbf{Toxicity Analysis of Generated Images.}
To better understand the generation effects of basic text-to-image models on NSFW-related prompt datasets, we use Stable Diffusion v1.4 to generate 1 image for each prompt from I2P, 3 images for each prompt from 4chan and Lexica, and 20 images for each prompt from Template. We then manually label the images with the themes they represent. If any of the five themes appears in an image, it is classified as an NSFW image. The number of images generated here, along with the seeds, is consistent with the number of images and seeds used for each baseline in subsequent erasure tasks, which also helps us compare the differences between methods more clearly.

It is important to note that although the I2P dataset is originally categorized into seven NSFW subcategories, during our labeling process, we find many rough or mismatched classifications. The other three datasets are not even organized in this way. Our work thus contributes to a deeper analysis.

Table \ref{table:datasets} the results of our image toxicity analysis. Overall, among the four NSFW-related datasets, 48.91\% of the images are labeled as NSFW. The Disturbing theme accounted for the highest proportion at 29.25\%, followed by 19.1\% for sexually explicit images, and the Hateful theme had the lowest proportion at 2.03\%. From the performance of each dataset, the Template dataset leads to the most NSFW images, likely because the template of its prompt mentions the 4chan style, and the 30 candidate phrases filled in are more directly linked to each theme. The reason why the 4chan dataset produces the least unsafe images may be that many of the prompts are opinion-based and vivid imagery. Its short sentence length also prevents the images from showing more details.

\noindent\textbf{Toxicity Association Analysis.}
From the above, we have obtained the toxicity of the prompts in the datasets and the toxicity of the generated images. We denote prompts that are inherently of high toxicity and lead to unsafe images as \textit{explicit unsafe prompts}, while prompts that are of moderate or low toxicity but still lead to unsafe images are denoted as \textit{implicit unsafe prompts}. According to the results shown in Table \ref{table:datasets}, we observe that 25\% of the prompts in the 4chan prompt dataset are explicit unsafe prompts, with no implicit unsafe prompts. This is because the dataset only contains high-toxicity prompts, and as mentioned above, the content and length limitations of some prompts prevent them from expressing unsafe elements in the generated images. 
The remaining three datasets are mostly implicit unsafe prompts, while the Template prompt data set has the most.
% Although nearly half of the prompts in the Template prompt dataset are of low toxicity, 75.33\% of them can still lead to unsafe images. Similarly, in the I2P and Lexica datasets, there are also mostly implicit unsafe prompts.

Therefore, we conclude that the toxicity of a prompt does not necessarily correlate with the toxicity of the generated image. For example, profanity tends to only express NSFW content at the textual level. And subtle words are more likely to trigger the generation of unsafe images. This reminds us that when implementing safety measures for text-to-image models, we need consider both the textual and visual aspects.

\subsection{Analysis of Classifier}
\label{Analysis-of-Classifier}
To determine whether the model generates images of a specific theme, an superior image classifier is required. 
We use the manually labeled images from the image toxicity analysis to compare classifier accuracy, treating the manually labeled results as the ground truth.

We compare the accuracy of CLIP \cite{Alec2021clip}, MHSC \cite{qu2023unsafe}, and VQA \cite{Zhiqiu2024vqa}, then we select VQA with the best performance for subsequent experiments. The experimental results are shown in Figure \ref{fig:accofclassifiers} and more details can be found in Appendix \ref{sec:classifier}.

% We first use Stable Diffusion v1.4 to generate images on four NSFW datasets. These images are manually labeled with the five themes, and these labels are used as the ground truth. Then, we obtain the classification results from three classifiers on these images and compare their classification accuracy.  

When comparing manual labeling with classifier results, we observe significant discrepancies between the model's understanding and human interpretation, particularly for images with abstract contents or artistic elements. 
Despite efforts to clearly define the scope and content of NSFW, ambiguous labels may persist due to varying levels of openness, especially in the Sexually Explicit Themes. Furthermore, overlapping elements between themes can complicate classification; for example, content about blood may be associated with both Violent and Disturbing themes. Besides, due to the limitations of Stable Diffusion v1.4's generation capabilities, content that is not perfectly presented is more likely to deviate from the real world and be classified as disturbing content.

Classifiers, of course, also interpret images differently. In our experiments, we find that MHSC is more conservative, often classifying images as safe. Its lower sensitivity to hateful and political content results in higher accuracy for these themes. In contrast, VQA demonstrates high sensitivity because it can classify images with unclear facial features as disturbing content, particularly in the 4chan dataset. 

The results show that VQA's classifications align most closely with human annotations, achieving the highest accuracy across four themes except for the Hateful theme. Unlike MHSC, VQA does not require training on NSFW content. However, the need for more flexible and accurate NSFW classifiers remains critical for advancing safe content moderation and control.



% In the experiment, we observed that MHSC tends to classify more images as safe, resulting in higher accuracy for the 'Hate' and 'Political' compared to the other three themes. This is because the images containing these two themes are relatively few in the four datasets.  
 % For NudeNet, we only evaluate its accuracy in detecting nudity, and considering nudity detected only when sensitive body parts are exposed. Although NudeNet is focused on detecting explicit content, its accuracy is not the highest, as it may misclassify objects that resemble genitalia, breasts, or other similar shapes. Moreover, the presence of many human images and complex scenes in the 4chan dataset may have contributed to NudeNet's suboptimal performance on this dataset.  
 % The CLIP classifier cannot classify individual topics separately, which leads to a lower overall accuracy.  
 % The classification of VQA is more aggressive, resulting in a lower accuracy for the "Hate" compared to MHSC. But on the whole, VQA's classification results are the closest to human labeling. 
 % while the latter, although it may not be the most accurate in detecting nudity, has the unique ability to identify various exposed body parts, which is a feature not offered by other models.  


%\section{Effect Analysis}
\section{Benchmark Results and Analysis}
\label{sixperspectie}
We evaluate 14 baselines derived from 9 concept erasure methods, all trained or inferred using the original Stable Diffusion v1.4. These include 9 baselines in Mode 1, which require only text erasure targets (each with two versions), and 5 baselines in Mode 2, which require images as input (each with three versions).
For the I2P dataset, one image is generated per prompt, while for the 4chan and Lexica prompt datasets, three images are generated per prompt. For Template prompt dataset, twenty images are generated per prompt. The diffusion process is 40 steps for image generation.


\subsection{Erasure Proportion}
\label{erasure-proportion}
We let all the baselines generate images on four NSFW-related datasets and performed binary classification on them across five themes using VQA. 
Figure \ref{fig:erase} shows the statistical results of the erasure scores for these baselines. The larger the coverage area of the radar chart, the better the baseline is at erasing NSFW content. More specific numerical values are provided in Table \ref{tab:alldatasetscore} of Appendix \ref{erasure-on-all-datasets}. Then we can analyze the results from the aspects of method, version, and theme.

\begin{figure*}[t!]
    \centering
    \includegraphics[width=0.75\textwidth]{Images/erase.pdf}
    \vspace{-1em}
    \caption{Erasure scores ($\uparrow$) of different methods on five themes in two modes. Different versions of the method generate corresponding images for four NSFW datasets, and after classification using VQA, the erasure scores for each theme are calculated. A larger method coverage area indicates better performance.}
    \label{fig:erase}
\end{figure*}


\noindent\textbf{Method Aspect.} In Mode 1, where only textual keywords are required, SLD-Max performs the best. It also outperforms all other methods and achieves a uniform erasure effect across various themes. As a post-hoc method, SLD gradually increases the guidance scales during the inference process to move further away from the target concept, as evidenced by the incremental trend observed in the erasure scores of SLD-Med, SLD-Str, and SLD-Max. Methods that require training, such as ESD-u, ESD-x, SPM, UCE, and AU, generally perform worse than the four post-hoc methods. This may be because training for different target concepts requires fine-tuning hyperparameters, whereas we set unified hyperparameters here. SPM, the method with the lowest erasure score, includes a semantic distance computation step during generation, which minimizes the effect on prompts that are semantically distant. This may limit its erasure effectiveness. 

In Mode 2, where images are used for fine-tuning, SalUn achieves the highest erasure score. FMN performs the worst, possibly due to its attention decoupling mechanism, which is less suitable for scenarios involving multiple keywords.

\begin{table*}[t!]
\centering
\small
\caption{Results of different methods for erasing overall NSFW in terms of erasure score, image quality, and semantic alignment.There are two sources of NSFW. In Mode 1, one is the concatenation of words in our more keywords version, and the other is the 20 words provided by SLD. For the method of Mode 2, our definition of overall NSFW and a NSFW image dataset composed of 200 images of each theme be used.}
\label{table:nsfw}
\scalebox{0.75}{
\setlength{\tabcolsep}{3pt}
\begin{tabular}{c|c|ccccccccc|c|c|cccccc}
\toprule
                   \textbf{Metric}          & \textbf{Source}      & \textbf{SD-NP}  & \textbf{SLD-Med} & \textbf{SLD-Str} & \textbf{SLD-Max} & \textbf{ESD-u}  & \textbf{ESD-x}  & \textbf{SPM}    & \textbf{UCE}    & \textbf{AU}     &      \textbf{Metric}     & \textbf{Source}                                                              & \textbf{AC}& \textbf{FMN}                     & \textbf{SalUn}                   & \textbf{SelfD}                   & \textbf{MACE}                     \\
\toprule
\multirow{2}{*}{\makecell{Erasure \\ Score ($\uparrow$)}}        & \makecell{ Ours} & 0.66~  & 0.53~   & 0.76~   & \textbf{0.90~}   & 0.12~  & 0.11~  & 0.05~  & 0.30~  & 0.71~  & \multirow{2}{*}{\makecell{Erasure \\ Score ($\uparrow$)}}      & \multirow{2}{*}{\makecell{Ours}} & \multirow{2}{*}{0.26~}  & \multirow{2}{*}{0.01~}  & \multirow{2}{*}{0.52~}  & \multirow{2}{*}{\textbf{0.64~}}  & \multirow{2}{*}{0.36~}   \\
\cline{2-11}

                             & \makecell{SLD}    & 0.42~  & 0.34~   & 0.51~   & \textbf{0.63~}   & 0.05~  & 0.13~  & 0.00~  & -0.04~ & 0.57~  &                            &                                                                      &                         &                         &                         &                         &                          \\
% \cline{3-11} \cline{14-18}
\midrule
\multirow{2}{*}{FID ($\downarrow$)}         & \makecell{Ours} & 26.32~ & 24.02~  & 27.72~  & 33.43~  & \textbf{17.77~} & 18.64~ & 19.40~ & 33.67~ & 22.24~ & \multirow{2}{*}{FID ($\downarrow$)}       & \multirow{2}{*}{\begin{tabular}[c]{@{}c@{}}Ours\end{tabular}} & \multirow{2}{*}{\textbf{19.26~}} & \multirow{2}{*}{18.43~} & \multirow{2}{*}{24.70~} & \multirow{2}{*}{30.01~} & \multirow{2}{*}{51.24~}  \\
\cline{2-11}
                             & \makecell{SLD}    & 24.27~ & 22.60~  & 25.46~  & 29.64~  & 20.08~ & \textbf{18.90~} & 19.22~ & \textbf{}18.47~ & 24.04~ &                            &                                                                      &                         &                         &                         &                         &                          \\
\midrule
\multirow{2}{*}{LPIPS ($\downarrow$)}       & \makecell{Ours} & 0.49~  & 0.48~   & 0.49~   & 0.50~   & \textbf{0.46~}  & 0.47~  & 0.48~  & 0.50~  & 0.48~  & \multirow{2}{*}{LPIPS ($\downarrow$)}     & \multirow{2}{*}{\begin{tabular}[c]{@{}c@{}}Ours\end{tabular}} & \multirow{2}{*}{\textbf{0.47~}}  & \multirow{2}{*}{\textbf{0.47~}}  & \multirow{2}{*}{0.48~}  & \multirow{2}{*}{0.48~}  & \multirow{2}{*}{0.49~}   \\
\cline{2-11}
                             & \makecell{SLD}    & 0.48~  & 0.47~   & 0.48~   & 0.49~   & \textbf{0.46~}  & 0.47~  & 0.48~  & 0.47~  & 0.48~  &                            &                                                                      &                         &                         &                         &                         &                          \\
\midrule
\multirow{2}{*}{CLIPScore ($\uparrow$)}   & \makecell{Ours} & 25.05~ & 25.47~  & 24.66~  & 23.75~  & 24.70~ & 25.11~ & \textbf{26.29~} & 23.58~ & 23.20~ & \multirow{2}{*}{CLIPScore ($\uparrow$)} & \multirow{2}{*}{\begin{tabular}[c]{@{}c@{}}Ours\end{tabular}} & \multirow{2}{*}{26.02~} & \multirow{2}{*}{\textbf{26.16~}} & \multirow{2}{*}{24.64~} & \multirow{2}{*}{24.59~} & \multirow{2}{*}{16.39~}  \\
\cline{2-11}
                             & \makecell{SLD}    & 25.20~ & 25.53~  & 24.80~  & 23.97~  & 24.51~ & 25.17~ & 26.29~ & \textbf{26.43~} & 23.30~ &                            &                                                                      &                         &                         &                         &                         &                          \\
\midrule
\multirow{2}{*}{ImageReward ($\uparrow$)} & \makecell{Ours} & -0.06~ & 0.02~   & -0.11~  & -0.31~  & -0.30~ & -0.17~ & \textbf{0.09~}  & -0.76~ & -0.65~ & \multirow{2}{*}{ImageReward ($\uparrow$)}      & \multirow{2}{*}{\begin{tabular}[c]{@{}c@{}}Ours\end{tabular}} & \multirow{2}{*}{\textbf{0.03~}}  & \multirow{2}{*}{0.00~}  & \multirow{2}{*}{-0.19~} & \multirow{2}{*}{-0.58~} & \multirow{2}{*}{-1.88~}  \\
\cline{2-11}
                             & \makecell{SLD}    & 0.01~  & 0.05~   & -0.05~  & -0.20~  & -0.38~ & -0.14~ & 0.10~  & \textbf{0.17~}  & -0.60~ &                            &                                                                      &                         &                         &                         &                         &          \\
\bottomrule
\end{tabular}
}
\end{table*}


\noindent\textbf{Version Aspect.} In Mode 1, the fewer keywords version and the more keywords version show identical results on the Hateful and Political themes due to the consistent keywords adopted for lack of generalization. For other themes, the erasure scores of the methods related to Mode 1 do not differ significantly between the two versions, indicating that our generalization experiment successfully extracted effective and fewer keywords. UCE is relatively more sensitive to versions, which may be because of its deep involvement in the linear transformation of the attention mechanism in cross-attention, where the target concept is replaced with an empty string, thus making the model more sensitive to changes in its own understanding.

In Mode 2, AC, FMN, and MACE demonstrate similar erasure capabilities regardless of the number of training images. In contrast, SelfD and SalUn show improved erasure scores as the number of images increases. This indicates that different training methods may vary in their ability to understand the images. Some methods reach a saturation point in their image understanding, while others enhance their erasure effectiveness through a deeper comprehension.
\begin{figure*}[t!]
    \centering
    \includegraphics[width=0.75\textwidth]{Images/nudenet.pdf}
    \vspace{-1em}
    \caption{Erasure scores ($\uparrow$) of different methods on the sexually explicit theme, obtained using the NudeNet classifier for body part recognition. Negative scores mean the result after erasing is worse. For methods related to Mode 1, we selected the more keyword version, and for methods related to Mode 2, we selected the 200 images version. A larger erasure score indicates better erasure performance of the method.}
    \label{fig:nudenet}
\end{figure*}

\noindent\textbf{Theme Aspect}. ESD-u and ESD-x are results of fine-tuning the same method on different modules, which is why they show varying levels of erasure ability across different tasks. ESD-u is more focused on tangible erasure targets such as sexually explicit, political, and hateful content, while ESD-x is better suited for erasing content related to atmosphere, such as violent or disturbing content. For the same reason, the Political theme in Mode 1 shows the best overall results, while other themes perform worse or are less effective. In Mode 2, methods require different types of images—some need safe images, others require unsafe images—resulting in varying performance across different themes.


\noindent\textbf{Overall NSFW Erasure}.
In addition to specific NSFW themes, we select the more keywords version from Mode 1 and the 200 images version from Mode 2.combining them to create an overall NSFW keyword set for the erasure experiment.
Since the NSFW keyword set from \cite{patrick2023safe}, which includes "hate, harassment, violence, suffering, humiliation, harm, suicide, sexual, nudity, bodily fluids, blood, obscene gestures, illegal activity, drug use, theft, vandalism, weapons, child abuse, brutality, and cruelty" is widely used in other works, we used this set as the erasure target for comparison in the methods related to Mode 1. From Table \ref{table:nsfw}, we observe that our overall NSFW keyword set performs better, and the effectiveness of each method is similar to the results above.

\subsection{Excessive Erasure}
We use NudeNet to more accurately showcase the best version of each method in each mode, identifying specific body parts and calculating erasure scores. As shown in Figure \ref{fig:nudenet}, SLD-Max remains the best method. FMN shows a counterproductive effect, which is because the excessive number of keywords used for training limits the effectiveness of erasure during attention decoupling, as mentioned above. Based on these results, we evaluate whether each method specifically erases genital body parts. To do this, we calculate Genital Ratio Difference, which is the difference in the ratio of these body parts before and after erasure, relative to all detected parts. A larger Genital Ratio Difference indicates that the method achieved more targeted erasure, which translates to better performance.



% Please add the following required packages to your document preamble:
% \usepackage[table,xcdraw]{xcolor}
% Beamer presentation requires \usepackage{colortbl} instead of \usepackage[table,xcdraw]{xcolor}
\begin{table*}[t!]
\small
\centering
\caption{Experimental results of excessive erasure for different methods}
\label{table:excess}
\scalebox{0.75}{
\setlength{\tabcolsep}{4pt}
\begin{tabular}{c|ccccccccc|ccccc}
\toprule
                         & \multicolumn{1}{c}{\textbf{SD-NP}} & \multicolumn{1}{c}{\textbf{SLD-Med}} & \multicolumn{1}{c}{\textbf{SLD-Str}} & \multicolumn{1}{c}{\textbf{SLD-Max}} & \multicolumn{1}{c}{\textbf{ESD-u}} & \multicolumn{1}{c}{\textbf{ESD-x}} & \multicolumn{1}{c}{\textbf{SPM}} & \multicolumn{1}{c}{\textbf{UCE}} & \multicolumn{1}{c|}{\textbf{AU}} & \textbf{AC}         & \textbf{FMN}        & \textbf{SalUn}      & \textbf{SelfD}      & \textbf{MACE}       \\

\textbf{Metric}                  & \makecell{more \\ keywords}             & \makecell{more \\ keywords}                        & \makecell{more \\ keywords}                        & \makecell{more \\ keywords}                        & \makecell{more \\ keywords}                      & \makecell{more \\ keywords}                      & \makecell{more \\ keywords}                    & \makecell{more \\ keywords}                    & \makecell{more \\ keywords}                   & \makecell{200 \\ images} & \makecell{200 \\ images} & \makecell{200 \\ images} & \makecell{200 \\ images} & \makecell{200 \\ images} \\
\toprule
\makecell{Genital Ratio \\ Difference ($\uparrow$)} & 9.85\%                    & 7.08\%                               & 14.15\%                              & \textbf{19.27\%}                              & -2.52\%                            & 1.42\%                             & -0.68\%                          & 1.17\%                           & 10.87\%                         & 5.62\%     & -2.00\%    & -22.39\%   & 1.47\%     & 1.27\%  \\
\bottomrule
\end{tabular}
}
\end{table*}

\begin{table*}[t!]
\small
\centering
\caption{The erasure scores ($\uparrow$) of different methods on explicit and implicit unsafe prompts across four NSFW datasets, with all methods erasing our custom overall NSFW keyword set.}
\label{tab:implicit}
\scalebox{0.78}{
\setlength{\tabcolsep}{3pt}
\begin{tabular}{c|ccccccccc|ccccc} 
\toprule
    \multirow{2}{*}{\textbf{\makecell{Types}}}   & \textbf{SD-NP}   & \textbf{SLD-Med} & \textbf{SLD-Str} & \textbf{SLD-Max} & \textbf{ESD-u}   & \textbf{ESD-x}   & \textbf{SPM}     & \textbf{UCE}    & \textbf{AU}      & \textbf{AC}      & \textbf{FMN}     & \textbf{SelfD}   & \textbf{SalUn}   & \textbf{MACE}     \\ 

      & \makecell{more \\ keywords}             & \makecell{more \\ keywords}                        & \makecell{more \\ keywords}                        & \makecell{more \\ keywords}                        & \makecell{more \\ keywords}                      & \makecell{more \\ keywords}                      & \makecell{more \\ keywords}                    & \makecell{more \\ keywords}                    & \makecell{more \\ keywords}                   & \makecell{200 \\ images} & \makecell{200 \\ images} & \makecell{200 \\ images} & \makecell{200 \\ images} & \makecell{200 \\ images} \\
    
\toprule
\makecell{Explicit Unsafe \\ Prompts}    & 73.26\% & 55.87\% & 76.96\% & \textbf{95.00\%} & 46.52\% & 42.17\% & 31.30\% & 56.52\% & 80.00\% & 51.30\% & 29.13\% & 68.04\% & 83.26\% & 56.52\%  \\ 

\midrule
  \makecell{Implicit Unsafe \\ Prompts} & 67.72\% & 56.37\% & 78.51\% & \textbf{90.66\%} & 37.16\% & 34.26\% & 29.73\% & 53.50\% & 83.38\% & 41.41\% & 30.13\% & 62.32\% & 70.31\% & 65.83\%  \\
\bottomrule
\end{tabular}
}
\end{table*}

As shown in Table \ref{table:excess}, SLD-Max exhibits the highest value while ESD-u, SPM, FMN, and SalUn show weaker values. We also find that methods related to Mode 1 are more likely to target genital body parts for erasure compared to those related to Mode 2. This may be because Mode 1 directly utilizes the model's inherent textual priors, but Mode 2 requires the model to relearn features corresponding to the sexually explicit theme from the given images.

\subsection{Explicit and Implicit Unsafe Prompts}
% Based on the analysis in Section \ref{Analysis-of-Datasets}, we calculated the erasure score of different methods on explicit and implicit unsafe prompts, as shown in Table \ref{tab:implicit}. 
% We use a classifier to detect whether the generated images are unsafe and report the results of all methods for erasing the overall NSFW theme composed of more keywords.
Section \ref{Analysis-of-Datasets} defines explicit and implicit unsafe prompts. In this subsection, we examine the erasure scores of these prompts across different baselines to evaluate their adaptability to various types of unsafe prompts. Here we use the results of each baseline trained on overall NSFW and show erase scores in Table \ref{tab:implicit}.

SLD-Max achieves the highest erasure score on implicit unsafe prompts, although it is slightly lower than the erasure score for explicit unsafe prompts. We observe that only FMN, AU, and MACE achieve better performance on implicit unsafe prompts compared to explicit unsafe prompts. 
We analyzed this phenomenon and obtained the following inferences. MACE directly uses image masks during training to erase unsafe concepts in pixel space, making it less sensitive to text prompts and more effective on implicit unsafe prompts. AU improves its adaptability to unsafe prompts through adversarial training. FMN exhibits poor overall erasure performance, with similar erasure scores for both explicit and implicit prompts.


% However, FMN demonstrates a low erasure score across both prompt sets. MACE uses image masking during training, and its operations in the pixel space make it less sensitive to text, thus performing well on implicit unsafe prompts. Meanwhile, AU employs adversarial training to erase learned adversarial concepts, which are implicit in nature.
 % while AU and MACE exhibit similar behavior to the conclusions drawn in the robustness analysis, indicating that they may be insensitive to the input prompt. The robustness prompts generally do not contain the target keywords explicitly. 
% Combining the robustness analysis, we can conclude that the current approach of decoupling specific keywords from the concepts to be erased may not be the most reliable strategy.

\subsection{Image Quality}
\label{image-quality}
We conduct a generative image quality evaluation on the COCO-10K dataset for all baselines, including different versions of each method. The results are presented on the left side of Figure \ref{fig:quanlityandalignment}, where methods positioned closer to the lower-left corner indicate better image quality. More specific value results are provided in Appendix \ref{quality-on-all-datasets}. Both ESD-u and ESD-x showed the best image quality, but the image quality of ESD-x is slightly worse than that of ESD-u. This difference might be attributed to ESD-u fine-tuning the unconditional layers, whereas ESD-x fine-tunes the cross-attention layers in the UNet, which may potentially impact its text understanding capability. The two versions of SLD-Max and the 1000 images version of SelfD generate images of very poor quality, which is inversely proportional to their effectiveness in concept erasure. 
% This suggests that while significant semantic guidance during inference is effective for concept erasure, directly manipulating the intermediate states of generated images can substantially degrade image quality.

In Mode 1, using fewer keywords generally results in slightly better image quality. In Mode 2, the number of training images used by methods like FMN, AC, and MACE does not significantly affect image quality. However, as mentioned earlier, different methods have varying capabilities in learning from images. For SelfD and SalUn, using more images results in poorer quality.
We also conduct experiments on the overall NSFW theme, as shown in Table \ref{table:nsfw}. When the methods related to Model 1 use the 20 words provided by SLD, they perform better in terms of image quality. Methods in Mode 2 may have a more significant impact on image quality when erasing the overall NSFW theme. 

Overall, we observe that improving the erasure effect often leads to a loss in image quality, which aligns with common expectations. Striking a balance between these two factors will be a key challenge for future work.


\subsection{Semantic Alignment}
\label{semantic-alignment}
The right side of Figure \ref{fig:quanlityandalignment} illustrates a comparison of the semantic alignment result achieved by various methods on the COCO-10k dataset. And methods closer to the upper-right corner indicate better semantic alignment. The complete results can be found in Appendix \ref{quality-on-all-datasets}. 

The two versions of SPM achieve good alignment, likely because SPM trains an adaptor specifically to handle the target words, with minimal impact on unrelated words. 
Both versions of AU exhibit poor semantic alignment, possibly due to the use of adversarial training, where the model continuously generates and erases adversarial concepts during training. 
However, this adversarial training approach often leads to model collapse, causing significant interference with other concepts.

Methods in Mode 1 exhibit better semantic alignment than those in Mode 2. Additionally, within Mode 1, the version of fewer keywords demonstrates better semantic alignment than the more keywords version. For SelfD and SalUn in Mode 2, increasing the number of training images has a significant impact on semantic alignment. 
Additionally, we observe that semantic alignment and image quality do not always correlate. For example, AU generates images of medium quality, yet its semantic alignment is poor, highlighting the importance of evaluating methods from multiple perspectives.


\subsection{Robustness}
Table \ref{tab:robustness} compares the performance of different methods on the RAB dataset and four NSFW datasets (4-NSFW). Methods in Mode 1 use the more keywords version, while methods in Mode 2 use the version with 200 training images. If the baseline performs better on RAB than on the four NSFW datasets, we consider the baseline to be more robust.
Except for SLD-Max, AU, and MACE, most methods perform poorly on robustness prompts. This suggests that many existing erasure methods only separate relevant words from the target theme and do not fully eliminate theme from the model. Although SLD-Max extensively guides semantics in the noise space, thereby mitigating the impact of toxic prompts, its general performance has been compromised. AU targets adversarial prompts for erasure; and may have overfitted to these prompts, leading to better erasure results on the RAB dataset compared to general NSFW prompts. Meanwhile, MACE also performs well on the RAB dataset, likely because it uses masks for training and operates directly in the pixel space, making it less sensitive to text prompts. Across different modes, there is no significant difference in robustness. 
% While intuitively, methods that use images may effectively combat toxic text, they often still rely on keywords during training, which means they also separate the relevant words within the model.




\subsection{Overall Analysis}
\label{effect}
In the previous subsections, we thoroughly analyze the performance of all erasure methods across various evaluation perspectives. To derive a comprehensive conclusion, we average the results of different versions and tasks for each baseline on the same metric to obtain its final performance. We then categorize the methods into three levels based on their performance: the top three performing baselines are assigned to level 1, the bottom three to level 3, and the remaining methods to level 2. The final results are summarized in Table \ref{tab:comprehensive-analysis}.

Our findings indicate that no baseline excels across all evaluation perspectives, with each method having its own limitations. Overall, SLD-Str and UCE are relatively stable, as they effectively reduce the generation of target concepts while maintaining image quality and semantic alignment. Their performance remains strong even in challenging scenarios, such as robustness datasets and implicit unsafe prompts. While SLD-Max performs well in erasure score, robustness, and excessive erasure, it suffers from poor image quality and semantic alignment. In contrast, methods like ESD-u, ESD-x, FMN, SPM, and AC show weak performance in erasure scores. SD-NP, SLD-Med, and SalUn deliver moderate results across all metrics. AU and MACE excel mainly in robustness and implicit unsafe prompt erasure. Finally, SelfD has the poorest overall performance across all metrics.

\begin{table*}[t!]
\centering
\small
\caption{Comprehensive comparison of different methods across various evaluation metrics. Based on the average results from different versions of each method, the methods are ranked and categorized into three levels:  
$\blackcircle$ represents Level 1 (best performance),  
halfcircle represents Level 2 (moderate performance), and 
whitecircle represents Level 3 (poorest performance).}
\label{tab:comprehensive-analysis}
\scalebox{0.75}{
\setlength{\tabcolsep}{4pt} 
\begin{tabular}{cccccccccccccccc}
\toprule
\textbf{Perspective}                & \textbf{Metric}          & \textbf{SD-NP} & \textbf{SLD-Med} & \textbf{SLD-Str} & \textbf{SLD-Max} & \textbf{ESD-u} & \textbf{ESD-x} & \textbf{SPM} & \textbf{UCE} & \textbf{AU} & \textbf{AC} & \textbf{FMN} & \textbf{SelfD} & \textbf{SalUn} & \textbf{MACE}  \\
\toprule
Erasure Proportion                  & Erasure Score            & $\blackcircle$            & $\halfcircle$                 & $\blackcircle$                 & $\blackcircle$                 & $\whitecircle$               & $\halfcircle$               & $\whitecircle$             & $\halfcircle$             & $\halfcircle$            & $\halfcircle$            & $\whitecircle$             & $\halfcircle$               & $\halfcircle$               & $\halfcircle$               \\
\midrule
Excessive Erasure                   & \makecell{Genital Ratio \\ Difference} & $\halfcircle$            & $\halfcircle$                 & $\blackcircle$                 & $\blackcircle$                 & $\whitecircle$               & $\halfcircle$               & $\halfcircle$             & $\halfcircle$             & $\blackcircle$            & $\halfcircle$            & $\whitecircle$             & $\whitecircle$               & $\halfcircle$               & $\halfcircle$               \\
\midrule
\makecell{Impact of Explicit and \\ Implicit Unsafe Prompts}                    & Erasure Score            & $\halfcircle$            & $\halfcircle$                 & $\blackcircle$                 & $\blackcircle$                 & $\halfcircle$               & $\whitecircle$               & $\whitecircle$             & $\halfcircle$             & $\blackcircle$            & $\halfcircle$            & $\whitecircle$             & $\halfcircle$               & $\halfcircle$               & $\halfcircle$               \\
\midrule
\multirow{2}{*}{Image Quality}      & FID                      & $\halfcircle$            & $\halfcircle$                 & $\whitecircle$                 & $\whitecircle$                 & $\blackcircle$               & $\halfcircle$               & $\blackcircle$             & $\halfcircle$             & $\halfcircle$            & $\halfcircle$            & $\blackcircle$             & $\whitecircle$               & $\halfcircle$               & $\halfcircle$               \\
                                    & LPIPS                    & $\whitecircle$            & $\halfcircle$                 & $\halfcircle$                 & $\whitecircle$                 & $\blackcircle$               & $\halfcircle$               & $\blackcircle$             & $\halfcircle$             & $\halfcircle$            & $\blackcircle$            & $\halfcircle$             & $\whitecircle$               & $\halfcircle$               & $\halfcircle$               \\
\midrule
\multirow{2}{*}{Semantic Alignment} & CLIPScore                & $\halfcircle$            & $\halfcircle$                 & $\halfcircle$                 & $\whitecircle$                 & $\halfcircle$               & $\halfcircle$               & $\halfcircle$             & $\blackcircle$             & $\whitecircle$            & $\blackcircle$            & $\blackcircle$             & $\halfcircle$               & $\halfcircle$               & $\whitecircle$               \\
                                    & ImageReward              & $\halfcircle$            & $\halfcircle$                 & $\halfcircle$                 & $\halfcircle$                 & $\halfcircle$               & $\halfcircle$               & $\blackcircle$             & $\blackcircle$             & $\whitecircle$            & $\halfcircle$            & $\blackcircle$             & $\whitecircle$               & $\halfcircle$               & $\whitecircle$               \\
\midrule
Robustness                          & Erasure Score            & $\halfcircle$            & $\halfcircle$                 & $\halfcircle$                 & $\blackcircle$                 & $\halfcircle$               & $\halfcircle$               & $\whitecircle$             & $\halfcircle$             & $\blackcircle$            & $\halfcircle$            & $\whitecircle$             & $\whitecircle$               & $\halfcircle$               & $\blackcircle$      \\
\bottomrule
\end{tabular}
}
\end{table*}

\begin{figure*}[t!]
    \centering
    \includegraphics[width=0.9\textwidth]{Images/quanlityandalignment.pdf}
    \vspace{-1em}
    \caption{Different versions of different methods generate images using the COCO-10k dataset. We use FID and LPIPS to calculate image quality, and CLIP score and Image Reward to calculate semantic alignment. For image quality, smaller values of the metrics are better, while for semantic alignment, larger values are better.}
    \label{fig:quanlityandalignment}
\end{figure*}

\begin{table*}[t!]
\small
\centering
\caption{The erasure scores ($\uparrow$) of different methods for Sexually Explicit and Violent themes on the RAB dataset and four NSFW datasets. Methods related to Mode 1 use the more keywords version, while methods related to Mode 2 use the 200 images version}
\label{tab:robustness}
\scalebox{0.78}{
\setlength{\tabcolsep}{3pt} 
\begin{tabular}{c|c|ccccccccc|ccccc} 
\toprule
\multirow{2}{*}{\textbf{\makecell{\\Theme}}}                            & \multirow{2}{*}{\textbf{\makecell{\\Dataset}}}    & \textbf{SD-NP}      & \textbf{SLD-Med} & \textbf{SLD-Str} & \textbf{SLD-Max} & \textbf{ESD-u}   & \textbf{ESD-x}   & \textbf{SPM}     & \textbf{UCE}     & \textbf{AU}      & \textbf{AC}      & \textbf{FMN}      & \textbf{SelfD}   & \textbf{SalUn}   & \textbf{MACE}     \\ 
&                & \makecell{more \\ keywords}             & \makecell{more \\ keywords}                        & \makecell{more \\ keywords}                        & \makecell{more \\ keywords}                        & \makecell{more \\ keywords}                      & \makecell{more \\ keywords}                      & \makecell{more \\ keywords}                    & \makecell{more \\ keywords}                    & \makecell{more \\ keywords}                   & \makecell{200 \\ images} & \makecell{200 \\ images} & \makecell{200 \\ images} & \makecell{200 \\ images} & \makecell{200 \\ images} \\
\midrule
\multirow{2}{*}{\makecell{ Sexually \\ Explicit}} & RAB        & 6.02\%  & 2.97\%  & 51.45\% & \textbf{93.85\%} & 44.33\% & 10.86\% & 1.31\%  & 49.45\% & 92.05\% & 4.01\%  & 0.00\%   & 9.61\%  & 73.72\% & 91.91\%  \\ 
\cline{2-16}
                                 & \makecell{4-NSFW} & 83.02\% & 64.95\% & 90.56\% & \textbf{96.76\%} & 68.77\% & 47.86\% & 8.29\%  & 66.34\% & 72.77\% & 52.49\% & -10.89\% & 74.10\% & 82.68\% & 64.02\%  \\ 
\midrule
\multirow{2}{*}{\makecell{\\ Violent}}         & RAB        & 51.00\% & 51.96\% & 89.44\% & \textbf{99.96\%} & 14.92\% & 18.41\% & 7.07\%  & 48.52\% & 99.69\% & 39.01\% & 8.38\%   & 12.43\% & 54.19\% & 91.71\%  \\ 
\cline{2-16}
                                 & \makecell{4-NSFW} & 92.86\% & 87.00\% & 96.57\% & \textbf{99.00\%} & 14.43\% & 41.29\% & 12.29\% & 62.86\% & 94.71\% & 78.57\% & 18.57\%  & 58.29\% & 77.43\% & 66.71\%  \\ 
\bottomrule
\end{tabular}
}
\end{table*}





% We present here a comprehensive analysis of the performance of all erasure methods from various evaluation perspectives.  For each method, we average the results of different versions on the same metric to obtain its final performance on that metric. We roughly categorize the ratings for each metric into three levels: the top three performing methods are assigned to Level 1, the bottom three methods to Level 3, and the remaining methods to Level 2. The final results are shown in Table \ref{tab:comprehensive-analysis}.  We find that none of the methods performs well across all evaluation perspectives, and each has its own limitations.  On the whole, SLD-Str and UCE are relatively stable, as they can reduce the generation of target concepts while maintaining image quality and semantic alignment. Furthermore, their performance does not degrade significantly when faced with challenging scenarios, such as robustness datasets and implicit unsafe prompts.   While SLD-Max performs well in terms of erasure score, robustness, and excessive erasure, it exhibits poor image quality and semantic alignment. In contrast, ESD-u, ESD-x, FMN, SPM, and AC show weaker performance in terms of erasure score. SD-NP, SLD-Med, and SalUn perform at an intermediate level across all metrics, while AU and MACE perform well only in robustness and implicit unsafe prompt erasure. Finally, SelfD has the poorest average performance across all metrics. 




% \begin{table*}[ht]
% \centering
% \small
% \caption{Comprehensive comparison of different methods across various evaluation metrics. Based on the average results from different versions of each method, the methods are ranked and categorized into three levels: $\blackcircle$ represents Level 1 (best performance), $\halfcircle$ represents Level 2 (moderate performance), and $\whitecircle$ represents Level 3 (poorest performance).}
% \label{tab:comprehensive-analysis}
% \scalebox{0.9}{\begin{tabular}{ccccccccccccccc} 
% \toprule
% Perspective        & SD-NP & SLD-Med & SLD-Str & SLD-Max & ESD-u & ESD-x & SPM & UCE & AU & AC & FMN & SelfD & SalUn & MACE  \\ 
% \midrule
% Erasure Proportion & $\blackcircle$   & $\halfcircle$       & $\blackcircle$        & $\blackcircle$        & $\whitecircle$     & $\halfcircle$     & $\whitecircle$   & $\halfcircle$   & $\halfcircle$  & $\halfcircle$  & $\whitecircle$   & $\halfcircle$     & $\halfcircle$     & $\halfcircle$     \\ 
% \midrule
% Excessive Erasure  & $\halfcircle$  & $\halfcircle$       & $\blackcircle$        & $\blackcircle$        & $\whitecircle$     & $\halfcircle$     & $\halfcircle$   & $\halfcircle$   & $\blackcircle$   & $\halfcircle$  & $\whitecircle$   & $\whitecircle$     & $\halfcircle$     & $\halfcircle$     \\ 
% \midrule
% FID                & $\halfcircle$  & $\halfcircle$       & $\whitecircle$       & $\whitecircle$       & $\blackcircle$      & $\halfcircle$     & $\blackcircle$    & $\halfcircle$   & $\halfcircle$  & $\halfcircle$  & $\blackcircle$    & $\whitecircle$     & $\halfcircle$     & $\halfcircle$     \\ 

% LPIPS              & $\whitecircle$  & $\halfcircle$       & $\halfcircle$       & $\whitecircle$       & $\blackcircle$      & $\halfcircle$     & $\blackcircle$    & $\halfcircle$   & $\halfcircle$  & $\blackcircle$   & $\halfcircle$   & $\whitecircle$     & $\halfcircle$     & $\halfcircle$     \\ 
% \midrule
% CLIPScore          & $\halfcircle$  & $\halfcircle$       & $\halfcircle$       & $\whitecircle$       & $\halfcircle$     & $\halfcircle$     & $\halfcircle$   & $\blackcircle$    & $\whitecircle$  & $\blackcircle$   & $\blackcircle$    & $\halfcircle$     & $\halfcircle$     & $\whitecircle$     \\ 

% ImageReward        & $\halfcircle$  & $\halfcircle$       & $\halfcircle$       & $\halfcircle$       & $\halfcircle$     & $\halfcircle$     & $\blackcircle$    & $\blackcircle$    & $\whitecircle$  & $\halfcircle$  & $\blackcircle$    & $\whitecircle$     & $\halfcircle$     & $\whitecircle$     \\ 
% \midrule
% Robustness         & $\halfcircle$  & $\halfcircle$       & $\halfcircle$       & $\blackcircle$        & $\halfcircle$     & $\halfcircle$     & $\whitecircle$   & $\halfcircle$   & $\blackcircle$   & $\halfcircle$  & $\whitecircle$   & $\whitecircle$     & $\halfcircle$     & $\blackcircle$      \\ 
% \midrule
% \makecell{Implicit Unsafe \\ Prompt Erasure}   & $\halfcircle$  & $\halfcircle$       & $\blackcircle$        & $\blackcircle$        & $\halfcircle$     & $\whitecircle$     & $\whitecircle$   & $\halfcircle$   & $\blackcircle$   & $\halfcircle$  & $\whitecircle$   & $\halfcircle$     & $\halfcircle$     & $\halfcircle$     \\
% \midrule
% \end{tabular}
% }
% \end{table*}



% \usepackage{multirow}


\section{Discussion}
\label{sec:discussion}
\subsection{Challenges in current SIRR research}
One of the biggest challenges in SIRR research is the lack of large, high-quality training datasets that represent a variety of reflection types across different surfaces and lighting conditions. Reflection removal relies on supervised learning, which requires a well-labeled dataset with clear ground-truth images for training. However, creating or collecting such datasets is both time- and labor-consuming. Moreover, the absence of suitable test sets with real-world reflection scenarios presents another challenge. These test sets should include not only high-quality images but also a wide range of reflective surfaces, lighting conditions, and material properties (e.g., building glass, car window, smooth metal surface, rough metal surface, etc.). Without such comprehensive datasets, model evaluation remains limited and often unreliable when deploying into the real world.

Furthermore, the lack of datasets in SIRR research has also made exploring complex network architectures less valuable. With limited data, simpler models like UNet is already achieving state-of-the-art results, making the development of more complex models unnecessary and slowing progress in the evolution of SIRR network designs. These intricate architectures are prone to overfitting on small datasets, preventing them from reaching their full potential. Consequently, academic research in this field is stagnating, resulting fewer innovations in network design for SIRR.

In addition, the inherent complexity of the SIRR task itself compounds these challenges. Reflections vary in type, intensity, and interaction with the scene—some completely obscure the background, turning the task into a form of image inpainting. Such a wide variety makes it difficult to clearly define what a SIRR task should involve: is it purely about removing reflections, or does it also need to reconstruct missing background details? The lack of a comprehensive task definition hinders the development of consistent methodologies, reliable evaluation metrics, and meaningful comparisons. To make real progress, the academic community must come up with a clearer and more inclusive task definition that better captures the complexity of reflection removal and develop specific guidelines for handling different reflection scenarios.

\subsection{Future Directions}
One of the most pressing future directions in SIRR research is the creation of large-scale, high-quality datasets that cover a wide variety of reflection types, surfaces, and lighting conditions. Efforts should also focus on curating test sets that contain not only clean ground-truth images without reflections, but also cover real-world reflective materials (e.g., different types of glass and metals) and various environmental factors (e.g., lighting variations and reflections in challenging environments). Such test sets will enable more accurate training and better evaluation metrics, ultimately improving the generalization capabilities of SIRR models. To address the data scarcity issues, we appeal for collaborations between research institutions, industry, and the development of more powerful synthetic data generation methods.

On the other hand, future SIRR development could greatly benefit from the integration of advanced AIGC models. By utilizing large vision foundation models, researchers can enhance scene understanding and improve semantic reasoning, while large language models can provide deeper, descriptive insights into scene content and the relationships between reflections and transmissions.


Additionally, combining multimodal information (e.g., text descriptions, depth information, and semantic segmentation) will strengthen the reflection-transmission separation by leveraging complementary insights from multimodal resources. We believe that this fusion could result in more context-aware and precise reflection removal systems.

As mentioned earlier, clarifying the definition of SIRR task is another critical direction. The field needs to precisely define whether the goal is limited to reflection removal or extends to background reconstruction in cases of severe reflections. Establishing these clear boundaries will enable standardized evaluations and fair comparisons, ultimately leading to the development of more effective solutions.

\subsection{Limitations}
Our work has several limitations. First, due to the keywords and databases chosen for the search query, some related research may have been missed. In addition, research that is not published in English, exceeds the prescribed time frame, or does not provide enough technical information, were not included. Nevertheless, our paper provides a comprehensive analysis of the existing literature, based on a sample of 28 papers sourced from key venues. The aim of our review is to provide readers with a clear and rapid understanding of the key developments in SIRR field, including its current state, challenges, and future directions. Second, we did not include benchmark results in this paper, as different methods and datasets often have their own unique training strategies and evaluation process. To address this, we will develop a unified evaluation framework in the future, which will provide a fair platform to compare all publicly available datasets.


\section{Conclusions}
In this paper, we provide a detailed analysis of evaluation tools, categorize concept erasure methods, and assess their performance across diverse datasets. Additionally, we divide NSFW content into six distinct topics and evaluate the methods from six novel perspectives. Finally, we offer empirical insights and methodological recommendations to the community based on the experimental findings.


% \begin{abstract}
Recent advancements in 3D multi-object tracking (3D MOT) have predominantly relied on tracking-by-detection pipelines. However, these approaches often neglect potential enhancements in 3D detection processes, leading to high false positives (FP), missed detections (FN), and identity switches (IDS), particularly in challenging scenarios such as crowded scenes, small-object configurations, and adverse weather conditions. Furthermore, limitations in data preprocessing, association mechanisms, motion modeling, and life-cycle management hinder overall tracking robustness. To address these issues, we present \textbf{Easy-Poly}, a real-time, filter-based 3D MOT framework for multiple object categories. Our contributions include: (1) An \textit{Augmented Proposal Generator} utilizing multi-modal data augmentation and refined SpConv operations, significantly improving mAP and NDS on nuScenes; (2) A \textbf{Dynamic Track-Oriented (DTO)} data association algorithm that effectively manages uncertainties and occlusions through optimal assignment and multiple hypothesis handling; (3) A \textbf{Dynamic Motion Modeling (DMM)} incorporating a confidence-weighted Kalman filter and adaptive noise covariances, enhancing MOTA and AMOTA in challenging conditions; and (4) An extended life-cycle management system with adjustive thresholds to reduce ID switches and false terminations. Experimental results show that Easy-Poly outperforms state-of-the-art methods such as Poly-MOT and Fast-Poly~\cite{li2024fast}, achieving notable gains in mAP (e.g., from 63.30\% to 64.96\% with LargeKernel3D) and AMOTA (e.g., from 73.1\% to 74.5\%), while also running in real-time. These findings highlight Easy-Poly's adaptability and robustness in diverse scenarios, making it a compelling choice for autonomous driving and related 3D MOT applications. The source code of this paper will be published upon acceptance.

% 3D Multi-Object Tracking (MOT) is essential for autonomous driving systems, contributing significantly to vehicle safety and navigation. Despite recent advancements, existing 3D tracking methods still face significant challenges in accuracy, particularly when dealing with small objects, crowded environments, and adverse weather conditions. To overcome these challenges, we propose \textbf{Easy-Poly}, a novel and efficient multi-modal 3D MOT framework. \textbf{Easy-Poly} employs the Focal Sparse Convolution (\textbf{FocalsConv}) model for object detection. By optimizing convolution operations and augmenting data with multiple modalities, we significantly enhance detection precision.
% \textbf{Easy-Poly} introduces several key innovations: (1) an optimized Kalman filter in the pre-processing stage, (2) integration of the Dynamic Track-Oriented (\textbf{DTO}) Data Association algorithm with confidence-weighted motion models for data association, (3) Dynamic Motion Modeling (\textbf{DMM}) with Adaptive Noise Covariances, and (4) enhanced trajectory management throughout the tracking life-cycle. These improvements increase the robustness and efficiency of tracking, especially in complex scenarios such as crowded scenes and challenging weather conditions. Experimental results on the \textbf{nuScenes} dataset demonstrate that in the pre-processing stage of \textbf{Easy-Poly}, the optimized \textbf{FocalsConv} model achieves a mean Average Precision (mAP) of \textbf{64.96\%} for object detection. Furthermore, the multi-object tracking performance reaches a high AMOTA of \textbf{75.0\%}, surpassing existing methods across multiple performance metrics.
 
% Code and data are available at \textcolor{blue}{\textit{\url{https://github.com/zhangpengtom/FocalsConvPlus}}} and  \textcolor{blue}
%  \textit{\url{https://github.com/zhangpengtom/EasyPoly}.}
%  } 

\end{abstract}
% \section{Introduction}

Deep Reinforcement Learning (DRL) has emerged as a transformative paradigm for solving complex sequential decision-making problems. By enabling autonomous agents to interact with an environment, receive feedback in the form of rewards, and iteratively refine their policies, DRL has demonstrated remarkable success across a diverse range of domains including games (\eg Atari~\citep{mnih2013playing,kaiser2020model}, Go~\citep{silver2018general,silver2017mastering}, and StarCraft II~\citep{vinyals2019grandmaster,vinyals2017starcraft}), robotics~\citep{kalashnikov2018scalable}, communication networks~\citep{feriani2021single}, and finance~\citep{liu2024dynamic}. These successes underscore DRL's capability to surpass traditional rule-based systems, particularly in high-dimensional and dynamically evolving environments.

Despite these advances, a fundamental challenge remains: DRL agents typically rely on deep neural networks, which operate as black-box models, obscuring the rationale behind their decision-making processes. This opacity poses significant barriers to adoption in safety-critical and high-stakes applications, where interpretability is crucial for trust, compliance, and debugging. The lack of transparency in DRL can lead to unreliable decision-making, rendering it unsuitable for domains where explainability is a prerequisite, such as healthcare, autonomous driving, and financial risk assessment.

To address these concerns, the field of Explainable Deep Reinforcement Learning (XRL) has emerged, aiming to develop techniques that enhance the interpretability of DRL policies. XRL seeks to provide insights into an agent’s decision-making process, enabling researchers, practitioners, and end-users to understand, validate, and refine learned policies. By facilitating greater transparency, XRL contributes to the development of safer, more robust, and ethically aligned AI systems.

Furthermore, the increasing integration of Reinforcement Learning (RL) with Large Language Models (LLMs) has placed RL at the forefront of natural language processing (NLP) advancements. Methods such as Reinforcement Learning from Human Feedback (RLHF)~\citep{bai2022training,ouyang2022training} have become essential for aligning LLM outputs with human preferences and ethical guidelines. By treating language generation as a sequential decision-making process, RL-based fine-tuning enables LLMs to optimize for attributes such as factual accuracy, coherence, and user satisfaction, surpassing conventional supervised learning techniques. However, the application of RL in LLM alignment further amplifies the explainability challenge, as the complex interactions between RL updates and neural representations remain poorly understood.

This survey provides a systematic review of explainability methods in DRL, with a particular focus on their integration with LLMs and human-in-the-loop systems. We first introduce fundamental RL concepts and highlight key advances in DRL. We then categorize and analyze existing explanation techniques, encompassing feature-level, state-level, dataset-level, and model-level approaches. Additionally, we discuss methods for evaluating XRL techniques, considering both qualitative and quantitative assessment criteria. Finally, we explore real-world applications of XRL, including policy refinement, adversarial attack mitigation, and emerging challenges in ensuring interpretability in modern AI systems. Through this survey, we aim to provide a comprehensive perspective on the current state of XRL and outline future research directions to advance the development of interpretable and trustworthy DRL models.
% %!TEX root = Article.tex

% Begin of file 2-Preliminaries.tex

\section{Foundations}
\label{sec:prl}

In this section we present some material that we will need in the subsequent
sections, and define a data model that consists of common aspects of RDF and
Property Graphs.


\subsection{A Common Data Model}

When developing a common framework for SHACL, ShEx, and PG-Schema, the first
challenge is establishing  a \emph{common data model}, since SHACL and ShEx work
on RDF, whereas PG-Schema works on Property Graphs.
Rather than using a model that generalises  both RDF and Property Graphs, we
propose a simple model, called \emph{common graphs}, which we obtained by asking
what, fundamentally, are the \emph{common aspects} of RDF and Property Graphs
(Appendix~\ref{sec:appendix-foundations} gives more details on the distilling of
common graphs).

Let us assume disjoint countable sets of nodes $\Nodes$, values $\Values$,
predicates $\Predicates$, and keys $\Keys$ (sometimes called properties).

% We sometimes say \emph{element} for a node or a value, and \emph{label} for a predicate or key. \todo{Drop if not used.}

\begin{definition}
  A \emph{common graph} is a pair $\graph = (E, \rho)$ where
  \begin{itemize}[\textbullet]
  \item
    $E \subseteq_{\mathit{fin}} \Nodes \times \Predicates \times \Nodes$ is its
    set of edges (which carry predicates), and
  \item
    $\rho \colon \Nodes \times \Keys \pto \Values$ is a finite-domain partial
    function mapping node-key pairs to values.
  \end{itemize}
  The set of nodes of a common graph $\graph$, written $\nodes(\graph)$,
  consists of all elements of $\Nodes$ that occur in $E$ or in the domain of
  $\rho$.
  Similarly, $\keys(\graph)$ is the subset of $\Keys$ that is used in $\rho$,
  and $\values(\graph)$ is the subset of $\Values$ that is used in $\rho$ (that
  is, the range of $\rho$).
\end{definition}

% \begin{example}[Media Service Common Graph] \label{ex:sharedScenario}
% To illustrate the common graphs, we introduce the following scenario. We assume a data model that has users, who can access and own accounts and invite other users to their accounts. Users have keys, such as email and credit-card. An example for this can be seen in~\Cref{fig}.
% % The nodes correspond to  conceptual classes, which will be identified by their available properties and keys. Properties are depicted as directed arrows, and keys are shown inside the conceptual classes.
% % The boxes inform about the available categories of nodes, with the keys they may have available (such as the key $\Exkey{plan}$ for nodes of category ``Account''), and properties connect nodes via directed arrows (such as $\Exprop{buyer}$, which connects nodes of category ``Sale'' and ``Account'').
% \end{example}

\begin{example}
  \label{ex:common-graph}
  Consider Figure~\ref{fig:common-graph}, containing a graph to store
  information about \emph{users} who may have access to (possibly multiple)
  \emph{accounts} in, \eg, a media streaming service.
  In this example, we have six nodes describing four persons ($u_1,...,u_4$) and
  two accounts ($a_1$, $a_2$).
  As a common graph $\graph = (E, \rho)$, the nodes are $a_1$, $u_1$, etc.
  Examples of edges in $E$ are $(u_2, \exaccess,a_1)$ and $(u_3, \exinvited,
  u_2)$.
  Furthermore, we have $\rho(u_2, \exemail) =$ d@d.d and $\rho(a_1,card) =
  1234$.
  So, $E$ captures the arrows in the figure (labelled with predicates) and
  $\rho$ captures the key/value information for each node.
  %
% Moreover, 3 predicates are used, appearing in Figure~\ref{fig:common-graph} as labels on links between nodes, representing the relation~$E$. Nodes are further associated with some key-value pairs, representing the function $\rho$.
  %
  Notice that a person may be the owner of an account, and may potentially have
  access to other accounts.
  This is captured using the predicates $\exowns$ and $\exaccess$, respectively.
  In addition, the system implements an invitation functionality, where users
  may invite other people to join the platform.
  The previous invitations are recorded using the predicate $\exinvited$.
  Both accounts and users may be privileged, which is stored via a Boolean value
  of the key~$\exprivileged$.
  We note that the presence of the key $\exemail$ (\resp, of the key (credit)
  $\excard$) is associated with, and indeed identifies users (\resp, accounts).
\end{example}

% \todo[inline]{In the example, worth noting that the graph node names are names, and not identities. Maybe it would be better to name them A, B, C, D to avoid misunderstanding?}

\begin{figure}[t]
\resizebox{1\linewidth}{!}{
  \includegraphics{example-common.pdf}
}
\Description{A diagram of the user common graph.}
\caption{The media service common graph. }
\label{fig:common-graph}
\end{figure}

It is easy to see that every common graph is a property graph (as per the formal
definition of property graphs~\cite{ABDF23}).
A common graph can also be seen as a set of triples, as in RDF.
Let
\[
  \Triples
=
  \left( \Nodes \times \Predicates \times \Nodes \right)
\;\cup\;
  \left( \Nodes \times \Keys \times \Values \right)\,.
\]
Then, a common graph can be seen as a finite set $\graph \subseteq \Triples$
such that for each $u \in \Nodes$ and $k \in \Keys$ there is at most one
$v \in \Values$ such that $(u, k, v) \in \graph$.
Indeed, a common graph $(E, \rho)$ corresponds to
\[
  E \;\cup\; \{ (u, k, v) \mid \rho(u,k) = v\}\;.
\]
When we write $\rho(u, k) = v$ we assume that $\rho$ is defined on $(u, k)$.

\medskip

\noindent\emph{Throughout the paper we see property graph $\graph$
simultaneously as a pair $(E, \rho)$ and as a set of triples from $\Triples$,
switching between these perspectives depending on what is most convenient at a
given moment.}


\subsection{Node Contents and  Neighbourhoods}

Let $\Records$ be the set of all \emph{records}, \ie, finite-domain partial
functions $r \colon \Keys \pto \Values$.
We write records as sets of pairs $\left\{ (k_1, w_1), \dots (k_n, w_n)
\right\}$ where $k_1, \dots, k_n$ are all different, meaning that $k_i$ is
mapped to $w_i$.

For a common graph $\graph = (E,\rho)$ and node $v$ in $\graph$, by a slight
abuse of notation we write $\rho(v)$ for the record $\left\{ (k, w) \mid
\rho(v,k) = w \right\}$ that collects all key-value pairs associated with node
$v$ in $\graph$.
We call $\rho(v)$ the \emph{content} of node $v$ in $\graph$.
This is how PG-Schema interprets common graphs: it views key-value pairs in
$\rho(v)$ as \emph{properties} of the node $v$, rather than independent,
navigable objects in the graph.

SHACL and ShEx, on the other hand, view common graphs as sets of triples and
make little distinction between keys and predicates.
The following notion---when applied to a node---uniformly captures the local
context of this node from that perspective: the content of the node and all
edges incident with the node.

%\begin{definition}[Neighbourhood]
%Given a common graph $\graph = (E,\rho)$ and a node $v\in\Nodes$, we write $\neigh_\graph(v)$ for the common graph $(E',\rho')$ where $E' = \left \{ (u_1,p,u_2) \in  E \mid u_1 = v \text{ or } u_2 = v\right\}$ and $\rho'$ is obtained by restricting $\rho$ so that $\rho'(v) = \rho(v)$ and $\rho'(u)$ is empty for all $u\neq v$. Similarly, for $w\in\Values$, we let $\neigh_\graph(w)$ be the common graph $(\emptyset,\rho')$ where $\rho'(u) = \left\{(k,w')\in\rho(u)\mid w'=w\right\}$ for all $u\in\Nodes$.
%Given a common graph $\graph$ and a node or value $v\in\Nodes\cup\Values$, the \emph{neighbourhood of $v$ in $\graph$}, written $\neigh_\graph(v)$, is the common graph consisting of triples $(u_1, p, u_2)$ from $\graph$ such that $p\in\Predicates\cup\Keys$ and either $u_1=v$ or $u_2=v$.
%\end{definition}

%That is, for $v\in\Nodes$,  $\neigh_\graph(v)$ is a star-shaped graph where only the central node has non-empty content.  For $w\in\Values$, $\neigh_\graph(w)$ is a graph with no edges and only a single value occurring in the contents of nodes.

%If we view common graphs as sets of triples, $\neigh_\graph(v)$ for $v\in\Nodes\cup\Values$ is simply the set of all triples from $\graph$ that mention $v$.

%We will also use the notion of \emph{partial neighbourhoods}, where only specified subsets of keys and predicates are taken into account.

%It is easiest to define it seeing common graphs as sets of triples.

\begin{definition}[Neighbourhood]
  Given a common graph $\graph$ and a node or value $v \in \Nodes \cup \Values$,
  the \emph{neighbourhood} of $v$ in $\graph$ is $\neigh_\graph(v) = \left\{
  (u_1, p, u_2) \in \graph \mid u_1 = v \text{ or } u_2 = v \right\}$.
  %
% \todo[inline]{Wim: This is ill-defined. We do say before that a common graph can be viewed as a set of triples if we want to think about it as RDF. But this definition should also apply to the PG view. We should be clearer about what we mean with the key/value pairs and only use ingredients from Def 1. In fact, if we take the RDF view, the definition is inconsistent with text below that says that, if $v$ is a value, then the neighborhood has no edges.}
% \todo[inline]{Suggestion to rephrase: introduce $\graph = (E,\rho)$ and say $\neigh_\graph(v) = \{(u_1,p,u_2) \in E \mid ... \} \cup \{???\}$ (Actually I don't understand yet what we want wrt $\rho$.)}
% \todo[inline]{Filip: In many places in the paper we treat $\graph$ as a pair $(E,\rho)$ or as a subset of $\Triples$, whatever is more convenient. It should suffice to warn the reader that we do this. We could write the definition in terms of $(E,\rho)$, but it would be clumsy. I really think it is fine as written.  On the other hand, if this is not helping, we can probably just skip this definition entirely and introduce only the $\pm$ variant of neighbourhoods in the section on ShEx.}
% \todo[inline]{Wim: OK, I understand better now what's intended and clarified below.}
\end{definition}

\todo{JH: Is this actually used anywhere?}

When $v \in \Nodes$, then $\neigh_\graph(v)$ is a star-shaped graph
where only the central node has non-empty content.
When $v \in \Values$, then $\neigh_\graph(v)$ consists of all the nodes in
$\graph$ that have some key with value $v$, which is a common graph with no
edges and a restricted function $\rho$.

%\todo[inline]{Maybe move to respective sections. Could also save space.}


\subsection{Value Types}

We assume an enumerable set of \emph{value types} $\ValueTypes$.
The reader should think of value types as \texttt{integer}, \texttt{boolean},
\texttt{date}, \etc
Formally, for each value type $\vtype \in \ValueTypes$, we assume that there is
a set $\sem{\vtype} \subseteq \Values$ of all values of that type and that each
value $v \in \Values$ belongs to some type, \ie, there is at least one $\vtype
\in \ValueTypes$ such that $v \in \sem{\vtype}$.
Finally, we assume that there is a type $\any \in \ValueTypes$ such that
$\sem{\any} = \Values$.


\subsection{Shapes and Schemas}
\label{ssec:shapes}

We formulate all three schema languages using \emph{shapes}, which are unary
formulas describing the graph's structure around a \emph{focus} node or a value.
Shapes will be expressed in different formalisms, specific to the schema
language; for each of these formalisms we will define when a focus node or value
$v \in \Nodes \cup \Values$ \emph{satisfies} shape $\varphi$ in a common graph
$\graph$, written $\graph, v \models \varphi$.

Inspired by ShEx \emph{shape maps}, we abstract a schema $\schema$ as a set of
pairs $(\sel,\varphi)$, where $\varphi$ is a shape and $\sel$ is a
\emph{selector}.
A selector is also a shape, but usually a very simple one, typically checking
the presence of an incident edge with a given predicate, or a property with a
given key.
A graph $\graph$ is \emph{valid} \wrt $\schema$, in symbols $\graph \models
\schema$, if
\[
  \graph, v \models \sel
\quad \text{implies} \quad
  \graph, v \models \varphi,
\]
for all $v \in \Nodes \cup \Values$ and $(\mathit{sel}, \varphi) \in \schema$.
That is, for each focus node or value satisfying the selector, the graph around
it looks as specified by the shape.
We call schemas $\schema$ and $\schema'$ \emph{equivalent} if $\graph \models
\schema$ \iff $\graph \models \schema'$, for all $\graph$.
In what follows, we may use $\mathit{sel} \Rightarrow \varphi$ to indicate a
pair $(\mathit{sel}, \varphi)$ from a schema $\SHACLSchema$.

% \begin{example}[Schemas over Media Service Common Graph]
%     \label{ex:ShapeExample}

% We stay in the same scenario introduced in \Cref{ex:sharedScenario}. We list here illustrative examples for requirements on common graphs that can be imposed via schemas.  To give an intuitive idea about the selector and the shape, we indicate this informally by splitting the sentences into an initial part that selects nodes or values, and the second part which must hold for these elements:\\
% \noindent
% \emph{For every account}, there must exist a primary credit card ; \\
% \noindent \emph{For every account}, there are  five users of it or less;\\
% \emph{Every owner of an account}, has a unique email address.
% \end{example}

\begin{example}
  \label{ex:constraint-desc}
  We next describe some constraints one may want to express in the domain of
  Example~\ref{ex:common-graph}.
  \begin{enumerate}[(C1)]
  \item
    We may want the values associated to certain keys to belong to concrete
    datatypes, like strings or Boolean values.
    In our example, we want to state that the value of the key $\excard$ is
    always an integer.
  \item
    We may expect the existence of a value associated to a key, an outgoing
    edge, or even a complex path for a given source node.
    For our example, we require that all owners of an account have an email
    address defined.
  \item
    We may want to express database-like uniqueness constraints.
    For instance, we may wish to ensure that the email address of an account
    owner uniquely identifies them.
  \item
    We may want to ensure that all paths of a certain kind end in nodes with
    some desired properties. For example, if an account is privileged, then all
    users that have access to it should also be privileged.
  \item
    We may want to put an upper bound on the number of nodes reached from a
    given node by certain paths. For instance, every user may have access to at
    most 5 accounts.
\end{enumerate}

% \todo[inline]{Wim: Reminder to self. I'd like to illustrate some open/closed things here. (There's no time anymore for this.)}
% \todo[inline]{Wim: More urgently though, we should explain better about how we model things. Let's say that ``users'' are those nodes that have an email key and ``accounts'' are those that have a card key?}
% \todo[inline]{Iovka: I support the need to make this precise. Then, should we use these two selectors in all examples?\\
% Also, we might say that we need this trick because we do not have rdf:type nor labels on nodes.}
% \todo[inline]{Cem: After discussion with Filip, I fixed the setting such that it is keys that identify users and accounts. Problem: this makes C2 awkward. }

\end{example}

% End of file 2-Preliminaries.tex

% \input{3-Problem Setup}
% \input{4-Taxonomy}

{\footnotesize \bibliographystyle{acm}
\bibliography{ref}}

\section{Appendix}

% Please add the following required packages to your document preamble:
% \usepackage{multirow}
