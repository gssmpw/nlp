\section{Message-level EI Attributes}
In this section, we provide prompt details for reflectiveness, grounding act, and empathy.
Also, we present model details for emotion, sentiment, and intimacy.


\subsection{Prompt for `Reflectiveness'}
\label{appendix:evaluation-reflective}

\begin{tcolorbox}[title=Reflectiveness Classification, myboxstyle, breakable]
You are an evaluator trained to determine if a speaker’s language is reflective, indicating self-awareness. Reflective language is characterized by self-observation, perspective-taking, and intentionality. This means that the speaker is not only aware of their thoughts, feelings, or actions but also able to express this awareness clearly. \\

A reflective response often includes one or more of the following traits:

\begin{itemize}
    \item \textbf{Self-observation:} The speaker describes their own emotional or cognitive state (e.g., “I feel uncertain about…” or “I’m aware that…”).
    \item \textbf{Perspective-taking:} The speaker shows an understanding of how their actions or emotions affect others or acknowledges another person’s perspective on the situation (e.g., “I understand that my response may seem…”).
    \item \textbf{Intentionality:} The speaker explains the reasoning behind their behavior or decisions, revealing their underlying motivations or goals (e.g., “I decided to respond this way because…”).
\end{itemize}

\textbf{Example Statements}
\begin{itemize} 
    \item "I realize I tend to get defensive when I receive feedback, and I think it’s because I want to do well."  \\
    \textbf{Reflective or Not Reflective:} Reflective  \\
    \textbf{Reason:} This statement shows self-observation (“I realize I tend to get defensive”) and insight into motivation (“because I want to do well”).

    \item "I did what I thought was best for the project."  \\
    \textbf{Reflective or Not Reflective:} Not Reflective  \\
    \textbf{Reason:} While the speaker describes their decision, they don’t analyze or acknowledge the emotions or motivations behind their choice or consider its impact on others.
\end{itemize}

Given this dialogue context:  
\texttt{{\{dialogue\_history\_within\_session\}}}

Determine whether the \texttt{{speaker}}'s last message (\texttt{\{turn}\}) is reflective or not.  \\
Reflective language includes phrases like 'I feel...', 'I think...', or similar reflective expressions.  \\
Respond only with \texttt{'True'} for reflective or \texttt{'False'} for not reflective.
\end{tcolorbox}

\subsection{Prompt for `Grounding Act'}
\label{appendix:evaluation-grounding}

\begin{tcolorbox}[title=Grounding Act Classification, myboxstyle, breakable]
You are an evaluator trained to determine if a speaker’s language demonstrates grounding, which reflects active engagement and a commitment to mutual understanding in conversation. Grounding acts are characterized by clarifying questions, follow-up inquiries, or statements that seek to confirm, clarify, or expand on shared information. These acts are essential for building common ground, ensuring that both participants have a clear understanding, and preventing misunderstandings. \\

A grounding response often includes one or more of the following traits: \\

\textbf{Clarifying questions:} The speaker asks questions that seek clarification or further information about the other person’s statements (e.g., “Could you explain that further?” or “What did you mean by...?”). \\

\textbf{Follow-up inquiries:} The speaker shows interest in exploring a point raised by the other person, prompting them to elaborate or continue sharing (e.g., “How did that make you feel?” or “Can you tell me more about...?”). \\

\textbf{Confirmation checks:} The speaker seeks to confirm their understanding of what the other person said (e.g., “So, you mean that...?” or “Are you saying that...?”). \\

\section*{Example Statements}

\textbf{"Can you tell me more about what happened at the event?"} \\
Grounding or Not Grounding: \textbf{Grounding} \\
Reason: This is a follow-up question that prompts the other person to provide more information, demonstrating interest and a desire to deepen mutual understanding. \\

\textbf{"I completely understand your point."} \\
Grounding or Not Grounding: \textbf{Not Grounding} \\
Reason: Although this statement indicates agreement, it does not actively seek further information or clarification and does not encourage continued dialogue.\\

\textbf{"So, you’re saying that this new policy will impact the timeline?"} \\
Grounding or Not Grounding: \textbf{Grounding} \\
Reason: This is a confirmation check, as the speaker seeks to ensure their understanding of the other person’s statement.\\

\textbf{"It sounds like you’ve already made your decision."} \\
Grounding or Not Grounding: \textbf{Not Grounding} \\
Reason: This statement reflects an observation rather than a clarifying or follow-up question, so it does not serve as a grounding act.
\end{tcolorbox}


\subsection{Prompt for `\texttt{Empathy}'}
\label{appendix:evaluation-empathy}

\begin{tcolorbox}[title=Empathy Assessment, myboxstyle, breakable]
You are an evaluator assessing the level of empathy conveyed in a response, based on three core components: \textbf{Emotional Reaction}, \textbf{Interpretation}, and \textbf{Exploration}. For each component, provide a score from 0–2, where 0 indicates no presence, 1 indicates partial presence, and 2 indicates explicit presence. Sum the scores from each component to obtain an overall empathy score.

\section*{Component 1: Emotional Reaction}
Does the response express or allude to warmth, compassion, concern, or similar feelings of the responder towards the seeker?
\begin{itemize}
    \item \textbf{0}: No.
    \item \textbf{1}: Yes, the response alludes to these feelings but the feelings are not explicitly expressed.
    \item \textbf{2}: Yes, the response has an explicit mention.
\end{itemize}

\section*{Component 2: Interpretation}
Does the response communicate an understanding of the seeker’s experiences and feelings? In what manner?
\begin{itemize}
    \item \textbf{0}: No.
    \item \textbf{1}: Yes, the response communicates an understanding of the seeker’s experiences and/or feelings.
    \begin{itemize}
        \item The response contains conjectures or speculations about the seeker’s experiences and/or feelings.
        \item The responder reflects back on similar experiences of their own or others.
        \item The responder describes similar experiences of their own or others.
        \item The response paraphrases the seeker’s experiences and/or feelings.
    \end{itemize}
    \item \textbf{2}: The response provides a deep, explicit understanding and validation of the seeker’s feelings or experiences, potentially using multiple sub-categories.
\end{itemize}

\section*{Component 3: Exploration}
Does the response make an attempt to explore the seeker’s experiences and feelings?
\begin{itemize}
    \item \textbf{0}: No.
    \item \textbf{1}: Yes, the exploration is present but remains generic.
    \item \textbf{2}: Yes, the exploration is present and is specific, delving into the seeker’s particular feelings or experiences.
\end{itemize}

\section*{Output Format}
Return output in the following JSON format:
\begin{verbatim}
{
    "emotional_reaction": [0–2],
    "interpretation": [0–2],
    "exploration": [0–2]
}
\end{verbatim}

\end{tcolorbox}




\definecolor{systembg}{RGB}{230, 255, 230} % Light green for system
\definecolor{userbg}{RGB}{255, 240, 230}   % Light peach for user
\definecolor{assistantbg}{RGB}{230, 230, 255} % Light blue for assistant

\section{Persona Simulation \& Memory Probing}  
In this section, we detail the prompts used to evaluate LLM performance on persona simulation and memory probing tasks.  
We specify the assigned roles (\textit{i.e.,} system, user, assistant) for each prompt in our experiments.  

\subsection{Persona Simulation}
\label{ssec:appendix-persona-simulation}
The input consists of a system and user prompt, where the system defines the user’s persona and task, while the user prompt provides the dialogue history.  
The output serves as ground truth for training.

\subsubsection{Input}
% System Prompt Box
\begin{tcolorbox}[colback=systembg, colframe=black, sharp corners]
\textbf{System:} \\
You are \texttt{\{opponent\_speaker\}}. Continue the conversation. \\
Output only the message, not the speaker name.
\end{tcolorbox}

% User Prompt Box
\begin{tcolorbox}[colback=userbg, colframe=black, sharp corners]
\textbf{User:} \\
\texttt{\{previous conversation\}} \\
\texttt{\{speaker\}}
\end{tcolorbox}

\subsubsection{Output (Ground Truth)}
\begin{tcolorbox}[colback=assistantbg, colframe=black, sharp corners]
\textbf{Assistant:} \\
\texttt{\{speaker's original message\}}
\end{tcolorbox}

\subsection{Memory Probing}
\label{ssec:appendix-memory-probing}
For the memory probing task, we use only user role messages.  
All conversation history is formatted consistently, with sessions divided by date, as shown in the following example.  
If a message contains an image, we convert it into a caption using BLIP.  

\begin{tcolorbox}[colback=userbg, colframe=black, sharp corners]  
\textbf{User:} \\  
Below is a conversation between Alice and Bob, spanning multiple days. Each session starts with the corresponding date. \\  

\textbf{DATE:} February 10, 2025  

\textbf{CONVERSATION:}  \\  
Alice: "Hey, how have you been?"  \\  
Bob: "I'm good, just busy with work. How about you?"  \\  
Alice: "Same here, lots of meetings this week." \\  
Alice shared an image of "papers on the desk." \\  

Based on the provided context, generate a concise short-phrase answer for the following question.  
If the question pertains to a date, infer the approximate timeframe (e.g., "In the 1800s", "Before Jan 2021", etc.). \\  

\textbf{Question:} What has Alice been busy with?  

\textbf{Answer:}  
\end{tcolorbox}  
