\section{Emotional Intelligence (EI) Evaluation}
\label{sec:ei-evaluation}
To systematically compare real-world and LLM-generated dialogues, and to evaluate persona consistency and simulation quality, we develop a comprehensive EI evaluation framework.

\subsection{Problem formulation.}
Each conversation $\mathcal{C}$ consists of multiple sessions $\mathcal{S}_i$, such that $\mathcal{C} = \{\mathcal{S}_1, \mathcal{S}_2, \dots, \mathcal{S}_m\}$, where $m$ is the total number of sessions in the conversation.
Each session $\mathcal{S}_i$ consists of a sequence of messages exchanged between two speakers such that $\mathcal{S}_i = \{\mathcal{M}_{s_1,1}, \mathcal{M}_{s_2,2}, \dots, \mathcal{M}_{s_n,n}\}$ where \( \mathcal{M}_{s_j,j} \) represents the \( j \)-th message in the session, sent by speaker \( s_j \in \{1,2\} \).
Unlike structured turn-taking, the order of messages can be variable, and a speaker may send consecutive messages (\textit{e.g.,} multiple chat bubbles in a row), reflecting the natural flow of real-world conversations.
To maintain coherence, we concatenate consecutive messages from the same speaker into a single message before analysis.



\subsection{Message-level EI Attributes.}
\label{ssec:message-level-ei}
We measure following EI attributes for each individual message $\mathcal{M}$:

\paragraph{Reflectiveness.} $R(\mathcal{M})$ is a boolean indicator that captures whether a speaker explicitly recognizes and describes their own emotions, thoughts, or reactions. 
For example, statements like ``I think I’m feeling this way because...'' signal self-reflection. 
We use an LLM-based classification to label each turn as reflective or not (see Appendix~\ref{appendix:evaluation-reflective}).

\paragraph{Grounding Act.}
$G(\mathcal{M})$ is a boolean indicator that captures whether a speaker use acts—such as clarifying questions or follow-ups—which are essential for building common ground and preventing misunderstandings~\cite{clark1996using, clark1989contributing, shaikh-etal-2024-grounding}.
We use an LLM-based classification to label each turn as grounding or not (see Appendix~\ref{appendix:evaluation-grounding}).

\paragraph{Emotion \& Sentiment \& Intimacy.}
$E(\mathcal{M})$ and $S(\mathcal{M})$ represent the speaker's emotion and sentiment labels in a message $\mathcal{M}$, respectively.
Emotion is classified into 11 predefined categories (e.g., anger, fear, joy), while sentiment is classified into 3 categories (positive, negative, neutral). 
$I(\mathcal{M})$ is a floating-point value that indicates the speaker's level of intimacy in the message $\mathcal{M}$.
Each of these attributes is measured using models trained specifically for emotion, sentiment, and intimacy classification, leveraging an annotated dataset from Twitter~\cite{antypas2023supertweeteval}.
While LLMs could be used for these tasks, fine-tuned models perform comparably or slightly better while being significantly more cost-efficient, making them a more practical choice for large-scale evaluation~\cite{rathje2024gpt}.


\paragraph{Empathy.}
$EP(\mathcal{M})$ is a floating-point score capturing the speaker's empathy in a message $\mathcal{M}$.
It consists of three components: emotional reaction, interpretation, and exploration. 
Each component is scored by the LLM on a 0–2 Likert scale based on the \textsc{EPITOME} framework~\cite{sharma-etal-2020-computational} (see Appendix~\ref{appendix:evaluation-empathy}). 
The empathy score for a message is the sum of the scores for these components.

\subsection{Speaker-level EI Evaluation.}
\label{ssec:speaker-level-ei}
To assess a speaker’s overall emotional intelligence (EI), we aggregate message-level EI attributes into five categories based on Goleman’s EI framework~\cite{goleman1998working}.

\paragraph{Self-awareness.}
It reflects a speaker’s ability to recognize and articulate emotions and perspectives. 
It is measured using two metrics: 

\vspace{0.2cm}\noindent
(1) \textit{Reflective frequency} measures how often a speaker engages in self-reflection:
\[
\text{reflective\_frequency}(s) = \frac{\sum_{\mathcal{M} \in \mathcal{M}_s} R(\mathcal{M})}{|\mathcal{M}_s|}
\]
where \(\mathcal{M}_s\) is the set of messages by speaker \(s\).

\vspace{0.2cm}\noindent
(2) \textit{Emotion \& Sentiment Diversity} captures the range of emotions or sentiments expressed by a speaker \(s\), calculated using entropy.
Let \(L\) represent the predefined sentiment labels (\textit{i.e.,} positive, negative, neutral), and \(p_s(x)\) denote the proportion of label \(x \in L\) in the speaker's messages \(\mathcal{M}_s\):
\[
p_s(x) = \frac{\sum_{\mathcal{M} \in \mathcal{M}_s} 1 (S(\mathcal{M}) = x)}{|\mathcal{M}_s|},
\]
where \(1 (S(\mathcal{M}) = x)\) equals 1 if the sentiment label of message \(\mathcal{M}\) is \(x\), and 0 otherwise.  
Using this, sentiment diversity is computed as:
\[
\text{sentiment\_diversity}(s) = - \sum_{x \in L} p_s(x) \cdot \log_2 p_s(x).
\]
The same applies to \textit{emotion\_diversity} by substituting \(S(\mathcal{M})\) with \(E(\mathcal{M})\) and using emotion labels.

% A higher entropy value indicates a wider range of emotional or sentiment expressions, enriching the dialogue’s depth and engagement.

\paragraph{Motivation}
A speaker’s engagement in maintaining the conversation, measured by \textit{grounding frequency}, the proportion of messages containing grounding acts:  
\[
\text{grounding\_frequency}(s) = \frac{\sum_{\mathcal{M} \in \mathcal{M}_s} G(\mathcal{M})}{|\mathcal{M}_s|}
\]


\paragraph{Social Skills}
A speaker’s ability to foster trust and engagement, measured via \textit{intimacy progression}, assuming intimacy naturally evolves over time.
The average intimacy for speaker $s$ per session $\mathcal{S}_i$ is:
\[
\text{intimacy\_average}(s, \mathcal{S}_i) = \frac{\sum_{\mathcal{M} \in \mathcal{M}_s^{\mathcal{S}_i}} I(\mathcal{M})}{|\mathcal{M}_s^{\mathcal{S}_i}|}
\] 
where \( \mathcal{M}_s^{\mathcal{S}_i} \) is the set of messages from speaker \( s \) in session \( \mathcal{S}_i \), and \( I(\mathcal{M}) \) denotes the intimacy score of message \( \mathcal{M} \).
The progression of average intimacy across sessions in \(\mathcal{C}\) is modeled as:  
\[
\text{linear\_progression}(s) = a, \text{ where } y = a \cdot x + b
\]  
\[
\text{exp\_progression}(s) = b, \text{ where } y = a \cdot e^{b \cdot x}
\]


\paragraph{Self-regulation}
It reflects a speaker’s ability to maintain emotional and sentiment stability while interacting with their conversational partner.
We evaluate this using two metrics:

\vspace{0.2cm}\noindent
(1) \textit{Emotion \& Sentiment Stability} measures the consistency of a speaker's emotions or sentiments across their messages:
\[
\text{stability}(s) = \frac{\sum_{k=2}^{|\mathcal{M}_s|} 1(E(\mathcal{M}_s^{(k)}) = E(\mathcal{M}_s^{(k-1)}))}{|\mathcal{M}_s| - 1}
\]  
where \( \mathcal{M}_s^{(k)} \) is the \( k \)-th message in speaker \( s \)'s ordered sequence of messages, \( 1(\cdot) \) is an indicator function that returns 1 if the emotion of \( \mathcal{M}_s^{(k)} \) matches the previous message \( \mathcal{M}_s^{(k-1)} \), and 0 otherwise.
The same formula applies for sentiment stability, replacing \( E(\mathcal{M}) \) with \( S(\mathcal{M}) \).  

\vspace{0.2cm}\noindent
(2) \textit{Emotion \& Sentiment Alignment} measures a speaker's synchronization with a partner:  
\[
\text{alignment}(s) = \frac{\sum_{k} 1(E(\mathcal{M}_s^{(k)}) = E(\mathcal{M}_p^{(k)}))}{|\mathcal{M}_s|}
\]  
where \( \mathcal{M}_s^{(k)} \) and \( \mathcal{M}_p^{(k)} \) are corresponding messages from speaker \( s \) and their partner \( p \), and \( 1(\cdot) \) returns 1 if their emotions match and 0 otherwise.  
The same formula applies for sentiment alignment, substituting \( E(\mathcal{M}) \) with \( S(\mathcal{M}) \).  








\paragraph{Empathy}  
It measures the average empathy score of a speaker \(s\)’s messages, calculated as:  
\[
\text{empathy}(s) = \frac{\sum_{\mathcal{M} \in \mathcal{M}_s} EP(\mathcal{M})}{|\mathcal{M}_s|},
\]  
where \(EP(\mathcal{M})\) is the empathy of message \(\mathcal{M}\).






