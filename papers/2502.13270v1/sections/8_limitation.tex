\section{Limitations}  
Our study has several limitations:  

\paragraph{Demographic constraints.}  
Participants are English speakers from the US, aged 18-25, limiting the dataset’s representativeness across diverse demographics.  
Additionally, we do not analyze how conversational dynamics vary based on factors such as gender, profession, geographic location, or age differences.  
Future studies should include a more diverse demographic range to better understand how factors like age, gender, and cultural background influence dialogue patterns.

\paragraph{Chit-chat focus.}  
Our dataset primarily consists of casual, open-domain conversations rather than goal-oriented interactions.  
This limits its applicability to domains where social intelligence involves structured reasoning, such as negotiations or counseling~\cite{zhou2023sotopia, wu2024longmemeval, liu-etal-2024-interintent}.  
Future work should extend EI modeling to structured dialogues where speakers need to achieve their social goals.  

\paragraph{Emotional intelligence (EI) measurement.}  
Our EI evaluation focuses on surface-level indicators, such as sentiment, reflectiveness, and empathy.  
However, these metrics may not fully capture deeper cognitive, cultural, and situational influences on emotional intelligence.  
Additionally, EI attributes are inherently subjective and challenging to quantify.  
To ensure reliability, we ground our evaluations in established literature.  

\paragraph{Lack of multi-modal analysis.}  
While our dataset includes images, our experiments focus solely on text-based interactions.  
We do not analyze how visual elements contribute to emotional alignment, persona perception, or memory retention.  
Future research could integrate multi-modal fusion techniques to enhance emotional intelligence modeling.  

