\section{Conclusion}  
In this work, we introduce \dataset{}, a 21-day corpus of authentic messaging app dialogues—the longest available dataset where the same two individuals engage in sustained conversations. 
Through a detailed analysis, we highlight key differences between human conversations and LLM-simulated dialogues, revealing that LLMs exhibit limited emotion diversity, excessive empathy, and reduced variability in EI expression compared to real interactions.
Using this data, we present two benchmarks:  
(1) \textbf{Persona simulation}, evaluating LLMs' ability to replicate an individual’s conversational style and showing that simple fine-tuning improves simulation accuracy.  
(2) \textbf{Memory probing}, assessing LLMs' limitations in applying real-world long-term context and highlighting challenges in existing memory systems.  
Our findings underscore the complexity of modeling real-world persona dynamics and memory, emphasizing the need for further research.  
We hope \dataset{} inspires advancements in socially intelligent AI, personalized dialogue systems, and adaptive memory strategies, driving more human-like interaction modeling.  