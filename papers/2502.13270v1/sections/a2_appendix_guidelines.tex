\onecolumn
\section{Annotation Guidelines}
\enlargethispage{1\baselineskip}

% \vspace{-0.4cm}  % Adjust as needed
\subsection{Chat Collection Guidelines}
\label{appendix:chat-guidelines}
\nopagebreak

\begin{tcolorbox}[title=Chat Guidelines for Participants, myboxstyle, breakable, before skip=0pt]
\begin{description}[align=left, labelwidth=5em, labelindent=0em, itemsep=1em]
    \item[\textbf{First-Time Interaction.}]
    You are interacting with the other speaker for the first time.

    \item[\textbf{Chit-chat.}]
    Engage in chit-chat that can include real events from your own life (\textit{e.g.,} taking a nap or cooking something).
    \textit{The content can be fictional.}

    \item[\textbf{Personal Information.}]
    Make the conversation personal from time to time by discussing topics like family, friends, likes, dislikes, and aspirations.
    \textit{The content can be fictional.}

    \item[\textbf{Time References.}]
    Include references to time (\textit{e.g.,} ‘last Friday’, ‘next month’, ‘when I was ten years old’) and specific places or locations.
    Consider the current time during the conversation; for example, if it’s after lunch, ask what the other participant had for lunch, or greet them with “Good Morning” if chatting in the morning.

    \item[\textbf{Dialog Style.}]
    Keep the dialogue style casual, as if talking to a friend.

    \item[\textbf{Daily Events and News.}]
    Discuss events from your life, news, social media highlights, or pop culture events (\textit{e.g.,} movies, concerts).
    Feel free to share opinions in a friendly, engaging way that may interest the other participant.

    \item[\textbf{Images.}]
    Share images at appropriate points in the conversation. Examples include:
    \begin{itemize}
        \item things you own (clothes, food, vehicles, etc.),
        \item things you've seen or want to see (\textit{e.g.,} food, people, events),
        \item things you like (\textit{e.g.,} a piece of clothing, a cute animal),
        \item things you need help with (\textit{e.g.,} how to fix something or navigate a relationship),
        \item images representing old memories.
    \end{itemize}

    \item[\textbf{Image Source.}]
    Do not share images from your camera. Instead, search for images on the internet using Google, save the image URL, and include it in square brackets within the conversation.
    \begin{quote}
        \texttt{[URL of the image]}
    \end{quote}
    Images should be licensed under Creative Commons to allow free use by others. In Google Search, such images can be found by selecting the Creative Commons license under Tools → Usage Rights. Using the dimension constraint \texttt{imagesize:3024x4032} in the query yields less generic images that can foster engaging conversations.

    \item[\textbf{Images Containing Faces.}]
    Avoid sharing images where a person's face is clearly visible and implies that it is you. However, you may share images where the face is obscured or clearly not you. Ensure any appearance remains approximately consistent throughout sessions.

    \item[\textbf{Image Context.}]
    Avoid sharing images that don’t add information to the conversation. For example, avoid generic images that lack context or relevance, but do share images that add context, like one showing a user kayaking.

    \item[\textbf{Pronoun Use.}]
    Refer to previously mentioned subjects in the conversation with pronouns, allowing the other participant to easily infer the reference.
\end{description}
\end{tcolorbox}

\subsection{Question \& Event Annotation Guidelines}
\label{appendix:qa-guidelines}

\begin{tcolorbox}[title=Question \& Event Annotation Guidelines, myboxstyle, breakable]
\begin{description}[align=left, labelwidth=5em, labelindent=0em, itemsep=1em]

\item[\textbf{Introduction.}] These guidelines provide instructions for annotating a dataset consisting of long-term conversations (approximately 21 days) between two individuals. The annotation task is divided into two sub-tasks:
    \begin{itemize}
        \item \textbf{Events Annotation:} Annotators document significant life events for each speaker in each session.
        \item \textbf{Question and Answer Creation:} Annotators generate questions that can only be answered by reading through one or more sessions of conversations.
    \end{itemize}
    The dataset includes both textual and image-based conversations.

\item[\textbf{Data Format.}] The conversation dataset consists of multiple interaction sessions between two speakers. Each session is timestamped.
    \begin{itemize}
        \item Conversations are structured in a table format with dedicated columns for each speaker.
        \item Speaker names appear at the top of the table.
        \item If a speaker shares an image, it is included in the dataset.
    \end{itemize}

\item[\textbf{Sub-Task 1: Writing Speaker Events.}] Annotators document key life events for each speaker as the conversation progresses. These events may include:
    \begin{itemize}
        \item Past events, ongoing events, and planned future events.
        \item Small-scale events (e.g., attending a cooking class) and major events (e.g., moving to a new city).
    \end{itemize}
    \textbf{Annotation Format:}  
    \begin{itemize}
        \item Each event should be recorded in a separate line within the allotted cell.
    \end{itemize}

\item[\textbf{Sub-Task 2: Creating Questions and Answers.}] Annotators generate questions that can only be answered by reviewing the conversation dataset.

    \textbf{Question Categories and Examples:}
    \begin{itemize}
        \item \textbf{Questions that Require Aggregation of Information.} \newline
            \textbf{Question:} Which countries has Joseph travelled to? \newline
            \textbf{Answer:} France, Japan \newline
            \textbf{Evidence:} D4:1, D1:8 \newline
            \textbf{Category:} 1
        \item \textbf{Questions that Require Reasoning About Time.} \newline
            \textbf{Question:} When did Kate start skiing? \newline
            \textbf{Answer:} 2013 \newline
            \textbf{Evidence:} D1:49 \newline
            \textbf{Category:} 2
        \item \textbf{Questions that Require Commonsense Reasoning or World Knowledge.} \newline
            \textbf{Question:} What kind of jobs might Joseph consider based on his recent ventures? \newline
            \textbf{Answer:} Jobs that combine software engineering and management, e.g., software product manager \newline
            \textbf{Evidence:} D2:9, D3:1 \newline
            \textbf{Category:} 3
    \end{itemize}

    \textbf{How to Annotate Questions:}  
    Each question entry includes the following fields:
    \begin{itemize}
        \item \textbf{Question:} The formulated question.
        \item \textbf{Answer:} The exact response extracted from the conversation.
        \item \textbf{Evidence:} Unique identifiers of dialogues used to derive the answer.
        \item \textbf{Category:} A label (1-3) indicating the question type.
    \end{itemize}
    Annotators should write at least \textbf{2 and up to 10 questions} from any of the categories at the end of each session.


\end{description}
\end{tcolorbox}


