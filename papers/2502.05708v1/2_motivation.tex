\section{Preliminary}\label{sec_motivation}

In this section, we provide the background on spatial spectrum and vanilla NeRF for wireless signal modeling.



\subsection{Spatial Spectrum}\label{sec_spatial_spectrum}
The spatial spectrum represents the received signal power distribution around the receiver for a signal transmitted by a transmitter. 
In this section, we describe the wireless channel and then outline the computation of the spatial spectrum.


\subsubsection{Wireless Channel}\label{sec_channel}
A wireless communication system consists of a transmitter and a receiver. 
The transmitted signal can be expressed as a complex number $x = A e^{j\psi}$, where $A$ represents the amplitude, and $\psi$ denotes the phase. 
As the signal propagates through the wireless channel, its amplitude is attenuated by $A_{\text{att}}$, and its phase is shifted by $\Delta \psi$. 
The signal received at the receiver with a single antenna can be written as:
\begin{equation}
y = x \cdot A_{\text{att}} e^{j\Delta \psi} = A \cdot A_{\text{att}} e^{j(\psi + \Delta \psi)}
\end{equation}



\textbf{Antenna Array.}  
For a receiver configured as an antenna array with $K \times K$ elements, as shown in Figure~\ref{fig_array}(a), the received signal is represented as a matrix $\mathbf{Y} = \big[y_{m,n}\big]_{m,n=0}^{K-1}$, where $y_{m,n}$ denotes the signal received at the $(m, n)$-th element of the array. 
Each element $y_{m,n}$ is given by:
\begin{equation}
y_{m,n} = A \cdot A_{\text{att}} e^{j(\psi + \Delta \psi + \Delta \sigma_{m,n})}
\end{equation}
where $A_{\text{att}} e^{j\Delta \psi}$ represents the attenuation and phase shift introduced by the wireless channel. This is assumed to be uniform across all elements due to the small array aperture relative to the propagation distance~\cite{van2002optimum}.
The term $\Delta \sigma_{m,n}$ is the geometric phase shift caused by the position of the $(m, n)$-th element relative to the array center at $(0, 0, 0)$. It is determined by the element's position and the angles of arrival of the incoming signal:
\begin{equation}
\label{eqn_geomtry}
\Delta \sigma_{m,n} = \frac{2\pi}{\lambda} \left( m d \sin\beta \cos\alpha + n d \sin\beta \sin\alpha \right)
\end{equation}
where $\lambda$ is the signal wavelength, $d = \frac{\lambda}{2}$ is the element spacing, and $\alpha$ and $\beta$ are the azimuth and elevation angles of the incoming signal, respectively, as shown in Figure~\ref{fig_array}(b).





\begin{figure}[t]
\centering
	\subfigure[Receiver: $4\times4$ antenna array]{
\includegraphics[width=.23\textwidth]{figs/array.pdf}}
	\subfigure[Azimuth $\alpha$ and elevation $\beta$]{
\includegraphics[width=.22\textwidth]{figs/array_alpha_beta.pdf}}
	\vspace{-0.15in}
\caption{Antenna array, each blue square as a single antenna.}
\label{fig_array}
\end{figure}


\subsubsection{Computing Spatial Spectrum}\label{sec_spectrum}
The spatial spectrum is computed from the received signal matrix $\mathbf{Y}$ to estimate power distribution over azimuth $\alpha$ and elevation $\beta$. 
For a given direction $(\alpha, \beta)$, the power is:
\begin{equation}
\mathbf{SS}(\alpha, \beta) = \left| \sum_{m=0}^{K-1} \sum_{n=0}^{K-1} y_{m,n} e^{-j \Delta \sigma_{m,n}} \right|^2
\end{equation}
where $\Delta \sigma_{m,n}$ is the theoretical geometric phase shift from Equation~(\ref{eqn_geomtry}).
With one-degree resolution and only the upper hemisphere considered, the spatial spectrum $\mathbf{SS}$ is a matrix of dimensions $(360, 90)$:
$\mathbf{SS} = \big[SS_{p,q}\big]_{p,q=0}^{359, 89}.$
Figure~\ref{fig_spectrum}(a) illustrates the 3D spatial spectrum with one-degree directional resolution. 
Figure~\ref{fig_spectrum}(b) then depicts the 2D spectrum, which is the 3D view projected onto the X-Y plane, with the radial distance representing $\cos(\beta)$.



\subsection{Vanilla NeRF for Wireless Signal Modeling}
\label{sec_nerf2_intro}
Vanilla NeRF has recently been adapted for wireless signal modeling~\cite{zhao2023nerf, lunewrf}. 
It models a scene using an MLP with 8-dimensional inputs:
\begin{equation}
\label{eqn_nerf2_voxel}
\mathcal{F}_{\Theta} : \left( \{x, y, z\}, \alpha, \beta, \{x_{tx}, y_{tx}, z_{tx}\}\right) \to \left( \delta, \xi \right)
\end{equation}
where $\{x, y, z\}$ are the 3D spatial coordinates of a voxel, and $(\alpha, \beta)$ are the azimuth and elevation angles defining the 2D view direction of a ray traced from the receiver.
The transmitter's 3D coordinates $\{x_{tx}, y_{tx}, z_{tx}\}$ are also considered, as each voxel's properties in the wireless domain are influenced by the signal source location.
The outputs $(\delta, \xi)$ represent wireless signal properties of the voxel, where $\delta$ is the attenuation and $\xi$ is the signal emission, with each voxel treated as a new wireless signal source according to the Huygens–Fresnel principle~\cite{born2013principles}.
To generate the spatial spectrum, \nerft~\cite{zhao2023nerf} performs ray tracing in each direction of the spectrum. 
Discrete sample voxels $\{V_1, ..., V_S\}$ are taken along each ray $\mathbf{r}$, and the MLP is queried with their coordinates to predict $\delta$ and $\epsilon$. 
The result for each ray $\mathbf{SS}\left(\mathbf{r}\right)$ is calculated as:
\begin{equation}
\label{eqn_tracing_rf}
\mathbf{SS}\left(\mathbf{r}\right) = \sum_{i=1}^{S} \exp \left( \sum_{j=1}^{i-1} \delta_j \right) \xi_i
\end{equation}
This equation calculates the aggregation of signals retransmitted by voxels along the ray, where each voxel acts as a new source, and transmissions are attenuated by intervening voxels between the source and the receiver.

\begin{figure}[t]
\centering
	\subfigure[3D view]{		\includegraphics[width=.225\textwidth]{figs/hemisphere_3d.pdf}}
	\subfigure[2D projection view]{
		\includegraphics[width=.205\textwidth]{figs/hemisphere_2d.pdf}}
	\vspace{-0.15in}
\caption{Illustration of 3D and 2D views of the spatial spectrum.}
\label{fig_spectrum}
\end{figure}



\begin{figure*}[t]
\centering
{\includegraphics[width=.93\textwidth]{figs/framework.pdf}}
	\vspace{-0.2in}
\caption{Architecture of \ourSystem. Each voxel is represented by a feature $\mathbf{v}^s_i$, where $i$ indexes the $M$ rays, and $s$ denotes the voxel's position along the $i$-th ray. A neural-driven ray tracing algorithm computes the received signal power for each ray (TX: transmitter, RX: receiver).}
	\label{fig_workflow}
\end{figure*}





