\section{Conclusion}\label{sec_conclusion}
This paper presents \ourSystem, a generalizable wireless radiance field framework for modeling wireless signal propagation. 
\ourSystem integrates a geometry-aware scene representation module to capture geometrically related voxel features. 
It then incorporates a neural-driven ray tracing algorithm to aggregate features for signal power computation. 
Extensive experiments validate the generalization of \ourSystem.




\section{Impact Statement}\label{sec_impact}

The proposed framework, \ourSystem, introduces a novel generalizable wireless radiance field for modeling wireless signal propagation, offering contributions to both machine learning research and practical applications in wireless communication and sensing. 
Below, we outline the broader scientific and societal impacts of this work:

\textbf{1) Advancing ML for Wireless Domains.}
\ourSystem bridges the gap between optical and wireless domains by extending NeRF to wireless signal propagation modeling. 
Addressing the generalization challenges in unseen scenes, it contributes to the broader field of machine learning through advancements in spatial interpolation and geometry-aware learning tailored for non-optical domains. 
This work highlights the versatility of neural frameworks in tackling domain-specific challenges beyond their original applications.



\textbf{2) Transforming Wireless Network Planning.}
Current approaches to network planning rely on expensive and time-intensive site surveys or Computer-Aided Design~(CAD)-based simulations, which are often impractical in real-world environments. 
\ourSystem provides a scalable, data-driven alternative that improves the accuracy of signal propagation modeling while reducing the need for extensive manual data collection. 
This transformation has the potential to accelerate the deployment and optimization of wireless networks, including 5G and future communication technologies.

\begin{table}[t]
\centering
\caption{Effect of wireless signal frequency bands on performance.}
\begin{tabular}{L{0.75in}C{0.6in}C{0.6in}C{0.6in}}
\toprule
     & 928\,MHz  & 2.412\,GHz & 5.805\,GHz     \\ 
  \midrule
PSNR$\uparrow$  & 25.70  & 24.53 & 24.91    \\ 
\bottomrule
\end{tabular}
\label{table_para_material}
\end{table}


\textbf{3) Improving Wireless Sensing Applications.}
Wireless sensing technologies, such as WiFi-based localization, rely on high-quality wireless signal data.  
By enabling generalizable spatial spectrum synthesis, \ourSystem provides high-quality data to train DNN models, improving the reliability and accuracy of these applications. 
Furthermore, \ourSystem can be utilized for quick evaluation of DNN models by providing accurate and generalizable signal data, reducing the need for extensive real-world data collection during the model development phase.


\textbf{4) Supporting Equity in Wireless Access.}  
By reducing the barriers to efficient network planning and sensing, \ourSystem has the potential to improve wireless coverage and sensing capabilities in underserved and remote areas. 


In summary, \ourSystem represents a significant step toward integrating machine learning innovations into the wireless domain, with far-reaching implications for technology, research, and society. 
It lays the groundwork for more flexible, efficient, and cost-effective wireless communication systems that can adapt to diverse and dynamic environments.

This work does not raise any ethical issues.



\section*{Acknowledgments}
Wan Du was partially supported by NSF Grant \#~2239458, a UC Merced Fall 2023 Climate Action Seed Competition grant, and a UC Merced Spring 2023 Climate Action Seed Competition grant. 
Kang Yang was partially supported at UC Merced by a financial assistance award approved by the Economic Development Administration’s Farms Food Future program.
