\section{Related Work}
\label{sec_relatedWork}

\textbf{Generalization on Optical NeRF.}  
Optical NeRF has revolutionized novel view synthesis by learning 3D scene representations from 2D images. 
However, traditional NeRF requires extensive per-scene training, limiting its generalization to new scenes. Generalizable NeRF approaches overcome this limitation by adapting to unseen scenes without retraining~\cite{trevithick2020grf, chen2021mvsnerf, 10377995, chou2024gsnerf, tian2023mononerf, liu2022neural}. 
For example, MVSNeRF integrates multi-view stereo with neural rendering to reconstruct radiance fields from a few input views~\cite{chen2021mvsnerf}, and WaveNeRF employs wavelet-based representations for generalization~\cite{10377995}. 
GSNeRF incorporates semantics to generate novel views and semantic maps for unseen scenes~\cite{chou2024gsnerf}, while NeuRay aggregates image features from multiple views to predict voxel properties~\cite{liu2022neural}. 
These methods, designed for visible light, cannot be directly applied to wireless signals due to fundamental differences in wavelength and propagation behaviors.
This work extends generalizable NeRF to the wireless domain for spatial wireless signal propagation modeling.


\textbf{NeRF for Wireless Signal Modeling.}
There are pioneering efforts that have explored NeRF for wireless signal modeling~\cite{zhao2023nerf, orekondy2022winert}.
For instance, \nerft~trains an MLP to compute voxel attributes and proposes Equation~(\ref{eqn_tracing_rf}) as the ray tracing algorithm to compute the received signal for each ray. 
However, it requires scene-specific training, which limits scalability and generalization to new environments.
NeWRF~\cite{lunewrf} leverages direction-of-arrivals~(DoA) as priors to reduce the required rays, thereby enhancing the efficiency of signal reception computation. 
However, this efficiency comes at the cost of requiring additional antenna arrays for DoA data collection, which are infeasible in complex, obstructed environments.
\ourSystem avoids this additional data requirement.
WiNeRT~\cite{orekondy2022winert} employs differentiable ray tracing with scene CAD models, which is impractical due to the difficulty of obtaining accurate CAD models. 
In contrast, \ourSystem eliminates this dependency, offering a more practical and flexible solution.
More importantly, these models lack generalization. 
\ourSystem introduces generalizable neural radiance fields for wireless signal propagation, adapting to unseen scenes without scene-specific training.