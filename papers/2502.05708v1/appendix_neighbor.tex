
\section{Why Incorporate Neighbors}\label{sec_design_neighbors}
The rationale is based on two observations highlighted in this subsection.
First, for a given target transmitter, neighboring transmitters typically traverse similar propagation paths, \ie they pass through many common voxels within the scene.
Second, the spectrum for the target transmitter can be perceived as an interpolation of the spectra from its neighbors.
Consequently, leveraging the spectra from these neighbors facilitates the spectrum generation for the target transmitter, thereby improving the generalization ability as long as neighbor data is available.
We conduct two empirical experiments to demonstrate these two observations.



\begin{figure}[t]
\centering
	\subfigure[Propagation path]{
		\includegraphics[width=.35\textwidth]{figs/path.pdf}}
	\subfigure[Reflection points distance]{
		\includegraphics[width=.35\textwidth]{figs/disance_txs.pdf}}
	\vspace{-0.15in}
\caption{Signal propagation path differences between two transmitters (TXs) and a receiver (RX).}
\label{fig_scene_moti_1}
\end{figure}



\textbf{Observation \#1.}
In a conference room, as depicted in Figure~\ref{fig_scene_moti_1}(a), we position the RX at a fixed location and place TX1 and TX2 at two closely situated positions.
We utilize the MATLAB ray tracing simulation~\cite{matlab_indoor_simulation} to conduct wireless ray tracing analysis.
This software calculates all the propagation paths between the transmitters and the receiver.
For brevity, we present one path from either TX1 or TX2 to the RX in Figure~\ref{fig_scene_moti_1}(a).
It is evident that the starting and reflection positions of the two paths are very close, and they both terminate at the RX.


For a more general case, we generate 500 pairs of closely positioned TX1 and TX2, execute the Wireless Insite ray tracing algorithm for each pair, and then compare the distances between the reflection points of their corresponding paths.
Figure~\ref{fig_scene_moti_1}(b) shows that approximately 80\% of the distances between reflection points are less than 0.1\,m.
Given their close starting positions, proximate reflection points, and identical end positions, these factors collectively indicate a similar propagation path.
Moreover, assuming the voxel size equals the wavelength, such as 0.13\,m for 2.4 GHz WiFi, the path differences are typically less than the voxel size.
Therefore, the propagation paths for closely positioned transmitters typically pass through many common voxels.



\begin{figure}[t]
\centering
	\begin{minipage}[t]{0.35\linewidth} 
    	\includegraphics[width=\textwidth]{figs/moti_pnsr.pdf}
	\vspace{-0.3in}
            \caption{PSNR for using neighbors' spectra.}
        \label{fig_scene_moti_2_psnr}
	\end{minipage}
 \hspace{0.05in}
 	\begin{minipage}[t]{0.35\linewidth} 
    	\includegraphics[width=\textwidth]{figs/moti_para.pdf}
	\vspace{-0.3in}
        \caption{Statistics of the interpolation matrix.}
        \label{fig_scene_moti_2_para}
	\end{minipage}
 \vspace{10pt}
\end{figure}






\textbf{Observation \#2.}
Given that closely positioned transmitters pass through many common voxels, their spectra may exhibit some correlations.
To verify this, we conduct an experiment using RFID dataset (details in Section~\ref{sec_experimental_setting}).
For each transmitter and its corresponding spectrum, we identify the 6 closest positions of other transmitters as its neighbors.
We then assign a weight matrix to each neighbor; since the spectrum size is $(360, 90)$, each weight matrix is also $(360, 90)$, with every pixel of the spectrum assigned a specific weight.
Using each neighbor's spectrum and its assigned weight matrix, we perform a weighted summation of all neighbors' spectra to predict the target transmitter's spectrum.
We employ the MSE loss to train the 6 weight matrices over 200 iterations for each target transmitter.
Once trained, we compare the generated spatial spectrum with the ground truth spectrum.


Figure~\ref{fig_scene_moti_2_psnr} illustrates that an average PSNR of 20.5\,dB with a standard deviation of 4.1\,dB is achieved for the interpolated spatial spectra.
To understand what the value of 20.5\,dB represents, refer to the second row, first column of Figure~\ref{fig_vis_d1}, which displays a PSNR of 19.5\,dB between two spatial spectra.
Visually, the similarity between them is apparent.
Thus, an average PSNR of 20.5\,dB suggests that the neighbors' spectra are indeed closely related to the target transmitter's spectrum.
Consequently, the target transmitter's spatial spectrum can be viewed as an interpolation of the spectra of its neighbors.
Explaining this phenomenon from a broader perspective, novel view synthesis in the context of visible light can be achieved by interpolating between images captured from different camera poses and views~\cite{chen2023view}.
Similarly, wireless signals may also exhibit such properties.
Because visible light and wireless signals are both forms of electromagnetic waves, they might share underlying interpolation characteristics.


You might wonder whether this interpolation method could be directly applied to synthesize spectra for a target transmitter.
The answer is NO.
First, although the generated spectrum shows promise, it is not sufficiently accurate.
Second, the weight matrices are highly dependent on the transmitter positions.
A set of weight matrices from one transmitter cannot be applied to another transmitter.
We normalize all learned weight matrices for all transmitters and present their mean and standard deviation in Figure~\ref{fig_scene_moti_2_psnr}.
It can be observed that the values range from $-1.0$ to $1.0$, covering all possible values.
Thus, the values of the weight matrix are highly dynamic, and a single set of weight matrices cannot be used for other transmitters.


