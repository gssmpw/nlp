\section{Introduction}\label{sec_introduction}



The evolution of wireless communication systems, such as WiFi and 5G cellular networks, has enabled transformative applications in areas like smart cities and precision agriculture~\cite{gadre2020frequency, tong2023citywide}. 
These networks serve as both communication backbones and sensing platforms. 
Reliable communication requires meticulous network planning to address issues like dead spots and outages~\cite{ahamed20215g}, while sensing applications—such as WiFi-based localization~\cite{ayyalasomayajula2020deep, yang2024orchloc, savvides2001dynamic}—depend on extensive, high-quality data for deep neural network~(DNN) training. 
Traditionally, site surveys~\cite{kar2018site} are conducted to collect signal strength data for planning and sensing purposes, requiring dense measurements at various locations. 
However, these site surveys are time-intensive and costly~\cite{site_survey_cisco}.


Wireless signal propagation modeling offers an efficient alternative by predicting the received signal at the receiver given the transmitter and receiver positions, while accounting for signal attenuation, phase rotation, and interference~\cite{1451581, na2022huygens}. 
For instance, in free space, Maxwell's equations~\cite{maxwell1873treatise} can accurately calculate the received signal~\cite{wong1984conductivity}. 
However, obstacles in real-world environments make these equations unsolvable~\cite{zhao2000integral}.
To address this, ray tracing simulations~\cite{yun2015ray, egea2021opal, matlab_indoor_simulation} attempt to model signal propagation paths using Computer-Aided Design~(CAD) scene models. 
Nevertheless, the difficulty of obtaining accurate and realistic CAD models makes these simulations both impractical and unreliable.



Recently, \nerft~\cite{zhao2023nerf} and NeWRF~\cite{lunewrf} have adapted vanilla Neural Radiance Fields~(NeRF), originally designed for novel view synthesis in the optical domain~\cite{tancik2022block, liu2020neural}, to model wireless signal propagation. 
These methods achieve state-of-the-art accuracy in predicting received signals by learning neural radiance fields specific to wireless signals.
However, like vanilla NeRF, they are prone to overfitting the scenes they are trained on, necessitating time-intensive training for each new scene~\cite{wang2021ibrnet, liu2022neural}.


This paper introduces \ourSystem, Generalizable Wireless Radiance Fields, which improves wireless signal propagation modeling in single scenes and enables generalization to unseen scenes.
In the optical domain, generalizable neural radiance fields~\cite{chen2021mvsnerf, trevithick2020grf, wang2021ibrnet, liu2022neural} have been developed based on view interpolation theory~\cite{chen2023view}. 
These models generalize by interpolating images across camera views, capturing shared geometric and appearance features from multiple perspectives.
In contrast, equivalent view interpolation theory for the wireless signal domain remains unexplored. 
To address this gap, we demonstrate that the spatial spectrum—representing received signal strength from all directions at the receiver~(see Section~\ref{sec_spatial_spectrum})—for a transmitter can be interpreted as an interpolation of the spectra of its geographically proximate neighboring transmitters. 
By leveraging the spectra of these neighbors, \ourSystem learns a generalizable wireless radiance field, enabling spatial spectrum inference in unseen scenes.


Constructing generalizable wireless radiance fields requires a fundamentally different approach from optical radiance fields due to their distinct propagation behaviors, driven by the nanometer-scale wavelengths of light compared to the centimeter-scale wavelengths of wireless signals.
Visible light primarily reflects off surfaces~\cite{pathak2015visible}, enabling optical radiance field methods to compute voxel attributes, \ie color and opacity, by integrating multi-view images~\cite{chen2021mvsnerf, trevithick2020grf}. 
In contrast, wireless signals interact with obstacles in more complex ways, including absorption, reflection, diffraction, and scattering~\cite{1451581, na2022huygens}.
This complexity makes two-attribute voxel representations insufficient for modeling wireless signal propagation. 
Such representations limit modeling precision within individual scenes and hinder generalization to unseen environments.
To address these challenges, we propose representing voxel attributes as latent vectors within a latent voxel feature space. 
This space is constructed using a geometry-aware Transformer encoder with a cross-attention mechanism, which processes the spatial spectra of neighboring transmitters and their positional relationships with the target transmitter. 
By capturing the intricate propagation behaviors of wireless signals and geometry-dependent features, \ourSystem improves voxel representation and facilitates generalization.



Furthermore, since each voxel's attributes are represented as a latent vector rather than two single values—such as signal emission and attenuation in \nerft~or color and opacity in optical radiance fields—we propose a novel neural-driven ray tracing algorithm to compute the received signal.
This algorithm emits rays in all directions from the receiver, with each ray sampling multiple voxels along its path. 
The sampled voxels are interpreted as sequential tokens, and a Transformer architecture aggregates their features to compute the received signal.
As voxel vectors may encode both amplitude and phase information inherent to wireless signals, the Transformer is designed as a multi-head model to capture diverse relationships among voxel features.



Extensive experimental results confirm that \ourSystem infers high-quality spatial spectra in both seen and unseen scenes. 
A case study on spectrum-based Angle of Arrival~(AoA) estimation showcases \ourSystem's effectiveness for downstream tasks.
Our key contributions are summarized as follows:

\begin{itemize}[leftmargin=*, align=left]

\item We present \ourSystem, the first generalizable wireless radiance field framework for modeling wireless signal propagation, capable of generalizing to unseen scenes.


\item We identify that the spatial spectrum of a transmitter can be interpreted as an interpolation of spectra from its geographically proximate transmitters.


\item We propose a geometry-aware scene representation module with a cross-attention layer to effectively fuse neighboring spatial spectra for learning voxel latent features.


\item We develop a neural-driven ray tracing algorithm that incorporates complex-valued wireless signal characteristics to achieve accurate spatial spectrum synthesis.

\end{itemize}

In summary, we propose \ourSystem, a generalizable framework for wireless signal propagation modeling, enabling more flexible and efficient wireless communication and sensing applications across diverse environments. 
Details of the impact statements are summarized in Section~\ref{sec_impact}.




