%The Model Driven Engineering (MDE) community also presents promising research for the automatic generation of architectural components. 
Over the years, a lot of promising work has been done in specifying software architecture using Architecture Description Languages (ADL) and further supporting the analysis and code generation of components using model transformations \cite{brun2008code, malavolta2012industry}. In particular, an approach for the generation of source code in the Go programming language from $\pi$-ADL was proposed by Cavalcante et al. \cite{cavalcante2014architectureadl}. Further, an ADL for architecture modeling, code generation, and simulation of Cyber-Physical-Systems was proposed by Sharaf et al. \cite{muccini2017caps,muccini_arduino}. Bardaro et al. in their recent work \cite{aadl-robotics} make use of AADL for modeling and further code generation in the robotics domain. 
However, to the best of our knowledge, not much work has been done in the serverless domain. Tankov et al.\cite{kotless} proposes Kotless which aims to simplify serverless development through the use of a DSL and a plugin to generate deployment code. However, the developer must still develop the application logic and code. There have been some works in using DSLs for supporting the specification and generation of microservices. Rademacher et al.\cite{lemma-microservices}
propose an extensible approach to generate adaptable microservice code and deployment specifications from models developed using the modeling language, LEMMA. Further, Suljkanovi{\'c} et al.\cite{silveria} proposed Silveria, a DSL that allows users to model the architecture of microservice-based systems and further generate executable code using model transformations. However as pointed out by Malavolta et al.\cite{malavolta2012industry}, ADLs in general have a higher learning curve which often hinders practical adoption. 

Significant work has been done on using LLMs for code generation, including \cite{bareiss2022codegen, gong2023intendedcodegen, chen2023improvingcodegen, wang2023review_llmcodegeneration, jiang2024surveylargelanguagemodels}. There also exist several tools such as GitHub Copilot and the Cursor IDE. However, these focus on the generation of code snippets, such as classes or methods, and not at an architectural component level. 

While Eskandani et al. \cite{eskandani2024towards_aisystems} describe the application design of serverless systems using GenAI as an open research direction, it is through the lens of selecting an optimal pattern based on requirements. We seek to evaluate the capability of LLMs in generating architectural components by choosing serverless as the architectural style. To this end, we also conduct evaluations on the functionality, code quality, and similarity between human-written and generated code. 
