\documentclass[conference]{IEEEtran}
\IEEEoverridecommandlockouts
% The preceding line is only needed to identify funding in the first footnote. If that is unneeded, please comment it out.
%Template version as of 6/27/2024

\usepackage{cite}
\usepackage{amsmath,amssymb,amsfonts}
\usepackage{algorithmic}
\usepackage{graphicx}
\usepackage{textcomp}
\usepackage{xcolor}
\usepackage{orcidlink}
\usepackage{tcolorbox}
\usepackage{todonotes}
\usepackage{booktabs}
\usepackage{hyperref}
\usepackage{xurl}
\usepackage{array}
\usepackage{tcolorbox}
\usepackage{fontawesome5}
\usepackage{makecell}
\usepackage{array, multirow, boldline}
\usepackage{caption}

\def\BibTeX{{\rm B\kern-.05em{\sc i\kern-.025em b}\kern-.08em
    T\kern-.1667em\lower.7ex\hbox{E}\kern-.125emX}}

\usepackage[T1]{fontenc}
\begin{document}

\title{LLMs for Generation of Architectural Components: An Exploratory Empirical Study in the Serverless World}

\author{\IEEEauthorblockN{Shrikara Arun* 
\orcidlink{0009-0005-8543-6130}
} 
\IEEEauthorblockA{
\textit{Software Engineering Research Centre} \\
% \textit{SERC, IIIT Hyderabad, India}\\
\textit{IIIT Hyderabad, India}\\
\textit{shrikara.a@students.iiit.ac.in}}
\thanks{*These authors contributed equally.}
\and
\IEEEauthorblockN{Meghana Tedla* 
\orcidlink{0009-0000-8540-8790}
}
\IEEEauthorblockA{
\textit{Software Engineering Research Centre} \\
% \textit{SERC, IIIT Hyderabad, India}\\
\textit{IIIT Hyderabad, India}\\
\textit{meghana.tedla@students.iiit.ac.in}}
\and
\IEEEauthorblockN{Karthik Vaidhyanathan 
\orcidlink{0000-0003-2317-6175}
}
\IEEEauthorblockA{
\textit{Software Engineering Research Centre} \\
% \textit{SERC, IIIT Hyderabad, India}\\
\textit{IIIT Hyderabad, India}\\
\textit{karthik.vaidhyanathan@iiit.ac.in}}
 }
% \vspace*{-1cm}
\maketitle
% \begingroup\renewcommand\thefootnote{}
% \footnotetext{*Equal contribution}

\newcommand{\magenta}[1]{\textcolor{magenta}{#1}}
\newcommand{\blue}[1]{\textcolor{blue}{#1}}
\newcommand{\Human}{\stackon{\faUser}{\textcolor{green}{\faCheck}}}
\newcommand{\NoHuman}{\stackon{\faUser}{\textcolor{red}{\faTimes}}}

\begin{abstract}
Recently, the exponential growth in capability and pervasiveness of Large Language Models (LLMs) has led to significant work done in the field of code generation. However, this generation has been limited to code snippets. Going one step further, our desideratum is to automatically generate architectural components. This would not only speed up development time, but would also enable us to eventually completely skip the development phase, moving directly from design decisions to deployment. To this end, we conduct an exploratory study on the capability of LLMs to generate architectural components for Functions as a Service (FaaS), commonly known as serverless functions. The small size of their architectural components make this architectural style amenable for generation using current LLMs compared to other styles like monoliths and microservices. We perform the study by systematically selecting open source serverless repositories, masking a serverless function and utilizing state of the art LLMs provided with varying levels of context information about the overall system to generate the masked function. We evaluate correctness through existing tests present in the repositories and use metrics from the Software Engineering (SE) and Natural Language Processing (NLP) domains to evaluate code quality and the degree of similarity between human and LLM generated code respectively. Along with our findings, we also present a discussion on the path forward for using GenAI in architectural component generation.
\end{abstract}

\begin{IEEEkeywords}
Architectural Component Generation, LLM, Serverless.
\end{IEEEkeywords}

\vspace{-12pt}
\section{Introduction}
\section{Introduction}
\label{sec:introduction}
The business processes of organizations are experiencing ever-increasing complexity due to the large amount of data, high number of users, and high-tech devices involved \cite{martin2021pmopportunitieschallenges, beerepoot2023biggestbpmproblems}. This complexity may cause business processes to deviate from normal control flow due to unforeseen and disruptive anomalies \cite{adams2023proceddsriftdetection}. These control-flow anomalies manifest as unknown, skipped, and wrongly-ordered activities in the traces of event logs monitored from the execution of business processes \cite{ko2023adsystematicreview}. For the sake of clarity, let us consider an illustrative example of such anomalies. Figure \ref{FP_ANOMALIES} shows a so-called event log footprint, which captures the control flow relations of four activities of a hypothetical event log. In particular, this footprint captures the control-flow relations between activities \texttt{a}, \texttt{b}, \texttt{c} and \texttt{d}. These are the causal ($\rightarrow$) relation, concurrent ($\parallel$) relation, and other ($\#$) relations such as exclusivity or non-local dependency \cite{aalst2022pmhandbook}. In addition, on the right are six traces, of which five exhibit skipped, wrongly-ordered and unknown control-flow anomalies. For example, $\langle$\texttt{a b d}$\rangle$ has a skipped activity, which is \texttt{c}. Because of this skipped activity, the control-flow relation \texttt{b}$\,\#\,$\texttt{d} is violated, since \texttt{d} directly follows \texttt{b} in the anomalous trace.
\begin{figure}[!t]
\centering
\includegraphics[width=0.9\columnwidth]{images/FP_ANOMALIES.png}
\caption{An example event log footprint with six traces, of which five exhibit control-flow anomalies.}
\label{FP_ANOMALIES}
\end{figure}

\subsection{Control-flow anomaly detection}
Control-flow anomaly detection techniques aim to characterize the normal control flow from event logs and verify whether these deviations occur in new event logs \cite{ko2023adsystematicreview}. To develop control-flow anomaly detection techniques, \revision{process mining} has seen widespread adoption owing to process discovery and \revision{conformance checking}. On the one hand, process discovery is a set of algorithms that encode control-flow relations as a set of model elements and constraints according to a given modeling formalism \cite{aalst2022pmhandbook}; hereafter, we refer to the Petri net, a widespread modeling formalism. On the other hand, \revision{conformance checking} is an explainable set of algorithms that allows linking any deviations with the reference Petri net and providing the fitness measure, namely a measure of how much the Petri net fits the new event log \cite{aalst2022pmhandbook}. Many control-flow anomaly detection techniques based on \revision{conformance checking} (hereafter, \revision{conformance checking}-based techniques) use the fitness measure to determine whether an event log is anomalous \cite{bezerra2009pmad, bezerra2013adlogspais, myers2018icsadpm, pecchia2020applicationfailuresanalysispm}. 

The scientific literature also includes many \revision{conformance checking}-independent techniques for control-flow anomaly detection that combine specific types of trace encodings with machine/deep learning \cite{ko2023adsystematicreview, tavares2023pmtraceencoding}. Whereas these techniques are very effective, their explainability is challenging due to both the type of trace encoding employed and the machine/deep learning model used \cite{rawal2022trustworthyaiadvances,li2023explainablead}. Hence, in the following, we focus on the shortcomings of \revision{conformance checking}-based techniques to investigate whether it is possible to support the development of competitive control-flow anomaly detection techniques while maintaining the explainable nature of \revision{conformance checking}.
\begin{figure}[!t]
\centering
\includegraphics[width=\columnwidth]{images/HIGH_LEVEL_VIEW.png}
\caption{A high-level view of the proposed framework for combining \revision{process mining}-based feature extraction with dimensionality reduction for control-flow anomaly detection.}
\label{HIGH_LEVEL_VIEW}
\end{figure}

\subsection{Shortcomings of \revision{conformance checking}-based techniques}
Unfortunately, the detection effectiveness of \revision{conformance checking}-based techniques is affected by noisy data and low-quality Petri nets, which may be due to human errors in the modeling process or representational bias of process discovery algorithms \cite{bezerra2013adlogspais, pecchia2020applicationfailuresanalysispm, aalst2016pm}. Specifically, on the one hand, noisy data may introduce infrequent and deceptive control-flow relations that may result in inconsistent fitness measures, whereas, on the other hand, checking event logs against a low-quality Petri net could lead to an unreliable distribution of fitness measures. Nonetheless, such Petri nets can still be used as references to obtain insightful information for \revision{process mining}-based feature extraction, supporting the development of competitive and explainable \revision{conformance checking}-based techniques for control-flow anomaly detection despite the problems above. For example, a few works outline that token-based \revision{conformance checking} can be used for \revision{process mining}-based feature extraction to build tabular data and develop effective \revision{conformance checking}-based techniques for control-flow anomaly detection \cite{singh2022lapmsh, debenedictis2023dtadiiot}. However, to the best of our knowledge, the scientific literature lacks a structured proposal for \revision{process mining}-based feature extraction using the state-of-the-art \revision{conformance checking} variant, namely alignment-based \revision{conformance checking}.

\subsection{Contributions}
We propose a novel \revision{process mining}-based feature extraction approach with alignment-based \revision{conformance checking}. This variant aligns the deviating control flow with a reference Petri net; the resulting alignment can be inspected to extract additional statistics such as the number of times a given activity caused mismatches \cite{aalst2022pmhandbook}. We integrate this approach into a flexible and explainable framework for developing techniques for control-flow anomaly detection. The framework combines \revision{process mining}-based feature extraction and dimensionality reduction to handle high-dimensional feature sets, achieve detection effectiveness, and support explainability. Notably, in addition to our proposed \revision{process mining}-based feature extraction approach, the framework allows employing other approaches, enabling a fair comparison of multiple \revision{conformance checking}-based and \revision{conformance checking}-independent techniques for control-flow anomaly detection. Figure \ref{HIGH_LEVEL_VIEW} shows a high-level view of the framework. Business processes are monitored, and event logs obtained from the database of information systems. Subsequently, \revision{process mining}-based feature extraction is applied to these event logs and tabular data input to dimensionality reduction to identify control-flow anomalies. We apply several \revision{conformance checking}-based and \revision{conformance checking}-independent framework techniques to publicly available datasets, simulated data of a case study from railways, and real-world data of a case study from healthcare. We show that the framework techniques implementing our approach outperform the baseline \revision{conformance checking}-based techniques while maintaining the explainable nature of \revision{conformance checking}.

In summary, the contributions of this paper are as follows.
\begin{itemize}
    \item{
        A novel \revision{process mining}-based feature extraction approach to support the development of competitive and explainable \revision{conformance checking}-based techniques for control-flow anomaly detection.
    }
    \item{
        A flexible and explainable framework for developing techniques for control-flow anomaly detection using \revision{process mining}-based feature extraction and dimensionality reduction.
    }
    \item{
        Application to synthetic and real-world datasets of several \revision{conformance checking}-based and \revision{conformance checking}-independent framework techniques, evaluating their detection effectiveness and explainability.
    }
\end{itemize}

The rest of the paper is organized as follows.
\begin{itemize}
    \item Section \ref{sec:related_work} reviews the existing techniques for control-flow anomaly detection, categorizing them into \revision{conformance checking}-based and \revision{conformance checking}-independent techniques.
    \item Section \ref{sec:abccfe} provides the preliminaries of \revision{process mining} to establish the notation used throughout the paper, and delves into the details of the proposed \revision{process mining}-based feature extraction approach with alignment-based \revision{conformance checking}.
    \item Section \ref{sec:framework} describes the framework for developing \revision{conformance checking}-based and \revision{conformance checking}-independent techniques for control-flow anomaly detection that combine \revision{process mining}-based feature extraction and dimensionality reduction.
    \item Section \ref{sec:evaluation} presents the experiments conducted with multiple framework and baseline techniques using data from publicly available datasets and case studies.
    \item Section \ref{sec:conclusions} draws the conclusions and presents future work.
\end{itemize}

\section{Background} \label{background}
\section{Background}\label{sec:backgrnd}

\subsection{Cold Start Latency and Mitigation Techniques}

Traditional FaaS platforms mitigate cold starts through snapshotting, lightweight virtualization, and warm-state management. Snapshot-based methods like \textbf{REAP} and \textbf{Catalyzer} reduce initialization time by preloading or restoring container states but require significant memory and I/O resources, limiting scalability~\cite{dong_catalyzer_2020, ustiugov_benchmarking_2021}. Lightweight virtualization solutions, such as \textbf{Firecracker} microVMs, achieve fast startup times with strong isolation but depend on robust infrastructure, making them less adaptable to fluctuating workloads~\cite{agache_firecracker_2020}. Warm-state management techniques like \textbf{Faa\$T}~\cite{romero_faa_2021} and \textbf{Kraken}~\cite{vivek_kraken_2021} keep frequently invoked containers ready, balancing readiness and cost efficiency under predictable workloads but incurring overhead when demand is erratic~\cite{romero_faa_2021, vivek_kraken_2021}. While these methods perform well in resource-rich cloud environments, their resource intensity challenges applicability in edge settings.

\subsubsection{Edge FaaS Perspective}

In edge environments, cold start mitigation emphasizes lightweight designs, resource sharing, and hybrid task distribution. Lightweight execution environments like unikernels~\cite{edward_sock_2018} and \textbf{Firecracker}~\cite{agache_firecracker_2020}, as used by \textbf{TinyFaaS}~\cite{pfandzelter_tinyfaas_2020}, minimize resource usage and initialization delays but require careful orchestration to avoid resource contention. Function co-location, demonstrated by \textbf{Photons}~\cite{v_dukic_photons_2020}, reduces redundant initializations by sharing runtime resources among related functions, though this complicates isolation in multi-tenant setups~\cite{v_dukic_photons_2020}. Hybrid offloading frameworks like \textbf{GeoFaaS}~\cite{malekabbasi_geofaas_2024} balance edge-cloud workloads by offloading latency-tolerant tasks to the cloud and reserving edge resources for real-time operations, requiring reliable connectivity and efficient task management. These edge-specific strategies address cold starts effectively but introduce challenges in scalability and orchestration.

\subsection{Predictive Scaling and Caching Techniques}

Efficient resource allocation is vital for maintaining low latency and high availability in serverless platforms. Predictive scaling and caching techniques dynamically provision resources and reduce cold start latency by leveraging workload prediction and state retention.
Traditional FaaS platforms use predictive scaling and caching to optimize resources, employing techniques (OFC, FaasCache) to reduce cold starts. However, these methods rely on centralized orchestration and workload predictability, limiting their effectiveness in dynamic, resource-constrained edge environments.



\subsubsection{Edge FaaS Perspective}

Edge FaaS platforms adapt predictive scaling and caching techniques to constrain resources and heterogeneous environments. \textbf{EDGE-Cache}~\cite{kim_delay-aware_2022} uses traffic profiling to selectively retain high-priority functions, reducing memory overhead while maintaining readiness for frequent requests. Hybrid frameworks like \textbf{GeoFaaS}~\cite{malekabbasi_geofaas_2024} implement distributed caching to balance resources between edge and cloud nodes, enabling low-latency processing for critical tasks while offloading less critical workloads. Machine learning methods, such as clustering-based workload predictors~\cite{gao_machine_2020} and GRU-based models~\cite{guo_applying_2018}, enhance resource provisioning in edge systems by efficiently forecasting workload spikes. These innovations effectively address cold start challenges in edge environments, though their dependency on accurate predictions and robust orchestration poses scalability challenges.

\subsection{Decentralized Orchestration, Function Placement, and Scheduling}

Efficient orchestration in serverless platforms involves workload distribution, resource optimization, and performance assurance. While traditional FaaS platforms rely on centralized control, edge environments require decentralized and adaptive strategies to address unique challenges such as resource constraints and heterogeneous hardware.



\subsubsection{Edge FaaS Perspective}

Edge FaaS platforms adopt decentralized and adaptive orchestration frameworks to meet the demands of resource-constrained environments. Systems like \textbf{Wukong} distribute scheduling across edge nodes, enhancing data locality and scalability while reducing network latency. Lightweight frameworks such as \textbf{OpenWhisk Lite}~\cite{kravchenko_kpavelopenwhisk-light_2024} optimize resource allocation by decentralizing scheduling policies, minimizing cold starts and latency in edge setups~\cite{benjamin_wukong_2020}. Hybrid solutions like \textbf{OpenFaaS}~\cite{noauthor_openfaasfaas_2024} and \textbf{EdgeMatrix}~\cite{shen_edgematrix_2023} combine edge-cloud orchestration to balance resource utilization, retaining latency-sensitive functions at the edge while offloading non-critical workloads to the cloud. While these approaches improve flexibility, they face challenges in maintaining coordination and ensuring consistent performance across distributed nodes.



\vspace{-7pt}
\section{Related Work} \label{related-work}
\section{RELATED WORK}
\label{sec:relatedwork}
In this section, we describe the previous works related to our proposal, which are divided into two parts. In Section~\ref{sec:relatedwork_exoplanet}, we present a review of approaches based on machine learning techniques for the detection of planetary transit signals. Section~\ref{sec:relatedwork_attention} provides an account of the approaches based on attention mechanisms applied in Astronomy.\par

\subsection{Exoplanet detection}
\label{sec:relatedwork_exoplanet}
Machine learning methods have achieved great performance for the automatic selection of exoplanet transit signals. One of the earliest applications of machine learning is a model named Autovetter \citep{MCcauliff}, which is a random forest (RF) model based on characteristics derived from Kepler pipeline statistics to classify exoplanet and false positive signals. Then, other studies emerged that also used supervised learning. \cite{mislis2016sidra} also used a RF, but unlike the work by \citet{MCcauliff}, they used simulated light curves and a box least square \citep[BLS;][]{kovacs2002box}-based periodogram to search for transiting exoplanets. \citet{thompson2015machine} proposed a k-nearest neighbors model for Kepler data to determine if a given signal has similarity to known transits. Unsupervised learning techniques were also applied, such as self-organizing maps (SOM), proposed \citet{armstrong2016transit}; which implements an architecture to segment similar light curves. In the same way, \citet{armstrong2018automatic} developed a combination of supervised and unsupervised learning, including RF and SOM models. In general, these approaches require a previous phase of feature engineering for each light curve. \par

%DL is a modern data-driven technology that automatically extracts characteristics, and that has been successful in classification problems from a variety of application domains. The architecture relies on several layers of NNs of simple interconnected units and uses layers to build increasingly complex and useful features by means of linear and non-linear transformation. This family of models is capable of generating increasingly high-level representations \citep{lecun2015deep}.

The application of DL for exoplanetary signal detection has evolved rapidly in recent years and has become very popular in planetary science.  \citet{pearson2018} and \citet{zucker2018shallow} developed CNN-based algorithms that learn from synthetic data to search for exoplanets. Perhaps one of the most successful applications of the DL models in transit detection was that of \citet{Shallue_2018}; who, in collaboration with Google, proposed a CNN named AstroNet that recognizes exoplanet signals in real data from Kepler. AstroNet uses the training set of labelled TCEs from the Autovetter planet candidate catalog of Q1–Q17 data release 24 (DR24) of the Kepler mission \citep{catanzarite2015autovetter}. AstroNet analyses the data in two views: a ``global view'', and ``local view'' \citep{Shallue_2018}. \par


% The global view shows the characteristics of the light curve over an orbital period, and a local view shows the moment at occurring the transit in detail

%different = space-based

Based on AstroNet, researchers have modified the original AstroNet model to rank candidates from different surveys, specifically for Kepler and TESS missions. \citet{ansdell2018scientific} developed a CNN trained on Kepler data, and included for the first time the information on the centroids, showing that the model improves performance considerably. Then, \citet{osborn2020rapid} and \citet{yu2019identifying} also included the centroids information, but in addition, \citet{osborn2020rapid} included information of the stellar and transit parameters. Finally, \citet{rao2021nigraha} proposed a pipeline that includes a new ``half-phase'' view of the transit signal. This half-phase view represents a transit view with a different time and phase. The purpose of this view is to recover any possible secondary eclipse (the object hiding behind the disk of the primary star).


%last pipeline applies a procedure after the prediction of the model to obtain new candidates, this process is carried out through a series of steps that include the evaluation with Discovery and Validation of Exoplanets (DAVE) \citet{kostov2019discovery} that was adapted for the TESS telescope.\par
%



\subsection{Attention mechanisms in astronomy}
\label{sec:relatedwork_attention}
Despite the remarkable success of attention mechanisms in sequential data, few papers have exploited their advantages in astronomy. In particular, there are no models based on attention mechanisms for detecting planets. Below we present a summary of the main applications of this modeling approach to astronomy, based on two points of view; performance and interpretability of the model.\par
%Attention mechanisms have not yet been explored in all sub-areas of astronomy. However, recent works show a successful application of the mechanism.
%performance

The application of attention mechanisms has shown improvements in the performance of some regression and classification tasks compared to previous approaches. One of the first implementations of the attention mechanism was to find gravitational lenses proposed by \citet{thuruthipilly2021finding}. They designed 21 self-attention-based encoder models, where each model was trained separately with 18,000 simulated images, demonstrating that the model based on the Transformer has a better performance and uses fewer trainable parameters compared to CNN. A novel application was proposed by \citet{lin2021galaxy} for the morphological classification of galaxies, who used an architecture derived from the Transformer, named Vision Transformer (VIT) \citep{dosovitskiy2020image}. \citet{lin2021galaxy} demonstrated competitive results compared to CNNs. Another application with successful results was proposed by \citet{zerveas2021transformer}; which first proposed a transformer-based framework for learning unsupervised representations of multivariate time series. Their methodology takes advantage of unlabeled data to train an encoder and extract dense vector representations of time series. Subsequently, they evaluate the model for regression and classification tasks, demonstrating better performance than other state-of-the-art supervised methods, even with data sets with limited samples.

%interpretation
Regarding the interpretability of the model, a recent contribution that analyses the attention maps was presented by \citet{bowles20212}, which explored the use of group-equivariant self-attention for radio astronomy classification. Compared to other approaches, this model analysed the attention maps of the predictions and showed that the mechanism extracts the brightest spots and jets of the radio source more clearly. This indicates that attention maps for prediction interpretation could help experts see patterns that the human eye often misses. \par

In the field of variable stars, \citet{allam2021paying} employed the mechanism for classifying multivariate time series in variable stars. And additionally, \citet{allam2021paying} showed that the activation weights are accommodated according to the variation in brightness of the star, achieving a more interpretable model. And finally, related to the TESS telescope, \citet{morvan2022don} proposed a model that removes the noise from the light curves through the distribution of attention weights. \citet{morvan2022don} showed that the use of the attention mechanism is excellent for removing noise and outliers in time series datasets compared with other approaches. In addition, the use of attention maps allowed them to show the representations learned from the model. \par

Recent attention mechanism approaches in astronomy demonstrate comparable results with earlier approaches, such as CNNs. At the same time, they offer interpretability of their results, which allows a post-prediction analysis. \par



\section{Study Design} \label{study-design}
\vspace{-2mm}
% !TEX root = main.tex
The \emph{goal} of this study is to empirically evaluate how code-comment coherence, through a quality-aware selection strategy grounded on SIDE, impacts the effectiveness and training efficiency of neural code summarization models.

More specifically, the study aims to address the following research questions:

\begin{itemize}[itemindent=0.25cm]
	\item[\textbf{RQ$_{0}$}:] \textit{How do code summarization datasets measure up in terms of code-comment coherence?}
	In this preliminary question, we assess the coherence of code-comment pairs of datasets commonly used in code summarization. As we aim to use a coherence-aware strategy to optimize training sets, first of all, we would like to see how the coherence is distributed.
	\item[\textbf{RQ$_{1}$}:] \textit{How does a coherence-aware strategy selection impact the performance of neural code summarization models?}
	In this research question, we investigate how a targeted selection of training data based on code-comment coherence impacts the performance of neural code summarization models.
	\item[\textbf{RQ$_{2}$}:] \textit{How does the coherence-aware strategy selection compare with a random baseline?}
	In this research question, we test our hypothesis that code-comment coherence is a quality attribute that can be used to select training instances.
\end{itemize}

\subsection{Context Selection}
\label{subsec:context_selection}
The \emph{context} of our study consists of datasets containing pairs of \java methods with the associated summaries. 
For fine-tuning the models, we consider the two most important datasets from the state of the art: TL-CodeSum \cite{hu2018summarizing}, and Funcom \cite{leclair2019neural}.

The \textit{TL-CodeSum} dataset \cite{hu2018summarizing} is specifically designed for the code summarization task. It consists of $\sim$87k instances $ \langle code, summary \rangle$ extracted from GitHub repositories created from 2015 to 2016, and having at least 20 stars. In detail, Hu \etal \cite{hu2018summarizing} extracted the first sentence---likely to describe the overall method functionality---from the doc of each pair.

Similarly to TL-CodeSum, the \textit{Funcom} dataset \cite{leclair2019neural} is also specifically designed for code summarization. \textit{Funcom} consists of over 2.1M $ \langle code, summary \rangle$ pairs collected from the Sourcerer repository. As for TL-CodeSum, LeClair \etal \cite{leclair2019neural} only consider methods with their javadoc, extracting the first sentence as corresponding \textit{summary}.

Shi \etal \cite{shi2022we} found many noisy instances and duplicates in the above-described datasets and cleaned them up using their heuristic-based dataset-cleaning approach. For this reason, we use the cleaned versions of \textit{TL-CodeSum} and \textit{Funcom} provided by Shi \etal \cite{shi2022we}. The cleaned \textit{TL-CodeSum} contains 53,597 training instances, while the cleaned \textit{Funcom} contains 1,184,438 training instances.

The above datasets are built automatically, and no manual check was performed, \ie there is no guarantee of their quality. For this reason, we use two additional, manually curated datasets to test the models. The first one is \textit{CoderEval} \cite{yu2024codereval}, which consists of 230 Python and 230 \java code generation problems collected from open-source, high-starred projects which include \textit{original} and \textit{human-labeled} docstrings that should act as prompt for Code Generation models to generate the corresponding \textit{code}. 
The instances have been subject to manual screening, for which the main criterion is the probability of appearing in real-development scenarios. We focus on the \java set of problems, inverting the input and the output \ie from $\langle docstring, code \rangle$ to $\langle code, docstring \rangle$. 
To align the format of the pairs format, we performed an additional manual analysis in which one of the authors checked all the triplets with a second author to confirm the analysis. 
We found that some of the \textit{docstring}(s) contained more than a sentence. Therefore, to make them consistent with the previous dataset format (\eg single sentence), we extracted the first sentence from each \textit{docstring}. Still, we found 12 occurrences in which the corresponding \textit{original docstring} does not describe the \textit{code} (\eg ``\texttt{{@inheritDoc}}'', ``\texttt{@param modelName model name of the entity}'', and similar). We also excluded \textit{docstring}: ``\texttt{Computes floor(\$log\_2 (n)\$) \$+ 1\$.}'' since it includes a formula not explained in natural language.
Again, to appropriately align the evaluation, we do not evaluate such instances, ending up with 218 \textit{original} instances.

The second manually-curated dataset we use is the one by Mastropaolo \etal \cite{mastropaolo2023robustness}. The dataset consists of 892 methods associated with their summary (\ie first sentence of the method documentation), collected from non-fork GitHub \java repositories with at least 300 commits, 50 contributors, and 25 stars. 
Such instances are in the form $\langle summary, code\rangle$ and, as for CoderEval \cite{yu2024codereval}, we inverted the input and the output \ie $\langle code, summary\rangle$. Mastropaolo \etal  analyzed such pairs to ensure their quality. We manually analyzed and cleaned them further (\eg ``\texttt{Adds an {@link CarrierService} to the {@linkCarrier}}'' into ``\texttt{Adds an CarrierService to the Carrier}''), as we had done for CoderEval. No instances were removed during such a manual analysis.

We remove the instances from the test sets which appear in the training sets of \textit{TL-CodeSum} and \textit{Funcom}. As a result, we remove ten instances from CoderEval, which are present only in the \textit{TL-CodeSum} training set.

\subsection{Study Methodology}
\label{subsec:exp_proc}
To answer RQ$_{0}$, we use SIDE to compute the degree to which the summaries of the studied datasets document their corresponding code. We did this for each instance of the training sets included in \textit{TL-CodeSum} and \textit{Funcom}. To understand the coherence of the training sets, we analyze the average and the distributions of the SIDE scores of the instances.\\

\addtolength{\extrarowheight}{\belowrulesep}
\aboverulesep=0pt
\belowrulesep=0pt
\begin{table}[t]
	\centering
	\caption{Different selections for \textit{TL-CodeSum} and \textit{Funcom} training sets.}
	\label{tab:dataset_w_strategies}
	\resizebox{0.6\columnwidth}{!}{%
		\begin{tabular}{lrrr}
			\toprule
			\cellcolor{black}\textcolor{white}{\textbf{Selection}} &  \cellcolor{black}\textcolor{white}{\textbf{TL-CodeSum}} & \cellcolor{black}\textcolor{white}{\textbf{Funcom}} \\
			\midrule
			Full & 53,597 & 1,184,438 \\
			\midrule
			\side{0.5} & 50,073 & 1,080,649 \\
			\side{0.6} & 48,146 & 1,031,647 \\
			\side{0.7} & 44,853 & 952,265 \\
			\side{0.8} & 38,733 & 813,998 \\
			\side{0.9} & 26,258 & 540,170 \\
			\bottomrule
	\end{tabular}}
\end{table}

To answer RQ$_{1}$, we use the SIDE-based filter we define in \secref{sec:selection_strategies}. We use five threshold values, \ie 0.5, 0.6, 0.7, 0.8, and 0.9. We do not use thresholds lower than 0.5 because they would result in negligible dataset reductions (lower than 10\% for both), as we will observe in the results of RQ$_{0}$.
We report information about the different datasets in \tabref{tab:dataset_w_strategies}. 
We apply each filter on the training sets of \textit{TL-CodeSum} and \textit{Funcom}. Such filtering leads to the definition of five new versions of both datasets.

We fine-tune a pre-trained Transformer-based model for each dataset version, \ie both the base one and its six filtered versions, producing 12 fine-tuned models.
We choose to leverage the pre-trained \emph{CodeT5+} \cite{wang2023codet5+} since it has been largely used for code-related tasks \cite{ahmed2024automatic,phan2024repohyper,yang2024important} and, more important, in the code summarization approaches described above. This model is built on the backbone of the well-known T5 model by Raffel \etal \cite{raffel2020exploring}, yet it benefits from specific enhancements tailored for code understanding and generation tasks. During the pre-training phase, \emph{CodeT5+} is first trained on unimodal data, which includes code and comments, employing a combination of pre-training objectives such as span-denoising \cite{raffel2020exploring} and Causal Language Modeling \cite{soltan2022alexatm,tay2022ul2}. Then, it is pre-trained on bi-modal data where pre-training objectives such as text-code contrastive learning, text-code matching, and text-code causal language modeling are employed. It comes with different variants: (i) \emph{CodeT5+} 220M, (ii) \emph{CodeT5+} 770M, (iii) \emph{CodeT5+} 2B, (iv) \emph{CodeT5+} 6B, and (vi) \emph{CodeT5+} 16B.
Since our experimental design would require training, validating and testing 12 models, we decided to fine-tune the \emph{CodeT5+} variant featuring 220M trainable parameters.
This choice aligns with the goal of our investigation: Rather than proposing a new code summarization technique, we aim to use a model that offers a favorable balance between size and training time while still allowing us to observe the relevant phenomenon (if present).

Considering the extensive array of our experiments, we fine-tune for 20 epochs using a batch size of 16. Additionally, we restrict the input length to 512 tokens and the output to 128 tokens, consistent with previous studies leveraging the two datasets we used \cite{mastropaolo2022using,zhou2022automatic,tufano2023automating}. In addition, we conduct the fine-tuning using the standard hyperparameters for \emph{CodeT5+}, which include the AdamW optimizer \cite{loshchilov2017decoupled} and a learning rate of 2e-5, which is the one recommended for (Code)T5 and also used in works leveraging such models \cite{mastropaolo2023towards,ciniselli2024generalizability,mastropaolo2024vul}.

To prevent overfitting, we employ early stopping \cite{prechelt2002early}. After each epoch, we assess the performance of the models by computing the number of correct predictions on the validation set. 
In line with similar research \cite{mastropaolo2023towards,ciniselli2024generalizability}, we implement early stopping with patience of 5 epochs and a delta of 0.01. This means that training will stop if the model's performance does not improve by at least 0.01 for five consecutive epochs. We then select the best-performing checkpoint before early stopping.
We fine-tune a \emph{CodeT5+} model for each training set derived from the selection strategy \ie 12 (2 datasets $\times$ six variants).

After training the models, we assess their performance on the test set dataset that, as previously explained, are the \textit{CoderEval} \cite{yu2024codereval}, and the one from Mastropaolo \etal \cite{mastropaolo2023robustness} which we refer to as the \textit{golden sets}.

In the inference phase, we employ a beam search decoding strategy. In detail, with $k \in \{1, 3, 5\}$, we allow each model to generate the $k$ most probable candidate \textit{summaries} for the given \textit{code}.
To evaluate the generated summaries of each model, we compute the following metrics: BLEU \cite{papineni2002bleu}, METEOR \cite{banerjee:acl2005}, and ROUGE-L \cite{lin2004rouge}.
\textbf{BLEU} is a metric that expresses, within a range from 0 to 1, the similarity between a generated text (candidate) and the target one (oracle). It computes the percentage of $n$-grams of the generated text that appear in the target, where $n \in \{1, 2, 3, 4\}$. 
\textbf{METEOR} is computed as the harmonic mean of unigram precision and recall, with the latter weighted higher than the former. It ranges from 0 to 1. 
\textbf{ROUGE-L} is computed as the length of the longest common subsequence (LCS) between the generated text and the target one and measures the recall by considering the proportion of the LCS relative to the length of the target text.
We do not use SIDE \cite{mastropaolo2024evaluating} as it was employed for selecting training instances and could therefore be unnaturally biased in favor of models trained on filtered datasets.
Also, we do not compute the percentage of exact matches for three reasons. First, exact matches might underestimate the actual performances of the model. Indeed, an exact match implies a correct summary, but many alternative summaries might be as correct (or even more correct, in theory) as the ones in the ground truth for the very nature of this task. Second (also related to the previous point), \textit{CoderEval} \cite{yu2024codereval} provides two summaries for each coding instance, namely \textit{original} (\ie the docstring collected from the original source code), and \textit{human} (\ie the docstring written from scratch by developers during the benchmark creation \cite{yu2024codereval}). The model could have correctly generated only one of them, which are, by definition, both correct alternatives, thus leading to inconsistent results. Third, the dataset provided by Mastropaolo \etal~\cite{mastropaolo2023robustness} includes three different yet semantically equivalent code summaries for each \java method. As previously noted, each of these alternative descriptions is a valid candidate summary.

\begin{figure}[t]
	\centering
	\includegraphics[width=0.65\linewidth]{distributions-plot.pdf}
	\caption{Distribution of SIDE scores for \textit{TL-CodeSum} and \textit{Funcom} training instances.}
	\label{fig:rq0_training_distribution}
\end{figure}

\begin{table*}[t]
	\centering
	\caption{Performance metrics on Top-1 predictions for CoderEval.}
	\label{tab:performance_metrics_codereval}
	\resizebox{\linewidth}{!}{%
		\begin{tabular}{c|l|r|r|rrr|rrr|rrr}
			\toprule
			\rowcolor{black}
			&  &  &  & \multicolumn{3}{c}{\textcolor{white}{\textbf{CoderEval-Original \cite{yu2024codereval}}}} & \multicolumn{3}{c}{\textcolor{white}{\textbf{CoderEval-Human \cite{yu2024codereval}}}} & \multicolumn{3}{c}{\textcolor{white}{\textbf{Mastropaolo \etal \cite{mastropaolo2023robustness}}}} \\
			\rowcolor{gray!20}
			\textbf{Dataset} & \textbf{Selection} & \textbf{\#Tokens} & \textbf{(\%) Saving} & \textbf{BLEU-4} & \textbf{METEOR} & \textbf{ROUGE} & \textbf{BLEU-4} & \textbf{METEOR} & \textbf{ROUGE} & \textbf{BLEU-4} & \textbf{METEOR} & \textbf{ROUGE} \\
			\midrule
			\multirow{6}{*}{\textit{TL-CodeSum \cite{hu2018summarizing}}}
			
			& \emph{Full}      & \cellcolor[HTML]{656565}\color[HTML]{FFFFFF} $\uparrow$ 7.9M &  \cellcolor[HTML]{a3070c}\color[HTML]{FFFFFF} --   & 11.72 & 17.84 & 35.01 & 6.41 & 14.28 & 30.51 & 6.37 & 13.04 & 27.09 \\
			\cmidrule(r){2-13}
			& \side{0.5} & 7.4M & 7\%  & 12.41 & 17.73 & 35.41 & 6.32 & 14.28 & 31.08 & 6.36 & 12.93 & 27.47\\
			& \side{0.6} & 7.1M & 10\% & 13.22 & 17.98 & 36.49 & 6.98 & 14.03 & 31.06 & 6.61 & 13.07 & 27.49\\
			& \side{0.7} & 6.6M & 16\% & 13.17 & 17.93 & 36.14 & 6.79 & 14.83 & 31.90 & 6.30 & 12.95 & 27.60\\
			& \side{0.8} & 5.6M & 28\% & 11.99 & 17.54 & 34.60 & 6.23 & 13.94 & 29.72 & 6.02 & 13.01 & 27.63\\
			& \side{0.9} & \cellcolor[HTML]{656565}\color[HTML]{FFFFFF}  $\downarrow$ 3.7M & \cellcolor[HTML]{026329}\color[HTML]{FFFFFF} 51\% & 11.60 & 17.07 & 33.67 & 6.56 & 14.19 & 30.53 & 5.62 & 12.75 & 27.26   \\
			\midrule
			\rowcolor[gray]{.85} & & & & & & & & & & & & \\
			\midrule
			\multirow{6}{*}{\textit{Funcom \cite{leclair2019neural}}} & \emph{Full}      & \cellcolor[HTML]{656565}\color[HTML]{FFFFFF} $\uparrow$ 108.7M  & \cellcolor[HTML]{a3070c}\color[HTML]{FFFFFF} --   & 14.25 & 17.61 & 36.53 & 5.93 & 12.95 & 28.36 & 6.77 & 12.94 & 28.00 \\
			\cmidrule(r){2-13}
			& \side{0.5} & 99.5M & 9\%  & 15.04 & 18.16 & 37.08 & 6.15 & 13.14 & 29.05 & 6.84 & 12.87 & 27.93 \\
			& \side{0.6} & 95.0M & 13\% & 16.19 & 19.30 & 38.38 & 7.02 & 14.07 & 30.25 & 7.03 & 12.99 & 28.33 \\
			& \side{0.7} & 87.7M & 20\% & 14.77 & 18.92 & 37.39 & 7.49 & 14.19 & 30.14 & 6.62 & 12.79 & 27.78 \\
			& \side{0.8} & 74.2M & 31\% & 14.10 & 18.34 & 37.04 & 6.53 & 13.38 & 28.45 & 6.93 & 12.78 & 27.99 \\
			& \side{0.9} & \cellcolor[HTML]{656565}\color[HTML]{FFFFFF} $\downarrow$ 49.0M & \cellcolor[HTML]{026329}\color[HTML]{FFFFFF} 54\% & 13.65 & 18.08 & 37.12 & 6.78 & 13.61 & 30.07 & 6.81 & 12.95 & 28.07 \\
			\bottomrule
	\end{tabular}}
\end{table*}

We also perform statistical hypothesis tests (Wilcoxon signed-rank test) \cite{wilcoxon1992individual} and Cliff's delta effect size \cite{grissom2005effect} to compare the distributions of the BLEU-4, METEOR, and ROUGE-L of the predictions generated by the different models trained on the filtered training sets with those of the models trained on the full training sets. We use Holm's correction \cite{holm1979simple} to adjust the \textit{p}-values for the multiple tests. We reject the \textit{null hypothesis} (there is no difference between the effectiveness of two given models) if the \emph{p}-value is lower than 0.05.

Finally, we study the Pareto front to analyze the cost-benefit trade-offs between the effectiveness of the models trained on the different selections of \textit{TL-CodeSum} and \textit{Funcom} (benefit, measured with \ie, BLEU, METEOR, and ROUGE-L) and the corresponding training dataset size (cost).

To answer RQ$_{2}$ we compare the selection strategy with \side{0.9} (\ie the most restrictive selection), with a \textit{Random} baseline. In detail, we randomly sample the same number of training instances as those selected with \side{0.9} from each dataset. We compare the effectiveness of the models trained with the training instances selected with \side{0.9} and \textit{Random} measured in terms of the previously described metrics (\ie BLEU-4, METEOR, and ROUGE-L). Again, we perform statistical hypothesis tests (Wilcoxon signed-rank test) \cite{wilcoxon1992individual} and compute the Cliff's delta effect size \cite{grissom2005effect} to compare the distributions of BLEU-4, METEOR, and ROUGE-L of the predictions generated by the \side{0.9} model and the \textit{Random} baseline. We use Holm's correction \cite{holm1979simple} to adjust the \textit{p}-values for the multiple tests. We reject the \textit{null hypothesis} (there is no difference between the two models) if the \emph{p}-value is lower than 0.05.


\vspace{-6pt}
\section{Results} \label{results}

\begin{table*}[t]
\centering
\fontsize{11pt}{11pt}\selectfont
\begin{tabular}{lllllllllllll}
\toprule
\multicolumn{1}{c}{\textbf{task}} & \multicolumn{2}{c}{\textbf{Mir}} & \multicolumn{2}{c}{\textbf{Lai}} & \multicolumn{2}{c}{\textbf{Ziegen.}} & \multicolumn{2}{c}{\textbf{Cao}} & \multicolumn{2}{c}{\textbf{Alva-Man.}} & \multicolumn{1}{c}{\textbf{avg.}} & \textbf{\begin{tabular}[c]{@{}l@{}}avg.\\ rank\end{tabular}} \\
\multicolumn{1}{c}{\textbf{metrics}} & \multicolumn{1}{c}{\textbf{cor.}} & \multicolumn{1}{c}{\textbf{p-v.}} & \multicolumn{1}{c}{\textbf{cor.}} & \multicolumn{1}{c}{\textbf{p-v.}} & \multicolumn{1}{c}{\textbf{cor.}} & \multicolumn{1}{c}{\textbf{p-v.}} & \multicolumn{1}{c}{\textbf{cor.}} & \multicolumn{1}{c}{\textbf{p-v.}} & \multicolumn{1}{c}{\textbf{cor.}} & \multicolumn{1}{c}{\textbf{p-v.}} &  &  \\ \midrule
\textbf{S-Bleu} & 0.50 & 0.0 & 0.47 & 0.0 & 0.59 & 0.0 & 0.58 & 0.0 & 0.68 & 0.0 & 0.57 & 5.8 \\
\textbf{R-Bleu} & -- & -- & 0.27 & 0.0 & 0.30 & 0.0 & -- & -- & -- & -- & - &  \\
\textbf{S-Meteor} & 0.49 & 0.0 & 0.48 & 0.0 & 0.61 & 0.0 & 0.57 & 0.0 & 0.64 & 0.0 & 0.56 & 6.1 \\
\textbf{R-Meteor} & -- & -- & 0.34 & 0.0 & 0.26 & 0.0 & -- & -- & -- & -- & - &  \\
\textbf{S-Bertscore} & \textbf{0.53} & 0.0 & {\ul 0.80} & 0.0 & \textbf{0.70} & 0.0 & {\ul 0.66} & 0.0 & {\ul0.78} & 0.0 & \textbf{0.69} & \textbf{1.7} \\
\textbf{R-Bertscore} & -- & -- & 0.51 & 0.0 & 0.38 & 0.0 & -- & -- & -- & -- & - &  \\
\textbf{S-Bleurt} & {\ul 0.52} & 0.0 & {\ul 0.80} & 0.0 & 0.60 & 0.0 & \textbf{0.70} & 0.0 & \textbf{0.80} & 0.0 & {\ul 0.68} & {\ul 2.3} \\
\textbf{R-Bleurt} & -- & -- & 0.59 & 0.0 & -0.05 & 0.13 & -- & -- & -- & -- & - &  \\
\textbf{S-Cosine} & 0.51 & 0.0 & 0.69 & 0.0 & {\ul 0.62} & 0.0 & 0.61 & 0.0 & 0.65 & 0.0 & 0.62 & 4.4 \\
\textbf{R-Cosine} & -- & -- & 0.40 & 0.0 & 0.29 & 0.0 & -- & -- & -- & -- & - & \\ \midrule
\textbf{QuestEval} & 0.23 & 0.0 & 0.25 & 0.0 & 0.49 & 0.0 & 0.47 & 0.0 & 0.62 & 0.0 & 0.41 & 9.0 \\
\textbf{LLaMa3} & 0.36 & 0.0 & \textbf{0.84} & 0.0 & {\ul{0.62}} & 0.0 & 0.61 & 0.0 &  0.76 & 0.0 & 0.64 & 3.6 \\
\textbf{our (3b)} & 0.49 & 0.0 & 0.73 & 0.0 & 0.54 & 0.0 & 0.53 & 0.0 & 0.7 & 0.0 & 0.60 & 5.8 \\
\textbf{our (8b)} & 0.48 & 0.0 & 0.73 & 0.0 & 0.52 & 0.0 & 0.53 & 0.0 & 0.7 & 0.0 & 0.59 & 6.3 \\  \bottomrule
\end{tabular}
\caption{Pearson correlation on human evaluation on system output. `R-': reference-based. `S-': source-based.}
\label{tab:sys}
\end{table*}



\begin{table}%[]
\centering
\fontsize{11pt}{11pt}\selectfont
\begin{tabular}{llllll}
\toprule
\multicolumn{1}{c}{\textbf{task}} & \multicolumn{1}{c}{\textbf{Lai}} & \multicolumn{1}{c}{\textbf{Zei.}} & \multicolumn{1}{c}{\textbf{Scia.}} & \textbf{} & \textbf{} \\ 
\multicolumn{1}{c}{\textbf{metrics}} & \multicolumn{1}{c}{\textbf{cor.}} & \multicolumn{1}{c}{\textbf{cor.}} & \multicolumn{1}{c}{\textbf{cor.}} & \textbf{avg.} & \textbf{\begin{tabular}[c]{@{}l@{}}avg.\\ rank\end{tabular}} \\ \midrule
\textbf{S-Bleu} & 0.40 & 0.40 & 0.19* & 0.33 & 7.67 \\
\textbf{S-Meteor} & 0.41 & 0.42 & 0.16* & 0.33 & 7.33 \\
\textbf{S-BertS.} & {\ul0.58} & 0.47 & 0.31 & 0.45 & 3.67 \\
\textbf{S-Bleurt} & 0.45 & {\ul 0.54} & {\ul 0.37} & 0.45 & {\ul 3.33} \\
\textbf{S-Cosine} & 0.56 & 0.52 & 0.3 & {\ul 0.46} & {\ul 3.33} \\ \midrule
\textbf{QuestE.} & 0.27 & 0.35 & 0.06* & 0.23 & 9.00 \\
\textbf{LlaMA3} & \textbf{0.6} & \textbf{0.67} & \textbf{0.51} & \textbf{0.59} & \textbf{1.0} \\
\textbf{Our (3b)} & 0.51 & 0.49 & 0.23* & 0.39 & 4.83 \\
\textbf{Our (8b)} & 0.52 & 0.49 & 0.22* & 0.43 & 4.83 \\ \bottomrule
\end{tabular}
\caption{Pearson correlation on human ratings on reference output. *not significant; we cannot reject the null hypothesis of zero correlation}
\label{tab:ref}
\end{table}


\begin{table*}%[]
\centering
\fontsize{11pt}{11pt}\selectfont
\begin{tabular}{lllllllll}
\toprule
\textbf{task} & \multicolumn{1}{c}{\textbf{ALL}} & \multicolumn{1}{c}{\textbf{sentiment}} & \multicolumn{1}{c}{\textbf{detoxify}} & \multicolumn{1}{c}{\textbf{catchy}} & \multicolumn{1}{c}{\textbf{polite}} & \multicolumn{1}{c}{\textbf{persuasive}} & \multicolumn{1}{c}{\textbf{formal}} & \textbf{\begin{tabular}[c]{@{}l@{}}avg. \\ rank\end{tabular}} \\
\textbf{metrics} & \multicolumn{1}{c}{\textbf{cor.}} & \multicolumn{1}{c}{\textbf{cor.}} & \multicolumn{1}{c}{\textbf{cor.}} & \multicolumn{1}{c}{\textbf{cor.}} & \multicolumn{1}{c}{\textbf{cor.}} & \multicolumn{1}{c}{\textbf{cor.}} & \multicolumn{1}{c}{\textbf{cor.}} &  \\ \midrule
\textbf{S-Bleu} & -0.17 & -0.82 & -0.45 & -0.12* & -0.1* & -0.05 & -0.21 & 8.42 \\
\textbf{R-Bleu} & - & -0.5 & -0.45 &  &  &  &  &  \\
\textbf{S-Meteor} & -0.07* & -0.55 & -0.4 & -0.01* & 0.1* & -0.16 & -0.04* & 7.67 \\
\textbf{R-Meteor} & - & -0.17* & -0.39 & - & - & - & - & - \\
\textbf{S-BertScore} & 0.11 & -0.38 & -0.07* & -0.17* & 0.28 & 0.12 & 0.25 & 6.0 \\
\textbf{R-BertScore} & - & -0.02* & -0.21* & - & - & - & - & - \\
\textbf{S-Bleurt} & 0.29 & 0.05* & 0.45 & 0.06* & 0.29 & 0.23 & 0.46 & 4.2 \\
\textbf{R-Bleurt} & - &  0.21 & 0.38 & - & - & - & - & - \\
\textbf{S-Cosine} & 0.01* & -0.5 & -0.13* & -0.19* & 0.05* & -0.05* & 0.15* & 7.42 \\
\textbf{R-Cosine} & - & -0.11* & -0.16* & - & - & - & - & - \\ \midrule
\textbf{QuestEval} & 0.21 & {\ul{0.29}} & 0.23 & 0.37 & 0.19* & 0.35 & 0.14* & 4.67 \\
\textbf{LlaMA3} & \textbf{0.82} & \textbf{0.80} & \textbf{0.72} & \textbf{0.84} & \textbf{0.84} & \textbf{0.90} & \textbf{0.88} & \textbf{1.00} \\
\textbf{Our (3b)} & 0.47 & -0.11* & 0.37 & 0.61 & 0.53 & 0.54 & 0.66 & 3.5 \\
\textbf{Our (8b)} & {\ul{0.57}} & 0.09* & {\ul 0.49} & {\ul 0.72} & {\ul 0.64} & {\ul 0.62} & {\ul 0.67} & {\ul 2.17} \\ \bottomrule
\end{tabular}
\caption{Pearson correlation on human ratings on our constructed test set. 'R-': reference-based. 'S-': source-based. *not significant; we cannot reject the null hypothesis of zero correlation}
\label{tab:con}
\end{table*}

\section{Results}
We benchmark the different metrics on the different datasets using correlation to human judgement. For content preservation, we show results split on data with system output, reference output and our constructed test set: we show that the data source for evaluation leads to different conclusions on the metrics. In addition, we examine whether the metrics can rank style transfer systems similar to humans. On style strength, we likewise show correlations between human judgment and zero-shot evaluation approaches. When applicable, we summarize results by reporting the average correlation. And the average ranking of the metric per dataset (by ranking which metric obtains the highest correlation to human judgement per dataset). 

\subsection{Content preservation}
\paragraph{How do data sources affect the conclusion on best metric?}
The conclusions about the metrics' performance change radically depending on whether we use system output data, reference output, or our constructed test set. Ideally, a good metric correlates highly with humans on any data source. Ideally, for meta-evaluation, a metric should correlate consistently across all data sources, but the following shows that the correlations indicate different things, and the conclusion on the best metric should be drawn carefully.

Looking at the metrics correlations with humans on the data source with system output (Table~\ref{tab:sys}), we see a relatively high correlation for many of the metrics on many tasks. The overall best metrics are S-BertScore and S-BLEURT (avg+avg rank). We see no notable difference in our method of using the 3B or 8B model as the backbone.

Examining the average correlations based on data with reference output (Table~\ref{tab:ref}), now the zero-shoot prompting with LlaMA3 70B is the best-performing approach ($0.59$ avg). Tied for second place are source-based cosine embedding ($0.46$ avg), BLEURT ($0.45$ avg) and BertScore ($0.45$ avg). Our method follows on a 5. place: here, the 8b version (($0.43$ avg)) shows a bit stronger results than 3b ($0.39$ avg). The fact that the conclusions change, whether looking at reference or system output, confirms the observations made by \citet{scialom-etal-2021-questeval} on simplicity transfer.   

Now consider the results on our test set (Table~\ref{tab:con}): Several metrics show low or no correlation; we even see a significantly negative correlation for some metrics on ALL (BLEU) and for specific subparts of our test set for BLEU, Meteor, BertScore, Cosine. On the other end, LlaMA3 70B is again performing best, showing strong results ($0.82$ in ALL). The runner-up is now our 8B method, with a gap to the 3B version ($0.57$ vs $0.47$ in ALL). Note our method still shows zero correlation for the sentiment task. After, ranks BLEURT ($0.29$), QuestEval ($0.21$), BertScore ($0.11$), Cosine ($0.01$).  

On our test set, we find that some metrics that correlate relatively well on the other datasets, now exhibit low correlation. Hence, with our test set, we can now support the logical reasoning with data evidence: Evaluation of content preservation for style transfer needs to take the style shift into account. This conclusion could not be drawn using the existing data sources: We hypothesise that for the data with system-based output, successful output happens to be very similar to the source sentence and vice versa, and reference-based output might not contain server mistakes as they are gold references. Thus, none of the existing data sources tests the limits of the metrics.  


\paragraph{How do reference-based metrics compare to source-based ones?} Reference-based metrics show a lower correlation than the source-based counterpart for all metrics on both datasets with ratings on references (Table~\ref{tab:sys}). As discussed previously, reference-based metrics for style transfer have the drawback that many different good solutions on a rewrite might exist and not only one similar to a reference.


\paragraph{How well can the metrics rank the performance of style transfer methods?}
We compare the metrics' ability to judge the best style transfer methods w.r.t. the human annotations: Several of the data sources contain samples from different style transfer systems. In order to use metrics to assess the quality of the style transfer system, metrics should correctly find the best-performing system. Hence, we evaluate whether the metrics for content preservation provide the same system ranking as human evaluators. We take the mean of the score for every output on each system and the mean of the human annotations; we compare the systems using the Kendall's Tau correlation. 

We find only the evaluation using the dataset Mir, Lai, and Ziegen to result in significant correlations, probably because of sparsity in a number of system tests (App.~\ref{app:dataset}). Our method (8b) is the only metric providing a perfect ranking of the style transfer system on the Lai data, and Llama3 70B the only one on the Ziegen data. Results in App.~\ref{app:results}. 


\subsection{Style strength results}
%Evaluating style strengths is a challenging task. 
Llama3 70B shows better overall results than our method. However, our method scores higher than Llama3 70B on 2 out of 6 datasets, but it also exhibits zero correlation on one task (Table~\ref{tab:styleresults}).%More work i s needed on evaluating style strengths. 
 
\begin{table}%[]
\fontsize{11pt}{11pt}\selectfont
\begin{tabular}{lccc}
\toprule
\multicolumn{1}{c}{\textbf{}} & \textbf{LlaMA3} & \textbf{Our (3b)} & \textbf{Our (8b)} \\ \midrule
\textbf{Mir} & 0.46 & 0.54 & \textbf{0.57} \\
\textbf{Lai} & \textbf{0.57} & 0.18 & 0.19 \\
\textbf{Ziegen.} & 0.25 & 0.27 & \textbf{0.32} \\
\textbf{Alva-M.} & \textbf{0.59} & 0.03* & 0.02* \\
\textbf{Scialom} & \textbf{0.62} & 0.45 & 0.44 \\
\textbf{\begin{tabular}[c]{@{}l@{}}Our Test\end{tabular}} & \textbf{0.63} & 0.46 & 0.48 \\ \bottomrule
\end{tabular}
\caption{Style strength: Pearson correlation to human ratings. *not significant; we cannot reject the null hypothesis of zero corelation}
\label{tab:styleresults}
\end{table}

\subsection{Ablation}
We conduct several runs of the methods using LLMs with variations in instructions/prompts (App.~\ref{app:method}). We observe that the lower the correlation on a task, the higher the variation between the different runs. For our method, we only observe low variance between the runs.
None of the variations leads to different conclusions of the meta-evaluation. Results in App.~\ref{app:results}.
% \vspace{-5pt}
\section{Discussion} \label{discussion}
\section{Discussion of Assumptions}\label{sec:discussion}
In this paper, we have made several assumptions for the sake of clarity and simplicity. In this section, we discuss the rationale behind these assumptions, the extent to which these assumptions hold in practice, and the consequences for our protocol when these assumptions hold.

\subsection{Assumptions on the Demand}

There are two simplifying assumptions we make about the demand. First, we assume the demand at any time is relatively small compared to the channel capacities. Second, we take the demand to be constant over time. We elaborate upon both these points below.

\paragraph{Small demands} The assumption that demands are small relative to channel capacities is made precise in \eqref{eq:large_capacity_assumption}. This assumption simplifies two major aspects of our protocol. First, it largely removes congestion from consideration. In \eqref{eq:primal_problem}, there is no constraint ensuring that total flow in both directions stays below capacity--this is always met. Consequently, there is no Lagrange multiplier for congestion and no congestion pricing; only imbalance penalties apply. In contrast, protocols in \cite{sivaraman2020high, varma2021throughput, wang2024fence} include congestion fees due to explicit congestion constraints. Second, the bound \eqref{eq:large_capacity_assumption} ensures that as long as channels remain balanced, the network can always meet demand, no matter how the demand is routed. Since channels can rebalance when necessary, they never drop transactions. This allows prices and flows to adjust as per the equations in \eqref{eq:algorithm}, which makes it easier to prove the protocol's convergence guarantees. This also preserves the key property that a channel's price remains proportional to net money flow through it.

In practice, payment channel networks are used most often for micro-payments, for which on-chain transactions are prohibitively expensive; large transactions typically take place directly on the blockchain. For example, according to \cite{river2023lightning}, the average channel capacity is roughly $0.1$ BTC ($5,000$ BTC distributed over $50,000$ channels), while the average transaction amount is less than $0.0004$ BTC ($44.7k$ satoshis). Thus, the small demand assumption is not too unrealistic. Additionally, the occasional large transaction can be treated as a sequence of smaller transactions by breaking it into packets and executing each packet serially (as done by \cite{sivaraman2020high}).
Lastly, a good path discovery process that favors large capacity channels over small capacity ones can help ensure that the bound in \eqref{eq:large_capacity_assumption} holds.

\paragraph{Constant demands} 
In this work, we assume that any transacting pair of nodes have a steady transaction demand between them (see Section \ref{sec:transaction_requests}). Making this assumption is necessary to obtain the kind of guarantees that we have presented in this paper. Unless the demand is steady, it is unreasonable to expect that the flows converge to a steady value. Weaker assumptions on the demand lead to weaker guarantees. For example, with the more general setting of stochastic, but i.i.d. demand between any two nodes, \cite{varma2021throughput} shows that the channel queue lengths are bounded in expectation. If the demand can be arbitrary, then it is very hard to get any meaningful performance guarantees; \cite{wang2024fence} shows that even for a single bidirectional channel, the competitive ratio is infinite. Indeed, because a PCN is a decentralized system and decisions must be made based on local information alone, it is difficult for the network to find the optimal detailed balance flow at every time step with a time-varying demand.  With a steady demand, the network can discover the optimal flows in a reasonably short time, as our work shows.

We view the constant demand assumption as an approximation for a more general demand process that could be piece-wise constant, stochastic, or both (see simulations in Figure \ref{fig:five_nodes_variable_demand}).
We believe it should be possible to merge ideas from our work and \cite{varma2021throughput} to provide guarantees in a setting with random demands with arbitrary means. We leave this for future work. In addition, our work suggests that a reasonable method of handling stochastic demands is to queue the transaction requests \textit{at the source node} itself. This queuing action should be viewed in conjunction with flow-control. Indeed, a temporarily high unidirectional demand would raise prices for the sender, incentivizing the sender to stop sending the transactions. If the sender queues the transactions, they can send them later when prices drop. This form of queuing does not require any overhaul of the basic PCN infrastructure and is therefore simpler to implement than per-channel queues as suggested by \cite{sivaraman2020high} and \cite{varma2021throughput}.

\subsection{The Incentive of Channels}
The actions of the channels as prescribed by the DEBT control protocol can be summarized as follows. Channels adjust their prices in proportion to the net flow through them. They rebalance themselves whenever necessary and execute any transaction request that has been made of them. We discuss both these aspects below.

\paragraph{On Prices}
In this work, the exclusive role of channel prices is to ensure that the flows through each channel remains balanced. In practice, it would be important to include other components in a channel's price/fee as well: a congestion price  and an incentive price. The congestion price, as suggested by \cite{varma2021throughput}, would depend on the total flow of transactions through the channel, and would incentivize nodes to balance the load over different paths. The incentive price, which is commonly used in practice \cite{river2023lightning}, is necessary to provide channels with an incentive to serve as an intermediary for different channels. In practice, we expect both these components to be smaller than the imbalance price. Consequently, we expect the behavior of our protocol to be similar to our theoretical results even with these additional prices.

A key aspect of our protocol is that channel fees are allowed to be negative. Although the original Lightning network whitepaper \cite{poon2016bitcoin} suggests that negative channel prices may be a good solution to promote rebalancing, the idea of negative prices in not very popular in the literature. To our knowledge, the only prior work with this feature is \cite{varma2021throughput}. Indeed, in papers such as \cite{van2021merchant} and \cite{wang2024fence}, the price function is explicitly modified such that the channel price is never negative. The results of our paper show the benefits of negative prices. For one, in steady state, equal flows in both directions ensure that a channel doesn't loose any money (the other price components mentioned above ensure that the channel will only gain money). More importantly, negative prices are important to ensure that the protocol selectively stifles acyclic flows while allowing circulations to flow. Indeed, in the example of Section \ref{sec:flow_control_example}, the flows between nodes $A$ and $C$ are left on only because the large positive price over one channel is canceled by the corresponding negative price over the other channel, leading to a net zero price.

Lastly, observe that in the DEBT control protocol, the price charged by a channel does not depend on its capacity. This is a natural consequence of the price being the Lagrange multiplier for the net-zero flow constraint, which also does not depend on the channel capacity. In contrast, in many other works, the imbalance price is normalized by the channel capacity \cite{ren2018optimal, lin2020funds, wang2024fence}; this is shown to work well in practice. The rationale for such a price structure is explained well in \cite{wang2024fence}, where this fee is derived with the aim of always maintaining some balance (liquidity) at each end of every channel. This is a reasonable aim if a channel is to never rebalance itself; the experiments of the aforementioned papers are conducted in such a regime. In this work, however, we allow the channels to rebalance themselves a few times in order to settle on a detailed balance flow. This is because our focus is on the long-term steady state performance of the protocol. This difference in perspective also shows up in how the price depends on the channel imbalance. \cite{lin2020funds} and \cite{wang2024fence} advocate for strictly convex prices whereas this work and \cite{varma2021throughput} propose linear prices.

\paragraph{On Rebalancing} 
Recall that the DEBT control protocol ensures that the flows in the network converge to a detailed balance flow, which can be sustained perpetually without any rebalancing. However, during the transient phase (before convergence), channels may have to perform on-chain rebalancing a few times. Since rebalancing is an expensive operation, it is worthwhile discussing methods by which channels can reduce the extent of rebalancing. One option for the channels to reduce the extent of rebalancing is to increase their capacity; however, this comes at the cost of locking in more capital. Each channel can decide for itself the optimum amount of capital to lock in. Another option, which we discuss in Section \ref{sec:five_node}, is for channels to increase the rate $\gamma$ at which they adjust prices. 

Ultimately, whether or not it is beneficial for a channel to rebalance depends on the time-horizon under consideration. Our protocol is based on the assumption that the demand remains steady for a long period of time. If this is indeed the case, it would be worthwhile for a channel to rebalance itself as it can make up this cost through the incentive fees gained from the flow of transactions through it in steady state. If a channel chooses not to rebalance itself, however, there is a risk of being trapped in a deadlock, which is suboptimal for not only the nodes but also the channel.

\section{Conclusion}
This work presents DEBT control: a protocol for payment channel networks that uses source routing and flow control based on channel prices. The protocol is derived by posing a network utility maximization problem and analyzing its dual minimization. It is shown that under steady demands, the protocol guides the network to an optimal, sustainable point. Simulations show its robustness to demand variations. The work demonstrates that simple protocols with strong theoretical guarantees are possible for PCNs and we hope it inspires further theoretical research in this direction.

\section{Conclusion and Future Work} \label{conclusion}
\section{Conclusion}
In this work, we propose a simple yet effective approach, called SMILE, for graph few-shot learning with fewer tasks. Specifically, we introduce a novel dual-level mixup strategy, including within-task and across-task mixup, for enriching the diversity of nodes within each task and the diversity of tasks. Also, we incorporate the degree-based prior information to learn expressive node embeddings. Theoretically, we prove that SMILE effectively enhances the model's generalization performance. Empirically, we conduct extensive experiments on multiple benchmarks and the results suggest that SMILE significantly outperforms other baselines, including both in-domain and cross-domain few-shot settings.

% \newpage

\bibliographystyle{ieeetr}
\bibliography{main}

\end{document}
