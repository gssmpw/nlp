\documentclass[conference]{IEEEtran}
\IEEEoverridecommandlockouts
% The preceding line is only needed to identify funding in the first footnote. If that is unneeded, please comment it out.
%Template version as of 6/27/2024

\usepackage{cite}
\usepackage{amsmath,amssymb,amsfonts}
\usepackage{algorithmic}
\usepackage{graphicx}
\usepackage{textcomp}
\usepackage{xcolor}
\usepackage{orcidlink}
\usepackage{tcolorbox}
\usepackage{todonotes}
\usepackage{booktabs}
\usepackage{hyperref}
\usepackage{xurl}
\usepackage{array}
\usepackage{tcolorbox}
\usepackage{fontawesome5}
\usepackage{makecell}
\usepackage{array, multirow, boldline}
\usepackage{caption}

\def\BibTeX{{\rm B\kern-.05em{\sc i\kern-.025em b}\kern-.08em
    T\kern-.1667em\lower.7ex\hbox{E}\kern-.125emX}}

\usepackage[T1]{fontenc}
\begin{document}

\title{LLMs for Generation of Architectural Components: An Exploratory Empirical Study in the Serverless World}

\author{\IEEEauthorblockN{Shrikara Arun* 
\orcidlink{0009-0005-8543-6130}
} 
\IEEEauthorblockA{
\textit{Software Engineering Research Centre} \\
% \textit{SERC, IIIT Hyderabad, India}\\
\textit{IIIT Hyderabad, India}\\
\textit{shrikara.a@students.iiit.ac.in}}
\thanks{*These authors contributed equally.}
\and
\IEEEauthorblockN{Meghana Tedla* 
\orcidlink{0009-0000-8540-8790}
}
\IEEEauthorblockA{
\textit{Software Engineering Research Centre} \\
% \textit{SERC, IIIT Hyderabad, India}\\
\textit{IIIT Hyderabad, India}\\
\textit{meghana.tedla@students.iiit.ac.in}}
\and
\IEEEauthorblockN{Karthik Vaidhyanathan 
\orcidlink{0000-0003-2317-6175}
}
\IEEEauthorblockA{
\textit{Software Engineering Research Centre} \\
% \textit{SERC, IIIT Hyderabad, India}\\
\textit{IIIT Hyderabad, India}\\
\textit{karthik.vaidhyanathan@iiit.ac.in}}
 }
% \vspace*{-1cm}
\maketitle
% \begingroup\renewcommand\thefootnote{}
% \footnotetext{*Equal contribution}

\newcommand{\magenta}[1]{\textcolor{magenta}{#1}}
\newcommand{\blue}[1]{\textcolor{blue}{#1}}
\newcommand{\Human}{\stackon{\faUser}{\textcolor{green}{\faCheck}}}
\newcommand{\NoHuman}{\stackon{\faUser}{\textcolor{red}{\faTimes}}}

\begin{abstract}
Recently, the exponential growth in capability and pervasiveness of Large Language Models (LLMs) has led to significant work done in the field of code generation. However, this generation has been limited to code snippets. Going one step further, our desideratum is to automatically generate architectural components. This would not only speed up development time, but would also enable us to eventually completely skip the development phase, moving directly from design decisions to deployment. To this end, we conduct an exploratory study on the capability of LLMs to generate architectural components for Functions as a Service (FaaS), commonly known as serverless functions. The small size of their architectural components make this architectural style amenable for generation using current LLMs compared to other styles like monoliths and microservices. We perform the study by systematically selecting open source serverless repositories, masking a serverless function and utilizing state of the art LLMs provided with varying levels of context information about the overall system to generate the masked function. We evaluate correctness through existing tests present in the repositories and use metrics from the Software Engineering (SE) and Natural Language Processing (NLP) domains to evaluate code quality and the degree of similarity between human and LLM generated code respectively. Along with our findings, we also present a discussion on the path forward for using GenAI in architectural component generation.
\end{abstract}

\begin{IEEEkeywords}
Architectural Component Generation, LLM, Serverless.
\end{IEEEkeywords}

\vspace{-12pt}
\section{Introduction}
\documentclass[../main.tex]{subfiles}
\graphicspath{{../images/}}
\makeatletter
\def\input@path{{../images/}}
\makeatother
\begin{document}
\section{Introduction}
\begin{figure}
\centering
\begin{tikzpicture}
\node[inner sep=0pt] (ws) at (0, 0) {
\includegraphics[height=.4\textwidth, trim={10cm 0 10cm 0},clip]{world_space.png}};
\node[inner sep=0pt] (cs) at (6,0) {\includegraphics[height=.4\textwidth, trim={10cm 1cm 10cm 4cm},clip]{conf_space.png}};
\end{tikzpicture}
\vspace{-5pt}
\label{fig:pbrm_intro}
\caption{\textbf{Left}: Shows world space obstacles as grey spheres. Robots start and goal configuration is colored red and green, respectively. Configurations along the computed path are colored transparent blue. \textbf{Right:} Mapped world space scenario to configuration space. Obstacle region is the grey mesh. Red spheres are collision-free regions computed by the neural SCDF. The optimized shortest path in the convex corridor is the blue curve.}
\vspace{-25pt}
\end{figure}
Motion planning is the problem of finding a collision-free trajectory that connects a given start and goal configuration. The planning takes place in the configuration space of the robot. For single body robots, like mobile robots or drones, the configuration space and the world space are usually the same. This simplifies the planning, since explicit obstacle representations are available which enables geometrical tools like separating hyperplanes, smallest distance to obstacles etc., to be used when designing motion planning algorithms. For multi-body robots like manipulators, the situation is completely different. The world space obstacles are usually mapped to non-convex regions, and to make the problem even harder, the mapping is usually not known. Forming explicit representations of the obstacle region in the configuration space is usually too expensive or intractable. Despite all of this, sampling based planners are used with great success, which mainly is due to their use of implicit representations of the obstacle region. The basic idea is to construct a graph in the configuration space that covers and connects the collision-free region. From this graph, a path can be extracted that connects a given start and goal configuration. The approach is computationally expensive, since the graph is constructed with the smallest geometrical building block available, points, which represents a collision-check. Furthermore, the extracted paths from the graph are non-smooth and jagged due to the stochastic nature of the approach. This adds an additional post-processing step to the process, where the paths are shortcutted and smoothened, before the path can be used for tracking. Clearly a lot of time is invested to form this graph and produce smooth paths. Thus, if the obstacles start to move, then all of this work is done in no use, since all points that make up this graph need to be re-verified, which is simply too time consuming to be done in real time.
\\\\
In this work, we want to address the existing drawbacks of the sampling based planners. Our main contribution is an improved motion planner where each vertex in the graph covers a collision-free region in the form of a sphere instead of a point and where the edges are formed with neighboring intersecting spheres. This representation has the advantage of instead of returning piecewise linear paths, returning a sequence of overlapping spheres, i.e. a convex corridor, that connects a given start and goal configuration, illustrated in Figure \ref{fig:pbrm_intro}. This convex corridor allows us to use convex optimization to produce smooth trajectories, instead of computationally expensive post-processing methods. The representation further allows us to estimate the coverage of the collision-free space, which gives us awareness and feedback in the offline roadmap construction phase. Finally, our representation is simple to adapt to moving obstacles, simply requery for the new radii and recheck for intersections. 
\\\\
The spherical collision-free regions are formed using a signed distance function (SDF), which is a function that returns the smallest distance from an arbitrary point to the boundary of an obstacle. As the name implies, the distance is signed, thus if the point is inside the obstacle it is negative otherwise positive. If the distance is positive, a sphere with radius equal to the distance is guaranteed to cover a collision-free region. Using an SDF in motion planning is not new, but what is novel about our approach is that we express the distance in the configuration space instead of the world space and by doing so allows us to form these convex collision-free regions. We refer to the resulting SDF as a signed configuration distance function (SCDF). Computing an SCDF analytically is non-trivial, our approach is therefore to parameterize the SCDF with a deep neural network and learn the mapping by supervised learning. Our resulting neural SCDF can compute distances for different parameter values of obstacle shapes and we also show how multiple distances can be combined, thus making our approach flexible.
\section{Related work}
Motion planning algorithms can roughly be divided into three families, grid-based, sampling based and optimization based methods. Grid-based methods (GBM) discretize the planning space from which a graph is then compiled. A standard search method is A$^\star$ \citep{a_star}, which is classified as an \textit{informed} search method, since it employs a heuristic function to speed up the search. A$^\star$ guarantees to return an optimal path at the level of discretization used. GBMs usually discretize the planning space by a regular lattice and this limits the GBMs to problems with low dimensionality due to the curse of dimensionality. Thus, GBMs are usually limited to single-body robots where the degrees of freedom (DOF) are low. To overcome the inherent scaling problem with the GBMs, stochastic methods are usually used for multi-body robots. These methods are termed as sampling-based methods (SBM) and core members within this family are the rapidly-exploring random trees (RRT) \citep{rrt} and the probabilistic roadmap (PRM) \citep{prm}. RRT grows a tree from the start configuration and explores the collision-free region in a rapid way until it is able to connect to the goal region. RRT is usually improved by bi-directional planning \citep{rrt_connect}, i.e. an additional tree is grown from the goal configuration and the trees are tested for connection after any tree has been expanded. RRT is a single-query method, thus it searches for a path from scratch each time it is queried. Contrary to this, PRM is a multi-query method, which solves for multiple queries without starting from scratch. PRM does this by creating a roadmap (graph) that covers the collision-free space as an offline step. The graph is then used to solve for multiple queries. PRMs are used in cases where the environment does not change since the extra offline step is too computationally costly and needs to be re-done if the environment is changed. In our work, we address this inherent issue by using a different roadmap representation. Our vertices in the graph cover a collision-free region in the form of spheres and we form the edges by checking for intersecting spheres. If something in the environment changes, we recompute the spheres radii and recheck the intersections, without relying on collision detection. We use a trained neural network to compute the sphere radius, therefore querying for the radius can be done fast, hence our representation enables the PRM for dynamic environments.
\\\\
In the recent decades, optimization based methods (OBM) \citep{chomp, schulman, itomp, stomp} have been introduced as an alternative to SBM for multi-body robots. Like the SBM, the OBMs scale well to higher dimensional problems and produce smoother motion. It is common to use a SDF in the optimization since it is a smooth function, thus enabling gradient-based methods. However, the standard way of expressing the SDF is in world space. The distance therefore needs to be mapped to the configuration space by the forward kinematics. This mapping makes the optimization problem a non-linear program (NLP), which is computationally expensive to solve. Recently, a different approach has been proposed. In \cite{mp_gcs} motion planning is formulated as a convex optimization problem by using the graph of convex sets framework \citep{gcs}. The underlying idea is to decompose the collision-free space into intersecting convex sets from which a convex optimization problem is formulated. In cases where an explicit representation of the obstacles in the configuration space exists, like for single-body robots, creating collision-free convex regions can be done fast \citep{iris}. For multi-body robots, this is non-trivial. Existing work does this successfully \citep{iris_nlp, iris_c} by an optimization based approach, but the methods are still too time consuming to be used in the presence of moving obstacles. Our approach is instead to use deep learning to learn an SDF expressed in the configuration space. With this, we can query for shortest distances to the collision boundary, which allows us to expand spherical regions which are collision-free. Our approach is fast and therefore enables our suggested roadmap planner to be used in dynamic environments.
\\\\
Recent research has focused on learning collision detection \citep{fk_kernel_distance, diffco, graphdistnet} by predicting the signed distance between the robot links and the surrounding obstacles in the world space. The learned SDF is used in trajectory optimization but since the distance is expressed in the world space, the problem becomes an NLP and therefore takes a long time to solve. We take a novel approach and suggest to instead express the signed distance in the configuration space. This allows us to improve the PRM at the same time as it enables convex optimization for trajectory optimization, which runs faster and is more reliable than NLP solvers. In \cite{cspf} a learned signed distance function in the configuration space is proposed similar to our approach. However, their approach is restricted to point cloud representations, while we propose to represent the obstacles as parameterized geometric shapes, e.g. spheres. Furthermore, we also show how to use our learned SCDF to improve an existing roadmap planner.
\section{Problem formulation}
A robot is located in the world space, $\W \subset \R^3 $. The unique location of the robot is given by its configuration $\q \in \C$, where $\C$ is the configuration space. The set of points covered by the robots bodies at a certain configuration is expressed as $\B(\q) \subset \W$. The robot is surrounded by $\NrObst$ obstacles $\O = \bigcup_{i=1}^{\NrObst} \O_i$, where  $\O_i \subset \W$. The representation of the obstacle in the configuration space is the set $\C\O_i = \{\q \in \C \: |\: \B(\q) \cap \O_i \neq \emptyset \}$. The obstacle space is formed as $\Co = \bigcup_{i=1}^{\NrObst} \C \O_i$. The complement is referred to as the free space, $\Cf = \C \setminus \Co$. The path planning problem is a tuple, ($\Cf$, $\qStart$, $\qGoal$), where we want to connect a query pair, consisting of a start, $\qStart$, and goal configuration, $\qGoal$, with a geometric path, $\q(s): [0, 1] \mapsto \Cf$, such that $\q(0)=\qStart$ and $\q(1)=\qGoal$, or report correctly when such a path does not exist.
\end{document}


\section{Background} \label{background}
\section{Basic Background: Supervised Learning and the PAC Model}
\label{sec:background}

At this point almost everyone has heard of machine learning (ML). Anyone likely to stumble upon this article will have also heard of its most influential special case, supervised learning, and those theoretically inclined will also be familiar with the PAC model. Nonetheless, I will set the stage by  recapping the basics.

\subsection{Basics of Supervised Learning}%Let's set the stage in any case

\emph{Supervised Learning} is the task of ``coming up'' with a function $f: \X \to \Y$ to ``explain'' or ``fit'' a sequence of input/output examples   $(x_1,y_1), \ldots, (x_n,y_n)$, with $x_i \in \X$ and $y_i \in \Y$.  Here $\X$ is a \emph{data domain} consisting of \emph{datapoints} $x \in \X$, $\Y$ is a \emph{label set} consisting of \emph{labels} $y \in \Y$, and the sequence $(x_1,y_1),\ldots,(x_n,y_n)$ is the \emph{training data} consisting of \emph{labeled examples (a.k.a. samples)}~$(x_i,y_i)$.  I~will refer to the chosen function $f$ as a \emph{predictor}, and to $n$ as the \emph{sample size}. A \emph{learning algorithm} takes as input training data, and outputs (some representation of) a predictor $f \in \Y^\X$.\footnote{Note that this describes the usual \emph{batch}, a.k.a.~\emph{offline}, setting of supervised learning. I do not discuss other paradigms such as online or active learning in this article.} 



Success in supervised learning is defined as \emph{generalization} to  future examples: For a typical \emph{test example}  $(x_{\tst},y_{\tst})$, the predicted label $y'_{\tst}=f(x_{\tst})$ should ``equal'' $y_{\tst}$, perhaps approximately. We usually assume the test example is drawn from the same  ``source'' as the training data  --- commonly, i.i.d.~from the same distribution. The quality of the prediction is quantified by $\ell(y'_{\tst},y_{\tst})$, where $\ell:~\Y~\times~\Y \to \RR_{\geq 0}$ is a \emph{loss function} chosen as part of the problem definition. Common loss functions include the 0-1 loss $\ell_{0-1}(y',y) = [y' \neq y]$ for \emph{classification} problems,\footnote{The notation $[P]$ denotes $1$ when predicate $P$ is true, and denotes $0$ when $P$ is false.} as well as the absolute loss $|y'-y|$ or squared loss $(y'-y)^2$ for \emph{regression problems} featuring $\Y  \sse \RR$.

Nontrivial generalization properties are typically only possible if one assumes something about the data.\footnote{The need for such an assumption is formalized by the  \emph{no free lunch theorems} of supervised learning \cite{wolpert_connection_1992,wolpert_lack_1996,schaffer_conservation_1994}.} The Bayesian approach to  machine learning, common in many applications, assumes some parametric form for the distribution generating the data, and postulates a prior on the parameters. This is not the approach I will take in this article. Instead, I will focus on the frequentist --- and some would say ``worst-case'' or ``adversarial'' ---  approach that is common in the computational learning theory community, embodied by the PAC model. Here we assume that the (training and test) data can be explained, perhaps approximately, by a function in some ``simple enough to learn'' class of functions $\H \sse \Y^\X$, often called the \emph{hypotheses}. Equivalently, we  seek a predictor which explains the unseen data roughly  as well as the best hypothesis $h^* \in \H$, whether or not we assume that $h^*$ itself provides a perfect explanation.



 \paragraph{Common Algorithmic Templates.} Perhaps the best known general-purpose supervised learning algorithm is \emph{empirical risk minimization (ERM)}, which chooses as its predictor a hypothesis $f \in \H$ minimizing $\frac{1}{n} \sum_{i=1}^n \ell(f(x_i),y_i)$ --- a quantity called the \emph{training error}, \emph{empirical error}, or \emph{empirical risk} of $f$. %\footnote{When multiple hypotheses minimize the empirical risk, we assume ERM breaks ties arbitrarily.}
A common template for generalizing ERM involves adding a \emph{regularization term} $\psi(f)$ to the  objective function, typically chosen to measure some notion of ``hypothesis complexity.'' An algorithm instantiating this template is known as a \emph{structural risk minimizer (SRM)}, and chooses as its predictor the hypothesis $f \in \H$ minimizing the \emph{structural risk} $\frac{1}{n} \sum_{i=1}^n \ell(f(x_i),y_i) + \psi(f)$. Other well-known algorithms, such as gradient descent and its variations,  can frequently be interpreted as approximate implementations of ERM or SRM.


\paragraph{Proper vs Improper Learning.} A learning algorithm is said to be \emph{proper} if its predictor $f$ is always chosen from the hypothesis class, i.e., $f \in \H$, otherwise it is said to be \emph{improper}. ERM  is an example of a proper learning algorithm, as are SRM algorithms of the form described above.  In the \emph{proper regime} of learning, algorithms are required to be proper. This article will be concerned with the more flexible \emph{improper regime} (a.k.a \emph{representation-independent learning}), where no such constraint is placed on the learner. In other words, all we care about is predictive power at test time, rather than any insights derived from the functional form or representation of the predictor~itself.


\subsection{The PAC Model}
A standard mathematical setup for evaluation of supervised learning algorithms, at least in the theoretical computer science community, is Valiant's \emph{Probably Approximately Correct (PAC) model} of learning (see e.g.~\cite{kearns_introduction_1994,mohri_foundations_2018}). Here, we assume there is an unknown distribution $\D$ on $\X \times \Y$ from which training and test data are  drawn.  Specifically, the labeled datapoints of the training set  $(x_1,y_1), \ldots, (x_n,y_n)$, as well as the test data  $(x_\tst,y_\tst)$, are i.i.d.~from $\D$. Often it is assumed that $\D$ lies in some class of distributions of interest. The \emph{true expected loss}, or simply \emph{loss}, of a predictor $f: \X \to \Y$ is the expected loss it incurs on draws from $\D$, written $L_\D(f) = \Ex_{(x,y) \sim \D} \ell(f(x),y)$.


There are two main ``settings'' in PAC learning. The  \emph{realizable setting} only requires that the data be perfectly explained by some hypothesis in $\H$. More generally, the \emph{agnostic setting} makes no assumption relating the data to the hypotheses, but shifts the goalposts as necessary to allow nontrivial guarantees: the expected loss at test time is evaluated only ``relative'' to that of the best hypothesis $h^* \in \H$. There are other settings which make more nuanced assumptions, such as $\D$ being of a particular parametric form or its support living in some (unknown) lower-dimensional space, etc. I will mostly discuss the realizable and agnostic settings in this article, those being the simplest and most studied from a theoretical perspective. %TODO:We will briefly discuss other settings in Section ??

The PAC model demands high probability guarantees of learners, in the worst case over distributions of interest. Consider first the realizable setting, where $\D$ is such that $\min_{h \in \H} L_{\D}(h) = 0$. A PAC learner has \emph{error} $\epsilon=\epsilon(n)$ and \emph{confidence} $\delta=\delta(n)$ if, when training data consists of $n$ i.i.d~samples from a realizable distribution $\D$, it produces a predictor $f$  satisfying $L_\D(f) \leq \epsilon$ with probability at least $1-\delta$. In the agnostic setting, where $\D$ can be arbitrary, we require $L_\D(f) - \min_{h \in \H} L_\D(h) \leq \epsilon$ with probability $1-\delta$.

In both the realizable and agnostic settings, we look for PAC learners with small $\epsilon$ and $\delta$ as a function of the sample size $n$. An equivalent perspective looks at the sample complexity $m(\epsilon,\delta)$, which is the minimum sample size which guarantees error  at most $\epsilon$ with probability at least $1-\delta$. We say a problem is \emph{PAC learnable} if its PAC sample complexity is finite whenever $\epsilon,\delta > 0$.

For most PAC learning problems, learnability and sample complexity are characterized in terms of a  ``dimension'' of the hypothesis class. Most prominently this is the \emph{VC dimension} for binary classification, the \emph{fat shattering dimension} for agnostic regression, and the \emph{DS dimension} for multiclass classification (see \cite{anthony_neural_1999,daniely_optimal_2014,brukhim_characterization_2022}). Treatment of these is beyond the scope of this article. The unfamiliar reader need not worry, however,  as dimensions will feature only tangentially in our~discussion.




%\paragraph{Learning settings: Realizable, Agnostic, etc.} In learning theory, evaluating a supervised learning algorithm requires specifying a data model and an objective. We will leave the details of the data model flexible for now, to allow for both the PAC model and the adversarial transductive model. Nonetheless we will describe two variations, which we call ``settings'', which cut across different models. The  \emph{realizable setting}  requires only that the data be perfectly explained by some hypothesis $h \in \H$ --- i.e., there exists a hypothesis which is guaranteed to suffer a loss of $0$ on training and test data. The performance of the learning algorithm is its expected loss at test time for some ``worst case'' realizable instance. More generally, the \emph{agnostic setting} makes no assumption relating the data to the hypotheses, but shifts the goalposts as necessary to allow nontrivial guarantees: the expected loss at test time is evaluated only ``relative'' to that of the best hypothesis $h^* \in \H$, again for some ``worst case'' instance. There are other settings which make more nuanced assumptions about the data, such as it is drawn from a distribution of a particular parametric form, or that it lives in some (unknown) lower-dimensional space, etc. We will mostly discuss the realizable and agnostic settings, those being the simplest and most studied from a theoretical perspective.




%%% Local Variables:
%%% mode: latex
%%% TeX-master: "learning_matching"
%%% End:


\vspace{-7pt}
\section{Related Work} \label{related-work}
\section{Related Work}
% \subsection{Vision Language Model}
% 시각장애인에서 상황을 설명할 DB가 없으니 만들었다. 그리고 이를 VLM에 튜닝했다.
\subsection{Technical approaches for assisting the visually-impaired}


\subsection{Datasets for visual instruction tuning}


\section{Study Design} \label{study-design}
\vspace{-2mm}
We conduct extensive experiments to demonstrate the efficacy of various features of \former~to allow it to be trained with only one hour of data. We also present our findings in a way that highlights \former’s differences compared to common practices in NLP and CV, where Transformer training practices have been extensively studied~\cite{xiong2020layer, xu2021optimizing, loshchilov2024ngpt, chen2021empirical, liu2021efficient, gani2022how, steiner2022how}. 
Therefore, our experiment results also serve as a guideline on how to optimize Transformer training for robotics, particularly in off-road navigation and mobility tasks with complex vehicle-terrain interactions under data-scarce conditions. 

\former's one hour of training data comes from human-teleoperated demonstration of driving an open-source four-wheeled ground vehicle~\cite{datar2024wheeled} on a custom-built off-road testbed composed of hundreds of rocks and boulders. The demonstrator mostly aims to drive the robot to safely and stably traverse the vertically challenging terrain, but still occasionally encounters dangerous situations such as large roll angles and getting stuck between rocks. Fortunately, those situations serve as explorations for \former~to understand a wider range of kinodynamic interactions. 
Direct application of standard Transformer training methodologies in NLP and CV to such a small robotics dataset proves challenging due to the inherent lack of inductive bias in Transformers~\cite{dosovitskiy2021image}, which necessitates substantial amounts of data for effective training. However, our experiments suggest that \former's judicious modifications to established MM and NTP training paradigms can facilitate effective Transformer training even with limited robotics data. 

We conduct our experiments based on three perspectives: Section~\ref{sec:basic_perspective} provides an analysis of basic factors to train Transformers in general; Section~\ref{sec:robotic_perspective} analyzes the best practices to train Transformers when dealing with off-road robot mobility data; Finally, Sec.~\ref{sec:objective_perspective} evaluates the effectiveness of each off-road mobility learning objective and compares \encoder, \decoder, and non-Transformer end-to-end model performances. For fairness, all experiments are conducted with the same hyper-parameters. 

\subsection{Experiment Results of Basic Transformer Factors} \label{sec:basic_perspective}

\begin{figure}[t]
  \centering
  \includegraphics[width=0.7\columnwidth]{figures/pos_encoding}
  \caption{\textbf{Positional Encoding:} Sinusoidal positional encoding achieves better model accuracy than learnable encoding for predicting $\mathbf{X}$, $\mathbf{Y}$, and $\mathbf{Z}$ components of the robot pose.}
  \label{fig:pos_encoding}
  % \vspace{-1.2em}
\end{figure}

\textbf{Positional encoding} is crucial for addressing the permutation equivariance of Transformers, which, by design, lacks inherent sensitivity to input sequence order. This characteristic necessitates the explicit provision of positional information to enable the model to effectively process sequential data. Learnable positional encodings, typically implemented as trainable vectors added to input embeddings, have found favor in CV applications~\cite{he2022masked}. Conversely, non-learnable encodings, such as the sinusoidal functions introduced in the seminal work by~\citet{vaswani2017attention}, have demonstrated efficacy in NLP tasks.
This divergence in methodological preference may stem from inherent differences in the statistical properties of data modalities. CV tasks often involve spatially structured data where absolute positional information may be less critical than relative relationships between local features. In such contexts, learnable encodings may offer greater flexibility in adapting to task-specific positional dependencies. Conversely, NLP tasks frequently rely on precise word order and long-range dependencies, where the fixed nature of non-learnable encodings may provide a beneficial inductive bias~\cite{weng2024navigating}.

To empirically investigate the relative merits of these approaches on robot mobility tasks, we conduct a comparative analysis of learnable positional encodings against sinusoidal encodings as shown in Fig.~\ref{fig:pos_encoding}. Our findings indicate that while both methods achieve comparable asymptotic performance levels, sinusoidal positional encodings exhibit a slight performance advantage.

\begin{figure}[t]
  \centering
  \includegraphics[width=0.7\columnwidth]{figures/last_layernorm}
  \caption{\textbf{Normalizing Output:} Normalizing the \tr~output before passing the embeddings to the task decoder improves model performance.}
  \label{fig:last_layernorm}
  % \vspace{-1.2em}
\end{figure}
\textbf{Normalization layers}, such as LayerNorm~\cite{ba2016layer} or RMSNorm~\cite{zhang2019root}, have been shown to play a crucial role in stabilizing the training of Large Language Models (LLMs)~\cite{loshchilov2024ngpt}. By normalizing the activations of hidden units, these layers help to address issues such as vanishing/exploding gradients and improve the overall stability of the training process~\cite{xiong2020layer}. In this study, we investigate the impact of applying RMSNorm layer immediately before the task head.

Our experiment results, depicted in Fig.~\ref{fig:last_layernorm}, demonstrate an advantage for a model incorporating RMSNorm layer before the task head. This configuration consistently exhibits improved generalization performance and enhanced training stability compared to a model without the final RMSNorm. This finding suggests that normalizing the final embedding vector before passing it to the task head can benefit model performance, potentially by facilitating more effective gradient flow and thus improving the robustness of the model's predictions.


\subsection{Experiment Results from a Robotics Perspective} \label{sec:robotic_perspective}

\begin{figure}[h]
  \centering
  \includegraphics[width=0.7\columnwidth]{figures/order_prediction}
  \caption{\textbf{Kinodynamics Understanding:} Without unified latent representation the model cannot capture temporal dependencies and understand kinodynamic transitions, resulting in an almost flat learning curve.}
  \label{fig:unified_state}
  % \vspace{-1.2em}
\end{figure}
\textbf{Unified latent space representation} offers a significant advantage in simultaneously addressing FKD, IKD, and BC. This unified approach facilitates a more holistic understanding of the robot's state and its interaction with the environment. To evaluate the efficacy of this unified representation, we perform a targeted ablation study. We train \coder~based on the objectives outlined by~\citet{nazeri2024vertiencoder} and augment them with additional objectives specifically designed to probe the model's capacity of kinodynamics understanding. 

A key component of this ablation involves the introduction of a sequence order prediction objective. This objective aims to assess whether the model can effectively discern the temporal evolution of robot and environment dynamics. During training, the model is presented with input sequences in two configurations: (1) 50\% of the time, the input sequence is presented in its natural temporal order; (2) the remaining 50\% of the time, the input sequence is randomly shuffled, disrupting the temporal coherence. The model is then tasked to classify whether an unseen sequence is presented in its original order or is shuffled, testing the model's ability to capture temporal dependencies and understand kinodynamic transitions.

As illustrated in Fig.~\ref{fig:unified_state}, our findings demonstrate a clear distinction in model performance based on the input representation. When the model is provided with separate, non-unified tokens, it exhibits a limited capacity of understanding the underlying kinodynamics and the learning loss barely drops. This suggests that processing information in a fragmented manner hinders the model's ability to capture temporal relationships and kinodynamic evolution, which is aligned with the findings by~\citet{zhou2024dinowm}. It may be possible to compensate by training with a larger dataset, which, however, is not always available in robotics.

Conversely, the utilization of a unified latent space representation significantly enhances the model's ability to discern temporal order and, consequently, understand the dynamics of the system. By consolidating relevant information into a single, cohesive representation, the model can effectively capture the interdependencies among different modalities and their evolution over time. This highlights the importance of a unified latent space representation in enabling robotic models to effectively learn and reason about complex dynamic systems when trained on limited data, in contrast to NLP and CV tasks where the data acquisition is easier.


\begin{figure}[h]
  \centering
  \includegraphics[width=1\columnwidth]{figures/steps_comparison}
  \caption{\textbf{Prediction Horizon:} \former~is capable of predicting a longer horizon without losing much accuracy due to its non-autoregressive nature.}
  \label{fig:pred_horizon}
  % \vspace{-1.2em}
\end{figure}
\textbf{Prediction horizon} is a critical factor in navigation planning. While longer prediction horizons can potentially lead to better planning by considering long-term effects, they also introduce greater uncertainty. This is because errors in early predictions can accumulate and lead to significant deviations in subsequent predictions. This issue is particularly relevant for autoregressive models such as the \vertidecoder~part of \former, where each prediction is based on the previous one. In such models, even a small error in the initial steps can propagate and amplify over time, causing the predicted trajectory to drift further away from the true path. To evaluate the impact of prediction horizon, we compare the performance of the autoregressive \vertidecoder~with the non-autoregressive \former, specifically focusing on their ability to maintain accuracy over long horizons. The results, shown in Fig.~\ref{fig:pred_horizon}, demonstrate that \former~is capable of predicting a longer horizon (two seconds) with less drift compared to its autoregressive counterpart even with a shorter horizon (one second). This highlights the advantage of non-autoregressive models in tasks requiring long-term prediction, as they are less susceptible to error accumulation.

\subsection{Experiment Results of Robotic Objective Functions} \label{sec:objective_perspective}
\begin{figure}[h]
  \centering
  \includegraphics[width=\columnwidth]{figures/patch_head}
  \caption{\textbf{Patch Prediction Head:} The inclusion of a patch reconstruction head results in a degradation of overall model performance. This counterintuitive result can be attributed to the inherent difficulty in accurately predicting the detailed structure of off-road terrain topography.}
  \label{fig:patch_head}
  % \vspace{-1.2em}
\end{figure}

\textbf{Patch prediction head}, as an auxiliary head to learn environment kinodynamics, was first introduced by~\citet{nazeri2024vertiencoder}. However, we find that the high complexity of off-road terrain topography and the potential presence of noise or occlusion within the input data create a challenging reconstruction task (see Fig.~\ref{fig:cover}). Consequently, the patch prediction head often generates inaccurate reconstructions, introducing noise into the learning process and negatively impacting the performance of the primary tasks, i.e., FKD, IKD, and BC. This suggests that the auxiliary task of patch reconstruction, in this specific domain, may introduce a conflicting learning signal that hinders the model's ability to effectively learn the desired representations for the main objectives (Fig.~\ref{fig:patch_head}).


\textbf{MM vs NTP vs End-to-End} (End2End) are currently the prominent approaches in CV, NLP, and robotics respectively. However, it is unclear what is the best approach for robot learning, especially learning off-road mobility. We present a comparative analysis of model performance utilizing the MM paradigm within an encoder architecture (\coder, Fig.~\ref{fig:former} left trained alone with MM), a decoder employing autoregressive NTP (\vertidecoder, Fig.~\ref{fig:former} right trained alone without cross-attention), and a non-Transformer-based End2End approach. We then further contrast these approaches with \former,  which adopts a non-autoregressive approach to NTP and MM (Fig.~\ref{fig:former}, trained end-to-end). 

\begin{figure}[t]
  \centering
  \includegraphics[width=\columnwidth]{figures/4model_comparison}
  \caption{\textbf{MM vs NTP  vs End2End:} \former~achieves best accuracy across FKD, IKD, and BC compared to \coder~(MM), \vertidecoder~(NTP), and End2End.}
  \label{fig:mm_vs_ntp}
  % \vspace{-1.2em}
\end{figure}

To be specific, an encoder model leverages the principles of MM, wherein portions of the input sequence (poses, actions, and terrain patches) are masked, and the model is trained to reconstruct the masked elements. This approach has demonstrated success in capturing contextual dependencies and learning robust representations~\cite{nazeri2024vertiencoder};
A decoder model employs NTP, a prevalent technique in autoregressive sequence generation. In this paradigm, the model predicts the subsequent element in a sequence conditioned on the preceding elements. For both encoder and decoder models, we use the same unified latent space representation presented in Sec.~\ref{sec:unified}. The specialized non-Transformer-based End2End approach uses Resnet-18~\cite{he2015deep} as a patch encoder and fully connected layers as the task heads. While more complex models might offer higher accuracy, we choose ResNet-18 to balance performance with the computational constraints of our robotic platform, making it well-suited for deployment on robots with limited on-board processing capabilities, compared to deeper networks like ResNet-50 or ResNet-101. More information about End2End model architecture is provided in Appendix~\ref{app:architecture}.


As illustrated in Fig.~\ref{fig:mm_vs_ntp}, our findings indicate that \former, a non-autoregressive Transformer, exhibits superior performance across various evaluation metrics, including FKD, IKD, and BC error rates, in the context of one-second prediction horizon. Compared to \vertidecoder, \former~predicts multiple future states simultaneously (i.e., non-autoregressively), which contributes to its better accuracy. These results suggest that the enhanced contextual awareness afforded by the non-autoregressive approach contributes to improved predictive accuracy. Note that \vertidecoder~cannot perform BC directly, as it has access to both action and pose at each step. Unlike \coder~\cite{nazeri2024vertiencoder}, \former~does not train different downstream heads separately each time and all tasks contribute to the performance of each other all together, which results in \former's lowest error rate in most cases (except for $\mathbf{Z}$ prediction). 
Across all kinodynamics tasks, End2End achieves the highest error rate, which shows the benefits of using Transformers for kinodynamic representation and understanding during off-road mobility tasks. 

Beyond the observed performance gains and training stability, \former~demonstrates the capacity of concurrent execution of multiple tasks, not only during training but also during inference. This is particularly relevant in robotics, where real-time control is required and sometimes some modalities may not be available during inference. For example, without a global planner, action sampler, or in the presence of sensor degradation, the robot may not always have access to desired future robot poses, candidate actions, or future terrain patches, respectively. 
Furthermore, the usage of a learned mask within the decoder part of \former~is posited to capture salient distributional characteristics of the data, effectively serving as a condensed representation during inference. This learned representation facilitates adaptation to new tasks where action or pose is missing.


\vspace{-6pt}
\section{Results} \label{results}
\begin{table}[ht!]
\centering
\caption{\textbf{Super Resolution Performance Results.} Our proposed WGAN EEG Spatial Upsampling method significantly outperforms a baseline of Bicubic Interpolation commonly used in EEG upsampling pipelines.}
\label{tab:results}
\resizebox{0.8\linewidth}{!}{%
\begin{tabular}{@{}cccccc@{}}
\toprule
\multirow{2}{*}{\textbf{Dataset}} & \multirow{2}{*}{\textbf{Scale}} & \multicolumn{2}{c}{\textbf{Bicubic}} & \multicolumn{2}{c}{\textbf{WGAN}} \\ \cmidrule(l){3-6} 
                      &   & \textbf{MSE} & \textbf{MAE} & \textbf{MSE}    & \textbf{MAE}   \\
\toprule
\multirow{2}{*}{Val}  & 2 & 3.71E7       & 3.89E3       & \textbf{2.01E3} & \textbf{24.38} \\
                      & 4 & 7.23E7       & 6.42E3       & \textbf{8.53E3} & \textbf{63.83} \\
\midrule
\multirow{2}{*}{Test} & 2 & 3.75E7       & 3.91E3       & \textbf{2.06E3} & \textbf{24.66} \\
                      & 4 & 7.30E7       & 6.45E3       & \textbf{8.68E3} & \textbf{64.39} \\
\bottomrule
\end{tabular}%
}
\end{table}
% \vspace{-5pt}
\section{Discussion} \label{discussion}
This work identifies signal collapse as a critical bottleneck in one-shot neural network pruning. Performance loss in pruned networks is due to \textbf{signal collapse} in addition to the removal of critical parameters. We propose \textbf{REFLOW} (\textbf{Re}storing \textbf{F}low of \textbf{Low}-variance signals), a simple yet effective method that mitigates signal collapse without computationally expensive weight updates. By focusing on signal preservation, REFLOW highlights the importance of mitigating signal collapse in sparse networks and enables magnitude pruning to match or surpass state-of-the-art one-shot pruning methods such as CHITA, CBS, and WF.

REFLOW consistently achieves state-of-the-art accuracy across diverse architectures, restoring ResNeXt-101 from under 4.1\% to 78.9\% top-1 accuracy at 80\% sparsity on ImageNet. Its lightweight design makes it a practical solution for both research and deployment, delivering high-quality sparse models without the overhead of traditional approaches. These findings challenge the traditional emphasis on weight selection strategies and underscore the critical role of signal propagation for achieving high-quality sparse networks in the context of one-shot pruning.




\section{Conclusion and Future Work} \label{conclusion}
\section*{Conclusion}
This paper aims to enhance our understanding of the computational complexity of computing various Shapley value variants. We found that for various ML models --- including decision trees, regression tree ensembles, weighted automata, and linear regression --- both local and global interventional and baseline SHAP can be computed in polynomial time under HMM modeled distributions. This extends popular algorithms, such as TreeSHAP, beyond their empirical distributional scope. We also establish strict complexity gaps between the various SHAP variants (baseline, interventional, and conditional) and prove the intractability of computing SHAP for tree ensembles and neural networks in simplified scenarios. Overall, we present SHAP as a versatile framework whose complexity depends on four key factors: \begin{inparaenum}[(i)] \item model type, \item SHAP variant, \item distribution modeling approach, \item and local vs. global explanations\end{inparaenum}. We believe this perspective provides deeper insight into the computational complexity of SHAP, paving the way for future work.




%We believe that our framework provides a more intricate understanding of SHAP computation complexity across different models, distributions, and variants, paving the way for further research.

Our work opens promising directions for future research. First, expanding our computational analysis to other SHAP-related metrics, such as asymmetric SHAP~\citep{frye20} and SAGE~\citep{covert2020understanding}, would be valuable. Additionally, we aim to explore more expressive distribution classes and relaxed assumptions beyond those in Section \ref{sec:tractable} while maintaining tractable SHAP computation. Finally, when exact computation is intractable (Section \ref{sec:intractable}), investigating the approximability of SHAP metrics through approximation and parameterized complexity theory~\citep{downey2012parameterized} is an important direction.

%Our work opens several promising avenues for future research on the computational properties of explainable AI methods, with a particular focus on SHAP. First, it would be interesting to broaden the computational analysis conducted in this work to include other popular SHAP-related metrics in the literature, such as asymmetric SHAP \cite{frye20} and SAGE \cite{covert2020understanding}. Also, in the future, we aim to explore more expressive distribution classes and relaxed distributional assumptions—extending beyond those examined in Section \ref{sec:tractable} —that still yield tractable SHAP computation. Finally, when exact computation proves intractable (Section \ref{sec:intractable}), it is worthwhile to theoretically investigate the question of the approximability of computing the SHAP metrics across various configurations, through the lens of approximation and parametrized complexity theory \cite{arora2009computational}.

%This paper aims to deepen our understanding of the computational complexity involved in obtaining different Shapley value variants. We found that for a variety of ML models, including decision trees, tree ensembles for regression, weighted automata, and linear regression models — computing both local and global interventional and baseline SHAP can be done in polynomial time when distributions are modeled by HMMs. This extends the distributional scope of popular algorithms like TreeSHAP, which is limited to empirical distributions. Additionally, we demonstrate a strict complexity gap between SHAP variants, showing that interventional and baseline SHAP can be strictly easier to compute than conditional SHAP. Despite these positive results, we uncovered intractability for various SHAP variants in neural networks and tree ensembles. Finally, we provided generalized complexity relations across SHAP variants. We believe that our framework offers a deeper understanding of the complexity involved in computing SHAP across various variants, models, distributions, as well as in both local and global computations, laying the groundwork for future research.

% \newpage

\bibliographystyle{ieeetr}
\bibliography{main}

\end{document}
