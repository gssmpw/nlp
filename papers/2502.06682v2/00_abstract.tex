\begin{abstract}

Self-driving cars relying solely on ego-centric perception face limitations in sensing, often failing to detect occluded, faraway objects. Collaborative autonomous driving (CAV) seems like a promising direction, but collecting data for development is non-trivial. It requires placing multiple sensor-equipped agents in a real-world driving scene, simultaneously! As such, existing datasets are limited in locations and agents. We introduce a novel surrogate to the rescue, which is to generate realistic perception from different viewpoints in a driving scene, conditioned on a real-world sample---the ego-car's sensory data. This surrogate has huge potential: it could potentially turn any ego-car dataset into a collaborative driving one to scale up the development of CAV.
We present the very first solution, using a combination of simulated collaborative data and real ego-car data. 
Our method \textbf{Transfer Your Perspective (\ours)} learns a conditioned diffusion model whose output samples are not only realistic but also consistent in both semantics and layouts with the given ego-car data. Empirical results demonstrate \ours's effectiveness in aiding in a CAV setting. In particular, \ours enables us to (pre-)train collaborative perception algorithms like early and late fusion with little or no real-world collaborative data, greatly facilitating downstream CAV applications.

\end{abstract}
