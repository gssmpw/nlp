%%%%%%%% ICML 2025 EXAMPLE LATEX SUBMISSION FILE %%%%%%%%%%%%%%%%%

\documentclass{article}

% Recommended, but optional, packages for figures and better typesetting:
\usepackage{microtype}
\usepackage{graphicx}
\usepackage{subfigure}
\usepackage{booktabs} % for professional tables

% hyperref makes hyperlinks in the resulting PDF.
% If your build breaks (sometimes temporarily if a hyperlink spans a page)
% please comment out the following usepackage line and replace
% \usepackage{icml2025} with \usepackage[nohyperref]{icml2025} above.
\usepackage{hyperref}


% Attempt to make hyperref and algorithmic work together better:
\newcommand{\theHalgorithm}{\arabic{algorithm}}

% Use the following line for the initial blind version submitted for review:
% \usepackage{icml2025}

% If accepted, instead use the following line for the camera-ready submission:
\usepackage[accepted]{icml2025}

% For theorems and such
\usepackage{amsmath}
\usepackage{amssymb}
\usepackage{mathtools}
\usepackage{amsthm}
\usepackage{multirow}

\usepackage{makecell}
% \usepackage{algpseudocode}
\usepackage{eqparbox}
\usepackage{array}
\usepackage{pifont}
\usepackage{wrapfig}
\usepackage{amsfonts}
\usepackage{colortbl}
\usepackage{fontenc}
% \usepackage{tikz}
\usepackage{xspace}
\usepackage{enumitem}
\usepackage{marvosym}

\definecolor{baselinecolor}{rgb}{0.9, 0.9, 1.}
\definecolor{graycolor}{gray}{0.9}
\definecolor{Green}{rgb}{0.0, 0.5, 0.0}
\definecolor{Green}{rgb}{0.0, 0.5, 0.0}
\definecolor{rebuttal}{rgb}{0.6, 0.6, 1.}
\newcommand{\baseline}[1]{\cellcolor{baselinecolor}{#1}}
\newcommand{\graybase}[1]{\cellcolor{graycolor}{#1}}


\newcommand{\cmark}{\ding{51}}%
\newcommand{\xmark}{\ding{55}}%


\makeatletter
\DeclareRobustCommand\onedot{\futurelet\@let@token\@onedot}
\def\@onedot{\ifx\@let@token.\else.\null\fi\xspace}
\def\eg{\emph{e.g}\onedot} \def\Eg{\emph{E.g}\onedot}
\def\ie{\emph{i.e}\onedot} \def\Ie{\emph{I.e}\onedot}
\def\cf{\emph{c.f}\onedot} \def\Cf{\emph{C.f}\onedot}
\def\etc{\emph{etc}\onedot} \def\vs{\emph{vs}\onedot}
\def\wrt{w.r.t\onedot} \def\dof{d.o.f\onedot}
\def\etal{\emph{et al}\onedot}
\makeatother

% if you use cleveref..
\usepackage[capitalize,noabbrev]{cleveref}

%%%%%%%%%%%%%%%%%%%%%%%%%%%%%%%%
% THEOREMS
%%%%%%%%%%%%%%%%%%%%%%%%%%%%%%%%
\theoremstyle{plain}
\newtheorem{theorem}{Theorem}[section]
\newtheorem{proposition}[theorem]{Proposition}
\newtheorem{lemma}[theorem]{Lemma}
\newtheorem{corollary}[theorem]{Corollary}
\theoremstyle{definition}
\newtheorem{definition}[theorem]{Definition}
\newtheorem{assumption}[theorem]{Assumption}
\theoremstyle{remark}
\newtheorem{remark}[theorem]{Remark}

% Todonotes is useful during development; simply uncomment the next line
%    and comment out the line below the next line to turn off comments
%\usepackage[disable,textsize=tiny]{todonotes}
\usepackage[textsize=tiny]{todonotes}

\newcommand*{\affaddr}[1]{#1} 
\newcommand*{\affmark}[1][*]{\textsuperscript{#1}}
\newcommand{\authormark}[2][]{%
  \begingroup
  \def\@thefnmark{#1}%
  \footnote{#2}%
  \endgroup
}



\author{%
\bf
Tao Zhang\affmark[1,3]~~~ 
Cheng Da\affmark[2]~~~ 
Kun Ding\affmark[1]~~~
Kun Jin\affmark[2]~~~ 
Yan Li\affmark[2]~~~ 
Tingting Gao\affmark[2]~~~ 
Di Zhang\affmark[2]~~~ \\
\bf
Shiming Xiang\affmark[1,3]~~~ 
Chunhong Pan\affmark[1]~~~ \\
\affaddr{\affmark[1]MAIS, CASIA~~~~~~~~~~}
\affaddr{\affmark[2]Kuaishou Technology~~~~~~~~~~} 
\affaddr{\affmark[3]School of Artificial Intelligence, UCAS~~~~~~~~~~} \\
\small
}


% The \icmltitle you define below is probably too long as a header.
% Therefore, a short form for the running title is supplied here:
% \icmltitlerunning{Submission and Formatting Instructions for ICML 2025}
\icmltitlerunning{Diffusion Model as a Noise-Aware Latent Reward Model for Step-Level Preference Optimization}

\begin{document}

\twocolumn[
\icmltitle{Diffusion Model as a Noise-Aware Latent Reward Model for Step-Level Preference Optimization}

\icmlkeywords{Diffusion Model, Preference Optimization, Reward Model, Image Generation}

\vskip 0.3in

\arxivauthor
]

\setlength{\abovedisplayskip}{5pt}
\setlength{\belowdisplayskip}{5pt}
% \setlength{\arraycolsep}{2pt}

% this must go after the closing bracket ] following \twocolumn[ ...

% This command actually creates the footnote in the first column
% listing the affiliations and the copyright notice.
% The command takes one argument, which is text to display at the start of the footnote.
% The \icmlEqualContribution command is standard text for equal contribution.
% Remove it (just {}) if you do not need this facility.

% \printAffiliationsAndNotice{}  % leave blank if no need to mention equal contribution
% \printAffiliationsAndNotice{\icmlEqualContribution} % otherwise use the standard text.

\begin{abstract}
% 视觉语言模型用于逐步偏好优化时的缺点--> 扩散模型具有这些能力--> 提出LRM,提出LPO --> 效果。

Preference optimization for diffusion models aims to align them with human preferences for images. Previous methods typically leverage Vision-Language Models (VLMs) as pixel-level reward models to approximate human preferences. However, when used for step-level preference optimization, these models face challenges in handling noisy images of different timesteps and require complex transformations into pixel space. In this work, we demonstrate that diffusion models are inherently well-suited for step-level reward modeling in the latent space, as they can naturally extract features from noisy latent images. Accordingly, we propose the \textbf{Latent Reward Model (LRM)}, which repurposes components of diffusion models to predict preferences of latent images at various timesteps. Building on LRM, we introduce \textbf{Latent Preference Optimization (LPO)}, a method designed for step-level preference optimization directly in the latent space. Experimental results indicate that LPO not only significantly enhances performance in aligning diffusion models with general, aesthetic, and text-image alignment preferences, but also achieves 2.5-28$\times$ training speedup compared to existing preference optimization methods. Our code and models are available at
\url{https://github.com/Kwai-Kolors/LPO}.
\end{abstract}

\section{Introduction}
\label{sec:intro}


\ps{Challenges of technology scaling}

The growing demand for computing performance has always been met by increasing the number of transistors per chip, which is only possible due to CMOS technology scaling.
However, as we keep pushing the boundaries of technology scaling, we encounter multiple challenges.
Firstly, whenever we transition to a more advanced technology node, the non-recurring cost due to physical design, verification, software, mask sets, and prototyping almost doubles \cite{cost-tech-node}.
As a result, designing a chip in an advanced technology node is only economically viable if the chip is manufactured in vast quantities.
Secondly, many chip components such as I/O drivers, analog circuits, or \gls{srams} have reached their scaling limit.
This means that we cannot shrink these components further, even if we use a more advanced technology with a smaller feature size.
Thirdly, advanced technology nodes suffer from high defect rates, diminishing the yield and inflating the recurring cost.
To tackle these challenges, new chip-design paradigms have been developed.

\ps{Why 2.5D integration?}

One of these new paradigms is 2.5D integration, where multiple silicon dies called chiplets are integrated into the same package.
Once designed, a single chiplet can be reused in multiple 2.5D stacked chips, which increases the ratio of production volume to non-recurring cost.
Another advantage is that multiple chiplets - fabricated in different technologies - can be integrated into the same package.
This means that only components that can take full advantage of technology scaling are built in bleeding-edge technologies.
Components that have reached their scaling limit are fabricated in more mature and hence less costly technology nodes.
Furthermore, chiplets are smaller than monolithic chips.
Therefore, manufacturing chiplets results in less silicon area loss due to fabrication defects and hence a higher yield.
Due to these economic advantages, chip vendors such as AMD \cite{amd-chiplet} and NVIDIA \cite{chiplet-book} have adopted the 2.5D integration paradigm.  

\ps{Challenges of 2.5D integration}

An important challenge when designing 2.5D stacked chips is the construction of a low-latency and high-throughput \gls{ici}. 
To build an \gls{ici}, we connect different chiplets using \gls{d2d} links.
These links are fabricated in an organic package substrate, silicon bridge, or silicon interposer, and they are connected to the chiplets using \gls{c4} bumps or microbumps.
The number of bumps per chiplet is limited, and so is the bandwidth of \gls{d2d} links.
In addition to having lower bandwidth than links in monolithic chips, \gls{d2d} links also have higher latency.
This latency is caused by wire delay and by \gls{phys} that are necessary in both the sending and the receiving chiplet.
\gls{phys} are needed to convert between protocols, voltage levels, and frequencies, which are usually different between on-chiplet links and \gls{d2d} links.
Due to these limitations, the \gls{ici} can quickly become a bottleneck.

\ps{How we solve these challenges differently than the related work does.}

Existing approaches to maximize the performance of the \gls{ici} either optimize the placement of chiplets (with potentially heterogeneous shapes) for a predetermined \gls{ici} topology 
\cite{ho,liu,seemuth,eris,osmolovskyi,tap25d,chiou}, select one topology out of a set of candidates \cite{coskun-1, coskun-2}, or they optimize the \gls{ici} topology for a 2D grid of homogeneously shaped chiplets on an active interposer \cite{butterdonut, cluscross, kite}.
To the best of our knowledge, there is no prior work on \gls{ici} topologies for chips with heterogeneously shaped chiplets or with passive silicon interposers or silicon bridges.
To fill this gap, we propose \name, a novel optimization methodology to jointly optimize the chiplet placement and \gls{ici} topology of such architectures.
\ifnb
\else
\newpage
\fi

\ps{Details on \name~and the key idea}

The key idea is as follows: 
We optimize the chiplet placement without a predetermined topology.
For each placement generated by an optimization algorithm, we infer a placement-based \gls{ici} topology by connecting chiplets that are in close proximity in that specific placement.
We then compute the latency and throughput of this combination of placement and topology for different traffic types.
These latencies and throughputs together with the total chip area are used to compute a user-defined quality-score of the placement, which is returned to the optimization algorithm.
Based on this quality score, the algorithm can further optimize the placement.
By following this iterative process, we jointly optimize the chiplet placement and the \gls{ici} topology.

\ps{Short evaluation-summary}

We provide our open-source framework implementing the proposed placement and topology co-optimization methodology, which we evaluate using both synthetic traffic and traffic traces.
A 2D grid of chiplets with a mesh topology is used as a baseline since many proposals for 2.5D stacked chips \cite{dataflow_accel_dnn, cifher, simba, hecaton, dojo} use such an architecture.
We reduce the latency of synthetic L1-to-L2 and L2-to-memory traffic, the two most important traffic types for cache coherency traffic, by up to 28\% and 62\% respectively.
For real traffic traces, we reduce the average packet latency for almost all traces and architectures considered (reduced by an 8\% or 18\% on average depending on the configuration of \gls{phys} within a chiplet).



\section{Related work}


Recent advances in single-image animatable head avatar generation can be categorized into mainly 2D-based and 3D-based approaches. 

\paragraph{\bf Image to 2D Animatable Avatar.}
2D-based methods, leveraging the power of convolutional neural networks (CNNs)~\cite{DBLP:conf/cvpr/KarrasLAHLA20,DBLP:conf/cvpr/IsolaZZE17,DBLP:conf/nips/GoodfellowPMXWOCB14}, often employ generative adversarial networks (GANs)~\cite{DBLP:conf/cvpr/StyleGAN} for direct image synthesis. Early approaches~\cite{DBLP:conf/cvpr/WangDYSW23,DBLP:conf/cvpr/BurkovPGL20,DBLP:conf/iccv/ZakharovSBL19} focus on injecting expression and pose features into the generator network, often utilizing architectures like U-Net or StyleGAN~\cite{DBLP:conf/cvpr/StyleGAN}.
Some other 2D methods~\cite{DBLP:journals/corr/abs-2407-03168,DBLP:conf/cvpr/ZhangQZZW0CW023,DBLP:conf/cvpr/HongZS022,DBLP:conf/mm/DrobyshevCKILZ22,DBLP:conf/cvpr/BurkovPGL20,DBLP:conf/nips/SiarohinLT0S19} represent expressions and poses as warping fields applied to the source image. 
Benefiting from advances in image and video diffusion networks, more recent 2D-based works~\cite{DBLP:journals/corr/abs-2410-07718,DBLP:journals/corr/abs-2406-08801,DBLP:conf/eccv/TianWZB24} get improved results with diffusion techniques. 
However, these methods still face challenges related to long generation times and significant computational resource demands. Audio-driven 2D control methods~\cite{DBLP:conf/cvpr/ZhangCWZSGSW23,DBLP:journals/corr/abs-2211-12368,DBLP:conf/iccv/GuoCLLBZ21} are easy to use but cannot explicitly control facial expressions and poses. 2D-based techniques often struggle with large pose or expression variations due to the lack of an explicit 3D structure, sometimes producing unrealistic distortions or identity changes. While some 2D methods~\cite{SadTalker,StyleHEAT,Pirenderer,DBLP:conf/cvpr/WangM021,MegaPortraits} incorporate 3D Morphable Models (3DMMs)~\cite{DBLP:conf/fgr/GerigMBELSV18,DBLP:journals/tog/LiBBL017,DBLP:conf/avss/PaysanKARV09,DBLP:conf/siggraph/BlanzV99} to mitigate these issues, they typically cannot achieve free-viewpoint rendering. 

\vspace{-0.1in}

\begin{figure*}[h]
    \centering
    \includegraphics[width=0.9\linewidth]{images/framework.pdf}
    \caption{\textbf{Overall Framework.} Our framework utilizes learnable query features attached to FLAME vertices to perform cross-attention with the extracted multi-level image features. The extracted features are then decoded to reconstruct the Gaussian avatar in the canonical space, which can be animated utilizing standard linear blend skinning (LBS) and corrective blendshapes as the FLAME model did and rendered in real-time on various platforms.}
    \label{fig:framework}
\end{figure*}

\paragraph{\bf Image to 3D Animatable Avatar.}
3D-aware methods offer improved geometric consistency and free-viewpoint rendering capabilities. Early 3D approaches~\cite{DBLP:conf/eccv/KhakhulinSLZ22,DBLP:conf/cvpr/XuYCWDJT20} utilize 3DMMs for head avatar reconstruction. With the advent of Neural Radiance Fields (NeRFs)~\cite{DBLP:conf/eccv/MildenhallSTBRN20}, many recent methods~\cite{DBLP:conf/siggraph/YuFZWYBCSWSW23,DBLP:conf/cvpr/MaZQLZ23,DBLP:conf/cvpr/LiZWZ0CZWB023,GPAvatar,ye2024real3d,deng2024portrait4d,deng2024portrait4d2,DBLP:conf/eccv/KiMC24,DBLP:conf/cvpr/BaiFWZSYS23,PointAvatar,Nerfies,INSTA} have adopted this representation for higher fidelity, particularly in modeling fine details like hair. However, NeRF-based~\cite{DBLP:conf/cvpr/ZhangZLHLWGCL024,HAvatar,DBLP:conf/cvpr/BaiTHSTQMDDOPTB23,AD-NeRF,DBLP:journals/tog/GaoZXHGZ22,DBLP:journals/tog/ParkSHBBGMS21,DBLP:conf/cvpr/AtharXSSS22,DBLP:journals/corr/abs-2112-05637,DBLP:conf/iccv/TretschkTGZLT21,DBLP:conf/cvpr/GafniTZN21,DBLP:conf/eccv/KiMC24,DBLP:conf/cvpr/BaiFWZSYS23,PointAvatar,Nerfies,DBLP:conf/siggraph/YuFZWYBCSWSW23,DBLP:conf/cvpr/MaZQLZ23,DBLP:conf/cvpr/LiZWZ0CZWB023} approaches often require extensive training data, including multi-view or single-view videos, raising privacy concerns and limiting generalization to unseen identities. Some methods~\cite{DBLP:conf/cvpr/SunWWLZZL23,DBLP:conf/3dim/ZhuangMKS22,DBLP:journals/pami/SunWZHWL24,DBLP:journals/tvcg/TangZYZCMW24,DBLP:conf/iclr/XuZLZBFS23} bypass this data requirement by training generators with random noise and then inverting them for identity-specific reconstruction, but inversion accuracy remains a challenge. Test-time optimization offers another alternative, but its computational cost limits practical applications. Several recent works~\cite{goha2023,hidenerf2023,gpavatar2024,ye2024real3d,ma2024cvthead,deng2024portrait4d,deng2024portrait4d2,GGHead} have explored one-shot 3D head reconstruction to address the limitations of data requirements and computational cost. These methods employ various techniques, such as tri-plane features, deformation fields, point-based expression fields, and vertex-feature transformers. Despite these advancements, NeRF-based methods often struggle with real-time rendering. 
Recently, 3D Gaussian Splatting~\cite{GaussianSplatting} has emerged as a promising alternative, offering both high-quality results and fast rendering speeds. However, existing Gaussian Splatting methods~\cite{GaussianAvatar,DBLP:conf/cvpr/XuCL00ZL24} typically rely on video data for training for each person, limiting their ability to generalize to new identities. Instead, the most recent work, GAGAvatar~\cite{GAGAvatar}, proposes a one-shot 3D Gaussian-based head avatar generation method. However, it still relies heavily on complex 2D neural post-processing to achieve optimal animation outcomes, thus it is not a pure 3D solution and the extra neural network hinders its application on various platforms. In contrast, our work generates Gaussian heads that are immediately animatable and renderable without additional networks or post-processing steps, enabling seamless integration into existing rendering pipelines for real-time animation and rendering across a wide range of platforms, including mobile phones. 


\section{Background}
\label{sec:background}

\paragraph{Causal ordering}
In the following, we consider a matrix $\mB \in \R^{p \times p}$ that encodes the structure of a directed acyclic graph (DAG), which means that the non-zero entries of $\mB$ represent the edges of the graph.
It follows that there exist a strictly lower triangular matrix $\mT$ and a permutation matrix $\mP$, referred to as the \emph{causal ordering}, such that $\mB = \mP^\top \mT \mP$.
In general, such an ordering is not unique.

\paragraph{LiNGAM, a model for causal discovery}
Let $\vx \in \R^p$ be a random vector of observations and let $\mB \in \R^{p \times p}$ represent a DAG.
We consider a structural equation model, also known as a functional causal model, where the data follows
\begin{align}
    \label{eq:causal_model_original}
    \vx = \mB \vx + \vs
\end{align}
where the entries $s_1,\ldots,s_p$ of the vector $\vs \in \R^p$ are independent noise terms, called disturbances.
Causal discovery consists in inferring the parameters $\mB$ of the model, from observations of $\vx$.
Yet, the identifiability of such a model and therefore the uniqueness of the inferred $\mB$ is not straightforward.
In fact, it is well-known that Gaussian noises make the model unidentifiable, in general~\citep{richardson2002identifiability,genin2021identifiability}.
A major advance was that \citet{shimizu2006linear} assumed the noise variables to be
\textit{non-Gaussian} leading to their Linear Non-Gaussian Acyclic Model (LiNGAM), which leads to identifiability, as discussed later.
In the model, we can then interpret $\mP$ as representing a reordering of the observations, such that the permuted entries $\mP \vx$ satisfy
\begin{align}
    \label{eq:causal_model}
    \mP \vx = \mT \mP \vx + \mP \vs
\end{align}
where the causal matrix between the $\mP\vx$ is now $\mT$ and thus strictly lower triangular.
This means any entry $(\mP \vx)_j$ is a weighted sum of previous entries $(\mP \vx)_{<j}$ and noise $(\mP \vs)_{j}$.
Because $(\mP \vx)_j$ does not depend on future entries $(\mP \vx)_{>j}$, we say that $\vx$ follows a causal ordering given by $\mP$.

\paragraph{Relation of LiNGAM to ICA}
The LiNGAM model, or any similar model based on~\eqref{eq:causal_model_original}, can be rewritten as a latent variable model, in particular an Independent Component Analysis (ICA) model \citep{Hyvabook} as
\begin{align}
    \label{eq:ica_model}
    \vx = \mA \vs
\end{align}
where, again, the entries in $\vs$ are independent and non-Gaussian, and the ``mixing" matrix $\mA$ expresses how the data is generated from the latents, and is given by $\mA = (\mI - \mB)^{-1}$, where $\mB$ is a DAG. Now, many methods developed for ICA can be used to estimate the matrix $\mA$, but it is important to take this special structure into account~\citep{shimizu2006linear}.
In particular, any ICA algorithm does not directly return the correct matrix $\mA$ but rather a related matrix where the columns of $\mA$ may appear in an arbitrary order.

\paragraph{Identifiability of LiNGAM} 
\citet{shimizu2006linear} showed that the LiNGAM model is identifiable in terms of the matrix $\mB$, with no indeterminacies unlike in basic ICA. A rigorous re-statement of this result is given in the following theorem which we prove for completeness in Appendix~\ref{app:ssec:lingam}.
\begin{theorem}[Identifiability of LiNGAM]
\label{theorem:lingam}
In the statistical model defined by~\eqref{eq:causal_model_original}, the parameter $\mB$ is identifiable, provided that the entries in $\vs$ are mutually independent, that at most one of them is Gaussian, and that $\mB$ is a DAG.
\end{theorem}
A further question that has received less attention is whether the model is identifiable in terms of the causal ordering $\mP$. In fact, it is not in general: specifically, there may exist many permutation matrices $\mP$ and strictly lower triangular matrices $\mT$ such that the generated data has the same distribution and the generating permutation cannot be identified.
For instance, in the degenerate case $\mB = \mT = \vzero$, any permutation matrix $\mP$ is equally valid and gives the same data distribution. As is well-known, a DAG in general defines only a partial order in the sense that for some pairs of variables, we cannot necessarily say which is ``earlier" and which is ``later". Thus, to make the causal ordering well-defined, we need further assumptions, as will be considered below.\footnote{One would argue that in some cases, $\mP$ is not even well-defined, and thus it cannot be identifiable, and it is not appropriate to use the concept of identiability. We prefer here to confound the concepts in the sense that we talk about identifiability of $\mP$ if it is uniquely defined and can be uniquely recovered from the data.
A more rigorous justification could be developed by assuming a hierarchical data generation process, where the $\mP$ and $\mT$ are generated first, and the $\mB$ is generated based on them. In that case, if the decomposition of $\mB$ is unique, and $\mB$ is identifiable, also $\mP$ is identifiable in the conventional statistical sense.\label{footnote:1}}


\paragraph{Multi-view ICA}
A multi-view version of ICA is of great practical interest.  One might obtain a number of views of the same data that might be, for example, different subjects in a biomedical context, different users in more technological applications, or different measurement systems of the same physical phenomenon. 
A multi-view extension of ICA
can then be defined in various ways. Here, we consider the case where the components are, at least partly, shared over views, while the mixing matrices (as well as optional noise terms) are view-dependent. 

This leads to the definition of Shared ICA \citep{richard2021sharedica,anderson2013multiviewidentifiability}:
\begin{align}
    \label{eq:multiview_ica_model}
    \vx^i = \mA^i (\vs + \vn^i)
\end{align}
where $\vx^i$ are the different views indexed by the view index $i$.
Recently, identifiability conditions have been explored for such multi-view ICA models. 
On the one hand, it is obvious that if the components are non-Gaussian, the Shared ICA model reduces to a standard ICA model when stacking the different views in a single vector, and hence identifiable. But \citet{richard2021sharedica,anderson2013multiviewidentifiability} showed the surprising result that even if more than one component is Gaussian, the model can still be identifiable if the variances of the noises $\vn^i$ are sufficiently diverse.


 \section{Method}
\label{sec:method}











Given a set $\{x_{1_i},c_i\}_{i=1}^m$ of input samples and their corresponding conditioning states, our goal is to construct a flow-matching model that samples from $q(x_1|c)$ that start from our conditional prior distribution (CPD). 

\subsection{Flow Matching from Conditional Prior Distribution}
\label{sec:conditional_fm_joint}

We generalize the framework of  Sec.~\ref{sec:flow_matching} to a construction that uses an arbitrary conditional joint distribution of $\rho(x_0, x_1, c)$ which satisfy the marginal constraints:
\begin{equation*}
\label{eq:conditional_marginal}
    \int \rho(x_0, x_1, c)dx_0 = q(x_1, c),  \int \rho(x_0, x_1, c)dx_1dc = p(x_0)
\end{equation*}
Then, building on flow matching, we propose to modify the conditional probability path so that at $t=0$, we define:
\begin{equation}
    \rho_0(x_0|x_1, c) = p(x_0|x_1, c) 
\end{equation}
where $p(x_0|x_1, c)$ is the conditional distribution $\frac{\rho(x_0, x_1, c)}{q(x_1, c)}$. 
Using this construction, we satisfy the boundary condition of Eq.~\ref{eq:boundary_conditions}: 
\begin{align}
    \rho_0(x_0) &= \int\rho_0(x_0|x_1, c)q(x_1, c)dx_1dc  \\
                &=  \int p(x_0|x_1, c)dx_1dc = p(x_0)
\end{align}




The conditional probability path $\rho_t(x|x_1, c)$ does not need to be explicitly formulated. Instead, only its corresponding conditional vector field $u_t(x|x_1, c)$ needs to be defined such that points $x_0$ drawn from the conditional prior distribution $\rho_0(x_0|x_1, c) $, reach $x_1$ at $t=1$, i.e., reach distribution $\rho_1(x|x_1, c) = \delta(x - x_1)$.  We thus purpose the \emph{Conditional Generation Joint FM} $\gL_{\rm cgjfm}(\theta)$ objective:
\begin{equation}\label{eq:conditionl_joint_cfm}
    \mathbb{E}_{t\sim \mathcal{U}(0,1), q(x_0,x_1,c)} \|v_\theta(t, x, c) - u_t(x | x_1, c)\|^2
\end{equation}
where $x = \psi_t(x_0|x_1,c)$.
Training only involves sampling from $q(x_0,x_1,c)$ and does not require explicitly defining the densities $q(x_0,x_1,c)$ and $\rho_t(x|x_1,c)$.
We note that this objective is reduced to the CGFM objective Eq.~\ref{eq_conditional_generative_fm_objective} when $q(x_0,x_1,c) = q(x_1, c)p(x_0)$.

\subsection{Conditional Prior Distribution}
\label{sec:prior_distribution}

We now describe our choice of a condition-specific prior distribution. 
When choosing a conditional prior distribution we want to adhere to the following design principles:
(i) \emph{Easy to sample}: can be efficiently sampled from.
(ii) Well represents the target conditional modes. 
We design a condition-specific prior distribution based on a parametric \emph{Mixture Model} where each mode of the mixture is correlated to a specific conditional distribution $p(x_1|c)$. 
Specifically, we choose the prior distribution to be the following, \emph{easy to sample}, \emph{Gaussian Mixture Model} (GMM):
\begin{equation}\label{eq:gmm_formula}
    p_0 = \mathrm{GMM}(\gN(\mu_i, \Sigma_i)_{i=1}^n, \pi)
\end{equation}

where $\pi\in\R^n$ is a probability vector associated with the number of conditions $n$ (could be $\infty$) and $\mu_i, \Sigma_i$ are parameters determined by the conditional distribution $q(x_1|c_i)$ statistics, \emph{i.e.} 
 \begin{equation}\label{eq:gmm_parameters}
     \mu_i = \E[x_1|c_i], \quad \Sigma_i = \mathrm{cov}[x_1|c_i]
 \end{equation}
To sample from the marginal distribution $p(x_0|x_1, c_i)$, we sample from the cluster $\gN(\mu_i, \Sigma_i)$ associated with the condition $c_i$.

\noindent \textbf{Obtaining a Lower Global Truncation Error.} \quad 
Our CPD fits a GMM to the data distribution in a favorable setting, where the association between samples and clusters is given. 
\begin{equation}\label{eq:wasserstein_definition}
    d_1 \left(X, Y \right) \coloneqq \sup_{h \in \mathrm{Lip_1}} \mathbb{E}[h(X) - h(Y)] .
\end{equation}

In this process, we fit a dedicated Gaussian distribution to data points with the same condition. If the latter are close to being unimodal, this approximation is expected to be tight, in terms of the average distances between samples from the condition data mode and the fitted Gaussian. 
Tab.~\ref{tab:wasserstein_table} provides the average distances between pairs of samples from the prior and data distributions (i.e. the \emph{transport cost}) of CondOT~\cite{lipman2022flow}, BatchOT~\cite{pooladian2023multisample} and our CPD over the ImageNet-64~\cite{deng2009imagenet} and MS-COCO~\cite{lin2014microsoft} datasets. 
As expected, BatchOT which minimizes this exact measure within mini-batches, obtains better scores than the naïve pairing used in CondOT, while our CPD, which approximates the data using a GMM exploits the conditioning available in these datasets, and obtains considerably lower average distances.

As noted in \cite{pooladian2023multisample}, lower transport cost is generally associated with straighter flow trajectories, more efficient sampling and lower training time. We want to substantiate this claim from the viewpoint of cumulative errors in numerical integration.
Sampling from flow-based models consists of solving a time-dependent ODE of the form $\dot{x}_t =u_t(x_t)$, where $u_t$ is the velocity field. This equation is solved by the following integral $x_t = \int_{0}^t u_s(x_s)ds$, where the initial condition $x_0 $ is sampled from the prior distribution. Numerical integration over discrete time steps accumulate an error at each step $n$ which is known as the \emph{local truncation error $\tau_n$}, which accumulates into what is known as the \emph{global truncation error $e_n$}.  This error is bounded by ~\cite{suli2003introduction}
\begin{equation}
    |e_n| \leq \frac{max_j\tau_j}{hL}\big(e^{L(t_n-t_0)} - 1\big)
\end{equation}\label{eq:truncation_error_bound} 
where $h$ is the step size and $L$ is the Lipschitz constant of the velocity $u_t$. 
Accordingly, the distance between the endpoints of a path $\Delta = |x_1  - x_0|$  is given by $|\int_0^1 u_s(x_s)ds|$ which can be interpreted as the magnitude of the average velocity along the path $x_t$. Hence, the longer the path $\Delta$ is, the larger the integrated flow vector field $u_t$ is.
For example, if we scale a path uniformly by a factor $C>1$, i.e., $x_t \rightarrow C(x_t)$, we get,  $\frac{d}{dt}C(x_t) = C(u_t)$ in which case the Lipschitz constant $L$ is also multiplied by $C$.

By shortening the distance between the prior and and data distribution, as our CPD does, we lower the integration errors which permits the use of coarser integration steps, which in turn yield smaller global errors. Thus, our construction allows for fewer integration steps during sampling.

\subsubsection{Construction}


Next, we explain how we construct $p_0$ (Eq.~\ref{eq:gmm_formula}) for both the discrete case (e.g., class conditional generation) and continuous case (e.g., text conditional generation). 

\noindent \textbf{Discrete Condition.} \quad
In the setup of discrete conditional generation, we are given data $\{x_{1_i}, c_i\}_{i=1}^m$ where there are a finite set of conditions $c_i$.
We approximate the statistics of Eq.~\ref{eq:gmm_parameters} using the training data statistics. That is, we compute the mean and covariance matrix of each class (potentially in some latent represntation of a pretrained auto-encoder).  Since the classes at inference time are the same as in training, we use the same statistics at inference. 

\noindent \textbf{Continuous Condition.} \quad
While in the discrete case we can directly approximate the statistics in Eq.~\ref{eq:gmm_parameters} from the training data, in the continuous case (\emph{e.g.} text-conditional) we need to find those statistics also for conditions that were not seen during training. To this end, we first consider a joint representation space for training samples $\{x_{1_i}, c_i\}_{i=1}^m$, which represents the semantic distances between the conditions $c_i$ and the samples $x_{1_i}$. In the setting where $c_i$ is text, we choose a pretrained CLIP embedding. 
$c_i$ is then mapped to this representation space, and then mapped to the 
data space (which could be a latent representation of an auto-encoder), using a learned mapper $\gP_\theta$. 
Specifically, $\gP_\theta$ is trained to minimize the objective:
\begin{equation}
    \gL_{\rm prior}(\theta) = \mathbb{E}_{q(x_1,c)} \|\gP_\theta(E(c)) - x_1\|^2_2.
\end{equation}
where $E$ is the pre-trained mapping to the joint condition-sample space (e.g. CLIP). $\gP_\theta$ can be seen as approximating $\E[x_1|c]$, which is used as the mean for the condition specific Gaussian.  
At inference, where new conditions (e.g., texts) may appear, we first encode the condition $c_i$ to the joint representation space (e.g., CLIP) followed by $\gP_\theta$. This mapping provides us with the center $\mu_i$ of each Gaussian. %
We also define $\Sigma_i = \sigma_i^2\mathrm{I}$ where $\sigma_i$ is a hyper-parameter, ablated in Sec.~\ref{sec:results_quantitative} 

\subsection{Training and Inference}

Given the prior $p_0$ (either using the data statistics or by training $\gP_\theta$), for each condition $c$, we have its associated Gaussian parameters $\mu_c$ and $\Sigma_c$. The map $\psi_t(x|x_1,c)$ must be defined in order to minimize Eq.~\ref{eq:conditionl_joint_cfm} above. This corresponds to the interpolating maps between this Gaussian at $t=0$ and a small Gaussian around $x_1$ at $t=1$, defined by:
\begin{align}
    \psi_{t}(x|x_1,c) &= \sigma_t(x_1,c)x + \mu_t(x_1,c), \\ 
    \sigma_t(x_1,c) &= t (\sigma_{\min} \mathrm{I}) + (1-t)\Sigma_{c}^{1/2}, \quad \text{and} \\
    \mu_t(x_1,c) &= t x_1 + (1-t) \mu_c.
\end{align}
This results in the following target flow vector field 
\begin{equation*}
    u_t(\psi_{t}(x|x_1,c)) = \frac{d}{dt}\psi_t (x|x_1,c)  =   \big(\sigma_{\min}  \mathrm{I} - \Sigma_c^{1/2}\big)x +  x_1 - \mu_c.
\end{equation*}

During inference we are given a condition $c$ and want to sample from $q(x_1|c)$. Similarly to the training, we sample $x_0\sim p(x_0|c)$ and solve the ODE 
\begin{equation}
    \frac{d}{dt} \psi_t(x) = v_\theta \left(t, \psi_t(x), c \right), \quad \psi_0(x) = x_0
\end{equation}
Training and implementation details are in the appendix.









\section{Experiments}
\label{sec:experiment}

\subsection{Experimental Setup}
\label{sec:exp_setup}
The experiments are mainly conducted on SD1.5 \cite{sd1} and SDXL \cite{sdxl} without refiner. The LRM is first trained on Pick-a-Pic and then used to fine-tune diffusion models through LPO. Unless otherwise specified, we employ \textit{homogeneous optimization}.

\textbf{LRM Training.} We denote the LRM based on SD1.5 and SDXL as LRM-1.5 and LRM-XL, respectively. They are trained on the filtered Pick-a-Pic v1 \cite{pickscore} as clarified in Sec.\;\ref{sec:lrm_train}. The $gs$ in the VFE module is set to 7.5. 
More details are in \cref{sec:experimental_detail}.

\textbf{LPO Training.} The same 4k prompts in SPO are used for the LPO training, randomly sampled from the training set of Pick-a-Pic v1. The DDIM scheduler \cite{ddim} with 20 inference steps is employed. We use all steps for sampling and training, \ie $t\in[0,50,...,900,950]$. The dynamic threshold range $[th_{min}, th_{max}]$ is set to $[0.35, 0.5]$ for SD1.5 and $[0.45, 0.6]$ for SDXL. The $\beta$ in Eqn.\;(\ref{eq:spo_loss}) is set to 500 and the $K$ in the sampling process is set to 4. Further details can be found in \cref{sec:experimental_detail}.

\begin{table}[t]
    \centering
    \vspace{-2.5mm}
    \caption{General and aesthetic preference scores on Pick-a-Pic validation unique set. $^*$ denotes the metrics are copied from \cite{spo}. Others are evaluated using the official model.}
    \vskip 0.05in
    \label{tab:preferenece_eval}
    \scriptsize
    \setlength{\tabcolsep}{1.0mm}{
    \scalebox{1.1}{
    \begin{tabular}{l c c c c c}
         \toprule
         Method & PickScore & ImageReward & HPSv2 & HPSv2.1 & Aesthetic \\
         \midrule
         \textcolor{gray}{SD1.5} & & & & & \\
         \hspace{1pt} Original & 20.56 & 0.0076 & 26.46 & 24.05 & 5.468 \\
         \hspace{1pt} $^*$DDPO & 21.06 & 0.0817 & - & 24.91 & 5.591 \\
         \hspace{1pt} $^*$D3PO & 20.76 & -0.1235 & - & 23.97 & 5.527 \\
         \hspace{1pt} Diff.-DPO & 20.99 & 0.3020 & 27.03 & 25.54 & 5.595 \\
         \hspace{1pt} SPO & 21.22 & 0.1678 & 26.73 & 25.83 & 5.927 \\
         \rowcolor{cyan!15}\hspace{1pt} LPO & \textbf{21.69} & \textbf{0.6588} & \textbf{27.64} & \textbf{27.86} & \textbf{5.945} \\
         \midrule
         \textcolor{gray}{SDXL} & & & & & \\
         \hspace{1pt} Original & 21.65 & 0.4780 & 27.06 & 26.05 & 5.920 \\
         \hspace{1pt} Diff.-DPO & 22.22 & 0.8527 & 28.10 & 28.47 & 5.939 \\
         \hspace{1pt} MaPO & 21.89 & 0.7660 & 27.61 & 27.44 & 6.095 \\
         \hspace{1pt} SPO & 22.70 & 0.9951 & 28.42 & 31.15 & 6.343 \\
         \rowcolor{cyan!15}\hspace{1pt} LPO & \textbf{22.86} & \textbf{1.2166} & \textbf{28.96} & \textbf{31.89} & \textbf{6.360} \\
         \bottomrule
    \end{tabular}}}
    % \vspace{-3mm}
    \vskip -0.15in
\end{table}


\begin{table*}[t]
    \vspace{-2.5mm}
    \caption{Quantitative results on T2I-CompBench++ \cite{t2i_compbench}.}
    \vskip 0.05in
    \label{tab:t2i_eval}
    \centering
    \scriptsize
    \setlength{\tabcolsep}{1.8mm}{
    \scalebox{1.1}{
    \begin{tabular}{c l c c c c c c c c}
         \toprule
         Model & Method & Color & Shape & Texture & 2D-Spatial & 3D-Spatial & Numeracy & Non-Spatial & Complex \\
         \midrule
         \multirow{4}{*}{SD1.5} & Original \cite{sd1} & 0.3783 & 0.3616 & 0.4172 & 0.1230 & 0.2967 & 0.4485 & 0.3104 & 0.2999 \\
         & Diff.-DPO \cite{diffusion_dpo} & 0.4090 & 0.3664 & 0.4253 & 0.1336 & 0.3124 & 0.4543 & \textbf{0.3115} & 0.3042 \\
         & SPO \cite{spo} & 0.4112 & 0.4019 & 0.4044 & 0.1301 & 0.2909 & 0.4372 & 0.3008 & 0.2988 \\
         & \cellcolor{cyan!15}LPO & 
         \cellcolor{cyan!15}\textbf{0.5042} &
         \cellcolor{cyan!15}\textbf{0.4522} & 
         \cellcolor{cyan!15}\textbf{0.5259} & 
         \cellcolor{cyan!15}\textbf{0.1928} & 
         \cellcolor{cyan!15}\textbf{0.3562} & 
         \cellcolor{cyan!15}\textbf{0.4845} & 
         \cellcolor{cyan!15}0.3110 &
         \cellcolor{cyan!15}\textbf{0.3308}\\
         \midrule
         \multirow{5}{*}{SDXL} & Original \cite{sdxl} & 0.5833 & 0.4782 & 0.5211 & 0.1936 & 0.3319 & 0.4874 & 0.3137 & 0.3327 \\
         & Diff.-DPO \cite{diffusion_dpo} & 0.6941 & 0.5311 & 0.6127 & 0.2153 & 0.3686 & 0.5304 & \textbf{0.3178} & 0.3525 \\
         & MaPO \cite{mapo} & 0.6090 & 0.5043 & 0.5485 & 0.1964 & 0.3473 & 0.5015 & 0.3154 & 0.3229 \\
         & SPO \cite{spo} & 0.6410 & 0.4999 & 0.5551 & 0.2096 & 0.3629 & 0.4931 & 0.3098 & 0.3467 \\
         & \cellcolor{cyan!15}LPO & 
         \cellcolor{cyan!15}\textbf{0.7351} & 
         \cellcolor{cyan!15}\textbf{0.5463} & \cellcolor{cyan!15}\textbf{0.6606} &
         \cellcolor{cyan!15}\textbf{0.2414} &
         \cellcolor{cyan!15}\textbf{0.4075} &
         \cellcolor{cyan!15}\textbf{0.5493} &
         \cellcolor{cyan!15}0.3152 &
         \cellcolor{cyan!15}\textbf{0.3801}\\
         \bottomrule
    \end{tabular}}}
    \vspace{-2mm}
    % \vskip -0.1in
\end{table*}


\begin{table*}[t]
    \begin{minipage}{0.63\linewidth}
        \vspace{-2mm}
        \caption{Quantitative results on GenEval \cite{geneval}.}
        \vskip 0.05in
        \label{tab:geneval}
        \centering
        \scriptsize
        \setlength{\tabcolsep}{1.1mm}{
        \scalebox{1.1}{
        \begin{tabular}{l l c c c c c c c}
             \toprule
             Model & Method & \makecell[c]{Single \\ Object} & \makecell[c]{Two \\ Object} & Counting & Colors & Position & \makecell[c]{Color \\ Attribution} & Overall \\
             \midrule
             \multirow{4}{*}{SD1.5} & Original & 97.50 & 37.12 & 34.69 & 75.53 & 3.75 & 6.75 & 42.56 \\
             & Diff.-DPO & \textbf{98.44} & 38.38 & 36.25 & 77.93 & 4.50 & 7.25 & 43.79 \\
             & SPO & 95.00 & 33.84 & 32.50 & 69.95 & 4.25 & 7.25 & 40.46 \\
             & \cellcolor{cyan!15}LPO & \cellcolor{cyan!15}97.81 &
             \cellcolor{cyan!15}\textbf{54.80}&
             \cellcolor{cyan!15}\textbf{40.94}&
             \cellcolor{cyan!15}\textbf{79.52}&
             \cellcolor{cyan!15}\textbf{7.00}& 
             \cellcolor{cyan!15}\textbf{10.25}&
             \cellcolor{cyan!15}\textbf{48.39}\\
             \midrule
             \multirow{5}{*}{SDXL} & Original & 93.75 & 63.38 & 30.94 & 80.05 & 9.25 & 19.00 & 49.40  \\
             & Diff.-DPO & 99.06 & 76.52 & \textbf{45.00} & 88.83 & 11.50 & 25.75 & 57.78 \\
             & MaPO & 95.63 & 68.94 & 32.19 & 83.51 & 11.50 & 17.75 & 51.59 \\
             & SPO & 94.38 & 69.44 & 31.88 & 81.65 & 10.25 & 15.50 & 50.52  \\
             & \cellcolor{cyan!15}LPO & \cellcolor{cyan!15}\textbf{99.69} &
             \cellcolor{cyan!15}\textbf{81.57} &
             \cellcolor{cyan!15}43.75 &
             \cellcolor{cyan!15}\textbf{89.10} &
             \cellcolor{cyan!15}\textbf{14.00} &
             \cellcolor{cyan!15}\textbf{27.50} & 
             \cellcolor{cyan!15}\textbf{59.27}\\
             \bottomrule
        \end{tabular}}}
        \vskip -0.1in
    \end{minipage}
    \hfill
    \begin{minipage}{0.35\linewidth}
        \vspace{-2mm}
        \caption{Comparisons of training speed.}
        \vskip 0.05in
        \label{tab:speed}
        \centering
        % \footnotesize
        \scriptsize
        \setlength{\tabcolsep}{1.1mm}{
        \scalebox{1.1}{
        \begin{tabular}{l c c c}
             \toprule
             Method & \makecell[c]{Reward \\ Modeling} & \makecell[c]{Preference \\ Optimization} & \makecell[c]{Total $\downarrow$ \\ (A100 h)} \\
             \midrule
             \textcolor{gray}{SD1.5} \\
             \hspace{1pt} Diff.-DPO & 0 & 240 & 240 \\
             \hspace{1pt} SPO & 32 & 48 & 80 \\
             \hspace{1pt} \cellcolor{cyan!15}LPO & \cellcolor{cyan!15}\textbf{15} & \cellcolor{cyan!15}\textbf{8} & \cellcolor{cyan!15}\textbf{23} \\
             \midrule
             \textcolor{gray}{SDXL} \\
             \hspace{1pt} Diff.-DPO & 0 & 2,560 & 2,560 \\
             \hspace{1pt} SPO & 116 & 118 & 234 \\
             \hspace{1pt} \cellcolor{cyan!15}LPO & \cellcolor{cyan!15}\textbf{52} & \cellcolor{cyan!15}\textbf{40} & \cellcolor{cyan!15}\textbf{92} \\
             \bottomrule
        \end{tabular}}}
        \vskip -0.1in
    \end{minipage}
    \vspace{-0.8mm}
\end{table*}

\begin{table}[t]
    \centering
    \vspace{-2mm}
    \caption{Heterogeneous optimization based on LRM-SD1.5. P-S and I-R denote the PickScore and ImageReward metrics.}
    \vskip 0.05in
    \label{tab:sd15_for_sd21}
    \scriptsize
    \setlength{\tabcolsep}{1.0mm}{
    \scalebox{1.0}{
    \begin{tabular}{c c c c c c c c}
         \toprule
         Model & Method & Aesthetic & GenEval & P-S & I-R & HPSv2 & HPSv2.1\\
         \midrule
         SD2.1 & Original & 5.673 & 48.59 & 20.92 & 0.3063 & 27.05 & 25.49 \\
         \tiny(Same VAE) & \cellcolor{cyan!15}LPO & \cellcolor{cyan!15}\textbf{5.969} & \cellcolor{cyan!15}\textbf{56.01}  & \cellcolor{cyan!15}\textbf{21.76} & \cellcolor{cyan!15}\textbf{0.7978} & \cellcolor{cyan!15}\textbf{28.05} & \cellcolor{cyan!15}\textbf{28.61} \\
         \midrule
         SDXL & Original & 5.920 & \textbf{49.40} & \textbf{21.65} & \textbf{0.4780} & 27.06 & 26.05\\
         \tiny(Diff. VAE) & \cellcolor{cyan!15}LPO & \cellcolor{cyan!15}\textbf{5.953} & \cellcolor{cyan!15}40.85 & \cellcolor{cyan!15}20.82 & \cellcolor{cyan!15}0.3919 & \cellcolor{cyan!15}\textbf{27.10} & \cellcolor{cyan!15}\textbf{26.69} \\
         \bottomrule
    \end{tabular}}}
    % \vspace{-2mm}
    \vskip -0.15in
\end{table}


\textbf{Baseline Methods.} We compare LPO with DDPO \cite{ddpo}, D3PO \cite{d3po}, Diffusion-DPO \cite{diffusion_dpo}, MaPO \cite{mapo}, and SPO \cite{spo}. These methods are trained on similar datasets, such as Pick-a-Pic v1 and v2, to ensure a fair comparison. Details are provided in \cref{sec:experimental_detail}.


\textbf{Evaluation Protocol.} We evaluate various diffusion models across three dimensions: general preference, aesthetic preference, and text-image alignment. The PickScore \cite{pickscore}, HPSv2 \cite{hpsv2}, HPSv2.1 \cite{hpsv2}, and ImageReward \cite{imagereward} are utilized to assess the general preference. The aesthetic preference is evaluated using the Aesthetic Score \cite{aesthetic}. Consistent with \cite{spo}, both general and aesthetic preferences are assessed on the validation unique split of Pick-a-Pic v1, which has 500 different prompts. For text-image alignment, we employ the GenEval \cite{geneval} and T2I-CompBench++ \cite{t2i_compbench} metrics. All images are generated using the DDIM scheduler with 20 steps. Additionally, to assess the correlations between the LRM and aesthetics as well as text-image alignment, we propose two corresponding metrics. Specifically, we calculate the score gaps $G_*,*\in\{A,C,L\}$ between winning and losing images, where $A$, $C$, $L$ represent Aesthetic, CLIP, and LRM. For LRM, the score is taken at $t=0$. Then the Pearson Correlation Coefficient \cite{pearson} between $G_L$ and $G_A$ is referred to as \textit{Aes-Corr} while that between $G_L$ and $G_C$ is termed \textit{CLIP-Corr}. They are evaluated on the validation unique and test unique splits of Pick-a-Pic v1.

\subsection{Main Results}


\textbf{Quantitative Comparison.} As indicated in Tab.\;\ref{tab:preferenece_eval}, Tab.\;\ref{tab:t2i_eval}, and Tab.\;\ref{tab:geneval}, Diffusion-DPO excels in enhancing the text-image alignment, while SPO focuses more on aesthetics. LPO outperforms both methods across three dimensions, achieving higher Aesthetic Scores and superior performance on T2I-CompBench++ and GenEval metrics, leading to improved general preference scores. The user study results indicate similar findings, as discussed in \cref{sec:add_exp}. Notably, the LPO-optimized SD1.5 even exhibits performance comparable to the original SDXL model across various metrics.  We further validate the effectiveness of \textit{heterogeneous optimization} in Tab.\;\ref{tab:sd15_for_sd21}. SD1.5 and SD2.1 \cite{sd1} share the same VAE encoder, but SD1.5 has a smaller text encoder. Remarkably, fine-tuning SD2.1 using LRM-1.5 still yields significant improvements across various aspects, demonstrating that a smaller and inferior diffusion model can effectively fine-tune a larger and more advanced model as long as they share the same VAE encoder. In contrast, applying LRM-1.5 for the LPO of SDXL is ineffective due to the distribution mismatch in their latent spaces, which arises from differences in their VAE encoders.

\textbf{Qualitative Comparison.} The qualitative comparisons of various methods are illustrated in Fig.\;\ref{fig:main_comparison} and Fig.\;\ref{fig:vis_15_1}-Fig.\;\ref{fig:vis_xl_4}. The images generated by Diffusion-DPO exhibit deficiencies in color and detail, whereas those produced by SPO demonstrate lower semantic relevance. Additionally, SPO's excessive focus on aesthetics may lead to an overabundance of details in some images, making them appear cluttered. In contrast, the images produced by LPO achieve a strong balance between text-image alignment and aesthetic quality, delivering a higher overall image quality.


\textbf{Training Efficiency Comparison.} LPO achieves significantly faster training speed. As shown in Tab.\;\ref{tab:speed}, considering the time required for both reward modeling and preference optimization, LPO requires only 23 A100 hours for SD1.5---just 1/10 of the training time needed for Diffusion-DPO and 1/3.5 of that for SPO. For SDXL, LPO's training time is reduced to 1/28 and 1/2.5 of that for Diffusion-DPO and SPO, respectively. This efficiency is primarily due to LPO performing reward modeling and preference optimization directly in the latent space, avoiding the additional computational overhead of converting to pixel space.

\subsection{Ablation Studies}
\label{sec:ablation_study}
If not specified, ablation experiments are conducted on SD1.5. Due to space limitations, we only use PickScore to reflect general preference in Tab.\;\ref{tab:ablation_data} and Tab.\;\ref{tab:ablation_lrm}.


\textbf{MPCF.} As shown in Tab.\;\ref{tab:ablation_data}, MPCF plays a critical role in LRM training. As discussed in Sec.\;\ref{sec:lrm_train}, the inconsistent preference issue makes training on the full dataset (wo MPCF) ineffective, since it hinders the LRM from adequately focusing on aesthetics or text-image alignment, resulting in inferior LPO performance. On the other hand, different filtering strategies can profoundly impact the preference patterns of both the LRM and LPO-optimized models. The first filtering strategy strictly requires that winning images score higher than losing images across all aspects. However, since the diffusion model lacks explicit text-image alignment pre-training like CLIP, it is prone to overfitting to the visual features of the images, as indicated by a higher Aes-Corr. This overfitting results in reduced attention to alignment, as reflected by lower CLIP-Corr and GenEval scores. The second and third strategies relax the aesthetic constraints to varying degrees. However, excessively lenient constraints (the 3rd strategy) may cause LRM to focus solely on text-image alignment while neglecting image quality, resulting in a negative Aes-Corr. In contrast, the second strategy balances these two aspects better, leading to the highest general preference scores.


\begin{table}[t]
    \centering
    \vspace{-2.5mm}
    \caption{Ablation results on MPCF of LRM's training data. The second strategy balances aesthetics and alignment better.}
    \vskip 0.05in
    \label{tab:ablation_data}
    \scriptsize
    \setlength{\tabcolsep}{1.0mm}{
    \scalebox{1.1}{
    \begin{tabular}{c c c c c c}
         \toprule
         \multirow{2}{*}{Strategy} & \multicolumn{2}{c}{LRM} & \multicolumn{3}{c}{LPO} \\
         \cmidrule(lr){2-3} \cmidrule(lr){4-6}
          & Aes-Corr & CLIP-Corr & Aesthetic & GenEval & PickScore \\
         \midrule
         wo MPCF & 0.1342 & 0.2274 & 5.772 & 45.66 & 21.49 \\
         1 & \textbf{0.4860} & 0.1011 & \textbf{6.390} & 45.77 & \underline{21.61} \\
         \rowcolor{cyan!15}2 & 0.1136 & 0.3588 & \underline{5.945} & \underline{48.39} & \textbf{21.69} \\
         3 & -0.1152 & \textbf{0.4480} & 5.750 & \textbf{48.62} & 21.47 \\
         \bottomrule
    \end{tabular}}}
    % \vspace{-2mm}
    \vskip -0.1in
\end{table}


\begin{table}[t]
    \centering
    \vspace{-2mm}
    \caption{Ablation results on the VFE module of LRM. Introducing VFE leads to better alignment and general preferences.}
    \vskip 0.05in
    \label{tab:ablation_lrm}
    \scriptsize
    \setlength{\tabcolsep}{1.0mm}{
    \scalebox{1.1}{
    \begin{tabular}{c c c c c c c }
         \toprule
         \multirow{2}{*}{VFE} & \multirow{2}{*}{$gs$} & \multicolumn{2}{c}{LRM} & \multicolumn{3}{c}{LPO} \\
         \cmidrule(lr){3-4} \cmidrule(lr){5-7}
          &  & Aes-Corr & CLIP-Corr & Aesthetic & GenEval & PickScore\\
         \midrule
         \xmark & 1.0 & \textbf{0.1712} & 0.3211 & \textbf{6.053} & 46.60 & 21.51  \\
         \cmark & 3.0 & 0.1233 & 0.3441 & 5.923 & 47.35 & 21.53 \\
         \rowcolor{cyan!15}\cmark & 7.5 & 0.1136 & 0.3588 & \underline{5.945} & \textbf{48.39} & \textbf{21.69}\\
         \cmark & 10.0 & 0.1063 & \textbf{0.3592} & 5.937 & \underline{48.13} & \underline{21.56}\\
         \bottomrule
    \end{tabular}}}
    % \vspace{-2mm}
    \vskip -0.1in
\end{table}


\begin{table}[t]
    \centering
    \vspace{-2.5mm}
    \caption{Ablation results on optimization timestep ranges in LPO.}
    \vskip 0.05in
    \label{tab:ablation_timestep}
    \scriptsize
    \setlength{\tabcolsep}{1.0mm}{
    \scalebox{1.1}{
    \begin{tabular}{c c c c c c c}
         \toprule
         Range of $t$ & Aesthetic & GenEval & P-S & I-R & HPSv2 & HPSv2.1 \\
         \midrule
         \texttt{[}0, 200\texttt{]} & 5.434 & 40.11 & 20.46 & -0.0987 & 26.25 & 23.61 \\
         \texttt{[}250, 450\texttt{]} & 5.527 & 43.00 & 20.76 & 0.1430 & 26.90 & 25.37 \\
         \texttt{[}500, 700\texttt{]} & 5.742 & 44.44 & 20.95 & 0.1591 & 26.71 & 25.16\\
         \texttt{[}750, 950\texttt{]} & \underline{5.853} & \underline{48.28} & 
         \underline{21.54} & \underline{0.6337} & \underline{27.47} & \underline{27.64} \\
         \midrule
         \texttt{[}0, 450\texttt{]} & 5.573 & 42.71 & 20.63 & 0.0204 & 26.69 & 24.88 \\
         \texttt{[}0, 700\texttt{]} & 5.765 & 44.93 & 21.02 & 0.3087 &  27.10 & 26.25\\
         \rowcolor{cyan!15}\texttt{[}0, 950\texttt{]} & \textbf{5.945} & \textbf{48.39} & \textbf{21.69} & \textbf{0.6588} & \textbf{27.64} & \textbf{27.86} \\
         \bottomrule
    \end{tabular}}}
    % \vspace{-2mm}
    \vskip -0.1in
\end{table}


\begin{table}[t]
    \centering
    \vspace{-2mm}
    \caption{Ablation results on different threshold strategies.}
    \vskip 0.05in
    \label{tab:ablation_threshold}
    \scriptsize
    \setlength{\tabcolsep}{1.0mm}{
    \scalebox{1.1}{
    \begin{tabular}{c c c c c c c }
         \toprule
          Threshold & Aesthetic & GenEval & P-S & I-R & HPSv2 & HPSv2.1\\
         \midrule
         0.3 & 5.853 & 46.75 & 21.22 & 0.5112  & 27.30 & 27.12 \\ 
         0.4 & 5.832 & 48.32 & 21.32 & 0.4789 & 27.08 & 26.37 \\
         0.5 & 5.900 & 48.39 & 21.57 & 0.6088 & 27.54 & \underline{27.42} \\
         0.6 & 5.877 & 47.97 & 21.35 & 0.5510 & 27.25 & 26.73 \\
         \midrule
         \texttt{[}0.3, 0.45\texttt{]} & \underline{5.916} & \textbf{49.43} & \underline{21.58} & \underline{0.6405} & \underline{27.55} & 27.33\\
         \rowcolor{cyan!15}\texttt{[}0.35, 0.5\texttt{]} & \textbf{5.945} & 48.39 & \textbf{21.69} & \textbf{0.6588} & \textbf{27.64} & \textbf{27.86} \\
         \texttt{[}0.4, 0.55\texttt{]} & 5.882 & \underline{48.77} & 21.48 & 0.4791 & 27.30 & 27.13\\
         \bottomrule
    \end{tabular}}}
    % \vspace{-2mm}
    \vskip -0.1in
\end{table}


\textbf{Structure of LRM.} As illustrated in Tab.\;\ref{tab:ablation_lrm}, the introduction of VFE ($gs>1$) leads to lower Aes-Corr values but higher CLIP-Corr values, indicating an enhanced emphasis on text-image alignment. This results in improvements in both the GenEval score and PickScore, with only a minor decline in the Aesthetic Score. As $gs$ increases, the LRM's correlation with alignment steadily improves, while its correlation with aesthetics decreases. When $gs$ is set to 7.5, the model achieves the best overall performance.

\textbf{Optimization Timesteps.} Tab.\;\ref{tab:ablation_timestep} ablates different optimization timestep ranges, indicating that larger timesteps lead to better performance. The results achieved within the range of $[750, 950]$ are nearly comparable to those achieved through optimization across the entire denoising process, \ie $[0,950]$. We suggest this is because diffusion models focus on low-frequency information, such as image layout and style, during larger timesteps, while emphasizing high-frequency texture details during smaller timesteps. The low-frequency components formed in higher timesteps play a decisive role in determining the overall quality of the generated images. This observation also demonstrates the effectiveness of LRM, even in very large timesteps. The qualitative comparison of different ranges is shown in Fig.\;\ref{fig:vis_timestep}.

\textbf{Dynamic Sampling Threshold.} The standard deviation $\sigma_t$ of samples at smaller timesteps is relatively small according to the DDPM scheduling \cite{ddpm}, making the constant threshold insufficient to accommodate all timesteps. As indicated in Tab.\;\ref{tab:ablation_threshold}, the dynamic threshold strategy generally outperforms the constant threshold across different intervals, effectively alleviating this problem. We further explore other dynamic strategies in \cref{sec:add_exp}.



\section{Discussion}

In this paper, we explored the relationship between human evaluations and NLP benchmarks of chat-finetuned language models (chat LMs). Our work is motivated by the recent shift towards human evaluations as the primary means of assessing chat LM performance, and the desire to determine the role that NLP benchmarks should play.

Through a large-scale study of the Chat Llama 2 model family on a diverse set of human and NLP evaluations, we demonstrated that NLP benchmarks are generally well-correlated with human judgments of chat LM quality. However, our analysis also reveals some notable exceptions to this overall trend. In particular, we find that adversarial and safety-focused evaluations, as well as language assistance and open question answering tasks, exhibit weaker or negative correlations respectively with NLP benchmarks. We also explored predicting human evaluation scores from NLP evaluation scores using overparameterized linear regression models. Our results suggest that NLP benchmarks can indeed be used to predict aggregate human preferences, although we caution that the limited sample size and the assumptions of our models may limit the generalizability of these findings. Our results suggest that NLP benchmarks can serve as fast and cheap proxies of slower and expensive human evaluations in assessing chat LMs.

Additionally, our work highlights the need for further research into NLP evaluations that can effectively capture important aspects of LM behavior, such as safety, robustness to adversarial inputs, and performance on complex, open-ended tasks. It is possible that new NLP benchmarks can provide signals on these topics, e.g., \citep{wang2023decodingtrust}. Of particular interest is developing human-interpretable and scaling-predictable evaluation processes, e.g., \citep{schaeffer2024emergent, ruan2024observational,schaeffer2024predictingdownstreamcapabilitiesfrontier}. Developing and refining such evaluation methods \citep{madaan2024quantifyingvarianceevaluationbenchmarks}, as well as detecting whether evaluations scores faithfully capture models' true performance \citep{oren2023proving,schaeffer2023pretrainingtestsetneed,roberts2023cutoff,jiang2024investigatingdatacontaminationpretraining,zhang2024careful,duan2024uncoveringlatentmemoriesassessing} will be crucial for ensuring that LMs are safe, reliable, and beneficial as they become increasingly integrated into society.

% In conclusion, our study provides insights into the relationship between human evaluations and NLP benchmarks of chat language models. By leveraging the complementary strengths of both human and NLP benchmarks, we can build a more complete understanding of LM capabilities and behaviors, ultimately enabling the development of models more capable, trustworthy, and beneficial to society.



\section*{Impact Statement}
\setlength{\itemsep}{5pt}
\setlength{\topsep}{0pt}
This work introduces a preference optimization method used for text-to-image diffusion models. The method may have the following impacts:
\begin{itemize}[left=0pt]
    \vspace{-1pt}
    \item The reward model plays a crucial role in shaping the preferences of diffusion models. However, if the training data for the reward model contains biases, the optimized model may inherit or even amplify these biases, leading to the generation of stereotypical or discriminatory content about certain groups. To mitigate this risk, it is essential to ensure that the training data is diverse, representative, and fair.
    \item This method offers a way to enhance the quality of generated images in terms of aesthetics, relevance, and other aspects by leveraging a well-designed reward model. It can also be applied to safety domains, optimizing the model to prevent the generation of negative content.
    \item The optimized model can generate highly realistic images, which may be used to create misinformation or misleading content, thereby impacting public opinion and social trust. Therefore, it is essential to develop effective detection tools and mechanisms to identify synthetic content.
\end{itemize}


\bibliography{example_paper}
\bibliographystyle{icml2025}


%%%%%%%%%%%%%%%%%%%%%%%%%%%%%%%%%%%%%%%%%%%%%%%%%%%%%%%%%%%%%%%%%%%%%%%%%%%%%%%
%%%%%%%%%%%%%%%%%%%%%%%%%%%%%%%%%%%%%%%%%%%%%%%%%%%%%%%%%%%%%%%%%%%%%%%%%%%%%%%
% APPENDIX
%%%%%%%%%%%%%%%%%%%%%%%%%%%%%%%%%%%%%%%%%%%%%%%%%%%%%%%%%%%%%%%%%%%%%%%%%%%%%%%
%%%%%%%%%%%%%%%%%%%%%%%%%%%%%%%%%%%%%%%%%%%%%%%%%%%%%%%%%%%%%%%%%%%%%%%%%%%%%%%
\newpage
\appendix
\onecolumn

\newpage
\appendix
\onecolumn

\renewcommand{\thetable}{A\arabic{table}} % Prefix table numbers with 'A'
\renewcommand{\thefigure}{A\arabic{figure}} % Prefix figure numbers with 'A'
\renewcommand{\theequation}{A\arabic{equation}} % Prefix equation numbers with 'A'

\setcounter{table}{0} % Reset table counter
\setcounter{figure}{0} % Reset figure counter
\setcounter{equation}{0} % Reset equation counter

\section*{Appendix}

\section{Optimal Brain Surgeon Derivation}
\label{OBS_ALGORITHM}

In the original setup in OBS, we have a local quadratic model for the loss $L$ given by:
$$
    \delta L = L(w + \delta w) \approx L(w) + \nabla_w L^T \delta w + \frac{1}{2} \delta w^T H \delta w
$$
Since OBS is a pruning-after-training approach, they discarded the 1-st order component. Reducing the expression for saliency as:
$$
    \delta L = \frac{1}{2} \delta w^T H \delta w
$$
To remove a single parameter, the authors of OBS introduced the constraint $e_q^T \delta w + w_q = 0$, with $e_q$ being the $q^{\text{th}}$ canonical basis vector. The pruning is defined as a constrained optimization problem of the form:
$$
    \min_{\delta w \in \mathbb{R^d}} \left( \frac{1}{2} \delta w^T H \delta w\right),
    ~~\text{s.t}~~
    e_q^T \delta w + w_q = 0.
$$
And the choice of which parameter to remove becomes:
$$
    \min_{q \in \mathcal{Q}} \left\{
        \min_{\delta w \in \mathbb{R^d}} \left( \frac{1}{2} \delta w^T H \delta w\right),
        ~~\text{s.t}~~
        e_q^T \delta w + w_q = 0
    \right\}.
$$
To solve the internal problem, we use a Lagrange multiplier $\lambda$ to write the problem as an unconstrained optimization case as follows:
$$
    \mathcal{L}(\delta w, \lambda) =
    \frac{1}{2} \delta w^T H \delta w +
    \lambda(e_q^T \delta w + w_q).
$$
Then, to find the stationary conditions, we compute the partial derivatives with respect to $\delta w$ and $\lambda$, and equate them to 0, obtaining:
$$
    \nabla_{\delta w} \mathcal{L} = 
    H \delta w + \lambda e_q = 0 
    \rightarrow
    \delta w = - \lambda H^{-1} e_q
$$
$$
    \nabla_{\lambda} \mathcal{L} =
    e_q^T \delta w + w_q = 0
    \rightarrow
    e_q^T \delta w = -w_q
$$
With some replacements, we get:
$$
    e_q^T \delta w = -w_q
    \rightarrow
    e_q^T \left( 
        - \lambda H^{-1} e_q
    \right) = -w_q
    \rightarrow
    - \lambda e_q^T H^{-1} e_q = -w_q
    \rightarrow
    \lambda = \frac{w_q}{e_q^T H^{-1} e_q} = \frac{w_q}{[H^{-1}]_{qq}}
$$
$$
    \delta w = - \frac{w_q H^{-1} e_q}{[H^{-1}]_{qq}}
$$
Replacing the expression for $\delta w$ in the saliency expression, we have:
\begin{align*}
    \delta L = \frac{1}{2} \delta w^T H \delta w
    &= \frac{1}{2}\left(
        - \frac{w_q H^{-1} e_q}{[H^{-1}]_{qq}}
    \right)^T
    H
    \left(
        - \frac{w_q H^{-1} e_q}{[H^{-1}]_{qq}}
    \right)
    \nonumber \\
    &= 
    \frac{w_q^2}{2[H^{-1}]_{qq}^2}
    \left(
        H^{-1} e_q
    \right)^T
    H
    \left(
        H^{-1} e_q
    \right)
    \nonumber \\
    &= 
    \frac{w_q^2}{2[H^{-1}]_{qq}^2}
    e_q ^T
    H^{-1}
    e_q
    = 
    \frac{w_q^2}{2[H^{-1}]_{qq}^2}
    [H^{-1}]_{qq}
    = 
    \frac{w_q^2}{2[H^{-1}]_{qq}}
    \nonumber \\
\end{align*}
%------------------------------------------------------------------------------------------------
\newpage
\section{Fisher Brain Surgeon Sensitivity Derivation}
\label{FBSS_ALGORITHM}
As we considered a PBT setting, it is not possible to ignore the first-order term in the local quadratic approximation of the error as it could still be informative. In this case, our model for sensitivity is given by: 
$$
    \delta L = \nabla_w L^T \delta w + \frac{1}{2} \delta w^T H \delta w
$$
The process to remove a single parameter remains similar; the constraint $e_q^T \delta w + w_q = 0$, with $e_q$ is still valid, redefining the optimization problem as:
$$
    \min_{\delta w \in \mathbb{R^d}} \left(
        \nabla_w L^T \delta w +  \frac{1}{2} \delta w^T H \delta w
    \right),
    ~~\text{s.t}~~
    e_q^T \delta w + w_q = 0.
$$
And the choice of which parameter to remove becomes:
$$
    \min_{q \in \mathcal{Q}} \left\{
        \min_{\delta w \in \mathbb{R^d}} \left(
            \nabla_w L^T \delta w + \frac{1}{2} \delta w^T H \delta w
        \right),
        ~~\text{s.t}~~
        e_q^T \delta w + w_q = 0
    \right\}.
$$
Using a Lagrange multiplier $\lambda$ as in the reference case, we solve the following unconstrained optimization problem:
$$
    \mathcal{L}(\delta w, \lambda) =
    \nabla_w L^T \delta w + 
    \frac{1}{2} \delta w^T H \delta w +
    \lambda(e_q^T \delta w + w_q).
$$
With the following stationary conditions:
$$
    \nabla_{\delta w} \mathcal{L} = 
    \nabla_w L + H \delta w + \lambda e_q = 0 
    \rightarrow
    \delta w = - (\lambda H^{-1}e_q + H^{-1} \nabla_w L)
$$
$$
    \nabla_{\lambda} \mathcal{L} =
    e_q^T \delta w + w_q = 0
    \rightarrow
    e_q^T \delta w = -w_q
$$
The expression for $\lambda$ is redefined as follows:
\begin{align*}
    e_q^T \left(
        - (\lambda H^{-1}e_q + H^{-1} \nabla_w L)
    \right) 
    &= -w_q
    \nonumber \\
    \lambda e_q^T H^{-1} e_q + e_q^T H^{-1} \nabla_w L
    &= w_q
    \nonumber \\
    \lambda [H^{-1}]_{qq} 
    &= w_q - e_q^T H^{-1} \nabla_w L
    \nonumber \\
    \lambda
    &= \frac{w_q - e_q^T H^{-1} \nabla_w L}{[H^{-1}]_{qq}}
\end{align*}
Replacing the expression for $\delta w$ in our sensitivity expression, we have:
\begin{align*}
    \delta L = \nabla_w L^T \delta w + \frac{1}{2} \delta w^T H \delta w
    &= 
    \nabla_w L^T \left[
        - (\lambda H^{-1}e_q + H^{-1} \nabla_w L)
    \right]
    \nonumber \\
    &+
    \frac{1}{2}\left[
        - (\lambda H^{-1}e_q + H^{-1} \nabla_w L)
    \right]^T
    H
    \left[
        - (\lambda H^{-1}e_q + H^{-1} \nabla_w L)
    \right]
    \nonumber \\
    &= 
    - \lambda \nabla_w L^T H^{-1}e_q - \nabla_w L^T H^{-1} \nabla_w L
    \nonumber \\
    &+
    \frac{1}{2}\left[
        (\lambda H^{-1}e_q)^T + (H^{-1} \nabla_w L)^T
    \right]
    \left[
        \lambda H H^{-1}e_q + H H^{-1} \nabla_w L)
    \right]
    \nonumber \\
    &= 
    - \lambda \nabla_w L^T H^{-1}e_q - \nabla_w L^T H^{-1} \nabla_w L
    \nonumber \\
    &+
    \frac{1}{2}\left[
        (\lambda H^{-1}e_q)^T + (H^{-1} \nabla_w L)^T
    \right]
    \left[
        \lambda e_q + \nabla_w L
    \right]
    \nonumber \\
    &= 
    - \lambda \nabla_w L^T H^{-1}e_q - \nabla_w L^T H^{-1} \nabla_w L
    \nonumber \\
    &+
    \frac{1}{2}\left[
        (\lambda H^{-1}e_q)^T \lambda e_q
        + (H^{-1} \nabla_w L)^T \lambda e_q
        + (\lambda H^{-1}e_q)^T \nabla_w L
        + (H^{-1} \nabla_w L)^T \nabla_w L
    \right]
    \nonumber \\
    &= 
    - \lambda \nabla_w L^T H^{-1}e_q - \nabla_w L^T H^{-1} \nabla_w L
    \nonumber \\
    &+
    \frac{1}{2}\left[
        \lambda^2 e_q^T H^{-1} e_q
        + \lambda \nabla_w L^T H^{-1} e_q
        + \lambda e_q^T H^{-1} \nabla_w L
        + \nabla_w L^T H^{-1} \nabla_w L
    \right]
    \nonumber \\
    &= 
    \frac{1}{2}\left[
        \lambda^2 [H^{-1}]_{qq}
        - \lambda \nabla_w L^T H^{-1} e_q
        + \lambda e_q^T H^{-1} \nabla_w L
        - \nabla_w L^T H^{-1} \nabla_w L
    \right]
    \nonumber \\
\end{align*}
Finally, replacing the $\lambda$:
\begin{align*}
    \delta L 
    &= 
    \frac{1}{2}\left[
        \lambda^2 [H^{-1}]_{qq}
        - \lambda \nabla_w L^T H^{-1} e_q
        + \lambda e_q^T H^{-1} \nabla_w L
        - \nabla_w L^T H^{-1} \nabla_w L
    \right]
    \nonumber \\
    &= 
    \frac{1}{2[H^{-1}]_{qq}}\left[
        (w_q - e_q^T H^{-1} \nabla_w L)^2 
        + (w_q - e_q^T H^{-1} \nabla_w L)(e_q^T H^{-1} \nabla_w L - \nabla_w L^T H^{-1} e_q)
        - \nabla_w L^T H^{-1} \nabla_w L
    \right]
    \nonumber \\
    &= 
    \frac{1}{2[H^{-1}]_{qq}}[
        w_q^2
        - 2 w_q (e_q^T H^{-1} \nabla_w L)
        + (e_q^T H^{-1} \nabla_w L)^2
        + w_q (e_q^T H^{-1} \nabla_w L)
    \nonumber \\
        &- w_q (\nabla_w L^T H^{-1} e_q)
        - (e_q^T H^{-1} \nabla_w L)(e_q^T H^{-1} \nabla_w L)
        + (e_q^T H^{-1} \nabla_w L)(\nabla_w L^T H^{-1} e_q)
        - \nabla_w L^T H^{-1} \nabla_w L
    ]
    \nonumber \\
    &= 
    \frac{1}{2[H^{-1}]_{qq}}[
        w_q^2
        - w_q (e_q^T H^{-1} \nabla_w L)
        + (e_q^T H^{-1} \nabla_w L)^2
    \nonumber \\
        &- w_q (\nabla_w L^T H^{-1} e_q)
        - (e_q^T H^{-1} \nabla_w L)^2
        + (e_q^T H^{-1} \nabla_w L)(\nabla_w L^T H^{-1} e_q)
        - \nabla_w L^T H^{-1} \nabla_w L
    ]
    \nonumber \\
    &= 
    \frac{1}{2[H^{-1}]_{qq}}\left[
        w_q^2
        - 2 w_q (e_q^T H^{-1} \nabla_w L)
        + (e_q^T H^{-1} \nabla_w L)^2
        - \nabla_w L^T H^{-1} \nabla_w L
    \right]
    \nonumber \\
    &= 
    \frac{1}{2[\hat{F}^{-1}]_{qq}}
    \left[
        w_q - (e_q^T \hat{F}^{-1} \nabla \mathcal{L}(w_0))
    \right]^2
\end{align*}

%------------------------------------------------------------------------------------------------

\newpage
\section{Training and Testing Details}
\label{appendix:training_parameters}

We perform an 80:20 stratified split, with a constant seed, on the CIFAR10/100 training dataset to obtain a validation set with the same class distribution. For both datasets, we have a training set with 40,000 samples, a validation set with 10,000 samples, and a testing set of 10,000 samples. Validation is performed after each training step, and the weights of the best-performing validation step (based on top-1 accuracy) are utilized for the final evaluation on the testing set. Table \ref{tab:table_training_parameters} summarizes the training parameters.

\begin{table}[h]
\caption{Training parameters used for ResNet18 and VGG19 on the CIFAR-10/100 datasets.}
\label{tab:table_training_parameters}
\vskip 0.15in
\begin{center}
\begin{small}
\begin{sc}
\begin{tabular}{lcc}
\toprule
Parameter & ResNet18 & VGG19 \\
\midrule
Number of steps       & 160 & 160 \\
Criterion             & CE & CE \\
Optimizer             & SGD & SGD \\
Learning rate         & 0.01 & 0.1 \\
Momentum              & 0.9 & 0.9 \\
Weight decay          & $5 \times 10^{-4}$ & $1 \times 10^{-4}$ \\
Learning rate drops   & [60, 120] & [60, 120] \\
Learning rate drop factor & 0.2 & 0.1 \\
\bottomrule
\end{tabular}
\end{sc}
\end{small}
\end{center}
\vskip -0.1in
\end{table}

%------------------------------------------------------------------------------------------------

\newpage
\section{Results CIFAR10}
\subsection{ResNet18}
\label{appendix:CIFAR10_ResNet18}

\begin{table}[h]
\caption{Performance of different sensitivity methods for pruning evaluated using ResNet18 on the CIFAR-10 testset. The right side of the table presents our proposed criteria. The mean accuracy and standard deviation are reported across three initialization seeds for various sparsity levels. Baseline, no pruning: $91.78 \pm 0.09$.}
\label{tab:resnet18_cifar10_compressors}
\vskip 0.15in
\begin{center}
\begin{small}
\begin{sc}
\resizebox{\textwidth}{!}{%
\begin{tabular}{lccccc|cccc}
\toprule
Sparsity  & Random & Magnitude & GN & SNIP & GraSP & FD & FP & FTS & FBSS \\
\midrule
0.10  & 91.71 ± 0.21 & 91.72 ± 0.07 & 91.57 ± 0.15 & 91.72 ± 0.07 & 89.16 ± 0.05 & 91.87 ± 0.13 & 91.63 ± 0.21 & 91.53 ± 0.12 & 91.76 ± 0.08 \\
0.20  & 91.63 ± 0.11 & 91.42 ± 0.12 & 91.51 ± 0.09 & 91.64 ± 0.16 & 88.69 ± 0.34 & 91.50 ± 0.12 & 91.65 ± 0.14 & 91.53 ± 0.15 & 91.54 ± 0.13 \\
0.30  & 91.45 ± 0.18 & 91.61 ± 0.13 & 91.68 ± 0.20 & 91.65 ± 0.08 & 88.67 ± 0.26 & 91.65 ± 0.18 & 91.44 ± 0.27 & 91.49 ± 0.05 & 91.62 ± 0.07 \\
0.40  & 91.59 ± 0.18 & 91.06 ± 0.16 & 91.61 ± 0.09 & 91.55 ± 0.08 & 88.24 ± 0.33 & 91.51 ± 0.05 & 91.38 ± 0.13 & 91.56 ± 0.28 & 91.39 ± 0.05 \\
0.50  & 91.60 ± 0.06 & 91.32 ± 0.13 & 91.44 ± 0.13 & 91.22 ± 0.07 & 87.69 ± 0.15 & 91.30 ± 0.18 & 91.58 ± 0.16 & 91.46 ± 0.19 & 91.41 ± 0.05 \\
0.60  & 91.10 ± 0.16 & 91.18 ± 0.16 & 91.59 ± 0.13 & 91.24 ± 0.04 & 87.48 ± 0.55 & 91.34 ± 0.07 & 91.35 ± 0.16 & 91.40 ± 0.11 & 91.38 ± 0.18 \\
0.70  & 91.17 ± 0.04 & 91.07 ± 0.07 & 91.19 ± 0.17 & 91.33 ± 0.18 & 87.26 ± 0.34 & 91.34 ± 0.23 & 91.42 ± 0.23 & 91.18 ± 0.18 & 91.27 ± 0.14 \\
0.80  & 90.78 ± 0.08 & 91.10 ± 0.12 & 90.95 ± 0.35 & 90.74 ± 0.10 & 87.18 ± 0.51 & 90.95 ± 0.11 & 91.08 ± 0.06 & 90.94 ± 0.22 & 90.73 ± 0.33 \\
0.90  & 89.35 ± 0.13 & 89.88 ± 0.28 & 90.39 ± 0.23 & 90.36 ± 0.34 & 86.60 ± 0.51 & 90.04 ± 0.21 & 90.20 ± 0.08 & 90.55 ± 0.23 & 89.22 ± 0.30 \\
0.95  & 87.59 ± 0.11 & 89.23 ± 0.19 & 89.00 ± 0.05 & 89.31 ± 0.17 & 86.50 ± 0.05 & 88.61 ± 0.28 & 89.50 ± 0.18 & 89.47 ± 0.32 & 87.58 ± 0.25 \\
0.98  & 83.47 ± 0.20 & 85.70 ± 0.33 & 86.43 ± 0.05 & 87.26 ± 0.28 & 85.99 ± 0.08 & 85.61 ± 0.20 & 86.97 ± 0.22 & 87.24 ± 0.32 & 83.40 ± 0.74 \\
0.99  & 78.28 ± 0.45 & 71.99 ± 0.28 & 83.47 ± 0.15 & 84.54 ± 0.04 & 84.56 ± 0.46 & 82.13 ± 0.28 & 83.74 ± 0.48 & 84.85 ± 0.18 & 77.60 ± 1.02 \\
\bottomrule
\end{tabular}}
\end{sc}
\end{small}
\end{center}
\vskip -0.1in
\end{table}

%------------------------------------------------------------------------------------------------
\clearpage
\subsection{VGG19}
\label{appendix:CIFAR10_VGG19}

As discussed earlier, introducing a warm-up phase effectively mitigates layer collapse in data-dependent pruning methods. Here, we evaluate the impact of different warm-up durations by comparing no warm-up, a single warm-up epoch, and five warm-up epochs. Table \ref{tab:VGG19_cifar10_compressors} demonstrates how performance drastically degrades with increasing sparsity, ultimately leading to layer collapse at 0.90 sparsity. However, as shown in the results, a single warm-up epoch is sufficient to prevent collapse and stabilize pruning performance. Moreover, as seen in Table \ref{tab:VGG19_cifar10_compressors_warmup5}, increasing the warm-up period to five epochs provides no substantial additional improvement. This indicates that prolonged warm-up training is not necessary; a single training step is enough to achieve gradient stabilization and overcome layer collapse.

\begin{table}[h]
\caption{Performance of different sensitivity methods for pruning evaluated using VGG19 on the CIFAR-10 test set. The right side of the table presents our proposed criteria. The mean accuracy and standard deviation are reported across three initialization seeds for various sparsity levels. Baseline, no pruning: $89.21 \pm 0.22$.}
\label{tab:VGG19_cifar10_compressors}
\vskip 0.15in
\begin{center}
\begin{small}
\begin{sc}
\resizebox{\textwidth}{!}{%
\begin{tabular}{lccccc|cccc}
\toprule
Sparsity  & Random & Magnitude & GN & SNIP & GraSP & FD & FP & FTS & FBSS \\
\midrule
0.10  & 88.40 ± 0.95 & 89.12 ± 0.55 & 90.14 ± 0.10 & 90.16 ± 0.18 & 87.81 ± 1.66 & 90.20 ± 0.29 & 90.21 ± 0.37 & 90.25 ± 0.38 & 89.06 ± 0.75 \\
0.20  & 89.19 ± 0.22 & 89.65 ± 0.60 & 89.59 ± 0.69 & 90.06 ± 0.04 & 89.57 ± 0.34 & 89.91 ± 0.28 & 90.28 ± 0.55 & 89.80 ± 0.28 & 88.89 ± 0.76 \\
0.30  & 88.93 ± 0.83 & 88.77 ± 1.07 & 90.23 ± 0.09 & 89.88 ± 0.59 & 89.14 ± 0.19 & 90.25 ± 0.09 & 89.97 ± 0.26 & 90.46 ± 0.41 & 89.06 ± 0.36 \\
0.40  & 88.28 ± 1.08 & 89.38 ± 0.53 & 90.50 ± 0.23 & 89.79 ± 0.67 & 88.20 ± 0.31 & 90.51 ± 0.12 & 90.37 ± 0.24 & 90.23 ± 0.14 & 10.00 ± 0.00 \\
0.50  & 88.96 ± 0.82 & 89.03 ± 0.59 & 90.46 ± 0.60 & 90.38 ± 0.25 & 88.67 ± 0.23 & 89.54 ± 0.86 & 90.47 ± 0.52 & 90.19 ± 0.31 & 10.00 ± 0.00 \\
0.60  & 88.15 ± 0.68 & 89.47 ± 0.18 & 89.95 ± 0.30 & 90.32 ± 0.25 & 88.82 ± 0.32 & 90.02 ± 0.40 & 90.18 ± 0.33 & 90.14 ± 0.36 & 10.00 ± 0.00 \\
0.70  & 88.02 ± 0.53 & 89.63 ± 0.44 & 89.69 ± 0.42 & 89.23 ± 0.19 & 89.62 ± 0.81 & 89.85 ± 0.08 & 90.01 ± 0.34 & 10.00 ± 0.00 & 10.00 ± 0.00 \\
0.80  & 88.28 ± 0.34 & 89.62 ± 0.91 & 85.72 ± 0.63 & 89.39 ± 0.43 & 88.82 ± 0.14 & 10.00 ± 0.00 & 88.29 ± 0.11 & 10.00 ± 0.00 & 10.00 ± 0.00 \\
0.90  & 85.82 ± 0.19 & 89.29 ± 0.79 & 10.00 ± 0.00 & 80.85 ± 0.62 & 24.28 ± 20.2 & 10.00 ± 0.00 & 10.00 ± 0.00 & 10.00 ± 0.00 & 10.00 ± 0.00 \\
0.95  & 84.41 ± 0.05 & 10.00 ± 0.00 & 10.00 ± 0.00 & 10.00 ± 0.00 & 10.00 ± 0.00 & 10.00 ± 0.00 & 10.00 ± 0.00 & 10.00 ± 0.00 & 10.00 ± 0.00 \\
0.98  & 80.04 ± 0.90 & 10.00 ± 0.00 & 10.00 ± 0.00 & 10.00 ± 0.00 & 10.00 ± 0.00 & 10.00 ± 0.00 & 10.00 ± 0.00 & 10.00 ± 0.00 & 10.00 ± 0.00 \\
0.99  & 76.89 ± 0.26 & 10.00 ± 0.00 & 10.00 ± 0.00 & 10.00 ± 0.00 & 10.00 ± 0.00 & 10.00 ± 0.00 & 10.00 ± 0.00 & 10.00 ± 0.00 & 10.00 ± 0.00 \\
\bottomrule
\end{tabular}}
\end{sc}
\end{small}
\end{center}
\vskip -0.1in
\end{table}
\newpage
%------------------------------------------------------------------------------------------------
\begin{table*}[h]
\caption{Performance of different compression methods evaluated after 1 warmup epoch using VGG19 on the CIFAR-10 dataset. We report the mean accuracy between three initialization seeds across various sparsity levels. Baseline, no pruning: $89.21 \pm 0.22$.}
\label{tab:VGG19_cifar10_compressors_warmup1}
\vskip 0.15in
\begin{center}
\begin{small}
\begin{sc}
\resizebox{\textwidth}{!}{%
\begin{tabular}{lccccc|cccc}
\toprule
Sparsity  & Random & Magnitude & GN & SNIP & GraSP & FD & FP & FTS & FBSS \\
\midrule
0.80  & 88.73 ± 0.38 & 88.35 ± 0.54 & 86.76 ± 0.27 & 87.39 ± 0.66 & 87.24 ± 0.25 & 87.14 ± 0.45 & 87.00 ± 0.87 & 87.68 ± 0.33 & 64.33 ± 15.91 \\
0.90  & 87.26 ± 0.42 & 88.62 ± 0.49 & 85.96 ± 0.75 & 86.75 ± 0.76 & 87.47 ± 0.33 & 86.69 ± 0.72 & 87.09 ± 0.31 & 87.42 ± 0.21 & 46.16 ± 7.62 \\
0.95  & 85.47 ± 0.64 & 87.68 ± 0.49 & 86.66 ± 0.27 & 86.00 ± 1.10 & 86.71 ± 1.24 & 85.71 ± 1.35 & 86.73 ± 0.36 & 87.56 ± 0.62 & 46.30 ± 5.32 \\
0.98  & 80.44 ± 0.30 & 86.61 ± 0.62 & 84.72 ± 1.69 & 87.22 ± 0.23 & 86.45 ± 0.64 & 80.34 ± 6.43 & 86.07 ± 0.39 & 86.36 ± 0.29 & 49.05 ± 4.31 \\
0.99  & 77.24 ± 0.73 & 83.69 ± 1.36 & 80.28 ± 2.04 & 83.49 ± 1.77 & 85.39 ± 0.43 & 75.11 ± 7.80 & 84.40 ± 1.27 & 85.35 ± 1.05 & 47.10 ± 4.41 \\
\bottomrule
\end{tabular}}
\end{sc}
\end{small}
\end{center}
\vskip -0.1in
\end{table*} 
%------------------------------------------------------------------------------------------------

\begin{table}[h]
\caption{Performance of different sensitivity methods for pruning evaluated after 5 warmup epochs using VGG19 on the CIFAR-10 testset. The right side of the table presents our proposed criteria. The mean accuracy and standard deviation are reported across three initialization seeds for various sparsity levels. Baseline, no pruning: $89.21 \pm 0.22$.}
\label{tab:VGG19_cifar10_compressors_warmup5}
\vskip 0.15in
\begin{center}
\begin{small}
\begin{sc}
\resizebox{\textwidth}{!}{%
\begin{tabular}{lccccc|cccc}
\toprule
Sparsity  & Random & Magnitude & GN & SNIP & GraSP & FD & FP & FTS & FBSS \\
\midrule
0.80  & 88.84 ± 0.43 & 88.41 ± 0.47 & 87.58 ± 0.52 & 88.15 ± 1.09 & 86.77 ± 1.14 & 87.28 ± 0.90 & 88.22 ± 0.82 & 86.68 ± 0.61 & 70.52 ± 9.25 \\
0.90  & 87.56 ± 0.62 & 88.60 ± 0.93 & 86.73 ± 0.37 & 87.89 ± 0.25 & 87.10 ± 0.47 & 87.50 ± 1.42 & 88.18 ± 0.47 & 86.98 ± 0.14 & 47.78 ± 1.26 \\
0.95 & 85.51 ± 0.69 & 87.66 ± 1.19 & 87.44 ± 0.46 & 87.71 ± 0.82 & 87.05 ± 0.16 & 86.83 ± 1.47 & 87.36 ± 0.52 & 87.00 ± 0.74 & 48.83 ± 2.52 \\
0.98 & 82.09 ± 0.17 & 86.24 ± 0.52 & 84.66 ± 1.33 & 86.55 ± 0.84 & 86.04 ± 0.66 & 85.44 ± 0.64 & 86.64 ± 0.13 & 84.89 ± 0.51 & 49.48 ± 0.85 \\
0.99 & 77.22 ± 1.03 & 83.93 ± 1.80 & 81.62 ± 2.17 & 84.53 ± 0.70 & 81.33 ± 5.77 & 81.71 ± 1.41 & 85.02 ± 0.69 & 83.78 ± 0.80 & 41.24 ± 1.55 \\
\bottomrule
\end{tabular}}
\end{sc}
\end{small}
\end{center}
\vskip -0.1in
\end{table}

%------------------------------------------------------------------------------------------------

\newpage
\section{Results CIFAR100}
\subsection{ResNet18}
\label{sec:resnet_cifar-100}

CIFAR-100 results exhibit a similar trend to those observed on CIFAR-10, further reinforcing the robustness of our proposed Fisher-Taylor Sensitivity (FTS) criterion. Across all evaluated sparsity levels, FTS consistently maintains strong performance, frequently ranking among the top-performing methods. This trend is particularly evident at extreme sparsities, where many pruning approaches suffer significant performance degradation. The stability of FTS across both datasets highlights its effectiveness in preserving network expressivity despite aggressive pruning.

\begin{table}[h]
\caption{Performance of different compression methods evaluated using ResNet18 on the CIFAR-100 dataset. We report the mean accuracy between three initialization seeds across various sparsity levels. Baseline, no pruning: $69.57 \pm 0.19$.}
\label{tab:resnet18_cifar100_compressors}
\vskip 0.15in
\begin{center}
\begin{small}
\begin{sc}
\resizebox{\textwidth}{!}{%
\begin{tabular}{lccccc|cccc}
\toprule
Sparsity  & Random & Magnitude & GN & SNIP & GraSP & FD & FP & FTS & FBSS \\
\midrule
0.10  & 69.16 ± 0.11 & 69.37 ± 0.14 & 69.63 ± 0.34 & 69.42 ± 0.07 & 64.26 ± 0.27 & 69.66 ± 0.30 & 69.08 ± 0.21 & 69.16 ± 0.11 & 69.07 ± 0.10 \\
0.20  & 69.16 ± 0.30 & 69.06 ± 0.24 & 69.19 ± 0.11 & 69.30 ± 0.08 & 63.28 ± 0.58 & 69.60 ± 0.30 & 69.35 ± 0.35 & 69.41 ± 0.43 & 69.07 ± 0.20 \\
0.30  & 69.36 ± 0.18 & 68.58 ± 0.36 & 69.37 ± 0.13 & 68.82 ± 0.17 & 62.02 ± 0.43 & 69.24 ± 0.40 & 68.84 ± 0.13 & 68.80 ± 0.55 & 68.96 ± 0.11 \\
0.40  & 69.41 ± 0.20 & 68.50 ± 0.29 & 69.16 ± 0.26 & 68.95 ± 0.19 & 61.18 ± 0.19 & 69.17 ± 0.16 & 68.88 ± 0.25 & 69.02 ± 0.21 & 68.92 ± 0.25 \\
0.50  & 69.12 ± 0.46 & 68.17 ± 0.20 & 68.94 ± 0.20 & 68.63 ± 0.11 & 61.11 ± 0.40 & 69.13 ± 0.13 & 68.68 ± 0.12 & 68.71 ± 0.12 & 68.71 ± 0.57 \\
0.60  & 68.66 ± 0.27 & 67.78 ± 0.35 & 68.77 ± 0.17 & 68.63 ± 0.42 & 61.40 ± 0.78 & 68.34 ± 0.43 & 67.98 ± 0.23 & 68.41 ± 0.14 & 68.60 ± 0.15 \\
0.70  & 67.95 ± 0.43 & 67.51 ± 0.24 & 68.29 ± 0.39 & 68.08 ± 0.18 & 59.43 ± 0.76 & 68.03 ± 0.46 & 67.96 ± 0.15 & 68.29 ± 0.06 & 68.16 ± 0.07 \\
0.80  & 67.26 ± 0.48 & 66.55 ± 0.19 & 67.20 ± 0.37 & 67.21 ± 0.38 & 59.08 ± 0.22 & 66.70 ± 0.05 & 67.05 ± 0.06 & 66.77 ± 0.65 & 66.62 ± 0.43 \\
0.90  & 64.75 ± 0.16 & 64.48 ± 0.18 & 64.87 ± 0.27 & 65.70 ± 0.08 & 59.16 ± 0.91 & 64.74 ± 0.44 & 65.46 ± 0.30 & 65.41 ± 0.13 & 63.90 ± 0.31 \\
0.95  & 61.01 ± 0.32 & 62.20 ± 0.06 & 62.20 ± 0.23 & 63.20 ± 0.20 & 57.91 ± 0.09 & 62.14 ± 0.42 & 63.22 ± 0.25 & 63.21 ± 0.47 & 61.25 ± 0.44 \\
0.98  & 54.72 ± 0.22 & 55.44 ± 0.18 & 57.34 ± 0.31 & 58.83 ± 0.35 & 54.85 ± 0.35 & 55.57 ± 0.17 & 58.05 ± 0.18 & 58.59 ± 0.12 & 55.02 ± 0.34 \\
0.99  & 45.62 ± 0.55 & 40.39 ± 0.36 & 50.46 ± 0.61 & 52.96 ± 0.10 & 49.13 ± 0.19 & 48.02 ± 0.32 & 49.98 ± 0.60 & 52.85 ± 0.24 & 44.91 ± 0.52 \\
\bottomrule
\end{tabular}}
\end{sc}
\end{small}
\end{center}
\vskip -0.1in
\end{table}

%------------------------------------------------------------------------------------------------
\clearpage
\subsection{VGG19}
The results on VGG19 with CIFAR-100 exhibit a similar trend to those observed on CIFAR-10, reinforcing the effectiveness of our proposed approach. Once again, we identify the occurrence of layer collapse at extreme sparsities when no warm-up is applied, leading to a significant drop in accuracy. Introducing a single warm-up epoch effectively resolves this issue, restoring pruning performance across all evaluated criteria. However, increasing the warm-up phase to five epochs does not yield any additional advantage, indicating that a brief warm-up period is sufficient to stabilize gradient-based importance scores and prevent collapse.

\label{sec:vgg_cifar-100}

\begin{table}[h]
\caption{Performance of different compression methods evaluated using VGG19 on the CIFAR-100 dataset. We report the mean accuracy between three initialization seeds across various sparsity levels. Baseline, no pruning: $58.96 \pm 2.30$.}
\label{tab:VGG19_cifar100_compressors}
\vskip 0.15in
\begin{center}
\begin{small}
\begin{sc}
\resizebox{\textwidth}{!}{%
\begin{tabular}{lccccc|cccc}
\toprule
Sparsity & Random & Magnitude & GN & SNIP & GraSP & FD & FP & FTS & FBSS \\
\midrule
0.10  & 60.31 ± 0.40 & 59.13 ± 1.29 & 61.93 ± 0.48 & 61.98 ± 0.29 & 59.32 ± 0.63 & 62.13 ± 0.61 & 60.45 ± 3.47 & 61.56 ± 1.04 & 58.79 ± 0.98 \\
0.20  & 60.43 ± 1.14 & 59.27 ± 0.34 & 62.64 ± 0.21 & 62.68 ± 0.24 & 61.21 ± 0.41 & 63.04 ± 0.43 & 62.71 ± 1.02 & 62.24 ± 0.44 & 60.48 ± 0.48 \\
0.30  & 58.32 ± 0.60 & 59.35 ± 1.43 & 62.61 ± 0.23 & 63.11 ± 0.35 & 59.30 ± 0.43 & 62.85 ± 0.42 & 61.43 ± 0.61 & 62.65 ± 0.54 & 58.77 ± 1.02 \\
0.40  & 56.50 ± 3.20 & 60.04 ± 1.02 & 62.36 ± 0.02 & 62.39 ± 0.55 & 56.34 ± 1.49 & 62.38 ± 0.75 & 61.56 ± 1.25 & 62.67 ± 0.06 & 1.00 ± 0.00 \\
0.50  & 58.47 ± 1.49 & 61.49 ± 1.22 & 62.02 ± 0.64 & 62.76 ± 0.50 & 54.43 ± 0.84 & 62.84 ± 0.33 & 62.25 ± 0.33 & 62.47 ± 0.42 & 1.00 ± 0.00 \\
0.60  & 57.54 ± 0.74 & 61.50 ± 0.30 & 62.55 ± 0.13 & 63.08 ± 0.55 & 56.76 ± 0.69 & 62.40 ± 0.57 & 62.70 ± 0.63 & 62.17 ± 0.23 & 1.00 ± 0.00 \\
0.70  & 57.63 ± 0.80 & 61.71 ± 0.25 & 60.85 ± 0.79 & 60.58 ± 0.39 & 57.76 ± 0.84 & 60.44 ± 0.34 & 60.92 ± 0.41 & 60.51 ± 1.67 & 1.00 ± 0.00 \\
0.80  & 57.84 ± 0.57 & 61.89 ± 1.02 & 55.09 ± 0.49 & 59.84 ± 0.29 & 58.39 ± 0.74 & 1.00 ± 0.00 & 43.16 ± 1.02 & 58.66 ± 2.28 & 1.00 ± 0.00 \\
0.90  & 58.41 ± 0.41 & 62.60 ± 0.91 & 1.00 ± 0.00 & 8.35 ± 10.39 & 42.88 ± 1.64 & 1.00 ± 0.00 & 1.00 ± 0.00 & 8.87 ± 11.13 & 1.00 ± 0.00 \\
0.95  & 54.84 ± 1.08 & 1.00 ± 0.00 & 1.00 ± 0.00 & 1.00 ± 0.00 & 1.00 ± 0.00 & 1.00 ± 0.00 & 1.00 ± 0.00 & 1.00 ± 0.00 & 1.00 ± 0.00 \\
0.98  & 50.21 ± 0.72 & 1.00 ± 0.00 & 1.00 ± 0.00 & 1.00 ± 0.00 & 1.00 ± 0.00 & 1.00 ± 0.00 & 1.00 ± 0.00 & 1.00 ± 0.00 & 1.00 ± 0.00 \\
0.99  & 46.69 ± 0.45 & 1.00 ± 0.00 & 1.00 ± 0.00 & 1.00 ± 0.00 & 1.00 ± 0.00 & 1.00 ± 0.00 & 1.00 ± 0.00 & 1.00 ± 0.00 & 1.00 ± 0.00 \\
\bottomrule
\end{tabular}}
\end{sc}
\end{small}
\end{center}
\vskip -0.1in
\end{table}

%------------------------------------------------------------------------------------------------

\begin{table}[h]
\caption{Performance of different compression methods evaluated after 1 warmup epoch using VGG19 on the CIFAR-100 dataset. We report the mean accuracy between three initialization seeds across various sparsity levels. Baseline, no pruning: $58.96 \pm 2.30$.}
\label{tab:VGG19_cifar100_compressors_warmup1}
\vskip 0.15in
\begin{center}
\begin{small}
\begin{sc}
\resizebox{\textwidth}{!}{%
\begin{tabular}{lccccc|cccc}
\toprule
Sparsity & Random & Magnitude & GN & SNIP & GraSP & FD & FP & FTS & FBSS \\
\midrule
0.80  & 60.39 ± 1.16 & 58.91 ± 0.41 & 52.81 ± 1.32 & 55.62 ± 2.27 & 55.15 ± 2.25 & 56.71 ± 0.31 & 58.03 ± 0.93 & 52.41 ± 3.07 & 52.74 ± 5.16 \\
0.90  & 58.90 ± 0.98 & 60.95 ± 0.81 & 50.56 ± 4.59 & 55.89 ± 2.05 & 56.01 ± 1.58 & 52.07 ± 3.24 & 53.65 ± 0.57 & 52.45 ± 3.75 & 19.65 ± 1.68 \\
0.95  & 56.10 ± 0.85 & 57.64 ± 2.63 & 50.34 ± 1.00 & 53.70 ± 3.60 & 56.16 ± 0.41 & 54.44 ± 1.38 & 53.24 ± 3.54 & 53.56 ± 1.26 & 17.24 ± 0.44 \\
0.98  & 50.97 ± 0.40 & 54.66 ± 2.56 & 43.43 ± 5.32 & 50.19 ± 1.59 & 54.64 ± 1.50 & 42.75 ± 1.91 & 50.59 ± 3.39 & 48.56 ± 5.25 & 16.42 ± 0.64 \\
0.99  & 46.52 ± 0.45 & 43.33 ± 5.83 & 33.90 ± 5.35 & 42.65 ± 5.32 & 45.98 ± 4.48 & 29.67 ± 8.49 & 49.11 ± 3.46 & 48.70 ± 2.59 & 13.25 ± 0.84 \\
\bottomrule
\end{tabular}}
\end{sc}
\end{small}
\end{center}
\vskip -0.1in
\end{table}


%------------------------------------------------------------------------------------------------

\begin{table}[h]
\caption{Performance of different compression methods evaluated after 5 warmup epochs using VGG19 on the CIFAR-100 dataset. We report the mean accuracy between three initialization seeds across various sparsity levels. Baseline, no pruning: $58.96 \pm 2.30$.}
\label{tab:VGG19_cifar100_compressors_warmup5}
\vskip 0.15in
\begin{center}
\begin{small}
\begin{sc}
\resizebox{\textwidth}{!}{%
\begin{tabular}{lccccc|cccc}
\toprule
Sparsity & Random & Magnitude & GN & SNIP & GraSP & FD & FP & FTS & FBSS \\
\midrule
0.80  & 60.41 ± 1.39 & 58.38 ± 0.85 & 60.86 ± 0.79 & 61.63 ± 0.45 & 56.25 ± 0.49 & 59.59 ± 0.76 & 59.37 ± 3.50 & 60.86 ± 0.53 & 46.93 ± 9.04 \\
0.90  & 60.32 ± 0.09 & 57.74 ± 1.64 & 57.77 ± 2.41 & 58.23 ± 4.07 & 56.27 ± 1.02 & 60.19 ± 0.63 & 61.23 ± 0.50 & 60.52 ± 0.37 & 21.66 ± 1.95 \\
0.95 & 57.86 ± 0.53 & 59.55 ± 1.15 & 56.09 ± 0.97 & 58.83 ± 0.65 & 55.26 ± 1.25 & 55.80 ± 2.77 & 59.83 ± 0.94 & 58.52 ± 1.32 & 19.98 ± 2.62 \\
0.98 & 51.75 ± 0.43 & 47.75 ± 7.63 & 52.26 ± 4.06 & 55.27 ± 1.69 & 54.59 ± 0.96 & 49.46 ± 4.98 & 57.40 ± 1.26 & 56.00 ± 1.08 & 17.59 ± 1.36 \\
0.99 & 47.59 ± 0.80 & 42.46 ± 7.95 & 46.58 ± 2.00 & 53.13 ± 0.84 & 53.91 ± 1.53 & 42.87 ± 4.63 & 53.17 ± 1.18 & 53.05 ± 2.14 & 13.92 ± 0.14 \\
\bottomrule
\end{tabular}}
\end{sc}
\end{small}
\end{center}
\vskip -0.1in
\end{table}


%------------------------------------------------------------------------------------------------
\clearpage

\section{Mask Batch Size for Other Sparsities}
The Effect of batch size on pruning performance across different sparsities. 
As sparsity increases, the effect of batch size on pruning performance becomes more pronounced. 
At lower sparsities (0.90, 0.95), the differences across batch sizes are less evident, suggesting that even smaller batches provide a reasonable estimation of parameter importance. However, at extreme sparsities (0.98, 0.99), we observe a clear trend where larger batch sizes consistently lead to better parameter selection, ultimately improving accuracy. This aligns with our hypothesis that larger batches help reduce variance in gradient estimation, leading to more stable and effective pruning decisions. 
\label{batch_size_heatmaps}

\begin{figure}[h]
    \centering
    \includegraphics[width=0.8\linewidth]{imgs/cifar10_resnet18_heatmap_warmup_0.png}
    \caption{Effect of batch size on pruning performance at increasing sparsities.}
    \label{fig:enter-label}
\end{figure}

%------------------------------------------------------------------------------------------------

\clearpage
\section{Comparison of our criteria with magnitude-based pruning}

Figure \ref{fig:our_criterion_vs_magnitude} illustrates the relationship between parameter magnitude and different sensitivity-based pruning metrics. Each point represents a model parameter, with red points indicating the top-ranked parameters selected for retention by each criterion. The green dashed line marks the 99th percentile of parameter magnitudes.

A key observation is that the most effective pruning criteria, such as Fisher-Taylor Sensitivity, tend to retain parameters with a broad range of magnitudes, including many that are relatively small (left of the green line). This shows that the estimated importance does not always prioritize parameters based on their magnitude. 


\begin{figure}[htp]
    \centering
    \includegraphics[width=0.9\linewidth]{imgs/cifar_10_mag_vs_criteria_s_99.png}
    \caption{Our criteria vs. Magnitude parameter selection for 99\% sparsity (ResNet18, CIFAR-10, Seed 0)} 
    \label{fig:our_criterion_vs_magnitude}
\end{figure}




\end{document}


