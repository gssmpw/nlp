%%%%%%%% ICML 2025 EXAMPLE LATEX SUBMISSION FILE %%%%%%%%%%%%%%%%%

\documentclass{article}

% Recommended, but optional, packages for figures and better typesetting:
\usepackage{microtype}
\usepackage{graphicx}
\usepackage{subfigure}
\usepackage{booktabs} % for professional tables

% hyperref makes hyperlinks in the resulting PDF.
% If your build breaks (sometimes temporarily if a hyperlink spans a page)
% please comment out the following usepackage line and replace
% \usepackage{icml2025} with \usepackage[nohyperref]{icml2025} above.
\usepackage{hyperref}


% Attempt to make hyperref and algorithmic work together better:
\newcommand{\theHalgorithm}{\arabic{algorithm}}

% Use the following line for the initial blind version submitted for review:
% \usepackage{icml2025}

% If accepted, instead use the following line for the camera-ready submission:
\usepackage[accepted]{icml2025}

% For theorems and such
\usepackage{amsmath}
\usepackage{amssymb}
\usepackage{mathtools}
\usepackage{amsthm}
\usepackage{multirow}

\usepackage{makecell}
% \usepackage{algpseudocode}
\usepackage{eqparbox}
\usepackage{array}
\usepackage{pifont}
\usepackage{wrapfig}
\usepackage{amsfonts}
\usepackage{colortbl}
\usepackage{fontenc}
% \usepackage{tikz}
\usepackage{xspace}
\usepackage{enumitem}
\usepackage{marvosym}

\definecolor{baselinecolor}{rgb}{0.9, 0.9, 1.}
\definecolor{graycolor}{gray}{0.9}
\definecolor{Green}{rgb}{0.0, 0.5, 0.0}
\definecolor{Green}{rgb}{0.0, 0.5, 0.0}
\definecolor{rebuttal}{rgb}{0.6, 0.6, 1.}
\newcommand{\baseline}[1]{\cellcolor{baselinecolor}{#1}}
\newcommand{\graybase}[1]{\cellcolor{graycolor}{#1}}


\newcommand{\cmark}{\ding{51}}%
\newcommand{\xmark}{\ding{55}}%


\makeatletter
\DeclareRobustCommand\onedot{\futurelet\@let@token\@onedot}
\def\@onedot{\ifx\@let@token.\else.\null\fi\xspace}
\def\eg{\emph{e.g}\onedot} \def\Eg{\emph{E.g}\onedot}
\def\ie{\emph{i.e}\onedot} \def\Ie{\emph{I.e}\onedot}
\def\cf{\emph{c.f}\onedot} \def\Cf{\emph{C.f}\onedot}
\def\etc{\emph{etc}\onedot} \def\vs{\emph{vs}\onedot}
\def\wrt{w.r.t\onedot} \def\dof{d.o.f\onedot}
\def\etal{\emph{et al}\onedot}
\makeatother

% if you use cleveref..
\usepackage[capitalize,noabbrev]{cleveref}

%%%%%%%%%%%%%%%%%%%%%%%%%%%%%%%%
% THEOREMS
%%%%%%%%%%%%%%%%%%%%%%%%%%%%%%%%
\theoremstyle{plain}
\newtheorem{theorem}{Theorem}[section]
\newtheorem{proposition}[theorem]{Proposition}
\newtheorem{lemma}[theorem]{Lemma}
\newtheorem{corollary}[theorem]{Corollary}
\theoremstyle{definition}
\newtheorem{definition}[theorem]{Definition}
\newtheorem{assumption}[theorem]{Assumption}
\theoremstyle{remark}
\newtheorem{remark}[theorem]{Remark}

% Todonotes is useful during development; simply uncomment the next line
%    and comment out the line below the next line to turn off comments
%\usepackage[disable,textsize=tiny]{todonotes}
\usepackage[textsize=tiny]{todonotes}

\newcommand*{\affaddr}[1]{#1} 
\newcommand*{\affmark}[1][*]{\textsuperscript{#1}}
\newcommand{\authormark}[2][]{%
  \begingroup
  \def\@thefnmark{#1}%
  \footnote{#2}%
  \endgroup
}



\author{%
\bf
Tao Zhang\affmark[1,3]~~~ 
Cheng Da\affmark[2]~~~ 
Kun Ding\affmark[1]~~~
Kun Jin\affmark[2]~~~ 
Yan Li\affmark[2]~~~ 
Tingting Gao\affmark[2]~~~ 
Di Zhang\affmark[2]~~~ \\
\bf
Shiming Xiang\affmark[1,3]~~~ 
Chunhong Pan\affmark[1]~~~ \\
\affaddr{\affmark[1]MAIS, CASIA~~~~~~~~~~}
\affaddr{\affmark[2]Kuaishou Technology~~~~~~~~~~} 
\affaddr{\affmark[3]School of Artificial Intelligence, UCAS~~~~~~~~~~} \\
\small
}


% The \icmltitle you define below is probably too long as a header.
% Therefore, a short form for the running title is supplied here:
% \icmltitlerunning{Submission and Formatting Instructions for ICML 2025}
\icmltitlerunning{Diffusion Model as a Noise-Aware Latent Reward Model for Step-Level Preference Optimization}

\begin{document}

\twocolumn[
\icmltitle{Diffusion Model as a Noise-Aware Latent Reward Model for Step-Level Preference Optimization}

\icmlkeywords{Diffusion Model, Preference Optimization, Reward Model, Image Generation}

\vskip 0.3in

\arxivauthor
]

\setlength{\abovedisplayskip}{5pt}
\setlength{\belowdisplayskip}{5pt}
% \setlength{\arraycolsep}{2pt}

% this must go after the closing bracket ] following \twocolumn[ ...

% This command actually creates the footnote in the first column
% listing the affiliations and the copyright notice.
% The command takes one argument, which is text to display at the start of the footnote.
% The \icmlEqualContribution command is standard text for equal contribution.
% Remove it (just {}) if you do not need this facility.

% \printAffiliationsAndNotice{}  % leave blank if no need to mention equal contribution
% \printAffiliationsAndNotice{\icmlEqualContribution} % otherwise use the standard text.

\begin{abstract}
% 视觉语言模型用于逐步偏好优化时的缺点--> 扩散模型具有这些能力--> 提出LRM,提出LPO --> 效果。

Preference optimization for diffusion models aims to align them with human preferences for images. Previous methods typically leverage Vision-Language Models (VLMs) as pixel-level reward models to approximate human preferences. However, when used for step-level preference optimization, these models face challenges in handling noisy images of different timesteps and require complex transformations into pixel space. In this work, we demonstrate that diffusion models are inherently well-suited for step-level reward modeling in the latent space, as they can naturally extract features from noisy latent images. Accordingly, we propose the \textbf{Latent Reward Model (LRM)}, which repurposes components of diffusion models to predict preferences of latent images at various timesteps. Building on LRM, we introduce \textbf{Latent Preference Optimization (LPO)}, a method designed for step-level preference optimization directly in the latent space. Experimental results indicate that LPO not only significantly enhances performance in aligning diffusion models with general, aesthetic, and text-image alignment preferences, but also achieves 2.5-28$\times$ training speedup compared to existing preference optimization methods. Our code and models are available at
\url{https://github.com/Kwai-Kolors/LPO}.
\end{abstract}

\section{Introduction}\label{sec:Intro} 


Novel view synthesis offers a fundamental approach to visualizing complex scenes by generating new perspectives from existing imagery. 
This has many potential applications, including virtual reality, movie production and architectural visualization \cite{Tewari2022NeuRendSTAR}. 
An emerging alternative to the common RGB sensors are event cameras, which are  
 bio-inspired visual sensors recording events, i.e.~asynchronous per-pixel signals of changes in brightness or color intensity. 

Event streams have very high temporal resolution and are inherently sparse, as they only happen when changes in the scene are observed. 
Due to their working principle, event cameras bring several advantages, especially in challenging cases: they excel at handling high-speed motions 
and have a substantially higher dynamic range of the supported signal measurements than conventional RGB cameras. 
Moreover, they have lower power consumption and require varied storage volumes for captured data that are often smaller than those required for synchronous RGB cameras \cite{Millerdurai_3DV2024, Gallego2022}. 

The ability to handle high-speed motions is crucial in static scenes as well,  particularly with handheld moving cameras, as it helps avoid the common problem of motion blur. It is, therefore, not surprising that event-based novel view synthesis has gained attention, although color values are not directly observed.
Notably, because of the substantial difference between the formats, RGB- and event-based approaches require fundamentally different design choices. %

The first solutions to event-based novel view synthesis introduced in the literature demonstrate promising results \cite{eventnerf, enerf} and outperform non-event-based alternatives for novel view synthesis in many challenging scenarios. 
Among them, EventNeRF \cite{eventnerf} enables novel-view synthesis in the RGB space by assuming events associated with three color channels as inputs. 
Due to its NeRF-based architecture \cite{nerf}, it can handle single objects with complete observations from roughly equal distances to the camera. 
It furthermore has limitations in training and rendering speed: 
the MLP used to represent the scene requires long training time and can only handle very limited scene extents or otherwise rendering quality will deteriorate. 
Hence, the quality of synthesized novel views will degrade for larger scenes. %

We present Event-3DGS (E-3DGS), i.e.,~a new method for novel-view synthesis from event streams using 3D Gaussians~\cite{3dgs} 
demonstrating fast reconstruction and rendering as well as handling of unbounded scenes. 
The technical contributions of this paper are as follows: 
\begin{itemize}
\item With E-3DGS, we introduce the first approach for novel view synthesis from a color event camera that combines 3D Gaussians with event-based supervision. 
\item We present frustum-based initialization, adaptive event windows, isotropic 3D Gaussian regularization and 3D camera pose refinement, and demonstrate that high-quality results can be obtained. %

\item Finally, we introduce new synthetic and real event datasets for large scenes to the community to study novel view synthesis in this new problem setting. 
\end{itemize}
Our experiments demonstrate systematically superior results compared to EventNeRF \cite{eventnerf} and other baselines. 
The source code and dataset of E-3DGS are released\footnote{\url{https://4dqv.mpi-inf.mpg.de/E3DGS/}}. 






\section{Related Work}
\label{lit_review}

\begin{highlight}
{

Our research builds upon {\em (i)} Assessing Web Accessibility, {\em (ii)} End-User Accessibility Repair, and {\em (iii)} Developer Tools for Accessibility.

\subsection{Assessing Web Accessibility}
From the earliest attempts to set standards and guidelines, web accessibility has been shaped by a complex interplay of technical challenges, legal imperatives, and educational campaigns. Over the past 25 years, stakeholders have sought to improve digital inclusion by establishing foundational standards~\cite{chisholm2001web, caldwell2008web}, enforcing legal obligations~\cite{sierkowski2002achieving, yesilada2012understanding}, and promoting a broader culture of accessibility awareness among developers~\cite{sloan2006contextual, martin2022landscape, pandey2023blending}. 
Despite these longstanding efforts, systemic accessibility issues persist. According to the 2024 WebAIM Million report~\cite{webaim2024}, 95.9\% of the top one million home pages contained detectable WCAG violations, averaging nearly 57 errors per page. 
These errors take many forms: low color contrast makes the interface difficult for individuals with color deficiency or low vision to read text; missing alternative text leaves users relying on screen readers without crucial visual context; and unlabeled form inputs or empty links and buttons hinder people who navigate with assistive technologies from completing basic tasks. 
Together, these accessibility issues not only limit user access to critical online resources such as healthcare, education, and employment but also result in significant legal risks and lost opportunities for businesses to engage diverse audiences. Addressing these pervasive issues requires systematic methods to identify, measure, and prioritize accessibility barriers, which is the first step toward achieving meaningful improvements.

Prior research has introduced methods blending automation and human evaluation to assess web accessibility. Hybrid approaches like SAMBA combine automated tools with expert reviews to measure the severity and impact of barriers, enhancing evaluation reliability~\cite{brajnik2007samba}. Quantitative metrics, such as Failure Rate and Unified Web Evaluation Methodology, support large-scale monitoring and comparative analysis, enabling cost-effective insights~\cite{vigo2007quantitative, martins2024large}. However, automated tools alone often detect less than half of WCAG violations and generate false positives, emphasizing the need for human interpretation~\cite{freire2008evaluation, vigo2013benchmarking}. Recent progress with large pretrained models like Large Language Models (LLMs)~\cite{dubey2024llama,bai2023qwen} and Large Multimodal Models (LMMs)~\cite{liu2024visual, bai2023qwenvl} offers a promising step forward, automating complex checks like non-text content evaluation and link purposes, achieving higher detection rates than traditional tools~\cite{lopez2024turning, delnevo2024interaction}. Yet, these large models face challenges, including dependence on training data, limited contextual judgment, and the inability to simulate real user experiences. These limitations underscore the necessity of combining models with human oversight for reliable, user-centered evaluations~\cite{brajnik2007samba, vigo2013benchmarking, delnevo2024interaction}. 

Our work builds on these prior efforts and recent advancements by leveraging the capabilities of large pretrained models while addressing their limitations through a developer-centric approach. CodeA11y integrates LLM-powered accessibility assessments, tailored accessibility-aware system prompts, and a dedicated accessibility checker directly into GitHub Copilot---one of the most widely used coding assistants. Unlike standalone evaluation tools, CodeA11y actively supports developers throughout the coding process by reinforcing accessibility best practices, prompting critical manual validations, and embedding accessibility considerations into existing workflows.
% This pervasive shortfall reflects the difficulty of scaling traditional approaches---such as manual audits and automated tools---that either demand immense human effort or lack the nuanced understanding needed to capture real-world user experiences. 
%
% In response, a new wave of AI-driven methods, many powered by large language models (LLMs), is emerging to bridge these accessibility detection and assessment gaps. Early explorations, such as those by Morillo et al.~\cite{morillo2020system}, introduced AI-assisted recommendations capable of automatic corrections, illustrating how computational intelligence can tackle the repetitive, common errors that plague large swaths of the web. Building on this foundation, Huang et al.~\cite{huang2024access} proposed ACCESS, a prompt-engineering framework that streamlines the identification and remediation of accessibility violations, while López-Gil et al.~\cite{lopez2024turning} demonstrated how LLMs can help apply WCAG success criteria more consistently---reducing the reliance on manual effort. Beyond these direct interventions, recent work has also begun integrating user experiences more seamlessly into the evaluation process. For example, Huq et al.~\cite{huq2024automated} translate user transcripts and corresponding issues into actionable test reports, ensuring that accessibility improvements align more closely with authentic user needs.
% However, as these AI-driven solutions evolve, researchers caution against uncritical adoption. Othman et al.~\cite{othman2023fostering} highlight that while LLMs can accelerate remediation, they may also introduce biases or encourage over-reliance on automated processes. Similarly, Delnevo et al.~\cite{delnevo2024interaction} emphasize the importance of contextual understanding and adaptability, pointing to the current limitations of LLM-based systems in serving the full spectrum of user needs. 
% In contrast to this backdrop, our work introduces and evaluates CodeA11y, an LLM-augmented extension for GitHub Copilot that not only mitigates these challenges by providing more consistent guidance and manual validation prompts, but also aligns AI-driven assistance with developers’ workflows, ultimately contributing toward more sustainable propulsion for building accessible web.

% Broader implications of inaccessibility—legal compliance, ethical concerns, and user experience
% A Historical Review of Web Accessibility Using WAVE
% "I tend to view ads almost like a pestilence": On the Accessibility Implications of Mobile Ads for Blind Users

% In the research domain, several methods have been developed to assess and enhance web accessibility. These include incorporating feedback into developer tools~\cite{adesigner, takagi2003accessibility, bigham2010accessibility} and automating the creation of accessibility tests and reports for user interfaces~\cite{swearngin2024towards, taeb2024axnav}. 

% Prior work has also studied accessibility scanners as another avenue of AI to improve web development practices~\cite{}.
% However, a persistent challenge is that developers need to be aware of these tools to utilize them effectively. With recent advancements in LLMs, developers might now build accessible websites with less effort using AI assistants. However, the impact of these assistants on the accessibility of their generated code remains unclear. This study aims to investigate these effects.

\subsection{End-user Accessibility Repair}
In addition to detecting accessibility errors and measuring web accessibility, significant research has focused on fixing these problems.
Since end-users are often the first to notice accessibility problems and have a strong incentive to address them, systems have been developed to help them report or fix these problems.

Collaborative, or social accessibility~\cite{takagi2009collaborative,sato2010social}, enabled these end-user contributions to be scaled through crowd-sourcing.
AccessMonkey~\cite{bigham2007accessmonkey} and Accessibility Commons~\cite{kawanaka2008accessibility} were two examples of repositories that store accessibility-related scripts and metadata, respectively.
Other work has developed browser extensions that leverage crowd-sourced databases to automatically correct reading order, alt-text, color contrast, and interaction-related issues~\cite{sato2009s,huang2015can}.

One drawback of collaborative accessibility approaches is that they cannot fix problems for an ``unseen'' web page on-demand, so many projects aim to automatically detect and improve interfaces without the need for an external source of fixes.
A large body of research has focused on making specific web media (e.g., images~\cite{gleason2019making,guinness2018caption, twitterally, gleason2020making, lee2021image}, design~\cite{potluri2019ai,li2019editing, peng2022diffscriber, peng2023slide}, and videos~\cite{pavel2020rescribe,peng2021say,peng2021slidecho,huh2023avscript}) accessible through a combination of machine learning (ML) and user-provided fixes.
Other work has focused on applying more general fixes across all websites.

Opportunity accessibility addressed a common accessibility problem of most websites: by default, content is often hard to see for people with visual impairments, and many users, especially older adults, do not know how to adjust or enable content zooming~\cite{bigham2014making}.
To this end, a browser script (\texttt{oppaccess.js}) was developed that automatically adjusted the browser's content zoom to maximally enlarge content without introducing adverse side-effects (\textit{e.g.,} content overlap).
While \texttt{oppaccess.js} primarily targeted zoom-related accessibility, recent work aimed to enable larger types of changes, by using LLMs to modify the source code of web pages based on user questions or directives~\cite{li2023using}.

Several efforts have been focused on improving access to desktop and mobile applications, which present additional challenges due to the unavailability of app source code (\textit{e.g.,} HTML).
Prefab is an approach that allows graphical UIs to be modified at runtime by detecting existing UI widgets, then replacing them~\cite{dixon2010prefab}.
Interaction Proxies used these runtime modification strategies to ``repair'' Android apps by replacing inaccessible widgets with improved alternatives~\cite{zhang2017interaction, zhang2018robust}.
The widget detection strategies used by these systems previously relied on a combination of heuristics and system metadata (\textit{e.g.,} the view hierarchy), which are incomplete or missing in the accessible apps.
To this end, ML has been employed to better localize~\cite{chen2020object} and repair UI elements~\cite{chen2020unblind,zhang2021screen,wu2023webui,peng2025dreamstruct}.

In general, end-user solutions to repairing application accessibility are limited due to the lack of underlying code and knowledge of the semantics of the intended content.

\subsection{Developer Tools for Accessibility}
Ultimately, the best solution for ensuring an accessible experience lies with front-end developers. Many efforts have focused on building adequate tooling and support to help developers with ensuring that their UI code complies with accessibility standards.

Numerous automated accessibility testing tools have been created to help developers identify accessibility issues in their code: i) static analysis tools, such as IBM Equal Access Accessibility Checker~\cite{ibm2024toolkit} or Microsoft Accessibility Insights~\cite{accessibilityinsights2024}, scan the UI code's compliance with predefined rules derived from accessibility guidelines; and ii) dynamic or runtime accessibility scanners, such as Chrome Devtools~\cite{chromedevtools2024} or axe-Core Accessibility Engine~\cite{deque2024axe}, perform real-time testing on user interfaces to detect interaction issues not identifiable from the code structure. While these tools greatly reduce the manual effort required for accessibility testing, they are often criticized for their limited coverage. Thus, experts often recommend manually testing with assistive technologies to uncover more complex interaction issues. Prior studies have created accessibility crawlers that either assist in developer testing~\cite{swearngin2024towards,taeb2024axnav} or simulate how assistive technologies interact with UIs~\cite{10.1145/3411764.3445455, 10.1145/3551349.3556905, 10.1145/3544548.3580679}.

Similar to end-user accessibility repair, research has focused on generating fixes to remediate accessibility issues in the UI source code. Initial attempts developed heuristic-based algorithms for fixing specific issues, for instance, by replacing text or background color attributes~\cite{10.1145/3611643.3616329}. More recent work has suggested that the code-understanding capabilities of LLMs allow them to suggest more targeted fixes.
For example, a study demonstrated that prompting ChatGPT to fix identified WCAG compliance issues in source code could automatically resolve a significant number of them~\cite{othman2023fostering}. Researchers have sought to leverage this capability by employing a multi-agent LLM architecture to automatically identify and localize issues in source code and suggest potential code fixes~\cite{mehralian2024automated}.

While the approaches mentioned above focus on assessing UI accessibility of already-authored code (\textit{i.e.,} fixing existing code), there is potential for more proactive approaches.
For example, LLMs are often used by developers to generate UI source code from natural language descriptions or tab completions~\cite{chen2021evaluating,GitHubCopilot,lozhkov2024starcoder,hui2024qwen2,roziere2023code,zheng2023codegeex}, but LLMs frequently produce inaccessible code by default~\cite{10.1145/3677846.3677854,mowar2024tab}, leading to inaccessible output when used by developers without sufficient awareness of accessibility knowledge.
The primary focus of this paper is to design a more accessibility-aware coding assistant that both produces more accessible code without manual intervention (\textit{e.g.,} specific user prompting) and gradually enables developers to implement and improve accessibility of automatically-generated code through IDE UI modifications (\textit{e.g.}, reminder notifications).

}
\end{highlight}



% Work related to this paper includes {\em (i)} Web Accessibility and {\em (ii)} Developer Practices in AI-Assisted Programming.

% \ipstart{Web Accessibility: Practice, Evaluation, and Improvements} Substantial efforts have been made to set accessibility standards~\cite{chisholm2001web, caldwell2008web}, establish legal requirements~\cite{sierkowski2002achieving, yesilada2012understanding}, and promote education and advocacy among developers~\cite{sloan2006contextual, martin2022landscape, pandey2023blending}. In the research domain, several methods have been developed to assess and enhance web accessibility. These include incorporating feedback into developer tools~\cite{adesigner, takagi2003accessibility, bigham2010accessibility} and automating the creation of accessibility tests and reports for user interfaces~\cite{swearngin2024towards, taeb2024axnav}. 
% % Prior work has also studied accessibility scanners as another avenue of AI to improve web development practices~\cite{}.
% However, a persistent challenge is that developers need to be aware of these tools to utilize them effectively. With recent advancements in LLMs, developers might now build accessible websites with less effort using AI assistants. However, the impact of these assistants on the accessibility of their generated code remains unclear. This study aims to investigate these effects.

% \ipstart{Developer Practices in AI-Assisted Programming}
% Recent usability research on AI-assisted development has examined the interaction strategies of developers while using AI coding assistants~\cite{barke2023grounded}.
% They observed developers interacted with these assistants in two modes -- 1) \textit{acceleration mode}: associated with shorter completions and 2) \textit{exploration mode}: associated with long completions.
% Liang {\em et al.} \cite{liang2024large} found that developers are driven to use AI assistants to reduce their keystrokes, finish tasks faster, and recall the syntax of programming languages. On the other hand, developers' reason for rejecting autocomplete suggestions was the need for more consideration of appropriate software requirements. This is because primary research on code generation models has mainly focused on functional correctness while often sidelining non-functional requirements such as latency, maintainability, and security~\cite{singhal2024nofuneval}. Consequently, there have been increasing concerns about the security implications of AI-generated code~\cite{sandoval2023lost}. Similarly, this study focuses on the effectiveness and uptake of code suggestions among developers in mitigating accessibility-related vulnerabilities. 


% ============================= additional rw ============================================
% - Paulina Morillo, Diego Chicaiza-Herrera, and Diego Vallejo-Huanga. 2020. System of Recommendation and Automatic Correction of Web Accessibility Using Artificial Intelligence. In Advances in Usability and User Experience, Tareq Ahram and Christianne Falcão (Eds.). Springer International Publishing, Cham, 479–489
% - Juan-Miguel López-Gil and Juanan Pereira. 2024. Turning manual web accessibility success criteria into automatic: an LLM-based approach. Universal Access in the Information Society (2024). https://doi.org/10.1007/s10209-024-01108-z
% - s
% - Calista Huang, Alyssa Ma, Suchir Vyasamudri, Eugenie Puype, Sayem Kamal, Juan Belza Garcia, Salar Cheema, and Michael Lutz. 2024. ACCESS: Prompt Engineering for Automated Web Accessibility Violation Corrections. arXiv:2401.16450 [cs.HC] https://arxiv.org/abs/2401.16450
% - Syed Fatiul Huq, Mahan Tafreshipour, Kate Kalcevich, and Sam Malek. 2025. Automated Generation of Accessibility Test Reports from Recorded User Transcripts. In Proceedings of the 47th International Conference on Software Engineering (ICSE) (Ottawa, Ontario, Canada). IEEE. https://ics.uci.edu/~seal/publications/2025_ICSE_reca11.pdf To appear in IEEE Xplore
% - Achraf Othman, Amira Dhouib, and Aljazi Nasser Al Jabor. 2023. Fostering websites accessibility: A case study on the use of the Large Language Models ChatGPT for automatic remediation. In Proceedings of the 16th International Conference on PErvasive Technologies Related to Assistive Environments (Corfu, Greece) (PETRA ’23). Association for Computing Machinery, New York, NY, USA, 707–713. https://doi.org/10.1145/3594806.3596542
% - Zsuzsanna B. Palmer and Sushil K. Oswal. 0. Constructing Websites with Generative AI Tools: The Accessibility of Their Workflows and Products for Users With Disabilities. Journal of Business and Technical Communication 0, 0 (0), 10506519241280644. https://doi.org/10.1177/10506519241280644
% ============================= additional rw ============================================

\section{Background}
\label{sec:background}

To present our work systematically, we formulate it as a two-level hierarchical solution of Markov Decision Processes (MDPs), corresponding to the two-phase pipeline. In contrast to some hierarchical methods targeting action levels \citep{mcgovern1998hierarchical, hauskrecht2013hierarchical}, we focus on the level of reward functions.

\textbf{Low-level MDP of Controlling Problem}
Specifically, the scene reconstruction phase is to construct the MDP $\mathcal{G} = \langle \mathcal{S}, \mathcal{A}, T, \mathcal{R}_{0|1} \rangle$ from videos, where $\mathcal{S} \in \mathbb{R}^m$ represents the states of the environment, $\mathcal{A}$ is the action space of the agent, and $T$ is the transition probability function. 
$\mathcal{R}_{0|1}$ is the 0-1 reward function that distinguish whether the trajectory is successful. $\mathcal{R}_{0|1}$ can be a scalar evaluation function. 
To solve this MDP, we will train a policy $\pi$ through reinforcement learning in the Isaac Gym simulation. To achieve high performance, we leverage the LLM to sample various reward functions to learn policies in a high-level manner, based on the evaluation results from $\mathcal{R}_{0|1}$ and some CoT \citep{wei2022chain} instructions.

\textbf{High-level MDP of Reward Designs} The aim of reward design is to create a shaped reward function that simplifies the optimization of a challenging given reward function, such as the sparse 0-1 reward function $\mathcal{R}_{0|1}$. Following the definition of Reward Design Problem (RDP) from previous works \citep{singh2009rewards, ma2023eureka}, we consider a high-level MDP $\hat{\mathcal{G}} = \langle \hat{\mathcal{S}}, \hat{\mathcal{A}}, \hat{T}, F \rangle$. Here, $\hat{\mathcal{A}}$ is the space of reward functions. Each time we choose an action $\hat{R} \in \mathcal{A}$ in the MDP $\hat{\mathcal{G}}$, we will train a policy $\pi$ by RL for the low-level MDP $\mathcal{G} = \langle \mathcal{S}, \mathcal{A}, T, \hat{\mathcal{R}} + \mathcal{R}_{0|1} \rangle$. The horizon of the MDP $\hat{\mathcal{G}}$ is the iteration number in the second phase.
$\hat{\mathcal{S}}$ includes the training and evaluation information during RL and the policy model $\pi$. $\hat{T}$ is the state transition function, and $F$ is the reward function that produces a scalar evaluation of any policy $\pi$. Specifically, $F$ is equal to $\mathcal{R}_{0|1}$. Thus, the high-level MDP's goal is to find a reward function $\hat{\mathcal{R}} \in \hat{A}$ to maximize the success rates of the low-level policies.


\section{The E-3DGS Method}\label{sec:Method} 
Our aim is to learn a 3D representation of a static scene using only a color event stream, where each pixel observes changes in brightness corresponding to one of the red, green, or blue channels according to a Bayer pattern, with known camera intrinsics $K_t~\in~\mathbb{R}^{3 \times 3}$, and noisy initial poses~$P_t~\in~\mathbb{R}^{3 \times 4}$, at reasonably high-frequency time steps indexed by $t$. 
Following 3DGS~\cite{3dgs}, we represent our scene by anisotropic 3D Gaussians. Our methodology comprises a technique to initialize Gaussians in the absence of a Structure from Motion (SfM) point cloud, adaptive event frame supervision of 3DGS, and a pose refinement module. 
An overview of our method is provided in Fig.~\ref{fig:methodology}.


Our E-3DGS method is not restricted to scenes of a certain size and can handle unbounded environments. It does not rely on any assumptions regarding the background color, type of camera motion, or speed. Thus, it ensures robust performance across a wide range of scenarios. 

\subsection{Event Stream Supervision} 

There are two main categories of approaches to learning 3D scene representations from event streams. 
Some apply the loss to single events~\cite{robust_enerf} based on Eq.~\eqref{eq:egm}. Others use the sum of events~\( E_{\x}(t_1,t_2) \) from Eq.~\eqref{eq:egm_sum}. We choose the second approach, as rasterization in 3DGS is well suited to efficiently render entire images rather than individual pixels. 

To optimize our Gaussian scene representation using event data, we can make a logical equivalence between the observed event stream and the scene renderings. 
To do so, we replace the true logarithmic intensities~\( L_{\x} \) in Eq.~\eqref{eq:egm_sum} with the rendered logarithmic intensities~\(\hat{L}_{\x} \) from our scene, and the times $\tau$ with the camera poses $P_t$ that were used to render the scene at the respective time steps. 
Following the approach used in~\cite{eventnerf}, the log difference is then point-wise multiplied with a Bayer filter $F$ to obtain the respective color channel. We can finally calculate the error between the logarithmic change from our model and the actual change observed from the event stream, and define the following per-pixel loss: 
\begin{equation}
    \begin{split}
    &\mathcal{L}_{\x}\left(t_1, t_2\right) = \\
    &\left\| 
    F \odot (\hat{L}_{\x}(P_{t_2}) - \hat{L}_{\x}(P_{t_1})) 
    - F \odot E_{\x}\left(t_1, t_2\right)\right\|_1, 
    \end{split}
    \label{eq:L_recon_per_pixel} 
\end{equation}
where ``$\odot$'' denotes pixelwise multiplication. 


\subsection{Frustum-Based Initialization}
\label{sec:frustum_init}

In the original 3DGS \cite{3dgs}, the Gaussians are initialized using a point cloud obtained from applying SfM on the input images. 
The authors also experimented with initializing the Gaussians at random locations within a cube. While this worked for them with a slight performance drop, it requires an assumption about the extent of the scene. 

Applying SfM directly to event streams is more challenging than RGB inputs \cite{Kim2016} and exploring this aspect is not the primary focus of this paper. 
In the absence of an SfM point cloud, we use the randomly initialized Gaussians and extend this approach to unbounded scenes. 
To this end, we initialize a specified number of Gaussians (on the order of \qty{d4}{}) in the frustum of each camera. 
This gives two benefits: 1) All the initialized Gaussians are within the observable area, and 2) We only need one loose assumption about the scene, which is the maximum depth $z_\mathrm{far}$. 


\subsection{Adaptive Event Window}\label{subsec:adaptive_window}

Rudnev et al.~\cite{eventnerf} demonstrated in EventNeRF that using a fixed event window duration results in suboptimal reconstruction. They find that larger windows are essential for capturing low-frequency color and structure, and smaller ones are essential for optimization of finer high-frequency details. While they randomly sampled the event window duration, a drawback is that it does not consider the camera speed and event rate, thus the sampled windows may contain too many or too few events.  
As our dataset features variable camera speeds, we improve upon this by sampling the number of events rather than the window duration.  
To achieve this, for each time step we randomly sample a target number of events from within the range $[N_\mathrm{min}, N_\mathrm{max}]$. 
Given a time step~$t$, we search for a previous time step~$t_s$ such that the number of events in the event frame $E(t_s, t)$ is approximately equal to the desired number. 

When determining $N_\mathrm{max}$, we find that for values where details and low-frequency structure are optimal, 3DGS tends to get unstable and sometimes prunes away Gaussians in homogeneous areas.
While this can be mitigated by choosing a much larger $N_\mathrm{max}$, this again deteriorates the details. 
Therefore, we propose a strategy to incorporate both, small and large windows. For each $t$, we choose two earlier time steps~$t_{s_1}$ and~$t_{s_2}$. The ranges for sampling the event counts for both are empirically chosen to be $[\frac{N_\mathrm{max}}{10}, N_\mathrm{max}]$ and $[\frac{N_{max}}{300}, \frac{N_\mathrm{max}}{30}]$. We then render frames from our model at times $t$, $t_{s_1}$ and $t_{s_2}$, and use two concurrent losses for the event windows $E_{\x}\left(t_{s_1}, t\right)$ and $E_{\x}\left(t_{s_2}, t\right)$. 

\subsection{As-Isotropic-As-Possible Regularization} 
\label{ssec:IsotropicReg} 

In 3DGS, Gaussians are unconstrained in the direction perpendicular to the image plane. 
This lack of constraint can result in elongated and overfitted Gaussians. 
And while they may appear correct from the training views, they introduce significant artifacts when rendered from novel views by manifesting as floaters and distortions of object surfaces. 
We also observe that the lack of multi-view consistency and tendency to overfit destabilize the pose refinement. 

To mitigate these issues, we draw inspiration from Gaussian Splatting SLAM~\cite{3dgsslam} and SplaTAM~\cite{splatam}, and apply isotropic regularization:
\begin{equation}
    \mathcal{L}_{\text{iso}} = \frac{1}{|\mathcal{G}|} \sum_{g \in \mathcal{G}} \left\| S_g - \bar{S}_g \right\|_1
    \label{eq:L_iso} \mathrm{\,,}
\end{equation}
where~$\mathcal{G}$ is the set of Gaussians visible in the image. Eq.~\eqref{eq:L_iso} imposes a soft constraint on the Gaussians to be as isotropic as possible.
We find that it helps to improve pose refinement, minimizes floaters and enhances generalizability. 

\subsection{Pose Refinement} 
\label{sec:pose_refinement}

To obtain the most accurate results, we allow the poses to be refined during optimization
by modeling the refined pose as $P'_t = P^e_t P_t$, where  $P^e_t$ is an error correction transform. 
Instead of directly optimizing~$P^e_t$ as a~$3 \times 3$ matrix, following Hempel et al.~\cite{6d_rotation} we represent it as $[r_1\,\, r_2\,\, T]$, where $r_1$ and $r_2$ represent two rotation vectors of the rotation matrix~$R = [r_1\,\, r_2\,\, r_3]$, while $T$ is the translation.
We can then obtain the~$P^e_t$ matrix from the representation using Gram-Schmidt orthogonalization (see details in Supplement~\ref{sec:supp_pose_refinement}), hence ensuring that during optimization, our error correction transform always represents a valid transformation matrix. 
$P^e_t$ is initialized to be the identity transform. Since the loss function from Eq.~\eqref{eq:L_recon_per_pixel} depends on the camera pose as well, it allows us to use the same loss to backpropagate and obtain gradients for pose refinement. 

As our goal is to refine the estimated noisy poses rather than perform SLAM, this training signal is sufficient for our needs. Moreover, we observe that poses tend to diverge with 3DGS due to the periodic opacity reset.
To combat this, we impose a soft constraint with an additional pose regularization, that encourages the matrices~$P^e_t$ to stay close to the identity matrix $I$:
\begin{equation}
    \mathcal{L}_{\text{pose}} = \| P^e_{t_{s_1}} - I \|_2 + \| P^e_{t_{s_2}} - I \|_2 + \| P^e_{t} - I \|_2
    \label{eq:L_pose} \mathrm{\,,}
\end{equation}
with all terms weighted equally. 


\subsection{Optimization}
\label{ssec:Optimization} 

Eq.~\eqref{eq:L_recon_per_pixel} defines the reconstruction loss per pixel for a single event frame. However, naively averaging these per-pixel losses over whole images leads to problems. For small event windows, most pixels have no events, which are not very informative but will then make up the majority of the loss. 
To address this, we compute separate averages of the losses for pixels with events~$\mathcal{X}_\text{evs}$ and pixels without events~$\mathcal{X}_\text{noevs}$. 
These averages are then scaled by the hyperparameter~$\alpha=0.3$ to obtain the complete weighted reconstruction loss:
\begin{equation}
    \begin{split}
        \mathcal{L}_{\text{recon}}\left(t_s, t\right) = \,\,&
        \frac{\alpha}{|\mathcal{X}_{\text{noevs}}|} \cdot 
        \left(\sum_{\x\in \mathcal{X}_{\text{noevs}}} \mathcal{L}_{\x}\left(t_s, t\right)\right) + \\
        + \,\,& \,\, \frac{1 - \alpha}{|\mathcal{X}_{\text{evs}}|} \,\,\, \cdot 
        \left(\sum_{\x\in \mathcal{X}_{\text{evs}}} \mathcal{L}_{\x}\left(t_s, t\right)\right). 
    \end{split}
    \label{eq:L_recon}
\end{equation}
To obtain the final loss, we take a weighted sum of the reconstruction losses for the two event windows from Sec.~\ref{subsec:adaptive_window} along with the isotropic and pose regularization: 
\begin{equation}
    \begin{split}
        \mathcal{L} =\,\,\,\, & 
        \lambda_1 \mathcal{L}_{\text{recon}}\left(t_{s_1}, t\right) \,\,+  \,\,
        \lambda_2 \mathcal{L}_{\text{recon}}\left(t_{s_2}, t\right)  \\&
        +\,\, \lambda_\text{iso} \mathcal{L}_{\text{iso}} \,\,+ \,\,
        \lambda_\text{pose} \mathcal{L}_{\text{pose}}
    \end{split}
    \label{eq:loss} \mathrm{\,,}
\end{equation}
where $\lambda_1$, $\lambda_2$ and $\lambda_{\text{iso}}$ are hyper-parameters. In our experiments, we use  $\lambda_1=\lambda_2=0.65$, and $\lambda_{\text{iso}}$ is set to $10$ initially and reduced to $1$ after $\qty{d4}{}$ iterations. 







\section{Numerical Experiments}\label{sec:experiments}



\begin{figure}[t]
    \centering
    \includegraphics[width=0.8\textwidth]{img/regret_vs_iter.pdf}
    \caption{
    Regret scaling for Tsallis-INF and two other bandit algorithms. Each configuration $(T)$ is run for 512 trials. The interval between the 10th and 90th percentile is overlaid. The thicker dashed line represents a linear fit on the $T\geq 10^5$ subset of the log-log data.}
    \label{fig:regret-comparison}
\end{figure}


To validate our theoretical results,
we conduct a few numerical experiments.

The first experiment compares
Tsallis-INF against two baselines in terms of the regret: 
the classical UCB1~\citep{auer2002finite} and Exp3~\citep{auer2002nonstochastic} algorithms, 
which are known to have $O(T)$ and $\tilde{O}(\sqrt{T})$ regret bounds respectively in the adversarial setting.
We compare them on the game associated with $A$ defined in \eqref{eq:example-2x2-game-matrix-simplified},
with varying $T$ and $\eps=T^{-1/3}$,
where feedback $r_t$ follows a Bernoulli distribution over $\{ -1, 1 \}$ such that $ \E[r_t \mid i_t, j_t] = A(i_t, j_t)$.
As discussed, Theorem~\ref{thm:general-bound-together} predicts a regret of $\tilde{O}(T^{1/3})$ for Tsallis-INF.
The result of the experiment agrees with all these bounds in Figure~\ref{fig:regret-comparison},
where the asymptotic slope in the log-log plot (shown with a linear fit on the $T\geq 10^5$ region) is close to the theoretical prediction.


\begin{figure}[t]
    \centering
    \includegraphics[width=0.8\textwidth]{img/identify_P_vs_iterations_by_H1.pdf}
    \caption{
        Experimental validation of Tsallis-INF's PSNE identification capability.
        The plot shows the algorithm's success rate in correctly identifying PSNE
        against the number of itrations.
        We use a hard instance of a $256\times 256$ matrix and $\Delta_1=0.1$,
        running 512 trials for each $\Delta_{\min}$ values
        over a horizon of $128\OPT$ iterations,
        where $\OPT$ is the theoretical lower bound for PSNE identification.
        The $x$-axis is scaled by $1/\OPT$.
        }
    \label{fig:PSNE-id-rate}
\end{figure}

We have discussed in Section~\ref{sec:PSNE_complexity} that Tsallis-INF needs $\frac{\omegar+\omegac}{\Delta_{\min}}$ iterations to identify the PSNE of a game. To validate our theoretical bounds, we conduct our second experiment using the following hard instance  introduced by \citet{maiti2024midsearch}:
\begin{equation}
    A=\begin{bNiceArray}{ccccc}[nullify-dots, margin, custom-line = {letter=I, tikz=dashed}, cell-space-limits = 4pt]
        0 & 2{\Delta_{\min}} & \Block{1-3}{} 2{\Delta_1} &\Cdots& 2{\Delta_1} \\
        -2{\Delta_{\min}} & \Block{4-4}{} 
                       0      & 1      & \Cdots & 1      \\
        \Block{3-1}{}
        -2{\Delta_1} & -1     & \Ddots & \Ddots & \Vdots \\
        \Vdots       & \Vdots & \Ddots & \Ddots & 1      \\
        -2{\Delta_1} & -1     & \Cdots & -1     & 0      \\
    \end{bNiceArray},
    \label{eq:psne-experiment-array}
\end{equation}
where the top-left entry is the PSNE. We set the number of actions $n=m=256$ and the gap $\Delta_1=0.1$, and vary the value of $\Delta_{\min}$.
Let $\OPT$ represent the theoretical optimal bound for identifying PSNE (ignoring log terms), defined as 
$\OPT=
\sum_{i \in [m] \setminus \{ \istar \}} \frac{1}{{\Deltar}^2_i}
+
\sum_{j \in [n] \setminus \{ \jstar \}} \frac{1}{{\Deltac}^2_j}
$,
which simplifies to $\frac{1}{2\Delta_{\min}^2}+\frac{m-2}{2\Delta_1^2}$ in this experiment.
\citepos{maiti2024midsearch} achieve the optimal $\tilde{O}(\OPT)$ sample complexity,
and their Figure~2 suggests that the sample complexity of Tsallis-INF divided by $\OPT$ is unbounded as $\Delta_{\min}$ decreases,
but our analysis in Section~\ref{sec:PSNE_complexity} disagrees with this trend.
As shown in Figure~\ref{fig:PSNE-id-rate},
the number of iterations needed to identify the PSNE divided by $\OPT$
decreases and then increases
as $\Delta_{\min}$ varies.
Lemma~\ref{lem:sqrtk-ratio} predicts the minimum ratio occurs when $\frac{\Delta_{\min}}{\Delta_1}=\frac{1}{\sqrt{m}+1}=1/17$,
and among the values we tested,
the minimum is reached when $\frac{\Delta_{\min}}{\Delta_1}=0.005/0.1=1/20$,
closely matching the prediction.
This supports our derived bound of $\tilde{O}\rbrm[\big]{\sqrt{m}\cdot \OPT}$.

The code for reproducing the experiments is available on 
\url{https://github.com/EtaoinWu/instance-dependent-game-learning}.


\section{Discussion and Conclusion}
\label{sec:discussion}


\textbf{Conclusion.} In this paper, we propose LRM to utilize diffusion models for step-level reward modeling, based on the insights that diffusion models possess text-image alignment abilities and can perceive noisy latent images across different timesteps. To facilitate the training of LRM, the MPCF strategy is introduced to address the inconsistent preference issue in LRM's training data. We further propose LPO, a method that employs LRM for step-level preference optimization, operating entirely within the latent space. LPO not only significantly reduces training time but also delivers remarkable performance improvements across various evaluation dimensions, highlighting the effectiveness of employing the diffusion model itself to guide its preference optimization. We hope our findings can open new avenues for research in preference optimization for diffusion models and contribute to advancing the field of visual generation.

\textbf{Limitations and Future Work.} (1) The experiments in this work are conducted on UNet-based models and the DDPM scheduling method. Further research is needed to adapt these findings to larger DiT-based models \cite{sd3} and flow matching methods \cite{flow_match}. (2) The Pick-a-Pic dataset mainly contains images generated by SD1.5 and SDXL, which generally exhibit low image quality. Introducing higher-quality images is expected to enhance the generalization of the LRM. (3) As a step-level reward model, the LRM can be easily applied to reward fine-tuning methods \cite{alignprop, draft}, avoiding lengthy inference chain backpropagation and significantly accelerating the training speed. (4) The LRM can also extend the best-of-N approach to a step-level version, enabling exploration and selection at each step of image generation, thereby achieving inference-time optimization similar to GPT-o1 \cite{gpt_o1}.


\section*{Impact Statement}
\setlength{\itemsep}{5pt}
\setlength{\topsep}{0pt}
This work introduces a preference optimization method used for text-to-image diffusion models. The method may have the following impacts:
\begin{itemize}[left=0pt]
    \vspace{-1pt}
    \item The reward model plays a crucial role in shaping the preferences of diffusion models. However, if the training data for the reward model contains biases, the optimized model may inherit or even amplify these biases, leading to the generation of stereotypical or discriminatory content about certain groups. To mitigate this risk, it is essential to ensure that the training data is diverse, representative, and fair.
    \item This method offers a way to enhance the quality of generated images in terms of aesthetics, relevance, and other aspects by leveraging a well-designed reward model. It can also be applied to safety domains, optimizing the model to prevent the generation of negative content.
    \item The optimized model can generate highly realistic images, which may be used to create misinformation or misleading content, thereby impacting public opinion and social trust. Therefore, it is essential to develop effective detection tools and mechanisms to identify synthetic content.
\end{itemize}


\bibliography{example_paper}
\bibliographystyle{icml2025}


%%%%%%%%%%%%%%%%%%%%%%%%%%%%%%%%%%%%%%%%%%%%%%%%%%%%%%%%%%%%%%%%%%%%%%%%%%%%%%%
%%%%%%%%%%%%%%%%%%%%%%%%%%%%%%%%%%%%%%%%%%%%%%%%%%%%%%%%%%%%%%%%%%%%%%%%%%%%%%%
% APPENDIX
%%%%%%%%%%%%%%%%%%%%%%%%%%%%%%%%%%%%%%%%%%%%%%%%%%%%%%%%%%%%%%%%%%%%%%%%%%%%%%%
%%%%%%%%%%%%%%%%%%%%%%%%%%%%%%%%%%%%%%%%%%%%%%%%%%%%%%%%%%%%%%%%%%%%%%%%%%%%%%%
\newpage
\appendix
\onecolumn

%%%%%%%%%%%%%%%%%%%%%%%%%%%%%%%%%%%%%%%%%%%%%%%%%%%%%%%%%%%%%%%%%%%%%%%%%%%%%%%
%%%%%%%%%%%%%%%%%%%%%%%%%%%%%%%%%%%%%%%%%%%%%%%%%%%%%%%%%%%%%%%%%%%%%%%%%%%%%%%
% APPENDIX
%%%%%%%%%%%%%%%%%%%%%%%%%%%%%%%%%%%%%%%%%%%%%%%%%%%%%%%%%%%%%%%%%%%%%%%%%%%%%%%
%%%%%%%%%%%%%%%%%%%%%%%%%%%%%%%%%%%%%%%%%%%%%%%%%%%%%%%%%%%%%%%%%%%%%%%%%%%%%%%
\newpage
\appendix
\onecolumn

\section{Related Work} \label{app:related_work}
\textbf{Personalized Generation} 
Due to the considerable success of large text-to-image models \cite{ramesh2022hierarchical, ramesh2021zero, saharia2022photorealistic, rombach2022high}, the field of personalized generation has been actively developed. The challenge is to customize a text-to-image model to generate specific concepts that are specified using several input images. Many different approaches \cite{DB, TI, CD, svdiff, ortogonal, profusion, elite, r1e} have been proposed to solve this problem and can be divided into the following groups: pseudo-token optimization \cite{TI, profusion, disenbooth, r1e}, diffusion fune-tuning \cite{DB, CD, profusion}, and encoder-based \cite{elite}. The pseudo-token paradigm adjusts the text encoder to convert the concept token into the proper embedding for the diffusion model. Such embedding can be optimized directly \cite{TI, r1e} or can be generated by other neural networks \cite{disenbooth, profusion}. Such approaches usually require a small number of parameters to optimize but lose the visual features of the target concept. Diffusion fine-tuning-based methods optimize almost all \cite{DB} or parts \cite{CD} of the model to reconstruct the training images of the concept. This allows the model to learn the input concept with high accuracy, but the model due to overfitting may lose the ability to edit it when generated with different text prompts. To reduce overfitting and memory usage, lightweight parameterizations \cite{svdiff, r1e, lora} have been proposed that preserve edibility but at the cost of degrading concept fidelity. Encoder-based methods \cite{elite} allow one forward pass of an encoder that has been trained on a large dataset of many different objects to embed the input concept. This dramatically speeds up the process of learning a new concept and such a model is highly editable, but the quality of recovering concept details may be low. Generally, the main problem with existing personalized generation approaches is that they struggle to simultaneously recover a concept with high quality and generate it in a variety of scenes.

\textbf{Sampling strategies}
Much research has been devoted to sampling techniques for text-to-image diffusion models, focusing not only on personalized generation but also on image editing. In this paper, we address a more specific question: how can the two trajectories -- superclass and concept -- be optimally combined to achieve both high concept fidelity and high editability? The ProFusion paper \cite{profusion} considered one way of combining these trajectories (Mixed sampling), which we analyze in detail in our paper (see Section \ref{sec:mixed_sampling}) and show its properties and problems. In ProFusion, authors additionally proposed a more complex sampling procedure, which we observed to be redundant compared to Mixed sampling, as can be seen in our experiments (see Section \ref{sec:experiments}). In Photoswap \cite{photoswap}, authors consider another way of combining trajectories by superclass and concept, which turns out to be almost identical to the Switching sampling strategy that we discuss in detail in Section \ref{sec:switching_sampling}. We show why this strategy fails to achieve simultaneous improvements in concept reconstruction and editability. In the paper, we propose a more efficient way of combining these two trajectories that achieves an optimal balance between the two key features of personalized generation: concept reconstruction and editability.

\section{Training details} \label{sec:training-details}
The Stable Diffusion-2-base model is used for all experiments. For the Dreambooth, Custom Diffusion, and Textual Inversion methods, we used the implementation from \url{https://github.com/huggingface/diffusers}.

\textbf{SVDiff} We implement the method based on \url{https://github.com/mkshing/svdiff-pytorch}. The parameterization is applied to all Text Encoder and U-Net layers. The models for all concepts were trained for $1600$ using Adam optimizer with $\text{batch size} = 1$, $\text{learning rate} = 0.001$, $\text{learning rate 1d} = 0.000001$, $\text{betas} = (0.9, 0.999)$, $\text{epsilon} = 1e\!-\!8$, and $\text{weight decay} = 0.01$. 

\textbf{Dreambooth} All query, key, and value layers in Text Encoder and U-Net were trained during fine-tuning. The models for all concepts were trained for $400$ steps using Adam optimizer with $\text{batch size} = 1$, $\text{learning rate} = 2e\!-\!5$, $\text{betas} = (0.9, 0.999)$, $\text{epsilon} = 1e\!-\!8$, and $\text{weight decay} = 0.01$. 

\textbf{Custom Diffusion} The models for all concepts were trained for $1600$ steps using Adam optimizer with $\text{batch size} = 1$, $\text{learning rate} = 0.00001$, $\text{betas} = (0.9, 0.999)$, $\text{epsilon} = 1e\!-\!8$, and $\text{weight decay} = 0.01$. 

\textbf{Textual Inversion} The models for all concepts were trained for $10000$ steps using Adam optimizer with $\text{batch size} = 1$, $\text{learning rate} = 0.005$, $\text{betas} = (0.9, 0.999)$, $\text{epsilon} = 1e\!-\!8$, and $\text{weight decay} = 0.01$. 

\textbf{ELITE} We used the pre-trained model from the official repo \url{https://github.com/csyxwei/ELITE} with $\lambda=0.6$ and inference hyperparams from the original paper.

\clearpage
\section{Superclass and concept trajectory choice}\label{app:hyper_theta}

 \begin{wrapfigure}{r}{0.45\textwidth}
    % \centering
    \includegraphics[trim={3cm 10cm 3cm 10cm},clip,width=\linewidth]{imgs/mixed_noft_nosup.pdf}
    \caption{The Pareto frontiers for original Mixed sampling and Mixed sampling in the Superclass, NoFT, and Empty Prompt setups. Mixed NoFT and Mixed Empty Prompt configurations overlap with the Pareto frontier of the original mixed sampling, but primarily in regions associated with low image similarity, which compromises concept fidelity.} \label{fig:mixed_noft_ep}
    \vspace{-0.14in}
\end{wrapfigure}

There are multiple ways to define sampling with maximized textual alignment to the prompt. However, the arbitrary choice can harm the alignment between Base sampling~\ref{eq:concept_sampling} and the selected trajectory. We use the Sampling with superclass (\ref{eq:superclass_sampling}) as it's the default choice in the literature and guarantees the maximized alignment between noise predictions $\tilde{\varepsilon}_{\theta}(p^C)$ and $\tilde{\varepsilon}_{\theta}(p^S)$. 

The several natural ways to adjust Sampling with superclass can be presented by varying $\theta$ and $p^{S}$ in (\ref{eq:superclass_sampling}). We explore two additional options with decreased alignment with (\ref{eq:concept_sampling}): (1) NoFT -- weights of base model $\theta^{\text{orig}}$ instead fine-tuned weights, (2) Empty Prompt -- prompt without any reference to a concept, even to its superclass category, i.e. $p^{\hat{S}} = \textit{"with a city in the background"}$ instead of $p^{S} = \textit{"a backpack with a city in the background"}$.

To validate the robustness of our framework for sampling method selection, we employ the original experimental protocol, supplementing the results shown in Figures~\ref{fig:examples} and~\ref{fig:profusion-photoswap}. Our analysis of Figures~\ref{fig:multi-stage_noft_ep} and~\ref{fig:masked_noft_ep} reveals that trajectories generated under the NoFT and Empty Prompt configurations (second and third columns, respectively) maintain identical method ordering to those produced by Superclass sampling ((\ref{eq:superclass_sampling}), first column).

Notably, Figure~\ref{fig:mixed_noft_ep} shows that Empty Prompt configuration demonstrates weaker alignment with Base sampling compared to NoFT, particularly at higher values of the superclass guidance scale $\omega_{s}$. This divergence manifests as reduced concept fidelity for Empty Prompt under large $\omega_{s}$. These findings highlight a practical adjustment: prioritizing smaller $\omega_{s}$ values in Empty Prompt setup preserves concept fidelity without altering the framework’s core selection logic. 

A key limitation of increased misalignment is the gradual erosion of superclass category information from generated images, which can lead to semantically inconsistent outputs. For instance, Figure~\ref{fig:examples_noft_ep} illustrates how the Mixed Empty Prompt setup, despite the strong animal prior in Base sampling, can produce human-like features in an image of a cat described as \textit{"in a chef outfit"}. This suggests that when superclass information is weakened, the model may introduce unexpected visual artifacts, impacting the fidelity of the intended concept.

Concept sampling (\ref{eq:concept_sampling}) can also be adjusted to better capture a concept’s visual characteristics, further decoupling fidelity from editability. For example, this can be achieved by (1) using the weights of a highly overfitted model (e.g., DreamBooth) or (2) selecting a prompt that omits contextual details, such as $p^{\hat{C}} = \textit{"a photo of V*"}$ instead of $p^{C} = \textit{"a V* with a city in the background"}$. Combining superclass sampling under NoFT or Empty Prompt with Base sampling configured via (1) or (2) could enhance both image and text similarity. We leave this direction for future work.

\begin{figure}[b]
    \centering
    \includegraphics[trim={3cm 10cm 3cm 10cm},clip,width=0.32\linewidth]{imgs/multi-stage_original.pdf}
    \hfill
    \includegraphics[trim={3cm 10cm 3cm 10cm},clip,width=0.32\linewidth]{imgs/multi-stage_noft.pdf}
    \hfill
    \includegraphics[trim={3cm 10cm 3cm 10cm},clip,width=0.32\linewidth]{imgs/multi-stage_nosup.pdf}
    \caption{Pareto Frontier Curves for Mixed, Switching, and Multi-Stage Sampling Methods in the Superclass, NoFT and Empty Prompt setups.
The NoFT and Empty Prompt configurations (second and third columns, respectively) preserve the same method ordering as those produced by Superclass sampling (first column).} \label{fig:multi-stage_noft_ep}
\end{figure}
\begin{figure}[t]
    \centering
    \includegraphics[trim={3cm 10cm 3cm 10cm},clip,width=0.32\linewidth]{imgs/masked_profusion.pdf}
    \hfill
    \includegraphics[trim={3cm 10cm 3cm 10cm},clip,width=0.32\linewidth]{imgs/masked_noft.pdf}
    \hfill
    \includegraphics[trim={3cm 10cm 3cm 10cm},clip,width=0.32\linewidth]{imgs/masked_nosup.pdf}
    \caption{Pareto Frontier Curves for Mixed, Switching, Masked, and ProFusion Sampling Methods in the Superclass, NoFT, and Empty Prompt setups.
The NoFT and Empty Prompt configurations (second and third columns, respectively) preserve the same method ordering as those produced by Superclass sampling (first column).} \label{fig:masked_noft_ep}
\end{figure}

\begin{figure}[b]
    \centering
    \includegraphics[width=\linewidth]{imgs/examples_noft_ep.pdf}
    \caption{Examples of the generation outputs for Mixed and ProFusion sampling methods for their optimal metrics point in the Superclass, NoFT, and Empty Prompt (EP) setups.} \label{fig:examples_noft_ep}
\end{figure}

\clearpage

\begin{figure}[ht!]
  \centering
  \includegraphics[trim={0 5cm 0 5cm},clip,width=0.95\linewidth]{imgs/us_example_new.pdf}
  \caption{An example of a task in the user study}
  \label{fig:us_ex}
  \vspace{-0.19in}
\end{figure}

\section{Data preparation}\label{app:data}
For each concept, we used inpainting augmentations to create the training dataset. We took an original image and automatically segmented it using the Segment Anything model on top of the CLIP cross-attention maps. Then we crop the concept from the original image, apply affine transformations to it, and inpaint the background. We used $10$ augmentation prompts, different from the evaluation prompts, and sampled $3$ images per prompt, resulting in a total of $30$ training images per concept. We commit to open-source the augmented datasets for each concept after publication.

\section{User Study}\label{app:us}

An example task from the user study is shown in Figure~\ref{fig:us_ex}. In total, we collected 48,864 responses from 200 unique users for 16,000 unique pairs. For each task, users were asked three questions: 1) "Which image is more consistent with the text prompt?" 2) "Which image better represents the original image?" 3) "Which image is generally better in terms of alignment with the prompt and concept identity preservation?" For each question, users selected one of three responses: "1", "2", or "Can't decide."

\section{Complex Prompts Setting}\label{app:long_prompts}

We conduct a comparison of different sampling methods using a set of complex prompts. For this analysis, we collected 10 prompts, each featuring multiple scene changes simultaneously, including stylization, background, and outfit:

\adjustbox{max width=\linewidth}{
\begin{lstlisting}
live_long = [
  "V* in a chief outfit in a nostalgic kitchen filled with vintage furniture and scattered biscuit",
  "V* sitting on a windowsill in Tokyo at dusk, illuminated by neon city lights, using neon color palette",
  "a vintage-style illustration of a V* sitting on a cobblestone street in Paris during a rainy evening, showcasing muted tones and soft grays",
  "an anime drawing of a V* dressed in a superhero cape, soaring through the skies above a bustling city during a sunset",
  "a cartoonish illustration of a V* dressed as a ballerina performing on a stage in the spotlight",
  "oil painting of a V* in Seattle during a snowy full moon night",
  "a digital painting of a V* in a wizard's robe in a magical forest at midnight, accented with purples and sparkling silver tones",
  "a drawing of a V* wearing a space helmet, floating among stars in a cosmic landscape during a starry night",
  "a V* in a detective outfit in a foggy London street during a rainy evening, using muted grays and blues",
  "a V* wearing a pirate hat exploring a sandy beach at the sunset with a boat floating in the background",
]

object_long = [
  "a digital illustration of a V* on a windowsill in Tokyo at dusk, illuminated by neon city lights, using neon color palette",
  "a sketch of a V* on a sofa in a cozy living room, rendered in warm tones",
  "a watercolor painting of a V* on a wooden table in a sunny backyard, surrounded by flowers and butterflies",
  "a V* floating in a bathtub filled with bubbles and illuminated by the warm glow of evening sunlight filtering through a nearby window",
  "a charcoal sketch of a giant V* surrounded by floating clouds during a starry night, where the moonlight creates an ethereal glow",
  "oil painting of a V* in Seattle during a snowy full moon night",
  "a drawing of a V* floating among stars in a cosmic landscape during a starry night with a spacecraft in the background",
  "a V* on a sandy beach next to the sand castle at the sunset with a floaing boat in the background",
  "an anime drawing V* on top of a white rug in the forest with a small wooden house in the background",
  "a vintage-style illustration of a V* on a cobblestone street in Paris during a rainy evening, showcasing muted tones and soft grays",
]
\end{lstlisting}
}

The results of this comparison are presented in Figures~\ref{fig:add_long},~\ref{fig:add_long_metrics}. We observe that Base sampling may struggle to preserve all the features specified by the prompts, whereas advanced sampling techniques effectively restore them. The overall arrangement of methods in the metric space closely mirrors that observed in the setting with simple prompts.

\begin{figure}[h!]
  \centering
  \includegraphics[width=\linewidth]{imgs/long_prompts_examples.pdf}
  \caption{Additional examples of the generation outputs for different sampling methods with \textbf{complex prompts}. We highlight parts of the prompt that are missing in Base sampling while appearing in other methods.}
  \label{fig:add_long}
\end{figure}

\clearpage
\section{Dreambooth results}\label{app:dreambooth}

We conduct additional analysis of different sampling methods in combination with Dreambooth. Figure~\ref{fig:add_db_metrics} shows that Mixed Sampling still overperforms Switching and Photoswap,  while Multi-stage and Masked struggle to provide an additional improvement over the simple baseline. Figure~\ref{fig:add_db} shows that all methods allow for improvement TS with a negligent decrease in IS while Mixed Sampling provides the best IS among all samplings.

\begin{figure*}[!ht]
\centering
\begin{minipage}{.477\textwidth}
  \centering
  \includegraphics[trim={3cm 10cm 3cm 10cm},clip,width=\linewidth]{imgs/long_prompts.pdf}
  \caption{CLIP metrics for different sampling methods estimated on \textbf{complex prompts}.}
  \label{fig:add_long_metrics}
\end{minipage}
\hfill
\begin{minipage}{.477\textwidth}
  \centering
  \includegraphics[trim={3cm 10cm 3cm 10cm},clip,width=\linewidth]{imgs/db_samplings.pdf}
  \caption{CLIP metrics for different sampling strategies on top of a Dreambooth fine-tuning method.}
  \label{fig:add_db_metrics}
\end{minipage}
\end{figure*} 

\begin{figure}[h!]
  \centering
  \includegraphics[trim={0 1cm 0 1cm},clip,width=\linewidth]{imgs/db_sampling_examples.pdf}
  \caption{Additional examples of the generation outputs for different sampling methods on top of a Dreambooth fine-tuning method.}
  \label{fig:add_db}
\end{figure}

\section{PixArt-alpha \& SD-XL}\label{app:add_backbones}
We conducted a series of experiments using different backbones. For SD-XL~\cite{podell2023sdxlimprovinglatentdiffusion}, we used SVDDiff as the fine-tuning method, while PixArt-alpha~\citep{chen2023pixartalphafasttrainingdiffusion} employed standard Dreambooth training. Hyperparameters for Switching, Masked, and ProFusion were selected in the same manner as in the experiments with SD2.

Figures~\ref{fig:pixart} and~\ref{fig:sdxl} demonstrate that Mixed Sampling follows a similar pattern to SD2, improving TS without a significant loss in IS. Notably, Mixed Sampling for SD-XL achieves simultaneous improvements in both IS and TS. ProFusion exhibits behavior consistent with SD2, enhancing IS more effectively than Mixed Sampling but performing worse at improving TS while also requiring twice the computational resources. 

\begin{figure}[h]
\centering
\begin{minipage}{.49\textwidth}
  \centering
  \includegraphics[trim={3cm 10cm 3cm 10cm},clip,width=\linewidth]{imgs/pixart.pdf}
  \captionof{figure}{CLIP metrics for different sampling methods estimated on PixArt model.}
  \label{fig:pixart}
\end{minipage}%
\hfill
\begin{minipage}{.49\textwidth}
  \centering
  \includegraphics[trim={3cm 10cm 3cm 10cm},clip,width=\linewidth]{imgs/sdxl.pdf}
  \captionof{figure}{CLIP metrics for different sampling methods estimated on SD-XL model.}
  \label{fig:sdxl}
\end{minipage}
\end{figure}

\clearpage
\section{Cross-Attention Masks}\label{app:cross_attn}

\begin{figure}[h!]
  \centering
  \includegraphics[trim={3cm 0cm 3cm 0cm},clip,width=\linewidth]{imgs/cross_attention_masks.pdf}
  \caption{Visualization of the cross-attention masks for Masked sampling examples. Here, $q$ defines the thresholding quantile and $t$ the denoising step.}
  \label{fig:cross_attn_add_ex}
\end{figure}

\clearpage
\section{Additional Examples}\label{app:add_example}

\begin{figure}[h!]
  \centering
  \includegraphics[trim={0 2cm 0 2cm},clip,width=\linewidth]{imgs/additional_examples.pdf}
  \caption{Additional examples of the generation outputs for different sampling methods.}
  \label{fig:add_ex}
\end{figure}

\begin{figure}[h!]
  \centering
  \includegraphics[trim={0 2cm 0 2cm},clip,width=\linewidth]{imgs/additional_examples_all.pdf}
  \caption{Additional examples of the generation outputs for Mixed and ProFusion sampling methods in comparison to the main personalized generation baselines.}
  \label{fig:add_ex_all}
\end{figure}

\clearpage
\section{DINO Image Similarity}\label{app:add_dino}

We compare CLIP-IS (left column) and DINO-IS~\citep{oquab2024dinov2learningrobustvisual} (right column) in Figures~\ref{fig:profusion_photoswap_dino},~\ref{fig:all_methods_dino}. We observe that despite the choice of metric, different sampling techniques and finetuning strategies have the same arrangement. The most noticeable difference is that SVDDiff superiority over ELITE and TI is more pronounced. That strengthens our motivation to select SVDDiff as the main backbone.

\begin{figure}[h]
\centering
\begin{minipage}{.49\textwidth}
  \centering
  \includegraphics[trim={3cm 10cm 3cm 10cm},clip,width=\linewidth]{imgs/profusion_photoswap.pdf}
\end{minipage}%
\hfill
\begin{minipage}{.49\textwidth}
  \centering
  \includegraphics[trim={3cm 10cm 3cm 10cm},clip,width=\linewidth]{imgs/profusion_photoswap_dino.pdf}
\end{minipage}
\caption{Pareto frontiers curves for Photoswap~\citep{photoswap} and ProFusion~\citep{profusion}.}\label{fig:profusion_photoswap_dino}
\end{figure}

\begin{figure}[h]
\centering
\begin{minipage}{.49\textwidth}
  \centering
  \includegraphics[trim={3cm 10cm 3cm 10cm},clip,width=\linewidth]{imgs/all_methods.pdf}
\end{minipage}%
\hfill
\begin{minipage}{.49\textwidth}
  \centering
  \includegraphics[trim={3cm 10cm 3cm 10cm},clip,width=\linewidth]{imgs/all_methods_dino.pdf}
\end{minipage}
\caption{The overall results of different sampling methods against main personalized generation baselines.}\label{fig:all_methods_dino}
\end{figure}




\end{document}


