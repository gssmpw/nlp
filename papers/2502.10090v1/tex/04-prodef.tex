\section{Problem Formulation}
\label{sec:method-formulation}
Given a complete set of 3D assembly parts and its assembly manual, our goal is to generate a physically feasible sequence of robotic assembly actions for autonomous furniture assembly. Manuals typically use schematic diagrams and symbols designed to depict step-by-step instructions in an abstract format that is universally understandable. We define the manual pages as a set of $N$ images. $\mathcal{I} = \{I_1, I_2, \cdots, I_N\}$, where each image $I_i$ illustrates a specific step in the assembly process, such as the merging of certain parts or subassemblies

The furniture consists of $M$ individual parts $\mathcal{P} = \{P_1, P_2, \cdots, P_M\}$. A \emph{part} is an individual element in $\mathcal{P}$ that remains disconnected from other parts until assembly. A \emph{subassembly} is any partially or fully assembled structure that forms a proper subset of $\mathcal{P}$ (for example, $\{P_1, P_2\}$). The term \emph{component} encompasses both parts and subassemblies.

Given the manual and 3D parts, the system generates an assembly plan. Each step corresponds to a manual image and specifies the involved parts and sub-assemblies, their spatial 6D poses, and the assembly actions or motion trajectories required for execution. 