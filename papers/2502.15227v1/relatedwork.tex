\section{Related Works}
\subsection{The Cause of Cybersickness}
Since Stanney et al. \cite{stanney_aftereffects_1998} identified the need to develop a better theoretical understanding of cybersickness, 
different models concerning sensory conflict, postural instability, etc., are in development, although this goal has yet to be fully achieved \cite{stanney_identifying_2020}. 
The sensory conflict theory claims that the cause of cybersickness is the discrepancy between the visual and inertial information.
The optical flow generated by virtual locomotion in VR specifies acceleration that does not correspond to the inertial force changes users can perceive \cite{lackner_motion_2014, oman_motion_1990, laviola_discussion_2000}.
On the other hand, the postural instability theory attributes cybersickness to prolonged exposure to postural instability \cite{riccio_ecological_1991}.
In VR, optical flow does not fully correspond to the vestibular/inertial information in accordance with physical laws, creating difficulties for the user in stabilizing their posture.
Researchers have had success using postural data (sometimes in combination with other physiological signals) to predict cybersickness 
%\dd{\cite{islam_cybersickness_2021, islam_towards_2022, wang_vr_2019, cortes_eeg-based_2023, monteiro_using_2021}}
\cite{islam_cybersickness_2021, cortes_eeg-based_2023}. 
Although these two hypotheses disagree with each other in many aspects, both of them agree that 
%\dd{an earth-based rest frame }
Earth referents, which means elements of VR displays that are stable relative to the Earth \cite{bailey_using_2022},
should mitigate cybersickness \cite{stanney_identifying_2020, bailey_using_2022}.

\subsection{Measuring Cybersickness}

Traditionally, researchers test the cybersickness level that a user experience with subjective questionnaires.
However, objective cybersickness measures are still under exploration.

\subsubsection{Subjective Measures}
The simulator sickness questionnaire (SSQ) is widely adopted to quantify the cybersickness elicited by virtual reality \cite{kennedy_simulator_1993}. 
Although SSQ can capture the symptoms that users experience at the end of the experiment, it does not provide information about users' discomfort level in-between the start and the end of exposure.
Repeating SSQ takes a significant amount of time and could lead to higher reported symptoms due to demanding characteristics \cite{ellis_demand_2007, bimberg_usage_2020}.
The Fast Motion Sickness Score (FMS) is an easy-to-administer, finely graded motion sickness measure where participants report their level of motion sickness on a scale from 0 (no sickness at all) to 20 (frank sickness) \cite{keshavarz_validating_2011}. 
FMS can be taken repeatedly throughout the exposure to a provocative stimulus so that the time course of motion sickness can be captured.
Fernandes and Feiner adapted the discomfort score developed by Rebenitsch and Owen, where participants were asked to rate their discomfort level from 0 (how they felt coming in) to 10 (severe discomfort) \cite{fernandes_combating_2016, rebenitsch_individual_2014}. 
Each participant's discomfort score ratings can be used to calculate their average discomfort score (ADS) and relative discomfort score (RDS).
RDS captures the participant's relative performance if they terminate early \cite{fernandes_combating_2016}, and has been widely used by researchers (e.g., \cite{cao_visually-induced_2018, wu_dont_2021, norouzi_assessing_2018}). While subjective measures are widely used in assessing cybersickness, they have several limitations that impact their reliability, including their dependence on the user's ability to judge and recall their discomfort, and their inability to capture real-time fluctuations in discomfort and sickness levels.


\subsubsection{Objective Measures}
Objective measures aim to provide implicit, real-time, and continuous data on a user's levels of discomfort and cybersickness.  
Previous studies have identified various physiological signals that can serve as objective indicators of cybersickness symptoms. 
Among these, gastric activity measured by electrogastrogram (EGG) \cite{dennison_use_2016}, 
heart rate or heart rate variability measured by electrocardiogram (ECG) or Photoplethysmogram (PPG), 
brain activity measured by electroencephalogram (EEG) or functional magnetic resonance imaging (fMRI) \cite{tian_who_2024, mimnaugh_virtual_2023}, % cite 
electrodermal activity (EDA) \cite{dennison_use_2016},   
eye movement behavior \cite{lopes_2020},  
and postural activity data 
%\dd{\cite{arcioni_postural_2019, palmisano_predicting_2018, risi_effects_2019, stoffregen_postural_1998}}
\cite{stoffregen_postural_1998}
have been explored as potential indicators of cybersickness.
Recent approaches have also used deep learning methods to analyze information about the visual scene, such as 3D motion flow, to estimate cybersickness symptoms \cite{zhao_2023}. 
Additionally, other approaches have integrated deep learning techniques with both physiological data and visual scene information to estimate and forecast cybersickness \cite{islam_cybersickness_2021}.
Despite their potential, objective measures are still not widely used due to their complexity, cost, and technical challenges in physiological data collection and analysis.


\subsection{Mitigating Cybersickness}
\subsubsection{Overview}
Researchers have proposed various techniques to mitigate cybersickness in VR (see Ang and Quarles \cite{ang_reduction_2023} for a review).
Most of the software-based techniques, such as the dynamic FOV restrictor, rest frame, and peripheral blurring, try to reduce peripheral optical flow.
Peripheral vision plays an important role in motion perception because the human visual system is more sensitive to motion in the peripheral retina \cite{stoffregen_flow_1985, lishman_vision_1981}, and the magnitude of optical flow is larger in the outer region of the image space \cite{gibson_parallax_1955}.

\subsubsection{Dynamic FOV Restriction}
The dynamic FOV restrictor, which reduces the FOV during virtual locomotion, is the most widely researched cybersickness mitigation technique \cite{bolas_dynamic_2014}. 
Many studies have confirmed its effectiveness in reducing cybersickness 
%\dd{\cite{fernandes_combating_2016, zhao_mitigation_2022, wu_dont_2021, wu_asymmetric_2022, wu_adaptive_2022, adhanom_effect_2020, bala_dynamic_2021}.}
(e.g., \cite{fernandes_combating_2016, zhao_mitigation_2022, wu_adaptive_2022}).
% Start with name
%\dd{Fernandes and Feiner, and Zhao et al.'s studies reduced the FOV to $80^\circ$ \cite{fernandes_combating_2016, zhao_mitigation_2022}, while other studies used $55^\circ$ \cite{wu_adaptive_2022, wu_asymmetric_2022, adhanom_effect_2020}.  Furthermore, Bala et al., Al Zayer et al., and Zielasko et al. have conducted studies with smaller FOVs between $40^{\circ}$ and $50^{\circ}$ \cite{bala_dynamic_2021, al_zayer_effect_2019, zielasko_dynamic_2018}.}
Previous studies on FOV restrictors used different FOVs, ranging from $80^\circ$ \cite{fernandes_combating_2016, zhao_mitigation_2022} to $55^\circ$ \cite{wu_adaptive_2022, adhanom_effect_2020}, or even smaller \cite{bala_dynamic_2021, al_zayer_effect_2019}.
Researchers have proposed different restrictor designs, including foveated restrictor \cite{adhanom_effect_2020}, fixed shape asymmetrical restrictors \cite{wu_asymmetric_2022, wu_dont_2021}, and optical-flow-reducing restrictors \cite{wu_adaptive_2022, bala_dynamic_2021}.
Others also evaluated its effect on spatial learning \cite{adhanom_field--view_2021} and navigation performance \cite{al_zayer_effect_2019}. 

\begin{figure*}[t]
    \centering
    \includegraphics[width = \textwidth, trim = 0px 330px 610px 50px, clip]{figures/PT-Effect.pdf}
    \caption{An overview of the peripheral teleportation technique. 
    Two extra rest frame cameras $RF_0$ and $RF_1$ were rendering images beside the main camera. 
    Their positions in the VE are illustrated in (f).
    The red arrow denotes the moving direction of the user.
    Figures (a-c) are the output of the main camera, $RF_0$, and $RF_1$.
    The rest frame (d) is a linear interpolation of images rendered by $RF_0$ and $RF_1$ (b, c).
    Finally, peripheral teleportation replaces the peripheral region of (a) with the rest frame (d)'s peripheral region, creating an output like figure (e).
    }
    \label{fig:pt_space}
\end{figure*}

\subsubsection{Rest Frames}
When the user is walking physically in VR, the artificial optical information is congruent with the user's physical motion.
In this case, people do not tend to get sick \cite{chance_locomotion_1998}. 
As a result, researchers have used rest frames, which are portions of the VE that remain fixed in relation to the real world, to mitigate cybersickness.
Most of the proposed RFs are fixed to the Earth.
% Start with name
%\dd{Duh et al. found a stationary visual background could reduce postural disturbance induced by virtual motion \cite{duh_independent_2001}.  Cao et al. found a static rest frame that is always on can significantly reduce cybersickness, while a dynamic one cannot \cite{cao_visually-induced_2018}.  Moreover, Cao et al. found granulated rest frames have better user performance on a visual search task \cite{cao_granulated_2021}.  Bala et al added a grid to the periphery but did not find significant cybersickness reduction \cite{bala_visually_2018}.  Wu and Suma Rosenberg proposed adding a wire frame of the user's physical space in the periphery to mitigate cybersickness and protect user safety in a poster \cite{wu_combining_2019}.}
A stationary visual background reduced postural disturbance induced by virtual motion \cite{duh_independent_2001}. A static metal net rest frame significantly reduced cybersickness, whereas a dynamic one did not \cite{cao_visually-induced_2018}. Researchers have also proposed other improved designs, such as adding random noise-like grains \cite{cao_granulated_2021} or a wireframe representation of the user's physical space in the periphery \cite{wu_combining_2019}, to enhance visual search task performance or ensure user safety.
However, some of the other proposed RFs are fixed to the head.
% Start with name
%\dd{Wienrich et al. found a virtual nose can significantly reduce cybersickness \cite{wienrich_virtual_2018}. Ang and Quarles proposed `AuthenticNose' with a more realistic appearance \cite{ang_gingervr_2020}. Shi et al. compared a head-fixed RF with FOV restriction and blurring but did not find a significant effect \cite{shi_virtual_2021}.}
Previously, it was recognized that a virtual nose can significantly reduce cybersickness \cite{wienrich_virtual_2018}. However, a recent experiment with a larger sample size failed to replicate this finding \cite{yip_preregistered_2024}. An alternative design, called \textit{AuthenticNose}, features a more realistic appearance \cite{ang_gingervr_2020}. A comparative experiment did not show any significant effect between a head-fixed RF, FOV restriction, and blurring \cite{shi_virtual_2021}.
The Circle effect proposed by Buhler et al. in a poster shares some similarities with the proposed peripheral teleportation technique by adding a stationary camera's view into the periphery \cite{buhler_reducing_2018}.
However, they did not teleport the stationary camera and chose to blend in the main camera's image instead, which introduced undesirable optical flow.
Additionally, their user study only involved 18 participants and did not show any significant difference in cybersickness reduction.



\subsubsection{Other Techniques}
Other than the techniques mentioned above, researchers have used other solutions to mitigate cybersickness.
Software-based techniques include blurring \cite{lin_how_2020, nie_analysis_2020}, contrast reduction \cite{zhao_mitigation_2022}, reverse optical flow \cite{park_mixing_2022, buhler_reducing_2018, xiao_augmenting_2016}, path modification \cite{hu_reducing_2019}, and geometry modification \cite{groth_cybersickness_2024, nie_like_2023, lou_geometric_2022}.
Hardware-based techniques include galvanic vestibular stimulation \cite{groth_omnidirectional_2022, sra_adding_2019}, vibration \cite{peng_walkingvibe_2020, jung_floor-vibration_2021} and haptic feedback \cite{liu_phantomlegs_2019}.
Distraction \cite{venkatakrishnan_effects_2023, venkatakrishnan_effects_2024}, action \cite{lin_intentional_2022}, or heartbeat feedback \cite{joo_effects_2024} can also reduce discomfort.
However, there was a lack of a comparison between the software-based techniques that produced significant results.
Some of the techniques, like reverse flow visualization or geometry modification, can introduce undesirable visual effects in the fovea.
Additionally, the adoption of hardware-based techniques is limited because they require extra devices.



\subsection{Discrete Movement}
First mentioned by Mine \cite{mine_virtual_1995}, teleportation is a target-based VR locomotion interface that involves discrete viewpoint movement and generally does not induce cybersickness compared to continuous locomotion \cite{adhikari_integrating_2022, rahimi_scene_2020, prithul_teleportation_2021, langbehn_evaluation_2018}.
% Start with name
%\dd{However, Bowman et al. were the first to evaluate teleportation alongside other locomotion techniques and found rapid view change disoriented users \cite{bowman_travel_1997}.  Kelley et al. found that rotational self-motion cues are important to spatial updating during teleportation \cite{kelly_teleporting_2020}.}
However, during teleportation, the rapid view change disoriented users \cite{bowman_travel_1997} because self-motion cues are important to spatial updating \cite{kelly_teleporting_2020}.
Researchers have proposed various solutions to improve the spatial awareness of teleportation (see Prithul et al. \cite{prithul_teleportation_2021} for a review).
% Start with name
%\dd{Weissker et al. proposed a range-restricted variant of teleportation called \textit{jumping} \cite{weissker_spatial_2018}. Rahimi et al. proposed pulsed interpolation, which shows a sequence of viewpoints along the path, to serve as a middle ground between teleportation and continuous locomotion. However, pulsed interpolation did not perform significantly better than teleportation in the spatial task \cite{rahimi_scene_2020}.}
\textit{Jumping} is a range-restricted variant of teleportation \cite{weissker_spatial_2018}. Pulsed interpolation shows a sequence of viewpoints along the path to serve as a middle ground between teleportation and continuous locomotion. However, it did not perform significantly better than teleportation in the spatial task \cite{rahimi_scene_2020}.
% Start with name
%\dd{Adhikari et al. proposed HyperJump by adding teleportation to continuous locomotion. HyperJump merges benefits from both techniques and allows faster locomotion compared to continuous locomotion \cite{adhikari_integrating_2022}. Sun et al. and Langbehn et al. explored subconsciously adding small changes to the viewpoint transform by utilizing saccade- or blink-induced suppression \cite{sun_towards_2018, langbehn_blink_2018}. Freiwald et al. proposed Smart Avatars and Stuttered Locomotion to enhance teleportation. Smart Avatars use avatar visualization to provide knowledge of the path after teleportation. Stuttered Locomotion keeps teleporting the user when they initiate locomotion \cite{freiwald_continuity_2022}. Zielasko et al. investigated different snap turning techniques and found that selection-based methods outperform directional methods in a search task \cite{zielasko_systematic_2022}.}
HyperJump adds teleportation to continuous locomotion, which merges benefits from both techniques and allows faster locomotion compared to continuous locomotion \cite{adhikari_integrating_2022}. 
Subconscious adjustments to the viewpoint transform during saccades or blinks are effective strategies for increasing gains during redirected walking \cite{sun_towards_2018, langbehn_blink_2018}. To enhance teleportation, techniques such as Smart Avatars and Stuttered Locomotion have been proposed, where the former provides path awareness through avatar visualization, and the latter repeatedly teleports users upon locomotion initiation \cite{freiwald_continuity_2022}. Comparative analyses of snap turning techniques revealed that selection-based methods outperform directional methods in search tasks \cite{zielasko_systematic_2022}.



\begin{figure*}
  \centering
  \includegraphics[width=\textwidth, trim = 0px 328px 165px 20px, clip]{figures/PT-RestrictorShape.pdf}
  \caption{
    During translation, the appearance of a symmetric FOV mask in the black FOV restrictor condition (a) and the peripheral teleportation condition (c) is a circle. 
    During turning, we used an asymmetric mask that shifted its center into the turn (b, d). 
    All images were for the left eye.
  }
  \label{fig:pt_shape}
\end{figure*}