\section{Background and Related Work}
% Challenges in stream learning and why existing solutions are not sufficient
Machine learning for data streams presents unique challenges that distinguish it from traditional batch learning. Unlike static datasets, streaming data arrives continuously and must be processed incrementally without storing the entire dataset. 
% This requires models that are both \textbf{computationally efficient} and \textbf{adaptable} to changing data distributions. 
Two key challenges in such scenarios are \textbf{concept drift}, where the statistical properties of the data change over time, requiring models to adjust dynamically to maintain predictive performance, and \textbf{computational constraints}, as models must process each instance upon arrival while operating under strict memory and processing limitations. To effectively address these challenges, practitioners and researchers need specialized tools for evaluating, analyzing, and experimenting with various algorithms.

% Discussion of existing frameworks and their limitations
Several frameworks have been developed to address online learning~\footnote{The terms online learning and stream learning are often used interchangeably, but they are not strictly identical. Online learning generally refers to models that update incrementally, instance by instance, without storing past data. Stream learning further emphasizes the dynamic nature of data streams, including challenges such as concept drift and strict computational constraints. In this paper, we use the terms contextually depending on the focus of discussion.}, the most popular ones being MOA____, River____ and Scikit-Multiflow____. 
% the most popular ones are 
% \begin{inparaitem}
%     \item MOA ____,
%     \item River____, and
%     \item Scikit-Multiflow____.
% \end{inparaitem}
% MOA ____, River____, and Scikit-Multiflow____. 

\textbf{MOA} is a well-established Java-based framework with a large range of efficient implementations, but its Java-based code limits accessibility for new users that often prefer Python. \textbf{River} is a fully Python-based framework that focuses on usability and flexibility but sacrifices computational efficiency due to its pure Python implementation. \textbf{Scikit-Multiflow} was initially developed to provide Python implementations of MOA algorithms but has since been largely replaced by River. In summary, while MOA remains the most efficient choice for large-scale stream learning, it lacks the accessibility of a Python-native tool, whereas River prioritizes usability but struggles with performance, highlighting a trade-off between efficiency and accessibility in existing frameworks. 

% Summary of CapyMOA’s key advantages (efficiency, interoperability, accessibility)
CapyMOA addresses these limitations by providing an efficient, extensible, and user-friendly framework for stream learning. It is built upon three core pillars: 
% \textbf{Efficiency}, taking advantage of efficient Java implementations through a modern Python with minimal memory overhead; \textbf{Interoperability}, with support for both native Python implementations and seamless access to algorithms from MOA and PyTorch, facilitating hybrid learning approaches; and \textbf{Accessibility}, through an intuitive Python API that simplifies stream learning tasks such as handling concept drift, configuring online pipelines, and evaluating models incrementally.
\textbf{Efficiency}, leveraging optimized Java implementations with minimal memory overhead in a modern Python framework; \textbf{Interoperability}, supporting both native Python implementations and seamless integration with MOA and PyTorch for hybrid learning; and \textbf{Accessibility}, providing an intuitive Python API that simplifies complex stream learning tasks, such as concept drift detection, simulation, and evaluation. 

% Candidate for removal if space is short
% By bridging the gap between performance and usability, CapyMOA allows researchers and practitioners to build scalable and adaptive machine learning systems for streaming data.


% \begin{itemize}
%     \item \textbf{Efficiency}: CapyMOA is optimized for high-speed processing of streaming data, ensuring minimal memory footprint and fast execution time.
%     \item \textbf{Interoperability}: The library provides seamless access to MOA implementations in Java while allowing users to work with Python, integrating easily with other popular machine learning frameworks such as scikit-learn and PyTorch.
%     \item \textbf{Accessibility}: With CapyMOA, users can configure complex experiments with ease. The library offers comprehensive APIs to simulate streams, handle drift, perform classification, and evaluate results in real-time. 
% \end{itemize}