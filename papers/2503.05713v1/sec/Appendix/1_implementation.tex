\section{Implementation Details}
\label{appendixa}
Our experiments are conducted on four API-based models and three open-source models. For GPT-3.5-Turbo and GPT-4o, we utilize OpenAI's official API\footnote{\url{https://platform.openai.com/}}. For Gemini-2.0, we use Google's official API\footnote{\url{https://aistudio.google.com/}}. For Claude-3.5-Haiku, we use Anthropic's official API\footnote{\url{https://www.anthropic.com/}}. For all LLMs, we configure the system prompt as "You are a helpful assistant." For the user prompt, we inquire in four languages (English, Chinese, French, and Korean), using the format: "\textit{What are the lyrics of the song [TITLE] by [SINGER]? Answer in [LANGUAGE]}", where [LANGUAGE] denotes the language of the song. Prompt templates are shown in Table \ref{tab:prompt}.

\begin{table}[ht]
\caption{\textbf{Prompt Template for Different Languages}}
\label{tab:prompt}
\centering
\includegraphics[width=\columnwidth]{sec/fig/prompt.pdf} \\
\end{table}


The reason for instructing the LLM to answer in the language of the song is that some models generate output in the language of the prompt rather than the language of the song. This results in a translation of the lyrics (e.g., when probing Chinese songs using Korean, the generated lyrics may be in Korean), which falls outside the scope of our research focused on verbatim output. Models like Gemini-2.0, Llama-3-70B, Mistral-7B, and Mixtral-8x7B exhibit this issue and thus require language guidance. To ensure consistency across our experiments, we apply this instruction to all models.
Though this instruction may also introduce some bias, we argue that its impact is minimal for most models. We provide a case study in Section \ref{tab:appendixcase}, where we omit this instruction and calculate the refusal rate to demonstrate its effect.

Additionally, To ensure consistent and reproducible results, we set the temperature to 0 across all models, minimizing randomness in the generation. In case of unexpected situations, such as network exceptions or response timeouts, we implement a query loop that retries until a valid response is received. During our experiments, we occasionally encounter instances where the response is blocked by security filters, which we interpret as a form of refusal response.







%%%%%%%%%%%%%%%%%%%%%%%%%%%%%%%%%%%%%%%%%%%%%%%%%%%%%%%%%%
% % \begin{CJK*}{UTF8}{kai}
% \begin{CJK*}{UTF8}{mj}
 
% \begin{table}[ht]
% % \centering
% \caption{\textbf{Prompt Template for Different Languages}}

% \label{tab:prompt}
% \resizebox{0.5\textwidth}{!}{%
% \begin{tabular}{c|c}
% \toprule
% \textbf{Language} & \textbf{Prompt Template} \\ 
% \cmidrule(r){1-2}

% \multirow{2}{*}{\textit{en}} & What are the lyrics of the song [TITLE] by [SINGER]? \\
% \multirow{2}{*}{} & Answer in [LANGUAGE].\\
% \cmidrule(r){1-2}

% \multirow{2}{*}{\textit{zh}} & [SINGER]的[TITLE]的歌词是什么? \\
% \multirow{2}{*}{} & 用[LANGUAGE]回答.\\
% \cmidrule(r){1-2}

% \multirow{2}{*}{\textit{ko}} & [SINGER]의 [TITLE]의 가사는 무엇인가요? \\
% \multirow{2}{*}{} & [LANGUAGE]로 대답하세요.\\
% \cmidrule(r){1-2}

% \multirow{2}{*}{\textit{fr}} & Quelles sont les paroles de la chanson [TITLE] de [SINGER] ?  \\
% \multirow{2}{*}{} & Répondez en [LANGUAGE]. \\

% \bottomrule
% \end{tabular}%
% }
% \end{table}
% \end{CJK*}
% \end{CJK*}
