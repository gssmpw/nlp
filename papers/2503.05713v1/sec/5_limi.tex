\section{Limitations}
This study focuses on multilingual bias in copyright protection, primarily using direct probing to identify copyrighted material. However, this approach may not fully capture the broader range of probing methods, such as prefix probing or jailbreaking. Furthermore, while copyrighted works span a wide variety of formats, including novels, poems, and news reports, we limit our dataset to song lyrics due to their online prevalence and the challenges of constructing multilingual datasets. We plan to expand our experiments in future work to include a wider array of probing methods and types of copyrighted content.
Additionally, our study focuses exclusively on verbatim output of copyrighted material by language models, excluding non-literal forms of copyright infringement, such as similar stories or translations \cite{chen2024copybench}.

\section{Ethics Statement}
Our research aims to explore how large language models handle copyright to safeguard authors' intellectual property against AI-generated copyright infringement. More importantly, we are committed to advancing the fairness of copyright protection, ensuring that the rights of authors and creators from any ethnic background are respected in the era of AI. Our use of these materials is fundamentally aimed at promoting progress in copyright protection. Throughout the experiment, we have implemented measures to ensure that the use of copyrighted materials aligns with legal requirements and ethical responsibilities. The dataset employed in our experiments, which includes copyrighted materials, will not be publicly released. It will only be made available upon request for research purposes, with the assurance that its usage will comply with ethical standards and guidelines.

