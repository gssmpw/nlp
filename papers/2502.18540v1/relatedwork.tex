\section{Related Work}
\textbf{LLMs for Graph}: Recent advancements in LLMs for graph tasks have led to significant contributions in methodology and evaluation. These tasks are often classified into Enhancer, Predictor, and Alignment types \citep{suvery}. Notably, \citep{2} presents a roadmap for unifying LLMs with Knowledge Graphs (KGs), while \citep{3} proposes an end-to-end method for solving graph-related problems,\cite{cao2024graphinsight} improves LLMs' understanding of graph structures by addressing positional biases and incorporating an external knowledge base. On the evaluation front, several benchmarks have been introduced. NLGraph \citep{NLgraph-6} offers a simple test dataset for graph tasks, and GPT4Graph \citep{gpt4graph-7} evaluates LLM capabilities on semantic tasks. GraCoRe\cite{yuan2025gracore} comprehensively verifies the graph understanding and reasoning capabilities of LLM. Other notable works include \citep{8}, which assesses LLMs in graph data analysis, and \citep{10}, which designs a hint method for graph tasks.

\noindent\textbf{LLM Agents}: In recent years, several multi-agent frameworks have been proposed to enhance the coordination and efficiency of language models in complex tasks. MetaGPT\cite{hong2023metagpt} reduces hallucinations in complex tasks by embedding human workflows into language models. CAMEL\cite{li2023camel} promotes autonomous cooperation among agents, guiding them to align with human goals and studying their interactions. AutoGen\cite{wu2023autogen} is a flexible framework that allows developers to customize agent interactions using both natural language and code, suitable for various fields. In addition, \cite{li2024graphteam} can be used to solve simple graph problems, \cite{li2024anim} is an autonomous agent that uses LLMs to create animated videos from simple narratives.
% \titlespacing*{\subsection}{0pt}{0.5ex plus 0.2ex minus 0.2ex}{0pt}
% \titlespacing*{