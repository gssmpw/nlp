\documentclass[10pt,conference]{IEEEtran}
\usepackage{pgfplots}
\pgfplotsset{compat=1.18}
%\documentclass[conference]{IEEEtran}
\IEEEoverridecommandlockouts
% The preceding line is only needed to identify funding in the first footnote. If that is unneeded, please comment it out.
\usepackage{cite}
\usepackage{amsmath,amssymb,amsfonts}
\usepackage{algorithmic}
\usepackage{graphicx}
\usepackage{svg}
\usepackage{lipsum} 
\usepackage{textcomp}


\usepackage[english]{babel}
\usepackage{lipsum}
\usepackage{soul}
\usepackage{multirow}
\PassOptionsToPackage{table,xcdraw}{xcolor}\usepackage{xcolor}
\usepackage{tikz}
\usepackage{mdframed}

\usepackage{pgfplots}

%\usepackage[noadjust]{cite}


\definecolor{mygray}{RGB}{211,211,211} % Definindo a cor cinza




\def\BibTeX{{\rm B\kern-.05em{\sc i\kern-.025em b}\kern-.08em
    T\kern-.1667em\lower.7ex\hbox{E}\kern-.125emX}}
\begin{document}

\title{Embracing Experiential Learning: Hackathons as an Educational Strategy for Shaping Soft Skills in Software Engineering
\thanks{This study was financed in part by the Coordenação de Aperfeiçoamento de Nível Superior - Brasil (CAPES) - Postdoctoral Institutional Program (PIPD).}
}

\author{\IEEEauthorblockN{Allysson Allex Araújo}
\IEEEauthorblockA{\textit{Center for Science and Technology} \\
\textit{Federal University of Cariri}\\
Juazeiro do Norte, Brazil \\
allysson.araujo@ufca.edu.br}
\and
\IEEEauthorblockN{Marcos Kalinowski}
\IEEEauthorblockA{\textit{Department of Informatics} \\
\textit{Pontifical Catholic University of Rio de Janeiro}\\
Rio de Janeiro, Brazil \\
kalinowski@inf.puc-rio.br}
\and
\IEEEauthorblockN{Maria Teresa Baldassarre}
\IEEEauthorblockA{\textit{Department of Computer Science} \\
\textit{University of Bari}\\
Bari, Italy \\
mariateresa.baldassarre@uniba.it}
}



\maketitle

\begin{abstract}
In recent years, Software Engineering (SE) scholars and practitioners have emphasized the importance of integrating soft skills into SE education. However, teaching and learning soft skills are complex, as they cannot be acquired passively through raw knowledge acquisition. On the other hand, hackathons have attracted increasing attention due to their experiential, collaborative, and intensive nature, which certain tasks could be similar to real-world software development. This paper aims to discuss the idea of hackathons as an educational strategy for shaping SE students' soft skills in practice. Initially, we overview the existing literature on soft skills and hackathons in SE education. Then, we report preliminary empirical evidence from a seven-day hybrid hackathon involving 40 students. We assess how the hackathon experience promoted innovative and creative thinking, collaboration and teamwork, and knowledge application among participants through a structured questionnaire designed to evaluate students' self-awareness. Lastly, our findings and new directions are a\-nalyzed through the lens of Self-Determination Theory, which offers a  psychological lens to understand human behavior. This paper contributes to academia by advocating the potential of hackathons in SE education and proposing concrete plans for future research within SDT. For industry, our discussion has implications around developing soft skills in future SE professionals, thereby enhancing their employability and readiness in the software market.
\end{abstract}

\begin{IEEEkeywords}
Hackathon, Soft Skills, Self-Determination Theory, Experiential Learning.
\end{IEEEkeywords}

\section{Introduction}
%Problem statement
%Research objectives
%Context
%valor da escala x diferente da percepçao das pessoas (quantidade de respostas?)

%Becoming a proficient software engineer involves diverse educational pathways and training approaches \cite{malik2012systematic}. From formal education programs to on-the-job training, from coding clubs to bootcamps, and from online resources to participation in programming contests, aspiring software engineers have numerous avenues to acquire the requisite skill set. However, a reflection remains: How can we best motivate students in their journey to become competent software engineers?

%Enhancing students' skills in Software Engineering (SE) education and training is a continuous process that calls for regular evaluation and research of practical approaches. In this ongoing endeavor, technical knowledge surely needs to be improved, but soft skills are also quite important. Soft skills are rooted in psychology and encompass a wide range of traits related to personality types, communication skills, social interaction abilities, and personal habits \cite{faheemsoft}. Soft skills like teamwork, communication, and innovative thinking are highly valued in the software business landscape  \cite{ahmed2012evaluating, ahmed2015soft}. According to \citet{capretz2018call}, software professionals should investigate these soft skills and recognize that the people involved in the software development process are as important as the processes and the technology itself. 

Alongside the last years, SE scholars \cite{capretz2018call, akdur2022analysis} and practitioners have been arguing about the relevance of soft skills in the practice of SE and the need to incorporate this topic in the context of SE education. As discussed by Sedelmaier and Landes \cite{sedelmaier2014practicing}, soft skills are core competencies of a software engineer and for this reason soft skills should be a core part of SE education. However, one of the difficulties is in selecting the right teaching or training approach that allows a person to gain new abilities in an environment that is somewhat similar to the nuances in the market \cite{matturro2019systematic}. In this sense, it became evident that the SE community has been working to figure out how to improve the soft skills of SE students. For instance, Dubey and Tewari \cite{dubey2016systematic}, reviewed SE curricula and found references to soft skills such as communication and leadership. Marques et al. \cite{marques2020stimulating} explored how Design Thinking provides practical experiences and it supports the development of students' soft skills from SE courses. Garcia et al. \cite{garcia2020effects}, in turn, investigated the effects of game-based learning in the acquisition of soft skills on undergraduate SE courses.

As one can notice, these pedagogical approaches to teaching soft skills exhibit a high degree of variability, as it is commonly accepted that soft skills cannot be learned passively through raw knowledge acquisition \cite{martins2007role}. Caeiro-Rodríguez et al. \cite{caeiro2021teaching}, for example, clarified that students must take an active role, experiencing their capabilities, strengths, and weaknesses in relation to soft skills. Given this scenario, one can highlight the potential of hackathons in leveraging soft skills for SE students by bridging theory to practice for analogous real-world nuances (time pressure, collaborative teamwork, innovative thinking, etc.) \cite{porras2018hackathons}. In summary, a hackathon is an intensive and collaborative event where participants, typically software developers, engage in focused, time-limited programming and problem-solving activities \cite{steglich2021online, sadovykh2020hackathons}. Hackathons usually range from a few hours to a few days and are often organized around a theme/challenge where the participants can work in teams to develop and present software solutions \cite{steglich2020hackathons}. These events are known for their fast-paced and dynamic nature, encouraging creativity and teamwork, as participants strive to create functional software within a condensed timeframe \cite{kumalakov2018hackathon}. 

Indeed, the prominence of hackathons as immersive and experiential learning experiences able to enhance students' skill sets has already garnered considerable interest among SE researchers \cite{steglich2020hackathons, steglich2021online, sadovykh2020hackathons, gama2018hackathon, kumalakov2018hackathon, afshar2022hackathon}. In particular, our paper specifically aims to discuss the idea of hackathons in enhancing soft skills of SE students. We believe that this effort can help raise awareness to this educational opportunity and its importance for both academia and industry. To achieve this objective, we have defined the following research question: ``\textit{Can participation in hackathons lead to a self-perception of improvement in soft skills for SE students?}''. To investigate this issue, we first briefly overview the literature on soft skills and hackathons in SE education. Additionally, we present a case study that brings initial findings from a hackathon conducted at a public university in Brazil. We also provide details on how we organized our hackathon in an educational setting, including operational details and the lessons we learned. Lastly, we discuss our findings in the light of Self-determination Theory (SDT) \cite{deci2012self}, which served as a psychological lens for deriving relevant implications for future research and practice.

Our investigated hackathon spanned seven days and blended online and in-person activities. It attracted 40 enrolled students, 79\% of whom were participating in a hackathon for the first time. This research reports a preliminary assessment with a formal evaluation strategy covering how engagement in the hackathon fostered creativity, innovative thinking, collaboration, teamwork, and knowledge application among students. These specific soft skills were investigated due to their critical relevance in enhancing overall skill development, particularly in enhancing the synergy between software development and business \cite{moreira2023analyzing}. Naturally, the empirical findings we report in this work are initial evidence, emphasizing the need for more investigation to understand the intricacies of this challenge. 

Furthermore, this paper is part of an ongoing project that focused on behavioral aspects related to hackathon. In contrast to this previous work \cite{araujo2024can}, our primary goal with this paper is to raise awareness and discuss a forward-thinking idea about the potential value of hackathons in helping SE students bridge the gap between technical proficiency and practical soft skills grounded on the experiential learning and SDT. We did not address the perspective of soft skills in our previous paper. 

%Furthermore, this paper is part of an ongoing project that focused on understanding whether hackathons can impact the motivation of SE students \cite{araujo2024can}. In contrast to this previous work, our primary goal with this position paper is to raise awareness and discuss a forward-thinking idea about the potential value of hackathons in helping SE students bridge the gap between technical proficiency and practical soft skills grounded on the experiential learning.

%Moreover, this paper is part of an ongoing project with previous results \cite{araujo2024can}, which focuses  whether hackathons can impact the motivation of SE students. Diffently from this work, our primary aim with this position paper is to raise awareness and discuss a forward-thinking idea of the value of hackathons in helping SE students bridge the gap between technical proficiency and soft skills in practice.

%, as well as discuss the advantages, drawbacks, and potential directions for further research.

%Naturally, the empirical findings we report in this work are only preliminary evidence, highlighting the need for more investigation to fully understand the nuances of this setting. Therefore, the primary objective of this position paper is to raise awareness and present a forward-thinking idea of the value of hackathons in helping SE students bridge the gap between technical proficiency and soft skills, as well as discuss the advantages, drawbacks, and potential directions for further research. 



%Unlike the majority of studies focusing on hackathons as a pedagogical strategy for SE education \cite{steglich2020hackathons, steglich2021online, sadovykh2020hackathons, gama2018hackathon, kumalakov2018hackathon, afshar2022hackathon}, our research aimed to assess how participating in the hackathon stimulated creativity, innovation, collaboration, teamwork, and knowledge application. These issues are particularly relevant to be investigated because they. This paper is part of a ongoing project whose initial results were shared with the community. However, we focused on the perspective on the results covering the academic motivation \cite{}, while at this paper we approach the results concerning the soft skills. 

This study offers three main contributions. For \textit{academia}, it discusses the potential of hackathons to enhance soft skills in SE students, supported by preliminary empirical evidence. For \textit{practice/industry}, it highlights the role of hackathons in developing soft skills in SE students, enhancing their employability and readiness for the  software market. As a \textit{forward-looking idea}, it suggests directions based on SDT to explore hackathons to develop soft skills in SE students, thus contributing to ongoing advancements in SE practice.


%The paper's structure is outlined as follows: Section 2 brings the background and related work. Section 3 details the case study design. Section 4 presents preliminary results, followed by an overall discussion looking to the future in Section 5. The paper concludes in Section 6 with final remarks.

\section{Background and Related Work}
%This section addresses the challenge of teaching soft skills in SE. Then, we explore the educational and promising role of hackathons in SE as an experiential learning strategy.

%intersection between hackathons and SE education. Ahead, we delve into the literature on learning soft skills in the context of SE.

%\subsection{Soft Skills in SE Education}
\subsubsection{Unveiling the Challenge: Enhancing Soft Skills in SE Students}

Traditionally, SE programs have primarily emphasized technical skills (also known as ``hard skills'') like programming, software testing, and system architecture. However, the constantly evolving nature of the software industry and the collaborative demands of modern workplaces have emphasized the importance of soft skills. These skills encompass a broad range of characteristics, including personality traits, social interaction abilities, communication, and personal habits \cite{ahmed2012evaluating}. According to Kechagias \cite{kechagias2011teaching}, soft skills can be defined as ``intra and inter personal (socio-emotional) skills'' that are essential for personal development, social participation, and workplace success. Capretz and Ahmed \cite{capretz2018call} advocate for promoting soft skills in SE to enhance career prospects and better prepare practitioners for the industry's challenges. This need is also driven by the increasing complexity and interdisciplinary nature of software projects \cite{matturro2019systematic, miranda2021compreendendo}.



%Capretz and Ahmed \cite{capretz2018call} advocate for promoting soft skills in SE to enhance career prospects and better prepare practitioners for the industry's challenges and rapid evolution. Another key driver of this shift is the complexity and interdisciplinary nature of software projects \cite{matturro2019systematic}.

%Consequently, SE scholars are striving to balance the development of both technical and soft skills to better prepare students for real-world challenges.

%Successful software development depends on how software practitioners perform their tasks and how they interact with their peers \cite{capretz2018call}. As stated by \citet{ahmed2012evaluating}, ``software is a by-product of human activities, such as problem solving capabilities, cognitive aspects, and social interaction''.

%The job requirements in the field of SE usually divide the skill requirements into the two categories of “hard skills” (technical requirements and knowledge a person should have to carry out a task) and “soft skills” (a broad range of characteristics involving personality types, social interaction abilities, communication, and personal habit) \cite{ahmed2012evaluating}. In line with this definition, \citet{kechagias2011teaching} claim that soft skills could be understood as “intra and inter personal (socio-emotional) skills” essential for personal development, social participation and workplace success. While much of our SE research is technical and comprise hard skills, the importance of addressing soft skills has been also attracting considerable attention both by industry and academy \cite{matturro2015soft}.

From an educational lens, teaching soft skills presents challenges because these skills cannot be passively acquired through theoretical knowledge alone \cite{martins2007role}. SE scholars have been investigated active learning methods to enhance students' soft skills. These proposals include, for example, project-based learning \cite{gonzalez2011teaching}, game-based learning \cite{garcia2020effects}, and problem-based learning \cite{yu2016developing}.  While these approaches are valuable and contribute to soft skill development, they may not fully replicate the fast-paced, high-pressure, and collaborative nature of software projects. Hackathons, in contrast, provide an immersive environment where students must rapidly adapt and collaborate under specific constraints, making them a promising complementary strategy for enhancing soft skills in SE education \cite{porras2018hackathons, steglich2021online}.




%One relevant reason for this shift is the complexity and interdisciplinary nature of software projects. Recently, Capretz and Ahmed \cite{capretz2018call} issued an open call to promote soft skills in SE. This perspective allows to enhance career prospects and contributes to the development of SE practitioners better prepared to deal with the challenges and rapid pace of the software industry \cite{matturro2019systematic}. %They argue that successful software development depends not only on the technical execution of tasks but also on how software practitioners interact with their peers. %Within a general scope, \citet{bolton2009people} reported that 80\% of individuals who fail at work do not fail due to their lack of technical skills but rather because of their inability to relate well with others.

%This research agenda on the importance of incorporating soft skills into SE education has been gaining large traction over the years. In this regard, Sedelmaier and Landes \cite{sedelmaier2014practicing} argue that soft skills are core competencies for software engineers and should be integral to SE education. This perspective allows to enhance career prospects and contributes to the development of SE practitioners better prepared to deal with the challenges and rapid pace of the software industry \cite{matturro2019systematic}.

%Recently, the Experiential Learning (ExL) have been growing as a means to cultivate soft skills in SE students. ExL is rooted in the concept of learning through direct experience, specifically defined as "learning through reflection on doing" By involving students in hands-on activities and encouraging reflection, ExL facilitates the connection between theoretical concepts taught in the classroom and real-world scenarios. As we can see, this approach closely mirrors the essence of hackathons, where participants engage in intense problem-solving sessions and subsequently reflect on their experiences.

%More recently, a particular stream of research has been advocating for Experiential Learning  (ExL) to foster soft skills in SE students. In summary, ExL ise based on the process of learning through experience, and is more narrowly defined as ``learning through reflection on doing''. By engaging students in hands-on experiences and reflection, they are better able to connect theories and knowledge learned in the classes to real-world situations. As we can see, this idea closely aligns with the context of hackathons.


%\citet{matturro2019systematic}, for example, reported the results of a systematic mapping study to identify existing research on soft skills in SE and to determine what soft skills are considered relevant to the practice. In another study, \citet{matturro2013soft} reviewed a set of jobs advertisements offering job positions related to SE in order to identify what soft skills are most in demand by software companies in Uruguay. This method of reviewing job requirements to understand the role of soft skills in software industry was also tackled by \citet{ahmed2012evaluating}. In their work, they reported some of the vital statistics from industry about the requirements of soft skills in various roles of software development phases. They also highlighted the variation in the types of skills required for each of the roles. More recently, \citet{galster2022soft} used a qualitative research method and manually analyzed 530 job adverts from New Zealand’s largest job portal to understand what the employers demand from professionals beyond “hard skills”.

%Lorem ipsum dolor sit amet, consectetur adipiscing elit. Nulla viverra enim a pretium luctus. In at erat leo. Cras non semper urna. Curabitur pellentesque quis tortor a varius. Sed in fermentum nisl. Nulla facilisi. Curabitur leo purus, tristique quis mauris vitae, ultrices pretium ipsum. Maecenas nec molestie quam. Interdum et malesuada fames ac ante ipsum primis in faucibus. 


\subsubsection{Hackathons in SE Education: Leveraging Experiential Learning}
The importance of developing soft skills in SE students has also been successfully linked to Experiential Learning (ExL) \cite{fioravanti2023software}. ExL is based on learning through direct experience, specifically defined as ``learning through reflection on doing'' \cite{bradberry2019learning}. By involving students in hands-on activities and encouraging reflection, ExL helps bridge the gap between theoretical concepts taught in the classroom and real-world scenarios. This educational approach resembles the essence of hackathons, where participants engage in intense problem-solving sessions and later reflect on their experiences.

Researchers have been closely studying the pedagogical potential of hackathons in SE education, focusing on various aspects such as skill development, student perceptions, and curriculum integration. This focus on hackathons stems from their ability to provide a rich experiential learning approach where students can apply knowledge in practice. Porras et al. \cite{porras2018hackathons} conducted a literature review, utilizing their extensive learning experiences and written and interview materials from students and industry participants to present an overview of hackathons. In another study, Porras et al. \cite{porras2019code} discussed a systematic literature review on the use of hackathons in SE and computer science education. 


%For example, \citet{porras2018hackathons} provided an overview through a literature review, drawing from various materials from students and industry participants. In another study, \citet{porras2019code} conducted a systematic review to explore the use of hackathons and code camps in SE and computer science education. 

Other works have focused on the impact of hackathons on student skills and perceptions.  Steglich et al. \cite{steglich2020hackathons} conducted a case study to investigate factors influencing student participation, their perceptions, and the SE practices they adopted. Steglich et al. \cite{steglich2021online} further examined the professional skills developed during hackathons and how intense collaboration occurs in atypical scenarios. In addition, Haque et al. \cite{haque2022effectiveness} assessed students' ability to apply SE best practices learned in their courses to hackathon settings, demonstrating the practical application of these skills.

Hackathons have also been integrated into SE curricula.  Sadovykh et al. \cite{sadovykh2020hackathons} introduced hackathons as part of an SE program, proposing a specific setup and evaluating outcomes through surveys from students, mentors, and industry representatives. They reported high satisfaction and suitability for final evaluations in Software Quality courses. Sadovykh et al. \cite{sadovykh4hackathon} also found that hackathons as exam replacements motivated students to learn beyond the course material. Gama et al. \cite{gama2018hackathon} proposed a methodology for project-oriented undergraduate courses, using hackathons as realistic scenarios to practice class knowledge, and evaluated its use in an IoT course through quantitative and qualitative analyses. Kumalakov et al. \cite{kumalakov2018hackathon} explored a blended hackathon model to increase student engagement and interest in computer science, with preliminary findings indicating positive results.

In terms of practical applications, Afshar et al. \cite{afshar2022hackathon} integrated a hackathon component into a software development and architecture course. Data collected from pre- and post-hackathon surveys and GitHub code commits demonstrated the impact of hackathon participation on student performance and understanding. La Place and Jordan \cite{la2022adapting} found that students who participated in hackathons and a project-based learning (PBL) SE degree developed transferable skills and employed various problem-solving approaches in both settings. Unlike traditional PBL, which focuses on long-term collaboration and gradual skill development, hackathons prioritize immediate problem-solving under time constraints \cite{szymanska2020effects}. This high-pressure setting inherent to hackathons enhances soft skills like communication, leadership, and conflict resolution, complementing the gradual learning process in PBL \cite{garcia2023fostering}.

\subsubsection{Research Gap} Although some papers have previously addressed this issue, there is a scarcity of studies specifically focusing on the role of hackathons in developing soft skills among SE students. We found no other work discussing these results through the lens of SDT. Further exploration of these aspects could help optimize hackathon design to enhance their educational impact and bridge the gap between technical proficiency and soft skills development in SE education.

%Even though some papers have already addressed this issue, there is a general scarcity of studies focusing specifically on the role of hackathons in developing soft skills among SE students. We also do not found other work discussing the results in the light of SDT. Continuing to examine these aspects can offer guidance on how to optimize hackathon structures to maximize their educational impact and, consequently, bridge the gap between technical proficiency and soft skills development in SE education.

%Even with som paper already addressing, there is a overall scarcity of studies whose primary aim is to investigate the role of hackathons in developing soft skills among SE students. More specifically, exploring these aspects can provide directions into optimizing hackathon structures to maximize their educational impact, thus bridging the gap between technical proficiency and soft skills development in SE education. 

%Sed in fermentum nisl. Nulla facilisi. Curabitur leo purus, tristique quis mauris vitae, ultrices pretium ipsum. Maecenas nec molestie quam. Interdum et malesuada fames ac ante ipsum primis in faucibus. 

%Researchers have been closely studying the pedagogical potential of hackathons in SE education. For example, \citet{porras2018hackathons} conducted a literature review, drawing from their years of learning experiences, as well as written and interview materials from students and industry participants, to provide an overview of hackathons and code camps. More ahead, \citet{porras2019code} conducted a systematic literature review to investigate the current state of using code camps and hackathons in SE and computer science education.

%\citet{steglich2020hackathons}, conducted a case study on an educational hackathon to understand the factors that may affect students' participation, their perceptions of the event, and the SE practices they adopt. They found that hackathons can serve practical pedagogical purposes since students believe they can improve their skills through participation in these competitions. In another study, \citet{steglich2021online} investigated the professional skills developed by students who participate in hackathons and how intense collaboration takes place in an atypical scenario while solving problematic situations. In turn, \citet{haque2022effectiveness} evaluated students' capability to apply the skills they learned in their courses to the hackathon setting, especially those related to SE best practices. 

%\citet{sadovykh2020hackathons} introduced hackathons as part of their SE program curriculum and proposed a specific hackathon setup. The outcome of the hackathon was evaluated based on a survey completed by students, mentors, and industry representatives. All groups reported high satisfaction with the hackathon. \citet{sadovykh4hackathon} also examined the educational contribution of hackathons as a replacement for exams in software engineering education. Their preliminary study at an IT university revealed a positive reception among students. They found that hackathons are suitable for final evaluations in Software Quality courses, as they motivate students to learn material beyond the course curriculum.

%\citet{yuen2021designing} showcased a real-life case study involving a hackathon for potential college students in a university's data science program. They offered a structured framework and suggests practical principles and guidelines for designing hackathons.  \citet{gama2018hackathon} introduced a methodology to be used in project-oriented undergraduate courses where a hackathon dynamic can be employed as a realistic scenario to practice the knowledge acquired in classes. They conducted a quantitative and qualitative analysis to evaluate the usage of the methodology in an IoT undergraduate course. \citet{kumalakov2018hackathon} investigated the proposition of a hackathon blended model as an appropriate method for increasing students’ interest and understanding in computer science and encouraging their engagement in using computing skills and knowledge. Preliminary findings have produced not significant, but positive results acceptable to apply hackathon to the curricular.

%\citet{afshar2022hackathon} integrated a hackathon component in a software development and architecture course curriculum to evaluate the effect of working on a real-world web development project in a hackathon setting on deepening the theoretical concepts learnt in lectures. The data is collected through two surveys which were accessible to students before and after the hackathon and students code commits on GitHub.  Research findings shown the importance of hackathon participation on students’ performance and state of knowledge.

%\citet{la2022adapting} builds upon a previously published work \cite{la2020wip} in progress, finding that students who attended hackathons and a project-based learning Software Engineering degree, built transferable skills between hackathons and capstones. Participants described the employment of software design methodologies in both hackathons and capstone projects, various problem solving and researching approaches used, and a distinct level of attention to the context of development, informal or formal, to determine their courses of actions.

%To the best of our knowledge, few studies claimed the usage of hackathons in the soft skills. Include here a concluding paragraph. 

%Lorem ipsum dolor sit amet, consectetur adipiscing elit. Nulla viverra enim a pretium luctus. In at erat leo. Cras non semper urna. Curabitur pellentesque quis tortor a varius. 

\section{Method}

\subsection{Research Design}
This investigation adopts an evaluative case study approach following the methodological guidelines proposed by Runeson and Höst \cite{runeson2009guidelines}. Our investigation seeks to tackle the following \textbf{research question} (1): \textit{Can participation in hackathons lead to a self-perception of improvement in soft skills for SE students?} As we can see, this research question  aligns with the open call concerning the relevance of approaching soft skills into SE education \cite{akdur2022analysis, sedelmaier2014practicing}.%In particular, we justify the use of case study as research method due its ability to generate contextual background material for a focused discussion on a specific problem under investigation \cite{gustafsson2017single}. Furthermore, it is important to note that case studies provide a richer knowledge of the phenomenon being studied even though they are not as successful as controlled experiments in demonstrating causal linkages  \cite{runeson2012case}.

%This investigation uses an evaluative case study following the methodological recommendations suggested by \citet{runeson2009guidelines}, which are discussed below. In this case, we justify the choice of a case study design due its capability to generate background material to a discussion about a concrete problem \cite{gustafsson2017single}. Hence, it is worth to notice that case studies do not generate the same results on e.g. causal relationships as controlled experiments do, but they provide deeper understanding of the phenomena under study \cite{runeson2009guidelines}.

%According to them, the case study design section has to clarify \textbf{five major issues}, which are discussed below.

%This investigation concerns an evaluative case study with students as the unit of analysis for assessing their motivation. We followed the methodological guidelines provided by \citet{runeson2009guidelines}. These guidelines emphasize \textbf{five main aspects} addressed below. According to them, the case study design section has to clarify \textbf{five major issues}. 


% Is it possible for students studying software engineering to enhance their soft skills through hackathon participation?


%we aim to bring attention and have a forward-thinking idea about the usefulness of hackathons in assisting SE students in closing the gap between their theoretical and practical soft skills. As we can see, this research questions is aligned with the open call about the relevance of incorporating soft skill in the context of SE education \cite{groeneveldsoftware, akdur2022analysis, sedelmaier2014practicing}.

%This study aims to address the following \textbf{research question} (1): \textit{Can participating in a hackathon improve the soft skills of software engineering students?”} This research question guides our investigation, delving into the intricate relationship between hackathons and student motivation. By exploring this dynamic, we aim to contribute with a better understanding of the existing body of knowledge on effective SE education and training strategies.

In the \textbf{case selection} (2), students from a public university in Brazil participated in a sponsored educational hackathon. This case was selected for its relevance to the study’s aim of exploring hackathons’ role in developing SE students’ soft skills. The sample included bachelor’s students who expressed interest, with selection based on event enrollment.


%students from a public university in Brazil participated in a sponsored and educational hackathon with the goal of tackling a practical problem. This instance was carefully chosen because it was pertinent to the study's goal of investigating the potential role of hackathons to leverage the soft skills of SE students. The pool of bachelor's students who indicated interest in taking part in the hackathon was used as sample. Hence, the selection criteria took into account their enrollment in the event.


%The \textbf{case selection} (2) involved a hackathon at a federal university in Brazil's northeast, which aimed to bring together students to address a real-world problem. The purposeful selection of this case was based on its relevance to the research aim of investigating the impact of a hackathon as a motivational tool for SE students. The participants were selected from the pool of bachelor students who expressed interest in participating in the hackathon. The selection criteria considered their enrollment in the event.

Concerning \textbf{data collection} (3), all the participants had to initially answer a pre-event questionnaire designed to collect demographic information like age, gender, course of study, and the duration of their enrollment at the university. %Additionally, this questionnaire contained questions about their self-assessed proficiency in software development (rated on a Likert scale of 1 to 5), self-assessment of their ability to work in a team (rated on a Likert scale of 1 to 5), and a question to determine if they had any prior experience with hackathons.
%Additionally, participants were administered the AMS (rated on a Likert scale of 1 to 7) as part of the pre-hackathon data collection. Considering that we dealt with Brazilian students, we administered the validated translation into Brazilian Portuguese of the AMS \cite{davoglio2016validaccao}. As depicted in Table \ref{tab:results}, the questions were also slightly adapted to the context of a hackathon. 
%Following the hackathon's conclusion, participants completed a post-event questionnaire featuring scaled questions designed to evaluate the extent of improvement in three key perspectives: i) innovative and creative thinking, ii) collaboration and teamwork, and iii) practical knowledge application. Each question was rated on a scale of 1 to 7.
Following the hackathon's conclusion, participants completed a post-event questionnaire featuring scaled questions (from 1 to 7) designed to evaluate the extent of improvement in three key perspectives: Q1) How much did participation in the hackathon stimulate your creativity and innovation?	Q2) How much did the hackathon provide a collaborative and teamwork environment? Q3) How much did the hackathon allow you to apply the knowledge acquired in the classroom? 

%Upon the hackathon's conclusion, participants were given a follow-up questionnaire that had scaled questions with the purpose to determine how much the hackathon enhanced the students' i) innovation and creativity thinking, ii) collaboration and teamwork, and iii) the practical application of knowledge learned. Every question has a scale from 1 to 7.

%Subsequent to the conclusion of the hackathon, participants were presented with a follow-up questionnaire, which also featured a series of scaled questions requiring self-assessment. The questions were designed to gauge the extent to which participation in the hackathon contributed to academic engagement, stimulated creativity and innovation, fostered a collaborative and teamwork-oriented environment, and facilitated the practical application of knowledge gained in the classroom. The scale for each question ranged from 1 to 7.

%An open-ended question was included to elicit additional comments based on the participants' experiences during the event. Again, participants were requested to complete the AMS as part of the post-hackathon data collection process. 

The \textbf{data analysis} (4) encompassed quantitative data from questionnaires responses, capturing demographic details and scaled questions related to soft skills. Descriptive statistical analysis was performed utilizing measures such as means, standard deviations, and frequency distributions. Lastly, we discuss our findings in the light of Self Determination Theory, connecting our empirical results with a theoretical background.


%The \textbf{data analysis} (4) included both quantitative data from the pre- and post-hackathon questionnaires, covering demographic information and scaled questions regarding soft skills. Descriptive statistics analysis was conducted on this data, using measures such as means, standard deviations, and frequency distributions. This analysis provided valuable insights into the overall distribution of motivation levels and any changes that occurred.

%The \textbf{data analysis} (4) encompassed  quantitative data derived from the pre- and post-hackathon questionnaires, encompassing demographic information and scaled questions covering the soft skills. This data underwent descriptive statistics analysis. Measures such as means, standard deviations, and frequency distributions were employed. This analysis provided insights into the overall distribution and changes in motivation levels. 

%while the qualitative analysis allowed for a deeper exploration of participants' individual experiences and perspectives.


To increase \textbf{validity of the findings} (5), we followed the well-known case study protocol recommended by Runeson and Höst \cite{runeson2009guidelines}. To enhance the verifiability and transparency of our results, we have also made all our data openly available through our supporting repository \cite{repo}. All participants were  informed about the study's context, including its objective, required data, potential risks, and other relevant details, with explicit consent obtained through consent form.

%Different procedures were implemented to ensure the \textbf{validity of the findings} (5). Data triangulation was employed by collecting data questionnaires, open-ended questions, and a literature review. This approach allowed for cross-validation of the findings and increased the reliability of the results \cite{ralph2021acm}. To uphold the standards of methodological rigor, reliability, and validity, we adhered to the case study protocol as recommended by \citet{runeson2009guidelines}. All the data underpinning this study is openly available via our supporting repository \cite{repo}, ensuring transparency and accessibility.

\subsection{Case Study Description and Participants Characterization}
Ten teams of four students participated in a seven-day hackathon from May 18 to May 24, 2023. Teams were selected based on their registration order, with the final list announced on May 15, 2023. The hybrid format required strong self-management skills for both online and in-person activities.
%Ten teams of four students participated in a seven-day hackathon that took place from May 18 to May 24, 2023. Teams were chosen using an application process that took into account registration order. The teams who were chosen were revealed on May 15, 2023. Due to the week's hybrid format, self-management abilities were heavily utilized in both online and in-person activities. 

Concerning the online activities, we set up a Discord server accessible to all event participants, including team members, mentors, and organizers. The server was organized into three categories, namely 1) Workspace, 2) Information, and 3) Organization. The \textit{Workspace} category included one text channel and one voice channel for each team. These channels were only accessible by the organizers and respective team members. In addition, the \textit{Information} comprised channels that were open to all participants, with specific channels dedicated to important information such as welcome messages, challenge details, sponsor advertise, off-topic discussions, and support. The \textit{Organization} included exclusive channels for members of the organizing committee and mentors. These channels were useful to follow up and manage communication.

In regard to in-person activities, teams were granted access to the university's laboratory facilities based on their specific demands. These demands were gathered through the enrollment process, during which participants registered their teams and specified whether they would need access to the laboratory and at what time. Each team was assigned mentors who were alumni of the university and currently employed in the software industry. This partnership with alumni was a strategic decision due to two primary reasons. First, it facilitated the connection between current students and recent graduates, fostering a culture of knowledge exchange and networking. Second, it evidenced a sense of gratitude from mentors who appreciated the opportunity to return to the university and contribute back to their community.

The problem addressed during the hackathon was the development of a meeting tracking software for undergraduate thesis supervision. A software requirements document was used to inform the participants about the product requirements specification. On the last day of the hackathon, each team had to present a minimally functioning prototype that included both front-end and back-end elements. The presentation took the form of a structured slide deck pitch. Only one team opted not to present their pitch for personal reasons. The solutions were evaluated by a panel of five invited judges, who assessed criteria such as creativity and innovation, quality of design and usability, functionality and feasibility, and presentation (oral and visual). The team with the highest average score was declared the winner. Members of the winning team were awarded personalized small 3D trophies crafted using a printer from the own university, along with sponsored gift vouchers as symbolic prizes. Additionally, all teams that presented their projects received a certificate of completion.% for their participation in the hackathon.

Out of 40 participants who enrolled, we obtained 34 responses (85\%) for the pre-hackathon questionnaire and 35 responses (87.5\%) for the post-hackathon questionnaire. 
Among the participants, the vast majority (over 80\%) were male, based on responses to the gender question, which offered the options: male, female, non-binary, prefer not to answer, and an open choice (to be filled out by the participant). Most of the respondents (79.40\%) fell within the age range of 18-21, whereas 14.70\% were aged between 22-25, and 2.90\% each were aged between 26 and older. Regarding the distribution of courses, 50\% of the participants were from the Information Systems course, 47.10\% were from Computer Science, and 2.90\% were from Mining Engineering. In terms of enrollment duration at university, 26.50\% of students had been enrolled for less than a year, 61.80\% for one to three years, and the remaining 11.80\% for over three years.

\section{Preliminary Empirical Findings}


%Regarding the case study description, the hackathon lasted seven days, from May 18 to May 24, 2023, and involved ten teams of four students. The selection of teams was carried out through an application process based on the order of registration. The announcement of the selected teams was made on May 15, 2023. The hybrid nature of the week allowed for both online and in-person activities, including intense use of self-management skills. Participants had access to the hackathon's Discord server, where communication and collaboration occurred among team members, mentors, and event organizers. Depending on the team's specific needs, they were granted access to the university's laboratory facilities. Mentors, alumni of the university (and are currently working in the software industry), were assigned to each team to provide guidance and support in refining their projects.

%The creation of a meeting tracking software for undergraduate thesis supervision was the problem that needed to be solved during the hackathon. The organizers of the hackathon determined that this issue was a demand that the university already had in practice. 

%In summary, the software aimed to manage and record meetings (tasks to be done, logs, etc.) between supervisors and students while allowing course coordination to track their activities. The product requirements were communicated to the participants through a software requirements document. On the final day of the hackathon, each team was required to present a minimally functional prototype with front-end and back-end, following a slide deck that covered specific points in the format of a pitch presentation. Only one team opted not to present their pitch for personal reasons. The solutions were evaluated by a panel of five invited judges, who assessed criteria such as creativity and innovation, quality of design and usability, functionality and feasibility, and presentation (oral and visual). The team with the highest average score was chosen as the winner. Each winning team member received a personalized small trophy (3D printed in the university) and a sponsored gift voucher as a symbolic prize. All teams that presented their projects received a hackathon completion certificate.



%In terms of enrollment duration at university, 26.50\% had been enrolled for less than one year, 61.80\% had been enrolled for one to three years, and the remaining 11.80\% had been enrolled for over three years.

%From 40 enrolled participants, 34 responses (85\%) were obtained for the pre-hackathon questionnaire, while 35 responses (87.5\%) were collected post-hackathon. The gender distribution among the participants indicated that 88.23\% were male and 11.76\% were female. In terms of age, most respondents fell within the 18-21 age range, accounting for 79.40\% of the participants. The 22-25 age range accounted for 14.70\% of the participants, while the 26-30 and over 30 age ranges represented 2.90\% each. Regarding the distribution of courses, 50\% of the participants were from the Information Systems course, 47.10\% were from Computer Science, and the remaining 2.90\% were from Mining Engineering. Regarding enrollment duration at university, 26.50\% had been enrolled for less than one year, 61.80\% had been enrolled for one to three years, and the remaining 11.80\% had been enrolled for over three years.

%Figure \ref{fig:results} shows the answers distribution of concerning to each analyzed questions, respectively covering : i) innovative and creative thinking (blue), ii) collaboration and teamwork (red), and iii) practical knowledge application (yellow). Concerning the elicited responses about the the innovative and creative thinking, we noticed that the answers ranged from 4 to 7, with the 85.7\% of the total responses, were clustered around the higher end of the scale (rated 6 or 7). This distribution reflects a generally positive perception among students regarding the hackathon's impact on their creative and innovative thinking. The average rating of approximately 6.125 indicates a strong overall level of stimulation experienced by the participants. The standard deviation of approximately 0.95 suggests that the responses were relatively tightly clustered around the mean, indicating a consistent perception among the participants regarding the hackathon's effectiveness in stimulating creativity and innovation. However, a closer examination of the data reveals nuances worth exploring. While most respondents rated their experience highly (mode = 7), the presence of a few lower ratings suggests variability in individual perceptions. Further analysis could delve into factors influencing these perceptions, such as team dynamics, project complexity, or personal creative inclinations, for example.

%Regarding how much did the hackathon provide a collaborative and teamwork environment, the answers ranged from 3 to 7, with a majority of responses clustered towards the higher end of the scale (rated 6 or 7). Specifically, 86.1\% of the responses fell into the 6 or 7 rating category, indicating a strong perception among participants regarding the hackathon's ability to provide a collaborative and teamwork-oriented environment. The average rating of approximately 6.29 further supports this positive perception, suggesting a significant level of collaboration experienced during the hackathon. The standard deviation, calculated to be approximately 1.36, indicates a moderate spread of responses around the mean, implying some variability in participants' experiences regarding collaborative and teamwork aspects. Further analysis could explore factors influencing these perceptions, such as team dynamics, communication strategies, or task allocation methodologies, providing valuable insights into the effectiveness of hackathons in fostering collaborative environments for software engineering students.

%Finally, the question covering how much did the hackathon allowed to apply the knowledge acquired in the classroom ranged from 1 to 7, with varying degrees of perceived ability to apply classroom knowledge during the hackathon. Notably, 47.1\% of responses fell into the lower end of the scale (rated 1 to 3), indicating a mixed perception regarding the extent to which the hackathon facilitated the application of classroom-acquired knowledge. The average rating of approximately 4.97 suggests a moderate level of success in applying classroom knowledge during the hackathon. However, the standard deviation, calculated to be approximately 2.03, indicates a considerable spread of responses around the mean, highlighting significant variability in participants' experiences regarding knowledge application. Further analysis could explore factors influencing these perceptions, such as the relevance of hackathon tasks to classroom learning, the depth of prior knowledge, or the availability of resources and support during the hackathon, offering valuable insights into optimizing the integration of classroom and practical learning experiences in software engineering education.

Figure \ref{fig:results} shows the distribution of responses for each question analyzed. The questions (Q1, Q2, and Q3) revolve around the innovative and creative thinking (blue), collaboration and teamwork (orange), and practical knowledge application (green). Looking at the responses for \textbf{innovative and creative thinking}, the answers ranged from 4 to 7. The majority of responses (85.7\%) were clustered around the higher end of the scale (rated 6 or 7), indicating a positive self-perception among students regarding the impact of the hackathon on their creativity and innovation. The average rating of approximately 6.12 suggests a strong overall level of stimulation experienced by the participants. The standard deviation of approximately 0.95 indicates that the responses were relatively tightly clustered around the mean, implying a consistent perception among the participants regarding the effectiveness of the hackathon in stimulating creativity and innovation. However, there were a few lower ratings, suggesting variability in individual perceptions. Of course, further analysis could explore factors influencing these perceptions, such as team dynamics, project complexity, or personal creative inclinations.

Moving on to the responses for \textbf{collaboration and teamwork}, the answers ranged from 3 to 7. The majority of responses (86.1\%) were clustered towards the higher end of the scale (rated 6 or 7), indicating a strong self-perception among participants regarding the hackathon's ability to enact collaboration and teamwork. The average rating of 6.29 further supports this positive perception, suggesting a higher level of collaboration experienced during the hackathon. The standard deviation, calculated to be approximately 1.36, indicates some variability in participants' experiences. Further analysis could explore factors influencing these perceptions, such as team dynamics, communication strategies, or task allocation, providing potential insights into the effectiveness of hackathons in fostering collaborative environments for SE students.

Finally, the responses for \textbf{practical knowledge application} ranged from 1 to 7. Notably, 47.1\% of responses fell into the lower end of the scale (rated 1 to 3), indicating a mixed perception regarding the extent to which the hackathon facilitated the application of classroom-acquired knowledge. The average rating of approximately 4.97 suggests a moderate level of success in applying classroom knowledge during the hackathon. However, the standard deviation, calculated to be approximately 2.03, indicates a certain variability in participants' experiences regarding knowledge application. One  reason for this variation is that 26\% of the participants were first-year students with limited experience in course subjects. Further analysis could investigate the correlation between classroom instruction and hackathon tasks.%, alongside the impact of prior knowledge.

%relevance of hackathon tasks to classroom learning and the depth of prior knowledge.


%One possible reason for this variation may be due to the fact that 26\% of the participants were in their first year of graduation, with little experience in the course subjects. Further analysis could explore factors influencing these perceptions, such as the relevance of hackathon tasks to classroom learning and the depth of prior knowledge

%, or the availability of resources and support during the hackathon.

%, offering valuable insights into optimizing the integration of classroom and practical learning experiences in software engineering education.

\begin{figure}[ht!]
  \centering
  \includegraphics[scale=1.07]{figures/plot.pdf}
      \caption{Distribution of responses for each question.}
  \label{fig:results}
\end{figure}

\vspace{-0.4cm}
\section{Looking to the Future}

%This study aims to raise awareness and discuss a forward-thinking idea of the value of hackathons in helping SE students leverage soft skills through experiential learning. 
The research question behind this study is: ``\textit{Can participation in hackathons lead to a self-perception of improvement in soft skills for SE students?}''. To answer this question, we provide here an analysis covering what we found in our preliminary case study backed by what the literature has been discussing about this subject. We opted to anchor this analysis within the Self-Determination Theory (SDT) framework \cite{deci2012self} due to its well-known success in depicting motivation as a driver for sustained engagement and learning. %The insights may contribute to depicting new directions concerning the role of hackathons on the shaping of soft skills in SE students.

%SE education, especially in the context of soft skills.


%In particular, our choice by the SDT derives from the fact that this theory emphasizes the fundamental role of motivation for sustained engagement and learning \cite{guay2022applying}. 

%encompassing two key aspects: a review of existing literature and insights from our preliminary study. We anchor our analysis within the Self-Determination Theory (SDT) framework \cite{deci2012self}. In particular, our choice by the SDT derives from the fact that this theory emphasizes the fundamental role of motivation for sustained engagement and learning \cite{guay2022applying}. 

%In this sense, several studies have shown that when people are autonomously motivated, either by intrinsic motivation or well-internalized (thus autonomous) forms of extrinsic motivation they display higher interest.

%Moreover, SDT fits very well with our research objective since this theory is focused on varied forms of motivation (from autonomous to controlled motivation) to predict outcomes such as performance, engagement, and learning \cite{}. 

%Developed by Edward Deci and Richard Ryan, 

%SDT has emerged as a highly important theory of human motivation and wellbeing \cite{ryan2022self}. 

In summary, SDT offers a psychological lens through which to examine human motivation and behavior, distinguished into \textit{intrinsic motivation} (doing something for its own sake, out of interest and enjoyment), \textit{extrinsic motivation} (doing something for an instrumental reason), and \textit{amotivation} (lacking any reason to engage in an activity) \cite{ryan2023self, ryan2022self}. According to Deci and Ryan \cite{deci2012self}, SDT posits that individuals are driven by three psychological needs: autonomy, competence, and relatedness. These needs impact motivation, which influences outcomes (in our research, soft skills improvement). In addition, both these needs and motivation might be influenced by the uncertainty and interdependence issues. Inspired by the work of Gagné et al. \cite{gagne2022understanding}, we approach below the SDT concepts previously explained (depicted in Figure \ref{fig:sdt}) in order to understand how hackathons may shape the soft skills of SE students.

\begin{figure}[ht!]
  \centering
  \includegraphics[scale=1]{figures/sdt.png}
      \caption{Primary concepts of Self-determination Theory \cite{gagne2022understanding}}
  \label{fig:sdt}
\end{figure}

%in the context of hackathons to shape the soft skills of SE students. 
Hackathons have been found to impact student motivation \cite{araujo2024can, steglich2020hackathons}, which may be influenced by factors such as \textbf{uncertainty} and \textbf{interdependence} \cite{ryan2022self}. As discussed by Griffin et al. \cite{griffin2007new}, higher levels of uncertainty require adaptive behaviors, while interdependence levels necessitate social, team-oriented, and network-oriented behaviors. Based on this comprehension, we may assume that the  dynamic nature of hackathons provides an ideal environment for students to develop and practice their soft skills, including how to handle uncertainty and interdependence effectively with their teams. Our preliminary results, for example, indicated that most participants (86.1\%) rated the hackathon as capable of providing a collaborative and teamwork-oriented environment.

The uncertainty and interdependence features, in turn, influence autonomy, competence, and relatedness \cite{ryan2022self}. \textbf{Autonomy} refers to feeling one has a choice and is willingly endorsing one's behavior \cite{wang2019competence, deci2000and}. In hackathons, autonomy could be reflected in students' ability to self-manage their capability of building a real-world solution in practice, including taking ownership of their learning process \cite{lifshitz2021minimal}. Thus, fostering autonomy in hackathons may promote intrinsic motivation, driving soft skill development as students demonstrate their abilities in a self-directed environment \cite{alkema2017agile}.

In addition, \textbf{competence} refers to the experience of mastery and being effective in one’s activity \cite{wang2019competence, deci2000and}. Hackathons are widely recognized for their ability to foster a multidisciplinary and competitive scenario in which students can showcase their skills \cite{mhlongo2020effectiveness, moshirpour2023multidisciplinary}. As students successfully overcome challenges and contribute meaningfully to their team projects, they experience a sense of competence, reinforcing their learning outcomes. Our case study found that students responded positively to improvements in innovative and creative thinking, collaboration and teamwork, and practical knowledge application, for example.  

Moreover, \textbf{relatedness} refers to the need to feel connected and a sense of belongingness with others \cite{wang2019competence, deci2000and}. In particular, hackathons leverage relatedness by encouraging collaboration, teamwork, and interaction with peers, mentors, and industry professionals \cite{kumalakov2018hackathon}. Through shared experiences, feedback exchanges, and collective problem-solving, students develop a sense of belonging and connectedness, which help to enhance their soft skills \cite{steglich2021online, afshar2022hackathon}.

As previously explained, fulfilling the psychological needs of competence, autonomy, and relatedness affects \textbf{motivation}, which in turn influences the \textbf{outcomes} achieved. Multiple factors could impact students' motivation and, consequently, their learning process \cite{reeve2002self}. In our case study, we noticed different lessons learned. Firstly, the involvement of professors and students in the organizational committee was important, promoting a sense of belonging and ownership among all the stakeholders. Moreover, since this hackathon was the first held on the university campus, most students experienced this type of event for the first time, potentially amplifying their engagement. Engaging alumni from the university enhanced the sense of community and facilitated knowledge sharing. However, alongside these benefits, different challenges emerged. Maintaining student motivation and communication during the hackathon was challenging due to concurrent demands (academic, personal, professional, etc.). Lastly, participants' li\-mited experience with hackathon formats also accentuated the need for eventual introductory workshops to previously cover fundamental topics (\textit{e.g.}, product management or programming), enhancing their readiness for such intensive events.

%Firstly, the fact of involving some professors and students in the organizational comittee. This fact promoted to the students a sense of belonging and sense of making the things work. Moreover, this hackathon was the first one in the campus. So, for the majority of the students was the first time of participating in a hackathon. Moreover, involving alumni of the university increased the sense of community and knowledge sharing. Of course, on the other hand, there were several challenges. For example, keeping students motivated and commmunicating between them to accomplish the hackathon alongside the week, even with other demands (classes, work, etc). Another challenges was the own lack of experience of the participants with the hackathon format. We noticed that some prior workshops, for example, could be useful to teach students some basic issues related to product management, software development, etc.

%As a concrete plan for the future, this study emphasizes the importance of ensuring the satisfaction of participants' psychological needs and avoiding factors that may undermine them during hackathons.

%Organizational research has consequently focused on managerial and leadership behaviours that support or thwart these needs and promote different types of work motivation. 

%There is also substantial research on the effects of work design (the nature and organization of people’s work tasks within a job or role, such as who makes what decisions, the extent to which people’s tasks are varied, or whether people work alone or in a team structure) and compensation systems on need satisfaction and work motivation. 

%we argue in favor of understanding these psychological needs to determine how to drive motivation and increase the possibility to enhance students' soft skills with the support of hackathons.
%\begin{framed}
%\noindent
As a plan for the future, we advocate for understanding psychological needs to drive motivation and enhance students' soft skills through hackathons. Grounded on the discussions derived from the literature, particularly the SDT, and our preliminary results, we may answer our research question by saying that hackathons are able to lead a self-perception of improvement in soft skills for SE students. However, the absence of statistical significance requires careful interpretation and hinders definitive conclusions. We acknowledge that these findings represent an initial stage, but they have also led us to consider another question for the future: \textit{How can hackathons be designed and implemented to better meet SE students' competencies, autonomy, and relatedness, and consequently enhance their soft skills?}


%\textit{How hackathons could be designed and implemented to be proactively adapted better to meet SE students' competencies, autonomy, and relatedness, and, consequently, enhance their soft skills}?


%\end{framed}
%As a concrete plan for the future, we argue in favor of understanding these psychological needs to determine how to drive motivation and increase the possibility to enhance students' soft skills in practice. Based on the previous discussions derived from the literature and our preliminary results, we may answer our research question by saying that participating in a hackathon may improve soft skills. We acknowledge that these results represent an initial stage, but they have also led us to consider another important question for the future: \textit{how hackathons could be designed and implemented to be proactively adapted better to meet students' competencies, autonomy, and relatedness, and, consequently, enhance their soft skills}?

%As a concrete plan for the future, we argue in favor on the importance of understanding and addressing these psychological needs to drive motivation influence the quality and extent of the capability of bridging the gap between technical proficiency and soft skills in practice. Based on the previous discussions derived by the literature and our preliminary case study, we may answer our research question by saying that participating in a hackathon may lead to an improvement in soft skills. However, looking to the future, another promising questions remains: \textit{how hackathons could be designed and implemented to be proactively adapted to better meet students competencies, needs and values, and, consequently enhance their soft skills}?

\section{Conclusion}
%Software Engineering (SE) education faces recurrent challenges due to the fast-paced dynamics intrinsic to the software industry. While the foundational principles of SE remain robust, new topics, demands, and educational approaches continually emerge to facilitate learning. More recently, SE scholars and practitioners have been emphasizing the relevance of soft skills in SE practice and the importance of students developing these skills throughout their undergraduate courses. However, teaching soft skills is complex as they cannot be acquired passively through raw knowledge. Consequently, different pedagogical approaches have been proposed to teach and practice soft skills.

%. Drawing on Self-Determination Theory and preliminary case study findings, we highlight how hackathons foster an intensive, collaborative environment where students engage in software development activitie

In this paper, we embrace the idea that hackathons could be a valuable experiential learning strategy for shaping soft skills in SE students in practice. Drawing on Self-Determination Theory and preliminary case study findings, we highlight how the investigated hackathon served as an enabler of an intensive, collaborative environment, where students engaged in software development activities that could be similar to real-world nuances, such as time pressure, collaborative teamwork, innovative thinking, etc. Lastly, future work could investigate how different hackathon formats (e.g., in-person vs. online, short-term vs. extended) potentially affect skill improvement.

%In this idea paper, we embrace that hackathons could be a valuable experiential learning strategy for developing soft skills in SE students. With the support of Self-Determination Theory, we approached the relevance of hackathons provide an intensive, dynamic, and collaborative environment where students engage in software development activities that simulate real-world contexts.  

%We delivered in this work an overview of the literature on soft skills and hackathons in SE education. Then, we reported preliminary empirical evidence from a seven-day hybrid hackathon involving 40 students. We assessed how the hackathon experience promoted innovative and creative thinking, collaboration and teamwork, and knowledge application among participants. Overall, we found that most responses were rated 5, 6, or 7, showing positive self-perception among students regarding hackathon impact on the evaluated skills. Lastly, we analyzed our findings through the lens of Self-Determination Theory (SDT), offering a psychological perspective to understand learning and human behavior. 

%Future work would explore integrating SDT to better understand how hackathons enhance technical and non-technical competencies in SE education. This perspective includes examining the long-term impact of hackathons on soft skill development in SE students as well as investigating how different hackathon formats (e.g., in-person vs. online, short-term vs. extended) potentially affect skill improvement.

%By embracing SDT to anchor our analysis, we reflected the growing attention the SE research community had given to behavioral aspects within the growing field known as Behavioral Software Engineering (BSE). 

%In this sense, SDT appeared particularly valuable due to its focus on understanding motivation as a fundamental driver for engagement and learning. Consequently, our forward-looking idea and future directions were grounded in a robust theoretical framework.

%This study delves into the motivations of Software Engineering (SE) students participating in a hackathon. The research examined the perceptions of participants both before and after the event, providing an understanding of the factors driving their participation. The pre-hackathon phase reveals a diverse range of motivations, including intrinsic growth aspirations and extrinsic factors, reflecting the complex nature of participants' intent. Post-hackathon responses highlight a remaining positive outlook on their motivations. These findings align with existing evidence emphasizing the benefits of hackathons in promoting intrinsic motivation, collaborative learning, and skill development. In the context of SE education, hackathons are particularly valuable as they equip students with technical expertise and soft skills demanded by industry. % and innovative thinking, crucial attributes for success in the software industry dynamics.

%However, this preliminary research has limitations. The reliance on self-reported data and a single-institution focus may limit the broader applicability of findings. The lack of a control group also prevents definitive conclusions about the causality of hackathons on motivation. Moreover, the lack of statistical significance needs cautious interpretation. This issue suggests the potential influence of factors like sample size on the ability to detect significant changes. However, students reported a high academic engagement, with an average rating of 5.97 out of 7. Despite these limitations, the use of the Academic Motivation Scale (AMS) enabled us to enhance our comprehension of hackathons as a motivational tool, also furnishing actionable inputs to explore avenues for improving the design and impact of hackathons in SE education. 
%Our study adds to the existing groundwork on behavioral aspects in SE education and paves the road for future studies exploring motivational factors. As for future work, we may expand our scope by investigating a wider range of hackathons and diverse settings (different institutions, skill levels, etc.).

%%
%% The next two lines define the bibliography style to be used, and
%% the bibliography file.

{\footnotesize\bibliography{bibliography}}
\bibliographystyle{IEEEtranS}
%\bibliographystyle{plain}

\end{document}
