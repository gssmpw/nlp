\section{Background and Related Work}
%This section addresses the challenge of teaching soft skills in SE. Then, we explore the educational and promising role of hackathons in SE as an experiential learning strategy.

%intersection between hackathons and SE education. Ahead, we delve into the literature on learning soft skills in the context of SE.

%\subsection{Soft Skills in SE Education}
\subsubsection{Unveiling the Challenge: Enhancing Soft Skills in SE Students}

Traditionally, SE programs have primarily emphasized technical skills (also known as ``hard skills'') like programming, software testing, and system architecture. However, the constantly evolving nature of the software industry and the collaborative demands of modern workplaces have emphasized the importance of soft skills. These skills encompass a broad range of characteristics, including personality traits, social interaction abilities, communication, and personal habits \cite{ahmed2012evaluating}. According to Kechagias \cite{kechagias2011teaching}, soft skills can be defined as ``intra and inter personal (socio-emotional) skills'' that are essential for personal development, social participation, and workplace success. Capretz and Ahmed \cite{capretz2018call} advocate for promoting soft skills in SE to enhance career prospects and better prepare practitioners for the industry's challenges. This need is also driven by the increasing complexity and interdisciplinary nature of software projects \cite{matturro2019systematic, miranda2021compreendendo}.



%Capretz and Ahmed \cite{capretz2018call} advocate for promoting soft skills in SE to enhance career prospects and better prepare practitioners for the industry's challenges and rapid evolution. Another key driver of this shift is the complexity and interdisciplinary nature of software projects \cite{matturro2019systematic}.

%Consequently, SE scholars are striving to balance the development of both technical and soft skills to better prepare students for real-world challenges.

%Successful software development depends on how software practitioners perform their tasks and how they interact with their peers \cite{capretz2018call}. As stated by \citet{ahmed2012evaluating}, ``software is a by-product of human activities, such as problem solving capabilities, cognitive aspects, and social interaction''.

%The job requirements in the field of SE usually divide the skill requirements into the two categories of “hard skills” (technical requirements and knowledge a person should have to carry out a task) and “soft skills” (a broad range of characteristics involving personality types, social interaction abilities, communication, and personal habit) \cite{ahmed2012evaluating}. In line with this definition, \citet{kechagias2011teaching} claim that soft skills could be understood as “intra and inter personal (socio-emotional) skills” essential for personal development, social participation and workplace success. While much of our SE research is technical and comprise hard skills, the importance of addressing soft skills has been also attracting considerable attention both by industry and academy \cite{matturro2015soft}.

From an educational lens, teaching soft skills presents challenges because these skills cannot be passively acquired through theoretical knowledge alone \cite{martins2007role}. SE scholars have been investigated active learning methods to enhance students' soft skills. These proposals include, for example, project-based learning \cite{gonzalez2011teaching}, game-based learning \cite{garcia2020effects}, and problem-based learning \cite{yu2016developing}.  While these approaches are valuable and contribute to soft skill development, they may not fully replicate the fast-paced, high-pressure, and collaborative nature of software projects. Hackathons, in contrast, provide an immersive environment where students must rapidly adapt and collaborate under specific constraints, making them a promising complementary strategy for enhancing soft skills in SE education \cite{porras2018hackathons, steglich2021online}.




%One relevant reason for this shift is the complexity and interdisciplinary nature of software projects. Recently, Capretz and Ahmed \cite{capretz2018call} issued an open call to promote soft skills in SE. This perspective allows to enhance career prospects and contributes to the development of SE practitioners better prepared to deal with the challenges and rapid pace of the software industry \cite{matturro2019systematic}. %They argue that successful software development depends not only on the technical execution of tasks but also on how software practitioners interact with their peers. %Within a general scope, \citet{bolton2009people} reported that 80\% of individuals who fail at work do not fail due to their lack of technical skills but rather because of their inability to relate well with others.

%This research agenda on the importance of incorporating soft skills into SE education has been gaining large traction over the years. In this regard, Sedelmaier and Landes \cite{sedelmaier2014practicing} argue that soft skills are core competencies for software engineers and should be integral to SE education. This perspective allows to enhance career prospects and contributes to the development of SE practitioners better prepared to deal with the challenges and rapid pace of the software industry \cite{matturro2019systematic}.

%Recently, the Experiential Learning (ExL) have been growing as a means to cultivate soft skills in SE students. ExL is rooted in the concept of learning through direct experience, specifically defined as "learning through reflection on doing" By involving students in hands-on activities and encouraging reflection, ExL facilitates the connection between theoretical concepts taught in the classroom and real-world scenarios. As we can see, this approach closely mirrors the essence of hackathons, where participants engage in intense problem-solving sessions and subsequently reflect on their experiences.

%More recently, a particular stream of research has been advocating for Experiential Learning  (ExL) to foster soft skills in SE students. In summary, ExL ise based on the process of learning through experience, and is more narrowly defined as ``learning through reflection on doing''. By engaging students in hands-on experiences and reflection, they are better able to connect theories and knowledge learned in the classes to real-world situations. As we can see, this idea closely aligns with the context of hackathons.


%\citet{matturro2019systematic}, for example, reported the results of a systematic mapping study to identify existing research on soft skills in SE and to determine what soft skills are considered relevant to the practice. In another study, \citet{matturro2013soft} reviewed a set of jobs advertisements offering job positions related to SE in order to identify what soft skills are most in demand by software companies in Uruguay. This method of reviewing job requirements to understand the role of soft skills in software industry was also tackled by \citet{ahmed2012evaluating}. In their work, they reported some of the vital statistics from industry about the requirements of soft skills in various roles of software development phases. They also highlighted the variation in the types of skills required for each of the roles. More recently, \citet{galster2022soft} used a qualitative research method and manually analyzed 530 job adverts from New Zealand’s largest job portal to understand what the employers demand from professionals beyond “hard skills”.

%Lorem ipsum dolor sit amet, consectetur adipiscing elit. Nulla viverra enim a pretium luctus. In at erat leo. Cras non semper urna. Curabitur pellentesque quis tortor a varius. Sed in fermentum nisl. Nulla facilisi. Curabitur leo purus, tristique quis mauris vitae, ultrices pretium ipsum. Maecenas nec molestie quam. Interdum et malesuada fames ac ante ipsum primis in faucibus. 


\subsubsection{Hackathons in SE Education: Leveraging Experiential Learning}
The importance of developing soft skills in SE students has also been successfully linked to Experiential Learning (ExL) \cite{fioravanti2023software}. ExL is based on learning through direct experience, specifically defined as ``learning through reflection on doing'' \cite{bradberry2019learning}. By involving students in hands-on activities and encouraging reflection, ExL helps bridge the gap between theoretical concepts taught in the classroom and real-world scenarios. This educational approach resembles the essence of hackathons, where participants engage in intense problem-solving sessions and later reflect on their experiences.

Researchers have been closely studying the pedagogical potential of hackathons in SE education, focusing on various aspects such as skill development, student perceptions, and curriculum integration. This focus on hackathons stems from their ability to provide a rich experiential learning approach where students can apply knowledge in practice. Porras et al. \cite{porras2018hackathons} conducted a literature review, utilizing their extensive learning experiences and written and interview materials from students and industry participants to present an overview of hackathons. In another study, Porras et al. \cite{porras2019code} discussed a systematic literature review on the use of hackathons in SE and computer science education. 


%For example, \citet{porras2018hackathons} provided an overview through a literature review, drawing from various materials from students and industry participants. In another study, \citet{porras2019code} conducted a systematic review to explore the use of hackathons and code camps in SE and computer science education. 

Other works have focused on the impact of hackathons on student skills and perceptions.  Steglich et al. \cite{steglich2020hackathons} conducted a case study to investigate factors influencing student participation, their perceptions, and the SE practices they adopted. Steglich et al. \cite{steglich2021online} further examined the professional skills developed during hackathons and how intense collaboration occurs in atypical scenarios. In addition, Haque et al. \cite{haque2022effectiveness} assessed students' ability to apply SE best practices learned in their courses to hackathon settings, demonstrating the practical application of these skills.

Hackathons have also been integrated into SE curricula.  Sadovykh et al. \cite{sadovykh2020hackathons} introduced hackathons as part of an SE program, proposing a specific setup and evaluating outcomes through surveys from students, mentors, and industry representatives. They reported high satisfaction and suitability for final evaluations in Software Quality courses. Sadovykh et al. \cite{sadovykh4hackathon} also found that hackathons as exam replacements motivated students to learn beyond the course material. Gama et al. \cite{gama2018hackathon} proposed a methodology for project-oriented undergraduate courses, using hackathons as realistic scenarios to practice class knowledge, and evaluated its use in an IoT course through quantitative and qualitative analyses. Kumalakov et al. \cite{kumalakov2018hackathon} explored a blended hackathon model to increase student engagement and interest in computer science, with preliminary findings indicating positive results.

In terms of practical applications, Afshar et al. \cite{afshar2022hackathon} integrated a hackathon component into a software development and architecture course. Data collected from pre- and post-hackathon surveys and GitHub code commits demonstrated the impact of hackathon participation on student performance and understanding. La Place and Jordan \cite{la2022adapting} found that students who participated in hackathons and a project-based learning (PBL) SE degree developed transferable skills and employed various problem-solving approaches in both settings. Unlike traditional PBL, which focuses on long-term collaboration and gradual skill development, hackathons prioritize immediate problem-solving under time constraints \cite{szymanska2020effects}. This high-pressure setting inherent to hackathons enhances soft skills like communication, leadership, and conflict resolution, complementing the gradual learning process in PBL \cite{garcia2023fostering}.

\subsubsection{Research Gap} Although some papers have previously addressed this issue, there is a scarcity of studies specifically focusing on the role of hackathons in developing soft skills among SE students. We found no other work discussing these results through the lens of SDT. Further exploration of these aspects could help optimize hackathon design to enhance their educational impact and bridge the gap between technical proficiency and soft skills development in SE education.

%Even though some papers have already addressed this issue, there is a general scarcity of studies focusing specifically on the role of hackathons in developing soft skills among SE students. We also do not found other work discussing the results in the light of SDT. Continuing to examine these aspects can offer guidance on how to optimize hackathon structures to maximize their educational impact and, consequently, bridge the gap between technical proficiency and soft skills development in SE education.

%Even with som paper already addressing, there is a overall scarcity of studies whose primary aim is to investigate the role of hackathons in developing soft skills among SE students. More specifically, exploring these aspects can provide directions into optimizing hackathon structures to maximize their educational impact, thus bridging the gap between technical proficiency and soft skills development in SE education. 

%Sed in fermentum nisl. Nulla facilisi. Curabitur leo purus, tristique quis mauris vitae, ultrices pretium ipsum. Maecenas nec molestie quam. Interdum et malesuada fames ac ante ipsum primis in faucibus. 

%Researchers have been closely studying the pedagogical potential of hackathons in SE education. For example, \citet{porras2018hackathons} conducted a literature review, drawing from their years of learning experiences, as well as written and interview materials from students and industry participants, to provide an overview of hackathons and code camps. More ahead, \citet{porras2019code} conducted a systematic literature review to investigate the current state of using code camps and hackathons in SE and computer science education.

%\citet{steglich2020hackathons}, conducted a case study on an educational hackathon to understand the factors that may affect students' participation, their perceptions of the event, and the SE practices they adopt. They found that hackathons can serve practical pedagogical purposes since students believe they can improve their skills through participation in these competitions. In another study, \citet{steglich2021online} investigated the professional skills developed by students who participate in hackathons and how intense collaboration takes place in an atypical scenario while solving problematic situations. In turn, \citet{haque2022effectiveness} evaluated students' capability to apply the skills they learned in their courses to the hackathon setting, especially those related to SE best practices. 

%\citet{sadovykh2020hackathons} introduced hackathons as part of their SE program curriculum and proposed a specific hackathon setup. The outcome of the hackathon was evaluated based on a survey completed by students, mentors, and industry representatives. All groups reported high satisfaction with the hackathon. \citet{sadovykh4hackathon} also examined the educational contribution of hackathons as a replacement for exams in software engineering education. Their preliminary study at an IT university revealed a positive reception among students. They found that hackathons are suitable for final evaluations in Software Quality courses, as they motivate students to learn material beyond the course curriculum.

%\citet{yuen2021designing} showcased a real-life case study involving a hackathon for potential college students in a university's data science program. They offered a structured framework and suggests practical principles and guidelines for designing hackathons.  \citet{gama2018hackathon} introduced a methodology to be used in project-oriented undergraduate courses where a hackathon dynamic can be employed as a realistic scenario to practice the knowledge acquired in classes. They conducted a quantitative and qualitative analysis to evaluate the usage of the methodology in an IoT undergraduate course. \citet{kumalakov2018hackathon} investigated the proposition of a hackathon blended model as an appropriate method for increasing students’ interest and understanding in computer science and encouraging their engagement in using computing skills and knowledge. Preliminary findings have produced not significant, but positive results acceptable to apply hackathon to the curricular.

%\citet{afshar2022hackathon} integrated a hackathon component in a software development and architecture course curriculum to evaluate the effect of working on a real-world web development project in a hackathon setting on deepening the theoretical concepts learnt in lectures. The data is collected through two surveys which were accessible to students before and after the hackathon and students code commits on GitHub.  Research findings shown the importance of hackathon participation on students’ performance and state of knowledge.

%\citet{la2022adapting} builds upon a previously published work \cite{la2020wip} in progress, finding that students who attended hackathons and a project-based learning Software Engineering degree, built transferable skills between hackathons and capstones. Participants described the employment of software design methodologies in both hackathons and capstone projects, various problem solving and researching approaches used, and a distinct level of attention to the context of development, informal or formal, to determine their courses of actions.

%To the best of our knowledge, few studies claimed the usage of hackathons in the soft skills. Include here a concluding paragraph. 

%Lorem ipsum dolor sit amet, consectetur adipiscing elit. Nulla viverra enim a pretium luctus. In at erat leo. Cras non semper urna. Curabitur pellentesque quis tortor a varius.