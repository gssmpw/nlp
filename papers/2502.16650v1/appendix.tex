\appendix

\section{Key Management} \label{sec:management}

Key management in NR V2X communications over the PC5 interface is essential for establishing secure unicast links between UEs. According to 3GPP~\cite{3gpp.33.536}, each UE possesses long-term credentials ($K_{long-term}$), such as symmetric keys or asymmetric key pairs, which serve as the root of trust for mutual authentication. Through mutual authentication procedures leveraging these long-term credentials, UEs derive a shared NR PC5 root key ($K_{NRP}$), forming the basis for subsequent key derivations. A 32-bit identifier known as the $KNRP_{ID}$ is associated with $K_{NRP}$ to uniquely identify the root key in communications between the pair of UEs.

From the $K_{NRP}$, UEs derive a session-specific key ($K_{NRP-sess}$) for each unicast link to ensure key freshness and session uniqueness. The derivation of $K_{NRP-sess}$ involves the exchange of nonces between the UEs—each UE generates a random nonce, and these nonces are combined during the key derivation process to introduce randomness and prevent replay attacks. A 16-bit $KNRP{sess}$ ID, constructed by combining bits selected by each UE, uniquely identifies the session key. Using $K_{NRP-sess}$, UEs derive the NR PC5 Encryption Key (NRPEK) and the NR PC5 Integrity Key (NRPIK) by applying standardized key derivation functions. These keys provide the necessary cryptographic material for confidentiality and integrity protection of both signaling and user plane data over the PC5 interface. According to [Clause 5.3.3.1.2.1]: 

\begin{fancyquote}[]
The NR PC5 Encryption Key (NRPEK) and NR PC5 Integrity Key (NRPIK) are used in the chosen confidentiality and integrity algorithms respectively for protecting PC5-S signalling, PC5 RRC signalling, and PC5 user plane data. They are derived from $K_{NRP-sess}$ and are refreshed automatically every time $K_{NRP-sess}$ is changed. 
\end{fancyquote}

Security contexts are established and maintained for each unicast link, encompassing the derived keys, selected algorithms, and replay protection parameters. UEs manage these security contexts throughout the communication session, updating them during rekeying procedures and securely deleting them upon session termination to prevent residual vulnerabilities. Rekeying procedures can be initiated by either UE and involve generating new nonces to derive a fresh $K_{NRP-sess}$. Additionally, identity privacy is preserved through procedures that allow UEs to change and randomize their Layer-2 IDs and KNRP IDs during active sessions, preventing tracking and linkability attacks. Figure~\ref{fig:keys} show the hierarchy of all the keys used for communication establishment based on the specifications.

\begin{figure}[!t]
     \centering
     \includegraphics[width=0.8\columnwidth]{images/keys.png}
     \caption{Key hierarchy for 5G V2X.}
     \label{fig:keys}
\end{figure}

\begin{table*}[t]
\centering
\caption{Comparison of Related Works on V2X Sidelink Security, Grouped by Approach}
\label{tab:related-comparison}
\renewcommand{\arraystretch}{1.2}
\small

\begin{tabularx}{\textwidth}{l l c c c c c}
\toprule
\multirow{2}{*}{\textbf{References}} & 
\multirow{2}{*}{\textbf{Approach}} &
\multicolumn{5}{c}{\textbf{NR Vulnerabilities Addressed}} \\
\cmidrule(lr){3-7}
& & 
\makecell{\textbf{Sync} \\ \textbf{Attacks}} &
\makecell{\textbf{Resource} \\ \textbf{Spoofing}} &
\makecell{\textbf{HARQ} \\ \textbf{Spoofing}} &
\makecell{\textbf{L2 ID} \\ \textbf{Exposure}} &
\makecell{\textbf{PC5} \\ \textbf{Exploits}} \\
\midrule

\multicolumn{7}{l}{\textbf{- Survey-Based Approaches (Overviews of V2X Security)}} \\
%\midrule
\makecell[l]{~\cite{Lu20secoverview, Huang20advances, Luo20phy, alnasser19challengessolutions, marojevic18requirementsprocedures, Mohan22threats, Boualouache23challenges} \\ ~\cite{Yoshizawa23v2xsurvey, Ying24review, Sedar23survey, Hasan20v2x, GHOSAL20:V2X-Survey, Lai20:Challenges}} &
\makecell[l]{Surveys, \\ Literature Reviews, \\ General Risks, Threats \\ \& Requirements.} &
$\bm{\times}$ & 
$\bm{\times}$ & 
$\bm{\times}$ & 
$\bm{\times}$ & 
$\bm{\times}$ \\

\midrule

\multicolumn{7}{l}{\textbf{- Cryptographic / Trust-Based Approaches}} \\
%\midrule
\makecell[l]{~\cite{Pizzi21securedelivery, Zhang15SeDS, Suraci21d2d, Ahmed18secure, Liu21plattoning, Alnasser20trust}} &
\makecell[l]{Key Exchange, \\ Trust Models, \\
Secure Data. }&
$\bm{\times}$ & 
$\bm{\times}$ & 
$\bm{\times}$ & 
$\bm{\times}$ & 
$\bm{\times}$ \\

\midrule

\multicolumn{7}{l}{\textbf{- Simulation-Based Approaches}} \\
%\midrule
\makecell[l]{~\cite{Yiliang20secrecy, Trkulja20DenialofServiceAO}} &
\makecell[l]{Math Analysis of DoS \\ on Resources. \\ V2V Secrecy Rate Maximization.}&
$\bm{\times}$ & 
$\bm{\times}$ & 
$\bm{\times}$ & 
$\bm{\times}$ & 
$\bm{\times}$ \\

\midrule

\multicolumn{7}{l}{\textbf{- Partially Experiment-Based Approaches}} \\
%\midrule
\makecell[l]{~\cite{Twardokus22dos, Twardokus23dos}} &
\makecell[l]{DoS Attacks against Scheduling, \\ via Jamming and Exhaustion. \\ Detection Mechanisms.} &
$\bm{\times}$ & 
$\bm{\times}$ & 
$\bm{\times}$ & 
$\bm{\times}$ & 
$\bm{\times}$ \\

\midrule

\multicolumn{7}{l}{\textbf{- Spec-Based Approaches (Detailed 3GPP NR V2X Analysis)}} \\
%\midrule
\textit{Our Work} &
Procedure \& Protocol Assessment &
\checkonly & 
\checkonly & 
\checkonly & 
\checkonly & 
\checkonly \\

\bottomrule
\end{tabularx}
\end{table*}

\section{Further Discussion on Countermeasures} \label{sec:assess}

In this section we continue the discussion about the potential countermeasures on 5G NR V2X.

\textbf{Synchronization Protection.} Authentication at the physical layer can help prevent false SyncRef signals. For instance, including cryptographic signatures or certificate-based tokens in the synchronization signals (e.g., SLSS) ensures authenticity. However, integrating such security at the physical layer introduces overhead in terms of extra bits for signatures and potential timing delays in verifying them. UEs may also need more powerful hardware or firmware support, impacting cost and battery life. This modification will require change in the design and implementation of UE stacks. An alternative is a partial integrity tag that is smaller than a full signature but still provides basic tampering detection.

Regardless, achieving this, requires embedding cryptographic material—such as a message authentication code or a short digital signature—within an extremely limited payload (i.e., signal format). According to design constraints, physical-layer sidelink messages typically have small bit budgets (e.g., tens of bits). SCIs could be more flexible in sidelink compared to the DCIs in conventional architectures (There is no DCI in sidelink, unless a gNodeB is involved in the Uu interface.), however the space limits are equally relevant. From a performance standpoint, adding any cryptographic field at the physical layer increases both computational and timing overhead. Even a small MAC calculation typically may require an extra hashing pass (e.g., HMAC with a 128-bit secret key), which must be computed by the SyncRef UE and verified by all receiving UEs within the tight synchronization window. Based on LTE/NR reference timescales, this check must be completed within milliseconds—any cryptographic validation that overshoots that boundary risks delaying the entire sidelink synchronization procedure. In resource-constrained UEs (especially in high-speed vehicular scenarios), these additional cycles could marginally raise the UE’s power consumption or reduce the effective throughput on other parallel sidelink channels. Design trade-offs need careful field evaluation to confirm that the overhead—both in bits and processing time—remains acceptable while still significantly reducing the risk of malicious synchronization injection. 

Protecting the physical layer has been discussed by past works~\cite{oh24:enablingphysicallocalization, Ross24:Broadcasts, ludant24unprotected4g5g} on conventional LTE/5G implementation, however it remains unclear whether such an implementation is applicable to sidelink and whether its adoption will be accepted. As we have already mentioned, sidelink introduces new uses cases, threats model and risks, consequently a thorough investigation of physical layer protection specifically on sidelink is paramount. Therefore, the exploration of physical layer security remains a future work.

\begin{table*}[!t]
\centering
\caption{SCI format 1-A fields for NR V2X and potential manipulations}
\label{tab:sci1a-format}
\renewcommand{\arraystretch}{1.3}
\small
\begin{tabularx}{\textwidth}{l l X}
\toprule
\textbf{Parameter} & \textbf{Bit Length} & \textbf{Attacker Manipulation \& Relevance} \\
\midrule

\textbf{Priority} &
3 bits &
Spoofing a higher/lower priority could mislead UEs about traffic importance. \\

\midrule

\textbf{Frequency Resource Assignment} &
\makecell[l]{relies on \texttt{sl-MaxNumPerReserve}} &
High impact: forging frequency allocations can cause UEs to perceive subchannels as occupied, leading to resource blocking or collisions. \\

\midrule

\textbf{Time Resource Assignment} &
\makecell[l]{5 or 9 bits \\ (relies on \texttt{sl-MaxNumPerReserve})} &
High impact: specifying multiple or extended time slots artificially reduces available resources for legitimate UEs. \\

\midrule

\textbf{Resource Reservation Period} &
\makecell[l]{$\log_2(\texttt{\#PeriodListEntries})$ bits \\ (if used)} &
High impact: setting a large reservation period (RRI) makes UEs believe resources remain taken for a long duration. \\

\midrule

\textbf{DMRS Pattern} &
$\log_2(N_{\text{DMRSPattern}})$ bits &
May affect demodulation reference signals; not crucial for blocking. \\

\midrule

\textbf{2nd-Stage SCI Format} &
2 bits &
This points to second-stage parameters, and could be used if an attacker wants to go further than just spoofing. \\

\midrule

\textbf{Beta\_offset Indicator} &
2 bits &
Modifies power offset for second-stage SCI; minimal effect on resource blocking. \\

\midrule

\textbf{Number of DMRS Port} &
1 bit &
Indicates rank-1 or rank-2 DMRS usage, not key for blocking. \\

\midrule

\textbf{Modulation and Coding Scheme (MCS)} &
5 bits &
Misrepresenting MCS might cause decoding issues, but doesn’t fundamentally block resources. \\

\midrule

\textbf{Additional MCS Table Indicator} &
\makecell[l]{1 bit (if one table) \\2 bits (if two tables) \\0 otherwise} &
References advanced MCS tables; not central for resource blocking. \\

\midrule

\textbf{PSFCH Overhead Indication} &
\makecell[l]{1 bit (if \texttt{sl-PSFCH-Period} = 2 or 4), \\ else 0} &
Might claim overhead is large, but frequency/time fields remain the main vector for blocking. \\

\bottomrule
\end{tabularx}
\end{table*}

\textbf{HARQ Protection.} Similarly, protecting HARQ feedback (ACK/NACK) with cryptographic material from spoofing entails adding integrity checks or authentication tokens to a message that is notoriously small and time-sensitive. Typical HARQ feedback bits must be transmitted and processed within a short feedback window—on the order of few milliseconds (depends on the implementation and device though)—to meet NR’s low-latency requirements, which leaves little room for cryptographic overhead. Even appending a minimal 16–32 bit integrity field (if feasible) could significantly increase the per-packet overhead, particularly since HARQ operates in rapid, repeated cycles. Apart from significant design modifications, hardware constraints further complicate this approach, as UEs must compute or verify any authentication field (e.g., a lightweight MAC) in near real-time, risking missed timing deadlines if cryptographic operations are too slow.

From a resource standpoint, HARQ feedback typically has only a few bits for signaling ACK/NACK bursts. Extending it to include cryptographic information might crowd out existing fields or require additional sidelink symbols, cutting into spectral efficiency. In addition, because HARQ processes occur repeatedly with each transmission block, even a modest increase in per-feedback processing can accumulate, raising UE power consumption and potentially lowering throughput if the UE or network must account for these extra checks. A possible intermediate solution would be to rely on physical layer anomaly detections (e.g., verifying consistent transmission power, scrambling patterns, or channel estimates from the legitimate UE), that are valid only within a strict bound time window. 

While physical-layer validation of parameters, such as transmission power and modulation consistency, and use of directional antennas could be effective in detecting anomalous ACK/NACK signals, timing constraints for HARQ feedback are still extremely tight, often within a few of milliseconds window. While not as robust as full digital signatures (due to potential false positives/negatives), these approaches could help maintain real-time performance better than time-consuming cryptographic operations without compromising reliability in NR V2X environments as much. Nevertheless, such measures need to tested under realistic V2X scenarios to determine their robustness, and their potential advantages.

\textbf{Securing PC5 Messages.} Completely encrypting or authenticating pre-security-establishment PC5 messages can be impractical due to design changes, limited overhead budgets and the need for rapid session initiation in sidelink Mode 2. Instead, applying robust verifications—e.g., through timestamps, short sequence numbers, or stateful connection checks—can catch replayed or conflicting messages at relatively low overhead. These measures involve maintaining lightweight state on each UE (e.g., tracking recent message IDs), which adds memory and processing cost but remains significantly less demanding than full cryptographic protection.

At the same time, traffic pattern analysis (e.g., verifying that message frequencies align with known V2X protocols) imposes additional computational overhead, especially in high-density scenarios where each UE sees numerous sidelink exchanges. However, such analysis could be integrated into existing MAC or RRC procedures with minimal modifications, providing a feasible way to detect anomalies without large cryptographic fields or repeated key negotiations. In dense vehicular networks, each UE must ensure that any extra checks do not inflate connection setup times beyond acceptable bounds—particularly if the sidelink interface is used for safety-critical messages. By combining these detection methods, UEs can reduce the likelihood of MitM or malicious attachment attacks while keeping the per-message overhead relatively small.

\textbf{Preventing UE Tracking.} Frequently changing Layer-2 identifiers (e.g., via the Direct Link Identifier Update procedure) is a crucial step in thwarting adversarial tracking. However, each identifier update generates additional signaling overhead—both in the link-layer control plane (e.g., updating mapping tables) and in the application layer (if connections must be re-established). In high-traffic NR V2X environments, performing these updates too often can lead to noticeable latency spikes, as UEs must temporarily pause or reconfigure ongoing transmissions to synchronize the new identifiers among peers. Moreover, cryptographically secure randomization of each new identifier requires on-device generation of random numbers, which may be hardware-accelerated or might rely on the UE’s CPU, thus potentially affecting battery life and throughput if done at short intervals.

For groupcast or broadcast-based services, using shared group identifiers can hide individual UE identities but could reduce the precision of certain procedures (e.g., selective HARQ or targeted resource allocation). This trade-off may increase collision risk or complicates error recovery, especially as the network or autonomous Mode 2 relies on acknowledging specific UEs’ receptions. Consequently, it is currently unclear how this measure can be precisely implemented and in which use cases shared group identifiers can be used to protect the sidelink network.

Consequently, we could adopt an intermediate approach—where the UE employs moderately timed privacy timers (e.g., tens of seconds) combined with partial randomization—strikes a compromise, limiting the exposure window while keeping overhead manageable for real-time vehicular operations. As already mentioned though, the lack of specific design instructions and directives (design deficiencies) open the room for implementation flaws. The current 3GPP specifications do not establish secure generation and management procedures of such identifiers, let alone evaluating a potential trade-off between security and performance.

\textbf{General Protocol Enhancements.} Enforcing stricter security policies—such as eliminating NULL ciphers for ordinary sidelink communications and mandating authentication over the PC5 interface—can significantly reduce exploitability and is significantly less impactful on performance compared to the aforementioned measures. Nonetheless, we should keep in mind that: (1) these countermeasures do not solve the previous security flaws at the physical and MAC layers, and (2) in extremely time-sensitive V2X contexts, authentication, ciphering and integrity-protection can still cause delays in session initiation and communication and burden UEs with more frequent cryptographic operations, potentially impacting real-time performance. 

By logging security-related events (e.g., suspicious message sequences, repeated failed integrity checks), UEs can detect and respond to incidents more effectively. However, storing logs in high-throughput vehicular environments may require on-device memory, and analyzing them in real-time can consume processing cycles, implying a trade-off between thorough incident tracing and maintaining low latency. Similarly, the ability to quarantine or isolate suspicious UEs demands either network coordination or robust local procedures. Nonetheless, such methods are crucial for long-term resilience: once a malicious UE is identified, promptly notifying relevant parties (e.g., a back-end security server or the local cluster of vehicles) can avert widespread disruption. 

While these protocol enhancements may impose extra overhead and complexity, they help against the cause of sidelink vulnerabilities stemming from insufficient security defaults and permissive configuration options. Nonetheless, more investigation is needed in order to determine their practicality in real V2X scenarios.

\begin{table*}[!t]
\centering
\caption{SCI format 2-A fields for NR V2X with potential security implications. This message may be used for additional HARQ manipulation, even though HARQ relies on PSFCH for ACK/NACK signaling.}
\label{tab:sci2a-format}
\renewcommand{\arraystretch}{1.3}
\small
\begin{tabularx}{\textwidth}{l l X}
\toprule
\textbf{Parameter} & \textbf{Bit Length} & \textbf{Attacker Manipulation \& Relevance} \\
\midrule

\textbf{HARQ Process Number} &
4 bits &
Identifies the HARQ process for the current data block. Spoofing might confuse the transmitter about which HARQ process is active, potentially causing retransmission misalignment. \\

\midrule

\textbf{New Data Indicator (NDI)} &
1 bit &
Signals if the current TB is a new transmission or a retransmission. Forging this bit could mislead the receiver into treating packets incorrectly (e.g., discarding a new TB or expecting old data). However, this is \emph{not} the ACK/NACK feedback. \\

\midrule

\textbf{Redundancy Version (RV)} &
2 bits &
Specifies which redundancy version (out of 4) is used if it is a retransmission. An attacker manipulating RV could corrupt the receiver’s decoding process, although it mainly impacts HARQ efficiency rather than directly blocking resources. \\

\midrule

\textbf{Source ID} &
8 bits &
Indicates the UE sending the transport block. Spoofing could impersonate or conflate multiple sources, enabling replay or identity-based confusion. \\

\midrule

\textbf{Destination ID} &
16 bits &
Indicates the target UE/group. \\

\midrule

\textbf{HARQ Feedback Enabled/Disabled Indicator} &
1 bit &
Tells whether HARQ feedback (ACK/NACK) is expected. Spoofing “disabled” could trick the transmitter into not waiting for feedback, losing reliability. Spoofing “enabled” could cause the transmitter to expect absent feedback and force timeouts. \\

\midrule

\textbf{Cast Type Indicator} &
2 bits &
Specifies whether the sidelink transmission is unicast, groupcast, or broadcast (per Table 8.4.1.1-1 in~\cite{3gpp.38.212}). Faking cast type may lead to unexpected reception behaviors or disrupt group membership filters. \\

\midrule

\textbf{CSI Request} &
1 bit &
Requests channel state information from the receiver. An attacker toggling this bit might prompt unnecessary overhead or hamper link adaptation if the legitimate transmitter/receiver rely on accurate CSI feedback. \\

\bottomrule
\end{tabularx}
\end{table*}

