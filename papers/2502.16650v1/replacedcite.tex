\section{Related Work}
\label{sec:related-work}

The design and implementation of LTE-V2X have been widely studied, providing an overview of long-term evolution-vehicle (LTE-V) communication and its benefits for vehicular applications____. Further research has focused on the simulation environments for both LTE____ and 5G networks____. The architecture and capabilities of NR have been extensively explored____, with several studies analyzing NR performance and design implications for V2X communications____. Additionally, NR has been proposed for public safety applications____, for military communications____ and in support of drone operations____, demonstrating its versatility across multiple domains, while also a few features have been implemented based on a custom Open Air Interface________.

Multiple works have covered high-level overview of the security challenges____. However, some of these studies primarily focus on "conventional" LTE/5G networks and do not dive into NR sidelink specific internal functions, while others focus on generic device-to-device communications. A survey by Yoshizawa et al.____ provide a valuable overview into V2X, though it does not investigate technically the NR ecosystem, while Ying et al.____ offer an updated literature overview. ____ give more insights regarding the general security posture of the V2X networks.

Various related cryptographic mechanisms and their performance in LTE and 5G communications have also been analyzed____, even though trust and protection is not examined holistically and at a macroscopic level, nor compatibility with 3GPP standards. Device-to-device secrecy improvements with radio resource and power management have been studied____, and DoS attacks have been mathematically simulated against C-V2X resources____. Finally, Twardokus et al.____ have notably explored resource exhaustion and jamming techniques for C-V2X targeting the resource scheduling leading to DoS, while also proposing countermeasures. On the contrary, our focus is mainly on SCI spoofing attacks (not jamming) for resources, while also proving a detailed protocol, message and parameter analysis (i.e., a unique and holistic approach) for this attack, specifically for 5G V2X Sidelink (not LTE mode 4-oriented).

Generally, our work offers the first in-depth examination of technical NR V2X procedures and protocols by focusing on their unique security implications of cellular V2X. As shown in Table~\ref{tab:related-comparison}, existing studies either address broader V2X concepts or rely on simulation/mathematical setups without fully exploring the specific functionalities.