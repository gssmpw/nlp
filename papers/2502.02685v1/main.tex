% \documentclass[journal,twoside,web]{ieeecolor}

\documentclass[lettersize,journal]{IEEEtran}
\usepackage{amsmath,amsfonts}
\usepackage{algorithmic}
\usepackage{algorithm}
\usepackage{array}
\usepackage[caption=false,font=normalsize,labelfont=sf,textfont=sf]{subfig}

% \usepackage{generic}
% \usepackage{amsmath,amssymb,amsfonts}
% \usepackage{algorithmic}
\usepackage{cite}
% % \usepackage{cite}
% \usepackage{url}
% \usepackage{graphicx}
% \usepackage{tikz}
% \usepackage{standalone}
% % \usepackage[settings]{markdown}
% \usepackage{textcomp}
% \def\BibTeX{{\rm B\kern-.05em{\sc i\kern-.025em b}\kern-.08em
%     T\kern-.1667em\lower.7ex\hbox{E}\kern-.125emX}}
% \markboth{\journalname, VOL. XX, NO. XX, XXXX 2024}
% {A Methodology for PDK Re-Centering Using TCAD and Experimental Data for Cryogenic Temperatures}
% Package for subfloat
\usepackage[caption=false]{subfig}
\usepackage[font=small]{caption}
\usepackage{sansmath}
\DeclareCaptionFont{sansmath}{\sansmath}
\captionsetup{textfont={sf,sansmath}}

\usepackage{booktabs}
\usepackage{xspace}
\usepackage{markdown}
% Shortcuts
\newcommand{\rsd}{$R_{\rm SD}$\xspace}
\newcommand{\tsi}{$t_{\rm Si}$\xspace}
\newcommand{\Next}{$N_{\rm ext}$\xspace}
\newcommand{\Lext}{$L_{\rm ext}$\xspace}
\newcommand{\vmin}{$V_{\rm min}$\xspace}
\newcommand{\vdd}{$V_{\rm DD}$\xspace}
% \newcommand{\vdd}{\ensuremath{V_{\rm DD}}\xspace}
\newcommand{\vdds}{$V_{\rm DD}$}
\newcommand{\vds}{$V_{\rm DS}$\xspace}
\newcommand{\vdss}{$V_{\rm DS}$\xspace}
\newcommand{\vgs}{$V_{\rm GS}$\xspace}
\newcommand{\vbg}{$V_{\rm BG}$\xspace}
\newcommand{\vbgs}{$V_{\rm BG}$}
\newcommand{\ion}{$I_{\rm ON}$\xspace}
\newcommand{\ioff}{$I_{\rm OFF}$\xspace}
\newcommand{\ieff}{$I_{\rm EFF}$\xspace}
\newcommand{\Lg}{$L_g$\xspace}
\newcommand{\tox}{$t_{ox}$\xspace}
\renewcommand{\ss}{${\rm SS}$\xspace}
\newcommand{\vt}{$V_t$\xspace}
\newcommand{\vth}{$V_{th}$\xspace}

\newcommand{\idvg}{$I_{\rm DS}-V_{\rm GS}$\xspace}
\newcommand{\TCryo}{$T_{\rm Cryo}$\xspace}

\newcommand{\fig}[1]{Fig.~\ref{#1}}
% \newcommand{\td}[1]{{\bf\color{blue} TD: #1}}
\newcommand{\blue}[1]{{\color{blue} #1}}
\newcommand{\red}[1]{{\color{black} #1}}
\newcommand{\tapas}[1]{{\color{black} #1}}

\usepackage[colorlinks]{hyperref}
\hypersetup{%
  colorlinks=true,
  linkcolor={blue},
  citecolor={red},
  urlcolor={blue},
  % bookmarksnumbered=true,%
  % bookmarksopen=true
  }
% \usepackage[colorlinks,pagebackref=true]{hyperref}
% \hypersetup{%
%   colorlinks=true,
%   linkcolor={red},
%   citecolor={red},
%   urlcolor={blue!80}
%   bookmarksnumbered=true,%
%   bookmarksopen=true
%   }
 
\begin{document}

\bstctlcite{IEEEexample:BSTcontrol} % for short referencing

\title{A Methodology for Process Design Kit Re-Centering Using TCAD and Experimental Data for Cryogenic Temperatures}

\author{Tapas Dutta, \IEEEmembership{Member, IEEE}, Fikru Adamu-Lema, Djamel Bensouiah, Asen Asenov, \IEEEmembership{Fellow, IEEE}
\thanks{T. Dutta, F. Adamu-Lema, and A. Asenov are with Semiwise Ltd., Glasgow, and also with the Device Modelling Group, University of Glasgow, UK. D. Bensouiah is with Semiwise Ltd., Glasgow.
(e-mail: tapas.dutta@glasgow.ac.uk)}}


% The paper headers
\markboth{}%
{Dutta \MakeLowercase{\textit{et al.}}: A Methodology for Process Design Kit Re-Centering Using TCAD and Experimental Data for Cryogenic Temperatures}

% \IEEEpubid{0000--0000/00\$00.00~\copyright~2021 IEEE}

\maketitle
\begin{abstract}
In this work, we describe and demonstrate a novel Technology Computer Aided Design (TCAD) driven methodology that allows measurement data from `non-ideal' silicon wafers to be used for re-centering a room temperature-based Process Design Kit (PDK) to cryogenic temperatures. This comprehensive approach holds promise for advancing cryogenic CMOS design in the absence of foundry supplied cryogenic PDKs.
\end{abstract}

\begin{IEEEkeywords}
Cryogenic, CMOS, TCAD, Compact model, PDK Re-centering
\end{IEEEkeywords}
% \vspace{-6mm}
%%%%%%
\section{Introduction}
Cryogenic CMOS design has attracted huge interest for reducing the power dissipation in data centers by operating at cryogenic temperatures and for interfacing CMOS analog and digital circuits to the quantum bits (qubits) in the same cryogenic chamber\cite{CharbonIEDM16,XueNature21} in quantum computing research. An important hurdle is that the foundry PDKs are designed for room temperature operation and \red{at the time of executing this work, and to the best of our knowledge, no PDKs are available for design at cryo temperature.} Measurements at cryogenic temperatures can be used to re-center\cite{WangSISPAD16} the room temperature PDK to cryogenic temperatures. However, this is a complicated process due to discrepancies between the characteristics of the Typical/Typical (TT) transistors from the foundry PDK and the transistors measured on the silicon wafers. The foundry provides no guarantee  to customers that the fabricated  devices will have the same characteristics as the TT transistors in the PDK. Instead, the foundry only guarantees that the transistor characteristics on the fabricated wafers will be in-between the characteristics of the fast/fast (FF) and slow/slow (SS) corner transistor characteristics in the PDK. 

\red{While several methods have been proposed to address cryogenic CMOS modeling, they usually are limited to direct extraction of compact models using data from a few cryogenic temperature measurements of one or in some cases a few devices. Some such recent efforts are: cryogenic modeling of 28nm bulk CMOS tuning two parameters of the PSP model\cite{JungICEIC24}, BSIM-CMG model extraction for FinFETs\cite{SinghLAEDC24} etc. However, the simplistic approach of fitting limited measurements and extracting a compact model cannot produce a cryogenic PDK as the measured die can be anywhere between FF, SS corners, or even outside. While for room temperature (RT) models, measurements across many die/wafers/lots are performed and PDK is extracted, but for cryogenic CMOS this is not as easy and is also much more expensive (an example of cryogenic PDK from scratch has been reported in\cite{AkturkBCICTS24}, but that is for an old 130nm CMOS technology). Further, this simplistic approach has been usually targeted at a specific device architecture or technology node, which is useful, but has limited applicability for real circuit design, in contrast to the generic approach that we have developed.}

This paper outlines a procedure to accurately re-center room temperature foundry PDKs for supporting circuit design at cryogenic temperatures, using a combination of \red{a minimum set of} experimental cryogenic measurements of CMOS transistors on test chips and TCAD simulations. The methodology delivers PDK quality models for TT, SS, FF, SF and FS transistors and include statistical variability (mismatch). A flowchart summarizing the entire methodology is shown in Fig. 1 (corners skipped for simplicity). The next sections describe the various steps involved in re-centering procedure.

\begin{figure}[!htbp]
\centering
\includegraphics[width=0.49\textwidth]{figures/flow1.png} 
\caption{Flowchart of the PDK re-centering procedure.\vspace{-3mm}}
\label{fig:recenteringFlow}
\end{figure}

\begin{figure}[!htbp]
\centering
\captionsetup[subfloat]
{captionskip=-2em,margin=1.2em,justification=raggedright,singlelinecheck=false,font=small}
\subfloat[]{\includegraphics[width=0.3\textwidth,trim={0cm 0cm 0.0cm 0cm},clip]{figures/TCAD1_300K_LD.png}}

\subfloat[]{\includegraphics[width=0.3\textwidth,trim={0cm 0cm 0.0cm 0cm},clip]{figures/TCAD1_300K_HD.png}}
\caption{Calibrated TCAD characteristics of the TT transistor against PDK data: (a) \vds=0.05V (b) \vds=\vdd. For comparison the characteristics of the slow-slow (SS) and the fast-fast (FF) corner transistors obtained from the PDK are also plotted. T=300K (a.u. stands for arbitrary units).}
\label{fig:calib}
\end{figure}

 
\section{TCAD Deck Calibration for TT Transistor}
The initial step in the PDK Re-Centering process is the creation of a TCAD deck, starting with the generation of a TCAD device structure of the typical-typical (TT) transistor, considering the room temperature PDK data \red{(SPICE model generated)}. \red{The structure is created using process simulations\cite{SentaurusProcess}), and device simulations are performed under the drift-diffusion formalism\cite{SentaurusManual22}}. This step includes device simulations the adjustments of the doping profiles and the transistor structure to match the electrostatic behavior of the TT transistors, which includes the threshold voltage (\vt), the Subthreshold Slope (SS), the Drain Induced Barrier Lowering (DIBL) at different bias conditions and their dependence on the transistor dimensions. This is followed by carrier mobility calibration aiming to represent accurately the transistor performance at low and high drain biases. The mobility models \red{(Philips unified mobility model\cite{KlassenSSE92},
% //considers impurity scattering and carrier–carrier scattering mechanisms. In addition to describing the temperature dependence of the mobility, the model takes into account electron–hole scattering, screening of ionized impurities by charge carriers, and clustering of impurities//
Lombardi model\cite{Lombardi}, and
% EnormalDependence.
Hansch \cite{HanschSSE89} model)
% HighFieldDependence / modified version of Canali (Caughey-Thomas) model)
}
are selected to represent as accurately as possible the temperature dependence. An example of such a TCAD simulated characteristics obtained for a TT $n$-channel transistor calibrated to TT transistor data from the PDK is illustrated in Fig. 2. For comparison, the characteristics of the slow-slow (SS) and the fast-fast (FF) transistors as obtained from the room temperature PDK are also plotted in the figure. For this particular demo, we have used 22nm FDSOI devices\cite{CarterIEDM16}, but the methodology presented is generic and not dependent on a particular device architecture, technology node, or compact model.

\section{TT Transistor TCAD Model Adjustment}
Next, we compare the room temperature characteristics of the TT transistor from TCAD (Fig. 2) and the corresponding measured data from a fabricated silicon wafer as illustrated in Fig. 3 (a). As expected, the measured transistor data is shifted with respect to the corresponding TT transistor. The TT transistor represents the average transistor characteristics across the wafer, across the lots and from lot to lot. The actual transistor characteristics on each wafer are different from the average transistor characteristics due to uncontrollable variations in the fabrication conditions. The main process parameters that are causing the process variation are the dose and energy of different implantation steps, the gate oxide thickness, the annealing temperatures, and the transistor dimensions. Typically, up to 5\%-10\% variations in the above process parameters are expected during the fabrication process. The different technology parameters have different impact on threshold voltage, electrostatic integrity and drive current and should be carefully tested and combined to give the desirable shift in transistor behavior from the TT to the measured transistor. \tapas{The result from the simulation of the process calibrated structure (in order to match the room temperature measurement) is illustrated in Fig. 3 (b).} The new TCAD deck represents a Shifted TT (STT) transistor, which will be used at the next stage to perform calibration of the mobility, density gradient (for quantum correction), and other cryogenic specific models at cryogenic temperatures.

\begin{figure}[!htbp]
\centering
\captionsetup[subfloat]{captionskip=-2.6em,margin=1.6em,justification=raggedright,singlelinecheck=false,font=normal}
\subfloat[]{\includegraphics[width=0.25\textwidth,trim={0cm 0cm 0.0cm 0cm},clip]{figures/TCAD1_300K_w_Exp_HD_Meas_woShift.png}}
\subfloat[]{\includegraphics[width=0.25\textwidth,trim={0cm 0cm 0.0cm 0cm},clip]{figures/TCAD2_300K_w_Exp_HD_wShifted.png}}

\subfloat[]{\includegraphics[width=0.3\textwidth,trim={0cm 0cm 0.0cm 0cm},clip]{figures/TCAD3_linlog.png}}

\caption{(a) Comparison of \idvg characteristics of the TT transistor in Fig. 2 and the corresponding measurement data at T=300K. (b) Comparison between the measurement data on a fabricated silicon wafer and the calibrated shifted TCAD model. T=300K. (c) Comparison between the measurement data and the calibrated shifted TCAD model both at cryogenic temperature (\TCryo) at low and high drain biases.}
\label{fig:shifted}
\end{figure}

\section{Adjusted Transistor at Cryogenic Temperature}
At this stage, the STT TCAD transistor simulations are calibrated to the cryogenic transistor measurements. The process includes two steps – electrostatic calibration and mobility calibration. The electrostatic calibration aims to reproduce \vt, SS, and DIBL of the STT at the cryogenic temperature (\TCryo). \red{Considering the impact of the band tail states\cite{KanePR63} was critical for calibrating the subthreshold characteristics.} The mobility calibration follows the calibration procedure at room temperature aiming to reproduce the current voltage characteristics above and below threshold. The results from the calibration of the STT transistor at \TCryo are illustrated in Fig. 3 (c).


\section{Target Data Generation at Cryogenic Temperature and Model Extraction}
After the calibration of the STT transistor, the TCAD structure of the ‘original’ TT transistor is used along with the calibrated band tail, incomplete ionization and mobility models at cryogenic temperature from the previous step to generate the target characteristics for the compact model\cite{BsimOnline} extraction of the TT transistor at cryogenic temperature. The model cards for the compact model are extracted to yield the cryogenic PDK using standard compact model extraction procedures. The generated target characteristics at cryogenic temperature and the extracted compact model (HSPICE \cite{HSPICEManual22} results) of the TT transistor are illustrated in Fig. 4.

\begin{figure}[!htbp]
\centering
\includegraphics[width=0.35\textwidth]{figures/TCAD4_linlog.png} 
\caption{Comparison between the target \idvg data obtained from TCAD simulation of the original TT transistor at cryogenic temperature and the extracted compact model at low and high drain biases.}
\label{fig:TargetIdVg}
\end{figure}

The same approach is applied to the FF and SS corners. \red{We have also developed a Monte-Carlo (MC) approach to consider variability in the cryogenic PDK, which will be described in a future publication.}
%. Threshold voltage distributions from 1000 MC runs for both nMOS and pMOS devices in the SPICE simulator are shown in Fig. 5 (b) illustrating the reduction of mean $V_{t,lin}$ and $V_{t,sat}$ with temperature.

% \begin{figure}[!htbp]
% \centering
% \subfloat[]{\includegraphics[width=0.35\textwidth,trim={-1cm 0cm 0.0cm 0cm},clip]{figures/nch_idvg77k_edl.png}}

% \subfloat[]{\includegraphics[width=0.45\textwidth,trim={0cm 0cm 0.0cm 0cm},clip]{figures/vth_hist_thin.png}}
% \caption{(a) Re-centering procedure applied to process corners. \textit{Solid curves:} Output of SPICE using re-centered PDK. \textit{Grey dots:} the Monte-Carlo output (100 runs) at discrete bias points. (b) Threshold voltage distributions from 1000 MC runs at RT and cryogenic temperatures for nMOS and pMOS devices at low and high \vds.}
% \label{fig:shifted}
% \end{figure}


\section{Conclusions}
In this paper we have described a new approach for PDK re-centering based on TCAD and experimental data at cryogenic temperatures suitable for circuit design applications in data centers and quantum computers. A similar procedure can be used to derive target data for cryogenic compact model extraction for SS and FF corners. Further work is in progress to extend this methodology for deriving statistical \red{SPICE} models, which are essential for verifying circuit performance at cryogenic temperatures.
%%%%%%

\section*{Acknowledgement}
We are grateful to GlobalFoundries for providing the 22FDX PDK and allowing us to customize it for cryogenic temperature operation. The device measurements were performed by Incize SRL, Belgium. This work was supported partially by Innovate UK funded project ``Development of Cryo-CMOS to enable the next generation of scalable quantum computers” under the grant number of 10006017 and was also partially supported by Semiwise Ltd, UK.

% References
\bibliographystyle{IEEEtranDOI}
\bibliography{biblio.bib}
% {
% \parindent0pt
% \small
% [1] E. Charbon, F. Sebastiano, A. Vladimirescu, H. Homulle, S. Visser, L. Song, et al., "Cryo-CMOS for quantum computing", 2016 IEEE International Electron Devices Meeting (IEDM), pp. 13.5.1-13.5.4, Dec 2016. DOI: 10.1109/IEDM.2016.7838410

% [2] X. Xue, B. Patra, J.P.G. van Dijk et al., “CMOS-based cryogenic control of silicon quantum circuits.” Nature. vol. 593, pp. 205–210, 2021. DOI: 10.1038/s41586-021-03469-4 

% [3] L. Wang, B. Cheng, P. Asenov, A. Pender, D. Reid, F. Adamu-Lema, C. Millar, A. Asenov, "TCAD proven compact modelling re-centering technology for early 0.x PDKs," 2016 International Conference on Simulation of Semiconductor Processes and Devices (SISPAD), pp. 157-160, September 2016. DOI: 10.1109/SISPAD.2016.7605171

% [4] Sentaurus Device Manual, Synopsys Inc., Mountain view, CA, USA, 2022.

% [5] D. B. M. Klaassen, “A Unified Mobility Model for Device Simulation—I. Model Equations and Concentration Dependence,” Solid-State Electronics, vol. 35, no. 7, pp. 953–959, 1992. DOI: 10.1016/0038-1101(92)90325-7

% [6] C. Lombardi et al., “A Physically Based Mobility Model for Numerical Simulation of Nonplanar Devices,” IEEE Transactions on Computer-Aided Design, vol. 7, no. 11, pp. 1164–1171, 1988. DOI: 10.1109/43.9186

% [7] W. Hänsch and M. Miura-Mattausch, “The hot-electron problem in small semiconductor devices,” Journal of Applied Physics, vol. 60, no. 2, pp. 650–656, 1986. DOI: 10.1063/1.337408

% [8] P. P. Altermatt, A. Schenk and G. Heiser, "A Simulation model for the density of states and for incomplete ionization in crystalline Silicon. I. establishing the model in Si:P," Journal of Applied Physics, vol. 100, pp. 113714, 2006. DOI: 10.1063/1.2386934

% [9] E. O. Kane, “Thomas-Fermi Approach to Impure Semiconductor Band Structure,” Physical Review, vol. 131, no. 1, pp. 79–88, 1963. DOI: 10.1103/PhysRev.131.79

% [10] [Online]. Available: https://bsim.berkeley.edu/models/

% [11] HSPICE User Guide, Synopsys, Inc., Mountain view, CA, USA, 2022.
% }


\end{document}