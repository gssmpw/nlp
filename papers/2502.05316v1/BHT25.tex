
\documentclass[a4paper,UKenglish,cleveref, autoref, thm-restate]{lipics-v2021}
\nolinenumbers
%This is a template for producing LIPIcs articles. 
%See lipics-v2021-authors-guidelines.pdf for further information.
%for A4 paper format use option "a4paper", for US-letter use option "letterpaper"
%for british hyphenation rules use option "UKenglish", for american hyphenation rules use option "USenglish"
%for section-numbered lemmas etc., use "numberwithinsect"
%for enabling cleveref support, use "cleveref"
%for enabling autoref support, use "autoref"
%for anonymousing the authors (e.g. for double-blind review), add "anonymous"
%for enabling thm-restate support, use "thm-restate"
%for enabling a two-column layout for the author/affilation part (only applicable for > 6 authors), use "authorcolumns"
%for producing a PDF according the PDF/A standard, add "pdfa"

%\pdfoutput=1 %uncomment to ensure pdflatex processing (mandatatory e.g. to submit to arXiv)
%\hideLIPIcs  %uncomment to remove references to LIPIcs series (logo, DOI, ...), e.g. when preparing a pre-final version to be uploaded to arXiv or another public repository

%\graphicspath{{./graphics/}}%helpful if your graphic files are in another directory

\bibliographystyle{plainurl}% the mandatory bibstyle

\title{Finding equilibria: simpler for pessimists, simplest for optimists} %TODO Please add
\newcommand{\thought}[1]{{\color[rgb]{0.2,0.39,0.66}(#1)}}
\newcommand{\todo}[1]{{\color[rgb]{1.0,0.0,0.0}(#1)}}
\newcommand{\hsh}[1]{{\color{green!50!black} Henrik: #1}}
\newcommand{\st}[1]{{\color{red!50!black} Sebastian: #1}}

\newcommand{\ulm}[1]{_{\scaleto{\mathrm{#1}}{3pt}}}
\newcommand\at[2]{\left.#1\right|_{#2}}











\newtheorem{assumption}{Assumption}

\DeclareMathOperator*{\argmax}{arg\,max}
\DeclareMathOperator*{\argmin}{arg\,min}

\newcommand{\swname}[1]{\texttt{#1}}
\newcommand{\ie}{i\/.\/e\/.,\/~}
\newcommand{\eg}{e\/.\/g\/.,\/~}
\newcommand{\cf}{cf\/.\/~}

\newcommand{\fig}{Fig\/.\/~}
\newcommand{\defn}{Def\/.\/~}
\newcommand{\sect}{Sec\/.\/~}
\newcommand{\tabl}{Tab\/.\/~}
\newcommand{\algo}{Algorithm~}
\newcommand{\theo}{Theorem~}

\newcommand{\bnnl}{3 hidden layers}
\newcommand{\bnnn}{50 neurons}
\newcommand{\bnna}{tanh activations}

\newcommand{\capt}[1]{\mdseries{\emph{#1}}}

\newcommand{\videolink}{at \url{https://youtu.be/_d7AqTRjz6g}}
\newcommand{\codelink}{\url{https://github.com/wheelbot/mini-wheelbot}}

\newcommand{\fakepar}[1]{\vspace{0mm}\noindent\textbf{#1.}}

\newcommand{\needref}{\textcolor{red}{[REF]}}

\newcommand{\plotfontsize}{9pt}

%\titlerunning{Dummy short title} %TODO optional, please use if title is longer than one line

% \author{Jane {Open Access}}{Dummy University Computing Laboratory, [optional: Address], Country \and My second affiliation, Country \and \url{http://www.myhomepage.edu} }{johnqpublic@dummyuni.org}{https://orcid.org/0000-0002-1825-0097}{(Optional) author-specific funding acknowledgements}%TODO mandatory, please use full name; only 1 author per \author macro; first two parameters are mandatory, other parameters can be empty. Please provide at least the name of the affiliation and the country. The full address is optional. Use additional curly braces to indicate the correct name splitting when the last name consists of multiple name parts.

\author{Léonard Brice}{Université Libre de Bruxelles, Belgium}{leonard.brice@ulb.be}{https://orcid.org/0000-0001-7748-7716}{}

\author{Thomas A. Henzinger}{Institute of Science \& Technology Austria}{tah@ist.ac.at}{https://orcid.org/0000-0002-2985-7724}{}

\author{K. S. Thejaswini}{Institute of Science \& Technology Austria}{thejaswini.k.s@ista.ac.at}{https://orcid.org/0000-0001-6077-7514}{}

\authorrunning{L. Brice, T. A. Henzinger, K. S. Thejaswini} %TODO mandatory. First: Use abbreviated first/middle names. Second (only in severe cases): Use first author plus 'et al.'

\Copyright{Léonard Brice, Thomas A. Henzinger, K. S. Thejaswini} %TODO mandatory, please use full first names. LIPIcs license is "CC-BY";  http://creativecommons.org/licenses/by/3.0/

\ccsdesc[100]{Software and its engineering: Formal methods; Theory of computation: Logic and verification; Theory of computation: Solution concepts in game theory.} %TODO mandatory: Please choose ACM 2012 classifications from https://dl.acm.org/ccs/ccs_flat.cfm 

\keywords{Equilibria, Nash equilibria,  stochastic games, graph games, risk measure, entropic risk measure, risk-sensitive equilibria} %TODO mandatory; please add comma-separated list of keywords

\category{} %optional, e.g. invited paper

\relatedversion{} %optional, e.g. full version hosted on arXiv, HAL, or other respository/website
%\relatedversiondetails[linktext={opt. text shown instead of the URL}, cite=DBLP:books/mk/GrayR93]{Classification (e.g. Full Version, Extended Version, Previous Version}{URL to related version} %linktext and cite are optional

%\supplement{}%optional, e.g. related research data, source code, ... hosted on a repository like zenodo, figshare, GitHub, ...
%\supplementdetails[linktext={opt. text shown instead of the URL}, cite=DBLP:books/mk/GrayR93, subcategory={Description, Subcategory}, swhid={Software Heritage Identifier}]{General Classification (e.g. Software, Dataset, Model, ...)}{URL to related version} %linktext, cite, and subcategory are optional

\funding{This work is a part of project VAMOS that has received funding from the European Research Council (ERC), grant agreement No 101020093}%optional, to capture a funding statement, which applies to all authors. Please enter author specific funding statements as fifth argument of the \author macro.

%\acknowledgements{I want to thank \dots}%optional

%\nolinenumbers %uncomment to disable line numbering



%Editor-only macros:: begin (do not touch as author)%%%%%%%%%%%%%%%%%%%%%%%%%%%%%%%%%%
% \EventEditors{John Q. Open and Joan R. Access}
% \EventNoEds{2}
% \EventLongTitle{42nd Conference on Very Important Topics (CVIT 2016)}
% \EventShortTitle{CVIT 2016}
% \EventAcronym{CVIT}
% \EventYear{2016}
% \EventDate{December 24--27, 2016}
% \EventLocation{Little Whinging, United Kingdom}
% \EventLogo{}
% \SeriesVolume{42}
% \ArticleNo{23}
%%%%%%%%%%%%%%%%%%%%%%%%%%%%%%%%%%%%%%%%%%%%%%%%%%%%%%

\begin{document}

\maketitle

%TODO mandatory: add short abstract of the document
\begin{abstract}
We consider simple stochastic games with terminal-node rewards and multiple players, who have differing perceptions of risk. Specifically, we study risk-sensitive equilibria (RSEs), where no player can improve their perceived reward---based on their risk parameter---by deviating from their strategy. We start with the entropic risk (ER) measure, which is widely studied in finance. ER characterises the players on a quantitative spectrum, with positive risk parameters representing optimists and negative parameters representing pessimists. Building on known results for Nash equilibira, we show that  RSEs exist under ER for all games with non-negative terminal rewards. However, using similar techniques, we also show that the corresponding \emph{constrained} existence problem---to determine whether an RSE exists under ER with the payoffs in given intervals---is undecidable.

To address this, we introduce a new, qualitative risk measure---called \emph{extreme risk} (XR)---which coincides with the limit cases of positively infinite and negatively infinite ER parameters. Under XR, every player is an extremist: an extreme optimist perceives their reward as the maximum payoff that can be achieved with positive probability, while an extreme pessimist expects the minimum payoff achievable with positive probability. Our first main result proves the existence of RSEs also under XR for non-negative terminal rewards. Our second main result shows that under XR the constrained existence problem is not only decidable, but $\NP$-complete. Moreover, when all players are extreme optimists, the problem becomes $\PTIME$-complete. Our algorithmic results apply to all rewards, positive or negative, establishing the first decidable fragment for equilibria in simple stochastic games with terminal objectives without restrictions on strategy types or number of players.
\end{abstract}




\section{Introduction}
Stochastic systems have been used extensively in several areas including  verification~\cite{FKNP11}, learning theory~\cite{AJKS21}, epidemic processes~\cite{Lef81} to name a few. Several real-world systems however do not work with a centralised control. Therefore, modelling using stochastic systems with multiple agents makes for more faithful abstractions of such systems without a centralised control. Some examples of fields in which multi-agents stochastic modelling include cyber physical systems~\cite{SEC16}, distributed and probabilistic computer programs~\cite{dAHJ01}, probabilistic planning~\cite{TKI10}. In such cases, the problem of reasoning about multiple agents with several, often times orthogonal objectives, becomes important. % However, for situations that are modelled as graph games, Nash equilibria come with its own down-sides and therefore several notions of equilibria have emerged in turn-based games on graphs to circumvent the problems posed by the natural definitions of Nash equilibira, like subgame-perfect equilibira, Stackleberg-equilibria. 
For multi-agent systems modelled with stochasticity on the underlying arena, a fundamental question to ask is the existence or finding of an equilibrium.
The most popular equilibria in literature are Nash equilibria~\cite{Nas50}. However, those come with their own downsides. The computational complexity for studying Nash equilibria over multi-agent systems is prohibitively expensive, and even undecidable in the general case, where systems have $10$ or more players~\cite{UW11}. 
Further, even if Nash equilibria could be computed efficiently, they do not faithfully model the agents in real world settings
%as each agent might perceive risk differently. With randomness arising from both the strategies of other agents, as well as the underlying model of the system, this might mean that risk-averse or risk-loving agents might have an incentive to deviate since their perceived values of outcome is different from expected value of the game.
since they do not consider their tolerance or averseness to risk.

Let us consider a $1$-player game where a protagonist is proposed two options: (a) earning \$1; (b) playing a lottery in which, with probability $\frac{1}{40}$, she gets \$40, and with probability $\frac{39}{40}$, she does not earn anything.
Classically, rational strategies would be maximising the expected payoff. From this perspective, both options yield an expected payoff of \$1, making them equivalent.
This approach is particularly justified when the game represents a scenario that can be repeated many times: the law of large numbers ensures that, in the long run, the average payoff will converge to the expected payoff. However, when the game is played only once, the protagonist may prioritise immediate needs. If she urgently requires \$1, the guaranteed option (a) becomes preferable.

Conversely, if she is a risk-taker or finds herself in a situation where only the \$40 can make a significant difference, she may prefer the high-risk option (b).
Although this choice might appear irrational, it mirrors the behaviour of millions of people who participate everyday in games with a negative expected payoff, driven by the allure of a potentially life-changing win, and generating an annual turnover of USD 536 billions~\cite{GamblingNewspaper23} for the gambling industry.
That industry, on the other hand, operates on a large scale where expected payoff becomes the key metric. 
This contrast underscores the importance of alternative measures to expected payoff that account for each agent's risk tolerance.%, offering a more nuanced understanding of decision-making in uncertain scenarios.

%We thus have an example of a two-player interaction, with apparently zero-sum payoffs, but where both players express a preference for the same option, because the context gives them a different tolerance to risk: this paradox underscores the importance of alternative measures to expected payoff that account for an agent's risk tolerance, offering a more nuanced understanding of decision-making in uncertain scenarios.
% This contrast underscores the relevance of generalising the notion of Nash equilibria: in a multi-agent system, the agents may have diverging perception of which risks can be taken.
% It makes sense, then, to study \emph{risk-sensitive equilibria}, in which players do not necessarily maximise their expected payoff, but their perception of what their payoff will be according to different risk measures.\leon{I'm actually not satisfied with this, I will modify it and move it.}


% Classically, we consider that a rational strategy would consist in maximising the expected payoff: from that perspective, the two choices are equivalent.
% Such an approach is justified especially when the game models a situation that can be repeated a large number of times, in which case the law of large numbers guarantees that the average payoff converges to the expected payoff.
% But when the game models a situation that is played only once, the protagonist may consider that she really needs her euro, and that the possibility of earning 40\$ is too unlikely to be taken into account: she would then have a justifiable preference for the option (a).
% On the contrary, if she is more of a gambler, or if she finds herself in a desperate situation in which only earning those 40\$ could save her, she could go all out and express a strict preference for option (b)\footnote{Even though this case seems more irrational, it explain why millions of people play everyday games in which they know that their expected payoff is negative.
% On the other side, the companies with whom they interact repeat the experience often enough to consider expected payoff as the relevant measure --- generating a yearly turnover of 536 billions of US\$.}.
% Hence the relevance of alternatives to expected payoff, that take into account the tolerance of the agent to risk.

\subparagraph*{Risk Measures.}
A \emph{risk measure} captures the perception that a player has of what their payoff will be. In that sense, they generalise the notion of expected payoff.
Various risk measures exist in the literature, and have been used extensively in the field of economics and finance. 
Some of these risk measures include expected shortfall (ES), value at risk (VaR)~\cite{Aue18}, variance~\cite{Bra99}, entropic risk measure (ER)~\cite{FS02}. 

%However, since the introduction of the characteristic of risk measure called \emph{coherence}~\cite{ADJH99}, it was expected that a ``good'' a risk measure must be coherent. A risk measure is coherent if it is monotonic, homogeneous, translational-invariance, and sub-additive.
%This automatically weeds out several of the above risk measures listed above like  Value at Risk or variance as a risk measure. 
A lot of work has been done in considering these risk measures over MDPs which use variance (along with mean) as a risk-measure~\cite{FK89, PSB22,MT11}, ES~\cite{RRS15,KM18,Meg22} (also referred to as conditional value at risk (CVaR), average value at risk (AVaR), expected tail loss (ETL), and superquantile in literature) and ER~\cite{HM72,BR14,BCMP24}. % have also been studied. 
Studying the entropic risk measure in MDPs appears more practical compared to expected shortfall  or using variance-penalised risk-measures. This impracticability of ES and variance-penalised measure in particular is due to the intractable exponential memory~\cite{HK15} and time required to compute optimal strategies~\cite{PSB22}, even for the one agent system of Markov decision processes (MDPs). On the other hand, when the risk measure used is ER, players have optimal positional strategies in MDPs~\cite{How72}, which makes it a prime candidate for consideration in multi-agent settings.

\subparagraph*{Entropic Risk Measure.}
The entropic risk measure is computed by assigning to each agent a risk parameter, i.e., a value $\rho \in \Rb$.
%Based on this risk parameter $\rho$, this measure first computes the expectation of the exponential function of the random variable and then re-normalises this.  
The entropic risk measure of a random variable $X$ is then defined as
$\re_\rho[X] = -\frac{1}{\rho} \log_e \left( \Eb \left[ e^{-\rho X}\right] \right)$. 
%For computational reasons, instead of the Euler's constant $e$, we use different bases sometimes.
If the risk parameter $\rho$ is positive, then more weight will be given to the bad payoffs: the corresponding player can then be considered as risk-averse.
Conversely, players with a negative $\rho$ are more risk-loving.
When $\rho$ tends to $0$, the entropic risk measure converges to the classical expectation $\Eb[X]$.

The game depicted by Figure~\ref{fig:example_gamma} extends the lottery example we discussed earlier. 
Black vertices are stochastic, and the circle vertex is controlled by player $\Circle$.
A play can be seen as an infinite sequence of moves of a token along the edges of the graph, starting from $a$: from a stochastic vertex, it takes one of the outgoing edges with the probabilities indicated on those, and from a vertex controlled by the player, she chooses which edge it takes.
The payoff $40$, $0$, or $1$ is obtained when the terminal vertex $t_1$, $t_2$, or $t_3$ is reached, respectively.
If no terminal vertex is reached, then the payoff is $0$.
Taking the red edge corresponds to option (a): then, her risk entropy is always $1$, for every risk parameter $\rho$.
But if she chooses option (b), that is, if she takes the blue edge, her risk entropy is $\re_\rho[\mu_{\circ}] = -\frac{1}{\rho} \log \left( \Eb \left[ e^{-\rho \mu_{\circ}}\right] \right) = -\frac{1}{\rho} \log \left(  e^{-40\rho } + \frac{39}{40} \right)$.
Both cases are illustrated with red and blue curves in \cref{fig:example_plot}.
The curves cross at abscissa $\rho = 0$, where the entropic risk measure corresponds to the expectation. Note that other strategies are possible if \emph{randomisation} is allowed---the player could, for example, toss a coin and participate in the lottery if the outcome is heads. The perceived reward of randomising between outermost red and blue edges are illustrated in the intermediate cases with mixtures of red and blue in \cref{fig:example_plot}.

%For this example, we  replace Euler's constant $e$ with instead the constant $2$, which makes $\re_\rho[X] = -\frac{1}{\rho} \log_2 \left( \Eb \left[ 2^{-\rho X}\right] \right)$.

%Player $\Circle$ gets the payoff $11$ if she reaches the terminal vertex $t_1$, the payoff $1$ if she reaches $t_2$, the payoff $2$ if she reaches $t_3$, and the payoff $0$ if she reaches none of those terminal states (i.e., if she loops on the vertex $a$ forever).

%If the player's strategy is to choose the red edge, she gets payoff $1$ with probability $1$.
%Therefore, her risk measure is equal to $1$ for every risk parameter $\rho$.

%If her strategy is to choose the blue edge going to $c$, then she gets the payoff $11$ with probability $\frac{1}{10}$, and $1$ with probability $\frac{9}{10}$.
%Thus her risk entropy is
%$\re_\rho[\mu_{\circ}] = \frac{1}{\rho} \log_2\left(\frac{1}{3} 2^{-4\rho} + \frac{2}{3} 2^{-1\rho}\right)$.
 
\begin{figure}[h] 
			\centering
            \begin{subfigure}[t]{0.4\textwidth}
			\begin{tikzpicture}[->,>=latex,shorten >=1pt, initial text={}, scale=1, every node/.style={scale=1}]
				\node[initial left, stoch] (a) at (0, 0) {$a$};
				\node[state] (b) at (1.5, 0) {$b$};
                \node[stoch] (c) at (2.5, 1) {$c$};
                \node (t1) at (4, 2) {$t_1:~\stack{\circ}{40}$};
                \node (t2) at (4, 0) {$t_2:~\stack{\circ}{0}$};
                \node (t3) at (3, -1) {$t_3:~\stack{\circ}{1}$};
                \path (a) edge[loop above] node[above] {$\frac{1}{2}$} (a);
				\path (a) edge node[above] {$\frac{1}{2}$} (b);
                \path (b) edge[blue, thick] (c);
                \path (b) edge[red, thick] (t3);
				\path (c) edge node[above] {$\frac{1}{40}$} (t1);
				\path (c) edge node[below] {$\frac{39}{40}$} (t2);
			\end{tikzpicture}
			\caption{A stochastic MDP}
			\label{fig:example_gamma}
            \end{subfigure}
            \begin{subfigure}[t]{0.55\textwidth}
			\begin{tikzpicture}
              \begin{axis}[
                xlabel={Risk parameter $\rho$},
                ylabel={$\re(\text{Outcome})$},
                domain=-3:5,
                samples=200,
                    width=8cm,
                   height=6cm,
                grid=major,
                ]
                \addplot [
                  red!8!blue,
                  thick
                ]
                {-1/x * log2((9/10)*e^(-1*x) + (1/400) * e^(-40*x) + (39/400))/log2(e)};
                \addplot [
                  red!16!blue,
                  thick
                ]
                {-1/x * log2((199/200)*e^(-1*x) + (1/8000) * e^(-40*x) + (39/8000))/log2(e)};
                \addplot [
                  red!24!blue,
                  thick
                ]
                {-1/x * log2((19999/20000)*e^(-1*x) + (1/800000) * e^(-40*x) + (39/800000))/log2(e)};
                \addplot [
                  red!32!blue,
                  thick
                ]
                {-1/x * log2((1999999/2000000)*e^(-1*x) + (1/80000000) * e^(-40*x) + (39/80000000))/log2(e)};
                \addplot [
                  red!40!blue,
                  thick
                ]
                {-1/x * log2((199999999/200000000)*e^(-1*x) + (1/8000000000) * e^(-40*x) + (39/8000000000))/log2(e)};
                \addplot [
                  red!48!blue,
                  thick
                ]
                {-1/x * log2((19999999999/20000000000)*e^(-1*x) + (1/800000000000) * e^(-40*x) + (39/800000000000))/log2(e)};
                \addplot [
                  red!56!blue,
                  thick
                ]
                {-1/x * log2((1999999999999/2000000000000)*e^(-1*x) + (1/80000000000000) * e^(-40*x) + (39/80000000000000))/log2(e)};
                \addplot [
                  red!64!blue,
                  thick
                ]
                {-1/x * log2((199999999999999/200000000000000)*e^(-1*x) + (1/8000000000000000) * e^(-40*x) + (39/8000000000000000))/log2(e)};
                \addplot [
                  red!72!blue,
                  thick
                ]
                {-1/x * log2((199999999999999999/200000000000000000)*e^(-1*x) + (1/8000000000000000000) * e^(-40*x) + (39/8000000000000000000))/log2(e)};
                \addplot [
                  red!80!blue,
                  thick
                ]
                {-1/x * log2((199999999999999999999/200000000000000000000)*e^(-1*x) + (1/8000000000000000000000) * e^(-40*x) + (39/8000000000000000000000))/log2(e)};
                \addplot [
                  red!88!blue,
                  thick
                ]
                {-1/x * log2((199999999999999999999999/200000000000000000000000)*e^(-1*x) + (1/8000000000000000000000000) * e^(-40*x) + (39/8000000000000000000000000))/log2(e)};
                \addplot [
                  red!94!blue,
                  thick
                ]
                {-1/x * log2((199999999999999999999999999/200000000000000000000000000)*e^(-1*x) + (1/8000000000000000000000000000) * e^(-40*x) + (39/8000000000000000000000000000))/log2(e)};
                \addplot [
                  blue,
                  thick
                ]
                {-1/x * log2((1/40) * e^(-40*x) + (39/40))/log2(e)};
                \addplot [
                  red,
                  thick
                ]
                {+1};
              \end{axis}
            \end{tikzpicture}
			\caption{Each curve represents the perceived reward of a player choosing only blue strategy, only red, or  randomising between both strategies. The percieved payoff for a player with risk parameter $\rho \in (-3,5)$ for these strategies are represented.}
			\label{fig:example_plot}
            \end{subfigure}
        \caption{Entropic risk measure}\label{fig:example_re}
\end{figure}
%\end{example}
Unfortunately, even for two player zero-sum stochastic games with total-reward objectives (payoff is the sum of the rewards seen along the way), computing optimal strategies can only be done in $\PSPACE$, when the base $e$ is replaced by an algebraic number; and if $e$ is the base of the exponent, then it is decidable only subject to Shanuel's conjecture~\cite{BCMP24}. % and inputs where the risk is computed using ER. 
Solving the two-player zero-sum case is a specific case of finding equilibria in two-agent systems where the payoffs of the two agents are exactly the negation of each others and so are the risk parameters of each of the agents.
Therefore, reasoning about multi-agent systems with ER also has potential to be computationally intractable.%\leon{I'm not sure I understand this sentence}


% \subparagraph*{Equilibria}
% Our example involves only one player.
% However, one might model it with a second player: the company that sells the lottery ticket, and therefore that made the choice of making the game possible.
% Of course, in the real world, companies only enable such games when its expected payoff is positive, that is, when the player's expected payoff is negative; which does not prevent millions of players to participate in such lotteries everyday, generating an annual turnover of USD 536 billion~\cite{h2_gambling_2023}.
% This can be explained by the fact that players are ready to take an important risk there, because they play a small number of times, and their likely loss remains acceptable, while their possible earning would be huge: in other words, players are efficiently modelled by a negative risk parameter.
% On the other hand, the company repeats the game a very large number of times, which is why, from its perspective, the expected payoff is the relevant metric.
% %This contrast underscores the importance of alternative measures to expected payoff that account for an agent's risk tolerance, offering a more nuanced understanding of decision-making in uncertain scenarios.
% This contrast underscores the relevance of generalising the notion of Nash equilibria: in a multi-agent system, the agents may have diverging perception of which risks can be taken.
% It makes sense, then, to study \emph{risk-sensitive equilibria}, in which players do not necessarily maximise their expected payoff, but their perception of what their payoff will be according to different risk measures.\leon{I'm actually not satisfied with this, I will modify it and move it.}



\subparagraph*{Extreme Risk Measure.} We introduce a new risk measure called extreme risk measure (XR) to identify tractable risk parameters. %\leon{Do we actually use that notation?} 
% If they have to choose between two options: (a) one which always gives him an outcome of 1, and the other option (b) that gives him a positive probability $p$ of 100, but  probability $(1-p)$ of -1, he would always chose option (a), 
%
%Let us say in a stochastic system, one agent is tasked with a safety-critical objective and wishes to avoid any positive probability of getting a payoff below some threshold, say $0$.
Consider an agent who wishes to maximise the lowest payoff received with positive probability.
In our example, 
by choosing option (a) her only payoff is $\$1$, whereas by choosing option (b), the payoffs that she receives with positive probability are $\$40$ and $\$0$. 
This agent would choose the option (a) since, then, the lowest reward she gets is $\$1$, instead of $\$0$. This would be her choice regardless of the probabilities or if the lottery amount in option (b) is increased.
%Even when the probabilities are changed for option (b) or if the lottery amount is increased, she would still prefer option (a).
Such agents can be considered ``extreme pessimists'' because
their perceived payoff can be thought of as the minimum among all the possible payoffs.
%We define the perceived reward of an extreme pessimist as the infimum of the payoffs that they get with positive probability. Therefore, extreme pessimists aim to maximise the smallest payoff that they receive with positive probability, and might be willing to deviate to achieve this objective.  
Similarly, one can define ``extreme optimists''  whose perceived reward is the best payoff that can be obtained with positive probability.
In the above scenario, an extreme optimist posed with the same options would choose option (b), no matter how small the probability is of receiving that payoff.

Extreme pessimists can be used to model safety-critical agents, where any positive probability of low reward or failure is unacceptable.
On the other hand, extreme optimists model naturally the opponents of such agents.
In a multiplayer setting, they can be an accurate modelling of agents like hackers in a system, who are happy with a small probability of success, or agents that have the possibility to restart their interactions with the same system, so that as long as there is a non-zero probability of achieving a high reward, they are guaranteed to receive that high reward. %\theju{If at all we discuss motivation here is the space.}





%\begin{example}
% Consider the same example game as in \cref{fig:example_gamma}. Here, the reward that the player perceives in the MDP can perceive on using the red strategy is exactly $2$ since that is the only payoff the player can get with a positive probability.
% However, if using the blue strategy, the perceived reward depends on if the player is an optimist or a pessimist. If the players is a pessimist, then the perceived reward is $4$, and if instead the player is an extreme pessimist, then the perceived reward is $1$.
%\end{example}
% \thejaswini{Introduce with examples some systems that need to be designed where some agent needs sure reward, and agents that some agents are happy with non-zero probability of reward}

%We capture this concept of extreme optimism and pessimism by introducing a new risk measure of pessimistic expectation and optimistic expectation.
\subparagraph*{Our results.}
We consider the problem of finding equilibria in a multiplayer stochastic game, that is, a game in which the payoffs that the players receive depend on the \emph{terminal vertex} that is reached, and in which an infinite play is associated to the zero payoff vector.

Our contributions are four fold. 
Firstly, we consider the problem of finding equilibria where entropic risk measure is used to determine the perceived reward of each player.  Each player has their own risk-sensitivity parameter, and we wish to find an equilibrium where no player has the incentive to deviate and increase their risk measure. We show that, when the rewards are all non-negative, such an equilibrium always exists.
We conjecture that this remains true when rewards can be negative.
Although some equilibria exist, not all equilibria are made the same, with some equilibria being more desirable than the others. One might want to find an equilibrium that maximises the overall social welfare, or want to minimise it for certain agents. A reasonably general setting is providing an interval for the risk measure for each agent and to check if there is an equilibrium satisfying these constraints. We call this problem \emph{constrained existence problem of risk-sensitive equilibria} (RSEs). 
We show (in \cref{sec:ERM}) that this problem is undecidable when the risk parameters of the players are rational values, with undecidability results extending from the constrained existence problem for Nash equilibria in the work of Ummels and Wojtczak~\cite{UW11}. However, we find restrictions on strategies to recover decidability. % for risk-sensitive equilibria in the cases where the risk parameters are finite.
If we restrict the memory requirements of each player, then for (small) finite memory strategies, we can solve the problem by encoding it using the existential theory of reals with exponentiation, giving us decidability subject to Shanuel's conjecture, and $\PSPACE$ algorithms when the base of the exponents are encoded as small algebraic instances, reminiscent of the two-player zero-sum case by Baier et al.~\cite{BCMP24}. 
%(\cref{proposition:Undecidable}).

Secondly, since the general problem is undecidable, and even in restricted cases, we obtain complexities that are $\PSPACE$ or higher, we pivot to searching for a more tractable risk measure that can be used to find equilibria in multi-agent systems. We define extreme risk measure (XR) as a novel risk measure to consider in multi-agent stochastic systems. We show (in \cref{sec:XR}) that our new definition is robust, since it exactly captures the well-studied entropic risk measure when the risk parameters tend to $\pm \infty$.
%This result (\cref{thm:RE=PEorOE}) in turn ensures that our risk measure is a robust definition since it is the limit of a well-studied risk measure. 
We further show the existence of 
such equilibria for games with non-negative rewards. Moreover, there exists a stationary strategy profile that can be algorithmically constructed in polynomial time. We conjecture, again, that this remains true when negative rewards are involved.
One further advantage of XR as a risk measure is that it is indifferent to the exact probabilities of the underlying stochastic model, since it only deals with events that occur with a positive probability and, therefore, can also be used in systems where the underlying probabilities are unknown. 

Thirdly, we show that the constrained existence problem of RSEs is decidable and also $\NP$-complete when the perceived payoff is calculated using XR, where each agent is either an extreme optimist or pessimist. The $\NP$ membership is nontrivial and follows several steps. First, we show that if there is a strategy that satisfies the constraints, then there is a finite abstraction of this strategy. Later, we show that this finite abstraction of the strategy has a polynomial representation. 
With this polynomial representation of the strategy, we show that verifying whether a given polynomially represented strategy is a risk-sensitive equilibrium that satisfies the constraints can also be done in polynomial time. 
Finally, we show that if all players are extreme optimists, this problem is $\PTIME$-complete.%, and provide a polynomial time algorithm for the constrained existence problem.
%\thejaswini{We add to this list by introducing a new risk measure that captures the above situation of extreme optimism and pessimism.}
%\thejaswini{We argue that our definition is robust, since this exactly captures the ERisk measure when the parameters are set to $-\infty$ and $+\infty$}
% \thejaswini{When the parameters are anywhere that are not $\pm\infty$, we show that computing Equilibria where ERisk is the outcome is undecidable. }
% \thejaswini{Argue that however, computational costs of precisely computing RSE for such values for stochastic games are undecidable}
% \thejaswini{This makes our definition the only decidable fragment for finding equilibria with entropic risk as a measure decidable}
\section{Preliminaries} 
We assume that the reader is familiar with the basics of probability and graph theory. However, we define some concepts for establishing notation. 

%\thejaswini{We need a good prelims set for probability here.}
%G
\subparagraph*{Probabilities.} Given a (finite or infinite) set of outcomes $\Omega$ and a probability measure $\prob$ over $\Omega$, let $X$ be a random variable over $\Omega$, that is, a mapping $X: \Omega \to \Rb$. We then write $\Eb_\prob[\X]$, or simply $\Eb[X]$, for the expectation of $X$, when it is defined.
%For a random variable $X$, we write $\Eb_M[X]$ to denote the expectation of $X$ to represent the exception of the variable $X$ when the probability measure or distribution over the set of outcomes is $M$. Mostly, when $M$ is known from context, we omit such a measure $M$.
Given a finite set $S$, a \emph{probability distribution} over $S$ is a mapping $d: S \to [0,1]$ that satisfies the equality $\sum_{x \in S} d(x) = 1$.
We write $\Supp (d)$ for the \emph{support} of the distribution $d$, that is, the set of elements $x \in S$ such that $d(x) > 0$.

\subparagraph{Risk measures.}
Given a set $\Omega$ of outcomes, a \emph{risk measure} over $\Omega$ is a mapping $M$ which maps a probability measure $\prob$ over $\Omega$ and a random variable $X$ to a real value $M^\prob[X]$.

Sometimes, in the literature, risk measures are expected to have the following three properties: (1) they are \emph{normalised}, i.e., we have $M^\prob[0] = 0$; (2) they are  \emph{monotone}, i.e., the pointwise inequality $X \leq Y$ implies $M^\prob[X] \leq M^\prob[Y]$; and (3) they are \emph{translative}, i.e., $M^\prob[X + c] = M^\prob[X] + c$ for every constant $c$.
In particular, the expectation of a random variable $\Eb$ is a risk measure.
We do not refer to the above properties again and only state them here to remark that all the risk measures we will consider satisfy the above properties.
We remark that the definition of translative sometimes instead refers to satisfying the opposite of the property we define as translative, $M^\prob[X + c] = M^\prob[X] - c$ for every constant $c$. This is a matter of whether we use our risk measure or its negation.
%This property will not be satisfied by the risk measures we consider, as we want our risk measures to capture the perceived payoff, and not its opposite.
%However, we choose to not use this definition of the translative, as we also want our risk measure to capture perceived reward. % so that they correspond to not inherent measure of risk, but rat  

%as an output such that it is normalised ($\mathbb{RM}(0) = 0$), translative ($\mathbb{RM}(X + c) = \mathbb{RM}(X) + c$) and monotone $X\leq Y\implies \mathbb{RM}(X)\leq \mathbb{RM}(X')$.
%The \emph{perceived reward} on a game $\Game$ for a fixed strategy $\bsigma$ of a risk-measure $\mathbb{RM}$, represented by  $\pReward_{\bsigma}^i$ is $\mathbb{RM}(\prob_{\bsigma\mu_i})$. This value is exactly  the risk measure of the random variable that takes the value equal to player $i$'s outcomes based on the probability distribution by the strategy profile $\bsigma$.


\subparagraph*{Graph, paths, games.}A directed graph $(V,E)$ consists of a set of \emph{vertices} $V$ and a set of ordered pair of vertices, called \emph{edges}, $E$. 
In a directed graph $(V, E)$, for each vertex $u$, we write $E(u)$ to represent the set $E \cap (\{u\} \times V)$.
For simplicity, we often write $uv$ for an edge $(u, v)\in E$.
A \emph{path} in the directed graph $(V, E)$ is a (finite or infinite) word $\pi = \pi_0 \pi_1 \dots$ over the alphabet $V$ such that $\pi_n\pi_{n+1} \in E$ for every $n$ such that $\pi_n$ and $\pi_{n+1}$ exist.
We write $\Occ(\pi)$ for the set of vertices occurring along $\pi$, and $\Inf(\pi)$ for those that occur infinitely often, if there are any.
The prefix $\pi_0 \dots \pi_n$ is written as $\pi_{\leq n}$ or $\pi_{< n+1}$, and the suffix $\pi_n \pi_{n+1} \dots$ is written as $\pi_{\geq n}$ or $\pi_{>n-1}$.
A finite path $\pi = \pi_0 \dots \pi_n$ is \emph{simple} if every vertex occurs at most once along $\pi$.
It is a \emph{cycle} if its last vertex $\pi_n$ is such that $\pi_n\pi_0 \in E$.
%\theju{deleted some definitions about $\Paths_{<\infty}$ here @L\'eonard, can you check? if you ever use it?}
%Given a graph $(V, E)$, we will write $\Paths_{<\infty}(V, E)$ (resp. $\Paths_\infty(V, E)$) the set of finite (resp. infinite) paths in $(V, E)$, and $\Paths_{<\infty}(V, E)_{\|v_0}$ (resp. $\Paths_\infty(V, E)_{\|v_0}$) for the set of finite (resp. infinite) paths in $(V, E)$ that start in the vertex $v_0 \in V$.


\begin{definition}[Game]
    A \emph{game} is a tuple $\Game = (V, E, \Pi, (V_i)_{i \in \Pi}, \p, \mu)$, where we have:
    \begin{itemize}
        \item a directed graph $(V, E)$, called the \emph{underlying} graph of $\Game$;

        \item a finite set $\Pi$ of \emph{players};

        \item a partition $(V_i)_{i \in \Pi \cup \{?\}}$ of the set $V$, where $V_i$ denotes the set of vertices \emph{controlled} by player $i$, and the vertices in $V_?$ are called \emph{stochastic vertices};

        \item a \emph{probability function} $\p: E(V_?) \to [0, 1]$, such that for each stochastic vertex $u$, the restriction of $\p$ to $E(u)$ is a probability distribution;

        \item a mapping $\mu: T \to \Rb^\Pi$ called \emph{payoff function}, where $T$ is the set of \emph{terminal vertices}, i.e. vertices of the graph $(V, E)$ that have no outgoing edges.
        We also write $\mu_i$, for each player $i$, for the function that maps a terminal vertex $t$ to the $i^\text{th}$ coordinate of the tuple $\mu(t)$.
    \end{itemize}
\end{definition}
%Any finite path that ends in a vertex in $T$ is said to give payoff $\mu_i$ for player $i$, and infinite paths and paths that do not end in $T$ give payoff $0$ to player $i$. We therefore often say payoff of a path for player $i$ to refer to the above.

In a more general framework, payoffs can be assigned to all infinite paths.
Here, we only focus on what is usually called \emph{simple quantitative games}, i.e. games in which the underlying graph contains terminal vertices and the payoffs depend only on which terminal vertex is eventually reached.
We thus extend the mapping $\mu$ to the set $(V \setminus T)^\omega \cup (V \setminus T)^* T$ by defining $\mu(v_1 \dots v_k t) = \mu(t)$, and $\mu(v_1 v_2 \dots) = (0)_{i \in \Pi}$ (if no terminal vertex is reached, everyone gets the payoff $0$).
A game is \emph{Boolean} if all payoffs belong to the set $\{0, 1\}$.

An \emph{initialised game} is a tuple $(\Game, v_0)$, usually written $\Game_{\|v_0}$, where $v_0 \in V$ is an \emph{initial vertex}.
In what follows, when the context is clear, we use the word \emph{game} also for an initialised game.
We often assume that we are given a game $\Game_{\|v_0}$ and implicitly use the same notations as in the definition above.

\subparagraph*{Histories and plays.} We call \emph{play} a path in the underlying graph that is infinite, or whose last vertex is terminal.
Other paths are called \emph{histories}.
We will then use the notations $\Hist(\Game)$ to denote finite paths in the graph of the game, and $\Plays(\Game)$ to denote both finite and infinite paths. For a history $h = h_0 \dots h_n$, we write $\last(h) = h_n$.
We will also write $\Hist_i(\Game)$ for the set of histories whose last vertex is controlled by player $i$.
%If the game $\Gc$ is clear from context, we omit it.
A history or play in an initialised game $\Game_{\|v_0}$ is a history or play in $\Game$ whose first vertex is $v_0$.




\begin{definition}[Markov decision process, Markov chain]
    A (initialised or not) \emph{Markov decision process} is a game with one player.
    A \emph{Markov chain} is a game with zero player.
\end{definition}




\paragraph*{Strategies, and strategy profiles}

% \subparagraph*{Stable set.} A subset of vertices $W$ is a \emph{stable} vertex set in $\Game_{\|v_0}$ if\theju{What about the word closed instead of stable? Can this be moved to the appendix, since we only need it for the proofs?}
% \begin{itemize}
%     \item every $v \in W$ is accessible from $v_0$ using only vertices of $W$, 
%     \item for every $v \in W \setminus V_?$, we have $E(v) \cap W \neq \emptyset$, and 
%     \item for every $v \in W \cap V_?$, we have $E(v) \subseteq W$.
% \end{itemize}

    In a game $\Game_{\|v_0}$, a \emph{strategy} for player $i$ is a mapping $\sigma_i$ that maps each history $hu \in \Hist_i(\Game_{\|v_0})$ to a probability distribution over $E(u)$.
The set of possible strategies for player $i$ in $\Game_{\|v_0}$ is written as $\Strat_i(\Game_{\|v_0})$.
A path $\pi_0 \pi_1 \dots$ (be it a history or a play) is \emph{compatible} with the strategy $\sigma_i$ if for each $k$ such that $\pi_k \in V_i$, we have the probability that the strategy $\sigma_i$ proposes $h_{k+1}$ after  history $h_{\leq k}$ is positive, that is, $\sigma_i(h_{\leq k})(h_{k+1}) > 0$.
A \emph{strategy profile} for a subset $P \subseteq \Pi$ is a tuple $(\sigma_i)_{i \in P}$.
A strategy profile for the set $P$ of players is written $\bsigma_P$, or simply $\bsigma$ when $P = \Pi$. We also write $\bsigma_{-i}$ for $\bsigma_P$ where $P = \Pi \setminus \{i\}$.
Similarly, we use $(\sigma_{-i}, \sigma'_i)$ to denote the strategy profile $\btau$ defined by $\tau_i = \sigma'_i$ and $\tau_j = \sigma_j$ for $j \neq i$. 
We sometimes write $\bsigma(hv)$ to mean $\sigma_i(hv)$ where $i$ is the player controlling $v$, or $\p(v)$ when $v \in V_?$.
We sometimes also write $V_{-i}$ instead of $\bigcup_{j \neq i} V_j$. %, or $\Hist_{-i}$ for $\bigcup_{j \neq i} \Hist_j$\leon{Do we?}.

For some history $h$, and a strategy $\sigma_i$, we define the strategy truncated to a history $h$, written $\sigma_{i\| hv}$, as the strategy $\sigma'_i: h' = \sigma_i(hh')$ in the game $\Game_{\|v}$.

A strategy profile $\bsigma_{-i}$ in the game $\Game_{\|v_0}$ defines an initialised Markov decision process $\Game_{\|v_0}(\bsigma_{-i})$, where the vertices of the (infinite) underlying graph are the histories of $\Game_{\|v_0}$ and the edges are added from $hu$ to each the history $huv$ iff  $uv \in E$.
Similarly, a strategy profile $\bsigma$ for $\Pi$ defines an initialised Markov chain $\Game_{\|v_0}(\bsigma)$.
Thus, it also defines a probability measure $\prob_\bsigma$ over plays --- which turns the payoff functions $\mu_i$ into random variables.
%We omit the game $\Gc$ and $v_0$ if that is clear from context. Alternatively, if we consider a strategy truncated to a history $h$, then we instead write $\prob_{\sigma_{\| h}}$.


    
\subparagraph{Pure, stationary, and positional strategies.}
We say that a strategy $\sigma_i$ is \emph{pure} when for each history $hu$, there is a vertex $v$ such that $\sigma_i(hu)(v) = 1$; then we often just write $\sigma_i(hu) = v$. 
%Consider the equivalence relation over histories, defined by $h \approx_{\sigma_i} h'$ if and only if the two strategies $\tau_i: h'' \mapsto \sigma_i(hh'')$ and $\tau_i': h'' \mapsto \sigma_i(h'h'')$ are identical. The number of such classes will be called the \emph{memory} of $\sigma_i$. 
%A strategy $\sigma_i$ is \emph{finite-memory} if the equivalence relation $\approx_{\sigma_i}$ has finitely many classes. 
We say that $\sigma_i$ is \emph{stationary} when for every two histories $hu, h'u \in \Hist_i(\Game_{\|v_0})$, we have $\sigma_i(hu) = \sigma_i(h'u)$.
In that case, we sometimes assume that strategy $\sigma_i$ is defined in every game $\Game_{\|u}$ and simply write $\sigma_i(u)$ for $\sigma_i(hu)$.
Finally, the strategy $\sigma_i$ is \emph{positional} when it is pure and stationary.
Those concepts are naturally generalised to strategy profiles.

\subparagraph*{Memory structures.}
%Observe that we can represent a stationary strategy as a function that maps every vertex to a distribution, and similarly, a positional strategy as a function that maps every vertex to another vertex.
% We represent the set of all stationary strategies for player $i$ using $\Mless_i(\Game)$ and the set of all positional strategies of player $i$ as $\Pos_i(\Game)$.
A \emph{memory structure} for player $i$ is a tuple $(S_i, s_0, \delta_i, \nu_i)$, where $S_i$ is a finite set of \emph{memory states}, where $s_0 \in S_i$ is an \emph{initial state}, where $\delta_i$ is a \emph{memory-update mapping} that maps each pair $(s, v) \in S_i \times V$ to a memory state $s'$, and where $\nu_i$ is an \emph{output mapping} that maps each pair $(s, v) \in S_i \times V_i$ to a distribution $d$ over $E(v)$.
The memory-update mapping can be extended to a mapping $\delta_i^*: \Hist(\Game_{\|v_0}) \to S_i$ with $\delta_i^*(\epsilon) = m_0$ and $\delta_i^*(hu) = \delta_i(\delta_i^*(h), u)$ for each history $hu$.
The memory structure then induces a strategy $\sigma_i$ defined by $\sigma_i(hu) = \nu_i(\delta^*_i(h), u)$ for each history $hu \in \Hist_i\Game_{\|v_0}$.
A strategy induced by a memory structure is called \emph{finite-memory strategy}.
Note that stationary strategies are exactly the strategies that are induced by a memory structure with $|S_i| = 1$.

%We analogously define memory structures for all players. 
A tuple of finite-memory strategy profiles for each player is a \emph{finite-memory strategy profile}  is given by a similar object $(S, s_0, \delta, \nu)$ to memory structure, where by replacing $\nu$ by its restriction to $S \times V_i$, we obtain a memory structure that induces the strategy $\sigma_i$.%\footnote{Here, it might be useful to point out that our model of memory structure is deterministic, even though the outputs are probability distributions: after a given history $h$, the memory state $\delta^*(h)$ is deterministically defined.
%This prevents the players from behaviours in which they would, for example, toss a common coin at the beginning of the game, and all play according to the outcome: such behaviours do not fit in our formalism of strategy profiles, which assumes implicitely that they cannot communicate during the game, since each action can be based only on the vertices that have been visited at a given point of time.}.



% Given any finite set $M_i$, and a memory update map $\nu_i$ that is a function $M_i\times V\to M_i\times \Dist(V)$, and an initial memory state $m_\init^i$, then we can construct a finite memory strategy $\sigma_i$
% which we define recursively. For the empty history $\epsilon$, the \emph{current memory state} is $m_\init^i$, 
% for a history $hv$, where the memory state $m$ of history $h$, the memory state of the history $hv$ is defined as $m'$.
% where 
% $(\Dc,m') = \nu_i(m,v)$. 
% We define the strategy $\sigma_i(h)$ for player $i$ as 
% and the distribution $\Dc$ obtained above if $v\in V_i$. 
% %Note that if $v\notin V_i$, then the distribution is the empty-distribution, otherwise, this does not represent a valid strategy.  %\theju{empty distribution to be defined}

% We therefore call such a tuple $\Mc_i = (M_i,\nu_i,m_\init^i)$ as a memory structure for player $i$, and $M_i$ memory states of the strategy $\sigma_i$. A memory structure can be smaller than the number of equivalence classes in $\approx_{\sigma_i}$ since all elements of an equivalence class must correspond to histories that have the same last vertex, whereas the same is not true for elements of $M_i$ in a memory structure.  

% For a finite set $M$ and a memory update function $\nu$ such that $M\times V\to M\times \Dist(V)$, along with  $p$ initial memory functions $\Bar{m_\init} =(m_\init^1,\dots,m_\init^p)$, we can also describe $p$ different memory-structures for $p$ players $\Mc_i = (M, \nu, m_\init^i)$ for each $i\in \{1,\dots,p\}$.
% We refer to such an $M$ as the memory states of a strategy of a \emph{coalition} $\{1,\dots,p\}$ of players. We define coalition and strategies profiles in more detail below. 

% Sometimes, we deal with finite memory strategies of specific types, composed of finitely many positional or stationary strategies. In such cases we call such objects memory structure.
% If $M = \Mc_i\times V$, such that $\mu_i(m, v) = $
% A memory structure $\Mc_i$ is defined 

% A specific kind of 
% finite memory strategies are those that with a memory-structure $M$ consists of a finite set $M\times \Mless(\Gc)$ along with transitions from $M\times $



% Suppose each player has a finite memory strategy with a memory structure and a memory-update function $(M_i,\nu_i)$, then this gives a memory structure $(\Bar{M},\Bar{\nu})$, where $M = \prod_i M_i$ and $\nu$ takes vertex $v$ and based on weather 
% To represent a strategy profile in which each player's strategy is finite memory, 
% \begin{itemize}
%     \item the vertices of the (infinite) underlying graph are the histories of $\Game_{\|v_0}$;

%     \item there is an edge from each vertex $hu$ to each vertex $huv$ such that $uv \in E$, and only there;

%     \item the initial vertex is $v_0$;

%     \item the unique player $i$ controls the set $\Hist_i\Game_{\|v_0}$;

%     \item the probability function $\p_{\sigma_i}$ is defined by $\p_{\sigma_i}(hu, huv) = \p(uv)$ if $u \in V_?$, and $\p_{\sigma_i}(hu, huv) = \bsigma_{-i}(hu)(v)$ if $u \in V_{-i}$;

%     \item the payoff function is $\mu_i$.
% \end{itemize}





\subparagraph{Risk-sensitive equilibria, constrained existence problem.}
In multiplayer stochastic games, we wish to study generalisations of the classical Nash equilibria where the expectation is replaced by other risk measures.
Such generalisations are called \emph{risk-sensitive equilibria}~\cite{Now05}. We define this for games played over graphs
%A strategy profile is a risk-sensitive equilibrium if no player can reduce their risk measure by deviating from the strategy.
%The concept is similar to Nash equilibria when the payoffs are instead replaced with the player's perceived risk.  

When $M$ is a risk measure and $\bsigma$ is a strategy profile, we write $M(\bsigma)$ for $M^{\prob_\bsigma}$.

\begin{definition}[Risk-sensitive equilibrium]
    Let $\Game_{\|v_0}$ be a game, and let $\bM = (M_i)_{i \in \Pi}$ be a tuple of risk measures.
    Let $\bsigma$ be a strategy profile in $\Game_{\|v_0}$, let $i$ be a player, and let $\sigma'_i$ be a strategy for player $i$, called \emph{deviation} of player $i$ from $\bsigma$.
    The deviation $\sigma'_i$ is \emph{profitable} with regards to the risk measure $M_i$ if we have $M_i(\bsigma_{-i}, \sigma'_i)[\mu_i] > M_i(\bsigma)[\mu_i]$
    The strategy profile $\bsigma$ is a $\bM$-\emph{risk-sensitive equilibrium}, or $\bM$-RSE, if no player $i$ has a profitable deviation from $\bsigma$ with regards to $M_i$.
\end{definition}


% We sometimes say that 
% in an MDP $\MDProc$, a strategy $\sigma_i$ is \emph{risk-optimal} for player $i$ if for every strategy $\sigma'_i$, we have $\M(\sigma'_i)[\mu_i] \leq \M(\sigma_i)[\mu_i]$. Notice above that in a risk-sensitive equilibria $\bsigma$ in a game, for every player, the MDP obtained by considering the game with the strategy $\sigma$ when viewed as an MDP against the strategies of other players, \leon{Do we use this?}
% \theju{I think this paragraph is not used here at all. To verify maybe(?). I never use it}\theju{@Leonard, check if this is needed anywhere and remove?}

The following problem is the main focus throughout our paper.

\begin{question}[Constrained existence of risk-sensitive equilibria]
    Given a game $\Game_{\|v_0}$, a tuple of risk measures $\bM$, and two payoff vectors $\bx, \by \in \Qb^\Pi$, does there exist a $\bM$-RSE $\bsigma$ in $\Game_{\|v_0}$ such that for each $i \in \Pi$, we have $x_i \leq M_i(\bsigma)[\mu_i] \leq y_i$?
\end{question}

To turn this problem into an algorithmic decision problem, we need to restrict it to some specific sets of risk measures that can be finitely encoded. 
That is what we do in the sequel of this paper, with the \emph{entropic risk measure}, and later with the \emph{extreme risk measure}.

%\thejaswini{Write a note on randomness used.}





\section{Entropic risk measure}\label{sec:ERM}
The entropic risk measure is a measure of the perceived payoff, which depends on the aversion or inclination of the player toward risk through the exponential utility function. 
It is defined using a \emph{risk parameter}, i.e. a real value $\rho\in\Rb \setminus \{0\}$: large positive values indicate risk-averseness, large negative values risk-inclination. To see a visual representation of the entropic risk measure, see \cref{fig:example_re} in the introduction.

\begin{definition}[Entropic risk measure]
Given a risk parameter $\rho$, the \emph{entropic risk measure} is defined for every probability measure $\prob$ and random variable $X$ as
$$\re_{\rho}^\prob[X] = -\frac{1}{\rho} \log_e \left( \Eb^\prob \left[ e^{-\rho X}\right] \right).$$
For computational reasons, this definition is generalised by allowing every base $\beta > 1$ instead of Euler's constant. The \emph{entropic risk measure with base $\beta$} is then defined by: 
$$\re^\prob_{\beta\rho}[X] = -\frac{1}{\rho} \log_\beta \left( \Eb^\prob \left[ \beta^{-\rho X}\right] \right).$$
\end{definition}

The three parameters $\prob$, $\rho$ and $\beta$ can be omitted when they are clear from the context.

\begin{remark}
\begin{itemize}
    \item For every $\beta$ and $\rho$, the entropic risk measure $\RM_{\beta\rho}$ is a risk measure.

    \item By enabling any base $\beta$, we obtain a definition that is more general only on a computational level, since handling Euler's constant may not be equivalent to handling rational values.
    Baring computational concerns, these definitions with different bases are equivalent, since for every $\beta$ we have $\RM_{\beta\rho} = \RM_{e\rho'}$, where $\rho' = \rho \log_e(\beta)$.

    \item The above definition implies that for $\rho = 0$, the function is not defined.
    However, it is known that for all $\prob$, $\beta$ and $X$, the quantity $\RM_{\rho}$ converges to $\Eb[X]$ when $\rho$ tends to $0$ (see e.g.~\cite{PDM20}).
    Therefore, we henceforth assume that $\RM_{0}[X] = \mathbb{E}[X]$ to make risk entropy defined for all finite risk parameters $\rho$.
\end{itemize}
\end{remark}

When we are given a profile $\brho = (\rho_i)_{i \in \Pi}$ of risk parameters, we will sometimes write $\M_{\beta\brho}[\mu]$ for the tuple $\left(\M_{\beta\rho_i}[\mu_i]\right)_{i \in \Pi}$.
Risk entropy defines a family of RSEs, namely the $(\M_{\beta\rho_i})_i$-RSEs, that we also call \emph{$(\beta, \brho)$-entropic risk-sensitive equilibria}, or $(\beta, \brho)$-ERSEs.
% \begin{restatable}{lemma}{ERzeroExp}\label{lemma:ERzeroExp}\theju{would be nice to add a citation, but can't seem to find any}
% The limit risk entropy of $X$ when $\rho$ tends to $0$ exists, and equals the expectation $\lim_{\rho \to 0} \re_{\beta,\rho} [X] = \Eb[X]$.
% \end{restatable}
% The above lemma shows that for each value $\beta$ and risk-sensitivity profiles profiles $\brho$, we can define a new risk measure $\re_{\beta,\rho}$ for each value of $\rho$. For a risk sensitivity profile $\brho$ which assigns a risk parameter for each player, we obtain 
% a risk measure $\re_{\beta,\rho}$ for each player. 
% We refer to an RSE where the risk-measures for player $i$ is $\re_{\beta,\rho_i}$ as a $(\beta,\brho)$-RSE. 
The following theorem states the existence of such an RSE that uses no randomness in its strategy profile, in cases where all the payoffs are non-negative. 

\begin{theorem}[Existence of ERSE]\label{thm:existanceRSE}
    Let $\Game_{\|v_0}$ be a simple stochastic game with only non-negative payoffs.
    Then, there exists a (pure) $(\beta,\rho)$-ERSE over $\Game_{\|v_0}$.
\end{theorem}

\begin{proof}
 Pure Nash equilibria always exists in a stochastic multi-player games with prefix-closed Boolean objectives~\cite[Theorem 3.10]{Umm10} (a correction of an existing proof~\cite{CMJ04}). It is known that simple stochastic games where rewards are all positive (or all negative) can be converted into a game with reachability objectives such that if there is an NE in one, there is an NE in the converted game with the reachability objective. Indeed, if all the rewards are positive, we can always scale the rewards for each player of a stochastic game to ensure they are in the unit interval $[0,1]$. If the rewards are within the unit interval, then for terminals with reward $p$, we can instead add a probabilistic node that reaches this terminal vertex with probability $p$. 
Therefore, with the same result, Nash equilibria always exist in simple stochastic games with non-negative rewards on the terminals. 

Then, we can conclude our theorem using the following lemma.

\begin{restatable}[App.~\ref{lemma:RSEtoQSSG}]{lemma}{RSEtoQSSG}\label{lemma:RSEtoQSSG}
Given a game $\Game_{\|v_0}$ and a tuple $\brho \in \Rb^\Pi$, there exists a game $\Game'_{\|v_0}$ with the same underlying graph, player set, and probability function (but possibly different payoff function), such that the $(\beta,\rho)$-ERSEs in $\Game_{\|v_0}$ are exactly the Nash equilibria in $\Game'_{\|v_0}$. \qedhere
\end{restatable}
\end{proof}

We conjecture that this result remains true when we remove the guarantee that rewards are non-negative.
We now turn to the constrained existence problem of $\tpl{\beta,\brho}$-ERSEs.
Unfortunately, it is undecidable in the general case.

\begin{proposition}\label{proposition:Undecidable}
    The constrained existence problem of $\tpl{\beta,\brho}$-ERSEs with $\brho \in \Qb^{\Pi}$ is undecidable, even for any fixed value of $\beta$, for $\brho = (0)_i$, and with only nonnegative payoffs. %Further, even when players are restricted to pure strategies, the problem remains undecidable
\end{proposition}

\begin{proof}
         The undecidability of the constrained existence problem follows from the work of Ummels and Wojtczak~\cite[Theorem 4.9]{UW11} where they show the undecidability of the constrained existence problem for Nash equilibria in the setting with 10 or more players. Since Nash equilibria is a specific instance of the setting of ERSEs where the risk parameters $\brho$ is $0$ for each player, the undecidability of our setting follows. 
\end{proof}

We therefore turn our attention to the constrained existence problem when the class of strategies considered is restricted. 

\begin{restatable}[App.~\ref{app:ERRSErestricted}]{theorem}{stationaryRSE}\label{thm:ERRSErestricted}
The constrained existence problem of $(\beta,\brho)$-ERSEs, in quantitative simple stochastic games:
\begin{enumerate}
    \item remains undecidable when players are restricted to pure strategies;\label{itm:ERRSEitmundec} %(which use no randomness but arbitrary amounts of memory) is undecidable;
    \item is decidable when players are restricted to stationary strategies\label{itm:ERRSEdecidable}
\begin{enumerate}
        \item subject to Shanuel's conjecture if $\beta = e$ and the risk-parameters $\rho_i$ are algebraic;\label{itm:ERRSEitmShanuel}
        \item in $\PSPACE$ if the risk parameters and the base $\beta$ are algebraic, in which case it is also $\NP$-hard and $\SQRTSUM$-hard.\label{itm:ERRSE:PSPACE}
        The $\NP$ lower bound also holds for the case where strategies are restricted to positional strategies.
    \end{enumerate}
\end{enumerate}
\end{restatable}
%The results are obtained as a combination of results from 

\begin{proof}[Proof Sketch]
    The undecidability of the case where pure strategies are considered is inherited from Nash equilibria~\cite[Theorem~4.9]{UW11}, since the reduction uses only pure strategies. 
        The decidability of this stationary case is reminiscent of similar results for the two-player zero-sum case, which was recently studied in the work of Baier et al.~\cite{BCMP24}.
    However, the techniques used are quite different and also require inspiration from the work of Ummels and Wojtczak~\cite[Theorem 4.5, Theorem 4.6]{UW11}, with significant modifications. 
    We write formulas in the existential theory of reals ($\exists\Rb$) which puts them in $\PSPACE$. 
    For the case $\beta = e$, this formula can be written in the existential theory of reals with exponentiation, which is decidable subject to Shanuel's conjecture, which is a well-known conjecture in the field of transcendental number theory~\cite{Lan66}. The lower bounds of $\NP$-hardness and $\SQRTSUM$-hardness  also follow from the works of Ummels and Wojtczak~\cite[Theorem~4.4,Theorem~4.6]{UW11}. The exact complexity of $\SQRTSUM$ (deciding, given a set $\{a_1, \dots, a_n\} \subseteq \Nb$ and an integer $t$, whether we have $\sum_i \sqrt{a_i} \leq t$) is open and is known to lie in the polynomial hierarchy and in the fourth level of the counting hierarchy~\cite{AKBM06}. 
\end{proof}


% \begin{lemma}\label{lemma:stationaryRSE}
%         The constrained existence problem for RSE when players are restricted to pure strategies for quantitative simple stochastic games is undecidable.
% \end{lemma}
% \begin{proof}
%\end{proof}
% \begin{restatable}{lemma}{stationaryRSE}\label{lemma:stationaryRSE}
%         The constrained existence problem for $(\beta,\brho)$-RSE when players are restricted to stationary (but stochastic) strategies for quantitative simple stochastic games is \begin{itemize}
%             \item in $\PSPACE$ if the risk-parameters for each player $\brho$ as well as the base $\beta$ is algebraic; 
%             \item decidable---subject to Shanuel's conjecture---if the risk-parameters for each player $\brho$  is algebraic and the base of the exponent is the Euler's constant $\beta=e$.
%             \end{itemize}
%         The problem is at least $\NP$-hard. When the strategies are stationary but also randomised, the problem is also $\SQRTSUM$-hard.
    

    % We know from \cref{lemma:RSEtoQSSG}, we show that %
     

    %Further, we need the results in Baier et al.,~\cite{BCMP24} to reason about computational related to checking if a strategy 

%\theju{to check if this doesn't hit really low values. Need to bound it}


% \begin{lemma}\label{lemma:positionalRSE}
%         The constrained existence problem for RSE when players are restricted to stationary pure strategies for quantitative simple stochastic games is  in $\PSPACE$ if the risk-parameters for each player $\brho$ as well as the base $\beta$ is algebraic. Further, the problem is $\NP$-hard.
% \end{lemma}
% \begin{proof}
%     For the lower bounds, we just remark that the corresponding problem of constrained existence of Nash equilibria, when players are restricted to positional strategies is both $\NP$-hard~\cite[Theorem 4.4]{UW11}
%     from the work of Ummels and Wojtczak. Since Nash equilibria is a specific instance of the setting of RSE, where the risk parameters of each player is $0$, the $\NP$-hardness for the more general case of RSE follows.
%     For the upper bound, we simply observe that the same proof as the one for the upper bound of~\cref{lemma:stationaryRSE} can be adapted to this situation. 
% \end{proof}

% From this point, we will therefore extend, by continuity the definition of $\RE_{\beta,\rho}$ for $\rho = 0$.




% \begin{definition}[Risk-sensitive Equilibria]
%     Let $\Game_{\|v_0}$ be a game, and let $\brho \in (\Rb \setminus \{0\})^\Pi$ be a \emph{risk-sensitivity profile}.
%     Then, the strategy profile $\bsigma$ is a \emph{$\beta\brho$-risk-sensitive equilibrium}, or \emph{$\beta\brho$-RSE} for short, if and only if for each player $i$, the strategy $\sigma_i$ is $\beta,\rho$-risk-optimal in the MDP $\MDProc(\bsigma_{-i})$.
% \end{definition}
\section{Extreme risk measure}\label{sec:XR}
This section introduces a new risk measure that provides a tractable alternative to existing risk measures available in the literature. We provide a simpler, yet robust, alternative that allows us later to tackle the constrained existence of equilibria in multiplayer settings.
 %In the general case where the base of the exponent is the Euler's constant $e$, the problem is decidable only subject to Shanuel's conjecture.
%The situation does not improve significantly, with the problem being both $\NP$-hard and $\SQRTSUM$-hard even when the base of the exponent is rational.  

%\thejaswini{To add an example. Or discuss previously discussed examples here. }

%\subsubsection*{Definition}
Let us consider a random variable $X$ that ranges over $\Rb$. 
%We define the \emph{extreme pessimistic risk-measure} or just \emph{pessimistic risk} below, but first provide an intuition. 
The \emph{pessimistic risk measure} of $X$ is the highest value $x$ such that $X$ almost-surely takes a value above $x$.
When $X$ takes finitely many values, that corresponds to the least value that it takes with positive probability.
In probability theory, that measure is sometimes referred to as \emph{essential infimum}, written $\essinf$.
The definition of \emph{optimistic risk measure} is symmetric.
%the lowest value that a random variable takes with positive probability. To capture this, we use the notion of $\essinf$.

\begin{definition}[Optimistic, pessimistic risk measure]
The pessimistic risk measure of a random variable $X$ is defined by
$\pexp[X] = \essinf(X) = \sup \{x \in X ~|~ \prob(X \geq x) = 1\}$.
Analogously,  the \emph{optimistic risk measure} of $X$ is $\oexp[X] = \esssup(X) =  \inf \{x \in X ~|~ \prob(X \leq x) = 1\}$.    
\end{definition}

%When the set of values that the variable $X$ can take with non-zero probability is finite, those simply correspond, respectively, to the minimum and the maximum of these values.



When we are given a game $\Game_{\|v_0}$, we can assign an risk measure for each player by defining a partition $(P, O)$ of $\Pi$, where the set $P$ represents the set of players that are \emph{pessimists}, whose perceived payoffs are defined by the pessimistic risk measure, while $O$ represents the \emph{optimists}, who intend to maximise their optimistic risk measure.
For convenience, we group both measures under the umbrella term \emph{extreme risk measure (XR)}, and often assume that $(P, O)$ is given; % where extreme risk measure corresponds to either the extreme pessimistic or the extreme optimistic risk measure, depending on player.  
then, we write $\X_i$ for $\pexp$ when $i \in P$, and for $\oexp$ when $i \in O$.
Since each player $i$ is usually interested only in the risk measure of their own payoff, we will also write $\X_i(\bsigma)$ for the quantity $\X_i(\bsigma)[\mu_i]$.
We define \emph{extreme risk-sensitive equilibria}, or XRSEs for short, as $(\X_i)_i$-RSEs.


% For a multi-player stochastic game with players $\Pi$, and a partition $(P,O)$ of the players $\Pi$ then assigns an extreme risk-measure for each player: the pessimistic risk measure for players $i\in P$ and the optimistic risk measure otherwise. 
%Therefore, any partition  $(P,O)$ of the players $\Pi$ gives rise to a risk measures for each player and therefore also a notion of risk sensitive equilibria. We call a strategy profile a $(P,O)$-Risk sensitive equilibrium or a $(P,O)$-RSE for a partition of the players $(P,O)$. 


\subsubsection*{Equivalence to limit of Entropic Risk Measure}
We show that our definition of extreme risk measure corresponds to the limit cases of entropic risk measure.
Observe that in \cref{fig:example_re}, following the  blue strategy, the only payoffs that are obtained with positive probability were $40$ and $0$, which are also the limits of the risk entropy when $\rho$ tends to infinite values.
On the other hand, in the red strategy, the only payoff obtained with positive probability is payoff $1$. Although the payoff $0$ is possible since the play $a^\omega$ is compatible with every strategy, this play must be ignored since it is realised with probability $0$.

\begin{restatable}[App.~\ref{app:RE=PEorOE}]{theorem}{REisPEOE}\label{thm:RE=PEorOE}
    Let $X$ be a random variable that ranges over $\Rb$, and let $\beta > 1$.

    \begin{itemize}
        \item The limit risk entropy of $X$ when $\rho$ tends to $+\infty$ exists and is equal to the pessimistic risk measure, that is, we have $\lim_{\rho \to +\infty} \re_{\beta\rho} [X] = \pexp[X]$.    
       \item Similarly, the limit risk entropy of $X$ when $\rho$ tends to $-\infty$ exists and is equal to the pessimistic risk measure, that is, we have $\lim_{\rho \to -\infty} \re_{\beta\rho} [X] = \oexp[X]$.
    \end{itemize}
\end{restatable}
%We henceforth refer to the perceived reward computed according to the risk-measure as $\xr$ to refer to $\pexp$ or $\oexp$ based on if the player is a pessimist or an optimist.\theju{I am not a fan of expectation necessarily in this context. Would rather call it extreme pessimistic risk or something}

\subsubsection*{Extreme risk-sensitive equilibria exist}
We now answer a fundamental question about equilibria, which is if one always exists. 
% This existence result does not trivially extend to quantitative rewards, where the rewards can be negative.
% Based on their result, we were also able to show in the previous section that there is always a risk-sensitive equilibrium when the risk-measures are the entropic risk measures and the rewards are non-negative.\leon{Maybe we could reduce this paragraph? Sounds like repetition now.}
We show that (stationary) XRSEs are guaranteed to exist in games with only non-negative rewards, similarly as ERSEs.
But our proof does not rely on the same arguments, and we instead give a constructive proof.
% Our result does not follow from any of the previous existence of equilibria, since our risk measure is the limit value of entropic risk measure.

\begin{restatable}[App.~\ref{app:XRSEexists}]{theorem}{XRSEexists}\label{thm:XRSEexists}
    Let $\Game_{\|v_0}$ be a game with only non-negative rewards, and let $(P,O)$ be a partition of $\Pi$.
    Then, there exists a stationary XRSE in $\Game_{\|v_0}$.
    Moreover, there exists an algorithm that, given such a game, outputs the representation of such an XRSE in time $\Oh(m^2 p)$, where $m$ is the number of edges, and $p$ the number of \emph{pessimistic} players.
\end{restatable}

\begin{proof}[Proof sketch.]
    Our algorithm generates an XRSE by constructing a decreasing sequence $E = E_0, E_1, \dots$ of sets of edges, and considering, for each $k$, the stationary strategy profile that randomises between all the outgoing edges in $E_k$ from all vertices.
    
    Let us illustrate it with the game depicted by Figure~\ref{fig:ex_extreme1}, which involves two pessimists, player $\Circle$ and player $\Square$.
    In that game, both players want to leave the cycle, but each of them would prefer the player to leave.
    If we first consider the strategy profile that always randomises between all the available edges, then both terminal vertices are reached with positive probability, and it is almost sure that one of them is reached: both players get therefore risk measure $1$.
    Then, player $\Square$ (and symmetrically player $\Circle$) has a profitable deviation by refusing to leave the cycle, and always going back to the vertex $a$.
    Note that player $\Circle$ cannot detect such a deviation of strategy, since she does not have access to the internal coins tossed by player $\Square$. %cannot know whether player $\Square$ went back to $a$ as a deviation, or as one of the possible outcomes of his randomised strategy: that deviation is \emph{undetectable}.
    Then, we remove the edge $bt_2$ (or $at_1$). This results in a set of edges where player $\Square$ gets the payoff $2$, and player $\Circle$ cannot get more than $1$, ensuring that the new strategy profile that we obtain is a (stationary) XRSE.
\end{proof}

Like in the case of ERSEs, we conjecture that existence, and even existence of a stationary strategy profile, remain true in the general case.


\begin{figure}[h] 
			\centering
            \begin{subfigure}[t]{0.3\textwidth}
			\begin{tikzpicture}[->,>=latex,shorten >=1pt, initial text={}, scale=1, every node/.style={scale=0.9}]
				\node[initial above, state] (a) at (0, 2) {$a$};
				\node[state, rectangle] (b) at (2, 2) {$b$};
                \node (t1) at (0, 0.75) {$t_1:~\stack{\circ}{1}~\stack{\square}{2}$};
                \node (t2) at (2, 0.75) {$t_2:~\stack{\circ}{2}~\stack{\square}{1}$};
                \path (a) edge[bend left] (b);
				\path (b) edge[bend left] (a);
                \path (a) edge (t1);
                \path (b) edge (t2);
			\end{tikzpicture}
			\caption{Two pessimists}
			\label{fig:ex_extreme1}
            \end{subfigure}
            \hfill
            \begin{subfigure}[t]{0.3\textwidth}
			\begin{tikzpicture}[->,>=latex,shorten >=1pt, initial text={}, scale=1, every node/.style={scale=0.9}]
                \node[initial above, state, stoch] (c) at (1, 3) {$c$};
				\node[state] (a) at (0, 2) {$a$};
				\node[state, rectangle] (b) at (2, 2) {$b$};
                \node (t1) at (0, 0.75) {$t_1:~\stack{\circ}{1}~\stack{\square}{2}$};
                \node (t2) at (2, 0.75) {$t_2:~\stack{\circ}{2}~\stack{\square}{1}$};
                \path (c) edge (a);
                \path (c) edge (b);
                \path (a) edge[bend left] (b);
				\path (b) edge[bend left] (a);
                \path (a) edge (t1);
                \path (b) edge (t2);
			\end{tikzpicture}
			\caption{Two pessimists with a common coin}
			\label{fig:ex_extreme2}
            \end{subfigure}
            \hfill
            \begin{subfigure}[t]{0.3\textwidth}
			\begin{tikzpicture}[->,>=latex,shorten >=1pt, initial text={}, scale=1, every node/.style={scale=0.9}]
                \node[initial above, state, stoch] (c) at (1, 4.25) {$c$};
                \node[state] (d) at (-0.5, 3.25) {$d$};
                \node[state, rectangle] (e) at (2.5, 3.25) {$e$};
				\node[state] (a) at (0, 2) {$a$};
				\node[state, rectangle] (b) at (2, 2) {$b$};
                \node (t1) at (0, 0.75) {$t_1:~\stack{\circ}{1}~\stack{\square}{2}$};
                \node (t2) at (2, 0.75) {$t_2:~\stack{\circ}{2}~\stack{\square}{1}$};
                \node (t3) at (1, 3.25) {$t_3:~\stack{\circ}{2}~\stack{\square}{2}$};
                \path (c) edge (d);
                \path (c) edge (e);
                \path (d) edge (t3);
                \path (e) edge (t3);
                \path (d) edge (a);
                \path (e) edge (b);
                \path (a) edge[bend left] (b);
				\path (b) edge[bend left] (a);
                \path (a) edge (t1);
                \path (b) edge (t2);
			\end{tikzpicture}
			\caption{Two pessimists with a common coin and some temptation}
			\label{fig:ex_extreme3}
            \end{subfigure}
        \caption{Some games involving two pessimistic players}\label{fig:ex_extreme}
\end{figure}


%\subsubsection*{Some results for $(P,O)$-equilibria}
% Given a Markov decision process $\MDProc$, a strategy $\sigma$ defines a probability distribution $\prob_\sigma$, and therefore a pessimistic expectation $\pexp(\sigma)$ and an optimistic expectation $\oexp(\sigma)$.



% Since all vertices are stochastic, a Markov chain defines a probability distribution $\prob_\MChain$ on the set of infinite paths in the graph $(V, E)$ starting from $v_0$, as follows: for every path $v_0 v_1 \dots v_n$, we have:
% $$\prob_\MChain\left(v_0 \dots v_{n-1} \Paths_\infty(\MChain_{\|v_n})\right) = \p(v_0v_1) \p(v_1v_2) \dots \p(v_{n-1}v_n).$$
% Note that the probability measure that is thus defined is unique, by TODO\theju{Check what this is? Move to prelims?}



% \begin{definition}[Pessimistic and optimistic optimality, $(P, O)$-equilibrium]
%     The strategy $\sigma$ is \emph{pessimistically optimal} in the MDP $\MDProc_{\|v_0}$ if and only if for every strategy $\sigma'$, we have $\pexp_{\sigma'}[\mu] \leq \pexp_{\sigma}[\mu]$.
%     Similarly, it is \emph{optimistically optimal} if for every alternative strategy $\sigma'$, we have $\oexp_{\sigma'}[\mu] \leq \oexp_{\sigma}[\mu]$.

%     Let us now consider a game $\Game_{\|v_0}$, and a partition $(P, O)$ of $\Pi$, in which the elements of $P$ are called \emph{pessimists}, and the elements of $O$ are called \emph{optimists}.
%     Then, a strategy profile $\bsigma$ is a $(P, O)$-equilibrium if and only if each strategy $\sigma_i$ with $i \in P$ is pessimistically optimal, and each strategy $\sigma_i$ with $i \in O$ is optimistically optimal.
% \end{definition}



\section{Constrained existence of extreme risk-sensitive equilibria}\label{sec:NPComplete}
We now study the computational complexity of the constrained existence problem of XRSEs.
The main result of this section is the following theorem, which proves that, contrary to the same problem with ERSEs, it is a decidable fragment of the constrained existence of RSEs.

\begin{theorem}\label{thm:NPcomplete}
    The constrained existence problem for XRSEs is $\NP$-complete and is $\NP$-hard even when all players are pessimistic and all rewards are non-negative.
\end{theorem}

First, we prove \cref{thm:memorysmall}, which shows that if there is an XRSE then there is one that uses finite memory. We show that this finite-memory strategy profile can be described only using polynomial size, which in turn proves $\NP$ membership (\cref{lemma:np_easy}).

Later, we consider the problem of XRSEs when the players are restricted to pure, stationary or positional strategies. We show that in all the above cases, the problem remains $\NP$-complete. The upperbound is similar to the general case, but the lower bound is shown in \cref{lemma:np_hardness} by showing a reduction from $\THREESAT$ to the constrained existence problem. 
\begin{restatable}{theorem}{restrictedStrategies}\label{thm:infinite_rho_restricted_strategy_np_easy}
    The constrained existence problem of XRSEs is also $\NP$-complete when the players are restricted to positional, stationary,  or pure strategies. 
\end{restatable}

Finally, we show in the following theorem that when all players are optimistic, the problem becomes $\PTIME$-complete. 
\begin{restatable}{theorem}{PTIMEcompleteThm}\label{thm:PTIMEcomplete}
    The constrained existence problem of XRSE is $\PTIME$-complete when all players are optimists, that is, when $P=\emptyset$.
\end{restatable} %, by giving an algorithm that incrementally removes edges of the underl until an XRSE is found
We dedicate the rest of the section to proving these three results. 
\subsection{Membership in $\NP$}
%\thejaswini{We need to discuss some examples to make the proof less opaque. To discuss and put good examples.}
%\paragraph*{Intuition}
$\NP$-membership is a consequence of the fact that when an XRSE exists, there also exists one with the same extreme risk measures that uses finite memory, with a number of states that is polynomial in the size of the game.
Let us therefore illustrate, with examples, how and why memory is required in such XRSEs.
We consider the following constrained existence question and analyse the same question on three example graphs. 
\begin{quote}$(*)$
   Is there an XRSE in the game in which both players have a risk measure $1$?
\end{quote}

\subparagraph*{Game in \cref{fig:ex_extreme1}.}Let us consider the game in \cref{fig:ex_extreme1} again.
The answer to Question~$(*)$ here is \emph{no}.
Intuitively, such an XRSE would require at least two plays of positive probability: one that ends in $t_1$, and one that ends in $t_2$.
%We say that the first one \emph{anchors} player $\Circle$'s payoff, and the second one anchors player $\Square$'s one.
For those two plays to occur with positive probability, the strategy profile must proceed to a randomised action at vertex $a$ or $b$: i.e., one of the players, at some point of time, must toss a coin to give the payoff $1$ to player $\Circle$ in one case and to $\Square$ in the other.
But then, since that player is the only one that can see that coin, they have a profitable deviation by lying about the outcome, and always choose the option that gives them the best payoff.
More randomisation will not help: as long as one of the players randomises, be it once, several times, or infinitely often, they have an incentive to deviate and stay in the cycle and wait until the other player leaves.

\subparagraph*{Game in \cref{fig:ex_extreme2}.} Consider now a slight modification, as shown in~\cref{fig:ex_extreme2}.
There, the first player that plays is determined at random by the edge that is taken from an initial stochastic vertex.
The answer to Question $(*)$ for this game is \emph{yes}. 
The random choice on which player gets payoff $1$ is decided by the stochastic vertex. Since both players can see which edge is taken from there, this serves as a source of unbiased randomness based on which they act. % their future action on the outcome of that random experiment.
For example, it can be decided that if the play visits the vertex $a$ immediately after $c$, then player $\Circle$ must visit the terminal $t_1$, and similarly, if it visits the vertex $b$, then player $\Square$ must visit the terminal $t_2$. If the edge $ab$ is taken, player $\Square$ punishes player $\Circle$ by always going back to $a$, and vice versa.
In other words, the stochastic vertex provides the players with a common coin.
%Note that we could also have replaced $c$ by a vertex controlled by a third player, for whom $t_1$ and $t_2$ would be equivalent.

\subparagraph*{Game in \cref{fig:ex_extreme3}.} Finally, consider the game depicted in~\cref{fig:ex_extreme3}.
Here, both players $\Circle$ and $\Square$ have the possibility of deviating to a terminal with payoff $2$ in one play.
The stationary strategy profile in which from vertex $d$, player $\Circle$ goes from $d$ to $a$ and then to $t_1$, and in which player $\Square$ goes from $e$ to $b$ and then to $t_2$, is therefore not an XRSE: both players have a profitable deviation that goes to terminal $t_3$.
But the answer to Question~$(*)$ still remains \emph{yes}!
If, from vertex $d$, player $\Circle$ goes from vertex $d$ to $a$ and then to $b$, from which player $\Square$ leaves to $t_2$, and symmetrically, from vertex $e$, player $\Square$ goes to $a$ through $b$ from which player $\Circle$ goes to $t_1$, then that strategy profile is an XRSE, in which everyone gets the risk measure $1$. 
This is because a player has a profitable deviation only if they can play in a way that guarantees them a risk measure better than $1$, i.e., that guarantees them \emph{almost surely} a payoff greater than~$1$ by deviating.
If there remains a play that occurs with nonzero probability and offers a lower reward, then the player does not increase their risk measure.  
Therefore, an XRSE where player $\Circle$ gets the extreme risk measure $1$ only needs to have one play with positive probability in which she gets the payoff $1$, \emph{and} in which she cannot increase her payoff by deviating. We say that such a play \emph{anchors} that player. In our example, the play $cebat_1$ anchors player $\Circle$. 

We see in this last example that memory is required to remember either the subset of players that are being anchored, or if a player has deviated from the strategy and must be punished. 
Given one or more players that are being anchored, the memory state of any of the players does not change unless either a player deviates or, more importantly, randomisation occurs. When randomisation occurs, the set of players that are anchored in each of the plays is a subset of the set of players anchored before this play \emph{split}.
In our examples, the set of players that are anchored at $c$ is both $\Circle$ and $\Square$, and it immediately splits.
After the splits, when we have only one player to anchor, the players can follow a positional strategy profile; and similarly when one player deviates and must be punished the players can follow a positional strategy profile.
%\theju{to do: i will add some lines about anchoring needing only positional strategies. Currently, anchoring is not a objective. it is a property of a play}%Since only a positional strategy is required to ensure a positive probability of a specific payoff, 
%In general, to ensure that a player receives a specific payoff is achieved using a positional strategy profile.%, therefore if the memory-state is anchoring, then the players may follow a positional strategy-profile until the memory state %Anchoring a subset of players or punishing a subset of players requires only a positional strategy.  

% Moreover, the same idea can be used to prove that such a limited amount of memory is also sufficient when we restrict our work to pure strategy profiles: in that case, the players are no longer allowed to proceed themselves to a randomisation that would induce a split between several anchoring plays, and can only use stochastic vertices to do so, as it was actually the case in our examples.

We prove a theorem that bounds the amount of memory required by a strategy to a polynomial in the number of players and vertices in the game. %We further show that the same idea can be used to prove that such a limited amount of memory is also sufficient when we restrict our work to pure strategy profiles.
\begin{restatable}[App.~\ref{app:memorysmall}]{theorem}{memorysmall}\label{thm:memorysmall}
    Let $\bsigma$ be an XRSE in the game $\Game_{\|v_0}$ with $n$ vertices and $p$ players,  and a partition $(P, O)$ of player $\Pi$.
    Then, there exists a finite-memory XRSE $\bsigma^\star$ with at most $3np-2n+p+1$ many memory states, and such that $\X(\bsigma^\star) = \X(\bsigma)$. Furthermore, if $\bsigma$ is pure, then there is such a strategy profile $\bsigma^\star$ that is pure.
\end{restatable}

\begin{proof}[Proof sketch]
We prove this theorem by formalising the idea of \emph{anchoring plays}.
To do so, we define a labelling function $\Lambda$, that maps each history $h$ compatible with $\bsigma$ to the set of players that is, after the history $h$, currently being \emph{anchored}. In \cref{lm:Lambda}, we show the existence of such a labelling, with some properties. 
In the sequel, we write $\bz$ to represented the risk measure of each player in the strategy profile $\bsigma$, that is, $\bz = (z_i)_i = \X(\bsigma)$.

     \begin{restatable}[The labelling $\Lambda$, App.~\ref{app:memorysmall}]{lemma}{finiteMemAbstraction}\label{lm:Lambda}
        There exists a labelling $\Lambda$ that maps each history $h \in \Hist\Game_{\|v_0}$ compatible with $\bsigma$ to a set $\Lambda(h) \subseteq \Pi$, such that for each such $h$, if we write $\{v_1, \dots, v_k\} = \Supp(\bsigma(h))$, the labelling $\Lambda$ satisfies the following properties.
        \begin{enumerate}
            \item\label{itm:splitsetsanchorwithouti} If the vertex $\last(h)$ is stochastic, or belongs to some player $i \not\in \Lambda(h)$, then the sets $\Lambda(hv_1), \dots, \Lambda(hv_k)$ form a partition of $\Lambda(h)$.

            \item\label{itm:splitsetsanchorwithi} If the vertex $\last(h)$ belongs to some player $i \in \Lambda(h)$, then the sets $\Lambda(hv_1) \setminus \{i\}, \dots, \Lambda(hv_k) \setminus \{i\}$ along with $\{i\}$ form a partition of $\Lambda(h) \setminus \{i\}$, and $i$ belongs to all sets $\Lambda(hv_1), \dots, \Lambda(hv_k)$.

            \item\label{itm:optimistanchor} For each optimistic player $i \in \Lambda(h)$, we have $\X_i(\bsigma_{\|h}) = z_i$.
            
            \item\label{itm:pessimistanchor} For each pessimistic $i \in \Lambda(h)$, for all strategies $\tau_i$ of player $i$, we have $\X_i(\bsigma_{-i\|h}, \tau_i) \leq z_i$.

            \item\label{itm:nosplit} If there is a successor $v_\l$ such that $\Lambda(hv_\l) = \Lambda(h)$, then all other successors $v_{\l'}$ are such that $\X_i(\bsigma_{\|hv_{\l'}}) < z_i$ for each optimist $i \in \Lambda(h)$, and there exists $\tau_i$ with $\X_i(\bsigma_{-i\|hv_{\l'}}, \tau_i) > z_i$ for each pessimist $i \in \Lambda(h)$.
        \end{enumerate}
    \end{restatable}

        
    %     \begin{enumerate}
    %     \item $\Lambda(h) = \bigcup_{v \in \Supp (\bsigma(h))} \Lambda(hv)$;~\label{itm:partitionanchor}
    %     \item if there is more than one vertex $v\in \Supp (\bsigma(h)) = \set{v_1,\dots,v_\l} = \Supp(\bsigma(h))$
    %     %and $\Lambda(h)$ that has more than two vertices,
    %     and if  player $i$ controls the vertex $\last(h)$
    %     \begin{itemize}
    %         \item where $i\in \Lambda(h)$, then player $i$ is in all sets $\Lambda(hv_1),\Lambda(hv_u), \dots,\Lambda(hv_\l)$, and the pairwise intersection of any two of the sets is exactly $\{i\}$; 
    %         \item  where $i\notin \Lambda(h)$ (or if it is a stochastic vertex) , then the sets $\Lambda(hv_1),\Lambda(hv_u), \dots,\Lambda(hv_\l)$ form a partition of $\Lambda(h)$. 
    %     \end{itemize}\label{itm:splitsetsanchor}
    %     \item if $i$ is an optimist, then  $\xr(\bsigma_{\|hv})[\mu_i] = z_i$;~\label{itm:optimistanchor}\theju{check if this notation is defined}

    %     \item if $i$ is a pessimist, then for every strategy $\tau_i$ for player $i$, we have $\xr(\bsigma_{-i\|hv}, \tau_i)[\mu_i] \leq z_i$.~\label{itm:pessimistanchor}
    % \end{enumerate}
    % \end{restatable}
%     \begin{proof}[Proof sketch]
%     \leonard{Do we still need this proof sketch? Maybe the examples above are sufficient now?}
%     Let $z_i$ represent the perceived reward $\xr(\bsigma)[\mu_i]$. 
%     Once we extract such a $\Lambda$ from the proposition above, we say that a subset $P$ of players is $\Lambda$-compatible if there is a history $h$ with $P = \Lambda(h)$. 
    
%     We define a strategy profile $\bsigma^\star$ by providing memory structure $\Mc = (M, \nu, \Bar{m}_\init)$.

%     We define here set $M$ to consist exactly of the state $\punish_i$ for each player $i$, for each vertex $v$ and each subset $P \subseteq \Pi$ that is $\Lambda$-compatible, the state $\anchor_{Pv}$, and also the state $\anchor_{\emptyset v}$ and finally the state $\anchor_{\Pi\bot}$. 

%     The initial memory states are exactly those of the form $\anchor_{\Pi v_0}$. 
    
%     When in the memory state of the form $\anchor_{i}v$ for singleton sets $\{i\}$ is intuitively a positional strategy that ensures player $i$ exactly the risk measure $z_i$. 
%     Similarly, when the memory state is of the form $\punish_i$, then the memory structure is the strategies of all players, that ensures player $i$ receives the worst value of perceived reward from every vertex. We remark that these punishing strategy idea was first defined first for repeated games~\cite{Aum85}.
%     We say \emph{$P$ splits at $v$} if there is a history  $hv$, we had $\Lambda(hv) = P$, and there is some successor $w_i$ such that $\Lambda(hvw_i) = P_i\neq P$. This implies that the original strategy $\sigma$ proposed a probability distribution over vertices $w_1,\dots,w_\ell$, that is,  $\Supp(\sigma(hv)) = w_1,\dots,w_\ell$. If $P$ splits at $v$, then we describe the intuition behind the memory state  $\anchor_{P v}$. Here, the strategy  is a positional strategy that reaches vertex $v$ and then randomises at $v$ between the states $w_1,\dots,w_\ell$, and switches to the memory state to a different memory state $\anchor_{P_iu}$, where $w_i$ is the result of the randomisation and $P_i$ splits at vertex $u$. 
%     If not, then we follow a positional strategy that gives all the players $i$ in $P$ the reward $z_i$, and we know one such exists from the definition of $\Lambda$.

%     Finally, we show that the strategy profile $\bsigma^\star$ constructed has the same perceived reward $\bsigma$ for each player as in the strategy profile in \cref{prop:ActualPayoff}. We also show that $\bsigma^\star$ is indeed a risk-sensitive equilibrium, where no player has an incentive to deviate in \cref{prop:Nodeviation}. 

%     To show that the strategy profile is small, we argue that the partial order structure induced by the subset of players that is $\Lambda$-compatible form the nodes of a split-DAG, and we prove a small combinatorial lemma that bounds the size of a split-DAG 
%     \end{proof}
% \begin{restatable}[A split directed acyclic graph]{definition}{splitDAG}
%     A graph $\Dc$ is a \emph{split directed acyclic graph} or a split-DAG of a finite set $S$, if the vertices of the DAG are substes of $S$, that is, $V(\Dc)\subseteq 2^S$, and
%     \begin{enumerate}
%         \item    the only source vertex of $\Dc$ is $S$, and all vertices of the DAG are reachable from $S$;\label{itm:splitDAGreachable}
%         \item    for any edge $(X, Y)$, then $X\supseteq Y$;\label{itm:splitDAGsubset}
%         % \item   if there are two distinct edges $(X,Y_1)$ and $(X, Y_2)$, then $X\supsetneq Y_1$ or $X\supsetneq Y_2$. 
%         \item   the union of all subsets outgoing from $X$  is $X$, that is, $\cup_{(X,Y_i)}{Y_i} = X$ \label{itm:splitDAGunion}
%         \item   for each vertex labelled by $X$, either $Y_i$s form a partition of $X$ or there is one element $s\in X$ such that for any pair of edges $(X,Y_i)$, $(X,Y_j)$ from $X$, we have $Y_i\cap Y_j = \{s\}$.\label{itm:splitDAGpartition} 
%     \end{enumerate}
% \end{restatable}

% \leonard{This notion and this lemma seem very abstract and technical to me.
% Since $\Lambda$ is already defined here, can't we simply count the number of sets $P$ such that there exist $h$ with $\Lambda(h) = P$?}

% We can prove this lemma  by a simple induction on the size of $S$.
% \begin{restatable}{lemma}{countingDAG}\label{lemma:counting_subset_dag}
%         Let $S$ be a finite set of size $n$. A split-DAG of $S$ has at most $4n-2$ many vertices.
% \end{restatable}

% The rest of the proof of \cref{thm:memorysmall} proceeds by reconstructing the strategy profile $\bsigma^\star$ from $\Lambda$, with memory states that remember which players are currently being anchored, and what was the last vertex seen, so that we can detect deviations.
% When a deviation is detected, the players switch to a punishing memory state, and follow a stationary strategy profile to punish the deviator.
With such a labelling $\Lambda$, we later show that there are at most $3p-2$ subsets $A$ such that $\lambda(h) = A$ for some history $h$ by an inductive argument (\cref{prop:combinatorial} in \cref{app:memorysmall}). 
We use subsets in the range of the function $\Lambda$  to create $3p-2$ memory states for each of the $n$ vertices to remember the anchoring plays at that vertex of the play, including one extra memory-state for the empty subset. In addition, the memory states also include $p$ punishing  strategies, one for each player, adding up to the number $3np-2n+p+1$. 
We construct a strategy $\bsigma^\star$ from $\Lambda$ that uses only these memory states defined above.
\end{proof}

Finally, using \cref{thm:memorysmall}, we can show the following lemma.

\begin{lemma}\label{lemma:np_easy}
    The constrained existence problem of XRSEs is in $\NP$. The same problem when players are restricted to pure strategies is still in $\NP$.
\end{lemma}

\begin{proof}
    Let $\Game_{\|v_0}$ be a simple quantitative stochastic game.
    Let $(P,O)$ be a partition of $\Pi$, and let $\bx$ and $\by$ be threshold vectors.
    By \cref{thm:memorysmall}, if there exists a (pure) XRSE with $\bx \leq \X(\bsigma) \leq \by$, then there exists one with at most $3np-2n+p+1$ memory states, where $p$ is the number of players and $n$ is the number of vertices.
    Such a strategy profile can be guessed in polynomial time.
    
    We now show that, once such a finite-memory strategy profile $\bsigma$ is guessed, one can check in polynomial time whether it is an XRSE, and satisfies the constraint $\bx \leq \X(\bsigma) \leq \by$.
    
    \begin{itemize}
        \item First, given $\bsigma$, for each player $i$, the quantity $\xr_i(\bsigma)$ can be computed in polynomial time, since it reduces to computing player $i$'s risk measure in the Markov chain induced by $\bsigma$ (which has polynomial size) (\cref{lm:secretlemma} in App.~\ref{appendix:secretlemma}).

        \item Second, checking that $\bx \leq \xr(\bsigma) \leq \by$ can be done in polynomial time.

        \item Third, for each player $i$, one must check that player $i$ has no profitable deviation.
        This can also be done in polynomial time (\cref{lm:secretlemma} in App.~\ref{appendix:secretlemma}) by computing the best risk measure player $i$ can get in the MDP induced by $\bsigma_{-i}$ (which has polynomial size).\qedhere \qedhere
    \end{itemize}
\end{proof}

\subsection{Restrictions on strategies}
We now consider subcases where the space of a strategies is restricted. 
%We will prove hardness for all those results later.
We show in \cref{thm:infinite_rho_restricted_strategy_np_easy} that restricting the memory or amount of randomness of the strategy still renders the problem only in $\NP$.
Later in this section, we prove that all these problems, including the general problem, are $\NP$-hard. This subsection therefore completes the proof of \cref{thm:NPcomplete,thm:infinite_rho_restricted_strategy_np_easy}.

We restrict the set of strategies of each player to stationary, positional or pure. 
We show that the problem is in $\NP$ for each of these cases.
%We show that even in such cases finding an RSE is $\NP$-complete. 
\begin{lemma}\label{lm:restrictionsNPeasy}
    The constrained existence problem, when all the players are restricted to positional, stationary, or pure strategies, is in $\NP$. 
\end{lemma}
\begin{proof}
    We show that we can still guess a strategy profile, and verify in polynomial time if it is indeed an XRSE.
    For the cases of positional and stationary strategies, guessing a strategy profile is straightforward, since such a strategy profile $\bsigma$ can be represented using polynomially many bits.%, since one only needs to guess the set of edges from each vertex that are being used with nonzero probability.
    We can  then verify that a given strategy profile $\bsigma$ gives risk measures within the constraints, and also is an XRSE in polynomial time (\cref{lm:secretlemma} in \cref{appendix:secretlemma}). 

    However, for pure strategies, memory might be required. But we showed with \cref{thm:memorysmall} that if there is a pure strategy profile, then there is one that requires polynomial memory, and therefore our results follow.  
\end{proof}
We now prove $\NP$-hardness of the constrained existence problem for the general setting as well as the cases where the players are restricted. 

\begin{restatable}[App.~\ref{app:np_hardness}]{lemma}{NPHard}\label{lemma:np_hardness}
    The constrained existence problem of XRSEs
    is $\NP$-hard, even when all players are pessimists and all rewards are non-negative.
    It remains $\NP$-hard when the strategies are reduced to stationary, pure, or positional ones.
\end{restatable}

\begin{proof}[Proof sketch]
     We reduce instances of $\THREESAT$ to a an instance of the problem. From a given formula $\Phi$, we construct  a game $\Game_\Phi$
    with no optimist and $4n+m+1$ pessimists, where $n$ is the number of literals and $m$ the number of clauses in $\Phi$.
    That game will contain an XRSE where a witness player gets risk measure $2$ if and only if $\Phi$ is satisfiable.
\end{proof}
%    We note that in our reduction, we can assume that all players are pessimistic.
    % , and ask if there is a constraint existence problem that asks if there is an XRSE where player $\diamond$'s perceived risk is exactly $2$. 
 
%     % Each satisfying assignment can be converted into an RSE by taking the positional stationary strategy where for a satisfied literal $\ell$, the vertex owned by $\Square\ell$ is never visited. 
% \end{proof}

This lemma, along with \cref{lemma:np_easy}, proves \cref{thm:NPcomplete}; and along with \cref{lm:restrictionsNPeasy}, it proves \cref{thm:infinite_rho_restricted_strategy_np_easy}.




\subsection{Things get easier when everyone is optimistic}
Since our $\NP$-hardness results involved only pessimistic players, we now show that
the constrained existence problem of XRSEs becomes $\PTIME$-complete when the perceived reward of each player is computed based on the risk measure $\oexp$, thus proving \cref{thm:PTIMEcomplete}.
%In this scenario, we show that the problem is $\PTIME$-complete. 
We first show an upperbound by giving a polynomial-time algorithm.  

\begin{restatable}[App.~\ref{app:ptimeupperbound}]{lemma}{ptimeupperbound}\label{lm:ptimeupperbound}
    If all players are optimists, then the constrained existence problem for XRSE is in $\PTIME$, and there is an algorithm for the decision problem, which runs in time $\Oh(pm^2)$, where $m$ is the number of edges in $\Game$ and $p$ the number of players.
    Moreover, the algorithm can be modified to output an XRSE that satisfies the constraints, if one exists in time $\Oh(pm^2 + m^3)$.  Moreover, there is an algorithm that runs in time $\Oh(pm^2)$ if the upper bounds $y_i \geq 0$ for all players $i$.
\end{restatable}

\begin{proof}[Proof Sketch.]
    We want to decide whether there exists an XRSE $\bsigma$ satisfying the constraints $\Bar{x}\leq \X(\bsigma) \leq \Bar{y}$.
    The algorithm considers and deals with two cases, that we call \emph{cycle-friendly} and \emph{cycle-averse} cases, separately.
    In the cycle-friendly case, we have $y_i \geq 0$ for all players~$i$.
    Then, an XRSE could have positive probability of reaching no terminal vertex.
    However, in the cycle-averse case, that is impossible, since there is a player $i$ such that $y_i < 0$.
    In this proof sketch, we describe only the algorithm in the cycle-friendly case.

    The algorithm constructs a decreasing sequence of sets of edges $E_0, E_1, \dots$ until it reaches a fixed point.
    For each set $E_k$, it considers the strategy profile $\bsigma^{E_k}$, defined as follows: from each non-stochastic vertex $v$, when $v$ is seen for the first time, it randomises uniformly between all edges $vw \in E_k$.
    Later, if $v$ is visited again, it always repeats the same choice.
    If some player $i$ deviates and takes an edge that they are not supposed to take, then all the players switch to a positional strategy profile designed to minimise their risk measure.
    Such a strategy profile is finite-memory, but requires $2^{|V|}|V| + p$ memory states to be represented as a memory structure: we therefore use the set $E_k$ as a succinct representation.

    At each iteration $k$, the algorithm identifies new sets of vertices $V_\bad^k$ that must be avoided. This includes the terminals that give some player $i$ a payoff that is larger than $y_i$, or vertices from which player $i$ can deviate and obtain a higher value than the value offered by the strategy profile $\X_i(\bsigma^{E_k})$. 
    If it is not possible to avoid reaching the set $V_\bad^k$, the answer $\No$ is returned.
    Otherwise, the set $E_{k+1}$ is defined from $E_k$ by removing 
    edges that ensure that $V_\bad^k$ is not reached with positive probability.
    The algorithm stops when there are no more edges to remove and answers $\Yes$ and if we have $\X_i(\bsigma^{E_k}) \geq x_i$ for each $i$, and $\No$ otherwise.

    Each iteration requires time $\Oh(mp)$ to identify and remove edges.
    Since there are $\Oh(m)$ many edges, the algorithm terminates in time $\Oh(pm^2)$.
    % Each step $k$ consists in identifying a new set of vertices $V_\bad^k$ that must be avoided.
    % At step $k=0$, it is the set of terminal vertices that give to some player $i$ a payoff that is larger than $y_i$, which would then make them have an off-constraints risk measure.
    % At step $k \geq 1$, it is the set of vertices $v$ whose adversarial value $\val(v)$ is greater than the risk measure $\X_i(\bsigma^{E_k})$, where $i$ is the player controlling $v$.
    % In other words, the vertices from which that player can have a profitable deviation.
    % Then, the algorithm computes the set $A_k$ of vertices from which, whatever the players play (using only edges of $E_k$), they have a positive probability of visiting $V_\bad^k$.
    % If $k \geq 1$ and $v_0 \in A_k$, i.e., if it is not possible to avoid reaching the set $V_\bad^k$, the answer $\No$ is returned.
    % Otherwise, the set $E_{k+1}$ is defined from $E_k$ by removing all the edges that lead from a vertex that does not belong to $A_k$ to a vertex that does, thus making sure that $V_\bad^k$ will never be reached.
    % The algorithm stops when there is no more edge to remove.
    % Then, the algorithm answers $\Yes$ and outputs the set $E_k$, as a succinct representation of the strategy profile $\bsigma^{k+1}$, if we have $\X_i(\bsigma^{E_k}) \geq x_i$ for each $i$, and answers $\No$ otherwise.
\end{proof}

Finally, we show that the problem is $\PTIME$-hard, even when there are only two players.

\begin{restatable}[App.~\ref{app:ptimelowerbound}]{lemma}{ptimelowerbound}\label{lm:ptimelowerbound}
    The constrained existence problem of XRSEs with optimistic players is $\PTIME$-hard even with only two players.
\end{restatable}

\begin{proof}[Proof sketch]
    We give a log-space reduction from the problem of deciding two-player zero-sum reachability games, which is known to be $\PTIME$-complete~\cite[Proposition~6]{Imm81}.
\end{proof}

%\section{Restrictions on players and strategies}\label{sec:Polynomial}
%% Finally, we consider two further restrictions on the problem to show the tractability border of this problem, and if there are cases where the problem can be solved more efficiently, 
% We show in \cref{thm:infinite_rho_restricted_strategy_np_easy} that restricting the memory or amount of randomness of the strategy still renders the problem $\NP$-complete. 

% Based on observation of the $\NP$-hardness proof requiring only pessimistic players, we show that if instead all the players are optimists, then the problem becomes solvable in polynomial time. 

% \subsection{Restrictions on strategies} 
% First, we consider restrictions on the kinds of strategies used. That is, if the strategy are memoryless, positional or pure. We show that even in such cases finding an RSE is $\NP$-complete. 
% \begin{theorem}\label{thm:infinite_rho_restricted_strategy_np_easy}
%     The constrained existence problem of $(P,O)$-RSEs is $\NP$-complete
%     when the number of players is fixed, even when the players are restricted to positional, memoryless,  or pure strategies. 
% \end{theorem}
% \begin{proof}
%     The lower bound for all three of the above cases follow from \cref{lemma:np_hardness}, with the observation that, in the reduction, if an RSE that satisfies the constraints, then the strategies of the players are positional strategies, and therefore are both pure and memoryless too. 

%     For the upper bound, we show that we can still guess a strategy and verify in polynomial time if it is indeed a strategy. For the cases of positional and memoryless strategies, guessing a strategy is straightforward. Whereas for pure strategies, this requires some work. 
%     \paragraph*{Positional strategies and memoryless strategies} The size of such a strategy $\sigma$ that is an RSE can be represented using polynomially many bits, since one only needs to guess the set of edges from each vertex that are being used with non-zero probability. From \cref{lm:mpd_ptime}, we can then verify if the given $\sigma$ indeed gives a simple quantitative payoff within the constraints, and also if it is indeed an RSE in polynomial time. 

%     \paragraph*{Pure strategies}
%     \thejaswini{Need to write. Maybe proof sketch is enough. }
%     For pure strategies, the strategies might require more memory to represent. We argue therefore that if there is a winning strategy, there is one that requires only 
% \end{proof}

% \subsection{Everyone is optimistic}
% We consider the case where the perceived reward of each player is computed based on the risk-measure $\pexp$ for optimistic players. In this scenario, we show that the problem is $\PTIME$-complete. 
% \begin{theorem}
%     The constrained existence problem of $(P,O)$-RSE is $\PTIME$-complete where all players are optimists, that is, $P=\emptyset$.
% \end{theorem}
% \begin{restatable}{lemma}{ptimeupperbound}
%     The constrained existence problem for $(P,O)$-RSE is in $\PTIME$ if all players are optimists.    
%     The running time of such an algorithm is at most $\Oh()$ where $n$ is the number of vertices in the graph and $p$ is the number of players.\theju{To compute and write here}
%     Moreover, this polynomial time decision algorithm can be modified to output a succinct representation of an RSE satisfying the constraints, when it exists, in $\Oh()$\theju{Copy}.
% \end{restatable}
% \begin{proof}[Proof Sketch.]
    
% \end{proof}

% Finally, we show that the problem is also $\PTIME$-hard even for the case where there are two players. 
% \begin{restatable}{lemma}{ptimelowerbound}\label{lemma:PTIMEHard}
%     The constrained existence problem for two players RSE is $\PTIME$-hard when both players are optimists.
% \end{restatable}
% We show $\PTIME$-hardness for the problem in the case by giving a log-space reduction from the two player zero sum reachability game. Deciding the winner in two player reachability games is $\PTIME$-complete, with hardness proved for the same problem of alternating graph reachability~\cite[Proposition~6]{Imm81}.
\section{Discussion}
In this work, we explore the database drafting approaches in speculative decoding, which do not require additional training or fine-tuning. Existing methods rely on a single database from a single source, resulting in inconsistent and suboptimal acceleration gains. To address this, we propose Hierarchical Drafting (HD), which optimally utilizes diverse sources by constructing multiple databases based on temporal locality. Our method hierarchically accesses these databases, prioritizing those with the highest locality for optimal acceleration. Experimental results show that HD consistently and effectively accelerates LLM inference across various scenarios, outperforming other database drafting methods. These findings demonstrate that our hierarchical framework maximizes the strengths of each database with minimal overhead, expanding the directions exploiting multiple databases for lossless acceleration in speculative decoding.
%% Bibliography
%%

%% Please use bibtex, 

\bibliography{biblio}

\appendix
\section{Appendix for \cref{sec:ERM}}\label{appendix:ERM}
% \ERzeroExp*
% \begin{proof}[Proof of \cref{lemma:ERzeroExp}]
% We have:
%         $$\re_{\beta,\rho}= -\frac{1}{\rho} \log_\beta \left( \int_{x \in \Rb} \beta^{-\rho x} \d \prob(X = x) \right).$$
%         When $\rho$ tends to $0$, the function $x \mapsto \beta^{-\rho x}$, for $x$ ranging in $\Rb$ and $\rho$ in any neighborhood of $0$, converges dominatedly to $x \mapsto 1$.
%         Therefore, the argument of the logarithm above converges to $1$, and according to the usual equivalence $\ln(1+t) \sim_{t \to 0} t$, that risk entropy has the same limit, if it has one, as:
%         $$-\frac{1}{\rho\ln(\beta)} \left( \int_{x \in \Rb} \beta^{-\rho x} \d \prob(X = x) - 1 \right) = \int_{x \in \Rb} \frac{1 - \beta^{-\rho x}}{\rho\ln(\beta)}  \d \prob(X = x).$$
%         Now, when $\rho$ tends to $0$, the function $x \mapsto \frac{1 - \beta^{-\rho x}}{\rho\ln(\beta)}$ also converges dominatedly to $x \mapsto x$, using the usual equivalence $e^t - 1 \sim_{t \to 0} t$.
%         Hence the risk entropy converges to the quantity:
%         $$\int_{x \in \Rb} x  \d \prob(X = x) = \Eb(X).$$
% \end{proof}

\subsection{Proof of \cref{lemma:RSEtoQSSG}}\label{app:RSEtoQSSG}

\RSEtoQSSG*
\begin{proof}[Proof of \cref{lemma:RSEtoQSSG}]
    Consider the simple stochastic game $\Game_{\|v_0} = \tpl{V,E,\Pi,(V_i)_{i\in \Pi},\p ,\mu}$. We will define a payoff  function $\mu'$ over the same set of terminals for game $\Game'_{v_0}$ such that $\Game'_{\|v_0} = \tpl{V,E,\Pi,(V_i)_{i\in \Pi},\p ,\mu'}$ has a Nash equilibrium if and only if $\Game$ has a $(\beta,\brho)$-ERSE.

    For a terminal vertex $t$, we simply define $\mu_i'(t) = 1-(\beta^{-\rho_i\mu_i(t)})$ if $\rho>0$, and $\mu_i'(t) = (\beta^{-\rho_i\mu_i(t)})-1$ if $\rho<0$.
    
    Consider the function $\modifiedreward{\beta}{\rho}\colon x\mapsto 1-(\beta^{-\rho x})$ if 
    $\rho>0$ and $x\mapsto (\beta^{-\rho x})-1$ if $\rho<0$
    as the modified reward function.  This function is similar to the negative utility function defined in the work of Baier et al.,~\cite{BCMP24}, where they replace terminal rewards with the negative value of $(\beta^{-\rho\mu_i(t)})$ (as they assume $\rho>0$), in order to compute the winner in a two-player zero-sum game with risk-averse players. We additionally add or subtract $1$ from their value  to ensure that besides monotonicity, this function also maps the play that does not reach a terminal in the original game to the payoff $0$, and therefore in the modified game to preserve that such plays are still mapped to $0$. 
    
    Observe that for any random variable $X$ and constant $r$, for a value $\rho>0$, we have that 
    \begin{equation}\label{inequality:REvsExp}    
    \re_{\beta,\rho}\left[X\right] \geq r\text{ if and only if }\Eb\left[\modifiedreward{\beta}{\rho}(X)\right] \geq 1-\beta^{-\rho r}\end{equation}
    since $\re_{\beta,\rho}\left[X\right] = \frac{-1}{\rho}\log_\beta\tpl{\Eb[\beta^{-\rho X}]}$.
    
    Therefore, any Nash equilibrium in the game $\Game'_{\|v_0}$ implies that there is a strategy profile $\bsigma$ such that, for all players $i\in\Pi$, in the MDP induced by $\bsigma_{-i}$, the strategy $\sigma_i$ of player $i$ is an optimal strategy. 

    We first consider the case of player $i$ where $\rho_i>0$. The case of $\rho_i<0$ is analogous, so we omit it. 
    For every strategy $\tau_i$ of player $i$, where $\rho_i>0$, if we write $\btau = (\bsigma_{-i}, \tau_i)$, we have 
    $\Eb(\bsigma)[\mu_i']\geq \Eb(\btau)[\mu_i']$, since $\bsigma$ is a Nash equilibrium. 
    Since the payoffs of $\Gc'$ at a terminal $t$ is just $\modifiedreward{\beta}{\rho}(\mu_i(v))$, we therefore have $\Eb(\bsigma)[\modifiedreward{\beta}{\rho_i}(\mu_i)]\geq \Eb(\btau)[\modifiedreward{\beta}{\rho_i}(\mu_i)]$. From  \cref{inequality:REvsExp}, we have 
    \begin{align*}
         \Eb(\bsigma)\left[\modifiedreward{\beta}{\rho_i}(\mu_i)\right]\geq \Eb(\btau)\left[\modifiedreward{\beta}{\rho_i}(\mu_i)\right]\text{ if and only if }\\
          \Eb(\bsigma)\left[1-\beta^{-\rho_i \mu_i}\right]\geq \Eb(\btau)\left[1-\beta^{\rho_i \mu_i}\right] \\
           \iff\Eb(\bsigma)\left[-\beta^{-\rho_i \mu_i}\right]\geq \Eb(\btau)\left[-\beta^{-\rho_i \mu_i}\right]
    \end{align*}
Taking $\frac{1}{\rho}\log_\beta$ on both sides,   we get the above is true iff
    \begin{align*}
         \re_{\beta,\rho_i}(\bsigma)[\mu_i] \geq -\frac{1}{\rho_i}\log_\beta\tpl{-\Eb\left[\modifiedreward{\beta}{\rho_i}(\btau)[\mu_i]\right] }\\ \text{if and only if }\re_{\beta,\rho_i}[\bsigma](\mu_i) \geq \re_{\beta,\rho_i}[\btau](\mu_i)     \end{align*}
      Therefore the strategy  $\bsigma$ is at least as good as (any strategy where one player deviates) $\btau$ for the player $i$ that deviates, when their rewards are the risk-entropy measure. Thus, the strategy profile $\bsigma$ is an ERSE.
    
%     then consider the probability distribution $\prob_{\btau}$ over plays. Define a random variable $X'$ where the value of the play in $\Game'$ sampled according to the probability distribution $\prob_{\btau}$. \theju{Need to check this again.}
%     Similarly, let the random variable $Y'$ be the value of a play in game $\Game'$ where the play is sampled according to the probability distribution $\prob_{\bsigma}$. Since $\bsigma$ is a Nash equilibrium, we have that 
%      $\Eb\left[Y'\right]\geq \Eb\left[X'\right]$.
%      Since the payoffs in $\Game'$ is just $\modifiedreward{\beta}{\rho}(\mu_i(v))$, this implies that $\Eb\left[\modifiedreward{\beta}{\rho}(Y)\right]\geq \Eb\left[\modifiedreward{\beta}{\rho}(X)\right]$, where 
%      $X$ and $Y$ are random variables where $X$ and $Y$ are both the value of the play in the original game $\Game$, however the plays sampled for  $X$, and $Y$ are according to the probability distribution $\prob_{\btau}$ and $\prob_{\bsigma}$, respectively. 
%      From the statement (\ref{inequality:REvsExp}), we have     
%      \begin{align*}
%          \Eb\left[\modifiedreward{\beta}{\rho}(Y)\right]\geq \Eb\left[\modifiedreward{\beta}{\rho}(X)\right]\text{ if and only if }\\
%           \Eb\left[1-\beta^{\rho Y}\right]\geq \Eb\left[1-\beta^{\rho X}\right] \\
%            \iff\Eb\left[-\beta^{\rho Y}\right]\geq \Eb\left[-\beta^{\rho X}\right]
%     \end{align*}
% Taking $\log_\beta$ on both sides and multiplying with $-1/\rho$, we get the above is true iff
%     \begin{align*}
%          \re_{\beta,\rho}[Y'] \geq -\frac{1}{\rho}\log_\beta\tpl{-\Eb\left[\modifiedreward{\beta}{\rho}(X)\right] }\\ \text{if and only if }\re_{\beta,\rho}[Y'] \geq \re_{\beta,\rho}[X']
%      \end{align*}
%       Therefore the strategy  $\bsigma$ is at least as good as (any strategy where one player deviates) $\bsigma'$ for the player $i$ that deviates, when their rewards are the risk-entropy measure. Thus, this makes $\sigma$ an risk-sensitive equilibrium. %\theju{badly written, need to revise. Will do it tomorrow.}
\end{proof}


\subsection{Proof of \cref{thm:ERRSErestricted}}\label{app:ERRSErestricted}


\stationaryRSE*
\begin{proof}[Proof of \cref{thm:ERRSErestricted}]
For the proof of \cref{thm:ERRSErestricted}, we first focus on \cref{itm:ERRSEitmundec}.
\begin{proposition}
    The constrained existence problem of $(\beta,\brho)$-ERSEs in quantitative stochastic games where players are restricted to pure strategies is undecidable.
\end{proposition}
\begin{claimproof}
The undecidability of the case where pure strategies are considered is inherited from Nash equilibria~\cite[Theorem~4.9]{UW11}, since the reduction for undecidability uses only pure strategies. Therefore, our current undecidability follows from \cref{proposition:Undecidable,lemma:RSEtoQSSG}.  
%We remark that the case of two player zero-sum games in the work of Baier et al.~\cite{BCMP24} is a specific sub-case of the multi-player setting with positional strategies we consider. Since deciding such a game is the same as finding $\tpl{\beta,\brho}$-ERSE, where the underlying game is a two player game, and the players have zero-sum rewards and $\brho = (\rho,-\rho)$.
\end{claimproof}


\paragraph*{Decidable subcases.} Now, we turn our attention to \cref{itm:ERRSEdecidable} to show decidability and conditional decidability results. 
\subparagraph*{Decidability lowerbounds.} The problem is $\NP$-hard in general for even two players, which follows from  the work of Ummels and Wojtczak.
Since Nash equilibria are a specific instance of the setting of ERSEs, where the risk parameters of each player is $0$, $\NP$-hardness follows~\cite[Theorem 4.4]{UW11}. 
Similarly, for stationary strategy profiles, they show $\SQRTSUM$-hardness~\cite[Theorem 4.6]{UW11} for Nash equilibria which also shows a lower bound for our case. 

\subparagraph*{Decidability upperbounds.} For decidability, we show in \cref{prop:ETRformula} that we can write a formula in the existential theory of reals (with exponentiation if $\beta=e$)
which is satisfied if and only if the constrained existence problem is satisfied. This gives the $\PSPACE$ upper bound when $\beta$ is algebraic since the existential theory of reals is in $\PSPACE$~\cite{Can88}, and decidability subject to Shanuel's conjecture when the base is $\beta = e$, since the existential theory of reals with exponentiation is decidable assuming Shanuel's conjecture~\cite{MW95}.
Our proof is similar to the one by Ummels and Wojtczak~\cite[Theorem 4.5]{UW11}, however, we need to do slightly more work to encode the payoff expressed by the entropic risk measure. %We provide the proof here for completeness. 



We only have to show that we can encode the constrained existence problem using the existential theory of reals. 
First, observe that it is enough to verify if there is a memoryless Nash equilibrium in the modified game obtained where all the terminal rewards $\mu_i(v)$ are replaced instead with $1-\tpl{\beta^{\rho\mu_i(v)}}$. This follows from \cref{lemma:RSEtoQSSG}. 

Since the players are restricted to strategies that are memoryless, we give a non-deterministic algorithm that uses the solution to sentences in $\exists\Rb$ if the values of $\beta$ and $\rho$s can be expressed in $\Qb$.  Since $\NPSPACE = \PSPACE$, and a non-deterministic procedure is still a deterministic procedure, this does not change the complexity. %If $\beta = e$, then we compute the satisfiability of such existential first order formulas expressed in $\exp\text{-}\exists\Rb$ (the existential theory of reals with exponentiation), subjec. We use solutions to such solutions as a black-box. %For the rational case, since $\exists\Rb$ is contained in $\PSPACE$ and further $\NPSPACE = \PSPACE$, this gives us a $\PSPACE$ algorithm~\cite{Can88} for the case where $\beta$ and $\rho$ are rationals. 
     
     For a game $\Game_{\|v_0} = \tpl{V,E,\Pi,(v_i)_{i\in \Pi}, \p,\mu}$, where $\mu$ is the payoff function from a terminal set of nodes $T$ to values in $\Qb$, and constraints $\Bar{x}$ and $\Bar{y}$ for each of the $n$ players in $\Pi$, our algorithm guesses, first, the support $S \subseteq E$ of the strategies that will be considered; that is, the set of edges that will be used with positive probability. %Henceforth, we assume that the mixed strategy profile therefore uses all outoging edges with positive probability. 

         %\subparagraph*{Writing values of terminals efficiently}
\begin{claim}
            For any $z$, which requires $\ell$ bits to encode, there is a formula in $\exists\Rb$ that uses only polynomially many variables in $\ell$ to encode $\modifiedreward{\beta}{\rho}(z) = 1-\beta^{-\rho z}$, where $\beta$ and $\rho$ can also be represented in $\exists \Rb$ using a polynomial formula.  If $\beta = e$, then $\modifiedreward{e}{\rho}(z) = 1-\beta^{-\rho z}$ can be expressed using the existential theory of reals with exponentiation using a formula of most polynomial length.
\end{claim}
\begin{claimproof}
            We assume without loss of generality that $\beta$ is a natural number. If $\beta$ is rational instead, and is represented by a value $a/b$, then individually find $a_1  = a^{\rho z}$ and $b_1 = b^{\rho z}$, and just find $b_1/a_1$, which is just written in $\exists\Rb$ by stating that $\exists  t_r \;\exists a_1'\colon a_1'\times a_1  = 1\land t_r = a_1'\times b_1$.
        Now that we assume that $\beta$ is a natural number, we deal with fixed finite exponentiation with rational values. Similarly, we can assume without loss of generality that $\rho z$ is a natural number. Indeed, any value $r^{a/b} = r^a\times z^{1/b}$ and $z^{1/b}$ can be written as $\exists y \colon y^b = z$.

        It suffices to show therefore that for two values $b,a$, both natural numbers, $b^a$ can be expressed in $\exists\Rb$ succinctly, using only a formula that has length that is not more a poly-log of $b$ or $a$. 
        Let $a = \sum_{i=0}^{\log_2{a}}a_i 2^i$, where $a_i\in\{0,1\}$. 
        This follows from the following observations. 
        \begin{itemize}
            \item $b^a = \prod_{i=1}^{\log_2{a}}\tpl{b^{a_i}b^{2^i}}$
            \item  $2^i$ can be expressed in a formula with at most $i+1$ many variables
            \item $b^{2^i}$ further requires at most $i$ many variables to express, because if $b_i$ represents $b^{2^{i}}$, then we have $b^{2^{i+1}} = b^{2^i}\times b^{2^i}$.
            \item Finally, using a similar trick, $b$ itself can be represented using at most $\log_2{b}$-many variables. 
        \end{itemize}
        If $\beta = e$, it naturally follows that $e^{-\rho z}$ is expressed using exponentiation with $e$. 
\end{claimproof}
        % For each value $z = \mu_i(t)$, which requires $\ell$ bits to encode, we ensured that there is a formula that uses only polynomially many variables in $\ell$ to encode $\modifiedreward{\beta}{\rho}(z) = -\beta^{-\rho z}$. 

    The above claim ensures we can efficiently represent the variables used for the payoffs of the modified game, we now can write an equation assuming that all terminal rewards are available to us as constants. 
    This will write the equation in three parts. Since we have guessed the support, we first ensure that, in fact, there are variables corresponding to the probabilities of the strategy that only take positive values on the edges corresponding to the support set that we guessed.  
    Then, we write equations using variables that compute the values of the induced Markov chain from this strategies. Finally, we also have a formula whose solution corresponds to the values of the MDP obtained for each player when playing against the strategies of all other players. Then we compare if the value of the MDP is at least as large as the underlying Markov chain for each player, to ensure that it is indeed an equilibrium. To write all of this in $\exists\Rb$, we introduce the following variables. 
     \begin{itemize}
         \item one variable  $p_{vw}$ for each pair of vertices $vw$, which corresponds to the probabilities corresponding to the strategy;
         \item a variable $r^i_v$ which corresponds to the entropic risk measure of player $i$ from vertex $v$ if they followed the strategy defined by the probabilities above; 
         \item a variable $m^i_v$ which corresponds to the value obtained by player $i$ if the game is treated as an MDP against other players.
     \end{itemize}
 
     \begin{proposition}\label{prop:ETRformula}
        For the constrained existance of ERSE in a game $\Game_{\|v_0}$ with constraints $\Bar{x},\Bar{y}$, and a subset $S$ of edges of the game, 
         \begin{itemize}
             \item there is a formula  in $\exists\Rb$ that is satisfied if and only if there is a stationary strategy that uses exactly edges in $S$, when $\beta$ and $\rho_i$ are rational values.
             \item there is a formula $\Gamma_S'$ in $\exists\Rb\text{-}\exp$ that is satisfied if and only if there is a stationary strategy that uses exactly the edges in $S$ when $\beta = e$ and $\rho_i$ are rational values.
         \end{itemize}
    \end{proposition}
    \begin{claimproof}
    The following part of the proof is similar to the one found in Ummels and Wojtczak~\cite[Theorem 4.5]{UW11}, but we provide it to suit our setting, for the sake of completeness. 
    First, we have a formula that states that the values $p_{vw}$ indeed describe a strategy. We further ensure that for stochastic vertices, the value $p_{vw}$ encodes exactly the value dictated by the probability function $\p$ by the stochastic vertex:
    \begin{align*}
    \Phi_S(\Bar{p})\:= \bigwedge_{v,w\in V} \left( p_{vw}\geq 0\right)\land \bigwedge_{v,w\in V} \left( p_{vw} \leq 1\right)
    \land \bigwedge_{i\in \Pi}\bigwedge_{v\in V_i} \tpl{\sum_{w\in E(v)}\p_{vw}=1}\land \\
    \bigwedge_{v\in V_?} \left( p_{vw} =  \p(vw)\right) \land \bigwedge_{vw\in S} \left(p_{vw}> 0 \right)
    \end{align*}
    For a fixed support $S$ of a strategy $\bsigma$, it is possible to compute the terminals $T_S$ that have non-zero probability of being reached in the underlying Markov chain that is formed, and the vertices $V_S$ from which such terminals can be reached with non-zero probability. We assign value $\mu_i'(v)$ as the reward for the terminal vertices for player $i$, and the reward $0$ for all vertices that cannot reach any terminal with positive probability. % (since $\modifiedreward{\beta}{\rho}(0) = 1$).
    \[\Omega_S^i(\Bar{p},\Bar{r}^i)\:= \bigwedge_{t\in T_S} \left(r^i_t =\mu_i'(t)\right) \land
                 \bigwedge_{v\notin V_S} \left( r^i_v = 1\right) \land
                 \bigwedge_{v\in V_S\setminus T_S} \tpl{r^i_v = \sum_{w\in E(v)}p_{vw} r^i_v}
                 \]     
    Finally, for computing the values of the MDP, we construct a similar FO statement
    \[\Psi_S^i(\Bar{p},\Bar{m}^i) \:= \bigwedge_{t\in T} \left( m^i =\mu_i'(t) \right) \land
                 \bigwedge_{v\in V_i, w\in E(v)} \left(m_v^i\geq m_w^i\right) \land
                 \bigwedge_{v\notin V\setminus V_i} \tpl{m^i_w = \sum_{w\in E(v)} p_{vw} m^i_v}\]
Finally our statement would be $$\exists \Bar{p}\:\exists\Bar{r}\:\exists\Bar{m}\colon \Phi(\Bar{p})\land \bigwedge_{i\in\Pi}\left(\tpl{x^i_{v_0}\leq r^i_{v_0}}\land \tpl{r^i_{v_0}\leq y^i_{v_0}}\land\Omega_S^i(\Bar{p},\Bar{r^i})\land \Psi_S^i(\Bar{p},\Bar{m^i})\land \left( m_{v_0}^i\leq r_{v_0}^i \right) \right)$$
For rewards that are represented by algebraic numbers that also can be expressed succinctly via $\exists\Rb$, we observe that our above reduction extends naturally. 
For $\beta = e$, we remark that the same formula is expressible using $\exists\Rb\text{-}\exp$.
    \end{claimproof}

\end{proof}

\section{A technical lemma}\label{appendix:secretlemma}
\begin{lemma}\label{lm:secretlemma}
    Let $\Game_{\|v_0}$ be a game with two players, called $i$ and $j$.
    We assume given a partition $(P, O)$ of $\{i, j\}$.
    Then, the quantity:
    $$\inf_{\sigma_j \in \Strat_j\Game_{\|v_0}} \sup_{\sigma_i \in \Strat_i\Game_{\|v_0}} \X_i(\sigma_i, \sigma_j)$$
    can be computed in time $\Oh(m)$, where $m$ is the number of edges in $\Game$.
    Moreover, the infimum is reached with a positional strategy of player $j$; and there is a positional strategy of player $i$ that realises the supremum for every strategy of player $j$.

    Consequently, the optimality of positional strategies and the $\Oh(m)$ upper bound also hold in Markov decision processes, and in Markov chains; and, on the other hand, it holds when $j$ is a fictional player that represents a coalition of players who all have as unique objective to minimise player $i$'s risk measure.
\end{lemma}

\begin{proof}
    The $\Oh(m)$ upper bound holds by a slight adaptation of the classical attractor algorithm~\cite[Chapter~5.3]{AG11}.
    Note that those algorithms run in time $\Oh(m+n)$, where $n$ is the number of vertices; but here, we assumed that each vertex (except possibly $v_0$) has at least one ingoing edge, hence $n \leq m+1$ and $m+n = \Oh(m)$.
    That algorithm immediately induces positional optimal strategies.
    Another way to obtain that second result, however, is the following: once the quantity $x = \inf_{\sigma_j} \sup_{\sigma_i} \X_i(\sigma_i, \sigma_j)$ is known, strategies that realise the infimum and the supremum can be seen as optimal strategies in the Boolean zero-sum game in which player $i$ wants with positive probability (if they are optimist) or with probability $1$ (if they are pessimist) to reach the set of terminals yielding them at least payoff $x$ (if $x > 0$) or to avoid the set of terminals yielding them less than payoff $x$ (if $x \leq 0$).
    This is then a reachability game (seen either from player $i$'s of from player $j$'s perspective), and it is well-known~\cite[Chapter~5.3]{AG011} that in such a game, for both players, positional strategies suffice to maximise the probability of winning.
    In particular, if one has a strategy to win that game with positive probability, or with probability $1$, there is also such a strategy that is positional.
\end{proof}
\section{Appendix for \cref{sec:XR}}\label{appendix:XR}
\subsection{Proof of \cref{thm:RE=PEorOE}}\label{app:RE=PEorOE}

\REisPEOE*
\begin{proof}[Proof of \cref{thm:RE=PEorOE}]
    \begin{itemize}

        \item First, let us note that for every $\rho$, we always have $\re_{\beta,\rho}[X] \geq \pexp[X]$.
        Let now $\epsilon > 0$.
        We want to prove that there exists $\rho_0 \in \Rb$ such that for every $\rho \geq \rho_0$, we have $\re_{\beta,\rho}[X] \leq \pexp[X] + \epsilon$.

        Let us first notice that we have:
       \begin{align*}
       \re_{\beta\rho}[X] &= -\frac{1}{\rho} \log_\beta \left( \int_{x \in \Rb} \beta^{-\rho x} \d \prob(X = x) \right)\\
        &= -\frac{1}{\rho} \log_\beta \left( \int_{x \in \Rb} \beta^{-\rho \pexp[X]} \beta^{-\rho (x-\pexp[X])} \d \prob(X = x) \right)\\
        &= \pexp[X] -\frac{1}{\rho} \log_\beta \left( \int_{x \in \Rb} \beta^{-\rho (x-\pexp[X])} \d \prob(X = x) \right)\\
        &= \pexp[X] -\frac{1}{\rho} \log_\beta \Bigg( \int_{x \leq \pexp[X] + \frac{\epsilon}{2}} \beta^{-\rho (x-\pexp[X])} \d \prob(X = x) \\
        &\qquad\qquad\qquad+ \int_{x \geq \pexp[X] + \frac{\epsilon}{2}} \beta^{-\rho (x-\pexp[X])} \d \prob(X = x) \Bigg)\\
         &\leq \pexp[X] - \frac{1}{\rho} \log_\beta \left( \int_{x \leq \pexp[X] + \frac{\epsilon}{2}} \beta^{-\rho \frac{\epsilon}{2}} \d \prob(X = x) + 0 \right)\\
        &= \pexp[X] - \frac{1}{\rho} \log_\beta \left( \prob\left(X \leq \pexp[X] + \frac{\epsilon}{2}\right) \beta^{-\rho \frac{\epsilon}{2}} \right)\\
        &= \pexp[X] - \frac{1}{\rho} \log_\beta \left( \prob\left(X \leq \pexp[X] + \frac{\epsilon}{2}\right)\right) + \frac{\epsilon}{2}.
        \end{align*}

        For $\rho$ large enough, this quantity is indeed smaller than $\pexp[X] + \epsilon$.


        \item Let us first notice that for every $\beta, \rho, X$, we have the following equality $\re_{\beta,\rho}[X] = -\re_{\beta(-\rho)}[-X]$.
        Thus, we can apply the previous result, and find:
        \begin{align*}
        \lim_{\rho \to -\infty} \re_{\beta,\rho} [X] 
        &= \lim_{\rho \to -\infty} -\re_{\beta(-\rho)} [-X]
        \\&= -\lim_{\rho \to +\infty} \re_{\beta,\rho} [-X]
        \\& =  - \pexp[-X]
        \\ &= - \inf \{x \in \Rb ~|~ \prob(-X \leq x) > 0\}
        \\ &= - \inf \{x \in \Rb ~|~ \prob(X \geq -x) > 0\}
        \\ &= \sup \{x \in \Rb ~|~ \prob(X \geq x) > 0\}
        \\ &= \oexp[X].
        \end{align*}
        \end{itemize}
\end{proof}

% \begin{remark}[Monotonicity of measurc risk]\theju{CHECK: Do we need this anywhere?}
%     If $\rho \leq \rho'$, then $\re_{\beta,\rho} \geq \re_{\beta \rho'}$.
% \end{remark}

% \begin{proof}
%     This result was proved in a text book on risk theory~\cite{KGD08} in a slightly different context.

%     Let $\rho, \rho' \in \Rb \setminus \{0\}$.
%     Let us assume $0 < \rho < \rho'$.
%     Then, the mapping $x \mapsto x^{\frac{\rho}{\rho'}}$ is concave, hence by Jensen's inequality we can write:
%     $$\Eb\left[ \left(\beta^{-\rho'X}\right)^{\frac{\rho}{\rho'}} \right] \leq \left( \Eb\left[ \beta^{-\rho X} \right] \right)^{\frac{\rho}{\rho'}}.$$
%     Then, by applying the mapping $-\frac{1}{\rho} \log_\beta$, we obtain:
%     $$-\frac{1}{\rho} \log_\beta \Eb\left[ \left(\beta^{-\rho X}\right) \right] \geq -\frac{1}{\rho} \frac{\rho}{\rho'} \log_\beta \Eb\left[ \beta^{-\rho X} \right],$$
%     i.e. $\re_\rho[X] \geq \re_{\rho'}[X]$.

%     The case where $\rho$ and $\rho'$ are negative is analogous, and the result can be generalised to $\Rb$ using the continuity in $0$.
% \end{proof}

\subsection{Proof of \cref{thm:XRSEexists}}\label{app:XRSEexists}

\XRSEexists*

\begin{proof}[Proof of \cref{thm:XRSEexists}]
    Throughout this proof, for a given set of edges $F \subseteq E$, we write $\Game^F$ for the game obtained from $\Game$ by removing all the edges that do not belong to $F$.
    In that game, we define $\bsigma^F$ as the stationary strategy profile that maps each vertex $v$ to some probability distribution whose support is $F(v)$.
    Note that the probabilities do not matter here: we are only interested in the support of the distribution of the strategy profile.

    \paragraph*{Algorithm.} We proceed by presenting the algorithm, \cref{algo:existence}, that takes as an input the game $\Game_{\|v_0}$ and the partition $(P, O)$, and returns a subset $F \subseteq E$ such that, as we will show, the strategy profile $\bsigma^F$ is always an XRSE.
    That algorithm defines a decreasing sequence $E_0, E_1, \dots$ of subsets of $E$, where $E_0 = E$.
    At each step $k$, for each pessimist $i$, it computes the risk measure $z_i^k$ of player $i$ in $\bsigma^{E_k}$, and then the set $W_i^k$ of vertices $v$ such that, from $v$, whatever player $i$ does, that player almost surely gets a payoff smaller than or equal to $z_i^k$.
    If we have $v_0 \in W_i^k$ for each $i$, then the algorithm stops there and returns the set $E_k$ (and we will show below that it means that $\bsigma^{E_k}$ is an XRSE).
    Otherwise, we pick player $i$ such that $v_0 \not\in W_i^k$ (a player who provably has a profitable deviation), and define $E_{k+1}$ by removing all the edges accessible from $v_0$ leading from $V \setminus W_i^k$ to $W_i^k$.


            \begin{algorithm}
            \begin{algorithmic}\caption{Exhibition of one stationary XRSE}\label{algo:existence}
                \Procedure{Existence}{$\Game_{\|v_0}, P, O$}
                    \State $k \gets 0$
                    \State $E_k \gets E$
                    \While{$\top$}
                        \State Compute $A^k = \{v \in V \mid v \text{ is accessible from } v_0 \text{ in } (V, E_k)\}$
                        \ForAll{$i \in P$}
                            \State Compute $z^k_i = \X_i(\bsigma^{E_k})$
                            \State Compute $W^k_i = \{v \in V \mid \forall \tau_i \in \Strat_i \Game^{E_k}_{\|v_0}, \text{we have } \prob_{\bsigma^{E_k}_{-i}, \tau_i}(\mu_i \leq z^k_i) > 0\}$
                        \EndFor
                        \If{$\exists i$ such that $v_0 \not\in W_i^k$}
                            \State Pick one such $i$
                            \State $E_{k+1} \gets E_k \setminus ((A^k \setminus W_i^k) \times W_i^k)$
                            \State $k \gets k+1$
                        \Else
                            \State \Return $E_k$
                        \EndIf
                    \EndWhile
                \EndProcedure
            \end{algorithmic}
        \end{algorithm}

    \paragraph*{Correctness} A first quick invariant that we need to prove is the following one, which will guarantee that the games $\Game^{E_k}$ and the strategies $\bsigma^{E_k}$ are well-defined.

    \begin{invariant}\label{inv:outgoingedges}
        For each $k$, each stochastic vertex $v$, we have $E(v) \subseteq E_k$, and for each non-stochastic vertex $v$, we have $E(v) \cap E_k \neq \emptyset$.
    \end{invariant}
    
\begin{claimproof}[Proof that \cref{algo:existence} satisfies \cref{inv:outgoingedges}]
    The set $E_0 = E$ trivially satisfies the invariant.

    Now, let us assume that $E_k$ satisfies the invariant.
    At step $k$, an edge is removed if and only if goes from a vertex $u \in A^k \setminus W^k_i$ to a vertex $v \in W^k_i$.
    Consider a stochastic vertex $u \in A^k$: if it has an edge that leads to vertex $v \in W^k_i$, then whatever player $i$ plays from $u$, with positive probability, the vertex $v$ is reached; and then, if the other players play the strategy profile $\bsigma^{E_k}$, then with positive probability, player $i$ gets the payoff $z^k_i$ or less.
    Hence $u \in W_i^k$, and the edge $uv$ is not removed, and remains in the set $E_{k+1}$.
    Similarly, if $u$ is not a stochastic vertex, but all its outgoing edges lead to a vertex that belong to $W_i^k$, then $u$ itself belongs to $W_i^k$, hence the outgoing edges of $u$ will not all be removed.
    The invariant is therefore still true at step $k+1$, and by induction, is true for all $k$.
\end{claimproof}
    
Each step of the algorithm is then also properly defined.
Moreover, we have termination.

\begin{proposition}
    \cref{algo:existence} terminates.
\end{proposition}

\begin{claimproof}
    With \cref{inv:outgoingedges}, we now know that \cref{algo:existence} successfully constructs a sequence $E_0, E_1, \dots$ of sets of edges until it stops and returns the last of those sets.
    Termination is an immediate consequence of the fact that this sequence is decreasing.

    Indeed, for each step $k$ at which nothing is returned, there exists a player $i$ with $v_0 \not\in W_i^k$.
    On the other hand, the set $W_i^k$ is necessarily accessible from $v_0$:

    \begin{claim}
        The set $W_i^k$ is nonempty, and accessible from $v_0$ in the graph $(V, E_k)$.
    \end{claim}

    \begin{claimproof}
        If $z^k_i$ is obtained by reaching a terminal vertex $t$, then we have $t \in W_i^k$, and $t$ is accessible from $v_0$.
        If now $z^k_i = 0$ is obtained by reaching no terminal vertex, then when following $\bsigma^k$, with positive probability, no terminal is reached.
        Then, there is in particular a vertex $u$ that has positive probability to be visited infinitely often.
        And when playing $\bsigma^k$ from $u$, the probability that some terminal is ever reached is actually $0$, since if it was some constant $q > 0$, then the probability of visiting $u$ infinitely often would be $\lim_\l (1-q)^\l = 0$.
        In other words, no terminal vertex is accessible from $u$ in $(V, E_k)$, and then, we have $u \in W_i^k$.
    \end{claimproof}

    Now, along a play that starts from $v_0 \not\in W_i^k$ and visits $W_i^k$, there exists at least one edge that goes from a vertex that does not belong to $W_i^k$, to a vertex that does.
    Such an edge is then removed in the set $E_{k+1}$, which is therefore strictly included in the set $E_k$.
    This holds for every $k$, ensuring termination.
\end{claimproof}

We now know that the algorithm terminates, i.e., constructs a finite decreasing sequence $E = E_0, E_1, \dots, E_n$, and then returns the set $E_n$, as a succinct representation of the stationary strategy profile $\bsigma^{E_n}$.
What remains to be proven is that that strategy profile is an XRSE.
Before proving that it is an XRSE in the game $\Game_{\|v_0}$, we first prove that it is one in the game $\Game^{E_n}_{\|v_0}$, i.e., when the edges that have been removed cannot be used to deviate.

\begin{proposition}\label{prop:xrseGEn}
    The strategy profile $\bsigma^{E_n}$ is an XRSE in the game $\Game^{E_n}_{\|v_0}$.
\end{proposition}

\begin{claimproof}
    Consider a player $i$, and a deviation $\sigma'_i$ of player $i$ from the strategy profile $\bsigma^{E_n}$ in the game $\Game_{\|v_0}^{E_n}$.
    Let $x = \X_i(\bsigma^{E_n}_{-i}, \sigma'_i)$.

    \subparagraph*{If player $i$ is an optimist.}
    If $x = 0$, then since all rewards are non-negative, we have $x \leq \X_i(\bsigma^{E_n})$.
    If $x > 0$, then the payoff $x$ is obtained by reaching a terminal vertex $t$.
    But then, that terminal vertex is accessible from $v_0$ in the graph $(V, E_n)$, and is therefore also reached with positive probability when all players follow the strategy profile $\bsigma^{E_n}$.
    Hence, again, the inequality $x \leq \X_i(\bsigma^{E_n})$.
    
    \subparagraph*{If player $i$ is a pessimist.}
    Then, since the algorithm terminated at step $n$, player $i$ is such that $v_0 \in W_i^k$.
    The strategy $\sigma'_i$, like every strategy $\tau_i$ for player $i$, satisfies therefore the inequality $\prob_{\bsigma^{E_n}_{-i}, \sigma'_i}(\mu_i \leq z_i^n) > 0$.
    Consequently, we have $x \leq z_i^k = \X_i(\bsigma^{E_n})$.

    In both cases, the deviation $\sigma'_i$ is not profitable, hence the conclusion.
\end{claimproof}


Let us now prove that putting back the removed edges does not change that result, and therefore conclude the correctness proof.

\begin{proposition}
    The strategy profile $\bsigma^{E_n}$ is an XRSE in the game $\Game_{\|v_0}$.
\end{proposition}

\begin{claimproof}
    Let $i$ be a player, and let $\sigma'_i$ be a deviation from $\bsigma^{E_n}$ for player $i$ in $\Game_{\|v_0}$.
    Since $\bsigma^n$ is stationary, we can assume that $\sigma'_i$ is positional by \cref{lm:secretlemma}.
    If the strategy $\sigma'_i$ uses only edges of $E_n$, then it can be considered as a deviation from $\bsigma^{E_n}$ in the game $\Game^{E_n}$, hence by \cref{prop:xrseGEn}, it is not a profitable deviation.
    
    Let us now assume that the strategy $\sigma'_i$ uses an edge that does not belong to $E_n$, i.e. there exists a vertex $v$ that is visited with positive probability in the strategy profile $(\bsigma^{E_n}_{-i}, \sigma'_i)$ and an edge $vw \in E \setminus E_n$ such that $w = \sigma'_i(v)$.
    Since only edges controlled by pessimists have been removed, we can immediately deduce that player $i$ is a pessimist.
    
    Now, among such edges, let us choose one whose removal occurred the earliest, that is, let us choose it in order to minimise the index $k$ such that $vw \in E_k \setminus E_{k+1}$.
    Thus, in the strategy profile $(\bsigma^{E_n}_{-i}, \sigma'_i)$, it is almost sure that only edges of $E_k$ are used.

    The fact that the edge $uv$ has been removed at step $k$ means that we had $u \not\in W_i^k$ and $v \in W_i^k$.
    Thus, from the vertex $v$, if player $i$ uses only edges of $E_k$ (which is the case when they follow $\sigma'_i$), and if the other players follow the strategy profile $\bsigma^{E_k}_{-i}$, player $i$ gets the payoff $z_i^k$ or less with positive probability.
    Let us show that it is also the case when the other players follow the strategy profile $\bsigma^{E_n}$ instead of $\bsigma^{E_k}$.

\begin{claim}
    From the vertex $v$, we have $\prob_{\bsigma^{E_n}_{-i}, \sigma'_i}(\mu_i \leq z_i^k) > 0$.
\end{claim}

\begin{claimproof}
    We proceed by proving that when the players follow, from the vertex $v$, the strategy profile $(\bsigma^{E_k}, \sigma'_i)$, there is a positive probability that player $i$ gets the payoff $z_i^k$ or less \emph{and} that the set $W_i^k$ is never left.

    Indeed, if in that strategy profile there is a positive probability that player $i$ gets a payoff smaller than or equal to $z_i^k$ by reaching a terminal vertex $t$ that yields such a payoff, then we have $t \in W_i^k$, and with positive probability the terminal vertex $t$ is reached without leaving $W_i^k$.
    Similarly, if such a payoff is obtained by reaching no terminal vertex, and therefore getting payoff $0$, then, using a reasoning that has already been used above, with positive probability a vertex $w$ is reached without leaving $W_i^k$, such that from $w$, no terminal vertex is accessible anymore in $(V, E_k)$; and, therefore, all the vertices accessible from $w$ in that graph belong to $W_i^k$, hence once $w$ is reached it is almost sure that $W_i^k$ is never left.

    Then, the set $E_{k+1}$ was defined so that $W_i^k$ is no longer accessible from $v_0$ in the graph $(V, E_{k+1})$.
    Therefore, those vertices are not accessible at any step $\l > k$, and therefore no outgoing edge of a vertex of $W_i^k$ is ever removed in the sequel, i.e. $E_n \cap (W_i^k \times V) = E_k \cap (W_i^k \times V)$.
    Consequently, since $\sigma'_i$ uses only edges of $E_k$, when the strategy profile $(\bsigma^n, \sigma'_i)$ is played from $v$, it is also true that with positive probability player $i$ gets the payoff $z_i^k$, or less.
\end{claimproof}
    
    This claim proves that we have $\X_i(\bsigma^{E_n}_{-i}, \sigma'_i) \leq z_i^k$.
    To conclude that the deviation $\bsigma'_i$ is not profitable, we still need to prove that that quantity $z_i^k$ is smaller than or equal to (actually strictly smaller) the risk measure $\X_i(\bsigma^n)$.
    That inequality is an immediate consequence of the following claim.

\begin{claim}
    For every pessimistic player $j$, the sequence $(z_j^\l)$ of player $j$'s risk measures is non-decreasing.
\end{claim}

\begin{claimproof}
    Let $\l$ be a step index, and let us prove that we have $z_j^\l < z_j^{\l+1}$.
    The quantity $z_j^{\l+1}$ is the pessimistic risk measure of player $j$ in the strategy profile $\bsigma^{E_{\l+1}}$: there is therefore a positive probability that player $j$ gets the payoff $z_j^{\l+1}$ when that strategy profile is followed.
    Player $j$ obtains that payoff either by reaching a terminal vertex to which that payoff is assigned, or by reaching no terminal vertex at all.

    Let us first show that the second case is actually impossible.
    If player $j$ gets the payoff $z_j^{\l+1} = 0$ by reaching no terminal vertex, with the same reasoning as above, there is, in particular, a vertex $v$ that has positive probability to be visited infinitely often when $\bsigma^{\l+1}$ is played from $v_0$, and therefore such that no terminal vertex is accessible from $v$ in $(V, E_{\l+1})$.
    But then, let $j'$ be the player that was controlling the edges that were removed at step $\l$.
    Let us consider a strategy $\tau_{j'}$ of player $j'$ that uses only edges of $E_\l$.
    Then, when the strategy profile $(\bsigma^\l_{-j'}, \tau_{j'})$ is played from the vertex $v$, it will almost surely be true that either no terminal vertex is reached, leading to the payoff $0$, or an edge of $E_\l \setminus E_{\l+1}$ is taken, leading therefore to a vertex of $W_i^\l$, and to a risk measure of $z_{j'}^\l$ or less.
    Thus, since all rewards are non-negative and therefore $z_{j'}^\l \geq 0$, the vertex $v$ belongs to the set $W_{j'}^\l$, which is impossible since it should then have been made unaccessible in the graph $(V, E_{\l+1})$.

    Therefore, player $j$ gets payoff $z_j^{\l+1}$ by reaching a terminal giving them that payoff, which means that such a terminal is accessible from $v_0$ in the graph $(V, E_{\l+1})$.
    Then, it is also accessible from $v_0$ in the graph $(V, E_\l)$, and therefore it is reached with positive probability when following the strategy profile $\bsigma^{E_\l}$.
    Consequently, we have $z_j^\l \leq z_j^{\l+1}$.
\end{claimproof}

    
    Consequently, with $j = i$, we have $z_i^k \leq z_i^n$, and therefore $\X_i(\bsigma^{E_n}_{-i}, \sigma'_i) \leq z_i^k \leq z_i^n = \X_i(\bsigma^{E_n})$.
    The strategy $\sigma'_i$ is not a profitable deviation from the strategy profile $\bsigma^{E_n}$, which is therefore a (stationary) XRSE.
\end{claimproof}


    
    \paragraph*{Complexity}

    We finally show that \cref{algo:existence} runs with time $\Oh(m^2p)$.
    At each iteration of the while loop, at least one edge is removed; we therefore have at most $m$ iterations of the while loop.

    Now, during the $k^\text{th}$ iteration of the loop, the algorithm computes the set $A^k$, and for each pessimistic player $i$, the algorithm also computes the quantity $z_i^k = \X_i(\bsigma^{E_k})$, and then the set $W^k_i$.
    All of those computations can be done in time $\Oh(m)$ using \cref{lm:secretlemma}. Since there are $p$ players, this step therefore takes $\Oh(mp)$ time.
    Finally, checking whether $v_0 \in W^k_i$ for each player $i$ takes time $\Oh(p)$, and removing all the edges leading to $W_i^k$ to define the set $E_{k+1}$ takes time $\Oh(m)$.
    Hence, the complexity of the algorithm is $\Oh(m^2 p)$.
\end{proof}
\section{Appendix for \cref{sec:NPComplete}}\label{appendix:NPComplete}

\subsection{Proof of \cref{thm:memorysmall}}\label{app:memorysmall}

% We first show some simple lemmas that we use during the proof of  \cref{thm:memorysmall}.\theju{to check if this is also needed for later on in PTIME case}
% \begin{lemma} \label{lm:mpd_ptime}
%     Given a simple quantitative MDP $\MDProc$, the quantities $\max_\sigma \pexp_\sigma$ and $\sup_\sigma \oexp_\sigma$ can be computed in polynomial time.
%     As a consequence, the pessimistic and optimistic expectation of a simple quantitative payoff function can also be computed in polynomial time in a Markov chain.
% \end{lemma}

% \begin{proof}
%     \leonard{todo}
% \end{proof}


\memorysmall*

\begin{proof}[Proof of \cref{thm:memorysmall}]
We first define the anchoring set of players $\Lambda$ given a strategy profile~$\bsigma$.
\subparagraph*{Definition of $\Lambda$.}

\finiteMemAbstraction*

\begin{claimproof}
    We define the labelling $\Lambda$ inductively. 

    \subparagraph*{Base case.}
    First, on the one-vertex history $v_0$, we define $\Lambda(v_0) = \Pi$: at the start, all players must be anchored.
    Let us notice that the history $v_0$ satisfies Property~\ref{itm:optimistanchor}, which states that the optimists get the optimistic expectation they are supposed to get, and Property~\ref{itm:pessimistanchor}, which states that the pessimists have no profitable deviations.
    The other properties will be checked in the inductive case.

    \subparagraph*{Inductive case.}
    Suppose $\Lambda(hv)$ has already been defined, where $hv$ is a history compatible with the strategy profile $\bsigma$, and that the five properties are satisfied by $\Lambda$ on all histories on which it is already defined.
    Let $w_1, \dots, w_k$ be the successors of $v$ that are chosen by the strategy $\bsigma(hv)$ with non-zero probability, that is, the support of $\bsigma(hv)$.
    If $k=1$, then we define $\Lambda(hvw_1) = \Lambda(hv)$.
    Note that Properties~\ref{itm:splitsetsanchorwithouti},~\ref{itm:splitsetsanchorwithi}, and~\ref{itm:nosplit} are immediately satisfied, and that Properties~\ref{itm:optimistanchor} and~\ref{itm:pessimistanchor} are satisfied by induction hypothesis.
    
    If $k>1$, we need to partition the set $\Lambda(h)$ between the $k$ successors.
    To do so, we will use the following claim.
    
    \begin{claim}\label{claim:successorAnchor}
    For each player $i \in \Lambda(hv)$ that does not control the vertex $v$, the follwing holds.
    \begin{itemize}
        \item If player $i$ is an optimist, and $\X_i(\bsigma_{\|hv}) = z_i$, then there is at least one successor $w_\l$ such that $\X_i(\bsigma_{\|hvw_\l}) = z_i$.
    
        \item If player $i$ is a pessimist, then there is a successor $w_\ell \in \Supp(\bsigma(hv))$, such that for every  strategy $\tau_i^\ell$ from $w_\l$, we have $\X_i(\bsigma_{-i\|hvw_\ell}, \tau_i) \leq z_i$.
    \end{itemize}
    \end{claim}
    
    \begin{claimproof}
        The first case follows from Property~\ref{itm:optimistanchor} in the induction hypothesis.
    
        As for the second case, we proceed by contradiction.
        Let us assume that for each $w_\l$, there exists a strategy $\tau_i^\l$ such that $\X_i(\bsigma_{-i\|hvw_\l}, \tau^\l_i) > z_i$.
        Then, the strategy $\tau_i$ defined by $\tau_{i\|vw_\l} = \tau^\l$ for each $\l$ is such that $\X_i(\bsigma_{-i\|hv}, \tau_j) > z_i$, which is impossible since $\Lambda(hv)$ is assumed to satisfy Property~\ref{itm:pessimistanchor}.
    \end{claimproof}

    We define each set $\Lambda(hvw_\l)$ by iterating through each element of $\Lambda(hv)$ as follows:
    \begin{itemize}
        \item \textbf{Initialisation.} For all $w_\l$, declare $\Lambda(hvw) = \emptyset$.
        \item \textbf{Iteration over players.} Consider each player $i\in \Lambda(hv)$ sequentially and proceed as follows:
        If $i$ controls the vertex $v$, then add $i$to every set $\Lambda(hvw)$.
        If $i$ does not control $v$, then, by the claim stated earlier, there exists a successor $w_\l$. In this case, add ii to $\Lambda(hvw_\l)$ corresponding to this specific $w_\l)$
    \end{itemize}
    
    Not that the first four properties are thus guaranteed to be satisfied.

    Moreover, when there are several successors $w_\l$ possible, we always favour those such that, at the moment where the decision is taken, the sets $\Lambda(hvw_\l)$ are the smallest.
    This suffices to guarantee Property~\ref{itm:nosplit}.
    \end{claimproof}




\paragraph*{Construction of the strategy profile $\bsigma^\star$}
Based on $\Lambda$ as in \cref{lm:Lambda}, we construct a strategy profile that finite memory states. 
Formally, the definition of the finite-memory strategy $\bsigma^\star$ will be done by defining its memory structure.

The memory states are the following:
    \begin{itemize}    
        \item for each player $i$, the state $\punish_i$;

        \item for each vertex $v$ and each subset $A \subseteq \Pi$ of players such that there exists $h$ with $A = \Lambda(h)$, the state $\anchor_{Av}$;

        \item the state $\anchor_{\Pi\bot}$.
    \end{itemize}
Observe that there are at most $2^p + \Oh(p)$ such memory states.
We now define the transitions from each of those states from each vertex. Observe that long as the memory state does not change, strategy profile corresponds to a positional strategy profile. So, we describe such memoryless strategy profiles and also describe when the memory state changes. 
For each set $A \subseteq \Pi$, we write $W_A$ for the set of vertices that $\bsigma$ may visit while anchoring the set $A$, i.e., the set of vertices $v$ such that there exists a history $hv$ with $\Lambda(hv) = A$.

    \subparagraph*{Punishing memory states $\punish_i$.}
    First, let us define what $\bsigma^\star$ does when in state $\punish_i$, for some player $i$. Those memory states will correspond to the \emph{punishing strategies}, followed when player $i$ deviates from the assigned strategy with the other memory states. 
    By \cref{lm:secretlemma}, there is a positional strategy profile $\btau^{\dag i}_{-i}$ that minimises, from every vertex of the game, the payoff that player $i$ can enforce.
    In addition, we pick an arbitrary positional strategy $\sigma^{\dag i}_i$.
    Then, when the strategy profile $\bsigma^\star$ is in the memory state $\punish_i$ on reads a vertex $v$, the memory update function outputs the vertex $\btau^{\dag i}(v)$ and the same memory state $\punish_i$.

    \subparagraph*{Anchoring states with no player to anchor.}
    Let us now define what happens in memory state $\anchor_{\emptyset v}$.
    Consider the objective of achieving a payoff vector that has positive probability to be achieved in $\bsigma$, while visiting only vertices of $W_\emptyset$.
    Using classical attractor-based proofs (or \cref{lm:secretlemma}), there exists a positional strategy profile that achieves that objective with probability $1$ from every vertex from which that is possible: let us call it $\btau^{\anch \emptyset}$.
    Then, when in memory state $\anchor_{\emptyset v}$ and reading the vertex $w$, we distinguish two cases.
    \begin{itemize}
        \item If $vw$ is an edge that is compatible with the strategy profile $\btau^{\anch \emptyset}$, then the strategy profile $\bsigma^\star$ outputs the vertex $\btau^{\dag i}(w)$, where $i$ is the player controlling $v$, and shifts to the memory state $\punish_i$.

        \item Otherwise, it outputs the vertex $\btau^{\anch \emptyset}(w)$ and moves to the memory state $\anchor_{\emptyset w}$.
    \end{itemize}

    \subparagraph*{Anchoring state with one player to anchor.}
    We can now move to singletons, and define what happens in the states of the form $\anchor_{\{i\} v}$.
    In such a state, we define the strategy based on the that gives player $i$ exactly the extreme risk measure $z_i$.
    More precisely, we want player $i$ to receive payoff $z_i$ with positive probability, and never leave the set of vertices $W_{\{i\}}$ with probability~$1$.
%    of giving, with positive probability, the payoff $z_i$ to player $i$, and maintaining to $0$ the probability of achieving a payoff vector (including $0$ for every player) that has probability zero to be achieved in $\bsigma$, or of visiting a vertex that has probability zero to be visited in $\bsigma$. 
    Using \cref{lm:secretlemma}, there exists a positional strategy profile $\btau^{\anch i}$ that satisfies that property from every vertex from which it is possible.
    Note that that objective is in particular satisfiable, and therefore satisfied by $\btau^{\anch i}$, from every vertex $v \in W_{\{i\}}$.
    Similar to the previous step, we define the strategy profile $\bsigma^\star$ in the states of the form $\anchor_{\{i\} v}$ so that it follows the strategy profile $\btau^{\anch i}$, remembers the last vertex that was visited and uses that memory to switch to the corresponding punishing state when some player $j$ deviates.

        \subparagraph*{Anchoring states with two or more players to anchor.}
    Now, let us consider the states of the form $\anchor_{A v}$, where $A$ has cardinality at least $2$.
    The existence of $\anchor_{A v}$ implies that there is a history $h$ such that $\Lambda(h) = A$.
    Moreover, since each randomisation splits the label of histories in sets that have at most one element in common (Properties~\ref{itm:splitsetsanchorwithouti} and~\ref{itm:splitsetsanchorwithi}), there is only one side of each split that can contain $A$, which implies that among such histories $h$, we can choose one that is a prefix of all others.
    After history $h$, the histories labelled by $A$ form a sequence $h, hv_1, hv_1v_2, \dots$ which may be infinite, end in a terminal vertex, or end with a new split.
    

    \textbf{If that sequence end with a split,} then there is a longest history $hv_1 \dots v_q$ with $\Lambda(hv_1 \dots v_q) = A$ and $k \geq 2$ vertices $w_1, \dots, w_k \in \Supp(\bsigma(hv_1 \dots v_q))$ such that we have $\Lambda(hv_1 \dots v_q w_\l) \neq \emptyset, A$ for each $\l$.
    We can then define a simple history $h'v_q$ that also goes from the vertex $\last(h)$ to the vertex $v_q$, with $\Occ(h'v_q) \subseteq \Occ(\last(h) v_1 \dots v_q)$.
    We then define the strategy profile $\bsigma^\star$ in each state $\anchor_{Av}$ so that it follows the history $h' v_q$ and remembers the last vertex visited, and switches to the state $\punish_i$ and follows the strategy profile $\btau^{\dag i}$ when a given player $i$ deviates and takes an edge that they are not supposed to take.
    Moreover, when an edge is taken that does belong to the history $h'v_q$, but not because a player deviated (it is then necessarily because of a stochastic vertex), the memory switches to the state $\anchor_{\emptyset v}$ (where $v$ is the last vertex seen) and immediately follows the corresponding strategy.
    Finally, when the vertex $v_q$ is reached and the memory is in state $\anchor_{A \last(h')}$, the strategy profile $\bsigma^\star$ chooses randomly between the edges $v_q w_1$, \dots, and $v_q w_k$, all with positive probability.
    Such action will often be referred to as a \emph{split}.


        \textbf{If that sequence is infinite or end in a terminal vertex,} then $\pi^A = v_1 v_2 \dots$ is a play, and satisfies $\Lambda(h\pi^A_{< k}) = A$ for each $k$.
        We can then consider a play $\pi^{A\star}$ with $\Occ(\pi^{A\star}) \subseteq \Occ(\pi^A)$ and $\Inf(\pi^{A\star}) \subseteq \Inf(\pi^A)$ that is either a simple path from $\pi^A_0$ to the terminal reached by $\pi^A$, or, if $\pi^A$ is infinite, a simple lasso (i.e., a play of the form $h'c^\omega$, where the history $h'c$ is simple).
        We can moreover choose $\pi^{A\star}$ so that the set of vertices visited infinitely often (if there are any) in $\pi^{A\star}$ is included in the set of vertices visited infinitely often in $\pi^A$.
Then, we can define $\bsigma^\star$ in the states of the form $\anchor_{A v}$ as following the play $\pi^{A\star}$, and remembering the last vertex seen.
    When a player $i$ deviates and takes an edge that should not be taken, the memory switches to the state $\punish_i$ and follows the strategy profile $\btau^{\dag i}$.
    Finally, when an edge is taken that does not belong to $\pi^{A\star}$ but does not correspond to a deviation either, we switch to the state $\anchor_{\emptyset w}$ where $w$ is the last vertex seen, and to the corresponding strategy profile.



\subparagraph*{Initialisation.}
    The strategy profile $\bsigma^\star$  has the state $\anchor_{\Pi\bot}$ as the initial memory state.
    In this state, it behaves exactly as in any state of the form $\anchor_{\Pi v}$, but without having memorised a last visited vertex $v$, since there is no such vertex.
    From that memory state therefore, it necessarily reads the vertex $v_0$, and starts acting as described in the previous case.


\subparagraph*{The pure case.}
In this construction, the vertices on which the strategy profile $\bsigma^\star$ proceeds to an actual randomisation (i.e., the vertices $v$ such that there exists a history $hv$ such that the support of the distribution $\bsigma^\star(hv)$ contains more than one element) are vertices on which $\bsigma$ also proceeds to such a randomisation.
Therefore, if $\bsigma$ is pure (i.e., if randomisations occur only on stochastic vertices), so is $\bsigma^\star$.





\paragraph*{A combinatorial break: counting states}

    Now that the strategy $\bsigma^\star$ is defined, let us bound the memory it uses.
    There are, obviously, exactly $p$ states of the form $\punish_i$, and one state $\anchor_{\Pi\bot}$.
    To prove that there are at most $3np-2n$ states of the form $\anchor_{Av}$, we need to prove that there are at most $3p-2$ sets $A$ such that there is a history $h$ with $\Lambda(h) = A$.

    Let us call \emph{$\Lambda$-anchored} all such sets $A$.
    By analogy with strategies, we write $\Lambda_{\|hv}$ for the labelling that maps each history $h' \in \Hist\Game_{\|v}$ compatible with $\bsigma_{\|hv}$ to the set $\Lambda(hh')$, and we will also use the notion of anchoredness for each of those labellings $\Lambda_{\|hv}$.
    We proceed by proving the following stronger result.
    
    \begin{proposition}\label{prop:combinatorial}
        For every history $h$ compatible with $\bsigma$, if $\Lambda(h)$ contains at least two elements, then there are at most $3|\Lambda(h)|-2$ sets that are $\Lambda_{\|h}$-anchored.
    \end{proposition}
    


\begin{claimproof}
    For each history $h$, we write $f(h)$ for the number of $\Lambda_{\|h}$-anchored sets that have cardinal at least $2$.
    There are $|\Lambda(h)|+1$ subsets of $\Lambda(h)$ that have cardinality $0$ or $1$: the result will therefore be proved if we prove $f(h) \leq 2|\Lambda(h)| - 3$.
    The proof goes by induction on $m = \Lambda(h) \geq 2$.

    \subparagraph*{Base case.}
    If $m = 2$, the set $\Lambda(h)$ is a pair $\Lambda(h) = \{i, j\}$.
    Then, since the $\Lambda_{\|h}$-anchored sets are all subsets of $\Lambda(h)$, the only set of cardinality at most $2$ that is $\Lambda_{\|h}$-anchored is the pair $\{i, j\}$ itself, hence we have $f(h) = 2 \times 2 - 3 = 1$, as desired.

    \subparagraph*{Inductive case.}
    If $m > 2$, and if we assume that the result is true for every history $h'$ with $2 \leq |\Lambda(h')| \leq m-1$, then let $\{v_1, \dots, v_k\} \subseteq \Supp(\bsigma(h))$ be the set of possible next vertices $v$ such that $|\Lambda(hv)| \geq 2$.

    If $k = 1$, i.e., if $\Lambda(hv_1) = \Lambda(h)$, then we have $f(h) = f(hv_\l)$ and the result for $h$ will be proved if we prove it for $hv_1$.
    Following that reasoning, we can extend the history $h$ until we are not in that case: if we always are, then the only $\Lambda_{\|h}$-anchored sets are $\Lambda(h)$ itself, and possibly the empty sets and some singletons, hence $f(h) = 1$ and the result is immediate.
    We can therefore assume that $k > 1$.

    Let $i$ be the player controlling the vertex $\last(h)$; we set $i = \bot$ if $\last(h)$ is a stochastic node.
    Then, by Properties~\ref{itm:splitsetsanchorwithouti} and~\ref{itm:splitsetsanchorwithi} of \cref{lm:Lambda}  guaranteed during the construction of $\Lambda$, the sets $\Lambda(hv_1) \setminus \{i\}, \dots, \Lambda(hv_k) \setminus \{i\}$ form a partition of $\Lambda(h) \setminus \{i\}$.
    Therefore, no set of cardinality at least $2$ can be simultaneously $\Lambda(hv_\l)$-anchored and $\Lambda(hv_{\l'})$-anchored for $\l \neq \l'$, hence the equality $f(h) = 1+\sum_\l f(hv_\l)$.
    Now, since each set $\Lambda(hv_\l)$ has at least $2$ and less than $m$ elements, we can apply the induction hypothesis to deduce:
    $$f(h) \leq 1 + \sum_{\l=1}^k (2|\Lambda(hv_\l)| - 3).$$
    Moreover, we have $\sum_\l |\Lambda(hv_\l)| \leq m + k - 1$ (each element of $\Lambda(h)$ occurs in one of the sets $\Lambda(hv_\l)$, except possibly one that would occur in all of them), hence the inequality above becomes:
    $$f(h) \leq 1 + 2(m+k-1) -3k$$
    $$= 2m - 1 - k$$
    and since we have assumed $k \geq 2$, we obtain $f(h) \leq 2m-3$.
\end{claimproof}    

Let us recall that $\Lambda(v_0) = \Pi$.
As a particular case of this claim, we obtain, if $p \geq 2$, that there are at most $3p-2$ sets that are $\Lambda$-anchored, as desired.
In the case $p = 1$, the game $\Game_{\|v_0}$ is an MDP, and using \cref{lm:secretlemma}, we can immediately construct $\bsigma^\star$ as a positional strategy (which has therefore $1 \leq 3n \times 1 - 2n + 1 + 1$ memory states) with the same risk measure as $\bsigma$.




    \paragraph*{The strategy profile $\bsigma^\star$ has the desired extreme risk measures.}

We now show that $\X(\bsigma^\star) = \X(\bsigma)$.
Let us recall that we defined $\bz = \X(\bsigma)$.
    
\begin{proposition}\label{prop:ActualPayoff}
    The strategy profile $\bsigma^\star$ satisfies the equality $\X(\bsigma^\star) = \bz$.
\end{proposition}

\begin{claimproof}
    Let $i$ be a player: we want to prove that $\X(\bsigma^\star) = z_i$.
    Let us first see how $z_i$ has positive probability of being obtained in the strategy profile $\bsigma$, and we will then show that no larger (respectively smaller) if $i$ is optimistic (respectively pessimistic) has a positive probability by the same strategy.

   

    \subparagraph*{Player $i$ gets payoff $z_i$ with positive probability.}
    Let $A \subseteq \Pi$ be one of the smallest sets (for the inclusion relation) containing $i$ such that there exists a history $hu$ with $\Lambda(hu) = A$.
    Then, by construction of $\Lambda$, there exists a finite sequence of sets $\Pi = A_0, A_1, \dots, A_m = A$ and of histories $h_1 v_1 w_1, \dots, h_m v_m w_m$ where for each $k$, the history $h_{k+1}$ starts from $w_k$, the history $h_1 v_1 \dots h_k v_k w_k$ is compatible with $\bsigma$, and we have $\Lambda(h_1 v_1 \dots h_k v_k) = A_{k-1}$ and $\Lambda(h_1 v_1 \dots h_k v_k w_k) = A_k$.
    We can then write $hu = h_1 v_1 \dots h_m v_m w_m$.
    
    Consider the strategy profile $\bsigma^\star$, which initially follows  the positional strategy profile $\btau^{\anch \Pi}$. This strategy profile generates, with nonzero probability, a history $h'_1 v_1$ starting from vertex $v_0$ to vertex $v_1$, based on our construction. 
    From that vertex $v_1$, it proceeds to a randomised action and, with positive probability, moves to the vertex $w_1$ and switches to the positional strategy profile $\btau^{\anch A_1}$, and so on: there is, therefore, a history $h'_1 v_1 h'_2 v_2 \dots h'_m v_m w_m$ that is compatible with the strategy profile $\bsigma^\star$ and after which the collective memory is in the state $\anchor_{A v_m}$, and plays accordingly.

    Since $A_m= A$ is the  subset of $\Pi$ where $i\in A = \Lambda(hu)$, we are in the case where the set $A$ is no longer split further by our labelling. That is, there is a play $\pi$ from $w_m$ such that $\Lambda(h \pi_{\leq k}) = A$ for every $k$ such that that is defined.
    Then, in the construction of the strategy profile $\bsigma^\star$ we have distinguished two cases: the one where $A$ was a singleton, and the one where it had at least two elements (the empty case is excluded, since $A$ contains the player $i$).

    \textbf{If $A$ is a singleton,} then after the history $hu$, without any player deviating, all players are following the strategy profile $\btau^{\anch i}$.
    By its definition, that strategy profile achieves the payoff $z_i$ for player $i$ with positive probability.

    \textbf{If $A$ has at least two elements,} then after that same history, all players are following the play $\pi^{A\star}$, which yields the same payoffs as $\pi^A$.
    However, we must still prove that player $i$ actually gets the payoff $z_i$ in $\pi^A$, and that the play $\pi^{A\star}$ is generated with positive probability (i.e. that it does not cross infinitely many stochastic vertices---which we must first show for $\pi^A$).
We do so in the following claim, which we will use again later.

\begin{claim}\label{claim:piA}
    The play $\pi^A$ is (eventually) generated with positive probability when the players follow the strategy profile $\bsigma$.
    Similarly, the play $\pi^{A\star}$ is generated with positive probability when they follow $\bsigma^\star$.
    Both plays yield to each player $j \in A$ the payoff $z_j$.
\end{claim}

\begin{claimproof}
    % Let us first not that using Property~\ref{itm:nosplit} of \cref{lm:Lambda}, we have that for each $k$, the vertex $\pi^A_{k+1}$ was actually the only element of $\Supp(\bsigma(h\pi^A_{<k}))$ satisfying the required properties in the construction of $\Lambda$.
    Let $j \in A$.
    Let us proceed by case disjunction according to the risk measure used by player $j$.

\emph{If player $j$ is an optimist,} then, by Property~\ref{itm:optimistanchor} of \cref{lm:Lambda}, we have $\X_j(\bsigma_{\|h}) = z_j$, and therefore $\prob_{\bsigma_{\|h}}(\mu_j = z_j) > 0$, i.e., by the law of total probability:
        $$\prob_{\bsigma_{\|hu}}(\pi^A) \prob_{\bsigma_{\|h}}(\mu_j = z_j \mid \pi^A) + \sum_k \sum_{w \in E(\pi_k) \setminus \{\pi_{k+1}\}} \prob_{\bsigma_{\|hu}}(\pi^A_{\leq k} w) \prob_{\bsigma_{\|hu}}(\mu_j = z_j \mid \pi^A_{\leq k}w) > 0.$$
        But using Property~\ref{itm:nosplit}, all the terms of the summation on the right are zero, hence the product $\prob_{\bsigma_{\|h}}(\pi^A) \prob_{\bsigma_{\|h}}(\mu_j = z_j \mid \pi^A) > 0$ is positive, i.e. the play $\pi^A$ has  a positive probability of being generated and $\mu_j(\pi^A) = z_j$.

   \emph{If player $j$ is a pessimist,} then because of Property~\ref{itm:nosplit} again, for every $k \geq 0$ and each $w \in \Supp\left(\bsigma\left(h\pi^A_{\leq k}\right)\right)$, there exists a strategy $\tau_j^{kw}$ such that $\X_j(\bsigma_{-j\|h\pi^A_{\leq k}w}, \tau_j^{kw}) > z_j$.
        By composing all those strategies, we obtain a deviation $\tau_j$ of the strategy $\sigma_{j\|h}$; which, by Property~\ref{itm:pessimistanchor}, satisfies the inequality $\X_j(\bsigma_{-j\|h}, \tau_j) \leq z_j$.
        Therefore, either:
        \begin{itemize}
            \item we have:
            $$\min_k \min_{w \in \Supp\left(\bsigma\left(h\pi^A_{\leq k}\right)\right) \setminus \{\pi^A_{k+1}\}} \X_j(\bsigma_{-j\|h\pi_{\leq k}w}, \tau^{kw}_j) \leq z_j,$$
            which is impossible by definition of the strategies $\tau_j^{kw}$;

            \item or we have $\prob_{\bsigma_{-j\|h}, \tau_j}(\pi^A) = \prob_{\bsigma_{\|h}}(\pi^A) \neq 0$ and $\mu_j(\pi^A) \leq z_j$, and then actually $\mu_j(\pi^A) = z_j$.
        \end{itemize}
        
         


We have thus proven that player $j$ gets the payoff $z_j$ in $\pi^A$, and that the play $\pi^A$ is generated with positive probability in $\bsigma$.
The analogous results about $\pi^{A\star}$ follow using the equalities $\Occ(\pi^{A\star}) = \Occ(\pi^A)$ and $\Inf(\pi^{A\star}) = \Inf(\pi^A)$.
\end{claimproof}

    In those two cases (if $A$ is a singleton or has several elements), we obtain that the strategy profile $\bsigma^\star$ is such that, with some positive probability, player $i$ gets the payoff $z_i$.


    \subparagraph*{Player $i$ gets risk measure $z_i$.}
    We still have to prove that player $i$ has zero probability of getting a lower payoff (if they are a pessimist) or a higher payoff (if they are an optimist).
    To show both cases, we prove the following claim:

    \begin{claim}
        Every payoff vector that has a positive probability of being achieved in the strategy profile $\bsigma^\star$ also has a positive probability of being achieved in the strategy profile $\bsigma$.
    \end{claim}

\begin{claimproof}
    Let $\bz'$ be such a payoff vector.
    Then, there is a history $hw$ compatible with $\bsigma^\star$ and a set $A \subseteq \Pi$ such that, after the history $hw$, the strategy profile $\bsigma^\star$ is in state $\anchor_{A \last(h)}$, and from that point it has a nonzero probability of achieving the payoff vector $\bz'$ while staying in states of the form $\anchor_{A v}$.
    
    \emph{If $A$ is empty}, then the strategy profile $\btau^{\anch \emptyset}$ has been defined as a strategy profile that almost surely generates a payoff vector that is generated with positive probability by $\bsigma$, from every vertex from which that is possible.
    That requirement is satisfiable, and therefore satisfied by $\btau^{\anch \emptyset}$, from the vertex $w$, since that vertex is itself reached with positive probability in the strategy profile $\bsigma$.
    Therefore, the payoff vector $\bz'$ is also achieved with positive probability in $\bsigma$.
    
    \emph{If $A$ is a singleton,} say $A = \{j\}$, then the strategy profile $\btau^{\anch j}$ has been defined so that from every vertex from which that is possible, on the one hand, it generates the payoff $z_j$ with positive probability, and on the other hand, it is almost sure that the payoff vector that will be generated has also positive probability to be generated in $\bsigma$.
    Similarly as above, that requirement is satisfiable from the vertex $w$, since $\bsigma_{\|hw}$ satisfies it.
    Therefore, again, the payoff vector $\bz'$ is also achieved with positive probability in $\bsigma$.
    
    \emph{If $A$ has at least two elements}, then the strategy profile $\bsigma^\star$ stays in states of the form $\anchor_{A v}$ only along one play, namely $\pi^{A\star}$, and that play generates a payoff vector that was also associated with the play $\pi^A$ that by \cref{claim:piA}, has positive probability to be generated in $\bsigma$, hence the same conclusion.
\end{claimproof}

This proves the equality $\X_i(\bsigma^\star) = z_i$.
\end{claimproof}


\paragraph*{The strategy profile $\bsigma^\star$ is an XRSE.}

We have now constructed the finite-memory strategy profile $\bsigma^\star$, showed that it had the expected number of memory states, and that it generates the expected risk measures.
We must now give the final argument for our construction: that strategy profile is also an extreme risk-sensitive equilibrium.
We will prove that result by showing separately that optimists have no profitable deviations, and then that neither do pessimists.

\begin{proposition}\label{prop:NodeviationOpt}
    No optimist has a profitable deviation in $\bsigma^\star$.
\end{proposition}

\begin{claimproof}
    Let $i$ be an optimist, and let us consider a deviation $\sigma'_i$ of that player from $\bsigma^\star$. Let us write $z'$ for the risk measure $z' = \X_i(\bsigma^\star_{-i}, \sigma'_i)$.

    Let us notice that along every play compatible with $\bsigma^\star_{-i}$, the transitions that are possible in the memory structure of the strategy profile $\bsigma^\star$ can be classified as follows:
    \begin{itemize}
        \item transitions among states of the form $\anchor_{A v}$ for a fixed $A$;
        
        \item transitions from a state of the form $\anchor_{A v}$ to a state of the form $\anchor_{B w}$ with $B \subset A$;

        \item transitions from a state of the form $\anchor_{A v}$ to the state $\punish_i$;

        \item and transitions from $\punish_i$ to itself
    \end{itemize}
    Therefore, any such play stabilises either in the state $\punish_i$, or among the states of the form $\anchor_{A v}$ for a fixed set $A$.
    Consequently, if in the strategy profile $(\bsigma^\star_{-i}, \sigma'_i)$ player $i$ gets the payoff $z'$ with positive probability, then we can also say that either:
    \begin{itemize}
        \item with positive probability, player $i$ gets the payoff $z'$ \emph{and} the state $\punish_i$ is reached;

        \item or there exists a set $A \subseteq \Pi$ such that with positive probability, player $i$ gets the payoff $z'$, and the collective memory remains in states of the form $\anchor_{A v}$.
    \end{itemize}

    \emph{In the first case,} let us consider a history $hv$ compatible with $\bsigma^\star_{-i}$ such that the collective memory is in an anchoring state after $h$ and in state $\punish_i$ after $hv$.
    If player $i$ can obtain the risk measure $z'$ by going to $v$ from that vertex against $\bsigma^\star_{-i\|h}$, and therefore, against the punishing strategy profile $\btau^{\dag i}_{-i}$, it means that they can enforce that risk measure against every possible strategy profile from $\last(h)$.
    On the other hand, if the collective memory is in an anchoring state after $h$, it means that the vertex $\last(h)$ is also visited with positive probability in the strategy profile $\bsigma$ (otherwise we would have switched to a punishing state earlier).
    There is therefore a history $h'$ compatible with $\bsigma$ such that $\last(h) = \last(h')$; and after that history, against the strategy profile $\bsigma_{\|h'}$, player $i$ also has the possibility of getting with positive probability the payoff $z'$.
    Since $\bsigma$ is an XRSE, that implies $z' \leq z_i$.

    \emph{In the second case,} let us notice that the strategy profiles of the form $\btau^{\anch A}$ are pure, and therefore that any deviation of player $i$ is immediately detected and leads to a switch to state $\punish_i$.
    Therefore, if the collective memory remains in states of the form $\anchor_{A v}$, it means that player $i$ is actually following the strategy $\sigma^\star_i$.
    Thus, we also have $z' \leq z_i$.
    
    The strategy $\sigma'_i$ is not a profitable deviation from $\bsigma^\star$.
\end{claimproof}

We can now end the proof with the dual proposition.

\begin{proposition}
    No pessimist has a profitable deviation in $\bsigma^\star$.
\end{proposition}

\begin{claimproof}
    Let $i$ be a pessimist, and consider a deviation $\sigma'_i$ of that player from $\bsigma^\star$.
    We intend to prove that the deviation $\sigma'_i$ is not profitable, that is, when following the strategy profile $(\bsigma^\star_{-i}, \sigma'_i)$, there is still a positive probability that player $i$ receives a payoff smaller than or equal to $z_i$.
    Using \cref{lm:secretlemma}, we can assume without loss of generality that $\sigma'_i$ is pure.

    First, we observe that for each history $hv$ compatible with $\bsigma^\star$ such that, after $hv$, the collective memory is in state $\anchor_{A \last(h)}$ with $i \in A$, the vertex $v$ is such that there also exists a history $h'v$ compatible with $\bsigma$ with $\Lambda(h'v) = A$.
    By Property~\ref{itm:pessimistanchor} of \cref{lm:Lambda}, we have $\X_i(\bsigma_{-i\|h'v}, \tau_i) \leq z_i$ for every $\tau_i$.
    Therefore, if player $i$ accepts to follow the history $hv$ and, then, deviates and takes an edge that makes the collective memory switch to the state $\punish_i$, then with positive probability player $i$ gets a payoff lesser than or equal to $z_i$.
    If such an action is ever performed, then the deviation $\sigma'_i$ is not profitable.

    Let us now assume that $\sigma'_i$ performs no such action: after every history $hv \in \Hist_i \Game_{\|v_0}$, if the collective memory is in a state of the form $\anchor_{A \last(h)}$ with $i \in A$, the vertex $\sigma'_i(hv)$ belongs to the set $\Supp(\sigma^\star_i(hv))$.
    Then, by Property~\ref{itm:splitsetsanchorwithi} of \cref{lm:Lambda}, we also have $i \in \Lambda(hv\sigma'_i(hv))$.
    Thus, there still exists a set $A$ with $i \in A$ such that, with positive probability, when following the strategy profile $(\bsigma^\star_{-i}, \sigma'_i)$, the strategy profile $\bsigma^\star_{-i}$ stabilises among memory states of the form $\anchor_{A v}$; and then, the strategy profile $\bsigma^\star$ only proceeds to pure actions, hence the strategy $\sigma'_i$ is actually following $\sigma^\star_i$.

    Using the same arguments as in the proof of \cref{prop:ActualPayoff} (definition of $\btau^{\anch i}$ in case $A = \{i\}$, and \cref{claim:piA} in case $A$ has more elements), we can then conclude that the player $i$ gets the payoff $z_i$ with positive probability and therefore the deviation $\sigma'_i$ is not profitable.
    \end{claimproof}

    The strategy profile $\bsigma^\star$ is an XRSE, satisfies the equality $\X(\bsigma^\star) = \X(\bsigma)$, and uses the desired number of memory states.
    Furthermore, if $\bsigma$ is pure, so is $\bsigma^\star$.
\end{proof}
\subsection{Proof of \cref{lemma:np_hardness}}\label{app:np_hardness}
\NPHard*
\begin{proof}[Proof of \cref{lemma:np_hardness}]
  We prove $\NP$-hardness by reducing from the problem $\THREESAT$. Consider a $\THREESAT$ formula $\Phi$, over the variables  $x_1,\dots,x_n$, where $\Phi = C_1\land C_2\land \dots\land C_m$, where for each $i$ we have $C_i = (\ell_{i1}\lor \ell_{i2}\lor \ell_{i3})$ and for $j=1,2,3$, we have $\ell_{ij} = x_k$ or $\ell_{ij} = \neg x_k$ for some $k\in \{1, \dots, n\}$. 
  We construct  a game $\Game_\Phi$ with two players for each literal $\ell$, denoted by $\Circle \ell$ and $\Square \ell$. The game is depicted in \cref{fig:NPhard}.
  For convenience, some terminal vertices have been represented several times.
  Each player $\Circle \ell$ controls one vertex, the vertex $\Circle \l$, of circled shape, and symmetrically, each player $\Square \ell$ controls the square-shaped vertex $\Square \l$.
  Further, we add a player $C_i$, who controls the vertex $C_i$, for each clause $C_i$. Finally, there is a player $\Diamond$ who does not control any vertex. There are also stochastic vertices, that are represented by the black circles.
  In each terminal vertex, the symbol $\forall$ should be understood as "every (other) player".

%The edges between the players are as in \cref{fig:NPhard}. From each clause $C_i$, we add edges to terminal vertices where the payoff of the player $\Circle \ell$ is $0$ and everyone else's is 2 if and only if $\ell$ is in the clause $C_i$. In the example, we assume $C_2 = x_2\lor x_4\lor \neg x_{11}$ and only draw edges from $C_2$ and not from other $C_i$. 

 We assume all players are pessimistic, and ask if there is an XRSE where player $\Diamond$'s risk measure is exactly $2$.
We give the formal definition of the game $\Game_\Phi$ below.

    \begin{figure}
        \centering
        
        \begin{tikzpicture}[shorten >=1pt, node distance=1.5cm and 2cm, on grid, auto, scale=1.1]
          %every node/.style={scale=0.6}
          % Smaller state style
          \tikzstyle{state}=[circle, draw, minimum size=20pt, inner sep=1pt]
          \tikzstyle{squarestate}=[rectangle, draw, minimum size=20pt, inner sep=1pt] 

            \node[state] (qnc1) at (0, 0) {$\neg x_{1}$};
            \node[state, initial,initial text=] (qc1) at (0, 1) {$x_{1}$};

            \node[scale=0.6] (tpunish1) at (0,-1) {$t_\dag:~\stack{\forall}{0}$};
            
            \node[stoch, scale=0.6] (stoc2) at (1, 0) {$s_{\neg x_1}$};
            \node[stoch, scale=0.6] (stoc1) at (1, 1) {$s_{x_1}$};

            \node[scale=0.6] (reward1) at (1,2) {$f_{x_1}:~\stack{\circ x_1}{1}$$\stack{\forall}{2}$};
            \node[scale=0.6] (reward2) at (1,-1) {$f_{\neg x_1}:~\stack{\circ \neg x_1}{1}$$\stack{\forall}{2}$};
            
            \node[squarestate] (qns1) at (2, 1) {$\neg x_{1}$};
            \node[squarestate] (qs1) at (2, 0) {$x_{1}$};

            \node[scale=0.6] (punishW1) at (2,2) {$t_\diamond:~\stack{\diamond}{0}$$\stack{\forall}{2}$};
            \node[scale=0.6] (punishW2) at (2,-1) {$t_\diamond:~\stack{\diamond}{0}$$\stack{\forall}{2}$};
            
            \node[state] (qnc2) at (3.5, 0) {$\neg x_{2}$};
            \node[state] (qc2) at (3.5, 1) {$x_{2}$};
            \node[stoch, scale=0.6] (stoc3) at (4.5, 1) {$s_{x_2}$};
            \node[stoch, scale=0.6] (stoc4) at (4.5, 0) {$s_{\neg x_2}$};

            \node[scale=0.6] (reward3) at (4.5,2) {$f_{x_2}:~\stack{\circ x_2}{1}$$\stack{\forall}{2}$};
            \node[scale=0.6] (reward4) at (4.5,-1) {$f_{\neg x_2}:~\stack{\circ \neg x_2}{1}$$\stack{\forall}{2}$};
            
            \node[squarestate] (qns2) at (5.5, 1) {$\neg x_{2}$};
            \node[squarestate] (qs2) at (5.5, 0) {$x_{2}$};

            \node[scale=0.6] (punishW3) at (5.5,2) {$t_\diamond:~\stack{\diamond}{0}$$\stack{\forall}{2}$};
            \node[scale=0.6] (punishW4) at (5.5,-1) {$t_\diamond:~\stack{\diamond}{0}$$\stack{\forall}{2}$};

            \node[scale=0.6] (tpunish2) at (3.5,-1) {$t_\dag:~\stack{\forall}{0}$};
            \node (qc3) at (6.5, 1) {};

            \node (dots) at (6.8,0.5) {$\dots$};

            \node[squarestate, initial,initial text=] (qnsn) at (8, 1) {$\neg x_{n}$};
            \node[squarestate, initial,initial text=] (qsn) at (8, 0) {$x_{n}$};

            \node[scale=0.6] (punishW5) at (8,2) {$t_\diamond:~\stack{\diamond}{0}$$\stack{\forall}{2}$};
            \node[scale=0.6] (punishW6) at (8,-1) {$t_\diamond:~\stack{\diamond}{0}$$\stack{\forall}{2}$};
            
            \node[stoch, scale=0.6] (stochfin) at (9,0.5) {$s_\mathsf{r}$};
            \node (fakenode) at (9,0.5) {};
            
            \node (c1) at (10,1.8) {$C_1$};
            \node (c2) at (10,1) {$C_2$};
            \node (cdots) at (10,0.4) {$\vdots$};
            \node (c3) at (10,-0.6) {$C_m$};

            \node[scale=0.6] (ter1) at (11.2, 1.7) {$t_{x_2}:~\stack{\Box x_2}{1}$ $\stack{\forall}{2}$};
            \node[scale=0.6] (ter2) at (11.2, 1) {$t_{x_4}:~\stack{\Box x_4}{1}$ $\stack{\forall}{2}$};
            \node[scale=0.6] (ter3) at (11.2, -0.3) {$t_{\neg x_{11}}:~\stack{\Box\neg x_{11}}{1}$ $\stack{\forall}{2}$};

          \path[->]
              (qc1) edge (stoc1)
              (qc1) edge (qnc1)
              (qnc1) edge (stoc2)
              (stoc1) edge (qns1)
              (stoc2) edge (qs1)
              (qns1) edge (qc2)
              (qs1) edge (qc2)
              (qc2) edge (stoc3)
              (qc2) edge (qnc2)
              (qnc2) edge (stoc4)
              (stoc3) edge (qns2)
              (stoc4) edge (qs2)
              (qns2) edge (qc3)
              (qs2) edge (qc3)
              (qnsn) edge (stochfin)
              (qsn) edge (stochfin)
              (fakenode) edge (c1)
              (fakenode) edge (c2)
              (fakenode) edge (c3)
              (c2) edge (ter1)
              (c2) edge (ter2)
              (c2) edge (ter3)
              (qnc1) edge (tpunish1)
              (qnc2) edge (tpunish2);

        \path[->]
            (stoc1) edge (reward1)
            (stoc2) edge (reward2)
            (stoc3) edge (reward3)
            (stoc4) edge (reward4)
            (qs1) edge (punishW2)
            (qns1) edge (punishW1)
            (qs2) edge (punishW4)
            (qns2) edge (punishW3)
            (qsn) edge (punishW6)
            (qnsn) edge (punishW5);
        
        \end{tikzpicture}
        \caption{Construction of a game $\Game_\Phi$ from a $\THREESAT$ formula $\Phi$}
        \label{fig:NPhard}
    \end{figure}


% We will construct a game $\Game_\Phi$ with $O(n+m)$-players, constraints $\Bar{x},\Bar{y}$ and where at least $2n$ players are pessimists, such that the game has a $\Bar{\rho}$-RSE if and only if $\Game_\Phi$ is satisfiable. 

\subparagraph*{Construction of the game $\Game_\Phi$: vertices, edges and payoffs.}
For each literal $\ell$, we define two players $\Square\ell$ and $\Circle\ell$. We add one other player $C_i$ for each clause $C_i$, and an additional constraining player $\Diamond$.
All players are pessimists.

Each player owns at most one vertex in the game, and therefore, we will refer to the player and vertex interchangeably. There is one vertex for each of the players mentioned above other than $\Diamond$, who owns no vertices. Further, there are $2n + 1$ many stochastic vertices: one for each literal $s_{x_1},s_{x_2},\dots,s_{x_n}$,  $s_{\neg x_1},s_{\neg x_2},\dots,s_{\neg x_n}$, and finally one clause-randomiser $s_\mathsf{r}$. 
There are also $2n + 2$ terminal vertices, written $f_{\ell}$ and $t_{\ell}$ for each literal $\ell$, and further the terminal vertices $t_\Diamond$ and $t_\dag$.

We now define the edges between the vertices of the graph for all $i\in \{1, \dots, n\}$:  there are edges from $\Circle x_i$ to $\Circle\neg x_i$, and edges from $\Circle\neg x_i$ to $t_\dag$.
    Further, for every literal $\ell = x_i$ or $\neg x_i$, there are edges:
    \begin{itemize}
        \item from $\Circle\ell$ to $s_{\ell}$;
        \item from $s_{\ell}$ to $f_\ell$ and to $\Square\Bar{\ell}$, where $\Bar{\ell} = \neg x_i$ if $\ell = x_i$ and $\Bar{\ell} = x_i$ if $\ell = \neg x_i$;
        \item from $\Square\ell$ to $t_\Diamond$;
        \item from $\Square\ell$ to $\Circle x_{i+1}$ if  $i<n$,  and  to $s_\mathsf{r}$ if  $i=n$.%, edges are added
    \end{itemize}
    Finally, for all clauses $C_j$, there are edges from $s_\mathsf{r}$ to $C_j$ and from $C_j$ to $t_{\ell}$ such that $\ell$ occurs positively in the clause $C_j$.

    The terminal vertices yield the following payoffs.
\begin{itemize}
    \item In terminal $t_\ell$, all players get payoff $2$, except the player $\Square \ell$ who gets payoff $1$. 
    \item In terminal $f_\ell$, all players get payoff $2$, except player $\Circle \ell$ who gets payoff $1$.
    \item In terminal $t_\dag$, all players get payoff $0$.
    \item In terminal $t_\Diamond$, all players get payoff $2$, except player $\Diamond$ who gets payoff $0$.% , and player $\Diamond$ get payoff $0$. 
\end{itemize}    

Finally, we let the constraints be that player $\Diamond$ gets a risk measure of exactly $2$.
Equivalently, we define $\bx$ and $\by$ by $\by = (2)_{i \in \Pi}$, $x_i = 0$ for each $i \in \Pi \setminus \{\Diamond\}$, and $x_\Diamond = 2$.

% and $\Square x_i$

% All players in the game are pessimistic players    \theju{Need to detail who needs to be pessimistic.}
    % We draw an example formula with 3 variables and two clauses. 

    % \begin{example}
    %     Consider clause $C_1 = (x_1\lor \neg x_2\lor x_3)$, and $C_1 = (\neg x_1\lor  x_2\lor \neg x_3)$
    %         \thejaswini{To do add an example}
    %         \leonard{Is that really necessary? Your figure above is pretty clear (and clearly pretty).}
    % \end{example}
% We first make a claim whose proof is straight forward and can be shown by the definition of RSE, and using the fact that player $\Square\ell$ is a pessimistic player.

% \begin{claim}
%     If vertex $\Square \ell$ is visited in an RSE, then terminal $t_\ell$ must be visited probability $0$. Similarly, if terminal $t_\ell$ is visited with non-zero probability, then the strategy 
% \end{claim}

    \subparagraph*{If $\Phi$ is satisfiable, then there is an XRSE satisfying the constraints.}
    Consider a satisfying assignment of the $\THREESAT$ formula, described by the assignment $\alpha$ from the set of all variables to $\{\top,\bot\}$.
    
    For each $i$, let $\ell_i$ denote the literal, among $x_i$ and $\neg x_i$, which is set to true by 
    the satisfying assignment $\alpha$.
    Let us define the (positional) strategy profile $\bsigma^\alpha$.
    
    \begin{itemize}
        \item Player $\Circle \ell_i$ goes to $s_{\ell_i}$.
        \item Player $\Circle x_i$ goes to $s_{x_i}$ if $\alpha(x_i) = \top$, and to $\Circle \neg x_i$ otherwise.
        \item Player $\Circle\neg x_i$ goes to $s_{\neg x_i}$ if $\alpha(x_i) = \top$ and to $t_\dag$ otherwise.
        \item For each player $\Square \ell$, the strategy is to chose the edge that does \emph{not} lead to $t_\Diamond$. That is, the edge to $\Circle x_{i+1}$ if $\ell = x_i$ or $\neg x_{i}$ and  $i<n$,  and  the edge to $s_\mathsf{r}$ if  $i=n$.
        \item Each clause player $C_i$ takes the edge to the vertex $\ell_j$ such that the litteral $\ell_j$ was set to true by the satisfying assignment $\alpha$. 
    \end{itemize}
     We now show that this is an XRSE that satisfies the constraint. First, we verify if the constraints are satisfied. Observe that following the strategy profile $\bsigma^\alpha$, it is almost sure that none of the terminals where player $\Diamond$ has payoff less than $2$ will be reached. Therefore this satisfies the constraints. 

     We now argue that $\bsigma^\alpha$ is an XRSE, i.e. that no player can get a better risk measure by deviating.
     The result is immediate for player $\Diamond$ and for the clause players, who all get risk measure $2$, the best they could hope for.
     
     For each literal $\l$, player $\Circle\ell$ gets risk measure $1$ if $\l$ is set to true, and risk measure $2$ if $\ell$ is set to false.
     The same argument as above holds therefore in the second case.
    In the first case, they get risk measure $0$, but they have no profitable deviation, since the only deviation available leads to $t_\dag$ and to the payoff $0$.
     
     Player $\Square \ell$  has also risk measure  $2$ when $\ell$ is set to false.
     Otherwise, they get payoff $1$. In that second case, the vertex owned by the player is not visited in any history of the game, hence they have no possibility of deviating.

     The (positional) strategy profile $\bsigma^\alpha$ is therefore an XRSE.
     
    \subparagraph*{If there is an XRSE satisfying the constraints, then $\Phi$ is satisfiable.} 
    Let us assume that there exists an XRSE $\bsigma$ in the game $\Game_\Phi$, such that player $\Diamond$ gets the risk measure $2$.
    We prove, first, that we can assume that $\bsigma$ is pure (and therefore positional, since there is then only one history leading to each vertex).

\begin{claim}
    There exists an XRSE $\bsigma^\star$ in $\Game_{\|v_0}$ where player $\Diamond$ gets risk measure $2$ that is positional.
\end{claim}

\begin{proof}
    Let us first focus on what happens in vertices that have positive probability of being reached.

    If the vertex $\Circle\neg x_i$ has a positive probability of being reached in $\bsigma$, then any strategy of the player $\Circle \neg x_i$ that goes to $t_\dag$ with positive probability gives the player $\Diamond$ the risk measure $0$.
    Therefore, necessarily, the strategy $\sigma_{\circ \neg x_i}$ consists of deterministically going to $s_{\neg x_i}$.
    The same argument holds for the vertices of the form $\Square \l$.
    
    If now the vertex $\Circle x_i$ has a positive probability of being reached and if the player $\Circle x_i$ randomises between the two edges available, then she gets the risk measure $1$, since the terminal vertex $f_{x_i}$ is reached with positive probability and $t_\dag$ with probability zero.
    But then, if she deviates and goes to the vertex $\Circle \neg x_i$ with probability $1$, she avoids the terminal vertex $f_{x_i}$, and the other players will not react since they do not detect the deviation.
    She therefore gets the risk measure $2$, and the deviation is profitable.
    Consequently, the strategy $\sigma_{\circ x_i}$ can only  deterministically select one of those two edges.

    At the end of the game, for each $j$, the player $C_j$ could play a randomised strategy. In such a case, her strategy can be replaced by a pure strategy that takes, deterministically, one of the edges that she was previously taking.
    Such a modification in her strategy can only increase the risk measure of some players (namely, those of the form $\Square \l$) without impacting player $\Diamond$'s risk measure or giving any player the possibility of profitably deviating.

    Finally, if one of those vertices is reached after a history that is not compatible with $\bsigma$, i.e. if one of those players deviates: it is necessarily due to a deviation of a player of the form $\Circle x_i$, since any other deviation would immediately lead to a terminal vertex.
    If she went to $s_{x_i}$ instead of $\Circle \neg x_i$, what the other players do afterwards does not matter, since such a deviation cannot be profitable: with positive probability, the terminal vertex $f_{x_i}$ is reached, and she gets payoff $1$.
    If she went to $\Circle x_i$ instead of $s_{x_i}$, then we can assume that player $\Circle \neg x_i$'s strategy consists of going to the terminal vertex $t_\dag$, giving her the payoff $0$.
    Those modifications do not impact the fact that $\bsigma$ is an XRSE.
\end{proof}
    % If the vertex $\Circle\neg x_i$ has a positive probability of being reached, then any strategy of player $\Circle \neg x_i$ that goes to $t_\dag$ with positive probability gives player $\Diamond$ the risk measure $0$.
    % Therefore, necessarily, the strategy $\sigma_{\circ \neg x_i}$ consists in deterministically going to $s_{\neg x_i}$.
    % A consequence of that fact is that player $\Circle x_i$, who can get the payoff $0$ only in the terminal vertex $t_\dag$, gets in the strategy profile $\bsigma$ a risk measure greater than or equal to $1$.

    % If now the vertex $\Circle x_i$ has positive probability to be reached, then player $\Circle x_i$ is also necessarily playing a pure strategy: if she randomises between the two actions available, then we argue that there is an undetectable deviation possible at the vertex $\Circle x_i$. Player $\Circle x_i$ would always prefer the edge to $\Circle\neg x_i$ since this removes entirely the path that gives the player $\Circle x_i$ the possibility of payoff $0$.
    
    % Since there is no randomisation possible at vertices $\Circle x_i$, to ensure that player $\Diamond$ gets payoff $2$, the terminals $t_\Diamond$ or $t_\dag$ cannot be visited from any of the states $\Square\ell$.  This means that the strategy from every vertex $\Square \ell$ must be to visit $\Circle x_{i+1}$ or $s_\mathsf{r}$ next. However, observe $t_\Diamond$ ensures a payoff of $2$ to player $\Square\ell$. 
    % To ensure that the strategy is an RSE, and that the player $\Square \ell$ does not have an incentive to deviate to this edge, the outcome of the strategy to player $\Square \ell$ should be $2$ (Any lower risk-expectation of player $\Square\ell$ and would ensure deviation). This is true for any vertex $\Square\ell$ that has a positive probability of being visited by the strategy.    
    % This also rules out the possibility of randomisation at vertices $\Square\ell$ for vertices that are visited with positive probability according to the strategy.
    % Since there is no randomisation at both $\Circle\ell$ as well as $\Square\ell$ vertices. If the stochastic vertex $s_\mathsf{r}$ is reached, then it is reached after visiting a unique path (assuming no deviations in the strategy).
% \begin{claim}
%      The vertex $\Square \ell$ is visited by a strategy $\sigma$ that is an RSE and  satisfies the constraint if and only if terminal $t_\ell$ is be visited probability $0$ by $\sigma$.% Similarly, if terminal $t_\ell$ is visited with non-zero probability, then an RSE that satisfies the constraints does not visit vertex $\Square\ell$ with non-zero probability. 
%  \end{claim}
%    This follows by observing that no literal $\ell$ such that  
 % \begin{claimproof}
 %        Let $\sigma$ denote such an RSE. 
 %     Consider the unique path based on strategy $\sigma$ that is used to reach the vertex $s_\mathsf{r}$. This path visits visits either $\Square x_i$ or $\Square\neg x_i$, based on the structure of the graph. We show that if a literal $\Square \ell$ is visited, then $\ell\notin S$. If $\Square\ell$ is visited, then player $\Square\ell$ has an incentive to deviate and instead chose edge $\Square \ell\rightarrow t_\Diamond$, thus making his risk-expectation $2$, but also reducing the risk expectation of player $\Diamond$.
 % \end{claimproof}   

We therefore assume that $\bsigma$ is positional.
Let us now define the assignment $\alpha$ as follows: for each variable $x_i$ we have $\alpha(x_i) = \top$ if $\sigma_{\circ x_i}(\Circle x_i) = s_{x_i}$, and $\alpha(x_i) = \bot$ if $\sigma_{\circ x_i}(\Circle x_i) = \Circle \neg x_i$.
Let then $C_j$ be a clause, and let us prove that it is satisfied by $\alpha$.
Let $t_\l = \sigma_{C_j}(C_j)$.
Then, the player $\Square \l$ gets risk measure $1$ in the XRSE $\bsigma$.
Consequently, the vertex $\Square \l$ is never reached: otherwise, the only play compatible with $\bsigma$ in which player $\Square \l$ gets payoff $1$ would traverse the vertex $\Square \l$, and player $\Square \l$ would have a profitable deviation by going to the terminal vertex $t_\Diamond$.
If that is the case, then the definition of $\alpha$ given above implies that the literal $\l$ is true.

The assignment $\alpha$ satisfies therefore the formula $\Phi$.

\subparagraph*{Conclusion.}
We have defined an instance of the constrained existence problem of XRSEs from an instance of $\THREESAT$ and proved that one is a positive instance if and only if the other is.
This proves the $\NP$-hardness of the constrained existence problem of XRSEs, since the game $\Game_\Phi$ can clearly be constructed in polynomial time.
Moreover, the game $\Game_\Phi$ is such that if an XRSE where player $\Diamond$ gets risk measure $2$ exists, then there also exists such an equilibrium that it positional, which proves also $\NP$-hardness when the players are restricted to pure, stationary or positional strategies.
\end{proof}

%\section{Appendix for \cref{sec:Polynomial}}\label{appendix:Polynomial}

\subsection{Proof of \cref{lm:ptimeupperbound}}\label{app:ptimeupperbound}

\ptimeupperbound*
% \begin{lemma}\label{lemma:PTIMEEasy}
%     The constrained existence problem of RSE can be solved in $\PTIME$ when all players are optimists.
%     The running time of such an algorithm is at most $\Oh()$ where $n$ is the number of vertices in the graph and $p$ is the number of players.\theju{To compute and write here}
%     Moreover, there is a functional version of that algorithms that outputs a succinct representation of an RSE satisfying the constraints, when it exists, in the same time.
% \end{lemma}

\begin{proof}[Proof of \cref{lm:ptimeupperbound}] \paragraph*{Preliminary remarks}

We are given the game $\Game_{\|v_0}$ and two threshold vectors $\bx, \by \in \Qb^{\Pi}$; we wish to find an XRSE $\bsigma$ such that $\bx \leq \X(\bsigma) \leq \by$. 

    Throughout the proof, when $W \subseteq V$ is a set of vertices and $F \subseteq E$ is a set of edges, we write $\Attr(W, F)$ for the \emph{positive probabilistic attractor} of $W$ in $(V, F)$, i.e. the set of vertices $v$ such that for every strategy profile $\bsigma$ in $\Game_{\|v}$ that uses only edges of $F$, there is a positive probability of reaching $W$.
    As a consequence of \cref{lm:secretlemma} (replacing the vertices of $W$ with terminal vertices), we have the following.
    
    \begin{claim}\label{claim:positiveattractorLinear}
      Given $W$, the set $\Attr(W, F)$ can be computed in time $\Oh(m)$.
    \end{claim}
% Such an algorithm for above can be computed using a trivial modification of the attractor computation algorithm.

% Similarly, we also first compute the value of a game where we  compute the extreme risk of each player when the other players work to reduce the player's extreme-risk value.    This can be viewed as the XRSE value when viewed as a zero-sum game starting at vertex for the player $i$ owning the vertex $v$.
% We represent it in the algorithm using $\val(v)$ for a vertex $v$, defined for all the non-stochastic vertices $v$. 
% \theju{need to check if these words are the ones we are using! I forgot already :////}

Similarly, \cref{lm:secretlemma} enables us to compute the \emph{adversarial values} of each vertex, i.e., the best risk measure that the player controlling that vertex can ensure from that vertex when the other players are fully hostile.

    \begin{claim}\label{claim:adverserialXRLinear}
        For each $i$ and $v \in V_i$, the quantity:
        $$\val(v) = \inf_{\btau_{-i} \in \Strat_{-i}\Game_{\|v}} \sup_{\tau_i \in \Strat_i\Game_{\|v}} \X_i(\btau)$$
        can be computed in time $\Oh(m)$. 
    \end{claim}

Then, computing all those values can be done in time $\Oh(m^2)$.
We can therefore assume that those quantities $\val(v)$ are given with the input.




    \paragraph*{Cycle-friendly and cycle-averse cases}

We differentiate the two types of instances. 
    If there exists a player $i$ such that we have $y_i < 0$, then the requirement $\bx \leq \X(\bsigma) \leq \by$ implies that $\bsigma$ must almost surely reach a terminal vertex: we call that case the \emph{cycle-averse} case.
    If there is no such player, we are in the \emph{cycle-friendly} case.
    Our algorithm will work slightly differently in those two cases.
    However, the fundamental idea is still the same in both cases: we prune iteratively the set of edges, and each of the subsets $F \subseteq E$ which we obtain will induce a strategy profile $\bsigma^F$, in which the profitable deviations will be detected and used to prune new edges.
    However, the definition of $\bsigma^F$ differs in the cycle-averse and the cycle-friendly case.

 
    
    \paragraph*{Algorithm in the cycle-friendly case.}
    In the cycle-friendly case, for a given set of edges $F$, the strategy profile $\bsigma^F$ in the game $\Game_{\|v_0}$, is defined as follows: from each non-stochastic vertex $v$, when $v$ is seen for the first time, the strategy profile randomises uniformly between all the edges $vw \in F$.
    Later, when $v$ is visited again, it always repeats the same choice.
    Equivalently, each player initially chooses, at random, a positional strategy, and then follows it.
    If some player $i$ deviates and takes an edge that they are not supposed to take (be it an edge that does not belong to $F$ or an outgoing edge of a vertex from which a different edge has already been taken), then all the players switch to the positional strategy profile $\btau^{\dag i}$, where $\btau^{\dag i}_{-i}$ minimises the best risk measure that player $i$ can get (a positional such strategy profile exists by \cref{lm:secretlemma}), and $\tau^{\dag i}_i$ is some positional strategy.

     
    Our algorithm in the cycle-friendly case is presented in~\cref{algo:cyclefriendly}.
    Each step $k$ consists of identifying a new set of vertices $V_\bad^k$ that must be avoided.
    At step $k=0$, it is the set of terminal vertices that give some player $i$ a payoff that is larger than $y_i$, which would then make them have an off-constraints risk measure.
    At step $k \geq 1$, it is the set of vertices $v$ whose adversarial value $\val(v)$ is greater than the risk $z_i^k = \X_i(\bsigma^{E_k})$, where $i$ is the player controlling $v$.
    In other words, the vertices from which that player can have a profitable deviation.
    Note that it that second case, the computation of $V_\bad^k$ requires the computation of $z_i^k$, which can be done in time $\Oh(m)$ by computing the set of terminals that are accessible from $v_0$ in $(V, E_k)$, and by deciding whether the probability of reaching no terminal is positive: that will be the case if and only if there exists a positional strategy profile that uses only edges of $E_k$ (and therefore that $\bsigma^{E_k}$ is following with positive probability) such that with positive probability no terminal vertex is reached, which can be decided in time $\Oh(m)$ using \cref{lm:secretlemma}.

     
    
    Then, the positive probabilistic attractor $A_k = \Attr(v_\bad^k, E_k)$ is computed.
    If $k \geq 1$ and $v_0 \in A_k$, i.e., if it is not possible to avoid reaching the set $V_\bad^k$, the answer $\No$ is returned.
    Otherwise, the set $E_{k+1}$ is defined from $E_k$ by removing all the edges that lead from a vertex that does not belong to $A_k$ to a vertex that does, thus making sure that $V_\bad^k$ will never be reached.
    The algorithm stops when there is no more edge to remove.
    Then, the algorithm answers $\Yes$ and outputs the set $E_k$, as a succinct representation of the strategy profile $\bsigma^{E_{k+1}}$, if we have $z_i^k \geq x_i$ for each $i$, and answers $\No$ otherwise.

     
    
            \begin{algorithm}
            \begin{algorithmic}\caption{Constrained existence problem with optimists in the cycle-friendly case}\label{algo:cyclefriendly}
                \Procedure{CycleFriendly}{$\Game, \Bar{x},\Bar{y}$}
                    \State $k \gets 0$
                    \State $E_k\gets E$
                    \State $V_\bad^k = \{t\in T \mid \mu_i(t)> y_i\}$ 
                    \State $A_k \gets \Attr(V^k_\bad, E_k)$
                    \If{$v_0 \in A_k$}
                        \Return{$\No$}
                    \Else
                        \State $E_{k+1} \gets E_k \setminus \{uv\in E_k\mid u\not\in A_k\text{ and }v\in A_k\}$
                    \While{$k=0$\text{ or }$E_{k+1}\neq E_k$}
                        \State $k\gets k+1$
                        \State Compute $z^k_i = \X_i(\bsigma^{E_k})$ for each $i \in \Pi$
                        \State $V^k_\bad \gets \{v \mid \val(v) > z_i^k \text{ for } i \in \Pi \text{ such that } v \in V_i\}$
                        \State $A_k \gets \Attr(V^k_\bad, E_k)$
                        \If{$v_0 \in A_k$}
                            \Return{$\No$}
                        \Else
                            \State $E_{k+1} \gets E_k \setminus \{uv\in E_k\mid u\notin A_k\text{ and }v\in A_k\}$
                        \EndIf
                    \EndWhile
                    \EndIf
                    \If{$z_i^k\geq x_i$ for all players $i$}
                        \Return $(\Yes, E_{k+1})$
                    \Else{\text{ }}\Return{$\No$}
                    \EndIf
                \EndProcedure
            \end{algorithmic}
        \end{algorithm}

     \subparagraph*{Correctness in the cycle-friendly case.}
To prove the correctness of \cref{algo:cyclefriendly}, we first need to prove that the edge removals are such that all vertices always keep at least one outgoing edge, and that the stochastic ones always keep all of them, so that the strategy $\bsigma^{E_k}$ is always properly defined.

\begin{invariant}
    At each step $k$, every vertex $v \not\in V_?$ is such that $E_k(v) \neq \emptyset$, and every vertex $v \in V_?$ is such that $E_k(v) = E(v)$.
\end{invariant}

The proof is left to the reader.
We also need termination.

 

\begin{claim}
    \cref{algo:cyclefriendly} terminates.
\end{claim}

\begin{claimproof}
    At each step $k \geq 1$, we either have that the algorithm terminates, or that an edge is removed.
    The sequence $E_1, E_2, \dots$ is therefore strictly decreasing (note that we might have $E_0 = E_1$), hence it cannot be infinite.
\end{claimproof}

 

Now, to prove correctness, we will first prove the following claim.

        \begin{claim} \label{claim:zik}
            For each player $i$ and each index $k$, every strategy profile $\bsigma'$ that uses only edges of $E_k$ is such that $\X_i(\bsigma') \leq z_i^k$.
        \end{claim}

        \begin{claimproof}
            This result is a consequence of the fact that every payoff vector that can be obtained with positive probability in the strategy profile $\bsigma'$ is obtained with positive probability in the strategy profile $\bsigma^{E_k}$.
            
            Indeed, consider some payoff vector $\bz$ that has a positive probability to be generated in $\bsigma'$.
            If $\bz$ is obtained by reaching a terminal vertex, then that terminal vertex is accessible from $v_0$ in the graph $(V, E_k)$, and it therefore has a positive probability to be reached in $\bsigma^{E_k}$.
            
            If $\bz = (0)_i$ is obtained by reaching no terminal vertex, then by \cref{lm:secretlemma}, there exists a positional strategy profile $\btau$ that uses only edges of $E_k$ such that with positive probability, no terminal vertex is reached.
            Then, when following the strategy profile $\bsigma^{E_k}$, there is a positive probability that the players actually follow $\btau$.
            And therefore, there is also a positive probability to get the payoff vector $\bz = (0)_i$ in the strategy profile $\bsigma^{E_k}$.

            Since all players are optimists, the claim follows.
        \end{claimproof}

Note that this claim implies that the sequence $(z_i^k)_k$, for each $i$, is nondecreasing. 

 

We can now prove correctness.
To do so, we need to prove two propositions: the algorithm recognises only positive instances, and recognises all of them.

\begin{proposition}
    The algorithm recognises only positive instances.
\end{proposition}

 

\begin{claimproof}
        Let us assume that the algorithm answers $\Yes$ at step $k$: let us show that the strategy profile $\bsigma^{E_k}$ is an XRSE that satisfies the desired constraints.
        Note that the algorithm does never answer $\Yes$ at step $0$, hence we necessarily have $k \geq 1$.

        
        \subparagraph*{The strategy profile $\bsigma^{E_k}$ satisfies $\bx \leq \X({\bsigma^{E_k} }) \leq \by$.}
        
        The lower bound is immediate since the algorithm answers $\Yes$ at step $k$ only if the strategy profile $\bsigma^{E_k}$ satisfies that constraint.

             
            
            Regarding the upper bound, observe that the set $E_1$ has been defined so that the set $\Attr(V_{\frownie}^0, E)$, and therefore the set $V_{\frownie}^0$, is not accessible from $v_0$ in the graph $(V, E_1)$, and therefore not in the graph $(V, E_{k+1})$.
            Thus, it is almost sure in $\bsigma^{E_{k+1}}$ that no vertex of $V_{\frownie}^0$ will ever be reached.
            In other words, all terminals that have a positive probability of being reached give each player $i$ a lower payoff than $y_i$.
            Now, if there is a positive probability that the play never will reach a terminal, that also does not give any player $i$ such a payoff, since we are in the cycle-friendly case.
             

        \subparagraph*{The strategy profile $\sigma^{E_k}$ is an XRSE.}
        
        Let $i$ be a player, and let $\sigma'_i$ be a deviation of player $i$ from $\bsigma^{E_k}$.
        We can assume without loss of generality that $\sigma'_i$ is pure.        
        Let $z' = \X_i({\bsigma^{E_k}_{-i}, \sigma'_i})$ be the extreme risk measure obtained by the player $i$.
        We want to prove that the deviation $\sigma'_i$ is not profitable, that is, we have $z' \leq z_i^k$.

        If the deviation $\sigma'_i$ uses only the edges of $E_k$, then it cannot be profitable by \cref{claim:zik}.
        But if it does use more edges, let us show that it cannot be a profitable deviation either.
 
\begin{claim}
    If there is a history $hv$ compatible with $\bsigma^{E_k}$ such that $v\sigma'_i(hv) \not\in E_k$, then we have $\X_i(\bsigma^{E_k}_{\|hv}, \sigma'_{i\|hv}) \leq z_i^k$.
\end{claim}

\begin{claimproof}
    After such a history, the strategy profile $\bsigma^{E_k}_{-i}$ follows the positional strategy profile $\btau^{\dag i}_{-i}$.
    By the definition of that strategy profile, we have $\X_i(\bsigma^{E_k}_{\|hv}, \sigma'_{i\|hv}) \leq \val(v)$.
    On the other hand, the vertex $v$ is accessible from $v_0$ in $(V, E_k)$, since it is visited with a positive probability in $\bsigma^{E_k}$.
    Therefore, it does not belong to the set $A_k$, and in particular not to the set $V_\bad^k$, which means that we have $\val(v) \leq z_i^k$.
    Hence, the conclusion follows. 
\end{claimproof}



 
In the general case, the payoffs that player $i$ obtains with positive probability in the strategy profile $(\bsigma^{E_k}_{-i}, \sigma'_i)$ are obtained either by using only edges that belong to $E_k$, or by using an edge that does not. In both cases, we have shown that player $i$ cannot get a payoff greater than $z_i^k$, which proves that the strategy profile $\bsigma^{E_k}$ is an XRSE.
\cref{algo:cyclefriendly} answers $\Yes$ only on positive instances, and outputs in that case a succinct representation of an XRSE matching the constraints.
\end{claimproof}

It now remains to prove the converse.
 
            \begin{proposition}
                \cref{algo:cyclefriendly} recognises all positive instances. 
            \end{proposition}
    
            \begin{claimproof}
           Let us assume that we have a positive instance, i.e., that there exists an XRSE $\bsigma$ with $\bx \leq \X(\bsigma) \leq \by$.
        Let us show that the algorithm will answer $\Yes$.
        To do so, we first prove the following invariant: if an edge is removed at some step, then it is never taken by the XRSE $\bsigma$.

        \begin{invariant} \label{inv:edgesnotused}
            For each $k \geq 0$, every edge that has positive probability to be eventually taken in $\bsigma$ belongs to $E_k$.
        \end{invariant}

        \begin{claimproof}
        We prove the invariant by induction. 
        
        \subparagraph*{Base case.} The case $k=0$ is immediate, since we have $E_0 = E$.
        
        Further, at step $k=1$, if the strategy profile $\bsigma$ uses eventually, with positive probability, an edge that does not belong to $E_1$, then it goes with positive probability to a vertex $v \in \Attr(V^0_\bad, E_0)$.
        Then, with positive probability, a terminal vertex will be reached that gives to some player $i$ a payoff greater than $y_i$, which is impossible.
        Therefore, such an edge cannot be taken in $\bsigma$.
        
         
        \subparagraph*{Induction step.} Let us assume that the invariant is true until step $k \geq 1$, and let us show that it holds at step $k+1$.    
        Let $uv$ be an edge that is used with positive probability when following $\bsigma$, and let us assume toward contradiction that it does not belong to $E_{k+1}$.
        Since the invariant is true at each step until $k$, we can assume that $uv$ has been removed at step $k$, i.e., that we have $uv \in E_k \setminus E_{k+1}$.
        Then, we have $u \not\in A_k$ and $v \in A_k$.
        The strategy profile $\bsigma$ has therefore positive probability of visiting the set $A_k$, and therefore the set $V_\bad^k$.
        Then, from a vertex of $V_\bad^k$, i.e., a vertex $v$ with $\val(v) > z_i^k$, player $i$ can deviate and get a risk measure strictly better than $z_i^k$.
        But since the invariant is true at step $k$, the strategy profile $\bsigma$ uses only vertices of $E_k$, and therefore, by \cref{claim:zik}, we have $\X_i(\bsigma) \leq z_i^k$: player $i$ has a profitable deviation in $\bsigma$, which is impossible.
\end{claimproof}

         

        We are now able to conclude.
        The answer $\No$ can be given in the two following cases:
            \begin{itemize}
                \item \emph{If at step $k$, we have $v_0 \in \Attr(V^k_\bad, E_k)$.}
                Then, the strategy profile $\bsigma$ visits the set $V^k_\bad$ with positive probability.
                With the same arguments that were used in the proof of \cref{inv:edgesnotused}, that is not possible.
                
                \item \emph{If during step $k$, no edge is removed, but we have $z^k_i < x_i$ for some player $i$.}
                Since $\bsigma$ uses only edges of $E_k$ by \cref{inv:edgesnotused}, we can apply \cref{claim:zik}, and obtain $\X_i(\bsigma) \leq z^k_i$, and therefore $\X_i(\bsigma) < x_i$: that case is therefore also impossible by definition of $\bsigma$.
            \end{itemize}
        None of those cases is possible, hence our algorithm will eventually answer $\Yes$.
    \end{claimproof}

 




    \paragraph*{Algorithm in the cycle-averse case}
    % We define the algorithm on the graph $\Gc$.
    % For a game graph $G = (V,E)$. We assume that the game graph is such that there is always a path from every vertex to a terminal in $G$. 

    The algorithm and the structure of the proof will be similar.
    However, we need some significant modifications, especially in the definition of the strategy profiles $\bsigma^F$.

    In the cycle-averse case, for a given set of edges $F$, the strategy profile $\bsigma^F$, in the game $\Game_{\|v_0}$, is defined as follows: from each vertex $v \not\in V_?$, it randomises uniformly between all the edges $vw \in F$. Contrary to the cycle-friendly case, the outcome of such a randomisation has no influence on what will happen if $v$ is seen again.
    If some player $i$ deviates and takes an edge that they are not supposed to take (an edge that does not belong to $E_k$, then), then all the players switch to the positional strategy profile $\btau^{\dag i}$, where $\btau^{\dag i}_{-i}$ minimises the best risk measure that player $i$ can get (a positional such strategy profile exists by \cref{lm:secretlemma}), and $\tau^{\dag i}_i$ is some positional strategy.
 
    Our algorithm in the cycle-averse case is presented in~\cref{algo:cycleaverse}.
    Again, each step $k$ identifies a new set of vertices that must be avoided.
    Their definition depends now on the parity of $k$.
    When $k$ is even, it is the same as in the cycle-friendly case: the set $V^k_\bad$ is the set of vertices $v$ such that $\val(v) > z_i^k$, where $i$ is the player controlling $v$, and $A_k$ is the positive probabilistic attractor of $V_\bad^k$.
    When $k$ is odd, we define directly $A_k$ as the set of vertices from which whatever the players play, there is a positive probability of reaching no terminal vertex.
    
    Again, the computation of $V_\bad^k$ for an even step $k \geq 2$ requires the computation of $z_i^k$, which can be done in time $\Oh(m)$ by computing the set of terminals that are accessible from $v_0$ in $(V, E_k)$, and by deciding whether the probability of reaching no terminal is positive: that will be the case, now, if and only if there exists a vertex from which no terminal vertex is accessible, which can also be decided in time $\Oh(m)$.
    As for odd steps, the computation of $A_k$ can also be done in $\Oh(m)$ using \cref{lm:secretlemma}.
    
    If $k \geq 1$ and $v_0 \in A_k$, i.e., if it is not possible to avoid reaching the set $V_\bad^k$, the answer $\No$ is returned.
    Otherwise, the set $E_{k+1}$ is defined from $E_k$ by removing all the edges that lead from a vertex that does not belong to $A_k$ to a vertex that does, thus making sure that $V_\bad^k$ will never be reached.

    The loop stops when there is no more edge to remove, i.e., when we get $E_{k+2} = E_k$.
    Then, the algorithm answers $\No$ if we have $z_i^k < x_i$ for some $i$.
    Otherwise, it  performs \emph{final refinements}, defined as follows: first, it defines $F_0 = E_k$.
    Then, once $F_\l$ is defined for some $\l$, it checks whether there exists an edge $uv$ that matches the following conditions in the graph $(V, E_\l)$:
        \begin{enumerate}
            \item\label{itm:cuttableedge} the vertex $u$ is not stochastic and has several outgoing edges;
            \item\label{itm:nolessterminals} all the terminal vertices accessible from $v$ are also accessible from $v_0$ without using $uv$;
            \item\label{itm:nocycle} at least one terminal vertex is accessible from $u$ without using $uv$.
        \end{enumerate}
    In the following, we will refer to those conditions as Conditions~\ref{itm:cuttableedge}, \ref{itm:nolessterminals}, and \ref{itm:nocycle}.
    If there exists such an edge, then we define $F_{\l+1} = F_\l \setminus \{uv\}$.
    If there is no such edge, the algorithm stops there, answers $\Yes$, and returns $F_\l$ as a succinct representation of $\bsigma^{F_\l}$.
    
    
            \begin{algorithm}[t]
        \begin{algorithmic}\caption{Constrained existence problem with optimists in the cycle-averse case}\label{algo:cycleaverse}
                \Procedure{CycleAverse}{$\Game, \Bar{x},\Bar{y}$}
                    \State $k \gets 0$
                    \State $E_k\gets E$
                    \State $V_\bad^k = \{t\in T \mid \mu_i(t)> y_i\}$ 
                    \State $A_k \gets \Attr(V^k_\bad, E_k)$
                    \If{$v_0 \in A_k$}
                        \Return{$\No$}
                    \Else
                        \State $E_{k+1} \gets E_k \setminus \{uv\in E_k\mid u\not\in A_k\text{ and }v\in A_k\}$
                    \While{$E_{k+2}\neq E_k$\text{ or }$k \leq 1$}
                        \State $k\gets k+1$
                        \If{$k$ is even}
                            \State Compute $z^k_i = \X_i(\bsigma^{E_k})$ for each $i \in \Pi$
                            \State $V^k_\bad \gets \{v \mid \val(v) > z_i^k \text{ for } i \in \Pi \text{ such that } v \in V_i\}$
                            \State $A_k \gets \Attr(V^k_\bad, E_k)$
                        \Else
                            \State $A^k_\bad \gets \{v \mid \forall \btau \in \Strat_\Pi\Game_{\|v}, \prob_\btau(\Occ \cap T = \emptyset) > 0\}$
                        \EndIf
                        \If{$v_0 \in A_k$}
                            \Return{$\No$}
                        \Else
                            \State $E_{k+1} \gets E_k \setminus \{uv\in E_k\mid u\notin A_k\text{ and }v\in A_k\}$
                        \EndIf
                    \EndWhile
                    \EndIf
                    \If{$z_i^k < x_i$ for some player $i$}
                        \Return{$\No$}
                    \Else
                        \State $\l \gets 0$
                        \State $F_\l \gets E_k$
                        \Comment{Final refinement steps}
                        \While{there exists $uv$ satisfying Conditions~\ref{itm:cuttableedge}, \ref{itm:nolessterminals}, and \ref{itm:nocycle}}
                            \State $\l \gets \l+1$
                            \State $F_{\l+1} \gets F_\l \setminus \{uv\}$
                        \EndWhile
                        \Return{$(\Yes, F_\l)$}
                    \EndIf
                \EndProcedure
            \end{algorithmic}
        \end{algorithm}

     \paragraph*{Correctness in the cycle-averse case.}
The fact that $\bsigma^{E_k}$ and $\bsigma^{F_\l}$ are always correctly defined can be proved with arguments similar as those that were used for the cycle-friendly case.
We now focus on correctness properly said.
We first need the following properties.

\begin{invariant}\label{inv:terminalsaccessible}
    For every even $k > 0$, and for every $\l$, the graph $(V, E_k)$, or $(V, F_\l)$, contains no vertex that is accessible from $v_0$ and from which no terminal vertex is accessible.
\end{invariant}

\begin{claimproof}
    In the graph $(V, E_k)$ (for $k>0$ even), no induction is required: the set $E_k$ has been obtained after an odd step, in which the set $A_{k-1}$ has been made inaccessible.
    Thus, if we have a vertex $v$ from which no terminal vertex is accessible, it means that in the graph $(V, E_{k-1})$, all paths from $v$ to a terminal vertex were traversing a vertex of $A_{k-1}$, which implies that $v$ itself belonged to $A_{k-1}$, and is therefore not accessible from $v_0$ in $(V, E_k)$.

    This also proves that the invariant is true during the final refinements at step $\l = 0$.
    If now we assume that it is true at some step $\l$, then Condition~\ref{itm:nocycle} guarantees that it remains true at step $\l+1$.
\end{claimproof}


\begin{invariant} \label{inv:finishingtouchesconstantpayoff}
    If the algorithm switches to final refinements after step $k$, then for each step $\l$ of final refinements and for each player $i$, we have $\X_i(\bsigma^{F_\l}) = z_i^k$.
\end{invariant}

\begin{claimproof}
    The invariant is immediate for $\l=0$, since we have $F_\l = E_k$.
    Then, if it is true at step $\l$, it remains true at step $\l+1$.
    Indeed, Condition~\ref{itm:nolessterminals} guarantees that the set of terminal vertices accessible from $v_0$ in $(V, F_\l)$ is the same as in $(V, F_{\l+1})$.
    In other words, the terminal vertices that are reached with positive probability in $\bsigma^{F_\l}$ and $\bsigma^{F_{\l+1}}$ are the same.
    Moreover, \cref{inv:terminalsaccessible} guarantees that it is almost sure that some terminal vertex will be reached, in $\bsigma^{F_\l}$ as well as in $\bsigma^{F_{\l+1}}$.
    Therefore, the set of payoff vectors that have positive probability to be obtained is the same in both strategy profiles, hence the risk measures are the same.
\end{claimproof}


We can now prove correctness.
To do so, we need to prove two propositions: the algorithm recognises only positive instances, and recognises all of them.


\begin{proposition}
    The algorithm recognises only positive instances.
\end{proposition}

\begin{claimproof}
        Let us assume that the algorithm answers $\Yes$ at step $\l$ of the final refinements, after having switched to the final refinements loop at step $k$: let us show that the strategy profile $\bsigma^{F_\l}$ is an XRSE that satisfies the desired constraints.
        Note that the algorithm does never answer $\Yes$ at step $0$, hence we necessarily have $k \geq 1$.

 \subparagraph*{The strategy profile $\bsigma^{F_\l}$ satisfies $\bx \leq \X({\bsigma^{F_\l} }) \leq \by$.}
        
        The algorithm switches to the final refinements at step $k$ only if the strategy profile $\bsigma^{F_\l}$ satisfies $\X(\bsigma^{E_k}) \geq \bx$.
        Then, by \cref{inv:finishingtouchesconstantpayoff}, we also have $\X(\bsigma^{F_\l}) \geq \bx$.
        
            
        As for the upper bound, observe that the set $E_1$ has been defined so that the set $\Attr(V_{\frownie}^0, E)$, and therefore the set $V_{\frownie}^0$, is not accessible from $v_0$ in the graph $(V, E_1)$, and therefore not in the graph $(V, F_\l)$ either.
        Thus, it is almost sure in $\bsigma^{F_\l}$ that no vertex of $V_{\frownie}^0$ will ever be reached.
        In other words, all terminals that have positive probability to be reached give to each player $i$ a payoff smaller than $y_i$.
        That is sufficient to prove the lower bound, because it is almost sure, when following $\bsigma^{F_\l}$, that some terminal vertex will eventually be reached, by \cref{inv:terminalsaccessible}.
            

        \subparagraph*{The strategy profile $\bsigma^{F_\l}$ is an XRSE.}
        
        Let $i$ be a player, and let $\sigma'_i$ be a deviation of player $i$ from $\bsigma^{F_\l}$.
        %Since the strategy profile $\bsigma^{F_\l}_{-i}$ is positional, we can now assume without loss of generality that $\sigma'_i$ is positional.        
        Let $z' = \X_i({\bsigma^{F_\l}_{-i}, \sigma'_i})$ be the extreme risk measure obtained by player $i$.
        We want to prove that the deviation $\sigma'_i$ is not profitable, i.e., that we have $z' \leq z_i^k$ (since we have $\X_i(\bsigma^{E_\l}) = z_i^k$ by \cref{inv:finishingtouchesconstantpayoff}).
        To do so, we first show that player $i$ cannot obtain a payoff better than $z_i^k$ after using an edge that does not belong to $F_\l$.

        We first show that if the deviation $\sigma'_i$ uses only edges of $F_\l$, then it cannot be profitable.

\begin{claim}\label{claim:finishingtouchescycleimpossible}
    If it is almost sure, when following $(\bsigma^{F_\l}_{-i}, \sigma'_i)$, that only edges of $F_\l$ will be used, then the deviation $\sigma'_i$ is not profitable.
\end{claim}

\begin{claimproof}
    First, let us note that as long as player $i$ uses only edges that belong to $F_\l$, the strategy profile $\bsigma^{F_\l}$ behaves in a stationary way, and we can therefore assume without loss of generality that $\sigma'_i$ is positional.

    The payoff $z'$ may be obtained by reaching a terminal vertex: in that case, that terminal vertex is accessible from $v_0$ in $(V, F_\l)$, and therefore also reached with positive probability when following the strategy profile $\bsigma^{F_\l}$, hence $z' \leq z_i^k$.

    Let us show that it cannot be obtained by reaching no terminal.
    We proceed by contradiction: if, in the strategy profile $(\bsigma^{F_\l}, \sigma'_i)$, there is a positive probability of reaching no terminal when following that strategy profile, then there is a vertex that has positive probability of being visited infinitely often.
    We can then define the set $W$ of such vertices, i.e., the set $W = \{v \in V \mid \prob_{\bsigma^{F_\l}, \sigma'_i}(v \in \Inf) > 0\}$.
    Thus, when the strategy profile $(\bsigma^{F_\l}, \sigma'_i)$ is followed from a vertex of $W$, it is almost sure that no terminal vertex is reached, and that the set $W$ will never be left.
    We can then choose $w \in W$ such that it has positive probability of being reached without visiting any other vertex of $W$ before, i.e., such that there exists a history $hw$ from $v_0$ with $\Occ(h) \cap W = \emptyset$ (note that $h$ can be empty).
    
    On the other hand, in the graph $(V, F_\l)$, there is at least one terminal vertex accessible from $w$: all vertices from which no terminal is accessible are made themselves inaccessible at odd steps, the switch to final refinements loop happens only if there is no more edge to remove in that perspective, and Condition~\ref{itm:nocycle} guarantees that the final refinements loop leave at least one terminal vertex accessible from every vertex accessible from $v_0$.

    From each terminal $t$ accessible from $w$, we pick a simple path $h_0^t \dots h_{q_t}^t$ from $h_0^t = w$ to $h_{q_t}^t = t$ in the graph $(V, F_\l)$.
    Those paths define a directed acyclic graph (DAG) $D = (V_D, E_D)$ rooted at $w$, where all non-terminal vertices have at least one outgoing edge, with $V_D \subseteq V$ and $E_D \subseteq F_\l$.
    Now, since $\sigma'_i$ guarantees that no terminal vertex will be reached, each branch $h^t$ of that DAG is such that there exists a (smallest) index $j$ with $h^t_j \in W \cap V_i$, and $\sigma'_i(h^t_j) \neq h^t_{j+1}$.
    It may be the case that $h^t_j \sigma'_i(h^t_j) \in E_D$, i.e., that from $h^t_j$, player $i$ proceeds to an undetectable deviation and takes another branch of the DAG.
    But that cannot be the case for all $t$, otherwise, there would be a branch $h^t$ that would be followed with positive probability when following $(\bsigma^{F_\l}_{-i}, \sigma'_i)$ from $w$, and therefore a terminal vertex $t$ that would be reached with positive probability, which contradicts the definition of $w$.

    There must therefore exist an edge $uv \in F_\l \setminus E_D$, with $u \in V_D \cap V_i \cap W$
    We will show that such an edge should have been removed during the final refinements loop.
    First, it immediately satisfies Condition~\ref{itm:cuttableedge}.
    Moreover, the vertex $u$ is necessarily on a branch $h^t$ of $D$ that leads to a terminal vertex $t$, hence it satisfies Condition~\ref{itm:nocycle}.
    Finally, since $v$ is accessible from $w$ in $(V, F_\l)$, the terminal vertices that are accessible from $v$ in $(V, F_\l)$ are all accessible from $w$ in that same graph, and therefore are accessible from $w$ in the DAG $D$.
    Since $w$ is accessible from $v_0$ without visiting any vertex of $W$, and it particular without visiting $u$, it means that the terminal vertices accessible from $v$ are also accessible from $v_0$ without using the edge $uv$.
    In other words, the edge $uv$ satisfies Condition~\ref{itm:nolessterminals}, and should have been removed during the final refinements.

    This case is therefore impossible: when the players follow the strategy profile $(\bsigma^{F_\l}, \sigma'_i)$, it is almost sure that some terminal vertex will be reached, and that concludes the proof.
\end{claimproof}
        
But now, if the strategy $\sigma'_i$ does use edges that do not belong to $F_\l$, let us show that it cannot be a profitable deviation either.

\begin{claim}
    If there is a history $hv$ compatible with $\bsigma^{F_\l}$ such that $v\sigma'_i(hv) \not\in F_\l$, then we have $\X_i(\bsigma^{F_\l}_{\|hv}, \sigma'_{i\|hv}) \leq z_i^k$.
\end{claim}

\begin{claimproof}
    After such a history, the strategy profile $\bsigma^{F_\l}_{-i}$ follows the positional strategy profile $\btau^{\dag i}_{-i}$.
    By definition of that strategy profile, we have $\X_i(\bsigma^{F_\l}_{\|hv}, \sigma'_{i\|hv}) \leq \val(v)$.
    On the other hand, the vertex $v$ is accessible from $v_0$ in $(V, F_\l)$, since it is visited with positive probability in $\bsigma^{F_\l}$.
    Therefore, it does not belong to the set $A_k$ (if $k$ is even) or $A_{k-1}$ (if $k$ is odd), and in particular not to the set $V_\bad^k$ or $V_\bad^{k-1}$, which means that we have $\val(v) \leq z_i^k$.
    Hence the conclusion.
\end{claimproof}



In the general case, the payoffs that player $i$ obtains with positive probability in the strategy profile $(\bsigma^{F_\l}_{-i}, \sigma'_i)$ are either obtained using only edges that belong to $F_\l$, or by using an edge that does not: in both cases, we have shown that player $i$ cannot get a payoff greater than $z_i^k$, which proves that the strategy profile $\bsigma^{F_\l}$ is an XRSE.
\cref{algo:cyclefriendly} answers $\Yes$ only on positive instances, and outputs in that case a succinct representation of an XRSE matching the constraints.
\end{claimproof}


We will now prove the converse.

            \begin{proposition}
                \cref{algo:cycleaverse} recognises all positive instances. 
            \end{proposition}
    
            \begin{claimproof}
           Let us assume that we have a positive instance, i.e., that there exists an XRSE $\bsigma$ with $\bx \leq \X(\bsigma) \leq \by$.
        Let us show that the algorithm will answer $\Yes$.
        To do so, we first prove the following invariant: if an edge is removed at some step \emph{before the final refinements}, then it is never taken by the XRSE $\bsigma$.

        \begin{invariant} \label{inv:edgesnotused_bis}
            For each $k \geq 0$, every edge that has positive probability to be eventually taken in $\bsigma$ belongs to $E_k$.
        \end{invariant}

        \begin{claimproof}
        We prove the invariant by induction. 
        
        \subparagraph*{Base case.} The case $k=0$ is immediate, since we have $E_0 = E$.
        
        Further, at step $k=1$, if the strategy profile $\bsigma$ uses eventually, with positive probability, an edge that does not belong to $E_1$, then it goes with positive probability to a vertex $v \in \Attr(V^0_\bad, E_0)$.
        Then, with positive probability, a terminal vertex will be reached that gives to some player $i$ a payoff greater than $y_i$, which is impossible.
        Therefore, such an edge cannot be taken in $\bsigma$.
        
        
        \subparagraph*{Induction step.} Let us assume that the invariant is true until step $k \geq 1$, and let us show that it holds at step $k+1$.    
        Let $uv$ be an edge that is used with positive probability when following $\bsigma$, and let us assume toward contradiction that it does not belong to $E_{k+1}$.
        Since the invariant is true at each step until $k$, we can assume that $uv$ has been removed at step $k$, i.e., that we have $uv \in E_k \setminus E_{k+1}$.
        Then, we have $u \not\in A_k$ and $v \in A_k$.
        The strategy profile $\bsigma$ has therefore positive probability of visiting the set $A_k$.
        
        We must now distinguish the cases where $k$ is even or odd.
        If $k$ is even, then the strategy profile $\bsigma$ has therefore positive probability of visiting a vertex $v \in V_\bad^k$.
        If player $i$ is the player controlling $v$, then that player has a deviation in which they get risk measure at least $\val(v) > z_i^k$.
        Let us now note that when the players follow the strategy profile $\bsigma$, it is almost sure that a terminal vertex will eventually be reached, and that all the terminal vertices that have positive probability of being reached also have positive probability of being reached in $\bsigma^{E_k}$, since the invariant is true at step $k$: therefore, we have $z_i^k \geq \X_i(\bsigma)$, and player $i$ has a profitable deviation after $v$, which contradicts the fact that $\bsigma$ is an XRSE.

        If $k$ is odd, then, by definition of $A_k$, when the strategy profile $\bsigma$ is followed, there is a positive probability of reaching no terminal.
        But that is impossible in the cycle-averse case.

        The invariant is therefore necessarily still true at step $k+1$.
\end{claimproof}

        

        We are now able to conclude.
        The answer $\No$ can be given in the two following cases:
            \begin{itemize}
                \item \emph{If at step $k$, we have $v_0 \in \Attr(V^k_\bad, E_k)$.}
                Then, the strategy profile $\bsigma$ visits the set $V^k_\bad$ with positive probability.
                With the same arguments that were used in the proof of \cref{inv:edgesnotused_bis}, that is not possible.
                
                \item \emph{If during steps $k-1$ and $k$, no edge is removed, but we have $z^k_i < x_i$ for some player $i$.}
                Since $\bsigma$ uses only edges of $E_k$ by \cref{inv:edgesnotused_bis} and reaches almost surely a terminal (since we are in the cycle-averse case), we have $\X_i(\bsigma) \leq z^k_i$, and therefore $\X_i(\bsigma) < x_i$: that case is therefore also impossible by definition of $\bsigma$.
            \end{itemize}
        None of those cases is possible, hence our algorithm will eventually answer $\Yes$.
    \end{claimproof}


\paragraph*{Complexities.}
We consider here the complexity of the two algorithms.
Since at least one edge is removed every two steps, there are $\Oh(m)$ steps.
In each of them, we need $\Oh(p)$ calls to simple algorithms: computation of $z_i^k$, of $V_\bad^k$, of $A_k$.
Hence, the complexity $\Oh(pm^2)$.

In the cycle-averse case, when an output is asked, we need to add the final refinements loop: which consist of $\Oh(m)$ additional steps in which we check, for each of the $\Oh(m)$ remaining edges, whether they satisfy Conditions~\ref{itm:cuttableedge},~\ref{itm:nolessterminals}, and~\ref{itm:nocycle} in the finishing touches), which can be done in time $\Oh(m)$.
Hence the complexity $\Oh(pm^2 + m^3)$.
\end{proof}





\subsection{Proof of \cref{lm:ptimelowerbound}}\label{app:ptimelowerbound}
\ptimelowerbound*
\begin{proof}[Proof of \cref{lm:ptimelowerbound}]
        Given a deterministic two-player (between players $\Circle$ and $\Square$) zero-sum reachability game $\Game_{\|v_0}$ with target set of vertices $T$, we construct a simple stochastic game (with no stochastic vertices) where there is an XRSE $\bsigma$ satisfying $\X_\circ(\bsigma) = 1$ and $\X_\Box(\bsigma) = -1$ if and only if player $\Circle$ wins the game.  

        The game is simply obtained by assigning rewards on the zero-sum two player game as follows: we make all nodes in the target set $T$ of the reachability game as a terminal node where player $\Circle$ gets reward $1$ and player $\Square$ the reward $-1$. Recall that if no terminal is reached, both players get reward $0$.  

        If $\Circle$ has a strategy to win the reachability game, it is easy to see that the same strategy for $\Circle$, along with any strategy for $\Square$, will be an XRSE in that new game, and that it satisfies the constraint.
        Similarly, if on the other hand, player $\Square$ has a strategy to avoid the states $T$, then no strategy of $\Circle$ that gives her payoff $+1$ and gives player $\Square$ the payoff $-1$ will be an equilibrium, since $\Square$ can always deviate to the winning strategy in the reachability game that offers him the better payoff of $0$.
\end{proof}



 % \section{Existence problem}\label{sec:Exists}
 % %\leonard{This is only the old proof, and should not appear in the paper.}

% \begin{theorem}[Existence of RSE]
%     Let $\Game_{\|v_0}$ be a multiplayer simple stochastic game with only non-negative rewards, and let $\brho \in \{\pm \infty\}^\Pi$.
%     Then, there exists a stationary strategy that is a $(P,O)$-RSE in $\Game_{\|v_0}$.
%     Moreover, there exists an algorithm that, given such a game, outputs the representation of such an RSE in time $\Oh(m^2 p)$, where $m$ is the number of edge and $p$ is the number of players.
% \end{theorem}

% \begin{proof}
%     \leonard{Should be a functional proof, feel free to check}

%     Let $\Game_{\|v_0}$ be a game, and let  $(P,O)$ represent the partition of the players into pessimists and optimists. 
%     We define a decreasing sequence $E_0, E_1, \dots$ of subsets of $E$, satisfying the invariant that for each $n$, each stochastic vertex has all its outgoing edges in $E_n$, and that each non-stochastic and non-terminal vertex has at least one.


%     First, we define $E_0 = E$, which immediately satisfies the invariant.\theju{invariant. Not induction hyp.}
%     When $E_n$ is defined, let us consider the game $\Game^n_{\|v_0}$ defined as equal to $\Game_{\|v_0}$ but where the edges that do not belong to $E_n$ have been removed (that is indeed a game structure by the invariant).
%     Let us consider the memoryless strategy profile $\bsigma^n$ that consists, from each vertex $v$, in taking with positive probability any edge $vw \in E_n$.
%     If $\bsigma^n$ is an RSE in $\Game_{\|v_0}^n$, we stop there.
%     If it is not, then let $i$ be a player that has a profitable deviation $\sigma'_i$ --- note that we are only considering the game $\Game_{\|v_0}^n$, and that therefore $\sigma'_i$ uses no more edges than $\sigma^n_i$.
%     By TODO, we can assume that $\sigma'_i$ is positional.

%     If player $i$ is optimistic, then there exists a payoff $x$ such that $\prob_{\bsigma^n}(\mu_i = x) = 0$, that $\prob_{\bsigma^n_{-i}, \sigma'_i}(\mu_i = x) > 0$, and that every payoff that player $i$ has positive probability to get in $\bsigma^n$ is strictly smaller than $x$.
%     Moreover, since we assume that all the rewards are positive, we necessarily have $x > 0$.
%     But, in the strategy profile $\bsigma^n$, all the terminal vertices that are accessible from $v_0$ in the game $\Game^n$ are reached with positive probability: that case is therefore impossible.

%     If player $i$ is pessimistic, let $y_n = \X(\bsigma^n)[\mu_i]$.
%     Then, when following the strategy profile $(\bsigma_{-i}^n, \sigma'_i)$, player $i$ gets almost surely more than the payoff $y_n$.
%     Let $W_n$ be the set of vertices $v$ from which whatever player $i$ does, their risk entropy is $y_n$ or less; formally, the set of vertices $v$ such that from $v$, in the game $\Game^n$, for every strategy $\tau_i$ of player $i$, we have $\prob_{\bsigma_{-i}^n, \tau_i}(\mu_i \leq y_n) > 0$.
%     The set $W_n$ is nonempty and accessible from $v_0$ in $(V, E_n)$.
%     Indeed, if $y_n$ is obtained by reaching a terminal vertex $t$, then we have $t \in W_n$, and $t$ is accessible from $v_0$.
%     If now $y_n = 0$ is obtained by reaching no terminal vertex, then when following $\bsigma^n$, with positive probability, no terminal is reached.
%     Then, there is in particular a vertex $u$ that has positive probability to be visited infinitely often.
%     And when playing $\bsigma^n$ from $u$, the probability that some terminal is ever reached is actually $0$, since if it was some constant $q > 0$, then the probability of visiting it infinitely often would be $\lim_k (1-q)^k = 0$.
%     In other words, no terminal vertex is accessible from $u$ in $(V, E_n)$, hence $u \in W_n$.

%     Now, let us note that since $\sigma'_i$ is a profitable deviation, we have $v_0 \not\in W_n$, and since $W_n$ is nonempty and accessible from $v_0$, there exists at least one edge $vw \in E_n$ such that $v \not\in W_n$ and $w \in W_n$.
%     For each such edge, we have $v \in V_i$, and there exists at least one other edge $vw' \in E_n$ with $w' \not\in W_n$: otherwise, for every strategy $\tau_i$ of player $i$, the strategy profile $(\btau^n_{-i}, \tau_i)$, played from $v$, would with positive probability reach the vertex $w$ and then give player $i$ the payoff $y_n$, which contradicts the fact that $v \not\in W_n$.
%     We can therefore define $E_{n+1} \subseteq E_n$ by removing all such edges and only them: the invariant still holds (TODO: check).

%     Now, since there is at least one vertex that is removed at each step, the sequence is necessarily finite: there is $n$ such that $\bsigma^n$ is an RSE in $\Game^n_{\|v_0}$.
%     Let us prove that it is also an RSE in $\Game_{\|v_0}$.

%     Let us assume that some player $i$ has a deviation $\sigma'_i$ from $\bsigma^n$ in $\Game_{\|v_0}$.
%     Since $\bsigma^n$ is memoryless, we can assume that $\sigma'_i$ is positional by TODO.
%     and since $\bsigma^n$ is an RSE in the game $\Game^n$, the strategy $\sigma'_i$ uses an edge that does not belong to $E_n$, i.e. there exists $v$ accessible from $v_0$ in $(V, E_n)$ and $vw \in E \setminus E_n$ such that $w = \sigma'_i(v)$.
%     Since only edges controlled by pessimists have been removed, we can immediately deduce that player $i$ is a pessimist.
    
%     Now, among such edges, we choose one whose removal is the most ancient, i.e. we choose it in order to minimize the index $k$ such that $vw \in E_k \setminus E_{k+1}$.
%     Thus, in the strategy profile $(\bsigma^n_{-i}, \sigma'_i)$, it is almost sure that only edges of $E_k$ are used.

%     The fact that the edge $uv$ has been removed at step $k$ means that we had $v \not\in W_k$ and $w \in W_k$.
%     Thus, from the vertex $w$, if player $i$ uses only edges of $E_k$ and the other players follow the strategy profile $\bsigma^k_{-i}$, with positive probability, player $i$ gets the payoff $y_k$ or less.
%     We can already note that the strategy $\sigma'_i$ uses only edges of $E_k$.
%     Moreover, using the reasoning with which we proved that $W_k$ was nonempty, if $y_k \neq 0$ or less is obtained by reaching a terminal $t$, then we have $t \in W_k$ and with positive probability the terminal vertex $t$ is reached without leaving $W_k$.
%     And similarly, if $y_k$ or less is obtained by reaching no terminal, then with positive probability a vertex $u$ is reached without leaving $W_k$, such that from $u$, no terminal is accessible anymore: in both cases, we can say that $w$ is such that if, from $w$, player $i$ uses only edges of $E_k$, and the other players follow the strategy profile $\bsigma^k_{-i}$, then with positive probability player $i$ gets the payoff $y_k$ or less \emph{and} the set $W_k$ is never left.
    
%     Now, the set $E_{k+1}$ was defined so that $W_k$ is no longer accessible from $v_0$ in the graph $(V, E_{k+1})$.
%     Therefore, those vertices are not accessible at any step $\l > k$, and therefore no outgoing edge of a vertex of $W_k$ is ever removed in the sequel, i.e. $E_n \cap (W_k \times V) = E_k \cap (W_k \times V)$.
%     Consequently, since $\sigma'_i$ uses only edges of $E_k$, when the strategy profile $(\bsigma^n, \sigma'_i)$ is played from $w$, it is also true that with positive probability player $i$ gets the payoff $y_k$.
%     And therefore, we have $\X(\bsigma^n_{-i}, \sigma'_i)[\mu_i] \leq y_k$.

%     Let us now prove that $y_k = \X(\bsigma^k)[\mu_i] < \X(\bsigma^n)[\mu_i]$.
%     That is a consequence of the fact that the sequence $(z_\l) = \left(\X(\bsigma^\l)[\mu_i]\right)_\l$ of player $i$'s risk entropies is strictly increasing.
%     Indeed, let us assume $z_{\l+1} \leq z_\l$ for some $\l$.
%     That implies that, in the strategy profile $\bsigma^{\l+1}$, player $i$ has a positive probability of getting the payoff $z_\l$ or less.
%     Then, either there is a positive probability of reaching a terminal vertex that gives them a payoff $z_\l$ or less, or there is a positive probability of reaching no terminal vertex at all.

%     The first case is impossible, because all the terminal vertices that have positive probability of being reached when following $\bsigma^{\l+1}$, i.e. that are accessible from $v_0$ in $(V, E_{\l+1})$, are also accessible from $v_0$ in $(V, E_\l)$, and therefore have positive probability to be reached when following $\bsigma^\l$.

%     In the second case, with the same reasoning as above, there is in particular, a vertex $v$ that has positive probability to be visited infinitely often when $\bsigma^{\l+1}$ is played from $v_0$, and therefore such that if $\bsigma^{\l+1}$ is played from $v$, the probability of reaching a terminal vertex is $0$, i.e. no terminal vertex is accessible from $v$ in $(V, E_{\l+1})$.
%     But then, the vertex $v$ belongs to the set $W_\l$.
%     Indeed, let $j$ be the player that was controlling the edges that were removed at step $\l$.
%     Consider a strategy $\tau_j$ of player $j$ that uses only edges of $E_\l$.
%     Then, when the strategy profile $(\bsigma^\l_{-j}, \tau_j)$ is played, it will almost surely be true that either no terminal vertex is reached, leading to the payoff $0$, or an edge of $E_\l \setminus E_{\l+1}$ is taken, leading therefore to a vertex of $W_\l$, and to a risk entropy of $y_\l$ or less.
%     Thus, since all rewards are non-negative and therefore $y_\l \geq 0$, the vertex $v$ belongs to $W_\l$, which is impossible since it should then have been made inaccessible in the graph $(V, E_k)$.

%     The sequence $(z_\l)_\l$ is therefore strictly increasing, and as a consequence we have $y_k = z_k < \X(\bsigma^n)[\mu_i]$.
%     But then, the deviation $\sigma'_i$ is not a profitable deviation, which is a contradiction.
%     The strategy profile $\bsigma^n$ is a memoryless RSE.

%     \paragraph*{Algorithm}

%     The proof that is given immediately yields an algorithm.
%     Let us comment on its complexity.
%     At each step, at least on edge is removed: we have therefore $\Oh(m)$ such steps.

%     Now, each step starts with checking whether $\bsigma^n$ is an RSE, i.e. by checking, for each player $i$, whether that player $i$ has a profitable deviation.
%     That can be done by computing the set $W_n$ that corresponds to player $i$: player $i$ has a profitable deviation if and only if $v_0 \in W_n$.
%     The computation of the set $W_n$, for a given player $i$, can be done in time $\Oh(m)$ by TODO; and there are $p$ players.
%     Then, the step is finished by removing all the edges leading to $W_n$, which also takes time $\Oh(m)$.

%     Hence the complexity $\Oh(m^2 p)$.
% \end{proof}

\end{document}
