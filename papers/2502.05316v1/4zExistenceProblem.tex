%\leonard{This is only the old proof, and should not appear in the paper.}

% \begin{theorem}[Existence of RSE]
%     Let $\Game_{\|v_0}$ be a multiplayer simple stochastic game with only non-negative rewards, and let $\brho \in \{\pm \infty\}^\Pi$.
%     Then, there exists a stationary strategy that is a $(P,O)$-RSE in $\Game_{\|v_0}$.
%     Moreover, there exists an algorithm that, given such a game, outputs the representation of such an RSE in time $\Oh(m^2 p)$, where $m$ is the number of edge and $p$ is the number of players.
% \end{theorem}

% \begin{proof}
%     \leonard{Should be a functional proof, feel free to check}

%     Let $\Game_{\|v_0}$ be a game, and let  $(P,O)$ represent the partition of the players into pessimists and optimists. 
%     We define a decreasing sequence $E_0, E_1, \dots$ of subsets of $E$, satisfying the invariant that for each $n$, each stochastic vertex has all its outgoing edges in $E_n$, and that each non-stochastic and non-terminal vertex has at least one.


%     First, we define $E_0 = E$, which immediately satisfies the invariant.\theju{invariant. Not induction hyp.}
%     When $E_n$ is defined, let us consider the game $\Game^n_{\|v_0}$ defined as equal to $\Game_{\|v_0}$ but where the edges that do not belong to $E_n$ have been removed (that is indeed a game structure by the invariant).
%     Let us consider the memoryless strategy profile $\bsigma^n$ that consists, from each vertex $v$, in taking with positive probability any edge $vw \in E_n$.
%     If $\bsigma^n$ is an RSE in $\Game_{\|v_0}^n$, we stop there.
%     If it is not, then let $i$ be a player that has a profitable deviation $\sigma'_i$ --- note that we are only considering the game $\Game_{\|v_0}^n$, and that therefore $\sigma'_i$ uses no more edges than $\sigma^n_i$.
%     By TODO, we can assume that $\sigma'_i$ is positional.

%     If player $i$ is optimistic, then there exists a payoff $x$ such that $\prob_{\bsigma^n}(\mu_i = x) = 0$, that $\prob_{\bsigma^n_{-i}, \sigma'_i}(\mu_i = x) > 0$, and that every payoff that player $i$ has positive probability to get in $\bsigma^n$ is strictly smaller than $x$.
%     Moreover, since we assume that all the rewards are positive, we necessarily have $x > 0$.
%     But, in the strategy profile $\bsigma^n$, all the terminal vertices that are accessible from $v_0$ in the game $\Game^n$ are reached with positive probability: that case is therefore impossible.

%     If player $i$ is pessimistic, let $y_n = \X(\bsigma^n)[\mu_i]$.
%     Then, when following the strategy profile $(\bsigma_{-i}^n, \sigma'_i)$, player $i$ gets almost surely more than the payoff $y_n$.
%     Let $W_n$ be the set of vertices $v$ from which whatever player $i$ does, their risk entropy is $y_n$ or less; formally, the set of vertices $v$ such that from $v$, in the game $\Game^n$, for every strategy $\tau_i$ of player $i$, we have $\prob_{\bsigma_{-i}^n, \tau_i}(\mu_i \leq y_n) > 0$.
%     The set $W_n$ is nonempty and accessible from $v_0$ in $(V, E_n)$.
%     Indeed, if $y_n$ is obtained by reaching a terminal vertex $t$, then we have $t \in W_n$, and $t$ is accessible from $v_0$.
%     If now $y_n = 0$ is obtained by reaching no terminal vertex, then when following $\bsigma^n$, with positive probability, no terminal is reached.
%     Then, there is in particular a vertex $u$ that has positive probability to be visited infinitely often.
%     And when playing $\bsigma^n$ from $u$, the probability that some terminal is ever reached is actually $0$, since if it was some constant $q > 0$, then the probability of visiting it infinitely often would be $\lim_k (1-q)^k = 0$.
%     In other words, no terminal vertex is accessible from $u$ in $(V, E_n)$, hence $u \in W_n$.

%     Now, let us note that since $\sigma'_i$ is a profitable deviation, we have $v_0 \not\in W_n$, and since $W_n$ is nonempty and accessible from $v_0$, there exists at least one edge $vw \in E_n$ such that $v \not\in W_n$ and $w \in W_n$.
%     For each such edge, we have $v \in V_i$, and there exists at least one other edge $vw' \in E_n$ with $w' \not\in W_n$: otherwise, for every strategy $\tau_i$ of player $i$, the strategy profile $(\btau^n_{-i}, \tau_i)$, played from $v$, would with positive probability reach the vertex $w$ and then give player $i$ the payoff $y_n$, which contradicts the fact that $v \not\in W_n$.
%     We can therefore define $E_{n+1} \subseteq E_n$ by removing all such edges and only them: the invariant still holds (TODO: check).

%     Now, since there is at least one vertex that is removed at each step, the sequence is necessarily finite: there is $n$ such that $\bsigma^n$ is an RSE in $\Game^n_{\|v_0}$.
%     Let us prove that it is also an RSE in $\Game_{\|v_0}$.

%     Let us assume that some player $i$ has a deviation $\sigma'_i$ from $\bsigma^n$ in $\Game_{\|v_0}$.
%     Since $\bsigma^n$ is memoryless, we can assume that $\sigma'_i$ is positional by TODO.
%     and since $\bsigma^n$ is an RSE in the game $\Game^n$, the strategy $\sigma'_i$ uses an edge that does not belong to $E_n$, i.e. there exists $v$ accessible from $v_0$ in $(V, E_n)$ and $vw \in E \setminus E_n$ such that $w = \sigma'_i(v)$.
%     Since only edges controlled by pessimists have been removed, we can immediately deduce that player $i$ is a pessimist.
    
%     Now, among such edges, we choose one whose removal is the most ancient, i.e. we choose it in order to minimize the index $k$ such that $vw \in E_k \setminus E_{k+1}$.
%     Thus, in the strategy profile $(\bsigma^n_{-i}, \sigma'_i)$, it is almost sure that only edges of $E_k$ are used.

%     The fact that the edge $uv$ has been removed at step $k$ means that we had $v \not\in W_k$ and $w \in W_k$.
%     Thus, from the vertex $w$, if player $i$ uses only edges of $E_k$ and the other players follow the strategy profile $\bsigma^k_{-i}$, with positive probability, player $i$ gets the payoff $y_k$ or less.
%     We can already note that the strategy $\sigma'_i$ uses only edges of $E_k$.
%     Moreover, using the reasoning with which we proved that $W_k$ was nonempty, if $y_k \neq 0$ or less is obtained by reaching a terminal $t$, then we have $t \in W_k$ and with positive probability the terminal vertex $t$ is reached without leaving $W_k$.
%     And similarly, if $y_k$ or less is obtained by reaching no terminal, then with positive probability a vertex $u$ is reached without leaving $W_k$, such that from $u$, no terminal is accessible anymore: in both cases, we can say that $w$ is such that if, from $w$, player $i$ uses only edges of $E_k$, and the other players follow the strategy profile $\bsigma^k_{-i}$, then with positive probability player $i$ gets the payoff $y_k$ or less \emph{and} the set $W_k$ is never left.
    
%     Now, the set $E_{k+1}$ was defined so that $W_k$ is no longer accessible from $v_0$ in the graph $(V, E_{k+1})$.
%     Therefore, those vertices are not accessible at any step $\l > k$, and therefore no outgoing edge of a vertex of $W_k$ is ever removed in the sequel, i.e. $E_n \cap (W_k \times V) = E_k \cap (W_k \times V)$.
%     Consequently, since $\sigma'_i$ uses only edges of $E_k$, when the strategy profile $(\bsigma^n, \sigma'_i)$ is played from $w$, it is also true that with positive probability player $i$ gets the payoff $y_k$.
%     And therefore, we have $\X(\bsigma^n_{-i}, \sigma'_i)[\mu_i] \leq y_k$.

%     Let us now prove that $y_k = \X(\bsigma^k)[\mu_i] < \X(\bsigma^n)[\mu_i]$.
%     That is a consequence of the fact that the sequence $(z_\l) = \left(\X(\bsigma^\l)[\mu_i]\right)_\l$ of player $i$'s risk entropies is strictly increasing.
%     Indeed, let us assume $z_{\l+1} \leq z_\l$ for some $\l$.
%     That implies that, in the strategy profile $\bsigma^{\l+1}$, player $i$ has a positive probability of getting the payoff $z_\l$ or less.
%     Then, either there is a positive probability of reaching a terminal vertex that gives them a payoff $z_\l$ or less, or there is a positive probability of reaching no terminal vertex at all.

%     The first case is impossible, because all the terminal vertices that have positive probability of being reached when following $\bsigma^{\l+1}$, i.e. that are accessible from $v_0$ in $(V, E_{\l+1})$, are also accessible from $v_0$ in $(V, E_\l)$, and therefore have positive probability to be reached when following $\bsigma^\l$.

%     In the second case, with the same reasoning as above, there is in particular, a vertex $v$ that has positive probability to be visited infinitely often when $\bsigma^{\l+1}$ is played from $v_0$, and therefore such that if $\bsigma^{\l+1}$ is played from $v$, the probability of reaching a terminal vertex is $0$, i.e. no terminal vertex is accessible from $v$ in $(V, E_{\l+1})$.
%     But then, the vertex $v$ belongs to the set $W_\l$.
%     Indeed, let $j$ be the player that was controlling the edges that were removed at step $\l$.
%     Consider a strategy $\tau_j$ of player $j$ that uses only edges of $E_\l$.
%     Then, when the strategy profile $(\bsigma^\l_{-j}, \tau_j)$ is played, it will almost surely be true that either no terminal vertex is reached, leading to the payoff $0$, or an edge of $E_\l \setminus E_{\l+1}$ is taken, leading therefore to a vertex of $W_\l$, and to a risk entropy of $y_\l$ or less.
%     Thus, since all rewards are non-negative and therefore $y_\l \geq 0$, the vertex $v$ belongs to $W_\l$, which is impossible since it should then have been made inaccessible in the graph $(V, E_k)$.

%     The sequence $(z_\l)_\l$ is therefore strictly increasing, and as a consequence we have $y_k = z_k < \X(\bsigma^n)[\mu_i]$.
%     But then, the deviation $\sigma'_i$ is not a profitable deviation, which is a contradiction.
%     The strategy profile $\bsigma^n$ is a memoryless RSE.

%     \paragraph*{Algorithm}

%     The proof that is given immediately yields an algorithm.
%     Let us comment on its complexity.
%     At each step, at least on edge is removed: we have therefore $\Oh(m)$ such steps.

%     Now, each step starts with checking whether $\bsigma^n$ is an RSE, i.e. by checking, for each player $i$, whether that player $i$ has a profitable deviation.
%     That can be done by computing the set $W_n$ that corresponds to player $i$: player $i$ has a profitable deviation if and only if $v_0 \in W_n$.
%     The computation of the set $W_n$, for a given player $i$, can be done in time $\Oh(m)$ by TODO; and there are $p$ players.
%     Then, the step is finished by removing all the edges leading to $W_n$, which also takes time $\Oh(m)$.

%     Hence the complexity $\Oh(m^2 p)$.
% \end{proof}
