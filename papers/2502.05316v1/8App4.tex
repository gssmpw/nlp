
\subsection{Proof of \cref{lm:ptimeupperbound}}\label{app:ptimeupperbound}

\ptimeupperbound*
% \begin{lemma}\label{lemma:PTIMEEasy}
%     The constrained existence problem of RSE can be solved in $\PTIME$ when all players are optimists.
%     The running time of such an algorithm is at most $\Oh()$ where $n$ is the number of vertices in the graph and $p$ is the number of players.\theju{To compute and write here}
%     Moreover, there is a functional version of that algorithms that outputs a succinct representation of an RSE satisfying the constraints, when it exists, in the same time.
% \end{lemma}

\begin{proof}[Proof of \cref{lm:ptimeupperbound}] \paragraph*{Preliminary remarks}

We are given the game $\Game_{\|v_0}$ and two threshold vectors $\bx, \by \in \Qb^{\Pi}$; we wish to find an XRSE $\bsigma$ such that $\bx \leq \X(\bsigma) \leq \by$. 

    Throughout the proof, when $W \subseteq V$ is a set of vertices and $F \subseteq E$ is a set of edges, we write $\Attr(W, F)$ for the \emph{positive probabilistic attractor} of $W$ in $(V, F)$, i.e. the set of vertices $v$ such that for every strategy profile $\bsigma$ in $\Game_{\|v}$ that uses only edges of $F$, there is a positive probability of reaching $W$.
    As a consequence of \cref{lm:secretlemma} (replacing the vertices of $W$ with terminal vertices), we have the following.
    
    \begin{claim}\label{claim:positiveattractorLinear}
      Given $W$, the set $\Attr(W, F)$ can be computed in time $\Oh(m)$.
    \end{claim}
% Such an algorithm for above can be computed using a trivial modification of the attractor computation algorithm.

% Similarly, we also first compute the value of a game where we  compute the extreme risk of each player when the other players work to reduce the player's extreme-risk value.    This can be viewed as the XRSE value when viewed as a zero-sum game starting at vertex for the player $i$ owning the vertex $v$.
% We represent it in the algorithm using $\val(v)$ for a vertex $v$, defined for all the non-stochastic vertices $v$. 
% \theju{need to check if these words are the ones we are using! I forgot already :////}

Similarly, \cref{lm:secretlemma} enables us to compute the \emph{adversarial values} of each vertex, i.e., the best risk measure that the player controlling that vertex can ensure from that vertex when the other players are fully hostile.

    \begin{claim}\label{claim:adverserialXRLinear}
        For each $i$ and $v \in V_i$, the quantity:
        $$\val(v) = \inf_{\btau_{-i} \in \Strat_{-i}\Game_{\|v}} \sup_{\tau_i \in \Strat_i\Game_{\|v}} \X_i(\btau)$$
        can be computed in time $\Oh(m)$. 
    \end{claim}

Then, computing all those values can be done in time $\Oh(m^2)$.
We can therefore assume that those quantities $\val(v)$ are given with the input.




    \paragraph*{Cycle-friendly and cycle-averse cases}

We differentiate the two types of instances. 
    If there exists a player $i$ such that we have $y_i < 0$, then the requirement $\bx \leq \X(\bsigma) \leq \by$ implies that $\bsigma$ must almost surely reach a terminal vertex: we call that case the \emph{cycle-averse} case.
    If there is no such player, we are in the \emph{cycle-friendly} case.
    Our algorithm will work slightly differently in those two cases.
    However, the fundamental idea is still the same in both cases: we prune iteratively the set of edges, and each of the subsets $F \subseteq E$ which we obtain will induce a strategy profile $\bsigma^F$, in which the profitable deviations will be detected and used to prune new edges.
    However, the definition of $\bsigma^F$ differs in the cycle-averse and the cycle-friendly case.

 
    
    \paragraph*{Algorithm in the cycle-friendly case.}
    In the cycle-friendly case, for a given set of edges $F$, the strategy profile $\bsigma^F$ in the game $\Game_{\|v_0}$, is defined as follows: from each non-stochastic vertex $v$, when $v$ is seen for the first time, the strategy profile randomises uniformly between all the edges $vw \in F$.
    Later, when $v$ is visited again, it always repeats the same choice.
    Equivalently, each player initially chooses, at random, a positional strategy, and then follows it.
    If some player $i$ deviates and takes an edge that they are not supposed to take (be it an edge that does not belong to $F$ or an outgoing edge of a vertex from which a different edge has already been taken), then all the players switch to the positional strategy profile $\btau^{\dag i}$, where $\btau^{\dag i}_{-i}$ minimises the best risk measure that player $i$ can get (a positional such strategy profile exists by \cref{lm:secretlemma}), and $\tau^{\dag i}_i$ is some positional strategy.

     
    Our algorithm in the cycle-friendly case is presented in~\cref{algo:cyclefriendly}.
    Each step $k$ consists of identifying a new set of vertices $V_\bad^k$ that must be avoided.
    At step $k=0$, it is the set of terminal vertices that give some player $i$ a payoff that is larger than $y_i$, which would then make them have an off-constraints risk measure.
    At step $k \geq 1$, it is the set of vertices $v$ whose adversarial value $\val(v)$ is greater than the risk $z_i^k = \X_i(\bsigma^{E_k})$, where $i$ is the player controlling $v$.
    In other words, the vertices from which that player can have a profitable deviation.
    Note that it that second case, the computation of $V_\bad^k$ requires the computation of $z_i^k$, which can be done in time $\Oh(m)$ by computing the set of terminals that are accessible from $v_0$ in $(V, E_k)$, and by deciding whether the probability of reaching no terminal is positive: that will be the case if and only if there exists a positional strategy profile that uses only edges of $E_k$ (and therefore that $\bsigma^{E_k}$ is following with positive probability) such that with positive probability no terminal vertex is reached, which can be decided in time $\Oh(m)$ using \cref{lm:secretlemma}.

     
    
    Then, the positive probabilistic attractor $A_k = \Attr(v_\bad^k, E_k)$ is computed.
    If $k \geq 1$ and $v_0 \in A_k$, i.e., if it is not possible to avoid reaching the set $V_\bad^k$, the answer $\No$ is returned.
    Otherwise, the set $E_{k+1}$ is defined from $E_k$ by removing all the edges that lead from a vertex that does not belong to $A_k$ to a vertex that does, thus making sure that $V_\bad^k$ will never be reached.
    The algorithm stops when there is no more edge to remove.
    Then, the algorithm answers $\Yes$ and outputs the set $E_k$, as a succinct representation of the strategy profile $\bsigma^{E_{k+1}}$, if we have $z_i^k \geq x_i$ for each $i$, and answers $\No$ otherwise.

     
    
            \begin{algorithm}
            \begin{algorithmic}\caption{Constrained existence problem with optimists in the cycle-friendly case}\label{algo:cyclefriendly}
                \Procedure{CycleFriendly}{$\Game, \Bar{x},\Bar{y}$}
                    \State $k \gets 0$
                    \State $E_k\gets E$
                    \State $V_\bad^k = \{t\in T \mid \mu_i(t)> y_i\}$ 
                    \State $A_k \gets \Attr(V^k_\bad, E_k)$
                    \If{$v_0 \in A_k$}
                        \Return{$\No$}
                    \Else
                        \State $E_{k+1} \gets E_k \setminus \{uv\in E_k\mid u\not\in A_k\text{ and }v\in A_k\}$
                    \While{$k=0$\text{ or }$E_{k+1}\neq E_k$}
                        \State $k\gets k+1$
                        \State Compute $z^k_i = \X_i(\bsigma^{E_k})$ for each $i \in \Pi$
                        \State $V^k_\bad \gets \{v \mid \val(v) > z_i^k \text{ for } i \in \Pi \text{ such that } v \in V_i\}$
                        \State $A_k \gets \Attr(V^k_\bad, E_k)$
                        \If{$v_0 \in A_k$}
                            \Return{$\No$}
                        \Else
                            \State $E_{k+1} \gets E_k \setminus \{uv\in E_k\mid u\notin A_k\text{ and }v\in A_k\}$
                        \EndIf
                    \EndWhile
                    \EndIf
                    \If{$z_i^k\geq x_i$ for all players $i$}
                        \Return $(\Yes, E_{k+1})$
                    \Else{\text{ }}\Return{$\No$}
                    \EndIf
                \EndProcedure
            \end{algorithmic}
        \end{algorithm}

     \subparagraph*{Correctness in the cycle-friendly case.}
To prove the correctness of \cref{algo:cyclefriendly}, we first need to prove that the edge removals are such that all vertices always keep at least one outgoing edge, and that the stochastic ones always keep all of them, so that the strategy $\bsigma^{E_k}$ is always properly defined.

\begin{invariant}
    At each step $k$, every vertex $v \not\in V_?$ is such that $E_k(v) \neq \emptyset$, and every vertex $v \in V_?$ is such that $E_k(v) = E(v)$.
\end{invariant}

The proof is left to the reader.
We also need termination.

 

\begin{claim}
    \cref{algo:cyclefriendly} terminates.
\end{claim}

\begin{claimproof}
    At each step $k \geq 1$, we either have that the algorithm terminates, or that an edge is removed.
    The sequence $E_1, E_2, \dots$ is therefore strictly decreasing (note that we might have $E_0 = E_1$), hence it cannot be infinite.
\end{claimproof}

 

Now, to prove correctness, we will first prove the following claim.

        \begin{claim} \label{claim:zik}
            For each player $i$ and each index $k$, every strategy profile $\bsigma'$ that uses only edges of $E_k$ is such that $\X_i(\bsigma') \leq z_i^k$.
        \end{claim}

        \begin{claimproof}
            This result is a consequence of the fact that every payoff vector that can be obtained with positive probability in the strategy profile $\bsigma'$ is obtained with positive probability in the strategy profile $\bsigma^{E_k}$.
            
            Indeed, consider some payoff vector $\bz$ that has a positive probability to be generated in $\bsigma'$.
            If $\bz$ is obtained by reaching a terminal vertex, then that terminal vertex is accessible from $v_0$ in the graph $(V, E_k)$, and it therefore has a positive probability to be reached in $\bsigma^{E_k}$.
            
            If $\bz = (0)_i$ is obtained by reaching no terminal vertex, then by \cref{lm:secretlemma}, there exists a positional strategy profile $\btau$ that uses only edges of $E_k$ such that with positive probability, no terminal vertex is reached.
            Then, when following the strategy profile $\bsigma^{E_k}$, there is a positive probability that the players actually follow $\btau$.
            And therefore, there is also a positive probability to get the payoff vector $\bz = (0)_i$ in the strategy profile $\bsigma^{E_k}$.

            Since all players are optimists, the claim follows.
        \end{claimproof}

Note that this claim implies that the sequence $(z_i^k)_k$, for each $i$, is nondecreasing. 

 

We can now prove correctness.
To do so, we need to prove two propositions: the algorithm recognises only positive instances, and recognises all of them.

\begin{proposition}
    The algorithm recognises only positive instances.
\end{proposition}

 

\begin{claimproof}
        Let us assume that the algorithm answers $\Yes$ at step $k$: let us show that the strategy profile $\bsigma^{E_k}$ is an XRSE that satisfies the desired constraints.
        Note that the algorithm does never answer $\Yes$ at step $0$, hence we necessarily have $k \geq 1$.

        
        \subparagraph*{The strategy profile $\bsigma^{E_k}$ satisfies $\bx \leq \X({\bsigma^{E_k} }) \leq \by$.}
        
        The lower bound is immediate since the algorithm answers $\Yes$ at step $k$ only if the strategy profile $\bsigma^{E_k}$ satisfies that constraint.

             
            
            Regarding the upper bound, observe that the set $E_1$ has been defined so that the set $\Attr(V_{\frownie}^0, E)$, and therefore the set $V_{\frownie}^0$, is not accessible from $v_0$ in the graph $(V, E_1)$, and therefore not in the graph $(V, E_{k+1})$.
            Thus, it is almost sure in $\bsigma^{E_{k+1}}$ that no vertex of $V_{\frownie}^0$ will ever be reached.
            In other words, all terminals that have a positive probability of being reached give each player $i$ a lower payoff than $y_i$.
            Now, if there is a positive probability that the play never will reach a terminal, that also does not give any player $i$ such a payoff, since we are in the cycle-friendly case.
             

        \subparagraph*{The strategy profile $\sigma^{E_k}$ is an XRSE.}
        
        Let $i$ be a player, and let $\sigma'_i$ be a deviation of player $i$ from $\bsigma^{E_k}$.
        We can assume without loss of generality that $\sigma'_i$ is pure.        
        Let $z' = \X_i({\bsigma^{E_k}_{-i}, \sigma'_i})$ be the extreme risk measure obtained by the player $i$.
        We want to prove that the deviation $\sigma'_i$ is not profitable, that is, we have $z' \leq z_i^k$.

        If the deviation $\sigma'_i$ uses only the edges of $E_k$, then it cannot be profitable by \cref{claim:zik}.
        But if it does use more edges, let us show that it cannot be a profitable deviation either.
 
\begin{claim}
    If there is a history $hv$ compatible with $\bsigma^{E_k}$ such that $v\sigma'_i(hv) \not\in E_k$, then we have $\X_i(\bsigma^{E_k}_{\|hv}, \sigma'_{i\|hv}) \leq z_i^k$.
\end{claim}

\begin{claimproof}
    After such a history, the strategy profile $\bsigma^{E_k}_{-i}$ follows the positional strategy profile $\btau^{\dag i}_{-i}$.
    By the definition of that strategy profile, we have $\X_i(\bsigma^{E_k}_{\|hv}, \sigma'_{i\|hv}) \leq \val(v)$.
    On the other hand, the vertex $v$ is accessible from $v_0$ in $(V, E_k)$, since it is visited with a positive probability in $\bsigma^{E_k}$.
    Therefore, it does not belong to the set $A_k$, and in particular not to the set $V_\bad^k$, which means that we have $\val(v) \leq z_i^k$.
    Hence, the conclusion follows. 
\end{claimproof}



 
In the general case, the payoffs that player $i$ obtains with positive probability in the strategy profile $(\bsigma^{E_k}_{-i}, \sigma'_i)$ are obtained either by using only edges that belong to $E_k$, or by using an edge that does not. In both cases, we have shown that player $i$ cannot get a payoff greater than $z_i^k$, which proves that the strategy profile $\bsigma^{E_k}$ is an XRSE.
\cref{algo:cyclefriendly} answers $\Yes$ only on positive instances, and outputs in that case a succinct representation of an XRSE matching the constraints.
\end{claimproof}

It now remains to prove the converse.
 
            \begin{proposition}
                \cref{algo:cyclefriendly} recognises all positive instances. 
            \end{proposition}
    
            \begin{claimproof}
           Let us assume that we have a positive instance, i.e., that there exists an XRSE $\bsigma$ with $\bx \leq \X(\bsigma) \leq \by$.
        Let us show that the algorithm will answer $\Yes$.
        To do so, we first prove the following invariant: if an edge is removed at some step, then it is never taken by the XRSE $\bsigma$.

        \begin{invariant} \label{inv:edgesnotused}
            For each $k \geq 0$, every edge that has positive probability to be eventually taken in $\bsigma$ belongs to $E_k$.
        \end{invariant}

        \begin{claimproof}
        We prove the invariant by induction. 
        
        \subparagraph*{Base case.} The case $k=0$ is immediate, since we have $E_0 = E$.
        
        Further, at step $k=1$, if the strategy profile $\bsigma$ uses eventually, with positive probability, an edge that does not belong to $E_1$, then it goes with positive probability to a vertex $v \in \Attr(V^0_\bad, E_0)$.
        Then, with positive probability, a terminal vertex will be reached that gives to some player $i$ a payoff greater than $y_i$, which is impossible.
        Therefore, such an edge cannot be taken in $\bsigma$.
        
         
        \subparagraph*{Induction step.} Let us assume that the invariant is true until step $k \geq 1$, and let us show that it holds at step $k+1$.    
        Let $uv$ be an edge that is used with positive probability when following $\bsigma$, and let us assume toward contradiction that it does not belong to $E_{k+1}$.
        Since the invariant is true at each step until $k$, we can assume that $uv$ has been removed at step $k$, i.e., that we have $uv \in E_k \setminus E_{k+1}$.
        Then, we have $u \not\in A_k$ and $v \in A_k$.
        The strategy profile $\bsigma$ has therefore positive probability of visiting the set $A_k$, and therefore the set $V_\bad^k$.
        Then, from a vertex of $V_\bad^k$, i.e., a vertex $v$ with $\val(v) > z_i^k$, player $i$ can deviate and get a risk measure strictly better than $z_i^k$.
        But since the invariant is true at step $k$, the strategy profile $\bsigma$ uses only vertices of $E_k$, and therefore, by \cref{claim:zik}, we have $\X_i(\bsigma) \leq z_i^k$: player $i$ has a profitable deviation in $\bsigma$, which is impossible.
\end{claimproof}

         

        We are now able to conclude.
        The answer $\No$ can be given in the two following cases:
            \begin{itemize}
                \item \emph{If at step $k$, we have $v_0 \in \Attr(V^k_\bad, E_k)$.}
                Then, the strategy profile $\bsigma$ visits the set $V^k_\bad$ with positive probability.
                With the same arguments that were used in the proof of \cref{inv:edgesnotused}, that is not possible.
                
                \item \emph{If during step $k$, no edge is removed, but we have $z^k_i < x_i$ for some player $i$.}
                Since $\bsigma$ uses only edges of $E_k$ by \cref{inv:edgesnotused}, we can apply \cref{claim:zik}, and obtain $\X_i(\bsigma) \leq z^k_i$, and therefore $\X_i(\bsigma) < x_i$: that case is therefore also impossible by definition of $\bsigma$.
            \end{itemize}
        None of those cases is possible, hence our algorithm will eventually answer $\Yes$.
    \end{claimproof}

 




    \paragraph*{Algorithm in the cycle-averse case}
    % We define the algorithm on the graph $\Gc$.
    % For a game graph $G = (V,E)$. We assume that the game graph is such that there is always a path from every vertex to a terminal in $G$. 

    The algorithm and the structure of the proof will be similar.
    However, we need some significant modifications, especially in the definition of the strategy profiles $\bsigma^F$.

    In the cycle-averse case, for a given set of edges $F$, the strategy profile $\bsigma^F$, in the game $\Game_{\|v_0}$, is defined as follows: from each vertex $v \not\in V_?$, it randomises uniformly between all the edges $vw \in F$. Contrary to the cycle-friendly case, the outcome of such a randomisation has no influence on what will happen if $v$ is seen again.
    If some player $i$ deviates and takes an edge that they are not supposed to take (an edge that does not belong to $E_k$, then), then all the players switch to the positional strategy profile $\btau^{\dag i}$, where $\btau^{\dag i}_{-i}$ minimises the best risk measure that player $i$ can get (a positional such strategy profile exists by \cref{lm:secretlemma}), and $\tau^{\dag i}_i$ is some positional strategy.
 
    Our algorithm in the cycle-averse case is presented in~\cref{algo:cycleaverse}.
    Again, each step $k$ identifies a new set of vertices that must be avoided.
    Their definition depends now on the parity of $k$.
    When $k$ is even, it is the same as in the cycle-friendly case: the set $V^k_\bad$ is the set of vertices $v$ such that $\val(v) > z_i^k$, where $i$ is the player controlling $v$, and $A_k$ is the positive probabilistic attractor of $V_\bad^k$.
    When $k$ is odd, we define directly $A_k$ as the set of vertices from which whatever the players play, there is a positive probability of reaching no terminal vertex.
    
    Again, the computation of $V_\bad^k$ for an even step $k \geq 2$ requires the computation of $z_i^k$, which can be done in time $\Oh(m)$ by computing the set of terminals that are accessible from $v_0$ in $(V, E_k)$, and by deciding whether the probability of reaching no terminal is positive: that will be the case, now, if and only if there exists a vertex from which no terminal vertex is accessible, which can also be decided in time $\Oh(m)$.
    As for odd steps, the computation of $A_k$ can also be done in $\Oh(m)$ using \cref{lm:secretlemma}.
    
    If $k \geq 1$ and $v_0 \in A_k$, i.e., if it is not possible to avoid reaching the set $V_\bad^k$, the answer $\No$ is returned.
    Otherwise, the set $E_{k+1}$ is defined from $E_k$ by removing all the edges that lead from a vertex that does not belong to $A_k$ to a vertex that does, thus making sure that $V_\bad^k$ will never be reached.

    The loop stops when there is no more edge to remove, i.e., when we get $E_{k+2} = E_k$.
    Then, the algorithm answers $\No$ if we have $z_i^k < x_i$ for some $i$.
    Otherwise, it  performs \emph{final refinements}, defined as follows: first, it defines $F_0 = E_k$.
    Then, once $F_\l$ is defined for some $\l$, it checks whether there exists an edge $uv$ that matches the following conditions in the graph $(V, E_\l)$:
        \begin{enumerate}
            \item\label{itm:cuttableedge} the vertex $u$ is not stochastic and has several outgoing edges;
            \item\label{itm:nolessterminals} all the terminal vertices accessible from $v$ are also accessible from $v_0$ without using $uv$;
            \item\label{itm:nocycle} at least one terminal vertex is accessible from $u$ without using $uv$.
        \end{enumerate}
    In the following, we will refer to those conditions as Conditions~\ref{itm:cuttableedge}, \ref{itm:nolessterminals}, and \ref{itm:nocycle}.
    If there exists such an edge, then we define $F_{\l+1} = F_\l \setminus \{uv\}$.
    If there is no such edge, the algorithm stops there, answers $\Yes$, and returns $F_\l$ as a succinct representation of $\bsigma^{F_\l}$.
    
    
            \begin{algorithm}[t]
        \begin{algorithmic}\caption{Constrained existence problem with optimists in the cycle-averse case}\label{algo:cycleaverse}
                \Procedure{CycleAverse}{$\Game, \Bar{x},\Bar{y}$}
                    \State $k \gets 0$
                    \State $E_k\gets E$
                    \State $V_\bad^k = \{t\in T \mid \mu_i(t)> y_i\}$ 
                    \State $A_k \gets \Attr(V^k_\bad, E_k)$
                    \If{$v_0 \in A_k$}
                        \Return{$\No$}
                    \Else
                        \State $E_{k+1} \gets E_k \setminus \{uv\in E_k\mid u\not\in A_k\text{ and }v\in A_k\}$
                    \While{$E_{k+2}\neq E_k$\text{ or }$k \leq 1$}
                        \State $k\gets k+1$
                        \If{$k$ is even}
                            \State Compute $z^k_i = \X_i(\bsigma^{E_k})$ for each $i \in \Pi$
                            \State $V^k_\bad \gets \{v \mid \val(v) > z_i^k \text{ for } i \in \Pi \text{ such that } v \in V_i\}$
                            \State $A_k \gets \Attr(V^k_\bad, E_k)$
                        \Else
                            \State $A^k_\bad \gets \{v \mid \forall \btau \in \Strat_\Pi\Game_{\|v}, \prob_\btau(\Occ \cap T = \emptyset) > 0\}$
                        \EndIf
                        \If{$v_0 \in A_k$}
                            \Return{$\No$}
                        \Else
                            \State $E_{k+1} \gets E_k \setminus \{uv\in E_k\mid u\notin A_k\text{ and }v\in A_k\}$
                        \EndIf
                    \EndWhile
                    \EndIf
                    \If{$z_i^k < x_i$ for some player $i$}
                        \Return{$\No$}
                    \Else
                        \State $\l \gets 0$
                        \State $F_\l \gets E_k$
                        \Comment{Final refinement steps}
                        \While{there exists $uv$ satisfying Conditions~\ref{itm:cuttableedge}, \ref{itm:nolessterminals}, and \ref{itm:nocycle}}
                            \State $\l \gets \l+1$
                            \State $F_{\l+1} \gets F_\l \setminus \{uv\}$
                        \EndWhile
                        \Return{$(\Yes, F_\l)$}
                    \EndIf
                \EndProcedure
            \end{algorithmic}
        \end{algorithm}

     \paragraph*{Correctness in the cycle-averse case.}
The fact that $\bsigma^{E_k}$ and $\bsigma^{F_\l}$ are always correctly defined can be proved with arguments similar as those that were used for the cycle-friendly case.
We now focus on correctness properly said.
We first need the following properties.

\begin{invariant}\label{inv:terminalsaccessible}
    For every even $k > 0$, and for every $\l$, the graph $(V, E_k)$, or $(V, F_\l)$, contains no vertex that is accessible from $v_0$ and from which no terminal vertex is accessible.
\end{invariant}

\begin{claimproof}
    In the graph $(V, E_k)$ (for $k>0$ even), no induction is required: the set $E_k$ has been obtained after an odd step, in which the set $A_{k-1}$ has been made inaccessible.
    Thus, if we have a vertex $v$ from which no terminal vertex is accessible, it means that in the graph $(V, E_{k-1})$, all paths from $v$ to a terminal vertex were traversing a vertex of $A_{k-1}$, which implies that $v$ itself belonged to $A_{k-1}$, and is therefore not accessible from $v_0$ in $(V, E_k)$.

    This also proves that the invariant is true during the final refinements at step $\l = 0$.
    If now we assume that it is true at some step $\l$, then Condition~\ref{itm:nocycle} guarantees that it remains true at step $\l+1$.
\end{claimproof}


\begin{invariant} \label{inv:finishingtouchesconstantpayoff}
    If the algorithm switches to final refinements after step $k$, then for each step $\l$ of final refinements and for each player $i$, we have $\X_i(\bsigma^{F_\l}) = z_i^k$.
\end{invariant}

\begin{claimproof}
    The invariant is immediate for $\l=0$, since we have $F_\l = E_k$.
    Then, if it is true at step $\l$, it remains true at step $\l+1$.
    Indeed, Condition~\ref{itm:nolessterminals} guarantees that the set of terminal vertices accessible from $v_0$ in $(V, F_\l)$ is the same as in $(V, F_{\l+1})$.
    In other words, the terminal vertices that are reached with positive probability in $\bsigma^{F_\l}$ and $\bsigma^{F_{\l+1}}$ are the same.
    Moreover, \cref{inv:terminalsaccessible} guarantees that it is almost sure that some terminal vertex will be reached, in $\bsigma^{F_\l}$ as well as in $\bsigma^{F_{\l+1}}$.
    Therefore, the set of payoff vectors that have positive probability to be obtained is the same in both strategy profiles, hence the risk measures are the same.
\end{claimproof}


We can now prove correctness.
To do so, we need to prove two propositions: the algorithm recognises only positive instances, and recognises all of them.


\begin{proposition}
    The algorithm recognises only positive instances.
\end{proposition}

\begin{claimproof}
        Let us assume that the algorithm answers $\Yes$ at step $\l$ of the final refinements, after having switched to the final refinements loop at step $k$: let us show that the strategy profile $\bsigma^{F_\l}$ is an XRSE that satisfies the desired constraints.
        Note that the algorithm does never answer $\Yes$ at step $0$, hence we necessarily have $k \geq 1$.

 \subparagraph*{The strategy profile $\bsigma^{F_\l}$ satisfies $\bx \leq \X({\bsigma^{F_\l} }) \leq \by$.}
        
        The algorithm switches to the final refinements at step $k$ only if the strategy profile $\bsigma^{F_\l}$ satisfies $\X(\bsigma^{E_k}) \geq \bx$.
        Then, by \cref{inv:finishingtouchesconstantpayoff}, we also have $\X(\bsigma^{F_\l}) \geq \bx$.
        
            
        As for the upper bound, observe that the set $E_1$ has been defined so that the set $\Attr(V_{\frownie}^0, E)$, and therefore the set $V_{\frownie}^0$, is not accessible from $v_0$ in the graph $(V, E_1)$, and therefore not in the graph $(V, F_\l)$ either.
        Thus, it is almost sure in $\bsigma^{F_\l}$ that no vertex of $V_{\frownie}^0$ will ever be reached.
        In other words, all terminals that have positive probability to be reached give to each player $i$ a payoff smaller than $y_i$.
        That is sufficient to prove the lower bound, because it is almost sure, when following $\bsigma^{F_\l}$, that some terminal vertex will eventually be reached, by \cref{inv:terminalsaccessible}.
            

        \subparagraph*{The strategy profile $\bsigma^{F_\l}$ is an XRSE.}
        
        Let $i$ be a player, and let $\sigma'_i$ be a deviation of player $i$ from $\bsigma^{F_\l}$.
        %Since the strategy profile $\bsigma^{F_\l}_{-i}$ is positional, we can now assume without loss of generality that $\sigma'_i$ is positional.        
        Let $z' = \X_i({\bsigma^{F_\l}_{-i}, \sigma'_i})$ be the extreme risk measure obtained by player $i$.
        We want to prove that the deviation $\sigma'_i$ is not profitable, i.e., that we have $z' \leq z_i^k$ (since we have $\X_i(\bsigma^{E_\l}) = z_i^k$ by \cref{inv:finishingtouchesconstantpayoff}).
        To do so, we first show that player $i$ cannot obtain a payoff better than $z_i^k$ after using an edge that does not belong to $F_\l$.

        We first show that if the deviation $\sigma'_i$ uses only edges of $F_\l$, then it cannot be profitable.

\begin{claim}\label{claim:finishingtouchescycleimpossible}
    If it is almost sure, when following $(\bsigma^{F_\l}_{-i}, \sigma'_i)$, that only edges of $F_\l$ will be used, then the deviation $\sigma'_i$ is not profitable.
\end{claim}

\begin{claimproof}
    First, let us note that as long as player $i$ uses only edges that belong to $F_\l$, the strategy profile $\bsigma^{F_\l}$ behaves in a stationary way, and we can therefore assume without loss of generality that $\sigma'_i$ is positional.

    The payoff $z'$ may be obtained by reaching a terminal vertex: in that case, that terminal vertex is accessible from $v_0$ in $(V, F_\l)$, and therefore also reached with positive probability when following the strategy profile $\bsigma^{F_\l}$, hence $z' \leq z_i^k$.

    Let us show that it cannot be obtained by reaching no terminal.
    We proceed by contradiction: if, in the strategy profile $(\bsigma^{F_\l}, \sigma'_i)$, there is a positive probability of reaching no terminal when following that strategy profile, then there is a vertex that has positive probability of being visited infinitely often.
    We can then define the set $W$ of such vertices, i.e., the set $W = \{v \in V \mid \prob_{\bsigma^{F_\l}, \sigma'_i}(v \in \Inf) > 0\}$.
    Thus, when the strategy profile $(\bsigma^{F_\l}, \sigma'_i)$ is followed from a vertex of $W$, it is almost sure that no terminal vertex is reached, and that the set $W$ will never be left.
    We can then choose $w \in W$ such that it has positive probability of being reached without visiting any other vertex of $W$ before, i.e., such that there exists a history $hw$ from $v_0$ with $\Occ(h) \cap W = \emptyset$ (note that $h$ can be empty).
    
    On the other hand, in the graph $(V, F_\l)$, there is at least one terminal vertex accessible from $w$: all vertices from which no terminal is accessible are made themselves inaccessible at odd steps, the switch to final refinements loop happens only if there is no more edge to remove in that perspective, and Condition~\ref{itm:nocycle} guarantees that the final refinements loop leave at least one terminal vertex accessible from every vertex accessible from $v_0$.

    From each terminal $t$ accessible from $w$, we pick a simple path $h_0^t \dots h_{q_t}^t$ from $h_0^t = w$ to $h_{q_t}^t = t$ in the graph $(V, F_\l)$.
    Those paths define a directed acyclic graph (DAG) $D = (V_D, E_D)$ rooted at $w$, where all non-terminal vertices have at least one outgoing edge, with $V_D \subseteq V$ and $E_D \subseteq F_\l$.
    Now, since $\sigma'_i$ guarantees that no terminal vertex will be reached, each branch $h^t$ of that DAG is such that there exists a (smallest) index $j$ with $h^t_j \in W \cap V_i$, and $\sigma'_i(h^t_j) \neq h^t_{j+1}$.
    It may be the case that $h^t_j \sigma'_i(h^t_j) \in E_D$, i.e., that from $h^t_j$, player $i$ proceeds to an undetectable deviation and takes another branch of the DAG.
    But that cannot be the case for all $t$, otherwise, there would be a branch $h^t$ that would be followed with positive probability when following $(\bsigma^{F_\l}_{-i}, \sigma'_i)$ from $w$, and therefore a terminal vertex $t$ that would be reached with positive probability, which contradicts the definition of $w$.

    There must therefore exist an edge $uv \in F_\l \setminus E_D$, with $u \in V_D \cap V_i \cap W$
    We will show that such an edge should have been removed during the final refinements loop.
    First, it immediately satisfies Condition~\ref{itm:cuttableedge}.
    Moreover, the vertex $u$ is necessarily on a branch $h^t$ of $D$ that leads to a terminal vertex $t$, hence it satisfies Condition~\ref{itm:nocycle}.
    Finally, since $v$ is accessible from $w$ in $(V, F_\l)$, the terminal vertices that are accessible from $v$ in $(V, F_\l)$ are all accessible from $w$ in that same graph, and therefore are accessible from $w$ in the DAG $D$.
    Since $w$ is accessible from $v_0$ without visiting any vertex of $W$, and it particular without visiting $u$, it means that the terminal vertices accessible from $v$ are also accessible from $v_0$ without using the edge $uv$.
    In other words, the edge $uv$ satisfies Condition~\ref{itm:nolessterminals}, and should have been removed during the final refinements.

    This case is therefore impossible: when the players follow the strategy profile $(\bsigma^{F_\l}, \sigma'_i)$, it is almost sure that some terminal vertex will be reached, and that concludes the proof.
\end{claimproof}
        
But now, if the strategy $\sigma'_i$ does use edges that do not belong to $F_\l$, let us show that it cannot be a profitable deviation either.

\begin{claim}
    If there is a history $hv$ compatible with $\bsigma^{F_\l}$ such that $v\sigma'_i(hv) \not\in F_\l$, then we have $\X_i(\bsigma^{F_\l}_{\|hv}, \sigma'_{i\|hv}) \leq z_i^k$.
\end{claim}

\begin{claimproof}
    After such a history, the strategy profile $\bsigma^{F_\l}_{-i}$ follows the positional strategy profile $\btau^{\dag i}_{-i}$.
    By definition of that strategy profile, we have $\X_i(\bsigma^{F_\l}_{\|hv}, \sigma'_{i\|hv}) \leq \val(v)$.
    On the other hand, the vertex $v$ is accessible from $v_0$ in $(V, F_\l)$, since it is visited with positive probability in $\bsigma^{F_\l}$.
    Therefore, it does not belong to the set $A_k$ (if $k$ is even) or $A_{k-1}$ (if $k$ is odd), and in particular not to the set $V_\bad^k$ or $V_\bad^{k-1}$, which means that we have $\val(v) \leq z_i^k$.
    Hence the conclusion.
\end{claimproof}



In the general case, the payoffs that player $i$ obtains with positive probability in the strategy profile $(\bsigma^{F_\l}_{-i}, \sigma'_i)$ are either obtained using only edges that belong to $F_\l$, or by using an edge that does not: in both cases, we have shown that player $i$ cannot get a payoff greater than $z_i^k$, which proves that the strategy profile $\bsigma^{F_\l}$ is an XRSE.
\cref{algo:cyclefriendly} answers $\Yes$ only on positive instances, and outputs in that case a succinct representation of an XRSE matching the constraints.
\end{claimproof}


We will now prove the converse.

            \begin{proposition}
                \cref{algo:cycleaverse} recognises all positive instances. 
            \end{proposition}
    
            \begin{claimproof}
           Let us assume that we have a positive instance, i.e., that there exists an XRSE $\bsigma$ with $\bx \leq \X(\bsigma) \leq \by$.
        Let us show that the algorithm will answer $\Yes$.
        To do so, we first prove the following invariant: if an edge is removed at some step \emph{before the final refinements}, then it is never taken by the XRSE $\bsigma$.

        \begin{invariant} \label{inv:edgesnotused_bis}
            For each $k \geq 0$, every edge that has positive probability to be eventually taken in $\bsigma$ belongs to $E_k$.
        \end{invariant}

        \begin{claimproof}
        We prove the invariant by induction. 
        
        \subparagraph*{Base case.} The case $k=0$ is immediate, since we have $E_0 = E$.
        
        Further, at step $k=1$, if the strategy profile $\bsigma$ uses eventually, with positive probability, an edge that does not belong to $E_1$, then it goes with positive probability to a vertex $v \in \Attr(V^0_\bad, E_0)$.
        Then, with positive probability, a terminal vertex will be reached that gives to some player $i$ a payoff greater than $y_i$, which is impossible.
        Therefore, such an edge cannot be taken in $\bsigma$.
        
        
        \subparagraph*{Induction step.} Let us assume that the invariant is true until step $k \geq 1$, and let us show that it holds at step $k+1$.    
        Let $uv$ be an edge that is used with positive probability when following $\bsigma$, and let us assume toward contradiction that it does not belong to $E_{k+1}$.
        Since the invariant is true at each step until $k$, we can assume that $uv$ has been removed at step $k$, i.e., that we have $uv \in E_k \setminus E_{k+1}$.
        Then, we have $u \not\in A_k$ and $v \in A_k$.
        The strategy profile $\bsigma$ has therefore positive probability of visiting the set $A_k$.
        
        We must now distinguish the cases where $k$ is even or odd.
        If $k$ is even, then the strategy profile $\bsigma$ has therefore positive probability of visiting a vertex $v \in V_\bad^k$.
        If player $i$ is the player controlling $v$, then that player has a deviation in which they get risk measure at least $\val(v) > z_i^k$.
        Let us now note that when the players follow the strategy profile $\bsigma$, it is almost sure that a terminal vertex will eventually be reached, and that all the terminal vertices that have positive probability of being reached also have positive probability of being reached in $\bsigma^{E_k}$, since the invariant is true at step $k$: therefore, we have $z_i^k \geq \X_i(\bsigma)$, and player $i$ has a profitable deviation after $v$, which contradicts the fact that $\bsigma$ is an XRSE.

        If $k$ is odd, then, by definition of $A_k$, when the strategy profile $\bsigma$ is followed, there is a positive probability of reaching no terminal.
        But that is impossible in the cycle-averse case.

        The invariant is therefore necessarily still true at step $k+1$.
\end{claimproof}

        

        We are now able to conclude.
        The answer $\No$ can be given in the two following cases:
            \begin{itemize}
                \item \emph{If at step $k$, we have $v_0 \in \Attr(V^k_\bad, E_k)$.}
                Then, the strategy profile $\bsigma$ visits the set $V^k_\bad$ with positive probability.
                With the same arguments that were used in the proof of \cref{inv:edgesnotused_bis}, that is not possible.
                
                \item \emph{If during steps $k-1$ and $k$, no edge is removed, but we have $z^k_i < x_i$ for some player $i$.}
                Since $\bsigma$ uses only edges of $E_k$ by \cref{inv:edgesnotused_bis} and reaches almost surely a terminal (since we are in the cycle-averse case), we have $\X_i(\bsigma) \leq z^k_i$, and therefore $\X_i(\bsigma) < x_i$: that case is therefore also impossible by definition of $\bsigma$.
            \end{itemize}
        None of those cases is possible, hence our algorithm will eventually answer $\Yes$.
    \end{claimproof}


\paragraph*{Complexities.}
We consider here the complexity of the two algorithms.
Since at least one edge is removed every two steps, there are $\Oh(m)$ steps.
In each of them, we need $\Oh(p)$ calls to simple algorithms: computation of $z_i^k$, of $V_\bad^k$, of $A_k$.
Hence, the complexity $\Oh(pm^2)$.

In the cycle-averse case, when an output is asked, we need to add the final refinements loop: which consist of $\Oh(m)$ additional steps in which we check, for each of the $\Oh(m)$ remaining edges, whether they satisfy Conditions~\ref{itm:cuttableedge},~\ref{itm:nolessterminals}, and~\ref{itm:nocycle} in the finishing touches), which can be done in time $\Oh(m)$.
Hence the complexity $\Oh(pm^2 + m^3)$.
\end{proof}





\subsection{Proof of \cref{lm:ptimelowerbound}}\label{app:ptimelowerbound}
\ptimelowerbound*
\begin{proof}[Proof of \cref{lm:ptimelowerbound}]
        Given a deterministic two-player (between players $\Circle$ and $\Square$) zero-sum reachability game $\Game_{\|v_0}$ with target set of vertices $T$, we construct a simple stochastic game (with no stochastic vertices) where there is an XRSE $\bsigma$ satisfying $\X_\circ(\bsigma) = 1$ and $\X_\Box(\bsigma) = -1$ if and only if player $\Circle$ wins the game.  

        The game is simply obtained by assigning rewards on the zero-sum two player game as follows: we make all nodes in the target set $T$ of the reachability game as a terminal node where player $\Circle$ gets reward $1$ and player $\Square$ the reward $-1$. Recall that if no terminal is reached, both players get reward $0$.  

        If $\Circle$ has a strategy to win the reachability game, it is easy to see that the same strategy for $\Circle$, along with any strategy for $\Square$, will be an XRSE in that new game, and that it satisfies the constraint.
        Similarly, if on the other hand, player $\Square$ has a strategy to avoid the states $T$, then no strategy of $\Circle$ that gives her payoff $+1$ and gives player $\Square$ the payoff $-1$ will be an equilibrium, since $\Square$ can always deviate to the winning strategy in the reachability game that offers him the better payoff of $0$.
\end{proof}
