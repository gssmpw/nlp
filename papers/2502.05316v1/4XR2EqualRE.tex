We show that our definition of extreme risk measure corresponds to the limit cases of entropic risk measure.
Observe that in \cref{fig:example_re}, following the  blue strategy, the only payoffs that are obtained with positive probability were $40$ and $0$, which are also the limits of the risk entropy when $\rho$ tends to infinite values.
On the other hand, in the red strategy, the only payoff obtained with positive probability is payoff $1$. Although the payoff $0$ is possible since the play $a^\omega$ is compatible with every strategy, this play must be ignored since it is realised with probability $0$.

\begin{restatable}[App.~\ref{app:RE=PEorOE}]{theorem}{REisPEOE}\label{thm:RE=PEorOE}
    Let $X$ be a random variable that ranges over $\Rb$, and let $\beta > 1$.

    \begin{itemize}
        \item The limit risk entropy of $X$ when $\rho$ tends to $+\infty$ exists and is equal to the pessimistic risk measure, that is, we have $\lim_{\rho \to +\infty} \re_{\beta\rho} [X] = \pexp[X]$.    
       \item Similarly, the limit risk entropy of $X$ when $\rho$ tends to $-\infty$ exists and is equal to the pessimistic risk measure, that is, we have $\lim_{\rho \to -\infty} \re_{\beta\rho} [X] = \oexp[X]$.
    \end{itemize}
\end{restatable}
%We henceforth refer to the perceived reward computed according to the risk-measure as $\xr$ to refer to $\pexp$ or $\oexp$ based on if the player is a pessimist or an optimist.\theju{I am not a fan of expectation necessarily in this context. Would rather call it extreme pessimistic risk or something}