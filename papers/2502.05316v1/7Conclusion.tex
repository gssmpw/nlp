Our definition opens up several promising directions for future research.
%With the tractability of equilibria in our extreme-risk measure, future work could address the limitations inherent in the equilibria for games.
One immediate extension of our work would be to study games with more sophisticated objectives, such as mean payoff or discounted sum. 
Another extension of our work is to study the concurrent version of such games, where players choose actions concurrently rather than in a turn-based setting. Concurrent stochastic multi-player games on stochastic graphs have traditionally been viewed as intractable, often requiring infinite memory to achieve optimal strategies, which has limited exploration of multi-player versions of the same problem. 

Finally, our definition of risk-sensitive equilibria is modelled after Nash equilibria and suffers from several of their limitations.
Like Nash equilibria, RSEs allow irrational behaviours when one player deviates and must be punished, %it does not account for rationality in scenarios where a player might avoid a worse action in a subgame.
as exemplified in our game \cref{fig:ex_extreme3}.
Exploring alternative definitions of risk-sensitive equilibria that are modelled after other equilibria concepts more suited for games on graphs~\cite[Section~7.1]{Osb04}, such as subgame-perfect equilibria, could provide a more rational framework for player decision-making.

%Although computing these equilibria is often at least as computationally expensive as Nash equilibria, our framework’s reduced complexity could make this an attractive direction to pursue.
% Specifically, the current definition of risk-sensitive equilibria, inspired by Nash equilibria, inherits similar shortcomings as it fails to account for rationality in scenarios where a player might avoid a decisive action in a subgame. Investigating extensions of the current definition of RSE to notions such as subgame-perfect equilibria, could provide a more rational framework for decision-making. These equilibria have been studied in the context of multi-player games on graphs~\cite{}. While computing subgame-perfect equilibria is often at least as computationally hard as Nash equilibria, our computationally easier-to-handle definitions risk-measure makes this an intriguing direction to explore.

% Another potential avenue involves studying concurrent versions of such games. Concurrent stochastic multi-player games on stochastic graphs have traditionally been viewed as intractable, leading to a lack of focus on the associated computational problems. Addressing these challenges could open up significant new possibilities.

% We introduced extreme optimism and extreme pessimism and  studied the concept of risk-sensitive equilibria when measuring risk using the entropic risk measure. We proved existence of risk-sensitive equilibria using extreme risk measure or using entropic risk measure only in cases where rewards were positive. The results where the rewards are negative is still an open problem, as it is for entropic risk-sensitive equilibria, and even for Nash equilibria.
% %. More fundamentally, we do not know if Nash equilibria exist in simple stochastic games where the terminal rewards are rational numbers. 

% Our definition opens several exciting avenues of future research. 
% Given the tractability of equilibria in our extreme-risk measure, we can focus future work on overcoming the  downsides equilibria on games suffer. Our current definition of risk-sensitive equilibria, modelled after Nash's equilibria, suffers from the same pitfalls since it does not take into account rationality of each player that might not do an action that is determinant to it in a subgame. Studying concepts such as subgame perfect equilibria offers a more rational framework for such players. Such equilibria have been studied for multi-player games on graphs.~\cite{}.  Computing subgame-perfect equilibria is usually computationally at least as expensive as Nash equilibria, but with our reduced computational cost, this might be an interesting avenue to consider.
% Another direction would be to study the concurrent version of such games. Concurrent stochastic multi-player games played on stochastic graphs were considered intractable and  therefore related computation problems have not been studied. %With newfound tractability in our setting, we could consider %In fact, it was explicitly left as an open problem to  leaving the study of concurrent games as an open problem in the work of Ummels and Wojtczak~\cite{UW11}.equilibria in two player settings. W

