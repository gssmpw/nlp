\subsection{Proof of \cref{lemma:np_hardness}}\label{app:np_hardness}
\NPHard*
\begin{proof}[Proof of \cref{lemma:np_hardness}]
  We prove $\NP$-hardness by reducing from the problem $\THREESAT$. Consider a $\THREESAT$ formula $\Phi$, over the variables  $x_1,\dots,x_n$, where $\Phi = C_1\land C_2\land \dots\land C_m$, where for each $i$ we have $C_i = (\ell_{i1}\lor \ell_{i2}\lor \ell_{i3})$ and for $j=1,2,3$, we have $\ell_{ij} = x_k$ or $\ell_{ij} = \neg x_k$ for some $k\in \{1, \dots, n\}$. 
  We construct  a game $\Game_\Phi$ with two players for each literal $\ell$, denoted by $\Circle \ell$ and $\Square \ell$. The game is depicted in \cref{fig:NPhard}.
  For convenience, some terminal vertices have been represented several times.
  Each player $\Circle \ell$ controls one vertex, the vertex $\Circle \l$, of circled shape, and symmetrically, each player $\Square \ell$ controls the square-shaped vertex $\Square \l$.
  Further, we add a player $C_i$, who controls the vertex $C_i$, for each clause $C_i$. Finally, there is a player $\Diamond$ who does not control any vertex. There are also stochastic vertices, that are represented by the black circles.
  In each terminal vertex, the symbol $\forall$ should be understood as "every (other) player".

%The edges between the players are as in \cref{fig:NPhard}. From each clause $C_i$, we add edges to terminal vertices where the payoff of the player $\Circle \ell$ is $0$ and everyone else's is 2 if and only if $\ell$ is in the clause $C_i$. In the example, we assume $C_2 = x_2\lor x_4\lor \neg x_{11}$ and only draw edges from $C_2$ and not from other $C_i$. 

 We assume all players are pessimistic, and ask if there is an XRSE where player $\Diamond$'s risk measure is exactly $2$.
We give the formal definition of the game $\Game_\Phi$ below.

    \begin{figure}
        \centering
        
        \begin{tikzpicture}[shorten >=1pt, node distance=1.5cm and 2cm, on grid, auto, scale=1.1]
          %every node/.style={scale=0.6}
          % Smaller state style
          \tikzstyle{state}=[circle, draw, minimum size=20pt, inner sep=1pt]
          \tikzstyle{squarestate}=[rectangle, draw, minimum size=20pt, inner sep=1pt] 

            \node[state] (qnc1) at (0, 0) {$\neg x_{1}$};
            \node[state, initial,initial text=] (qc1) at (0, 1) {$x_{1}$};

            \node[scale=0.6] (tpunish1) at (0,-1) {$t_\dag:~\stack{\forall}{0}$};
            
            \node[stoch, scale=0.6] (stoc2) at (1, 0) {$s_{\neg x_1}$};
            \node[stoch, scale=0.6] (stoc1) at (1, 1) {$s_{x_1}$};

            \node[scale=0.6] (reward1) at (1,2) {$f_{x_1}:~\stack{\circ x_1}{1}$$\stack{\forall}{2}$};
            \node[scale=0.6] (reward2) at (1,-1) {$f_{\neg x_1}:~\stack{\circ \neg x_1}{1}$$\stack{\forall}{2}$};
            
            \node[squarestate] (qns1) at (2, 1) {$\neg x_{1}$};
            \node[squarestate] (qs1) at (2, 0) {$x_{1}$};

            \node[scale=0.6] (punishW1) at (2,2) {$t_\diamond:~\stack{\diamond}{0}$$\stack{\forall}{2}$};
            \node[scale=0.6] (punishW2) at (2,-1) {$t_\diamond:~\stack{\diamond}{0}$$\stack{\forall}{2}$};
            
            \node[state] (qnc2) at (3.5, 0) {$\neg x_{2}$};
            \node[state] (qc2) at (3.5, 1) {$x_{2}$};
            \node[stoch, scale=0.6] (stoc3) at (4.5, 1) {$s_{x_2}$};
            \node[stoch, scale=0.6] (stoc4) at (4.5, 0) {$s_{\neg x_2}$};

            \node[scale=0.6] (reward3) at (4.5,2) {$f_{x_2}:~\stack{\circ x_2}{1}$$\stack{\forall}{2}$};
            \node[scale=0.6] (reward4) at (4.5,-1) {$f_{\neg x_2}:~\stack{\circ \neg x_2}{1}$$\stack{\forall}{2}$};
            
            \node[squarestate] (qns2) at (5.5, 1) {$\neg x_{2}$};
            \node[squarestate] (qs2) at (5.5, 0) {$x_{2}$};

            \node[scale=0.6] (punishW3) at (5.5,2) {$t_\diamond:~\stack{\diamond}{0}$$\stack{\forall}{2}$};
            \node[scale=0.6] (punishW4) at (5.5,-1) {$t_\diamond:~\stack{\diamond}{0}$$\stack{\forall}{2}$};

            \node[scale=0.6] (tpunish2) at (3.5,-1) {$t_\dag:~\stack{\forall}{0}$};
            \node (qc3) at (6.5, 1) {};

            \node (dots) at (6.8,0.5) {$\dots$};

            \node[squarestate, initial,initial text=] (qnsn) at (8, 1) {$\neg x_{n}$};
            \node[squarestate, initial,initial text=] (qsn) at (8, 0) {$x_{n}$};

            \node[scale=0.6] (punishW5) at (8,2) {$t_\diamond:~\stack{\diamond}{0}$$\stack{\forall}{2}$};
            \node[scale=0.6] (punishW6) at (8,-1) {$t_\diamond:~\stack{\diamond}{0}$$\stack{\forall}{2}$};
            
            \node[stoch, scale=0.6] (stochfin) at (9,0.5) {$s_\mathsf{r}$};
            \node (fakenode) at (9,0.5) {};
            
            \node (c1) at (10,1.8) {$C_1$};
            \node (c2) at (10,1) {$C_2$};
            \node (cdots) at (10,0.4) {$\vdots$};
            \node (c3) at (10,-0.6) {$C_m$};

            \node[scale=0.6] (ter1) at (11.2, 1.7) {$t_{x_2}:~\stack{\Box x_2}{1}$ $\stack{\forall}{2}$};
            \node[scale=0.6] (ter2) at (11.2, 1) {$t_{x_4}:~\stack{\Box x_4}{1}$ $\stack{\forall}{2}$};
            \node[scale=0.6] (ter3) at (11.2, -0.3) {$t_{\neg x_{11}}:~\stack{\Box\neg x_{11}}{1}$ $\stack{\forall}{2}$};

          \path[->]
              (qc1) edge (stoc1)
              (qc1) edge (qnc1)
              (qnc1) edge (stoc2)
              (stoc1) edge (qns1)
              (stoc2) edge (qs1)
              (qns1) edge (qc2)
              (qs1) edge (qc2)
              (qc2) edge (stoc3)
              (qc2) edge (qnc2)
              (qnc2) edge (stoc4)
              (stoc3) edge (qns2)
              (stoc4) edge (qs2)
              (qns2) edge (qc3)
              (qs2) edge (qc3)
              (qnsn) edge (stochfin)
              (qsn) edge (stochfin)
              (fakenode) edge (c1)
              (fakenode) edge (c2)
              (fakenode) edge (c3)
              (c2) edge (ter1)
              (c2) edge (ter2)
              (c2) edge (ter3)
              (qnc1) edge (tpunish1)
              (qnc2) edge (tpunish2);

        \path[->]
            (stoc1) edge (reward1)
            (stoc2) edge (reward2)
            (stoc3) edge (reward3)
            (stoc4) edge (reward4)
            (qs1) edge (punishW2)
            (qns1) edge (punishW1)
            (qs2) edge (punishW4)
            (qns2) edge (punishW3)
            (qsn) edge (punishW6)
            (qnsn) edge (punishW5);
        
        \end{tikzpicture}
        \caption{Construction of a game $\Game_\Phi$ from a $\THREESAT$ formula $\Phi$}
        \label{fig:NPhard}
    \end{figure}


% We will construct a game $\Game_\Phi$ with $O(n+m)$-players, constraints $\Bar{x},\Bar{y}$ and where at least $2n$ players are pessimists, such that the game has a $\Bar{\rho}$-RSE if and only if $\Game_\Phi$ is satisfiable. 

\subparagraph*{Construction of the game $\Game_\Phi$: vertices, edges and payoffs.}
For each literal $\ell$, we define two players $\Square\ell$ and $\Circle\ell$. We add one other player $C_i$ for each clause $C_i$, and an additional constraining player $\Diamond$.
All players are pessimists.

Each player owns at most one vertex in the game, and therefore, we will refer to the player and vertex interchangeably. There is one vertex for each of the players mentioned above other than $\Diamond$, who owns no vertices. Further, there are $2n + 1$ many stochastic vertices: one for each literal $s_{x_1},s_{x_2},\dots,s_{x_n}$,  $s_{\neg x_1},s_{\neg x_2},\dots,s_{\neg x_n}$, and finally one clause-randomiser $s_\mathsf{r}$. 
There are also $2n + 2$ terminal vertices, written $f_{\ell}$ and $t_{\ell}$ for each literal $\ell$, and further the terminal vertices $t_\Diamond$ and $t_\dag$.

We now define the edges between the vertices of the graph for all $i\in \{1, \dots, n\}$:  there are edges from $\Circle x_i$ to $\Circle\neg x_i$, and edges from $\Circle\neg x_i$ to $t_\dag$.
    Further, for every literal $\ell = x_i$ or $\neg x_i$, there are edges:
    \begin{itemize}
        \item from $\Circle\ell$ to $s_{\ell}$;
        \item from $s_{\ell}$ to $f_\ell$ and to $\Square\Bar{\ell}$, where $\Bar{\ell} = \neg x_i$ if $\ell = x_i$ and $\Bar{\ell} = x_i$ if $\ell = \neg x_i$;
        \item from $\Square\ell$ to $t_\Diamond$;
        \item from $\Square\ell$ to $\Circle x_{i+1}$ if  $i<n$,  and  to $s_\mathsf{r}$ if  $i=n$.%, edges are added
    \end{itemize}
    Finally, for all clauses $C_j$, there are edges from $s_\mathsf{r}$ to $C_j$ and from $C_j$ to $t_{\ell}$ such that $\ell$ occurs positively in the clause $C_j$.

    The terminal vertices yield the following payoffs.
\begin{itemize}
    \item In terminal $t_\ell$, all players get payoff $2$, except the player $\Square \ell$ who gets payoff $1$. 
    \item In terminal $f_\ell$, all players get payoff $2$, except player $\Circle \ell$ who gets payoff $1$.
    \item In terminal $t_\dag$, all players get payoff $0$.
    \item In terminal $t_\Diamond$, all players get payoff $2$, except player $\Diamond$ who gets payoff $0$.% , and player $\Diamond$ get payoff $0$. 
\end{itemize}    

Finally, we let the constraints be that player $\Diamond$ gets a risk measure of exactly $2$.
Equivalently, we define $\bx$ and $\by$ by $\by = (2)_{i \in \Pi}$, $x_i = 0$ for each $i \in \Pi \setminus \{\Diamond\}$, and $x_\Diamond = 2$.

% and $\Square x_i$

% All players in the game are pessimistic players    \theju{Need to detail who needs to be pessimistic.}
    % We draw an example formula with 3 variables and two clauses. 

    % \begin{example}
    %     Consider clause $C_1 = (x_1\lor \neg x_2\lor x_3)$, and $C_1 = (\neg x_1\lor  x_2\lor \neg x_3)$
    %         \thejaswini{To do add an example}
    %         \leonard{Is that really necessary? Your figure above is pretty clear (and clearly pretty).}
    % \end{example}
% We first make a claim whose proof is straight forward and can be shown by the definition of RSE, and using the fact that player $\Square\ell$ is a pessimistic player.

% \begin{claim}
%     If vertex $\Square \ell$ is visited in an RSE, then terminal $t_\ell$ must be visited probability $0$. Similarly, if terminal $t_\ell$ is visited with non-zero probability, then the strategy 
% \end{claim}

    \subparagraph*{If $\Phi$ is satisfiable, then there is an XRSE satisfying the constraints.}
    Consider a satisfying assignment of the $\THREESAT$ formula, described by the assignment $\alpha$ from the set of all variables to $\{\top,\bot\}$.
    
    For each $i$, let $\ell_i$ denote the literal, among $x_i$ and $\neg x_i$, which is set to true by 
    the satisfying assignment $\alpha$.
    Let us define the (positional) strategy profile $\bsigma^\alpha$.
    
    \begin{itemize}
        \item Player $\Circle \ell_i$ goes to $s_{\ell_i}$.
        \item Player $\Circle x_i$ goes to $s_{x_i}$ if $\alpha(x_i) = \top$, and to $\Circle \neg x_i$ otherwise.
        \item Player $\Circle\neg x_i$ goes to $s_{\neg x_i}$ if $\alpha(x_i) = \top$ and to $t_\dag$ otherwise.
        \item For each player $\Square \ell$, the strategy is to chose the edge that does \emph{not} lead to $t_\Diamond$. That is, the edge to $\Circle x_{i+1}$ if $\ell = x_i$ or $\neg x_{i}$ and  $i<n$,  and  the edge to $s_\mathsf{r}$ if  $i=n$.
        \item Each clause player $C_i$ takes the edge to the vertex $\ell_j$ such that the litteral $\ell_j$ was set to true by the satisfying assignment $\alpha$. 
    \end{itemize}
     We now show that this is an XRSE that satisfies the constraint. First, we verify if the constraints are satisfied. Observe that following the strategy profile $\bsigma^\alpha$, it is almost sure that none of the terminals where player $\Diamond$ has payoff less than $2$ will be reached. Therefore this satisfies the constraints. 

     We now argue that $\bsigma^\alpha$ is an XRSE, i.e. that no player can get a better risk measure by deviating.
     The result is immediate for player $\Diamond$ and for the clause players, who all get risk measure $2$, the best they could hope for.
     
     For each literal $\l$, player $\Circle\ell$ gets risk measure $1$ if $\l$ is set to true, and risk measure $2$ if $\ell$ is set to false.
     The same argument as above holds therefore in the second case.
    In the first case, they get risk measure $0$, but they have no profitable deviation, since the only deviation available leads to $t_\dag$ and to the payoff $0$.
     
     Player $\Square \ell$  has also risk measure  $2$ when $\ell$ is set to false.
     Otherwise, they get payoff $1$. In that second case, the vertex owned by the player is not visited in any history of the game, hence they have no possibility of deviating.

     The (positional) strategy profile $\bsigma^\alpha$ is therefore an XRSE.
     
    \subparagraph*{If there is an XRSE satisfying the constraints, then $\Phi$ is satisfiable.} 
    Let us assume that there exists an XRSE $\bsigma$ in the game $\Game_\Phi$, such that player $\Diamond$ gets the risk measure $2$.
    We prove, first, that we can assume that $\bsigma$ is pure (and therefore positional, since there is then only one history leading to each vertex).

\begin{claim}
    There exists an XRSE $\bsigma^\star$ in $\Game_{\|v_0}$ where player $\Diamond$ gets risk measure $2$ that is positional.
\end{claim}

\begin{proof}
    Let us first focus on what happens in vertices that have positive probability of being reached.

    If the vertex $\Circle\neg x_i$ has a positive probability of being reached in $\bsigma$, then any strategy of the player $\Circle \neg x_i$ that goes to $t_\dag$ with positive probability gives the player $\Diamond$ the risk measure $0$.
    Therefore, necessarily, the strategy $\sigma_{\circ \neg x_i}$ consists of deterministically going to $s_{\neg x_i}$.
    The same argument holds for the vertices of the form $\Square \l$.
    
    If now the vertex $\Circle x_i$ has a positive probability of being reached and if the player $\Circle x_i$ randomises between the two edges available, then she gets the risk measure $1$, since the terminal vertex $f_{x_i}$ is reached with positive probability and $t_\dag$ with probability zero.
    But then, if she deviates and goes to the vertex $\Circle \neg x_i$ with probability $1$, she avoids the terminal vertex $f_{x_i}$, and the other players will not react since they do not detect the deviation.
    She therefore gets the risk measure $2$, and the deviation is profitable.
    Consequently, the strategy $\sigma_{\circ x_i}$ can only  deterministically select one of those two edges.

    At the end of the game, for each $j$, the player $C_j$ could play a randomised strategy. In such a case, her strategy can be replaced by a pure strategy that takes, deterministically, one of the edges that she was previously taking.
    Such a modification in her strategy can only increase the risk measure of some players (namely, those of the form $\Square \l$) without impacting player $\Diamond$'s risk measure or giving any player the possibility of profitably deviating.

    Finally, if one of those vertices is reached after a history that is not compatible with $\bsigma$, i.e. if one of those players deviates: it is necessarily due to a deviation of a player of the form $\Circle x_i$, since any other deviation would immediately lead to a terminal vertex.
    If she went to $s_{x_i}$ instead of $\Circle \neg x_i$, what the other players do afterwards does not matter, since such a deviation cannot be profitable: with positive probability, the terminal vertex $f_{x_i}$ is reached, and she gets payoff $1$.
    If she went to $\Circle x_i$ instead of $s_{x_i}$, then we can assume that player $\Circle \neg x_i$'s strategy consists of going to the terminal vertex $t_\dag$, giving her the payoff $0$.
    Those modifications do not impact the fact that $\bsigma$ is an XRSE.
\end{proof}
    % If the vertex $\Circle\neg x_i$ has a positive probability of being reached, then any strategy of player $\Circle \neg x_i$ that goes to $t_\dag$ with positive probability gives player $\Diamond$ the risk measure $0$.
    % Therefore, necessarily, the strategy $\sigma_{\circ \neg x_i}$ consists in deterministically going to $s_{\neg x_i}$.
    % A consequence of that fact is that player $\Circle x_i$, who can get the payoff $0$ only in the terminal vertex $t_\dag$, gets in the strategy profile $\bsigma$ a risk measure greater than or equal to $1$.

    % If now the vertex $\Circle x_i$ has positive probability to be reached, then player $\Circle x_i$ is also necessarily playing a pure strategy: if she randomises between the two actions available, then we argue that there is an undetectable deviation possible at the vertex $\Circle x_i$. Player $\Circle x_i$ would always prefer the edge to $\Circle\neg x_i$ since this removes entirely the path that gives the player $\Circle x_i$ the possibility of payoff $0$.
    
    % Since there is no randomisation possible at vertices $\Circle x_i$, to ensure that player $\Diamond$ gets payoff $2$, the terminals $t_\Diamond$ or $t_\dag$ cannot be visited from any of the states $\Square\ell$.  This means that the strategy from every vertex $\Square \ell$ must be to visit $\Circle x_{i+1}$ or $s_\mathsf{r}$ next. However, observe $t_\Diamond$ ensures a payoff of $2$ to player $\Square\ell$. 
    % To ensure that the strategy is an RSE, and that the player $\Square \ell$ does not have an incentive to deviate to this edge, the outcome of the strategy to player $\Square \ell$ should be $2$ (Any lower risk-expectation of player $\Square\ell$ and would ensure deviation). This is true for any vertex $\Square\ell$ that has a positive probability of being visited by the strategy.    
    % This also rules out the possibility of randomisation at vertices $\Square\ell$ for vertices that are visited with positive probability according to the strategy.
    % Since there is no randomisation at both $\Circle\ell$ as well as $\Square\ell$ vertices. If the stochastic vertex $s_\mathsf{r}$ is reached, then it is reached after visiting a unique path (assuming no deviations in the strategy).
% \begin{claim}
%      The vertex $\Square \ell$ is visited by a strategy $\sigma$ that is an RSE and  satisfies the constraint if and only if terminal $t_\ell$ is be visited probability $0$ by $\sigma$.% Similarly, if terminal $t_\ell$ is visited with non-zero probability, then an RSE that satisfies the constraints does not visit vertex $\Square\ell$ with non-zero probability. 
%  \end{claim}
%    This follows by observing that no literal $\ell$ such that  
 % \begin{claimproof}
 %        Let $\sigma$ denote such an RSE. 
 %     Consider the unique path based on strategy $\sigma$ that is used to reach the vertex $s_\mathsf{r}$. This path visits visits either $\Square x_i$ or $\Square\neg x_i$, based on the structure of the graph. We show that if a literal $\Square \ell$ is visited, then $\ell\notin S$. If $\Square\ell$ is visited, then player $\Square\ell$ has an incentive to deviate and instead chose edge $\Square \ell\rightarrow t_\Diamond$, thus making his risk-expectation $2$, but also reducing the risk expectation of player $\Diamond$.
 % \end{claimproof}   

We therefore assume that $\bsigma$ is positional.
Let us now define the assignment $\alpha$ as follows: for each variable $x_i$ we have $\alpha(x_i) = \top$ if $\sigma_{\circ x_i}(\Circle x_i) = s_{x_i}$, and $\alpha(x_i) = \bot$ if $\sigma_{\circ x_i}(\Circle x_i) = \Circle \neg x_i$.
Let then $C_j$ be a clause, and let us prove that it is satisfied by $\alpha$.
Let $t_\l = \sigma_{C_j}(C_j)$.
Then, the player $\Square \l$ gets risk measure $1$ in the XRSE $\bsigma$.
Consequently, the vertex $\Square \l$ is never reached: otherwise, the only play compatible with $\bsigma$ in which player $\Square \l$ gets payoff $1$ would traverse the vertex $\Square \l$, and player $\Square \l$ would have a profitable deviation by going to the terminal vertex $t_\Diamond$.
If that is the case, then the definition of $\alpha$ given above implies that the literal $\l$ is true.

The assignment $\alpha$ satisfies therefore the formula $\Phi$.

\subparagraph*{Conclusion.}
We have defined an instance of the constrained existence problem of XRSEs from an instance of $\THREESAT$ and proved that one is a positive instance if and only if the other is.
This proves the $\NP$-hardness of the constrained existence problem of XRSEs, since the game $\Game_\Phi$ can clearly be constructed in polynomial time.
Moreover, the game $\Game_\Phi$ is such that if an XRSE where player $\Diamond$ gets risk measure $2$ exists, then there also exists such an equilibrium that it positional, which proves also $\NP$-hardness when the players are restricted to pure, stationary or positional strategies.
\end{proof}
