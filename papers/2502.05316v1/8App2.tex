\subsection{Proof of \cref{thm:RE=PEorOE}}\label{app:RE=PEorOE}

\REisPEOE*
\begin{proof}[Proof of \cref{thm:RE=PEorOE}]
    \begin{itemize}

        \item First, let us note that for every $\rho$, we always have $\re_{\beta,\rho}[X] \geq \pexp[X]$.
        Let now $\epsilon > 0$.
        We want to prove that there exists $\rho_0 \in \Rb$ such that for every $\rho \geq \rho_0$, we have $\re_{\beta,\rho}[X] \leq \pexp[X] + \epsilon$.

        Let us first notice that we have:
       \begin{align*}
       \re_{\beta\rho}[X] &= -\frac{1}{\rho} \log_\beta \left( \int_{x \in \Rb} \beta^{-\rho x} \d \prob(X = x) \right)\\
        &= -\frac{1}{\rho} \log_\beta \left( \int_{x \in \Rb} \beta^{-\rho \pexp[X]} \beta^{-\rho (x-\pexp[X])} \d \prob(X = x) \right)\\
        &= \pexp[X] -\frac{1}{\rho} \log_\beta \left( \int_{x \in \Rb} \beta^{-\rho (x-\pexp[X])} \d \prob(X = x) \right)\\
        &= \pexp[X] -\frac{1}{\rho} \log_\beta \Bigg( \int_{x \leq \pexp[X] + \frac{\epsilon}{2}} \beta^{-\rho (x-\pexp[X])} \d \prob(X = x) \\
        &\qquad\qquad\qquad+ \int_{x \geq \pexp[X] + \frac{\epsilon}{2}} \beta^{-\rho (x-\pexp[X])} \d \prob(X = x) \Bigg)\\
         &\leq \pexp[X] - \frac{1}{\rho} \log_\beta \left( \int_{x \leq \pexp[X] + \frac{\epsilon}{2}} \beta^{-\rho \frac{\epsilon}{2}} \d \prob(X = x) + 0 \right)\\
        &= \pexp[X] - \frac{1}{\rho} \log_\beta \left( \prob\left(X \leq \pexp[X] + \frac{\epsilon}{2}\right) \beta^{-\rho \frac{\epsilon}{2}} \right)\\
        &= \pexp[X] - \frac{1}{\rho} \log_\beta \left( \prob\left(X \leq \pexp[X] + \frac{\epsilon}{2}\right)\right) + \frac{\epsilon}{2}.
        \end{align*}

        For $\rho$ large enough, this quantity is indeed smaller than $\pexp[X] + \epsilon$.


        \item Let us first notice that for every $\beta, \rho, X$, we have the following equality $\re_{\beta,\rho}[X] = -\re_{\beta(-\rho)}[-X]$.
        Thus, we can apply the previous result, and find:
        \begin{align*}
        \lim_{\rho \to -\infty} \re_{\beta,\rho} [X] 
        &= \lim_{\rho \to -\infty} -\re_{\beta(-\rho)} [-X]
        \\&= -\lim_{\rho \to +\infty} \re_{\beta,\rho} [-X]
        \\& =  - \pexp[-X]
        \\ &= - \inf \{x \in \Rb ~|~ \prob(-X \leq x) > 0\}
        \\ &= - \inf \{x \in \Rb ~|~ \prob(X \geq -x) > 0\}
        \\ &= \sup \{x \in \Rb ~|~ \prob(X \geq x) > 0\}
        \\ &= \oexp[X].
        \end{align*}
        \end{itemize}
\end{proof}

% \begin{remark}[Monotonicity of measurc risk]\theju{CHECK: Do we need this anywhere?}
%     If $\rho \leq \rho'$, then $\re_{\beta,\rho} \geq \re_{\beta \rho'}$.
% \end{remark}

% \begin{proof}
%     This result was proved in a text book on risk theory~\cite{KGD08} in a slightly different context.

%     Let $\rho, \rho' \in \Rb \setminus \{0\}$.
%     Let us assume $0 < \rho < \rho'$.
%     Then, the mapping $x \mapsto x^{\frac{\rho}{\rho'}}$ is concave, hence by Jensen's inequality we can write:
%     $$\Eb\left[ \left(\beta^{-\rho'X}\right)^{\frac{\rho}{\rho'}} \right] \leq \left( \Eb\left[ \beta^{-\rho X} \right] \right)^{\frac{\rho}{\rho'}}.$$
%     Then, by applying the mapping $-\frac{1}{\rho} \log_\beta$, we obtain:
%     $$-\frac{1}{\rho} \log_\beta \Eb\left[ \left(\beta^{-\rho X}\right) \right] \geq -\frac{1}{\rho} \frac{\rho}{\rho'} \log_\beta \Eb\left[ \beta^{-\rho X} \right],$$
%     i.e. $\re_\rho[X] \geq \re_{\rho'}[X]$.

%     The case where $\rho$ and $\rho'$ are negative is analogous, and the result can be generalised to $\Rb$ using the continuity in $0$.
% \end{proof}

\subsection{Proof of \cref{thm:XRSEexists}}\label{app:XRSEexists}

\XRSEexists*

\begin{proof}[Proof of \cref{thm:XRSEexists}]
    Throughout this proof, for a given set of edges $F \subseteq E$, we write $\Game^F$ for the game obtained from $\Game$ by removing all the edges that do not belong to $F$.
    In that game, we define $\bsigma^F$ as the stationary strategy profile that maps each vertex $v$ to some probability distribution whose support is $F(v)$.
    Note that the probabilities do not matter here: we are only interested in the support of the distribution of the strategy profile.

    \paragraph*{Algorithm.} We proceed by presenting the algorithm, \cref{algo:existence}, that takes as an input the game $\Game_{\|v_0}$ and the partition $(P, O)$, and returns a subset $F \subseteq E$ such that, as we will show, the strategy profile $\bsigma^F$ is always an XRSE.
    That algorithm defines a decreasing sequence $E_0, E_1, \dots$ of subsets of $E$, where $E_0 = E$.
    At each step $k$, for each pessimist $i$, it computes the risk measure $z_i^k$ of player $i$ in $\bsigma^{E_k}$, and then the set $W_i^k$ of vertices $v$ such that, from $v$, whatever player $i$ does, that player almost surely gets a payoff smaller than or equal to $z_i^k$.
    If we have $v_0 \in W_i^k$ for each $i$, then the algorithm stops there and returns the set $E_k$ (and we will show below that it means that $\bsigma^{E_k}$ is an XRSE).
    Otherwise, we pick player $i$ such that $v_0 \not\in W_i^k$ (a player who provably has a profitable deviation), and define $E_{k+1}$ by removing all the edges accessible from $v_0$ leading from $V \setminus W_i^k$ to $W_i^k$.


            \begin{algorithm}
            \begin{algorithmic}\caption{Exhibition of one stationary XRSE}\label{algo:existence}
                \Procedure{Existence}{$\Game_{\|v_0}, P, O$}
                    \State $k \gets 0$
                    \State $E_k \gets E$
                    \While{$\top$}
                        \State Compute $A^k = \{v \in V \mid v \text{ is accessible from } v_0 \text{ in } (V, E_k)\}$
                        \ForAll{$i \in P$}
                            \State Compute $z^k_i = \X_i(\bsigma^{E_k})$
                            \State Compute $W^k_i = \{v \in V \mid \forall \tau_i \in \Strat_i \Game^{E_k}_{\|v_0}, \text{we have } \prob_{\bsigma^{E_k}_{-i}, \tau_i}(\mu_i \leq z^k_i) > 0\}$
                        \EndFor
                        \If{$\exists i$ such that $v_0 \not\in W_i^k$}
                            \State Pick one such $i$
                            \State $E_{k+1} \gets E_k \setminus ((A^k \setminus W_i^k) \times W_i^k)$
                            \State $k \gets k+1$
                        \Else
                            \State \Return $E_k$
                        \EndIf
                    \EndWhile
                \EndProcedure
            \end{algorithmic}
        \end{algorithm}

    \paragraph*{Correctness} A first quick invariant that we need to prove is the following one, which will guarantee that the games $\Game^{E_k}$ and the strategies $\bsigma^{E_k}$ are well-defined.

    \begin{invariant}\label{inv:outgoingedges}
        For each $k$, each stochastic vertex $v$, we have $E(v) \subseteq E_k$, and for each non-stochastic vertex $v$, we have $E(v) \cap E_k \neq \emptyset$.
    \end{invariant}
    
\begin{claimproof}[Proof that \cref{algo:existence} satisfies \cref{inv:outgoingedges}]
    The set $E_0 = E$ trivially satisfies the invariant.

    Now, let us assume that $E_k$ satisfies the invariant.
    At step $k$, an edge is removed if and only if goes from a vertex $u \in A^k \setminus W^k_i$ to a vertex $v \in W^k_i$.
    Consider a stochastic vertex $u \in A^k$: if it has an edge that leads to vertex $v \in W^k_i$, then whatever player $i$ plays from $u$, with positive probability, the vertex $v$ is reached; and then, if the other players play the strategy profile $\bsigma^{E_k}$, then with positive probability, player $i$ gets the payoff $z^k_i$ or less.
    Hence $u \in W_i^k$, and the edge $uv$ is not removed, and remains in the set $E_{k+1}$.
    Similarly, if $u$ is not a stochastic vertex, but all its outgoing edges lead to a vertex that belong to $W_i^k$, then $u$ itself belongs to $W_i^k$, hence the outgoing edges of $u$ will not all be removed.
    The invariant is therefore still true at step $k+1$, and by induction, is true for all $k$.
\end{claimproof}
    
Each step of the algorithm is then also properly defined.
Moreover, we have termination.

\begin{proposition}
    \cref{algo:existence} terminates.
\end{proposition}

\begin{claimproof}
    With \cref{inv:outgoingedges}, we now know that \cref{algo:existence} successfully constructs a sequence $E_0, E_1, \dots$ of sets of edges until it stops and returns the last of those sets.
    Termination is an immediate consequence of the fact that this sequence is decreasing.

    Indeed, for each step $k$ at which nothing is returned, there exists a player $i$ with $v_0 \not\in W_i^k$.
    On the other hand, the set $W_i^k$ is necessarily accessible from $v_0$:

    \begin{claim}
        The set $W_i^k$ is nonempty, and accessible from $v_0$ in the graph $(V, E_k)$.
    \end{claim}

    \begin{claimproof}
        If $z^k_i$ is obtained by reaching a terminal vertex $t$, then we have $t \in W_i^k$, and $t$ is accessible from $v_0$.
        If now $z^k_i = 0$ is obtained by reaching no terminal vertex, then when following $\bsigma^k$, with positive probability, no terminal is reached.
        Then, there is in particular a vertex $u$ that has positive probability to be visited infinitely often.
        And when playing $\bsigma^k$ from $u$, the probability that some terminal is ever reached is actually $0$, since if it was some constant $q > 0$, then the probability of visiting $u$ infinitely often would be $\lim_\l (1-q)^\l = 0$.
        In other words, no terminal vertex is accessible from $u$ in $(V, E_k)$, and then, we have $u \in W_i^k$.
    \end{claimproof}

    Now, along a play that starts from $v_0 \not\in W_i^k$ and visits $W_i^k$, there exists at least one edge that goes from a vertex that does not belong to $W_i^k$, to a vertex that does.
    Such an edge is then removed in the set $E_{k+1}$, which is therefore strictly included in the set $E_k$.
    This holds for every $k$, ensuring termination.
\end{claimproof}

We now know that the algorithm terminates, i.e., constructs a finite decreasing sequence $E = E_0, E_1, \dots, E_n$, and then returns the set $E_n$, as a succinct representation of the stationary strategy profile $\bsigma^{E_n}$.
What remains to be proven is that that strategy profile is an XRSE.
Before proving that it is an XRSE in the game $\Game_{\|v_0}$, we first prove that it is one in the game $\Game^{E_n}_{\|v_0}$, i.e., when the edges that have been removed cannot be used to deviate.

\begin{proposition}\label{prop:xrseGEn}
    The strategy profile $\bsigma^{E_n}$ is an XRSE in the game $\Game^{E_n}_{\|v_0}$.
\end{proposition}

\begin{claimproof}
    Consider a player $i$, and a deviation $\sigma'_i$ of player $i$ from the strategy profile $\bsigma^{E_n}$ in the game $\Game_{\|v_0}^{E_n}$.
    Let $x = \X_i(\bsigma^{E_n}_{-i}, \sigma'_i)$.

    \subparagraph*{If player $i$ is an optimist.}
    If $x = 0$, then since all rewards are non-negative, we have $x \leq \X_i(\bsigma^{E_n})$.
    If $x > 0$, then the payoff $x$ is obtained by reaching a terminal vertex $t$.
    But then, that terminal vertex is accessible from $v_0$ in the graph $(V, E_n)$, and is therefore also reached with positive probability when all players follow the strategy profile $\bsigma^{E_n}$.
    Hence, again, the inequality $x \leq \X_i(\bsigma^{E_n})$.
    
    \subparagraph*{If player $i$ is a pessimist.}
    Then, since the algorithm terminated at step $n$, player $i$ is such that $v_0 \in W_i^k$.
    The strategy $\sigma'_i$, like every strategy $\tau_i$ for player $i$, satisfies therefore the inequality $\prob_{\bsigma^{E_n}_{-i}, \sigma'_i}(\mu_i \leq z_i^n) > 0$.
    Consequently, we have $x \leq z_i^k = \X_i(\bsigma^{E_n})$.

    In both cases, the deviation $\sigma'_i$ is not profitable, hence the conclusion.
\end{claimproof}


Let us now prove that putting back the removed edges does not change that result, and therefore conclude the correctness proof.

\begin{proposition}
    The strategy profile $\bsigma^{E_n}$ is an XRSE in the game $\Game_{\|v_0}$.
\end{proposition}

\begin{claimproof}
    Let $i$ be a player, and let $\sigma'_i$ be a deviation from $\bsigma^{E_n}$ for player $i$ in $\Game_{\|v_0}$.
    Since $\bsigma^n$ is stationary, we can assume that $\sigma'_i$ is positional by \cref{lm:secretlemma}.
    If the strategy $\sigma'_i$ uses only edges of $E_n$, then it can be considered as a deviation from $\bsigma^{E_n}$ in the game $\Game^{E_n}$, hence by \cref{prop:xrseGEn}, it is not a profitable deviation.
    
    Let us now assume that the strategy $\sigma'_i$ uses an edge that does not belong to $E_n$, i.e. there exists a vertex $v$ that is visited with positive probability in the strategy profile $(\bsigma^{E_n}_{-i}, \sigma'_i)$ and an edge $vw \in E \setminus E_n$ such that $w = \sigma'_i(v)$.
    Since only edges controlled by pessimists have been removed, we can immediately deduce that player $i$ is a pessimist.
    
    Now, among such edges, let us choose one whose removal occurred the earliest, that is, let us choose it in order to minimise the index $k$ such that $vw \in E_k \setminus E_{k+1}$.
    Thus, in the strategy profile $(\bsigma^{E_n}_{-i}, \sigma'_i)$, it is almost sure that only edges of $E_k$ are used.

    The fact that the edge $uv$ has been removed at step $k$ means that we had $u \not\in W_i^k$ and $v \in W_i^k$.
    Thus, from the vertex $v$, if player $i$ uses only edges of $E_k$ (which is the case when they follow $\sigma'_i$), and if the other players follow the strategy profile $\bsigma^{E_k}_{-i}$, player $i$ gets the payoff $z_i^k$ or less with positive probability.
    Let us show that it is also the case when the other players follow the strategy profile $\bsigma^{E_n}$ instead of $\bsigma^{E_k}$.

\begin{claim}
    From the vertex $v$, we have $\prob_{\bsigma^{E_n}_{-i}, \sigma'_i}(\mu_i \leq z_i^k) > 0$.
\end{claim}

\begin{claimproof}
    We proceed by proving that when the players follow, from the vertex $v$, the strategy profile $(\bsigma^{E_k}, \sigma'_i)$, there is a positive probability that player $i$ gets the payoff $z_i^k$ or less \emph{and} that the set $W_i^k$ is never left.

    Indeed, if in that strategy profile there is a positive probability that player $i$ gets a payoff smaller than or equal to $z_i^k$ by reaching a terminal vertex $t$ that yields such a payoff, then we have $t \in W_i^k$, and with positive probability the terminal vertex $t$ is reached without leaving $W_i^k$.
    Similarly, if such a payoff is obtained by reaching no terminal vertex, and therefore getting payoff $0$, then, using a reasoning that has already been used above, with positive probability a vertex $w$ is reached without leaving $W_i^k$, such that from $w$, no terminal vertex is accessible anymore in $(V, E_k)$; and, therefore, all the vertices accessible from $w$ in that graph belong to $W_i^k$, hence once $w$ is reached it is almost sure that $W_i^k$ is never left.

    Then, the set $E_{k+1}$ was defined so that $W_i^k$ is no longer accessible from $v_0$ in the graph $(V, E_{k+1})$.
    Therefore, those vertices are not accessible at any step $\l > k$, and therefore no outgoing edge of a vertex of $W_i^k$ is ever removed in the sequel, i.e. $E_n \cap (W_i^k \times V) = E_k \cap (W_i^k \times V)$.
    Consequently, since $\sigma'_i$ uses only edges of $E_k$, when the strategy profile $(\bsigma^n, \sigma'_i)$ is played from $v$, it is also true that with positive probability player $i$ gets the payoff $z_i^k$, or less.
\end{claimproof}
    
    This claim proves that we have $\X_i(\bsigma^{E_n}_{-i}, \sigma'_i) \leq z_i^k$.
    To conclude that the deviation $\bsigma'_i$ is not profitable, we still need to prove that that quantity $z_i^k$ is smaller than or equal to (actually strictly smaller) the risk measure $\X_i(\bsigma^n)$.
    That inequality is an immediate consequence of the following claim.

\begin{claim}
    For every pessimistic player $j$, the sequence $(z_j^\l)$ of player $j$'s risk measures is non-decreasing.
\end{claim}

\begin{claimproof}
    Let $\l$ be a step index, and let us prove that we have $z_j^\l < z_j^{\l+1}$.
    The quantity $z_j^{\l+1}$ is the pessimistic risk measure of player $j$ in the strategy profile $\bsigma^{E_{\l+1}}$: there is therefore a positive probability that player $j$ gets the payoff $z_j^{\l+1}$ when that strategy profile is followed.
    Player $j$ obtains that payoff either by reaching a terminal vertex to which that payoff is assigned, or by reaching no terminal vertex at all.

    Let us first show that the second case is actually impossible.
    If player $j$ gets the payoff $z_j^{\l+1} = 0$ by reaching no terminal vertex, with the same reasoning as above, there is, in particular, a vertex $v$ that has positive probability to be visited infinitely often when $\bsigma^{\l+1}$ is played from $v_0$, and therefore such that no terminal vertex is accessible from $v$ in $(V, E_{\l+1})$.
    But then, let $j'$ be the player that was controlling the edges that were removed at step $\l$.
    Let us consider a strategy $\tau_{j'}$ of player $j'$ that uses only edges of $E_\l$.
    Then, when the strategy profile $(\bsigma^\l_{-j'}, \tau_{j'})$ is played from the vertex $v$, it will almost surely be true that either no terminal vertex is reached, leading to the payoff $0$, or an edge of $E_\l \setminus E_{\l+1}$ is taken, leading therefore to a vertex of $W_i^\l$, and to a risk measure of $z_{j'}^\l$ or less.
    Thus, since all rewards are non-negative and therefore $z_{j'}^\l \geq 0$, the vertex $v$ belongs to the set $W_{j'}^\l$, which is impossible since it should then have been made unaccessible in the graph $(V, E_{\l+1})$.

    Therefore, player $j$ gets payoff $z_j^{\l+1}$ by reaching a terminal giving them that payoff, which means that such a terminal is accessible from $v_0$ in the graph $(V, E_{\l+1})$.
    Then, it is also accessible from $v_0$ in the graph $(V, E_\l)$, and therefore it is reached with positive probability when following the strategy profile $\bsigma^{E_\l}$.
    Consequently, we have $z_j^\l \leq z_j^{\l+1}$.
\end{claimproof}

    
    Consequently, with $j = i$, we have $z_i^k \leq z_i^n$, and therefore $\X_i(\bsigma^{E_n}_{-i}, \sigma'_i) \leq z_i^k \leq z_i^n = \X_i(\bsigma^{E_n})$.
    The strategy $\sigma'_i$ is not a profitable deviation from the strategy profile $\bsigma^{E_n}$, which is therefore a (stationary) XRSE.
\end{claimproof}


    
    \paragraph*{Complexity}

    We finally show that \cref{algo:existence} runs with time $\Oh(m^2p)$.
    At each iteration of the while loop, at least one edge is removed; we therefore have at most $m$ iterations of the while loop.

    Now, during the $k^\text{th}$ iteration of the loop, the algorithm computes the set $A^k$, and for each pessimistic player $i$, the algorithm also computes the quantity $z_i^k = \X_i(\bsigma^{E_k})$, and then the set $W^k_i$.
    All of those computations can be done in time $\Oh(m)$ using \cref{lm:secretlemma}. Since there are $p$ players, this step therefore takes $\Oh(mp)$ time.
    Finally, checking whether $v_0 \in W^k_i$ for each player $i$ takes time $\Oh(p)$, and removing all the edges leading to $W_i^k$ to define the set $E_{k+1}$ takes time $\Oh(m)$.
    Hence, the complexity of the algorithm is $\Oh(m^2 p)$.
\end{proof}