% Finally, we consider two further restrictions on the problem to show the tractability border of this problem, and if there are cases where the problem can be solved more efficiently, 
% We show in \cref{thm:infinite_rho_restricted_strategy_np_easy} that restricting the memory or amount of randomness of the strategy still renders the problem $\NP$-complete. 

% Based on observation of the $\NP$-hardness proof requiring only pessimistic players, we show that if instead all the players are optimists, then the problem becomes solvable in polynomial time. 

% \subsection{Restrictions on strategies} 
% First, we consider restrictions on the kinds of strategies used. That is, if the strategy are memoryless, positional or pure. We show that even in such cases finding an RSE is $\NP$-complete. 
% \begin{theorem}\label{thm:infinite_rho_restricted_strategy_np_easy}
%     The constrained existence problem of $(P,O)$-RSEs is $\NP$-complete
%     when the number of players is fixed, even when the players are restricted to positional, memoryless,  or pure strategies. 
% \end{theorem}
% \begin{proof}
%     The lower bound for all three of the above cases follow from \cref{lemma:np_hardness}, with the observation that, in the reduction, if an RSE that satisfies the constraints, then the strategies of the players are positional strategies, and therefore are both pure and memoryless too. 

%     For the upper bound, we show that we can still guess a strategy and verify in polynomial time if it is indeed a strategy. For the cases of positional and memoryless strategies, guessing a strategy is straightforward. Whereas for pure strategies, this requires some work. 
%     \paragraph*{Positional strategies and memoryless strategies} The size of such a strategy $\sigma$ that is an RSE can be represented using polynomially many bits, since one only needs to guess the set of edges from each vertex that are being used with non-zero probability. From \cref{lm:mpd_ptime}, we can then verify if the given $\sigma$ indeed gives a simple quantitative payoff within the constraints, and also if it is indeed an RSE in polynomial time. 

%     \paragraph*{Pure strategies}
%     \thejaswini{Need to write. Maybe proof sketch is enough. }
%     For pure strategies, the strategies might require more memory to represent. We argue therefore that if there is a winning strategy, there is one that requires only 
% \end{proof}

% \subsection{Everyone is optimistic}
% We consider the case where the perceived reward of each player is computed based on the risk-measure $\pexp$ for optimistic players. In this scenario, we show that the problem is $\PTIME$-complete. 
% \begin{theorem}
%     The constrained existence problem of $(P,O)$-RSE is $\PTIME$-complete where all players are optimists, that is, $P=\emptyset$.
% \end{theorem}
% \begin{restatable}{lemma}{ptimeupperbound}
%     The constrained existence problem for $(P,O)$-RSE is in $\PTIME$ if all players are optimists.    
%     The running time of such an algorithm is at most $\Oh()$ where $n$ is the number of vertices in the graph and $p$ is the number of players.\theju{To compute and write here}
%     Moreover, this polynomial time decision algorithm can be modified to output a succinct representation of an RSE satisfying the constraints, when it exists, in $\Oh()$\theju{Copy}.
% \end{restatable}
% \begin{proof}[Proof Sketch.]
    
% \end{proof}

% Finally, we show that the problem is also $\PTIME$-hard even for the case where there are two players. 
% \begin{restatable}{lemma}{ptimelowerbound}\label{lemma:PTIMEHard}
%     The constrained existence problem for two players RSE is $\PTIME$-hard when both players are optimists.
% \end{restatable}
% We show $\PTIME$-hardness for the problem in the case by giving a log-space reduction from the two player zero sum reachability game. Deciding the winner in two player reachability games is $\PTIME$-complete, with hardness proved for the same problem of alternating graph reachability~\cite[Proposition~6]{Imm81}.