
\subsection{Proof of \cref{thm:memorysmall}}\label{app:memorysmall}

% We first show some simple lemmas that we use during the proof of  \cref{thm:memorysmall}.\theju{to check if this is also needed for later on in PTIME case}
% \begin{lemma} \label{lm:mpd_ptime}
%     Given a simple quantitative MDP $\MDProc$, the quantities $\max_\sigma \pexp_\sigma$ and $\sup_\sigma \oexp_\sigma$ can be computed in polynomial time.
%     As a consequence, the pessimistic and optimistic expectation of a simple quantitative payoff function can also be computed in polynomial time in a Markov chain.
% \end{lemma}

% \begin{proof}
%     \leonard{todo}
% \end{proof}


\memorysmall*

\begin{proof}[Proof of \cref{thm:memorysmall}]
We first define the anchoring set of players $\Lambda$ given a strategy profile~$\bsigma$.
\subparagraph*{Definition of $\Lambda$.}

\finiteMemAbstraction*

\begin{claimproof}
    We define the labelling $\Lambda$ inductively. 

    \subparagraph*{Base case.}
    First, on the one-vertex history $v_0$, we define $\Lambda(v_0) = \Pi$: at the start, all players must be anchored.
    Let us notice that the history $v_0$ satisfies Property~\ref{itm:optimistanchor}, which states that the optimists get the optimistic expectation they are supposed to get, and Property~\ref{itm:pessimistanchor}, which states that the pessimists have no profitable deviations.
    The other properties will be checked in the inductive case.

    \subparagraph*{Inductive case.}
    Suppose $\Lambda(hv)$ has already been defined, where $hv$ is a history compatible with the strategy profile $\bsigma$, and that the five properties are satisfied by $\Lambda$ on all histories on which it is already defined.
    Let $w_1, \dots, w_k$ be the successors of $v$ that are chosen by the strategy $\bsigma(hv)$ with non-zero probability, that is, the support of $\bsigma(hv)$.
    If $k=1$, then we define $\Lambda(hvw_1) = \Lambda(hv)$.
    Note that Properties~\ref{itm:splitsetsanchorwithouti},~\ref{itm:splitsetsanchorwithi}, and~\ref{itm:nosplit} are immediately satisfied, and that Properties~\ref{itm:optimistanchor} and~\ref{itm:pessimistanchor} are satisfied by induction hypothesis.
    
    If $k>1$, we need to partition the set $\Lambda(h)$ between the $k$ successors.
    To do so, we will use the following claim.
    
    \begin{claim}\label{claim:successorAnchor}
    For each player $i \in \Lambda(hv)$ that does not control the vertex $v$, the follwing holds.
    \begin{itemize}
        \item If player $i$ is an optimist, and $\X_i(\bsigma_{\|hv}) = z_i$, then there is at least one successor $w_\l$ such that $\X_i(\bsigma_{\|hvw_\l}) = z_i$.
    
        \item If player $i$ is a pessimist, then there is a successor $w_\ell \in \Supp(\bsigma(hv))$, such that for every  strategy $\tau_i^\ell$ from $w_\l$, we have $\X_i(\bsigma_{-i\|hvw_\ell}, \tau_i) \leq z_i$.
    \end{itemize}
    \end{claim}
    
    \begin{claimproof}
        The first case follows from Property~\ref{itm:optimistanchor} in the induction hypothesis.
    
        As for the second case, we proceed by contradiction.
        Let us assume that for each $w_\l$, there exists a strategy $\tau_i^\l$ such that $\X_i(\bsigma_{-i\|hvw_\l}, \tau^\l_i) > z_i$.
        Then, the strategy $\tau_i$ defined by $\tau_{i\|vw_\l} = \tau^\l$ for each $\l$ is such that $\X_i(\bsigma_{-i\|hv}, \tau_j) > z_i$, which is impossible since $\Lambda(hv)$ is assumed to satisfy Property~\ref{itm:pessimistanchor}.
    \end{claimproof}

    We define each set $\Lambda(hvw_\l)$ by iterating through each element of $\Lambda(hv)$ as follows:
    \begin{itemize}
        \item \textbf{Initialisation.} For all $w_\l$, declare $\Lambda(hvw) = \emptyset$.
        \item \textbf{Iteration over players.} Consider each player $i\in \Lambda(hv)$ sequentially and proceed as follows:
        If $i$ controls the vertex $v$, then add $i$to every set $\Lambda(hvw)$.
        If $i$ does not control $v$, then, by the claim stated earlier, there exists a successor $w_\l$. In this case, add ii to $\Lambda(hvw_\l)$ corresponding to this specific $w_\l)$
    \end{itemize}
    
    Not that the first four properties are thus guaranteed to be satisfied.

    Moreover, when there are several successors $w_\l$ possible, we always favour those such that, at the moment where the decision is taken, the sets $\Lambda(hvw_\l)$ are the smallest.
    This suffices to guarantee Property~\ref{itm:nosplit}.
    \end{claimproof}




\paragraph*{Construction of the strategy profile $\bsigma^\star$}
Based on $\Lambda$ as in \cref{lm:Lambda}, we construct a strategy profile that finite memory states. 
Formally, the definition of the finite-memory strategy $\bsigma^\star$ will be done by defining its memory structure.

The memory states are the following:
    \begin{itemize}    
        \item for each player $i$, the state $\punish_i$;

        \item for each vertex $v$ and each subset $A \subseteq \Pi$ of players such that there exists $h$ with $A = \Lambda(h)$, the state $\anchor_{Av}$;

        \item the state $\anchor_{\Pi\bot}$.
    \end{itemize}
Observe that there are at most $2^p + \Oh(p)$ such memory states.
We now define the transitions from each of those states from each vertex. Observe that long as the memory state does not change, strategy profile corresponds to a positional strategy profile. So, we describe such memoryless strategy profiles and also describe when the memory state changes. 
For each set $A \subseteq \Pi$, we write $W_A$ for the set of vertices that $\bsigma$ may visit while anchoring the set $A$, i.e., the set of vertices $v$ such that there exists a history $hv$ with $\Lambda(hv) = A$.

    \subparagraph*{Punishing memory states $\punish_i$.}
    First, let us define what $\bsigma^\star$ does when in state $\punish_i$, for some player $i$. Those memory states will correspond to the \emph{punishing strategies}, followed when player $i$ deviates from the assigned strategy with the other memory states. 
    By \cref{lm:secretlemma}, there is a positional strategy profile $\btau^{\dag i}_{-i}$ that minimises, from every vertex of the game, the payoff that player $i$ can enforce.
    In addition, we pick an arbitrary positional strategy $\sigma^{\dag i}_i$.
    Then, when the strategy profile $\bsigma^\star$ is in the memory state $\punish_i$ on reads a vertex $v$, the memory update function outputs the vertex $\btau^{\dag i}(v)$ and the same memory state $\punish_i$.

    \subparagraph*{Anchoring states with no player to anchor.}
    Let us now define what happens in memory state $\anchor_{\emptyset v}$.
    Consider the objective of achieving a payoff vector that has positive probability to be achieved in $\bsigma$, while visiting only vertices of $W_\emptyset$.
    Using classical attractor-based proofs (or \cref{lm:secretlemma}), there exists a positional strategy profile that achieves that objective with probability $1$ from every vertex from which that is possible: let us call it $\btau^{\anch \emptyset}$.
    Then, when in memory state $\anchor_{\emptyset v}$ and reading the vertex $w$, we distinguish two cases.
    \begin{itemize}
        \item If $vw$ is an edge that is compatible with the strategy profile $\btau^{\anch \emptyset}$, then the strategy profile $\bsigma^\star$ outputs the vertex $\btau^{\dag i}(w)$, where $i$ is the player controlling $v$, and shifts to the memory state $\punish_i$.

        \item Otherwise, it outputs the vertex $\btau^{\anch \emptyset}(w)$ and moves to the memory state $\anchor_{\emptyset w}$.
    \end{itemize}

    \subparagraph*{Anchoring state with one player to anchor.}
    We can now move to singletons, and define what happens in the states of the form $\anchor_{\{i\} v}$.
    In such a state, we define the strategy based on the that gives player $i$ exactly the extreme risk measure $z_i$.
    More precisely, we want player $i$ to receive payoff $z_i$ with positive probability, and never leave the set of vertices $W_{\{i\}}$ with probability~$1$.
%    of giving, with positive probability, the payoff $z_i$ to player $i$, and maintaining to $0$ the probability of achieving a payoff vector (including $0$ for every player) that has probability zero to be achieved in $\bsigma$, or of visiting a vertex that has probability zero to be visited in $\bsigma$. 
    Using \cref{lm:secretlemma}, there exists a positional strategy profile $\btau^{\anch i}$ that satisfies that property from every vertex from which it is possible.
    Note that that objective is in particular satisfiable, and therefore satisfied by $\btau^{\anch i}$, from every vertex $v \in W_{\{i\}}$.
    Similar to the previous step, we define the strategy profile $\bsigma^\star$ in the states of the form $\anchor_{\{i\} v}$ so that it follows the strategy profile $\btau^{\anch i}$, remembers the last vertex that was visited and uses that memory to switch to the corresponding punishing state when some player $j$ deviates.

        \subparagraph*{Anchoring states with two or more players to anchor.}
    Now, let us consider the states of the form $\anchor_{A v}$, where $A$ has cardinality at least $2$.
    The existence of $\anchor_{A v}$ implies that there is a history $h$ such that $\Lambda(h) = A$.
    Moreover, since each randomisation splits the label of histories in sets that have at most one element in common (Properties~\ref{itm:splitsetsanchorwithouti} and~\ref{itm:splitsetsanchorwithi}), there is only one side of each split that can contain $A$, which implies that among such histories $h$, we can choose one that is a prefix of all others.
    After history $h$, the histories labelled by $A$ form a sequence $h, hv_1, hv_1v_2, \dots$ which may be infinite, end in a terminal vertex, or end with a new split.
    

    \textbf{If that sequence end with a split,} then there is a longest history $hv_1 \dots v_q$ with $\Lambda(hv_1 \dots v_q) = A$ and $k \geq 2$ vertices $w_1, \dots, w_k \in \Supp(\bsigma(hv_1 \dots v_q))$ such that we have $\Lambda(hv_1 \dots v_q w_\l) \neq \emptyset, A$ for each $\l$.
    We can then define a simple history $h'v_q$ that also goes from the vertex $\last(h)$ to the vertex $v_q$, with $\Occ(h'v_q) \subseteq \Occ(\last(h) v_1 \dots v_q)$.
    We then define the strategy profile $\bsigma^\star$ in each state $\anchor_{Av}$ so that it follows the history $h' v_q$ and remembers the last vertex visited, and switches to the state $\punish_i$ and follows the strategy profile $\btau^{\dag i}$ when a given player $i$ deviates and takes an edge that they are not supposed to take.
    Moreover, when an edge is taken that does belong to the history $h'v_q$, but not because a player deviated (it is then necessarily because of a stochastic vertex), the memory switches to the state $\anchor_{\emptyset v}$ (where $v$ is the last vertex seen) and immediately follows the corresponding strategy.
    Finally, when the vertex $v_q$ is reached and the memory is in state $\anchor_{A \last(h')}$, the strategy profile $\bsigma^\star$ chooses randomly between the edges $v_q w_1$, \dots, and $v_q w_k$, all with positive probability.
    Such action will often be referred to as a \emph{split}.


        \textbf{If that sequence is infinite or end in a terminal vertex,} then $\pi^A = v_1 v_2 \dots$ is a play, and satisfies $\Lambda(h\pi^A_{< k}) = A$ for each $k$.
        We can then consider a play $\pi^{A\star}$ with $\Occ(\pi^{A\star}) \subseteq \Occ(\pi^A)$ and $\Inf(\pi^{A\star}) \subseteq \Inf(\pi^A)$ that is either a simple path from $\pi^A_0$ to the terminal reached by $\pi^A$, or, if $\pi^A$ is infinite, a simple lasso (i.e., a play of the form $h'c^\omega$, where the history $h'c$ is simple).
        We can moreover choose $\pi^{A\star}$ so that the set of vertices visited infinitely often (if there are any) in $\pi^{A\star}$ is included in the set of vertices visited infinitely often in $\pi^A$.
Then, we can define $\bsigma^\star$ in the states of the form $\anchor_{A v}$ as following the play $\pi^{A\star}$, and remembering the last vertex seen.
    When a player $i$ deviates and takes an edge that should not be taken, the memory switches to the state $\punish_i$ and follows the strategy profile $\btau^{\dag i}$.
    Finally, when an edge is taken that does not belong to $\pi^{A\star}$ but does not correspond to a deviation either, we switch to the state $\anchor_{\emptyset w}$ where $w$ is the last vertex seen, and to the corresponding strategy profile.



\subparagraph*{Initialisation.}
    The strategy profile $\bsigma^\star$  has the state $\anchor_{\Pi\bot}$ as the initial memory state.
    In this state, it behaves exactly as in any state of the form $\anchor_{\Pi v}$, but without having memorised a last visited vertex $v$, since there is no such vertex.
    From that memory state therefore, it necessarily reads the vertex $v_0$, and starts acting as described in the previous case.


\subparagraph*{The pure case.}
In this construction, the vertices on which the strategy profile $\bsigma^\star$ proceeds to an actual randomisation (i.e., the vertices $v$ such that there exists a history $hv$ such that the support of the distribution $\bsigma^\star(hv)$ contains more than one element) are vertices on which $\bsigma$ also proceeds to such a randomisation.
Therefore, if $\bsigma$ is pure (i.e., if randomisations occur only on stochastic vertices), so is $\bsigma^\star$.





\paragraph*{A combinatorial break: counting states}

    Now that the strategy $\bsigma^\star$ is defined, let us bound the memory it uses.
    There are, obviously, exactly $p$ states of the form $\punish_i$, and one state $\anchor_{\Pi\bot}$.
    To prove that there are at most $3np-2n$ states of the form $\anchor_{Av}$, we need to prove that there are at most $3p-2$ sets $A$ such that there is a history $h$ with $\Lambda(h) = A$.

    Let us call \emph{$\Lambda$-anchored} all such sets $A$.
    By analogy with strategies, we write $\Lambda_{\|hv}$ for the labelling that maps each history $h' \in \Hist\Game_{\|v}$ compatible with $\bsigma_{\|hv}$ to the set $\Lambda(hh')$, and we will also use the notion of anchoredness for each of those labellings $\Lambda_{\|hv}$.
    We proceed by proving the following stronger result.
    
    \begin{proposition}\label{prop:combinatorial}
        For every history $h$ compatible with $\bsigma$, if $\Lambda(h)$ contains at least two elements, then there are at most $3|\Lambda(h)|-2$ sets that are $\Lambda_{\|h}$-anchored.
    \end{proposition}
    


\begin{claimproof}
    For each history $h$, we write $f(h)$ for the number of $\Lambda_{\|h}$-anchored sets that have cardinal at least $2$.
    There are $|\Lambda(h)|+1$ subsets of $\Lambda(h)$ that have cardinality $0$ or $1$: the result will therefore be proved if we prove $f(h) \leq 2|\Lambda(h)| - 3$.
    The proof goes by induction on $m = \Lambda(h) \geq 2$.

    \subparagraph*{Base case.}
    If $m = 2$, the set $\Lambda(h)$ is a pair $\Lambda(h) = \{i, j\}$.
    Then, since the $\Lambda_{\|h}$-anchored sets are all subsets of $\Lambda(h)$, the only set of cardinality at most $2$ that is $\Lambda_{\|h}$-anchored is the pair $\{i, j\}$ itself, hence we have $f(h) = 2 \times 2 - 3 = 1$, as desired.

    \subparagraph*{Inductive case.}
    If $m > 2$, and if we assume that the result is true for every history $h'$ with $2 \leq |\Lambda(h')| \leq m-1$, then let $\{v_1, \dots, v_k\} \subseteq \Supp(\bsigma(h))$ be the set of possible next vertices $v$ such that $|\Lambda(hv)| \geq 2$.

    If $k = 1$, i.e., if $\Lambda(hv_1) = \Lambda(h)$, then we have $f(h) = f(hv_\l)$ and the result for $h$ will be proved if we prove it for $hv_1$.
    Following that reasoning, we can extend the history $h$ until we are not in that case: if we always are, then the only $\Lambda_{\|h}$-anchored sets are $\Lambda(h)$ itself, and possibly the empty sets and some singletons, hence $f(h) = 1$ and the result is immediate.
    We can therefore assume that $k > 1$.

    Let $i$ be the player controlling the vertex $\last(h)$; we set $i = \bot$ if $\last(h)$ is a stochastic node.
    Then, by Properties~\ref{itm:splitsetsanchorwithouti} and~\ref{itm:splitsetsanchorwithi} of \cref{lm:Lambda}  guaranteed during the construction of $\Lambda$, the sets $\Lambda(hv_1) \setminus \{i\}, \dots, \Lambda(hv_k) \setminus \{i\}$ form a partition of $\Lambda(h) \setminus \{i\}$.
    Therefore, no set of cardinality at least $2$ can be simultaneously $\Lambda(hv_\l)$-anchored and $\Lambda(hv_{\l'})$-anchored for $\l \neq \l'$, hence the equality $f(h) = 1+\sum_\l f(hv_\l)$.
    Now, since each set $\Lambda(hv_\l)$ has at least $2$ and less than $m$ elements, we can apply the induction hypothesis to deduce:
    $$f(h) \leq 1 + \sum_{\l=1}^k (2|\Lambda(hv_\l)| - 3).$$
    Moreover, we have $\sum_\l |\Lambda(hv_\l)| \leq m + k - 1$ (each element of $\Lambda(h)$ occurs in one of the sets $\Lambda(hv_\l)$, except possibly one that would occur in all of them), hence the inequality above becomes:
    $$f(h) \leq 1 + 2(m+k-1) -3k$$
    $$= 2m - 1 - k$$
    and since we have assumed $k \geq 2$, we obtain $f(h) \leq 2m-3$.
\end{claimproof}    

Let us recall that $\Lambda(v_0) = \Pi$.
As a particular case of this claim, we obtain, if $p \geq 2$, that there are at most $3p-2$ sets that are $\Lambda$-anchored, as desired.
In the case $p = 1$, the game $\Game_{\|v_0}$ is an MDP, and using \cref{lm:secretlemma}, we can immediately construct $\bsigma^\star$ as a positional strategy (which has therefore $1 \leq 3n \times 1 - 2n + 1 + 1$ memory states) with the same risk measure as $\bsigma$.




    \paragraph*{The strategy profile $\bsigma^\star$ has the desired extreme risk measures.}

We now show that $\X(\bsigma^\star) = \X(\bsigma)$.
Let us recall that we defined $\bz = \X(\bsigma)$.
    
\begin{proposition}\label{prop:ActualPayoff}
    The strategy profile $\bsigma^\star$ satisfies the equality $\X(\bsigma^\star) = \bz$.
\end{proposition}

\begin{claimproof}
    Let $i$ be a player: we want to prove that $\X(\bsigma^\star) = z_i$.
    Let us first see how $z_i$ has positive probability of being obtained in the strategy profile $\bsigma$, and we will then show that no larger (respectively smaller) if $i$ is optimistic (respectively pessimistic) has a positive probability by the same strategy.

   

    \subparagraph*{Player $i$ gets payoff $z_i$ with positive probability.}
    Let $A \subseteq \Pi$ be one of the smallest sets (for the inclusion relation) containing $i$ such that there exists a history $hu$ with $\Lambda(hu) = A$.
    Then, by construction of $\Lambda$, there exists a finite sequence of sets $\Pi = A_0, A_1, \dots, A_m = A$ and of histories $h_1 v_1 w_1, \dots, h_m v_m w_m$ where for each $k$, the history $h_{k+1}$ starts from $w_k$, the history $h_1 v_1 \dots h_k v_k w_k$ is compatible with $\bsigma$, and we have $\Lambda(h_1 v_1 \dots h_k v_k) = A_{k-1}$ and $\Lambda(h_1 v_1 \dots h_k v_k w_k) = A_k$.
    We can then write $hu = h_1 v_1 \dots h_m v_m w_m$.
    
    Consider the strategy profile $\bsigma^\star$, which initially follows  the positional strategy profile $\btau^{\anch \Pi}$. This strategy profile generates, with nonzero probability, a history $h'_1 v_1$ starting from vertex $v_0$ to vertex $v_1$, based on our construction. 
    From that vertex $v_1$, it proceeds to a randomised action and, with positive probability, moves to the vertex $w_1$ and switches to the positional strategy profile $\btau^{\anch A_1}$, and so on: there is, therefore, a history $h'_1 v_1 h'_2 v_2 \dots h'_m v_m w_m$ that is compatible with the strategy profile $\bsigma^\star$ and after which the collective memory is in the state $\anchor_{A v_m}$, and plays accordingly.

    Since $A_m= A$ is the  subset of $\Pi$ where $i\in A = \Lambda(hu)$, we are in the case where the set $A$ is no longer split further by our labelling. That is, there is a play $\pi$ from $w_m$ such that $\Lambda(h \pi_{\leq k}) = A$ for every $k$ such that that is defined.
    Then, in the construction of the strategy profile $\bsigma^\star$ we have distinguished two cases: the one where $A$ was a singleton, and the one where it had at least two elements (the empty case is excluded, since $A$ contains the player $i$).

    \textbf{If $A$ is a singleton,} then after the history $hu$, without any player deviating, all players are following the strategy profile $\btau^{\anch i}$.
    By its definition, that strategy profile achieves the payoff $z_i$ for player $i$ with positive probability.

    \textbf{If $A$ has at least two elements,} then after that same history, all players are following the play $\pi^{A\star}$, which yields the same payoffs as $\pi^A$.
    However, we must still prove that player $i$ actually gets the payoff $z_i$ in $\pi^A$, and that the play $\pi^{A\star}$ is generated with positive probability (i.e. that it does not cross infinitely many stochastic vertices---which we must first show for $\pi^A$).
We do so in the following claim, which we will use again later.

\begin{claim}\label{claim:piA}
    The play $\pi^A$ is (eventually) generated with positive probability when the players follow the strategy profile $\bsigma$.
    Similarly, the play $\pi^{A\star}$ is generated with positive probability when they follow $\bsigma^\star$.
    Both plays yield to each player $j \in A$ the payoff $z_j$.
\end{claim}

\begin{claimproof}
    % Let us first not that using Property~\ref{itm:nosplit} of \cref{lm:Lambda}, we have that for each $k$, the vertex $\pi^A_{k+1}$ was actually the only element of $\Supp(\bsigma(h\pi^A_{<k}))$ satisfying the required properties in the construction of $\Lambda$.
    Let $j \in A$.
    Let us proceed by case disjunction according to the risk measure used by player $j$.

\emph{If player $j$ is an optimist,} then, by Property~\ref{itm:optimistanchor} of \cref{lm:Lambda}, we have $\X_j(\bsigma_{\|h}) = z_j$, and therefore $\prob_{\bsigma_{\|h}}(\mu_j = z_j) > 0$, i.e., by the law of total probability:
        $$\prob_{\bsigma_{\|hu}}(\pi^A) \prob_{\bsigma_{\|h}}(\mu_j = z_j \mid \pi^A) + \sum_k \sum_{w \in E(\pi_k) \setminus \{\pi_{k+1}\}} \prob_{\bsigma_{\|hu}}(\pi^A_{\leq k} w) \prob_{\bsigma_{\|hu}}(\mu_j = z_j \mid \pi^A_{\leq k}w) > 0.$$
        But using Property~\ref{itm:nosplit}, all the terms of the summation on the right are zero, hence the product $\prob_{\bsigma_{\|h}}(\pi^A) \prob_{\bsigma_{\|h}}(\mu_j = z_j \mid \pi^A) > 0$ is positive, i.e. the play $\pi^A$ has  a positive probability of being generated and $\mu_j(\pi^A) = z_j$.

   \emph{If player $j$ is a pessimist,} then because of Property~\ref{itm:nosplit} again, for every $k \geq 0$ and each $w \in \Supp\left(\bsigma\left(h\pi^A_{\leq k}\right)\right)$, there exists a strategy $\tau_j^{kw}$ such that $\X_j(\bsigma_{-j\|h\pi^A_{\leq k}w}, \tau_j^{kw}) > z_j$.
        By composing all those strategies, we obtain a deviation $\tau_j$ of the strategy $\sigma_{j\|h}$; which, by Property~\ref{itm:pessimistanchor}, satisfies the inequality $\X_j(\bsigma_{-j\|h}, \tau_j) \leq z_j$.
        Therefore, either:
        \begin{itemize}
            \item we have:
            $$\min_k \min_{w \in \Supp\left(\bsigma\left(h\pi^A_{\leq k}\right)\right) \setminus \{\pi^A_{k+1}\}} \X_j(\bsigma_{-j\|h\pi_{\leq k}w}, \tau^{kw}_j) \leq z_j,$$
            which is impossible by definition of the strategies $\tau_j^{kw}$;

            \item or we have $\prob_{\bsigma_{-j\|h}, \tau_j}(\pi^A) = \prob_{\bsigma_{\|h}}(\pi^A) \neq 0$ and $\mu_j(\pi^A) \leq z_j$, and then actually $\mu_j(\pi^A) = z_j$.
        \end{itemize}
        
         


We have thus proven that player $j$ gets the payoff $z_j$ in $\pi^A$, and that the play $\pi^A$ is generated with positive probability in $\bsigma$.
The analogous results about $\pi^{A\star}$ follow using the equalities $\Occ(\pi^{A\star}) = \Occ(\pi^A)$ and $\Inf(\pi^{A\star}) = \Inf(\pi^A)$.
\end{claimproof}

    In those two cases (if $A$ is a singleton or has several elements), we obtain that the strategy profile $\bsigma^\star$ is such that, with some positive probability, player $i$ gets the payoff $z_i$.


    \subparagraph*{Player $i$ gets risk measure $z_i$.}
    We still have to prove that player $i$ has zero probability of getting a lower payoff (if they are a pessimist) or a higher payoff (if they are an optimist).
    To show both cases, we prove the following claim:

    \begin{claim}
        Every payoff vector that has a positive probability of being achieved in the strategy profile $\bsigma^\star$ also has a positive probability of being achieved in the strategy profile $\bsigma$.
    \end{claim}

\begin{claimproof}
    Let $\bz'$ be such a payoff vector.
    Then, there is a history $hw$ compatible with $\bsigma^\star$ and a set $A \subseteq \Pi$ such that, after the history $hw$, the strategy profile $\bsigma^\star$ is in state $\anchor_{A \last(h)}$, and from that point it has a nonzero probability of achieving the payoff vector $\bz'$ while staying in states of the form $\anchor_{A v}$.
    
    \emph{If $A$ is empty}, then the strategy profile $\btau^{\anch \emptyset}$ has been defined as a strategy profile that almost surely generates a payoff vector that is generated with positive probability by $\bsigma$, from every vertex from which that is possible.
    That requirement is satisfiable, and therefore satisfied by $\btau^{\anch \emptyset}$, from the vertex $w$, since that vertex is itself reached with positive probability in the strategy profile $\bsigma$.
    Therefore, the payoff vector $\bz'$ is also achieved with positive probability in $\bsigma$.
    
    \emph{If $A$ is a singleton,} say $A = \{j\}$, then the strategy profile $\btau^{\anch j}$ has been defined so that from every vertex from which that is possible, on the one hand, it generates the payoff $z_j$ with positive probability, and on the other hand, it is almost sure that the payoff vector that will be generated has also positive probability to be generated in $\bsigma$.
    Similarly as above, that requirement is satisfiable from the vertex $w$, since $\bsigma_{\|hw}$ satisfies it.
    Therefore, again, the payoff vector $\bz'$ is also achieved with positive probability in $\bsigma$.
    
    \emph{If $A$ has at least two elements}, then the strategy profile $\bsigma^\star$ stays in states of the form $\anchor_{A v}$ only along one play, namely $\pi^{A\star}$, and that play generates a payoff vector that was also associated with the play $\pi^A$ that by \cref{claim:piA}, has positive probability to be generated in $\bsigma$, hence the same conclusion.
\end{claimproof}

This proves the equality $\X_i(\bsigma^\star) = z_i$.
\end{claimproof}


\paragraph*{The strategy profile $\bsigma^\star$ is an XRSE.}

We have now constructed the finite-memory strategy profile $\bsigma^\star$, showed that it had the expected number of memory states, and that it generates the expected risk measures.
We must now give the final argument for our construction: that strategy profile is also an extreme risk-sensitive equilibrium.
We will prove that result by showing separately that optimists have no profitable deviations, and then that neither do pessimists.

\begin{proposition}\label{prop:NodeviationOpt}
    No optimist has a profitable deviation in $\bsigma^\star$.
\end{proposition}

\begin{claimproof}
    Let $i$ be an optimist, and let us consider a deviation $\sigma'_i$ of that player from $\bsigma^\star$. Let us write $z'$ for the risk measure $z' = \X_i(\bsigma^\star_{-i}, \sigma'_i)$.

    Let us notice that along every play compatible with $\bsigma^\star_{-i}$, the transitions that are possible in the memory structure of the strategy profile $\bsigma^\star$ can be classified as follows:
    \begin{itemize}
        \item transitions among states of the form $\anchor_{A v}$ for a fixed $A$;
        
        \item transitions from a state of the form $\anchor_{A v}$ to a state of the form $\anchor_{B w}$ with $B \subset A$;

        \item transitions from a state of the form $\anchor_{A v}$ to the state $\punish_i$;

        \item and transitions from $\punish_i$ to itself
    \end{itemize}
    Therefore, any such play stabilises either in the state $\punish_i$, or among the states of the form $\anchor_{A v}$ for a fixed set $A$.
    Consequently, if in the strategy profile $(\bsigma^\star_{-i}, \sigma'_i)$ player $i$ gets the payoff $z'$ with positive probability, then we can also say that either:
    \begin{itemize}
        \item with positive probability, player $i$ gets the payoff $z'$ \emph{and} the state $\punish_i$ is reached;

        \item or there exists a set $A \subseteq \Pi$ such that with positive probability, player $i$ gets the payoff $z'$, and the collective memory remains in states of the form $\anchor_{A v}$.
    \end{itemize}

    \emph{In the first case,} let us consider a history $hv$ compatible with $\bsigma^\star_{-i}$ such that the collective memory is in an anchoring state after $h$ and in state $\punish_i$ after $hv$.
    If player $i$ can obtain the risk measure $z'$ by going to $v$ from that vertex against $\bsigma^\star_{-i\|h}$, and therefore, against the punishing strategy profile $\btau^{\dag i}_{-i}$, it means that they can enforce that risk measure against every possible strategy profile from $\last(h)$.
    On the other hand, if the collective memory is in an anchoring state after $h$, it means that the vertex $\last(h)$ is also visited with positive probability in the strategy profile $\bsigma$ (otherwise we would have switched to a punishing state earlier).
    There is therefore a history $h'$ compatible with $\bsigma$ such that $\last(h) = \last(h')$; and after that history, against the strategy profile $\bsigma_{\|h'}$, player $i$ also has the possibility of getting with positive probability the payoff $z'$.
    Since $\bsigma$ is an XRSE, that implies $z' \leq z_i$.

    \emph{In the second case,} let us notice that the strategy profiles of the form $\btau^{\anch A}$ are pure, and therefore that any deviation of player $i$ is immediately detected and leads to a switch to state $\punish_i$.
    Therefore, if the collective memory remains in states of the form $\anchor_{A v}$, it means that player $i$ is actually following the strategy $\sigma^\star_i$.
    Thus, we also have $z' \leq z_i$.
    
    The strategy $\sigma'_i$ is not a profitable deviation from $\bsigma^\star$.
\end{claimproof}

We can now end the proof with the dual proposition.

\begin{proposition}
    No pessimist has a profitable deviation in $\bsigma^\star$.
\end{proposition}

\begin{claimproof}
    Let $i$ be a pessimist, and consider a deviation $\sigma'_i$ of that player from $\bsigma^\star$.
    We intend to prove that the deviation $\sigma'_i$ is not profitable, that is, when following the strategy profile $(\bsigma^\star_{-i}, \sigma'_i)$, there is still a positive probability that player $i$ receives a payoff smaller than or equal to $z_i$.
    Using \cref{lm:secretlemma}, we can assume without loss of generality that $\sigma'_i$ is pure.

    First, we observe that for each history $hv$ compatible with $\bsigma^\star$ such that, after $hv$, the collective memory is in state $\anchor_{A \last(h)}$ with $i \in A$, the vertex $v$ is such that there also exists a history $h'v$ compatible with $\bsigma$ with $\Lambda(h'v) = A$.
    By Property~\ref{itm:pessimistanchor} of \cref{lm:Lambda}, we have $\X_i(\bsigma_{-i\|h'v}, \tau_i) \leq z_i$ for every $\tau_i$.
    Therefore, if player $i$ accepts to follow the history $hv$ and, then, deviates and takes an edge that makes the collective memory switch to the state $\punish_i$, then with positive probability player $i$ gets a payoff lesser than or equal to $z_i$.
    If such an action is ever performed, then the deviation $\sigma'_i$ is not profitable.

    Let us now assume that $\sigma'_i$ performs no such action: after every history $hv \in \Hist_i \Game_{\|v_0}$, if the collective memory is in a state of the form $\anchor_{A \last(h)}$ with $i \in A$, the vertex $\sigma'_i(hv)$ belongs to the set $\Supp(\sigma^\star_i(hv))$.
    Then, by Property~\ref{itm:splitsetsanchorwithi} of \cref{lm:Lambda}, we also have $i \in \Lambda(hv\sigma'_i(hv))$.
    Thus, there still exists a set $A$ with $i \in A$ such that, with positive probability, when following the strategy profile $(\bsigma^\star_{-i}, \sigma'_i)$, the strategy profile $\bsigma^\star_{-i}$ stabilises among memory states of the form $\anchor_{A v}$; and then, the strategy profile $\bsigma^\star$ only proceeds to pure actions, hence the strategy $\sigma'_i$ is actually following $\sigma^\star_i$.

    Using the same arguments as in the proof of \cref{prop:ActualPayoff} (definition of $\btau^{\anch i}$ in case $A = \{i\}$, and \cref{claim:piA} in case $A$ has more elements), we can then conclude that the player $i$ gets the payoff $z_i$ with positive probability and therefore the deviation $\sigma'_i$ is not profitable.
    \end{claimproof}

    The strategy profile $\bsigma^\star$ is an XRSE, satisfies the equality $\X(\bsigma^\star) = \X(\bsigma)$, and uses the desired number of memory states.
    Furthermore, if $\bsigma$ is pure, so is $\bsigma^\star$.
\end{proof}
\subsection{Proof of \cref{lemma:np_hardness}}\label{app:np_hardness}
\NPHard*
\begin{proof}[Proof of \cref{lemma:np_hardness}]
  We prove $\NP$-hardness by reducing from the problem $\THREESAT$. Consider a $\THREESAT$ formula $\Phi$, over the variables  $x_1,\dots,x_n$, where $\Phi = C_1\land C_2\land \dots\land C_m$, where for each $i$ we have $C_i = (\ell_{i1}\lor \ell_{i2}\lor \ell_{i3})$ and for $j=1,2,3$, we have $\ell_{ij} = x_k$ or $\ell_{ij} = \neg x_k$ for some $k\in \{1, \dots, n\}$. 
  We construct  a game $\Game_\Phi$ with two players for each literal $\ell$, denoted by $\Circle \ell$ and $\Square \ell$. The game is depicted in \cref{fig:NPhard}.
  For convenience, some terminal vertices have been represented several times.
  Each player $\Circle \ell$ controls one vertex, the vertex $\Circle \l$, of circled shape, and symmetrically, each player $\Square \ell$ controls the square-shaped vertex $\Square \l$.
  Further, we add a player $C_i$, who controls the vertex $C_i$, for each clause $C_i$. Finally, there is a player $\Diamond$ who does not control any vertex. There are also stochastic vertices, that are represented by the black circles.
  In each terminal vertex, the symbol $\forall$ should be understood as "every (other) player".

%The edges between the players are as in \cref{fig:NPhard}. From each clause $C_i$, we add edges to terminal vertices where the payoff of the player $\Circle \ell$ is $0$ and everyone else's is 2 if and only if $\ell$ is in the clause $C_i$. In the example, we assume $C_2 = x_2\lor x_4\lor \neg x_{11}$ and only draw edges from $C_2$ and not from other $C_i$. 

 We assume all players are pessimistic, and ask if there is an XRSE where player $\Diamond$'s risk measure is exactly $2$.
We give the formal definition of the game $\Game_\Phi$ below.

    \begin{figure}
        \centering
        
        \begin{tikzpicture}[shorten >=1pt, node distance=1.5cm and 2cm, on grid, auto, scale=1.1]
          %every node/.style={scale=0.6}
          % Smaller state style
          \tikzstyle{state}=[circle, draw, minimum size=20pt, inner sep=1pt]
          \tikzstyle{squarestate}=[rectangle, draw, minimum size=20pt, inner sep=1pt] 

            \node[state] (qnc1) at (0, 0) {$\neg x_{1}$};
            \node[state, initial,initial text=] (qc1) at (0, 1) {$x_{1}$};

            \node[scale=0.6] (tpunish1) at (0,-1) {$t_\dag:~\stack{\forall}{0}$};
            
            \node[stoch, scale=0.6] (stoc2) at (1, 0) {$s_{\neg x_1}$};
            \node[stoch, scale=0.6] (stoc1) at (1, 1) {$s_{x_1}$};

            \node[scale=0.6] (reward1) at (1,2) {$f_{x_1}:~\stack{\circ x_1}{1}$$\stack{\forall}{2}$};
            \node[scale=0.6] (reward2) at (1,-1) {$f_{\neg x_1}:~\stack{\circ \neg x_1}{1}$$\stack{\forall}{2}$};
            
            \node[squarestate] (qns1) at (2, 1) {$\neg x_{1}$};
            \node[squarestate] (qs1) at (2, 0) {$x_{1}$};

            \node[scale=0.6] (punishW1) at (2,2) {$t_\diamond:~\stack{\diamond}{0}$$\stack{\forall}{2}$};
            \node[scale=0.6] (punishW2) at (2,-1) {$t_\diamond:~\stack{\diamond}{0}$$\stack{\forall}{2}$};
            
            \node[state] (qnc2) at (3.5, 0) {$\neg x_{2}$};
            \node[state] (qc2) at (3.5, 1) {$x_{2}$};
            \node[stoch, scale=0.6] (stoc3) at (4.5, 1) {$s_{x_2}$};
            \node[stoch, scale=0.6] (stoc4) at (4.5, 0) {$s_{\neg x_2}$};

            \node[scale=0.6] (reward3) at (4.5,2) {$f_{x_2}:~\stack{\circ x_2}{1}$$\stack{\forall}{2}$};
            \node[scale=0.6] (reward4) at (4.5,-1) {$f_{\neg x_2}:~\stack{\circ \neg x_2}{1}$$\stack{\forall}{2}$};
            
            \node[squarestate] (qns2) at (5.5, 1) {$\neg x_{2}$};
            \node[squarestate] (qs2) at (5.5, 0) {$x_{2}$};

            \node[scale=0.6] (punishW3) at (5.5,2) {$t_\diamond:~\stack{\diamond}{0}$$\stack{\forall}{2}$};
            \node[scale=0.6] (punishW4) at (5.5,-1) {$t_\diamond:~\stack{\diamond}{0}$$\stack{\forall}{2}$};

            \node[scale=0.6] (tpunish2) at (3.5,-1) {$t_\dag:~\stack{\forall}{0}$};
            \node (qc3) at (6.5, 1) {};

            \node (dots) at (6.8,0.5) {$\dots$};

            \node[squarestate, initial,initial text=] (qnsn) at (8, 1) {$\neg x_{n}$};
            \node[squarestate, initial,initial text=] (qsn) at (8, 0) {$x_{n}$};

            \node[scale=0.6] (punishW5) at (8,2) {$t_\diamond:~\stack{\diamond}{0}$$\stack{\forall}{2}$};
            \node[scale=0.6] (punishW6) at (8,-1) {$t_\diamond:~\stack{\diamond}{0}$$\stack{\forall}{2}$};
            
            \node[stoch, scale=0.6] (stochfin) at (9,0.5) {$s_\mathsf{r}$};
            \node (fakenode) at (9,0.5) {};
            
            \node (c1) at (10,1.8) {$C_1$};
            \node (c2) at (10,1) {$C_2$};
            \node (cdots) at (10,0.4) {$\vdots$};
            \node (c3) at (10,-0.6) {$C_m$};

            \node[scale=0.6] (ter1) at (11.2, 1.7) {$t_{x_2}:~\stack{\Box x_2}{1}$ $\stack{\forall}{2}$};
            \node[scale=0.6] (ter2) at (11.2, 1) {$t_{x_4}:~\stack{\Box x_4}{1}$ $\stack{\forall}{2}$};
            \node[scale=0.6] (ter3) at (11.2, -0.3) {$t_{\neg x_{11}}:~\stack{\Box\neg x_{11}}{1}$ $\stack{\forall}{2}$};

          \path[->]
              (qc1) edge (stoc1)
              (qc1) edge (qnc1)
              (qnc1) edge (stoc2)
              (stoc1) edge (qns1)
              (stoc2) edge (qs1)
              (qns1) edge (qc2)
              (qs1) edge (qc2)
              (qc2) edge (stoc3)
              (qc2) edge (qnc2)
              (qnc2) edge (stoc4)
              (stoc3) edge (qns2)
              (stoc4) edge (qs2)
              (qns2) edge (qc3)
              (qs2) edge (qc3)
              (qnsn) edge (stochfin)
              (qsn) edge (stochfin)
              (fakenode) edge (c1)
              (fakenode) edge (c2)
              (fakenode) edge (c3)
              (c2) edge (ter1)
              (c2) edge (ter2)
              (c2) edge (ter3)
              (qnc1) edge (tpunish1)
              (qnc2) edge (tpunish2);

        \path[->]
            (stoc1) edge (reward1)
            (stoc2) edge (reward2)
            (stoc3) edge (reward3)
            (stoc4) edge (reward4)
            (qs1) edge (punishW2)
            (qns1) edge (punishW1)
            (qs2) edge (punishW4)
            (qns2) edge (punishW3)
            (qsn) edge (punishW6)
            (qnsn) edge (punishW5);
        
        \end{tikzpicture}
        \caption{Construction of a game $\Game_\Phi$ from a $\THREESAT$ formula $\Phi$}
        \label{fig:NPhard}
    \end{figure}


% We will construct a game $\Game_\Phi$ with $O(n+m)$-players, constraints $\Bar{x},\Bar{y}$ and where at least $2n$ players are pessimists, such that the game has a $\Bar{\rho}$-RSE if and only if $\Game_\Phi$ is satisfiable. 

\subparagraph*{Construction of the game $\Game_\Phi$: vertices, edges and payoffs.}
For each literal $\ell$, we define two players $\Square\ell$ and $\Circle\ell$. We add one other player $C_i$ for each clause $C_i$, and an additional constraining player $\Diamond$.
All players are pessimists.

Each player owns at most one vertex in the game, and therefore, we will refer to the player and vertex interchangeably. There is one vertex for each of the players mentioned above other than $\Diamond$, who owns no vertices. Further, there are $2n + 1$ many stochastic vertices: one for each literal $s_{x_1},s_{x_2},\dots,s_{x_n}$,  $s_{\neg x_1},s_{\neg x_2},\dots,s_{\neg x_n}$, and finally one clause-randomiser $s_\mathsf{r}$. 
There are also $2n + 2$ terminal vertices, written $f_{\ell}$ and $t_{\ell}$ for each literal $\ell$, and further the terminal vertices $t_\Diamond$ and $t_\dag$.

We now define the edges between the vertices of the graph for all $i\in \{1, \dots, n\}$:  there are edges from $\Circle x_i$ to $\Circle\neg x_i$, and edges from $\Circle\neg x_i$ to $t_\dag$.
    Further, for every literal $\ell = x_i$ or $\neg x_i$, there are edges:
    \begin{itemize}
        \item from $\Circle\ell$ to $s_{\ell}$;
        \item from $s_{\ell}$ to $f_\ell$ and to $\Square\Bar{\ell}$, where $\Bar{\ell} = \neg x_i$ if $\ell = x_i$ and $\Bar{\ell} = x_i$ if $\ell = \neg x_i$;
        \item from $\Square\ell$ to $t_\Diamond$;
        \item from $\Square\ell$ to $\Circle x_{i+1}$ if  $i<n$,  and  to $s_\mathsf{r}$ if  $i=n$.%, edges are added
    \end{itemize}
    Finally, for all clauses $C_j$, there are edges from $s_\mathsf{r}$ to $C_j$ and from $C_j$ to $t_{\ell}$ such that $\ell$ occurs positively in the clause $C_j$.

    The terminal vertices yield the following payoffs.
\begin{itemize}
    \item In terminal $t_\ell$, all players get payoff $2$, except the player $\Square \ell$ who gets payoff $1$. 
    \item In terminal $f_\ell$, all players get payoff $2$, except player $\Circle \ell$ who gets payoff $1$.
    \item In terminal $t_\dag$, all players get payoff $0$.
    \item In terminal $t_\Diamond$, all players get payoff $2$, except player $\Diamond$ who gets payoff $0$.% , and player $\Diamond$ get payoff $0$. 
\end{itemize}    

Finally, we let the constraints be that player $\Diamond$ gets a risk measure of exactly $2$.
Equivalently, we define $\bx$ and $\by$ by $\by = (2)_{i \in \Pi}$, $x_i = 0$ for each $i \in \Pi \setminus \{\Diamond\}$, and $x_\Diamond = 2$.

% and $\Square x_i$

% All players in the game are pessimistic players    \theju{Need to detail who needs to be pessimistic.}
    % We draw an example formula with 3 variables and two clauses. 

    % \begin{example}
    %     Consider clause $C_1 = (x_1\lor \neg x_2\lor x_3)$, and $C_1 = (\neg x_1\lor  x_2\lor \neg x_3)$
    %         \thejaswini{To do add an example}
    %         \leonard{Is that really necessary? Your figure above is pretty clear (and clearly pretty).}
    % \end{example}
% We first make a claim whose proof is straight forward and can be shown by the definition of RSE, and using the fact that player $\Square\ell$ is a pessimistic player.

% \begin{claim}
%     If vertex $\Square \ell$ is visited in an RSE, then terminal $t_\ell$ must be visited probability $0$. Similarly, if terminal $t_\ell$ is visited with non-zero probability, then the strategy 
% \end{claim}

    \subparagraph*{If $\Phi$ is satisfiable, then there is an XRSE satisfying the constraints.}
    Consider a satisfying assignment of the $\THREESAT$ formula, described by the assignment $\alpha$ from the set of all variables to $\{\top,\bot\}$.
    
    For each $i$, let $\ell_i$ denote the literal, among $x_i$ and $\neg x_i$, which is set to true by 
    the satisfying assignment $\alpha$.
    Let us define the (positional) strategy profile $\bsigma^\alpha$.
    
    \begin{itemize}
        \item Player $\Circle \ell_i$ goes to $s_{\ell_i}$.
        \item Player $\Circle x_i$ goes to $s_{x_i}$ if $\alpha(x_i) = \top$, and to $\Circle \neg x_i$ otherwise.
        \item Player $\Circle\neg x_i$ goes to $s_{\neg x_i}$ if $\alpha(x_i) = \top$ and to $t_\dag$ otherwise.
        \item For each player $\Square \ell$, the strategy is to chose the edge that does \emph{not} lead to $t_\Diamond$. That is, the edge to $\Circle x_{i+1}$ if $\ell = x_i$ or $\neg x_{i}$ and  $i<n$,  and  the edge to $s_\mathsf{r}$ if  $i=n$.
        \item Each clause player $C_i$ takes the edge to the vertex $\ell_j$ such that the litteral $\ell_j$ was set to true by the satisfying assignment $\alpha$. 
    \end{itemize}
     We now show that this is an XRSE that satisfies the constraint. First, we verify if the constraints are satisfied. Observe that following the strategy profile $\bsigma^\alpha$, it is almost sure that none of the terminals where player $\Diamond$ has payoff less than $2$ will be reached. Therefore this satisfies the constraints. 

     We now argue that $\bsigma^\alpha$ is an XRSE, i.e. that no player can get a better risk measure by deviating.
     The result is immediate for player $\Diamond$ and for the clause players, who all get risk measure $2$, the best they could hope for.
     
     For each literal $\l$, player $\Circle\ell$ gets risk measure $1$ if $\l$ is set to true, and risk measure $2$ if $\ell$ is set to false.
     The same argument as above holds therefore in the second case.
    In the first case, they get risk measure $0$, but they have no profitable deviation, since the only deviation available leads to $t_\dag$ and to the payoff $0$.
     
     Player $\Square \ell$  has also risk measure  $2$ when $\ell$ is set to false.
     Otherwise, they get payoff $1$. In that second case, the vertex owned by the player is not visited in any history of the game, hence they have no possibility of deviating.

     The (positional) strategy profile $\bsigma^\alpha$ is therefore an XRSE.
     
    \subparagraph*{If there is an XRSE satisfying the constraints, then $\Phi$ is satisfiable.} 
    Let us assume that there exists an XRSE $\bsigma$ in the game $\Game_\Phi$, such that player $\Diamond$ gets the risk measure $2$.
    We prove, first, that we can assume that $\bsigma$ is pure (and therefore positional, since there is then only one history leading to each vertex).

\begin{claim}
    There exists an XRSE $\bsigma^\star$ in $\Game_{\|v_0}$ where player $\Diamond$ gets risk measure $2$ that is positional.
\end{claim}

\begin{proof}
    Let us first focus on what happens in vertices that have positive probability of being reached.

    If the vertex $\Circle\neg x_i$ has a positive probability of being reached in $\bsigma$, then any strategy of the player $\Circle \neg x_i$ that goes to $t_\dag$ with positive probability gives the player $\Diamond$ the risk measure $0$.
    Therefore, necessarily, the strategy $\sigma_{\circ \neg x_i}$ consists of deterministically going to $s_{\neg x_i}$.
    The same argument holds for the vertices of the form $\Square \l$.
    
    If now the vertex $\Circle x_i$ has a positive probability of being reached and if the player $\Circle x_i$ randomises between the two edges available, then she gets the risk measure $1$, since the terminal vertex $f_{x_i}$ is reached with positive probability and $t_\dag$ with probability zero.
    But then, if she deviates and goes to the vertex $\Circle \neg x_i$ with probability $1$, she avoids the terminal vertex $f_{x_i}$, and the other players will not react since they do not detect the deviation.
    She therefore gets the risk measure $2$, and the deviation is profitable.
    Consequently, the strategy $\sigma_{\circ x_i}$ can only  deterministically select one of those two edges.

    At the end of the game, for each $j$, the player $C_j$ could play a randomised strategy. In such a case, her strategy can be replaced by a pure strategy that takes, deterministically, one of the edges that she was previously taking.
    Such a modification in her strategy can only increase the risk measure of some players (namely, those of the form $\Square \l$) without impacting player $\Diamond$'s risk measure or giving any player the possibility of profitably deviating.

    Finally, if one of those vertices is reached after a history that is not compatible with $\bsigma$, i.e. if one of those players deviates: it is necessarily due to a deviation of a player of the form $\Circle x_i$, since any other deviation would immediately lead to a terminal vertex.
    If she went to $s_{x_i}$ instead of $\Circle \neg x_i$, what the other players do afterwards does not matter, since such a deviation cannot be profitable: with positive probability, the terminal vertex $f_{x_i}$ is reached, and she gets payoff $1$.
    If she went to $\Circle x_i$ instead of $s_{x_i}$, then we can assume that player $\Circle \neg x_i$'s strategy consists of going to the terminal vertex $t_\dag$, giving her the payoff $0$.
    Those modifications do not impact the fact that $\bsigma$ is an XRSE.
\end{proof}
    % If the vertex $\Circle\neg x_i$ has a positive probability of being reached, then any strategy of player $\Circle \neg x_i$ that goes to $t_\dag$ with positive probability gives player $\Diamond$ the risk measure $0$.
    % Therefore, necessarily, the strategy $\sigma_{\circ \neg x_i}$ consists in deterministically going to $s_{\neg x_i}$.
    % A consequence of that fact is that player $\Circle x_i$, who can get the payoff $0$ only in the terminal vertex $t_\dag$, gets in the strategy profile $\bsigma$ a risk measure greater than or equal to $1$.

    % If now the vertex $\Circle x_i$ has positive probability to be reached, then player $\Circle x_i$ is also necessarily playing a pure strategy: if she randomises between the two actions available, then we argue that there is an undetectable deviation possible at the vertex $\Circle x_i$. Player $\Circle x_i$ would always prefer the edge to $\Circle\neg x_i$ since this removes entirely the path that gives the player $\Circle x_i$ the possibility of payoff $0$.
    
    % Since there is no randomisation possible at vertices $\Circle x_i$, to ensure that player $\Diamond$ gets payoff $2$, the terminals $t_\Diamond$ or $t_\dag$ cannot be visited from any of the states $\Square\ell$.  This means that the strategy from every vertex $\Square \ell$ must be to visit $\Circle x_{i+1}$ or $s_\mathsf{r}$ next. However, observe $t_\Diamond$ ensures a payoff of $2$ to player $\Square\ell$. 
    % To ensure that the strategy is an RSE, and that the player $\Square \ell$ does not have an incentive to deviate to this edge, the outcome of the strategy to player $\Square \ell$ should be $2$ (Any lower risk-expectation of player $\Square\ell$ and would ensure deviation). This is true for any vertex $\Square\ell$ that has a positive probability of being visited by the strategy.    
    % This also rules out the possibility of randomisation at vertices $\Square\ell$ for vertices that are visited with positive probability according to the strategy.
    % Since there is no randomisation at both $\Circle\ell$ as well as $\Square\ell$ vertices. If the stochastic vertex $s_\mathsf{r}$ is reached, then it is reached after visiting a unique path (assuming no deviations in the strategy).
% \begin{claim}
%      The vertex $\Square \ell$ is visited by a strategy $\sigma$ that is an RSE and  satisfies the constraint if and only if terminal $t_\ell$ is be visited probability $0$ by $\sigma$.% Similarly, if terminal $t_\ell$ is visited with non-zero probability, then an RSE that satisfies the constraints does not visit vertex $\Square\ell$ with non-zero probability. 
%  \end{claim}
%    This follows by observing that no literal $\ell$ such that  
 % \begin{claimproof}
 %        Let $\sigma$ denote such an RSE. 
 %     Consider the unique path based on strategy $\sigma$ that is used to reach the vertex $s_\mathsf{r}$. This path visits visits either $\Square x_i$ or $\Square\neg x_i$, based on the structure of the graph. We show that if a literal $\Square \ell$ is visited, then $\ell\notin S$. If $\Square\ell$ is visited, then player $\Square\ell$ has an incentive to deviate and instead chose edge $\Square \ell\rightarrow t_\Diamond$, thus making his risk-expectation $2$, but also reducing the risk expectation of player $\Diamond$.
 % \end{claimproof}   

We therefore assume that $\bsigma$ is positional.
Let us now define the assignment $\alpha$ as follows: for each variable $x_i$ we have $\alpha(x_i) = \top$ if $\sigma_{\circ x_i}(\Circle x_i) = s_{x_i}$, and $\alpha(x_i) = \bot$ if $\sigma_{\circ x_i}(\Circle x_i) = \Circle \neg x_i$.
Let then $C_j$ be a clause, and let us prove that it is satisfied by $\alpha$.
Let $t_\l = \sigma_{C_j}(C_j)$.
Then, the player $\Square \l$ gets risk measure $1$ in the XRSE $\bsigma$.
Consequently, the vertex $\Square \l$ is never reached: otherwise, the only play compatible with $\bsigma$ in which player $\Square \l$ gets payoff $1$ would traverse the vertex $\Square \l$, and player $\Square \l$ would have a profitable deviation by going to the terminal vertex $t_\Diamond$.
If that is the case, then the definition of $\alpha$ given above implies that the literal $\l$ is true.

The assignment $\alpha$ satisfies therefore the formula $\Phi$.

\subparagraph*{Conclusion.}
We have defined an instance of the constrained existence problem of XRSEs from an instance of $\THREESAT$ and proved that one is a positive instance if and only if the other is.
This proves the $\NP$-hardness of the constrained existence problem of XRSEs, since the game $\Game_\Phi$ can clearly be constructed in polynomial time.
Moreover, the game $\Game_\Phi$ is such that if an XRSE where player $\Diamond$ gets risk measure $2$ exists, then there also exists such an equilibrium that it positional, which proves also $\NP$-hardness when the players are restricted to pure, stationary or positional strategies.
\end{proof}
