We now answer a fundamental question about equilibria, which is if one always exists. 
% This existence result does not trivially extend to quantitative rewards, where the rewards can be negative.
% Based on their result, we were also able to show in the previous section that there is always a risk-sensitive equilibrium when the risk-measures are the entropic risk measures and the rewards are non-negative.\leon{Maybe we could reduce this paragraph? Sounds like repetition now.}
We show that (stationary) XRSEs are guaranteed to exist in games with only non-negative rewards, similarly as ERSEs.
But our proof does not rely on the same arguments, and we instead give a constructive proof.
% Our result does not follow from any of the previous existence of equilibria, since our risk measure is the limit value of entropic risk measure.

\begin{restatable}[App.~\ref{app:XRSEexists}]{theorem}{XRSEexists}\label{thm:XRSEexists}
    Let $\Game_{\|v_0}$ be a game with only non-negative rewards, and let $(P,O)$ be a partition of $\Pi$.
    Then, there exists a stationary XRSE in $\Game_{\|v_0}$.
    Moreover, there exists an algorithm that, given such a game, outputs the representation of such an XRSE in time $\Oh(m^2 p)$, where $m$ is the number of edges, and $p$ the number of \emph{pessimistic} players.
\end{restatable}

\begin{proof}[Proof sketch.]
    Our algorithm generates an XRSE by constructing a decreasing sequence $E = E_0, E_1, \dots$ of sets of edges, and considering, for each $k$, the stationary strategy profile that randomises between all the outgoing edges in $E_k$ from all vertices.
    
    Let us illustrate it with the game depicted by Figure~\ref{fig:ex_extreme1}, which involves two pessimists, player $\Circle$ and player $\Square$.
    In that game, both players want to leave the cycle, but each of them would prefer the player to leave.
    If we first consider the strategy profile that always randomises between all the available edges, then both terminal vertices are reached with positive probability, and it is almost sure that one of them is reached: both players get therefore risk measure $1$.
    Then, player $\Square$ (and symmetrically player $\Circle$) has a profitable deviation by refusing to leave the cycle, and always going back to the vertex $a$.
    Note that player $\Circle$ cannot detect such a deviation of strategy, since she does not have access to the internal coins tossed by player $\Square$. %cannot know whether player $\Square$ went back to $a$ as a deviation, or as one of the possible outcomes of his randomised strategy: that deviation is \emph{undetectable}.
    Then, we remove the edge $bt_2$ (or $at_1$). This results in a set of edges where player $\Square$ gets the payoff $2$, and player $\Circle$ cannot get more than $1$, ensuring that the new strategy profile that we obtain is a (stationary) XRSE.
\end{proof}

Like in the case of ERSEs, we conjecture that existence, and even existence of a stationary strategy profile, remain true in the general case.


\begin{figure}[h] 
			\centering
            \begin{subfigure}[t]{0.3\textwidth}
			\begin{tikzpicture}[->,>=latex,shorten >=1pt, initial text={}, scale=1, every node/.style={scale=0.9}]
				\node[initial above, state] (a) at (0, 2) {$a$};
				\node[state, rectangle] (b) at (2, 2) {$b$};
                \node (t1) at (0, 0.75) {$t_1:~\stack{\circ}{1}~\stack{\square}{2}$};
                \node (t2) at (2, 0.75) {$t_2:~\stack{\circ}{2}~\stack{\square}{1}$};
                \path (a) edge[bend left] (b);
				\path (b) edge[bend left] (a);
                \path (a) edge (t1);
                \path (b) edge (t2);
			\end{tikzpicture}
			\caption{Two pessimists}
			\label{fig:ex_extreme1}
            \end{subfigure}
            \hfill
            \begin{subfigure}[t]{0.3\textwidth}
			\begin{tikzpicture}[->,>=latex,shorten >=1pt, initial text={}, scale=1, every node/.style={scale=0.9}]
                \node[initial above, state, stoch] (c) at (1, 3) {$c$};
				\node[state] (a) at (0, 2) {$a$};
				\node[state, rectangle] (b) at (2, 2) {$b$};
                \node (t1) at (0, 0.75) {$t_1:~\stack{\circ}{1}~\stack{\square}{2}$};
                \node (t2) at (2, 0.75) {$t_2:~\stack{\circ}{2}~\stack{\square}{1}$};
                \path (c) edge (a);
                \path (c) edge (b);
                \path (a) edge[bend left] (b);
				\path (b) edge[bend left] (a);
                \path (a) edge (t1);
                \path (b) edge (t2);
			\end{tikzpicture}
			\caption{Two pessimists with a common coin}
			\label{fig:ex_extreme2}
            \end{subfigure}
            \hfill
            \begin{subfigure}[t]{0.3\textwidth}
			\begin{tikzpicture}[->,>=latex,shorten >=1pt, initial text={}, scale=1, every node/.style={scale=0.9}]
                \node[initial above, state, stoch] (c) at (1, 4.25) {$c$};
                \node[state] (d) at (-0.5, 3.25) {$d$};
                \node[state, rectangle] (e) at (2.5, 3.25) {$e$};
				\node[state] (a) at (0, 2) {$a$};
				\node[state, rectangle] (b) at (2, 2) {$b$};
                \node (t1) at (0, 0.75) {$t_1:~\stack{\circ}{1}~\stack{\square}{2}$};
                \node (t2) at (2, 0.75) {$t_2:~\stack{\circ}{2}~\stack{\square}{1}$};
                \node (t3) at (1, 3.25) {$t_3:~\stack{\circ}{2}~\stack{\square}{2}$};
                \path (c) edge (d);
                \path (c) edge (e);
                \path (d) edge (t3);
                \path (e) edge (t3);
                \path (d) edge (a);
                \path (e) edge (b);
                \path (a) edge[bend left] (b);
				\path (b) edge[bend left] (a);
                \path (a) edge (t1);
                \path (b) edge (t2);
			\end{tikzpicture}
			\caption{Two pessimists with a common coin and some temptation}
			\label{fig:ex_extreme3}
            \end{subfigure}
        \caption{Some games involving two pessimistic players}\label{fig:ex_extreme}
\end{figure}