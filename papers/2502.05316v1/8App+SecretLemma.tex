\begin{lemma}\label{lm:secretlemma}
    Let $\Game_{\|v_0}$ be a game with two players, called $i$ and $j$.
    We assume given a partition $(P, O)$ of $\{i, j\}$.
    Then, the quantity:
    $$\inf_{\sigma_j \in \Strat_j\Game_{\|v_0}} \sup_{\sigma_i \in \Strat_i\Game_{\|v_0}} \X_i(\sigma_i, \sigma_j)$$
    can be computed in time $\Oh(m)$, where $m$ is the number of edges in $\Game$.
    Moreover, the infimum is reached with a positional strategy of player $j$; and there is a positional strategy of player $i$ that realises the supremum for every strategy of player $j$.

    Consequently, the optimality of positional strategies and the $\Oh(m)$ upper bound also hold in Markov decision processes, and in Markov chains; and, on the other hand, it holds when $j$ is a fictional player that represents a coalition of players who all have as unique objective to minimise player $i$'s risk measure.
\end{lemma}

\begin{proof}
    The $\Oh(m)$ upper bound holds by a slight adaptation of the classical attractor algorithm~\cite[Chapter~5.3]{AG11}.
    Note that those algorithms run in time $\Oh(m+n)$, where $n$ is the number of vertices; but here, we assumed that each vertex (except possibly $v_0$) has at least one ingoing edge, hence $n \leq m+1$ and $m+n = \Oh(m)$.
    That algorithm immediately induces positional optimal strategies.
    Another way to obtain that second result, however, is the following: once the quantity $x = \inf_{\sigma_j} \sup_{\sigma_i} \X_i(\sigma_i, \sigma_j)$ is known, strategies that realise the infimum and the supremum can be seen as optimal strategies in the Boolean zero-sum game in which player $i$ wants with positive probability (if they are optimist) or with probability $1$ (if they are pessimist) to reach the set of terminals yielding them at least payoff $x$ (if $x > 0$) or to avoid the set of terminals yielding them less than payoff $x$ (if $x \leq 0$).
    This is then a reachability game (seen either from player $i$'s of from player $j$'s perspective), and it is well-known~\cite[Chapter~5.3]{AG011} that in such a game, for both players, positional strategies suffice to maximise the probability of winning.
    In particular, if one has a strategy to win that game with positive probability, or with probability $1$, there is also such a strategy that is positional.
\end{proof}