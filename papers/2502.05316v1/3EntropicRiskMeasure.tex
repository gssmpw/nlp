The entropic risk measure is a measure of the perceived payoff, which depends on the aversion or inclination of the player toward risk through the exponential utility function. 
It is defined using a \emph{risk parameter}, i.e. a real value $\rho\in\Rb \setminus \{0\}$: large positive values indicate risk-averseness, large negative values risk-inclination. To see a visual representation of the entropic risk measure, see \cref{fig:example_re} in the introduction.

\begin{definition}[Entropic risk measure]
Given a risk parameter $\rho$, the \emph{entropic risk measure} is defined for every probability measure $\prob$ and random variable $X$ as
$$\re_{\rho}^\prob[X] = -\frac{1}{\rho} \log_e \left( \Eb^\prob \left[ e^{-\rho X}\right] \right).$$
For computational reasons, this definition is generalised by allowing every base $\beta > 1$ instead of Euler's constant. The \emph{entropic risk measure with base $\beta$} is then defined by: 
$$\re^\prob_{\beta\rho}[X] = -\frac{1}{\rho} \log_\beta \left( \Eb^\prob \left[ \beta^{-\rho X}\right] \right).$$
\end{definition}

The three parameters $\prob$, $\rho$ and $\beta$ can be omitted when they are clear from the context.

\begin{remark}
\begin{itemize}
    \item For every $\beta$ and $\rho$, the entropic risk measure $\RM_{\beta\rho}$ is a risk measure.

    \item By enabling any base $\beta$, we obtain a definition that is more general only on a computational level, since handling Euler's constant may not be equivalent to handling rational values.
    Baring computational concerns, these definitions with different bases are equivalent, since for every $\beta$ we have $\RM_{\beta\rho} = \RM_{e\rho'}$, where $\rho' = \rho \log_e(\beta)$.

    \item The above definition implies that for $\rho = 0$, the function is not defined.
    However, it is known that for all $\prob$, $\beta$ and $X$, the quantity $\RM_{\rho}$ converges to $\Eb[X]$ when $\rho$ tends to $0$ (see e.g.~\cite{PDM20}).
    Therefore, we henceforth assume that $\RM_{0}[X] = \mathbb{E}[X]$ to make risk entropy defined for all finite risk parameters $\rho$.
\end{itemize}
\end{remark}

When we are given a profile $\brho = (\rho_i)_{i \in \Pi}$ of risk parameters, we will sometimes write $\M_{\beta\brho}[\mu]$ for the tuple $\left(\M_{\beta\rho_i}[\mu_i]\right)_{i \in \Pi}$.
Risk entropy defines a family of RSEs, namely the $(\M_{\beta\rho_i})_i$-RSEs, that we also call \emph{$(\beta, \brho)$-entropic risk-sensitive equilibria}, or $(\beta, \brho)$-ERSEs.
% \begin{restatable}{lemma}{ERzeroExp}\label{lemma:ERzeroExp}\theju{would be nice to add a citation, but can't seem to find any}
% The limit risk entropy of $X$ when $\rho$ tends to $0$ exists, and equals the expectation $\lim_{\rho \to 0} \re_{\beta,\rho} [X] = \Eb[X]$.
% \end{restatable}
% The above lemma shows that for each value $\beta$ and risk-sensitivity profiles profiles $\brho$, we can define a new risk measure $\re_{\beta,\rho}$ for each value of $\rho$. For a risk sensitivity profile $\brho$ which assigns a risk parameter for each player, we obtain 
% a risk measure $\re_{\beta,\rho}$ for each player. 
% We refer to an RSE where the risk-measures for player $i$ is $\re_{\beta,\rho_i}$ as a $(\beta,\brho)$-RSE. 
The following theorem states the existence of such an RSE that uses no randomness in its strategy profile, in cases where all the payoffs are non-negative. 

\begin{theorem}[Existence of ERSE]\label{thm:existanceRSE}
    Let $\Game_{\|v_0}$ be a simple stochastic game with only non-negative payoffs.
    Then, there exists a (pure) $(\beta,\rho)$-ERSE over $\Game_{\|v_0}$.
\end{theorem}

\begin{proof}
 Pure Nash equilibria always exists in a stochastic multi-player games with prefix-closed Boolean objectives~\cite[Theorem 3.10]{Umm10} (a correction of an existing proof~\cite{CMJ04}). It is known that simple stochastic games where rewards are all positive (or all negative) can be converted into a game with reachability objectives such that if there is an NE in one, there is an NE in the converted game with the reachability objective. Indeed, if all the rewards are positive, we can always scale the rewards for each player of a stochastic game to ensure they are in the unit interval $[0,1]$. If the rewards are within the unit interval, then for terminals with reward $p$, we can instead add a probabilistic node that reaches this terminal vertex with probability $p$. 
Therefore, with the same result, Nash equilibria always exist in simple stochastic games with non-negative rewards on the terminals. 

Then, we can conclude our theorem using the following lemma.

\begin{restatable}[App.~\ref{lemma:RSEtoQSSG}]{lemma}{RSEtoQSSG}\label{lemma:RSEtoQSSG}
Given a game $\Game_{\|v_0}$ and a tuple $\brho \in \Rb^\Pi$, there exists a game $\Game'_{\|v_0}$ with the same underlying graph, player set, and probability function (but possibly different payoff function), such that the $(\beta,\rho)$-ERSEs in $\Game_{\|v_0}$ are exactly the Nash equilibria in $\Game'_{\|v_0}$. \qedhere
\end{restatable}
\end{proof}

We conjecture that this result remains true when we remove the guarantee that rewards are non-negative.
We now turn to the constrained existence problem of $\tpl{\beta,\brho}$-ERSEs.
Unfortunately, it is undecidable in the general case.

\begin{proposition}\label{proposition:Undecidable}
    The constrained existence problem of $\tpl{\beta,\brho}$-ERSEs with $\brho \in \Qb^{\Pi}$ is undecidable, even for any fixed value of $\beta$, for $\brho = (0)_i$, and with only nonnegative payoffs. %Further, even when players are restricted to pure strategies, the problem remains undecidable
\end{proposition}

\begin{proof}
         The undecidability of the constrained existence problem follows from the work of Ummels and Wojtczak~\cite[Theorem 4.9]{UW11} where they show the undecidability of the constrained existence problem for Nash equilibria in the setting with 10 or more players. Since Nash equilibria is a specific instance of the setting of ERSEs where the risk parameters $\brho$ is $0$ for each player, the undecidability of our setting follows. 
\end{proof}

We therefore turn our attention to the constrained existence problem when the class of strategies considered is restricted. 

\begin{restatable}[App.~\ref{app:ERRSErestricted}]{theorem}{stationaryRSE}\label{thm:ERRSErestricted}
The constrained existence problem of $(\beta,\brho)$-ERSEs, in quantitative simple stochastic games:
\begin{enumerate}
    \item remains undecidable when players are restricted to pure strategies;\label{itm:ERRSEitmundec} %(which use no randomness but arbitrary amounts of memory) is undecidable;
    \item is decidable when players are restricted to stationary strategies\label{itm:ERRSEdecidable}
\begin{enumerate}
        \item subject to Shanuel's conjecture if $\beta = e$ and the risk-parameters $\rho_i$ are algebraic;\label{itm:ERRSEitmShanuel}
        \item in $\PSPACE$ if the risk parameters and the base $\beta$ are algebraic, in which case it is also $\NP$-hard and $\SQRTSUM$-hard.\label{itm:ERRSE:PSPACE}
        The $\NP$ lower bound also holds for the case where strategies are restricted to positional strategies.
    \end{enumerate}
\end{enumerate}
\end{restatable}
%The results are obtained as a combination of results from 

\begin{proof}[Proof Sketch]
    The undecidability of the case where pure strategies are considered is inherited from Nash equilibria~\cite[Theorem~4.9]{UW11}, since the reduction uses only pure strategies. 
        The decidability of this stationary case is reminiscent of similar results for the two-player zero-sum case, which was recently studied in the work of Baier et al.~\cite{BCMP24}.
    However, the techniques used are quite different and also require inspiration from the work of Ummels and Wojtczak~\cite[Theorem 4.5, Theorem 4.6]{UW11}, with significant modifications. 
    We write formulas in the existential theory of reals ($\exists\Rb$) which puts them in $\PSPACE$. 
    For the case $\beta = e$, this formula can be written in the existential theory of reals with exponentiation, which is decidable subject to Shanuel's conjecture, which is a well-known conjecture in the field of transcendental number theory~\cite{Lan66}. The lower bounds of $\NP$-hardness and $\SQRTSUM$-hardness  also follow from the works of Ummels and Wojtczak~\cite[Theorem~4.4,Theorem~4.6]{UW11}. The exact complexity of $\SQRTSUM$ (deciding, given a set $\{a_1, \dots, a_n\} \subseteq \Nb$ and an integer $t$, whether we have $\sum_i \sqrt{a_i} \leq t$) is open and is known to lie in the polynomial hierarchy and in the fourth level of the counting hierarchy~\cite{AKBM06}. 
\end{proof}


% \begin{lemma}\label{lemma:stationaryRSE}
%         The constrained existence problem for RSE when players are restricted to pure strategies for quantitative simple stochastic games is undecidable.
% \end{lemma}
% \begin{proof}
%\end{proof}
% \begin{restatable}{lemma}{stationaryRSE}\label{lemma:stationaryRSE}
%         The constrained existence problem for $(\beta,\brho)$-RSE when players are restricted to stationary (but stochastic) strategies for quantitative simple stochastic games is \begin{itemize}
%             \item in $\PSPACE$ if the risk-parameters for each player $\brho$ as well as the base $\beta$ is algebraic; 
%             \item decidable---subject to Shanuel's conjecture---if the risk-parameters for each player $\brho$  is algebraic and the base of the exponent is the Euler's constant $\beta=e$.
%             \end{itemize}
%         The problem is at least $\NP$-hard. When the strategies are stationary but also randomised, the problem is also $\SQRTSUM$-hard.
    

    % We know from \cref{lemma:RSEtoQSSG}, we show that %
     

    %Further, we need the results in Baier et al.,~\cite{BCMP24} to reason about computational related to checking if a strategy 

%\theju{to check if this doesn't hit really low values. Need to bound it}


% \begin{lemma}\label{lemma:positionalRSE}
%         The constrained existence problem for RSE when players are restricted to stationary pure strategies for quantitative simple stochastic games is  in $\PSPACE$ if the risk-parameters for each player $\brho$ as well as the base $\beta$ is algebraic. Further, the problem is $\NP$-hard.
% \end{lemma}
% \begin{proof}
%     For the lower bounds, we just remark that the corresponding problem of constrained existence of Nash equilibria, when players are restricted to positional strategies is both $\NP$-hard~\cite[Theorem 4.4]{UW11}
%     from the work of Ummels and Wojtczak. Since Nash equilibria is a specific instance of the setting of RSE, where the risk parameters of each player is $0$, the $\NP$-hardness for the more general case of RSE follows.
%     For the upper bound, we simply observe that the same proof as the one for the upper bound of~\cref{lemma:stationaryRSE} can be adapted to this situation. 
% \end{proof}

% From this point, we will therefore extend, by continuity the definition of $\RE_{\beta,\rho}$ for $\rho = 0$.




% \begin{definition}[Risk-sensitive Equilibria]
%     Let $\Game_{\|v_0}$ be a game, and let $\brho \in (\Rb \setminus \{0\})^\Pi$ be a \emph{risk-sensitivity profile}.
%     Then, the strategy profile $\bsigma$ is a \emph{$\beta\brho$-risk-sensitive equilibrium}, or \emph{$\beta\brho$-RSE} for short, if and only if for each player $i$, the strategy $\sigma_i$ is $\beta,\rho$-risk-optimal in the MDP $\MDProc(\bsigma_{-i})$.
% \end{definition}