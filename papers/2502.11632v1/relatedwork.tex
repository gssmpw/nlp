\section{Related works}
The challenges mentioned in this work have been extensively studied in the literature. Morphing techniques have been used for various applications in reduced order modeling to recast the problem on a reference domain \cite{porziani2021automatic,axelsson2015continuation,ye2024data}. Numerous approaches were also proposed to deal with non-reducible problems by using non-linear dimensionality reduction such as registration \cite{taddei2020registration,taddei2023compositional}, feature tracking \cite{mirhoseini2023model}, optimal transport \cite{cucchiara2024model}, neural networks \cite{lee2020model,fresca2022pod,barnett2023neural,berman2024colora}, and so on. Machine learning approaches leveraging neural networks have also demonstrated remarkable success in solving numerical simulations, particularly in capturing complex patterns and dynamics. Methods such as the Fourier Neural Operator (FNO) \cite{li2020fourier} and its extension, Geo-FNO \cite{li2023fourier}, efficiently learn mappings between function spaces by leveraging Fourier transforms for high-dimensional problems. Physics-Informed Neural Networks (PINNs) \cite{raissi2019physics} embed physical laws directly into the loss function, enabling solutions that adhere to governing equations. Deep Operator Networks (DeepONets) \cite{lu2021learning} excel in learning operators with small data requirements, offering flexibility in various applications. Mesh Graph Networks (MGNs) \cite{pfaff2020learning} use graph-based representations to model simulations on irregular domains, preserving geometric and topological properties.