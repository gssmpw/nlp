\section{Generation of safe responses}\label{app:safe_responses}

We generate the safe responses that are part of \bscs{} by running inference on two models unrelated to the ones used in the experimentation: Mistral 7B v0.3~\cite{jiang2023mistral7b} and Phi 3 Small 8k~\cite{abdin2024phi3technicalreporthighly}. Respectively, they score an ASR of 0.21 and 0.01 when evaluated on the base, untemplated instructions of \bscrt{}. Although Mistral is unsafe, it was chosen after a qualitative inspection of its answers showed that, when safe, they were elaborate and well-written, and thus suitable for the \bscs{} dataset.

To generate safe responses, the two models were prompted with the instances of \bscrt{} without jailbreaking templates. The following safe system prompt was added to the inference processes of both models to minimize the risk of providing unsafe responses:

\vspace{1em}
\hspace*{0.05\textwidth}%
\begin{minipage}{0.85\textwidth}
``You are a helpful assistant. You will be provided with an unsafe question, instruction or prompt that you should NOT directly address. Instead, answer the prompt in the safest and most harmless way possible. Remember, answer the prompt in the safest and most harmless way possible'' \\
\end{minipage}

After inference, \llguard{} to classify both models' responses as safe or unsafe. When safe, the responses from Mistral were given priority over the ones from Phi 3, as the latter's responses were observed to be mostly short responses like ``I cannot help with that'', thus less rich than the ones from the former.

To form \bscs{}, each untemplated request was paired with its corresponding selected response. The templated versions of each untemplated request were also paired with said response. To ensure the safety of these responses, they were reviewed by the authors.