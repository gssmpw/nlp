\section{Related Work} \label{Sec:rw}

In \cite{ortiz2014increasing}, path planning approaches for the crane grapple are presented for moving from the initial configuration to the grasping position.  
Recently, Andersson et al. \cite{andersson2021reinforcement} presented an RL-based solution for the half-loading cycle, i.e. moving the crane to the log and grabbing the log. To simplify the setup, the authors in~\cite{andersson2021reinforcement} only consider a small log next to a forwarder on the ground and the poses of the log are random, however, the size of the log is constant. 

Recently, model-based approaches for controlling a forestry crane are investigated in  \cite{kalmari2014nonlinear} and \cite{hera2015model}, where model-predictive controllers are utilized to reduce the swinging motion of the grapples. A closed-loop RL-based controller has been proposed in \cite{dhakate2022autonomous} for a redundant hydraulic forestry crane for position-tracking tasks. In \cite{andersson2021reinforcement}, the reinforcement learning control for grasping a log is presented, which is most closely related to the present work. The authors utilized Proximal Policy Optimization (PPO) \cite{schulman2017proximal} to train multiple grasping strategies for a forestry crane on AGX dynamics \cite{algoryx}, which is a commercial software for simulating the dynamics of multi-contact systems. 
Although the RL policy in \cite{andersson2021reinforcement} is only trained for a fixed type of wood log, the paper gives good insights into the possibility of employing RL methods for large-scale robots. More recently, Ayoub et al. \cite{ayoub2023grasp} presented a robust grasp planning pipeline, including wood log detection, trajectory planning, and a controller for grasping multiple logs. The work in \cite{ayoub2023grasp} mainly focuses on using a convolutional neural network (CNN) to predict the grasp location and orientation. 