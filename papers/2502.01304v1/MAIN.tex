%%%%%%%%%%%%%%%%%%%%%%%%%%%%%%%%%%%%%%%%%%%%%%%%%%%%%%%%%%%%%%%%%%%%%%%%%%%%%%%%
%2345678901234567890123456789012345678901234567890123456789012345678901234567890
%        1         2         3         4         5         6         7         8

\documentclass[letterpaper, 10 pt, conference]{class/ieeeconf}  % Comment this line out
                                                          % if you need a4paper

\IEEEoverridecommandlockouts                              % This command is only
                                                          % needed if you want to
                                                          % use the \thanks command
\overrideIEEEmargins
\usepackage{xcolor}
\usepackage{siunitx}  
% definition of notes command
\setlength{\marginparwidth}{0.5cm}
\setlength{\marginparpush}{0.5cm}
\newcommand{\note}[1]{ \marginpar{\color{red!60!gray} \small #1} }
%\renewcommand{\note}[1]{}
\usepackage{graphicx}
 \graphicspath{./figures/robot_demonstration}
% The following packages can be found on http:\\www.ctan.org

\usepackage{amsmath}
\usepackage{amssymb}
\usepackage{latexsym}
\usepackage{url}
\usepackage{cite}
\usepackage{relsize}
\usepackage{multirow}
\usepackage{afterpage}
\usepackage{ifthen}
\usepackage{graphicx}
\usepackage{algpseudocode,algorithm,algorithmicx}
\usepackage{subfigure}
%\usepackage[keeplastbox]{flushend}
\usepackage{epstopdf}
\usepackage{stackengine}
\usepackage{textcomp} % new



\usepackage{gensymb}
%\usepackage[bookmarks=false, linkcolor=blue, urlcolor=blue, citecolor=blue]{hyperref} 
\usepackage{booktabs}
\usepackage{makecell}
\usepackage{hhline}
\usepackage{courier}
\usepackage{lipsum}
\usepackage{bbm}
\usepackage{bm}
\usepackage{pifont}% http://ctan.org/pkg/pifont
%\newcommand{\norm}[1]{\left \lVert #1 \right \rVert}

\newcommand{\xmark}{\ding{55}}%

\usepackage{xcolor} % new
            
%\hypersetup{pdfstartview={XYZ null null 1.00}}
%\hypersetup{pdfstartview={FitH}}

%%%% for reference blue color
\makeatletter
\let\NAT@parse\undefined
\makeatother
\usepackage[bookmarks=false, linkcolor=blue, urlcolor=blue, citecolor=blue]{hyperref} 
\usepackage{etoolbox}
\makeatletter
\patchcmd{\@makecaption}
  {\scshape}
  {}
  {}
  {}
\makeatother
\hypersetup{
    colorlinks=true,
    linkcolor=red,
    filecolor=magenta,      
    urlcolor=blue,
    pdfstartview={FitH},
    citecolor =blue
    }

    
\setlength{\floatsep}{0.1in}
\setlength{\dblfloatsep}{0.1in}
\setlength{\textfloatsep}{0.1in}
\setlength{\dbltextfloatsep}{0.1in}
\setlength{\intextsep}{0.1in}
\setlength{\abovecaptionskip}{-0.1in}

\renewcommand{\UrlFont}{\scriptsize}
\DeclareRobustCommand{\vec}[1]{ 				
	\ifthenelse{\equal{#1}{\omega} \OR \equal{#1}{\varphi} \OR \equal{#1}{\alpha} \OR \equal{#1}{\beta} \OR \equal{#1}{\chi} \OR \equal{#1}{\delta} \OR \equal{#1}{\varepsilon} \OR \equal{#1}{\phi} \OR \equal{#1}{\epsilon} \OR \equal{#1}{\gamma} \OR \equal{#1}{\eta} \OR \equal{#1}{\iota} \OR \equal{#1}{\kappa} \OR \equal{#1}{\lambda} \OR \equal{#1}{\mu} \OR \equal{#1}{\nu} \OR \equal{#1}{\pi} \OR \equal{#1}{\theta} \OR \equal{#1}{\vartheta} \OR \equal{#1}{\rho} \OR \equal{#1}{\sigma} \OR \equal{#1}{\varsigma} \OR \equal{#1}{\tau} \OR \equal{#1}{\upsilon} \OR \equal{#1}{\xi} \OR \equal{#1}{\psi} \OR \equal{#1}{\zeta}}{
		% Fuer griechische Kleinbuchstaben muss boldsymbol verwendet werden (deckt mathbf nicht ab)
		\boldsymbol{#1}
	}{
		% Alle anderen Symbole verwenden mathbf
		\mathbf{#1}
	}
}
\providecommand{\FullStop}{\text{~\@.\xspace}}
\providecommand{\Comma}{\text{~,\xspace}}
\newcommand{\transpose}[1]{#1^\mathrm{T}}
\newcommand{\numset}[1]{\mathbbm{#1}}

\usepackage{xspace}
\DeclareMathOperator*{\argmax}{arg\,max}
\providecommand{\kuka}{\textsc{KUKA} LBR iiwa R820\xspace}

\newcommand{\etal}{\textit{et al.}}
 \graphicspath{{./figures/}}

\usepackage{tikz}
\usepackage{pgfplots}
\usepackage{pgfplotstable}
\newcommand\corr[1]{\textcolor{black}{#1}}  % For additions/corrections/suggestions
\newcommand\marc[1]{\textcolor{blue}{#1}}  % For additions/corrections/suggestions

%\title{\LARGE \bf Open-Vocabulary Affordance Segmentation}

%\title{\LARGE \bf Open-Vocabulary Affordance Detection on 3D Point Clouds}

\title{\LARGE \bf Towards Autonomous Wood-Log Grasping with a Forestry Crane: Simulator and Benchmarking}

\iffalse
    \author{M.N. Vu$^{1,2}$, A. Wachter$^{1}$, G. Ebmer$^{1}$, M. Ecker$^{1,2}$, T. Gl{\"u}ck$^{2}$, A. Nguyen$^{3}$, \\W. Kemmetm{\"u}ller$^1$, and A. Kugi$^{1,2}$% <-this % stops a space
    \thanks{$^{1}$ M.N. Vu, A. Watcher, G. Ebmer, M. Ecker, W. Kemmetm{\"u}ller, and A. Kugi are with the Automation \& Control Institute (ACIN), TU Wien, 1040 Vienna, Austria {\tt\small \{vu,watcher,ebmer,ecker,\}@acin.tuwien.ac.at}}%
    \thanks{$^{2}$ M.N.Vu, M. Ecker,  T. Gl{\"u}ck, and A. Kugi are with AIT Austrian Institute of Technology GmbH, 1210 Vienna, Austria {\tt\small \{minh.vu, marc.ecker,tobias.glueck,andreas.kugi\}@ait.ac.at}}%     
    \thanks{$^{3}$ A. Nguyen is with Department of Computer Science, University of Liverpool {\tt\small a.nguyen@liverpool.ac.uk}}
    }
\fi
\author{M.N. Vu$^{1,2}$, A. Wachter$^{1}$, G. Ebmer$^{1}$, M. Ecker$^{1,2}$, T. Gl{\"u}ck$^{2}$, A. Nguyen$^{3}$, \\W. Kemmetm{\"u}ller$^1$, and A. Kugi$^{1,2}$% <-this % stops a space
    \thanks{$^{1}$Automation \& Control Institute (ACIN), TU Wien, 1040 Vienna, Austria {\tt\small vu@acin.tuwien.ac.at}}%
    \thanks{$^{2}$AIT Austrian Institute of Technology GmbH, 1210 Vienna, Austria}%     
    \thanks{$^{3}$Department of Computer Science, University of Liverpool}
    }
\begin{document}
\input{0_localmacros}

\maketitle
\thispagestyle{empty}
\pagestyle{empty}

%%%%%%%%%%%%%%%%%%%%%%%%%%%%%%%%%%%%%%%%%%%%%%%%%%%%%%%%%%%%%%%%%%%%%%%%%%%%%%%%
\begin{abstract}
Forestry machines operated in forest production environments face challenges when performing manipulation tasks, especially regarding the complicated dynamics of underactuated crane systems and the heavy weight of logs to be grasped. 
This study investigates the feasibility of using reinforcement learning for forestry crane manipulators in grasping and lifting heavy wood logs autonomously. We first build a simulator using Mujoco physics engine to create realistic scenarios, including modeling a forestry crane with $8$ degrees of freedom from CAD data and wood logs of different sizes. 
We further implement a velocity controller for autonomous log grasping with deep reinforcement learning using a curriculum strategy. Utilizing our new simulator, the proposed control strategy exhibits a success rate of 96\% when grasping logs of different diameters and under random initial configurations of the forestry crane. 
In addition, reward functions and reinforcement learning baselines are implemented to provide an open-source benchmark for the community in large-scale manipulation tasks. A video with several demonstrations can be seen at \href{https://www.acin.tuwien.ac.at/en/d18a/}{https://www.acin.tuwien.ac.at/en/d18a/}. 
\end{abstract}

%%%%%%%%%%%%%%%%%%%%%%%%%%%%%%%%%%%%%%%%%%%%%%%%%%%%%%%%%%%%%%%%%%%%%%%%%%%%%%%%

\section{Introduction}
\label{sec:intro}
% Image editing methods in diffusion models depend on user-defined control directions - users can unlock their creativity using these methods by specifying the desired manipulation through prompts~\cite{gandikota2023concept}, reference images~\cite{ruiz2022dreambooth, kumari2022customdiffusion, gal2022image, chen2024trainingfreeregionalpromptingdiffusion}, or attribute vectors~\cite{parmar2023zero,hertz2022prompt}. In this work, we ask a fundamentally different question: \emph{Can we automatically discover the underlying visual structure of a concept within diffusion model's knowledge?} %Rather than requiring user-specified controls, we aim to decompose the model's internal knowledge into meaningful directions.

% This question touches on a fundamental limitation in how we interact with diffusion models. Current control methods ~\cite{zhang2023addingconditionalcontroltexttoimage, gandikota2023concept, ye2023ipadaptertextcompatibleimage,ye2023ipadaptertextcompatibleimage, hertz2024stylealignedimagegeneration, li2023photomaker, shi2024instantbooth, chen2024trainingfreeregionalpromptingdiffusion} require users to specify their desired manipulations in advance, limiting interactive creativity. This contrasts with natural human artistic workflows, where creators dynamically explore creative ideas while jointly refining them toward meaningful artistic outcomes~\cite{hoffmann2016modeling}. This synergy between specification and exploration is not new to generative models. Early GAN architectures naturally developed disentangled latent spaces that enabled continuous\cite{harkonen2020ganspace,radford2015unsupervised, wu2021stylespace, shen2020interfacegan}, compositional control over generated images. Users could explore these spaces to discover interesting variations that would be difficult to describe in words~\cite{wu2021stylespace}, then combine them to achieve their creative goals~\cite{grabe2022towards}. 


% While diffusion models have largely superseded GANs in conditional image synthesis~\cite{dhariwal2021diffusion},  their underlying structure remains less understood. Diffusion models achieve remarkable diversity through high-dimensional latents, unlike GANs' compact latent spaces.  With a single prompt, diffusion models can generate radically different variations through different random initializations of input noise. We ask - Is it possible to discover interpretable structure within this vast space of variations?

Text-to-image diffusion models are capable of generating remarkable visual variations from a single prompt through different random initializations. However, this vast creative potential remains largely opaque to users---while we can generate diverse images, we lack understanding of the underlying structure of these variations. This presents a fundamental challenge: how can we discover and expose the latent visual capabilities encoded within these models?

\let\thefootnote\relax \footnote{$^{*}$Correspondence to \texttt{gandikota.ro@northeastern.edu}}

The challenge touches on a key limitation in how we interact with diffusion models today. Current control methods require users to explicitly specify their desired edits in advance through prompts~\cite{gandikota2023concept}, reference images~\cite{zhang2023addingconditionalcontroltexttoimage, chen2024trainingfreeregionalpromptingdiffusion, ruiz2022dreambooth,kumari2022customdiffusion, Ryu_lora, hu2021lora}, or attribute vectors~\cite{ye2023ipadaptertextcompatibleimage, hertz2024stylealignedimagegeneration, li2023photomaker, shi2024instantbooth,parmar2023zero,hertz2022prompt}. That contrasts sharply with natural human creative workflows, where artists dynamically explore creative ideas and jointly refine them toward meaningful artistic outcomes~\cite{hoffmann2016modeling}. The need for pre-specified controls creates a barrier between users and the full creative potential of these models.

Interestingly, earlier generative models like GANs~\cite{gans,karras2019style,brock2018large} naturally developed more interpretable internal structures. Their compact latent spaces often exhibited emergent disentanglement~\cite{harkonen2020ganspace,radford2015unsupervised, wu2021stylespace, shen2020interfacegan}, enabling continuous and compositional control over generated images. Users could explore these spaces to discover interesting variations that would be difficult to describe in words~\cite{wu2021stylespace}, then combine them to achieve their creative goals~\cite{grabe2022towards}.

Diffusion models have largely superseded GANs in conditional image synthesis~\cite{dhariwal2021diffusion}, achieving greater diversity through much higher-dimensional latents. And yet an understanding of the underlying structure of these larger latent spaces has remained elusive. In this work, we ask a fundamental question: \emph{Can we automatically discover the visual structure within a diffusion model's knowledge of a concept?} Rather than requiring user-specified controls, we aim to decompose the model's internal representations into expressive directions that users can explore and combine.

To address these needs, we present \textbf{SliderSpace}, a framework that brings systematic explorability to diffusion models. Given just a text prompt, SliderSpace discovers a canonical set of meaningful, diverse, and controllable directions within the model's knowledge of that concept. Each direction is implemented as a low-rank adapter~\cite{hu2021lora} that can be scaled and composed with others, allowing users to explore and smoothly combine different aspects of variation, as shown in Figure~\ref{fig:intro}.

We ground SliderSpace discovery in three key requirements for meaningful decomposition of a diffusion model's visual manifold: 
\begin{enumerate}
    \item \textbf{Unsupervised Discovery:} The decomposition process should emerge from the intrinsic structure of the model's learned representation, rather than being guided by predefined attributes. This ensures we capture the true topology of the model's knowledge space rather than projecting our assumptions onto it.
    
    \item \textbf{Semantic Orthogonality:} Each discovered control must represent a distinct semantic direction. This is enforced in a semantic feature space, like CLIP, where every slider has an orthogonal effect in embeddings. This prevents discovering multiple controls that create similar semantic effects, making the system more efficient and easier.
    
    \item \textbf{Distribution Consistency:} Directions must induce consistent transformations across both random seeds and prompt variations. 
\end{enumerate}

These requirements naturally lead to our proposed framework, which we formalize in Section~\ref{sec:method}. As we show in our experiments, SliderSpace is architecture-agnostic, working with both conventional U-Net based models like Stable Diffusion~\cite{rombach2022high, rombach2022sd20, podell2023sdxl, turbo, dmd} and recent transformer-based architectures like Flux~\cite{flux}.

We demonstrate the expressiveness of SliderSpace through three applications: First, we show how SliderSpace can decompose high-level concepts into diverse and expressive components, revealing the natural axes of variation in the model's understanding. Second, we explore artistic style variation, where SliderSpace discovers directions that match or exceed the diversity of manually curated artist lists while being judged more useful by human evaluators. Finally, we show how SliderSpace can help reverse the mode collapse commonly observed in distilled diffusion models, restoring diversity while maintaining generation speed.

Beyond providing practical creative control, SliderSpace opens new avenues for understanding and utilizing the latent capabilities of diffusion models. By mapping these models' visual potential into intuitive, composable directions, we take a step toward making their creative possibilities more accessible and interpretable to users.

% Image editing methods in diffusion models unlock the creativity of users. In this work we ask an alternate question: \emph{Can we organize and expose what of the diffusion model is already capable of?}.
% Existing methods for controlling image generation typically require users to manually specify edit directions for desired changes. This process is time-consuming, requires technical expertise, and limits the spontaneity of the creative process. For instance, if a user wants to adjust the smile of a generated person, they must explicitly request this edit, often through imprecise prompt engineering or model fine-tuning. This approach of predefined controls or manual specifications restricts users from fully exploring the latent capabilities of the model. There may be interesting stylistic variations or attributes that the model can generate, but users have no easy way to discover or utilize these.

% Natural visual disentanglement was an emergent property in the latent space of Generative Adversarial Models (GANs) \cite{harkonen2020ganspace,radford2015unsupervised, wu2021stylespace, shen2020interfacegan}. In particular, it has been observed that StyleGAN~\cite{karras2019style} stylespace neurons offer detailed control over many meaningful aspects of images that would be difficult to describe in words~\cite{wu2021stylespace}. However, diffusion models do not share such a compact latent space~\cite{park2023unsupervised}; and efforts to uncover such a space in the semantic embeddings of the text conditioning have met with limited success \nik{Nick - is there a specific citation you were thinking about?}.

% In this work we introduce \textbf{SliderSpace}, which takes a step towards uncovering an analogous low dimensional representation of diffusion models' visual breadth; in essence treating the diffusion model as many generators sharing parameters, where a particular generator is defined by a specific prompt. For a given prompt we sample many random seeds (and optionally prompt expansions using an LLM), generate the corresponding images, and apply an off the shelf feature extractor (in this work CLIP, but our method can be applied to any differentiable feature extractor). We use PCA to analyze these features, and for each of the leading $k$ principal components we train a LoRA \cite{} which causes the diffusion model to produces images which increase the feature magnitude along that component when passed back through the same feature extractor. This leads to a 'Slider' for each principal component, because each LoRA can be scaled and applied to the original diffusion model, continuously varying those visual features in the generated results (as measured, in our case, by CLIP).

% There are many other works that enhance the controllability of diffusion models. One common approach is enabling users to add spatial constraints to a generation either manually, or via a reference image \cite{zhang2023addingconditionalcontroltexttoimage, chen2024trainingfreeregionalpromptingdiffusion}, a second is leveraging more abstract embeddings (e.g. identity, style) extracted from a reference image \cite{ye2023ipadaptertextcompatibleimage, hertz2024stylealignedimagegeneration, li2023photomaker, shi2024instantbooth}, a third is finetuning a foundation model to better generate a concept important to the user \cite{ruiz2022dreambooth, kumari2022customdiffusion, Ryu_lora, hu2021lora}, and a fourth (most relevant to this work) is finding low-rank adaptors of the model based on a prompt or small training set which can be scaled to provide continous control over one aspect of generated image (e.g. night vs day, basic vs luxury, etc.) \cite{gandikota2023concept}. SliderSpace is complementary to all of these methods and offers something distinct. All of the other methods we are aware require the user (and / or model designer) to know in advance what type of control they want. In contrast SliderSpace assists users in discovering and controlling hidden capabilities present in the diffusion model's distribution of possible generations.

%We propose that truly intuitive creative control in a text-to-image model should meet three key criteria: \emph{discoverability}, \emph{intuitiveness}, and \emph{specificity}. The model should reveal controllable attributes that may not be immediately obvious, offer controls that are easy to understand and manipulate, and ensure each control affects a distinct attribute of the generated image.

% We demonstrate the utility and power of SliderSpace using three applications built on top of SDXL-DMD \cite{dmd}, because its fast generation speed lends itself well to the continuous control offered by SliderSpace.

% First, we study concept decomposition (Section \ref{sec:concept_exp}), where we learn sliders for a specific concept (e.g. 'monster', 'waterfall', 'car'). Through quantitative metrics of diversity and text alignment we demonstrate that the learned sliders dramatically boost the diversity of generations when randomly applied without harming text alignment; we also ask humans to qualitatively judge these results in a user study where they find the SliderSpace results to be more 'Diverse', 'Useful', and 'Creative' than our baselines.

% Second, we attempt to compare the automatic discoveries of SliderSpace to a large scale manual study of artistic styles (Section \ref{sec:art_exp}), open-sourced by ParrotZone \cite{parrotzone}. In this study SDXL was prompted with over 4300 artist names,  and based on visual inspection the cases of successful stylistic mimicry recorded. Quantitatively SliderSpace more closely matches the distribution of artistic variation discovered by ParrotZone than other baselines, and in our user studies was judged to be significantly more 'Diverse' and 'Useful' than the baselines. To our surprise humans even judged SliderSpace results to be slightly more 'Diverse' than the results generated by the manually discovered artist names of \cite{parrotzone}.

% Third, we attempt to use SliderSpace to reverse the mode collapse commonly observed in distilled few-step diffusion models relative to the original teacher model (Section \ref{sec:diverse_exp}). We quantitatively demonstrate that applying SliderSpace to SDXL-DMD leads to more closely matching the distribution of images by the original teacher, SDXL.

%Through extensive experiments on various state-of-the-art text-to-image models, we demonstrate that SliderSpace significantly enhances user control and creative expression in AI-assisted image generation tasks. Our method enables a range of applications, including concept decomposition and control, diversity improvement in generated images, customization dissection and edits, and the exploration of artistic styles inherent in the model.

% SliderSpace goes beyond providing a practical tool for enhanced creative control. By mapping the visual potential of diffusion models it can open new avenues for generative creativity and deepens our understanding of each model's hidden potential.
\section{Related work}

In recent years, diffusion models have gained prominence as a powerful generative framework, excelling in tasks such as image~\citep{rombach2022latentdiff,nichol2021improvedddpm,nichol2021glide,ramesh2022hierarchicaldiff} and video synthesis~\citep{blattmann2023alignlatentvd,an2023latentshiftvd,ge2023preservecovd,guo2023animatediffvd,singer2022makeavideovd}. These models generate data by progressively denoising randomly initialized samples until a coherent structure or scene emerges. Leveraging the flexibility and effectiveness of generative models, they have been adapted to a wide range of tasks~\cite{zheng2023ddcot, tang2023cotdet, shi2024part2object, tang2023temporal, tang2023contrastive}, including 3D and 4D content generation. 
% In this section, we review the related works on diffusion models and their applications in this field.
In this section, we will review three parts: diffusion models, 4D scene representations, and 4D generation with diffusion models.


\paragraph{Diffusion for Generation}
Recently, diffusion models, pre-trained on large-scale datasets~\citep{schuhmann2022laion}, have made significant strides in generating high-quality and diverse visual content for both 2D image and video tasks ~\citep{rombach2022latentdiff,nichol2021improvedddpm,blattmann2023alignlatentvd,an2023latentshiftvd,huang2024free}. Leveraging aligned vision-language representations~\cite{shi2024plain, dai2024curriculum, shi2024devil, shi2023logoprompt, shi2023edadet, shi2022spatial}, these models can produce various forms of visual content with impressive diversity and realism conditioned on text or images. To adapt 2D diffusion models for 3D generation, some methods utilize Score Distillation Sampling Loss~\citep{poole2022dreamfusion,lin2023magic3d,chen2023fantasia3d,wang2024prolificdreamer} to distill 3D priors and train a neural radiance field~\citep{mildenhall2020nerf} for 3D asset creation. However, this approach often faces challenges such as slow training speeds and multi-face artifacts~\citep{shi2023mvdream}. To address these limitations, another strategy involves fine-tuning pre-trained 2D diffusion models to directly generate multi-view consistent images~\citep{shi2023mvdream,liu2023zero123,long2024wonder3d,liu2023syncdreamer,li2024era3d} from large-scale multi-view datasets~\citep{deitke2023objaverse}. These images are then processed through 3D reconstruction algorithms~\citep{wang2021neus,kerbl20233dgs,liu2023nero} to produce high-quality 3D assets. Despite these advancements, efficiently leveraging these techniques for 4D generation, ensuring both spatial and temporal coherence, remains a challenging problem.


\paragraph{4D Scene Representation}
Current 4D scene representations can be broadly categorized into two types based on their underlying 3D scene representation: 1) NeRF-based \citep{mildenhall2020nerf} and 2) 3D Gaussian Splatting (3DGS)-based \citep{kerbl20233dgs}. Both approaches extend static 3D scene representations into the temporal domain by introducing deformable fields or animation-driven training frameworks.
NeRF (Neural Radiance Fields) was initially proposed to encode the geometry and appearance of static scenes using implicit models with MLPs. Building upon this, many works have extended static NeRF to handle dynamic scenes, either by modeling a dynamic deformation field on the top of a canonical static scene representation~\citep{pons2021dnerf,tretschk2021nonrigidnerf,yuan2021stardnerf,park2021nerfies,fang2022fastdnerf} or by directly learning a time-conditioned radiance field~\citep{li2022neural3dvideo,gao2021dynamicviewsynthesis,park2021hypernerf,xian2021spacenerfvideo}. Despite its success, NeRF-based methods often face limitations in training and inference speed, making them less suitable for real-time applications.
Recently, 3D Gaussian Splatting (3DGS) has shown impressive performance due to its efficient training and real-time novel view synthesis capabilities. This method represents static scenes as a set of Gaussian primitives and employs a fast Gaussian differentiable rasterizer with adaptive density control. As an explicit representation, 3DGS also simplifies tasks such as scene editing. 
3DGS then has been applied to model dynamic scenes with the similar idea of building a deformation field~\citep{luiten2023dynamic3dgauss,wu20244dgauss,yang2024deformable4dgauss,zeng2024stag4d,wu2024sc4d}.
For example, Dynamic 3D Gaussians~\citep{luiten2023dynamic3dgauss} enable the Gaussians to move and rotate over time under local rigid constraints. This approach efficiently models fine details and temporal changes, making it highly effective for 4D content creation.
Together, these representations offer a robust framework for generating realistic and temporally coherent dynamic scenes in 4D space, supporting applications such as animation, scene reconstruction, and motion capture.

\paragraph{4D Generation} By efficiently integrating advanced diffusion techniques with 4D scene representations, significant progress has been made toward 4D generation. One approach in this direction leverages Score Distillation Sampling~\citep{poole2022dreamfusion} to distill spatial and temporal prior knowledge from multiple diffusion models into a 4D scene representation, producing spatially and temporally consistent 4D objects, including text-to-video and text-to-image generation. 
A pioneering work, MAV4D~\citep{singer2023mav3d} introduced a multi-stage training pipeline for dynamic scene generation, utilizing a Text-to-Image (T2I) model to initialize static scenes and a Text-to-Video (T2V)~\citep{singer2022makeavideovd} model to handle motion dynamics.
Building on this paradigm, several methods have sought to improve 4D generation quality by incorporating image conditions ~\citep{zhao2023animate124}, hybrid Score Distillation Sampling~\citep{bahmani20244dfy}, strategies that decouple static elements from dynamic ones~\citep{zheng2024dreamin4d}, and related techniques. However, these methods are largely based on NeRF variants, which suffer from issues like over-saturated appearance and long optimization times. To overcome these limitations,  Align-Your-Gaussians~\citep{ling2024alignyourgauss} proposed using dynamic 3D Gaussian Splatting (3DGS)~\citep{kerbl20233dgs} as the underlying 4D scene representation to learn a deformation field~\citep{park2021nerfies,pons2021dnerf}, offering faster training and better real-time capabilities. Despite this, the reliance on SDS loss in these methods leads to slow optimization speeds, limiting their applicability in downstream tasks.
Another approach uses video as guidance. Several video-to-4D frameworks~\citep{jiang2023consistent4d,yin20234dgen,pan2024fastdy4d} have been introduced that use video inputs as references to guide 4D generation. These methods attempt to generate dynamic scenes by leveraging video-driven information for more precise motion dynamics.
Additionally, to ensure multi-view consistency, recent works have focused on retraining multi-view video diffusion models~\citep{zhang20244diffusion,liang2024diffusion4d,li2024vividzoo,ren2024l4gm,jiang2024animate3d} with 4D datasets, integrating both spatial and temporal modules. However, these models often require large amounts of data and are computationally intensive.
 
\section{Forestry Crane Simulator} \label{Sec:method}

\subsection{Kinematics}

Figure \ref{fig:crane_scematics} illustrates the schematic of the forestry crane. 
It has eight degrees of freedom (DoFs) $\mathbf{q}^\mathrm{T} = [\mathbf{q}_A^\mathrm{T},\mathbf{q}_U^\mathrm{T}] $ consisting of six actuated DoFs $\mathbf{q}_A^\mathrm{T} = [q_1,q_2,q_3,q_4,q_7,q_8]$ and two unactuated joints $\mathbf{q}_U^\mathrm{T} = [{q}_5,{q}_6]$. Note that there are two pairs of synchronized joints, i.e., the prismatic joint $q_4$ and the revolute joint $q_8$. \corr{In each pair of synchronized joints, the same input is applied to the corresponding actuators; for example, the joint angle $q_8$ at the left- and right-jaw of the grapple in Figure \ref{fig:crane_scematics}. }

%The characteristics of all joints are listed in Table \ref{tab:crane_joints}.  
\begin{figure}
    \centering
    \scalebox{0.8}{
    \includegraphics[trim=5cm 3.5cm 0cm 2cm,clip,scale=0.44]{figures/KinematicChain.pdf}
    }
    \caption{Schematic of the forestry crane \cite{ecker2022iterative}.}
    \label{fig:crane_scematics}
\end{figure}
\iffalse

    \begin{table}
        \caption[abc]{List of the forestry crane joints.}
        \label{tab:crane_joints}
        \begin{center} 
            
                \begin{tabular}{c c c c c}
                    \hline
                    Coordinate & Name & Actuated & Range & Unit\\
                    \hline
                     $q_1$ & Slewing joint & \checkmark & [-3.71,\:3.71] & \si{rad}\\ 
                     $q_2$ & Boom joint & \checkmark & [-1.2,\:1.56] & \si{rad} \\  
                     $q_3$ & Arm joint & \checkmark & [-0.91,\:4.6] & \si{rad} \\
                     $q_4$ & Prismatic joint & \checkmark & [0,\:4.47] & \si{m} \\
                     $q_5$ & Tip joint & \xmark & [-1.57,\:1.57] & \si{rad} \\
                     $q_6$ & Tilt joint & \xmark & [-0.79,\:2.36] & \si{rad} \\
                     $q_7$ & Rotate joint & \checkmark & $ [-\infty,\:\infty]$ & \si{rad}\\
                     $q_8$ & Grapple jaws & \checkmark & [0,\:3] & \si{rad}\\
                    \hline 
                \end{tabular}
        \end{center}        
    \end{table}
\fi    

The kinematics of the forestry crane are described by transformations from a coordinate Frame $\mathcal{F}_i$ attached to joint $i$ to a coordinate frame $\mathcal{F}_{i-1}$ attached to joint $i-1$
\begin{align}
	\vec{H}^i_{i-1} = \begin{bmatrix}
		\vec{R}_{i-1}^i & \vec{d}_{i-1}^i\\
		\transpose{\vec{0}} & 1
	\end{bmatrix}\in\mathcal{SE}(3) \Comma
\end{align}
where $\vec{R}_{i-1}^i\in\mathcal{SO}(3)$ and $\vec{d}_{i-1}^i\in\mathbb{R}^3$ are the three-dimensional rotation matrix and the three-dimensional translation vector, respectively. 
The coordinate frames are illustrated in Figure~\ref{fig:crane_scematics} according to the \textit{Denavit-Hartenberg (DH) convention} \cite{spong:2006}. 
\iffalse

    Note that frame $\mathcal{F}_{11}$ is defined by DH convention w.r.t. frame $\mathcal{F}_{8}$ instead of $\mathcal{F}_{10}$ due to the kinematic structure depicted in Figure \ref{fig:crane_scematics}.
    %, but rather from frame $\mathcal{F}_8$. 
    Hence, homogeneous transformations $\vec{H}_{i-1}^i$, $i=1,\dots,10,12$ and $\vec{H}_{8}^{11}$ can be described using four DH parameters $\theta_i$, $d_i$, $a_i$ and $\alpha_i$ as
    \begin{align}
    	\vec{H}_{i-1}^i = \vec{H}_{Rz}(\theta_i)\vec{H}_{Tz}(d_i)\vec{H}_{Tx}(a_i)\vec{H}_{Rx}(\alpha_i)\Comma
    \end{align}
    where $\vec{H}_{Ri}$ is a pure rotation around the $i$-axis and $\vec{H}_{Ti}$ is a pure translation in direction of the $i$-axis. 
    The transformation from $\mathcal{F}_j$ to $\mathcal{F}_i$, $0\leq i < j$ can be computed using
    \begin{align}
    	\vec{H}_i^j=\begin{cases}
    		\prod_{l=i+1}^j\vec{H}_{l-1}^l&,\text{ for }j\leq 10\\
    		\Big(\prod_{l=i+1}^8\vec{H}_{l-1}^l\Big)\vec{H}_{8}^{11}\vec{H}_{11}^j&,\text{ for }11\leq j\leq 12
    	\end{cases}\Comma
    \end{align}
    where $\prod_{l=i+1}^j\vec{H}_{l-1}^l$ being the identity for $j\leq i$.
    \begin{table}
        \caption{Denavit-Hartenberg parameters of the timber crane.}\label{tab:DHParams}
    	\centering
    	\begin{tabular}{c|cccc}
                \hline
    		$i$ & $\theta_i$ [rad] & $d_i$ [m] & $a_i$ [m] & $\alpha_i$ [rad]\\
    		\hline
    		1   &            $q_1$ & 2.425     & 0.1800      &$\pi/2$\\
    		2   &            $q_2$ & 0         & 3.4931    &0\\
    		3   &            $q_3$ & 0         & -0.3925   &$\pi/2$\\
    		4   &                0 & $q_4$ + 3.157     & 0         &0\\
    		5   &                0 & $q_4$     & 0         &$-\pi/2$\\
    		6   &            $q_5$ & 0         & -0.2130   &$-\pi/2$\\
    		7   &            $q_6$ & 0         & 0         &$-\pi/2$\\
    		8   &            $q_7$ & 0.578     & 0         &0\\
    		9   & $-\pi/2$ & 0         & 0.3402    &$\pi/2$\\
    		10  &           $\pi/2$ & 0         & 0.8566    &0\\
    		11  &  $\pi/2$ & 0         & 0.3248    &$\pi/2$\\
    		12  &           $\pi/2$ & 0         & 0.8566    &0\\
                 \hline
    	\end{tabular}
    \end{table}
\begin{table}[h]
    
    \caption{Denavit-Hartenberg parameters of the forestry crane.}\label{tab:DHParams}
        	\begin{center}
            	\begin{tabular}{c|cccc}
                        \hline
            		$i$ & $\theta_i$ in \SI{}{\radian} & $d_i$ in \SI{}{\meter} & $a_i$ in \SI{}{\meter} & $\alpha_i$ in \SI{}{\radian}\\
            		\hline
            		1   &            $q_1$ & 2.4     & 0.18      &$\pi/2$\\
            		2   &            $q_2$ & 0         & 3.5    &0\\
            		3   &            $q_3$ & 0         & -0.4   &$\pi/2$\\
            		4   &                0 & $q_4$ + 3.1     & 0         &0\\
            		5   &                0 & $q_4$     & 0         &$-\pi/2$\\
            		6   &            $q_5$ & 0         & -0.21   &$-\pi/2$\\
            		7   &            $q_6$ & 0         & 0         &$-\pi/2$\\
            		8   &            $q_7$ & 0.58     & 0         &0\\
                    %9   & $-\pi/2$ & 0         & 0.3402    &$\pi/2$\\
                    \hline
            	\end{tabular}
        	\end{center}
    
    \end{table}
\fi    
    %The DH parameters for the forestry crane are given in Table~\ref{tab:DHParams}. 

\iffalse
    Using the above kinematic relations, the wrist position of the grapple, $\mathbf{d}_g^\mathrm{T} = [g_x,g_y,g_z]$, is taken from  
    %Using this formalism the calculation of the center point coordinates $g_x, g_y$ and $g_z$ of the grapple reads as
    \begin{equation}
        \vec{H}_{0}^{8} = 
        \begin{bmatrix}
            \mathbf{R}_0^8 & \mathbf{d}_g \\
            \mathbf{0}^\mathrm{T} & 1
        \end{bmatrix} \:.
        \label{eq:transformation}
    \end{equation}
\fi    
%In the used scenarios the grapple is already close to the logs and the crane is unfolded, therefore the special hydraulic kinematic is neglected. 
%Instead, all rotary joints are driven with velocity-controlled rotary motors and the telescopic boom is driven by a linear motor.

\subsection{Simulator}
\label{sec: b simulator}

The system dynamics and contacts with the environment are simulated using the open-source MuJoCo \cite{todorov2012mujoco} physics engine. 
An example of the simulated environment is illustrated in Figure \ref{fig: example mujoco}. 
The assembled model of the forestry crane (including the truck) consists of $38$ rigid bodies and $10$ active joints\corr{, including two pairs of synchronized joints}. The total operating weight of the forestry crane is approximately \SI{1981}{\kilo\gram}. 
On standard forestry cranes, hydraulic actuators powered by a pump that is driven by a combustion engine drive the slewing ($q_1$), boom ($q_2$), arm ($q_3$), and prismatic ($q_4$) joints, respectively. 
In order to simplify the model for training purposes, the hydraulic actuators are not explicitly modeled. 
%Instead, two linear motors are modeled for the synchronized joint $q_4$, and rotational motors are utilized to drive other actuated joints. 
\corr{We assume an (ideal) underlying velocity controller, with the reference velocity for the prismatic joint $q_4$ and the reference rotational velocity for the other joints as inputs. }
Thus, in the simulation environment, the grasping controller for the modeled forestry crane is a fine-tuned PID controller with reference velocities for the actuated joints $\mathbf{q}_A$.  

The wood log position is randomized in a reachable region of the forestry crane, depicted as the yellow region in Fig. \ref{fig: example mujoco}. 
The center of this region is approximately \SI{6.5}{\meter} from the crane's base. 
Additionally, logs are modeled as cylinders with a length of $\SI{2.75}{\meter}$, and the log's diameter varies in the range of $[0.3,0.8] \SI{}{\meter}$. 
In order to prevent overfitting during the training process, the slew angle of the crane is varied in the range $[-2\pi/3, -\pi/3] \SI{}{\radian}$. 
The 6-dimensional contact forces between the grapple and the wood log are computed using the signed distance field (SDF) collision primitive \cite{reiner2011interactive}. 
This is particularly important to maintain the robustness of the simulation since the inner and outer jaw of the grapple have curvy shapes. 










\section{Wood-Log Grasping Method}
\label{sec: method}
Since a varied-diameter wood log is considered for the grasping task, the latent Markov decision process (L-MDP) \cite{chen2021understanding,vuong2019pick} is utilized in this work. The reward function and the policy gradient method are briefly discussed in this section. 


\subsection{Latent-MDP for forestry crane with varied-diameter wood logs}
The control problem for the grasping task can be modeled as a Markov decision process (MDP), which emulates the interactive learning between the agent (the grasping controller) and the simulated environment. An MDP consists of the $4$-element tuple $(\mathcal{S},\mathcal{A},\mathbf{P},R)$ referring to the state space $\mathcal{S}$, the action space $\mathcal{A}$, the transition probability density function $\mathbf{P}$, and the reward function $R$, respectively. 
%At a given discrete time step $t$, the forestry crane system is in state $\mathbf{s}_t \in \mathcal{S}$ and the agent perceives observations  $\mathbf{o}_t \in \mathcal{O}$ 
At a given time step $t$, the agent, in the state $\mathbf{s}_t \in \mathcal{S}$, selects an action $\mathbf{a}_t \in \mathcal{A}$ to transition to the next state $\mathbf{s}_{t+1}$ %\marc{according to the distribution (instead of "with the function")} 
with the transition probability $\mathbf{P}(\mathbf{s}_{t+1}|\mathbf{s}_t,\mathbf{a}_t)$. 
This results in the immediate reward $R_t$. Note that $\mathbf{P}(\mathbf{s}_{t+1}|\mathbf{s}_t,\mathbf{a}_t)$ is obtained from the simulator introduced in Section \ref{sec: b simulator}. It is worth noting that the transition function $\mathbf{P}$ is uncertain since the contact dynamics are not the same for different wood log dimensions.  
%\marc{(The transition PDF in the RL setting always describes an uncertain process. Do you mean that it additionally depends on the log dimensions?)}. 
Details on the state $\mathbf{s}_t$, the action $\mathbf{a}_t$, and the reward function $R_t$ for the grasping task with the forestry crane are introduced in the following subsection. 

Since the diameter $d$ of a wood log and its mass vary during the training process, the latent MDP (L-MDP) is utilized, where the log's size can be considered a latent variable. Additionally, individual training episodes are considered as single MDPs with finite length $H$. We denote the L-MDP as $\{\mathcal{L}, p(d)\}$, where $\mathcal{L}$ is the set of single MDPs with different diameters $d$, and $p(d)$ is the \corr{uniform} distribution of the diameter $d$ over $\mathcal{L}$. To this end, the objective of the grasping task is as follows
\begin{equation}
    J(\bm{\pi_\theta}) = \mathrm{E}_{d\sim p(d)}\bigg[\mathrm{E}_{\mathbf{a}_t\sim \bm{\pi_\theta}(.|\mathbf{s}_t)} \Bigg[\sum_{t=0}^{H}\gamma R_t\Bigg] \bigg]\:,
    \label{eq: RL objective}
\end{equation}
\corr{where $\mathrm{E}$ is the expectation function, the symbol ``$\sim$'' denotes the sampling process from the corresponding distribution $\bm{\pi_\theta}$ over the action space $\mathcal{A}$, and $0<\gamma<1$ is the discount factor. The normal distribution is typically used to model $\bm{\pi_\theta}$.} 
Additionally, $\bm{\theta}$ combines the policy parameters that can be weights and biases of a neural network. 
An RL method is utilized to find the optimal policy $\bm{\pi}^*$ that maximizes (\ref{eq: RL objective}) 
\begin{equation}
    \bm{\pi_\theta}^* = \argmax_{\bm{\pi_\theta}} J(\bm{\pi_\theta}) \:.
    \label{eq: RL policy}
\end{equation}
To find the optimal policy (\ref{eq: RL policy}), we utilize the modified version of Proximal Policy Optimization (PPO) as discussed in Subsection \ref{sec: RPPO}. The following subsection presents the details of the observation space, action space, and reward function. 
%Acting $a_t \in \mathcal{A}$ according to the policy distribution $\pi(a|s)$, the agent receives an immediate scalar reward $r_t(s_t, a_t)$ according to the specified reward function $R(s, a)$. 
%The goal of RL algorithms is to find the optimal policy $\pi(a|s)^*$ such that the agent takes the optimal action at any given state to maximize the expected return. 
%Here, the deep RL approach involves parameterizing the policy $\pi$ as a neural network $\pi_\theta$ with parameters $\theta$. The resulting policy approximator outputs 
\subsection{Learning environment for the forestry crane}
\subsubsection{Observations and actions} 
\label{sec: observation}
The observations consist of joint angles of the forestry crane and the actuated joint velocities, i.e., $\mathcal{O} = \{\mathbf{q},\dot{\mathbf{q}}_{A}\}$.  
From the 6 DoF poses of the log's pose w.r.t the crane's base obtained by other algorithms \cite{wen2023bundlesdf,vuong2023grasp}, we consider the reduced poses of 4 DoF $\mathbf{q}_l = [x_{l},y_{l}, z_{l}, \psi_l]^\mathrm{T}$, consisting the 3D Cartesian position of the log's center point $\mathbf{p}_l = [x_{l},y_{l}, z_{l}]^\mathrm{T}$ and the yaw angle $\psi_l$, see Figure \ref{fig: rl explained}. The augmented relative Cartesian distance is computed as
\begin{equation}
    \bm{\Delta}_p =  [x_{l},y_{l}, z_{l} - (d_{max}-z_l)/2]^\mathrm{T} - \mathbf{p}_\mathrm{C}(\mathbf{q}) \:,
    \label{eq: relative distance}
\end{equation}
where $\mathbf{p}_\mathrm{C}(\mathbf{q}) = [p_{\mathrm{C},x},p_{\mathrm{C},y},p_{\mathrm{C},z}]^\mathrm{T}$ results from the forward kinematics
\begin{equation}
    \mathbf{H}_\mathrm{C}(\mathbf{q}) = 
    \begin{bmatrix}
        \mathbf{e}_{\mathrm{C},x} & \mathbf{e}_{\mathrm{C},y} & \mathbf{e}_{\mathrm{C},z} & \mathbf{p}_\mathrm{C} \\
        0 & 0 & 0 & 1
    \end{bmatrix}
\end{equation}
and is located at the center of the grapple, see Figure \ref{fig: rl explained}. Note that $\mathbf{e}_{\mathrm{C},x}$, $\mathbf{e}_{\mathrm{C},y}$, and $\mathbf{e}_{\mathrm{C},z}$ are column vectors of the orientation of the grapple's center. The term $d_{off} = (d_{max}-z_l)/2$ represents an offset in $z$-direction for different log sizes where $d_{max} = 0.8$ is the maximum diameter of the wood log. 
Since $\psi_l$ is the yaw rotation around the $z$-axis of the crane base, illustrated in Fig. \ref{fig: rl explained}, the unit vector $\mathbf{e}_{l,y}$ along the length of the log is computed as
\begin{equation}
    \mathbf{e}_{l,y} = [-\sin(\psi_l), \cos(\psi_l), 0]^\mathrm{T}
\end{equation}
In order to successfully grasp the wood log, the orientation of the grapple must be well-aligned with the wood log, as defined in the following condition
\begin{equation}
     \mathrm{mod}\bigg[\widehat{(\mathbf{e}_{l,y},\mathbf{e}_{\mathrm{C},x})},\pi \bigg] \approx 0 \:, 
     \label{eq: orientation condition}
\end{equation}
where the $\widehat{(\mathbf{e}_{l,y},\mathbf{e}_{\mathrm{C},x})}$ presents the angle between the two vectors $\mathbf{e}_{l,y}$ and $\mathbf{e}_{\mathrm{C},x}$. 
The condition (\ref{eq: orientation condition}) can be normalized as the angle distance function in the form
\begin{equation}
    %\Delta_{\psi} = \dfrac{\mathbf{e}_{\mathrm{C},x}\cdot \mathbf{e}_{l,y}}{\norm{\mathbf{e}_{\mathrm{C},x}} \norm{\mathbf{e}_{\mathrm{l},y}}} \:\:,
    \Delta_{\psi} = 1- |\mathbf{e}_{\mathrm{C},x}\cdot \mathbf{e}_{l,y} |\:\:,
    \label{eq: angle distance}
\end{equation}
where the symbol ``$\cdot$'' denotes the dot product between two vectors. 
To this end, the observation space also includes the relative distance (\ref{eq: relative distance}) and the angle distance (\ref{eq: angle distance}), i.e., $\mathcal{O} = \{\mathbf{q},\dot{\mathbf{q}}_{A},\bm{\Delta}_p,\Delta_\psi\}$. As ideal underlying velocity controllers are assumed, the action space consists of desired actuated joint velocities $\mathcal{A} = \{\dot{\mathbf{q}}_{A,d}\}$. 
    %The observation and action spaces are listed in Table \ref{tab:crane_obs_act}. 

\begin{figure}[t]
\centering
\scalebox{0.6}{
\def\svgwidth{1\columnwidth}
\input{figures/explain_rl_env_2.pdf_tex}
}
\vspace{0.1cm}
\caption{Details of variables used for constructing the observations and reward function.}
\label{fig: rl explained}
\vspace{-0.2ex}
\end{figure}
    
\iffalse
    \begin{table}
        
        \caption[abc]{Summary of the observations and actions.}
        \begin{center} 
        \begin{tabular}{c | c c}
        \hline
        & Observations & Actions\\
        \hline
        \multirow{8}{*}{Joint angles and angle rates} & $q_1,\:  \dot{q}_{1}$ & $\dot{q}_{1,d}$\\
         & $q_2 \:  \dot{q}_{2}$ & $\dot{q}_{2,d}$\\
         & $q_3 \:  \dot{q}_{3}$ & $\dot{q}_{3,d}$\\
         & $q_4\:  \dot{q}_{4}$ & $\dot{q}_{4,d}$\\
         & $q_5$ \\
         & $q_6$ \\
         & $q_7\:  \dot{q}_{7}$ & $\dot{q}_{7,d}$\\
         & $q_8\:  \dot{q}_{8}$ & $\dot{q}_{8,d}$\\
         \hline
        \multirow{3}{*}{Relative distance $\bm{\Delta}_p$} & $x_l - p_{\mathrm{C},x}$ \\
         & $y_l - p_{\mathrm{C},y}$\\ 
         & $z_l - d_{off} - p_{\mathrm{C},z}$ \\
              \hline
        \multirow{1}{*}{Angle distance} & $\Delta_\psi$ \\
         
         \hline 
        \end{tabular}
        \end{center}
        \label{tab:crane_obs_act}
    \end{table}
\fi
\subsubsection{Reward function}    
The reward function $R$ is designed to gradually guide the grapple along the actions, e.g., approaching, grasping, lifting, and balancing, to achieve the final goal. 
First, the forestry crane can grasp the wood log when the combined weighted distance 
\begin{equation}
    d_\mathrm{combine} = \norm{\bm{\Delta}_p} + \omega_1 \Delta_\psi
    \label{eq: d combine}
\end{equation}
is small enough. Consequently, the associated reward function term reads as
%on this factor is expressed as
\begin{equation}
    r_{\mathrm{distance}}  = \mathrm{exp}(-\omega_2 d_\mathrm{combine}) \:,
    \label{eq: r_distance}
\end{equation}
where $\omega_1 > 0 $ and $\omega_2 > 0$ are user-defined parameters. 
When the crane approaches the target, the RL agent is encouraged to close the grapple to hold the wood log. The reward function term for this behavior is
\begin{equation}
    r_{\mathrm{grapple}}  = r_{\mathrm{distance}}({q_{8}}/{\overline{q_8}}) + (1-{q_{8}}/{\overline{q_8}})(1-r_{\mathrm{distance}})\:,
\end{equation}
with $\overline{q_8} = \SI{3}{\radian}$ as the limit of the joint angle $q_8$. 
After holding the wood log inside the grapple, the forestry crane proceeds with the lifting action, represented by the reward function term%by giving the following reward 
\begin{equation}
    r_{\mathrm{lift}} = (1- \mathrm{tanh}(\omega_3|z_l-z_{l,d}|))(1-r_\mathrm{grapple})\:\:,
\end{equation}
where $z_{l,d}$ is the desired height of the log and $\omega_3 >0$ is a user-defined parameter. Finally, we encourage the forestry crane to stabilize after grasping the log by using
\begin{equation}
    r_{\mathrm{balance}} = (1 - \mathrm{tanh}(\norm{\dot{\mathbf{q}}_{A,d}}))(1-r_\mathrm{lift}) \:\:.
\end{equation}
Combining all parts, the overall reward function takes the form
\begin{equation}
    R = r_{\mathrm{distance}} + r_{\mathrm{grapple}} + r_{\mathrm{lift}} + r_{\mathrm{balance}} \:\:.
\end{equation}
%r_lift = 1 - np.tanh(z_log_desire_distance*4)
%distance_combine = cartersian_distance + w_angle_distance*angle_diff_distance
%r_distance_combine = np.exp(-distance_combine * w_distance)
%r_jaw_opening = (-jaw_angle / jaw_angle_max + 1) * (1 - r_distance_combine)*0.5
%r_jaw_closing = (jaw_angle / jaw_angle_max)

\subsubsection{Episode termination}
\label{sec: early termination}
Each training episode has a time limitation of $t_{max} = \SI{9}{\second}$. 
Additionally, other termination criteria are listed in the following: 
\begin{itemize}
    \item If the grapple point $\mathbf{p}_C$ is not close to the log's center point $\mathbf{p}_l$,  i.e., $d_\mathrm{combine} < \epsilon$, within $t_{limit} = \SI{6}{\second}$, the episode is early terminated. 
    \item One of the joint limits is violated. 
    \item The log is located more than \SI{8}{\meter} away from the grapple. 
    \item The velocity of the actuated joints exceeds the physical limits, i.e., $|\dot{\mathbf{q}}_A| > \dot{\mathbf{q}}_{A,max}$. 
    %This can happen when the forestry crane tries to push away the log. 
\end{itemize}
\subsection{Modified proximal policy optimization (mPPO) utilizing Beta distribution}
\label{sec: RPPO}

\begin{figure}[t]
\centering
\scalebox{0.8}{

\def\svgwidth{1\columnwidth}
\input{figures/learning_architecture-new.pdf_tex}
}
\vspace{1ex}
\caption{Overview of the learning process. $m$ randomized environments with different wood log sizes and poses are generated by our crane simulator, presented in Subsection \ref{sec: b simulator}. }
\label{fig: overview learning}
\end{figure}
The overview of the learning process is illustrated in Figure \ref{fig: overview learning}. Using the crane simulator in Mujoco, $m$ parallel environments are sampled to generate rollouts (trajectories) for training the agent. A standard architecture of an actor-critic network used in the PPO algorithm is illustrated on the right-hand side of Figure \ref{fig: overview learning}. Since the details on PPO are omitted, readers are referred to \cite{schulman2017proximal}. Only modifications of the PPO, named mPPO, are presented below. 
    
In deep RL, the policy $\bm{\pi}$ is a neural network with the parameter vector $\bm{\theta}$ that takes the state $\mathbf{s}_t$ as input and outputs the distribution of actions $\bm{\pi_{\theta}}$ modeled as Gaussian distribution in the form
\begin{equation}
    \bm{\pi_{\theta}}(\mathbf{a}_t|\mathbf{s}_t) = \dfrac{1}{\sqrt{2\pi}\bm{\sigma_\theta}}\mathrm{exp}
    \Bigg(
    \dfrac{-(\mathbf{a}_t-\bm{\mu_\theta})^2}{2\bm{\sigma_\theta}}
    \Bigg)\:.
\end{equation}
The control actions $\mathbf{a}_t$ can be sampled in the backpropagation process or the inference process as follows
\begin{equation}
    \mathbf{a}_t = \bm{\mu_\theta}(\mathbf{s}_t) + \bm{\sigma_\theta}(\mathbf{s}_t)\mathcal{N}(0,1)   \:,
\end{equation}
\corr{
where $\bm{\mu_\theta}$ and $\bm{\sigma_\theta}$ are the mean and standard deviation of the Gaussian distribution. Since the control input in $\mathbf{a}_t$ is modeled as a separate distribution, an element-wise product is used for all equations in this subsection.}

However, the control actions of the considered forestry crane are constrained in an admissible range $\underline{\mathbf{a}_t} \leq \mathbf{a}_t \leq \overline{\mathbf{a}_t}$ for safety reasons. Thus, in the mPPO algorithm, the Beta distribution is employed with the probability density function (PDF) \cite{chou2017improving}
\begin{equation}
    \bm{\pi_\theta}(\mathbf{a}_t|\mathbf{s}_t) = \bm{\mathrm{Beta}}(\mathbf{a}_{t,n};\bm{\alpha_\theta},\bm{\beta_\theta})\:\:, \bm{\alpha_\theta}>1, \:\bm{\beta_\theta} >1 \:\:,
\end{equation}
where $\mathbf{a}_{t,n} = \dfrac{\mathbf{a}_t- \overline{\mathbf{a}_t}}{\overline{\mathbf{a}_t} - \underline{\mathbf{a}_t}}$ is the normalized action and 
\begin{equation}
    \bm{\mathrm{Beta}}(\mathbf{a}_{t,n};\bm{\alpha_\theta},\bm{\beta_\theta}) = \dfrac{\bm{\Gamma}(\bm{\alpha_\theta}+\bm{\beta_\theta})}{\bm{\Gamma}(\bm{\alpha_\theta})\bm{\Gamma}(\bm{\beta_\theta})}\mathbf{a}_{t,n}^{\bm{\alpha_\theta}-1}(1-\mathbf{a}_{t,n})^{\bm{\beta_\theta}-1}\:\:. 
\end{equation}
Note that ${\Gamma}(i) = \int_0^\infty j^{i-1}\mathrm{exp}(-j)\mathrm{d}j$ is the Gamma function \cite{davis1959leonhard}. In this way, the action $\mathbf{a}_t$ is always sampled in the admissible range by using the mean of the Beta distribution 
\begin{equation}
    \bm{\mu}_{t,n} = \dfrac{\bm{\alpha_\theta}(\mathbf{s}_t)}{\bm{\alpha_\theta}(\mathbf{s}_t) + \bm{\beta_\theta}(\mathbf{s}_t)} \:\:\:. 
    \label{eq: sampling}
\end{equation}

In the PPO algorithm, the loss function consists of three parts, i.e., the surrogate loss to constrain the policy update, the error term of the value function, and the entropy term to encourage exploration, see \cite{weng2018policy}. 
In addition, in a conventional PPO algorithm, the agent shows more randomness in its actions, but the surrogate loss can constrain the updating policy during the training process. 
In a complex environment with large search areas, exploration is important for an agent like this forestry crane to complete the task. 
Inspired by Robust Policy Optimization (RPO) \cite{huang2022cleanrl}, at each step of the training process, we perturb the sampling action (\ref{eq: sampling}) by random values in the uniform distribution $\mathbf{g} \sim \mathcal{U(-\epsilon,\epsilon)}$ in the form
\begin{equation}
    \mathbf{a}_{t,n} \leftarrow \mathrm{clip}(\mathbf{a}_{t,n} + \mathbf{g},0,1) \:.
    \label{eq: robust ppo}
\end{equation}
The function $\mathrm{clip}$ limits the value of $\mathbf{a}_{t,n}$ in the range of $[0,1]$. In this work, $\epsilon$ is set to $0.1$. 

%\lipsum[1]
%\subsection{Part}
 

\begin{table*}[ht]
  \centering
  \small
  \begin{tabular}{lcccccccccc}
    \toprule
    \multirow{2}{*}{Method} & \multirow{2}{*}{\#Params} & \multicolumn{6}{c}{Pre-trained tasks} & \multicolumn{1}{c}{Target task} & \multicolumn{2}{c}{Metrics} \\
    \cmidrule(lr){3-8} \cmidrule(lr){9-9}  \cmidrule(lr){10-11}
                    &        & \textbf{GQA}              & \textbf{VizWiz}                & \textbf{SQA}                   & \textbf{TextVQA} & \textbf{POPE} & \textbf{MMBench} & \textbf{Flickr30k} & \textbf{Avg} & \textbf{Hscore} \\
    \midrule
    \textbf{Zero-shot}       & --      & 61.94            & 50.00                 & 66.80                 & 58.27 & 85.90 & 64.30 & 3.5 & 55.82  & 59.86 \\
    \midrule
    \textbf{Fine-tune}       & 1.2B   & 56.26            & 44.45                 & 28.34                 & 38.98 & 38.40 & 50.56 & \textbf{78.82} & 47.97 & 45.26 \\
    \textbf{LoRA}            & 29M   & 17.74            & 40.63                 & 5.38                  & 30.48 & 2.40  & 9.55  & 64.18 & 24.33 & 20.49 \\
    \textbf{Model Tailor}    & 273M   & 52.49            & 42.28                 & \underline{67.15}     & 43.89 & 82.88 & 63.40 & \underline{75.40} & 61.07 & 59.85 \\
    \midrule
    \textbf{MDGD}             & 1.2B   & \underline{67.71}  & \underline{48.18} & \textbf{69.05}         & \underline{57.32} & \textbf{85.12} & \underline{65.43} & 73.47 & \textbf{66.61} & \textbf{66.03} \\
    ~~w/o visual align     & 1.2B   & 57.64           & 36.95                 & 53.96                 & 32.84 & 30.43 & 56.66 & 65.58 & 47.72 & 46.19 \\
    \textbf{MDGD-GM } & 124M   & \textbf{69.89}  & \textbf{51.22}        & 65.87                 & \textbf{58.18} & \underline{84.39} & \textbf{66.42} & 64.18  & \underline{65.74} & \underline{65.86} \\
    \bottomrule
    \toprule

    \multirow{2}{*}{Method} & \multirow{2}{*}{\#Params} & \multicolumn{6}{c}{Pre-trained tasks} & \multicolumn{1}{c}{Target task} & \multicolumn{2}{c}{Metrics} \\
    \cmidrule(lr){3-8} \cmidrule(lr){9-9}  \cmidrule(lr){10-11}
    &                       & \textbf{GQA} & \textbf{VizWiz} & \textbf{SQA} & \textbf{TextVQA} & \textbf{POPE} & \textbf{MMBench} & \textbf{OKVQA} & \textbf{Avg} & \textbf{Hscore} \\
    \midrule
    \textbf{Zero-shot}       & --     & 61.94 & 50.00 & 66.80 & 58.27 & 85.90 & 64.30 & 0.14 & 55.34 & 59.58 \\
    \midrule
    \textbf{Fine-tune}     & 1.2B  & 62.98 & 40.59 & 59.84 & 48.38 & 71.42 & 51.98 & \underline{69.10} & 57.76 & 56.79 \\
    \textbf{LoRA}            & 29M   & 63.44 & 41.61 & 51.29 & 48.02 & 75.27 & 37.31 & \textbf{71.46} & 55.49 & 54.12 \\
    \textbf{Model Tailor}    & 273M   & 60.39              & \textbf{46.49} & \textbf{69.51} & \textbf{54.88} & \textbf{85.44} & \underline{63.32} & 38.10 & 59.73 & 61.48 \\
    \midrule
    \textbf{MDGD}          & 1.2B     & \textbf{66.55}     & 42.72 & 64.60 & 52.54 & \underline{85.17} & 61.73 & 62.29 & \underline{62.23} & \underline{62.22} \\
    ~~w/o visual align & 1.2B  & \underline{66.39} & 39.89 & 60.19 & 52.40 & 84.92 & 62.97 & 62.39 & 61.31 & 61.22 \\
    \textbf{MDGD-GM}  & 124M  & 66.02              & \underline{43.97} & \underline{67.91} & \underline{52.80} & 84.70 & \textbf{63.97} & 61.04 & \textbf{62.92} & \textbf{63.07} \\
    \bottomrule
  \end{tabular}
  \caption{
  Performance on various pre-trained tasks of LLaVA-1.5 models fine-tuned on Flickr30K and OKVQA. 
  We report the best performance for each task in a \textbf{bold font} while the second best performance \underline{underlined}. 
  }
  \label{tab:llava-main}
  \vspace{-.8cm}
\end{table*}

\section{Experiments}
In this section, we conduct experiments on various datasets and backbone MLLMs to investigate the following research questions:
\begin{enumerate}
    \item \textbf{RQ1 (Overall Performance)} Can MDGD prevent visual forgetting while improving downstream tasks?
    \item \textbf{RQ2 (Ablation Study)} How do the visual alignment and gradient masking affect the MDGD's performance?
    \item \textbf{RQ3 (Representation Learning)} How does MDGD benefit effective multimodal representation learning in MLLMs?
    \item \textbf{RQ4 (Sensitivity Study)} How does gradient masking ratio $\alpha$ affects the learning of MDGD?
\end{enumerate}


\noindent\textbf{Datasets}
To evaluate the effectiveness of MDGD in mitigating catastrophic forgetting, we used two models of different sizes. 
Our experimental design follows the settings from the work of \citet{zhu2024model}. 
For each model, datasets were categorized into two types: 
\textbf{pre-trained tasks}, which assess the model's ability to retain inherent knowledge after fine-tuning, 
and \textbf{fine-tuning tasks}, consisting of unseen datasets used to test adaptability. 
After fine-tuning, we evaluated performance on both task types to measure forgetting and generalization. 
Below, we detail the datasets used for each model. 
\textbf{LLaVA-1.5 (Vicuna-7B) \citep{liu2024improved}}: This model has 7 billion parameters. In line with \citet{liu2024improved}, we used the following datasets:
\begin{itemize}
    \item \textbf{Pre-trained Tasks}: VQAv2 \citep{goyal2017making}, GQA \citep{hudson2019gqa}, VizWiz \citep{gurari2018vizwiz}, SQA \citep{lu2022learn}, TextVQA \citep{singh2019towards}, POPE \citep{li2023evaluating}, and MM-Bench \citep{liu2023mmbench}.
        
    \item \textbf{Fine-tuning}: Flickr30k \citep{young2014image} and OKVQA \citep{marino2019ok}, which were not encountered in the pre-training stage.
\end{itemize}
\textbf{MiniCPM-V-2.0 \citep{yao2024minicpmvgpt4vlevelmllm}}: This model has 2.8 billion parameters. We evaluated its performance on:
\begin{itemize}
    \item \textbf{Pre-trained Tasks}: VizWiz, OKVQA, A-OKVQA \cite{schwenk2022okvqa}, Text-VQA, IconQA \cite{lu2021iconqa}, POPE, and MM-Bench.
    \item \textbf{Fine-tuning}: TextCaps \citep{sidorov2020textcaps} and PathVQA \citep{he2020pathvqa}, which were not part of its pre-training exposure.
\end{itemize}



\noindent\textbf{Baselines} We compare our approach against several baselines: 
\begin{itemize}
\item \textbf{Standard Fine-Tuning.} For a fair comparison, we follow the setting of Model-Tailor \cite{zhu2024model}, where LLaVA-1.5 is fine-tuned on the last 6 layers and its feature adapter, with a total of 1.2B parameters.
MiniCPM is fine-tuned on the last 8 layers and its feature resampler, with 517M parameters. 
\item \textbf{LoRA-based Fine-Tuning \citep{hu2021lora}.} LoRA introduces low-rank matrices to update only a small subset of parameters, reducing memory consumption and computational cost. In our experiments, LLaVA-1.5 and MiniCPM are fine-tuned by modifying the query and key projection layers within the attention mechanism. 
\item \textbf{Model Tailor \citep{zhu2024model}.} This baseline employs a hybrid strategy that mitigates catastrophic forgetting by identifying and adjusting the most critical parameters for adaptation. It has been evaluated through experiments on multimodal large language models (MLLMs). As the method is not open source, we report only the original results of the LLaVA-1.5 experiments provided in the original paper as a baseline.
\end{itemize}


\noindent\textbf{Implementation Details}
We use the official Huggingface implementations of the LLaVA-1.5 and the MiniCPM-V-2.0 models and their LoRA adapters. 
For model fine-tuning, we use BFloat16 precision for memory-efficient training. 
Experiments are conducted using 2 NVIDIA A100-SXM4-80GB GPUs.

\subsection{Overall Performance (RQ1)}\label{sec:main-results}
\subsubsection{Larger MLLM adapts better to downstream tasks but is more prone to visual forgetting.}
We study the visual forgetting problem on the LLaVA-1.5 MLLM which contains 7B model parameters 
and report performance comparison results in Table~\ref{tab:llava-main}.
We observe that the pre-trained LLaVA enables efficient instruction tuning on target tasks,
where the zero-shot performance is near zero.
When the model is fine-tuned on the image caption task, Flickr30K, which largely differs from the pre-trained tasks of visual question-answering,
the model can learn a degraded multimodal representation,
which causes visual forgetting in its projected visual representation space (in Section~\ref{sec:visual_forget}) 
and its average performance on pre-trained tasks drops 33.63\% compared with zero-shot performance. 
Fine-tuning on visual question-answering task OKVQA, which is similar to the pre-trained tasks, can also cause a 13.44\% performance drop,
due to the limited image-text pairs existing in the downstream task, which potentially leads to MLLM's visual understanding drift.

\subsubsection{Smaller MLLM also experiences visual forgetting while limited in downstream task improvements.}
To validate the observation on a smaller MLLM, we report the comparison results of MiniCPM-V-2.0 with 2.8B model parameters in Table~\ref{tab:minicpm}.
We observe that compared with the LLaVA MLLM, MiniCPM suffers from less prominent visual forgetting.
The average performance drop of the model limits to 6.28\% and 4.25\% when fine-tuning on PathVQA and TextCaps, respectively.
We attribute this observation to MiniCPM learning a more compact and constrained visual representation space during pre-training, 
causing the visual representations of target task images to be less aligned with those of the pre-trained MLLM. 
Consequently, MiniCPM exhibits limited improvement in downstream tasks, 
as its restricted ability to acquire additional visual knowledge leads to ineffective instruction tuning.

\subsubsection{MDGD prevents visual forgetting while maintaining downstream task improvements.}
By employing MDGD in MLLM instruction tuning,
we observe the LLaVA's average performance drop on pre-trained tasks reduces to 3.59\% when fine-tuned on OKVQA
and also achieves a 1.45\% improvement when fine-tuned on Flickr30K,
which demonstrates the efficiency of MDGD in mitigating visual forgetting.
For the smaller MLLM, MiniCPM, MDGD achieves comparable downstream task improvements with direct fine-tuning,
while completely eliminating visual forgetting in the pre-trained tasks.
MDGD and its variants consistently achieve the best average performance for both MLLMs,
demonstrating its great potential for incremental learning on individual downstream tasks.

\subsubsection{Comparison with baseline methods.}
In Table~\ref{tab:llava-main}, we compare MDGD with LoRA fine-tuning and Model Tailor \cite{zhu2024model} on LLaVA-1.5, 
which are designed for parameter-efficient fine-tuning.
We observe that LoRA fine-tuning can suffer from significant visual forgetting on Flickr30K and OKVQA.
Since LoRA introduces additional representation projections in intermediate layers,
the pre-trained multimodal representations can be projected into a lower-rank subspace leading to visual forgetting (in Section~\ref{sec:visual_forget}),
due to the limitation of image-text pairs in the target dataset.
Model Tailor is designed for MLLM anti-forgetting, 
which identifies ``patches'' of sub-model parameters significantly affected by fine-tuning on the target task.
However, since the method is not specifically designed for MLLMs, 
the unique challenge of visual forgetting cannot be effectively mitigated while maintaining robust performance on the target task.Thus, we observe that Model Tailor’s performance is sensitive to the target task datasets (e.g., better on Flickr30K than OKVQA), 
whereas MDGD consistently outperforms Model Tailor in terms of both average task scores and H-scores across the two datasets.
In Table~\ref{tab:minicpm}, we also report the results of the MDGD comparison with LoRA fine-tuning on MiniCPM.
We observe consistent improvements on the average task performance of MDGD when fine-tuned on PathVQA and TextCaps,
especially MDGD achieves 2.43\% and 1.83\% on the pre-trained tasks of PathVQA and TextCaps, respectively,
which demonstrates the effectiveness of MDGD in mitigating visual forgetting. 

\begin{table*}[ht]
  \centering
  \small
  \begin{tabular}{lccccccccccc}
    \toprule   
    \multirow{2}{*}{Method} & \multirow{2}{*}{\#Params} & \multicolumn{7}{c}{Pre-trained tasks} & \multicolumn{1}{c}{Target task} & \multicolumn{2}{c}{Metrics} \\
    \cmidrule(lr){3-9} \cmidrule(lr){10-10} \cmidrule(lr){11-12}
    &                       & \textbf{VizWiz} & \textbf{A-OKVQA} & \textbf{OKVQA} & \textbf{TextVQA} & \textbf{IconQA} & \textbf{POPE} & \textbf{MMBench} & \textbf{PathVQA} & \textbf{Avg} & \textbf{Hscore} \\
    \midrule
    \textbf{Zero-shot}       & -     & 55.27  & 79.39 & 64.86 & 77.98 & 79.01 & 88.93 & 70.98 & 5.44 & 65.23   & 10.04\\
    \midrule
    \textbf{Fine-tune}       & 517M  & 52.91  & 76.94 & 59.06 & 58.34 & 76.96 & \textbf{89.60} & 70.16 & \underline{11.04} & 61.88  & \underline{18.74}\\
    \textbf{LoRA}            & 35M & 52.95  & 76.24 & \textbf{64.45} & 77.18 & 77.80 & 88.08 & 67.47 & \textbf{15.03} & 64.90 & \textbf{24.41}\\
    \midrule
    \textbf{MDGD}           &  517M  & \textbf{55.73}  & 78.25 & \underline{64.33} & \underline{77.54} & \textbf{79.45} & \underline{89.19} & \textbf{71.94} & 9.09 & \textbf{65.69}   & 15.97\\
    ~~w/o visual align & 517M  & 54.92  & \underline{78.52} & 64.17 & 77.42 & \underline{79.37} & 89.10 & 70.96 & 8.49 & \underline{65.37}  & 15.03\\
   \textbf{ MDGD-GM}  & 52M  & \underline{55.04}  & \textbf{78.78} & 64.31 & \textbf{77.78} & 79.10 & 88.76 & \underline{70.98} & 5.72 & 65.06  & 10.52 \\
    
    \bottomrule
    \toprule
    
    \multirow{2}{*}{Method} & \multirow{2}{*}{\#Params} & \multicolumn{7}{c}{Pre-trained tasks} & \multicolumn{1}{c}{Target task} & \multicolumn{2}{c}{Metrics} \\
    \cmidrule(lr){3-9} \cmidrule(lr){10-10} \cmidrule(lr){11-12}
    &                        & \textbf{VizWiz} & \textbf{A-OKVQA} & \textbf{OKVQA} & \textbf{TextVQA} & \textbf{IconQA} & \textbf{POPE} & \textbf{MMBench} & \textbf{TextCaps} & \textbf{Avg} & \textbf{Hscore} \\
    \midrule
    \textbf{Zero-shot}        & -     & 55.27  & 79.39 & 64.86 & 77.98 & 79.01 & 88.93 & 70.98 & 15.77 & 66.52 & 25.50  \\
    \midrule
    \textbf{Fine-tune}        & 517M  & 52.03  & 77.73 & 59.16 & 67.24 & 78.67 & 88.20 & 71.42 & \textbf{33.85} & 66.04 & \textbf{44.76} \\
    \textbf{LoRA}             & 35M  & 53.30  & \underline{78.17}          & \underline{63.99} & \underline{77.68} & 78.28 & 87.31 & 69.23 & \underline{32.41} & 67.55 & \underline{43.80} \\
    \midrule
    \textbf{MDGD}             & 517M  & \textbf{55.17}  & \underline{78.17} & 63.67          & 76.08 & \underline{79.40} & \textbf{89.11} & \underline{71.58} & 28.90 & \underline{67.76} & 40.52 \\
    ~~w/o visual align & 517M  & 51.35           & 78.08            & 63.06          & 76.48 & 78.99                & \underline{88.98} & 71.30 & 25.93 & 66.77 & 37.35 \\
    \textbf{MDGD-GM}  & 52M   & \underline{55.04}  & \textbf{78.43} & \textbf{65.26} & \textbf{78.08} & \textbf{79.65} & 88.93           & \textbf{71.88} & 29.14 & \textbf{68.30} & 40.85 \\
    \bottomrule
  \end{tabular}
  \caption{
  Performance on various pre-trained tasks of MiniCPM-V2.5 models fine-tuned on PathVQA and TextCaps. 
  We report the best performance for each task in a \textbf{bold font} while the second best performance \underline{underlined}.
  }
  \label{tab:minicpm}
  \vspace{-.8cm}
\end{table*}


\section{Conclusion}\label{Sec:con}
This work introduces a benchmarking for model-free varying-diameter log-grasping with a forestry crane, including the structure of the environment, design of reward functions, and a modified proximal policy optimization (mPPO) algorithm. Under the assumption that the log pose is given, extensive simulations are presented to show the effectiveness of the reward shape and the exploration capability of the mPPO over other algorithms. The overall success rate of the grasping task of varying-diameter wood logs, varying log poses, and randomized initial configurations of the forestry crane exceeds $96\%$. 

\textbf{Limitation.} Although our method shows promising results, we recognize many aspects that require further attention, particularly regarding the sim-to-real gap. For instance, while the simulation offers many benefits, real-world uncertainties such as sensor noise, actuation delays, and unexpected disturbances will require more robust handling. The computational efficiency, especially the training time, can be further optimized by leveraging GPU acceleration. Additionally, incorporating transfer learning techniques may help improve the generalization to physical systems. In future work, we will focus on deploying the learned model in real-world demonstrations and aim to refine the agent’s ability to adapt to dynamic, unpredictable conditions. 


%Additionally, the training process can integrate object estimation and imitation learning. 
%Closing the sim-to-real gap is not a trivial problem since we consider a large-scale hydraulically actuated robot. This is also our main focus for the future work. 




%\section*{Acknowledgment}
%\addcontentsline{toc}{section}{Acknowledgment}
%\lipsum[1]


\bibliographystyle{class/IEEEtran}
\bibliography{class/reference}
\end{document}