%%%%%%%%%%%%%%%%%%%%%%%%%%%%%%%%%%%%%%%%%%%%%%%%%%%%%%%%%%%%%%%%%%%%%%%%%%%%%%%%
%2345678901234567890123456789012345678901234567890123456789012345678901234567890
%        1         2         3         4         5         6         7         8

\documentclass[letterpaper, 10 pt, conference]{class/ieeeconf}  % Comment this line out
                                                          % if you need a4paper

\IEEEoverridecommandlockouts                              % This command is only
                                                          % needed if you want to
                                                          % use the \thanks command
\overrideIEEEmargins
\usepackage{xcolor}
\usepackage{siunitx}  
% definition of notes command
\setlength{\marginparwidth}{0.5cm}
\setlength{\marginparpush}{0.5cm}
\newcommand{\note}[1]{ \marginpar{\color{red!60!gray} \small #1} }
%\renewcommand{\note}[1]{}
\usepackage{graphicx}
 \graphicspath{./figures/robot_demonstration}
% The following packages can be found on http:\\www.ctan.org

\usepackage{amsmath}
\usepackage{amssymb}
\usepackage{latexsym}
\usepackage{url}
\usepackage{cite}
\usepackage{relsize}
\usepackage{multirow}
\usepackage{afterpage}
\usepackage{ifthen}
\usepackage{graphicx}
\usepackage{algpseudocode,algorithm,algorithmicx}
\usepackage{subfigure}
%\usepackage[keeplastbox]{flushend}
\usepackage{epstopdf}
\usepackage{stackengine}
\usepackage{textcomp} % new



\usepackage{gensymb}
%\usepackage[bookmarks=false, linkcolor=blue, urlcolor=blue, citecolor=blue]{hyperref} 
\usepackage{booktabs}
\usepackage{makecell}
\usepackage{hhline}
\usepackage{courier}
\usepackage{lipsum}
\usepackage{bbm}
\usepackage{bm}
\usepackage{pifont}% http://ctan.org/pkg/pifont
%\newcommand{\norm}[1]{\left \lVert #1 \right \rVert}

\newcommand{\xmark}{\ding{55}}%

\usepackage{xcolor} % new
            
%\hypersetup{pdfstartview={XYZ null null 1.00}}
%\hypersetup{pdfstartview={FitH}}

%%%% for reference blue color
\makeatletter
\let\NAT@parse\undefined
\makeatother
\usepackage[bookmarks=false, linkcolor=blue, urlcolor=blue, citecolor=blue]{hyperref} 
\usepackage{etoolbox}
\makeatletter
\patchcmd{\@makecaption}
  {\scshape}
  {}
  {}
  {}
\makeatother
\hypersetup{
    colorlinks=true,
    linkcolor=red,
    filecolor=magenta,      
    urlcolor=blue,
    pdfstartview={FitH},
    citecolor =blue
    }

    
\setlength{\floatsep}{0.1in}
\setlength{\dblfloatsep}{0.1in}
\setlength{\textfloatsep}{0.1in}
\setlength{\dbltextfloatsep}{0.1in}
\setlength{\intextsep}{0.1in}
\setlength{\abovecaptionskip}{-0.1in}

\renewcommand{\UrlFont}{\scriptsize}
\DeclareRobustCommand{\vec}[1]{ 				
	\ifthenelse{\equal{#1}{\omega} \OR \equal{#1}{\varphi} \OR \equal{#1}{\alpha} \OR \equal{#1}{\beta} \OR \equal{#1}{\chi} \OR \equal{#1}{\delta} \OR \equal{#1}{\varepsilon} \OR \equal{#1}{\phi} \OR \equal{#1}{\epsilon} \OR \equal{#1}{\gamma} \OR \equal{#1}{\eta} \OR \equal{#1}{\iota} \OR \equal{#1}{\kappa} \OR \equal{#1}{\lambda} \OR \equal{#1}{\mu} \OR \equal{#1}{\nu} \OR \equal{#1}{\pi} \OR \equal{#1}{\theta} \OR \equal{#1}{\vartheta} \OR \equal{#1}{\rho} \OR \equal{#1}{\sigma} \OR \equal{#1}{\varsigma} \OR \equal{#1}{\tau} \OR \equal{#1}{\upsilon} \OR \equal{#1}{\xi} \OR \equal{#1}{\psi} \OR \equal{#1}{\zeta}}{
		% Fuer griechische Kleinbuchstaben muss boldsymbol verwendet werden (deckt mathbf nicht ab)
		\boldsymbol{#1}
	}{
		% Alle anderen Symbole verwenden mathbf
		\mathbf{#1}
	}
}
\providecommand{\FullStop}{\text{~\@.\xspace}}
\providecommand{\Comma}{\text{~,\xspace}}
\newcommand{\transpose}[1]{#1^\mathrm{T}}
\newcommand{\numset}[1]{\mathbbm{#1}}

\usepackage{xspace}
\DeclareMathOperator*{\argmax}{arg\,max}
\providecommand{\kuka}{\textsc{KUKA} LBR iiwa R820\xspace}

\newcommand{\etal}{\textit{et al.}}
 \graphicspath{{./figures/}}

\usepackage{tikz}
\usepackage{pgfplots}
\usepackage{pgfplotstable}
\newcommand\corr[1]{\textcolor{black}{#1}}  % For additions/corrections/suggestions
\newcommand\marc[1]{\textcolor{blue}{#1}}  % For additions/corrections/suggestions

%\title{\LARGE \bf Open-Vocabulary Affordance Segmentation}

%\title{\LARGE \bf Open-Vocabulary Affordance Detection on 3D Point Clouds}

\title{\LARGE \bf Towards Autonomous Wood-Log Grasping with a Forestry Crane: Simulator and Benchmarking}

\iffalse
    \author{M.N. Vu$^{1,2}$, A. Wachter$^{1}$, G. Ebmer$^{1}$, M. Ecker$^{1,2}$, T. Gl{\"u}ck$^{2}$, A. Nguyen$^{3}$, \\W. Kemmetm{\"u}ller$^1$, and A. Kugi$^{1,2}$% <-this % stops a space
    \thanks{$^{1}$ M.N. Vu, A. Watcher, G. Ebmer, M. Ecker, W. Kemmetm{\"u}ller, and A. Kugi are with the Automation \& Control Institute (ACIN), TU Wien, 1040 Vienna, Austria {\tt\small \{vu,watcher,ebmer,ecker,\}@acin.tuwien.ac.at}}%
    \thanks{$^{2}$ M.N.Vu, M. Ecker,  T. Gl{\"u}ck, and A. Kugi are with AIT Austrian Institute of Technology GmbH, 1210 Vienna, Austria {\tt\small \{minh.vu, marc.ecker,tobias.glueck,andreas.kugi\}@ait.ac.at}}%     
    \thanks{$^{3}$ A. Nguyen is with Department of Computer Science, University of Liverpool {\tt\small a.nguyen@liverpool.ac.uk}}
    }
\fi
\author{M.N. Vu$^{1,2}$, A. Wachter$^{1}$, G. Ebmer$^{1}$, M. Ecker$^{1,2}$, T. Gl{\"u}ck$^{2}$, A. Nguyen$^{3}$, \\W. Kemmetm{\"u}ller$^1$, and A. Kugi$^{1,2}$% <-this % stops a space
    \thanks{$^{1}$Automation \& Control Institute (ACIN), TU Wien, 1040 Vienna, Austria {\tt\small vu@acin.tuwien.ac.at}}%
    \thanks{$^{2}$AIT Austrian Institute of Technology GmbH, 1210 Vienna, Austria}%     
    \thanks{$^{3}$Department of Computer Science, University of Liverpool}
    }
\begin{document}
% Macros

\newtheorem{problem}{Problem}
\newtheorem{lemma}{Lemma}
\newtheorem{theorem}[lemma]{Theorem}
\newtheorem{claim}{Claim}
\newtheorem{corollary}[lemma]{Corollary}
\newtheorem{definition}[lemma]{Definition}
\newtheorem{proposition}[lemma]{Proposition}
\newtheorem{remark}[lemma]{Remark}
\newenvironment{LabeledProof}[1]{\noindent{\it Proof of #1: }}{\qed}

\def\beq#1\eeq{\begin{equation}#1\end{equation}}
\def\bea#1\eea{\begin{align}#1\end{align}}
\def\beg#1\eeg{\begin{gather}#1\end{gather}}
\def\beqs#1\eeqs{\begin{equation*}#1\end{equation*}}
\def\beas#1\eeas{\begin{align*}#1\end{align*}}
\def\begs#1\eegs{\begin{gather*}#1\end{gather*}}

\newcommand{\poly}{\mathrm{poly}}
\newcommand{\eps}{\epsilon}
\newcommand{\e}{\epsilon}
\newcommand{\polylog}{\mathrm{polylog}}
\newcommand{\rob}[1]{\left( #1 \right)} %Round Brackets
\newcommand{\sqb}[1]{\left[ #1 \right]} %square Brackets
\newcommand{\cub}[1]{\left\{ #1 \right\} } %curly brackets
\newcommand{\rb}[1]{\left( #1 \right)} %Round
\newcommand{\abs}[1]{\left| #1 \right|} %| |
\newcommand{\zo}{\{0, 1\}}
\newcommand{\zonzo}{\zo^n \to \zo}
\newcommand{\zokzo}{\zo^k \to \zo}
\newcommand{\zot}{\{0,1,2\}}
\newcommand{\en}[1]{\marginpar{\textbf{#1}}}
\newcommand{\efn}[1]{\footnote{\textbf{#1}}}
\newcommand{\vecbm}[1]{\boldmath{#1}} %more general (handles greek letters)
\newcommand{\uvec}[1]{\hat{\vec{#1}}}
\newcommand{\thv}{\vecbm{\theta}}
\newcommand{\junk}[1]{}
\newcommand{\var}{\mathop{\mathrm{var}}}
\newcommand{\rank}{\mathop{\mathrm{rank}}}
\newcommand{\diag}{\mathop{\mathrm{diag}}}
\newcommand{\tr}{\mathop{\mathrm{tr}}}
\newcommand{\acos}{\mathop{\mathrm{acos}}}
\newcommand{\atantwo}{\mathop{\mathrm{atan2}}}
\newcommand{\SVD}{\mathop{\mathrm{SVD}}}
\newcommand{\quadf}{\mathop{\mathrm{q}}}
\newcommand{\linterp}{\mathop{\mathrm{l}}}
\newcommand{\sgn}{\mathop{\mathrm{sign}}}
\newcommand{\sym}{\mathop{\mathrm{sym}}}
\newcommand{\avg}{\mathop{\mathrm{avg}}}
\newcommand{\mean}{\mathop{\mathrm{mean}}}
\newcommand{\erf}{\mathop{\mathrm{erf}}}
\newcommand{\grad}{\nabla}
\newcommand{\R}{\mathbb{R}}
\newcommand{\defeq}{\triangleq}
\newcommand{\dims}[2]{[#1\!\times\!#2]}
\newcommand{\sdims}[2]{\mathsmaller{#1\!\times\!#2}}
\newcommand{\udims}[3]{#1}
\newcommand{\udimst}[4]{#1}
\newcommand{\com}[1]{\rhd\text{\emph{#1}}}
\newcommand{\ind}{\hspace{1em}}
\newcommand{\argmin}[1]{\underset{#1}{\operatorname{argmin}}}
\newcommand{\floor}[1]{\left\lfloor{#1}\right\rfloor}
\newcommand{\step}[1]{\vspace{0.5em}\noindent{#1}}
\newcommand{\quat}[1]{\ensuremath{\mathring{\mathbf{#1}}}}
\newcommand{\norm}[1]{\left\lVert#1\right\rVert}
\newcommand{\ignore}[1]{}
\newcommand{\specialcell}[2][c]{\begin{tabular}[#1]{@{}c@{}}#2\end{tabular}}
\newcommand*\Let[2]{\State #1 $\gets$ #2}
\newcommand{\algorithmicbreak}{\textbf{break}}
\newcommand{\Break}{\State \algorithmicbreak}
\newcommand{\ra}[1]{\renewcommand{\arraystretch}{#1}}

\renewcommand{\vec}[1]{\mathbf{#1}} %looks better

\algdef{S}[FOR]{ForEach}[1]{\algorithmicforeach\ #1\ \algorithmicdo}
\algnewcommand\algorithmicforeach{\textbf{for each}}
\algrenewcommand\algorithmicrequire{\textbf{Require:}}
\algrenewcommand\algorithmicensure{\textbf{Ensure:}}
\algnewcommand\algorithmicinput{\textbf{Input:}}
\algnewcommand\INPUT{\item[\algorithmicinput]}
\algnewcommand\algorithmicoutput{\textbf{Output:}}
\algnewcommand\OUTPUT{\item[\algorithmicoutput]}

\maketitle
\thispagestyle{empty}
\pagestyle{empty}

%%%%%%%%%%%%%%%%%%%%%%%%%%%%%%%%%%%%%%%%%%%%%%%%%%%%%%%%%%%%%%%%%%%%%%%%%%%%%%%%
\begin{abstract}
Forestry machines operated in forest production environments face challenges when performing manipulation tasks, especially regarding the complicated dynamics of underactuated crane systems and the heavy weight of logs to be grasped. 
This study investigates the feasibility of using reinforcement learning for forestry crane manipulators in grasping and lifting heavy wood logs autonomously. We first build a simulator using Mujoco physics engine to create realistic scenarios, including modeling a forestry crane with $8$ degrees of freedom from CAD data and wood logs of different sizes. 
We further implement a velocity controller for autonomous log grasping with deep reinforcement learning using a curriculum strategy. Utilizing our new simulator, the proposed control strategy exhibits a success rate of 96\% when grasping logs of different diameters and under random initial configurations of the forestry crane. 
In addition, reward functions and reinforcement learning baselines are implemented to provide an open-source benchmark for the community in large-scale manipulation tasks. A video with several demonstrations can be seen at \href{https://www.acin.tuwien.ac.at/en/d18a/}{https://www.acin.tuwien.ac.at/en/d18a/}. 
\end{abstract}

%%%%%%%%%%%%%%%%%%%%%%%%%%%%%%%%%%%%%%%%%%%%%%%%%%%%%%%%%%%%%%%%%%%%%%%%%%%%%%%%

\section{Introduction}
Backdoor attacks pose a concealed yet profound security risk to machine learning (ML) models, for which the adversaries can inject a stealth backdoor into the model during training, enabling them to illicitly control the model's output upon encountering predefined inputs. These attacks can even occur without the knowledge of developers or end-users, thereby undermining the trust in ML systems. As ML becomes more deeply embedded in critical sectors like finance, healthcare, and autonomous driving \citep{he2016deep, liu2020computing, tournier2019mrtrix3, adjabi2020past}, the potential damage from backdoor attacks grows, underscoring the emergency for developing robust defense mechanisms against backdoor attacks.

To address the threat of backdoor attacks, researchers have developed a variety of strategies \cite{liu2018fine,wu2021adversarial,wang2019neural,zeng2022adversarial,zhu2023neural,Zhu_2023_ICCV, wei2024shared,wei2024d3}, aimed at purifying backdoors within victim models. These methods are designed to integrate with current deployment workflows seamlessly and have demonstrated significant success in mitigating the effects of backdoor triggers \cite{wubackdoorbench, wu2023defenses, wu2024backdoorbench,dunnett2024countering}.  However, most state-of-the-art (SOTA) backdoor purification methods operate under the assumption that a small clean dataset, often referred to as \textbf{auxiliary dataset}, is available for purification. Such an assumption poses practical challenges, especially in scenarios where data is scarce. To tackle this challenge, efforts have been made to reduce the size of the required auxiliary dataset~\cite{chai2022oneshot,li2023reconstructive, Zhu_2023_ICCV} and even explore dataset-free purification techniques~\cite{zheng2022data,hong2023revisiting,lin2024fusing}. Although these approaches offer some improvements, recent evaluations \cite{dunnett2024countering, wu2024backdoorbench} continue to highlight the importance of sufficient auxiliary data for achieving robust defenses against backdoor attacks.

While significant progress has been made in reducing the size of auxiliary datasets, an equally critical yet underexplored question remains: \emph{how does the nature of the auxiliary dataset affect purification effectiveness?} In  real-world  applications, auxiliary datasets can vary widely, encompassing in-distribution data, synthetic data, or external data from different sources. Understanding how each type of auxiliary dataset influences the purification effectiveness is vital for selecting or constructing the most suitable auxiliary dataset and the corresponding technique. For instance, when multiple datasets are available, understanding how different datasets contribute to purification can guide defenders in selecting or crafting the most appropriate dataset. Conversely, when only limited auxiliary data is accessible, knowing which purification technique works best under those constraints is critical. Therefore, there is an urgent need for a thorough investigation into the impact of auxiliary datasets on purification effectiveness to guide defenders in  enhancing the security of ML systems. 

In this paper, we systematically investigate the critical role of auxiliary datasets in backdoor purification, aiming to bridge the gap between idealized and practical purification scenarios.  Specifically, we first construct a diverse set of auxiliary datasets to emulate real-world conditions, as summarized in Table~\ref{overall}. These datasets include in-distribution data, synthetic data, and external data from other sources. Through an evaluation of SOTA backdoor purification methods across these datasets, we uncover several critical insights: \textbf{1)} In-distribution datasets, particularly those carefully filtered from the original training data of the victim model, effectively preserve the model’s utility for its intended tasks but may fall short in eliminating backdoors. \textbf{2)} Incorporating OOD datasets can help the model forget backdoors but also bring the risk of forgetting critical learned knowledge, significantly degrading its overall performance. Building on these findings, we propose Guided Input Calibration (GIC), a novel technique that enhances backdoor purification by adaptively transforming auxiliary data to better align with the victim model’s learned representations. By leveraging the victim model itself to guide this transformation, GIC optimizes the purification process, striking a balance between preserving model utility and mitigating backdoor threats. Extensive experiments demonstrate that GIC significantly improves the effectiveness of backdoor purification across diverse auxiliary datasets, providing a practical and robust defense solution.

Our main contributions are threefold:
\textbf{1) Impact analysis of auxiliary datasets:} We take the \textbf{first step}  in systematically investigating how different types of auxiliary datasets influence backdoor purification effectiveness. Our findings provide novel insights and serve as a foundation for future research on optimizing dataset selection and construction for enhanced backdoor defense.
%
\textbf{2) Compilation and evaluation of diverse auxiliary datasets:}  We have compiled and rigorously evaluated a diverse set of auxiliary datasets using SOTA purification methods, making our datasets and code publicly available to facilitate and support future research on practical backdoor defense strategies.
%
\textbf{3) Introduction of GIC:} We introduce GIC, the \textbf{first} dedicated solution designed to align auxiliary datasets with the model’s learned representations, significantly enhancing backdoor mitigation across various dataset types. Our approach sets a new benchmark for practical and effective backdoor defense.



\section{Related Work} \label{Sec:rw}

In \cite{ortiz2014increasing}, path planning approaches for the crane grapple are presented for moving from the initial configuration to the grasping position.  
Recently, Andersson et al. \cite{andersson2021reinforcement} presented an RL-based solution for the half-loading cycle, i.e. moving the crane to the log and grabbing the log. To simplify the setup, the authors in~\cite{andersson2021reinforcement} only consider a small log next to a forwarder on the ground and the poses of the log are random, however, the size of the log is constant. 

Recently, model-based approaches for controlling a forestry crane are investigated in  \cite{kalmari2014nonlinear} and \cite{hera2015model}, where model-predictive controllers are utilized to reduce the swinging motion of the grapples. A closed-loop RL-based controller has been proposed in \cite{dhakate2022autonomous} for a redundant hydraulic forestry crane for position-tracking tasks. In \cite{andersson2021reinforcement}, the reinforcement learning control for grasping a log is presented, which is most closely related to the present work. The authors utilized Proximal Policy Optimization (PPO) \cite{schulman2017proximal} to train multiple grasping strategies for a forestry crane on AGX dynamics \cite{algoryx}, which is a commercial software for simulating the dynamics of multi-contact systems. 
Although the RL policy in \cite{andersson2021reinforcement} is only trained for a fixed type of wood log, the paper gives good insights into the possibility of employing RL methods for large-scale robots. More recently, Ayoub et al. \cite{ayoub2023grasp} presented a robust grasp planning pipeline, including wood log detection, trajectory planning, and a controller for grasping multiple logs. The work in \cite{ayoub2023grasp} mainly focuses on using a convolutional neural network (CNN) to predict the grasp location and orientation.  
\section{Model}
\begin{figure*}[t]
  \centering
  \includegraphics[width=\textwidth]{figures/framework_fig2.pdf}
   \caption{
   The pipeline of our \Model framework. We first generate an initial task instruction using LLMs with in-context learning and sample trajectories aligned with the initial language instructions in the environment. Next, we use the LLM to summarize the sampled trajectories and generate refined task instructions that better match these trajectories. We then modify specific actions within the trajectories to perform new actions in the environment, collecting negative trajectories in the process. Using the refined task instructions, along with both positive and negative trajectories, we train a lightweight reward model to distinguish between matching and non-matching trajectories. The learned reward model can then collaborate with various LLM agents to improve task planning.
   }
   \label{fig:pipeline}
\end{figure*}

In this section, we provide a detailed introduction to our framework, autonomous Agents from automatic Reward Modeling And Planning (\Model). The framework includes automated reward data generation in section~\ref{sec:data}, reward model design in section~\ref{sec:model}, and planning algorithms in section~\ref{sec:plan}.

\subsection{Background}
The planning tasks for LLM agents can be typically formulated as a Partially Observable Markov Decision Process (POMDP): $(\mathcal{X}, \mathcal{S}, \mathcal{A}, \mathcal{O}, \mathcal{T})$, where:
\begin{itemize}
    \item $\mathcal{X}$ is the set of text instructions;
    \item $\mathcal{S}$ is the set of environment states;
    \item $\mathcal{A}$ is the set of available actions at each state;
    \item $\mathcal{O}$ represents the observations available to the agents, including text descriptions and visual information about the environment in our setting;
    \item $\mathcal{T}: \mathcal{S} \times \mathcal{A} \rightarrow \mathcal{S}$ is the transition function of states after taking actions, which is given by the environment in our settings. 
\end{itemize}

Given a task instruction $\mathit{x} \in \mathcal{X}$ and the initial environment state $\mathit{s_0} \in \mathcal{S}$, planning tasks require the LLM agents to propose a sequence of actions ${\{a_n\}_{n=1}^{N}}$ that aim to complete the given task, where $a_n \in \mathcal{A}$ represents the action taken at time step $n$, and $N$ is the total number of actions executed in a trajectory.
Following the $n$-th action, the environment transitions to state $\mathit{s_{n}}$, and the agent receives a new observation $\mathit{o_{n}}$. Based on the accumulated state and action histories, the task evaluator determines whether the task is completed.

An important component of our framework is the learned reward model $\mathcal{R}$, which estimates whether a trajectory $h$ has successfully addressed the task:
\begin{equation}
    r = \mathcal{R}(\mathit{x}, h),
\end{equation}
where $h = \{\{a_n\}_{n=1}^N, \{o_n\}_{n=0}^{N}\}$, $\{a_n\}_{n=1}^N$ are the actions taken in the trajectory, $\{o_n\}_{n=0}^{N}$ are the corresponding environment observations, and $r$ is the predicted reward from the reward model.
By integrating this reward model with LLM agents, we can enhance their performance across various environments using different planning algorithms.

\subsection{ Automatic Reward Data Generation.}
\label{sec:data}
To train a reward model capable of estimating the reward value of history trajectories, we first need to collect a set of training language instructions $\{x_m\}_{m=1}^M$, where $M$ represents the number of instruction goals. Each instruction corresponds to a set of positive trajectories $\{h_m^+\}_{m=1}^M$ that match the instruction goals and a set of negative trajectories $\{h_m^-\}_{m=1}^M$ that fail to meet the task requirements. This process typically involves human annotators and is time-consuming and labor-intensive~\citep{christiano2017deep,rafailov2024direct}. As shown in Fig.~\ref{fig:instruction_generation_sciworld} of the Appendix. we automate data collection by using Large Language Model (LLM) agents to navigate environments and summarize the navigation goals without human labels.

\noindent\textbf{Instruction Synthesis.} The first step in data generation is to propose a task instruction for a given observation. We achieve this using the in-context learning capabilities of LLMs. The prompt for instruction generation is shown in Fig.~\ref{fig:instruction_refinement_sciworld} of the Appendix. Specifically, we provide some few-shot examples in context along with the observation of an environment state to an LLM, asking it to summarize the observation and propose instruction goals. In this way, we collect a set of synthesized language instructions $\{x_m^{raw}\}_{m=1}^M$, where $M$ represents the total number of synthesized instructions.

\noindent\textbf{Trajectory Collection.} Given the synthesized instructions $x_m^{raw}$ and the environment, an LLM-based agent is instructed to take actions and navigate the environment to generate diverse trajectories $\{x_m^{raw}, h_m\}_{m=0}^M$ aimed at accomplishing the task instructions. Here, $h_m$ represents the $m$-th history trajectory, which consists of $N$ actions $\{a_n\}_{n=1}^N$ and $N+1$ environment observations $\{o_n\}_{n=0}^N$.
Due to the limited capabilities of current LLMs, the generated trajectories $h_m$ may not always align well with the synthesized task instructions $x_m$. To address this, we ask the LLM to summarize the completed trajectory $h_m$ and propose a refined goal $x_m^r$. This process results in a set of synthesized demonstrations $\{x_m^r, h_m\}_{m=0}^{M_r}$, where $M_r$ is the number of refined task instructions.

\noindent\textbf{Pairwise Data Construction.} 
To train a reward model capable of distinguishing between good and poor trajectories, we also need trajectories that do not satisfy the task instructions. To create these, we sample additional trajectories that differ from $\{x_m^r, h_m\}$ and do not meet the task requirements by modifying actions in $h_m$ and generating corresponding negative trajectories $\{h_m^-\}$. For clarity, we refer to the refined successful trajectories as $\{x_m, h_m^+\}$ and the unsuccessful ones as $\{x_m, h_m^-\}$. These paired data will be used to train the reward model described in Section~\ref{sec:model}, allowing it to estimate the reward value of any given trajectory in the environment.

\subsection{ Reward Model Design.} 
\label{sec:model}
\noindent\textbf{Reward Model Architectures.}
Theoretically, we can adopt any vision-language model that can take a sequence of visual and text inputs as the backbone for the proposed reward model. In our implementation, we use the recent VILA model~\citep{lin2023vila} as the backbone for reward modeling since it has carefully maintained open-source code, shows strong performance on standard vision-language benchmarks like~\citep{fu2023mme,balanced_vqa_v2,hudson2018gqa}, and support multiple image input. 

The goal of the reward model is to predict a reward score to estimate whether the given trajectory $(x_m, h_m)$  has satisfied the task instruction or not, which is different from the original goal of VILA models that generate a series of text tokens to respond to the task query. To handle this problem, we additionally add a fully-connected layer for the model, which linearly maps the hidden state of the last layer into a scalar value. 

\noindent\textbf{Optimazation Target.}
Given the pairwise data that is automatically synthesized from the environments in Section~\ref{sec:data}, we optimize the reward model by distinguishing the good trajectories $(x_m, h^+_m)$ from bad ones $(x_m, h^-_m)$. Following standard works of reinforcement learning from human feedback~\citep{bradley1952rank,sun2023salmon,sun2023aligning}, we treat the optimization problem of the reward model as a binary classification problem and adopt a cross-entropy loss. Formally, we have 
\begin{equation}
    \mathcal{L(\theta)} = -\mathbf{E}_{(x_m,h_m^+,h_m^-)}[\log\sigma(\mathcal{R}_\theta(x_m, h_m^+)-\mathcal{R}_\theta(x_m, h_m^-))],
\end{equation}
where $\sigma$ is the sigmoid function and $\theta$ are the learnable parameters in the reward model $\mathcal{R}$.
By optimizing this target, the reward model is trained to give higher value scores to the trajectories that are closer to the goal described in the task instruction. 

\subsection{ Planning with Large Vision-Langauge Reward Model.}
After getting the reward model to estimate how well a sampled trajectory match the given task instruction, we are able to combine it with different planning algorithms to improve LLM agents' performance. Here, we summarize the typical algorithms we can adopt in this paper.

\noindent\textbf{Best of N.} This is a simple algorithm that we can adopt the learned reward model to improve the LLM agents' performances. We first prompt the LLM agent to generate $n$ different trajectories independently and choose the one with the highest predicted reward score as the prediction for evaluation. Note that this simple method is previously used in natural language generation~\citep{zhang2024improving} and we adopt it in the context of agent tasks to study the effectiveness of the reward model for agent tasks.

\noindent\textbf{Reflexion.} Reflexion~\citep{shinn2024reflexion} is a planning framework that enables large language models (LLMs) to learn from trial-and-error without additional fine-tuning. Instead of updating model weights, Reflexion agents use verbal feedback derived from task outcomes. This feedback is converted into reflective summaries and stored in an episodic memory buffer, which informs future decisions. Reflexion supports various feedback types and improves performance across decision-making, coding, and reasoning tasks by providing linguistic reinforcement that mimics human self-reflection and learning. %This approach yields significant gains over baseline methods in several benchmarks.

\noindent\textbf{MCTS.} 
We also consider tree search-based planning algorithms like Monte Carlo Tree Search (MCTS)~\citep{coulom2006efficient,silver2017mastering} to find the optimal policy. 
There is a tree structure constructed by the algorithm, where each node represents a state and each edge signifies an action.
Beginning at the initial state of the root node, the algorithm navigates the state space to identify action and state trajectories with high rewards, as predicted by our learned reward model. 

The algorithm tracks 1) the frequency of visits to each node and 2) a value function that records the maximum predicted reward obtained from taking action ${a}$ in state ${s}$.
MCTS would visit and expand nodes with either higher values (as they lead to high predicted reward trajectory) or with smaller visit numbers (as they are under-explored).
We provide more details in the implementation details and the appendix section.


\label{sec:plan}
\section{Wood-Log Grasping Method}
\label{sec: method}
Since a varied-diameter wood log is considered for the grasping task, the latent Markov decision process (L-MDP) \cite{chen2021understanding,vuong2019pick} is utilized in this work. The reward function and the policy gradient method are briefly discussed in this section. 


\subsection{Latent-MDP for forestry crane with varied-diameter wood logs}
The control problem for the grasping task can be modeled as a Markov decision process (MDP), which emulates the interactive learning between the agent (the grasping controller) and the simulated environment. An MDP consists of the $4$-element tuple $(\mathcal{S},\mathcal{A},\mathbf{P},R)$ referring to the state space $\mathcal{S}$, the action space $\mathcal{A}$, the transition probability density function $\mathbf{P}$, and the reward function $R$, respectively. 
%At a given discrete time step $t$, the forestry crane system is in state $\mathbf{s}_t \in \mathcal{S}$ and the agent perceives observations  $\mathbf{o}_t \in \mathcal{O}$ 
At a given time step $t$, the agent, in the state $\mathbf{s}_t \in \mathcal{S}$, selects an action $\mathbf{a}_t \in \mathcal{A}$ to transition to the next state $\mathbf{s}_{t+1}$ %\marc{according to the distribution (instead of "with the function")} 
with the transition probability $\mathbf{P}(\mathbf{s}_{t+1}|\mathbf{s}_t,\mathbf{a}_t)$. 
This results in the immediate reward $R_t$. Note that $\mathbf{P}(\mathbf{s}_{t+1}|\mathbf{s}_t,\mathbf{a}_t)$ is obtained from the simulator introduced in Section \ref{sec: b simulator}. It is worth noting that the transition function $\mathbf{P}$ is uncertain since the contact dynamics are not the same for different wood log dimensions.  
%\marc{(The transition PDF in the RL setting always describes an uncertain process. Do you mean that it additionally depends on the log dimensions?)}. 
Details on the state $\mathbf{s}_t$, the action $\mathbf{a}_t$, and the reward function $R_t$ for the grasping task with the forestry crane are introduced in the following subsection. 

Since the diameter $d$ of a wood log and its mass vary during the training process, the latent MDP (L-MDP) is utilized, where the log's size can be considered a latent variable. Additionally, individual training episodes are considered as single MDPs with finite length $H$. We denote the L-MDP as $\{\mathcal{L}, p(d)\}$, where $\mathcal{L}$ is the set of single MDPs with different diameters $d$, and $p(d)$ is the \corr{uniform} distribution of the diameter $d$ over $\mathcal{L}$. To this end, the objective of the grasping task is as follows
\begin{equation}
    J(\bm{\pi_\theta}) = \mathrm{E}_{d\sim p(d)}\bigg[\mathrm{E}_{\mathbf{a}_t\sim \bm{\pi_\theta}(.|\mathbf{s}_t)} \Bigg[\sum_{t=0}^{H}\gamma R_t\Bigg] \bigg]\:,
    \label{eq: RL objective}
\end{equation}
\corr{where $\mathrm{E}$ is the expectation function, the symbol ``$\sim$'' denotes the sampling process from the corresponding distribution $\bm{\pi_\theta}$ over the action space $\mathcal{A}$, and $0<\gamma<1$ is the discount factor. The normal distribution is typically used to model $\bm{\pi_\theta}$.} 
Additionally, $\bm{\theta}$ combines the policy parameters that can be weights and biases of a neural network. 
An RL method is utilized to find the optimal policy $\bm{\pi}^*$ that maximizes (\ref{eq: RL objective}) 
\begin{equation}
    \bm{\pi_\theta}^* = \argmax_{\bm{\pi_\theta}} J(\bm{\pi_\theta}) \:.
    \label{eq: RL policy}
\end{equation}
To find the optimal policy (\ref{eq: RL policy}), we utilize the modified version of Proximal Policy Optimization (PPO) as discussed in Subsection \ref{sec: RPPO}. The following subsection presents the details of the observation space, action space, and reward function. 
%Acting $a_t \in \mathcal{A}$ according to the policy distribution $\pi(a|s)$, the agent receives an immediate scalar reward $r_t(s_t, a_t)$ according to the specified reward function $R(s, a)$. 
%The goal of RL algorithms is to find the optimal policy $\pi(a|s)^*$ such that the agent takes the optimal action at any given state to maximize the expected return. 
%Here, the deep RL approach involves parameterizing the policy $\pi$ as a neural network $\pi_\theta$ with parameters $\theta$. The resulting policy approximator outputs 
\subsection{Learning environment for the forestry crane}
\subsubsection{Observations and actions} 
\label{sec: observation}
The observations consist of joint angles of the forestry crane and the actuated joint velocities, i.e., $\mathcal{O} = \{\mathbf{q},\dot{\mathbf{q}}_{A}\}$.  
From the 6 DoF poses of the log's pose w.r.t the crane's base obtained by other algorithms \cite{wen2023bundlesdf,vuong2023grasp}, we consider the reduced poses of 4 DoF $\mathbf{q}_l = [x_{l},y_{l}, z_{l}, \psi_l]^\mathrm{T}$, consisting the 3D Cartesian position of the log's center point $\mathbf{p}_l = [x_{l},y_{l}, z_{l}]^\mathrm{T}$ and the yaw angle $\psi_l$, see Figure \ref{fig: rl explained}. The augmented relative Cartesian distance is computed as
\begin{equation}
    \bm{\Delta}_p =  [x_{l},y_{l}, z_{l} - (d_{max}-z_l)/2]^\mathrm{T} - \mathbf{p}_\mathrm{C}(\mathbf{q}) \:,
    \label{eq: relative distance}
\end{equation}
where $\mathbf{p}_\mathrm{C}(\mathbf{q}) = [p_{\mathrm{C},x},p_{\mathrm{C},y},p_{\mathrm{C},z}]^\mathrm{T}$ results from the forward kinematics
\begin{equation}
    \mathbf{H}_\mathrm{C}(\mathbf{q}) = 
    \begin{bmatrix}
        \mathbf{e}_{\mathrm{C},x} & \mathbf{e}_{\mathrm{C},y} & \mathbf{e}_{\mathrm{C},z} & \mathbf{p}_\mathrm{C} \\
        0 & 0 & 0 & 1
    \end{bmatrix}
\end{equation}
and is located at the center of the grapple, see Figure \ref{fig: rl explained}. Note that $\mathbf{e}_{\mathrm{C},x}$, $\mathbf{e}_{\mathrm{C},y}$, and $\mathbf{e}_{\mathrm{C},z}$ are column vectors of the orientation of the grapple's center. The term $d_{off} = (d_{max}-z_l)/2$ represents an offset in $z$-direction for different log sizes where $d_{max} = 0.8$ is the maximum diameter of the wood log. 
Since $\psi_l$ is the yaw rotation around the $z$-axis of the crane base, illustrated in Fig. \ref{fig: rl explained}, the unit vector $\mathbf{e}_{l,y}$ along the length of the log is computed as
\begin{equation}
    \mathbf{e}_{l,y} = [-\sin(\psi_l), \cos(\psi_l), 0]^\mathrm{T}
\end{equation}
In order to successfully grasp the wood log, the orientation of the grapple must be well-aligned with the wood log, as defined in the following condition
\begin{equation}
     \mathrm{mod}\bigg[\widehat{(\mathbf{e}_{l,y},\mathbf{e}_{\mathrm{C},x})},\pi \bigg] \approx 0 \:, 
     \label{eq: orientation condition}
\end{equation}
where the $\widehat{(\mathbf{e}_{l,y},\mathbf{e}_{\mathrm{C},x})}$ presents the angle between the two vectors $\mathbf{e}_{l,y}$ and $\mathbf{e}_{\mathrm{C},x}$. 
The condition (\ref{eq: orientation condition}) can be normalized as the angle distance function in the form
\begin{equation}
    %\Delta_{\psi} = \dfrac{\mathbf{e}_{\mathrm{C},x}\cdot \mathbf{e}_{l,y}}{\norm{\mathbf{e}_{\mathrm{C},x}} \norm{\mathbf{e}_{\mathrm{l},y}}} \:\:,
    \Delta_{\psi} = 1- |\mathbf{e}_{\mathrm{C},x}\cdot \mathbf{e}_{l,y} |\:\:,
    \label{eq: angle distance}
\end{equation}
where the symbol ``$\cdot$'' denotes the dot product between two vectors. 
To this end, the observation space also includes the relative distance (\ref{eq: relative distance}) and the angle distance (\ref{eq: angle distance}), i.e., $\mathcal{O} = \{\mathbf{q},\dot{\mathbf{q}}_{A},\bm{\Delta}_p,\Delta_\psi\}$. As ideal underlying velocity controllers are assumed, the action space consists of desired actuated joint velocities $\mathcal{A} = \{\dot{\mathbf{q}}_{A,d}\}$. 
    %The observation and action spaces are listed in Table \ref{tab:crane_obs_act}. 

\begin{figure}[t]
\centering
\scalebox{0.6}{
\def\svgwidth{1\columnwidth}
\input{figures/explain_rl_env_2.pdf_tex}
}
\vspace{0.1cm}
\caption{Details of variables used for constructing the observations and reward function.}
\label{fig: rl explained}
\vspace{-0.2ex}
\end{figure}
    
\iffalse
    \begin{table}
        
        \caption[abc]{Summary of the observations and actions.}
        \begin{center} 
        \begin{tabular}{c | c c}
        \hline
        & Observations & Actions\\
        \hline
        \multirow{8}{*}{Joint angles and angle rates} & $q_1,\:  \dot{q}_{1}$ & $\dot{q}_{1,d}$\\
         & $q_2 \:  \dot{q}_{2}$ & $\dot{q}_{2,d}$\\
         & $q_3 \:  \dot{q}_{3}$ & $\dot{q}_{3,d}$\\
         & $q_4\:  \dot{q}_{4}$ & $\dot{q}_{4,d}$\\
         & $q_5$ \\
         & $q_6$ \\
         & $q_7\:  \dot{q}_{7}$ & $\dot{q}_{7,d}$\\
         & $q_8\:  \dot{q}_{8}$ & $\dot{q}_{8,d}$\\
         \hline
        \multirow{3}{*}{Relative distance $\bm{\Delta}_p$} & $x_l - p_{\mathrm{C},x}$ \\
         & $y_l - p_{\mathrm{C},y}$\\ 
         & $z_l - d_{off} - p_{\mathrm{C},z}$ \\
              \hline
        \multirow{1}{*}{Angle distance} & $\Delta_\psi$ \\
         
         \hline 
        \end{tabular}
        \end{center}
        \label{tab:crane_obs_act}
    \end{table}
\fi
\subsubsection{Reward function}    
The reward function $R$ is designed to gradually guide the grapple along the actions, e.g., approaching, grasping, lifting, and balancing, to achieve the final goal. 
First, the forestry crane can grasp the wood log when the combined weighted distance 
\begin{equation}
    d_\mathrm{combine} = \norm{\bm{\Delta}_p} + \omega_1 \Delta_\psi
    \label{eq: d combine}
\end{equation}
is small enough. Consequently, the associated reward function term reads as
%on this factor is expressed as
\begin{equation}
    r_{\mathrm{distance}}  = \mathrm{exp}(-\omega_2 d_\mathrm{combine}) \:,
    \label{eq: r_distance}
\end{equation}
where $\omega_1 > 0 $ and $\omega_2 > 0$ are user-defined parameters. 
When the crane approaches the target, the RL agent is encouraged to close the grapple to hold the wood log. The reward function term for this behavior is
\begin{equation}
    r_{\mathrm{grapple}}  = r_{\mathrm{distance}}({q_{8}}/{\overline{q_8}}) + (1-{q_{8}}/{\overline{q_8}})(1-r_{\mathrm{distance}})\:,
\end{equation}
with $\overline{q_8} = \SI{3}{\radian}$ as the limit of the joint angle $q_8$. 
After holding the wood log inside the grapple, the forestry crane proceeds with the lifting action, represented by the reward function term%by giving the following reward 
\begin{equation}
    r_{\mathrm{lift}} = (1- \mathrm{tanh}(\omega_3|z_l-z_{l,d}|))(1-r_\mathrm{grapple})\:\:,
\end{equation}
where $z_{l,d}$ is the desired height of the log and $\omega_3 >0$ is a user-defined parameter. Finally, we encourage the forestry crane to stabilize after grasping the log by using
\begin{equation}
    r_{\mathrm{balance}} = (1 - \mathrm{tanh}(\norm{\dot{\mathbf{q}}_{A,d}}))(1-r_\mathrm{lift}) \:\:.
\end{equation}
Combining all parts, the overall reward function takes the form
\begin{equation}
    R = r_{\mathrm{distance}} + r_{\mathrm{grapple}} + r_{\mathrm{lift}} + r_{\mathrm{balance}} \:\:.
\end{equation}
%r_lift = 1 - np.tanh(z_log_desire_distance*4)
%distance_combine = cartersian_distance + w_angle_distance*angle_diff_distance
%r_distance_combine = np.exp(-distance_combine * w_distance)
%r_jaw_opening = (-jaw_angle / jaw_angle_max + 1) * (1 - r_distance_combine)*0.5
%r_jaw_closing = (jaw_angle / jaw_angle_max)

\subsubsection{Episode termination}
\label{sec: early termination}
Each training episode has a time limitation of $t_{max} = \SI{9}{\second}$. 
Additionally, other termination criteria are listed in the following: 
\begin{itemize}
    \item If the grapple point $\mathbf{p}_C$ is not close to the log's center point $\mathbf{p}_l$,  i.e., $d_\mathrm{combine} < \epsilon$, within $t_{limit} = \SI{6}{\second}$, the episode is early terminated. 
    \item One of the joint limits is violated. 
    \item The log is located more than \SI{8}{\meter} away from the grapple. 
    \item The velocity of the actuated joints exceeds the physical limits, i.e., $|\dot{\mathbf{q}}_A| > \dot{\mathbf{q}}_{A,max}$. 
    %This can happen when the forestry crane tries to push away the log. 
\end{itemize}
\subsection{Modified proximal policy optimization (mPPO) utilizing Beta distribution}
\label{sec: RPPO}

\begin{figure}[t]
\centering
\scalebox{0.8}{

\def\svgwidth{1\columnwidth}
\input{figures/learning_architecture-new.pdf_tex}
}
\vspace{1ex}
\caption{Overview of the learning process. $m$ randomized environments with different wood log sizes and poses are generated by our crane simulator, presented in Subsection \ref{sec: b simulator}. }
\label{fig: overview learning}
\end{figure}
The overview of the learning process is illustrated in Figure \ref{fig: overview learning}. Using the crane simulator in Mujoco, $m$ parallel environments are sampled to generate rollouts (trajectories) for training the agent. A standard architecture of an actor-critic network used in the PPO algorithm is illustrated on the right-hand side of Figure \ref{fig: overview learning}. Since the details on PPO are omitted, readers are referred to \cite{schulman2017proximal}. Only modifications of the PPO, named mPPO, are presented below. 
    
In deep RL, the policy $\bm{\pi}$ is a neural network with the parameter vector $\bm{\theta}$ that takes the state $\mathbf{s}_t$ as input and outputs the distribution of actions $\bm{\pi_{\theta}}$ modeled as Gaussian distribution in the form
\begin{equation}
    \bm{\pi_{\theta}}(\mathbf{a}_t|\mathbf{s}_t) = \dfrac{1}{\sqrt{2\pi}\bm{\sigma_\theta}}\mathrm{exp}
    \Bigg(
    \dfrac{-(\mathbf{a}_t-\bm{\mu_\theta})^2}{2\bm{\sigma_\theta}}
    \Bigg)\:.
\end{equation}
The control actions $\mathbf{a}_t$ can be sampled in the backpropagation process or the inference process as follows
\begin{equation}
    \mathbf{a}_t = \bm{\mu_\theta}(\mathbf{s}_t) + \bm{\sigma_\theta}(\mathbf{s}_t)\mathcal{N}(0,1)   \:,
\end{equation}
\corr{
where $\bm{\mu_\theta}$ and $\bm{\sigma_\theta}$ are the mean and standard deviation of the Gaussian distribution. Since the control input in $\mathbf{a}_t$ is modeled as a separate distribution, an element-wise product is used for all equations in this subsection.}

However, the control actions of the considered forestry crane are constrained in an admissible range $\underline{\mathbf{a}_t} \leq \mathbf{a}_t \leq \overline{\mathbf{a}_t}$ for safety reasons. Thus, in the mPPO algorithm, the Beta distribution is employed with the probability density function (PDF) \cite{chou2017improving}
\begin{equation}
    \bm{\pi_\theta}(\mathbf{a}_t|\mathbf{s}_t) = \bm{\mathrm{Beta}}(\mathbf{a}_{t,n};\bm{\alpha_\theta},\bm{\beta_\theta})\:\:, \bm{\alpha_\theta}>1, \:\bm{\beta_\theta} >1 \:\:,
\end{equation}
where $\mathbf{a}_{t,n} = \dfrac{\mathbf{a}_t- \overline{\mathbf{a}_t}}{\overline{\mathbf{a}_t} - \underline{\mathbf{a}_t}}$ is the normalized action and 
\begin{equation}
    \bm{\mathrm{Beta}}(\mathbf{a}_{t,n};\bm{\alpha_\theta},\bm{\beta_\theta}) = \dfrac{\bm{\Gamma}(\bm{\alpha_\theta}+\bm{\beta_\theta})}{\bm{\Gamma}(\bm{\alpha_\theta})\bm{\Gamma}(\bm{\beta_\theta})}\mathbf{a}_{t,n}^{\bm{\alpha_\theta}-1}(1-\mathbf{a}_{t,n})^{\bm{\beta_\theta}-1}\:\:. 
\end{equation}
Note that ${\Gamma}(i) = \int_0^\infty j^{i-1}\mathrm{exp}(-j)\mathrm{d}j$ is the Gamma function \cite{davis1959leonhard}. In this way, the action $\mathbf{a}_t$ is always sampled in the admissible range by using the mean of the Beta distribution 
\begin{equation}
    \bm{\mu}_{t,n} = \dfrac{\bm{\alpha_\theta}(\mathbf{s}_t)}{\bm{\alpha_\theta}(\mathbf{s}_t) + \bm{\beta_\theta}(\mathbf{s}_t)} \:\:\:. 
    \label{eq: sampling}
\end{equation}

In the PPO algorithm, the loss function consists of three parts, i.e., the surrogate loss to constrain the policy update, the error term of the value function, and the entropy term to encourage exploration, see \cite{weng2018policy}. 
In addition, in a conventional PPO algorithm, the agent shows more randomness in its actions, but the surrogate loss can constrain the updating policy during the training process. 
In a complex environment with large search areas, exploration is important for an agent like this forestry crane to complete the task. 
Inspired by Robust Policy Optimization (RPO) \cite{huang2022cleanrl}, at each step of the training process, we perturb the sampling action (\ref{eq: sampling}) by random values in the uniform distribution $\mathbf{g} \sim \mathcal{U(-\epsilon,\epsilon)}$ in the form
\begin{equation}
    \mathbf{a}_{t,n} \leftarrow \mathrm{clip}(\mathbf{a}_{t,n} + \mathbf{g},0,1) \:.
    \label{eq: robust ppo}
\end{equation}
The function $\mathrm{clip}$ limits the value of $\mathbf{a}_{t,n}$ in the range of $[0,1]$. In this work, $\epsilon$ is set to $0.1$. 

%\lipsum[1]
%\subsection{Part}
 
\section{Experiments and analysis} \label{sec:5}
This section presents comprehensive experiments on the ATR2-HUTD dataset to evaluate the effectiveness of the proposed method. 
Section~\ref{sec:4.1} outlines the experimental metrics used. 
Section~\ref{sec:4.2} details the network architecture, comparison methods, experimental setup, and parameter configurations. 
To highlight the superiority of the proposed method, Section~\ref{sec:4.3} provides both quantitative analysis and visual evaluations across all comparison methods. 
Section~\ref{sec:4.4} includes ablation studies to assess the contributions of different model components, while Section~\ref{sec:4.5} presents a parameter sensitivity analysis.
\subsection{Evaluation Indicators}\label{sec:4.1}
To quantitatively assess the performance of the proposed method, we employ three widely recognized evaluation metrics in the HTD field.
\par
\textbf{(\romannumeral1) Receiver Operating Characteristic (ROC)~\cite{ROC, ROC3D}:} 
The ROC curve offers an unbiased, threshold-independent evaluation of detection performance. This paper presents three 2D ROC curves: $( \mathrm{P}_{\mathrm{d}}, \mathrm{P}_{\mathrm{f}})$, $( \mathrm{P}_{\mathrm{d}}, \tau)$, and $( \mathrm{P}_{\mathrm{f}}, \tau)$, along with a 3D ROC curve~\cite{ROC3D} of $(\tau, \mathrm{P}_{\mathrm{d}}, \mathrm{P}_{\mathrm{f}})$ for a comprehensive performance evaluation. A detector with ROC curves closer to the upper left, upper right, and lower left corners generally exhibits superior HTD performance.
\par
\textbf{(\romannumeral2) Area Under the ROC Curve (AUC)~\cite{Zhang2015}:} 
To address challenges in visually comparing ROC curves, we compute the area under each of the three 2D ROC curves: $\text{AUC}_{( \mathrm{P}_{\mathrm{d}}, \mathrm{P}_{\mathrm{f}})}$, $\text{AUC}_{( \mathrm{P}_{\mathrm{d}}, \tau)}$, and $\text{AUC}_{( \mathrm{P}_{\mathrm{f}}, \tau)}$. Larger AUC values indicate better performance, with $\text{AUC}_{( \mathrm{P}_{\mathrm{d}}, \mathrm{P}_{\mathrm{f}})} \to 1$, $\text{AUC}_{( \mathrm{P}_{\mathrm{d}}, \tau)} \to 1$, and $\text{AUC}_{( \mathrm{P}_{\mathrm{f}}, \tau)} \to 0$ signifying superior detection performance. Additionally, two AUC-based metrics are introduced for a more comprehensive evaluation:
\begin{equation}
    \mathrm{AUC}_{\mathrm{OA}} = \mathrm{AUC}_{\left(P_f, P_d\right)} + \mathrm{AUC}_{\left(\tau, P_d\right)} - \mathrm{AUC}_{\left(\tau, P_f\right)},
\end{equation}
\begin{equation}
    \mathrm{AUC}_{\mathrm{SNPR}} = \frac{\mathrm{AUC}_{\left(\tau, P_d\right)}}{\mathrm{AUC}_{\left(\tau, P_f\right)}},
\end{equation}
where higher values of $\mathrm{AUC}_{\mathrm{OA}} \to 2$ and $\mathrm{AUC}_{\mathrm{SNPR}} \to +\infty$ indicate improved detector performance.
% \textbf{(\romannumeral3) Separability Map~\cite{Liu2022}:} The degree of separation between the targets and backgrounds in the detection map is a critical performance indicator for UTD methods. 
% Thus, we also utilize the separability map for quantitative comparison in this study. 
% Specifically, the separability map uses green and blue boxes to represent the statistics of the target and background, respectively. 
% The horizontal line within each box indicates the median value, while the upper and lower whiskers denote the maximum and minimum values, providing a clear representation of the data range and central tendency. 
% \par
% A larger overlap between the two boxes suggests that the statistics of the target and background are similar, indicating poor separation between them. 
% Conversely, less overlap indicates better separation. 
% Moreover, background suppression is considered more effective when the blue box is closer to the ordinate 0, while higher target prominence is indicated when the green box is closer to ordinate 1.
% \clearpage
\subsection{Experimental Details and Settings}\label{sec:4.2}
\textbf{(\romannumeral1) Experimental Details:} 
The experimental setup and details of the proposed method are as follows. Unless otherwise specified, the parameters are applied consistently across all sub-datasets. The method consists of three core components: the RGC module, the HLCL module, and the SPL strategy, each contributing significantly to performance.

In the RGC module, unsupervised clustering is performed using the K-Means~\cite{Sinaga2020} algorithm, with cluster numbers set to 36, 39, and 42 for the lake, river, and sea sub-datasets, respectively, based on environmental complexity and waterbed characteristics.

The HLCL module employs the 3D-ResNet50~\cite{Jiang2019} network for spectral-spatial feature extraction. To enhance robustness and contrastive learning, untargeted FGSM~\cite{GoodfellowSS14} data augmentation is applied with a maximum perturbation of $\epsilon=0.1$ under the $l_{\infty}$ norm. The hybrid-level contrastive learning framework is trained for 50 epochs per SPL iteration. The Adam optimizer is used with a batch size of 256. The initial learning rate is $5\times10^{-3}$, decaying to $5\times10^{-5}$ through a cosine annealing schedule after 100 epochs, and a weight decay of $1\times10^{-4}$ is applied to reduce overfitting.

The SPL strategy is executed for 10 iterations across all sub-datasets to ensure convergence and computational efficiency.

For HUTD, as described in Section~\ref{sec3.4}, we use learned representations combined with basic hyperspectral detectors. To isolate the effect of detectors on performance, we employ two classic detectors, CEM~\cite{KRUSE1993145} and SAM~\cite{Manolakis2002}, as baseline methods.

\textbf{(\romannumeral2) Experimental Settings:} 
We compare the proposed method against several state-of-the-art (SOTA) HTD and HUTD methods, including two traditional HTD detectors (CEM and SAM), two advanced HTD methods (IEEPST~\cite{IEEPST} and MCLT~\cite{Wang2024}), and four HUTD methods (UTD-Net~\cite{Qi2021}, TUTDF~\cite{LiZheyong2023}, TDSS-UTD~\cite{Li2023}, and NUN-UTD~\cite{Liu2024}).

To ensure fairness, each method is executed with the original hyperparameter settings as specified in their respective publications. All experiments are conducted on a machine equipped with seven NVIDIA A6000 GPUs, an AMD Ryzen 5995WX CPU, and 128 GB of RAM, running Ubuntu 22.04.

\subsection{Main Results} \label{sec:4.3}
\textbf{(\romannumeral1) Detection Maps:} Figs.~\ref{fig:C1-1} to~\ref{fig:C1-2} present detection maps from the ATR2-HUTD-Lake sub-dataset, offering a qualitative comparison of the evaluated methods.
The detection maps of other sub-datasets are provided in the supplementary material.
\par
\begin{figure*}[!t]                 
    \centering                    
    \includegraphics[width=2\columnwidth]{images/C1-1.jpg}                     
    \caption{Detection maps of ATR2-HUTD Lake Scene1. (a) Pseudo-color image. (b) Ground truth. (c) CEM. (d) SAM. (e) IEEPST. (f) MCLT. (g) UTD-Net. (h) TUTDF. (i) TDSS-UTD. (j) NUN-UTD. (m) HUCLNet+CEM. (n) HUCLNet+SAM.}                  
    \label{fig:C1-1}    
\end{figure*}
\begin{figure*}[!t]                 
    \centering                    
    \includegraphics[width=2\columnwidth]{images/C1-2.jpg}                     
    \caption{Detection maps of ATR2-HUTD Lake Scene2. (a) Pseudo-color image. (b) Ground truth. (c) CEM. (d) SAM. (e) IEEPST. (f) MCLT. (g) UTD-Net. (h) TUTDF. (i) TDSS-UTD. (j) NUN-UTD. (m) HUCLNet+CEM. (n) HUCLNet+SAM.}                    
    \label{fig:C1-2}    
\end{figure*}
% \begin{figure*}[!t]                 
%     \centering                    
%     \includegraphics[width=2\columnwidth]{images/C1-3.jpg}                     
%     \caption{Detection maps of ATR2-HUTD River Scene1. (a) Pseudo-color image. (b) Ground truth. (c) CEM. (d) SAM. (e) IEEPST. (f) MCLT. (g) UTD-Net. (h) TUTDF. (i) TDSS-UTD. (j) NUN-UTD. (m) HUCLNet+CEM. (n) HUCLNet+SAM.}                      
%     \label{fig:C1-3}    
% \end{figure*}
% \begin{figure*}[!t]                 
%     \centering                    
%     \includegraphics[width=2\columnwidth]{images/C1-4.jpg}                     
%     \caption{Detection maps of ATR2-HUTD River Scene2. (a) Pseudo-color image. (b) Ground truth. (c) CEM. (d) SAM. (e) IEEPST. (f) MCLT. (g) UTD-Net. (h) TUTDF. (i) TDSS-UTD. (j) NUN-UTD. (m) HUCLNet+CEM. (n) HUCLNet+SAM.}                     
%     \label{fig:C1-4}    
% \end{figure*}
% \begin{figure*}[!t]                 
%     \centering                    
%     \includegraphics[width=2\columnwidth]{images/C1-5.jpg}                     
%     \caption{Detection maps of ATR2-HUTD Sea Scene1. (a) Pseudo-color image. (b) Ground truth. (c) CEM. (d) SAM. (e) IEEPST. (f) MCLT. (g) UTD-Net. (h) TUTDF. (i) TDSS-UTD. (j) NUN-UTD. (m) HUCLNet+CEM. (n) HUCLNet+SAM.}                 
%     \label{fig:C1-5}    
% \end{figure*}
% \begin{figure*}[!t]                 
%     \centering                    
%     \includegraphics[width=2\columnwidth]{images/C1-6.jpg}                     
%     \caption{Detection maps of ATR2-HUTD Sea Scene2. (a) Pseudo-color image. (b) Ground truth. (c) CEM. (d) SAM. (e) IEEPST. (f) MCLT. (g) UTD-Net. (h) TUTDF. (i) TDSS-UTD. (j) NUN-UTD. (m) HUCLNet+CEM. (n) HUCLNet+SAM.}                    
%     \label{fig:C1-2}    
% \end{figure*}
Traditional methods, such as CEM and SAM, exhibit significant limitations in underwater environments. CEM struggles with background noise suppression, resulting in false positives, while SAM fails to delineate target boundaries and often misses targets, especially in complex scenarios like the ATR2-HUTD River dataset. Its sensitivity to spectral noise and limited adaptability to spectral variations lead to incomplete detection and poor target-background separation.
\par
\begin{figure*}[!t]                 
    \centering                    
    \includegraphics[width=2\columnwidth]{images/C2-1.jpg}                     
    \caption{ROC curves comparison on ATR2-HUTD Lake Scene1. (a) 3-D ROC curve. (b) 2-D ROC curve of $(P_d, P_f)$. (c) 2-D ROC curve of $(P_f, \tau)$. (d) 2-D ROC curve of $(P_d, \tau)$.}                 
    \label{fig:C2-1}    
\end{figure*}
\begin{figure*}[!t]                 
    \centering                    
    \includegraphics[width=2\columnwidth]{images/C2-2.jpg}                     
    \caption{ROC curves comparison on ATR2-HUTD Lake Scene2. (a) 3-D ROC curve. (b) 2-D ROC curve of $(P_d, P_f)$. (c) 2-D ROC curve of $(P_f, \tau)$. (d) 2-D ROC curve of $(P_d, \tau)$.}                                  
    \label{fig:C2-2}    
\end{figure*}
Advanced land-cover detection methods, including IEEPST and MCLT, also underperform in underwater environments. IEEPST struggles to suppress background interference, particularly when water column spectral signatures overlap with target signatures in the ATR2-HUTD River sub-dataset. While MCLT leverages contrastive learning for feature enhancement, it shows reduced sensitivity to small or low-reflectance targets, hindered by the nonlinearities and spectral noise typical of underwater HSI data. These results underscore the necessity of specialized techniques for HUTD.

Among SOTA HUTD methods, UTD-Net demonstrates notable improvements by effectively unmixing target-water mixed pixels. However, it faces challenges with background interference in scenes with extensive non-target bottom areas, leading to high false positive rates. NUN-UTD improves target identification by preserving weak target spectral signals, yet remains susceptible to background interference when spectral characteristics of the background resemble those of the target, leading to false positives in spectrally overlapping environments.

Physical-based methods, such as TUTDF and TDSS-UTD, enhance background suppression using underwater imaging models and predicted depth values. However, TUTDF's performance declines in complex environments due to depth estimation inaccuracies, leading to inconsistent detection. Similarly, TDSS-UTD struggles in environments with substantial depth variation, such as the ATR2-HUTD River dataset, where depth errors degrade detection accuracy. Variations in underwater imaging mechanisms between deep and nearshore scenes further limit their effectiveness.

In contrast, HUCLNet-based methods consistently outperform the alternatives. By integrating instance-level and prototype-level contrastive learning, these methods effectively detect faint and deeply submerged targets with minimal false positives, enhancing background suppression and detection accuracy. HUCLNet+CEM and HUCLNet+SAM show resilience to spectral variability, capturing subtle target features while maintaining clear target-background separation, even under significant underwater bottom interference. These methods provide the most comprehensive target coverage and background suppression in challenging environments, such as the ATR2-HUTD River dataset, demonstrating the superior effectiveness of HUCLNet in mitigating spectral variability and improving detection accuracy.
\par 
\textbf{(\romannumeral2) ROC Curves:} Subjective analysis of detection maps may be insufficient for comprehensive evaluation. Therefore, 3-D ROC curves and their 2-D projections: ($P_d$, $P_f$), ($P_d$, $\tau$), and ($P_f$, $\tau$) were used to objectively assess detection performance on the ATR2-HUTD dataset, enabling a detailed evaluation of detection efficiency, target preservation, and background suppression. 
The ROC curves of ATR-HUTD-Lake sub-dataset are provided in Figs~\ref{fig:C2-1} to~\ref{fig:C2-2}, while those of the ATR-HUTD-River and ATR-HUTD-Sea sub-datasets are provided in the supplementary material.
\par
Figs.~\ref{fig:C2-1} (a) to~\ref{fig:C2-2} (a) show the 3-D ROC curves, highlighting the relationship between the true positive rate ($P_d$), false alarm probability ($P_f$), and detection threshold ($\tau$). HUCLNet+CEM and HUCLNet+SAM consistently outperform other methods, exhibiting higher $P_d$ and lower $P_f$ over a wide range of $\tau$, demonstrating superior adaptability.
\par
% \begin{figure*}[!t]                 
%     \centering                    
%     \includegraphics[width=2\columnwidth]{images/C2-3.jpg}                     
%     \caption{ROC curves comparison on ATR2-HUTD River Scene1. (a) 3-D ROC curve. (b) 2-D ROC curve of $(P_d, P_f)$. (c) 2-D ROC curve of $(P_f, \tau)$. (d) 2-D ROC curve of $(P_d, \tau)$.}                                   
%     \label{fig:C2-3}    
% \end{figure*}
% \begin{figure*}[!t]                 
%     \centering                    
%     \includegraphics[width=2\columnwidth]{images/C2-4.jpg}                     
%     \caption{ROC curves comparison on ATR2-HUTD River Scene2. (a) 3-D ROC curve. (b) 2-D ROC curve of $(P_d, P_f)$. (c) 2-D ROC curve of $(P_f, \tau)$. (d) 2-D ROC curve of $(P_d, \tau)$.}                                 
%     \label{fig:C2-4}    
% \end{figure*}
% \begin{figure*}[!t]                 
%     \centering                    
%     \includegraphics[width=2\columnwidth]{images/C2-5.jpg}                     
%     \caption{ROC curves comparison on ATR2-HUTD Sea Scene1. (a) 3-D ROC curve. (b) 2-D ROC curve of $(P_d, P_f)$. (c) 2-D ROC curve of $(P_f, \tau)$. (d) 2-D ROC curve of $(P_d, \tau)$.}                                   
%     \label{fig:C2-5}    
% \end{figure*}
% \begin{figure*}[!t]                 
%     \centering                    
%     \includegraphics[width=2\columnwidth]{images/C2-6.jpg}                     
%     \caption{ROC curves comparison on ATR2-HUTD Sea Scene2. (a) 3-D ROC curve. (b) 2-D ROC curve of $(P_d, P_f)$. (c) 2-D ROC curve of $(P_f, \tau)$. (d) 2-D ROC curve of $(P_d, \tau)$.}                                
%     \label{fig:C2-2}    
% \end{figure*}
Figs.~\ref{fig:C2-1} (b) to~\ref{fig:C2-2} (b) present the 2-D ROC curves of ($P_d$, $P_f$). HUCLNet-based methods occupy the top-left region, indicating superior detection accuracy. In contrast, traditional HTD methods, such as CEM and SAM, struggle to balance $P_d$ and $P_f$, particularly for targets with varying spectral properties. Although advanced HTD and SOTA HUTD methods show moderate performance, they fail to suppress false alarms in complex river environments, compromising detection accuracy.

Figs.~\ref{fig:C2-1}(c) to~\ref{fig:C2-2}(c) depict the 2-D ROC curves of ($P_f$, $\tau$), assessing background suppression. NUN-UTD shows high $P_f$ across thresholds, indicating poor background-target discrimination. While methods like MCLT and TUTDF show some improvement, they still struggle with high false alarm rates due to spectral overlap. \textbf{UTD-Net performs well in background suppression but largely by classifying all pixels as background}, as reflected in detection maps (Figs.~\ref{fig:C1-1} to~\ref{fig:C1-2}) and AUC$_{P_{d}, \tau}$ values (Tabs.~\ref{auc_lake} to~\ref{auc_sea}). In comparison, HUCLNet+CEM and HUCLNet+SAM exhibit superior background suppression with low $P_f$ and high AUC$_{P_{d}, \tau}$ values.

Figs.~\ref{fig:C2-1}(d) to~\ref{fig:C2-2}(d) present the 2-D ROC curves of ($P_d$, $\tau$), evaluating target preservation. Traditional methods, such as SAM, show significant drops in $P_d$ as $\tau$ increases, indicating poor target preservation. Advanced HTD and SOTA HUTD methods, such as MCLT and TDSS-UTD, show some improvement but still lag behind NUN-UTD and TUTDF. However, \textbf{the improved performance of NUN-UTD and TUTDF primarily results from misclassifying all pixels as targets}, as shown by high false alarm rates in detection maps (Figs.~\ref{fig:C1-1} to~\ref{fig:C1-2}) and increased AUC$_{P_{f}, \tau}$ values. In contrast, HUCLNet+CEM and HUCLNet+SAM maintain high $P_d$ at lower $\tau$, demonstrating robust and reliable target preservation.
\par
\begin{table*}[!t] 
    \centering
    \footnotesize   
    \caption{Quantitative comparison results on the ATR2-HUTD-Lake Sub-dataset. The best and second best results are in \textbf{bold} and with \underline{underline}.} \label{auc_lake}
    \renewcommand{\arraystretch}{1.5}
    \setlength{\tabcolsep}{1.85mm}
    \scalebox{0.875}
    {
        \begin{tabular}{ccccccccccc}
            \hline
            \multirow{2.4}{*}{\textbf{Method}} & \multicolumn{5}{c}{\cellcolor{tablecolor7!60}\textbf{ATR2-HUTD-Lake Scene1}}       & \multicolumn{5}{c}{\cellcolor{tablecolor8}\textbf{ATR2-HUTD-Lake Scene2}}       \\ \cmidrule(lr){2-6} \cmidrule(lr){7-11}
                                    & $\text{AUC}_{( \mathrm{P}_{\mathrm{d}},\mathrm{P}_{\mathrm{f}})}\textcolor{red}{\uparrow }$ & $\text{AUC}_{( \mathrm{P}_{\mathrm{f}}, \tau)}\textcolor{green}{\downarrow }$ & $\text{AUC}_{( \mathrm{P}_{\mathrm{d}},\tau)}\textcolor{red}{\uparrow }$ & $\mathrm{AUC}_{\mathrm{OA}} \textcolor{red}{\uparrow }$ & $\mathrm{AUC}_{\mathrm{SNPR}}\textcolor{red}{\uparrow }$ & $\text{AUC}_{( \mathrm{P}_{\mathrm{d}},\mathrm{P}_{\mathrm{f}})}\textcolor{red}{\uparrow }$ & $\text{AUC}_{( \mathrm{P}_{\mathrm{f}}, \tau)}\textcolor{green}{\downarrow }$ & $\text{AUC}_{( \mathrm{P}_{\mathrm{d}},\tau)}\textcolor{red}{\uparrow }$ & $\mathrm{AUC}_{\mathrm{OA}} \textcolor{red}{\uparrow }$ & $\mathrm{AUC}_{\mathrm{SNPR}}\textcolor{red}{\uparrow }$ \\ \hline
                                    CEM         & 0.671          & 0.250          & 0.258          & 0.678          & 1.028          & 0.489          & 0.524          & 0.520          & 0.485          & 0.994          \\
                                    SAM         & 0.670          & \underline{0.129}    & 0.151          & 0.692          & 1.170          & 0.480          & \underline{0.143}    & 0.025          & 0.362          & 0.172          \\
                                    IEEPST      & 0.424          & 0.204          & 0.075          & 0.295          & 0.369          & 0.417          & 0.187          & 0.036          & 0.266          & 0.193          \\
                                    MCLT        & 0.401          & 0.422          & 0.377          & 0.357          & 0.894          & 0.365          & 0.243          & 0.173          & 0.296          & 0.715          \\
                                    UTD-Net     & 0.846          & \textbf{0.013} & 0.019          & 0.853          & 1.510          & 0.944          & \textbf{0.041} & 0.073          & 0.976          & 1.773          \\
                                    TUTDF       & \underline{0.990}          & 0.634          & \underline{0.726}    & 1.081          & 1.145          & \underline{0.998}          & 0.461          & \underline{0.768}    & 1.306          & 1.667          \\
                                    TDSS-UTD    & 0.964          & 0.215          & 0.369          & 1.117          & 1.712          & \textbf{0.999} & 0.166          & 0.444          & 1.277          & 2.676          \\
                                    NUN-UTD     & 0.758          & 0.913          & \textbf{0.994} & 0.838          & 1.088          & 0.765          & 0.792          & \textbf{0.995} & 0.968          & 1.257          \\
            \rowcolor{tablecolor13!60}HUCLNet+CEM & 0.958    & 0.302          & 0.642          & \underline{1.298}    & \underline{2.126}    & 0.989    & 0.226          & 0.634          & \underline{1.397}    & \underline{2.805}    \\
            \rowcolor{tablecolor14!60}HUCLNet+SAM & \textbf{0.995} & 0.209          & 0.710          & \textbf{1.501} & \textbf{3.393} & \textbf{0.999} & 0.265          & 0.765          & \textbf{1.501} & \textbf{2.891} \\ \hline
        \end{tabular}}
\end{table*}
\begin{table*}[!t] 
    \centering
    \footnotesize   
    \caption{Quantitative comparison results on the ATR2-HUTD-River Sub-dataset. The best and second best results are in \textbf{bold} and with \underline{underline}.} \label{auc_river}
    \renewcommand{\arraystretch}{1.5}
    \setlength{\tabcolsep}{1.85mm}
    \scalebox{0.875}
    {
        \begin{tabular}{ccccccccccc}
            \hline
            \multirow{2.4}{*}{\textbf{Method}} & \multicolumn{5}{c}{\cellcolor{tablecolor9}\textbf{ATR2-HUTD-River Scene1}}       & \multicolumn{5}{c}{\cellcolor{tablecolor10}\textbf{ATR2-HUTD-River Scene2}}       \\ \cmidrule(lr){2-6} \cmidrule(lr){7-11}
                                    & $\text{AUC}_{( \mathrm{P}_{\mathrm{d}},\mathrm{P}_{\mathrm{f}})}\textcolor{red}{\uparrow }$ & $\text{AUC}_{( \mathrm{P}_{\mathrm{f}}, \tau)}\textcolor{green}{\downarrow }$ & $\text{AUC}_{( \mathrm{P}_{\mathrm{d}},\tau)}\textcolor{red}{\uparrow }$ & $\mathrm{AUC}_{\mathrm{OA}} \textcolor{red}{\uparrow }$ & $\mathrm{AUC}_{\mathrm{SNPR}}\textcolor{red}{\uparrow }$ & $\text{AUC}_{( \mathrm{P}_{\mathrm{d}},\mathrm{P}_{\mathrm{f}})}\textcolor{red}{\uparrow }$ & $\text{AUC}_{( \mathrm{P}_{\mathrm{f}}, \tau)}\textcolor{green}{\downarrow }$ & $\text{AUC}_{( \mathrm{P}_{\mathrm{d}},\tau)}\textcolor{red}{\uparrow }$ & $\mathrm{AUC}_{\mathrm{OA}} \textcolor{red}{\uparrow }$ & $\mathrm{AUC}_{\mathrm{SNPR}}\textcolor{red}{\uparrow }$ \\ \hline
                                    CEM         & 0.746          & 0.280          & 0.300          & 0.765          & 1.070          & 0.650          & 0.544          & 0.553          & 0.659          & 1.016          \\
                                    SAM         & 0.657          & 0.214          & 0.186          & 0.629          & 0.871          & 0.656          & \underline{0.078}    & 0.066          & 0.645          & 0.854          \\
                                    IEEPST      & 0.455          & 0.203          & 0.033          & 0.286          & 0.163          & 0.594          & 0.274          & 0.236          & 0.556          & 0.861          \\
                                    MCLT        & 0.550          & 0.989          & 0.990          & 0.552          & 1.001          & 0.531          & 0.970          & \underline{0.971}    & 0.533          & 1.002          \\
                                    UTD-Net     & \underline{0.843}          & \underline{0.080}    & 0.096          & 0.860          & 1.209          & \underline{0.889}          & \textbf{0.075} & 0.088          & \underline{0.903}          & 1.176          \\
                                    TUTDF       & 0.568          & 0.822          & \underline{0.824}    & 0.570          & 1.003          & 0.659          & 0.356          & 0.363          & 0.667          & 1.022          \\
                                    TDSS-UTD    & 0.402          & 0.438          & 0.415          & 0.379          & 0.948          & 0.539 & 0.179          & 0.174          & 0.534          & 0.974          \\
                                    NUN-UTD     & 0.632          & 0.968          & \textbf{0.999} & 0.663          & 1.032          & 0.503          & 0.977          & \textbf{0.980} & 0.505          & 1.002          \\
            \rowcolor{tablecolor13!60}HUCLNet+CEM & 0.794    & 0.353          & 0.518          & \underline{0.959}    & \underline{1.468}    & 0.753    & 0.354          & 0.481          & 0.880    & \underline{1.360}    \\
            \rowcolor{tablecolor14!60}HUCLNet+SAM & \textbf{0.966} & \textbf{0.055} & 0.175          & \textbf{1.086} & \textbf{3.206} & \textbf{0.924} & 0.178          & 0.327          & \textbf{1.073} & \textbf{1.837} \\ \hline
        \end{tabular}}
\end{table*}
\begin{table*}[!t] 
    \centering
    \footnotesize   
    \caption{Quantitative comparison results on the ATR2-HUTD-Sea Sub-dataset. The best and second best results are in \textbf{bold} and with \underline{underline}.} \label{auc_sea}
    \renewcommand{\arraystretch}{1.5}
    \setlength{\tabcolsep}{1.85mm}
    \scalebox{0.875}
    {
        \begin{tabular}{ccccccccccc}
            \hline
            \multirow{2.4}{*}{\textbf{Method}} & \multicolumn{5}{c}{\cellcolor{tablecolor11}\textbf{ATR2-HUTD-Sea Scene1}}       & \multicolumn{5}{c}{\cellcolor{tablecolor12!50}\textbf{ATR2-HUTD-Sea Scene2}}       \\ \cmidrule(lr){2-6} \cmidrule(lr){7-11}
                                    & $\text{AUC}_{( \mathrm{P}_{\mathrm{d}},\mathrm{P}_{\mathrm{f}})}\textcolor{red}{\uparrow }$ & $\text{AUC}_{( \mathrm{P}_{\mathrm{f}}, \tau)}\textcolor{green}{\downarrow }$ & $\text{AUC}_{( \mathrm{P}_{\mathrm{d}},\tau)}\textcolor{red}{\uparrow }$ & $\mathrm{AUC}_{\mathrm{OA}} \textcolor{red}{\uparrow }$ & $\mathrm{AUC}_{\mathrm{SNPR}}\textcolor{red}{\uparrow }$ & $\text{AUC}_{( \mathrm{P}_{\mathrm{d}},\mathrm{P}_{\mathrm{f}})}\textcolor{red}{\uparrow }$ & $\text{AUC}_{( \mathrm{P}_{\mathrm{f}}, \tau)}\textcolor{green}{\downarrow }$ & $\text{AUC}_{( \mathrm{P}_{\mathrm{d}},\tau)}\textcolor{red}{\uparrow }$ & $\mathrm{AUC}_{\mathrm{OA}} \textcolor{red}{\uparrow }$ & $\mathrm{AUC}_{\mathrm{SNPR}}\textcolor{red}{\uparrow }$ \\ \hline
                                    CEM         & 0.805          & 0.309          & 0.349          & 0.845          & 1.128           & 0.845          & 0.332          & 0.351          & 0.864          & 1.057          \\
                                    SAM         & 0.866          & 0.125    & 0.188          & 0.929          & 1.503           & 0.819          & 0.099          & 0.033          & 0.753          & 0.333          \\
                                    IEEPST      & 0.850          & 0.252          & 0.363          & 0.961          & 1.441           & 0.580          & 0.326          & 0.269          & 0.523          & 0.826          \\
                                    MCLT        & 0.895          & 0.980          & \underline{0.994} & 0.909          & 1.014           & 0.317          & 0.953          & \underline{0.944}    & 0.309          & 0.991          \\
                                    UTD-Net     & 0.762          & \underline{0.050}          & 0.083          & 0.796          & 1.682           & 0.774          & \textbf{0.043} & 0.070          & 0.801          & 1.634          \\
                                    TUTDF       & 0.952          & 0.841          & 0.872          & 0.984          & 1.037           & 0.903          & 0.426          & 0.482          & 0.959          & 1.131          \\
                                    TDSS-UTD    & 0.861          & 0.310          & 0.371          & 0.923          & 1.199           & 0.984          & 0.218          & 0.425          & 1.192          & 1.948          \\
                                    NUN-UTD     & \underline{0.979}    & 0.534          & \textbf{0.999} & \textbf{1.445} & 1.872           & 0.975          & 0.959          & \textbf{0.984} & 0.999          & 1.025          \\
            \rowcolor{tablecolor13!60}HUCLNet+CEM & 0.972          & 0.133          & 0.569          & \underline{1.409}    & \underline{4.284}     & \underline{0.987}    & 0.111          & 0.401          & \underline{1.287}    & \underline{3.620}    \\
            \rowcolor{tablecolor14!60}HUCLNet+SAM & \textbf{0.985} & \textbf{0.019} & 0.325          & 1.292          & \textbf{17.501} & \textbf{0.989} & \underline{0.053}    & 0.474          & \textbf{1.420} & \textbf{8.857} \\ \hline
        \end{tabular}}
\end{table*}
\textbf{(\romannumeral3) AUC Values:} The AUC values for each sub-dataset of the ATR2-HUTD dataset are computed using five key metrics: $\text{AUC}_{( \mathrm{P}_{\mathrm{d}}, \mathrm{P}_{\mathrm{f}})}$, $\text{AUC}_{( \mathrm{P}_{\mathrm{d}}, \tau)}$, $\text{AUC}_{( \mathrm{P}_{\mathrm{f}}, \tau)}$, $\text{AUC}_{SNPR}$, and $\text{AUC}_{OA}$, as detailed in Tabs.~\ref{auc_lake} to~\ref{auc_sea}. These metrics quantitatively assess detection accuracy, target preservation, background suppression, signal-to-noise ratio, and overall performance in varied underwater environments.
\par
\begin{table*}[!ht] 
    \centering
    \footnotesize   
    \caption{Quantitative results of ablation studies on the ATR2-HUTD dataset.} \label{ablation study}
    \renewcommand{\arraystretch}{2}
    \setlength{\tabcolsep}{2.5mm}
    \begin{threeparttable}
        \scalebox{0.975}
        { 
    \begin{tabular}{ccccccc}
        \hline
        \textbf{Module Name}                  & \textbf{Design}                                                      & $\text{AUC}_{( \mathrm{P}_{\mathrm{d}},\mathrm{P}_{\mathrm{f}})}\textcolor{red}{\uparrow }$ & $\text{AUC}_{( \mathrm{P}_{\mathrm{f}}, \tau)}\textcolor{green}{\downarrow }$ & $\text{AUC}_{( \mathrm{P}_{\mathrm{d}},\tau)}\textcolor{red}{\uparrow }$ & $\mathrm{AUC}_{\mathrm{OA}} \textcolor{red}{\uparrow }$ & $\mathrm{AUC}_{\mathrm{SNPR}}\textcolor{red}{\uparrow }$  \\ \hline
        \rowcolor{tablecolor0!50}
        \textbf{HUCLNet}                                          & N/A & 0.943 & 0.188 & 0.502 & 1.258 & 4.446 \\
        \rowcolor{tablecolor1!50}
        \cellcolor{tablecolor1!50}                             & w/o Cluster Refinement Strategy                             & 0.823 & 0.206 & 0.388 & 1.005 & 3.141 \\
        \rowcolor{tablecolor1!50}
        \multirow{-2}{*}{\cellcolor{tablecolor1!50}\textbf{RGC module}}  & w/o Reference Spectrum based Clustering Method & 0.737 & 0.211 & 0.375 & 0.901 & 2.616 \\
        \rowcolor{tablecolor2!50} 
        \cellcolor{tablecolor2!50}                              & w/o Instance-level Contrastive Learning                     & 0.878 & 0.199 & 0.438 & 1.117 & 3.513 \\
        \rowcolor{tablecolor2!50} 
        \cellcolor{tablecolor2!50}                              & w/o Prototype-level Contrastive Learning                    & 0.728 & 0.239 & 0.359 & 0.848 & 2.359 \\
        \rowcolor{tablecolor2!50}
        \cellcolor{tablecolor2!50}                              & w/o Hyperspectral-Oriented Data Augmentation                    & 0.883 & 0.195 & 0.452 & 1.165 & 3.584 \\
        \rowcolor{tablecolor2!50} 
        \multirow{-4}{*}{\cellcolor{tablecolor2!50}\textbf{HLCL module}} & w/o HLCL module$^{1}$                                             & 0.696 & 0.252 & 0.248 & 0.692 & 0.933 \\
        \rowcolor{tablecolor3!50} 
        \textbf{SPL Paradigm}                                          & w/o SPL Paradigm                                            & 0.743 & 0.217 & 0.388 & 0.914 & 2.864 \\ \hline
        \end{tabular}}
        \begin{tablenotes}
            \scriptsize
            \item[1] This experimental design is analogous to the baseline HTD methods, as the RGC module and SPL paradigm are dependent on the HLCL module for functionality.
        \end{tablenotes}
        \end{threeparttable}
\end{table*}
The $\text{AUC}_{( \mathrm{P}_{\mathrm{d}}, \mathrm{P}_{\mathrm{f}})}$ metric, which quantifies the trade-off between the true positive rate ($P_d$) and false alarm probability ($P_f$), is critical for evaluating detection performance. HUCLNet+SAM leads with an average score of 0.976, followed by HUCLNet+CEM at 0.909. Traditional methods, such as SAM (0.701) and MCLT (0.691), underperform significantly, while SOTA HUTD methods like TUTDF and NUN-UTD fall short of HUCLNet-based methods in detection capability.
\par
For background suppression, assessed by $\text{AUC}_{( \mathrm{P}_{\mathrm{f}}, \tau)}$, HUCLNet+SAM achieves the highest performance in the ATR2-HUTD-River Scene1 and ATR2-HUTD-Sea sub-datasets, the most complex nearshore environments. It also demonstrates robust performance across other sub-datasets. In contrast, SOTA HUTD methods, including TUTDF and NUN-UTD, show elevated values, suggesting overfitting due to high false positive rates.
\par
The $\text{AUC}_{( \mathrm{P}_{\mathrm{d}}, \tau)}$ metric, assessing target preservation, reveals HUCLNet-based methods performing well, though NUN-UTD leads. This may be attributed to the HLCL module in HUCLNet, which compromises target-background feature separation, impacting target preservation. Additionally, NUN-UTD's higher false positive rate boosts $P_d$ but hinders background suppression.
\par
The $\text{AUC}_{OA}$ metric, combining $\text{AUC}_{( \mathrm{P}_{\mathrm{d}}, \mathrm{P}_{\mathrm{f}})}$, $\text{AUC}_{( \mathrm{P}_{\mathrm{d}}, \tau)}$, and $\text{AUC}_{( \mathrm{P}_{\mathrm{f}}, \tau)}$, further emphasizes HUCLNet's superiority. HUCLNet+SAM achieves the highest average score of 1.312, with HUCLNet+CEM following at 1.205. In contrast, traditional and SOTA HUTD methods score between 0.492 and 0.928, underscoring HUCLNet's effectiveness in background suppression, target preservation, and detection accuracy in complex nearshore environments.
\par
Finally, the $\text{AUC}_{SNPR}$ metric, which measures robustness under varying signal-to-noise ratios, underscores HUCLNet+SAM's superior performance, achieving the highest scores across all sub-datasets, including 17.501 in ATR2-HUTD-Sea Scene1. HUCLNet+CEM consistently ranks second, while traditional HTD and SOTA HUTD methods show lower scores, indicating reduced robustness in fluctuating signal conditions.
\par
% \begin{figure*}[!t]                 
%     \centering                    
%     \includegraphics[scale=0.65]{images/C3-1.jpg}                     
%     \caption{Target-background separability boxplots for ATR2-HUTD Lake sub-dataset. (a) Scene1. (b) Scene2.}                    
%     \label{fig:C3-1}    
% \end{figure*}


% \begin{figure*}[!t]                 
%     \centering                    
%     \includegraphics[scale=0.65]{images/C3-2.jpg}                     
%     \caption{Target-background separability boxplots for ATR2-HUTD River sub-dataset. (a) Scene1. (b) Scene2.}                                 
%     \label{fig:C3-2}    
% \end{figure*}
% \begin{figure*}[!t]                 
%     \centering                    
%     \includegraphics[scale=0.65]{images/C3-3.jpg}                     
%     \caption{Target-background separability boxplots for ATR2-HUTD Sea sub-dataset. (a) Scene1. (b) Scene2.}                                      
%     \label{fig:C3-3}    
% \end{figure*}
% \textbf{(\romannumeral4) Separability Maps:} To assess the effectiveness of the comparison methods in distinguishing targets from the background, target-background separability is analyzed using boxplots, providing a clear visual representation of the separation. Figs.~\ref{fig:C3-1} to \ref{fig:C3-3} present these separability boxplots for all methods across the ATR2-HUTD sub-datasets.
% \par
% Traditional HTD methods, CEM and SAM, show limited separability, with SAM slightly outperforming CEM. In all sub-datasets, target boxes overlap with background boxes, despite some background suppression, indicating poor separation of targets from the background in underwater environments.
% Advanced HTD methods, IEEPST and MCLT, show minor improvement over traditional methods. 
% However, except for the ATR2-HUTD Sea sub-dataset (scene 1), target boxes still overlap with background boxes in most sub-datasets. 
% This suggests that even with advanced techniques, suppressing background noise and achieving clear target separation remains challenging in complex underwater environments.
% \par
% HUTD methods show improved separability. Specifically, UTD-Net achieves significant background suppression, though some overlap remains. 
% Additionally, UTD-Net exhibits a detection range near 0 in certain sub-datasets, indicating a high false positive rate. 
% NUN-UTD, an enhanced version of UTD-Net, improves target highlighting but still struggles with background noise suppression, leading to suboptimal performance in more complex scenes such as those in the ATR2-HUTD River and Sea sub-datasets.
% Compared to unmixing-based HUTD methods, TUTDF and TDSS-UTD demonstrate better separability, with detection ranges closer to 1, indicating more effective reduction of target-background correlation. 
% However, both methods still exhibit considerable target-background overlap and limited suppression in complex scenes, such as the ATR2-HUTD River sub-dataset.
% \par
% In contrast, the proposed HUCLNet-based methods, HUCLNet+CEM and HUCLNet+SAM, exhibit superior separability, with target boxes generally fully separated from the background. 
% These methods effectively suppress background noise, enabling reliable target detection in underwater environments. 
% The detection range for HUCLNet-based methods is close to 1, while the background range is near 0, indicating a low false positive rate. Compared to CEM and SAM, HUCLNet significantly enhances target-background separability, demonstrating its effectiveness in underwater hyperspectral target detection.
\subsection{Ablation Studies}\label{sec:4.4}
To evaluate the efficacy of each component in our method, we conducted ablation studies on the ATR2-HUTD dataset. These studies aim to confirm that the observed improvements stem not only from the increased number of parameters but also from the architectural design, which enhances the HUTD task. The HUCLNet framework is divided into three components for experimental validation. 
Corresponding results are presented in Tab.~\ref{ablation study}.
\par
\textbf{(\romannumeral1) Analysis of the RGC Module:} We validate the RGC with the following designs: 
\begin{itemize}
    \item \textbf{w/o Cluster Refinement Strategy:} This design excludes the cluster refinement strategy, relying solely on the reference spectrum-based clustering method. 
    \item \textbf{w/o Reference Spectrum-based Clustering:} This design omits the reference spectrum-based clustering approach from the RGC module.
\end{itemize}
\par
Without the cluster refinement strategy, the RGC module directly uses the original clustering results, often misclassifying pixels and negatively impacting prototype-level learning. As seen in Tab.~\ref{ablation study}, this leads to lower average metric values compared to the full HUCLNet-based methods, demonstrating the importance of refined pseudo-labels. Removing the RGC module entirely, the HLCL module uses pixel instances from the original HSI for instance-level contrastive learning, focusing on individual pixel spectra and neglecting the target-background relationships. Performance improves slightly over baseline HTD methods but remains significantly inferior to complete HUCLNet-based methods, highlighting the critical role of the RGC module in providing reliable pseudo-labels.
\par
\begin{figure*}[!t]                 
    \centering                    
    \includegraphics[width=2\columnwidth]{images/C4.jpg}                     
    \caption{Parameter analysis results on the ATR2-HUTD dataset. (a) Number of clusters in the RCG module; (b) Batch size in the HLCL module; (c) Attack method in the HLCL module. Red and yellow points indicate the maximum and minimum values, respectively.}                 
    \label{fig:C4-1}    
\end{figure*}
\textbf{(\romannumeral2) Analysis of the HLCL Module:} We evaluate the HLCL module with the following designs:
\begin{itemize}
    \item \textbf{w/o Instance-level Contrastive Learning:} This design removes instance-level contrastive learning, relying only on refined cluster labels from the RGC module.
    \item \textbf{w/o Prototype-level Contrastive Learning:} This design removes prototype-level contrastive learning, retaining only instance-level contrastive learning.
    \item \textbf{w/o Hyperspectral-Oriented Data Augmentation:} This design removes hyperspectral-specific data augmentation from the HLCL module.
    \item \textbf{w/o HLCL Module:} This design excludes the entire HLCL module.
\end{itemize}
\par
According to Tab.~\ref{ablation study}, we can draw the following conclusions.
When the HLCL module operates without instance-level contrastive learning, HUCLNet relies solely on the cluster labels, leading to performance degradation. However, prototype-level contrastive learning alone still outperforms baseline HTD methods, emphasizing the importance of target-background separability. The removal of prototype-level contrastive learning results in poorer performance compared to the instance-level design, indicating its greater impact on separability. When hyperspectral-oriented data augmentation is excluded, traditional augmentation methods lead to observable performance degradation, confirming the importance of hyperspectral-specific augmentation in enhancing feature discriminability and HUCLNet's performance. Finally, removing the HLCL module entirely reduces HUCLNet to baseline HTD methods, resulting in substantial performance loss, reinforcing the HLCL module's primary contribution to performance improvement.
\par
\textbf{(\romannumeral3) Analysis of the SPL Paradigm:} We evaluate the SPL paradigm with the following design: 
\begin{itemize}
    \item \textbf{w/o SPL Paradigm:} This design trains the model using the traditional self-supervised learning framework, which consists of a single reliable-guided clustering step followed by hybrid-level contrastive learning.
\end{itemize}
\par
Without the SPL paradigm, inaccurate clustering due to limited spectral discriminability hinders contrastive learning effectiveness, resulting in error propagation and performance degradation. Tab.~\ref{ablation study} confirms that the SPL paradigm significantly enhances HUCLNet's performance, underscoring the importance of the self-paced strategy in guiding model training and improving detection accuracy.
\par
\subsection{Parameter Analysis}\label{sec:4.5}
The key hyperparameters of the HUCLNet architecture, including the number of clusters in the RGC module, batch size, and attack method in the HLCL module, were analyzed through experiments on the ATR2-HUTD dataset. The results, primarily focusing on the $\text{AUC}_{\text{OA}}$ metric, are presented in Fig.~\ref{fig:C4-1}, as it is the most critical indicator of overall detection performance.
\par
\textbf{(\romannumeral1) Number of Clusters in the RGC Module:} The number of clusters in the RGC module plays a crucial role in clustering accuracy and overall HUCLNet performance. The number of clusters was varied between 30 and 48, with a step size of 3 (Fig.~\ref{fig:C4-1} (a)). Performance improves with an increasing number of clusters up to an optimal point, after which it deteriorates due to over-segmentation, where target pixels are fragmented into multiple clusters. This fragmentation hinders prototype-level contrastive learning, leading to inconsistent target representations. For the ATR2-HUTD Lake, River, and Sea sub-datasets, the optimal number of clusters was 36, 39, and 42, respectively. Even with suboptimal cluster numbers, HUCLNet outperforms baseline methods.

\textbf{(\romannumeral2) Batch Size in the HLCL Module:} The batch size in the HLCL module is another critical parameter affecting HUCLNet performance. Varying the batch size from 32 to 512 with a step size of 64, results (Fig.~\ref{fig:C4-1} (b)) show that larger batch sizes generally improve performance by increasing the number of negative samples, enhancing feature discriminability. This is consistent with prior work~\cite{Chen2020}, which indicates that larger batch sizes benefit contrastive learning. However, performance gains plateau at higher batch sizes, and larger sizes impose greater memory and computational demands. A batch size of 256 provides an optimal balance between performance and resource usage across all ATR2-HUTD sub-datasets.

\textbf{(\romannumeral3) Attack Method in the HLCL Module:} The choice of attack method in the HLCL module influences the generation of adversarial samples for contrastive learning. Four attack methods—FGSM~\cite{GoodfellowSS14}, PGD~\cite{MadryMSTV18}, FAB~\cite{Croce020}, and SPSA~\cite{SPSA}—were tested with a perturbation limit of $\epsilon = 0.1$. As shown in Fig.~\ref{fig:C4-1} (c), performance across attack methods is similar, suggesting that the specific choice of attack method has minimal impact, as long as the generated adversarial samples are effective. Given its computational efficiency and comparable performance, we adopt the FGSM attack method for HUCLNet.

% \subsection{Visualization of the effect of HUCLNet}\label{sec:4.5}
\section{Conclusions}\label{sec-conclusions}
We present Interaction-aware Conformal Prediction (ICP) to explicitly address the mutual influence between robot and humans in crowd navigation problems. We achieve interaction awareness by proposing an iterative process of robot motion planning based on human motion uncertainty and conformal prediction of the human motion dependent on the robot motion plan. Our crowd navigation simulation experiments show ICP strikes a good balance of performance among navigation efficiency, social awareness, and uncertainty quantification compared to previous works. ICP generalizes well to navigation tasks across different crowd densities, and its fast runtime and manageable memory usage indicates potential for real-world applications.

In future work, we will address infeasible robot planning solutions with adaptive failure probability and conduct real-world crowd navigation experiments to evaluate the effectiveness of ICP. As ICP is a task-agnostic algorithm, we would like to explore its applications in manipulation settings, such as collaborative manufacturing.

\begin{credits}
\subsubsection{\ackname} This work was supported by the National Science Foundation under Grant No. 2143435 and by the National Science Foundation under Grant CCF 2236484.
\end{credits}

\bibliographystyle{class/IEEEtran}
\bibliography{class/reference}
\end{document}