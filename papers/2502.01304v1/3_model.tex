\section{Forestry Crane Simulator} \label{Sec:method}

\subsection{Kinematics}

Figure \ref{fig:crane_scematics} illustrates the schematic of the forestry crane. 
It has eight degrees of freedom (DoFs) $\mathbf{q}^\mathrm{T} = [\mathbf{q}_A^\mathrm{T},\mathbf{q}_U^\mathrm{T}] $ consisting of six actuated DoFs $\mathbf{q}_A^\mathrm{T} = [q_1,q_2,q_3,q_4,q_7,q_8]$ and two unactuated joints $\mathbf{q}_U^\mathrm{T} = [{q}_5,{q}_6]$. Note that there are two pairs of synchronized joints, i.e., the prismatic joint $q_4$ and the revolute joint $q_8$. \corr{In each pair of synchronized joints, the same input is applied to the corresponding actuators; for example, the joint angle $q_8$ at the left- and right-jaw of the grapple in Figure \ref{fig:crane_scematics}. }

%The characteristics of all joints are listed in Table \ref{tab:crane_joints}.  
\begin{figure}
    \centering
    \scalebox{0.8}{
    \includegraphics[trim=5cm 3.5cm 0cm 2cm,clip,scale=0.44]{figures/KinematicChain.pdf}
    }
    \caption{Schematic of the forestry crane \cite{ecker2022iterative}.}
    \label{fig:crane_scematics}
\end{figure}
\iffalse

    \begin{table}
        \caption[abc]{List of the forestry crane joints.}
        \label{tab:crane_joints}
        \begin{center} 
            
                \begin{tabular}{c c c c c}
                    \hline
                    Coordinate & Name & Actuated & Range & Unit\\
                    \hline
                     $q_1$ & Slewing joint & \checkmark & [-3.71,\:3.71] & \si{rad}\\ 
                     $q_2$ & Boom joint & \checkmark & [-1.2,\:1.56] & \si{rad} \\  
                     $q_3$ & Arm joint & \checkmark & [-0.91,\:4.6] & \si{rad} \\
                     $q_4$ & Prismatic joint & \checkmark & [0,\:4.47] & \si{m} \\
                     $q_5$ & Tip joint & \xmark & [-1.57,\:1.57] & \si{rad} \\
                     $q_6$ & Tilt joint & \xmark & [-0.79,\:2.36] & \si{rad} \\
                     $q_7$ & Rotate joint & \checkmark & $ [-\infty,\:\infty]$ & \si{rad}\\
                     $q_8$ & Grapple jaws & \checkmark & [0,\:3] & \si{rad}\\
                    \hline 
                \end{tabular}
        \end{center}        
    \end{table}
\fi    

The kinematics of the forestry crane are described by transformations from a coordinate Frame $\mathcal{F}_i$ attached to joint $i$ to a coordinate frame $\mathcal{F}_{i-1}$ attached to joint $i-1$
\begin{align}
	\vec{H}^i_{i-1} = \begin{bmatrix}
		\vec{R}_{i-1}^i & \vec{d}_{i-1}^i\\
		\transpose{\vec{0}} & 1
	\end{bmatrix}\in\mathcal{SE}(3) \Comma
\end{align}
where $\vec{R}_{i-1}^i\in\mathcal{SO}(3)$ and $\vec{d}_{i-1}^i\in\mathbb{R}^3$ are the three-dimensional rotation matrix and the three-dimensional translation vector, respectively. 
The coordinate frames are illustrated in Figure~\ref{fig:crane_scematics} according to the \textit{Denavit-Hartenberg (DH) convention} \cite{spong:2006}. 
\iffalse

    Note that frame $\mathcal{F}_{11}$ is defined by DH convention w.r.t. frame $\mathcal{F}_{8}$ instead of $\mathcal{F}_{10}$ due to the kinematic structure depicted in Figure \ref{fig:crane_scematics}.
    %, but rather from frame $\mathcal{F}_8$. 
    Hence, homogeneous transformations $\vec{H}_{i-1}^i$, $i=1,\dots,10,12$ and $\vec{H}_{8}^{11}$ can be described using four DH parameters $\theta_i$, $d_i$, $a_i$ and $\alpha_i$ as
    \begin{align}
    	\vec{H}_{i-1}^i = \vec{H}_{Rz}(\theta_i)\vec{H}_{Tz}(d_i)\vec{H}_{Tx}(a_i)\vec{H}_{Rx}(\alpha_i)\Comma
    \end{align}
    where $\vec{H}_{Ri}$ is a pure rotation around the $i$-axis and $\vec{H}_{Ti}$ is a pure translation in direction of the $i$-axis. 
    The transformation from $\mathcal{F}_j$ to $\mathcal{F}_i$, $0\leq i < j$ can be computed using
    \begin{align}
    	\vec{H}_i^j=\begin{cases}
    		\prod_{l=i+1}^j\vec{H}_{l-1}^l&,\text{ for }j\leq 10\\
    		\Big(\prod_{l=i+1}^8\vec{H}_{l-1}^l\Big)\vec{H}_{8}^{11}\vec{H}_{11}^j&,\text{ for }11\leq j\leq 12
    	\end{cases}\Comma
    \end{align}
    where $\prod_{l=i+1}^j\vec{H}_{l-1}^l$ being the identity for $j\leq i$.
    \begin{table}
        \caption{Denavit-Hartenberg parameters of the timber crane.}\label{tab:DHParams}
    	\centering
    	\begin{tabular}{c|cccc}
                \hline
    		$i$ & $\theta_i$ [rad] & $d_i$ [m] & $a_i$ [m] & $\alpha_i$ [rad]\\
    		\hline
    		1   &            $q_1$ & 2.425     & 0.1800      &$\pi/2$\\
    		2   &            $q_2$ & 0         & 3.4931    &0\\
    		3   &            $q_3$ & 0         & -0.3925   &$\pi/2$\\
    		4   &                0 & $q_4$ + 3.157     & 0         &0\\
    		5   &                0 & $q_4$     & 0         &$-\pi/2$\\
    		6   &            $q_5$ & 0         & -0.2130   &$-\pi/2$\\
    		7   &            $q_6$ & 0         & 0         &$-\pi/2$\\
    		8   &            $q_7$ & 0.578     & 0         &0\\
    		9   & $-\pi/2$ & 0         & 0.3402    &$\pi/2$\\
    		10  &           $\pi/2$ & 0         & 0.8566    &0\\
    		11  &  $\pi/2$ & 0         & 0.3248    &$\pi/2$\\
    		12  &           $\pi/2$ & 0         & 0.8566    &0\\
                 \hline
    	\end{tabular}
    \end{table}
\begin{table}[h]
    
    \caption{Denavit-Hartenberg parameters of the forestry crane.}\label{tab:DHParams}
        	\begin{center}
            	\begin{tabular}{c|cccc}
                        \hline
            		$i$ & $\theta_i$ in \SI{}{\radian} & $d_i$ in \SI{}{\meter} & $a_i$ in \SI{}{\meter} & $\alpha_i$ in \SI{}{\radian}\\
            		\hline
            		1   &            $q_1$ & 2.4     & 0.18      &$\pi/2$\\
            		2   &            $q_2$ & 0         & 3.5    &0\\
            		3   &            $q_3$ & 0         & -0.4   &$\pi/2$\\
            		4   &                0 & $q_4$ + 3.1     & 0         &0\\
            		5   &                0 & $q_4$     & 0         &$-\pi/2$\\
            		6   &            $q_5$ & 0         & -0.21   &$-\pi/2$\\
            		7   &            $q_6$ & 0         & 0         &$-\pi/2$\\
            		8   &            $q_7$ & 0.58     & 0         &0\\
                    %9   & $-\pi/2$ & 0         & 0.3402    &$\pi/2$\\
                    \hline
            	\end{tabular}
        	\end{center}
    
    \end{table}
\fi    
    %The DH parameters for the forestry crane are given in Table~\ref{tab:DHParams}. 

\iffalse
    Using the above kinematic relations, the wrist position of the grapple, $\mathbf{d}_g^\mathrm{T} = [g_x,g_y,g_z]$, is taken from  
    %Using this formalism the calculation of the center point coordinates $g_x, g_y$ and $g_z$ of the grapple reads as
    \begin{equation}
        \vec{H}_{0}^{8} = 
        \begin{bmatrix}
            \mathbf{R}_0^8 & \mathbf{d}_g \\
            \mathbf{0}^\mathrm{T} & 1
        \end{bmatrix} \:.
        \label{eq:transformation}
    \end{equation}
\fi    
%In the used scenarios the grapple is already close to the logs and the crane is unfolded, therefore the special hydraulic kinematic is neglected. 
%Instead, all rotary joints are driven with velocity-controlled rotary motors and the telescopic boom is driven by a linear motor.

\subsection{Simulator}
\label{sec: b simulator}

The system dynamics and contacts with the environment are simulated using the open-source MuJoCo \cite{todorov2012mujoco} physics engine. 
An example of the simulated environment is illustrated in Figure \ref{fig: example mujoco}. 
The assembled model of the forestry crane (including the truck) consists of $38$ rigid bodies and $10$ active joints\corr{, including two pairs of synchronized joints}. The total operating weight of the forestry crane is approximately \SI{1981}{\kilo\gram}. 
On standard forestry cranes, hydraulic actuators powered by a pump that is driven by a combustion engine drive the slewing ($q_1$), boom ($q_2$), arm ($q_3$), and prismatic ($q_4$) joints, respectively. 
In order to simplify the model for training purposes, the hydraulic actuators are not explicitly modeled. 
%Instead, two linear motors are modeled for the synchronized joint $q_4$, and rotational motors are utilized to drive other actuated joints. 
\corr{We assume an (ideal) underlying velocity controller, with the reference velocity for the prismatic joint $q_4$ and the reference rotational velocity for the other joints as inputs. }
Thus, in the simulation environment, the grasping controller for the modeled forestry crane is a fine-tuned PID controller with reference velocities for the actuated joints $\mathbf{q}_A$.  

The wood log position is randomized in a reachable region of the forestry crane, depicted as the yellow region in Fig. \ref{fig: example mujoco}. 
The center of this region is approximately \SI{6.5}{\meter} from the crane's base. 
Additionally, logs are modeled as cylinders with a length of $\SI{2.75}{\meter}$, and the log's diameter varies in the range of $[0.3,0.8] \SI{}{\meter}$. 
In order to prevent overfitting during the training process, the slew angle of the crane is varied in the range $[-2\pi/3, -\pi/3] \SI{}{\radian}$. 
The 6-dimensional contact forces between the grapple and the wood log are computed using the signed distance field (SDF) collision primitive \cite{reiner2011interactive}. 
This is particularly important to maintain the robustness of the simulation since the inner and outer jaw of the grapple have curvy shapes. 









