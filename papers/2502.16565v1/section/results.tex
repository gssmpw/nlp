\section{Results and Analysis}
\label{sec:results}

\begin{figure}
    \centering
    \includegraphics[width=1\linewidth]{fig/timeseries_main.pdf}
    \vspace{-1.5em}
    \caption{Performance Comparison between Implicit Consensus and Explicit Consensus.}
    \vspace{-1.5em}
    \label{fig:timeseries-main}
\end{figure}

\begin{figure}
    \centering
    \includegraphics[width=1\linewidth]{fig/timeseries_ablation.pdf}
    \vspace{-1.5em}
    \caption{Performance Comparison between Implicit Consensus and other baselines.}
    \vspace{-1.5em}
    \label{fig:timeseries-ablation}
\end{figure}

\subsection{Overall Performance Comparison (RQ1)}

\paragraph{Coverage Rate, Misallocation, and Delay.} As summarized in {Table~\ref{tab:consensus} for the disaster response scenario, implicit consensus (IC) achieves a substantially higher coverage rate (CR) on average compared to explicit consensus (EC). This gap becomes especially pronounced under moderate or high volatility levels, where disaster zones shift unpredictably. For instance, under high volatility, the CR for IC remains above 0.95 in many trials, whereas EC drops below 0.65. Examination of the system logs reveals that EC teams tend to commit collectively to a single zone and often fail to reallocate in time when new disasters emerge. By contrast, IC teams exhibit ongoing discussion each round, allowing individual drones to deviate when they suspect an overlooked or more urgent location.

Table~\ref{tab:consensus} also shows that misallocation penalty is significantly lower for IC. Frequent clustering on the same zone---even as other crises unfold---remains a recurring problem for EC, inflating the penalty. The implicit approach, however, demonstrates a self-correcting mechanism: once one or two drones have already committed to a zone in the discussion phase, other drones often decide to target secondary or newly emerged zones. This dynamic ensures broader coverage. Another notable metric is the response delay, the time it takes for at least one drone to attend a newly appearing high-severity zone. As shown in the table, IC systematically outperforms EC with a lower mean RD, especially in the moderate/high volatility regimes. We observe that if an agent in the IC group receives partial or contradictory messages (e.g., unconfirmed reports about a new fire), it may deviate from the apparent group majority to investigate. Hence, while no single agent has full certainty, this partial autonomy accelerates detection and initial response to new disasters. Figure~\ref{fig:timeseries-main} shows the average performance comparison between the two methods by time.

\paragraph{Comparison to Other Baselines (Ablation Studies).} To place these findings in context, Figure~\ref{fig:timeseries-main} compares only IC vs. EC over time, while Figure~\ref{fig:timeseries-ablation} extends to additional baselines such as no diversity (all drones share identical prompts), no interaction (no agent-agent discussion), random strategy, and a single LLM system. As expected, all these baselines generally underperform compared to IC. Notably:

\begin{itemize} [itemsep=1pt, parsep=1pt, leftmargin=*]
    \item \emph{No Diversity} suffers because, in shifting environments, the entire team repeatedly makes the same decision and misses opportunities for localized exploration.
    \item \emph{No Interaction} leads to uncoordinated allocations, exhibiting high misallocation penalty even though each drone is individually rational.
    \item \emph{Random Strategy} performs erratically, occasionally hitting a decent coverage rate but typically failing to adapt to real-time severity changes.
    \item \emph{Single LLM} has neither diversity nor parallel coverage advantage, resulting in the lowest final coverage rate.
\end{itemize}

Overall, these comparisons confirm that \emph{implicit consensus} maintains superior adaptability to dynamic disaster conditions (\textbf{answering RQ1} affirmatively). Additionally, we report that the agent scale has no significant impact on the experimental results under the current setup.

\begin{figure}
    \centering
    \includegraphics[width=1\linewidth]{fig/deviation_plot.pdf}
    \vspace{-1.5em}
    \caption{Deviation-Performance Correlation Plot: Validating the Inverted U-Shaped Hypothesis.}
    \vspace{-1.5em}
    \label{fig:deviation}
\end{figure}

\begin{table}[t]
\centering
\small
\caption{Overall results of dynamic disaster response scenario: Comparison between Explicit and Implicit (Main) Consensus. Metrics include Coverage Rate (CR, higher is better), Misallocation Penalty (MP, lower is better), and Response Delay (RD, lower is better).}
\renewcommand{\arraystretch}{1.2}
\setlength{\tabcolsep}{4pt}
\resizebox{\linewidth}{!}{
\begin{tabular}{l l ccc ccc}
\toprule
\multirow{2}{*}{\textbf{Condition}} & \multirow{2}{*}{\textbf{Level}} & \multicolumn{3}{c}{\textbf{Explicit Consensus}} & \multicolumn{3}{c}{\textbf{Implicit Consensus}} \\
\cmidrule(lr){3-5} \cmidrule(lr){6-8}
& & \textbf{CR} & \textbf{MP} & \textbf{RD} & \textbf{CR} & \textbf{MP} & \textbf{RD} \\
\midrule
\textbf{Overall} & - & 0.679 & 2.847 & 1.324 & \textbf{0.952} & \textbf{0.208} & \textbf{0.222} \\
\midrule
\multirow{3}{*}{\textbf{Diversity}} 
& Low    & 0.671 & 2.958 & 1.352 & \textbf{0.918} & \textbf{0.542} & \textbf{0.436} \\
& Medium & 0.661 & 2.750 & 1.438 & \textbf{0.968} & \textbf{0.042} & \textbf{0.134} \\
& High   & 0.706 & 2.833 & 1.183 & \textbf{0.969} & \textbf{0.042} & \textbf{0.094} \\
\midrule
\multirow{3}{*}{\textbf{Volatility}} 
& Low     & 0.771 & 2.000 & \textbf{0.743} & \textbf{0.928} & \textbf{0.458} & 0.389 \\
& Moderate & 0.628 & 3.292 & 1.860 & \textbf{0.969} & \textbf{0.042} & \textbf{0.118} \\
& High    & 0.639 & 3.250 & 1.370 & \textbf{0.958} & \textbf{0.167} & \textbf{0.159} \\
\bottomrule
\end{tabular}
}
\label{tab:consensus}
\end{table}

\begin{table*}[t]
\centering
\small
\caption{Ablation Study: Comparing different configurations, including Main, No Diversity, No Interaction, Random Strategy, and Single LLM.}
\renewcommand{\arraystretch}{1.2}
\setlength{\tabcolsep}{4pt}
\resizebox{\linewidth}{!}{
\begin{tabular}{l l ccc ccc ccc ccc ccc}
\toprule
\multirow{2}{*}{\textbf{Condition}} & \multirow{2}{*}{\textbf{Level}} & \multicolumn{3}{c}{\textbf{Implicit Consensus}} & \multicolumn{3}{c}{\textbf{No Diversity}} & \multicolumn{3}{c}{\textbf{No Interaction}} & \multicolumn{3}{c}{\textbf{Random Strategy}} & \multicolumn{3}{c}{\textbf{Single LLM}} \\
\cmidrule(lr){3-5} \cmidrule(lr){6-8} \cmidrule(lr){9-11} \cmidrule(lr){12-14} \cmidrule(lr){15-17}
& & \textbf{CR} & \textbf{MP} & \textbf{RD} & \textbf{CR} & \textbf{MP} & \textbf{RD} & \textbf{CR} & \textbf{MP} & \textbf{RD} & \textbf{CR} & \textbf{MP} & \textbf{RD} & \textbf{CR} & \textbf{MP} & \textbf{RD} \\
\midrule
\textbf{Overall} & - & \textbf{0.952} & \textbf{0.167} & 0.222 & 0.892 & 0.750 & 0.316 & 0.843 & 0.278 & 0.246 & 0.754 & 0.208 & \textbf{0.173} & 0.628 & - & 1.619 \\
\midrule
\multirow{3}{*}{\textbf{Diversity}} 
& Low    & \textbf{0.918} & \textbf{0.458} & 0.436 & 0.906 & 0.667 & \textbf{0.227} & 0.782 & 0.750 & 0.553 & 0.714 & 0.542 & 0.330 & 0.729 & - & 1.400 \\
& Medium & \textbf{0.968} & \textbf{0.000} & 0.134 & 0.917 & 0.625 & 0.296 & 0.878 & \textbf{0.000} & 0.088 & 0.771 & 0.042 & \textbf{0.068} & 0.524 & - & 2.089 \\
& High   & \textbf{0.969} & \textbf{0.042} & 0.094 & 0.853 & 0.958 & 0.426 & 0.868 & 0.083 & 0.095 & 0.776 & \textbf{0.042} & \textbf{0.120} & 0.632 & - & 1.368 \\
\midrule
\multirow{3}{*}{\textbf{Volatility}} 
& Low     & 0.928 & \textbf{0.246} & 0.389 & \textbf{0.942} & 0.375 & 0.261 & 0.858 & 0.250 & \textbf{0.207} & 0.747 & 0.458 & 0.259 & 0.592 & - & 2.171 \\
& Moderate & \textbf{0.969} & 0.333 & \textbf{0.118} & 0.865 & 0.958 & 0.414 & 0.839 & 0.250 & 0.210 & 0.746 & \textbf{0.000} & 0.189 & 0.622 & - & 1.459 \\
& High    & \textbf{0.958} & \textbf{0.000} & 0.159 & 0.868 & 0.917 & 0.273 & 0.831 & \textbf{0.333} & 0.320 & 0.768 & 0.167 & \textbf{0.070} & 0.671 & - & 1.227 \\
\bottomrule
\end{tabular}
}
\label{tab:ablation}
\end{table*}

\subsection{Deviation-Performance Correlation (RQ2)}
\label{subsec:deviation}

While higher coverage rates and lower penalties favor IC over EC, a crucial question is \emph{why}. The \emph{dynamic consensus-diversity tradeoff} suggests that moderate agent-level deviations encourage exploration and rapid environmental adaptation. To verify this, we measure each agent’s deviation from the group mean, $d_i(t)$, and then plot the average $\bar{d}(t)$ against performance metrics.

\paragraph{Inverted-U Pattern.} In Figure~\ref{fig:deviation}, we observe a clear inverted-U relationship between mean deviation (\(\bar{d}\)) and coverage rate as the performance in scenario 1. When $\bar{d} \approx 0$ (full uniformity), coverage is suboptimal (under $0.80$ on average), because the system becomes over-committed to a single or obvious priority and overlooks secondary crises. In contrast, when $\bar{d}$ is extremely large (e.g., agents rarely agree on any zone), coverage also dips, reflecting disorganized duplication or spread-thin allocations. The peak of coverage near \(\bar{d}\approx 2 \) to \(3.5\) exemplifies how partial disagreement fosters just enough diversity to handle multiple emergent zones simultaneously while still maintaining broad consensus on urgent tasks.

\paragraph{Role of Diversity Levels.} Table~\ref{tab:consensus} and the ablation in Table~\ref{tab:ablation} break down performance for \emph{low}, \emph{medium}, and \emph{high} diversity. We see that:

\begin{itemize} [itemsep=1pt, parsep=1pt, leftmargin=*]
    \item \textbf{Medium/High Diversity} tends to yield the highest coverage rates (often exceeding 0.95) and minimal penalties. This suggests that having distinct role prompts (e.g., \emph{“Medical drone prioritizes casualties,” “Logistics drone focuses on transport cost,” “Infrastructure drone defends critical assets”}) can effectively divide labor when new or multiple disasters appear.
    \item \textbf{Low Diversity} yields more uniform decisions (\(\bar{d}\) is near zero), which can handle stable or low-volatility environments adequately, but fails to adapt quickly under frequent environment shifts.
\end{itemize}

Hence, moderate or strong heterogeneity among agents directly contributes to higher performance, validating the notion that \textit{some level of viewpoint deviation is beneficial in dynamic tasks} (\textbf{answering RQ2}).

\begin{figure}
    \centering
    \includegraphics[width=1\linewidth]{fig/random_shock.pdf}
    \vspace{-1.5em}
    \caption{Simplified Theoretical Model: Performance vs. Beta by Agent Count (N) Under Various Shock\_freq \& Alpha.}
    \vspace{-1.5em}
    \label{fig:toy-model-example}
\end{figure}

As stated in Section~\ref{subsec:cross-validation}, we perform a comparison between empirical findings and a simplified theoretical model, as shown in Figure~\ref{fig:toy-model-example}. In short, the toy model demonstrates that purely random noise typically degrades performance, indicating that the improvements from ‘partial disagreement’ truly hinge on meaningful role-based diversity rather than unstructured perturbations---consistent with our overall conclusion (see Appendix~\ref{sec:appendix:c} for details).

\subsection{Discussion and Key Insights}

\begin{enumerate} [itemsep=1pt, parsep=1pt, leftmargin=*]
    \item \textbf{Implicit vs.\ Explicit Coordination:} Figure~\ref{fig:timeseries-main} and Table~\ref{tab:consensus} confirm that \emph{implicit consensus} adapts faster to shifting disasters, achieving up to 95\% coverage in high-volatility settings.
    \item \textbf{Moderate Deviations Enhance Coverage:} Figure~\ref{fig:deviation} shows that coverage peaks at intermediate $\bar{d}$, forming a strong empirical basis for the inverted-U claim. Excessive or minimal deviations undermine synergy.
    \item \textbf{Impact of Diversity:} Medium or high diversity roles notably outperform low diversity or “no diversity,” underscoring that distinct heuristics and perspectives allow drones to intercept multiple threats simultaneously rather than following a single script.
\end{enumerate}

\vspace{-0.5em}

Overall, these findings highlight the properties of dynamic consensus-diversity tradeoff. Section~\ref{subsec:deviation}'s analysis strongly supports our hypothesis that partial autonomy and role-based heterogeneity help LLM agents respond more flexibly to evolving scenarios, confirming both \textbf{RQ1} and \textbf{RQ2}. We defer \textbf{RQ3} regarding reasoning analysis to Appendix~\ref{sec:analysis-rq3}, where we examine agent dialogues. And for performance of different base models, please refer to Appendix~\ref{sec:appendix:f}.