\section{Related Works}
\label{sec:related}
In the literature, related tasks in brain imaging analysis have been extensively studied. Conventional methods primarily focus on designing methods for brain extraction~\cite{kleesiek2016deep,lucena2019convolutional}, registration~\cite{sokooti2017nonrigid, su2022abn}, segmentation~\cite{akkus2017deep, kamnitsas2017efficient, chen2018voxresnet}, parcellation~\cite{thyreau2020learning,lim2022deepparcellation}, network generation~\cite{vskoch2022human, yin2023multi} and classification~\cite{li2021braingnn,kawahara2017brainnetcnn, kan2022brain} separately under supervised settings. However, in brain imaging studies, the collection of voxel-level annotations, transformations between images, and task-specific brain networks often prove to be expensive, as it demands extensive expertise, effort, and time to produce accurate labels, especially for high-dimensional neuroimaging data, \eg 3D MRI. To reduce this high demand for annotations, recent works have utilized automatic extraction tools~\cite{smith2002fast,cox1996afni,shattuck2002brainsuite, segonne2004hybrid}, unsupervised registration models~\cite{balakrishnan2018unsupervised,su2022abn}, inverse warping~\cite{jaderberg2015spatial}, and correlation-based metrics~\cite{liang2012effects} for performing extraction, registration, segmentation, parcellation and network generation. Nevertheless, these pipeline-based approaches frequently rely on manual quality control to correct intermediate results before performing subsequent tasks. Conducting such visual inspections is not only time-consuming and labor-intensive but also suffers from intra- and inter-rater variability, thereby impeding the overall efficiency and performance. More recently, joint extraction and registration~\cite{su2022ernet}, joint registration and segmentation~\cite{xu2019deepatlas}, joint extraction, registration and segmentation~\cite{su2023one}, and joint network generation and classification~\cite{kan2022fbnetgen} have been developed for collective learning. However, partial joint learning overlooks the potential interrelationships among these tasks, which can adversely affect overall performance and limit generalizability. There is a pressing need for more integrated, automated and robust methodologies that can seamlessly integrate and optimize all stages of raw brain imaging-to-graph analysis within a unified framework.