\section{Related Works}
\label{sec:related}
In the literature, related tasks in brain imaging analysis have been extensively studied. Conventional methods primarily focus on designing methods for brain extraction____, registration____, segmentation____, parcellation____, network generation____ and classification____ separately under supervised settings. However, in brain imaging studies, the collection of voxel-level annotations, transformations between images, and task-specific brain networks often prove to be expensive, as it demands extensive expertise, effort, and time to produce accurate labels, especially for high-dimensional neuroimaging data, \eg 3D MRI. To reduce this high demand for annotations, recent works have utilized automatic extraction tools____, unsupervised registration models____, inverse warping____, and correlation-based metrics____ for performing extraction, registration, segmentation, parcellation and network generation. Nevertheless, these pipeline-based approaches frequently rely on manual quality control to correct intermediate results before performing subsequent tasks. Conducting such visual inspections is not only time-consuming and labor-intensive but also suffers from intra- and inter-rater variability, thereby impeding the overall efficiency and performance. More recently, joint extraction and registration____, joint registration and segmentation____, joint extraction, registration and segmentation____, and joint network generation and classification____ have been developed for collective learning. However, partial joint learning overlooks the potential interrelationships among these tasks, which can adversely affect overall performance and limit generalizability. There is a pressing need for more integrated, automated and robust methodologies that can seamlessly integrate and optimize all stages of raw brain imaging-to-graph analysis within a unified framework.