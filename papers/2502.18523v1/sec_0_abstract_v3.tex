

\begin{abstract}
Brain imaging analysis is fundamental in neuroscience, providing valuable insights into brain structure and function. 
Traditional workflows follow a sequential pipeline—brain extraction, registration, segmentation, parcellation, network generation, and classification—treating each step as an independent task. These methods rely heavily on task-specific training data and expert intervention to correct intermediate errors, making them particularly burdensome for high-dimensional neuroimaging data, where annotations and quality control are costly and time-consuming. 
We introduce UniBrain, a unified end-to-end framework that integrates all processing steps into a single optimization process, allowing tasks to interact and refine each other. Unlike traditional approaches that require extensive task-specific annotations, UniBrain operates with minimal supervision, leveraging only low-cost labels (\ie classification and extraction) and a single labeled atlas. By jointly optimizing extraction, registration, segmentation, parcellation, network generation, and classification, UniBrain enhances both accuracy and computational efficiency while significantly reducing annotation effort. Experimental results demonstrate its superiority over existing methods across multiple tasks, offering a more scalable and reliable solution for neuroimaging analysis.
% Brain imaging analysis is fundamental in neuroscience, supporting the study of brain structure and function. Traditional workflows treat preprocessing (\eg brain extraction, registration, segmentation), graph-based analysis (\eg parcellation, brain network analysis), and diagnostics (\eg classification) as independent tasks, relying heavily on task-specific training data and manual intervention by experts to address intermediate errors. This stepwise approach is especially burdensome for high-dimensional neuroimages (\eg 3D MRI), where annotations and quality control are costly and time-consuming.
% We propose UniBrain, a unified end-to-end framework that jointly optimizes all processing steps, enabling tasks to mutually reinforce each other. Using only low-cost labels (\ie classification and extraction) and a single labeled atlas, UniBrain integrates modules for extraction, registration, segmentation, parcellation, network generation, and classification. Experiments show UniBrain outperforms existing methods across multiple tasks, improving both accuracy and efficiency while reducing annotation and computational costs.

%Brain imaging analysis has emerged as a prevalent paradigm in neuroscience, aimed at extracting and analyzing information from neuroimaging data for structural and functional studies of the human brain.
%The process usually starts with image preprocessing tasks (\eg brain extraction, registration, segmentation), progresses to graph-related tasks (\eg parcellation, brain network analysis), and concludes with diagnostic tasks (\eg classification). 
%Conventionally, these tasks are performed separately, requiring large amounts of task-specific training data and extensive visual inspection by experts at intermediate steps for error correction.
%This approach is particularly burdensome for high-dimensional neuroimages (\eg 3D MRI), where acquiring detailed voxel-level annotations and conducting manual quality control are both expensive and time-consuming.
%In this paper, we study the problem of end-to-end learning for brain imaging tasks using only low-cost labels (\ie classification and extraction labels) and a single labeled template image (\emph{a.k.a.} atlas) for training. 
%We propose UniBrain, a unified end-to-end framework that jointly optimizes all processing steps, allowing tasks to mutually reinforce each other.
%UniBrain integrates interconnected modules to learn extraction masks, transformations, segmentation masks, parcellation masks, brain networks, and classification labels in a collective manner.
%Our experiments on real-world datasets show that UniBrain significantly and efficiently outperforms existing methods in brain extraction, registration, segmentation, parcellation, and classification tasks.


%Brain imaging analysis has emerged as a prevalent paradigm in neuroscience. The goal is to extract and analyze information from neuroimaging data for brain's structural and functional studies.
%However, efficiently discovering knowledge from neuroimaging data is impeded by complex and labor-intensive processing steps (\eg brain extraction, registration, segmentation, parcellation, network generation and classification). 
%Conventionally, such steps are performed separately in a supervised manner, where their performance heavily relies on the quantity of training data and extensive visual inspection performed by experts for error correction.
%These processes are particularly burdensome for high-dimensional neuroimages (\eg 3D MRI), where acquiring detailed voxel-level annotations and conducting manual quality control are both expensive and time-consuming, presenting substantial obstacles in many medical studies.
%In this paper, we study the problem of end-to-end brain imaging analysis, leveraging only low-cost labels ({\ie} classification and extraction labels) and one labeled template image (\emph{a.k.a.} atlas) for training guidance. 
%We propose a unified end-to-end framework, called UniBrain, to jointly optimize all processing steps, allowing feedback among them.
%Specifically, UniBrain consists of interconnected modules to learn extraction mask, transformation, segmentation mask, parcellation mask, brain network and classification label, with all modules being mutually reinforced by collective learning.
%Experimental results on real-world datasets demonstrate that our proposed method excels in brain extraction, registration, segmentation, parcellation and classification tasks.

\end{abstract}

