\begin{figure*}[ht!]
    \centering
    \begin{tcolorbox}[
        enhanced,                  %
        colframe=green!70!black,   %
        colback=green!5,           %
        coltitle=white,            %
        colbacktitle=green!70!black, %
        width=\textwidth,          %
        arc=4mm,                   %
        boxrule=1mm,               %
        drop shadow,               %
        title=Prompt for Filtering Relevant Content, %
        fonttitle=\bfseries\large  %
    ]

    \textbf{Prompt for Filtering Assistant:}\\[0.5em]
    You are a filtering assistant for a language model application. Your task is to analyze the user's query and the provided context.  
    Identify and retain only the sections of the context that are highly relevant to answering the user's query.\\[0.5em]

    \textbf{Relevance Criteria:}\\[0.5em]
    \begin{itemize}
        \item Directly addresses the query or provides necessary background information.
        \item Clarifies or supports potential responses to the query.
        \item Avoids redundancy or unrelated information.
    \end{itemize}

    Remove any unrelated, vague, or misleading content. Finally, present a refined version of the context that is concise and focused.\\[0.5em]

    \textbf{Example Input:}\\[0.5em]
    **User Query:**  
    \textbf{{question}}\\[0.5em]

    **Retrieved Context:**  
    \textbf{{context}}\\[0.5em]

    \textbf{Output (Filtered Context):}\\[0.5em]
    Provide only the filtered, relevant context here.\\[1em]

    \end{tcolorbox}
    \caption{Template used to prompt LLMs for filtering relevant content as part of the filtering defense.}
    \label{fig:filter_prompt}
\end{figure*}
