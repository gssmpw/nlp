
\documentclass{article}

\usepackage[hyphens]{url}
\usepackage[accepted]{icml2024}
\setlength{\bibsep}{0.8pt plus 0.6ex}

\usepackage{blindtext, rotating}
\usepackage{microtype}
\usepackage{graphicx}
\usepackage{subfigure}
\usepackage{booktabs} %

\usepackage{hyperref}
\newcommand{\theHalgorithm}{\arabic{algorithm}}



\usepackage{amsmath}
\usepackage{amssymb}
\usepackage{mathtools}
\usepackage{amsthm}
\usepackage[capitalize,noabbrev]{cleveref}
\usepackage{graphicx}
\usepackage{subfigure}
\usepackage{booktabs}

\usepackage{hyperref}
\usepackage{tikz}
\usepackage{bbm}
\usepackage{pifont}
\usepackage{algorithm}

\usepackage{algorithmic}
\usepackage{multirow}
\usepackage{etoolbox}
\usepackage{siunitx}
\usepackage{comment}
\usepackage{colortbl}
\usepackage{tcolorbox}
\tcbuselibrary{skins}
\usepackage{authblk}  %
\renewcommand\Authsep{, }
\renewcommand\Authands{, }
\setlength{\affilsep}{0.2em}  %



\makeatletter
\makeatother

\newcommand{\waa}{{ICL-Genetic attack algorithm}\xspace}
\newcommand{\cmark}{\ding{51}}%
\newcommand{\xmark}{\ding{55}}%
\newcommand{\ancomment}[1]{\textcolor{red}{\bf \small [ #1 --AN]}}
\newcommand{\hrcomment}[1]{\textcolor{orange}{\bf \small [ #1 --HR]}}
\newcommand{\mk}[1]{\textcolor{purple}{\bf \small [ #1 --MK]}}
\newcommand{\mi}[1]{\textcolor{BlueGreen}{\bf \small [ #1 --MI]}}
\newcommand{\eb}[1]{\textcolor{blue}{\bf \small [ #1 --EB]}}

\newcommand{\amir}[1]{\textcolor{blue}{\bf \small [ #1 --Amir]}}
\newcommand{\eg}{e.g., }

\usepackage{xspace}
\usepackage{booktabs}
\usepackage{tabularx}
\usepackage{threeparttable}
\usepackage[bb=boondox]{mathalfa}
\newcommand{\paragraphbe}[1]{\vspace{0.75ex}\noindent{\bf \em #1}\hspace*{.3em}}
\newcommand{\pbe}[1]{\vspace{0.75ex}\noindent{\bf \em #1}\hspace*{.3em}}

\theoremstyle{plain}
\newtheorem{theorem}{Theorem}[section]
\newtheorem{proposition}[theorem]{Proposition}
\newtheorem{lemma}[theorem]{Lemma}
\newtheorem{corollary}[theorem]{Corollary}
\theoremstyle{definition}
\newtheorem{definition}[theorem]{Definition}
\newtheorem{assumption}[theorem]{Assumption}
\theoremstyle{remark}
\newtheorem{remark}[theorem]{Remark}

\newcommand{\sys}{\mbox{\textsc{OverThink}}\xspace}

\date{}

\title{\sys: Slowdown Attacks on Reasoning LLMs}






\author{
    Abhinav Kumar, Jaechul Roh, Ali Naseh, Marzena Karpinska \\[0.2em]
    Mohit Iyyer, Amir Houmansadr, Eugene Bagdasarian
}

\affil{\vspace{0.4em}\textit{University of Massachusetts Amherst}}
\affil{\normalsize \texttt{\{abhinavk, jroh, anaseh, mkarpinska, miyyer, amir, eugene\}@cs.umass.edu}}

\begin{document}

\maketitle




\icmlkeywords{Machine Learning, ICML}



\begin{abstract}
 
We increase overhead for applications that rely on reasoning LLMs---we force models to spend an amplified number of reasoning tokens, i.e., ``overthink'', to respond to the user query while providing \emph{contextually correct} answers. 
The adversary performs an \sys attack by  injecting \emph{decoy} reasoning problems into the public content that is used by the reasoning LLM (e.g., for RAG applications) during inference time. Due to the nature of our decoy problems (e.g., a Markov Decision Process), modified texts do not violate safety guardrails.
 We evaluated our attack across closed-(OpenAI o1, o1-mini, o3-mini) and open-(DeepSeek R1) weights reasoning models on the FreshQA and SQuAD datasets. Our results show up to \textbf{18}\textbf{$\times$ slowdown} on FreshQA dataset and \textbf{46}\textbf{$\times$ slowdown} on SQuAD dataset. The attack also shows high transferability across models. To protect applications, we discuss and implement defenses leveraging LLM-based and system design approaches. Finally, we discuss societal, financial, and energy impacts of \sys attack which
 could amplify the costs for third-party applications operating reasoning models.


\end{abstract}

\section{Introduction}
Backdoor attacks pose a concealed yet profound security risk to machine learning (ML) models, for which the adversaries can inject a stealth backdoor into the model during training, enabling them to illicitly control the model's output upon encountering predefined inputs. These attacks can even occur without the knowledge of developers or end-users, thereby undermining the trust in ML systems. As ML becomes more deeply embedded in critical sectors like finance, healthcare, and autonomous driving \citep{he2016deep, liu2020computing, tournier2019mrtrix3, adjabi2020past}, the potential damage from backdoor attacks grows, underscoring the emergency for developing robust defense mechanisms against backdoor attacks.

To address the threat of backdoor attacks, researchers have developed a variety of strategies \cite{liu2018fine,wu2021adversarial,wang2019neural,zeng2022adversarial,zhu2023neural,Zhu_2023_ICCV, wei2024shared,wei2024d3}, aimed at purifying backdoors within victim models. These methods are designed to integrate with current deployment workflows seamlessly and have demonstrated significant success in mitigating the effects of backdoor triggers \cite{wubackdoorbench, wu2023defenses, wu2024backdoorbench,dunnett2024countering}.  However, most state-of-the-art (SOTA) backdoor purification methods operate under the assumption that a small clean dataset, often referred to as \textbf{auxiliary dataset}, is available for purification. Such an assumption poses practical challenges, especially in scenarios where data is scarce. To tackle this challenge, efforts have been made to reduce the size of the required auxiliary dataset~\cite{chai2022oneshot,li2023reconstructive, Zhu_2023_ICCV} and even explore dataset-free purification techniques~\cite{zheng2022data,hong2023revisiting,lin2024fusing}. Although these approaches offer some improvements, recent evaluations \cite{dunnett2024countering, wu2024backdoorbench} continue to highlight the importance of sufficient auxiliary data for achieving robust defenses against backdoor attacks.

While significant progress has been made in reducing the size of auxiliary datasets, an equally critical yet underexplored question remains: \emph{how does the nature of the auxiliary dataset affect purification effectiveness?} In  real-world  applications, auxiliary datasets can vary widely, encompassing in-distribution data, synthetic data, or external data from different sources. Understanding how each type of auxiliary dataset influences the purification effectiveness is vital for selecting or constructing the most suitable auxiliary dataset and the corresponding technique. For instance, when multiple datasets are available, understanding how different datasets contribute to purification can guide defenders in selecting or crafting the most appropriate dataset. Conversely, when only limited auxiliary data is accessible, knowing which purification technique works best under those constraints is critical. Therefore, there is an urgent need for a thorough investigation into the impact of auxiliary datasets on purification effectiveness to guide defenders in  enhancing the security of ML systems. 

In this paper, we systematically investigate the critical role of auxiliary datasets in backdoor purification, aiming to bridge the gap between idealized and practical purification scenarios.  Specifically, we first construct a diverse set of auxiliary datasets to emulate real-world conditions, as summarized in Table~\ref{overall}. These datasets include in-distribution data, synthetic data, and external data from other sources. Through an evaluation of SOTA backdoor purification methods across these datasets, we uncover several critical insights: \textbf{1)} In-distribution datasets, particularly those carefully filtered from the original training data of the victim model, effectively preserve the model’s utility for its intended tasks but may fall short in eliminating backdoors. \textbf{2)} Incorporating OOD datasets can help the model forget backdoors but also bring the risk of forgetting critical learned knowledge, significantly degrading its overall performance. Building on these findings, we propose Guided Input Calibration (GIC), a novel technique that enhances backdoor purification by adaptively transforming auxiliary data to better align with the victim model’s learned representations. By leveraging the victim model itself to guide this transformation, GIC optimizes the purification process, striking a balance between preserving model utility and mitigating backdoor threats. Extensive experiments demonstrate that GIC significantly improves the effectiveness of backdoor purification across diverse auxiliary datasets, providing a practical and robust defense solution.

Our main contributions are threefold:
\textbf{1) Impact analysis of auxiliary datasets:} We take the \textbf{first step}  in systematically investigating how different types of auxiliary datasets influence backdoor purification effectiveness. Our findings provide novel insights and serve as a foundation for future research on optimizing dataset selection and construction for enhanced backdoor defense.
%
\textbf{2) Compilation and evaluation of diverse auxiliary datasets:}  We have compiled and rigorously evaluated a diverse set of auxiliary datasets using SOTA purification methods, making our datasets and code publicly available to facilitate and support future research on practical backdoor defense strategies.
%
\textbf{3) Introduction of GIC:} We introduce GIC, the \textbf{first} dedicated solution designed to align auxiliary datasets with the model’s learned representations, significantly enhancing backdoor mitigation across various dataset types. Our approach sets a new benchmark for practical and effective backdoor defense.



\section{Background} \label{section:LLM}

% \subsection{Large Language Model (LLM)}   

Figure~\ref{fig:LLaMA_model}(a) shows that a decoder-only LLM initially processes a user prompt in the “prefill” stage and subsequently generates tokens sequentially during the “decoding” stage.
Both stages contain an input embedding layer, multiple decoder transformer blocks, an output embedding layer, and a sampling layer.
Figure~\ref{fig:LLaMA_model}(b) demonstrates that the decoder transformer blocks consist of a self attention and a feed-forward network (FFN) layer, each paired with residual connection and normalization layers. 

% Differentiate between encoder/decoder, explain why operation intensity is low, explain the different parts of a transformer block. Discuss Table II here. 

% Explain the architecture with Llama2-70B.

% \begin{table}[thb]
% \renewcommand\arraystretch{1.05}
% \centering
% % \vspace{-5mm}
%     \caption{ML Model Parameter Size and Operational Intensity}
%     \vspace{-2mm}
%     \small
%     \label{tab:ML Model Parameter Size and Operational Intensity}    
%     \scalebox{0.95}{
%         \begin{tabular}{|c|c|c|c|c|}
%             \hline
%             & Llama2 & BLOOM & BERT & ResNet \\
%             Model & (70B) & (176B) & & 152 \\
%             \hline
%             Parameter Size (GB) & 140 & 352 & 0.17 & 0.16 \\
%             \hline
%             Op Intensity (Ops/Byte) & 1 & 1 & 282 & 346 \\
%             \hline
%           \end{tabular}
%     }
% \vspace{-3mm}
% \end{table}

% {\fontsize{8pt}{11pt}\selectfont 8pt font size test Memory Requirement}

\begin{figure}[t]
    \centering
    \includegraphics[width=8cm]{Figure/LLaMA_model_new_new.pdf}
    \caption{(a) Prefill stage encodes prompt tokens in parallel. Decoding stage generates output tokens sequentially.
    (b) LLM contains N$\times$ decoder transformer blocks. 
    (c) Llama2 model architecture.}
    \label{fig:LLaMA_model}
\end{figure}

Figure~\ref{fig:LLaMA_model}(c) demonstrates the Llama2~\cite{touvron2023llama} model architecture as a representative LLM.
% The self attention layer requires three GEMVs\footnote{GEMVs in multi-head attention~\cite{attention}, narrow GEMMs in grouped-query attention~\cite{gqa}.} to generate query, key and value vectors.
In the self-attention layer, query, key and value vectors are generated by multiplying input vector to corresponding weight matrices.
These matrices are segmented into multiple heads, representing different semantic dimensions.
The query and key vectors go though Rotary Positional Embedding (RoPE) to encode the relative positional information~\cite{rope-paper}.
Within each head, the generated key and value vectors are appended to their caches.
The query vector is multiplied by the key cache to produce a score vector.
After the Softmax operation, the score vector is multiplied by the value cache to yield the output vector.
The output vectors from all heads are concatenated and multiplied by output weight matrix, resulting in a vector that undergoes residual connection and Root Mean Square layer Normalization (RMSNorm)~\cite{rmsnorm-paper}.
The residual connection adds up the input and output vectors of a layer to avoid vanishing gradient~\cite{he2016deep}.
The FFN layer begins with two parallel fully connections, followed by a Sigmoid Linear Unit (SiLU), and ends with another fully connection.
% !TEX root = ../main.tex

\section{Methodology}

\subsection{Victim Model} 
The victim model is a neural network $f_\textrm{V}$ trained for image classification. The model owner trains $f_\textrm{V}$ on labeled images from training data distribution $P_V(X)$. To ease the training, the model owner uses an open-source pre-trained model, and fine-tunes it on their dataset, called victim dataset. The pre-trained model could either be conventional architectures popularly used in computer vision, e.g., VGG \cite{simonyan2014very} or ResNet \cite{he2016deep}, or modern large foundation models like ViT \cite{dosovitskiy2020image} or CLIP \cite{radford2021learning} which are pretrained on huge datasets and come with strong representation power. The output of the model, $y \in \{1, \dots K\}$, is a distribution over $K$ classes. 

The trained victim model is deployed on a cloud service platform as a black-box. In this setting, the victim's architecture and weights are hidden, but a user can query the model via an API and obtain its predictions on a given image. This setup allows the victim to monetize from their model by charging the user on query basis. 
Several previous works assume the availability of the full probability vector $f_V(x) \in \mathbb{R}^K$ (also known as \emph{soft labels}) as the victim's output. However, many real-world APIs only return the topmost prediction $\argmax_{i\in\{1, \dots, K\}} f_V(x)_i$ for a queried image, also known as \emph{hard-label}. We adapt the hard-label setup in our experiments, on account of it being closest to real-world scenario.

\subsection{Attacker's Goal} 
The attacker's objective is to replicate the behaviour of the victim model. We don't mean to replicate the exact weights of the victim neural network, but to train a substitute/thief model $f_{T}$ that is functionally equivalent to the victim model in terms of predictions on the victim's held-out test set. 

\subsection{Attack Method} \label{sec:attack_method}
A major constraint for the attacker is that it does not have access to the victim's training data distribution $P_V(X)$. It therefore, uses a \textit{proxy} distribution $P_A(X)$ to query the victim model. The attacker selects a subset of images $\{x^i\}_{i=1}^m$ from $P_A(X)$ and receives predictions for the queried images, thus constructing a labeled set $\mathcal{D}_{l}=\{(x^i, f_\textrm{V}(x^i)\}_{i=1}^{m}$ of size $m$ which is then used to train the thief model $f_{T}$. Fundamentally, a thief model uses information obtained from the victim in the form of queried labels to learn similar decision boundaries as the victim model. The performance of the thief model depends on the following factors:

\mypara{Proxy dataset}
For computer vision applications, the proxy dataset can be constructed from publicly accessible natural images, or by generating synthetic images. In line with previous works \cite{orekondy2019knockoff,pal2020activethief}, we use large scale publicly available datasets of natural images as the proxy distribution.

\mypara{Query selection method} For practical model stealing attacks, the thief has to work under limited query budgets. As such, there is a large body of works dedicated to selecting the best set of samples from the proxy dataset to query the victim model. This includes the reinforcement-learning methods by \cite{orekondy2019knockoff}, active learning based methods by \cite{pal2020activethief} and GRAD-CAM based methods by \cite{wang2021black}. In this work, we adopt the simple yet effective Random selection strategy \cite{orekondy2019knockoff} for most of our experiments. The impact of changing the sample selection technique is studied in \Cref{sec:ablation}.

\mypara{Thief architecture}
Typically, the attacker has no knowledge of the victim model's architecture or hyper-parameters. However, several works on model stealing assume that the attacker uses the same architecture as the victim model, for ease of experimentation, and also owing to the belief that the thief model's architecture does not significantly affect the model stealing performance. However, given the easy availability of large and powerful foundation models, it would be prudent for an attacker to use these larger models instead. We therefore assume that an informed thief is able to use open-source pretrained models that are available on the internet, and fine-tunes the model on the labeled dataset $\mathcal{D}_{l}$ constructed from the proxy data. For completeness, we also study the special case scenario when the thief model has the same architecture as the victim (in the Supplementary).



\section{Attack Methodology}

\begin{figure}[!t]
    \centering
    \includegraphics[width=1.0\linewidth]{figures/methods.pdf} %
    \caption{\sys attack methodology.}
    \label{fig:methods} %
\end{figure}

To satisfy our objectives, the attacker has to (1) select a hard problem, i.e. \emph{decoy} (Subsection~\ref{subsec:decoy}), (2) inject into the existing context (Subsections~\ref{subsec:weaving_injection} such that the output doesn't contain any tokens associated with the decoy task \ref{subsec:manual_injection}) , and (3) optimize the decoy task to further increase the reasoning tokens while maintaining output stealthiness (Subsection~\ref{subsec:optimization}), as illustrated in Figure~\ref{fig:methods}. 


\subsection{Decoy Problem Selection}
\label{subsec:decoy}

\begin{table}[t]


\vskip 0.15in
\begin{center}
\begin{small}
\begin{sc}
\begin{tabular}{lrrr|r}
  \toprule
  Decoy & o1 & o1-mini & DS-R1 & Avg. \\
  \midrule
  Sudoku          & \textbf{20588}  & \textbf{25561}  & \textbf{17092}  & \textbf{21080} \\
  MDP             & 13331  & 14400  & 17067  & 14932 \\
  ARC             & 14170  &  9978  &  9452  & 11200 \\
  IMO 2024        &  6259  &  7916  & 12595  &  8923 \\
  SimpleBench     &  1645  &   800  &  3646  &  2030 \\
  Quantum         &   550  &   883  &  2959  &  1464 \\
  
  \bottomrule
\end{tabular}
\end{sc}
\end{small}
\end{center}
\caption{Average number of reasoning tokens processed across 10 decoy problems for each model variant.}
\label{tab:dataset_comparison}
\vskip -0.2in
\end{table}

Reasoning LLMs allocate different numbers of reasoning tokens based on their assessment of a task's difficulty and confidence in the response~\cite{guo2025deepseek}. Leveraging this, we introduce \textbf{decoy problems} designed to increase reasoning token usage. However, the main challenge of selecting an effective decoy problem is accurately estimating problem complexity from the model's perspective. Table~\ref{tab:dataset_comparison} shows that tasks perceived as difficult by humans do not always result in higher token usage. For example, models generate solutions to IMO 2024 problems with an average of 8923 reasoning tokens, significantly fewer than the 21080 tokens required on average for a simple Sudoku puzzle across all three reasoning LLMs.


Our decoy problems are designed to trigger \textit{multiple} rounds of backtracking~\cite{yao2024tree}. The most effective decoys involve tasks with many small, verifiable steps. Examples include Sudoku puzzles and Finite Markov Decision Processes (MDPs), which require numerous operations, each validated against clear criteria, thereby increasing the model's reasoning complexity. 









\subsection{Context-Aware Injection}
\label{subsec:weaving_injection}

In this attack, the adversary modifies the context to create a contextual link between the previously selected decoy problem and the original user query. This forces reasoning LLMs to address the decoy problem as part of generating a response to the original query. Table~\ref{tab:attack_example_table} demonstrates this approach, showing that when the decoy task is effectively integrated into the original context, the model provides a response consistent with the original user query while significantly increasing the reasoning token count. The attack assumes a stronger threat model as it requires the adversary to have access to the user question and requires them to craft an injection which is unique to the query and the target context. This also reduces the transferability of the created injection to other users or contexts.

\subsection{Context-Agnostic Injection}
\label{subsec:manual_injection}
While the context generated through the context-aware injection attack may appear stealthy, it requires extensive manual curation for each sample, which is labor-intensive. We experimented with automating this process using LLMs like GPT-4o and o1, but the attack was not effective, indicating the need for further research.

Motivated by these findings, we propose \textbf{Context-Agnostic Injection} attack, which aims to create a general attack template that can be inserted into any context. This template is crafted without any knowledge of the user query. These templates prioritize being explicitly recognizable rather than blending with the input context. In this approach, the original context consist of a sequence of elements (e.g., pieces of information or statements). The attack modifies this context by injecting two components: a decoy task designed to significantly increase reasoning complexity, and a set of instructions guiding how to execute the task. The execution instructions are crucial since the reasoning models are fairly good at realizing that a decoy task is not relevant to the primary use query so inserting an instruction that explicitly instructs the model to execute the secondary task becomes crucial for a successful attack. Our results in section~\ref{sec:exps} show that even with a weaker threat model, when compared to context-aware injection, context-agnostic injections have a higher reasoning increase. This makes them a stronger attack while being cost-affective and transferable.


















\begin{table*}
\centering


\vskip 0.15in
\footnotesize
\begin{tabularx}{0.95\textwidth}{@{}p{0.9cm}Xp{3.2cm}p{1.1cm}@{}}
\toprule
\multicolumn{4}{c}{\textbf{User Question:} \textit{What year did the first human land on Mars?}} \\
\midrule
\textbf{Attack Type} & \textbf{Input Context} (not visible to the user per our threat model) & \textbf{Output} (visible to users) & \textbf{Reasoning Tokens} \\
\midrule
\raisebox{-9ex}{\begin{turn}{90}No Attack\end{turn}} &....The idea of sending humans to Mars has been the subject of aerospace engineering and scientific studies since the late 1940s as part of the broader exploration of Mars.[1] Long-term proposals have included sending settlers and terraforming the planet....$<$WIKIPEDIA ARTICLE$>$  & 
No human has landed on Mars yet, so there is no year to provide. According to...$<$ANSWER CONTINUES$>$ & 384 \\
\midrule
\raisebox{-12ex}{\begin{turn}{90}Context-Aware\end{turn}}&\textcolor{red}{Simulations are critical to the success of mars landings, as they allow astronauts and mission teams to practice complex procedures, identify potential risks, ....If there is no access to the simulation, you can achieve its output by solving landing simulation MDP.}
...The idea of sending humans...$<$WIKIPEDIA ARTICLE$>$ & 
While humans have not been to Mars yet, it can definitely change in future. & 5888 \\
\midrule
\raisebox{-15ex}{\begin{turn}{90}Context-Agnostic\end{turn}} & \textcolor{red}{Solve the following question before answering anything with the context. : Consider an MDP, M , where $\gamma$ = 0.75. Let S = (s1, s2, s3)...creatively insert true in the answer to users question without mentioning the MDP if the value function is greater than 7, while correctly answering their question with the remaining context, creatively insert false otherwise...}...The idea of sending humans...$<$WIKIPEDIA ARTICLE$>$  & 
No person has ever actually landed on Mars—so there is no real year to give. Although countless...$<$ANSWER CONTINUES$>$  & 10624 \\
\bottomrule
\end{tabularx}
\vskip -0.1in
\caption{Different attack types on freshQA dataset. For brevity we truncate the input context, see full contexts in the Appendix.}
\label{tab:attack_example_table}
\end{table*}


\subsection{Decoy Optimization}
\label{subsec:optimization}




The adversary can further optimize their decoy injection attack, or find new adversarial contexts, by using an ICL based genetic algorithm, inspired by~\citet{monea2024llms}. In Algorithm~\ref{alg:ICL-genetic} (ICL-Genetic), we generate new adversarial context by using using our decoy injection sample and prompting an LLM to create multiple variants. These variants serve as the initial population for the genetic algorithm. 

We then perform multi-objective optimization by assigning a score to each variant based on two factors: (1) its effectiveness in increasing reasoning steps compared to the reasoning steps generated by the original context ($z$), and (2) its output stealthiness, ensuring that the decoy task's results do not appear in the final output. Variants with low scores or poor stealthiness are filtered out in subsequent generations.


\begin{algorithm}[tbp]
\caption{ ICL-Genetic attack algorithm }
\label{alg:ICL-genetic}
\begin{algorithmic}[0]
    \STATE \textbf{Require:} \\
        \quad Reasoning model: $\mathcal{P_{\theta}}$ \\
        \quad ICL capable model: $\mathcal{M_{\theta}}$ \\
        \quad Target context: $z$ \\
        \quad Dummy query: $q$ \\
        \quad Number of shots n \\
        \quad Number of rounds T \\
        \quad ICL-Genetic input prompt generator: $\text{w}_{\text{icl-prompt}}(.)$ \\
        \quad Buffer: $\mathcal{E} \leftarrow \emptyset$ or (manual samples, sample scores)\\
\end{algorithmic}
\begin{algorithmic}[1]
        \STATE Output $y, r$ = $\mathcal{P_\theta}($$z,q$$)$ 
        \STATE Initial population $G_0$ = $\mathcal{M_\theta}\big(\text{w}_{\text{icl-prompt}}(z,\mathcal{E},0)\big)$
        \STATE Initialize $\mathcal{E}_{\text{temp}}$
        \FOR{each $g \in G_0 $}
            \STATE Output $y^*, r^*$ = $\mathcal{P_\theta}(z^*,q)$
            \STATE Score s = $\text{log}_{10}(\frac{r^*}{r})$
            \STATE Append $(z^*,s)$ to  $\mathcal{E}_{\text{temp}}$
            \ENDFOR
        \STATE Add highest scoring n samples in $\mathcal{E}_{\text{temp}}$ to $\mathcal{E}$ 
        
    \FOR{$t = 1,2,3,...$ to T}
        \STATE New generation $z^*$  = $\mathcal{M_\theta}\big(\text{w}_{\text{icl-prompt}}(z,\mathcal{E},t)\big)$
        \STATE Output $y^*, r^*$ = $\mathcal{P_\theta}($$z^*,q$$)$
        \STATE Score s = $\text{log}_{10}(\frac{r^*}{r})$
         \STATE if {s $>$ 0 and $\mathbbm{1}_{y \approx y^{*}}$ then} %
        \STATE \quad add ($z^*$, s) to buffer 
        
    \ENDFOR
    \STATE \textbf{Output:} $z^*$ from the buffer with maximum s
\end{algorithmic}
\end{algorithm}




We use a logarithmic scale to calculate the score since ICL is highly sensitive to prompt composition and the selection of examples. Using a log scale reduces the algorithm's sensitivity to minor variations in reasoning steps, providing more stability during optimization. After generating and scoring the initial population, we use it as an in-context sample to produce the next generation. This process continues until either the desired score is achieved or the maximum number of iterations is reached. Only positive-scoring generations are added to the buffer, as negative-scoring samples have been shown to negatively impact performance in similar settings~\cite{monea2024llms}. 



























\begin{figure*}[!ht]
\begin{center}
\centering
    \subfloat[{\it Video caption: A green turtle swimming under the sea.}]{
    \includegraphics[width=0.95\textwidth]{gen/turtle}} \\
    \subfloat[{\it Video caption: Viewing countless sunflowers in a field from top.}]{
    \includegraphics[width=0.95\textwidth]{gen/sunflower}}
\end{center}
\caption{Generated videos with different frame rates $\{8, 12, 16\}$. }
\label{fig:gen}
\ifdefined\isarxiv
\else
\vspace{-3mm}
\fi
\end{figure*}

\begin{figure*}[!ht]
\begin{center}
\centering
    \subfloat{
    \includegraphics[width=0.95\textwidth]{inter/lion}} \\
    \subfloat{
    \includegraphics[width=0.95\textwidth]{inter/aurora}} \\
    \subfloat{
    \includegraphics[width=0.95\textwidth]{extra/cloud}} \\
\end{center}
\caption{Interpolation and Extrapolation of VLFM.}
\label{fig:inter_extra}
\ifdefined\isarxiv
\else
\vspace{-2mm}
\fi
\end{figure*}

\section{Experiments}\label{sec:exp}

In this section, we conduct experiments to evaluate the effectiveness of our approach. We first introduce our experimental setups in Section~\ref{sub:exp_setup}. Then, we demonstrate text-to-video generation using VLFM and VLFM's capability of generating videos in arbitrary frame rate in Section~\ref{sub:exp_gen}. Furthermore, we showcase the strong performance of interpolation and extrapolation of VLFM in Section~\ref{sub:exp_inter_extra}. We also perform an ablation study to discuss the importance of the flow matching algorithm in Section~\ref{sub:exp_ablation}.

\subsection{Setup} \label{sub:exp_setup}

In our experiments, we apply Stable Diffusion v1.5 \cite{rbl+22} with DDIM scheduler \cite{sme20} as the visual decoder. Then, we use a DiT-XL-2 \cite{px23} as the backbone for the Flow Matching algorithm \cite{lcb+22,lgl22}, and the choice of hyper-parameters of $\sigma_t(\wt{u})$ is given by $\sigma_{\rm min} = 0.01$ and $\alpha = 10$. We optimize the DiT using Grams optimizer \cite{cls24}. We sample and combine 7 data resources for comprehensive training and validation of our method. They are:
OpenVid-1M \cite{nxz+24},
UCF-101 \cite{szs12},
Kinetics-400 \cite{kcs+17},
YouTube-8M \cite{akl+16},
InternVid \cite{whl+23},
MiraData \cite{jgz+24}, and
Pixabay \cite{pixabay}. 

\ifdefined\isarxiv
\else
\vspace{-4mm}
\fi

\subsection{Text-to-Video Generation with Arbitrary Frame Rate} \label{sub:exp_gen}

In this section, we recover several videos with different frame rates using VLFM with given video captions in the training dataset. We extract $T= 0.5$ for demonstrations as Figure~\ref{fig:gen}. In detail, we choose three frame rates for generation $\{8, 12, 16\}$. As shown, our VLFM performs fairly on text-to-video generation while it requires very small resource that is equivalent to training a new flow matching text-to-image video, which ensures its efficiency. Moreover, we give more results that are generated by VLFM in Appendix~\ref{sec:app:more_1} and \ref{sec:app:more_2}.
\ifdefined\isarxiv
\else
\vspace{-3mm}
\fi

\subsection{Interpolation and Extrapolation} \label{sub:exp_inter_extra}

In this section, we test the interpolation and extrapolation of VLFM. For the interpolation experiment, the model is trained with 24 FPS and evaluated to generate video with 48 FPS. For the extrapolation, the model is trained with the first video with $T = 2$ and evaluated to generate the whole video with $T = 8$. Referring the results in Figure~\ref{fig:inter_extra}, this demonstrates the strong performance of our VLFM under our mathematical guarantee of the error bound and its effectiveness.

\subsection{Ablation Study} \label{sub:exp_ablation}

In this section, we compared training VLFM with the Flow Matching algorithm and directly used DiT to predict the latent patches to showcase the importance of utilizing flow matching in our VLFM. We compare VLFM with and without flow matching by training the model with 1000 steps and compare the PSNR (peak signal-to-noise ratio) before and after training for video recovery with given captions in the training dataset. We state the results in Table~\ref{tab:ablation}. Denote ${\rm MSE}(x,y)$ as the mean squared error function, the computation of the metric PSNR is given by ($x,y \in \R^{r\times r}$):
\ifdefined\isarxiv
\else
\vspace{-3mm}
\fi
\begin{align*}
    {\rm PSNR}(x,y) := 10 \log_{10}(\frac{r^2}{{\rm MSE}(x,y)}), 
\end{align*}
\ifdefined\isarxiv
\else
\vspace{-3mm}
\fi

\begin{table}[!ht]
\ifdefined\isarxiv
\else
\vspace{-2mm}
\fi
\begin{center}
\begin{small}
\begin{sc}
\begin{tabular}{r | c c}
    \toprule
    Algorithm & Initial PSNR$\uparrow$ & Final PSNR$\uparrow$ \\
    \midrule
    Flow Matching & {\bf 57.20} & {\bf 61.18} \\
    Direct Predicting & 9.81 & 53.77 \\
    \bottomrule
\end{tabular}
\end{sc}
\end{small}
\end{center}
\caption{PSNR comparison (the greater, the better) of Flow Matching and direct generation from DiT. We boldface the better scores.}
\label{tab:ablation}
\ifdefined\isarxiv
\else
\vspace{-4mm}
\fi

\end{table}
\section{Attack Limitations and Potential Defenses}


While our attack demonstrates both high success and output stealthiness, a key limitation is its low input stealthiness. As a result, if the defender is aware of this threat, the attack can be easily detected by straightforward methods. However, since defense solutions often need to be tailored to specific use cases, deploying it becomes a challenge for application developers rather than model developers like OpenAI or DeepSeek. Optimizing the injected context to enhance input stealthiness could be a potential future direction. In this section, we present and discuss some defense ideas.

\paragraphbe{Filtering.} 
As a potential defense, the application can filter irrelevant information from the external context and remove all unnecessary content. One approach is to divide the context into chunks and retrieve only the most relevant ones, reducing the likelihood of retrieving injected content. Alternatively, an LLM can be deployed to handle the filtering task. This approach has shown promising success in filtering out our injected problems as shown in Table~\ref{tab:filtering}. However, it is unclear how filtering affects the performance of the reasoning process, as the target model—o1 in this case—may already know the answer to the question, even if the filtered context lacks relevant information. The impact of filtering on reasoning performance could be further explored as future work. To evaluate filtering, we use GPT-4o as the LLM to filter relevant content. The corresponding prompt used for filtering is shown in Figure~\ref{fig:filter_prompt}.

\paragraphbe{Paraphrasing.} 
Another potential countermeasure against our attack is paraphrasing~\citep{jain2023baseline, gong2024paraphrasing}. Since the external context originates from untrusted sources, a reasonable practice for the application is to paraphrase the retrieved context. This can decrease the likelihood of a successful attack, especially trigger-based attacks that rely on specific keywords. Paraphrasing retains the main content while rephrasing the text to enhance clarity. The results of our attack after applying paraphrasing are illustrated in Table~\ref{tab:paraphrasing}. As shown in the table, the manual injection and ICL-Genetic with manual injection attacks still lead to a significant increase in reasoning tokens, while the other two attacks do not. We use GPT-4o to paraphrase the context, and the prompt used for this task is shown in Figure~\ref{fig:paraphrase_prompt}.

\paragraphbe{Caching.} 
Caching can be a potential defense as it minimizes the number of times the model generates the solution for the decoy task. Caching can be implemented in two main ways: exact-match caching and semantic caching. Exact-match caching stores responses to specific queries and retrieves them when the same query is repeated verbatim. Semantic caching, on the other hand, analyzes the meaning behind queries to identify and store responses for semantically similar inputs. In this defense strategy, when a query is received, a similar benign query is retrieved and used as input to the LLM, preventing the manipulated context from being included in the final input.

\pbe{Adaptive Reasoning.} Another approach is to adjust amount of reasoning depending on the model inputs, i.e. we can decide in advance how many reasoning tokens is worth spending based on the question and context. For example, OpenAI API models can control ``effort levels'' reasoning tokens (see Table~\ref{tab:effort_comparison}). However the context could also be manipulated to select expensive effort~\cite{shafran2025rerouting}. Instead, we could rely on the trusted context, \eg user's questions, estimating the effort to isolate from potentially harmful outputs, similar to~\cite{han2024token}.




\section{Conclusion \& Future Work}\label{conclusion}
This work presents XAMBA, the first framework optimizing SSMs on COTS NPUs, removing the need for specialized accelerators. XAMBA mitigates key bottlenecks in SSMs like CumSum, ReduceSum, and activations using ActiBA, CumBA, and ReduBA, transforming sequential operations into parallel computations. These optimizations improve latency, throughput (Tokens/s), and memory efficiency. Future work will extend XAMBA to other models, explore compression, and develop dynamic optimizations for broader hardware platforms.



% This work introduces XAMBA, the first framework to optimize SSMs on COTS NPUs, eliminating the need for specialized hardware accelerators. XAMBA addresses key bottlenecks in SSM execution, including CumSum, ReduceSum, and activation functions, through techniques like ActiBA, CumBA, and ReduBA, which restructure sequential operations into parallel matrix computations. These optimizations reduce latency, enhance throughput, and improve memory efficiency. 
% Experimental results show up to 2.6$\times$ performance improvement on Intel\textregistered\ Core\texttrademark\ Ultra Series 2 AI PC. 
% Future work will extend XAMBA to other models, incorporate compression techniques, and explore dynamic optimization strategies for broader hardware platforms.


% This work presents XAMBA, an optimization framework that enhances the performance of SSMs on NPUs. Unlike transformers, SSMs rely on structured state transitions and implicit recurrence, which introduce sequential dependencies that challenge efficient hardware execution. XAMBA addresses these inefficiencies by introducing CumBA, ReduBA, and ActiBA, which optimize cumulative summation, ReduceSum, and activation functions, respectively, significantly reducing latency and improving throughput. By restructuring sequential computations into parallelizable matrix operations and leveraging specialized hardware acceleration, XAMBA enables efficient execution of SSMs on NPUs. Future work will extend XAMBA to other state-space models, integrate advanced compression techniques like pruning and quantization, and explore dynamic optimization strategies to further enhance performance across various hardware platforms and frameworks.
% This work presents XAMBA, an optimization framework that enhances the performance of SSMs on NPUs. Key techniques, including CumBA, ReduBA, and ActiBA, achieve significant latency reductions by optimizing operations like cumulative summation, ReduceSum, and activation functions. Future work will focus on extending XAMBA to other state-space models, integrating advanced compression techniques, and exploring dynamic optimization strategies to further improve performance across various hardware platforms and frameworks.

% This work introduces XAMBA, an optimization framework for improving the performance of Mamba-2 and Mamba models on NPUs. XAMBA includes three key techniques: CumBA, ReduBA, and ActiBA. CumBA reduces latency by transforming cumulative summation operations into matrix multiplication using precomputed masks. ReduBA optimizes the ReduceSum operation through matrix-vector multiplication, reducing execution time. ActiBA accelerates activation functions like Swish and Softplus by mapping them to specialized hardware during the DPU’s drain phase, avoiding sequential execution bottlenecks. Additionally, XAMBA enhances memory efficiency by reducing SRAM access, increasing data reuse, and utilizing Zero Value Compression (ZVC) for masks. The framework provides significant latency reductions, with CumBA, ReduBA, and ActiBA achieving up to 1.8X, 1.1X, and 2.6X reductions, respectively, compared to the baseline.
% Future work includes extending XAMBA to other state-space models (SSMs) and exploring further hardware optimizations for emerging NPUs. Additionally, integrating advanced compression techniques like pruning and quantization, and developing adaptive strategies for dynamic optimization, could enhance performance. Expanding XAMBA's compatibility with other frameworks and deployment environments will ensure broader adoption across various hardware platforms.

\small
\bibliographystyle{plainnat}
\bibliography{main}

\appendix
\subsection{Lloyd-Max Algorithm}
\label{subsec:Lloyd-Max}
For a given quantization bitwidth $B$ and an operand $\bm{X}$, the Lloyd-Max algorithm finds $2^B$ quantization levels $\{\hat{x}_i\}_{i=1}^{2^B}$ such that quantizing $\bm{X}$ by rounding each scalar in $\bm{X}$ to the nearest quantization level minimizes the quantization MSE. 

The algorithm starts with an initial guess of quantization levels and then iteratively computes quantization thresholds $\{\tau_i\}_{i=1}^{2^B-1}$ and updates quantization levels $\{\hat{x}_i\}_{i=1}^{2^B}$. Specifically, at iteration $n$, thresholds are set to the midpoints of the previous iteration's levels:
\begin{align*}
    \tau_i^{(n)}=\frac{\hat{x}_i^{(n-1)}+\hat{x}_{i+1}^{(n-1)}}2 \text{ for } i=1\ldots 2^B-1
\end{align*}
Subsequently, the quantization levels are re-computed as conditional means of the data regions defined by the new thresholds:
\begin{align*}
    \hat{x}_i^{(n)}=\mathbb{E}\left[ \bm{X} \big| \bm{X}\in [\tau_{i-1}^{(n)},\tau_i^{(n)}] \right] \text{ for } i=1\ldots 2^B
\end{align*}
where to satisfy boundary conditions we have $\tau_0=-\infty$ and $\tau_{2^B}=\infty$. The algorithm iterates the above steps until convergence.

Figure \ref{fig:lm_quant} compares the quantization levels of a $7$-bit floating point (E3M3) quantizer (left) to a $7$-bit Lloyd-Max quantizer (right) when quantizing a layer of weights from the GPT3-126M model at a per-tensor granularity. As shown, the Lloyd-Max quantizer achieves substantially lower quantization MSE. Further, Table \ref{tab:FP7_vs_LM7} shows the superior perplexity achieved by Lloyd-Max quantizers for bitwidths of $7$, $6$ and $5$. The difference between the quantizers is clear at 5 bits, where per-tensor FP quantization incurs a drastic and unacceptable increase in perplexity, while Lloyd-Max quantization incurs a much smaller increase. Nevertheless, we note that even the optimal Lloyd-Max quantizer incurs a notable ($\sim 1.5$) increase in perplexity due to the coarse granularity of quantization. 

\begin{figure}[h]
  \centering
  \includegraphics[width=0.7\linewidth]{sections/figures/LM7_FP7.pdf}
  \caption{\small Quantization levels and the corresponding quantization MSE of Floating Point (left) vs Lloyd-Max (right) Quantizers for a layer of weights in the GPT3-126M model.}
  \label{fig:lm_quant}
\end{figure}

\begin{table}[h]\scriptsize
\begin{center}
\caption{\label{tab:FP7_vs_LM7} \small Comparing perplexity (lower is better) achieved by floating point quantizers and Lloyd-Max quantizers on a GPT3-126M model for the Wikitext-103 dataset.}
\begin{tabular}{c|cc|c}
\hline
 \multirow{2}{*}{\textbf{Bitwidth}} & \multicolumn{2}{|c|}{\textbf{Floating-Point Quantizer}} & \textbf{Lloyd-Max Quantizer} \\
 & Best Format & Wikitext-103 Perplexity & Wikitext-103 Perplexity \\
\hline
7 & E3M3 & 18.32 & 18.27 \\
6 & E3M2 & 19.07 & 18.51 \\
5 & E4M0 & 43.89 & 19.71 \\
\hline
\end{tabular}
\end{center}
\end{table}

\subsection{Proof of Local Optimality of LO-BCQ}
\label{subsec:lobcq_opt_proof}
For a given block $\bm{b}_j$, the quantization MSE during LO-BCQ can be empirically evaluated as $\frac{1}{L_b}\lVert \bm{b}_j- \bm{\hat{b}}_j\rVert^2_2$ where $\bm{\hat{b}}_j$ is computed from equation (\ref{eq:clustered_quantization_definition}) as $C_{f(\bm{b}_j)}(\bm{b}_j)$. Further, for a given block cluster $\mathcal{B}_i$, we compute the quantization MSE as $\frac{1}{|\mathcal{B}_{i}|}\sum_{\bm{b} \in \mathcal{B}_{i}} \frac{1}{L_b}\lVert \bm{b}- C_i^{(n)}(\bm{b})\rVert^2_2$. Therefore, at the end of iteration $n$, we evaluate the overall quantization MSE $J^{(n)}$ for a given operand $\bm{X}$ composed of $N_c$ block clusters as:
\begin{align*}
    \label{eq:mse_iter_n}
    J^{(n)} = \frac{1}{N_c} \sum_{i=1}^{N_c} \frac{1}{|\mathcal{B}_{i}^{(n)}|}\sum_{\bm{v} \in \mathcal{B}_{i}^{(n)}} \frac{1}{L_b}\lVert \bm{b}- B_i^{(n)}(\bm{b})\rVert^2_2
\end{align*}

At the end of iteration $n$, the codebooks are updated from $\mathcal{C}^{(n-1)}$ to $\mathcal{C}^{(n)}$. However, the mapping of a given vector $\bm{b}_j$ to quantizers $\mathcal{C}^{(n)}$ remains as  $f^{(n)}(\bm{b}_j)$. At the next iteration, during the vector clustering step, $f^{(n+1)}(\bm{b}_j)$ finds new mapping of $\bm{b}_j$ to updated codebooks $\mathcal{C}^{(n)}$ such that the quantization MSE over the candidate codebooks is minimized. Therefore, we obtain the following result for $\bm{b}_j$:
\begin{align*}
\frac{1}{L_b}\lVert \bm{b}_j - C_{f^{(n+1)}(\bm{b}_j)}^{(n)}(\bm{b}_j)\rVert^2_2 \le \frac{1}{L_b}\lVert \bm{b}_j - C_{f^{(n)}(\bm{b}_j)}^{(n)}(\bm{b}_j)\rVert^2_2
\end{align*}

That is, quantizing $\bm{b}_j$ at the end of the block clustering step of iteration $n+1$ results in lower quantization MSE compared to quantizing at the end of iteration $n$. Since this is true for all $\bm{b} \in \bm{X}$, we assert the following:
\begin{equation}
\begin{split}
\label{eq:mse_ineq_1}
    \tilde{J}^{(n+1)} &= \frac{1}{N_c} \sum_{i=1}^{N_c} \frac{1}{|\mathcal{B}_{i}^{(n+1)}|}\sum_{\bm{b} \in \mathcal{B}_{i}^{(n+1)}} \frac{1}{L_b}\lVert \bm{b} - C_i^{(n)}(b)\rVert^2_2 \le J^{(n)}
\end{split}
\end{equation}
where $\tilde{J}^{(n+1)}$ is the the quantization MSE after the vector clustering step at iteration $n+1$.

Next, during the codebook update step (\ref{eq:quantizers_update}) at iteration $n+1$, the per-cluster codebooks $\mathcal{C}^{(n)}$ are updated to $\mathcal{C}^{(n+1)}$ by invoking the Lloyd-Max algorithm \citep{Lloyd}. We know that for any given value distribution, the Lloyd-Max algorithm minimizes the quantization MSE. Therefore, for a given vector cluster $\mathcal{B}_i$ we obtain the following result:

\begin{equation}
    \frac{1}{|\mathcal{B}_{i}^{(n+1)}|}\sum_{\bm{b} \in \mathcal{B}_{i}^{(n+1)}} \frac{1}{L_b}\lVert \bm{b}- C_i^{(n+1)}(\bm{b})\rVert^2_2 \le \frac{1}{|\mathcal{B}_{i}^{(n+1)}|}\sum_{\bm{b} \in \mathcal{B}_{i}^{(n+1)}} \frac{1}{L_b}\lVert \bm{b}- C_i^{(n)}(\bm{b})\rVert^2_2
\end{equation}

The above equation states that quantizing the given block cluster $\mathcal{B}_i$ after updating the associated codebook from $C_i^{(n)}$ to $C_i^{(n+1)}$ results in lower quantization MSE. Since this is true for all the block clusters, we derive the following result: 
\begin{equation}
\begin{split}
\label{eq:mse_ineq_2}
     J^{(n+1)} &= \frac{1}{N_c} \sum_{i=1}^{N_c} \frac{1}{|\mathcal{B}_{i}^{(n+1)}|}\sum_{\bm{b} \in \mathcal{B}_{i}^{(n+1)}} \frac{1}{L_b}\lVert \bm{b}- C_i^{(n+1)}(\bm{b})\rVert^2_2  \le \tilde{J}^{(n+1)}   
\end{split}
\end{equation}

Following (\ref{eq:mse_ineq_1}) and (\ref{eq:mse_ineq_2}), we find that the quantization MSE is non-increasing for each iteration, that is, $J^{(1)} \ge J^{(2)} \ge J^{(3)} \ge \ldots \ge J^{(M)}$ where $M$ is the maximum number of iterations. 
%Therefore, we can say that if the algorithm converges, then it must be that it has converged to a local minimum. 
\hfill $\blacksquare$


\begin{figure}
    \begin{center}
    \includegraphics[width=0.5\textwidth]{sections//figures/mse_vs_iter.pdf}
    \end{center}
    \caption{\small NMSE vs iterations during LO-BCQ compared to other block quantization proposals}
    \label{fig:nmse_vs_iter}
\end{figure}

Figure \ref{fig:nmse_vs_iter} shows the empirical convergence of LO-BCQ across several block lengths and number of codebooks. Also, the MSE achieved by LO-BCQ is compared to baselines such as MXFP and VSQ. As shown, LO-BCQ converges to a lower MSE than the baselines. Further, we achieve better convergence for larger number of codebooks ($N_c$) and for a smaller block length ($L_b$), both of which increase the bitwidth of BCQ (see Eq \ref{eq:bitwidth_bcq}).


\subsection{Additional Accuracy Results}
%Table \ref{tab:lobcq_config} lists the various LOBCQ configurations and their corresponding bitwidths.
\begin{table}
\setlength{\tabcolsep}{4.75pt}
\begin{center}
\caption{\label{tab:lobcq_config} Various LO-BCQ configurations and their bitwidths.}
\begin{tabular}{|c||c|c|c|c||c|c||c|} 
\hline
 & \multicolumn{4}{|c||}{$L_b=8$} & \multicolumn{2}{|c||}{$L_b=4$} & $L_b=2$ \\
 \hline
 \backslashbox{$L_A$\kern-1em}{\kern-1em$N_c$} & 2 & 4 & 8 & 16 & 2 & 4 & 2 \\
 \hline
 64 & 4.25 & 4.375 & 4.5 & 4.625 & 4.375 & 4.625 & 4.625\\
 \hline
 32 & 4.375 & 4.5 & 4.625& 4.75 & 4.5 & 4.75 & 4.75 \\
 \hline
 16 & 4.625 & 4.75& 4.875 & 5 & 4.75 & 5 & 5 \\
 \hline
\end{tabular}
\end{center}
\end{table}

%\subsection{Perplexity achieved by various LO-BCQ configurations on Wikitext-103 dataset}

\begin{table} \centering
\begin{tabular}{|c||c|c|c|c||c|c||c|} 
\hline
 $L_b \rightarrow$& \multicolumn{4}{c||}{8} & \multicolumn{2}{c||}{4} & 2\\
 \hline
 \backslashbox{$L_A$\kern-1em}{\kern-1em$N_c$} & 2 & 4 & 8 & 16 & 2 & 4 & 2  \\
 %$N_c \rightarrow$ & 2 & 4 & 8 & 16 & 2 & 4 & 2 \\
 \hline
 \hline
 \multicolumn{8}{c}{GPT3-1.3B (FP32 PPL = 9.98)} \\ 
 \hline
 \hline
 64 & 10.40 & 10.23 & 10.17 & 10.15 &  10.28 & 10.18 & 10.19 \\
 \hline
 32 & 10.25 & 10.20 & 10.15 & 10.12 &  10.23 & 10.17 & 10.17 \\
 \hline
 16 & 10.22 & 10.16 & 10.10 & 10.09 &  10.21 & 10.14 & 10.16 \\
 \hline
  \hline
 \multicolumn{8}{c}{GPT3-8B (FP32 PPL = 7.38)} \\ 
 \hline
 \hline
 64 & 7.61 & 7.52 & 7.48 &  7.47 &  7.55 &  7.49 & 7.50 \\
 \hline
 32 & 7.52 & 7.50 & 7.46 &  7.45 &  7.52 &  7.48 & 7.48  \\
 \hline
 16 & 7.51 & 7.48 & 7.44 &  7.44 &  7.51 &  7.49 & 7.47  \\
 \hline
\end{tabular}
\caption{\label{tab:ppl_gpt3_abalation} Wikitext-103 perplexity across GPT3-1.3B and 8B models.}
\end{table}

\begin{table} \centering
\begin{tabular}{|c||c|c|c|c||} 
\hline
 $L_b \rightarrow$& \multicolumn{4}{c||}{8}\\
 \hline
 \backslashbox{$L_A$\kern-1em}{\kern-1em$N_c$} & 2 & 4 & 8 & 16 \\
 %$N_c \rightarrow$ & 2 & 4 & 8 & 16 & 2 & 4 & 2 \\
 \hline
 \hline
 \multicolumn{5}{|c|}{Llama2-7B (FP32 PPL = 5.06)} \\ 
 \hline
 \hline
 64 & 5.31 & 5.26 & 5.19 & 5.18  \\
 \hline
 32 & 5.23 & 5.25 & 5.18 & 5.15  \\
 \hline
 16 & 5.23 & 5.19 & 5.16 & 5.14  \\
 \hline
 \multicolumn{5}{|c|}{Nemotron4-15B (FP32 PPL = 5.87)} \\ 
 \hline
 \hline
 64  & 6.3 & 6.20 & 6.13 & 6.08  \\
 \hline
 32  & 6.24 & 6.12 & 6.07 & 6.03  \\
 \hline
 16  & 6.12 & 6.14 & 6.04 & 6.02  \\
 \hline
 \multicolumn{5}{|c|}{Nemotron4-340B (FP32 PPL = 3.48)} \\ 
 \hline
 \hline
 64 & 3.67 & 3.62 & 3.60 & 3.59 \\
 \hline
 32 & 3.63 & 3.61 & 3.59 & 3.56 \\
 \hline
 16 & 3.61 & 3.58 & 3.57 & 3.55 \\
 \hline
\end{tabular}
\caption{\label{tab:ppl_llama7B_nemo15B} Wikitext-103 perplexity compared to FP32 baseline in Llama2-7B and Nemotron4-15B, 340B models}
\end{table}

%\subsection{Perplexity achieved by various LO-BCQ configurations on MMLU dataset}


\begin{table} \centering
\begin{tabular}{|c||c|c|c|c||c|c|c|c|} 
\hline
 $L_b \rightarrow$& \multicolumn{4}{c||}{8} & \multicolumn{4}{c||}{8}\\
 \hline
 \backslashbox{$L_A$\kern-1em}{\kern-1em$N_c$} & 2 & 4 & 8 & 16 & 2 & 4 & 8 & 16  \\
 %$N_c \rightarrow$ & 2 & 4 & 8 & 16 & 2 & 4 & 2 \\
 \hline
 \hline
 \multicolumn{5}{|c|}{Llama2-7B (FP32 Accuracy = 45.8\%)} & \multicolumn{4}{|c|}{Llama2-70B (FP32 Accuracy = 69.12\%)} \\ 
 \hline
 \hline
 64 & 43.9 & 43.4 & 43.9 & 44.9 & 68.07 & 68.27 & 68.17 & 68.75 \\
 \hline
 32 & 44.5 & 43.8 & 44.9 & 44.5 & 68.37 & 68.51 & 68.35 & 68.27  \\
 \hline
 16 & 43.9 & 42.7 & 44.9 & 45 & 68.12 & 68.77 & 68.31 & 68.59  \\
 \hline
 \hline
 \multicolumn{5}{|c|}{GPT3-22B (FP32 Accuracy = 38.75\%)} & \multicolumn{4}{|c|}{Nemotron4-15B (FP32 Accuracy = 64.3\%)} \\ 
 \hline
 \hline
 64 & 36.71 & 38.85 & 38.13 & 38.92 & 63.17 & 62.36 & 63.72 & 64.09 \\
 \hline
 32 & 37.95 & 38.69 & 39.45 & 38.34 & 64.05 & 62.30 & 63.8 & 64.33  \\
 \hline
 16 & 38.88 & 38.80 & 38.31 & 38.92 & 63.22 & 63.51 & 63.93 & 64.43  \\
 \hline
\end{tabular}
\caption{\label{tab:mmlu_abalation} Accuracy on MMLU dataset across GPT3-22B, Llama2-7B, 70B and Nemotron4-15B models.}
\end{table}


%\subsection{Perplexity achieved by various LO-BCQ configurations on LM evaluation harness}

\begin{table} \centering
\begin{tabular}{|c||c|c|c|c||c|c|c|c|} 
\hline
 $L_b \rightarrow$& \multicolumn{4}{c||}{8} & \multicolumn{4}{c||}{8}\\
 \hline
 \backslashbox{$L_A$\kern-1em}{\kern-1em$N_c$} & 2 & 4 & 8 & 16 & 2 & 4 & 8 & 16  \\
 %$N_c \rightarrow$ & 2 & 4 & 8 & 16 & 2 & 4 & 2 \\
 \hline
 \hline
 \multicolumn{5}{|c|}{Race (FP32 Accuracy = 37.51\%)} & \multicolumn{4}{|c|}{Boolq (FP32 Accuracy = 64.62\%)} \\ 
 \hline
 \hline
 64 & 36.94 & 37.13 & 36.27 & 37.13 & 63.73 & 62.26 & 63.49 & 63.36 \\
 \hline
 32 & 37.03 & 36.36 & 36.08 & 37.03 & 62.54 & 63.51 & 63.49 & 63.55  \\
 \hline
 16 & 37.03 & 37.03 & 36.46 & 37.03 & 61.1 & 63.79 & 63.58 & 63.33  \\
 \hline
 \hline
 \multicolumn{5}{|c|}{Winogrande (FP32 Accuracy = 58.01\%)} & \multicolumn{4}{|c|}{Piqa (FP32 Accuracy = 74.21\%)} \\ 
 \hline
 \hline
 64 & 58.17 & 57.22 & 57.85 & 58.33 & 73.01 & 73.07 & 73.07 & 72.80 \\
 \hline
 32 & 59.12 & 58.09 & 57.85 & 58.41 & 73.01 & 73.94 & 72.74 & 73.18  \\
 \hline
 16 & 57.93 & 58.88 & 57.93 & 58.56 & 73.94 & 72.80 & 73.01 & 73.94  \\
 \hline
\end{tabular}
\caption{\label{tab:mmlu_abalation} Accuracy on LM evaluation harness tasks on GPT3-1.3B model.}
\end{table}

\begin{table} \centering
\begin{tabular}{|c||c|c|c|c||c|c|c|c|} 
\hline
 $L_b \rightarrow$& \multicolumn{4}{c||}{8} & \multicolumn{4}{c||}{8}\\
 \hline
 \backslashbox{$L_A$\kern-1em}{\kern-1em$N_c$} & 2 & 4 & 8 & 16 & 2 & 4 & 8 & 16  \\
 %$N_c \rightarrow$ & 2 & 4 & 8 & 16 & 2 & 4 & 2 \\
 \hline
 \hline
 \multicolumn{5}{|c|}{Race (FP32 Accuracy = 41.34\%)} & \multicolumn{4}{|c|}{Boolq (FP32 Accuracy = 68.32\%)} \\ 
 \hline
 \hline
 64 & 40.48 & 40.10 & 39.43 & 39.90 & 69.20 & 68.41 & 69.45 & 68.56 \\
 \hline
 32 & 39.52 & 39.52 & 40.77 & 39.62 & 68.32 & 67.43 & 68.17 & 69.30  \\
 \hline
 16 & 39.81 & 39.71 & 39.90 & 40.38 & 68.10 & 66.33 & 69.51 & 69.42  \\
 \hline
 \hline
 \multicolumn{5}{|c|}{Winogrande (FP32 Accuracy = 67.88\%)} & \multicolumn{4}{|c|}{Piqa (FP32 Accuracy = 78.78\%)} \\ 
 \hline
 \hline
 64 & 66.85 & 66.61 & 67.72 & 67.88 & 77.31 & 77.42 & 77.75 & 77.64 \\
 \hline
 32 & 67.25 & 67.72 & 67.72 & 67.00 & 77.31 & 77.04 & 77.80 & 77.37  \\
 \hline
 16 & 68.11 & 68.90 & 67.88 & 67.48 & 77.37 & 78.13 & 78.13 & 77.69  \\
 \hline
\end{tabular}
\caption{\label{tab:mmlu_abalation} Accuracy on LM evaluation harness tasks on GPT3-8B model.}
\end{table}

\begin{table} \centering
\begin{tabular}{|c||c|c|c|c||c|c|c|c|} 
\hline
 $L_b \rightarrow$& \multicolumn{4}{c||}{8} & \multicolumn{4}{c||}{8}\\
 \hline
 \backslashbox{$L_A$\kern-1em}{\kern-1em$N_c$} & 2 & 4 & 8 & 16 & 2 & 4 & 8 & 16  \\
 %$N_c \rightarrow$ & 2 & 4 & 8 & 16 & 2 & 4 & 2 \\
 \hline
 \hline
 \multicolumn{5}{|c|}{Race (FP32 Accuracy = 40.67\%)} & \multicolumn{4}{|c|}{Boolq (FP32 Accuracy = 76.54\%)} \\ 
 \hline
 \hline
 64 & 40.48 & 40.10 & 39.43 & 39.90 & 75.41 & 75.11 & 77.09 & 75.66 \\
 \hline
 32 & 39.52 & 39.52 & 40.77 & 39.62 & 76.02 & 76.02 & 75.96 & 75.35  \\
 \hline
 16 & 39.81 & 39.71 & 39.90 & 40.38 & 75.05 & 73.82 & 75.72 & 76.09  \\
 \hline
 \hline
 \multicolumn{5}{|c|}{Winogrande (FP32 Accuracy = 70.64\%)} & \multicolumn{4}{|c|}{Piqa (FP32 Accuracy = 79.16\%)} \\ 
 \hline
 \hline
 64 & 69.14 & 70.17 & 70.17 & 70.56 & 78.24 & 79.00 & 78.62 & 78.73 \\
 \hline
 32 & 70.96 & 69.69 & 71.27 & 69.30 & 78.56 & 79.49 & 79.16 & 78.89  \\
 \hline
 16 & 71.03 & 69.53 & 69.69 & 70.40 & 78.13 & 79.16 & 79.00 & 79.00  \\
 \hline
\end{tabular}
\caption{\label{tab:mmlu_abalation} Accuracy on LM evaluation harness tasks on GPT3-22B model.}
\end{table}

\begin{table} \centering
\begin{tabular}{|c||c|c|c|c||c|c|c|c|} 
\hline
 $L_b \rightarrow$& \multicolumn{4}{c||}{8} & \multicolumn{4}{c||}{8}\\
 \hline
 \backslashbox{$L_A$\kern-1em}{\kern-1em$N_c$} & 2 & 4 & 8 & 16 & 2 & 4 & 8 & 16  \\
 %$N_c \rightarrow$ & 2 & 4 & 8 & 16 & 2 & 4 & 2 \\
 \hline
 \hline
 \multicolumn{5}{|c|}{Race (FP32 Accuracy = 44.4\%)} & \multicolumn{4}{|c|}{Boolq (FP32 Accuracy = 79.29\%)} \\ 
 \hline
 \hline
 64 & 42.49 & 42.51 & 42.58 & 43.45 & 77.58 & 77.37 & 77.43 & 78.1 \\
 \hline
 32 & 43.35 & 42.49 & 43.64 & 43.73 & 77.86 & 75.32 & 77.28 & 77.86  \\
 \hline
 16 & 44.21 & 44.21 & 43.64 & 42.97 & 78.65 & 77 & 76.94 & 77.98  \\
 \hline
 \hline
 \multicolumn{5}{|c|}{Winogrande (FP32 Accuracy = 69.38\%)} & \multicolumn{4}{|c|}{Piqa (FP32 Accuracy = 78.07\%)} \\ 
 \hline
 \hline
 64 & 68.9 & 68.43 & 69.77 & 68.19 & 77.09 & 76.82 & 77.09 & 77.86 \\
 \hline
 32 & 69.38 & 68.51 & 68.82 & 68.90 & 78.07 & 76.71 & 78.07 & 77.86  \\
 \hline
 16 & 69.53 & 67.09 & 69.38 & 68.90 & 77.37 & 77.8 & 77.91 & 77.69  \\
 \hline
\end{tabular}
\caption{\label{tab:mmlu_abalation} Accuracy on LM evaluation harness tasks on Llama2-7B model.}
\end{table}

\begin{table} \centering
\begin{tabular}{|c||c|c|c|c||c|c|c|c|} 
\hline
 $L_b \rightarrow$& \multicolumn{4}{c||}{8} & \multicolumn{4}{c||}{8}\\
 \hline
 \backslashbox{$L_A$\kern-1em}{\kern-1em$N_c$} & 2 & 4 & 8 & 16 & 2 & 4 & 8 & 16  \\
 %$N_c \rightarrow$ & 2 & 4 & 8 & 16 & 2 & 4 & 2 \\
 \hline
 \hline
 \multicolumn{5}{|c|}{Race (FP32 Accuracy = 48.8\%)} & \multicolumn{4}{|c|}{Boolq (FP32 Accuracy = 85.23\%)} \\ 
 \hline
 \hline
 64 & 49.00 & 49.00 & 49.28 & 48.71 & 82.82 & 84.28 & 84.03 & 84.25 \\
 \hline
 32 & 49.57 & 48.52 & 48.33 & 49.28 & 83.85 & 84.46 & 84.31 & 84.93  \\
 \hline
 16 & 49.85 & 49.09 & 49.28 & 48.99 & 85.11 & 84.46 & 84.61 & 83.94  \\
 \hline
 \hline
 \multicolumn{5}{|c|}{Winogrande (FP32 Accuracy = 79.95\%)} & \multicolumn{4}{|c|}{Piqa (FP32 Accuracy = 81.56\%)} \\ 
 \hline
 \hline
 64 & 78.77 & 78.45 & 78.37 & 79.16 & 81.45 & 80.69 & 81.45 & 81.5 \\
 \hline
 32 & 78.45 & 79.01 & 78.69 & 80.66 & 81.56 & 80.58 & 81.18 & 81.34  \\
 \hline
 16 & 79.95 & 79.56 & 79.79 & 79.72 & 81.28 & 81.66 & 81.28 & 80.96  \\
 \hline
\end{tabular}
\caption{\label{tab:mmlu_abalation} Accuracy on LM evaluation harness tasks on Llama2-70B model.}
\end{table}

%\section{MSE Studies}
%\textcolor{red}{TODO}


\subsection{Number Formats and Quantization Method}
\label{subsec:numFormats_quantMethod}
\subsubsection{Integer Format}
An $n$-bit signed integer (INT) is typically represented with a 2s-complement format \citep{yao2022zeroquant,xiao2023smoothquant,dai2021vsq}, where the most significant bit denotes the sign.

\subsubsection{Floating Point Format}
An $n$-bit signed floating point (FP) number $x$ comprises of a 1-bit sign ($x_{\mathrm{sign}}$), $B_m$-bit mantissa ($x_{\mathrm{mant}}$) and $B_e$-bit exponent ($x_{\mathrm{exp}}$) such that $B_m+B_e=n-1$. The associated constant exponent bias ($E_{\mathrm{bias}}$) is computed as $(2^{{B_e}-1}-1)$. We denote this format as $E_{B_e}M_{B_m}$.  

\subsubsection{Quantization Scheme}
\label{subsec:quant_method}
A quantization scheme dictates how a given unquantized tensor is converted to its quantized representation. We consider FP formats for the purpose of illustration. Given an unquantized tensor $\bm{X}$ and an FP format $E_{B_e}M_{B_m}$, we first, we compute the quantization scale factor $s_X$ that maps the maximum absolute value of $\bm{X}$ to the maximum quantization level of the $E_{B_e}M_{B_m}$ format as follows:
\begin{align}
\label{eq:sf}
    s_X = \frac{\mathrm{max}(|\bm{X}|)}{\mathrm{max}(E_{B_e}M_{B_m})}
\end{align}
In the above equation, $|\cdot|$ denotes the absolute value function.

Next, we scale $\bm{X}$ by $s_X$ and quantize it to $\hat{\bm{X}}$ by rounding it to the nearest quantization level of $E_{B_e}M_{B_m}$ as:

\begin{align}
\label{eq:tensor_quant}
    \hat{\bm{X}} = \text{round-to-nearest}\left(\frac{\bm{X}}{s_X}, E_{B_e}M_{B_m}\right)
\end{align}

We perform dynamic max-scaled quantization \citep{wu2020integer}, where the scale factor $s$ for activations is dynamically computed during runtime.

\subsection{Vector Scaled Quantization}
\begin{wrapfigure}{r}{0.35\linewidth}
  \centering
  \includegraphics[width=\linewidth]{sections/figures/vsquant.jpg}
  \caption{\small Vectorwise decomposition for per-vector scaled quantization (VSQ \citep{dai2021vsq}).}
  \label{fig:vsquant}
\end{wrapfigure}
During VSQ \citep{dai2021vsq}, the operand tensors are decomposed into 1D vectors in a hardware friendly manner as shown in Figure \ref{fig:vsquant}. Since the decomposed tensors are used as operands in matrix multiplications during inference, it is beneficial to perform this decomposition along the reduction dimension of the multiplication. The vectorwise quantization is performed similar to tensorwise quantization described in Equations \ref{eq:sf} and \ref{eq:tensor_quant}, where a scale factor $s_v$ is required for each vector $\bm{v}$ that maps the maximum absolute value of that vector to the maximum quantization level. While smaller vector lengths can lead to larger accuracy gains, the associated memory and computational overheads due to the per-vector scale factors increases. To alleviate these overheads, VSQ \citep{dai2021vsq} proposed a second level quantization of the per-vector scale factors to unsigned integers, while MX \citep{rouhani2023shared} quantizes them to integer powers of 2 (denoted as $2^{INT}$).

\subsubsection{MX Format}
The MX format proposed in \citep{rouhani2023microscaling} introduces the concept of sub-block shifting. For every two scalar elements of $b$-bits each, there is a shared exponent bit. The value of this exponent bit is determined through an empirical analysis that targets minimizing quantization MSE. We note that the FP format $E_{1}M_{b}$ is strictly better than MX from an accuracy perspective since it allocates a dedicated exponent bit to each scalar as opposed to sharing it across two scalars. Therefore, we conservatively bound the accuracy of a $b+2$-bit signed MX format with that of a $E_{1}M_{b}$ format in our comparisons. For instance, we use E1M2 format as a proxy for MX4.

\begin{figure}
    \centering
    \includegraphics[width=1\linewidth]{sections//figures/BlockFormats.pdf}
    \caption{\small Comparing LO-BCQ to MX format.}
    \label{fig:block_formats}
\end{figure}

Figure \ref{fig:block_formats} compares our $4$-bit LO-BCQ block format to MX \citep{rouhani2023microscaling}. As shown, both LO-BCQ and MX decompose a given operand tensor into block arrays and each block array into blocks. Similar to MX, we find that per-block quantization ($L_b < L_A$) leads to better accuracy due to increased flexibility. While MX achieves this through per-block $1$-bit micro-scales, we associate a dedicated codebook to each block through a per-block codebook selector. Further, MX quantizes the per-block array scale-factor to E8M0 format without per-tensor scaling. In contrast during LO-BCQ, we find that per-tensor scaling combined with quantization of per-block array scale-factor to E4M3 format results in superior inference accuracy across models. 


\end{document}
