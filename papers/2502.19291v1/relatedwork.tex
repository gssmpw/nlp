\section{Related Works}
\subsection{Incomplete Multi-view Clustering}
Incomplete multi-view clustering can be roughly classified into two categories: imputation-based methods and imputation-free methods. For imputation-based IMVC methods, the missing data is often recovered as the first step, followed by the usage of classic multi-view clustering methods for clustering.~\citeauthor{wang2021generative} use GAN to recover missing data and fuse features across views using adaptive fusion mechanism~\cite{wang2021generative}.~\citeauthor{li2023incomplete} use prototypes and sample-prototype relationships to recover missing samples, which preserve sample commonality and view diversity~\cite{li2023incomplete}.~\citeauthor{wang2024joint} use graph topological structure to recover the missing data and learn distinctive features by graph contrastive learning~\cite{wang2024joint}. For imputation-free IMVC methods, they usually perform clustering operations directly based on the available view data. For instance,~\citeauthor{xu2022deep} project embedded representations of complete data to a higher dimensional space for linear separability identification~\cite{xu2022deep}.~\citeauthor{xu2023adaptive} project all available data into a consensus feature space and explore consensus cluster information by maximizing mutual information~\cite{xu2023adaptive}.~\citeauthor{xu2024deep} use variational autoencoders to extract the information of each view, and use Product-of-Experts to obtain a consensus representation~\cite{xu2024deep}.

\subsection{Graph-based Incomplete Multi-view Clustering}
While many methods focus on learning high-level representations of data, the structural relations of the data are largely ignored. Since the graph contains the topological relations between the data and can better represent the structural relations between the data, graph-based IMVC method has also become an effective IMVC method.~\citeauthor{li2021incomplete} use a consensus graph to obtain clustering results by adaptively weighting the stretched base partition~\cite{li2021incomplete}. Instead of imputing missing samples directly,~\citeauthor{liu2023self} impute missing instances by completing the similarity graph for a clustering task~\cite{liu2023self}. With the development of deep learning technology, GCN-based IMVC methods are also widely used, which can capture the structure of data. SDIMC-net~\cite{wen2021structural} uses view-specific GCN encoders to simultaneously mine high-level representations and higher-order structural information of data.~\citeauthor{wang2022incomplete} propose a module for completing cross-view relationship transfer, which transfers the similarity between existing samples to missing samples and utilizes GCN to impute missing data~\cite{wang2022incomplete}. ~\citeauthor{yang2024geometric} propose the weight assignment method in a geometric perspective for the graph aggregation algorithm~\cite{yang2024geometric}.