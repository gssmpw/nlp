
\begin{figure*}[t!]
    \centering

    \begin{minipage}[c]{0.48\linewidth}
        \centering
        \parbox{1\linewidth}{\centering ``Samurai koala bear.''}\vspace{-3mm}
        \subfloat[VSD]{
            \begin{minipage}[c]{0.45\linewidth}
                \includegraphics[width=\linewidth]{./resource/validation/ori/ori_bear.png}
            \end{minipage}
        }\hspace{-2mm}
        \subfloat[USD]{
            \begin{minipage}[c]{0.45\linewidth}
                \includegraphics[width=\linewidth]{./resource/validation/ori/usd_bear.png}
            \end{minipage}
        }
    \end{minipage}
    \begin{minipage}[c]{0.48\linewidth}
        \centering
        \parbox{1\linewidth}{\centering ``A kangaroo wearing boxing gloves.''}\vspace{-3mm}
        \subfloat[VSD]{
            \begin{minipage}[c]{0.45\linewidth}
                \includegraphics[width=\linewidth]{./resource/validation/ori/ori_kangaroo.png}
            \end{minipage}
        }\hspace{-2mm}
        \subfloat[USD]{
            \begin{minipage}[c]{0.45\linewidth}
                \includegraphics[width=\linewidth]{./resource/validation/ori/usd_kangaroo.png}
            \end{minipage}
        }
    \end{minipage}


    \caption{2D score distillation comparing VSD~\citep{wang2024prolificdreamer} and USD. The prompts are augmented with auxiliary view descriptions (``from side view, from back view'') to capture multi-perspective information. Due to the original distribution's bias toward back-view angles, VSD generates predominantly back-view results, while USD successfully rectifies this distributional bias to produce more balanced viewpoints.}
    \label{fig:app_val_2d}
\end{figure*}

