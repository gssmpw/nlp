\begin{figure*}[t!]
    \centering
    \parbox{1\linewidth}{\centering ``Samurai koala bear.''}
    % \vspace{-7mm}

    \begin{minipage}[c]{0.22\linewidth}
        \subfloat[$p_0(c|y)$]{
            \begin{minipage}[c]{1\linewidth}
                \centering
                \includegraphics[width=0.95\linewidth]{./resource/ablation/qx/scene.png}
            \end{minipage}
        }
    \end{minipage}
    \begin{minipage}[c]{0.77\linewidth}
        \centering


        \subfloat{
            \begin{minipage}[c]{1\linewidth}
                \centering
                \rotatebox[origin=l]{90}{\parbox[c][0.03\linewidth]{0.135\linewidth}{\centering $q$}}
                \includegraphics[width=0.95\linewidth]{./resource/ablation/qx/qx_bear.png}
            \end{minipage}
        }\vspace{-5mm}\\\setcounter{subfigure}{1}
        \subfloat[Results]{
            \begin{minipage}[c]{1\linewidth}
                \centering
                \rotatebox[origin=l]{90}{\parbox[c][0.03\linewidth]{0.135\linewidth}{\centering USD}}
                \includegraphics[width=0.95\linewidth]{./resource/ablation/pyxt/pyx0_bear.png}
            \end{minipage}
        }
    \end{minipage}



    \caption{Comparison with the $q$ sampling and our USD. $q$ sampling modifies uniform view sampling using estimated pose probabilities $p_0(c|y)$. (a) shows the estimated $p_0(c|y)$ distribution for the prompt, with probabilities approximately $[0.1, 0.6, 0.15, 0.15]$ for the front (red), back (blue), left (yellow), and right (green) views. (b) shows the corresponding generation results.}
    \label{fig:app_main_qx}
\end{figure*}
