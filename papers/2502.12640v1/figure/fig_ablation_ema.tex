
\begin{figure*}[t!]
    \centering

    \begin{minipage}[c]{1\linewidth}
        \centering
        \parbox{1\linewidth}{\centering $n_{ema}=10,000$}\vspace{-3mm}
        \subfloat{
            \begin{minipage}[c]{1\linewidth}
                \includegraphics[width=\linewidth]{./resource/ablation/ema/sep_0.pdf}
            \end{minipage}
        }

        \parbox{1\linewidth}{\centering $n_{ema}=1,000$}\vspace{-3mm}
        \subfloat{
            \begin{minipage}[c]{1\linewidth}
                \includegraphics[width=\linewidth]{./resource/ablation/ema/sep_1.pdf}
            \end{minipage}
        }

        \parbox{1\linewidth}{\centering $n_{ema}=100$}\vspace{-3mm}
        \subfloat{
            \begin{minipage}[c]{1\linewidth}
                \includegraphics[width=\linewidth]{./resource/ablation/ema/sep_2.pdf}
            \end{minipage}
        }
        \\
        % \vspace{-4mm}
        
    \end{minipage}

    \caption{The impact of varying the number of valid EMA steps $n_{ema}$ on pose probability distributions over time. Each curve shows $\bar{p}_t(\bar{c}|y)$, representing how pose distributions evolve during training. For visualization clarity, we map the continuous time $t$ to discrete indices $t'$, where each index spans 100 timesteps (e.g., $t'=0$ corresponds to $t \in [0, 100]$). A lower $n_{ema}$ enables timely correction of distribution bias, resulting in more stable probability distributions (\ie, please zoom in for a better view of the y-axis).}
    \label{fig:app_main_hyperema}
\end{figure*}