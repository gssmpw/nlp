
\documentclass{article} % For LaTeX2e
\usepackage{iclr2025_conference,times}

% Optional math commands from https://github.com/goodfeli/dlbook_notation.
%%%%% NEW MATH DEFINITIONS %%%%%

\usepackage{amsmath,amsfonts,bm}
\usepackage{derivative}
% Mark sections of captions for referring to divisions of figures
\newcommand{\figleft}{{\em (Left)}}
\newcommand{\figcenter}{{\em (Center)}}
\newcommand{\figright}{{\em (Right)}}
\newcommand{\figtop}{{\em (Top)}}
\newcommand{\figbottom}{{\em (Bottom)}}
\newcommand{\captiona}{{\em (a)}}
\newcommand{\captionb}{{\em (b)}}
\newcommand{\captionc}{{\em (c)}}
\newcommand{\captiond}{{\em (d)}}

% Highlight a newly defined term
\newcommand{\newterm}[1]{{\bf #1}}

% Derivative d 
\newcommand{\deriv}{{\mathrm{d}}}

% Figure reference, lower-case.
\def\figref#1{figure~\ref{#1}}
% Figure reference, capital. For start of sentence
\def\Figref#1{Figure~\ref{#1}}
\def\twofigref#1#2{figures \ref{#1} and \ref{#2}}
\def\quadfigref#1#2#3#4{figures \ref{#1}, \ref{#2}, \ref{#3} and \ref{#4}}
% Section reference, lower-case.
\def\secref#1{section~\ref{#1}}
% Section reference, capital.
\def\Secref#1{Section~\ref{#1}}
% Reference to two sections.
\def\twosecrefs#1#2{sections \ref{#1} and \ref{#2}}
% Reference to three sections.
\def\secrefs#1#2#3{sections \ref{#1}, \ref{#2} and \ref{#3}}
% Reference to an equation, lower-case.
\def\eqref#1{equation~\ref{#1}}
% Reference to an equation, upper case
\def\Eqref#1{Equation~\ref{#1}}
% A raw reference to an equation---avoid using if possible
\def\plaineqref#1{\ref{#1}}
% Reference to a chapter, lower-case.
\def\chapref#1{chapter~\ref{#1}}
% Reference to an equation, upper case.
\def\Chapref#1{Chapter~\ref{#1}}
% Reference to a range of chapters
\def\rangechapref#1#2{chapters\ref{#1}--\ref{#2}}
% Reference to an algorithm, lower-case.
\def\algref#1{algorithm~\ref{#1}}
% Reference to an algorithm, upper case.
\def\Algref#1{Algorithm~\ref{#1}}
\def\twoalgref#1#2{algorithms \ref{#1} and \ref{#2}}
\def\Twoalgref#1#2{Algorithms \ref{#1} and \ref{#2}}
% Reference to a part, lower case
\def\partref#1{part~\ref{#1}}
% Reference to a part, upper case
\def\Partref#1{Part~\ref{#1}}
\def\twopartref#1#2{parts \ref{#1} and \ref{#2}}

\def\ceil#1{\lceil #1 \rceil}
\def\floor#1{\lfloor #1 \rfloor}
\def\1{\bm{1}}
\newcommand{\train}{\mathcal{D}}
\newcommand{\valid}{\mathcal{D_{\mathrm{valid}}}}
\newcommand{\test}{\mathcal{D_{\mathrm{test}}}}

\def\eps{{\epsilon}}


% Random variables
\def\reta{{\textnormal{$\eta$}}}
\def\ra{{\textnormal{a}}}
\def\rb{{\textnormal{b}}}
\def\rc{{\textnormal{c}}}
\def\rd{{\textnormal{d}}}
\def\re{{\textnormal{e}}}
\def\rf{{\textnormal{f}}}
\def\rg{{\textnormal{g}}}
\def\rh{{\textnormal{h}}}
\def\ri{{\textnormal{i}}}
\def\rj{{\textnormal{j}}}
\def\rk{{\textnormal{k}}}
\def\rl{{\textnormal{l}}}
% rm is already a command, just don't name any random variables m
\def\rn{{\textnormal{n}}}
\def\ro{{\textnormal{o}}}
\def\rp{{\textnormal{p}}}
\def\rq{{\textnormal{q}}}
\def\rr{{\textnormal{r}}}
\def\rs{{\textnormal{s}}}
\def\rt{{\textnormal{t}}}
\def\ru{{\textnormal{u}}}
\def\rv{{\textnormal{v}}}
\def\rw{{\textnormal{w}}}
\def\rx{{\textnormal{x}}}
\def\ry{{\textnormal{y}}}
\def\rz{{\textnormal{z}}}

% Random vectors
\def\rvepsilon{{\mathbf{\epsilon}}}
\def\rvphi{{\mathbf{\phi}}}
\def\rvtheta{{\mathbf{\theta}}}
\def\rva{{\mathbf{a}}}
\def\rvb{{\mathbf{b}}}
\def\rvc{{\mathbf{c}}}
\def\rvd{{\mathbf{d}}}
\def\rve{{\mathbf{e}}}
\def\rvf{{\mathbf{f}}}
\def\rvg{{\mathbf{g}}}
\def\rvh{{\mathbf{h}}}
\def\rvu{{\mathbf{i}}}
\def\rvj{{\mathbf{j}}}
\def\rvk{{\mathbf{k}}}
\def\rvl{{\mathbf{l}}}
\def\rvm{{\mathbf{m}}}
\def\rvn{{\mathbf{n}}}
\def\rvo{{\mathbf{o}}}
\def\rvp{{\mathbf{p}}}
\def\rvq{{\mathbf{q}}}
\def\rvr{{\mathbf{r}}}
\def\rvs{{\mathbf{s}}}
\def\rvt{{\mathbf{t}}}
\def\rvu{{\mathbf{u}}}
\def\rvv{{\mathbf{v}}}
\def\rvw{{\mathbf{w}}}
\def\rvx{{\mathbf{x}}}
\def\rvy{{\mathbf{y}}}
\def\rvz{{\mathbf{z}}}

% Elements of random vectors
\def\erva{{\textnormal{a}}}
\def\ervb{{\textnormal{b}}}
\def\ervc{{\textnormal{c}}}
\def\ervd{{\textnormal{d}}}
\def\erve{{\textnormal{e}}}
\def\ervf{{\textnormal{f}}}
\def\ervg{{\textnormal{g}}}
\def\ervh{{\textnormal{h}}}
\def\ervi{{\textnormal{i}}}
\def\ervj{{\textnormal{j}}}
\def\ervk{{\textnormal{k}}}
\def\ervl{{\textnormal{l}}}
\def\ervm{{\textnormal{m}}}
\def\ervn{{\textnormal{n}}}
\def\ervo{{\textnormal{o}}}
\def\ervp{{\textnormal{p}}}
\def\ervq{{\textnormal{q}}}
\def\ervr{{\textnormal{r}}}
\def\ervs{{\textnormal{s}}}
\def\ervt{{\textnormal{t}}}
\def\ervu{{\textnormal{u}}}
\def\ervv{{\textnormal{v}}}
\def\ervw{{\textnormal{w}}}
\def\ervx{{\textnormal{x}}}
\def\ervy{{\textnormal{y}}}
\def\ervz{{\textnormal{z}}}

% Random matrices
\def\rmA{{\mathbf{A}}}
\def\rmB{{\mathbf{B}}}
\def\rmC{{\mathbf{C}}}
\def\rmD{{\mathbf{D}}}
\def\rmE{{\mathbf{E}}}
\def\rmF{{\mathbf{F}}}
\def\rmG{{\mathbf{G}}}
\def\rmH{{\mathbf{H}}}
\def\rmI{{\mathbf{I}}}
\def\rmJ{{\mathbf{J}}}
\def\rmK{{\mathbf{K}}}
\def\rmL{{\mathbf{L}}}
\def\rmM{{\mathbf{M}}}
\def\rmN{{\mathbf{N}}}
\def\rmO{{\mathbf{O}}}
\def\rmP{{\mathbf{P}}}
\def\rmQ{{\mathbf{Q}}}
\def\rmR{{\mathbf{R}}}
\def\rmS{{\mathbf{S}}}
\def\rmT{{\mathbf{T}}}
\def\rmU{{\mathbf{U}}}
\def\rmV{{\mathbf{V}}}
\def\rmW{{\mathbf{W}}}
\def\rmX{{\mathbf{X}}}
\def\rmY{{\mathbf{Y}}}
\def\rmZ{{\mathbf{Z}}}

% Elements of random matrices
\def\ermA{{\textnormal{A}}}
\def\ermB{{\textnormal{B}}}
\def\ermC{{\textnormal{C}}}
\def\ermD{{\textnormal{D}}}
\def\ermE{{\textnormal{E}}}
\def\ermF{{\textnormal{F}}}
\def\ermG{{\textnormal{G}}}
\def\ermH{{\textnormal{H}}}
\def\ermI{{\textnormal{I}}}
\def\ermJ{{\textnormal{J}}}
\def\ermK{{\textnormal{K}}}
\def\ermL{{\textnormal{L}}}
\def\ermM{{\textnormal{M}}}
\def\ermN{{\textnormal{N}}}
\def\ermO{{\textnormal{O}}}
\def\ermP{{\textnormal{P}}}
\def\ermQ{{\textnormal{Q}}}
\def\ermR{{\textnormal{R}}}
\def\ermS{{\textnormal{S}}}
\def\ermT{{\textnormal{T}}}
\def\ermU{{\textnormal{U}}}
\def\ermV{{\textnormal{V}}}
\def\ermW{{\textnormal{W}}}
\def\ermX{{\textnormal{X}}}
\def\ermY{{\textnormal{Y}}}
\def\ermZ{{\textnormal{Z}}}

% Vectors
\def\vzero{{\bm{0}}}
\def\vone{{\bm{1}}}
\def\vmu{{\bm{\mu}}}
\def\vtheta{{\bm{\theta}}}
\def\vphi{{\bm{\phi}}}
\def\va{{\bm{a}}}
\def\vb{{\bm{b}}}
\def\vc{{\bm{c}}}
\def\vd{{\bm{d}}}
\def\ve{{\bm{e}}}
\def\vf{{\bm{f}}}
\def\vg{{\bm{g}}}
\def\vh{{\bm{h}}}
\def\vi{{\bm{i}}}
\def\vj{{\bm{j}}}
\def\vk{{\bm{k}}}
\def\vl{{\bm{l}}}
\def\vm{{\bm{m}}}
\def\vn{{\bm{n}}}
\def\vo{{\bm{o}}}
\def\vp{{\bm{p}}}
\def\vq{{\bm{q}}}
\def\vr{{\bm{r}}}
\def\vs{{\bm{s}}}
\def\vt{{\bm{t}}}
\def\vu{{\bm{u}}}
\def\vv{{\bm{v}}}
\def\vw{{\bm{w}}}
\def\vx{{\bm{x}}}
\def\vy{{\bm{y}}}
\def\vz{{\bm{z}}}

% Elements of vectors
\def\evalpha{{\alpha}}
\def\evbeta{{\beta}}
\def\evepsilon{{\epsilon}}
\def\evlambda{{\lambda}}
\def\evomega{{\omega}}
\def\evmu{{\mu}}
\def\evpsi{{\psi}}
\def\evsigma{{\sigma}}
\def\evtheta{{\theta}}
\def\eva{{a}}
\def\evb{{b}}
\def\evc{{c}}
\def\evd{{d}}
\def\eve{{e}}
\def\evf{{f}}
\def\evg{{g}}
\def\evh{{h}}
\def\evi{{i}}
\def\evj{{j}}
\def\evk{{k}}
\def\evl{{l}}
\def\evm{{m}}
\def\evn{{n}}
\def\evo{{o}}
\def\evp{{p}}
\def\evq{{q}}
\def\evr{{r}}
\def\evs{{s}}
\def\evt{{t}}
\def\evu{{u}}
\def\evv{{v}}
\def\evw{{w}}
\def\evx{{x}}
\def\evy{{y}}
\def\evz{{z}}

% Matrix
\def\mA{{\bm{A}}}
\def\mB{{\bm{B}}}
\def\mC{{\bm{C}}}
\def\mD{{\bm{D}}}
\def\mE{{\bm{E}}}
\def\mF{{\bm{F}}}
\def\mG{{\bm{G}}}
\def\mH{{\bm{H}}}
\def\mI{{\bm{I}}}
\def\mJ{{\bm{J}}}
\def\mK{{\bm{K}}}
\def\mL{{\bm{L}}}
\def\mM{{\bm{M}}}
\def\mN{{\bm{N}}}
\def\mO{{\bm{O}}}
\def\mP{{\bm{P}}}
\def\mQ{{\bm{Q}}}
\def\mR{{\bm{R}}}
\def\mS{{\bm{S}}}
\def\mT{{\bm{T}}}
\def\mU{{\bm{U}}}
\def\mV{{\bm{V}}}
\def\mW{{\bm{W}}}
\def\mX{{\bm{X}}}
\def\mY{{\bm{Y}}}
\def\mZ{{\bm{Z}}}
\def\mBeta{{\bm{\beta}}}
\def\mPhi{{\bm{\Phi}}}
\def\mLambda{{\bm{\Lambda}}}
\def\mSigma{{\bm{\Sigma}}}

% Tensor
\DeclareMathAlphabet{\mathsfit}{\encodingdefault}{\sfdefault}{m}{sl}
\SetMathAlphabet{\mathsfit}{bold}{\encodingdefault}{\sfdefault}{bx}{n}
\newcommand{\tens}[1]{\bm{\mathsfit{#1}}}
\def\tA{{\tens{A}}}
\def\tB{{\tens{B}}}
\def\tC{{\tens{C}}}
\def\tD{{\tens{D}}}
\def\tE{{\tens{E}}}
\def\tF{{\tens{F}}}
\def\tG{{\tens{G}}}
\def\tH{{\tens{H}}}
\def\tI{{\tens{I}}}
\def\tJ{{\tens{J}}}
\def\tK{{\tens{K}}}
\def\tL{{\tens{L}}}
\def\tM{{\tens{M}}}
\def\tN{{\tens{N}}}
\def\tO{{\tens{O}}}
\def\tP{{\tens{P}}}
\def\tQ{{\tens{Q}}}
\def\tR{{\tens{R}}}
\def\tS{{\tens{S}}}
\def\tT{{\tens{T}}}
\def\tU{{\tens{U}}}
\def\tV{{\tens{V}}}
\def\tW{{\tens{W}}}
\def\tX{{\tens{X}}}
\def\tY{{\tens{Y}}}
\def\tZ{{\tens{Z}}}


% Graph
\def\gA{{\mathcal{A}}}
\def\gB{{\mathcal{B}}}
\def\gC{{\mathcal{C}}}
\def\gD{{\mathcal{D}}}
\def\gE{{\mathcal{E}}}
\def\gF{{\mathcal{F}}}
\def\gG{{\mathcal{G}}}
\def\gH{{\mathcal{H}}}
\def\gI{{\mathcal{I}}}
\def\gJ{{\mathcal{J}}}
\def\gK{{\mathcal{K}}}
\def\gL{{\mathcal{L}}}
\def\gM{{\mathcal{M}}}
\def\gN{{\mathcal{N}}}
\def\gO{{\mathcal{O}}}
\def\gP{{\mathcal{P}}}
\def\gQ{{\mathcal{Q}}}
\def\gR{{\mathcal{R}}}
\def\gS{{\mathcal{S}}}
\def\gT{{\mathcal{T}}}
\def\gU{{\mathcal{U}}}
\def\gV{{\mathcal{V}}}
\def\gW{{\mathcal{W}}}
\def\gX{{\mathcal{X}}}
\def\gY{{\mathcal{Y}}}
\def\gZ{{\mathcal{Z}}}

% Sets
\def\sA{{\mathbb{A}}}
\def\sB{{\mathbb{B}}}
\def\sC{{\mathbb{C}}}
\def\sD{{\mathbb{D}}}
% Don't use a set called E, because this would be the same as our symbol
% for expectation.
\def\sF{{\mathbb{F}}}
\def\sG{{\mathbb{G}}}
\def\sH{{\mathbb{H}}}
\def\sI{{\mathbb{I}}}
\def\sJ{{\mathbb{J}}}
\def\sK{{\mathbb{K}}}
\def\sL{{\mathbb{L}}}
\def\sM{{\mathbb{M}}}
\def\sN{{\mathbb{N}}}
\def\sO{{\mathbb{O}}}
\def\sP{{\mathbb{P}}}
\def\sQ{{\mathbb{Q}}}
\def\sR{{\mathbb{R}}}
\def\sS{{\mathbb{S}}}
\def\sT{{\mathbb{T}}}
\def\sU{{\mathbb{U}}}
\def\sV{{\mathbb{V}}}
\def\sW{{\mathbb{W}}}
\def\sX{{\mathbb{X}}}
\def\sY{{\mathbb{Y}}}
\def\sZ{{\mathbb{Z}}}

% Entries of a matrix
\def\emLambda{{\Lambda}}
\def\emA{{A}}
\def\emB{{B}}
\def\emC{{C}}
\def\emD{{D}}
\def\emE{{E}}
\def\emF{{F}}
\def\emG{{G}}
\def\emH{{H}}
\def\emI{{I}}
\def\emJ{{J}}
\def\emK{{K}}
\def\emL{{L}}
\def\emM{{M}}
\def\emN{{N}}
\def\emO{{O}}
\def\emP{{P}}
\def\emQ{{Q}}
\def\emR{{R}}
\def\emS{{S}}
\def\emT{{T}}
\def\emU{{U}}
\def\emV{{V}}
\def\emW{{W}}
\def\emX{{X}}
\def\emY{{Y}}
\def\emZ{{Z}}
\def\emSigma{{\Sigma}}

% entries of a tensor
% Same font as tensor, without \bm wrapper
\newcommand{\etens}[1]{\mathsfit{#1}}
\def\etLambda{{\etens{\Lambda}}}
\def\etA{{\etens{A}}}
\def\etB{{\etens{B}}}
\def\etC{{\etens{C}}}
\def\etD{{\etens{D}}}
\def\etE{{\etens{E}}}
\def\etF{{\etens{F}}}
\def\etG{{\etens{G}}}
\def\etH{{\etens{H}}}
\def\etI{{\etens{I}}}
\def\etJ{{\etens{J}}}
\def\etK{{\etens{K}}}
\def\etL{{\etens{L}}}
\def\etM{{\etens{M}}}
\def\etN{{\etens{N}}}
\def\etO{{\etens{O}}}
\def\etP{{\etens{P}}}
\def\etQ{{\etens{Q}}}
\def\etR{{\etens{R}}}
\def\etS{{\etens{S}}}
\def\etT{{\etens{T}}}
\def\etU{{\etens{U}}}
\def\etV{{\etens{V}}}
\def\etW{{\etens{W}}}
\def\etX{{\etens{X}}}
\def\etY{{\etens{Y}}}
\def\etZ{{\etens{Z}}}

% The true underlying data generating distribution
\newcommand{\pdata}{p_{\rm{data}}}
\newcommand{\ptarget}{p_{\rm{target}}}
\newcommand{\pprior}{p_{\rm{prior}}}
\newcommand{\pbase}{p_{\rm{base}}}
\newcommand{\pref}{p_{\rm{ref}}}

% The empirical distribution defined by the training set
\newcommand{\ptrain}{\hat{p}_{\rm{data}}}
\newcommand{\Ptrain}{\hat{P}_{\rm{data}}}
% The model distribution
\newcommand{\pmodel}{p_{\rm{model}}}
\newcommand{\Pmodel}{P_{\rm{model}}}
\newcommand{\ptildemodel}{\tilde{p}_{\rm{model}}}
% Stochastic autoencoder distributions
\newcommand{\pencode}{p_{\rm{encoder}}}
\newcommand{\pdecode}{p_{\rm{decoder}}}
\newcommand{\precons}{p_{\rm{reconstruct}}}

\newcommand{\laplace}{\mathrm{Laplace}} % Laplace distribution

\newcommand{\E}{\mathbb{E}}
\newcommand{\Ls}{\mathcal{L}}
\newcommand{\R}{\mathbb{R}}
\newcommand{\emp}{\tilde{p}}
\newcommand{\lr}{\alpha}
\newcommand{\reg}{\lambda}
\newcommand{\rect}{\mathrm{rectifier}}
\newcommand{\softmax}{\mathrm{softmax}}
\newcommand{\sigmoid}{\sigma}
\newcommand{\softplus}{\zeta}
\newcommand{\KL}{D_{\mathrm{KL}}}
\newcommand{\Var}{\mathrm{Var}}
\newcommand{\standarderror}{\mathrm{SE}}
\newcommand{\Cov}{\mathrm{Cov}}
% Wolfram Mathworld says $L^2$ is for function spaces and $\ell^2$ is for vectors
% But then they seem to use $L^2$ for vectors throughout the site, and so does
% wikipedia.
\newcommand{\normlzero}{L^0}
\newcommand{\normlone}{L^1}
\newcommand{\normltwo}{L^2}
\newcommand{\normlp}{L^p}
\newcommand{\normmax}{L^\infty}

\newcommand{\parents}{Pa} % See usage in notation.tex. Chosen to match Daphne's book.

\DeclareMathOperator*{\argmax}{arg\,max}
\DeclareMathOperator*{\argmin}{arg\,min}

\DeclareMathOperator{\sign}{sign}
\DeclareMathOperator{\Tr}{Tr}
\let\ab\allowbreak



\usepackage{url}
\usepackage{hyperref}

\usepackage{graphicx}
\usepackage{subfig}
\usepackage{tikz}
\usepackage{tikz-dependency}

\usepackage{array}
\usepackage{multirow}
\usepackage{amsmath}
\usepackage{amsthm}    % 定理环境
\usepackage{color}    % 临时用于高亮

\usepackage{algorithm}
\usepackage{algorithmic}

\newtheorem{theorem}{Theorem}
\newtheorem{lemma}{Lemma}
\newtheorem{corollary}{Corollary}
\newtheorem{proposition}{Proposition}
\newtheorem{remark}{Remark}



% \usepackage{longtable}
% \usepackage[table]{xcolor}
\usepackage{colortbl}
\usepackage{todonotes}

\setlength{\textfloatsep}{4mm} % algorithm 与正文之间的间距


\def\etal{\textit{et al}.}
\def\ie{\textit{i.e.}}
\def\eg{\textit{e.g.}}
\def\etc{\textit{etc}}
\def\wrt{\textit{w.r.t. }}



\title{RecDreamer: Consistent Text-to-3D \\Generation via Uniform Score Distillation}

% Authors must not appear in the submitted version. They should be hidden
% as long as the \iclrfinalcopy macro remains commented out below.
% Non-anonymous submissions will be rejected without review.

% \author{Chenxi~Zheng, Yihong~Lin, Bangzhen~Liu, Xuemiao~Xu, Yongwei~Nie, Shengfeng~He \thanks{ Use footnote for providing further information
% about author (webpage, alternative address)---\emph{not} for acknowledging
% funding agencies.  Funding acknowledgements go at the end of the paper.} \\
% Department of Computer Science\\
% Cranberry-Lemon University\\
% Pittsburgh, PA 15213, USA \\
% \texttt{\{hippo,brain,jen\}@cs.cranberry-lemon.edu} \\
% \And
% Ji Q. Ren \& Yevgeny LeNet \\
% Department of Computational Neuroscience \\
% University of the Witwatersrand \\
% Joburg, South Africa \\
% \texttt{\{robot,net\}@wits.ac.za} \\
% \AND
% Coauthor \\
% Affiliation \\
% Address \\
% \texttt{email}
% }

\renewcommand{\thefootnote}{\fnsymbol{footnote}}
% \footnotetext[2]{\url{https://github.com/chansey0529/LSO}}
\footnotetext[2]{Corresponding authors. Code: \texttt{https://github.com/chansey0529/RecDreamer}.}

\author{Chenxi~Zheng\textsuperscript{1}, Yihong~Lin\textsuperscript{1}, Bangzhen~Liu\textsuperscript{1}, Xuemiao~Xu\textsuperscript{1}\footnotemark[2] , Yongwei~Nie\textsuperscript{1}\footnotemark[2] , Shengfeng~He\textsuperscript{2}\\
\textsuperscript{1}South China University of Technology, \textsuperscript{2}Singapore Management University\\
\{chansey0529, amcsyihonglin, liubz.scut\}@gmail.com,
\{xuemx, nieyongwei\}@scut.edu.cn,\\
shengfenghe@smu.edu.sg
}

% The \author macro works with any number of authors. There are two commands
% used to separate the names and addresses of multiple authors: \And and \AND.
%
% Using \And between authors leaves it to \LaTeX{} to determine where to break
% the lines. Using \AND forces a linebreak at that point. So, if \LaTeX{}
% puts 3 of 4 authors names on the first line, and the last on the second
% line, try using \AND instead of \And before the third author name.

\newcommand{\fix}{\marginpar{FIX}}
\newcommand{\new}{\marginpar{NEW}}

% \def\lyh{\textcolor{black}}



% \definecolor{customred}{HTML}{A80404}
\definecolor{customred}{HTML}{cc2936}
% \definecolor{customred}{HTML}{c3272b}
% \newcommand{\rebpara}{\color{customred}}

% \def\reb{\textcolor{customred}}

\iclrfinalcopy % Uncomment for camera-ready version, but NOT for submission.
\begin{document}


\maketitle

\begin{abstract}
   % original version
Current text-to-3D generation methods based on score distillation often suffer from geometric inconsistencies, leading to repeated patterns across different poses of 3D assets. This issue, known as the Multi-Face Janus problem, arises because existing methods struggle to maintain consistency across varying poses and are biased toward a canonical pose. While recent work has improved pose control and approximation, these efforts are still limited by this inherent bias, which skews the guidance during generation.
To address this, we propose a solution called RecDreamer, which reshapes the underlying data distribution to achieve a more consistent pose representation. The core idea behind our method is to rectify the prior distribution, ensuring that pose variation is uniformly distributed rather than biased toward a canonical form. By modifying the prescribed distribution through an auxiliary function, we can reconstruct the density of the distribution to ensure compliance with specific marginal constraints. In particular, we ensure that the marginal distribution of poses follows a uniform distribution, thereby eliminating the biases introduced by the prior knowledge.
We incorporate this rectified data distribution into existing score distillation algorithms, a process we refer to as uniform score distillation. To efficiently compute the posterior distribution required for the auxiliary function, RecDreamer introduces a training-free classifier that estimates pose categories in a plug-and-play manner. Additionally, we utilize various approximation techniques for noisy states, significantly improving system performance.
Our experimental results demonstrate that RecDreamer effectively mitigates the Multi-Face Janus problem, leading to more consistent 3D asset generation across different poses.


\end{abstract}



\section{Introduction}

Node classification is a fundamental task in graph analysis, with a wide range of applications such as item tagging \cite{Mao2020ItemTF}, user profiling \cite{Yan2021RelationawareHG}, and financial fraud detection \cite{Zhang2022eFraudComAE}. Developing effective algorithms for node classification is crucial, as they can significantly impact commercial success. For instance, US banks lost 6 billion USD to fraudsters in 2016. Therefore, even a marginal improvement in fraud detection accuracy could result in substantial financial savings.

Given its practical importance, node classification has been a long-standing research focus in both academia and industry. The earliest attempts to address this task adopted techniques such as Laplacian regularization \cite{belkin2006manifold}, graph embeddings \cite{yang2016revisiting}, and label propagation \cite{zhu2003semi}. Over the past decade, GNN-based methods have been developed and have quickly become prominent due to their superior performance, as demonstrated by works such as \citet{kipf2017GCN}, \citet{velickovic2018GAT}, and \citet{hamilton2017SAGE}. Additionally, the incorporation of encoded textual information has been shown to further complement GNNs' node features, enhancing their effectiveness \cite{jin2023patton, zhao2022GLEM}.

Inspired by the recent success of LLMs, there has been a surge of interest in leveraging LLMs for node classification \cite{li2023survey}. LLMs, pre-trained on extensive text corpora, possess context-aware knowledge and superior semantic comprehension, overcoming the limitations of the non-contextualized shallow embeddings used by traditional GNNs. Typically, supervised methods fall into three categories: Encoder, Reasoner, and Predictor. In the Encoder paradigm, LLMs employ their vast parameters to encode nodes' textual information, producing more expressive features that surpass shallow embeddings \cite{Zhu2024ENGINE}. The Reasoner approach utilizes LLMs' reasoning capabilities to enhance node attributes and the task descriptions with a more detailed text \cite{chen2024exploring, he2023TAPE}. This generated text augments the nodes' original information, thereby enriching their attributes. Lastly, the Predictor role involves LLMs integrating graph context through graph encoders, enabling direct text-based predictions  \cite{chen23llaga,tang2023graphgpt,chai2023graphllm,Huang2024GraphAdapter}. For zero-shot learning with LLMs, methods can be categorized into two types: Direct Inference and Graph Foundation Models (GFMs). Direct Inference involves guiding LLMs to directly perform classification tasks via crafted prompts \cite{Huang2023CanLE}. In contrast, GFMs entail pre-training on extensive graph corpora before applying the model to target graphs, thereby equipping the model with specialized graph intelligence \cite{li2024zerog}. An illustration of these methods is shown in Figure \ref{fig:llm_role}. 

Despite tremendous efforts and promising results, the design principles for LLM-based node classification algorithms remain elusive. Given the significant training and inference costs associated with LLMs, practitioners may opt to deploy these algorithms only when they provide substantial performance enhancements compared to costs. This study, therefore, seeks to identify \textbf{(1) the most suitable settings for each algorithm category, and (2) the scenarios where LLMs surpass traditional LMs such as BERT}. While recent work like GLBench \cite{Li2024GLBench} has evaluated various methods using consistent data splits in semi-supervised and zero-shot settings, differences in backbone architectures and implementation codebases still hinder fair comparisons and rigorous conclusions. To address these limitations, we introduce a new benchmark that further standardizes backbones and codebases. Additionally, we extend GLBench by incorporating three new E-Commerce datasets relevant to practical applications and expanding the evaluation settings. Specifically, we assess the impact of supervision signals (e.g., supervised, semi-supervised), different language model backbones (e.g., RoBERTa, Mistral, LLaMA, GPT-4o), and various prompt types (e.g., CoT, ToT, ReAct). These enhancements enable a more detailed and reliable analysis of LLM-based node classification methods. In summary, our contributions to the field of LLMs for graph analysis are as follows:


% A fair comparison necessitates a benchmark that evaluates all methods using consistent data splitting ratios, learning paradigms, backbone architectures, and implementation codebases. A very recent work, GLBench~\cite{Li2024GLBench}, tested various methods on several datasets in a semi-supervised/zero-shot setting, maintaining the same data splits. However, differences in the underlying backbones and implementation codebases still pose challenges for a fair comparison and drawing rigorous conclusions of the above questions. This paper introduces a benchmark that further standardizes the backbones and implementation codebases. Moreover, we expand upon GLBench by providing additional datasets and evaluation settings. Specifically, we include three new datasets from the E-Commerce sector, which are more relevant for practical commercial applications. We also assess the influence of supervision signals (e.g., supervised or semi-supervised), various language model backbones (e.g., RoBERTa, Mistral, GPT-4o), and prompts (e.g., CoT, ToT, and ReAct). These datasets and settings enable a detailed analysis of the aforementioned questions. 



% However, existing works lack the necessary standardization for such comparisons. An algorithm that performs exceptionally well in its original paper might underperform when used as a baseline in subsequent studies. This discrepancy often arises from variations in data splitting, learning paradigms, backbone architectures, and implementation codebases.  The backbone architecture and implementations are adopted from the original papers, which 

% To address this issue, this paper introduces a testbed for LLM-based node classification algorithms and conducts extensive experiments to derive insights and guidelines. 

\begin{itemize}
    \item \textbf{A Testbed:} We release LLMNodeBed, a PyG-based testbed designed to facilitate reproducible and rigorous research in LLM-based node classification algorithms. The initial release includes ten datasets, eight LLM-based algorithms, and three learning configurations. LLMNodeBed allows for easy addition of new algorithms or datasets, and a single command to run all experiments, and to automatically generate all tables included in this work.
    
    \item \textbf{Comprehensive Experiments:} By training and evaluating over 2,200 models, we analyzed how the learning paradigm, homophily, language model type and size, and prompt design impact the performance of each algorithm category.
    
    \item \textbf{Insights and Tips:} Detailed experiments were conducted to analyze each influencing factor. We identified the settings where each algorithm category performs best and the key components for achieving this performance. Our work provides intuitive explanations, practical tips, and insights about the strengths and limitations of each algorithm category.
\end{itemize}




%It has been a research focus in both academia and industry due to its wide range of applications, including item tagging \cite{Mao2020ItemTF}, user profiling \cite{Yan2021RelationawareHG}, and financial fraud detection \cite{Zhang2022eFraudComAE}. 


%Building effective algorithms for node classification is a long-standing topic as it has a direct impact on commercial success \cite{Lo2022InspectionLSG}.

%Before the popularity of LLMs, node classification is typically tackled by graph neural networks (GNNs) or language models (LMs) such as BERT \cite{Devlin2019BERTPO}. GNNs \cite{kipf2017GCN,velickovic2018GAT,hamilton2017SAGE} enhance node representations by aggregating information from neighboring nodes, thereby capturing the structural context essential for accurate classification. In contrast, LMs \cite{Wang2022e5-large, Liu2019roberta} focus on semantic representations by encoding the textual information associated with each node, transforming the node classification into a text classification task. The encoded textual information can further complement GNNs' node features \cite{jin2023patton, zhao2022GLEM}. Yifei: I think the current intro is too long, to move it to related works

%Over the past decade, we have witnessed great progress in node classification algorithms. The classical ones include Graph Neural Networks (GNNs) \cite{kipf2017GCN,velickovic2018GAT,hamilton2017SAGE} and additional language modeling to enhance the node features \cite{jin2023patton, zhao2022GLEM}. Recently, there has been a surge of interest in applying LLMs for node classification \cite{li2023survey}. In these studies, the roles performed by LLMs can be primarily 


% Despite the importance of this area, the literature of LLM-based node classification is scattered: the algorithms are evaluated under different datasets, learning paradigms, baselines, and implementation codebases. The purpose of this work is to perform rigorous comparisons among algorithms, as well as to open-source our software for anyone to replicate and extend our analysis. This manuscript investigates the question: \emph{How useful are LLMs for node classification under a fair setting?}

% To answer this question, we implement and tune eight LLM-based node classification algorithms, to compare them across ten datasets and three learning paradigms.  There are four major takeaways from our investigations: (1) \textbf{LLM-as-Encoder is effective for low-homophily graphs:} These methods outperform classic LM counterparts on low-homophily graphs, with the advantages being more obvious under limited supervision.
% (2) \textbf{LLM-as-Reasoner is the most effective when LLMs have prior knowledge of the target graph:} These methods achieve superior performance on datasets where the LLMs possess prior knowledge like academic and web link datasets, and benefit from more powerful models like GPT-4o. 
% (3) \textbf{LLM-as-Predictor methods is highly effective when labeled data is abundant}: Predictor methods require extensive supervision for model training, with their performance improving as larger LLMs adhering to scaling laws \cite{Kaplan2020ScalingLF} are utilized. Among different LLMs, Mistral-7B \cite{Jiang2023Mistral7B} consistently serves as a robust backbone. (4) \textbf{Zero-shot methods are most effective when neighbor information is injected:} Although Graph Foundation Models (GFMs) \cite{liu2023one, li2024zerog, Zhu2024GraphCLIPET} outperform open-source LLMs in zero-shot settings, they still lag behind advanced models like GPT-4o. The most effective zero-shot approaches involve injecting neighbor information to guide LLMs for direct inference.

% As a result of this paper, we release LLMNodeBed, a PyTorch-based testbed designed to facilitate reproducible and rigorous research in node classification algorithms. The initial release includes ten datasets, eight algorithms, three learning configurations, and the infrastructure to run all experiments. Our experimental framework can be easily extended to include new methods and datasets. We are committed to updating this repository with new algorithms and datasets and welcome pull requests from fellow researchers to ensure its ongoing development.


%While a myriad of algorithms exists, diverse datasets, architectures, learning configurations, and implementation codebases, rendering fair and realistic comparisons difficult and conclusions inconsistent. Inspired by standardized benchmarks in computer vision like ImageNet, this paper conducts a rigorous comparison of various LLM-based node classification methods to assess the true efficacy of LLMs. This investigation addresses the following research question:

%\textit{Under What Circumstances do LLMs Help Node Classification Task?}

%At a first step, we implement LLMNodeBed, a codebase and testbed for node classification with LLMs. It includes ten multi-domain graph datasets with varying scales and levels of homophily, supports eight representative algorithms that represent diverse LLM roles, and offers three learning configurations: semi-supervised, fully-supervised, and zero-shot. Through extensive experiments, we provide empirical insights into when LLMs contribute to node classification performance: 



% In summary, we make the following contributions: 

% \begin{enumerate}
%     \item \textbf{LLMNodeBed:} We introduce LLMNodeBed, a comprehensive and extensible testbed for evaluating LLM-based node classification algorithms. It comprises ten datasets, eight representative algorithms, and three learning scenarios, and can easily accommodate new datasets, methods, and backbones.
%     \item \textbf{Comprehensive Evaluation:} We conduct extensive empirical analysis across different datasets, algorithms, and learning settings to elucidate the efficacy of different LLM roles in node classification performance. 
%     \item \textbf{Practical Guidelines:} Based on our findings, we provide actionable guidelines for effectively applying LLMs to diverse real-world node classification tasks, enhancing their performance and applicability in various scenarios.
% \end{enumerate}

% \section{Background}

% In this work, we focus on two different model families: random Fourier features (RFFs) and deep neural networks (DNNs) for transfer learning with informative priors.
% What these model families have in common is that they can be overparameterized.

%\subsection{Random Fourier features}

% MCH: MOVED TO CASE A

%\subsection{Transfer learning with informative priors}

% MCH: MOVED TO CASE B




\section{Method}


The primary goal of our \textit{RecDreamer} is to mitigate the Multi-Face Janus problem through rectification of underlying data distribution in the pre-trained diffusion models. In the following sections, we will first theoretically illustrate the idea of how we rectify the data density via an auxiliary function to ensure a uniform pose distribution~(Sec.~\ref{method:rec}). Based on the former theoretical analysis, we introduce a \textit{uniform score distillation} approach for optimizing 3D representations in aligning with the rectified distribution~(Sec.~\ref{method:usd}). Furthermore, a series of designed components for implementing the auxiliary function is detailly discussed in Sec.~\ref{method:recdreamer}, including a pose classifier, approximation of the posterior distribution of pose, and estimation of pose-relevant statistics.



\subsection{Rectification of Data Distribution}\label{method:rec}


To directly analyze the relationship between data and pose, we eliminate redundant variables and simplify the text-conditioned probability $p_t(\boldsymbol{x}_t|y)$ to an unconditional density $p(\boldsymbol{x})$, removing the influence of the time step.
We denote the data with a general variable $\boldsymbol{x}$.
Assuming that $p(\boldsymbol{x}, c)$ represents the joint distribution, the pose distribution can be expressed as $p(c) = \int p(\boldsymbol{x}, c) \mathrm{d}\boldsymbol{x} = \int p(\boldsymbol{x}) p(c|\boldsymbol{x}) \mathrm{d}\boldsymbol{x}$, which is not a uniform distribution.
To mitigate this bias, we frame the simplified problem as follows: given the data distribution $p(\boldsymbol{x})$ and the target attribute distribution $f(c)$, \textit{how can we adjust $p(\boldsymbol{x})$ to a new distribution $\tilde{p}(\boldsymbol{x})$ such that $\tilde{p}(c) = \int \tilde{p}(\boldsymbol{x}) p(c|\boldsymbol{x}) \mathrm{d}\boldsymbol{x}=f(c)$ holds.}


By introducing a weighting function to the joint probability $p(\boldsymbol{x}, c)$, we establish that the original data density can be adjusted as follows.

\begin{theorem}[Proof in Appendix~\ref{app:method_theorem}]\label{thm:rpx}
 Let $p(\boldsymbol{x})$ denote the data density, $p(c | \boldsymbol{x})$ the conditional distribution of the attribute $c$ given data $\boldsymbol{x}$, and $p(c)$ the marginal distribution of $c$ induced by $p(\boldsymbol{x})$. Given a target distribution $f(c)$ for the attribute $c$, we can construct a new data density $\tilde{p}(\boldsymbol{x})$ such that the marginal distribution of $c$ under $\tilde{p}(\boldsymbol{x})$ matches the target distribution $f(c)$. This new density is given by:
    \begin{equation}
 \tilde{p}(\boldsymbol{x}) = p(\boldsymbol{x}) \int \frac{f(c)}{p(c)} p(c | \boldsymbol{x}) \, dc.
    \end{equation}
\end{theorem}

Theorem~\ref{thm:rpx} reveals that the new data density that features a uniformly distributed marginal $f(c)$ can be computed by the original data distribution and an auxiliary function. Furthermore, Theorem~\ref{thm:rpx} can be naturally extended to conditional distributions, as demonstrated in Corollary~\ref{cr:rpx_cond} (see Appendix~\ref{app:method_theorem}). So far, we have derived the rectified distribution for clean images, $\tilde{p}(\boldsymbol{x}_0|y)$.

However, since score distillation operates in the noise space, our ultimate goal is to reach the rectified density of the noisy data. Given the transition $p_t(\boldsymbol{x}_t|y) = \int p_0(\boldsymbol{x}_0|y)p_{t0}(\boldsymbol{x}_t|\boldsymbol{x}_0)\mathrm{d}\boldsymbol{x}_0$ where $p_{t0}(\boldsymbol{x}_t|\boldsymbol{x}_0)=\mathcal{N}(\boldsymbol{x}_t|\alpha_t\boldsymbol{x}_0,\sigma_t^2\boldsymbol{I})$, we prove that the rectified distributions for any time step share a unified form, as presented in the following theorem.


\begin{theorem}[Proof in Appendix~\ref{app:method_theorem}]\label{thm:rpx0t_cond}
 For any $t \sim \mathcal{U}[0, T]$, the rectified density of $\boldsymbol{x}_t$ is given by:
    \begin{equation}\label{eq:rpx0t_cond}
 \tilde{p}_t(\boldsymbol{x}_t|y) = p(\boldsymbol{x}_t|y) \int \frac{f(c|y)}{p_t(c|y)} p(c | \boldsymbol{x}_t, y) dc.
    \end{equation}
\end{theorem}
Theorem~\ref{thm:rpx0t_cond} reveals that the noisy density of the rectified text-to-image distribution can be expressed as the original noisy density multiplied by an auxiliary function, denoted as $r(\boldsymbol{x}_t|y)$. Specifically, $r(\boldsymbol{x}_t|y) = \int \frac{f(c|y)}{p_t(c|y)} p(c | \boldsymbol{x}_t, y) dc$.

\subsection{Uniform Score Distillation}\label{method:usd}


We now return to the original variational distillation problem. First, we define a set of 3D representations $\{\theta^i\}_{i=0}^n$, also named particles in the later gradient flow simulation. Given the distribution $\mu(\theta|y)$ composed of the set $\{\theta^i\}_{i=0}^n$, the camera pose $c$, and the text prompt $y$, the distribution of noisy rendered images is computed as $q_t^\mu(\boldsymbol{x}_t|c,y) = \int q_0^\mu(\boldsymbol{x}_0|c,y)p_{t0}(\boldsymbol{x}_t|\boldsymbol{x}_0)\mathrm{d}\boldsymbol{x}_0$, where $\boldsymbol{x}_0=\boldsymbol{g}(\theta,c)$. Given the rectified distribution $\tilde{p}_t(\boldsymbol{x}_t|y)$, the objective is as follows:
\begin{equation}\label{eq:usd_rkl}
 \min_\mu \mathbb{E}_{t,c}\left[(\sigma_t/\alpha_t)\omega(t)D_{\mathrm{KL}}(q_t^\mu(\boldsymbol{x}_t|c,y)\parallel \tilde{p}_t(\boldsymbol{x}_t|y))\right].
\end{equation}
We refer to this as \textit{uniform score distillation} (USD), as it seeks to approximate the score of the rectified distribution, which is uniformly distributed across the camera poses. To optimize the particles, we derive a corollary based on Theorem 2 from VSD~\citep{wang2024prolificdreamer}:
\begin{corollary}[Corollary to Theorem 2 from VSD]\label{cr:usd_gradient}
 For Wasserstein gradient flow minimizing~\eqref{eq:usd_rkl}, the gradient for the particles is given by:
    \begin{equation}\label{eq:usd_grad}
        \nabla_\theta\mathcal{L}_\text{USD} = \nabla_\theta \mathcal{L}_\text{VSD}^\prime(\theta) - \mathbb{E}_{t,\boldsymbol{\epsilon},c}\left[\omega(t)\frac{\sigma_t}{\alpha_t}\nabla_{\theta}\log r(\boldsymbol{x}_t|y)\right],
    \end{equation}
 where
    \begin{equation}
        \nabla_\theta\mathcal{L}_\text{VSD}^\prime = \mathbb{E}_{t,\boldsymbol{\epsilon},c}\left[\omega(t)\left(\boldsymbol{\epsilon}_\text{pretrain}(\boldsymbol{x}_t,t,y)-\boldsymbol{\epsilon}_\phi(\boldsymbol{x}_t,t,c,y)\right)\frac{\partial\boldsymbol{g}(\theta,c)}{\partial\theta}\right],
    \end{equation}    
 and $\boldsymbol{x}_t=\alpha_t\boldsymbol{g}(\theta,c)+\sigma_t\boldsymbol{\epsilon}$.
\end{corollary}




Since the rectification algorithm is based on reweighting the sub-distributions, it cannot generate content that was not present in the original distribution. To ensure that $y$ provides a comprehensive distribution including contents of multiple perspectives, we detail the construction techniques in the Appendix~\ref{app:method_others}.


In the optimization process, we follow VSD by iteratively optimizing the U-Net $\epsilon_{\phi}$ and the particles $\{\theta^i\}_{i=0}^n$ using~\eqref{eq:vsd_lora} and~\eqref{eq:usd_rkl}.

\subsection{RecDreamer}\label{method:recdreamer}

The previous sections derive the analytical solution of the rectified distribution and introduce a parameter optimization scheme based on score distillation.
To apply this scheme to optimize the 3D scene, we must also compute the rectification function $r(\boldsymbol{x}_t|y)$.
We design an effective \textit{classifier to accurately categorize image poses}.
Finally, we account for the effects of noisy states and \textit{estimate the posterior distribution of noisy images and its expected value}.



\begin{figure}[t]
    \centering
    \includegraphics[width=1\linewidth]{resource/classifier/classifier_pipeline.pdf} % Built-in example image
    \caption{The architecture of our classifier combines orientation and texture similarities in a differential ``and-gate'' manner. Orientation similarity is evaluated using a patch-matching distance metric, while texture similarity is calculated via cosine similarity of the $[cls]$ token.}
    % \vspace{-3mm}
    \label{fig:app_method_classifier}
\end{figure} 

\textbf{Discretization of the pose space.}
The crux to computing the auxiliary function $r(\boldsymbol{x}_t|y)$ lies in estimating both $p(c | \boldsymbol{x}_t, y)$ and $p_t(c | y)$.
Since $p_t(c | y)= \mathbb{E}_{\boldsymbol{x}_t \sim p(\boldsymbol{x}_t|y)}p(c | \boldsymbol{x}_t, y)$ is a term that depends on $p(c | \boldsymbol{x}_t, y)$, we begin by analyzing $p(c | \boldsymbol{x}_t, y)$, which can be interpreted as a pose estimator of noisy images.
However, obtaining an estimator for noisy images requires additional data and fine-tuning. To address this, we relate the noisy predictor to the clean predictor by following the DPS~\citep{chung2022diffusion} formulation: $p(c | \boldsymbol{x}_t, y) = \int p(\boldsymbol{x}_0|\boldsymbol{x}_t,y) p(c|\boldsymbol{x}_0,y) d \boldsymbol{x}_0$.
Thus, we prioritize the design of a clean estimator $p(c|\boldsymbol{x}_0, y)$ before tackling the noisy case.


Instead of explicitly estimating the camera's extrinsic parameters, we propose modeling a simplified pose by categorizing the images into broad pose categories, such as ``front'', ``back'', ``left'' and ``right''. In the context of USD, these global categories help maintain a rough balance between different poses and promote 3D consistency. Accordingly, we define the auxiliary function in a discrete form as follows: $r_\xi(\boldsymbol{x}_t|y) = \sum_{\bar{c}} \frac{f(\bar{c}|y)}{p_t(\bar{c}|y)} p_\xi(\bar{c} | \boldsymbol{x}_t, y)$, where $\bar{c}$ represents the discrete pose category, $f(\bar{c}|y) \sim \mathcal{U}\{\bar{c}_i\}_{i=0}^{k}$, and $p_\xi$ is the parameterized classifier.


\textbf{Pose classifier.} Building on this formulation, our goal is to create a lightweight pose classifier without the need for training. To achieve this, we propose a matching-based pose classifier that leverages a pretrained feature extractor and user-provided image templates for each category. Given an input image, the class probabilities are computed by assessing the similarity between the input and the templates. Empirically, the main challenge is distinguishing between 2D orientations (i.e., ``left-middle-right'') and classifying textures (i.e., ``front-back'').


To address this, we compute the overall similarity by combining orientation similarity and texture similarity in a differential ``and-gate'' manner. The pipeline of our classifier is shown in Fig.~\ref{fig:app_method_classifier}. Drawing inspiration from dense matching techniques~\citep{zhang2024telling, zhang2024tale}, we propose using a patch-matching distance metric to evaluate orientation similarity. Texture similarity is determined by calculating the cosine similarity of the $[cls]$ token between the input and template images. Orientation and texture similarities are then multiplied after normalization. Finally, the combined similarity is normalized using a low-temperature softmax function~\citep{goodfellow2016deep}. For more details on the patch-matching distance and the architecture, please refer to Appendix~\ref{app:method_classifier}.


\textbf{Estimating $p(c|\boldsymbol{x}_t, y)$ and $p_t(c|y)$.}
By establishing the calculation of $p(c|\boldsymbol{x}_0, y)$ with a plug-and-play pose classifier, we can now introduce the computation of $p(c | \boldsymbol{x}_t, y)$ and $p_t(c | y)$.
To compute $p(c | \boldsymbol{x}_t, y) = \int p(\boldsymbol{x}_0|\boldsymbol{x}_t,y) p(c|\boldsymbol{x}_0,y) d \boldsymbol{x}_0$, we follow DPS~\citep{chung2022diffusion} by replacing the calculation of probability with expectation $\mathbb{E}_{\boldsymbol{x}_0{\sim}p(\boldsymbol{x}_0|\boldsymbol{x}_t,y)}p(c|\boldsymbol{x}_0, y)$ and further approximating the expectation with Tweedie's formula~\citep{robbins1992empirical}. Formally, $p(c | \boldsymbol{x}_t, y)\approx p(c|\hat{\boldsymbol{x}}_0, y)$, where $\hat{\boldsymbol{x}}_0 = \left(\boldsymbol{x}_t-\sigma_t\epsilon_{pretrain}(\boldsymbol{x}_t, t, y)\right)/\alpha_t$. Beyond this approximation, we provide an on-the-fly estimate of the marginal density $p_t(c | y)$, avoiding any form of distribution estimation~\citep{robert1999monte}. Concretely, since $p_t(c | y)$ is the expected value of $p(c | \boldsymbol{x}_t, y)$ over $\boldsymbol{x}_t$, we update a distribution $\bar{p}_t(\bar{c} | y)$ using exponential moving average (EMA) of $p(c | \boldsymbol{x}_t, y)$ during optimization, with an update rate $\alpha_{ema}$, to approximate $p(c | \boldsymbol{x}_t, y)$. To enable the in-time estimate of the current pose distribution, we propose a time-interval EMA to capture the distribution. Technical details are left in Appendix~\ref{app:method_recfunc}.



The proposed scheme allows for the accurate estimation of the auxiliary function $r_\xi$, facilitating the adjustment of the initial distribution so that the sampling results align with the assumption of a uniform pose distribution. The implementation of uniform score distillation is presented in Algorithm~\ref{alg:usd}, and we refer to this systematic approach as \textit{RecDreamer}.


\begin{algorithm}[t]
    % \vspace{-3mm}
    \caption{Uniform Score Distillation}
    \begin{algorithmic}[1]\label{alg:usd}
        \REQUIRE A pretrained diffusion model $\epsilon_{pretrain}$, a noise predictor $\epsilon_\phi$ with optimizable parameters $\phi$, a set of particles $\{\theta^i\}_{i=0}^n$, a text prompt $y$, learning rates $\eta_1$ and $\eta_2$, a rectify function $r_\xi$ and a classifier $p_{\xi}(\bar{c}|\boldsymbol{x}_{t},y)$ parameterized by $\xi$, the number of discrete pose categories $n_{\bar{c}}$, the number of time steps $n_{\bar{t}}$, EMA update rate $\alpha_{ema}$.

        Initialize the EMA probabilities $\{\bar{p}_t(\bar{c}|y)\}_{t=0}^{n_t}$, with $\bar{p}_t(\bar{c}|y) = 1 / n_{\bar{c}}$.

        \WHILE {not converged}
            \STATE Randomly sample $\{\theta^i\}_{i=0}^n$ and $c$, render the image $\boldsymbol{x}_0=\boldsymbol{g}(\theta,c)$.
            \STATE Apply a forward step $\boldsymbol{x}_t=\mathcal{N}(\boldsymbol{x}_t|\alpha_t\boldsymbol{x}_0,\sigma_t^2\boldsymbol{I})$
            \STATE $\theta\leftarrow\theta-\eta_1\mathbb{E}_{t,\boldsymbol{\epsilon},c}\left[\omega(t)\left(\boldsymbol{\epsilon}_{\mathrm{pretrain}}(\boldsymbol{x}_t,t,y)-\boldsymbol{\epsilon}_\phi(\boldsymbol{x}_t,t,c,y)\right)\frac{\partial\boldsymbol{g}(\theta,c)}{\partial\theta}\right]$ \\
            \hspace{10mm} $+ \eta_1\mathbb{E}_{t,\boldsymbol{\epsilon},c}\left[\omega(t)\frac{\sigma_t}{\alpha_t}\nabla_{\theta}\log r_\xi (\boldsymbol{x}_t|y)\right]$
            \STATE $\bar{p}_t(\bar{c}|y) \leftarrow \alpha_{ema}p_{\xi}(\bar{c}|\boldsymbol{x}_{t},y) + (1 - \alpha_{ema})\bar{p}_t(\bar{c}|y)$
            \STATE $\phi\leftarrow\phi-\eta_2\nabla_\phi\mathbb{E}_{t,\epsilon}||\boldsymbol{\epsilon}_\phi(\boldsymbol{x}_t,t,c,y)-\boldsymbol{\epsilon}||_2^2.$
        \ENDWHILE
        \RETURN
    \end{algorithmic}
\end{algorithm}












\section{Experiments
\label{sec:experiments}
}

\begin{figure*}[t]
\centering
\begin{tabular}{ccc}
\includegraphics[scale=0.29]
{toy_prediction_exact.pdf} &
\includegraphics[scale=0.29]
{toy_prediction_trace.pdf} &
\includegraphics[scale=0.29]
{toy_prediction_log.pdf} \\
(a) & (b) & (c) \\                
\includegraphics[scale=0.29]
{toy_all_predictions.pdf} &
\includegraphics[scale=0.25]
{toy_all_losses.pdf} &
\includegraphics[scale=0.25]
{toy_all_variances.pdf} \\
(d) & (e) & (f)              
\end{tabular}
\caption{First row shows posterior predictions (means with 2-standard deviations) after
  fitting the exact GP (a), and the sparse GPs with either the standard collapsed SGPR bound (b) or the proposed SGPR-new collapsed bound (c). In panels (b),(c) the seven inducing points are intiliazed to the same random locations (shown on top with crosses) while the optimized values are shown at the bottom.
  Panel (d) superimposes all predictions in order to provide a more comparative visualization.
  Finally, panel (e) shows the ELBO (or exact log marginal likelihood for the exact GP) values across optimization steps while (f) shows the corresponding values for the noise variance $\sigma^2$.}
\label{fig:toy}
\end{figure*}


\subsection{Illustration in 1-D Regression}

In the first regression experiment we consider the 1-D  Snelson dataset \cite{Snelson2006}. We took a subset of 40 examples of this dataset and we fitted the exact GP with the squared exponential kernel $k(x, x') = \sigma_f^2 \exp( - \frac{ (x - x')^2}{2 \ell^2})$. We also fitted sparse variational GPs %, denoted as SGPR, 
with either the standard collapsed bound \cite{titsias2009variational} from \Cref{eq:collapsedbound_old} (SGPR) or the new collapsed bound from \Cref{eq:newcollapsedbound} (SGPR-new).
Both sparse GP methods use seven inducing points initialized at the
same values as shown in Figure \ref{fig:toy}. All methods are initialized to the same hyperparameter values; see \Cref{app:furtherresults}.

Figure \ref{fig:toy} shows the results. %Specifically,
Note that both SGPR and SGPR-new find similar inducing point locations. But SGPR-new,  as a tighter bound (see panel (e)), is able to reduce some bias when estimating
the hyperparameters since it finds a noise variance $\sigma^2$ closer to the one by exact GP (see panel (f)).  
This results in better predictions that match better the exact GP, as shown by the comparative visualization in panel (d). From panel (d), observe that both the mean and variances of SGPR-new are closer to the exact GP than SGPR.  


\subsection{Medium Size Regression Datasets
\label{sec:mediumregress}
}

To further investigate the findings from the previous section, we consider three medium size real-world UCI regression datasets (Pol, Bike, and Elevators)
with roughly 10k training data points each, and for which we can still run the exact GP. We choose the ARD squared exponential kernel $k(\bx, \bx') = \sigma_f^2 \exp( - \sum_{i=1}^d \frac{(x_i - x_i')^2}{2 \ell_i^2})$.
We run all three previous methods (Exact GP, SGPR, SGPR-new) five times with different random train-test splits;
see \Cref{app:furtherresults} for experimental details. We also include
in the comparison a fourth method (discussed in Related Work)
which is the \citet{artemevburt2021cglb}'s bound  (SGPR-artemev) that does training using the collapsed bound from \Cref{eq:artemvecollapsedbound} in \Cref{app:artemevbound}. 
All sparse GP methods use $M=1024$ or $M=2048$ inducing points initialized by k-means.  Figure \ref{fig:mediumsize1024} shows the objective function and the noise variance $\sigma^2$ across $10k$ optimization steps using Adam with base learning rate $0.01$ and for $M=1024$.  \Cref{fig:mediumsize2048} in \Cref{app:mediumsizeRegress} shows the corresponding plots for $M=2048$.  We observe that for Pol and Bike, SGPR-new matches closer the exact GP training than SGPR and SGPR-artemev. Specifically, SGPR-new gives higher ELBO and estimates the noise variance with reduced underfitting bias.
For the Elevators dataset, $M=1024$ inducing points were enough for sparse GP methods to closely match exact GP training. This happens because in this case $\bQ_{\f \f}$ accurately approximates $\bK_{\f \f}$, i.e., the elements $k_{ii} - q_{ii}$ get close to zero. Table \ref{table:smalldatasetsTestLL} reports test log-likelihood predictions which show that 
SGPR-new outperforms SGPR and SGPR-artemev.  

\begin{table}[t]
  \caption{Average test log-likelihoods for the medium size regression datasets.
  The numbers in parentheses are standard errors.
    %The SGPR methods used $M=1024$ inducing points.
  }
\label{table:smalldatasetsTestLL}
\vskip 0.15in
%\begin{small}
\begin{center}
%  \begin{sc}
\resizebox{\linewidth}{!}{%
\begin{tabular}{lcccr}
\toprule
& Pol  & Bike & Elevators \\
\midrule
Exact GP & $1.089(0.011)$ & $3.105(0.022)$ & $-0.386(0.001)$ \\
% Exact GP & $1.089(0.011)$ & $3.105(0.022)$ & $-0.386(0.001)$  \\
\midrule
 $M=1024$ & & & \\
SGPR & $0.821(0.008)$ & $2.176(0.020)$ & $-0.387(0.001)$\\
% SGPR-trace & $0.958(0.008)$  & $2.337(0.030)$ & $-0.387(0.001)$ \\
SGPR-artemev & $0.859(0.007)$ & $2.199(0.024)$ & $-0.387(0.001)$  \\
SGPR-new & $0.920(0.006)$ & $2.326(0.026)$  & $-0.387(0.001)$  \\
%SGPR-log & $0.998(0.008)$  & $2.511(0.021)$ & $-0.387(0.001)$ \\
\midrule
$M=2048$ & & & \\
% SGPR-trace & $0.821(0.008)$ & $2.176(0.020)$ & $-0.387(0.001)$\\
SGPR & $0.958(0.008)$  & $2.337(0.030)$ & $-0.387(0.001)$ \\
% SGPR-log & $0.920(0.006)$ & $2.326(0.026)$  & $-0.387(0.001)$  \\
SGPR-artemev & $0.976(0.008)$ & $2.356(0.029)$ & $-0.387(0.001)$  \\
SGPR-new & $0.998(0.008)$  & $2.511(0.021)$ & $-0.387(0.001)$ \\
\bottomrule
\end{tabular}}
%\end{sc}
%\end{small}
\end{center}
\vskip -0.1in
\end{table}


\begin{figure*}[t]
\centering
\begin{tabular}{ccc}
\includegraphics[scale=0.25]
{smallscale_elbo_pol_1024.pdf} &
\includegraphics[scale=0.25]
{smallscale_elbo_bike_1024.pdf} &
\includegraphics[scale=0.25]
{smallscale_elbo_elevators_1024.pdf} \\
\includegraphics[scale=0.25]
{smallscale_sigma2_pol_1024.pdf} &
\includegraphics[scale=0.25]
{smallscale_sigma2_bike_1024.pdf} &
\includegraphics[scale=0.25]
{smallscale_sigma2_elevators_1024.pdf} 
\end{tabular}
\caption{The two plots in each column correspond to the same dataset: first row shows the ELBO (or log-likelihood)
 for all four methods (Exact GP, SGPR, SGPR-new and SGPR-artemev) with the number of iterations, and the plot in the second row shows the
  corresponding values for $\sigma^2$. SGPR methods use $M=1024$ inducing points initialized by k-means. For each line we plot the mean and standard error
  after repeating the experiment five times with different train-test dataset splits; see \Cref{app:furtherresults} for further experimental details.       
}
\label{fig:mediumsize1024}
\end{figure*}


\subsection{Large Scale Regression Datasets
\label{sec:largeregress}
}

\begin{table*}[t]
\caption{Test log-likelihoods for the large scale regression datasets with standard errors in parentheses. Best mean values are highlighted.} 
% Uses random 80\% / 20\% training and test splits, repeated 5 times. }
\label{table:largescaleTestLL}
\makebox[\textwidth][c]{
\resizebox{1.02\textwidth}{!}{
\setlength\tabcolsep{2pt}
\begin{tabular}{ l l cc cc cc cc}
\toprule
& & Kin40k &  Protein & \footnotesize KeggDirected & KEGGU &  3dRoad & Song &  Buzz & \footnotesize HouseElectric \\
\cmidrule(lr){3-10}
& $N$ & 25,600 & 29,267 & 31,248 & 40,708 & 278,319 & 329,820 & 373,280 & 1,311,539  \\
& $d$ & 8 & 9 & 20 & 27 & 3 & 90 & 77 & 9  \\
\midrule
%\multirow{2}{*}{SVGP}
%& $1024$  
%& 0.094(0.003) & -0.963(0.006) & 0.967(0.005) & 0.678(0.004) & -0.698(0.002) & -1.193(0.001) & -0.079(0.002) & 1.304(0.002)  \\
%& $1536$  
%& 0.129(0.003) & -0.949(0.005) & 0.944(0.006) & 0.673(0.004) & -0.674(0.003) & -1.193(0.001) & -0.079(0.002) & 1.304(0.003) \\
%\midrule
From \citet{shietal2020} \\ 
ODVGP & $1024+1024$ 
& 0.137(0.003) & -0.956(0.005) & -0.199(0.067) & 0.105(0.033) & -0.664(0.003) & -1.193(0.001) & -0.078(0.001) & 1.317(0.002) \\
& $1024+8096$  
& 0.144(0.002) & -0.946(0.005) & -0.136(0.063) & 0.109(0.033) & -0.657(0.003) & -1.193(0.001) & -0.079(0.001) & 1.319(0.004) \\
SOLVE-GP & $1024 + 1024$ & 0.187(0.002) & -0.943(0.005) &  0.973(0.003) &  0.680(0.003) & -0.659(0.002) & -1.192(0.001) &  -0.071(0.001) & 1.333(0.003) \\
%\midrule
%SVGP
% \\
%& $2048$
%& 0.137(0.003) & {\bf -0.940}(0.005) & 0.907(0.003) & 0.665(0.004) & -0.669(0.002) & {\bf -1.192}(0.001) & -0.079(0.002) & 1.304(0.003) \\
\midrule
SVGP [ours] & 1024 & $0.108(0.002)$ & $-0.969(0.006)$ & $1.042(0.009)$ & $0.699(0.005)$ & $-0.704(0.003)$ & $-1.192(0.001)$ & $-0.069(0.002)$ & $1.383(0.002)$ \\
& 2048 & $0.237(0.002)$ & $-0.944(0.006)$ & ${\bf 1.050}(0.009)$ & ${\bf 0.703}(0.005)$ & ${\bf -0.650}(0.003)$ & ${\bf -1.190}(0.001)$ & $-0.063(0.001)$ & $1.419(0.002)$ \\
SVGP-new [ours]  & 1024 & $0.152(0.003)$ & $-0.965(0.006)$ & $1.044(0.009)$ & $0.699(0.005)$ & $-0.701(0.003)$ & $-1.192(0.001)$ & $-0.065(0.002)$ & $1.387(0.003)$ \\
 & 2048 & ${\bf 0.286}(0.002)$ & ${\bf -0.938}(0.006)$ & $1.051(0.009)$ & ${\bf 0.703}(0.005)$ & $-0.651(0.004)$ & ${\bf -1.190}(0.001)$ & ${\bf -0.060}(0.001)$ & ${\bf 1.421}(0.002)$ \\
\bottomrule 
\end{tabular}
}
}
\end{table*}


We consider 8 UCI regression datasets, with training data sizes ranging from tens of thousands to millions. 
%Results of exact GP regression have been reported on these datasets with distributed training~\citep{wang2019exact}. 
We implemented the stochastic optimization versions of the two scalable sparse GP methods: (i) the one that trains using the previous uncollapsed bound from
 \citet{hensman2013gaussian} (SVGP) and (ii) our new bound from    
\Cref{eq:newuncollapsedbound} (SVGP-new). We denote these stochastic optimization versions by SVGP to distinguish them from the corresponding
SGPR methods that use the more expensive collapsed bounds. We run the SVGP methods with $M=1024$ and $2048$ inducing points, Matern3/2 kernel with common lengthscale, minibatch size $1024$, Adam with base learning rate $0.01$ and $100$ epochs. These experimental settings match the ones in \citet{wang2019exact} and \citet{shietal2020} as further described  in \Cref{app:largescaleRegress}. Table \ref{table:largescaleTestLL} reports the test log likelihood scores
for all datasets. In the comparison we also included two strong baselines from Table 2 in \citet{shietal2020}, i.e., SOLVE-GP and ODVGP \cite{salimbeni2018orthogonally}.


\begin{figure*}
\centering
\begin{tabular}{ccc}
\includegraphics[scale=0.24]
{poisson_toy_all_predictions.pdf} &
\includegraphics[scale=0.24]
{poisson_toy_all_losses.pdf} &
\includegraphics[scale=0.24]
{poisson_elbo_nybicycle_16.pdf} \\
% (a) & (b) & (c)
\end{tabular}
\caption{({\bf left}) shows the % posterior 
predictions (means with 2-standard deviations) over counts (black dots) in the artificial data example  after
  fitting the Full GP, and the two SVGPs. This plot superimposes all predictions in order to provide a comparative visualization.
  %; see \Cref{app:poisson} for individual plots. 
  ({\bf middle})  shows the ELBO  across optimization steps for the artificial data example. ({\bf right}) shows the ELBO for the NYBikes dataset and $M=16$.}
\label{fig:poisson}
\end{figure*}


From the predictive log likelihood scores in Table \ref{table:largescaleTestLL} and also the corresponding Root Mean Squared Error (RMSE)
scores reported in  \Cref{table:largescaleRMSE} in \Cref{app:largescaleRegress}, we can conclude that training with the new SVGP-new variational bound
provides a clear improvement compared to training with the previous SVGP bound. Note that this improvement requires no change in the computational
cost, and in fact there is only a minor modification needed to be done in an existing SVGP implementation in order to run SVGP-new.  

\vspace{-1mm}

\subsection{Poisson Regression
  \label{sec:poisson}
}

\vspace{-1mm}

We consider a non-Gaussian likelihood example where the output data are counts modeled  by a Poisson likelihood 
$p(\y | \f) = \prod_{i=1}^N \frac{e^{f_i}}{y_i !} e^{-e^{f_i}}$  where the log intensities values follow a GP prior. For such 
case the new variational approximation includes a single additional variational parameter denoted by $v$, which is optimized together 
with the remaining parameters; see \cref{sec:nongaussian}. We will compare training with the new ELBO 
 from \Cref{eq:nonGaussian_bound_tractable}  (we denote this method by SVGP-new) with the standard ELBO that is obtained by restricting  $v=1$  (SVGP). 
 
Firstly, we consider an artificial example of $50$ observations with 1-D inputs placed in the grid $[-10, 10]$ where counts are
generated using Poisson intensities given by $\lambda(x) = 3.5 + 3  \sin(x)$. We train the GP model with the SVGP bound and the proposed SVGP-new bound using $6$ inducing points initialized to the same values for both methods; see \Cref{app:poisson}. 
\Cref{fig:poisson}(left) shows the 
% observed counts together with the 
predictions obtained by SVGP, SVGP-new 
and non-sparse %or full 
variational % inference 
GP (Full GP). From
this figure and from the
ELBO values, 
we observe that SVGP-new
remains closer to Full GP.  

Secondly, we consider a real dataset (NYBikes) about bicycles crossings going over bridges in New York City\footnote{This dataset is freely available from
\url{https://www.kaggle.com/datasets/new-york-city/nyc-east-river-bicycle-crossings}.}.
This dataset is a daily record of the number of bicycles crossing into or out of Manhattan via one of the East River bridges over a period 9 months. The data contains $210$  points and we randomly choose $90\%$ for training and $10\%$ for test.   
We apply GP Poisson regression for the Brooklyn bridge counts where the input vector $\bx$ is taken to be two-dimensional consisted of 
 maximum and minimum daily temperatures.  We train the sparse GPs with either SVGP or SVGP-new and with $M=8,16,32$ 
 inducing points initialized by k-means.  Since the dataset is small  we also run the non-sparse  Full GP. The ELBO across iterations in \Cref{fig:poisson} (right) and the test log likelihood scores (\Cref{table:poisson_nybikes} 
 in \Cref{app:poisson})
 indicate that  SVGP-new provides a better approximation than SVGP.  
 
  
   
   










\vspace{-2mm}
\section{Conclusion}\label{sec:conclusion}
In this paper, we presented RecDreamer, a novel approach to mitigating the Multi-Face Janus problem in text-to-3D generation. Our solution introduces a rectification function to modify the prior distribution, ensuring that the resulting joint distribution achieves uniformity across poses. By expressing the modified data distribution as the product of the original density and the rectification function, we seamlessly integrate this adjustment into the score distillation algorithm. This allows us to derive a particle optimization framework for uniform score distillation. Additionally, we developed a pose classifier and implemented reliable approximations and simulations to enhance the particle optimization process. Extensive experiments on both 2D and 3D synthesis tasks demonstrate the effectiveness of our approach in addressing the Multi-Face Janus problem, resulting in more consistent geometries and textures across different views.

\textbf{Limitations.} While our method significantly reduces bias in prior distributions, further exploration of 3D modeling with multi-view priors could improve geometric and texture consistency. Extending our approach through deeper research into conditional control presents another promising avenue for addressing these challenges in future work. 




% \subsubsection*{Author Contributions}
% If you'd like to, you may include  a section for author contributions as is done
% in many journals. This is optional and at the discretion of the authors.

\subsubsection*{Acknowledgments}
The work is supported by Guangdong Provincial Natural Science Foundation for Outstanding Youth Team Project (No. 2024B1515040010), NSFC Key Project (No. U23A20391), China National Key R\&D Program (Grant No. 2023YFE0202700, 2024YFB4709200), Key-Area Research and Development Program of Guangzhou City (No. 202206030007, 2023B01J0022), the Guangdong Natural Science Funds for Distinguished Young Scholars (No. 2023B1515020097), the AI Singapore Programme under the National Research Foundation Singapore (No. AISG3-GV-2023-011), and the Lee Kong Chian Fellowships.


\bibliography{iclr2025_conference}
\bibliographystyle{iclr2025_conference}

\newpage
\appendix


\section{Additional Related Works}
\label{app:related}


\textbf{Structured compositional generative models.} Structured generative models leverage architectural inductive biases in an encoder-decoder framework, such as recurrent attention mechanisms \cite{gregor2015drawrecurrentneuralnetwork} or slot-attention \cite{Wang2023SlotVAEOS}. These models decompose scenes into background and parts-based representations in an unsupervised manner guided by modeling priors. While these approaches can flexibly generate scenes with single or multiple objects, they are not explicitly controllable, and require specific model pre-training on datasets containing compositions of interest.

\textbf{Controllable generation.} Composition at inference-time is one potential mechanism for exerting control over the generation process. Another way to modify compositions of style and/or content attributes is through spatial conditioning a pre-trained diffusion model on a structural attribute (e.g., pose or depth) as in  \citet{zhang2023adding}, or on multiple attributes of style and/or content as in \citet{conditional-loradapter}. Another option is control through resampling, as in \citet{liu2024correcting}. These methods are complementary to single or multiple model conditioning mechanisms based on score composition that we study in the current work.

\textbf{Single model conditioning.} We distinguish the kind of composition we study in this paper from approaches that rely on a single model but use OOD conditioners to achieve novel combinations of concepts never seen together during training; for example, passing OOD text prompts to text-to-image models \citep{nichol2021glide, podell2023sdxl}, or works like \citet{okawa2024compositional, park2024emergence} where a single model conditions simultaneously on multiple attributes like shape and color, with some combinations held out during training.
In contrast, the compositions we study recombine the outputs of multiple separate models at inference time.
Though less powerful, this can still be surprisingly effective, and is more amenable to theoretical study since it disentangles the potential role of conditional embeddings.

\textbf{Multiple model composition.} Among compositions involving multiple separate models, many different variants have been explored with different goals and applications.
Some definitions of composition are inspired by logical operators like AND and OR, usually taken to mean that the composed distribution should have high probability under all of the conditional distributions to be composed, or at least one of them, respectively.
Given two conditional probabilities $p_0(x), p_1(x)$, AND is typically implemented as the product $p_0(x)p_1(x)$ and OR as sum $p_0(x) + p_1(x)$
(though these only loosely correspond to the logical operators and other implementations are also possible).
Some composition methods are based on diffusion models and use the learned scores (mainly for product compositions), others use energy-based models (which allows for OR-inspired sum compositions, as well as more sophisticated samplers, in particular sampling at $t=0$ \citep{du2020visualenergy, du2023reduce, liu2021learning}, and still others work directly with the densities \cite{skreta2024superposition} (enabling an even greater variety of compositions, including a different style of AND, taken to mean $p_0(x) = p_1(x)$). \citet{mcallister2025decentralized} explore another type of OR composition. \cite{wiedemer2024compositional} take a different approach of taking the final rendered images generated by separate diffusion models and ``adding them up'' in pixel-space, as part of a study on generalization of data-generating processes. Task-arithmetic \cite{zhang2023composing, ilharco2022editing}, often using LoRAs \cite{hu2021lora}, is a kind of composition in weight-space that has had significant practical impact.

\textbf{Product compositions.} In this work, we focus specifically on product compositions (broadly defined to allow for a ``background'' distribution, i.e. compositions of the form $\hat{p}(x) = p_b(x) \prod_i \frac{p_i(x)}{p_b(x)}$) implemented with diffusion models, which allows the composition to be implemented via a linear combinations of scores as in \citet{du2023reduce, liu2022compositional}. Our goal is not to propose
a wholly new method of composition but rather to improve theoretical understanding of existing methods.

\textbf{Learning and Generalization.}
Recently, \citet{kamb2024analytic}
demonstrated how a type of compositional generalization
arises from inductive bias in the learning procedure (equivariance
and locality).
Their findings are relevant to our broader motivation,
but complementary to the focus of this work.
Specifically, we focus only on mathematical aspects
of defining and sampling from compositional distributions,
and we do not consider any learning-theoretic aspects
such as inductive bias or sample complexity.
This allows us to study the behavior of
compositional sampling methods
even assuming perfect knowledge of the underlying distributions.

\subsection{Sampling with \mtd and History Guidance}
\newcommand{\algcomment}[1]{\small{\hfill \(\triangleright\) #1}}
\begin{algorithm}[t]
    \caption{\textbf{Flexible Sampling with \mtd and (optionally) History Guidance}}
    \label{alg:sampling}
    \begin{algorithmic}
    \STATE {\bfseries Task:} specified by indices $\cH$, $\cG = \cT \setminus \cH$, and history frames $\xH$.
    \STATE {\bfseries Input:} diffusion process defined by $\alpha_k, \sigma_k$, diffusion sampler $\mathcal{S}$ with sampling steps $N$,\\
    \textbf{\mtd} model $\rvs_\vtheta(\cdot, \cdot)$, and
    \textbf{History Guidance} scheme specified by $\{(\cH_i, k_{\cH_i}, \omega_i)\}_{i=1}^I$.
    \STATE Sample $\rvx_\cG ~\sim \mathcal{N}(0, I)$, then $\rvx_{\cT} \gets \rvx_{\cH} \oplus \rvx_{\cG}$ \algcomment{Sample random noise for generation frames}
    \FOR{$n=N, N-1, \ldots, 1$}
        \STATE $k_{\cT} \gets (k_t)_{t=1}^T$ where {\small$\begin{cases} k_t = \frac{n}{N} & \text{if } t \in \cG \\ k_t = 1 & \text{if } t \in \cH \end{cases}$}
        \STATE $\hat{\rvx}_{\cT} \gets \rvx_{\cT}$, then \emph{replace} $\hat{\rvx}_{\cH} \gets \beps$ where $\beps \sim \mathcal{N}(0, I)$ \algcomment{Fully mask history}
        \STATE $\hat{\rvs}^{\varnothing} \gets \rvs_\vtheta(\hat{\rvx}_{\cT}, k_{\cT})$ \algcomment{Estimate unconditional score}
        \FOR{$i=1, \ldots, I$}
            \STATE $k_{\cT} \gets (k_t)_{t=1}^T$ where {\small$\begin{cases} k_t = \frac{n}{N} & \text{if } t \in \cG \\ k_t = k_{\cH_i} & \text{if } t \in \cH_i \\ k_t = 1 & \text{if } t \in \cH \setminus \cH_i \end{cases}$}
            \STATE $\hat{\rvx}_{\cT} \gets \rvx_{\cT}$, then \emph{replace} $\begin{cases} \hat{\rvx}_{\cH_i} \gets \alpha_{k_{\cH_i}} \hat{\rvx}_{\cH_i} + \sigma_{k_{\cH_i}} \beps \text{ where } \beps \sim \mathcal{N}(0, I) \\ \hat{\rvx}_{\cH \setminus \cH_i} \gets \beps \text{ where } \beps \sim \mathcal{N}(0, I) \end{cases}$ \algcomment{Mask history based on $\cH_i$ and $k_{\cH_i}$}
            \STATE $\hat{\rvs}^i \gets \rvs_\vtheta(\hat{\rvx}_{\cT}, k_{\cT})$ \algcomment{Estimate $i$-th conditional score}
        \ENDFOR
        \STATE $\hat{\rvs} \gets \hat{\rvs}^{\varnothing} + \sum_{i=1}^I \omega_i \cdot (\hat{\rvs}^i - \hat{\rvs}^{\varnothing})$ \algcomment{Compose scores}
        \STATE $\rvx_{\cG} \gets \mathcal{S}(\rvx_{\cG}, \hat{\rvs}_{\cG}; \frac{n}{N}, \frac{n-1}{N})$ \algcomment{Denoise $k = \frac{n}{N} \rightarrow \frac{n-1}{N}$}
        
    \ENDFOR
    \STATE {\bfseries Output:} $\rvx_\cG$
    \end{algorithmic}
\end{algorithm}

\mtd is capable of flexible sampling conditioning on \emph{arbitrary history}, and is further capable of performing \emph{history guidance}, a family of guidance methods we propose. In \cref{alg:sampling}, we provide a detailed sampling procedure for \mtd and history guidance, where any score-based sampler such as DDPM~\cite{ho2020denoising} or DDIM~\cite{ddim} can be used for $\mathcal{S}$. Importantly, when estimating a score conditioned on a masked history, it is crucial to pass the corresponding noise levels $k_\cT$ and to \emph{replace} the clean history frames with noisy frames, which are created by diffusing the clean history to the noise levels. This ensures that the model input is consistent with what it encounters during training time. Note that \cref{alg:sampling} can be applied given arbitrary history frames. For instance, to \emph{extrapolate} the history of length $\tau$ to $T$ frames, set $\cH = \{1, \ldots, \tau\}$ and $\cG = \{\tau+1, \ldots, T\}$; to \emph{interpolate} between two frames, set $\cH = \{1, T\}$ and $\cG = \{2, \ldots, T-1\}$. Below we provide several representative examples of how the algorithm is applied:
\begin{itemize}[topsep=0pt, itemsep=0pt]
    \item \textbf{Conditional Sampling without History Guidance}: $\{(\cH_i, k_{\cH_i}, \omega_i)\}_{i=1}^I = \{(\cH, 0, 1)\}$
    \item \textbf{Vanilla History Guidance} with a guidance scale $\omega > 1$: $\{(\cH_i, k_{\cH_i}, \omega_i)\}_{i=1}^I = \{(\cH, 0, \omega)\}$
    \item \textbf{Temporal History Guidance} with $I$ subsequences $\{\cH_i\}_{i=1}^I$ and guidance scales $\{\omega_i\}_{i=1}^I$: $\{(\cH_i, k_{\cH_i}, \omega_i)\}_{i=1}^I = \{(\cH_i, 0, \omega_i)\}_{i=1}^I$
    \item \textbf{Fractional History Guidance} with a guidance scale $\omega$ and fractional masking level $k_\cH$: $\{(\cH_i, k_{\cH_i}, \omega_i)\}_{i=1}^I = \{(\cH, 0, 1), (\cH, k_\cH, \omega - 1)\}$
\end{itemize}

\subsection{Simplifying Training Objective}
\label{app:method_details_objective_causal}
Diffusion Forcing~\cite{chen2024diffusion} proposes to train the entire sequence with independent noise per frame. A natural question to ask is whether this mixed objective includes too many tasks compared to what one actually needs. Here we provide some insights from our experiments throughout the project: When the number of frames is small e.g. $10$ latent frames, there is no noticeable decrease in training efficiency - Diffusion Forcing seems to converge as fast as standard diffusion from both training and validation curves. However, when we grow the number of latent frames to $50$, we start to witness decreased performance at sampling time. While we firmly believe that binary dropout is not the ideal way to achieve objective reduction from our experiments, we believe that one can easily reduce our training objective by only applying independent noise up to the maximum training length one wants to support. In particular, if one wants to generate the next $10$ frames from previous $1-10$ frames, it doesn't seem necessary for frame $11$ to be independently masked as noise from time to time, since we will never need to mask it out for flexible conditioning. In addition, one may want to consider treating the number of history frames as a random variable at training time, sampling a length first and then applying uniform levels of masking to the history, though independent from the noise level of the generation target. We didn't investigate these simplifications in detail because we simply find Diffusion Forcing's training objective very versatile for many of the tasks we want to do, e.g. interpolation, and varying noise level sampling. However, we do believe that these schemes could worth more exploration if one is to scale up our method to a much bigger number of context frames.

\subsection{Causal Variant}
In principle, one can implement \mtd and History Guidance with a causal transformer as well. For example, CausVid~\cite{yin2024slow} has proved the effectiveness of Diffusion Forcing on fast causal video synthesis and doesn't conflict with History Guidance. However, we'd like to highlight that one can also use our non-causal \mtd to achieve causal sampling. Different from traditional transformer-based models, \mtd doesn't need to enforce an attention mask to achieve causality. Instead, at generation time, one can mask out the future with white noise to prevent any information from the future from leaking into the neural network. In fact, there might be use cases when one may want some low-frequency information from the future, and then one can fractionally mask out the future via noise as masking to achieve so. On the other hand, the motivation behind causal video diffusion models is often speed and real-time generation using KV caching. In that case, one either needs to train a causal \mtd directly or consult advanced techniques like attention sink~\cite{xiao2023efficient} to perform windowed attention effectively.

\subsection{Incorporating Other Conditioning}
Throughout our discussions in the main paper, conditioning is history exclusively. What if one wants to integrate the \method into a text-conditioned diffusion model? One claim of the \mtd is that it doesn't require architectural changes so one can fine-tune an existing model into a \mtd model. This is still the case here: if one already has a text-conditioned video diffusion model, presumably built to accept such conditioning via an adaptive layer norm, one simply take \mtd as an add on to their existing architecture to obtain a \mtd model that accepts both text and history as conditioning. \mtd's Figure~\ref{fig:architecture} does not assert that one cannot use an external AdaLN layer with \mtd, but is rather saying no architectural changes is needed.

\subsection{Extended Temporal History Guidance}
\label{app:method_details_temporal}

Temporal history guidance addresses the challenge of out-of-distribution (OOD) history by composing scores conditioned on different, shorter history subsequences, which are closer to being in-distribution. However, since the model receives the entire video sequence as input during sampling—including both the history and the noisy frames being generated—the OOD problem can arise throughout the entire video sequence, not just in the history portion. To mitigate this, we propose further decomposing the generation $\cG$ into generation subsequences $\cG_1, \cG_2, \ldots, \cG_{J} \subset \cG$. In line with the original temporal history guidance, the history $\cH$ is already decomposed into history subsequences $\cH_1, \cH_2, \ldots, \cH_{I} \subset \cH$. This allows us to compose scores conditioned on even shorter, and thus more in-distribution, subsequences in $\{\cH_i\}_{i=1}^{I} \times \{\cG_j\}_{j=1}^{J}$. Specifically, the composed score is given by:
\begin{equation}
    \scalebox{1.0}{$
    \bigoplus_{j=1}^{J} \sum_{i=1}^{I} \score p_k(\rvx_{\cG_j}^k | \rvx_{\cH_i})$}
\end{equation}
where $\bigoplus$ denotes a frame-wise averaging operation. We refer to this method as \emph{Extended Temporal History Guidance}, as it extends the concept of temporal history guidance by composing both history and generation subsequences. Empirically, we find this method to be more effective than the original temporal history guidance when the video sequence is clearly OOD (e.g., RealEstate10K OOD history experiment), and thus requires shorter subsequences to be in-distribution.

\section{Discrete Langevin Proposal}\label{appndx:dlp_proposal}
Our proposed controlled text generation leverages the gradient-based discrete sampling algorithm in \citet{zhang2022langevinlike}, which is further investigated by \citet{pynadath2024gradientbaseddiscretesamplingautomatic}. Using the same notation as in the Main Body of the paper, we put the original proposal distribution from \citet{zhang2022langevinlike} below:
\[
\text{Categorical} \left( \underset{j \in |V|}{\softmax} \left( \frac{1}{2} \nabla f(\hat{B} | X)_i (\text{Onehot}_j - \hat{b}_i) - \frac{||\text{Onehot}_j - \hat{b}_i ||^2_2}{2\alpha}\right) \right)
\]
Here, $\hat{b}_i$ corresponds to the one-hot vector in sequence position $i$. Similarly, $\text{Onehot}_j$ corresponds to the one-hot vector for the $j$th token in $V$. This proposal function defines a distribution over the vocabulary for the $i$th sequence position in the sequence by taking the softmax over all possible tokens.

As discussed in \citet{pynadath2024gradientbaseddiscretesamplingautomatic}, this proposal is locally balanced, or optimal for very small step-sizes. For the task of controlled text generation, we would prefer a proposal function that is optimal for large step-sizes, which allow for superior exploration of the space of potential sequences. The globally balanced proposal can be written as follows: 
\[
\text{Categorical} \left( \underset{j \in |V|}{\softmax} \left( \nabla f(\hat{B} | X)_i (\text{Onehot}_j- \hat{b}_i) \right) \right)
\]
In terms of the gradient computation, the one-hot representation  enables the use of automatic differentiation packages to compute $\nabla f(\hat{B} | X)$. We observe that the term 
$(\text{Onehot}_j - \hat{b}_i)$ corresponds to the distance between the proposed token $j$ and the original token $b_i$. We choose to represent this distance term as hamming distance, given the discrete nature of the space we wish to sample. For a token $j$, the hamming distance to the original token in position $i$ is 0 if the $j$th coordinate $\hat{b}_{ij} = 1$ as they are the same token; and 1 if the $j$th coordinate is 0. Thus we can represent the distances between the tokens as $1 - \hat{b}_{ij}$. This leads us to the proposal function in \ref{eq:dlp_prop}, which we place below for convenience: 
\[
b'_i \sim
    \categorical\left(\underset{j \in V}{\softmax} \left( \frac{1}{\tau} (\nabla f(\hat{B} | X))_{ij} (1 - \hat{b}_{ij}) \right) \right)
\]
Here, $b'_i$ refers to the token we sample from the categorical distribution over $V$. 

\section{Algorithmic Details} \label{appndx:algrthm-details}
Here we provide the full pseudo-code for our algorithm. 
\begin{algorithm}
    \caption{Discrete Autoregressive Biasing}
    \begin{algorithmic}[1]
    \REQUIRE Constraint function $f$, $P^{LM}$, prompt $X$, number steps $s$, sequence length $n$, embedding table $M$
    \STATE $\tilde{B} \gets \vec{0}, f_\text{min} \gets -\infty$, $Y_\text{best} \gets \{\}$ \LineComment{Initialize constraint violation as being maximal and current best generation as empty}
    \FOR{step $s$}
        \FOR{position $i$ in range($n$)} 
            \STATE $\tilde{y_i} \gets \log P^{LM} (\cdot | y_{<i}, X)$ \LineComment{Initial auto-regressive distribution over $V$}
            \STATE Calculate normalizing factor $r_i$ if $s > 1$, else $r_i \gets 1$
            \STATE $y_i \gets \text{argmax}_{j \in |V|} \left(\tilde{y}_{i, j} - w_i \cdot r_i \cdot \tilde{b}_{i, j} \right)$ \LineComment{Sample from $P(Y | X, B)$}
        \ENDFOR
        \STATE $B \gets Y$ \LineComment{Initialize $B$ as $Y$}
        \STATE Evaluate $f(B | X)$, update $f_\text{min}$, $Y_\text{best}$
        \STATE $B' \sim q_\tau(\cdot | B)$ as in \eqref{eq:dlp_prop} \LineComment{Approximately sample from $P(B | X, Y)$}
        \STATE Compute $\tilde{B}$ as in \eqref{eq:bias-vec-def}
    \ENDFOR
    \STATE return $Y_\text{best}$
    \end{algorithmic}
\label{alg:text-gen}
\end{algorithm}

DAB takes as input the external constraint $f$, the base language model $P^{LM}$, prompt $X$, number of steps $s$, sequence length $n$, and embedding table $M$. Given these inputs, our proposed algorithm alternates between auto-regressively generating the response sequence and sampling the bias sequence using Discrete Langevin Proposal (DLP) \citep{zhang2022langevinlike}. 

\section{Ablation Study}
\label{appndx:ablation}
% \begin{table}[!t]
% \centering
% \scalebox{0.68}{
%     \begin{tabular}{ll cccc}
%       \toprule
%       & \multicolumn{4}{c}{\textbf{Intellipro Dataset}}\\
%       & \multicolumn{2}{c}{Rank Resume} & \multicolumn{2}{c}{Rank Job} \\
%       \cmidrule(lr){2-3} \cmidrule(lr){4-5} 
%       \textbf{Method}
%       &  Recall@100 & nDCG@100 & Recall@10 & nDCG@10 \\
%       \midrule
%       \confitold{}
%       & 71.28 &34.79 &76.50 &52.57 
%       \\
%       \cmidrule{2-5}
%       \confitsimple{}
%     & 82.53 &48.17
%        & 85.58 &64.91
     
%        \\
%        +\RunnerUpMiningShort{}
%     &85.43 &50.99 &91.38 &71.34 
%       \\
%       +\HyReShort
%         &- & -
%        &-&-\\
       
%       \bottomrule

%     \end{tabular}
%   }
% \caption{Ablation studies using Jina-v2-base as the encoder. ``\confitsimple{}'' refers using a simplified encoder architecture. \framework{} trains \confitsimple{} with \RunnerUpMiningShort{} and \HyReShort{}.}
% \label{tbl:ablation}
% \end{table}
\begin{table*}[!t]
\centering
\scalebox{0.75}{
    \begin{tabular}{l cccc cccc}
      \toprule
      & \multicolumn{4}{c}{\textbf{Recruiting Dataset}}
      & \multicolumn{4}{c}{\textbf{AliYun Dataset}}\\
      & \multicolumn{2}{c}{Rank Resume} & \multicolumn{2}{c}{Rank Job} 
      & \multicolumn{2}{c}{Rank Resume} & \multicolumn{2}{c}{Rank Job}\\
      \cmidrule(lr){2-3} \cmidrule(lr){4-5} 
      \cmidrule(lr){6-7} \cmidrule(lr){8-9} 
      \textbf{Method}
      & Recall@100 & nDCG@100 & Recall@10 & nDCG@10
      & Recall@100 & nDCG@100 & Recall@10 & nDCG@10\\
      \midrule
      \confitold{}
      & 71.28 & 34.79 & 76.50 & 52.57 
      & 87.81 & 65.06 & 72.39 & 56.12
      \\
      \cmidrule{2-9}
      \confitsimple{}
      & 82.53 & 48.17 & 85.58 & 64.91
      & 94.90&78.40 & 78.70& 65.45
       \\
      +\HyReShort{}
       &85.28 & 49.50
       &90.25 & 70.22
       & 96.62&81.99 & \textbf{81.16}& 67.63
       \\
      +\RunnerUpMiningShort{}
       % & 85.14& 49.82
       % &90.75&72.51
       & \textbf{86.13}&\textbf{51.90} & \textbf{94.25}&\textbf{73.32}
       & \textbf{97.07}&\textbf{83.11} & 80.49& \textbf{68.02}
       \\
   %     +\RunnerUpMiningShort{}
   %    & 85.43 & 50.99 & 91.38 & 71.34 
   %    & 96.24 & 82.95 & 80.12 & 66.96
   %    \\
   %    +\HyReShort{} old
   %     &85.28 & 49.50
   %     &90.25 & 70.22
   %     & 96.62&81.99 & 81.16& 67.63
   %     \\
   % +\HyReShort{} 
   %     % & 85.14& 49.82
   %     % &90.75&72.51
   %     & 86.83&51.77 &92.00 &72.04
   %     & 97.07&83.11 & 80.49& 68.02
   %     \\
      \bottomrule

    \end{tabular}
  }
\caption{\framework{} ablation studies. ``\confitsimple{}'' refers using a simplified encoder architecture. \framework{} trains \confitsimple{} with \RunnerUpMiningShort{} and \HyReShort{}. We use Jina-v2-base as the encoder due to its better performance.
}
\label{tbl:ablation}
\end{table*}

\subsection{Efficiency}
\label{appndx:efficiency}
\begin{table}[t!]
\centering
    \scriptsize
    \setlength{\tabcolsep}{0.0035\linewidth}
    \caption{\textbf{Computational efficiency of EvSSC across different datasets.} Memory denotes training memory usage.}
    %\vskip-1ex
\setlength{\tabcolsep}{4pt} %5pt 设定列之间的宽度
\resizebox{\columnwidth}{!}{%
\begin{tabular}{l|>{\columncolor{gray!10}}l>{\columncolor{blue!8}}l|>{\columncolor{gray!10}}l>{\columncolor{blue!8}}l}
\toprule
\textbf{ } & \textbf{VoxFormer-S} & \textbf{EvSSC (VoxFormer)} & \textbf{SGN-S} & \textbf{EvSSC (SGN)}\\
\midrule\midrule
\multicolumn{5}{c}{\textit{DSEC-SSC}} \\ \midrule 
\textbf{mIoU} &25.62 & 26.34 & 29.06 & 29.55\\ 
\textbf{IoU}  & 47.25 & 47.29 & 43.70 & 43.99\\ 
\textbf{Memory}  & 9.74G & 10.52G & 10.19G & 10.70G\\ 
\textbf{Latency} & 0.732s & 0.836s & 0.941s & 1.193s \\\midrule 
\multicolumn{5}{c}{\textit{SemanticKITTI-E}} \\ \midrule 
\textbf{mIoU} & 12.86 & 13.61 & 14.55 & 15.15\\ 
\textbf{IoU}  & 44.42 & 45.01 & 43.60 & 43.17\\ 
\textbf{Memory} & 14.87G & 15.78G & 15.29G & 17.79G \\ 
\textbf{Latency}  & 0.996s & 1.005s & 0.855s &1.005s\\ \midrule 
\multicolumn{5}{c}{\textit{SemanticKITTI-C Shot Noise}} \\ \midrule 
\textbf{mIoU} & 8.29 & 12.64 & 13.62 & 14.32\\ 
\textbf{IoU}  & 44.26 & 45.04 & 42.05 & 42.54\\ 
\textbf{Memory} & 14.87G & 15.78G & 15.29G & 17.79G \\ 
\textbf{Latency}  & 0.996s & 1.005s & 0.855s &1.005s\\
\bottomrule
\end{tabular}
}
\label{table:efficiency}
%\vskip-3ex
\end{table}



\section{Experimental Details}
Here we include additional details on the experiment setup. We provide the hyper-parameter settings for our algorithm  for each experiment in Table \ref{appndx:tab:exp-hyperparam}. It should be noted that for Sampling Steps, we pick values to maintain roughly the same time cost as BOLT: given that our algorithm is roughly twice as fast, we use around twice the number of sampling steps. Furthermore, given the use of early stopping in BOLT, further computational budget doesn't necessarily provide any advantage. 

For the weight value, we use a schedule by \citet{liu2023bolt} as it was shown to be effective in terms of incorporating the bias term into auto-regressive generation. Thus for each position $t$, we have $w_t = w(1 - \frac{t}{L})$, where $w$ is the value we put in Table \ref{appndx:tab:exp-hyperparam}. 
\begin{table}[h]
\caption{Hyper-parameter settings used for DAB on Sentiment-directed generation, language detoxification, and topic-constrained generation.}
\centering
\label{appndx:tab:exp-hyperparam}
\begin{tabular}{l|cccccl}\toprule
     \textit{Hyper-parameter} & \textit{Sentiment} & \textit{Detoxify} & \textit{Topic} \\ \midrule 
Proposal Temp & .1 & .1 & .1\\ 
Top-k & 250 & 250 & 250 \\
Bias Weight Value & 1.05 & 1.05 & 1.4 \\
Number Sample Steps & 20 & 20 & 200 \\ 
\bottomrule
\end{tabular}
\end{table} 

\subsection{Fluency Metrics}
\label{appndx:fluency-metrics}
Here we provide more details as to the metrics we use to evaluate the fluency of text generations. 
\paragraph{CoLA Score} To assess the grammatical correctness of a generation, we use a fine-tuned RoBERTa model from \citet{morris2020textattack} to predict the probability of the sample being labelled as grammatically correct. While a similar metric was used in \citet{kumar2022gradient}, we compute the average predicted probability as opposed to the percentage over generations predicted as fluent since this provides more insight into the degree of grammatical correctness. 

\paragraph{Repeated Tri-grams} To compute the number of repeated tri-grams, we simply count all the tri-grams that were repeated and divide them by the total number of tri-grams per generation. We show the average across all generations for each metric. 

\paragraph{Perplexity} For perplexity, we use the built-in function within the Hugging Face evaluate package to compute the perplexity of each generation according to GPT2-XL \citep{wolf2020huggingfacestransformersstateoftheartnatural}. We show the perplexity of the \textbf{entire} generation, as opposed to conditioning on the prompt as done in \citet{han2023lm, kumar2022gradient, liu2023bolt}. 

\subsection{Sentiment Controlled Generation}
\label{appndx:senti-details}
\paragraph{Experiment Design} We use the same experimental design from \citet{liu2023bolt}, where the sampler uses an internal classifier to produce the generations. The internal model is a RoBERTA with GPT2-Large Embeddings fine-tuned on the yelp polarity dataset. We use two external models to provide additional evaluation: we use another RoBERTA trained on the same dataset but with the original embeddings, as well as a RoBERTa fine-tuned on Stanford Sentiment Treebank 2. 

We include the hyper-parameters we use for DAB in Table \ref{appndx:tab:exp-hyperparam}. For the baselines, we run the code within their codebase. While we minimize the changes made to the original code, we note that there are some necessary modifications needed in order to ensure that the experimental setting is consistent across all methods evaluated. This due to the fact that all the evaluated methods consider similar but slightly different experiments from ours in their original work \citep{qin2022cold, liu2023bolt, han2023lm, kumar2022gradient}.  

In regards to LM-Steer, which requires training data, we train the steering matrix using the SST-2 dataset, as done in \citet{han2023lm}. While this is a different dataset from what was used to fine-tune the internal classifiers for the EBM sampling methods, we choose this dataset as obtained worse results when training the steer matrix on yelp polarity. Furthermore, we include an external classifier fine-tuned on SST-2 to use as an evaluation criteria. This makes our experiments fair, as all the methods are evaluated with classifiers that are fine-tuned on a different dataset than used for sampling. Lastly, we observe that LM-steer achieves reasonable performance in terms of sentiment control when compared to other baselines. 

Here we list the prompts we use for this experiment: 

\paragraph{External Constraint} To represent the internal constraint, we use a RoBERTA with GPT-2 large embeddings fine-tuned on Yelp-Polarity for COLD, BOLT, MuCOLA, and DAB. We train this model following the codebase of \citet{liu2023bolt}. Since we require the embedding table to be the same between the base LM, we use the GPT2-large embeddings for the classifier, as done in \citet{liu2023bolt, kumar2022gradient}. 
% The final classifier achieves an accuracy of $96\%$ on the hold-out. 

We use a slightly different function to represent the constraint imposed by the fine-tuned model when compared to BOLT. Given the discriminator $h: |V| \to \mathbf{R}^2$, where the results represent the logits for both the desired class $c_{+}$ and the undesired class $c_{-}$, we define the final constraint function as follows: 
\begin{align*}
    f(Y) = (h(Y)_{+} - h(Y)_{-})
\end{align*}
Intuitively, this pushes the unnormalized logits between the desired class and the opposite class away from each other.

This differs from the constraint function in BOLT, which is the typical cross-entropy loss of the discriminator logits where the correct label is the desired sentiment: 
\begin{align*}
    f(Y) &= \log \softmax (h(Y)_{+})
\end{align*}

We find that our formulation of the constraint function enables more effective gradients for our specific method. 
Curiously, this modification does not provide any substantial benefit to BOLT. 
It is possible that the $\log \softmax$ of BOLT's method smooths out the directional information of the gradient. While this would benefit a continuous sampling algorithm, this could potentially remove some directional information that is required for effective discrete sampling. 

\paragraph{Example Generations} In Table \ref{appendix:tab:senti-gens} we include examples of generations for all methods evaluated.  
\begin{table}[!h]
\caption{Example text for Sentiment-guided generation. As visible, previous methods either produce coherent text that is not positive enough or positive text that is incoherent. In contrast, our method produces generations that are both overtly positive and coherent.}
\centering
\resizebox{\textwidth}{!}{
\begin{tabular}{p {.2 \textwidth}|p {.8 \textwidth}}
\toprule
 \textbf{Prompt} & \textit{The horse} \\
\toprule
COLD & The horse head was still in the water but the horse still had a good head. The horse \\ \midrule
MuCOLA & The horse is not only a beautiful and well-crafted piece of art, but it is also a great way \\ \midrule
BOLT & The horseback riding course is a great way to get acquainted with the trails and the terrain. The course is \\  \midrule
LM-Steer & The horseman delivers a stunningly beautiful, wonderfully lyrical, beautifully tender, powerfully moving, beautifully lyrical \\ \midrule
DAB (Ours) & The horse is also a very good and reliable companion. It has been used to carry the family's two- \\ 
\toprule
 \textbf{Prompt} & \textit{The chicken} \\
\toprule
COLD & The chicken is a good source of protein, but it's a very high fat, cholesterol and salt. \\ \midrule
MuCOLA & The chicken is a very healthy, and high in protein, and the other two chicken is a very good, \\ \midrule
BOLT & The chicken, which is named after the popular street food in Mexico, was a gift from a friend who had \\ \midrule 
LM-Steer & The chicken is beautifully tender, beautifully sweet, wonderfully colorful. the crisp, crispiness is wonderful. and the \\  \midrule
DAB (Ours) & The chicken is a great way to get the kids hooked on delicious, healthy food. It's also a good \\ 
\bottomrule
\end{tabular}
}
%\end{small}
% \vspace{-10pt}
\label{appendix:tab:senti-gens}
\end{table}


\subsection{Toxicity Controlled Generation}\textcolor{red}{Content Warning: The following section contains examples of LLM generations that some may find disturbing or offensive, including content that is hateful or violent in nature}

\label{appndx:toxicity-details}
\textbf{Experiment Design} Similar to the sentiment control experiment, we use an internal classifier to guide generation towards less toxic models. 
We use a RoBERTa classifier fine-tuned on the Jigsaw toxic comment dataset, following \citet{kumar2022gradient, liu2023bolt}. 


\textbf{External Constraint} We represent the external constraint by a RoBERTa model with GPT-2 large embeddings that was fine-tuned on the Jigsaw dataset, following the same training protocol as used in \citet{liu2023bolt}. The final classifier achieves an accuracy of $94\%$ on the hold-out dataset.  

\paragraph{Examples} In Table \ref{appendix:tab:detox-gens} we show examples of generations for the various samplers we examine. 
\begin{table}[!h]
\centering
\caption{Examples for Language Detoxification. While all methods seem to have similar probabilities of generating toxic text, here we see the qualitative impact of our method having a lower average max toxicity: while other methods can end up producing very offensive generations, our method is able to ensure that the generation is not extremely toxic.}
\resizebox{\textwidth}{!}{
\begin{tabular}{p {.2\textwidth}|p {.8 \textwidth}}
\toprule
 \textbf{Prompt} & One student’s complaint; however, has led to the logo being associated with \\
\toprule
COLD & One student’s complaint; however, has led to the logo being associated  with the school. The school has since been forced to take the logo and the school's \\ \midrule
MuCOLA & One student’s complaint; however, has led to the logo being associated in various East Semitic cultures. What can you do to help save the culture?Who \\ \midrule
BOLT & One student’s complaint; however, has led to the logo being associated with a negative image. The school chose to remove it from their website after complaints from students and parents \\ \midrule
LM-Steer & One student’s complaint; however, has led to the logo being associated with the campaign.  And Turner had previously written Turner School Turner ’. Turner School Turner Turner \\ \midrule
DAB (Ours) & One student’s complaint; however, has led to the logo being associated with a site that is not affiliated with the University of California. The university's website and its social \\ \toprule
 \textbf{Prompt} & During a show in Sydney in 2013, Tyler went on a tirade against a member of the group, \\
\toprule
COLD & During a show in Sydney in 2013, Tyler went on a tirade against a member of the group,  calling him a "fucking faggot" and a "faggot" and a " \\ \midrule
MuCOLA & During a show in Sydney in 2013, Tyler went on a tirade against a member of the group, telling him: "I want you to go back to your f***ing country." He also apologised \\ \midrule
BOLT & During a show in Sydney in 2013, Tyler went on a tirade against a member of the group, referring to him as "a piece of shit." The singer also publicly criticized the band's decision to \\ \midrule
LM-Steer & During a show in Sydney in 2013, Tyler went on a tirade against a member of the group, writing,As for the woman who called Tyler a slut, I'm afraid you don't \\ \midrule
DAB (Ours) & During a show in Sydney in 2013, Tyler went on a tirade against a member of the group, who was considered to be his friend. The man had been seen with the band at a party and \\ 
\bottomrule
\end{tabular}
}
\label{appendix:tab:detox-gens}
\end{table}



\subsection{Keyword Controlled Generation}
\label{appndx:keywords-details}
\paragraph{Experiment Design}
We largely follow the experimental design introduced in \citet{liu2023bolt}. We list the topics and keywords in \ref{table:kw-topics}. 

\paragraph{Constraint Function} Following \citet{liu2023bolt}, \citet{qin2022cold}, we use the differentiable BLEU score introduced by \citet{liu-etal-2022-dont}. 
This function measures the uni-gram similarity between the generated sentences and the target key-words, using an operation very similar to convolution. 


\begin{table}
\caption{List of topics and correspending keywords.}
\label{table:kw-topics}
\centering
\begin{tabular}{p {.15 \textwidth} | p {.25 \textwidth}}\toprule
      \textbf{Topic}& \textbf{Keywords} \\\midrule
computer & router, Linux, keyboard, server \\ \midrule
legal & plea, subpoena, transcript, bankrupt \\ \midrule
military & torpedo, headquarters, infantry, battlefield \\ \midrule
politics & court, culture, communism, capitilism\\ \midrule
religion & Bible, church, priest, saint \\ \midrule
science & microscope, mass, mineral, scientist\\ \midrule
space & meteor, planet, satellite, astronaut
\\\bottomrule
\end{tabular}
\end{table}

\paragraph{Reference Text Generation} We use GPT-4o to generate high-quality reference text to use in the BertScore computation. For a given topic t and keyword k, we query GPT-4o with the following prompt: 

\textit{Given the topic t and the keyword k, write 30 different, unique sentences using the keyword and relevant to the topic.}

We do this for each topic and for every keyword for that topic. This produces 120 different, unique sentences to use as a reference text in the BertScore computation. 

\paragraph{BertScore Computation Details}
We use the BertScore computation introduced in \citet{zhang2020bertscoreevaluatingtextgeneration} to evaluate the topicality of the generations. Since BertScore relies on the contextualized embedding of the candidate generations and the reference text, this provides insight into how well the methods use the keyword in the desired context. 

For each generation, we compute the BertScore against all the 120 reference sentences for the corresponding prompt and keyword. Because some of the reference text will not contain the keyword used in the generation, we use report the precision metric calculated in BertScore instead of the overall F1 score, as the precision metric matches tokens in the candidate generation to tokens in the reference text. This is preferable as we want to assess whether the generation is similar to any of the reference texts, as opposed to measuring whether all the reference texts are similar to the candidate generation.  

\paragraph{Implementation Details}
We found that in order to obtain good results with DAB on this task, it was necessary to include a string containing the keywords prior to the prompt. More specifically, we included the following string before the initial prompt for keywords $K$ and topic $t$: 

\textit{Include the following keywords: K relevant to t.}

By including the target keywords and topic before the prompt, this increases the probability of these words and similar words in the underlying language model distribution. This enables the bias vectors computed in our method to have a more impact on auto-regressive generation process and thus satisfy the external constraint. 

In order to ensure that this was not providing our method with an unfair advantage, we applied the same trick to BOLT in order to determine whether this would improve the performance of BOLT as well. We provide results in Table \ref{table:kw-prompted-comp}. 
\begin{table}[t]
\caption{Comparison on topic-guided generation between the original BOLT method, the prompted BOLT method, and DAB. As visible, even if the prompt manages to improve the success rate by $.7\%$, this comes at the cost of worse fluency and slightly worse topicality. Furthermore, our method still outperforms this baseline.
}
\label{table:kw-prompted-comp}
\centering

\resizebox{\textwidth}{!}{\begin{tabular}{lcc|ccc}
\Xhline{1pt}\\[-1ex]
& \multicolumn{2}{c|}{\textbf{Control}} & \multicolumn{3}{c}{\textbf{Fluency}} \\ 
\textbf{Topic}& \textit{BertScore} $\uparrow$ & \textit{Success Rate} $\uparrow$ & \textit{CoLA} $\uparrow$ & \textit{REP-3gram} $\downarrow$ & \textit{PPL} $\downarrow$ \\
\midrule
BOLT      & $.8291 \pm .0003$     & $99.1\%$                  & $.705 \pm .006$    & $.005 \pm .005$      & $32.019 \pm 1.593$   \\
BOLT (Prompted)       & $.8123 \pm .0002$     & $\mathbf{99.7\%}$          & $.705 \pm .005$    & $.005 \pm .001$      & $38.22 \pm .951$  \\
DAB  \textit{(Ours)}    & $\mathbf{.8303 \pm .0003}$ & $99.0\%$             & $\mathbf{.726 \pm .005}$ & $\mathbf{.004 \pm .001}$ & $\mathbf{23.424 \pm .317}$ \\
\bottomrule
\end{tabular}}
\end{table}

As visible, while the prompt does improve the success rate marginally, it does not improve any other metrics for BOLT. In fact, we see that this degrades BOLT's fluency slightly through a higher perplexity value. 
\paragraph{Examples} In Table \ref{appendix:tab:kw-gens} we show examples of generations for the various samplers we examine. 
\begin{table}[h]
\centering
\caption{Examples for Topic-Constrained Generation. As visible, while previous methods include the keyword, they tend to either repeat the keyword too many times or misuse the keyword. In contrast, our method is able to include the keyword in a meaningful way relevant to the given topic.}
\resizebox{\textwidth}{!}{
\begin{tabular}{p {.2 \textwidth}| p {.8 \textwidth}}
\toprule
\textbf{Prompt} & Once upon a time \\
\textbf{Topic} & Military \\ 
\textbf{Keywords} & torpedo, headquarters, infantry, battlefield \\
\toprule
\textit{COLD} & Once upon a time, the world was a peaceful place. People were \textbf{headquarters} of the world \textbf{headquarters} of the world \textbf{torpedo}- \\ \midrule
\textit{MuCOLA} & Once upon a time, the world was a world of the great \textbf{battlefield} the powerful \textbf{headquarters} a \textbf{torpedo} of the good and \textbf{infantry}\\ \midrule
\textit{BOLT} & Once upon a time, there was a man named John Smith who had a dream that he would be able to \textbf{infantry} his \\  \midrule
\textit{DAB} (Ours) & Once upon a time, there was a small group of officers who were in charge of the modern \textbf{infantry} and logistics. They \\ \toprule
\textbf{Prompt} &  The book \\
\textbf{Topic} &  Science \\
\textbf{Keywords} & microscope, mass, mineral, scientist \\
\midrule
\textit{COLD} & The book is scientist-driven, and is a scientist mineralogist, \textbf{microscope}, \textbf{microscope}, \textbf{mineral} \textbf{microscope}, \\ \toprule
\textit{MuCOLA} & The book also has \textbf{mass}ive properties, like the Alabaster House, which features extensive characters from Alabaster \\ \midrule
\textit{BOLT} & The book is divided into three parts, each of which contains a chapter \textbf{mass} mineral scientist relevant to science. scientist \\  \midrule
\textit{DAB} (Ours) & The book is a good introduction to the field of \textbf{mass} spectrometry and is an excellent resource for hands- \\ 
\bottomrule
\end{tabular}}
\label{appendix:tab:kw-gens}
\end{table}



\section{Classifier Experiments}\label{app:cls_exps}
\begin{table}[t!]
    \centering
    \caption{Classification performance and ablation studies. We compare three variants: orientation classifier without texture score, texture classifier without orientation score, and the complete model using both scores.}
    \label{table:class}
    \begin{tabular}{lccc}
    \hline
    \textbf{Metric} & \textbf{Full} & \textbf{Orient} & \textbf{Texture} \\ \hline
    Average Accuracy           & 0.7846 & 0.6328 & 0.7699 \\ 
    Average Precision     & 0.7986 & 0.6718 & 0.8013 \\ 
    Average Recall        & 0.7426 & 0.5934 & 0.7396 \\ 
    Average F1 Score      & 0.7439 & 0.5942 & 0.7308 \\ \hline
    \end{tabular}
    
    \end{table}
    

In this appendix, we evaluate our pose classifier's performance using the annotations described in Sec.~\ref{sec:exp_settings}. We assess both the texture and orientation branches independently. Table~\ref{table:class} presents the classification performance of the main classifier and its two variants. This ablation study validates our chosen classifier architecture.
\section{Validation Experiments on 2D Sampling}\label{app:val_exps}





We validate the performance of uniform score distillation using a set of 2D particles. Starting with 16 initialized particles, we conduct training over 4,000 iterations. The comparison between USD and variational score distillation is illustrated in Fig.~\ref{fig:app_val_2d}. Our results demonstrate that USD successfully achieves a more balanced distribution compared to the biased pre-trained distribution.


\begin{figure*}[t!]
    \centering

    \begin{minipage}[c]{0.48\linewidth}
        \centering
        \parbox{1\linewidth}{\centering ``Samurai koala bear.''}\vspace{-3mm}
        \subfloat[VSD]{
            \begin{minipage}[c]{0.45\linewidth}
                \includegraphics[width=\linewidth]{./resource/validation/ori/ori_bear.png}
            \end{minipage}
        }\hspace{-2mm}
        \subfloat[USD]{
            \begin{minipage}[c]{0.45\linewidth}
                \includegraphics[width=\linewidth]{./resource/validation/ori/usd_bear.png}
            \end{minipage}
        }
    \end{minipage}
    \begin{minipage}[c]{0.48\linewidth}
        \centering
        \parbox{1\linewidth}{\centering ``A kangaroo wearing boxing gloves.''}\vspace{-3mm}
        \subfloat[VSD]{
            \begin{minipage}[c]{0.45\linewidth}
                \includegraphics[width=\linewidth]{./resource/validation/ori/ori_kangaroo.png}
            \end{minipage}
        }\hspace{-2mm}
        \subfloat[USD]{
            \begin{minipage}[c]{0.45\linewidth}
                \includegraphics[width=\linewidth]{./resource/validation/ori/usd_kangaroo.png}
            \end{minipage}
        }
    \end{minipage}


    \caption{2D score distillation comparing VSD~\citep{wang2024prolificdreamer} and USD. The prompts are augmented with auxiliary view descriptions (``from side view, from back view'') to capture multi-perspective information. Due to the original distribution's bias toward back-view angles, VSD generates predominantly back-view results, while USD successfully rectifies this distributional bias to produce more balanced viewpoints.}
    \label{fig:app_val_2d}
\end{figure*}




Fig.~\ref{fig:app_val_pc} shows the pose distribution statistics across 10 intervals for both USD and VSD. Here, $\bar{p}_t(\bar{c}|y)$, which represents the expectation of $\bar{p}_t(\bar{c}|\boldsymbol{x},y)$ over $\boldsymbol{x}_t$, indicates the current pose distribution and reveals training bias progression. While VSD exhibits a strong bias toward specific distributions during training (due to the usage of auxiliary prompts), our method maintains an approximately uniform distribution throughout the process.


\begin{figure*}[t!]
    \centering

    \begin{minipage}[c]{1\linewidth}
        \centering
        \parbox{1\linewidth}{\centering $\bar{p}_t(\bar{c}|y)$ for VSD}\vspace{-3mm}
        \subfloat{
            \begin{minipage}[c]{1\linewidth}
                \includegraphics[width=\linewidth]{./resource/validation/2d/vsd_bear_curves.pdf}
            \end{minipage}
        }

        \parbox{1\linewidth}{\centering $\bar{p}_t(\bar{c}|y)$ for USD}\vspace{-3mm}
        \subfloat{
            \begin{minipage}[c]{1\linewidth}
                \includegraphics[width=\linewidth]{./resource/validation/2d/usd_bear_curves.pdf}
            \end{minipage}
        }

    %    \vspace{-2mm}
        
    \end{minipage}

    \caption{Comparison of pose probability distributions $\bar{p}_{t}(\bar{c}|y)$ between VSD~\citep{wang2024prolificdreamer} and USD across different timestep intervals $t$. While VSD converges to a biased distribution, USD maintains an approximately uniform distribution across camera poses.}
    \label{fig:app_val_pc}
\end{figure*}



\section{Discussion on Limitations and Future Works}\label{app:ext_pose}

\begin{figure*}[t!]
    \centering


    \begin{minipage}[c]{0.95\linewidth}
        \centering
        \parbox{1\linewidth}{\centering ``A person's face.''}\vspace{-3mm}
        \subfloat[Original (VSD)]{
            \begin{minipage}[c]{0.24\linewidth}
                \includegraphics[width=\linewidth]{./resource/other_bias/ori_person.png}
            \end{minipage}
        }\hspace{-2mm}
        \subfloat[Rectified (USD)]{
            \begin{minipage}[c]{0.24\linewidth}
                \includegraphics[width=\linewidth]{./resource/other_bias/rec_person.png}
            \end{minipage}
        }\hspace{-2mm}
        \subfloat[Control (male)]{
            \begin{minipage}[c]{0.24\linewidth}
                \includegraphics[width=\linewidth]{./resource/other_bias/male_person.png}
            \end{minipage}
        }\hspace{-2mm}
        \subfloat[Contorl (female)]{
            \begin{minipage}[c]{0.24\linewidth}
                \includegraphics[width=\linewidth]{./resource/other_bias/female_person.png}
            \end{minipage}
        }
    \end{minipage}

    \caption{Addressing semantic distributional bias using a CLIP~\citep{radford2021learning} classifier. (a) Results from VSD~\citep{wang2024prolificdreamer} exhibit inherent gender bias, predominantly generating female subjects. (b) By incorporating a CLIP-based male/female classifier, our method achieves balanced gender distribution. (c) and (d) demonstrate fine-grained control over specific gender attributes, enabling targeted generation of male and female subjects respectively.}
    \label{fig:app_bias}
\end{figure*}



\begin{figure*}[t!]
    \centering
    \parbox{1\linewidth}{\centering ``A platypus, dressed in a video game pixelated costume, steps on a pixelated surfboard and holds a squid weapon that emits 8-bit light effects.''}
    \begin{minipage}[c]{0.49\linewidth}
        \centering
        \subfloat[Tripo AI v2]{
            \begin{minipage}[c]{0.25\linewidth}
                \includegraphics[width=\linewidth]{./resource/3D+2D/tripoai/render_0}\vspace{-1.mm}\\
                \includegraphics[width=\linewidth]{./resource/3D+2D/tripoai/render_4}
            \end{minipage}\hspace{-1.mm}
            \begin{minipage}[c]{0.25\linewidth}
                \includegraphics[width=\linewidth]{./resource/3D+2D/tripoai/render_1}\vspace{-1.mm}\\
                \includegraphics[width=\linewidth]{./resource/3D+2D/tripoai/render_5}
            \end{minipage}\hspace{-1.mm}
            \begin{minipage}[c]{0.25\linewidth}
                \includegraphics[width=\linewidth]{./resource/3D+2D/tripoai/render_2}\vspace{-1.mm}\\
                \includegraphics[width=\linewidth]{./resource/3D+2D/tripoai/render_6}
            \end{minipage}\hspace{-1.mm}
            \begin{minipage}[c]{0.25\linewidth}
                \includegraphics[width=\linewidth]{./resource/3D+2D/tripoai/render_3}\vspace{-1.mm}\\
                \includegraphics[width=\linewidth]{./resource/3D+2D/tripoai/render_7}
            \end{minipage}\hspace{-1.mm}
        }
    \end{minipage}
    \begin{minipage}[USD with control]{0.49\linewidth}
        \centering
        \subfloat[USD+control]{
            \begin{minipage}[c]{0.25\linewidth}
                \includegraphics[width=\linewidth]{./resource/3D+2D/ours/rgb_0}\vspace{-1.mm}\\
                \includegraphics[width=\linewidth]{./resource/3D+2D/ours/rgb_4}
            \end{minipage}\hspace{-1.mm}
            \begin{minipage}[c]{0.25\linewidth}
                \includegraphics[width=\linewidth]{./resource/3D+2D/ours/rgb_1}\vspace{-1.mm}\\
                \includegraphics[width=\linewidth]{./resource/3D+2D/ours/rgb_5}
            \end{minipage}\hspace{-1.mm}
            \begin{minipage}[c]{0.25\linewidth}
                \includegraphics[width=\linewidth]{./resource/3D+2D/ours/rgb_2}\vspace{-1.mm}\\
                \includegraphics[width=\linewidth]{./resource/3D+2D/ours/rgb_6}
            \end{minipage}\hspace{-1.mm}
            \begin{minipage}[c]{0.25\linewidth}
                \includegraphics[width=\linewidth]{./resource/3D+2D/ours/rgb_3}\vspace{-1.mm}\\
                \includegraphics[width=\linewidth]{./resource/3D+2D/ours/rgb_7}
            \end{minipage}\hspace{-1.mm}
        }
    \end{minipage}
    \caption{Demonstration of the integration of USD with 3D-based methods (Tripo AI).}
    \label{fig:app_demo_future}
\end{figure*}




% \begin{figure*}[t!]
%     \centering

%     \begin{minipage}[c]{1\linewidth}
%         \centering
%         \parbox{1\linewidth}{\centering ``Samurai koala bear.''}\vspace{-3mm}
%         \subfloat{
%             \begin{minipage}[c]{0.45\linewidth}
%                 \includegraphics[width=\linewidth]{./resource/samples/sd/camera/front.png}
%             \end{minipage}
%         }\hspace{-2mm}
%         \subfloat{
%             \begin{minipage}[c]{0.45\linewidth}
%                 \includegraphics[width=\linewidth]{./resource/samples/sd/camera/back.png}
%             \end{minipage}
%         }\hspace{-2mm}
%         \subfloat{
%             \begin{minipage}[c]{0.45\linewidth}
%                 \includegraphics[width=\linewidth]{./resource/samples/sd/camera/left.png}
%             \end{minipage}
%         }\hspace{-2mm}
%         \subfloat{
%             \begin{minipage}[c]{0.45\linewidth}
%                 \includegraphics[width=\linewidth]{./resource/samples/sd/camera/right.png}
%             \end{minipage}
%         }
%     \end{minipage}
%     \caption{Reference images.}
%     \label{fig:app_ref_img}
% \end{figure*}


This discussion examines our work's boundaries while identifying promising paths for subsequent research. We identify several key limitations and opportunities for advancement.

A primary limitation of this work is generation speed. The bottleneck lies in the U-Net~\citep{ronneberger2015u} gradient back-propagation introduced by the rectifier function, which requires further optimization. Future research could explore methods to effectively bypass U-Net gradient back-propagation or develop a score-free optimization framework similar to MicroDreamer~\citep{chen2024microdreamer}.


Another significant challenge concerns 3D consistency of localized features. While USD eliminates bias in the overall data distribution, its reliance on the score distillation algorithm, which lacks explicit geometric consistency supervision, can lead to geometrically inconsistent content, potentially limiting practical applications. Addressing this limitation requires incorporating multi-perspective supervision during generation. Notably, the special case discussed in Appendix~\ref{app:main_exps_control} demonstrates a potential supervision mechanism for score distillation that warrants further investigation.


Beyond current technical limitations, we propose new directions for control-based synthesis that expand on cross-modal approaches (see Appendix~\ref{app:main_exps_cross}). For instance, Fig.~\ref{fig:app_demo_future} demonstrates an experiment using an imaginative prompt. As shown in Fig.~\ref{fig:app_demo_future}(a), the 3D generative model Tripo AI v2\footnote{https://lumalabs.ai/genie} captures basic geometric elements effectively, but still faces challenges when interpreting more abstract or imaginative descriptions (\ie, ``pixelated costume'' and ``pixelated surfboard'') due to its 3D modeling constraints. In contrast, our approach leverages selected Tripo AI renderings for pose control (Fig.~\ref{fig:app_demo_future}(a)), resulting in a more accurate prancing effect that better matches the text description, as demonstrated in Fig.~\ref{fig:app_demo_future}(b). While our model is trained from scratch and may lack geometric refinement, fine-tuning it from a geometrically consistent base model~\citep{zheng2024learning} can yield results that excel in both geometric accuracy and textural detail.


Finally, highlighting the versatility of our approach, our USD algorithm demonstrates considerable extensibility beyond pose classification and 3D generation. Fig.~\ref{fig:app_bias} showcases its application in addressing gender distribution bias in image generation using the CLIP~\citep{radford2021learning} classifier, enabling independent control over gender representation. This adaptability suggests that USD could be applied to address other forms of algorithmic bias with different classifier architectures.






\begin{figure*}[t]
    \centering
    \includegraphics[width=0.95\linewidth]{latex//image/crop_appendix_prompt.pdf}
    \caption{Prompts Used for Different Tasks in Our M$^2$RAG Benchmark.}
    \label{fig:prompts}
\end{figure*}
\section{Evaluation Results}

\begin{table*}[t]
\centering
\renewcommand{\arraystretch}{1.5} 
\resizebox{2\columnwidth}{!}{%
%\rotatebox{90}{
\begin{tabular}{l|ccc|ccc|ccc}
\toprule
\textbf{Model} & \multicolumn{9}{c}{\textbf{Open-ended Medical Questions}} \\
 & \multicolumn{3}{c|}{\textbf{\careqa{}-Open}} & \multicolumn{3}{c|}{\textbf{MedDialog Raw}} & \multicolumn{3}{c}{\textbf{MediQA2019}} \\
 & \textbf{Bits per Byte $\downarrow$} & \textbf{Byte Perplexity $\downarrow$} & \textbf{Word Perplexity $\downarrow$} & \textbf{Bits per Byte $\downarrow$} & \textbf{Byte Perplexity $\downarrow$} & \textbf{Word Perplexity $\downarrow$} & \textbf{Bits per Byte $\downarrow$} & \textbf{Byte Perplexity $\downarrow$} & \textbf{Word Perplexity $\downarrow$}\\
\midrule 
BioMistral-MedMNX & 1.302 & 2.465 & 467.349 &  1.043 & 2.060 & 74.760 & 0.416 & 1.335 & 6.044 \\
JSL-MedLlama-3-8B-v2.0 & 1.33 & 2.514 & 534.372 & 1.179 & 2.265 & 131.509 & 0.517 & 1.431 & 9.312 \\
Llama3-Med42-8B & 1.311 & 2.482 & 489.199 & 1.069 & 2.097 & 83.115 & 0.405 & 1.324 & 5.754 \\

Meta-Llama-3.1-70B-Instruct & 1.295 & 2.453 & 452.335 & 0.993 & 1.991 & 60.907 & 0.245 & 1.185 & 2.886 \\
Meta-Llama-3.1-8B-Instruct & 1.346 & 2.543 & 573.723 & 1.060 & 2.085 & 80.124 & 0.430 & 1.347 & 6.407 \\
Mistral-7B-Instruct-v0.3 & 1.442 & 2.717 & 907.864 & 1.073 & 2.104 & 84.603 & 0.420 & 1.338 & 6.145 \\
Mixtral-8x7B-Instruct-v0.1 & 1.453 & 2.738 & 956.752 & 1.028 & 2.039 & 70.258 & 0.300 & 1.232 & 3.662 \\
Phi-3-medium-4k-instruct &  1.255 & 2.387 & 375.453 & 1.068 & 2.097 & 82.957 & 0.410 & 1.329 & 5.884 \\
Phi-3-mini-4k-instruct & 1.342 & 2.535 & 566.127 & 1.082 & 2.117 & 87.936 & 0.444 & 1.360 & 6.796 \\
Qwen2-7B-Instruct & 1.468 & 2.766 & 1024.433 & 1.044 & 2.063 & 75.218 & 0.447 & 1.363 & 6.895 \\
Yi-1.5-34B-Chat & 1.533 & 2.893 & 1392.39 & 1.101 & 2.145 & 95.042 & 0.485 & 1.399 & 8.112 \\
Yi-1.5-9B-Chat & 1.537 & 2.901 & 1416.845 & 1.123 & 2.178 & 104.205 & 0.532 & 1.446 & 9.968 \\
%gemma-2-27b-it & 8.600 & 388.080 & 417929466779407872.000 & & & & 7.230	& 150.161	& 36615253021450.047 \\
%gemma-2-9b-it & 3.908 & 15.009 & 101542723.787 & 3.647 & 12.528 & 3563621.691 & 3.400 & 10.555 & 2387337.872 \\

\bottomrule
\end{tabular}%
}
%}
\caption{Perplexity results for Open-ended Medical Questions.}
\end{table*}

%\begin{table*}[h]
\centering
\renewcommand{\arraystretch}{2} 
\resizebox{2\columnwidth}{!}{%
%\rotatebox{90}{
\begin{tabular}{l|ccc|ccc|ccc|ccc}
\toprule
\textbf{Model} & \multicolumn{6}{c|}{\textbf{Clinical Note-taking}} & \multicolumn{6}{c}{\textbf{Open-ended Medical Questions}} \\
 & \multicolumn{3}{c|}{\textbf{ACI Bench}} & \multicolumn{3}{c|}{\textbf{MTS Dialog}} & \multicolumn{3}{c|}{\textbf{MedDialog Raw}} & \multicolumn{3}{c}{\textbf{MediQA2019}} \\
 & \textbf{Bits per Byte $\downarrow$} & \textbf{Byte Perplexity $\downarrow$} & \textbf{Word Perplexity $\downarrow$} & \textbf{Bits per Byte $\downarrow$} & \textbf{Byte Perplexity $\downarrow$} & \textbf{Word Perplexity $\downarrow$} & \textbf{Bits per Byte $\downarrow$} & \textbf{Byte Perplexity $\downarrow$} & \textbf{Word Perplexity $\downarrow$} & \textbf{Bits per Byte $\downarrow$} & \textbf{Byte Perplexity $\downarrow$} & \textbf{Word Perplexity $\downarrow$} \\
\midrule 
BioMistral-MedMNX & 0.601 & 1.517 & 13.894 & 1.059 & 2.083 & 132.827 & 1.043 & 2.060 & 74.760 & 0.416 & 1.335 & 6.044 \\
JSL-MedLlama-3-8B-v2.0 & 0.703 & 1.628 & 21.725 & 1.099 & 2.143 & 160.188 & 1.179 & 2.265 & 131.509 & 0.517 & 1.431 & 9.312 \\
Llama3-Med42-8B & 0.485 & 1.399 & 8.357 & 1.060 & 2.085 & 133.416 & 1.069 & 2.097 & 83.115 & 0.405 & 1.324 & 5.754 \\
Meta-Llama-3.1-70B-Instruct & - & - & - & 0.984 & 1.978 & 93.943 & 0.993 & 1.991 & 60.907 & 0.245 & 1.185 & 2.886 \\
Meta-Llama-3.1-8B-Instruct & 0.612 & 1.529 & 14.618 & 1.074 & 2.105 & 142.211 & 1.060 & 2.085 & 80.124 & 0.430 & 1.347 & 6.407 \\
Mistral-7B-Instruct-v0.3 & 0.596 & 1.512 & 13.628 & 1.053 & 2.074 & 129.076 & 1.073 & 2.104 & 84.603 & 0.420 & 1.338 & 6.145 \\
Mixtral-8x7B-Instruct-v0.1 & 0.566 & 1.481 & 11.933 & 1.046 & 2.064 & 125.070 & 1.028 & 2.039 & 70.258 & 0.300 & 1.232 & 3.662 \\
Phi-3-medium-4k-instruct & 0.642 & 1.560 & 16.600 & 0.971 & 1.960 & 88.447 & 1.068 & 2.097 & 82.957 & 0.410 & 1.329 & 5.884 \\
Phi-3-mini-4k-instruct & 0.599 & 1.514 & 13.754 & 0.972 & 1.962 & 89.163 & 1.082 & 2.117 & 87.936 & 0.444 & 1.360 & 6.796 \\
%Qwen2-72B-Instruct & - & - & - & 0.954 & 1.938 & 82.027 & 0.938 & 1.916 & 48.395 & 0.350 & 1.274 & 4.533 \\
Qwen2-7B-Instruct & 0.619 & 1.535 & 15.009 & 1.063 & 2.089 & 135.111 & 1.044 & 2.063 & 75.218 & 0.447 & 1.363 & 6.895 \\
Yi-1.5-34B-Chat & 0.728 & 1.657 & 24.270 & 1.099 & 2.143 & 160.265 & 1.101 & 2.145 & 95.042 & 0.485 & 1.399 & 8.112 \\
Yi-1.5-9B-Chat & 0.711 & 1.636 & 22.456 & 1.180 & 2.265 & 232.073 & 1.123 & 2.178 & 104.205 & 0.532 & 1.446 & 9.968 \\
%gemma-2-27b-it & - &- & -& -& -&- & - & - & - & -& -&-\\
%gemma-2-9b-it & - & - & - & - &  - & & 3.647 & 12.528 & 3563621.691 & 3.400& 10.555 & - \\

\bottomrule
\end{tabular}%
}
%}
\caption{Perplexity results for Open-ended Medical Questions.}
\end{table*}
\begin{table*}[h]
\centering
\renewcommand{\arraystretch}{1.5} 
\resizebox{2\columnwidth}{!}{%
\begin{tabular}{l|ccc|ccc|ccc}
\toprule
\textbf{Model} & \multicolumn{6}{c}{\textbf{Clinical Note-taking}} & \multicolumn{3}{c}{\textbf{Medical factuality}}\\
 & \multicolumn{3}{c|}{\textbf{ACI Bench}} & \multicolumn{3}{c|}{\textbf{MTS Dialog}} & \multicolumn{3}{c}{\textbf{OLAPH}}\\
 & \textbf{Bits per Byte $\downarrow$} & \textbf{Byte Perplexity $\downarrow$} & \textbf{Word Perplexity $\downarrow$} & \textbf{Bits per Byte $\downarrow$} & \textbf{Byte Perplexity $\downarrow$} & \textbf{Word Perplexity $\downarrow$} & \textbf{Bits per Byte $\downarrow$} & \textbf{Byte Perplexity $\downarrow$} & \textbf{Word Perplexity $\downarrow$} \\
\midrule 
BioMistral-MedMNX & 0.601 & 1.517 & 13.894 & 1.059 & 2.083 & 132.827 & 0.447 & 1.363 & 7.138\\
JSL-MedLlama-3-8B-v2.0 & 0.703 & 1.628 & 21.725 & 1.099 & 2.143 & 160.188 & 0.523& 1.437 & 9.978  \\
Llama3-Med42-8B & 0.485 & 1.399 & 8.357 & 1.060 & 2.085 & 133.416 & 0.450 & 1.366 & 7.211 \\
Meta-Llama-3.1-70B-Instruct & - & - & - & 0.984 & 1.978 & 93.943 & 2.202 & 4.601 & 15946.837 \\
Meta-Llama-3.1-8B-Instruct & 0.612 & 1.529 & 14.618 & 1.074 & 2.105 & 142.211 & 2.181 & 4.533 & 14513.067 \\
Mistral-7B-Instruct-v0.3 & 0.596 & 1.512 & 13.628 & 1.053 & 2.074 & 129.076 & 0.438 & 1.355 & 6.858 \\
Mixtral-8x7B-Instruct-v0.1 & 0.566 & 1.481 & 11.933 & 1.046 & 2.064 & 125.070 & 3.643 & 12.497 & 8992823.856 \\
Phi-3-medium-4k-instruct & 0.642 & 1.560 & 16.600 & 0.971 & 1.960 & 88.447 & 0.393 & 1.313 & 5.620 \\
Phi-3-mini-4k-instruct & 0.599 & 1.514 & 13.754 & 0.972 & 1.962 & 89.163 & 0.407 & 1.326 & 5.986 \\
Qwen2-7B-Instruct & 0.619 & 1.535 & 15.009 & 1.063 & 2.089 & 135.111 & 0.455 & 1.371 & 7.384 \\
Yi-1.5-34B-Chat & 0.728 & 1.657 & 24.270 & 1.099 & 2.143 & 160.265 & 2.798 & 6.955 & 218855.290 \\
Yi-1.5-9B-Chat & 0.711 & 1.636 & 22.456 & 1.180 & 2.265 & 232.073 & 0.571 & 1.485 & 12.281 \\
%gemma-2-27b-it & - &- &- &- &- &- \\
%gemma-2-9b-it &- & -&- &- &- & -\\
\bottomrule
\end{tabular}%
}
\caption{Perplexity results for clinical note-taking and medical factuality.}
\end{table*}
\begin{table*}[h]
\centering
\renewcommand{\arraystretch}{1.5} 
\resizebox{2\columnwidth}{!}{%
%\rotatebox{90}{
\begin{tabular}{l|ccc|ccc|ccc}
\toprule
\textbf{Model} & \multicolumn{3}{c|}{\textbf{Making treatment recommendations}} & \multicolumn{3}{c|}{\textbf{Question Entailment}} & \multicolumn{3}{c}{\textbf{Summarization}} \\
 & \multicolumn{3}{c|}{\textbf{MedText}} & \multicolumn{3}{c|}{\textbf{MedDialog Qsumm}} & \multicolumn{3}{c}{\textbf{Mimic-III}} \\
 & \textbf{Bits per Byte $\downarrow$} & \textbf{Byte Perplexity $\downarrow$} & \textbf{Word Perplexity $\downarrow$} & \textbf{Bits per Byte $\downarrow$} & \textbf{Byte Perplexity $\downarrow$} & \textbf{Word Perplexity $\downarrow$} & \textbf{Bits per Byte $\downarrow$} & \textbf{Byte Perplexity $\downarrow$} & \textbf{Word Perplexity $\downarrow$} \\
\midrule 
BioMistral-MedMNX & 0.499 & 1.413 & 10.605 & 1.471 & 2.772 & 275.846 & 1.771 & 3.413 & 4697.580 \\
JSL-MedLlama-3-8B-v2.0 & 0.556 & 1.470 & 13.868 & 1.715 & 3.282 & 699.785 & 2.035 & 4.099 & 16607.943 \\
Llama3-Med42-8B & 0.455 & 1.370 & 8.593 & 1.359 & 2.564 & 179.527 & 1.839 & 3.577 & 6489.224 \\
Meta-Llama-3.1-70B-Instruct & 0.447 & 1.364 & 8.298 & 1.280 & 2.428 & 132.988 &  - & - & - \\
Meta-Llama-3.1-8B-Instruct & 0.534 & 1.448 & 12.501 & 1.371 & 2.587 & 188.513 & 1.826 & 3.545 & 6106.099 \\
Mistral-7B-Instruct-v0.3 & 0.510 & 1.424 & 11.163 & 1.447 & 2.727 & 251.938 & 1.790 & 3.457 & 5138.524 \\
Mixtral-8x7B-Instruct-v0.1 & 0.491 & 1.405 & 10.194 & 1.370 & 2.586 & 187.912 & 1.679 & 3.202 & 3028.534 \\
Phi-3-medium-4k-instruct & 0.423 & 1.341 & 7.400 & 1.332 & 2.517 & 162.163 & 2.084 & 4.239 & 20901.351 \\
Phi-3-mini-4k-instruct & 0.438 & 1.355 & 7.956 & 1.311 & 2.481 & 149.718 & 1.902 & 3.737 & 8784.663 \\
%Qwen2-72B-Instruct & 0.459 & 1.375 & 8.791 & 1.283 & 2.433 & 134.531 & -  & -  &  - \\
Qwen2-7B-Instruct & 0.527 & 1.441 & 12.106 & 1.383 & 2.608 & 197.167 & 1.878 & 3.676 & 7839.132 \\
Yi-1.5-34B-Chat & 0.556 & 1.470 & 13.875 & 1.437 & 2.708 & 242.427 & 2.202  & 4.600 & 36704.322 \\
Yi-1.5-9B-Chat & 0.559 & 1.473 & 14.052 & 1.470 & 2.771 & 275.222 & 2.341 & 5.067 & 71436.330 \\
%gemma-2-27b-it & 6.629 & 99.000 & 41665378260772.586 & 7.945 &  246.410 & 15225223182540.287 & -& -&- \\
%gemma-2-9b-it & 3.125 & 8.721 & 2626055.604 & 3.836& 14.277 & 2313011.794 &- & - & - \\
\bottomrule
\end{tabular}%
}
%}
\caption{Perplexity results for the following tasks: making diagnosis and treatment recommendation, question entailment and summarization tasks.}
\end{table*}
%\begin{table*}[h]
\centering
\renewcommand{\arraystretch}{1.5} 
\resizebox{2.2\columnwidth}{!}{%
%\rotatebox{90}{
\begin{tabular}{l|ccc|ccc|ccc|ccc|ccc}
\toprule
\textbf{Model} & \multicolumn{6}{c|}{\textbf{Clinical Note-taking}} & \multicolumn{3}{c|}{\textbf{Open-ended Medical Questions}} & \multicolumn{3}{c|}{\textbf{Question Entailment}} & \multicolumn{3}{c}{\textbf{Summarization}} \\
 & \multicolumn{3}{c|}{\textbf{ACI Bench}} & \multicolumn{3}{c|}{\textbf{MTS Dialog}} & \multicolumn{3}{c|}{\textbf{MedDialog Raw}} & \multicolumn{3}{c|}{\textbf{MedDialog Qsumm}} & \multicolumn{3}{c}{\textbf{Mimic-III}} \\
 & \textbf{Bits per Byte $\downarrow$} & \textbf{Byte Perplexity $\downarrow$} & \textbf{Word Perplexity $\downarrow$} & \textbf{Bits per Byte $\downarrow$} & \textbf{Byte Perplexity $\downarrow$} & \textbf{Word Perplexity $\downarrow$} & \textbf{Bits per Byte $\downarrow$} & \textbf{Byte Perplexity $\downarrow$} & \textbf{Word Perplexity $\downarrow$} & \textbf{Bits per Byte $\downarrow$} & \textbf{Byte Perplexity $\downarrow$} & \textbf{Word Perplexity $\downarrow$} & \textbf{Bits per Byte $\downarrow$} & \textbf{Byte Perplexity $\downarrow$} & \textbf{Word Perplexity $\downarrow$} \\
\midrule 
BioMistral-MedMNX & 601 & 1517 & 13894 & 1059 & 2083 & 132827 & 1043 & 2060 & 74760 & 1471 & 2772 & 275846 & 1771 & 3413 & 4697580 \\
JSL-MedLlama-3-8B-v2.0 & 703 & 1628 & 21725 & 1099 & 2143 & 160188 & 1179 & 2265 & 131509 & 1715 & 3282 & 699785 & 2035 & 4099 & 16607943 \\
Llama3-Med42-8B & 485 & 1399 & 8357 & 1060 & 2085 & 133416 & 1069 & 2097 & 83115 & 1359 & 2564 & 179527 & 1839 & 3577 & 6489224 \\
Meta-Llama-3.1-70B-Instruct &  &  &  & 984 & 1978 & 93943 & 993 & 1991 & 60907 & 1280 & 2428 & 132988 &  &  &  \\
Meta-Llama-3.1-8B-Instruct & 612 & 1529 & 14618 & 1074 & 2105 & 142211 & 1060 & 2085 & 80124 & 1371 & 2587 & 188513 & 1826 & 3545 & 6106099 \\
Mistral-7B-Instruct-v0.3 & 596 & 1512 & 13628 & 1053 & 2074 & 129076 & 1073 & 2104 & 84603 & 1447 & 2727 & 251938 & 1790 & 3457 & 5138524 \\
Mixtral-8x7B-Instruct-v0.1 & 566	& 1481	& 11933	& 1046	& 2064	& 125070	& 1028	& 2039	& 70258	& 1370	& 2586	& 187912	& 1679	& 3202	& 3028534 \\
Phi-3-medium-4k-instruct & 642 & 1560 & 16600 & 971 & 1960 & 88447 & 1068 & 2097 & 82957 & 1332 & 2517 & 162163 & 2084 & 4239 & 20901351 \\
Phi-3-mini-4k-instruct & 599 & 1514 & 13754 & 972 & 1962 & 89163 & 1082 & 2117 & 87936 & 1311 & 2481 & 149718 & 1902 & 3737 & 8784663 \\
Qwen2-72B-Instruct &  &  &  & 954 & 1938 & 82027 & 938 & 1916 & 48395 & 1283 & 2433 & 134531 &  &  &  \\
Qwen2-7B-Instruct & 619 & 1535 & 15009 & 1063 & 2089 & 135111 & 1044 & 2063 & 75218 & 1383 & 2608 & 197167 & 1878 & 3676 & 7839132 \\
Yi-1.5-34B-Chat &  &  &  & 1099 & 2143 & 160265 & 1101 & 2145 & 95042 & 1437 & 2708 & 242427 &  &  &  \\
Yi-1.5-9B-Chat &  
711	& 1636	& 22456 & 1180 & 2265 & 232073 & 1123 & 2178 & 104205 & 1470 & 2771 & 275222 & 2341 & 5067 & 71436330 \\
\bottomrule
\end{tabular}%
}
%}
\caption{Perplexity results}
\end{table*}

\begin{table*}[h]
\centering
\renewcommand{\arraystretch}{1.5} 
\resizebox{0.7\columnwidth}{!}{%
\begin{tabular}{l|ccc|ccc}
\toprule
\textbf{Model} & \multicolumn{2}{c}{\textbf{Medical factuality}}\\
&  \multicolumn{2}{c}{\textbf{OLAPH}}\\
 & \textbf{Relaxed perplexity logprobs $\uparrow$} & \textbf{Relaxed perplexity $\downarrow$}\\
\midrule 

BioMistral-MedMNX & -33.122 & 81.532 \\
JSL-MedLlama-3-8B-v2.0 & -39.281 & 12.324 \\
Llama3-Med42-8B & -37.015 & 32.38 \\
Meta-Llama-3.1-70B-Instruct & - & - \\
Meta-Llama-3.1-8B-Instruct & -35.989 & 129.07 \\
Mistral-7B-Instruct-v0.3 & -34.513 & 27.64 \\
Mixtral-8x7B-Instruct-v0.1 & -33.810 & 23.045 \\
Phi-3-medium-4k-instruct & -33.157 & 44.207 \\
Phi-3-mini-4k-instruct & -33.567 & 74.641 \\
Qwen2-7B-Instruct & -37.247 & 133.359 \\
Yi-1.5-34B-Chat & -44.076 & 198.635 \\
Yi-1.5-9B-Chat & -44.501 & 352.381 \\
%gemma-2-9b-it & -41.824 & 142.559 \\
\bottomrule
\end{tabular}%
}
\caption{Relaxed perplexity results for medical factuality.}
\end{table*}

%\begin{table*}[H]
\centering
\renewcommand{\arraystretch}{1.5} 
\resizebox{2\columnwidth}{!}{%
%\rotatebox{90}{
\begin{tabular}{l|c|c|c|c|c}
\toprule
\textbf{Model} & \textbf{Treatment recommendations} & \textbf{Question Entailment} & \textbf{Summarization} & \multicolumn{2}{c|}{\textbf{Clinical Note-Taking}}\\
 & \textbf{MedText} & \textbf{MedDialog Qsumm} & \textbf{Mimic-III} & \textbf{ACI Bench} & \textbf{MTS Dialog} \\
& \multicolumn{5}{c}{\textbf{Prometheus $\uparrow$}}\\
\midrule 
BioMistral-MedMNX           & 0.225 ± 0.063 & 0.163 ± 0.005 & 0.330 ± 0.016 & 0.273 ± 0.027 & 0.297 ± 0.009  \\
JSL-MedLlama-3-8B-v2.0      & 0.263 ± 0.084 & 0.087 ± 0.004 & 0.298 ± 0.017 & 0.365 ± 0.031 & 0.172 ± 0.008 \\
Llama3-Med42-8B             & 0.138 ± 0.062 & 0.241 ± 0.007 & 0.213 ± 0.016 & 0.157 ± 0.024 & 0.130 ± 0.008 \\
Meta-Llama-3.1-70B-Instruct  & 0.062 ± 0.043 & 0.314 ± 0.007 & 0.342 ± 0.016 & 0.313 ± 0.026 & 0.281 ± 0.009 \\
Meta-Llama-3.1-8B-Instruct   & 0.188 ± 0.063 & 0.156 ± 0.005 & 0.263 ± 0.015 & 0.245 ± 0.027 & 0.237 ± 0.008  \\
Mistral-7B-Instruct-v0.3     & 0.050 ± 0.029 & 0.194 ± 0.006 & 0.187 ± 0.015 & 0.087 ± 0.018 & 0.055 ± 0.005 \\
Mixtral-8x7B-Instruct-v0.1   & 0.075 ± 0.036 & 0.112 ± 0.005 & 0.252 ± 0.016 & 0.090 ± 0.017 & 0.198 ± 0.009 \\
Phi-3-medium-4k-instruct     & 0.175 ± 0.064 & 0.168 ± 0.005 & 0.358 ± 0.017 & 0.190 ± 0.023 & 0.219 ± 0.008  \\
Phi-3-mini-4k-instruct       & 0.125 ± 0.057 & 0.126 ± 0.005 & 0.376 ± 0.016 & 0.287 ± 0.027 & 0.280 ± 0.009 \\
%Qwen2-72B-Instruct           & -            & 0.260 ± 0.006 & -            & 0.332 ± 0.026 & -            & -            & -            \\
Qwen2-7B-Instruct            & 0.125 ± 0.052 & 0.177 ± 0.006 & 0.267 ± 0.014 & 0.255 ± 0.026 & 0.144 ± 0.007  \\
Yi-1.5-34B-Chat              & 0.287 ± 0.069 & 0.179 ± 0.006 & 0.372 ± 0.016 & 0.342 ± 0.030 & 0.420 ± 0.008  \\
Yi-1.5-9B-Chat               & 0.138 ± 0.067 & 0.405 ± 0.007            & 0.550 ± 0.015 & 0.362 ± 0.026 & 0.397 ± 0.008 \\
%gemma-2-27b-it               & 0.000 ± 0.000 & 0.001 ± 0.000 & 0.002 ± 0.001 & 0.000 ± 0.000 & 0.000 ± 0.000  \\
%gemma-2-9b-it                & 0.000 ± 0.000 & 0.002 ± 0.001 & 0.011 ± 0.004 & 0.000 ± 0.000 & 0.000 ± 0.000 \\

\bottomrule
\end{tabular}%
}
%}
\caption{Prometheus results for the following tasks: making diagnosis and treatment recommendations, question entailment, summarization and clinical note-taking.
}
\end{table*}

\begin{table*}[h]
\centering
\renewcommand{\arraystretch}{1.5} 
\resizebox{2\columnwidth}{!}{%
%\rotatebox{90}{
\begin{tabular}{l|c|c|c|c|c}
\toprule
\textbf{Model} & \textbf{Question Entailment} & \multicolumn{3}{c|}{\textbf{Open-ended Medical Questions}}& \textbf{Treatment recommendations}\\
 & \textbf{MedDialog Qsumm} & \textbf{MedDialog Raw} & \textbf{MediQA2019} & \textbf{\careqa{}-Open}& \textbf{MedText} \\
& \multicolumn{5}{c}{\textbf{Prometheus $\uparrow$}}\\
\midrule 
BioMistral-MedMNX           & 0.163 ± 0.005 & 0.330 ± 0.016 & 0.273 ± 0.027 & 0.240 ± 0.007  & 0.297 ± 0.009  \\
JSL-MedLlama-3-8B-v2.0      & 0.087 ± 0.004 & 0.298 ± 0.017 & 0.365 ± 0.031 & 0.302 ± 0.008 & 0.172 ± 0.008 \\
Llama3-Med42-8B            & 0.241 ± 0.007 & 0.213 ± 0.016 & 0.157 ± 0.024 & 0.105 ± 0.005  & 0.130 ± 0.008 \\
Meta-Llama-3.1-70B-Instruct  & 0.314 ± 0.007 & 0.342 ± 0.016 & 0.313 ± 0.026 & 0.313 ± 0.007  & 0.281 ± 0.009 \\
Meta-Llama-3.1-8B-Instruct   & 0.156 ± 0.005 & 0.263 ± 0.015 & 0.245 ± 0.027 & 0.227 ± 0.007 & 0.237 ± 0.008  \\
Mistral-7B-Instruct-v0.3     & 0.194 ± 0.006 & 0.187 ± 0.015 & 0.087 ± 0.018 & 0.088 ± 0.005 & 0.055 ± 0.005 \\
Mixtral-8x7B-Instruct-v0.1   & 0.112 ± 0.005 & 0.252 ± 0.016 & 0.090 ± 0.017 & 0.130 ± 0.006 & 0.198 ± 0.009 \\
Phi-3-medium-4k-instruct     & 0.168 ± 0.005 & 0.358 ± 0.017 & 0.190 ± 0.023 & 0.319 ± 0.008  & 0.219 ± 0.008  \\
Phi-3-mini-4k-instruct       & 0.126 ± 0.005 & 0.376 ± 0.016 & 0.287 ± 0.027 & 0.185 ± 0.007 & 0.280 ± 0.009 \\
%Qwen2-72B-Instruct           & -            & 0.260 ± 0.006 & -            & 0.332 ± 0.026 & -            & -            & -            \\
Qwen2-7B-Instruct            & 0.177 ± 0.006 & 0.267 ± 0.014 & 0.255 ± 0.026 & 0.462 ± 0.008 & 0.144 ± 0.007  \\
Yi-1.5-34B-Chat              & 0.179 ± 0.006 & 0.372 ± 0.016 & 0.342 ± 0.030 & 0.492 ± 0.008 & 0.420 ± 0.008  \\
Yi-1.5-9B-Chat               & 0.405 ± 0.007            & 0.550 ± 0.015 & 0.362 ± 0.026 & 0.588 ± 0.007 & 0.397 ± 0.008 \\
%gemma-2-27b-it               & 0.000 ± 0.000 & 0.001 ± 0.000 & 0.002 ± 0.001 & 0.000 ± 0.000 & 0.000 ± 0.000  \\
%gemma-2-9b-it                & 0.000 ± 0.000 & 0.002 ± 0.001 & 0.011 ± 0.004 & 0.000 ± 0.000 & 0.000 ± 0.000 \\

\bottomrule
\end{tabular}%
}
%}
\caption{Prometheus results for the following tasks: question entailment, open-ended medical questions and treatment recommendations. }
\end{table*}

%\begin{table*}[H]
\centering
\renewcommand{\arraystretch}{1.7} 
\resizebox{1\columnwidth}{!}{%
%\rotatebox{90}{
\begin{tabular}{l|c|c|c}
\toprule
\textbf{Model} & \multicolumn{3}{c}{\textbf{Open-ended Medical Questions}} & \textbf{Medical Factuality}  \\
& \textbf{MedDialog Raw} & \textbf{MediQA2019} & \textbf{\careqa{}-Open}\\% & \textbf{OLAPH}\\
& \multicolumn{3}{c}{\textbf{Prometheus $\uparrow$}}\\
\midrule 
BioMistral-MedMNX  &0.535 ± 0.005 & 0.342 ± 0.007 & 0.272 ± 0.006\\
JSL-MedLlama-3-8B-v2.0   & 0.304 ± 0.005 & 0.459 ± 0.008 & 0.307 ± 0.007\\
Llama3-Med42-8B             & 0.138 ± 0.062 & 0.241 ± 0.007 & 0.230 ± 0.006\\
Meta-Llama-3.1-70B-Instruct  &  0.293 ± 0.005 & 0.326 ± 0.008 & 0.296 ± 0.006 \\
Meta-Llama-3.1-8B-Instruct   & 0.375 ± 0.005 & 0.229 ± 0.007 & 0.225 ± 0.006\\
Mistral-7B-Instruct-v0.3  & 0.476 ± 0.005 & 0.384 ± 0.008 & 0.087 ± 0.004 \\
Mixtral-8x7B-Instruct-v0.1   & 0.543 ± 0.005 & 0.361 ± 0.008 & 0.030 ± 0.002 \\
Phi-3-medium-4k-instruct  & 0.249 ± 0.005 & 0.281 ± 0.008 & 0.308 ± 0.007\\
Phi-3-mini-4k-instruct & 0.353 ± 0.005 & 0.328 ± 0.008 & 0.299 ± 0.006\\
%Qwen2-72B-Instruct           & -            & 0.260 ± 0.006 & -            & 0.332 ± 0.026 & -            & -            & -            \\
Qwen2-7B-Instruct   & 0.541 ± 0.005 & 0.267 ± 0.007 & 0.427 ± 0.007\\
Yi-1.5-34B-Chat & 0.508 ± 0.005 & 0.347 ± 0.009 & 0.488 ± 0.007 \\
Yi-1.5-9B-Chat & 0.288 ± 0.005 & 0.417 ± 0.009 & 0.612 ± 0.006 \\
%gemma-2-27b-it  & 0.001 ± 0.000 & 0.000 ± 0.000 & 0.003 ± 0.001\\
%gemma-2-9b-it & 0.002 ± 0.000 & 0.003 ± 0.001 & 0.006 ± 0.001\\

\bottomrule
\end{tabular}%
}
%}
\caption{Prometheus results for the open-ended medical questions.% and medical factuality.
}
\end{table*}

\begin{table*}[h]
\centering
\renewcommand{\arraystretch}{1.7} 
\resizebox{1\columnwidth}{!}{%
%\rotatebox{90}{
\begin{tabular}{l|c|c|c}
\toprule
\textbf{Model} & \textbf{Summarization} & \multicolumn{2}{c}{\textbf{Clinical Note-Taking}}  \\
& \textbf{Mimic-III} & \textbf{MTS Dialog} & \textbf{ACI Bench} \\% & \textbf{OLAPH}\\
& \multicolumn{3}{c}{\textbf{Prometheus $\uparrow$}}\\
\midrule 
BioMistral-MedMNX  &0.535 ± 0.005 & 0.342 ± 0.007 & 0.225 ± 0.063 \\
JSL-MedLlama-3-8B-v2.0   & 0.304 ± 0.005 & 0.459 ± 0.008 & 0.263 ± 0.084\\
Llama3-Med42-8B             & 0.138 ± 0.062 & 0.241 ± 0.007 & 0.138 ± 0.062  \\
Meta-Llama-3.1-70B-Instruct  &  0.293 ± 0.005 & 0.326 ± 0.008 & 0.062 ± 0.043 \\
Meta-Llama-3.1-8B-Instruct   & 0.375 ± 0.005 & 0.229 ± 0.007 & 0.188 ± 0.063\\
Mistral-7B-Instruct-v0.3  & 0.476 ± 0.005 & 0.384 ± 0.008 & 0.050 ± 0.029\\
Mixtral-8x7B-Instruct-v0.1   & 0.543 ± 0.005 & 0.361 ± 0.008 & 0.075 ± 0.036\\
Phi-3-medium-4k-instruct  & 0.249 ± 0.005 & 0.281 ± 0.008 & 0.175 ± 0.064\\
Phi-3-mini-4k-instruct & 0.353 ± 0.005 & 0.328 ± 0.008 & 0.125 ± 0.057\\
%Qwen2-72B-Instruct           & -            & 0.260 ± 0.006 & -            & 0.332 ± 0.026 & -            & -            & -            \\
Qwen2-7B-Instruct   & 0.541 ± 0.005 & 0.267 ± 0.007 & 0.125 ± 0.052\\
Yi-1.5-34B-Chat & 0.508 ± 0.005 & 0.347 ± 0.009 & 0.287 ± 0.069\\
Yi-1.5-9B-Chat & 0.288 ± 0.005 & 0.417 ± 0.009 & 0.138 ± 0.067\\
%gemma-2-27b-it  & 0.001 ± 0.000 & 0.000 ± 0.000 & 0.003 ± 0.001\\
%gemma-2-9b-it & 0.002 ± 0.000 & 0.003 ± 0.001 & 0.006 ± 0.001\\

\bottomrule
\end{tabular}%
}
%}
\caption{Prometheus results for summarization and clinical-note taking tasks.% and medical factuality.
}
\end{table*}



\begin{table*}[h]
\centering
\renewcommand{\arraystretch}{1.5} 
\resizebox{2.1\columnwidth}{!}{%
\begin{tabular}{l|ccccccc|ccccccc}
\toprule
\textbf{Model} & \multicolumn{14}{c}{\textbf{Clinical Note-taking}} \\ 
& \multicolumn{7}{c|}{\textbf{ACI Bench}} & \multicolumn{7}{c}{\textbf{MTS Dialog}} \\
 & \textbf{BERTScore $\uparrow$} & \textbf{BLEU $\uparrow$} & \textbf{BLEURT $\uparrow$} & \textbf{MoverScore $\uparrow$} & \textbf{ROUGE1 $\uparrow$} & \textbf{ROUGE2 $\uparrow$} & \textbf{ROUGEL $\uparrow$} & \textbf{BERTScore $\uparrow$} & \textbf{BLEU $\uparrow$} & \textbf{BLEURT $\uparrow$} & \textbf{MoverScore $\uparrow$} & \textbf{ROUGE1 $\uparrow$} & \textbf{ROUGE2 $\uparrow$} & \textbf{ROUGEL $\uparrow$} \\
\midrule
BioMistral-MedMNX & 0.839 ± 0.007 & 0.012 ± 0.005 & -0.834 ± 0.057 & 0.537 ± 0.006 & 0.171 ± 0.016 & 0.039 ± 0.009 & 0.130 ± 0.014 & 0.800 ± 0.001 & 0.001 ± 0.000 & -1.304 ± 0.006 & 0.493 ± 0.001 & 0.040 ± 0.001 & 0.003 ± 0.000 & 0.036 ± 0.001 \\
JSL-MedLlama-3-8B-v2.0 & 0.853 ± 0.011 & 0.033 ± 0.016 & -0.810 ± 0.143 & 0.549 ± 0.013 & 0.212 ± 0.050 & 0.083 ± 0.026 & 0.173 ± 0.040 & 0.801 ± 0.001 & 0.002 ± 0.000 & -1.279 ± 0.007 & 0.492 ± 0.001 & 0.048 ± 0.001 & 0.006 ± 0.001 & 0.043 ± 0.001 \\
Llama3-Med42-8B & 0.863 ± nan & 0.059 ± 0.019 & -0.608 ± nan & 0.564 ± nan & 0.285 ± nan & 0.114 ± nan & 0.224 ± nan & 0.803 ± 0.001 & 0.003 ± 0.001 & -1.290 ± 0.011 & 0.495 ± 0.001 & 0.048 ± 0.002 & 0.007 ± 0.001 & 0.043 ± 0.002 \\
Meta-Llama-3.1-70B-Instruct & 0.852 ± nan & 0.019 ± nan & -0.613 ± nan & 0.548 ± nan & 0.201 ± nan & 0.056 ± nan & 0.154 ± nan & 0.798 ± 0.001 & 0.000 ± 0.000 & -1.350 ± 0.007 & 0.492 ± 0.001 & 0.041 ± 0.001 & 0.002 ± 0.000 & 0.038 ± 0.001 \\
Meta-Llama-3.1-8B-Instruct & 0.829 ± 0.011 & 0.017 ± 0.007 & -0.870 ± 0.068 & 0.538 ± 0.006 & 0.188 ± 0.024 & 0.047 ± 0.013 & 0.138 ± 0.019 & 0.797 ± 0.001 & 0.001 ± 0.000 & -1.364 ± 0.007 & 0.490 ± 0.001 & 0.044 ± 0.001 & 0.003 ± 0.000 & 0.040 ± 0.001 \\
Mistral-7B-Instruct-v0.3 & 0.812 ± nan & 0.000 ± nan & -1.138 ± nan & 0.522 ± nan & 0.046 ± nan & 0.004 ± nan & 0.037 ± nan & 0.800 ± 0.001 & 0.000 ± 0.000 & -1.322 ± 0.007 & 0.491 ± 0.001 & 0.042 ± 0.001 & 0.002 ± 0.000 & 0.039 ± 0.001 \\
Mixtral-8x7B-Instruct-v0.1 & 0.832  ± nan	 &  0.013  ± 0.006	&  -0.881  ± nan & 0.540  ± nan & 0.168  ± nan	& 0.038  ± nan	& 0.119  ± nan & 0.800  ± 0.001	& 0.001  ± 0.000	& -1.349  ± 0.007 & 0.492  ± 0.001
 & 0.042  ± 0.001 & 0.003  ± 0.000	& 0.039  ± 0.001 \\
Phi-3-medium-4k-instruct & 0.824 ± 0.007 & 0.014 ± 0.014 & -1.005 ± 0.067 & 0.528 ± 0.005 & 0.111 ± 0.023 & 0.023 ± 0.011 & 0.086 ± 0.017 & 0.800 ± 0.001 & 0.001 ± 0.000 & -1.346 ± 0.007 & 0.494 ± 0.001 & 0.040 ± 0.001 & 0.003 ± 0.000 & 0.037 ± 0.001 \\
Phi-3-mini-4k-instruct & 0.821 ± nan & 0.015 ± 0.007 & -1.026 ± nan & 0.529 ± nan & 0.135 ± nan & 0.035 ± nan & 0.111 ± nan & 0.800 ± 0.001 & 0.000 ± 0.000 & -1.312 ± 0.007 & 0.494 ± 0.001 & 0.039 ± 0.001 & 0.002 ± 0.000 & 0.036 ± 0.001 \\
%Qwen2-72B-Instruct &  &  &  &  &  &  &  &  &  &  &  &  &  &  \\
Qwen2-7B-Instruct & 0.841 ± nan & 0.015 ± 0.007 & -0.861 ± nan & 0.538 ± nan & 0.167 ± nan & 0.051 ± nan & 0.133 ± nan & 0.798 ± 0.001 & 0.000 ± 0.000 & -1.334 ± 0.006 & 0.489 ± 0.001 & 0.040 ± 0.001 & 0.002 ± 0.000 & 0.037 ± 0.001 \\
Yi-1.5-34B-Chat & 0.840 ± 0.009 & 0.015 ± 0.009 & -0.814 ± 0.085 & 0.533 ± 0.007 & 0.163 ± 0.024 & 0.046 ± 0.015 & 0.126 ± 0.019 & 0.806 ± 0.001 & 0.004 ± 0.001 & -1.266 ± 0.011 & 0.498 ± 0.001 & 0.063 ± 0.003 & 0.012 ± 0.001 & 0.056 ± 0.002 \\
Yi-1.5-9B-Chat & 0.836 ± nan & 0.030 ± 0.024 & -0.892 ± nan & 0.531 ± nan & 0.159 ± nan & 0.063 ± nan & 0.140 ± nan & 0.803 ± 0.001 & 0.003 ± 0.001 & -1.320 ± 0.009 & 0.494 ± 0.001 & 0.053 ± 0.002 & 0.007 ± 0.001 & 0.048 ± 0.002 \\
%gemma-2-27b-it & 0.815 ± nan	& 0.000 ± nan	& -1.016 ± nan	 & 0.521 ± nan & 0.077 ± nan	& 0.000 ± nan	 & 0.077 ± nan  & 0.790 ± 0.004	& 0.000 ± 0.000	& -1.226 ± 0.036	& 0.504 ± 0.005	& 0.026 ± 0.008	& 0.000 ± 0.000	& 0.026 ± 0.008 \\
%gemma-2-9b-it & 0.789 ± 0.005 & 0.000 ± 0.000 & -1.081 ± 0.041 & 0.506 ± 0.004 & 0.080 ± 0.017 & 0.000 ± 0.000 & 0.061 ± 0.011 & 0.795 ± 0.001 & 0.000 ± 0.000 & -1.231 ± 0.010 & 0.491 ± 0.001 & 0.041 ± 0.002 & 0.000 ± 0.000 & 0.040 ± 0.002 \\
\bottomrule
\end{tabular}%
}
\caption{Clinical note-taking results.}
\label{tab:evaluation_results}
\end{table*}

\begin{table*}[h]
\centering
\renewcommand{\arraystretch}{1.5} 
\resizebox{2.1\columnwidth}{!}{%
\begin{tabular}{l|ccccccc}
\toprule
\multirow{2}{*}{\textbf{Model}} & \multicolumn{7}{c}{\textbf{Making Treatment Recommendations}} \\  
                                  & \multicolumn{7}{c}{\textbf{Medtext}} \\ %\cline{2-8}
                                  & \textbf{BERTScore $\uparrow$} & \textbf{BLEU $\uparrow$} & \textbf{BLEURT $\uparrow$} & \textbf{MoverScore $\uparrow$} & \textbf{ROUGE1 $\uparrow$} & \textbf{ROUGE2 $\uparrow$} & \textbf{ROUGEL $\uparrow$} \\ 
\midrule
BioMistral-MedMNX & 0.855 ± 0.001 & 0.013 ± 0.001 & -0.650 ± 0.007 & 0.547 ± 0.001 & 0.177 ± 0.002 & 0.037 ± 0.001 & 0.136 ± 0.002 \\
JSL-MedLlama-3-8B-v2.0 & 0.856 ± 0.001 & 0.021 ± 0.002 & -0.652 ± 0.012 & 0.546 ± 0.001 & 0.185 ± 0.003 & 0.045 ± 0.002 & 0.146 ± 0.003 \\
Llama3-Med42-8B & 0.865 ± 0.001 & 0.018 ± 0.002 & -0.546 ± 0.015 & 0.557 ± 0.001 & 0.204 ± 0.005 & 0.052 ± 0.003 & 0.158 ± 0.004 \\
Meta-Llama-3.1-70B-Instruct & 0.859 ± 0.001 & 0.022 ± 0.002 & -0.644 ± 0.008 & 0.547 ± 0.001 & 0.196 ± 0.003 & 0.048 ± 0.002 & 0.150 ± 0.002 \\
Meta-Llama-3.1-8B-Instruct & 0.843 ± 0.001 & 0.010 ± 0.001 & -0.839 ± 0.007 & 0.535 ± 0.001 & 0.155 ± 0.002 & 0.032 ± 0.001 & 0.120 ± 0.002 \\
Mistral-7B-Instruct-v0.3 & 0.870 ± 0.002 & 0.038 ± 0.005 & -0.467 ± 0.022 & 0.562 ± 0.002 & 0.230 ± 0.008 & 0.072 ± 0.006 & 0.183 ± 0.007 \\
Mixtral-8x7B-Instruct-v0.1 & 0.868  ± 0.001	& 0.029  ± 0.002	& -0.502  ± 0.011	& 0.559  ± 0.001	& 0.220  ± 0.003	& 0.060  ± 0.002	& 0.172  ± 0.003 \\ 
Phi-3-medium-4k-instruct & 0.869 ± 0.001 & 0.033 ± 0.002 & -0.504 ± 0.011 & 0.560 ± 0.001 & 0.231 ± 0.004 & 0.069 ± 0.003 & 0.182 ± 0.003 \\
Phi-3-mini-4k-instruct & 0.863 ± 0.001 & 0.027 ± 0.002 & -0.551 ± 0.009 & 0.555 ± 0.001 & 0.213 ± 0.003 & 0.060 ± 0.002 & 0.165 ± 0.003 \\
%Qwen2-72B-Instruct & - & - & - & - & - & - & - \\
Qwen2-7B-Instruct & 0.859 ± 0.001 & 0.021 ± 0.002 & -0.634 ± 0.012 & 0.547 ± 0.001 & 0.193 ± 0.004 & 0.049 ± 0.002 & 0.147 ± 0.003 \\
Yi-1.5-34B-Chat & 0.867 ± 0.001	& 0.033 ± 0.002	& -0.580 ± 0.008	& 0.559 ± 0.001	& 0.245 ± 0.003	& 0.074 ± 0.002	& 0.189 ± 0.002 \\
Yi-1.5-9B-Chat & 0.863 ± 0.000 & 0.022 ± 0.001 & -0.513 ± 0.006 & 0.555 ± 0.001 & 0.213 ± 0.002 & 0.054 ± 0.002 & 0.163 ± 0.002 \\
%gemma-2-27b-it & nan ± nan & 0.000 ± 0.000 & -1.190 ± 0.024 & 0.495 ± 0.002 & 0.071 ± 0.007 & 0.003 ± 0.001 & 0.062 ± 0.005 \\
%gemma-2-9b-it & 0.819 ± 0.001 & 0.000 ± 0.000 & -1.146 ± 0.009 & 0.498 ± 0.001 & 0.077 ± 0.003 & 0.002 ± 0.000 & 0.067 ± 0.002 \\
\bottomrule
\end{tabular}}
\caption{Making diagnosis and treatment recommendations results.}
\label{tab:treatment_recommendations}
\end{table*}
%\begin{table*}[h]
\centering
\renewcommand{\arraystretch}{1.5} 
\resizebox{2.1\columnwidth}{!}{%
\begin{tabular}{l|ccccccc}
\toprule
\textbf{Model} & \multicolumn{6}{c}{\textbf{Medical Translation}} \\ 
& \multicolumn{6}{c}{\textbf{Med English to Spanish}}  \\
& \textbf{BLEU $\uparrow$} & \textbf{BLEURT $\uparrow$} & \textbf{MoverScore $\uparrow$} & \textbf{ROUGE1 $\uparrow$} & \textbf{ROUGE2 $\uparrow$} & \textbf{ROUGEL $\uparrow$} \\  
\midrule
BioMistral-MedMNX & 0.000 ± 0.000 & -1.233 ± 0.005 & nan ± nan & 0.037 ± 0.001 & 0.003 ± 0.000 & 0.032 ± 0.001 \\
JSL-MedLlama-3-8B-v2.0 & 0.000 ± 0.000 & -1.132 ± 0.005 & nan ± nan & 0.029 ± 0.001 & 0.002 ± 0.000 & 0.025 ± 0.001 \\
Llama3-Med42-8B & 0.000 ± 0.000 & -1.134 ± 0.006 & nan ± nan & 0.021 ± 0.001 & 0.001 ± 0.000 & 0.018 ± 0.001 \\
Meta-Llama-3.1-70B-Instruct & 0.000 ± 0.000 & -1.173 ± 0.005 & nan ± nan & 0.014 ± 0.001 & 0.001 ± 0.000 & 0.012 ± 0.001 \\
Meta-Llama-3.1-8B-Instruct & 0.000 ± 0.000 & -1.306 ± 0.005 & nan ± nan & 0.036 ± 0.001 & 0.003 ± 0.000 & 0.030 ± 0.001 \\
Mistral-7B-Instruct-v0.3 & 0.000 ± 0.000 & -0.824 ± 0.005 & nan ± nan & 0.104 ± 0.002 & 0.007 ± 0.000 & 0.087 ± 0.001 \\
Mixtral-8x7B-Instruct-v0.1 & 0.000  ± 0.000	& -0.888  ± 0.006	& nan  ± nan	& 0.097  ± 0.001	& 0.007  ± 0.000	&0.080  ± 0.001 \\
Phi-3-medium-4k-instruct & 0.000 ± 0.000 & -1.128 ± 0.005 & nan ± nan & 0.032 ± 0.001 & 0.002 ± 0.000 & 0.027 ± 0.001 \\
Phi-3-mini-4k-instruct & 0.000 ± 0.000 & -1.177 ± 0.005 & nan ± nan & 0.029 ± 0.001 & 0.002 ± 0.000 & 0.025 ± 0.001 \\
Qwen2-72B-Instruct & 0.000 ± 0.000 & -0.903 ± 0.006 & nan ± nan & 0.087 ± 0.002 & 0.007 ± 0.000 & 0.072 ± 0.001 \\
Qwen2-7B-Instruct & - & - & - & - & - & - \\
Yi-1.5-34B-Chat & 0.000 ± 0.000 & -1.099 ± 0.006 & nan ± nan & 0.047 ± 0.001 & 0.003 ± 0.000 & 0.040 ± 0.001 \\
Yi-1.5-9B-Chat & 0.000 ± 0.000 & -0.999 ± 0.005 & nan ± nan & 0.072 ± 0.001 & 0.006 ± 0.000 & 0.062 ± 0.001 \\
gemma-2-27b-it & 0.000 ± 0.000	& -1.169 ± 0.004	& nan ± nan & 0.005 ± 0.000	& 0.000 ± 0.000	& 0.005 ± 0.000 \\
gemma-2-9b-it & - & - & - & - & - & - \\
\bottomrule
\end{tabular}%
}
\caption{Medical Translation}
\label{tab:medical_translation}
\end{table*}

\begin{table*}[h]
\centering
\renewcommand{\arraystretch}{1.5} 
\resizebox{2.1\columnwidth}{!}{%
\begin{tabular}{l|ccccccc}
\toprule
\multirow{2}{*}{\textbf{Model}} & \multicolumn{7}{c}{\textbf{Medical factuality}} \\  
                                  & \multicolumn{7}{c}{\textbf{OLAPH}} \\ %\cline{2-8}
                                  & \textbf{BERTScore $\uparrow$} & \textbf{BLEU $\uparrow$} & \textbf{BLEURT $\uparrow$} & \textbf{MoverScore $\uparrow$} & \textbf{ROUGE1 $\uparrow$} & \textbf{ROUGE2 $\uparrow$} & \textbf{ROUGEL $\uparrow$} \\ 
\midrule
BioMistral-MedMNX & 0.864 ± 0.001 & 0.022 ± 0.002 & -0.557 ± 0.014 & 0.555 ± 0.001 & 0.211 ± 0.004 & 0.058 ± 0.002 & 0.166 ± 0.003 \\
JSL-MedLlama-3-8B-v2.0 & 0.868 ± 0.001 & 0.031 ± 0.003 & -0.544 ± 0.019 & 0.558 ± 0.002 & 0.230 ± 0.005 & 0.071 ± 0.004 & 0.183 ± 0.005 \\
Llama3-Med42-8B & 0.876 ± 0.001 & 0.024 ± 0.002 & -0.387 ± 0.015 & 0.567 ± 0.001 & 0.239 ± 0.005 & 0.069 ± 0.004 & 0.185 ± 0.005 \\
Meta-Llama-3.1-70B-Instruct & 0.866 ± 0.001 & 0.021 ± 0.002 & -0.538 ± 0.017 & 0.559 ± 0.001 & 0.225 ± 0.005 & 0.064 ± 0.004 & 0.178 ± 0.005 \\
Meta-Llama-3.1-8B-Instruct & 0.845 ± 0.001 & 0.009 ± 0.001 & -0.792 ± 0.015 & 0.538 ± 0.001 & 0.166 ± 0.004 & 0.038 ± 0.002 & 0.129 ± 0.003 \\
Mistral-7B-Instruct-v0.3 & 0.886 ± 0.001 & 0.056 ± 0.005 & -0.285 ± 0.022 & 0.581 ± 0.002 & 0.293 ± 0.008 & 0.110 ± 0.006 & 0.240 ± 0.007 \\
Mixtral-8x7B-Instruct-v0.1 & 0.810 ± 0.003 & 0.000 ± 0.000 & -1.148 ± 0.015 & 0.501 ± 0.001 & 0.081 ± 0.004 & 0.003 ± 0.001 & 0.067 ± 0.003 \\
Phi-3-medium-4k-instruct & 0.880 ± 0.002 & 0.047 ± 0.005 & -0.369 ± 0.022 & 0.574 ± 0.002 & 0.274 ± 0.007 & 0.096 ± 0.006 & 0.221 ± 0.007 \\
Phi-3-mini-4k-instruct & 0.867 ± 0.002 & 0.025 ± 0.003 & -0.494 ± 0.022 & 0.559 ± 0.002 & 0.220 ± 0.007 & 0.063 ± 0.004 & 0.177 ± 0.006 \\
Qwen2-7B-Instruct & 0.876 ± 0.001 & 0.033 ± 0.003 & -0.349 ± 0.014 & 0.570 ± 0.001 & 0.250 ± 0.005 & 0.076 ± 0.003 & 0.200 ± 0.004 \\
Yi-1.5-34B-Chat & 0.879 ± 0.001 & 0.041 ± 0.003 & -0.371 ± 0.016 & 0.570 ± 0.002 & 0.269 ± 0.006 & 0.092 ± 0.004 & 0.216 ± 0.005 \\
Yi-1.5-9B-Chat & 0.878 ± 0.001 & 0.037 ± 0.002 & -0.349 ± 0.012 & 0.569 ± 0.001 & 0.253 ± 0.004 & 0.083 ± 0.003 & 0.203 ± 0.004 \\
%gemma-2-27b-it & 0.825 ± 0.002 & 0.000 ± 0.000 & -1.185 ± 0.030 & 0.501 ± 0.002 & 0.091 ± 0.007 & 0.007 ± 0.002 & 0.076 ± 0.006 \\
%gemma-2-9b-it & 0.821 ± 0.002 & 0.000 ± 0.000 & -1.181 ± 0.014 & 0.502 ± 0.001 & 0.083 ± 0.005 & 0.005 ± 0.001 & 0.069 ± 0.004 \\
\bottomrule
\end{tabular}}
\caption{Medical factuality results.}
\label{tab:medical_factuality}
\end{table*}
%\begin{table*}[h]
\centering
\renewcommand{\arraystretch}{1.5} 
\resizebox{2.1\columnwidth}{!}{%
\begin{tabular}{l|ccccccc}
\toprule
\textbf{Model} & \multicolumn{7}{c}{\textbf{Open-ended medical questions}} \\ 
& \multicolumn{7}{c}{\textbf{\careqa{}-Open}} \\
& \textbf{BERTScore $\uparrow$} & \textbf{BLEU $\uparrow$} & \textbf{BLEURT $\uparrow$} & \textbf{MoverScore $\uparrow$} & \textbf{ROUGE1 $\uparrow$} & \textbf{ROUGE2 $\uparrow$} & \textbf{ROUGEL $\uparrow$} \\  
\midrule
BioMistral-MedMNX & 0.835 ± 0.001 & 0.003 ± 0.000 & -1.040 ± 0.007 & 0.517 ± 0.001 & 0.069 ± 0.001 & 0.017 ± 0.001 & 0.057 ± 0.001 \\
JSL-MedLlama-3-8B-v2.0 & 0.844 ± 0.001 & 0.007 ± 0.001 & -0.908 ± 0.010 & 0.527 ± 0.001 & 0.083 ± 0.002 & 0.025 ± 0.001 & 0.070 ± 0.002 \\
Llama3-Med42-8B & 0.839 ± 0.001 & 0.005 ± 0.001 & -0.989 ± 0.012 & 0.521 ± 0.001 & 0.090 ± 0.003 & 0.022 ± 0.002 & 0.074 ± 0.002 \\
Meta-Llama-3.1-70B-Instruct & 0.843 ± 0.001 & 0.007 ± 0.001 & -0.969 ± 0.012 & 0.526 ± 0.001 & 0.134 ± 0.003 & 0.030 ± 0.002 & 0.113 ± 0.003 \\
Meta-Llama-3.1-8B-Instruct & 0.830 ± 0.001 & 0.003 ± 0.000 & -1.128 ± 0.007 & 0.514 ± 0.001 & 0.065 ± 0.001 & 0.017 ± 0.001 & 0.052 ± 0.001 \\
Mistral-7B-Instruct-v0.3 & 0.849 ± 0.002 & 0.006 ± 0.002 & -0.892 ± 0.023 & 0.529 ± 0.002 & 0.145 ± 0.007 & 0.038 ± 0.004 & 0.129 ± 0.006 \\
Mixtral-8x7B-Instruct-v0.1 & 0.814 ± 0.001 & 0.000 ± 0.000 & -1.264 ± 0.007 & 0.492 ± 0.001 & 0.051 ± 0.002 & 0.002 ± 0.000 & 0.045 ± 0.002 \\
Phi-3-medium-4k-instruct & 0.847 ± 0.001 & 0.008 ± 0.001 & -0.883 ± 0.013 & 0.529 ± 0.001 & 0.145 ± 0.003 & 0.034 ± 0.002 & 0.122 ± 0.003 \\
Phi-3-mini-4k-instruct & 0.839 ± 0.001 & 0.004 ± 0.001 & -0.986 ± 0.009 & 0.520 ± 0.001 & 0.115 ± 0.003 & 0.022 ± 0.001 & 0.097 ± 0.002 \\
Qwen2-7B-Instruct & 0.844 ± 0.001 & 0.006 ± 0.001 & -0.896 ± 0.009 & 0.527 ± 0.001 & 0.133 ± 0.003 & 0.028 ± 0.001 & 0.111 ± 0.002 \\
Yi-1.5-34B-Chat & 0.845 ± 0.001 & 0.008 ± 0.001 & -0.887 ± 0.009 & 0.526 ± 0.001 & 0.141 ± 0.003 & 0.034 ± 0.001 & 0.119 ± 0.002 \\
Yi-1.5-9B-Chat & 0.843 ± 0.000 & 0.007 ± 0.000 & -0.906 ± 0.007 & 0.525 ± 0.001 & 0.136 ± 0.002 & 0.030 ± 0.001 & 0.113 ± 0.002 \\
%gemma-2-27b-it	& 0.820 ± 0.003	& 0.000 ± 0.000	& -1.179 ± 0.020	& 0.500 ± 0.002	& 0.058 ± 0.007	& 0.000 ± 0.000	& 0.051 ± 0.006 \\
%gemma-2-9b-it	& 0.819 ± 0.001 & 0.000 ± 0.000 &	-1.167 ± 0.007 & 0.496 ± 0.001	& 0.050 ± 0.002	& 0.001 ± 0.000	& 0.045 ± 0.002 \\
\bottomrule
\end{tabular}
} 
\caption{Results for \careqa{}-Open \textcolor{red}{quitar}}
\label{tab:question_entailment_results}
\end{table*}
\begin{table*}[h]
\centering
\renewcommand{\arraystretch}{1.5} 
\resizebox{2.1\columnwidth}{!}{%
\begin{tabular}{l|ccccccc}
\toprule
\textbf{Model} & \multicolumn{7}{c}{\textbf{Open-ended medical questions}} \\ 
& \multicolumn{7}{c}{\textbf{\careqa{}-Open}} \\
& \textbf{BERTScore $\uparrow$} & \textbf{BLEU $\uparrow$} & \textbf{BLEURT $\uparrow$} & \textbf{MoverScore $\uparrow$} & \textbf{ROUGE1 $\uparrow$} & \textbf{ROUGE2 $\uparrow$} & \textbf{ROUGEL $\uparrow$} \\  
\midrule
    BioMistral-MedMNX & 0.816 ± 0.002 & 0.002 ± 0.000 & -1.329 ± 0.009 & 0.492 ± 0.001 & 0.066 ± 0.002 & 0.017 ± 0.001 & 0.058 ± 0.002 \\ 
    JSL-MedLlama-3-8B-v2.0 & 0.827 ± 0.001 & 0.003 ± 0.000 & -1.234 ± 0.009 & 0.493 ± 0.001 & 0.069 ± 0.002 & 0.019 ± 0.001 & 0.060 ± 0.002 \\ 
    Llama3-Med42-8B & 0.293 ± 0.010 & 0.002 ± 0.001 & -1.441 ± 0.010 & 0.503 ± 0.001 & 0.030 ± 0.002 & 0.006 ± 0.001 & 0.027 ± 0.002 \\ 
    Meta-Llama-3.1-70B-Instruct & 0.660 ± 0.007 & 0.005 ± 0.001 & -1.283 ± 0.010 & 0.508 ± 0.001 & 0.096 ± 0.003 & 0.031 ± 0.002 & 0.087 ± 0.003 \\ 
    Meta-Llama-3.1-8B-Instruct & 0.761 ± 0.004 & 0.002 ± 0.000 & -1.496 ± 0.007 & 0.485 ± 0.001 & 0.049 ± 0.001 & 0.013 ± 0.001 & 0.042 ± 0.001 \\ 
    Mistral-7B-Instruct-v0.3 & 0.841 ± 0.002 & 0.004 ± 0.001 & -1.212 ± 0.026 & 0.501 ± 0.003 & 0.109 ± 0.008 & 0.037 ± 0.006 & 0.098 ± 0.008 \\ 
    Mixtral-8x7B-Instruct-v0.1 & 0.768 ± 0.010 & 0.008 ± 0.001 & -1.140 ± 0.022 & 0.515 ± 0.003 & 0.126 ± 0.007 & 0.040 ± 0.004 & 0.114 ± 0.007 \\ 
    Phi-3-medium-4k-instruct & 0.814 ± 0.003 & 0.005 ± 0.001 & -1.276 ± 0.010 & 0.499 ± 0.001 & 0.089 ± 0.003 & 0.028 ± 0.001 & 0.077 ± 0.002 \\ 
    Phi-3-mini-4k-instruct & 0.684 ± 0.008 & 0.003 ± 0.001 & -1.277 ± 0.010 & 0.500 ± 0.001 & 0.064 ± 0.002 & 0.016 ± 0.001 & 0.054 ± 0.002 \\ 
    Qwen2-7B-Instruct & 0.755 ± 0.005 & 0.003 ± 0.000 & -1.229 ± 0.008 & 0.496 ± 0.001 & 0.067 ± 0.002 & 0.018 ± 0.001 & 0.057 ± 0.001 \\ 
    Yi-1.5-34B-Chat & 0.809 ± 0.003 & 0.005 ± 0.001 & -1.186 ± 0.008 & 0.496 ± 0.001 & 0.078 ± 0.002 & 0.024 ± 0.001 & 0.067 ± 0.002 \\ 
    Yi-1.5-9B-Chat & 0.831 ± 0.001 & 0.004 ± 0.000 & -1.180 ± 0.008 & 0.491 ± 0.001 & 0.079 ± 0.002 & 0.023 ± 0.001 & 0.066 ± 0.002 \\ 
\bottomrule
\end{tabular}
} 
\caption{Results for \careqa{}-Open.}
\label{tab:question_entailment_results}
\end{table*}

\begin{table*}[h]
\centering
\renewcommand{\arraystretch}{1.5} 
\resizebox{2.1\columnwidth}{!}{%

\begin{tabular}{l|ccccccc|ccccccc}
\toprule
\textbf{Model} & \multicolumn{14}{c}{\textbf{Open-ended Medical Questions}} \\ 
& \multicolumn{7}{c|}{\textbf{MedDialog Raw}} & \multicolumn{7}{c}{\textbf{MEDIQA2019}} \\
& \textbf{BERTScore $\uparrow$} & \textbf{BLEU $\uparrow$} & \textbf{BLEURT $\uparrow$} & \textbf{MoverScore $\uparrow$} & \textbf{ROUGE1 $\uparrow$} & \textbf{ROUGE2 $\uparrow$} & \textbf{ROUGEL $\uparrow$} & \textbf{BERTScore $\uparrow$} & \textbf{BLEU $\uparrow$} & \textbf{BLEURT $\uparrow$} & \textbf{MoverScore $\uparrow$} & \textbf{ROUGE1 $\uparrow$} & \textbf{ROUGE2 $\uparrow$} & \textbf{ROUGEL $\uparrow$} \\ 
\midrule
BioMistral-MedMNX & 0.833 ± 0.001 & 0.001 ± 0.000 & -0.898 ± 0.012 & 0.526 ± 0.001 & 0.113 ± 0.003 & 0.010 ± 0.001 & 0.088 ± 0.002 & 0.850 ± 0.002 & 0.005 ± 0.002 & -0.660 ± 0.024 & 0.547 ± 0.002 & 0.169 ± 0.007 & 0.032 ± 0.003 & 0.132 ± 0.005 \\
JSL-MedLlama-3-8B-v2.0 & 0.832 ± 0.001 & 0.000 ± 0.000 & -0.875 ± 0.015 & 0.524 ± 0.001 & 0.109 ± 0.003 & 0.009 ± 0.001 & 0.087 ± 0.002 & 0.849 ± 0.002 & 0.008 ± 0.002 & -0.688 ± 0.027 & 0.543 ± 0.002 & 0.164 ± 0.006 & 0.030 ± 0.003 & 0.130 ± 0.005 \\
Llama3-Med42-8B & 0.834 ± 0.001 & 0.000 ± 0.000 & -0.887 ± 0.019 & 0.527 ± 0.001 & 0.108 ± 0.004 & 0.010 ± 0.001 & 0.085 ± 0.003 & 0.850 ± 0.003 & 0.008 ± 0.003 & -0.646 ± 0.043 & 0.546 ± 0.004 & 0.166 ± 0.012 & 0.026 ± 0.005 & 0.129 ± 0.010 \\
Meta-Llama-3.1-70B-Instruct & 0.835 ± 0.001 & 0.000 ± 0.000 & -0.875 ± 0.014 & 0.525 ± 0.001 & 0.115 ± 0.003 & 0.011 ± 0.001 & 0.089 ± 0.002 & 0.856 ± 0.002 & 0.010 ± 0.003 & -0.630 ± 0.030 & 0.547 ± 0.002 & 0.176 ± 0.008 & 0.037 ± 0.004 & 0.139 ± 0.007 \\
Meta-Llama-3.1-8B-Instruct & 0.824 ± 0.001 & 0.000 ± 0.000 & -1.013 ± 0.011 & 0.521 ± 0.001 & 0.096 ± 0.003 & 0.008 ± 0.001 & 0.074 ± 0.002 & 0.843 ± 0.002 & 0.005 ± 0.001 & -0.775 ± 0.024 & 0.538 ± 0.002 & 0.154 ± 0.007 & 0.028 ± 0.003 & 0.117 ± 0.005 \\
Mistral-7B-Instruct-v0.3 & 0.841 ± 0.001 & 0.000 ± 0.000 & -0.762 ± 0.024 & 0.530 ± 0.001 & 0.121 ± 0.005 & 0.014 ± 0.002 & 0.095 ± 0.004 & 0.852 ± 0.004 & 0.016 ± 0.008 & -0.661 ± 0.061 & 0.541 ± 0.005 & 0.158 ± 0.016 & 0.046 ± 0.011 & 0.132 ± 0.015 \\

Mixtral-8x7B-Instruct-v0.1 & 0.838  ± 0.001	& 0.001  ± 0.000	& -0.819  ± 0.020	& 0.529  ± 0.001	& 0.119  ± 0.004	& 0.012  ± 0.001	& 0.093  ± 0.003 & 0.846 ± 0.004	& 0.006 ± 0.003	& -0.837 ± 0.058 &	0.536 ± 0.004	& 0.135 ± 0.015	& 0.022 ± 0.009	& 0.110 ± 0.014 \\



Phi-3-medium-4k-instruct & 0.837 ± 0.001 & 0.001 ± 0.000 & -0.854 ± 0.016 & 0.528 ± 0.001 & 0.121 ± 0.004 & 0.013 ± 0.001 & 0.093 ± 0.003 & 0.859 ± 0.003 & 0.011 ± 0.004 & -0.552 ± 0.042 & 0.551 ± 0.004 & 0.197 ± 0.013 & 0.049 ± 0.009 & 0.157 ± 0.012 \\
Phi-3-mini-4k-instruct & 0.834 ± 0.001 & 0.000 ± 0.000 & -0.891 ± 0.013 & 0.526 ± 0.001 & 0.103 ± 0.003 & 0.009 ± 0.001 & 0.082 ± 0.002 & 0.850 ± 0.003 & 0.008 ± 0.004 & -0.682 ± 0.036 & 0.543 ± 0.003 & 0.163 ± 0.011 & 0.032 ± 0.007 & 0.129 ± 0.009 \\
%Qwen2-72B-Instruct & - & - & - & - & - & - & - & 0.855 ± 0.002 & 0.006 ± 0.002 & -0.626 ± 0.029 & 0.546 ± 0.002 & 0.184 ± 0.007 & 0.041 ± 0.004 & 0.143 ± 0.006 \\
Qwen2-7B-Instruct & 0.833 ± 0.001 & 0.000 ± 0.000 & -0.939 ± 0.015 & 0.526 ± 0.001 & 0.109 ± 0.004 & 0.010 ± 0.001 & 0.084 ± 0.003 & 0.851 ± 0.002 & 0.005 ± 0.002 & -0.673 ± 0.031 & 0.542 ± 0.002 & 0.155 ± 0.008 & 0.029 ± 0.005 & 0.120 ± 0.007 \\
Yi-1.5-34B-Chat & 0.839 ± 0.001 & 0.000 ± 0.000 & -0.785 ± 0.014 & 0.529 ± 0.001 & 0.131 ± 0.004 & 0.016 ± 0.001 & 0.101 ± 0.003 & 0.858 ± 0.002 & 0.008 ± 0.002 & -0.524 ± 0.031 & 0.551 ± 0.003 & 0.185 ± 0.009 & 0.039 ± 0.005 & 0.147 ± 0.008 \\
Yi-1.5-9B-Chat & 0.837 ± 0.001 & 0.001 ± 0.000 & -0.804 ± 0.012 & 0.528 ± 0.001 & 0.123 ± 0.003 & 0.014 ± 0.001 & 0.096 ± 0.002 & 0.857 ± 0.002 & 0.011 ± 0.003 & -0.540 ± 0.026 & 0.549 ± 0.002 & 0.197 ± 0.007 & 0.043 ± 0.004 & 0.159 ± 0.006 \\
%gemma-2-27b-it & 0.815 ± 0.003	& 0.000 ± 0.000	& -1.209 ± 0.045	& 0.502 ± 0.003	& 0.049 ± 0.009	& 0.000 ± 0.000	& 0.045 ± 0.008 & 0.829 ± nan & 0.000 ± 0.000 & -1.134 ± nan & 0.515 ± nan & 0.093 ± nan & 0.000 ± nan & 0.085 ± nan \\
%gemma-2-9b-it & 0.818 ± 0.002 & 0.000 ± 0.000 & -1.115 ± 0.018 & 0.510 ± 0.001 & 0.066 ± 0.004 & 0.003 ± 0.001 & 0.057 ± 0.004 & 0.822 ± 0.004 & 0.000 ± 0.000 & -1.108 ± 0.045 & 0.513 ± 0.004 & 0.084 ± 0.010 & 0.004 ± 0.002 & 0.070 ± 0.007 \\
\bottomrule
\end{tabular}
} 
\caption{Open-ended medical questions results.}
\label{tab:open_ended_medical_questions}
\end{table*}
\begin{table*}[h]
\centering
\renewcommand{\arraystretch}{1.5} 
\resizebox{2.1\columnwidth}{!}{%
\begin{tabular}{l|ccccccc}
\toprule
\textbf{Model} & \multicolumn{7}{c}{\textbf{Question Entailment}} \\ 
& \multicolumn{7}{c}{\textbf{MedDialog Qsumm}} \\
& \textbf{BERTScore $\uparrow$} & \textbf{BLEU $\uparrow$} & \textbf{BLEURT $\uparrow$} & \textbf{MoverScore $\uparrow$} & \textbf{ROUGE1 $\uparrow$} & \textbf{ROUGE2 $\uparrow$} & \textbf{ROUGEL $\uparrow$} \\  
\midrule
BioMistral-MedMNX & 0.839 ± 0.000 & 0.005 ± 0.000 & -1.056 ± 0.003 & 0.520 ± 0.000 & 0.093 ± 0.001 & 0.018 ± 0.001 & 0.081 ± 0.001 \\
JSL-MedLlama-3-8B-v2.0 & 0.840 ± 0.000 & 0.004 ± 0.000 & -0.967 ± 0.004 & 0.522 ± 0.000 & 0.085 ± 0.001 & 0.013 ± 0.001 & 0.074 ± 0.001 \\
Llama3-Med42-8B & 0.845 ± 0.000 & 0.004 ± 0.000 & -1.020 ± 0.005 & 0.521 ± 0.000 & 0.099 ± 0.002 & 0.019 ± 0.001 & 0.084 ± 0.001 \\
Meta-Llama-3.1-70B-Instruct & 0.849 ± 0.000 & 0.008 ± 0.001 & -1.013 ± 0.005 & 0.525 ± 0.000 & 0.120 ± 0.002 & 0.029 ± 0.001 & 0.102 ± 0.001 \\
Meta-Llama-3.1-8B-Instruct & 0.836 ± 0.000 & 0.005 ± 0.000 & -1.097 ± 0.004 & 0.518 ± 0.000 & 0.091 ± 0.001 & 0.017 ± 0.001 & 0.078 ± 0.001 \\
Mistral-7B-Instruct-v0.3 & 0.852 ± 0.001 & 0.010 ± 0.001 & -0.966 ± 0.007 & 0.526 ± 0.001 & 0.122 ± 0.003 & 0.031 ± 0.002 & 0.106 ± 0.002 \\
Mixtral-8x7B-Instruct-v0.1 & 0.848 ± 0.001	& 0.004 ± 0.000	& -0.984 ± 0.006	& 0.525 ± 0.000& 	0.099 ± 0.002	& 0.020 ± 0.001	& 0.086 ± 0.002 \\
Phi-3-medium-4k-instruct & 0.839 ± 0.000 & 0.004 ± 0.000 & -1.086 ± 0.004 & 0.522 ± 0.000 & 0.093 ± 0.001 & 0.017 ± 0.001 & 0.081 ± 0.001 \\
Phi-3-mini-4k-instruct & 0.840 ± 0.000 & 0.003 ± 0.000 & -1.041 ± 0.004 & 0.521 ± 0.000 & 0.083 ± 0.001 & 0.012 ± 0.001 & 0.072 ± 0.001 \\
%Qwen2-72B-Instruct & 0.846 ± 0.000 & 0.007 ± 0.001 & -1.011 ± 0.004 & 0.525 ± 0.000 & 0.108 ± 0.002 & 0.024 ± 0.001 & 0.093 ± 0.001 \\
Qwen2-7B-Instruct & 0.844 ± 0.000 & 0.006 ± 0.001 & -1.007 ± 0.004 & 0.524 ± 0.000 & 0.102 ± 0.002 & 0.020 ± 0.001 & 0.088 ± 0.001 \\
Yi-1.5-34B-Chat & 0.842 ± 0.001	&0.006 ± 0.001	&-1.010 ± 0.005	&0.522 ± 0.000	&0.100 ± 0.002	&0.021 ± 0.001	&0.087 ± 0.002 \\
Yi-1.5-9B-Chat & 0.852 ± 0.000 & 0.010 ± 0.001 & -0.979 ± 0.004 & 0.525 ± 0.000 & 0.128 ± 0.001 & 0.033 ± 0.001 & 0.109 ± 0.001 \\
%gemma-2-27b-it & 0.817 ± 0.002 & 0.000 ± 0.000 & -1.151 ± 0.016 & 0.513 ± 0.001 & 0.052 ± 0.006 & 0.002 ± 0.001 & 0.049 ± 0.006 \\
%gemma-2-9b-it & 0.821 ± 0.001 &	0.000 ± 0.000	& -1.135 ± 0.006  & 0.510 ± 0.001 & 0.057 ± 0.002	& 0.002 ± 0.000	& 0.053 ± 0.002 \\ %
\bottomrule
\end{tabular}
} 
\caption{Question entailment results.}
\label{tab:question_entailment_results}
\end{table*}
\begin{table*}[h]
\centering
\renewcommand{\arraystretch}{1.5} 
\resizebox{2.1\columnwidth}{!}{%
\begin{tabular}{l|cccccccc}
\toprule
\textbf{Model} & \multicolumn{8}{c}{\textbf{Summarization}} \\ 
& \multicolumn{8}{c}{\textbf{MIMIC-III}} \\
&\textbf{F1-RadGraph $\uparrow$} & \textbf{BERTScore $\uparrow$} & \textbf{BLEU $\uparrow$} & \textbf{BLEURT $\uparrow$} & \textbf{MoverScore $\uparrow$} & \textbf{ROUGE1 $\uparrow$} & \textbf{ROUGE2 $\uparrow$} & \textbf{ROUGEL $\uparrow$} \\ 
\midrule
BioMistral-MedMNX & 0.089 ± 0.001 & 0.837 ± 0.000 & 0.009 ± 0.000 & -0.796 ± 0.003 & 0.551 ± 0.000 & 0.130 ± 0.001 & 0.031 ± 0.001 & 0.110 ± 0.001 \\
JSL-MedLlama-3-8B-v2.0 & 0.079 ± 0.002 & 0.841 ± 0.000 & 0.014 ± 0.001 & -0.780 ± 0.005 & 0.556 ± 0.001 & 0.143 ± 0.002 & 0.041 ± 0.001 & 0.124 ± 0.002 \\
Llama3-Med42-8B & 0.093 ± 0.002 & 0.843 ± 0.000 & 0.013 ± 0.001 & -0.729 ± 0.005 & 0.557 ± 0.001 & 0.152 ± 0.002 & 0.041 ± 0.001 & 0.129 ± 0.002 \\
Meta-Llama-3.1-70B-Instruct & 0.059 ± 0.002 & 0.836 ± 0.000 & 0.009 ± 0.001 & -0.811 ± 0.005 & 0.547 ± 0.001 & 0.130 ± 0.002 & 0.031 ± 0.001 & 0.110 ± 0.002 \\
Meta-Llama-3.1-8B-Instruct & 0.065 ± 0.001 & 0.830 ± 0.000 & 0.007 ± 0.000 & -0.834 ± 0.004 & 0.542 ± 0.000 & 0.115 ± 0.001 & 0.025 ± 0.001 & 0.097 ± 0.001 \\
Mistral-7B-Instruct-v0.3 & 0.082 ± 0.002 & 0.845 ± 0.000 & 0.013 ± 0.001 & -0.753 ± 0.005 & 0.558 ± 0.000 & 0.157 ± 0.002 & 0.044 ± 0.001 & 0.134 ± 0.002 \\
Mixtral-8x7B-Instruct-v0.1 &
0.088 ± 0.002	& 0.844 ± 0.000	& 0.015 ± 0.001	& -0.762 ± 0.004	& 0.557 ± 0.000	& 0.157 ± 0.002	& 0.044 ± 0.001	& 0.134 ± 0.002 \\
Phi-3-medium-4k-instruct & 0.038 ± 0.002	& 0.838 ± 0.001	& 0.010 ± 0.001	& -0.771 ± 0.008 & 0.550 ± 0.001	& 0.137 ± 0.003	& 0.034 ± 0.001	& 0.116 ± 0.002 \\ 
Phi-3-mini-4k-instruct & 0.066 ± 0.002	&0.836 ± 0.000	& 0.008 ± 0.001	& -0.767 ± 0.005 & 0.548 ± 0.001	& 0.123 ± 0.002	& 0.029 ± 0.001	& 0.104 ± 0.002 \\ 
%Qwen2-72B-Instruct & - & - & - & - & - & - & - & - \\ 
Qwen2-7B-Instruct & 0.078 ± 0.001 & 0.843 ± 0.000 & 0.009 ± 0.000 & -0.761 ± 0.004 & 0.555 ± 0.000 & 0.142 ± 0.002 & 0.035 ± 0.001 & 0.120 ± 0.001 \\
Yi-1.5-34B-Chat & 0.065 ± 0.001	& 0.839 ± 0.000	& 0.009 ± 0.001	& -0.775 ± 0.004	& 0.550 ± 0.000 &	0.137 ± 0.002	& 0.033 ± 0.001	& 0.116 ± 0.001 \\ 
Yi-1.5-9B-Chat & 0.080 ± 0.002 & 0.840 ± 0.000 & 0.012 ± 0.001 & -0.806 ± 0.005 & 0.554 ± 0.001 & 0.136 ± 0.002 & 0.035 ± 0.001 & 0.117 ± 0.002 \\
%gemma-2-27b-it & 0.000 ± 0.000 & 0.802 ± 0.001 & 0.000 ± 0.000 & -1.139 ± 0.014 & 0.519 ± 0.002 & 0.038 ± 0.004 & 0.001 ± 0.000 & 0.033 ± 0.004 \\
%gemma-2-9b-it & 0.000 ± 0.000 & 0.806 ± 0.001 & 0.000 ± 0.000 & -1.051 ± 0.007 & 0.509 ± 0.001 & 0.046 ± 0.002 & 0.001 ± 0.000 & 0.042 ± 0.002 \\
\bottomrule
\end{tabular}
}
\caption{Summarization results.}
\label{tab:summarization_results}
\end{table*}


%\subsection{Close-ended}
\begin{table*}[h]
\centering
\renewcommand{\arraystretch}{1.5} 
\resizebox{2.1\columnwidth}{!}{%
\begin{tabular}{l|cccccccccc}
\toprule
\textbf{Model} & \multicolumn{10}{c}{\textbf{Close-ended}} \\ 
 & \textbf{MedMCQA $\uparrow$} & \textbf{MedQA $\uparrow$} & \textbf{\careqa{} (en) $\uparrow$} & \textbf{\careqa{} (es) $\uparrow$} & \textbf{multimedqa $\uparrow$} & \textbf{PubMedQA $\uparrow$} & \textbf{Med Text Classification $\uparrow$} & \textbf{Med Transcriptions $\uparrow$} & \textbf{BioRED $\uparrow$} & \textbf{MMLU $\uparrow$} \\ 
\midrule
BioMistral-MedMNX & 0.495 ± 0.008 & 0.515 ± 0.014 & 0.629 ± 0.006	& 0.546 ± 0.007 & 0.547 ± 0.006 & 0.776 ± 0.019 & 0.202 ± 0.011 & 0.356 ± 0.007 & 0.216 ± 0.013 & 0.6784 ± 0.034 \\
JSL-MedLlama-3-8B-v2.0 & 0.613 ± 0.008 & 0.617 ± 0.014 & 0.672 ± 0.006	& 0.572 ± 0.007  & 0.648 ± 0.006 & 0.742 ± 0.020 & 0.191 ± 0.010 & 0.361 ± 0.007 & 0.254 ± 0.014 & 0.7739 ± 0.0305 \\
Llama3-Med42-8B & 0.603 ± 0.008 & 0.626 ± 0.014 & 0.683 ± 0.006	& 0.575 ± 0.007  & 0.642 ± 0.006 & 0.772 ± 0.019 & 0.202 ± 0.011 & 0.377 ± 0.007 & 0.203 ± 0.013 & 0.7525 ± 0.0315 \\
Meta-Llama-3.1-70B-Instruct & 0.722 ± 0.007 & 0.798 ± 0.011 & 0.837 ± 0.005	& 0.825 ± 0.005 &  0.764 ± 0.005 & 0.800 ± 0.018 & 0.145 ± 0.003 & 0.381 ± 0.007 & 0.515 ± 0.016 & 0.8711 ± 0.0236 \\
Meta-Llama-3.1-8B-Instruct & 0.593 ± 0.008 & 0.637 ± 0.013 & 0.700 ± 0.006	& 0.592 ± 0.007  & 0.638 ± 0.006 & 0.752 ± 0.019 & 0.161 ± 0.003 & 0.334 ± 0.007  & 0.232 ± 0.013 & 0.7621 ± 0.031 \\
Mistral-7B-Instruct-v0.3 & 0.482 ± 0.008 & 0.523 ± 0.014 & 0.607 ± 0.007	& 0.529 ± 0.007   & 0.538 ± 0.006 & 0.774 ± 0.019 & 0.178 ± 0.010 & 0.356 ± 0.007 & 0.358 ± 0.015 & 0.661 ± 0.0345 \\
Mixtral-8x7B-Instruct-v0.1 & 0.564  ± 0.008	&  0.614  ± 0.014 & 0.725 ± 0.006	& 0.688 ± 0.006   & 0.622  ± 0.006 & 0.796  ± 0.018 & 0.207  ± 0.011 & 0.344 ± 0.007 & 0.352  ± 0.015 & 0.7766 ± 0.0304  \\
Phi-3-medium-4k-instruct & 0.623 ± 0.007 & 0.596 ± 0.014 & 0.769 ± 0.006	& 0.718 ± 0.006  & 0.661 ± 0.006 & 0.782 ± 0.018 & 0.048 ± 0.002 & 0.365 ± 0.007 & 0.261 ± 0.014 & 0.8237 ± 0.0275 \\
Phi-3-mini-4k-instruct & 0.572 ± 0.008 & 0.537 ± 0.014 &  0.701 ± 0.006	& 0.585 ± 0.007  & 0.604 ± 0.006 & 0.752 ± 0.019 & 0.192 ± 0.003 & 0.367 ± 0.007 & 0.262 ± 0.014 & 0.7398 ± 0.0321 \\
%Qwen2-72B-Instruct & 0.693 ± 0.007 & 0.768 ± 0.012 & 0.833 ± 0.005 & 0.741 ± 0.005 & 0.794 ± 0.018 & - & - & 0.428 ± 0.016  & - & - & 0.8656 ± 0.0247 \\
Qwen2-7B-Instruct & 0.551 ± 0.008 & 0.570 ± 0.014 & 0.680 ± 0.006	& 0.621 ± 0.006   & 0.596 ± 0.006 & 0.742 ± 0.020 & 0.225 ± 0.011 & 0.363 ± 0.007 & 0.197 ± 0.013 & 0.7337 ± 0.032 \\
Yi-1.5-34B-Chat & 0.575 ± 0.008 & 0.614 ± 0.014 & 0.733 ± 0.006	& 0.632 ± 0.006  & 0.628 ± 0.006 & 0.774 ± 0.019 & 0.301 ± 0.012 & 0.345 ± 0.007 & 0.543 ± 0.016 & 0.7806 ± 0.0298 \\
Yi-1.5-9B-Chat & 0.488 ± 0.008 & 0.515 ± 0.014 & 0.650 ± 0.006	& 0.507 ± 0.007  & 0.546 ± 0.006 & 0.774 ± 0.019 & 0.227 ± 0.011 & 0.330 ± 0.007 & 0.537 ± 0.016 & 0.7007 ± 0.0329 \\
%gemma-2-27b-it & 0.616 ± 0.008 & 0.669 ± 0.013 & 0.781 ± 0.006 & 0.670 ± 0.006 & 0.800 ± 0.018 & 0.222 ± 0.011 & 0.005 ± 0.001 &  0.536 ± 0.016 & 0.220 ± 0.004	& 0.331 ± 0.004 & 0.8163 ± 0.0278 \\
%gemma-2-9b-it & 0.566 ± 0.008 & 0.629 ± 0.014 & 0.745 ± 0.006 & 0.625 ± 0.006 & 0.766 ± 0.019 & 0.209 ± 0.011 & 0.221 ± 0.006 & 0.536 ± 0.016 & 0.230  ± 0.004 & 0.399  ± 0.005 & 0.7796 ± 0.0303 \\
\bottomrule
\end{tabular}
} 
\caption{Close-ended results.}
\label{tab:close_ended_benchmarks}
\end{table*}


\input{}



\end{document}
