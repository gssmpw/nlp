\section{Introduction}


\begin{figure*}[]
    \centering
    \begin{minipage}[c]{1\linewidth}
    \centering

    \setcounter{subfigure}{0}
    \subfloat[Statistics for pose classification in images generated by diffusion models. The plots show that the overall distribution is biased toward front-facing poses, even when directional text prompts are used to specify other poses.]{
        \centering
        \begin{minipage}[c]{0.32\linewidth}
            \centering
            \includegraphics[width=1\textwidth]{./resource/teaser/statistics_0.pdf}
        \end{minipage}\hspace{1.5mm}
        \begin{minipage}[c]{0.32\linewidth}
            \centering
            \includegraphics[width=1\textwidth]{./resource/teaser/statistics_1.pdf}
        \end{minipage}\hspace{1.5mm}
        \begin{minipage}[c]{0.32\linewidth}
            \centering
            \includegraphics[width=1\textwidth]{./resource/teaser/statistics_2.pdf}
        \end{minipage}
    }


    \subfloat{
        \begin{tikzpicture}
            \node[anchor=south west,inner sep=0] at (0,0) {\includegraphics[width=0.165\textwidth]{./resource/render/vsd/bear/rgb_0.png}};
            \node[anchor=south west,inner sep=0] at (1.6,0) {\includegraphics[width=0.05\textwidth]{./resource/render/vsd/bear/normal_0.png}};
        \end{tikzpicture}
        \hspace{-1.7mm}
        \begin{tikzpicture}
            \node[anchor=south west,inner sep=0] at (0,0) {\includegraphics[width=0.165\textwidth]{./resource/render/vsd/bear/rgb_1.png}};
            \node[anchor=south west,inner sep=0] at (1.6,0) {\includegraphics[width=0.05\textwidth]{./resource/render/vsd/bear/normal_1.png}};
        \end{tikzpicture}
        \hspace{-1.7mm}
        \begin{tikzpicture}
            \node[anchor=south west,inner sep=0] at (0,0) {\includegraphics[width=0.165\textwidth]{./resource/render/vsd/bear/rgb_2.png}};
            \node[anchor=south west,inner sep=0] at (1.6,0) {\includegraphics[width=0.05\textwidth]{./resource/render/vsd/bear/normal_2.png}};
        \end{tikzpicture}
        \hspace{-1.7mm}
        \begin{tikzpicture}
            \node[anchor=south west,inner sep=0] at (0,0) {\includegraphics[width=0.165\textwidth]{./resource/render/vsd/bear/rgb_3.png}};
            \node[anchor=south west,inner sep=0] at (1.6,0) {\includegraphics[width=0.05\textwidth]{./resource/render/vsd/bear/normal_3.png}};
        \end{tikzpicture}
        \hspace{-1.7mm}
        \begin{tikzpicture}
            \node[anchor=south west,inner sep=0] at (0,0) {\includegraphics[width=0.165\textwidth]{./resource/render/vsd/bear/rgb_4.png}};
            \node[anchor=south west,inner sep=0] at (1.6,0) {\includegraphics[width=0.05\textwidth]{./resource/render/vsd/bear/normal_4.png}};
        \end{tikzpicture}
        \hspace{-1.7mm}
        \begin{tikzpicture}
            \node[anchor=south west,inner sep=0] at (0,0) {\includegraphics[width=0.165\textwidth]{./resource/render/vsd/bear/rgb_5.png}};
            \node[anchor=south west,inner sep=0] at (1.6,0) {\includegraphics[width=0.05\textwidth]{./resource/render/vsd/bear/normal_5.png}};
        \end{tikzpicture}
    }\vspace{-4mm}

    \subfloat{
        \begin{tikzpicture}
            \node[anchor=south west,inner sep=0] at (0,0) {\includegraphics[width=0.165\textwidth]{./resource/render/vsd/camera/rgb_0.png}};
            \node[anchor=south west,inner sep=0] at (1.6,0) {\includegraphics[width=0.05\textwidth]{./resource/render/vsd/camera/normal_0.png}};
        \end{tikzpicture}
        \hspace{-1.7mm}
        \begin{tikzpicture}
            \node[anchor=south west,inner sep=0] at (0,0) {\includegraphics[width=0.165\textwidth]{./resource/render/vsd/camera/rgb_1.png}};
            \node[anchor=south west,inner sep=0] at (1.6,0) {\includegraphics[width=0.05\textwidth]{./resource/render/vsd/camera/normal_1.png}};
        \end{tikzpicture}
        \hspace{-1.7mm}
        \begin{tikzpicture}
            \node[anchor=south west,inner sep=0] at (0,0) {\includegraphics[width=0.165\textwidth]{./resource/render/vsd/camera/rgb_2.png}};
            \node[anchor=south west,inner sep=0] at (1.6,0) {\includegraphics[width=0.05\textwidth]{./resource/render/vsd/camera/normal_2.png}};
        \end{tikzpicture}
        \hspace{-1.7mm}
        \begin{tikzpicture}
            \node[anchor=south west,inner sep=0] at (0,0) {\includegraphics[width=0.165\textwidth]{./resource/render/vsd/camera/rgb_3.png}};
            \node[anchor=south west,inner sep=0] at (1.6,0) {\includegraphics[width=0.05\textwidth]{./resource/render/vsd/camera/normal_3.png}};
        \end{tikzpicture}
        \hspace{-1.7mm}
        \begin{tikzpicture}
            \node[anchor=south west,inner sep=0] at (0,0) {\includegraphics[width=0.165\textwidth]{./resource/render/vsd/camera/rgb_4.png}};
            \node[anchor=south west,inner sep=0] at (1.6,0) {\includegraphics[width=0.05\textwidth]{./resource/render/vsd/camera/normal_4.png}};
        \end{tikzpicture}
        \hspace{-1.7mm}
        \begin{tikzpicture}
            \node[anchor=south west,inner sep=0] at (0,0) {\includegraphics[width=0.165\textwidth]{./resource/render/vsd/camera/rgb_5.png}};
            \node[anchor=south west,inner sep=0] at (1.6,0) {\includegraphics[width=0.05\textwidth]{./resource/render/vsd/camera/normal_5.png}};
        \end{tikzpicture}
    }\vspace{-4mm}


    \setcounter{subfigure}{1}
    \subfloat[Score distillation from a biased distribution leads to the Multi-Face Janus problem. The generated 3D assets tend to overemphasize frontal features to align with the prior distribution, resulting in repeated patterns.]{
        \begin{tikzpicture}
            \node[anchor=south west,inner sep=0] at (0,0) {\includegraphics[width=0.165\textwidth]{./resource/render/vsd/kangaroo/rgb_0.png}};
            \node[anchor=south west,inner sep=0] at (1.6,0) {\includegraphics[width=0.05\textwidth]{./resource/render/vsd/kangaroo/normal_0.png}};
        \end{tikzpicture}
        \hspace{-1.7mm}
        \begin{tikzpicture}
            \node[anchor=south west,inner sep=0] at (0,0) {\includegraphics[width=0.165\textwidth]{./resource/render/vsd/kangaroo/rgb_1.png}};
            \node[anchor=south west,inner sep=0] at (1.6,0) {\includegraphics[width=0.05\textwidth]{./resource/render/vsd/kangaroo/normal_1.png}};
        \end{tikzpicture}
        \hspace{-1.7mm}
        \begin{tikzpicture}
            \node[anchor=south west,inner sep=0] at (0,0) {\includegraphics[width=0.165\textwidth]{./resource/render/vsd/kangaroo/rgb_2.png}};
            \node[anchor=south west,inner sep=0] at (1.6,0) {\includegraphics[width=0.05\textwidth]{./resource/render/vsd/kangaroo/normal_2.png}};
        \end{tikzpicture}
        \hspace{-1.7mm}
        \begin{tikzpicture}
            \node[anchor=south west,inner sep=0] at (0,0) {\includegraphics[width=0.165\textwidth]{./resource/render/vsd/kangaroo/rgb_3.png}};
            \node[anchor=south west,inner sep=0] at (1.6,0) {\includegraphics[width=0.05\textwidth]{./resource/render/vsd/kangaroo/normal_3.png}};
        \end{tikzpicture}
        \hspace{-1.7mm}
        \begin{tikzpicture}
            \node[anchor=south west,inner sep=0] at (0,0) {\includegraphics[width=0.165\textwidth]{./resource/render/vsd/kangaroo/rgb_4.png}};
            \node[anchor=south west,inner sep=0] at (1.6,0) {\includegraphics[width=0.05\textwidth]{./resource/render/vsd/kangaroo/normal_4.png}};
        \end{tikzpicture}
        \hspace{-1.7mm}
        \begin{tikzpicture}
            \node[anchor=south west,inner sep=0] at (0,0) {\includegraphics[width=0.165\textwidth]{./resource/render/vsd/kangaroo/rgb_5.png}};
            \node[anchor=south west,inner sep=0] at (1.6,0) {\includegraphics[width=0.05\textwidth]{./resource/render/vsd/kangaroo/normal_5.png}};
        \end{tikzpicture}
    }\vspace{-4mm}

    \subfloat{
        \begin{tikzpicture}
            \node[anchor=south west,inner sep=0] at (0,0) {\includegraphics[width=0.165\textwidth]{./resource/render/usd/bear/rgb_0.png}};
            \node[anchor=south west,inner sep=0] at (1.6,0) {\includegraphics[width=0.05\textwidth]{./resource/render/usd/bear/normal_0.png}};
        \end{tikzpicture}
        \hspace{-1.7mm}
        \begin{tikzpicture}
            \node[anchor=south west,inner sep=0] at (0,0) {\includegraphics[width=0.165\textwidth]{./resource/render/usd/bear/rgb_1.png}};
            \node[anchor=south west,inner sep=0] at (1.6,0) {\includegraphics[width=0.05\textwidth]{./resource/render/usd/bear/normal_1.png}};
        \end{tikzpicture}
        \hspace{-1.7mm}
        \begin{tikzpicture}
            \node[anchor=south west,inner sep=0] at (0,0) {\includegraphics[width=0.165\textwidth]{./resource/render/usd/bear/rgb_2.png}};
            \node[anchor=south west,inner sep=0] at (1.6,0) {\includegraphics[width=0.05\textwidth]{./resource/render/usd/bear/normal_2.png}};
        \end{tikzpicture}
        \hspace{-1.7mm}
        \begin{tikzpicture}
            \node[anchor=south west,inner sep=0] at (0,0) {\includegraphics[width=0.165\textwidth]{./resource/render/usd/bear/rgb_3.png}};
            \node[anchor=south west,inner sep=0] at (1.6,0) {\includegraphics[width=0.05\textwidth]{./resource/render/usd/bear/normal_3.png}};
        \end{tikzpicture}
        \hspace{-1.7mm}
        \begin{tikzpicture}
            \node[anchor=south west,inner sep=0] at (0,0) {\includegraphics[width=0.165\textwidth]{./resource/render/usd/bear/rgb_4.png}};
            \node[anchor=south west,inner sep=0] at (1.6,0) {\includegraphics[width=0.05\textwidth]{./resource/render/usd/bear/normal_4.png}};
        \end{tikzpicture}
        \hspace{-1.7mm}
        \begin{tikzpicture}
            \node[anchor=south west,inner sep=0] at (0,0) {\includegraphics[width=0.165\textwidth]{./resource/render/usd/bear/rgb_5.png}};
            \node[anchor=south west,inner sep=0] at (1.6,0) {\includegraphics[width=0.05\textwidth]{./resource/render/usd/bear/normal_5.png}};
        \end{tikzpicture}
    }\vspace{-4mm}

    \subfloat{
        \begin{tikzpicture}
            \node[anchor=south west,inner sep=0] at (0,0) {\includegraphics[width=0.165\textwidth]{./resource/render/usd/camera/rgb_0.png}};
            \node[anchor=south west,inner sep=0] at (1.6,0) {\includegraphics[width=0.05\textwidth]{./resource/render/usd/camera/normal_0.png}};
        \end{tikzpicture}
        \hspace{-1.7mm}
        \begin{tikzpicture}
            \node[anchor=south west,inner sep=0] at (0,0) {\includegraphics[width=0.165\textwidth]{./resource/render/usd/camera/rgb_1.png}};
            \node[anchor=south west,inner sep=0] at (1.6,0) {\includegraphics[width=0.05\textwidth]{./resource/render/usd/camera/normal_1.png}};
        \end{tikzpicture}
        \hspace{-1.7mm}
        \begin{tikzpicture}
            \node[anchor=south west,inner sep=0] at (0,0) {\includegraphics[width=0.165\textwidth]{./resource/render/usd/camera/rgb_2.png}};
            \node[anchor=south west,inner sep=0] at (1.6,0) {\includegraphics[width=0.05\textwidth]{./resource/render/usd/camera/normal_2.png}};
        \end{tikzpicture}
        \hspace{-1.7mm}
        \begin{tikzpicture}
            \node[anchor=south west,inner sep=0] at (0,0) {\includegraphics[width=0.165\textwidth]{./resource/render/usd/camera/rgb_3.png}};
            \node[anchor=south west,inner sep=0] at (1.6,0) {\includegraphics[width=0.05\textwidth]{./resource/render/usd/camera/normal_3.png}};
        \end{tikzpicture}
        \hspace{-1.7mm}
        \begin{tikzpicture}
            \node[anchor=south west,inner sep=0] at (0,0) {\includegraphics[width=0.165\textwidth]{./resource/render/usd/camera/rgb_4.png}};
            \node[anchor=south west,inner sep=0] at (1.6,0) {\includegraphics[width=0.05\textwidth]{./resource/render/usd/camera/normal_4.png}};
        \end{tikzpicture}
        \hspace{-1.7mm}
        \begin{tikzpicture}
            \node[anchor=south west,inner sep=0] at (0,0) {\includegraphics[width=0.165\textwidth]{./resource/render/usd/camera/rgb_5.png}};
            \node[anchor=south west,inner sep=0] at (1.6,0) {\includegraphics[width=0.05\textwidth]{./resource/render/usd/camera/normal_5.png}};
        \end{tikzpicture}
    }\vspace{-4mm}


    \setcounter{subfigure}{2}
    \subfloat[Score distillation using a uniform distribution with our RecDreamer. The process relies on a rectified prior distribution that incorporates guidance from various poses, effectively alleviating geometric inconsistencies.]{
        \begin{tikzpicture}
            \node[anchor=south west,inner sep=0] at (0,0) {\includegraphics[width=0.165\textwidth]{./resource/render/usd/kangaroo/rgb_0.png}};
            \node[anchor=south west,inner sep=0] at (1.6,0) {\includegraphics[width=0.05\textwidth]{./resource/render/usd/kangaroo/normal_0.png}};
        \end{tikzpicture}
        \hspace{-1.7mm}
        \begin{tikzpicture}
            \node[anchor=south west,inner sep=0] at (0,0) {\includegraphics[width=0.165\textwidth]{./resource/render/usd/kangaroo/rgb_1.png}};
            \node[anchor=south west,inner sep=0] at (1.6,0) {\includegraphics[width=0.05\textwidth]{./resource/render/usd/kangaroo/normal_1.png}};
        \end{tikzpicture}
        \hspace{-1.7mm}
        \begin{tikzpicture}
            \node[anchor=south west,inner sep=0] at (0,0) {\includegraphics[width=0.165\textwidth]{./resource/render/usd/kangaroo/rgb_2.png}};
            \node[anchor=south west,inner sep=0] at (1.6,0) {\includegraphics[width=0.05\textwidth]{./resource/render/usd/kangaroo/normal_2.png}};
        \end{tikzpicture}
        \hspace{-1.7mm}
        \begin{tikzpicture}
            \node[anchor=south west,inner sep=0] at (0,0) {\includegraphics[width=0.165\textwidth]{./resource/render/usd/kangaroo/rgb_3.png}};
            \node[anchor=south west,inner sep=0] at (1.6,0) {\includegraphics[width=0.05\textwidth]{./resource/render/usd/kangaroo/normal_3.png}};
        \end{tikzpicture}
        \hspace{-1.7mm}
        \begin{tikzpicture}
            \node[anchor=south west,inner sep=0] at (0,0) {\includegraphics[width=0.165\textwidth]{./resource/render/usd/kangaroo/rgb_4.png}};
            \node[anchor=south west,inner sep=0] at (1.6,0) {\includegraphics[width=0.05\textwidth]{./resource/render/usd/kangaroo/normal_4.png}};
        \end{tikzpicture}
        \hspace{-1.7mm}
        \begin{tikzpicture}
            \node[anchor=south west,inner sep=0] at (0,0) {\includegraphics[width=0.165\textwidth]{./resource/render/usd/kangaroo/rgb_5.png}};
            \node[anchor=south west,inner sep=0] at (1.6,0) {\includegraphics[width=0.05\textwidth]{./resource/render/usd/kangaroo/normal_5.png}};
        \end{tikzpicture}
    }
\end{minipage}



    \caption{The Multi-Face Janus problem arises from an imbalance in the pose distribution of pretrained models, which tend to generate predominantly frontal images. This bias results in excessive faces appearing in the generated 3D assets. RecDreamer addresses this issue by producing a distribution with a uniform pose marginal, enabling more diverse pose generation and mitigating the Multi-Face Janus problem.}
    \label{fig:teaser}

\end{figure*} 

Text-to-3D generation has become a transformative technology with broad applications, enabling the creation of 3D models from natural language descriptions. By lowering the technical barriers, it allows non-experts to generate intricate 3D objects without specialized tools or expertise. This advancement significantly enhances productivity in fields such as gaming, virtual reality (VR), and augmented reality (AR), where manual 3D model creation is often labor-intensive. Current methods~\citep{wang2024prolificdreamer, chen2023fantasia3d, lin2023magic3d} rely on score distillation techniques~\citep{poole2022dreamfusion, wang2023score, graikos2022diffusion} to leverage text-to-image priors from diffusion models, generating high-quality 3D assets with remarkable visual fidelity, precise alignment to text descriptions, and strong conceptual integrity.

However, despite these advances, generated 3D assets frequently suffer from geometric inconsistencies, particularly in the form of repeated patterns or textures across different camera poses, a problem known as the \emph{Multi-Face Janus} issue (see Fig.~\ref{fig:teaser}(b)). This arises from biases in the underlying data distribution (see Fig.~\ref{fig:teaser}(a)), which current methods fail to fully address. Efforts to tackle this issue, such as modifying directional text descriptions through gradient-based adjustments~\citep{hong2023debiasing, armandpour2023re}, have yielded limited success, often introducing unwanted artifacts or irrelevant patterns. Other approaches~\citep{huang2024dreamcontrol, wang2024taming} attempt to impose constraints on the rendered 3D assets, but they still fall short of resolving the core bias present in text-to-image distributions.

To address this, we propose \emph{RecDreamer}, a novel solution designed to eliminate the biases in pretrained models by modifying the underlying data distribution. The rationale behind our approach is to reconstruct the original data distribution so that the marginal distribution of the pose becomes uniform, thus removing the bias toward a canonical pose. We achieve this by introducing a weighting function that reweights the density of the original distribution, ensuring it meets specific marginal constraints. Specifically, we derive a rectified distribution where the pose component in the joint distribution follows a uniform distribution across all possible poses.

This rectified distribution is then incorporated into the score distillation framework~\citep{wang2024prolificdreamer}. The use of reverse Kullback-Leibler divergence~\citep{kullback1951information} in score distillation allows the integration of the modified distribution without altering the overall sampling process or gradient derivation. As a result, we develop a process known as uniform score distillation (USD), which aligns the target distribution with a uniform distribution, effectively improving pose consistency in the generated 3D assets.

To compute the auxiliary function necessary for rectifying the distribution, RecDreamer introduces a training-free classifier that estimates pose categories by discretizing the continuous pose space. This classifier predicts pose based on orientation score and texture similarity, leveraging a pretrained feature extractor without the need for additional fine-tuning. Furthermore, we dynamically handle noisy image estimates, ensuring robust pose estimation and reliable performance even in the optimization process.

Experiments demonstrate the effectiveness of our method in alleviating the Multi-Face Janus problem and improving geometric consistency, while maintaining rendering quality comparable to baseline methods, as shown in Fig.~\ref{fig:teaser}(c). We also conduct additional experiments on 2D images and a toy dataset to further validate our algorithm. Additionally, we showcase further applications of the pose classifier.