
\begin{figure*}[t]
\newcommand\gap{0.25}
\newcommand\gapd{0.48}
  \centering
  \begin{subfigure}[b]{\gapd\linewidth}
      \centering
      \includegraphics[width=0.49\textwidth]{image/workplace_1.png}
      \hfill
      \includegraphics[width=0.49\textwidth]{image/workplace_2.png}
      \caption{Multiple Near-Optimal Strategies}
      \label{fig: workplace 1}
  \end{subfigure}
  \hfill  
  \begin{subfigure}[b]{\gap\linewidth}
      \centering
      \includegraphics[width=0.95\textwidth]{image/workplace_sub.png}
      \caption{Suboptimal Teamwork}
      \label{fig: workplace suboptimal}
  \end{subfigure}
  \hfill  
  \begin{subfigure}[b]{\gap\linewidth}
      \centering
      \includegraphics[width=0.95\textwidth]{image/workplace_po.png}
      \caption{Partial Observability}
      \label{fig: workplace partially observe}
  \end{subfigure}

  \captionsetup{subrefformat=parens}
  \caption{Motivating Example: Consider a team whose members must coordinate on the fly to complete subtasks at two conveyor belts. Each member has limited observability, perceiving only their immediate surroundings. For example, the unshaded area for the blue person in \subref{fig: workplace partially observe}. As shown in \subref{fig: workplace 1}, this task allows multiple near-optimal strategies, enabling teams to execute it in different ways based on their shared preferences. However, practical constraints -- such as partial observability -- can lead to suboptimal coordination and team performance. For instance, if multiple members gather at the same subtask location, it results in inefficient task allocation, where one subtask remains unattended while two members redundantly perform the same task \subref{fig: workplace suboptimal}. Like many real-world scenarios, this task engenders heterogeneous and potentially suboptimal demonstrations of teamwork. This paper focuses on learning models of team behavior in this challenging setting from demonstrations.}
  \label{fig: workplace}
\end{figure*}