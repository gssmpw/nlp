
%%% Use this environment to specify a short abstract for your paper.

\begin{abstract}
Successful collaboration requires team members to stay aligned, especially in complex sequential tasks. Team members must dynamically coordinate which subtasks to perform and in what order. However, real-world constraints like partial observability and limited communication bandwidth often lead to suboptimal collaboration. Even among expert teams, the same task can be executed in multiple ways. To develop multi-agent systems and human-AI teams for such tasks, we are interested in data-driven learning of multimodal team behaviors. Multi-Agent Imitation Learning (MAIL) provides a promising framework for data-driven learning of team behavior from demonstrations, but existing methods struggle with heterogeneous demonstrations, as they assume that all demonstrations originate from a single team policy.
Hence, in this work, we introduce \ouralg: a hierarchical MAIL algorithm designed to learn multimodal team behaviors in complex sequential tasks. \ouralg represents each team member with a hierarchical policy and learns these policies from heterogeneous team demonstrations in a factored manner. By employing a distribution-matching approach, \ouralg mitigates compounding errors and scales effectively to long horizons and continuous state representations. Experimental results show that \ouralg outperforms MAIL baselines and accurately models team behavior across a variety of collaborative scenarios.
\end{abstract}

%%% The code below was generated by the tool at http://dl.acm.org/ccs.cfm.
%%% Please replace this example with code appropriate for your own paper.
\begin{CCSXML}
<ccs2012>
   <concept>
       <concept_id>10010147.10010257.10010293.10010319</concept_id>
       <concept_desc>Computing methodologies~Learning latent representations</concept_desc>
       <concept_significance>300</concept_significance>
       </concept>
   <concept>
       <concept_id>10010147.10010257.10010282.10010290</concept_id>
       <concept_desc>Computing methodologies~Learning from demonstrations</concept_desc>
       <concept_significance>500</concept_significance>
       </concept>
   <concept>
       <concept_id>10010147.10010178.10010219.10010220</concept_id>
       <concept_desc>Computing methodologies~Multi-agent systems</concept_desc>
       <concept_significance>300</concept_significance>
       </concept>
   <concept>
       <concept_id>10010147.10010257.10010258.10010261.10010274</concept_id>
       <concept_desc>Computing methodologies~Apprenticeship learning</concept_desc>
       <concept_significance>500</concept_significance>
       </concept>
   <concept>
       <concept_id>10010147.10010257.10010282.10011305</concept_id>
       <concept_desc>Computing methodologies~Semi-supervised learning settings</concept_desc>
       <concept_significance>300</concept_significance>
       </concept>
 </ccs2012>
\end{CCSXML}

\ccsdesc[300]{Computing methodologies~Learning latent representations}
\ccsdesc[500]{Computing methodologies~Learning from demonstrations}
\ccsdesc[300]{Computing methodologies~Multi-agent systems}
\ccsdesc[500]{Computing methodologies~Apprenticeship learning}
\ccsdesc[300]{Computing methodologies~Semi-supervised learning settings}

%%% Use this command to specify a few keywords describing your work.
%%% Keywords should be separated by commas.
\keywords{Multi-Agent Imitation Learning, Teamwork, Behavior Modeling} 