\section{Literature Review and The Contribution}
%%%%%%%%%%%%%%%%%%%%%%%%%%%%%%%%%%%%%%%%%%%%%%%%%%%%%%%%%%%%%%%%%%%%%%%%%%%%%%%%%%%%%%%%%%%%%%%%%%%%%%%%%%%%%%%%%%%%%

AI systems have become increasingly popular and widely used across many fields. With advancements in AI technology, demonstrating the reliability of these systems is essential for their confident use. \shortciteN{werner2022leveraging} proposed a framework for developing, qualifying, and releasing reliable and assured AI systems by applying design for reliability tools and techniques during the design and development phases. \shortciteN{blood2023reliability} highlighted that traditional reliability tools need to be transformed to address the reliability of AI systems. \shortciteN{hong2023statistical} provided a compressive discussion on statistical reliability for AI systems. Existing research has made significant contributions in the field of AI systems, highlighting the importance of exploring the data used in AI reliability research.


To further emphasize the importance of data exploration for AI reliability research, several data collection methods from the field of traditional reliability analysis have already been investigated. Some of these methods could be further extended to AI system reliability studies. \shortciteN{smith2021reliability} illustrated a method for failure data collection, as well as a structured approach to recording the data using a formal document from the field, which can be used for reliability analysis. \shortciteN{meeker2022statistical} introduced data collection strategies that can be applied to planning reliability studies, as well as to data analysis and modeling in reliability research. \shortciteN{inel2023collect} developed a responsible AI methodology designed to guide data collection, which can be used to assess the robustness of data used for AI applications in the real world. However, a gap remains in the detailed introduction of data collection methods specifically for AI system reliability research.


Although AI system reliability research has emerged as a growing field in recent years, the availability of data for this research remains limited. For AI system reliability analysis, \shortciteN{hong2023statistical} used the public \citeANP{AIIncidentDB} database~(2021), which primarily collects AI incidents from news reports, and applied a text mining method to identify variables that contribute the most to AI incident data to illustrate the importance of AI reliability. \shortciteN{MinHongKingMeeker2020} and \shortciteN{Zheng2023-testplan} both used publicly accessible data from the California Department of Motor Vehicles (DMV). Specifically, \shortciteN{MinHongKingMeeker2020} focused on parametric and non-parametric models to describe disengagement events from autonomous vehicles, while \shortciteN{Zheng2023-testplan} focused on test planning for reliability assurance tests. \shortciteN{Panetal2024} used data from a physics-based AV simulation platform to demonstrate the reliability prediction performance and interpretability of an error propagation model. In terms of assessing the robustness of advanced ML algorithms, \shortciteN{Lianetal2021Robustness} collected prediction performance results by conducting a comprehensive set of mixture experiments to assess the robustness classification algorithms. \shortciteN{Faddietal2024} conducted experiments to collect datasets that capture the behavior of machine learning image classifiers on both clean and perturbed inputs to evaluate the reliability of AI algorithms. Some available data can be used for AI reliability studies; however, it remains limited and requires further exploration.


In summary, the currently available datasets for AI reliability research remain limited. Therefore, we aim to address this gap by creating a publicly accessible repository focused on AI reliability. The established public online repository provides several contributions. First, it is a valuable resource for AI reliability researchers by providing public access to reliability data, which can serve as a starting point for AI reliability research. Second, it facilitates communication between ML and reliability researchers, as well as researchers from other fields, by enabling collaboration across various academic domains. Third, highlighting the role of AI reliability in ensuring safety can help attract a broader range of researchers from the academic community, fostering further research and the development of new methods in this emerging field.



%%%%%%%%%%%%%%%%%%%%%%%%%%%%%%%%%%%%%%%%%%%%%%%%%%%%%%%%%%%%%%%%%%%%%%%%%%%%%%%%%%%%%%%%%%%%%%%%%%%%%%%%%%%%%%%%%%%%%