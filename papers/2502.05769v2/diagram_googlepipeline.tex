\begin{figure}[htpb]
\centering
\resizebox{0.45\textwidth}{!}{
\begin{tikzpicture}[
    node distance=0.75cm and 2.cm,
    every node/.style={align=center, font=\small},
    arrow/.style={-{Latex}, thick},
    process/.style={align=center, draw, font=\small},
    process2/.style={align=center, draw, font=\small, fill=blue!30},
    outprocess/.style={align=center, rounded corners, draw, font=\small, fill=red!30}
]
% Nodes
\node (input) {One of \{\textit{address, proper name, postal code, geographic coordinates}\}};
\node[below=of googleES,process] (geocoding) {Geocoding/Reverse Geocoding \\API call};
\node [below= 0.5cm of geocoding](gbmout) {\textit{Address, coordinates},\\ \textit{and geometry data}};
\node[below right=0.75cm and -0.7cm of gbmout,draw, fill=white, double copy shadow] (elevation) {Elevation and \\other API call};
\node[below left=0.75cm and -0.4cm of gbmout,process] (maps) {Static Maps \\API call};
\node [below=0.4cm of elevation](elevationout) {Ground elevation data,\\ additional data of interest};
\node [below right=0.75cm and -0.75cm of maps](mapsout1) {Street map(s)};
\node [below left=0.75cm and -1.0cm of maps](mapsout2) {Satellite image(s)};

% Arrows
\draw[arrow] (input) -- (geocoding);
\draw[arrow] (geocoding) -- (gbmout);
\draw[arrow] (gbmout) -- (elevation);
\draw[arrow] (gbmout) -- (maps);
\draw[arrow] (maps) -- (mapsout1);
\draw[arrow] (maps) -- (mapsout2);
\draw[arrow] (elevation) -- (elevationout);

\end{tikzpicture}
}
\caption{Diagram of our Google Map Services Integration Tool.}\label{fig:googlepipeline}
\end{figure}