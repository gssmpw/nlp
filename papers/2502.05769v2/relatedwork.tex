\section{Background and Related Works}
\subsection{ChatGPT/Deepseek and API}
Large Language Models (LLMs) are neural networks, typically Transformer-based \cite{transformer}, pre-trained on extensive, diverse text/image corpora, typically sourced from web crawls. These models, designed for Natural Language Processing (NLP), typically interpret text-based prompts and generate text-based outputs. Certain models, such as "DeepseekV3/R1" and their variants \cite{deepseekv3, deepseekr1}, support object character recognition (OCR, i.e., reading text from images). Models like "ChatGPT-4o" \cite{gpt4} and its variants additionally support full interpretation and analysis of image content.

LLMs have achieved widespread adoption since 2023. Beyond basic image and text interpretation, these models recently exhibited expert-level problem-solving in various scientific and engineering domains \cite{gpqa, math500}.

Due to their large size, LLMs often face hardware constraints for local deployment. While popular LLM providers such as OpenAI and Deepseek, provide web browser interfaces for their models, they also offer Application Programming Interfaces (APIs). These APIs enable client-side software or code to query LLMs hosted on OpenAI or Deepseek servers, facilitating large-scale data processing without requiring human-in-the-loop manipulations via browser interfaces. Unlike traditional local deep learning, which necessitates GPUs for both training and inference, API-based LLM querying requires minimal local hardware and can be deloyed on devices such as mobile phones. 
\begin{table*}[ht]
\centering
\captionof{table}{TABLE OF IMPORTANT DEEPSEEK AND OPENAI LLMs}
\begin{tabular}{l|l|l|l|c|c|c}
\hline
Model Name            & Model Class          & Model Type    & Image Processing & Parameters    & \begin{tabular}[c]{@{}l@{}}API call price/1M\\ Input Tokens (USD)\end{tabular} & \begin{tabular}[c]{@{}l@{}}API call price/1M\\ Output Tokens (USD)\end{tabular} \\ \hline
chatgpt4o-latest      & GPT4o                & Autoregressive & Analysis         & $\sim$ 1000+B       & 2.5                                                                & 10                                                                \\ 
gpt-4o-mini           & GPT4o Mini          & Autoregressive & Analysis         & $\sim$10's of B        & 0.15                                                               & 0.6                                                               \\ 
deepseek-chat         & Deepseek V3          & Autoregressive & OCR               & 617B          & *0.14 $\times$ 0.1*                                                    & 1.10                                                              \\ 
deepseek-reasoner     & Deepseek R1 (V3-base) & Reasoning & OCR               & 617B          & 0.14                                                               & 2.19                                                              \\ 
\textit{gpt-o1$^{1}$}                & GPT-o1 (GPT4-base) & Reasoning     & None              & $\sim$175B        & 15                                                                 & 60                                                                \\ \hline
\end{tabular}\par
\smallskip
\justifying
\noindent
Table compiled on 2025-01-31.  OpenAI models are not open-sourced, their model sizes (parameters) are estimated (B = billions, M = millions). $^1$We did not include gpt-o1 in our experiments due to cost, but we include its specifications for comparison. *The Deepseek V3 API call input token price is discounted by 90\% if input caching is used for repeated identical prompting.* \label{Tab:models}
\end{table*}
\subsection{Google Maps Platform API}
Google Map Platform is a cloud-based mapping service and a part of Google Cloud. Its API allows the client device to connect to various cloud-based GIS, mapping, and remote sensing services hosted on the Google Cloud servers. 

The services utilized in this research include remote sensing image retrieval, map retrieval, elevation data retrieval, geocoding/reverse geocoding, and building polygon retrieval. However, Google Maps Platform also offers other APIs for urban and environmental research, including real-time traffic data, solar potential data, air quality data, and plant pollen data, in addition to the full suite of commonly used Google Maps navigation and mapping tools. 

Although less known in the remote sensing and GIS community than its sister application Google Earth Engine, Google Map Platform has been used in a variety of GIS research including navigation, object tracking, city modeling, image and map retrieval, geospatial data analysis for commercial and industrial applications \cite{maps1,maps2,maps3,maps4}. It is also used as part of many commercial software for cloud-based mapping integration. 

\subsection{Google Earth Studio}
Google Earth Studio \cite{google_earth_studio} is a web-based animation tool that leverages Google Earth's satellite imagery and 3D terrain data. The tool is especially useful for creating geospatial visualizations, as it is integrated with Google Earth’s geographic data. It allows for the retrieval of images from user-specified camera poses at user-specified locations. In this research, we use Google Earth Studio to retrieve 360 $\degree$ multi-view remote sensing images of a building from its address, postal code, place name, or geographic coordinates following \cite{gao_3dgs,gbm}.



%\vspace{0.5cm}
%\begin{tikzpicture}[node distance=1.5cm]

% Nodes
\node (start) [startstop] {Google Earth Studio};
\node (process1) [process, below of=start] {SAM2 Mask Extraction};
\node (process2) [process, below of=process1] {Mask Refinement};
\node (process3) [process, below of=process2] {2D Gaussian Splatting};
\node (process4) [process, below of=process3] {TSDF Fusion};

% Arrows
\draw [arrow] (start) --  (process1);
\draw [arrow] (process1) -- (process2);
\draw [arrow] (process2) -- (process3);
\draw [arrow] (process3) -- (process4);
\draw [arrow, bend left] (start.east) to node[midway, right] {Multi-view remote sensing images} (process1.north east);
\draw [arrow, bend left] (process1.east) to node[midway, right] {Images and masks} (process2.north east);
\draw [arrow, bend left] (process2.east) to node[midway, right] {Images and refined masks} (process3.north east);
\draw [arrow, bend left] (process3.east) to node[midway, right] {Depth and color maps} (process4.north east);


\draw [arrow] (process3.east) -- ++(1cm,0) node[right] {Synthesized 2D images};
\draw [arrow] (process4.east) -- ++(1cm,0) node[right] {3D colored mesh};
\end{tikzpicture}
\captionof{figure}{Flowchart of GBM 3D mesh extraction. The output data modality at each step is denoted on the right. Processes and modules are boxed in blue. Figure sourced from \cite{gbm}.} \label{fig:gbmpipeline}

%\vspace{0.5cm}