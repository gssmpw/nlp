\begin{figure*}[htpb]
\centering
\begin{tikzpicture}[
    node distance=1.cm and 2.cm,
    every node/.style={align=center, font=\small},
    arrow/.style={-{Latex}, thick},
    process/.style={align=center, draw, font=\small, fill=blue!30},
    process2/.style={align=center, draw, font=\small, fill=blue!15},
    outprocess/.style={align=center, rounded corners, draw, font=\small, fill=red!30}
]

% Nodes
\node [outprocess](googleES) {Google Earth Studio};
\node[right=of googleES, process] (gptAPI) {Multi-Agent \\ LLM Module};
\node[right=of gptAPI,process] (gee) {Google Map Platform \\  Integration};
\node[above=1cm of gptAPI] (input) {One of \{\textit{address, place name, postal code,  geographic coordinates}\}};
\node[below=of googleES,process2] (gbm) {Gaussian Building Mesh (GBM)};
\node[below left=1cm and -2.0cm of gbm] (mesh) {\hspace{-5em}3D colored mesh};
\node[right=0.4cm of mesh] (2d) {Synthesized 2D image \\ from new viewpoints};
\node[below=1.5cm of gptAPI](semantic out) {Building semantic descriptions};
\node[below=1.5cm of gee] (gis out){Cloud-based building GIS\\ information + 2D maps/images};


% Arrows
\draw[arrow] (input) -- (googleES);
\draw[arrow] (gee) -- (gptAPI);
\draw[arrow] (googleES) -- (gbm);
\draw[arrow] (googleES) -- (gptAPI);
\draw[arrow] (input) -- (gee);
\draw[arrow] (gbm) -- (mesh);
\draw[arrow] (gbm) -- (2d);
\draw[arrow] (gptAPI) -- (semantic out);
\draw[arrow] (gee) -- (gis out);
\end{tikzpicture}
\caption{Diagram of our Digital Twin Building framework. External modules are boxed in red. Our own tools/modules are boxed in blue. The aspects specifically presented in this paper are in dark blue. Data (both inputs and outputs) are drawn in plain text.}\label{fig:framework}
\end{figure*}