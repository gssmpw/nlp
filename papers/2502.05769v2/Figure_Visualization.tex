\begin{figure*}[htbp]
    \centering
    % First row of subfigures
    \begin{subfigure}{0.4\textwidth}
        \centering
        \includegraphics[trim=200pt 200pt 200pt 100pt, clip, width=\textwidth]{Figures/Mesh.png}  % Replace with your image path% Caption for image (a)
        \label{fig:subfig1}
    \end{subfigure}
    \begin{subfigure}{0.42\textwidth}
        \centering
        \includegraphics[trim=0pt 0pt 50 0pt, clip,width=\textwidth]{Figures/sat_img.png}  % Replace with your image path % Caption for image (b)
        \label{fig:subfig2}
    \end{subfigure}
    \\
    \begin{subfigure}[b]{0.4\textwidth}
        \centering
        \includegraphics[trim=200pt 100pt 200pt 100pt, clip,width=\textwidth]{Figures/depth.png} 
 % Caption for image (c)
        \label{fig:subfig3}
    \end{subfigure}
    \begin{subfigure}[b]{0.43\textwidth}
        \centering
        \includegraphics[trim=0pt 0pt 50 0, clip,width=\textwidth]{Figures/map.png}  % Replace with your image  % Caption for image (d)
        \label{fig:subfig4}
    \end{subfigure}
    \caption{Visualization of results. Top Left: colored 3D mesh; Bottom Left: depth map; Top Right: aerial image with keywords and captions and retrieved polygon mask. Bottom Right: retrieved map with map information (entrance is labelled with a red place marker).}
    \label{fig:visualizationFig}
\end{figure*}