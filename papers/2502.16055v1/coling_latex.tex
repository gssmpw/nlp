% This must be in the first 5 lines to tell arXiv to use pdfLaTeX, which is strongly recommended.
\pdfoutput=1
% In particular, the hyperref package requires pdfLaTeX in order to break URLs across lines.

\documentclass[11pt]{article}

% Change "review" to "final" to generate the final (sometimes called camera-ready) version.
% Change to "preprint" to generate a non-anonymous version with page numbers.
\usepackage[preprint]{coling}

% Standard package includes
\usepackage{times}
\usepackage{latexsym}

% For proper rendering and hyphenation of words containing Latin characters (including in bib files)
\usepackage[T1]{fontenc}
% For Vietnamese characters
% \usepackage[T5]{fontenc}
% See https://www.latex-project.org/help/documentation/encguide.pdf for other character sets

% This assumes your files are encoded as UTF8
\usepackage[utf8]{inputenc}

% This is not strictly necessary, and may be commented out,
% but it will improve the layout of the manuscript,
% and will typically save some space.
\usepackage{microtype}

% This is also not strictly necessary, and may be commented out.
% However, it will improve the aesthetics of text in
% the typewriter font.
\usepackage{inconsolata}

%Including images in your LaTeX document requires adding
%additional package(s)
\usepackage{graphicx}
\usepackage{amsmath}
\usepackage{amssymb}
\usepackage{booktabs}
\usepackage[T1]{fontenc}

\usepackage{array}

\usepackage{multirow}
\usepackage{xcolor}
\usepackage{geometry} %
\usepackage{amsmath}
\usepackage{amssymb}
\usepackage{hyperref}
\usepackage{booktabs}
\usepackage{wrapfig}
\usepackage{url}
\usepackage{xcolor}
\usepackage{algorithm} 
\usepackage{stfloats}
\usepackage{algpseudocode} 
\renewcommand{\algorithmicrequire}{ \textbf{Input:}}     %
\renewcommand{\algorithmicensure}{ \textbf{Output:}}    %

\newcolumntype{P}[1]{>{\centering\arraybackslash}p{#1}}

\pdfobjcompresslevel=0
% If the title and author information does not fit in the area allocated, uncomment the following
%
%\setlength\titlebox{<dim>}
%
% and set <dim> to something 5cm or larger.

%%%%%%%%% TITLE - PLEASE UPDATE
\title{MedForge: Building Medical Foundation Models Like Open Source Software Development}

\author{
\textbf{Zheling Tan}$^{1*}$, 
\textbf{Kexin Ding}$^{2*}$, 
\textbf{Jin Gao}$^{1}$, 
\textbf{Mu Zhou}$^{2}$, \\
\textbf{Dimitris Metaxas}$^{2}$, 
\textbf{Shaoting Zhang}$^{1}$,
\textbf{Dequan Wang}$^{1\dagger}$\\
$^1$Shanghai Jiao Tong University \quad
$^2$Rutgers University\\
$*$Equal contribution \quad
$\dagger$Corresponding author \\
}

% \author[1*]{\textbf{Zheling Tan}}
% \author[2*]{\textbf{Kexin Ding}}
% \author[1]{\textbf{Jin Gao}}
% \author[2]{\textbf{Mu Zhou}}
% \author[2]{\textbf{Dimitris Metaxas}}
% \author[1]{\textbf{Shaoting Zhang}}
% \author[1$\dagger$]{\textbf{Dequan Wang}}

% % \affil[1]{Georgia Institute of Technology}
% % \affil[2]{Stanford University}
% % \affil[3]{Amazon GenAI}
% % \affil[ ]{\texttt{nsinha68@gatech.edu}, \texttt{hi@vinija.ai}, \texttt{hi@aman.ai}}


% % \author[1*]{Zheling Tan}
% % \author[2*]{Kexin Ding}
% % \author[1]{Jin Gao}
% % \author[2]{Mu Zhou}
% % \author[2]{Dimitris Metaxas}
% % \author[1]{Shaoting Zhang}
% % \author[1$\dagger$]{Dequan Wang}

% \affil[1]{Shanghai Jiao Tong University}
% \affil[2]{Rutgers University}
% \affil[*]{Equal contribution}
% \affil[$\dagger$]{Corresponding author}
\begin{document}

\maketitle
%%%%%%%%% ABSTRACT
\begin{abstract}
Foundational models (FMs) have made significant strides in the healthcare domain.
Yet the data silo challenge and privacy concern remain in healthcare systems, hindering safe medical data sharing and collaborative model development among institutions. The collection and curation of scalable clinical datasets increasingly become the bottleneck for training strong FMs. In this study, we propose \textbf{Med}ical \textbf{Fo}undation Models Me\textbf{rg}ing (\textbf{MedForge}), a cooperative framework enabling a community-driven medical foundation model development, meanwhile preventing the information leakage of raw patient data and mitigating synchronization model development issues across clinical institutions. MedForge offers a bottom-up model construction mechanism by flexibly merging task-specific Low-Rank Adaptation (LoRA) modules, which can adapt to downstream tasks while retaining original model parameters. Through an asynchronous LoRA module integration scheme, the resulting composite model can progressively enhance its comprehensive performance on various clinical tasks. 
MedForge shows strong performance on multiple clinical datasets (e.g., breast cancer, lung cancer, and colon cancer) collected from different institutions. 
Our major findings highlight the value of collaborative foundation models in advancing multi-center clinical collaboration effectively and cohesively.
Our code is publicly available at \url{https://github.com/TanZheling/MedForge}.
\end{abstract}

% \footnote{Code and dataset can be found here: \href{https://github.com/neelabhsinha/vlm-selection-tasks-domains-knowledge-type}{https://github.com/neelabhsinha/vlm-selection-tasks-domains-knowledge-type}}

%\footnote{Dataset and code is shared at the following Anonymous repository link (will be changed to Github link/web page in camera-ready version): \href{https://anonymous.4open.science/r/vlm-selection-tasks-domains-knowledge-type-7668}{https://anonymous.4open.science/r/vlm-selection-tasks-domains-knowledge-type-7668}. Code and Appendix is available in Supplementary material zip (Data isn't there in zip only due to size limits).}

%!TEX root = gcn.tex
\section{Introduction}
Graphs, representing structural data and topology, are widely used across various domains, such as social networks and merchandising transactions.
Graph convolutional networks (GCN)~\cite{iclr/KipfW17} have significantly enhanced model training on these interconnected nodes.
However, these graphs often contain sensitive information that should not be leaked to untrusted parties.
For example, companies may analyze sensitive demographic and behavioral data about users for applications ranging from targeted advertising to personalized medicine.
Given the data-centric nature and analytical power of GCN training, addressing these privacy concerns is imperative.

Secure multi-party computation (MPC)~\cite{crypto/ChaumDG87,crypto/ChenC06,eurocrypt/CiampiRSW22} is a critical tool for privacy-preserving machine learning, enabling mutually distrustful parties to collaboratively train models with privacy protection over inputs and (intermediate) computations.
While research advances (\eg,~\cite{ccs/RatheeRKCGRS20,uss/NgC21,sp21/TanKTW,uss/WatsonWP22,icml/Keller022,ccs/ABY318,folkerts2023redsec}) support secure training on convolutional neural networks (CNNs) efficiently, private GCN training with MPC over graphs remains challenging.

Graph convolutional layers in GCNs involve multiplications with a (normalized) adjacency matrix containing $\numedge$ non-zero values in a $\numnode \times \numnode$ matrix for a graph with $\numnode$ nodes and $\numedge$ edges.
The graphs are typically sparse but large.
One could use the standard Beaver-triple-based protocol to securely perform these sparse matrix multiplications by treating graph convolution as ordinary dense matrix multiplication.
However, this approach incurs $O(\numnode^2)$ communication and memory costs due to computations on irrelevant nodes.
%
Integrating existing cryptographic advances, the initial effort of SecGNN~\cite{tsc/WangZJ23,nips/RanXLWQW23} requires heavy communication or computational overhead.
Recently, CoGNN~\cite{ccs/ZouLSLXX24} optimizes the overhead in terms of  horizontal data partitioning, proposing a semi-honest secure framework.
Research for secure GCN over vertical data  remains nascent.

Current MPC studies, for GCN or not, have primarily targeted settings where participants own different data samples, \ie, horizontally partitioned data~\cite{ccs/ZouLSLXX24}.
MPC specialized for scenarios where parties hold different types of features~\cite{tkde/LiuKZPHYOZY24,icml/CastigliaZ0KBP23,nips/Wang0ZLWL23} is rare.
This paper studies $2$-party secure GCN training for these vertical partition cases, where one party holds private graph topology (\eg, edges) while the other owns private node features.
For instance, LinkedIn holds private social relationships between users, while banks own users' private bank statements.
Such real-world graph structures underpin the relevance of our focus.
To our knowledge, no prior work tackles secure GCN training in this context, which is crucial for cross-silo collaboration.


To realize secure GCN over vertically split data, we tailor MPC protocols for sparse graph convolution, which fundamentally involves sparse (adjacency) matrix multiplication.
Recent studies have begun exploring MPC protocols for sparse matrix multiplication (SMM).
ROOM~\cite{ccs/SchoppmannG0P19}, a seminal work on SMM, requires foreknowledge of sparsity types: whether the input matrices are row-sparse or column-sparse.
Unfortunately, GCN typically trains on graphs with arbitrary sparsity, where nodes have varying degrees and no specific sparsity constraints.
Moreover, the adjacency matrix in GCN often contains a self-loop operation represented by adding the identity matrix, which is neither row- nor column-sparse.
Araki~\etal~\cite{ccs/Araki0OPRT21} avoid this limitation in their scalable, secure graph analysis work, yet it does not cover vertical partition.

% and related primitives
To bridge this gap, we propose a secure sparse matrix multiplication protocol, \osmm, achieving \emph{accurate, efficient, and secure GCN training over vertical data} for the first time.

\subsection{New Techniques for Sparse Matrices}
The cost of evaluating a GCN layer is dominated by SMM in the form of $\adjmat\feamat$, where $\adjmat$ is a sparse adjacency matrix of a (directed) graph $\graph$ and $\feamat$ is a dense matrix of node features.
For unrelated nodes, which often constitute a substantial portion, the element-wise products $0\cdot x$ are always zero.
Our efficient MPC design 
avoids unnecessary secure computation over unrelated nodes by focusing on computing non-zero results while concealing the sparse topology.
We achieve this~by:
1) decomposing the sparse matrix $\adjmat$ into a product of matrices (\S\ref{sec::sgc}), including permutation and binary diagonal matrices, that can \emph{faithfully} represent the original graph topology;
2) devising specialized protocols (\S\ref{sec::smm_protocol}) for efficiently multiplying the structured matrices while hiding sparsity topology.


 
\subsubsection{Sparse Matrix Decomposition}
We decompose adjacency matrix $\adjmat$ of $\graph$ into two bipartite graphs: one represented by sparse matrix $\adjout$, linking the out-degree nodes to edges, the other 
by sparse matrix $\adjin$,
linking edges to in-degree nodes.

%\ie, we decompose $\adjmat$ into $\adjout \adjin$, where $\adjout$ and $\adjin$ are sparse matrices representing these connections.
%linking out-degree nodes to edges and edges to in-degree nodes of $\graph$, respectively.

We then permute the columns of $\adjout$ and the rows of $\adjin$ so that the permuted matrices $\adjout'$ and $\adjin'$ have non-zero positions with \emph{monotonically non-decreasing} row and column indices.
A permutation $\sigma$ is used to preserve the edge topology, leading to an initial decomposition of $\adjmat = \adjout'\sigma \adjin'$.
This is further refined into a sequence of \emph{linear transformations}, 
which can be efficiently computed by our MPC protocols for 
\emph{oblivious permutation}
%($\Pi_{\ssp}$) 
and \emph{oblivious selection-multiplication}.
% ($\Pi_\SM$)
\iffalse
Our approach leverages bipartite graph representation and the monotonicity of non-zero positions to decompose a general sparse matrix into linear transformations, enhancing the efficiency of our MPC protocols.
\fi
Our decomposition approach is not limited to GCNs but also general~SMM 
by 
%simply 
treating them 
as adjacency matrices.
%of a graph.
%Since any sparse matrix can be viewed 

%allowing the same technique to be applied.

 
\subsubsection{New Protocols for Linear Transformations}
\emph{Oblivious permutation} (OP) is a two-party protocol taking a private permutation $\sigma$ and a private vector $\xvec$ from the two parties, respectively, and generating a secret share $\l\sigma \xvec\r$ between them.
Our OP protocol employs correlated randomnesses generated in an input-independent offline phase to mask $\sigma$ and $\xvec$ for secure computations on intermediate results, requiring only $1$ round in the online phase (\cf, $\ge 2$ in previous works~\cite{ccs/AsharovHIKNPTT22, ccs/Araki0OPRT21}).

Another crucial two-party protocol in our work is \emph{oblivious selection-multiplication} (OSM).
It takes a private bit~$s$ from a party and secret share $\l x\r$ of an arithmetic number~$x$ owned by the two parties as input and generates secret share $\l sx\r$.
%between them.
%Like our OP protocol, o
Our $1$-round OSM protocol also uses pre-computed randomnesses to mask $s$ and $x$.
%for secure computations.
Compared to the Beaver-triple-based~\cite{crypto/Beaver91a} and oblivious-transfer (OT)-based approaches~\cite{pkc/Tzeng02}, our protocol saves ${\sim}50\%$ of online communication while having the same offline communication and round complexities.

By decomposing the sparse matrix into linear transformations and applying our specialized protocols, our \osmm protocol
%($\prosmm$) 
reduces the complexity of evaluating $\numnode \times \numnode$ sparse matrices with $\numedge$ non-zero values from $O(\numnode^2)$ to $O(\numedge)$.

%(\S\ref{sec::secgcn})
\subsection{\cgnn: Secure GCN made Efficient}
Supported by our new sparsity techniques, we build \cgnn, 
a two-party computation (2PC) framework for GCN inference and training over vertical
%ly split
data.
Our contributions include:

1) We are the first to explore sparsity over vertically split, secret-shared data in MPC, enabling decompositions of sparse matrices with arbitrary sparsity and isolating computations that can be performed in plaintext without sacrificing privacy.

2) We propose two efficient $2$PC primitives for OP and OSM, both optimally single-round.
Combined with our sparse matrix decomposition approach, our \osmm protocol ($\prosmm$) achieves constant-round communication costs of $O(\numedge)$, reducing memory requirements and avoiding out-of-memory errors for large matrices.
In practice, it saves $99\%+$ communication
%(Table~\ref{table:comm_smm}) 
and reduces ${\sim}72\%$ memory usage over large $(5000\times5000)$ matrices compared with using Beaver triples.
%(Table~\ref{table:mem_smm_sparse}) ${\sim}16\%$-

3) We build an end-to-end secure GCN framework for inference and training over vertically split data, maintaining accuracy on par with plaintext computations.
We will open-source our evaluation code for research and deployment.

To evaluate the performance of $\cgnn$, we conducted extensive experiments over three standard graph datasets (Cora~\cite{aim/SenNBGGE08}, Citeseer~\cite{dl/GilesBL98}, and Pubmed~\cite{ijcnlp/DernoncourtL17}),
reporting communication, memory usage, accuracy, and running time under varying network conditions, along with an ablation study with or without \osmm.
Below, we highlight our key achievements.

\textit{Communication (\S\ref{sec::comm_compare_gcn}).}
$\cgnn$ saves communication by $50$-$80\%$.
(\cf,~CoGNN~\cite{ccs/KotiKPG24}, OblivGNN~\cite{uss/XuL0AYY24}).

\textit{Memory usage (\S\ref{sec::smmmemory}).}
\cgnn alleviates out-of-memory problems of using %the standard 
Beaver-triples~\cite{crypto/Beaver91a} for large datasets.

\textit{Accuracy (\S\ref{sec::acc_compare_gcn}).}
$\cgnn$ achieves inference and training accuracy comparable to plaintext counterparts.
%training accuracy $\{76\%$, $65.1\%$, $75.2\%\}$ comparable to $\{75.7\%$, $65.4\%$, $74.5\%\}$ in plaintext.

{\textit{Computational efficiency (\S\ref{sec::time_net}).}} 
%If the network is worse in bandwidth and better in latency, $\cgnn$ shows more benefits.
$\cgnn$ is faster by $6$-$45\%$ in inference and $28$-$95\%$ in training across various networks and excels in narrow-bandwidth and low-latency~ones.

{\textit{Impact of \osmm (\S\ref{sec:ablation}).}}
Our \osmm protocol shows a $10$-$42\times$ speed-up for $5000\times 5000$ matrices and saves $10$-2$1\%$ memory for ``small'' datasets and up to $90\%$+ for larger ones.

\section{Related Works}

\textbf{Enhancing LLMs' Theory of Mind.} There has been systematic evaluation that revealed LLMs' limitations in achieving robust Theory of Mind inference \citep{ullman2023large, shapira2023clever}. To enhance LLMs' Theory of Mind capacity, recent works have proposed various prompting techniques. For instance, SimToM \citep{wilf2023think} encourages LLMs to adopt perspective-taking, PercepToM \citep{jung2024perceptions} improves perception-to-belief inference by extracting relevant contextual details, and \citet{huang2024notion} utilize an LLM as a world model to track environmental changes and refine prompts. Explicit symbolic modules also seem to improve LLM's accuracy through dynamic updates based on inputs. Specifically, TimeToM \citep{hou2024timetom} constructs a temporal reasoning framework to support inference, while SymbolicToM \citep{sclar2023minding} uses graphical representations to track characters' beliefs. Additionally, \citet{wagner2024mind} investigates ToM's necessity and the level of recursion required for specific tasks. However, these approaches continue to exhibit systematic errors in long contexts, complex behaviors, and recursive reasoning due to inherent limitations in inference and modeling \citep{jin2024mmtom,shi2024muma}. Most of them rely on domain-specific designs, lacking open-endedness.


\textbf{Model-based Theory of Mind inference.} Model-based Theory of Mind inference, in particular, Bayesian inverse planning (BIP) \citep{baker2009action,ullman2009help,baker2017rational,zhi2020online}, explicitly constructs representations of agents' mental states and how mental states guide agents' behavior via Bayesian Theory of Mind (BToM) models. These methods can reverse engineer human ToM inference in simple domains \citep[e.g.,][]{baker2017rational,netanyahu2021phase,shu2021agent}. Recent works have proposed to combine BIP with LLMs to achieve robust ToM inference in more realistic settings \citep{ying2023neuro, jin2024mmtom, shi2024muma}. However, these methods require manual specification of the BToM models as well as rigid, domain-specific implementations of Bayesian inference, limiting their adaptability to open-ended scenarios. To overcome this limitation, we propose \ours, a method capable of automatically modeling mental variables across diverse conditions and conducting automated BIP without domain-specific knowledge or implementations.


\begin{figure*}[ht]
  \centering
  \includegraphics[width=\linewidth]{figures/benchmarks_and_models.pdf}
    \vspace{-15pt}
  \caption{Examples questions (top panels) and the necessary Bayesian Theory of Mind (BToM) model for Bayesian inverse planning (bottom panels) in diverse Theory of Mind benchmarks. \ours aims to answer any Theory of Mind question in a variety of benchmarks, encompassing different mental variables, observable contexts, numbers of agents, the presence or absence of utterances, wording styles, and modalities. It proposes and iteratively adjusts an appropriate BToM and conducts automated Bayesian inverse planning based on the model.
  There can be more types of questions/models in each benchmark beyond the examples shown in this figure.}
  \label{fig:benchmarks_and_models}
  %\vspace{-0.75em}
  \vspace{-10pt}
\end{figure*}



\textbf{Automated Modeling with LLMs.} There has been an increasing interest in integrating LLMs with inductive reasoning and probabilistic inference for automated modeling. \citet{piriyakulkij2024doing} combine LLMs with Sequential Monte Carlo to perform probabilistic inference about underlying rules. Iterative hypothesis refinement techniques \citep{qiu2023phenomenal} further enhance LLM-based inductive reasoning by iteratively proposing, selecting, and refining textual hypotheses of rules. Beyond rule-based hypotheses, \citet{wang2023hypothesis} prompt LLMs to generate natural language hypotheses that are then implemented as verifiable programs, while \citet{li2024automated} propose a method in which LLMs construct, critique, and refine statistical models represented as probabilistic programs for data modeling. \citet{cross2024hypothetical} leverage LLMs to propose and evaluate agent strategies for multi-agent planning but do not specifically infer individual mental variables. Our method also aims to achieve automated modeling with LLMs. Unlike prior works, we propose a novel automated model discovery approach for Bayesian inverse planning, where the objective is to confidently infer any mental variable given any context via constructing a suitable Bayesian Theory of Mind model.
\section{Method}

In Fig. \ref{fig:overview}, we illustrate two major stages of MedForge for collaborative model development, including feature branch development (Sec~\ref{branch}) and model merging (Sec~\ref{forging}). In the feature branch development, individual contributors (i.e., medical centers) could make individual knowledge contributions asynchronously. Our MedForge allows each contributor to develop their own plugin module and distilled data locally without the need to share any private data. In the model merging stage, MedForge enables multi-task knowledge integration by merging the well-prepared plugin module asynchronously. This key integration process is guided by the distilled dataset produced by individual branch contributors, resulting in a generalizable model that performs strongly among multiple tasks.


\subsection{Preliminary}
\label{pre}
In MedForge, the development of a multi-capability model relies on the multi-center and multi-task knowledge introduced by branch plugin modules and the distilled datasets.
The relationship between the main base model and branch plugin modules in our proposed MedForge is conceptually similar to the relationship between the main repository and its branches in collaborative software version control platforms (e.g., GitHub~\cite{github}). 
To facilitate plugin module training on branches and model merging, we use the parameter-efficient finetuning (PEFT) technique~\cite{hu2021lora} for integrating knowledge from individual contributors into the branch plugin modules. 

\subsubsection{Parameter-efficient Finetuning}
Compared to resource-intensive full-parameter finetuning, parameter-efficient finetuning (PEFT) only updates a small fraction of the pretrained model parameters to reduce computational costs and accelerate training on specific tasks. These benefits are particularly crucial in medical scenarios where computational resources are often limited.
As the representative PEFT technique, LoRA (Low-Rank Adaptation)~\cite{hu2021lora} is widely utilized in resource-constrained downstream finetuning scenarios. In our MedForge, each contributor trains a lightweight LoRA on a specific task as the branch plugin module. LoRA decomposes the weight matrices of the target layer into two low-rank matrices to represent the update made to the main model when adapting to downstream tasks. If the target weight matrix is $W_0 \in R^{d \times k}$, during the adaptation, the updated weight matrix can be represented as $W_0+\Delta W=W_0+B A$, where $B \in \mathbb{R}^{d \times r}, A \in \mathbb{R}^{r \times k}$ are the low-rank matrices with rank $r \ll  \min (d, k)$ and $AB$ constitute the LoRA module. 



\subsubsection{Dataset Distillation}
Dataset distillation~\cite{wang2018dataset, yu2023dataset, lei2023comprehensive} is particularly valuable for medicine scenarios that have limited storage capabilities, restricted transmitting bandwidth, and high concerns for data privacy~\cite{li2024dataset}. 
We leverage the power of dataset distillation to synthesize a small-scale distilled dataset from the original data.

The distilled datasets serve as the training set in the subsequent merging stage to allow multi-center knowledge integration. Models trained on this distilled dataset maintain comparable performance to those trained on the original dataset (\ref{tab:main_res}). Moreover, the distinctive visual characteristics among images of the raw dataset are blurred (see \ref{fig:overview}(a)), which alleviates the potential patient information leakage. 

To perform dataset distillation, we define the original dataset as $\mathcal{T}=\{x_i,y_i\}^N_{i=1}$ and the model parameters as $\theta$. The dataset distillation aims to synthesize a distilled dataset ${\mathcal{S}=\{{s_i},\tilde{y_i}\}^M_{i=1}}$ with a much smaller scale (${M \ll N}$), while models trained on $\mathcal{S}$ can show similar performance as models trained on $\mathcal{T}$. 
This process is achieved by narrowing the performance gap between the real dataset $\mathcal{T}$ and the synthesized dataset $\mathcal{S}$. In MedForge, we utilize the distribution matching (DM)~\cite{zhao2023dataset}, which increases data distribution similarity between the synthesized distilled data and the real dataset
The distribution similarity between the real and synthesized dataset is evaluated through the empirical estimate of the Maximum Mean Discrepancy (MMD)~\cite{gretton2012kernel}:
\begin{equation}
\mathbb{E}_{\boldsymbol{\vartheta} \sim P_{\vartheta}}\left\|\frac{1}{|\mathcal{T}|} \sum_{i=1}^{|\mathcal{T}|} \psi_{\boldsymbol{\vartheta}}\left(\boldsymbol{x}_i\right)-\frac{1}{|\mathcal{S}|} \sum_{j=1}^{|\mathcal{S}|} \psi_{\boldsymbol{\vartheta}}\left(\boldsymbol{s}_j\right)\right\|^2
\end{equation}

where $P_\vartheta$ is the distribution of network parameters, $\psi_{\boldsymbol{\vartheta}}$ is a feature extractor. Then the distillation loss $\mathcal{L}_{DM}$ is:
\begin{equation}\scalebox{0.9}{$
\mathcal{L}_{\mathrm{DM}}(\mathcal{T},\mathcal{S},\psi_{\boldsymbol{\vartheta}})=\sum_{c=0}^{C-1}\left\|\frac{1}{\left|\mathcal{T}_c\right|} \sum_{\mathbf{x} \in \mathcal{T}_c} \psi(\mathbf{x})-\frac{1}{\left|\mathcal{S}_c\right|} \sum_{\mathbf{s} \in \mathcal{S}_c} \psi(\mathbf{s})\right\|^2$}
\end{equation}

We also applied the Differentiable Siamese Augmentation (DSA) strategy~\cite{zhao2021dataset} in the training process of distilled data to enhance the quality of the distilled data. DSA could ensure the distilled dataset is representative of the original data by exploiting information in real data with various transformations. The distilled images extract invariant and critical features from these augmented real images to ensure the distilled dataset remains representative.
\begin{figure}[t]
    \centering
    \includegraphics[width=\linewidth]{assets/img/model_arch.png}
    \caption{\textbf{Main model architecture.} We adopt CLIP as the base module and attach LoRA modules to the visual encoder and visual projection as the plugin module. During all the procedures, only the plugin modules are tuned while the rest are frozen. We get the classification result by comparing the cosine similarity of the visual and text embeddings.}
    \label{fig:model_arch}
\end{figure}

\subsection{Feature Branch Development}
\label{branch}
In the feature branch development stage, the branch contributors are responsible for providing the locally trained branch plugin modules and the distilled data to the MedForge platform, as shown in Fig~\ref{fig:overview} (a).
In collaborative software development, contributors work on individual feature branches, push their changes to the main platform, and later merge the changes into the main branch to update the repository with new features. Inspired by such collaborative workflow, branch contributors in MedForge follow similar preparations before the merging stage, enabling the integration of diverse branch knowledge into the main branch while effectively utilizing local resources.

MedForge consistently keeps a base module and a forge item as the main branch. The base module preserves generative knowledge of the foundation model pretrained on natural image datasets (i.e., ImageNet~\cite{deng2009imagenet}), while the forge item contains model merging information that guides the integration of feature branch knowledge (i.e., a merged plugin module or the merging coefficients assigned to plugin modules). 
Similar to individual software developers working in their own branches, each branch contributor (e.g., individual medical centers) trains a task-specific plugin module using their private data to introduce feature branch knowledge into the main branch. These branch plugin modules are then committed and pushed to update the forge items of the main branch in the merging stage, thus enhancing the model's multi-task capabilities.


\begin{figure*}
    \centering
    \includegraphics[width=\textwidth]{assets/img/fusion.png}
    \caption{\textbf{The detailed methodology of the proposed Fusion.} Branch contributors can asynchronously commit and push their branch plugin modules and the distilled datasets. the plugin modules will then be weighted fused to the current main plugin module.}

    \label{fig:merge}
\end{figure*}


Regarding model architecture, MedForge contains a base module and a plugin module (Fig ~\ref{fig:model_arch}). The base module is pretrained on general datasets (e.g., ImageNet) and remains the model parameters frozen in all processes and branches (main and feature branches) to avoid catastrophic forgetting of foundational knowledge acquired from pretraining. Meanwhile, the plugin module is adaptable for knowledge integration and can be flexibly added or removed from the base module, allowing updates without affecting the base model. In our study, we use the pretrained CLIP~\cite{radford2021learning} model as the base module. For the language encoder and projection layer of the CLIP model, all the parameters are frozen, which enables us to directly leverage the language capability of the original CLIP model. For the visual encoder, we apply LoRA on weight matrices of query ($W_q$) and value ($W_v$), following the previous study~\cite{hu2021lora}. To better adapt the model to downstream visual tasks, we apply the LoRA technique to both the visual encoder and the visual projection, and these LoRA modules perform as the plugin module. During the training, only the plugin module (LoRA modules) participates in parameter updates, while the base module (the original CLIP model) remains unchanged. 

In addition to the plugin modules, the feature branch contributors also develop a distilled dataset based on their private local data, which encapsulates essential patterns and features, serving as the foundation for training the merging coefficients in the subsequent merging stage~\ref{forging}. Compared to previous model merging approaches that rely on whole datasets or few-shot sampling, distilled data is lightweight and representative, mitigating the privacy risks associated with sharing raw data. 
We illustrate our distillation procedure in Algorithm~\ref{algorithm:alg1}. In each distillation step, the synthesized data $\mathcal{S}$ will be updated by minimizing $\mathcal{L}_{DM}$.
\begin{algorithmic}[1]
    \STATE \textbf{Input:} A list of clauses $C$
    \STATE \textbf{Output:} List of primary outputs $PO$, primary inputs $PI$, intermediate variables $IV$, and Boolearn expressions $BE$
    \STATE $SC$ = [] \COMMENT{List of sub-clauses}
    \FOR{$i = 1$ to length($C$)}
        % \IF{$C[i] \cap SC = \emptyset$}
        %     \STATE Append \text{Simplify}(\text{FindBooleanExpression}([], $SC$)) to $BE$
        %     %\COMMENT{Simplify Boolean expression}
        %     \FOR{each item $w$ in $SC$}
        %         \IF{$w \notin IV$ and $w \neq v$}
        %             \STATE Append $w$ to $PI$
        %         \ENDIF
        %     \ENDFOR
        %     \STATE $SC$ = []
        % \ELSE
            \STATE Append $C[i]$ to $SC$
            \FOR{each item $v$ in $SC$}
                \IF{$v \notin PI$ and $v \notin IV$}
                    \STATE $f \gets \text{FindBooleanExpression}(v, SC)$ %\COMMENT{Find Boolean expression for $v$}
                    \STATE $g \gets \text{FindBooleanExpression}(\neg v, SC)$ %\COMMENT{Find Boolean expression for $\neg v$}
                    \IF{$f = \neg g$}
                        \STATE Append \text{Simplify}($f$) to $BE$ %\COMMENT{Simplify Boolean expression}
                        \IF{$f = True$ or $f = False$}
                            \STATE Append $v$ to $PO$
                        \ELSE
                            \STATE Append $v$ to $IV$
                        \ENDIF
                        \FOR{each item $w$ in $SC$}
                            \IF{$w \notin IV$ and $w \neq v$}
                                \STATE Append $w$ to $PI$
                            \ENDIF
                        \ENDFOR
                        \STATE SC = []
                        \STATE \textbf{break}
                    \ENDIF
                \ENDIF    
            \ENDFOR
        % \ENDIF
    \ENDFOR
    \STATE \textbf{Return} $PO, PI, IV, BE$
    \vspace{-0.65cm}
\end{algorithmic}



\subsection{MedForge Merging Stage}
\label{forging}
Following the feature branch development stage illustrated in Fig~\ref{fig:overview} (a), branch contributors push and merge their branch plugin modules along with the corresponding distilled dataset into the main branch, as shown in Fig~\ref{fig:overview} (b). Our MedForge allows an incremental capability accumulation from branches to construct a comprehensive medical model that can handle multiple tasks.

In the merging stage, the $i^{th}$ branch contributor is assigned a coefficient $w'_i$ for the contribution of merging, while the coefficient for the current main branch is $w_i$. By adaptively adjusting the value of coefficients, the main branch can balance and coordinate updates from different contributors, ultimately enhancing the overall performance of the model across multiple tasks.
The optimization of the coefficients is done by minimizing the cross-entropy loss for classification based on the distilled datasets. We also add $L1$ regularization to the loss to regulate the weights to avoid outlier coefficient values (e.g., extremely large or small coefficient values)~\cite{huang2023lorahub}. During optimization, following~\cite{huang2023lorahub}, we utilize Shiwa algorithm~\cite{liu2020versatile} to enable model merging under gradient-free conditions, with lower computational and time costs. The optimizer selector~\cite{liu2020versatile} automatically chooses the most suitable optimization method for coefficient optimization. 

In the following sections, we introduce the two merging methods used in our MedForge: Fusion and Mixture. In MedForge-Fusion, the parameters of the branch plugin modules are fused into the main branch after each round of the merging stage. For MedForge-Mixture, the outputs of the branch modules are weighted and summed based on their respective coefficients rather than directly applying the weighted sum to the model parameters. This largely preserves the internal parameter structure of each branch module.

\paragraph{MedForge-Fusion}
In MedForge, forge items are utilized to facilitate the integration of branch knowledge into the main branch.
For MedForge-Fusion, the forge item refers to adaptable main plugin modules. When the $i^{th}$ branch contributor pushes its branch plugin module $\theta'_i=A'_iB'_i$ to the main branch, the current main plugin module $\theta_{i-1}=A_{i-1}B_{i-1}$ will be updated to $\theta_{i}=A_{i}B_{i}$. The parameters of the branch and the current main plugin modules are weighted with coefficients and added to fuse a new version. The $A_i$, $B_i$ are the low-rank matrices composing the LoRA module $\theta_i$. The detailed fusion process can be represented as:
\begin{equation}
\theta_{i}=(w_i A_{i-1}+w'_i A'_i)(w_i B_{i-1}+ w'_i B'_i)
\end{equation}
Where $w_i$ is the coefficient assigned to the current main branch, while $w'_i$ is the coefficient assigned to the branch contributor. After this round of merging, the resulting plugin module $\theta_{i}$ is the updated version of main forging item, thus the main model is able to obtain new capacity introduced by the current branch contributor. When new contributors push their plugin modules and distilled datasets, the main branch can be incrementally updated through merging stages, and the optimization of the coefficients is guided by distilled data.
As shown in Fig.~\ref{fig:merge}, though multiple contributors commit their branch plugin modules and distilled datasets at different times, they can flexibly merge their plugin modules with the current main branch. After each merging round, the plugin module of the main branch will be updated, and thus the version iteration has been achieved.
\begin{figure*}[t]
    \centering
    \includegraphics[width=\textwidth]{assets/img/mixture.png}
    \caption{\textbf{The detailed methodology of the proposed Mixture.} Branch contributors can asynchronously commit and push their branch plugin modules and the distilled datasets. the outputs of different plugin modules will be weighted aggregated. The weights of each merging step will be saved.}

    \label{fig:mixmerge}
\end{figure*}


\paragraph{MedForge-Mixture}
To further improve the model merging performance, inspired by~\cite{zhao2024loraretriever}, we also propose medForge-mixture. For MedForge-Mixture, the forge items refer to the optimized coefficients.
As shown in Fig.~\ref{fig:mixmerge}, for MedForge-Mixture, the coefficient of each branch contributor is acquired and optimized based on distilled datasets. Then the outputs of plugin modules will be weighted combined with these coefficients to get the merged output. 

For each merging round, with branch contributor $i$, the branch coefficient is $w'_i$, the main coefficient is $w_i$, the branch plugin module is $\theta'_i=A'_iB'_i$, and the current main plugin module is $\theta_i=A_iB_i$. With the input $x$, the resulted MedForge-Mixture output can be represented as:
\begin{equation}
y_{i}=w_i A_{i-1} B_{i-1} x+w'_i A'_i B'_i x
\end{equation}

In this way, MedForge encourages additional contributors as the workflow supports continuous incremental knowledge updates.

Overall, both MedForge merging strategies greatly improve the communication efficiency among contributors. We use this design to build a multi-task medical foundation model that enhances the full utilization of resources in the medical community. For the MedForge-Fusion strategy, the main plugin module is updated after each merging round, thus avoiding storing the previous plugin modules and saving space. Meanwhile, the MedForge-Mixture strategy avoids directly updating the parameters of each plugin module, thus preserving their original structure and preventing the introduction of additional noise, which enhances the robustness and stability of the models.

\section{Experiments and Results}
\subsection{Experiment Settings}

\begin{table*}[ht]
    \centering
    % \small
    \caption{The main results of our experimentation. Each row group corresponds to the results for the given dataset, with each row showcasing the metric results for each model. The columns include all the main approaches, with \textbf{bold} highlighting the best result across all approaches.}
    \small
    \begin{tabular}{llccccc}
      \toprule
      Dataset & Model & Baseline & RAG & CoT & RaR & \rephrase \\
      \midrule
      \multirow[l]{3}{*}{TriviaQA}
          & Llama-3.2 3B  & 59.5 & 82.0 & 87.5  & 86.0 &  \textbf{88.5}    \\
          & Llama-3.1 8B  & 76.5 & 89.5 & 90.5  & 89.5 &  \textbf{92.5}    \\
          & GPT-4o    & 88.7 & 92.7 & 92.7  & 94.7 &  \textbf{96.7}    \\
      \midrule
      \multirow[l]{3}{*}{HotpotQA}
          & Llama-3.2 3B  &  17.5  & 26.0  & 26.5   & 25.0  &  \textbf{31.5}   \\
          & Llama-3.1 8B  &  23.0  & 26.5  & 31.0   & 28.5  &  \textbf{33.5}   \\
          & GPT-4o    &  44.0  & 45.3  & 46.7   & \textbf{47.3}  &  46.7   \\
      \midrule
      \multirow[l]{3}{*}{ASQA}
          & Llama-3.2 3B  &  14.2 & 21.5  & 21.9  & 23.5  &  \textbf{26.6}   \\ 
          & Llama-3.1 8B  &  14.6 & 23.1  & 24.8  & 25.5  &  \textbf{28.8}   \\ 
          & GPT-4o    &  26.8 & 30.4  & \textbf{31.9}  & 30.1 & 31.7 \\ 
      \bottomrule
    \end{tabular}
    \label{tab:main}
\end{table*}



\textbf{Datasets}. We conduct experiments on two datasets: CC-news\footnote{\href{https://huggingface.co/datasets/vblagoje/cc_news}{Huggingface: vblagoje/cc\_news}} and Wikipedia\footnote{\href{https://huggingface.co/datasets/legacy-datasets/wikipedia}{Huggingface: legacy-datasets/Wikipedia}}. CC-news is a large collection of news articles which includes diverse topics and reflects real-world temporal events. Meanwhile, Wikipedia covers general knowledge across a wide range of disciplines, such as history, science, arts, and popular culture.\\
\textbf{LLMs}: We experiment on three models including \gpt~(124M)~\cite{gpt2radford}, \pythia~(1.4B)~\cite{pythia}, and \llama~(7B)~\cite{llama2touvron2023}. This selection of models ensures a wide range of model sizes from small to large that allows us to analyze scaling effects and generalizability across different capacities. \\
\textbf{Evaluation Metrics}. For evaluating language modeling performance, we measure perplexity (PPL), as it reflects the overall effectiveness of the model and is often correlated with improvements in other downstream tasks~\cite{kaplan2020scalinglaws, lmsfewshot}. For defense effectiveness, we consider the attack area under the curve (AUC) value and True Positive Rate (TPR) at low False Positive Rate (FPR). In total, we perform 4 MIAs with different MIA signals. Given the sample $x$, the MIA signal function $f$ is formulated as follows: \\
$\bullet$ Loss~\cite{8429311} utilizes the negative cross entropy loss as the MIA signal. 
    \[f_\text{Loss}(x) = \mathcal{L}_\text{CE}(\theta; x) \]
$\bullet$ Ref-Loss~\cite{Carlini2020ExtractingTD} considers the loss differences between the target model and the attack reference model. To enhance the generality, our experiments ensure there is no data contamination between the training data of the target, reference, and attack models.
    \[f_\text{Ref}(x) = \mathcal{L}_\text{CE}(\theta; x) - \mathcal{L}_\text{CE}(\theta_\text{attack}; x) \]
$\bullet$ Min-K~\cite{shi2024detecting} leverages top K tokens that have the lowest loss values.
    \[f_\text{min-K}(x) = \frac{1}{|\text{min-K(x)}|} \sum_{t_i \in \text{min-K(x)}} -\log(P(t_i|t_{<i};\theta) \]
$\bullet$ Zlib~\cite{Carlini2020ExtractingTD} calibrates the loss signal with the zlib compression size.
    \[ f_\text{zlib}(x) = \mathcal{L}_\text{CE}(\theta; x) / \text{zlib}(x) \]

\noindent \textbf{Baselines}. We present the results of four baselines. \textit{Base} refers to the pretrained LLM without fine tuning. \textit{FT} represents the standard causal language modeling without protection. \textit{Goldfish}~\cite{hans2024be} implements a masking mechanism. \textit{DPSGD}~\cite{abadi2016deep, yu2022differentially} applies gradient clipping and injects noise to achieve  sample-level differential privacy.

\noindent \textbf{Implementation}. We conduct full fine-tuning for \gpt and \pythia. For computing efficiency, \llama fine-tuning is implemented using Low-Rank Adaptation (LoRA)~\cite{hu2022lora} which leads to \textasciitilde4.2M trainable parameters. Additionally, we use subsets of 3K samples to fine-tune the LLMs. We present other implementation details in Appendix~\ref{sec:app-implementation}.

\subsection{Overall Evaluation}
Table~\ref{tab:main_result} provides the overall evaluation compared to several baselines across large language model architectures and datasets. Among these two datasets, CCNews is more challenging, which  leads to higher perplexity  for all LLMs and fine-tuning methods. Additionally, the reference-model-based attack performs the best and demonstrates high privacy risks with attack AUC on the conventional fine-tuned models at 0.95 and 0.85 for Wikipedia and CCNews, respectively. Goldfish achieves similar PPL to the conventional FT method but fails to defend against MIAs. This aligns with the reported results by \citet{hans2024be} that Goldfish resists exact match attacks but only marginally affects MIAs. DPSGD provides a very strong protection in all settings (AUC < 0.55) but with a significant PPL tradeoff. Our proposed \methodname guarantees a robust protection, even slightly better than DPSGD, but with a notably smaller tradeoff on language modeling performance. For example, on the Wikipedia dataset, \methodname delivers perplexity reduction by 15\% to 27\%. Moreover, Table~\ref{tab:tpr} (Appendix~\ref{sec:app-add-res}) provides the TPR at 1\% FPR. Both DPSGD and \methodname successfully reduce the TPR to $\sim$0.02 for all LLMs and datasets. \textit{Overall, across multiple LLM architectures and datasets, \methodname consistently offers ideal privacy protection with  little trade-off in language modeling performance.}

\noindent \textbf{Privacy-Utility Trade-off.}
To investigate the privacy-utility trade-off of the methods, we vary the hyper-parameters of the fine-tuning methods. Particularly, for DPSGD, we adjust the privacy budget $\epsilon$ from (8, 1e-5)-DP to (100, 1e-5)-DP. We modify the masking percentage of Goldfish from 25\% to 50\%. Additionally, we vary the loss weight $\alpha$ from 0.2 to 0.8 for \methodname. Figure~\ref{fig:priv-ult-tradeoff} depicts the privacy-utility trade-off for GPT2 on the CCNews dataset. Goldfish, with very large masking rate (50\%), can slightly reduce the risk of the reference attack but can increase the risks of other attacks. By varying the weight $\alpha$, \methodname offers an adjustable trade-off between privacy protection and language modeling performance. \methodname largely dominates DPSGD and improves the language modeling performance by around 10\% with the ideal privacy protection against MIAs.

\begin{figure}[h]
    \centering
    \includegraphics[width=\linewidth]{figs/privacy-ultility-tradeoff.pdf}
    \caption{Privacy-utility trade-off of the methods while varying hyper-parameters. The Goldfish masking rate is set to 25\%, 33\%, and 50\%. The privacy budget $\epsilon$ of DPSGD is evaluated at 8, 16, 50, and 100. The weight $\alpha$ of \methodname is configured at 0.2, 0.5, and 0.8.}
    \label{fig:priv-ult-tradeoff}
\end{figure}


\subsection{Ablation Study}
\textbf{\methodname without reference models.} To study the impact of the reference model, we adapt \methodname to a non-reference version which directly uses the loss of the current training model (i.e., $s(t_i) = \mathcal{L}_{CE}(\theta; t_i)$) to select the learning and unlearning tokens. This means the unlearning tokens are the tokens that have smallest loss values. Figure~\ref{fig:ppl-auc-noref} presents the training loss and testing perplexity. There is an inconsistent trend of the training loss and testing perplexity. Although the training loss decreases overtime, the test perplexity increases. This result indicates that identifying appropriate unlearning tokens  without a reference model is challenging and conducting unlearning on an incorrect set hurts the language modeling performance.

\begin{figure}[htp]
    \centering
    \includegraphics[width=0.35\textwidth]{figs/train_loss_ppl_noref.pdf}
    \caption{Training Loss and Test Perplexity of \methodname without a reference model.
    % (\lrx{If time permits, it would be better to compare with our training curve here)}
    }
    \label{fig:ppl-auc-noref}
\end{figure}

\noindent \textbf{\methodname with out-of-domain reference models.} To examine the influence of the distribution gap in the reference model, we replace the in-domain trained reference model with the original pretrained base model. 
Figure~\ref{fig:ppl-auc-base-woasc} depicts the language modeling performance and privacy risks in this study. \methodname with an out-of-domain reference model can reduce the privacy risks but yield a significant gap in language modeling performance compared to \methodname using an in-domain reference model.

\noindent \textbf{\methodname without Unlearning.} To study the effects of unlearning tokens, we implement \methodname which use the first term of the loss only ({$\mathcal{L}_{\theta} = \mathcal{L}_{CE}(\theta; \mathcal{T}_h)$}). Figure~\ref{fig:ppl-auc-base-woasc} provides the perplexity and MIA AUC scores in this setting. Generally, without gradient ascent, \methodname can marginally reduce membership inference risks while slightly improving the language modeling performance. The token selection serves as a regularizer that helps to improve the language modeling performance. Additionally, tokens that are learned well in previous epochs may not be selected in the next epochs. This slightly helps to not amplify the memorization on these tokens over epochs.

\begin{figure}[htp]
    \centering
    \includegraphics[width=0.28\textwidth]{figs/auc_vs_ppl_base_woasc.pdf}
    \caption{Privacy-utility trade-off of \methodname with different settings: in-domain reference model, out-domain reference model, and without unlearning}
    \label{fig:ppl-auc-base-woasc}
\end{figure}


\subsection{Training Dynamics}
\textbf{Memorization and Generalization Dynamics}. Figure~\ref{fig:training-dynamics} (left) illustrates the training dynamics of conventional fine tuning and \methodname, while Figure~\ref{fig:training-dynamics} (middle) depicts the membership inference risks. Generally, the gap between training and testing loss of conventional fine-tuning steadily increases overtime, leading to model overfitting and high privacy risks. In contrast, \methodname maintains a stable equilibrium where the gap remains more than 10 times smaller. This equilibrium arises from the dual-purpose loss, which balances learning on hard tokens while actively unlearning memorized tokens. By preventing excessive memorization, \methodname mitigates membership inference risks and enhances generalization.

\begin{figure*}[htp]
    \centering
    \includegraphics[width=0.29\linewidth]{figs/loss_vs_steps_ft_duolearn.pdf}
    \includegraphics[width=0.29\linewidth]{figs/auc_vs_steps_ft_duolearn.pdf}
    \includegraphics[width=0.316\linewidth]{figs/cosine.pdf}
    \caption{Training dynamics of \methodname and the conventional fine-tuning approach. The left and middle figures provide the training-testing gap and membership inference risks, respectively. The testing~$\mathcal{L}_{CE}$ of FT and training~$\mathcal{L}_{CE}$ of \methodname are significantly overlapping, we provide the breakdown in Figure~\ref{fig:add-overlap-breakdown} in Appendix~\ref{sec:app-add-res}. The right figure depicts the cosine similarity of the learning and unlearning gradients of \methodname. Cosine similarity of 1 means entire alignment, 0 indicates orthogonality, and -1 presents full conflict.}
    \label{fig:training-dynamics}
\end{figure*}

\noindent \textbf{Gradient Conflicts}. To study the conflict between the learning and unlearning objectives in our dual-purpose loss function, we compute the gradient for each objective separately. We then calculate the cosine similarity of these two gradients. Figure~\ref{fig:training-dynamics} (right) provides the cosine similarity between two gradients over time. During training, the cosine similarity typically ranges from -0.15 to 0.15. This indicates a mix of mild conflicts and near-orthogonal updates. On average, it decreases from 0.05 to -0.1. This trend reflects increasing gradient misalignment. Early in training, the model may not have strongly learned or memorized specific tokens, so the conflicts are weaker. Overtime, as the model learns more and memorization grows, the divergence between hard and memorized tokens increases, making the gradients less aligned. This gradient conflict is the root of the small degradation of language modeling performance of \methodname compared to the conventional fine tuning approach.

\noindent \textbf{Token Selection Dynamics}. Figure~\ref{fig:token-selection} illustrates the token selection dynamics of \methodname during training. The figure shows that the token selection process is dynamic and changes over epochs. In particular, some tokens are selected as an unlearning from the beginning to the end of the training. This indicates that a token, even without being selected as a learning token initially, can be learned and memorized through the connections with other tokens. This also confirms that simple masking as in Goldfish is not sufficient to protect against MIAs. Additionally, there are a significant number of tokens that are selected for learning in the early epochs but unlearned in the later epochs. This indicates that the model learned tokens and then memorized them over epochs, and the during-training unlearning process is essential to mitigate the memorization risks.

\begin{figure}[htp]
    \centering
    \includegraphics[width=0.7\linewidth]{figs/token-selection-dynamics.pdf}
    \caption{Token Selection Dynamics of \methodname}
    \label{fig:token-selection}
    \vspace{-4mm}
\end{figure}

\subsection{Privacy Backdoor}
To study the worst case of privacy attacks and defense effectiveness under the state-of-the-art MIA, we perform a privacy backdoor -- Precurious~\cite{precurious}. In this setup, the target model undergoes continual fine-tuning from a warm-up model. The attacker then applies a reference-based MIA that leverages the warm-up model as the attack's reference. Table~\ref{tab:backdoor} shows the language modeling and MIA performance on CCNews with GPT-2. Precurious increases the MIA AUC score by 5\%. Goldfish achieves the lowest PPL, aligning with~\citet{hans2024be}, where the Goldfish masking mechanism acts as a regularizer that potentially enhances generalization. Both DPSGD and \methodname provide strong privacy protection, with \methodname offering slightly better defense while maintaining lower perplexity than DPSGD.

% \begin{table}[h]
%     \centering
%     \begin{tabular}{c|cc|cc}
%        \multirow{2}{*}{\textbf{Method}}  & \multicolumn{2}{c}{\textbf{CCNews}} & \multicolumn{2}{c}{\textbf{Wikipedia}} \\ 
%        & \textbf{PPL} & \textbf{AUC} & \textbf{PPL} & \textbf{AUC} \\ \hline
%        \textbf{FT}        & 21.593 & 0.911 \\
%        \textbf{Goldfish}  & \textbf{21.074} & 0.886 \\
%        \textbf{DPSGD}     & 23.279 & 0.533 \\
%        \textbf{DuoLearn}  & 22.296 & \textbf{0.499} \\
%     \end{tabular}
%     \caption{Caption}
%     \label{tab:my_label}
% \end{table}

\begin{table}[h]
    \centering
    \resizebox{\columnwidth}{!}{\begin{tabular}{c|cccccc}
        \textbf{Metric} & \textbf{WU} & \textbf{FT} & \textbf{GF} & \textbf{DP} & \textbf{DuoL} \\ \hline
        \textbf{PPL} & \textit{23.318} & 21.593 & \textbf{21.074} & 23.279 & 22.296  \\
        \textbf{AUC} & \textit{0.500} & 0.911 & 0.886 & 0.533 & \textbf{0.499} \\
    \end{tabular}}
    \caption{Experimental results of privacy backdoor for GPT2 on the CC-news dataset. WU stands for the warm-up model leveraged by Precurious. GF, DP, and DuoL are abbreviations of Goldfish, DPSGD, and \methodname}
    \label{tab:backdoor}
\end{table}

% \subsubsection{Hyperparameter Study}

% \subsubsection{Full fine-tuning versus Parameter efficent fine tuning}

% \subsubsection{Extending to Vision Language Models}



\section{Ablation Analysis}\label{sec:ablation}

In this section, we study the impact of different parallel techniques on performance.

\subsection{Experiment Setting}

We conducted experiments using the Llama-2 model in three sizes: 7B, 13B, and 70B, to evaluate the impact of various parallelism strategies and memory management techniques. The experiments were executed on GPU clusters composed of NVIDIA A800 GPUs, with configurations ranging from 64 to 4096 GPUs. We employed a hybrid parallelism strategy (data parallelism + tensor parallelism) unless otherwise specified. Four key techniques were analyzed: parallelism strategies, system scale, memory offloading, recomputation, and overlap strategies.

We tested the following setups for each technique:
\begin{itemize}[noitemsep,topsep=1pt, leftmargin=*]
    \item \textbf{Parallelism Strategy}: We tested data parallelism (DP), tensor parallelism (TP), pipeline parallelism (PP), and hybrid parallelism strategies across different GPU counts and model sizes.
    \item \textbf{System Scale}: We varied the number of GPUs (64, 128, 256, 1024, and 4096) while keeping the model architecture fixed to isolate the impact of system scale on training efficiency.
    \item \textbf{Memory Offloading}: We tested configurations with no offloading, enable offloading, and memory bandwidth variations (DDR4 vs. DDR5).
    %\item \textbf{Recomputation}: We tested no recomputation, selective recomputation (recomputing specific layers), full recomputation, and hybrid recomputation.
    \item \textbf{Overlap Strategy}: We tested no overlap, gradient communication overlap, parameter gather overlap, and combined overlap strategies.
\end{itemize}

\begin{figure*}[htbp]
  \centering
    \subfloat[Llama-2-7B]{\includegraphics[width=0.33\textwidth]{figs/fig-dp-Llama-2-7B.pdf}}
    %\\
    \subfloat[Llama-2-13B]{\includegraphics[width=0.33\textwidth]{figs/fig-dp-Llama-2-13B.pdf}}
    %\\
    \subfloat[Llama-3-8B]{\includegraphics[width=0.33\textwidth]{figs/fig-dp-Llama-3-8B.pdf}}
  \caption{
  We compare \sysname's performance with all parallelism methods allowed and only data parallelism allowed.
  }
  \label{fig:ablation:dp}
\end{figure*}

\begin{figure*}[htbp]
  \centering
    \subfloat[Llama-2]{\includegraphics[width=0.33\textwidth]{figs/fig-scale-Llama-2.pdf}}
    %\\
    \subfloat[Llama-3]{\includegraphics[width=0.33\textwidth]{figs/fig-scale-Llama-3.pdf}}
    %\\
    \subfloat[GLM]{\includegraphics[width=0.33\textwidth]{figs/fig-scale-GLM.pdf}}
  \caption{
  We understand the system scale impact on training efficiency by training \sysname\ on different GPU numbers.
  }
  \label{fig:ablation:scale}
\end{figure*}

\sssec{Metrics}
We utilized three key metrics:
\begin{itemize}[noitemsep,topsep=1pt, leftmargin=*]
    \item \textbf{Training Throughput (samples/second)}: The primary metric for evaluating training efficiency, measuring the number of samples processed per second.
    \item \textbf{Scaling Efficiency (\%)}: This metric compares the achieved throughput at each scale to the ideal linear scaling expected as the number of GPUs increases.
    \item \textbf{Communication Overhead}: We measured the data transferred between GPUs to quantify the additional cost incurred by increasing the system size, focusing on inter-GPU and inter-node communication costs.
\end{itemize}

\sssec{Statistical Validation}
All experiments were run for a fixed number of iterations to ensure that the strategies had sufficient time to reach stable performance metrics. Each experiment was repeated three times to account for variability, and the results are reported as averages with standard deviations.

\subsection{Parallelism Strategy In-Depth Analysis}

\sssec{Method}. We analyzed the impact of different parallelism strategies (DP, TP, PP, and hybrid) on training performance across different GPU counts and model sizes. We monitored trade-offs between communication overhead and computation efficiency and identified cases where hybrid parallelism yielded improvements over single-strategy configurations.

\sssec{Results}.
Figure \ref{fig:ablation:dp} compares \sysname’s performance with all parallelism methods enabled against data parallelism (DP) alone across different system scales for Llama-2-7B, Llama-2-13B, and Llama-3-8B models. In all cases, \sysname consistently outperforms DP as the GPU count increases. The results show that \sysname maintains higher throughput by combining multiple parallelism strategies, effectively mitigating the communication overhead that causes DP's performance to degrade at larger scales. This demonstrates \sysname's superior scalability and adaptability, highlighting the benefits of hybrid parallelism over relying solely on data parallelism.

\subsection{System Scale Impact on Training Efficiency}

\sssec{Method}.
We explored the relationship between system scale (i.e., the number of GPUs) and training efficiency by analyzing changes in training throughput and communication overhead as the number of GPUs increased while keeping the model architecture fixed.

\sssec{Results}.
The results presented in Figure \ref{fig:ablation:scale} demonstrate a clear trend of diminishing per-GPU throughput as GPUs increase for each model (Llama-2, Llama-3, and GLM). For smaller models such as Llama-2 7B and 13B, the decrease in throughput per GPU is gradual, indicating that these models scale more efficiently with the number of GPUs. However, for larger models such as Llama-2 70B and GLM 130B, the throughput drops significantly as the GPU count exceeds 512, suggesting that these models experience higher communication overhead and resource contention at larger scales. 
This decline in efficiency highlights the limitations of scaling large models across many GPUs. As the system grows, the communication cost between GPUs begins to outweigh the computational benefits, leading to suboptimal utilization of the hardware resources. The sharp decrease in throughput for models like Llama-2 70B and GLM 130B at 1024 GPUs underscores the need for careful consideration of parallelization strategies to mitigate the impact of inter-GPU communication on training performance. These findings are consistent with previous observations in distributed training, where scaling efficiency degrades as the system size increases due to increased synchronization and data transfer costs.

\begin{figure}[htbp]
  \centering
    \subfloat{\includegraphics[width=0.4\textwidth]{figs/fig-offload-legend.pdf}}\\
    \addtocounter{subfigure}{-1}
    
    \subfloat[Llama-2-7B]{\includegraphics[width=0.16\textwidth]{figs/fig-offload-Llama-2-7B.pdf}}
    \subfloat[Llama-2-13B]{\includegraphics[width=0.16\textwidth]{figs/fig-offload-Llama-2-13B.pdf}}
    \subfloat[Llama-2-70B]{\includegraphics[width=0.16\textwidth]{figs/fig-offload-Llama-2-70B.pdf}}
    \\
    \subfloat[Llama-3-8B]{\includegraphics[width=0.24\textwidth]{figs/fig-offload-Llama-3-8B.pdf}}
    \subfloat[Llama-3-70B]{\includegraphics[width=0.24\textwidth]{figs/fig-offload-Llama-3-70B.pdf}}
    \\
    \subfloat[GLM-67B]{\includegraphics[width=0.24\textwidth]{figs/fig-offload-Llama-GLM-67B.pdf}}
    \subfloat[GLM-130B]{\includegraphics[width=0.24\textwidth]{figs/fig-offload-Llama-GLM-130B.pdf}}
  \caption{
  We compare \sysname's performance with offload allowed with unallowed
  }
  \label{fig:ablation:offload}
\end{figure}

\subsection{Memory Offloading Technique Analysis}

\sssec{Method}. We evaluated the effectiveness of memory offloading techniques, comparing disable/enable offloading, and the impact of different memory bandwidths on training performance, particularly in memory-constrained environments.

\sssec{Results}.
Figure \ref{fig:ablation:offload} compares \sysname's performance with and without memory offloading across different models and system scales. The results show that memory offloading becomes increasingly important as model size grow. For smaller models like Llama-2-7B and Llama-2-13B, the performance impact of offloading is minimal, but as models scale up (e.g., Llama-70B and GLM-130B), enabling offloading significantly improves throughput by alleviating memory bottlenecks. Without offloading, larger models experience sharp performance declines as GPU count increases. These findings highlight the critical role of memory offloading in maintaining efficient scaling for larger models across large GPU configurations.

%\subsection{Recomputation Technique Analysis}

%We analyzed the effects of recomputation techniques on training efficiency, focusing on trade-offs between memory savings and computational overhead across different recomputation configurations: no recomputation, selective recomputation, full recomputation, and hybrid recomputation.

%\sssec{Results}
%\textcolor{red}{@Haibin, please fill in.}

\begin{figure}[htbp]
  \centering
    \subfloat{\includegraphics[width=0.4\textwidth]{figs/fig-overlap-legend.pdf}}\\
    \addtocounter{subfigure}{-1}
    
    \subfloat[Llama-2-7B]{\includegraphics[width=0.16\textwidth]{figs/fig-overlap-Llama-2-7B.pdf}}
    \subfloat[Llama-2-13B]{\includegraphics[width=0.16\textwidth]{figs/fig-overlap-Llama-2-13B.pdf}}
    \subfloat[Llama-2-70B]{\includegraphics[width=0.16\textwidth]{figs/fig-overlap-Llama-2-70B.pdf}}
    \\
    \subfloat[Llama-3-8B]{\includegraphics[width=0.24\textwidth]{figs/fig-overlap-Llama-3-8B.pdf}}
    \subfloat[Llama-3-70B]{\includegraphics[width=0.24\textwidth]{figs/fig-overlap-Llama-3-70B.pdf}}
    \\
    \subfloat[GLM-67B]{\includegraphics[width=0.24\textwidth]{figs/fig-overlap-Llama-GLM-67B.pdf}}
    \subfloat[GLM-130B]{\includegraphics[width=0.24\textwidth]{figs/fig-overlap-Llama-GLM-130B.pdf}}
  \caption{
  We compare \sysname's performance with communication overlap allowed with unallowed
  }
  \label{fig:ablation:overlap}
\end{figure}

\subsection{Overlap Strategy Technique Analysis}

\sssec{Method}. We examined the effectiveness of overlapping communication with computation, comparing no overlap, gradient communication overlap, parameter gather overlap, and combined overlap strategies.

\sssec{Results}.
Figure \ref{fig:ablation:overlap} compares \sysname’s performance with and without communication overlap across different models and GPU scales. The results show that enabling communication overlap improves throughput, especially for larger models and higher GPU counts. For smaller models like Llama-2-7B and Llama-2-13B, the benefits are modest but noticeable, while for larger models such as Llama-2-70B and GLM-130B, overlap significantly reduces communication delays, resulting in better throughput. This highlights the importance of overlap strategies in optimizing performance and scalability, particularly for large-scale models where communication overhead becomes a bottleneck.
\section{Discussion}
The development of foundation models has increasingly relied on accessible data support to address complex tasks~\cite{zhang2024data}. Yet major challenges remain in collecting scalable clinical data in the healthcare system, such as data silos and privacy concerns. To overcome these challenges, MedForge integrates multi-center clinical knowledge sources into a cohesive medical foundation model via a collaborative scheme. MedForge offers a collaborative path to asynchronously integrate multi-center knowledge while maintaining strong flexibility for individual contributors.
This key design allows a cost-effective collaboration among clinical centers to build comprehensive medical models, enhancing private resource utilization across healthcare systems.

Inspired by collaborative open-source software development~\cite{raffel2023building, github}, our study allows individual clinical institutions to independently develop branch modules with their data locally. These branch modules are asynchronously integrated into a comprehensive model without the need to share original data, avoiding potential patient raw data leakage. Conceptually similar to the open-source collaborative system, iterative module merging development ensures the aggregation of model knowledge over time while incorporating diverse data insights from distributed institutions. In particular, this asynchronous scheme alleviates the demand for all users to synchronize module updates as required by conventional methods (e.g., LoRAHub~\cite{huang2023lorahub}).


MedForge's framework addresses multiple data challenges in the cycle of medical foundation model development, including data storage, transmission, and leakage. As the data collection process requires a large amount of distributed data, we show that dataset distillation contributes greatly to reducing data storage capacity. In MedForge, individual contributors can simply upload a lightweight version of the dataset to the central model developer. As a result, the remarkable reduction in data volume (e.g., 175 times less in LC25000) alleviates the burden of data transfer among multiple medical centers. For example, we distilled a 10,500 image training set into 60 representative distilled data while maintaining a strong model performance. We choose to use a lightweight dataset as a transformed representation of raw data to avoid the leakage of sensitive raw information.
Second, the asynchronous collaboration mode in MedForge allows flexible model merging, particularly for users from various local medical centers to participate in model knowledge integration. 
Third, MedForge reformulates the conventional top-down workflow of building foundational models by adopting a bottom-up approach. Instead of solely relying on upstream builders to predefine model functionalities, MedForge allows medical centers to actively contribute to model knowledge integration by providing plugin modules (i.e., LoRA) and distilled datasets. This approach supports flexible knowledge integration and allows models to be applicable to wide-ranging clinical tasks, addressing the key limitation of fixed functionalities in traditional workflows.

We demonstrate the strong capacity of MedForge via the asynchronous merging of three image classification tasks. MedForge offered an incremental merging strategy that is highly flexible compared to plain parameter average~\cite{wortsman2022model} and LoRAHub~\cite{huang2023lorahub}. Specifically, plain parameter averaging merges module parameters directly and ignores the contribution differences of each module. Although LoRAHub allows for flexible distribution of coefficients among modules, it lacks the ability to continuously update, limiting its capacity to incorporate new knowledge during the merging process. In contrast, MedForge shows its strong flexibility of continuous updates while considering the contribution differences among center modules. The robustness of MedForge has been demonstrated by shuffling merging order (Tab~\ref{tab:order}), which shows that merging new-coming modules will not hurt the model ability of previous tasks in various orders, mitigating the model catastrophic forgetting. 
MedForge also reveals a strong generality on various choices of component modules. Our experiments on dataset distillation settings (such as DC and without DSA technique) and PEFT techniques (such as DoRA) emphasize the extensible ability of MedForge's module settings. 

To fully exploit multi-scale clinical data, it will be necessary to include broader data modalities (e.g., electronic health records and radiological images). Managing these diverse data formats and standards among numerous contributors can be challenging due to the potential conflict between collaborators. 
Moreover, since MedForge integrates multiple clinical tasks that involve varying numbers of classification categories, conventional classifier heads with fixed class sizes are not applicable. However, the projection head of the CLIP model, designed to calculate similarities between image and text, is well-suited for this scenario. It allows MedForge to flexibly handle medical datasets with different category numbers, thus overcoming the challenge of multi-task classification. That said, this design choice also limits the variety of model architectures that can be utilized, as it depends specifically on the CLIP framework. Future investigations will explore extensive solutions to make the overall architecture more flexible. Additionally, incorporating more sophisticated data anonymization, such as synthetic data generation~\cite{ding2023large}, and encryption methods can also be considerable. To improve data privacy protection, test-time adaptation technique~\cite{wang2020tent, liang2024comprehensive} without substantial training data can be considered to alleviate the burden of data sharing in the healthcare system.



             

\section{Conclusion}
We reveal a tradeoff in robust watermarks: Improved redundancy of watermark information enhances robustness, but increased redundancy raises the risk of watermark leakage. We propose DAPAO attack, a framework that requires only one image for watermark extraction, effectively achieving both watermark removal and spoofing attacks against cutting-edge robust watermarking methods. Our attack reaches an average success rate of 87\% in detection evasion (about 60\% higher than existing evasion attacks) and an average success rate of 85\% in forgery (approximately 51\% higher than current forgery studies). 


%%%%%%%%% REFERENCES
\bibliography{egbib}

\end{document}
