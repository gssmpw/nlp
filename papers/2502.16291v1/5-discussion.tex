\section{Opportunities}
\label{sec:oppotunities}

% Include brainwriting, luminate, and sensecape as tools that have opportunities to be used in research even though they are not necessarily research-driven tools

GenAI-powered research tools present a significant opportunity to design truly mixed-initiative systems that put cognitive engagement as a key design consideration. These systems can balance user agency and control with the generative capabilities of LLMs, thereby encouraging both divergent ideation and convergent refinement while ensuring that researchers remain at the core of the inquiry process. Particularly, designers should remain mindful of automation bias, where users may become overly reliant on AI-generated outputs. 

In designing GenAI-powered research tools, it is crucial for designers to actively consider several interconnected dimensions: supporting user agency, encouraging both creative exploration and systematic refinement, and ensuring adaptability and precision. Instead of anticipating users to stay engaged, these deliberate design choices empower researchers to interact critically with AI outputs, reducing risks such as automation bias. This approach invites users to actively shape their interaction, paving the way for a more innovative and rigorous future in scientific discovery.