\section{Method}

Our literature search focused on AI-assisted creative research tools in contrast to AI-assisted writing tools, creative research tools assist with the co-creation of concepts and ideas in the research process rather than merely improving stylistic or rhetorical choices in written research. In total, we surveyed 11 systems papers published in top HCI venues (i.e., CHI, CSCW, UIST, and ToCHI) over the last three years (2022-2024)\footnote{We include a relevant CHI 2025 paper made available on arXiv.}; details can be found in Table~\ref{tab:tool_classification}. 

As LLMs became widely used in 2022 with the release of ChatGPT \cite{openai2022chatgpt}, this timeframe was chosen to reflect the period of significant growth in LLM popularity and adoption, allowing us to capture the most relevant and impactful developments in GenAI-driven research tools. Of the systems surveyed, six of the systems integrate LLM-based functionalities, while the other five represent a more traditional AI approach and employ machine learning techniques (e.g., Seq2Seq, BERT, RNN). By examining both GenAI and traditional AI approaches, we aim to understand to what extent GenAI tools represent a fundamental shift in capabilities and design considerations compared to more established AI approaches.

The thematic dimensions presented in Section~\ref{sec:design-space} resulted from an iterative process among the authors. We engaged in extensive internal discussions and consulted with an external expert specializing in knowledge spaces and the role of AI in fostering creativity and sensemaking. This collaborative and iterative approach resulted in the four dimensions presented in the next section.