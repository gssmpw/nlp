Generative AI (GenAI) tools are radically expanding the scope and capability of automation in knowledge work such as academic research. AI-assisted research tools show promise for augmenting human cognition and streamlining research processes, but could potentially increase automation bias and stifle critical thinking. We surveyed the past three years of publications from leading HCI venues. We closely examined 11 AI-assisted research tools, five employing traditional AI approaches and six integrating GenAI, to explore how these systems envision novel capabilities and design spaces. We consolidate four design recommendations that inform cognitive engagement when working with an AI research tool: Providing user agency and control; enabling divergent and convergent thinking; supporting adaptability and flexibility; and ensuring transparency and accuracy. We discuss how these ideas mark a shift in AI-assisted research tools from mimicking a researcher's established workflows to generative co-creation with the researcher and the opportunities this shift affords the research community.