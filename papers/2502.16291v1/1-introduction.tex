\section{Introduction}
Generative AI (GenAI) is transforming the landscape of research and creative workflows by radically expanding the scope and capabilities of automation. Recent advancements in large language models (LLMs) are redefining the ways in which knowledge workers -- particularly researchers -- engage in the process of ideation, sensemaking, and scientific creativity. At the same time, the integration of GenAI into research workflows raises critical questions about its effects on human cognition. Although these systems promise to augment human intelligence and streamline research processes, there is mounting evidence that overreliance on automated output can lead to reductions in critical thinking and increased automation bias~\cite{lee2025impact,passi2022overreliance}. Furthermore, concerns have been raised about the potential of GenAI to steer users too heavily, thus diminishing human capacity for interpretation and reflection~\cite{10.1145/3544548.3581066}. 

Latest developments in the AI industry further emphasize these trends: OpenAI's release of Deep Research~\cite{openai2025deepresearch}, an AI agent capable of independently synthesizing information from hundreds of online sources to generate comprehensive research reports, highlights the rapid push toward automation of knowledge work. Soon after, Google unveiled its AI co-scientist~\cite{google2025aicoScientist}, designed as a multi-agent system to accelerate scientific breakthroughs by augmenting human ingenuity with iterative hypothesis refinement, illustrates the expanding role of AI in shaping scientific discovery.

In response to these challenges, a human‐centered approach to AI~\cite{shneiderman2020human} offers a promising alternative to preserve and improve cognitive engagement by having machines provide computational power while humans guide, interpret, and refine outputs.

To investigate these dynamics, we surveyed a collection of the latest AI-powered research tools. We identified four dimensions to examine the level of cognitive engagement provided by AI-powered research tools: User agency and control (Section ~\ref{sec:agency}); divergent and convergent thinking (Section ~\ref{sec:divergent-convergent}); adaptability (Section ~\ref{sec:adaptability}); and accuracy (Section ~\ref{sec:accuracy}). By analyzing existing tools through this framework, we identify a space of design choices and potential pitfalls. We highlight a widening gap between GenAI tools, which often guide users toward discovery; in contrast with traditional AI tools, which tend to mimic user workflows and provide greater user autonomy (Section~\ref{sec:genai-vs-ai}). Awareness of this distinction opens opportunities for creating ``tools for thought'' (Section~\ref{sec:oppotunities}) that truly empower researchers, fostering deep cognitive engagement and maximizing the benefits of GenAI while mitigating its risks.
% \begin{enumerate}
%     \item User Agency \& Control: The degree to which researchers can intervene, override, or refine AI-generated output; assessing whether the tools empower users to maintain decision-making authority throughout the research process. (Section~\ref{sec:agency})
%     \item Divergent \& Convergent Thinking: The balance between expanding the research space (divergent thinking) and narrowing down ideas through critical evaluation and selection (convergent thinking). We explore how tools facilitate both the generation of novel ideas and the systematic refinement of research directions. (Section~\ref{sec:divergent-convergent})
%     \item Adaptability: The extent to which a tool supports a diversity of research tasks, workflows, and individual user preferences. This dimension considers the flexibility of AI systems to accommodate diverse approaches to ideation and sensemaking. (Section~\ref{sec:adaptability})
%     \item Accuracy: The system's capability to deliver reliable, unbiased, and contextually relevant information. Ensuring accuracy is fundamental for building trust and enabling effective human-AI collaboration. (Section~\ref{sec:accuracy})
% \end{enumerate}