%%
%% This is file `sample-sigconf-authordraft.tex',
%% generated with the docstrip utility.
%%
%% The original source files were:
%%
%% samples.dtx  (with options: `all,proceedings,bibtex,authordraft')
%% 
%% IMPORTANT NOTICE:
%% 
%% For the copyright see the source file.
%% 
%% Any modified versions of this file must be renamed
%% with new filenames distinct from sample-sigconf-authordraft.tex.
%% 
%% For distribution of the original source see the terms
%% for copying and modification in the file samples.dtx.
%% 
%% This generated file may be distributed as long as the
%% original source files, as listed above, are part of the
%% same distribution. (The sources need not necessarily be
%% in the same archive or directory.)
%%
%%
%% Commands for TeXCount
%TC:macro \cite [option:text,text]
%TC:macro \citep [option:text,text]
%TC:macro \citet [option:text,text]
%TC:envir table 0 1
%TC:envir table* 0 1
%TC:envir tabular [ignore] word
%TC:envir displaymath 0 word
%TC:envir math 0 word
%TC:envir comment 0 0
%%
%% The first command in your LaTeX source must be the \documentclass
%% command.
%%
%% For submission and review of your manuscript please change the
%% command to \documentclass[manuscript, screen, review]{acmart}.
%%
%% When submitting camera ready or to TAPS, please change the command
%% to \documentclass[sigconf]{acmart} or whichever template is required
%% for your publication.
%%
%%
\documentclass[sigconf, screen, nonacm]{acmart}
%%
%% \BibTeX command to typeset BibTeX logo in the docs
\AtBeginDocument{%
  \providecommand\BibTeX{{%
    Bib\TeX}}}

%% Rights management information.  This information is sent to you
%% when you complete the rights form.  These commands have SAMPLE
%% values in them; it is your responsibility as an author to replace
%% the commands and values with those provided to you when you
%% complete the rights form.
\setcopyright{acmlicensed}
\copyrightyear{2018}
\acmYear{2018}
\acmDOI{XXXXXXX.XXXXXXX}
%% These commands are for a PROCEEDINGS abstract or paper.
\acmConference[Conference acronym 'XX]{Make sure to enter the correct
  conference title from your rights confirmation email}{June 03--05,
  2018}{Woodstock, NY}
%%
%%  Uncomment \acmBooktitle if the title of the proceedings is different
%%  from ``Proceedings of ...''!
%%
%%\acmBooktitle{Woodstock '18: ACM Symposium on Neural Gaze Detection,
%%  June 03--05, 2018, Woodstock, NY}
\acmISBN{978-1-4503-XXXX-X/2018/06}


%%
%% Submission ID.
%% Use this when submitting an article to a sponsored event. You'll
%% receive a unique submission ID from the organizers
%% of the event, and this ID should be used as the parameter to this command.
%%\acmSubmissionID{123-A56-BU3}

%%
%% For managing citations, it is recommended to use bibliography
%% files in BibTeX format.
%%
%% You can then either use BibTeX with the ACM-Reference-Format style,
%% or BibLaTeX with the acmnumeric or acmauthoryear sytles, that include
%% support for advanced citation of software artefact from the
%% biblatex-software package, also separately available on CTAN.
%%
%% Look at the sample-*-biblatex.tex files for templates showcasing
%% the biblatex styles.
%%

%%
%% The majority of ACM publications use numbered citations and
%% references.  The command \citestyle{authoryear} switches to the
%% "author year" style.
%%
%% If you are preparing content for an event
%% sponsored by ACM SIGGRAPH, you must use the "author year" style of
%% citations and references.
%% Uncommenting
%% the next command will enable that style.
%%\citestyle{acmauthoryear}

\usepackage{balance}
\usepackage{xcolor}
\usepackage{pgffor}
\usepackage[most]{tcolorbox} % For side text boxes
\usepackage{graphicx}       % For including images
\usepackage{booktabs}       % For better tables
\usepackage{amsmath}        % For mathematical equations
\usepackage{hyperref}       % For hyperlinks
\usepackage{tikz}
\usepackage{forest}

\newtcolorbox{designRecom}[2][]{%
    enhanced,
    breakable,
    colback=blue!3,          % Lighter blue background for better readability
    colframe=blue!40!black,  % Slightly lighter frame
    coltitle=white,          % Title text remains white for contrast
    colbacktitle=blue!30!black, % Lighter title background
    boxrule=0.6pt,           % Keep the box outline clear
    arc=3pt,                 % Rounded corners for a modern look
    fontupper=\small, % Maintain italicized content text
    left=8pt, right=8pt,      % Padding adjustments
    top=6pt, bottom=6pt,
    title=\textbf{Design Recommendation:}\\\textnormal{#2},
    #1
}


%%
%% end of the preamble, start of the body of the document source.
\begin{document}

%%
%% The "title" command has an optional parameter,
%% allowing the author to define a "short title" to be used in page headers.
\title{The Design Space of Recent AI-assisted Research Tools for Ideation, Sensemaking, and Scientific Creativity}

%%
%% The "author" command and its associated commands are used to define
%% the authors and their affiliations.
%% Of note is the shared affiliation of the first two authors, and the
%% "authornote" and "authornotemark" commands
%% used to denote shared contribution to the research.
\author{Runlong (Harry) Ye}
\orcid{0000-0003-1064-2333}
\email{harryye@cs.utoronto.ca}
\affiliation{
  \institution{Computer Science, \\University of Toronto}
  \city{Toronto}
  \state{Ontario}
  \country{Canada}
}

\author{Matthew Varona}
\orcid{0009-0005-6201-973X}
\email{varona@cs.toronto.edu}
\affiliation{
  \institution{Computer Science, \\University of Toronto}
  \city{Toronto}
  \state{Ontario}
  \country{Canada}
}

\author{Oliver Huang}
\orcid{0009-0007-1585-1229}
\email{oliver@cs.toronto.edu}
\affiliation{
  \institution{Computer Science, \\University of Toronto}
  \city{Toronto}
  \state{Ontario}
  \country{Canada}
}

\author{Patrick Yung Kang Lee}
\orcid{0000-0002-3385-5756}
\email{patricklee@cs.toronto.edu}
\affiliation{
  \institution{Computer Science, \\University of Toronto}
  \city{Toronto}
  \state{Ontario}
  \country{Canada}
}

\author{Michael Liut}
\orcid{0000-0003-2965-5302}
\email{michael.liut@utoronto.ca}
\affiliation{
  \institution{Mathematical and Computational Sciences, \\University of Toronto Mississauga}
  \city{Mississauga}
  \state{Ontario}
  \country{Canada}
}

\author{Carolina Nobre}
\orcid{0000-0002-2892-0509}
\email{cnobre@cs.toronto.edu}
\affiliation{
  \institution{Computer Science, \\University of Toronto}
  \city{Toronto}
  \state{Ontario}
  \country{Canada}
}

%%
%% By default, the full list of authors will be used in the page
%% headers. Often, this list is too long, and will overlap
%% other information printed in the page headers. This command allows
%% the author to define a more concise list
%% of authors' names for this purpose.
\renewcommand{\shortauthors}{Trovato et al.}

%%
%% The abstract is a short summary of the work to be presented in the
%% article.
\begin{abstract}
\begin{abstract}
Out-of-distribution (OOD) detection and OOD generalization are widely studied in Deep Neural Networks (DNNs), yet their relationship remains poorly understood. We empirically show that the degree of Neural Collapse (NC) in a network layer is inversely related with these objectives: stronger NC improves OOD detection but degrades generalization, while weaker NC enhances generalization at the cost of detection. This trade-off suggests that a single feature space cannot simultaneously achieve both tasks. To address this, we develop a theoretical framework linking NC to OOD detection and generalization. We show that entropy regularization mitigates NC to improve generalization, while a fixed Simplex Equiangular Tight Frame (ETF) projector enforces NC for better detection. Based on these insights, we propose a method to control NC at different DNN layers. In experiments, our method excels at both tasks across OOD datasets and DNN architectures. 

\begin{comment}   

Out-of-distribution (OOD) detection and OOD generalization are critical for deploying machine learning models in real-world scenarios. While substantial progress has been made in addressing these problems independently, few works have attempted to tackle them jointly. However, existing methods often rely on auxiliary OOD training data and primarily focus on covariate-shifted OOD data that share labels with in-distribution (ID) data. In contrast, we tackle the more realistic and challenging task of jointly detecting and generalizing to semantic OOD data with disjoint labels from the ID data, without auxiliary OOD training data.
Achieving both objectives simultaneously is inherently difficult due to a fundamental conflict — OOD generalization requires enhanced transferability, while OOD detection necessitates the inhibition of transfer.
To address this, we leverage insights from neural collapse (NC) — a phenomenon in deep networks where top-layer representations suppress feature variability and adopt a Simplex Equiangular Tight Frame (ETF) structure, impairing transferability. By controlling NC, we unify OOD detection and generalization: preventing NC enhances OOD transfer while inducing NC improves OOD detection.
Our proposed method excels at both tasks across various OOD datasets and architectures. 

\end{comment}


\end{abstract}
\end{abstract}

%%
%% The code below is generated by the tool at http://dl.acm.org/ccs.cfm.
%% Please copy and paste the code instead of the example below.
%%
\begin{CCSXML}
<ccs2012>
   <concept>
       <concept_id>10003120.10003121</concept_id>
       <concept_desc>Human-centered computing~Interactive systems and tools</concept_desc>
       <concept_significance>500</concept_significance>
   </concept>
   <concept>
       <concept_id>10003120.10003123</concept_id>
       <concept_desc>Human-centered computing~Interaction techniques</concept_desc>
       <concept_significance>500</concept_significance>
   </concept>
   <concept>
       <concept_id>10003120.10003122</concept_id>
       <concept_desc>Human-centered computing~Empirical studies in HCI</concept_desc>
       <concept_significance>500</concept_significance>
   </concept>
</ccs2012>
\end{CCSXML}

\ccsdesc[500]{Human-centered computing~Interactive systems and tools}
\ccsdesc[500]{Human-centered computing~Interaction techniques}
\ccsdesc[500]{Human-centered computing~Empirical studies in HCI}

%%
%% Keywords. The author(s) should pick words that accurately describe
%% the work being presented. Separate the keywords with commas.
\keywords{information seeking; multilevel exploration; sensemaking; levels of abstraction; abstraction hierarchy; large language models; systems thinking; human-AI interaction}
%% A "teaser" image appears between the author and affiliation
%% information and the body of the document, and typically spans the
%% page.


%%
%% This command processes the author and affiliation and title
%% information and builds the first part of the formatted document.
\maketitle

\section{Introduction}

% State of the world (robots for creative activites)
The term ``robot,'' originally signifying `forced labor,' has long been associated with labor and work. Robots have demonstrated their utility in various automated productive and social contexts, where the primary goals are improving productivity, safety, and fostering social interactions with humans~\cite{simoes2022designing, weidemann2021role, honig2018understanding}. However, an increasing number of cases feature using of robots in creative settings. Unlike productive contexts, where the focus is on efficiency and task completion~\cite{arents2022smart}, or social contexts, where communication and trust are prioritized~\cite{nam2020trust, saunderson2019robots}, creative environments prioritize artistic innovation and expression~\cite{hsueh2024counts}. This shift fundamentally alters the dynamics of human-robot interaction, redefining the roles and expectations for both humans and robots.

For instance, robots’ social behaviors are leveraged to support the generation and expression of creative ideas~\cite{hu2021exploring, sandoval2022human, alves2020creativity}, and programmable robotic movements and trajectories are employed to inspire artistic activities such as sketching~\cite{lin2020your}. These studies often engage participants from creative fields who possess limited prior experience with robotics, and are typically conducted in short-term, experimental settings. Consequently, the findings from these studies remain constrained since much can be learned from professional practitioners' experiences to inform system design such as digital fabrication~\cite{hirsch2023nothing}. There is a notable gap in research examining the long-term, active, and practical experience of integrating robotic systems into the creative processes. As a result, the deeper insights into how robots facilitate and shape creative processes, beyond simply augmenting human creativity, remain underexplored. In this study, we aim to better understand the impacts of robots on creative processes and outcomes.

As early as Leonardo da Vinci's 16th century ``Automaton,'' artists have explored the creative affordances of robotic systems~\cite{shanken2002cybernetics, pagliarini2009development, jeon2017robotic}. The artistic creation process typically encompasses various stages, including the exploration of materials and techniques, ongoing experimentation and iteration, and the continual refinement of the artists' insights into their creative subjects~\cite{lewis2023art, sturdee2022state}. Therefore, investigating the artistic process involving robots offers an opportunity to gain deeper insights into robots' creative potential. Robotic art, in particular, provides a compelling case for this exploration.

We define robotic art as artworks that utilize robotic or automated machines to create artistic experiences and tangible artifacts. One example is robotic installation art, in which robots are programmed to follow specific rules that embody the artist’s expression (\autoref{fig:teaser} (a)). Another example is responsive art, in which robots react to their environment, with behaviors that change over time or in response to spectators (\autoref{fig:teaser} (b)). Additionally, there are robotic creators, which possess a degree of agency, allowing them to collaborate with human artists and produce works that extend beyond mere replication of human-created art (\autoref{fig:teaser} (c) and (d)). As such, robotic art becomes a rich case for exploring human-machine interactions in creative contexts. Gaining a deeper understanding of how robots facilitate artistic expression can provide insights for designing computing systems to support creative activities~\cite{gomez2021robot}.

% Therefore, we did...
We draw on semi-structured, in-depth interviews with renowned professional robotic artists to investigate the use of robots in artistic practice. Specifically, our goal is to understand how artistic exploration of robotic systems challenges conventional assumptions about the functions of robots, such as their roles in automating repetitive tasks or serving human needs. We also explore the implications of robots in the artistic process and examine how creativity may emerge within robotic art. To address these interrelated inquiries, our study focuses on the practice of robotic art, posing the research question: \textit{How do robotic artists utilize robots in their artistic practice?} We approach this inquiry through the perspectives and experiences of robotic artists, who creatively design, modify, and repurpose robotic systems for artistic expression and exploration.

% The key findings are...
Our findings highlight the social, material, and temporal dimensions of artists' practices that shape their creativity and artistic outcomes. The creation of robotic art is largely a social process, as artists receive both explicit and implicit feedback through the audience's reactions and reception of their work. Simultaneously, the embodiment and malfunctions inherent to robotic systems drive artistic experimentation. The temporal processes of creation and exhibition, beyond just the final product, further enhance the creative value. Our empirical analysis presents how creativity emerges through the interplay of social, material, and temporal interactions among artists, robots, audiences, and the environment.

% The contributions of this work are...
We make two main contributions to HCI in this study. 
First, we elucidate the interactive mechanisms among key actors---human creators, machines, audiences, and environments---within the practice of robotic art, a topic that remains underexplored in HCI. Our findings reveal the significance of sociality (e.g., interactions between artists and audiences), materiality (e.g., the embodiment and malfunctions of robots), and temporality (e.g., the processes of creation and exhibition) in shaping creative values. We propose that these three facets are central to the creative process and facilitate the emergence of creativity in robotic art.
Second, drawing from the findings, we offer implications for \textit{socially informed}, \textit{material-attentive}, and \textit{process-oriented} creation with computing systems. We suggest leveraging these three aspects to enhance creativity and the creative experience. Specifically, we discuss the value of incorporating implicit audience feedback, designing with technical malfunctions, and focusing on the post-creation process to foster alternative creative experiences with machines~\cite{alter2010designing, juarez2022glitch}.



\section{Notations}

We consider an auto-regressive language model $M$ with parameters $\theta$. We use $p_\theta(\cdot \vert x)$ to denote $M$'s distribution over the next token given the provided context $x$. 
Given a question $q$ (e.g., \nl{Jane had 4 apples and ate half of her apples. How many apples she has now?}), we denote the model's response as $(\textbf{r}, \textbf{a})$,
where $\textbf{a}$ is the answer (e.g., \nl{2}) and $\textbf{r}$ is a \emph{reasoning path} (or chain-of-thought),  a sequence of logical steps supposedly leading up to this answer (e.g., \nl{If Jane ate half her apples, this means she ate 2 apples. 4 minus 2 is 2.}).

\section{Confidence-Informed Self-Consistency}
\label{sec:cisc}

In this section we present \textit{Confidence-Informed Self-Consistency} (CISC). 
When designing CISC, we hypothesized that it is possible to reduce self-consistency's computational costs by generating a \emph{confidence score} for each reasoning path, and performing a weighted majority vote.

As an intuitive example, consider a hypothetical setting where there exist only two possible answers, one correct and one incorrect. For a model that responds with the correct answer $60\%$ of the time, standard majority voting will require \emph{40 samples} to reach $90\%$ accuracy\footnote{Calculated using the binomial distribution. All the technical details are included in Appendix \ref{appendix:example}}. However, a weighted majority vote that weights correct answers twice as much as incorrect ones, will achieve 90\% accuracy with less than \emph{10 samples}. 

With this motivation in mind, we build on recent findings suggesting that LLMs are capable of judging the correctness of their own outputs \cite{kadavath2022language, tian2023just, zhang2024small}, and incorporate the model’s self-assessment of its reasoning paths into the final answer selection:

\begin{definition}[Confidence-Informed Self-Consistency]
\label{def:cisc}
Given a question $q$ and responses $\{(\textbf{r}_1, \textbf{a}_1), \dots, (\textbf{r}_m, \textbf{a}_m) \}$, CISC involves:

\begin{itemize}
    \item \textbf{Confidence Extraction}: A self-assessed confidence score $c_i\in\R$ is derived for each $(\textbf{r}_i, \textbf{a}_i)$.
    \item \textbf{Confidence Normalization}: The confidence scores are normalized
    using Softmax: $\tilde{c}_i = \frac{\exp\!\bigl(\frac{c_i}{T}\bigr)}{\sum_{j=1}^m \exp\!\bigl(\tfrac{c_j}{T}\bigr)}$, where $T$ is a tunable temperature hyper-parameter (see discussion below).
    \item \textbf{Aggregation}:  The final answer is selected using a confidence-weighted majority vote: $\hat{a}_{CISC} = \arg\max_a\sum_{i=1}^m \textbf{1}[\textbf{a}_i = a]\cdot \tilde{c}_i$. 
\end{itemize}
\end{definition}

The temperature parameter $T$ controls the relative importance of the answer frequency versus the confidence scores. Namely, as $T\to \infty$, the distribution of normalized confidence scores approaches the uniform distribution, and CISC collapses to vanilla self-consistency. Conversely, as $T\to 0$,  the softmax normalization approaches the hard maximum function, prioritizing the single response with the highest confidence and disregarding the overall frequency of answers. This may lead CISC to select a different answer than self-consistency (see Figure \ref{fig:high-level}). 

\section{Design Space}
\label{sec:design-space}

% \subsection{Is the User Actively Engaged with the Tool?}

A fundamental challenge in integrating AI-powered research tools is ensuring that users remain cognitively ``in the loop'' rather than passively accepting AI-generated output. While automation promises efficiency and rapid idea generation, it also risks encouraging users to rely too heavily on system outputs, which can diminish human critical thinking and decision-making processes. We explore four design dimensions critical to navigating this tension.

% We explore 4 design dimensions that inform cognitive engagement when developing AI-assisted research tools: User Agency \& Control, Divergent \& Convergent Thinking, Adaptability, and Accuracy. In doing so, we highlight design patterns that promote meaningful human-AI collaboration while identifying potential pitfalls that could limit researcher agency, cognitive engagement, or trust in AI-driven insights.

% We expect a tool with good user agency that allows users to intervene, override, or refine AI outputs. They take an active role in evaluating and steering the research process. This active involvement forces researchers to check the validity of automated suggestions. It helps to strengthen their critical thinking and decision-making. For example, systems like CoQuest require users to constantly evaluate AI outputs and guide the research process \cite{CoQuest}

% The balance of AI contribution in divergent and convergent thinking affects cognition by guiding how researchers balance searching for new ideas with refining those ideas. The tools should support both creative exploration and systematic analysis. This dual approach keeps users mentally agile and engaged. In brainwriting sessions, participants must justify their choices and debate various ideas, which reinforces both divergent and convergent modes of thought \cite{brainwriting}. Similarly, IdeaSynth uses clarifying questions to prompt deeper idea refinement \cite{IdeaSynth}.

% We define adaptability as a tool’s flexibility to support different research tasks and user preferences. A tool that adapts to diverse approaches aligns with a user’s natural cognitive strategies. This design enhances various problem-solving styles. It lets users work in a way that suits their thinking patterns.

% Lastly, accuracy is a crucial consideration for building trust in a system. A tool must deliver reliable and contextually relevant information. When researchers trust the tool, they integrate its insights with confidence. For example, Paperweaver directly provide the transparency on all relevant information that researchers can depend on \cite{paperweaver}. This trust leads to more profound results and focuses cognitive effort on meaningful analysis rather than refining LLM-generated outputs. 

% These dimensions create a framework that continuously shapes cognitive engagement through design and functionality. LLM tool designs must balance these factors to enhance human cognition. Their goal is to keep users active and critical throughout the research process. The following subsections will examine all existing tools under each dimension and discuss how they contribute to overall cognitive engagement.


% "is user actively engaged in the work"
% We want to discuss how current tools engage users, passively or actively. The design challenge is to ensure that users remain cognitively “in the loop,” regardless of the automation level, especially on critical/high-stake intersections.

% TODO: we don't need this many example; a couple of high-level ones would be enough; we will mainly use the rest of the dimensions to articulate the appropriate amount of cognitive engagement.


% \subsubsection{Example of good use cases}
% \begin{itemize}
%     \item \cite{AnalogicalSearchEngine} The core design principle of the system is to surface articles that are \textit{analogically} relevant, potentially from very different domains. The fact that user need to be cognitively active to search for relavant article. And engaged in search results and refine the search query again.
    
%     \item \cite{brainwriting} Convergence Stage: The discussion and selection process necessitates active cognitive engagement. Participants must justify their choices, debate the merits of different ideas, and synthesize information from both human and AI sources.
    
%     \item \cite{CoQuest} RQ Flow Editor: The user is constantly required to evaluate the AI's output, provide feedback, and make decisions about the direction of the RQ development. This is not a passive consumption of AI-generated content; it's an active, intellectually demanding process. The user must remain "in the loop" to guide the AI.

%     \item \cite{CoQuest} AI Thoughts Panel: is specifically design for explaining LLM's reasoning and rationale. This is to engage user actively, and make sense of system's decision making process. Also, the wait time introduced, indirectly force user to engage with other component/interaction, further increase Cognitive Engagement.

%     \item \cite{IdeaSynth} Node-Based Canvas Visualization (Fig. 2, 4): The canvas interface with interconnected nodes provides a visual representation of the research idea's structure and development. his externalizes the user's thought process, making it easier to see connections, identify gaps, and explore different variations. It promotes active cognitive engagement by requiring users to organize and structure their ideas. The visual representation can reduce cognitive load compared to a purely text-based approach.

%     \item \cite{IdeaSynth} Clarifying Questions from AI (Fig 4, 5): The AI prompts users to clarify aspects of their ideas. This forces users to actively think about the details and nuances of their ideas, promoting deeper cognitive engagement. It's not just passive reception of suggestions; it's an interactive process of refinement.

%     \item \cite{paperweaver} This directly addresses cognitive engagement by providing concise, relevant information. It helps users quickly assess relevance without reading the full abstract, addressing the problem of information overload highlighted in the paper's introduction and formative study. It moves beyond simple keyword matching to provide a more semantically meaningful connection.

%     \item \cite{scholastic} Document Map: The Document Map encourages active cognitive engagement. Users are not passively presented with results; they must actively explore the clusters, hover for information, and make decisions about which documents to sample. The visual metaphor of a map encourages exploration and the formation of mental models. (Section 6.1)

%     \item \cite{scholastic} Document Reader: The Document Reader promotes active engagement through the coding process. Highlighting text and selecting keywords requires careful reading and interpretation. The ability to add memos further encourages reflection and documentation of the user's reasoning. (Section 6.2)

%     \item \cite{scholastic} Code Examiner: The Code Examiner encourages active engagement by requiring users to navigate the hierarchical word clusters and make decisions about which documents to sample based on their codes. The ability to refine codes and categories further promotes reflection and iterative analysis. (Section 6.3)

%     \item \cite{synergi} Combined PDF Viewer and Highlighter. Requires active reading and selection of relevant text. (Section 3.1 and shown in Figure 3A). Highlighting is an active reading strategy. The user must constantly evaluate the text for relevance to their research question.

%     \item \cite{synergi} Drag-and-Drop Outline Editor: Requires the user to actively synthesize and organize information. (Section 4.4 and Figure 5) This is not a passive activity. The user must make decisions about the structure of their outline, the relevance of different threads, and the overall narrative of their literature review.

%     \item \cite{threddy} Active Reading and Highlighting: The core interaction of Threddy – highlighting text and extracting references – requires active reading and engagement with the source material. Users are not passively consuming; they are actively selecting and curating.

%     \item \cite{threddy} Thread Organization: The drag-and-drop interface for organizing threads forces users to actively consider the relationships between different pieces of information and construct a mental model. This promotes deeper understanding.


% \end{itemize}

% \subsubsection{Mixed}
% \begin{itemize}
%     \item \cite{CoQuest}Paper Graph Visualizer: provides the tools for cognitive engagement (access to papers, summaries, citation links), but it doesn't actively enforce or guide users towards critical thinking. It relies on the user's own initiative and research skills to avoid the pitfalls you've mentioned. This is a significant difference. It's a good tool, but it's not inherently good for cognitive engagement if used passively. Fail to consider alternative perspectives: The tool might present a biased or incomplete view of the literature. Users might not actively seek out papers outside the Tool's suggestions. Skip deep reading because it's readily accessible: User might engage superficially, and think they have fully understood the main idea, while skipping deeper cognitive engagement.

%     \item \cite{disciplink} By presenting EQs rather than definitive answers, the system encourages users to think critically about different perspectives. The theme extraction and information scent features help users process information efficiently without getting lost in details. But passively present papers may be risk. 


%     \item \cite{IdeaSynth} Edge-Based Connection Validation (Fig.4): User can expand the edge between two nodes, and see how the connection can be improved. 

%     \cite{threddy} Recommendation: While recommendations are presented, the user must actively evaluate their relevance. This requires cognitive engagement, but the limited control over the recommendation criteria means this engagement might be less efficient or focused than it could be. The "FOMO" expressed by one participant (Section 6.2) suggests that the recommendations, while engaging, might also be overwhelming.
% \end{itemize}

% \subsubsection{Example of bad use cases}
% \begin{itemize}
%     \item \cite{brainwriting} Divergence stage: risk of over-reliance on the LLM, leading to passive acceptance of its suggestions. If users become "prompt engineers" rather than "idea generators," their cognitive engagement shifts from understanding the problem domain to understanding the LLM's quirks. 
%     \item 
% \end{itemize}

\subsection{User Agency and Control}
\label{sec:agency}

Human-AI collaboration systems aim to improve productivity and creativity by offloading certain tasks from humans to AI systems while keeping the user in the driver's seat. Providing user agency and control via source material participation, the ability to refine AI output, and the ability to reject and override automated actions are crucial to maintaining cognitive engagement from the user.

\subsubsection{Engagement with source material.} For tools aimed at making sense of existing text content (such as qualitative coding, thematic analysis, or literature reviews), there is a delicate balance to be struck between AI assistance and user agency. While recent advancements in LLM capabilities show promise in processing and summarizing large amounts of text, users still have to engage with the source text to build their own understanding and avoid model overreliance. One way to foster engagement is for user-selected text highlights to drive content generation. Systems such as Synergi~\cite{synergi} and Threddy~\cite{threddy} integrate PDF readers that transform user-selected highlights into seeds for AI-driven research thread generation, while Relatedly~\cite{Relatedly} further refine this process by organizing and highlighting overlapping research themes. Meanwhile, platforms like Scholastic~\cite{scholastic} and SenseMate~\cite{sensemate} scaffold existing analysis methods with cluster suggestions and strategic sampling. The role of AI in these systems (such as Scholastic~\cite{scholastic} and SenseMate~\cite{sensemate}) is to scaffold existing analysis methods with cluster suggestions and strategic sampling.

\subsubsection{Refining AI output.} Human-centric AI tool design assumes the user is the expert, giving them the power to modify the output provided by the AI system. Most tools we reviewed allow users to edit and refine AI-generated artifacts to ensure that they meet the user's goals. In Threddy~\cite{threddy}, for example, users can manually clean up errors in references and links extracted from a paper snippet. Beyond fixing errors, user editing can be designed into human-AI sensemaking systems to varying degrees. Arranging AI outputs into a more interpretable structure (such as node-link diagrams~\cite{IdeaSynth, CoQuest} or outlines~\cite{synergi}) can foster deeper engagement with suggestions. Finally, iterating on prompts and queries can help users gradually incorporate new ideas and discoveries into AI output. This is notably useful in cases such as the Analogical Search Engine~\cite{AnalogicalSearchEngine}, where users may not know exactly how to initially prompt but can gain and apply new information with each re-prompt. 

% Systems can provide the user with multiple ways of engaging with and refining AI suggestions. In Synergi~\cite{synergi}, for instance, users extract clips from papers which are then used to recommend research threads that map to the broader literature. The resulting threads can be curated and reorganized into new research outlines in an iterative process.

\subsubsection{Rejection and overriding.} One aspect of control that merits further consideration is the ability of users to reject, override, or ignore model output. Rejection can manifest in systems implicitly. For example, editing or curating AI output implies rejection of the original content in part or whole. However, most of the research papers we reviewed had AI assistance embedded in the system, with minimal ability to ``turn off'' AI suggestions. One notable exception is SenseMate, which explicitly aims to provide AI theme suggestions on demand rather than by default~\cite{sensemate}. In SenseMate, AI suggestions are hidden by default; users can also see the reasoning for the theme suggestions and explicitly reject them. Similarly, Scholastic's text clustering algorithm does not impose keywords on clusters. Instead, users have the option of either developing internal mental models of the meaning of the cluster or providing the algorithm with explicit codes that can be iterated on~\cite{scholastic}.

\begin{designRecom}{Provide User Agency and Control}
\textit{Design Insight:} Users should have meaningful control and agency within AI-assisted systems to ensure that they remain the primary decision-makers. Users should be able to engage with the source text and edit, refine, or reject AI output.
\smallskip

\textit{Implications:} 
\begin{itemize} 
\item \textbf{Customizable AI Assistance:} Users should have control over when and how AI assistance is applied, including the ability to disable suggestions when not needed.
\item \textbf{Iterative Refinement:} Enabling users to edit and refine AI output can lead to more relevant and intentional output.
\item \textbf{Engage the User:} Support user understanding and agency by giving users control over the structure of collaboratively generated insights and providing access to source text where relevant.
\end{itemize}

% \textit{Recommendation example:} Disciplink~\cite{disciplink} highlights a limitation where users can guide the generation of new exploratory research questions but cannot directly edit existing ones. Expanding user control to include full-text editing would enhance agency, foster iterative refinement, and enable LLM suggestions to evolve alongside the user's understanding and mental model.
\end{designRecom}

% \begin{itemize}
%     \item \cite{AnalogicalSearchEngine} The search system gives the user full freedom to continue progressing, user does not need to rely only on this search result.
%     \item \cite{brainwriting} The freeform and plug-and-play LLM usage gives full control to the user. Also, in the Convergence Stage: The selection process is entirely human-driven. The LLM is used as a tool for further development after humans have made the key decisions about which ideas are worth pursuing. This maintains a high level of user agency, as the humans are "calling the shots" on which direction the ideation process takes.
%     \item \cite{IdeaSynth} Facet Node Creation and Editing (Fig. 5): Users can create, title, and edit the content of idea facet nodes (Problem Description, Solution, Evaluation, Contribution). Users are in full control of expressing their ideas within the defined facets. This allows for personalized input and avoids the feeling of being dictated to by the AI.
%     \item \cite{IdeaSynth} Manual Node Linking (Fig. 5)Users can manually link nodes to create relationships between idea facets. This is a good use case. This empowers users to define the structure of their research idea, representing their mental model. It's not forced by the AI, giving the user ultimate control over the conceptual relationships.

%     \item \cite{IdeaSynth} Adopting or Rejecting AI Suggestions (Fig. 4, 5): Users can choose to accept, modify, or ignore AI-generated suggestions for node content, alternatives, and expansions. User is not forced to accept AI output. The system acts as an assistant, offering options, but the user makes the final decision. This prevents overreliance on the AI and maintains user control over the ideation process.

%     \item \cite{IdeaSynth} Literature Search and Collection Management (Fig 3): User can search for relevant scientific papers, add to personal collection, and remove papers from collection, or get AI recommendation if needed. User decide what literature background is relevant to the current ideation context.

%     \item \cite{paperweaver} Suggested Topic Description (Section 4.2.1): The system allow user's direct input for folder description, or generate an default one using LLM. They can override the suggestion, refining it to better reflect their interests.

%     \item \cite{scholastic} Document Sampling (Breadth-First - Document Map): Users can choose to sample randomly (initial exploration) or strategically based on the visualization of document clusters (geographical treemap). They control the granularity of the clusters using a slider. The ability to hover for titles and right-click for previews gives users control over which documents to examine further. The system does not force a particular sampling method. This aligns well with interpretive research, where the researcher's judgment is important.

%     \item \cite{scholastic} Document Sampling (Depth-First - Code Examiner): The Code Examiner allows depth-first sampling based on user-defined codes and categories. Users select codes, and the system presents documents ranked by relevance to those codes (using the underlying model). This provides high agency, as the sampling is directly driven by the researcher's evolving understanding. (Section 6.3, Figure 6) This reinforces user control by allowing them to directly leverage their coding schema to find relevant documents. It's a form of querying the corpus based on the user's interpretive framework.

%     \item \cite{sensemate} Theme Recommendations (On-Demand): AI suggests themes for each unit of analysis (snippet of text). These are not automatically applied. Users must explicitly choose to view and then accept, reject, or mark them as unsure. The system presents themes one at a time to encourage thoughtful review. Note: SenseMate is designed to help users apply this existing codebook more efficiently and consistently, not to develop a new one from scratch. The focus is on deductive coding, where a pre-defined codebook is used, rather than inductive coding, where codes are developed from the data. The authors state: "High rates of intercoder reliability are especially helpful during deductive coding where sensemakers start with a predefined set of thematic codes, which are assigned to qualitative data. The AI support in SenseMate can help users stay grounded in the codebook and minimize coding errors, which can speed up the analysis." (Page 12)

%     \item \cite{sensemate} Rationale Explanations: For each theme recommendation, users can click a "View Reason" button to see why the AI suggested that theme. The rationale is presented as bolded words within the original text snippet (Figure 3). 

%     \item \cite{sensemate}  Feedback Mechanisms (Quick Questions \& Highlighting): If a user rejects a recommendation, they are presented with up to five yes/no questions about whether specific phrases in the rationale are relevant (Figure 4). Users can select or deselect words within the rationale to indicate which parts are relevant or irrelevant (Figure 5). This allows for more nuanced feedback than simple acceptance/rejection. (page 8). feedback system is excellent for several reasons. It increases user agency by allowing them to directly influence the model's behavior (although retraining wasn't implemented in this version). It's adaptable because it provides different levels of feedback granularity (quick questions vs. highlighting). It fosters cognitive engagement by making users actively participate in refining the AI's understanding. \textbf{also in cog engagement (user engage in refining / enhance models), adaptability (provides different levels of feedback granularity (quick questions vs. highlighting), accuracy (refine existing coding to enhance model accuracy))}

    
%     \item \cite{synergi} Outline Editor: The user can drag and drop threads, edit labels, remove irrelevant information, and add their own insights. This allows for direct manipulation and customization, restoring agency after the relatively low-agency thread generation step.

%     \item \cite{threddy} Highlighting and Selection: Threddy allows users to highlight text and areas within PDFs, triggering the extraction of references. This gives users direct control over the starting point of thread creation. The user actively chooses what content is relevant. (Section 4, "Highlighting and Selection")

%     \item \cite{threddy} Reference Deselection: Within the holding tank, users can "deselect any reference they do not want to include or to fix any extraction error." (Section 4) This provides fine-grained control over the automatically extracted information, allowing users to curate the input to the thread.

%     \item \cite{threddy} Thread Creation and Organization: Users can create new threads, add content to existing threads, and (re-)organize threads using drag-and-drop. This offers significant control over the structure and content of their knowledge organization. (Section 4, "Organization")
    

% \end{itemize}

% \subsubsection{Mixed use cases}
% \begin{itemize}
%     \item \cite{CoQuest} The user has direct input at multiple stages. They initiate the process with keywords, provide feedback to refine AI suggestions, and can directly edit the structure of the RQ flow by adding, deleting, or moving nodes. The concern of artificially limiting the expansion of related RQ can be mitigated by re-prompting and re-generating. However, users may still experience unintentional bias from LLM-recommended RQs, and users do not have the option to override manually.
    
%     \item \cite{disciplink} The lack of direct EQ editing, addition, and deletion is a limitation. The user's control is primarily through influencing the generation of new EQs, not manipulating existing ones. This is less agency than a system that would allow full text editing of EQs.

%     \item \cite{scholastic} Coding (Document Reader): Users have high agency in applying codes. They highlight text, apply existing codes, or create new ones. They can also add memos, which is crucial for interpretive research. The "in vivo" keyword selection, while offering some automation, is still under user control (they choose which keywords to select). But user cannot apply multiple code at the same time on a single selection, it is acknowledge as a limitation by original author.

%     \item \cite{synergi} Loopy Belief Propagation (LBP) for Retrieval: The user sets the starting point, but the algorithm determines the expansion. There's no mention of parameters the user can adjust to tune the LBP process (e.g., weighting factors, stopping criteria). This lack of transparency and fine-grained control reduces agency. (Fig 1-B, Section 4.1)

%     \item \cite{synergi} Hierarchical Thread Generation:  The user receives a pre-structured hierarchy generated by GPT-4. While useful for overview, the user has limited input on how this hierarchy is created. The user doesn't choose the clustering method, similarity thresholds, or summarization prompts. This is a "black box" from the user's perspective. It's efficient, but the lack of control could lead to issues if the generated hierarchy doesn't align with the user's mental model. (Fig 1-C, Section 4.2)

%     \item \cite{threddy} Thread Selector Suggestions: The thread selector uses an algorithm to suggest the most relevant thread for new content. (Appendix A). While helpful, this is a recommendation, not a forced action. The user retains the ultimate choice. However, the algorithm's reliance on similarity metrics (and the potential for "high dispersion, low cohesion" as noted in Appendix A) could lead to less-than-ideal suggestions, subtly nudging users in certain directions.


% \end{itemize}
% \subsubsection{Example of bad use cases}
% \begin{itemize}
%     \item \cite{brainwriting} the LLM only suggests relevant diverging topics, without clear reason on the process. If it produces redundant, irrelevant, or biased output (as some students reported), the users can only try different prompts – they can't "debug" the AI's reasoning. Prone to automation bias.
%     \item \cite{disciplink} While users have control over EQs, the process of query generation is somewhat opaque. Section 4.2.2 describes query expansion using LLMs, but the user doesn't directly see or modify these queries. This is a potential limitation, as users might want more transparency or control over the search terms used, especially in unfamiliar disciplines.

%     \item \cite{threddy} Recommendation Control: The "Overview and Discovery" panel provides recommendations based on citation coverage. While users can click "Refresh" to regenerate recommendations, they have limited direct control over the recommendation criteria. The paper mentions a desire for "additional mechanisms for specifying which context is personally more important" (Section 7.1), acknowledging this limitation. The user can only accept or reject recommendations, not shape how they are generated. This is a significant weakness. They can't, for example, prioritize recency, specific authors, or keywords.
% \end{itemize}

\subsection{Divergent and Convergent Thinking}
\label{sec:divergent-convergent}
% Talk about when AI tool helps in divergent thinking (to expand space) and convergent thinking (to filter and narrow down). We try to show that AI can be useful to take over more work in divergent stages, so users can be exposed to more previously unknown knowledge, but it is less okay if we are in convergent stages where the user should be taking charge in terms of direction and criteria.

% CoQuest: divergent thinking: balanced cognitive engagement, effective for both divergent and convergent tasks

% Alalogical Search engine: more cognitive engagement, user are asked to actively adpat and apply ideas from analogical articles, the ai are maining focused on divergent thinking 

% brainswriting: the system fosters collaboration and critical engagement with AI, it is designed to help users to generate many idea, which is AI partcipated in a divergent thinking

% Disciplink: divergent thinking, DiscipLink generates diverse exploratory questions (EQs) from various disciplines, and crafts queries to broaden the search, users can create their own EQs based on findings.

% ideasynth: ai contribute into divergent thinking, it helps users generate and connect research ideas by using AI to suggest alternative ideas and connections, AI gives suggestions on improving the idea. It also helps users refine their research questions. 
% also provides some convergent thinking, provides literature analysis to summarize and connect papers with the nodes.

% luminate: divergent thinking, users diverge by generating multiple responses from a single prompt, the system is designed to help with the generation of the design space rather than individual artifacts

% paperweaver: mixed approach, PaperWeaver surfaces connections between papers that may be relevant to the user, which helps to expand their view of the literature (divergent), and then the system provides contextualized descriptions and helps users decide which papers are worth saving (convergent)

% personalFlow: divergent thinking from ai, uses ai generated personas with different expertise, which helps users explore diverse perspectives and direction. then user can refine their research questions after evaluating personas. The ability to customize personas promotes better engagement with the system and helps users feel in control.

% scholastic: 

% sensemate: convergent thinking from ai, AI is used for semi-automating qualitative coding and supports transparency and modification. 

% synergi: divergent thinking, increase cursitisity from engagement, very good for broadening perspectives with citation graphs and LLM to epand and structure research threads.

% threddy: supports divergent thinking by helping users to expand their understanding of a research area through the discovery of new connections and threads of research, human are the ones that do convergent thinking by organizing information into threads, consolidate their knowledge and identity key themes.

% sensecape: mixed thinking, in the divergent phase, the AI generates a wide range of subtopics and questions that allows users to expand their understanding of a topic, in the convergent phase, the AI organizes these ideas into a hierarchical structure, which helps users filter and make sense of the information
\subsubsection{Divergent Thinking}
 Divergent thinking is essential for expanding the horizons of research. AI tools support this by generating exploratory questions and novel insights that researchers may not have thought about. For example, the Analogical Search Engine~\cite{AnalogicalSearchEngine}, DiscipLink~\cite{disciplink}, and IdeaSynth~\cite{IdeaSynth} provide functionalities to explore creative connections in seemingly unrelated fields. However, when using an AI tool for divergent thinking is used in isolation, this approach can overwhelm users with many novel ideas without providing the guidance necessary to narrow them down or refine them effectively.
\subsubsection{Convergent Thinking}
In convergent thinking, AI tools transform raw data into clear and structured insights, acting as intelligent partners that help researchers distill complex information into coherent narratives. These systems guide scholars in filtering and organizing diverse inputs so that essential themes emerge with clarity and precision. 

For example, SenseMate~\cite{sensemate} leverages rationale extraction models to generate transparent theme recommendations and human-interpretable explanations for qualitative coding. By grounding its suggestions in data-driven rationale rather than relying on large language models, SenseMate empowers even novice users to engage deeply with the source material while retaining full control over their coding decisions. Focusing on the same task, CollabCoder~\cite{CollabCoder} utilizes LLMs to automatically generate qualitative code suggestions and facilitate structured group discussions, thereby bridging individual insights into a collective consensus. Complementing these approaches, Scholastic~\cite{scholastic} employs advanced visual analytics to help teams organize and interpret complex datasets, while PaperWeaver~\cite{paperweaver} presents contextualized links that highlight the most relevant insights. These convergent functionalities make thematic grouping and filtering more efficient and transparent. 

\subsubsection{Mixed-Thinking}
Some tools strike a balance by supporting both divergent and convergent thinking. Threddy~\cite{threddy}, for example, allows users to input various requests so that LLM systems can organize ideas into coherent themes. They also leverage hierarchical structures to discover new connections based on user input. Similarly, Synergi~\cite{synergi} uses citation graphs and language models to expand research threads and consolidate them. However, these mixed-thinking approaches often leave users with limited control over the balance between exploration and refinement, which may hinder effective sensemaking. So while too much convergence might stifle creative exploration, too much divergence can cause cognitive overload. A well-designed research tool should support either or both modalities, but its user-facing capabilities should be made transparent. In the design space of research tools, allowing users to control the levels of divergent and convergent functionality is key to researcher autonomy.


\begin{designRecom}{Enabling Divergent and Convergent Thinking}
\textit{Design Insight:} Consider how users may need different types of support during the research process. Users may need more creative support from GenAI when they are discovering a new research area to help them understand the space. Following this, GenAI should provide guidance and structure when users are converging their existing research material.
\smallskip

\textit{Implications:} Future systems should be engineered with a design space that considers:

\begin{itemize}
\item \textbf{Transparency of AI Contributions:} Users should clearly understand whether the AI is supporting divergent exploration, convergent refinement, or both.  
\item \textbf{User Control Over AI Assistance:} Users should be able to adjust the level of AI support for idea generation and refinement based on their needs.  
\end{itemize}

% \textit{Recommendation example:} 
% CoQuest\cite{CoQuest} guides users through both expansive idea generation and systematic refinement.  The tool provides functionality for divergent thinking by generating exploratory questions and provides analogical search outputs to expose users to diverse research perspectives. CoQuest aggregates and refines these outputs in a convergent manner to help researchers critically evaluate and select the most relevant information.

\end{designRecom}


\subsection{Adaptability}
\label{sec:adaptability}
% "is there enough variation/ways  this tool can be used for?"
% We want to discuss how the research tools are adopting a flexible design philosophy, aiming to create systems that support different types of tasks, different workflows, and different personal preferences. It asks whether the tool can adjust its mode of operation depending on the context—say, leaning more on machine initiative when the task is exploratory and the user lacks domain expertise, but shifting to a mode that grants more manual control when the task requires careful evaluation or detailed refinement.
% \subsubsection{Example of good use cases}
% \begin{itemize}
%     \item \cite{AnalogicalSearchEngine} The fact of a search-based system makes it adaptable for a lot of unknown scenarios, and using "purposes" and "mechanisms" as a goal to prioritize make it even more adaptable, as the goal and way of doing will be captured for different research projects.
%     \item \cite{brainwriting} Since the process are entirely driven by user, including main topics, key diverging and converging steps. And it is not subject/process specific, user can plug-in and get additional AI assistance when needed.
%     \item \cite{CoQuest} RQ Flow Editor: The system doesn't restrict the user to predefined categories or specific input formats. This allows it to be used for a variety of research topics and at different stages of the research process. The user can start with a vague idea and gradually refine it, demonstrating adaptability to different user needs and workflows.

%     \item \cite{IdeaSynth} Literature-Based vs. Idea-Based Exploration: The system supports both literature-driven exploration (through paper recommendations and summaries) and idea-driven exploration (through facet generation and refinement) (section 4.1). 
%     \item \cite{IdeaSynth} Prompt Customization (Fig 5): The ability to customize prompts for AI suggestions allows for some adaptation to user needs.

%     \item \cite{scholastic} Support for different tasks and free workflow:  The tool is designed to support a complete qualitative analysis workflow, from initial exploration to the development of a refined coding schema. The tool allows for flexible workflows. Users can move freely between the Document Map, Document Reader, and Code Examiner. They can start with random sampling and then move to strategic sampling, or vice versa. They can code and categorize iteratively.

%     \item \cite{synergi} Mixed-Initiative Workflow: The system supports both bottom-up (starting from specific papers) and top-down (exploring the generated hierarchy) approaches.

%     \item \cite{synergi} Input Flexibility: user can focus and tag different part of the paper, depending on the interest and research goal. And thus can be expanded to facilitate different research need.

%     \item \cite{threddy} Multiple Entry Points: Users can start from any PDF and build threads from there. They are not forced into a specific starting point or workflow. This supports diverse research approaches.

%     \item \cite{threddy} Flexible Thread Structure: The nested thread structure, combined with the ability to reorganize threads, allows for a high degree of flexibility in how users organize their knowledge. This can adapt to different research styles and evolving mental models.
%  \end{itemize}

% \subsubsection{Mixed use cases}
% \begin{itemize}
%     \item \cite{disciplink} The system's ability to generate EQs from multiple disciplines is a core strength, directly addressing the needs of interdisciplinary information seeking (IIS). The ability to refine the exploration based on user input shows responsiveness to the user's evolving understanding. But the paper acknowledges limitations in adaptability. Section 7.3 discusses the need for more nuanced understanding of user contexts, including research phase and preferences (e.g., "why" vs. "what" questions). While DiscipLink adapts to user input during a session, it doesn't seem to learn user preferences across sessions. Also, it does not allow users to set a specific range of discipline to start with.
%     \item \cite{IdeaSynth} The system's core functionality is built around a specific model of research. While prompt customization offers some flexibility, it doesn't fundamentally change the underlying structure. A truly adaptable system would need to be able to handle different types of research questions, different forms of evidence, and different ways of structuring arguments. The lack of explicit support for different research paradigms limits its adaptability.
% \end{itemize}
    
    

% \subsubsection{Example of bad use cases}
% \begin{itemize}
%     \item  \cite{AnalogicalSearchEngine} The system treats all users and all search tasks the same way. It does not adapt to individual research styles, levels of expertise, or specific project goals. The lack of personalization and customization makes the system a "one-size-fits-all" tool, limiting its ability to support diverse.
% \end{itemize}

Adaptability in AI-assisted research tools refers to a system’s capacity to support the diversity of tasks, workflows, and preferences of researchers. Ways this could be supported in tooling include flexible input mechanisms; fluid, nonlinear workflows; and context-sensitive design.

\subsubsection{Flexible Input \& Customization} Systems that prioritize flexible input mechanisms empower users to tailor the tool’s behavior from the outset. For example, the Analogical Search Engine~\cite{AnalogicalSearchEngine} leverages a custom ranking algorithm that focuses on user-defined “purposes” and “mechanisms” to modulate search results according to varying research objectives, allowing researchers to specify the kind of analogical relationships they are seeking. Similarly, the Brainwriting tool~\cite{brainwriting} allows users to dictate main topics in addition to divergent and convergent steps through a user-agnostic approach. These approaches underscore the importance of customization as a means of preserving user agency and aligning system outputs with specific investigative goals.

\subsubsection{Fluid, Non-Linear Workflows} A second facet of adaptability is found in tools that support non-linear, iterative workflows. The RQ Flow Editor in CoQuest~\cite{CoQuest} exemplifies this by eschewing predefined categories and instead promoting continuous refinement of ideas through a mixed-initiative interaction where AI suggests new research questions and users can provide feedback. Likewise, IdeaSynth~\cite{IdeaSynth} supports both literature-driven and idea-driven explorations through dynamic facet generation and prompt customization, enabling users to decompose an initial idea into finer-grained aspects and explore variations of them. Tools like Scholastic~\cite{scholastic} further demonstrate adaptability by allowing researchers to shift seamlessly between exploration, strategic sampling, and coding — highlighting the value of fluid transitions in non-linear research processes via its interactive document and word clustering.

\subsubsection{Mixed-Initiative \& Context-Sensitive Design} 
Adaptability also manifests in systems that accommodate varied research approaches through mixed-initiative interactions and context-sensitive features. For instance, Synergi~\cite{synergi} offers a mixed-initiative workflow that caters to both bottom-up and top-down strategies, allowing users to engage with content according to their preferred mode of inquiry and seamlessly combine machine-generated summaries and user-curated threads. Interdisciplinary platforms like DiscipLink~\cite{disciplink} contribute to this design space by facilitating the generation and refinement of research questions across multiple disciplines, albeit with some limitations regarding persistent user customization.

% Collectively, these examples emphasize a design space in which adaptability is achieved through varied strategies that maintain user agency while supporting a wide range of research tasks. By enabling flexible inputs, fluid workflows, and context-sensitive interactions, these tools foster environments where researchers can tailor system behavior to their evolving needs. This design philosophy is essential not only for accommodating diverse research paradigms but also for promoting deep cognitive engagement and innovative inquiry.

\begin{designRecom}{Ensure Adaptability and Workflow Flexibility} \textit{Design Insight:}
Research tools must provide flexible workflows and customizable interfaces to address varied researcher needs.

\smallskip

\textit{Implications:}
Future systems should be engineered with a design space that considers:
\begin{itemize}
\item \textbf{Flexible Input \& Customization:}
Enable users to tailor key parameters and input formats to align system behavior with their individual research goals.
\item \textbf{Fluid, Non-Linear Workflows:} Provide multiple entry points and modular interfaces that allow researchers to navigate, reorganize, and iteratively refine their inquiry as new insights emerge.
\item \textbf{Mixed-Initiative \& Context-Sensitive Design:}  Implement context-aware features that respond to shifts in research focus, ensuring a balanced mix of automated support and user control.
\end{itemize}

% \textit{Recommendation Example:}
% The Analogical Search Engine~\cite{AnalogicalSearchEngine} currently relies on a token-level ranking algorithm that identifies `purposes' (the problems addressed) and `mechanisms' (the solutions offered) across research articles. While effective for large-scale analogical reasoning, this uniform approach limits search flexibility. By enabling configurable search options -- such as toggling between analogical ranking, concept-based similarity matching, and methodological pattern exploration -- the system can better accommodate diverse research interests and methodologies.
\end{designRecom}

\subsection{Accuracy} 
\label{sec:accuracy} 
Ensuring that users receive accurate, unbiased, and contextually relevant information is paramount in AI-assisted research tools. To address challenges such as hallucination and contextual drift inherent to large language models, researchers have developed multifaceted strategies that combine human oversight, contextual grounding, and carefully managed efficiency-accuracy trade-offs.

\subsubsection{Interactive Interfaces for Accuracy Validation} A key design pattern embeds interactive mechanisms that enable real-time verification of AI outputs by linking inferences directly to their original sources. For instance, SenseMate \cite{sensemate} employs a ``View Reason'' feature that highlights source phrases underlying a theme recommendation, thus promoting local explainability and inviting critical evaluation rather than passive acceptance. Similarly, Synergi~\cite{synergi} and PaperWeaver~\cite{paperweaver} enhance their LLM-generated summaries by providing citation contexts and contextualized descriptions. Synergi groups related information to offer clear reference, while PaperWeaver uses aspect-based summaries (e.g., problem, method, findings) alongside paper comparisons to help researchers quickly assess the relevance of new publications.

\subsubsection{Iterative Refinement and Human-in-the-Loop Strategies} Complementing interactive validation, iterative refinement processes further emphasize human oversight. Scholastic~\cite{scholastic} demonstrates this approach by utilizing a machine-in-the-loop framework for qualitative text coding, wherein user feedback continuously refines coding decisions through rationale extraction models that enhance transparency and trust. Likewise, Threddy~\cite{threddy} and IdeaSynth~\cite{IdeaSynth} empower researchers to actively shape AI outputs. Threddy facilitates the extraction and iterative organization of research threads, while IdeaSynth leverages LLMs to propose new ideas that users can further refine. Together, these strategies enable researchers to combine automated suggestions with their own expert judgment.

% Collectively, these design strategies both ensure output accuracy and facilitate a reflective, engaged research practice. By integrating interactive validation and iterative human-centered refinement, AI-assisted research tools can minimize misinformation while promoting critical interrogation of outputs, which further reinforces the core principles of human-centered AI design \cite{shneiderman2020human}. In doing so, these systems encourage researchers to actively integrate practices that promote accuracy through ongoing validation and refinement, ensuring that their work remains reliable and contextually grounded throughout the ideation and sensemaking process.


\begin{designRecom}{Prioritize Transparency and Accuracy} \textit{Design Insight:} In scientific research, accuracy is non-negotiable. Systems must balance automation with robust mechanisms that empower users to verify AI outputs.

\textit{Implications:} 
\begin{itemize} 
\item \textbf{Transparent Rationale Explanations:} Allow users to understand the underlying reasoning behind AI outputs. 
\item \textbf{Citation Contexts:} Provide direct access to source materials, enabling quick cross-referencing and validation. 
\item \textbf{Human-in-the-Loop Reviews:} Facilitate iterative feedback that mitigates risks such as hallucinations and inadvertent biases. 
\end{itemize}

% \textit{Recommendation:} Avoid relying solely on fully automated LLM summarization for mission-critical tasks. As illustrated by PaperWeaver~\cite{paperweaver}, while summarization can quickly surface key insights, it also introduces uncertainty that may lead to misinformation or automation bias. A preferable approach, as demonstrated in Synergi~\cite{synergi}, employs recursive summarization techniques that preserve meaning across layers and present exact text snippets from original sources. This strategy enhances transparency and facilitates easier, more reliable validation by researchers. 
\end{designRecom}
\section{Experiments and Results}
We evaluate the performance of \ac{dfm} in terms of tracking accuracy relative to reference motions and the naturalness of transitions between motions. 
Additionally, we demonstrate the multitasking capabilities of our approach through locomotion and gaze control during stylized dancing.

\subsection{Tracking Accuracy}
To assess tracking accuracy, we calculate the difference between the reference motion and the observed joint positions on the robot hardware.
For this evaluation, we select a dancing motion that involved lifting the rear legs.
\figref{fig:moveup_rear_leg} illustrates the height of the rear right leg during this stylized dance.
Although \ac{dfm} does not perfectly replicate the height of the reference motion, it significantly outperforms the \ac{fld} baseline.
To provide a more quantitative comparison between the baseline and our method, we analyze three types of joint angles as shown in \figref{fig:joint_angle_tracking_acc}.
The results indicate that the motion reconstructed by \ac{fld} is overly smooth due to its strong enforcement of quasi-constant parameterization and periodicity assumption with $N = 100$.
In contrast, \ac{dfm} achieves a more accurate reconstruction with $N = 0$, preserving intricate details that may not follow periodic patterns.
When examining the joint encoder data measured from the robot, \ac{fld} again shows excessive smoothing, which we attribute to the overly strong periodic assumptions applied to the local time during \ac{rl} training, as described in \eqnref{eqn:periodic_assumption}.
\Figref{fig:latent_parameters} presents the $\sin{\phi}$ and frequency values derived from the latent parameters across eight channels during the same dancing motion.
On the left side, \ac{fld} shows that all channels of $\sin{\phi}$ are periodic, with little change in frequency.
In contrast, \ac{dfm} demonstrates variability in some channels of $\sin{\phi}$ and frequency during the upward movement of the rear leg, retaining non-periodic features that characterize the dance motions.
Finally, \tabref{table:tracking_accuracy} reports the mean absolute tracking error (MAE) across all joints for all 170 evaluated motions using the real aibo hardware.
Additionally, we test \ac{dfm} using the MIT Humanoid environment~\cite{chignoli2021humanoid} in Isaac Gym.
Our method consistently demonstrates superior tracking accuracy in both robot environments compared to \ac{fld}.

\begin{figure}[t]
    \vspace{1ex}
    \centering
    \includegraphics[trim={0 0 0 0}, width=\linewidth]{figures/fig3.pdf}
    \caption{Height reached by rear right leg. Left, middle and right depict reference motion, \ac{fld} and \ac{dfm} motions respectively. The red dash line illustrates the height of the right rear leg at reference motion.}
    \label{fig:moveup_rear_leg}
    \vspace{-1ex}
\end{figure}

\begin{figure}[t]
    % \vspace{-3ex}
    \centering
    %\includegraphics[trim={0 6ex 0 0}, width=\linewidth]
    \includegraphics[trim={0 2ex 0 0}, width=\linewidth]{figures/tracking_acc.pdf}
    \caption{Comparison of tracking accuracy for the \ac{fld} and \ac{dfm}. Blue: reference motion created by the motion designer. Orange: reconstructed motions from motion representation parts by conditioning the reference motion. Green: joint encoder reading activated by the \ac{rl} policy.}
    % \caption{Comparison of tracking accuracy for \ac{fld} and \ac{dfm}. Motion reconstruction by FLD is overly smooth. In contrast, DFM achieves a more accurate reconstruction, preserving intricate details that may not follow periodic patterns. }
    \label{fig:joint_angle_tracking_acc}
    \vspace{-3ex}
\end{figure}


\begin{figure}[!t]
    \vspace{1ex}
    \centering
    \includegraphics[trim={0 0 0 0}, width=\linewidth]{figures/periodic.pdf}
    \caption{Comparison of 8 channel latent parameters for \ac{fld} at the left and \ac{dfm}  at the right side by conditioning the same dancing motion as \figref{fig:moveup_rear_leg}. The upper and bottom of plots are $\sin{\phi}$ and frequency for each.}
    \label{fig:latent_parameters}
    % \vspace{-3ex}
\end{figure}

\begin{table}[!t]
\caption{Mean Absolute Tracking Accuracy}
\label{table:tracking_accuracy}
\begin{center}
\begin{tabular}{llcc}
\toprule
\textbf{Robot} & \textbf{reference motion} & \textbf{FLD} & \textbf{DFM (ours)} \\
\midrule
aibo & dance & $0.132$ $\rm{rad}$ & $0.094$ 
 $\rm{rad}$ \\
% aibo & locomotion & 0.141 $\rm{rad}$ & 0.123 $\rm{rad}$ \\
MIT humanoid  & locomotion &  $0.125$ $\rm{rad}$  & $0.103$ $\rm{rad}$   \\
\bottomrule
\end{tabular}
\end{center}
\vspace{-5ex}
\end{table}

\subsection{Natural Transition}
The motion representation employed by \ac{dfm} enables continuous frequency interpolation and smooth transitions between different dancing motions.

\Figref{fig:frequency_interpolation} shows the estimated latent frequency parameters conditioned on the reference motion, which primarily involves head movements transitioning from higher to lower dancing frequencies.
% , as demonstrated in the supplementary video.
While most frequency channels remain relatively constant, channels 3 and 4 exhibit gradual changes as shown in \figref{fig:frequency_interpolation}.
The linear interpolation of frequencies in these channels adjusts in response to the changing frequency of the reference dancing motion.
Even though the training dataset consists of discrete frequency types, the motion representation allows for continuous frequency interpolation.
This capability results in smooth, periodic changes in joint positions without abrupt movements, even for previously unseen datasets as shown at the bottom plot in \figref{fig:frequency_interpolation}.

\Figref{fig:natural_transient} illustrates the joint angular velocity at the head pitch and yaw during the transition from motion A to motion B, which primarily involves the head pitch and yaw actuators, as shown in the supplementary video.
During the transition times at 1 and 2.5 seconds, joint positions experience abrupt changes with switches between reference motions.
We compare the transition performance of \ac{dfm} with DeepMimic~\cite{deepmimic}, a well-known learning from demonstration approach that yields high tracking performance on single trajectories but lacks capabilities to deal with multiple motions.
Jerky transitions are observed in this case if the reference dataset and its representation are not carefully crafted.
In contrast, \ac{dfm} achieves smooth transitions without abrupt movements by interpolating in the latent space using \eqnref{eqn:natural_transient}.

\begin{figure}[!t]
    % \vspace{1ex}
    \centering
    \includegraphics[trim={0 0 0 0}, width=\linewidth]{figures/frequency_interpolation.pdf}
    \caption{Frequency modulation during head-moving dance. The upper plot displays the frequency of two representative latent channels out of eight. Solid and dashed curves represent raw and linearly interpolated data, respectively. The bottom plot shows head pitch (HP) and the head yaw (HY) joint angles.}
    \label{fig:frequency_interpolation}
    \vspace{-1ex}
\end{figure}


        % \small {
        % {\color{ourred}\rule[.5ex]{1em}{1pt}$\bullet$\rule[.5ex]{1em}{1pt}} step in place \qquad
        % {\color{ourorange}\rule[.5ex]{1em}{1pt}$\bullet$\rule[.5ex]{1em}{1pt}} forward run \qquad
        % {\color{ourgreen}\rule[.5ex]{1em}{1pt}$\bullet$\rule[.5ex]{1em}{1pt}} forward stride
        % }

% Continuous frequency transient. All plot are shown during dancing frequency which moves mostly head angle positions. Upper plot is all 8 channel of frequency from latent parameter. Solid and dash are raw and linear interpolation for each. middle plots is two plot (3 and 4 channel) are picked up and zoomed up from upper plot. Bottom plot are the joint angle of head yaw (HY) and head pitch (HP) actuator.

\begin{figure}[!t]
    %\vspace{-3ex}
    \centering
    \includegraphics[trim={0 0 0 0}, width=\linewidth]{figures/natural_transient.pdf}
    \caption{Transition between different dance types. The background color indicates the dance motion type. The left and right plots demonstrate hard switches between Dance A and Dance B, with DeepMimic and \ac{dfm}, respectively. Angular velocities at the head pitch (HP) and the head yaw (HY) are shown.}
    \label{fig:natural_transient}
    \vspace{-3ex}
\end{figure}

\subsection{Multi-task Demonstration}
We evaluate the multitasking capability of \ac{dfm} with auxiliary tasks, including locomotion and gaze, respectively.

\Figref{fig:dancing_locomotion} illustrates the locomotion policy during dancing, where an angular velocity command is used to facilitate in-place rotation.
In the reference motion, only the rear legs move alternately, while the forelegs remain stationary.
After training the policy with the reward structure defined in the locomotion curriculum (\tabref{table:reward}), aibo learns to rotate in response to the angular velocity command in the base frame.
To allow this rotation without hindering the movement of the rear legs, the right foreleg is lifted, enabling the execution of the stylized dancing, as shown in the supplementary video.

Similarly, a policy for auxiliary gaze control is trained using the reward scale from the gaze curriculum in \tabref{table:reward}.
This policy enables aibo to adjust its head orientation in response to pitch and yaw commands during dancing, as demonstrated in \figref{fig:dancing_gaze}.
The supplementary video shows that the dance sequence continues smoothly while the robot adjusts its pitch and yaw angles.
aibo utilizes its head and legs to track the commanded pitch and yaw angles, as illustrated in \figref{fig:dancing_gaze_plot}.
For instance, when a pitch of $0.3$ $\rm{rad}$ and a yaw of $0.0$ $\rm{rad}$ are commanded, both the directions of head are moved up with legs.
In contrast, a $-0.5$ $\rm{rad}$ pitch command prompts the head and legs to move in opposite directions.

\begin{figure}[!t]
    %\vspace{-3ex}
    \centering
    \includegraphics[trim={0 0 0 0}, width=\linewidth]{figures/fig8.pdf}
    \caption{Locomotion during dance. The reference dance motion alternates lifting the rear legs while keeping the forelegs stationary. Applying an angular velocity command results in locomotion by lifting the left foreleg.}
    \label{fig:dancing_locomotion}
    %\vspace{-3ex}
\end{figure}

\begin{figure}[!t]
    \vspace{-2ex}
    \centering
    \includegraphics[trim={0 0 0 0}, width=\linewidth]{figures/fig9.pdf}
    \caption{Gaze during dance. The reference motion is the same as in \figref{fig:dancing_locomotion}. The images depict commanded pitch angles of $0.0$, $0.3$ $\rm{rad}$, and $-0.5$ $\rm{rad}$, respectively. The command of the yaw angle is held at zero.}
    \label{fig:dancing_gaze}
    % \vspace{-3ex}
\end{figure}

\begin{figure}[!t]
    % \vspace{-1ex}
    \centering
    \includegraphics[trim={0 0 0 0}, width=\linewidth]{figures/dancing_gaze_plot.pdf}
    \caption{Joint readings for fore left shoulder pitch (FLSP), fore left shoulder roll (FLSR), fore left foot pitch (FLFP), and head pitch (HP) during dancing gaze are shown. The background color of the plot indicates command for pitch angle at the head frame (yellow: $0.0$, red: $0.3$ $\rm{rad}$, blue: $-0.5$ $\rm{rad}$).}
    \label{fig:dancing_gaze_plot}
    \vspace{-3ex}
\end{figure}


\section{Discussions}

\subsection{Transparency in Ride-Sharing Platform Algorithms}
The publicly available Chicago Transportation Network Provider dataset helped us answer many research questions, but ride-sharing platforms still make many of their mechanisms opaque. The lack of transparency in key platform mechanisms---such as pricing models, driver--rider matching algorithms, and driver ranking systems---makes it difficult to pinpoint the exact causes of these disparities. Without greater visibility into these proprietary algorithms, drivers also remain at an information disadvantage, unable to anticipate fare fluctuations or optimize their work schedules effectively.

Pricing models remain opaque, with our analysis revealing that fare adjustments over time have failed to keep pace with inflation, effectively reducing real driver earnings (\cref{sec:results-pricing-stablization}). While platforms advertise dynamic pricing mechanisms that respond to demand surges, drivers have limited insight into how much of the fare they actually receive after platform fees~\cite{santos2020dynamic}. Previous research has shown that drivers tend to work more during peaks for higher compensation~\cite{chen2016dynamic}. A real-time, large-scale understanding of the surge pricing model can help drivers become more informed in planning and organizing their workday, beyond anecdotal observations. Furthermore, researchers can provide prediction models of price surges, helping both drivers and riders adjust plans accordingly. Another key limitation of using the Chicago dataset is the lack of driver earning information. As a result, our analysis can only use the trip fare as a proxy for driver earning. Making such information available can significantly increase transparency into platform operations.

Similarly, the driver-rider matching algorithm remains a black box. Our inferred driver profiles suggest that trip assignments may systematically disadvantage certain groups, particularly those operating in lower-income areas. If the matching algorithm disproportionately favors drivers in high-demand or high-fare regions, it could reinforce existing geographic disparities in earnings. However, such analysis is hard to conduct without access to driver-level information. As discussed in \cref{sec:methods-driver-simulation}, releasing such data may lead to privacy concerns. Our approach is an effort to approximate driver working conditions without needing detailed driver data. However, researchers should still work with ride-sharing platforms to come up with privacy-preserving ways to analyze such data for insights. Also, driver ranking algorithms---which determine access to high-value trips---are equally opaque. While platforms often cite factors such as acceptance rate, customer ratings, and trip history, the lack of public accountability raises concerns regarding potential biases. Accessing such information can support researchers in identifying potential biases, also help drivers provide more desired services to riders.

In all, we call for increased regulatory oversight and platform-level efforts to improve algorithmic transparency. Without clear disclosures on how these systems operate, ride-sharing drivers remain vulnerable to unfair decision-making and fluctuating incomes that they cannot predict or control.

\subsection{Data Analysis Methodology Improvements}
Our study demonstrates the feasibility of simulating reasonable driver profiles from trip-level data, even in the absence of driver-related information. By leveraging a simulation-based approach, we were able to approximate driver earnings, work patterns, and geographic activity. However, there are still areas for improvement for our methodology.

First, a robust evaluation benchmark is needed to validate the accuracy of inferred driver profiles. While our approach provides valuable insights and matches previous empirical findings, the lack of direct ground truth data means we rely on approximations. We need alternative data sources to cross-verify our inferred driver activities. Tools for driver task management, such as Driver's Seat~\cite{calacci2023access}, asks drivers to upload their work tasks and can serve as a potential data source. More autonomous approaches that uses UI understanding techniques and directly collects data from drivers' phones can also scale up this effort~\cite{lu2024crepe}. 

Moreover, expanding the scope of inferred information would provide deeper insights into platform operations. Currently, we infer earnings and work patterns for drivers. Newer algorithms can be developed to analyze additional opaque platform mechanisms as discussed above. Future studies could aim to reconstruct other aspects of opaque platform algorithms, as discussed above, directly from publicly available, large-scale datasets.

Given a large-scale dataset that misses key information aspects, a potential future approach is to self-collect a smaller dataset that contains the necessary details and conduct a joint analysis of both datasets. For example, a smaller dataset that we collect directly from drivers, containing both driver and trip information, can serve both as a benchmark and a basis for use to train machine learning models that predict driver profiles from existing large-scale datasets. Future research can investigate effective measures to combine these different data sources~\cite{harris2018federal} for joint analysis. These methodological advancements can help us to use large-scale ridesharing datasets more effectively and accurately while maintaining driver and rider privacy.


\subsection{Societal Implications: Ride-Sharing as a Reflection of Broader Inequalities}

Our findings revealed regional ride-sharing disparities in the city of Chicago, which largely reflect the broader existing societal inequalities. Drivers working in lower-income neighborhoods---in our case, drivers that service the southern regions of Chicago---consistently earn less, even despite longer work hours. Structural disadvantages, such as lower infrastructure quality, longer wait times, and increased safety concerns---compound the challenges faced by gig workers. Chicago South Side, as a community suffering from violence and poverty, has been an example of social segregation and studied by numerous researchers~\cite{moore2016south, bachin2004building, bell1993community}. As an aspect of a deep-rooted societal issue, ride-sharing inequality in lower-income neighborhoods calls for holistic policymaking efforts from multiple stakeholders.

Our findings provide practical implications for labor activists and policy makers. By providing a more transparent view of drivers’ potential workday experiences, policymakers can better evaluate the labor conditions these platforms create, ensuring that emerging mobility systems align with equity goals. Urban planners and regulators can use these insights to inform policy interventions---such as driver support programs, driver caps, or incentive structures---that promote fairness and mitigate algorithmic biases. Similarly, platform operators themselves might harness these findings to improve their matching algorithms, advancing a more equitable ecosystem that benefits both drivers and passengers.

Research has shown that transportation access can have a positive impact on regional economic growth and productivity~\cite{targa2005economic, banerjee2020road, alstadt2012relationship}. Ride-sharing, as an increasingly critical way of transportation, especially where public transportation is scarce, can support individual and community access to growth opportunities. The persistence of regional earning gaps raises important questions about equity in urban transportation. If ride-sharing platforms are designed primarily to maximize efficiency and revenue, they may inadvertently exacerbate existing economic inequalities by steering high-value rides away from underserved areas~\cite{durand2022access, bocarejo2012transport}.

To address these issues, we call for policy interventions aimed at ensuring fair compensation and equitable access to earning opportunities. Regulators should consider implementing transparency mandates, income stability measures, and algorithmic accountability frameworks to prevent platforms from disproportionately disadvantaging certain driver groups. Moreover, these efforts should be in orchestration with existing efforts to promote infrastructural improvements and public safety in underserved regions. Collaborative initiatives between policymakers, ride-sharing companies, and community organizations can help create a more inclusive transportation ecosystem that benefits both drivers and passengers alike~\cite{baber2022new}.

%%
%% The acknowledgments section is defined using the "acks" environment
%% (and NOT an unnumbered section). This ensures the proper
%% identification of the section in the article metadata, and the
%% consistent spelling of the heading.
% \begin{acks}
% We thank Dr. Sangho Suh for the insightful discussion and constructive feedback during our theme ideation phase. 

We acknowledge the support of the Natural Sciences and Engineering Research Council of Canada (NSERC), [funding reference number RGPIN-2024-04348 and RGPIN-2024-06005].
% \end{acks}

%%
%% The next two lines define the bibliography style to be used, and
%% the bibliography file.
\bibliographystyle{ACM-Reference-Format}
\bibliography{reference}


%%
%% If your work has an appendix, this is the place to put it.
\appendix

\newpage
\appendix
\onecolumn
\section{Full Results on Longbench}
\label{appendix}
% \renewcommand{\arraystretch}{1.2} % 设置行高
\begin{table*}[ht]
\setlength{\tabcolsep}{2.5pt} % 设置列间距
\caption{\textbf{Result on Longbench.} The highest score in each task is marked in bold (except for "Full"). We also note the relative error of Twilight when integrated with the corresponding base algorithm. Green indicates an increase in score, while red indicates a decrease.}
\label{table:longbench}
    \centering
    \scalebox{0.69}{
    \begin{tabular}{lcccccccccccccc}
        \toprule
        \multirow{2}*{\textbf{Methods}} &
        \multirow{2}*{\textbf{Budget }} &
        \multicolumn{2}{c}{\textbf{Single-Doc. QA}} & \multicolumn{3}{c}{\textbf{Multi-Doc. QA}} & \multicolumn{3}{c}{\textbf{Summarization}} & \multicolumn{1}{c}{\textbf{Few-shot}} & \multicolumn{2}{c}{\textbf{Code}} & \multicolumn{1}{c}{\textbf{Synthetic}} & \multirow{2}*{\textbf{Avg. Score}}  \\
        \cmidrule(lr){3-4}\cmidrule(lr){5-7}\cmidrule(lr){8-10} \cmidrule(lr){11-11} \cmidrule(lr){12-13} \cmidrule(lr){14-14} 
        & & \textit{Qasper} & \textit{MF-en} & \textit{HotpotQA} & \textit{2WikiMQA} &  \textit{Musique} & \textit{GovReport} & \textit{QMSum} & \textit{MultiNews} & \textit{TriviaQA} &  \textit{LCC} & \textit{Repobench-P} & \textit{PR-en} \\
        \midrule
        \multicolumn{15}{c}{\textsc{Longchat-7B-32k}} \\
        \midrule
        \multirow{2}*{Full} & 32k & 29.48 & 42.11 & 30.97 & 23.74 & 13.11 & 31.03 & 22.77 & 26.09 & 83.25 & 30.50 & 52.70 & 55.62 & 36.78 \\
         & \textbf{Twilight (Avg. 146)} & 31.74 & \textbf{43.91} & 33.59 & \textbf{25.65} & \textbf{13.93} & 32.19 & \textbf{23.15} & 26.30 & 85.14 & 34.50 & 54.98 & 57.12 & 38.52\textcolor{teal}{(+4.7\%)}\\
        \midrule
        \multirow{5}*{Quest}
         & 256 & 26.00 & 32.83 & 23.23 & 22.14 & 7.45 & 22.64 & 20.98 & 25.05 & 67.40 & 33.60 & 48.70 & 45.07 & 31.26 \\
      & 1024 & 31.63 & 42.36 & 30.47 & 24.42 & 10.11 & 29.94 & 22.70 & 26.39 & 84.21 & 34.5 & 51.52 & 53.95 & 36.85 \\
       & 4096 & 29.77 & 42.71 & 32.94 & 23.94 & 13.24 & 31.54 & 22.86 & 26.45 & 84.37 & 31.50 & 53.17 & 55.52 & 37.33 \\
        & 8192 & 29.34 & 41.70 & 33.27 & 23.46 & 13.51 & 31.18 & 23.02 & 26.48 & 84.70 & 30.00 & 53.02 & 55.57 & 37.10 \\
             & \textbf{Twilight (Avg. 131)} & 31.95 & 43.28 & 31.62 & 24.87 & 13.48 & \textbf{32.21} & 22.79 & 26.33 & 84.93 & 33.50 & 54.86 & 56.70 & 38.04\textcolor{teal}{(+2.5\%)} \\
        \midrule
    \multirow{5}*{DS}
         & 256 & 28.28 & 39.78 & 27.10 & 20.75 & 9.34 & 29.68 & 21.79 & 25.69 & 83.97 & 32.00 & 52.01 & 53.44 & 35.32 \\
      & 1024 & 30.55 & 41.27 & 30.85 & 21.87 & 7.27 & 26.82 & 22.95 & 26.51 & 83.22 & 31.50 & 53.23 & 55.50 & 35.96 \\
       & 4096 & 28.95 & 41.90 & 32.52 & 23.65 & 8.07 & 29.68 & 22.75 & \textbf{26.55} & 83.34 & 30.00 & 52.77 & 55.48 & 36.31 \\
        & 8192 & 29.05 & 41.42 & 31.79 & 22.95 & 12.50 & 30.44 & 22.50 & 26.43 & 83.63 & 30.50 & 52.87 & 55.33 & 36.62 \\
             & \textbf{Twilight (Avg. 126)} & \textbf{32.34} & 43.89 & \textbf{34.67} & 25.43 & 13.84 & 31.88 & 23.01 & 26.32 & \textbf{85.29} & \textbf{35.50} & \textbf{55.03} & \textbf{57.27} & \textbf{38.71}\textcolor{teal}{(+5.7\%)} \\
        \midrule
        \multicolumn{15}{c}{\textsc{Llama-3.1-8B-Instruct}} \\
        \midrule
        \multirow{2}*{Full} & 128k & 46.17 & 53.33 & 55.36 & 43.95 & 27.08 & 35.01 & 25.24 & 27.37 & 91.18 & 99.50 & 62.17 & 57.76 & 52.01 \\
         & \textbf{Twilight (Avg. 478)} & 43.08 & 52.99 & 52.22 & 44.83 & 25.79 & 34.21 & \textbf{25.47} & 26.98 & 91.85 & \textbf{100.00} & \textbf{64.06} & 58.22 & 51.64\textcolor{red}{(-0.7\%)} \\
        \midrule
        \multirow{5}*{Quest}
         & 256 & 24.65 & 37.50 & 30.12 & 23.60 & 12.93 & 27.53 & 20.11 & 26.59 & 65.34 & 95.00 & 49.70 & 45.27 & 38.20 \\
      & 1024 & 38.47 & 49.32 & 47.43 & 38.48 & 20.59 & 33.71 & 23.67 & 26.60 & 81.94 & 99.50 & 60.78 & 52.96 & 47.79 \\
       & 4096 & 43.97 & 53.64 & 51.94 & 42.54 & 24.00 & 34.34 & 24.36 & 26.75 & 90.96 & 99.50 & 62.03 & 55.49 & 50.79 \\
        & 8192 &\textbf{44.34} & 53.25 & 54.72 & 44.84 & \textbf{25.98} & 34.62 & 24.98 & 26.70 & 91.61 & \textbf{100.00} & 62.02 & 54.20 & 51.44 \\
         & \textbf{Twilight (Avg. 427)} & 43.44 & 53.2 & 53.77 & 43.56 & 25.42 & 34.39 & 25.23 & 26.99 & 91.25 & 100.0 & 63.55 & 58.06 & 51.57\textcolor{teal}{(+0.3\%)} \\
        \midrule
    \multirow{5}*{DS}
         & 256 & 38.24 & 49.58 & 43.38 & 31.98 & 15.52 & 33.40 & 24.06 & 26.86 & 84.41 & 99.50 & 53.28 & 48.64 & 45.74 \\
      & 1024 & 42.97 & \textbf{54.65} & 51.75 & 33.92 & 20.39 & 34.50 & 24.92 & 26.71 & \textbf{92.81} & 99.50 & 62.66 & 48.37 & 49.43 \\
       & 4096 & 43.50 & 53.17 & 54.21 & 44.70 & 23.14 & \textbf{34.73} & 25.40 & 26.71 & 92.78 & 99.50 & 62.59 & 51.31 & 50.98 \\
        & 8192 & 43.82 & 53.71 & 54.19 & \textbf{45.13} & 23.72 & 34.27 & 24.98 & 26.69 & 91.61 & \textbf{100.00} & 62.40 & 52.87 & 51.14 \\
             & \textbf{Twilight (Avg. 446)} & 43.08 & 52.89 & \textbf{54.68} & 44.86 & 24.88 & 34.09 & 25.20 & \textbf{27.00} & 91.20 & \textbf{100.00} & 63.95 & \textbf{58.93} & \textbf{51.73}\textcolor{teal}{(+1.2\%)} \\
\bottomrule
\end{tabular}
}
\end{table*}

\end{document}
\endinput
%%
%% End of file `sample-sigconf-authordraft.tex'.
