%%
%% This is file `sample-sigconf-authordraft.tex',
%% generated with the docstrip utility.
%%
%% The original source files were:
%%
%% samples.dtx  (with options: `all,proceedings,bibtex,authordraft')
%% 
%% IMPORTANT NOTICE:
%% 
%% For the copyright see the source file.
%% 
%% Any modified versions of this file must be renamed
%% with new filenames distinct from sample-sigconf-authordraft.tex.
%% 
%% For distribution of the original source see the terms
%% for copying and modification in the file samples.dtx.
%% 
%% This generated file may be distributed as long as the
%% original source files, as listed above, are part of the
%% same distribution. (The sources need not necessarily be
%% in the same archive or directory.)
%%
%%
%% Commands for TeXCount
%TC:macro \cite [option:text,text]
%TC:macro \citep [option:text,text]
%TC:macro \citet [option:text,text]
%TC:envir table 0 1
%TC:envir table* 0 1
%TC:envir tabular [ignore] word
%TC:envir displaymath 0 word
%TC:envir math 0 word
%TC:envir comment 0 0
%%
%% The first command in your LaTeX source must be the \documentclass
%% command.
%%
%% For submission and review of your manuscript please change the
%% command to \documentclass[manuscript, screen, review]{acmart}.
%%
%% When submitting camera ready or to TAPS, please change the command
%% to \documentclass[sigconf]{acmart} or whichever template is required
%% for your publication.
%%
%%
\documentclass[sigconf, screen, nonacm]{acmart}
%%
%% \BibTeX command to typeset BibTeX logo in the docs
\AtBeginDocument{%
  \providecommand\BibTeX{{%
    Bib\TeX}}}

%% Rights management information.  This information is sent to you
%% when you complete the rights form.  These commands have SAMPLE
%% values in them; it is your responsibility as an author to replace
%% the commands and values with those provided to you when you
%% complete the rights form.
\setcopyright{acmlicensed}
\copyrightyear{2018}
\acmYear{2018}
\acmDOI{XXXXXXX.XXXXXXX}
%% These commands are for a PROCEEDINGS abstract or paper.
\acmConference[Conference acronym 'XX]{Make sure to enter the correct
  conference title from your rights confirmation email}{June 03--05,
  2018}{Woodstock, NY}
%%
%%  Uncomment \acmBooktitle if the title of the proceedings is different
%%  from ``Proceedings of ...''!
%%
%%\acmBooktitle{Woodstock '18: ACM Symposium on Neural Gaze Detection,
%%  June 03--05, 2018, Woodstock, NY}
\acmISBN{978-1-4503-XXXX-X/2018/06}


%%
%% Submission ID.
%% Use this when submitting an article to a sponsored event. You'll
%% receive a unique submission ID from the organizers
%% of the event, and this ID should be used as the parameter to this command.
%%\acmSubmissionID{123-A56-BU3}

%%
%% For managing citations, it is recommended to use bibliography
%% files in BibTeX format.
%%
%% You can then either use BibTeX with the ACM-Reference-Format style,
%% or BibLaTeX with the acmnumeric or acmauthoryear sytles, that include
%% support for advanced citation of software artefact from the
%% biblatex-software package, also separately available on CTAN.
%%
%% Look at the sample-*-biblatex.tex files for templates showcasing
%% the biblatex styles.
%%

%%
%% The majority of ACM publications use numbered citations and
%% references.  The command \citestyle{authoryear} switches to the
%% "author year" style.
%%
%% If you are preparing content for an event
%% sponsored by ACM SIGGRAPH, you must use the "author year" style of
%% citations and references.
%% Uncommenting
%% the next command will enable that style.
%%\citestyle{acmauthoryear}

\usepackage{balance}
\usepackage{xcolor}
\usepackage{pgffor}
\usepackage[most]{tcolorbox} % For side text boxes
\usepackage{graphicx}       % For including images
\usepackage{booktabs}       % For better tables
\usepackage{amsmath}        % For mathematical equations
\usepackage{hyperref}       % For hyperlinks
\usepackage{tikz}
\usepackage{forest}

\newtcolorbox{designRecom}[2][]{%
    enhanced,
    breakable,
    colback=blue!3,          % Lighter blue background for better readability
    colframe=blue!40!black,  % Slightly lighter frame
    coltitle=white,          % Title text remains white for contrast
    colbacktitle=blue!30!black, % Lighter title background
    boxrule=0.6pt,           % Keep the box outline clear
    arc=3pt,                 % Rounded corners for a modern look
    fontupper=\small, % Maintain italicized content text
    left=8pt, right=8pt,      % Padding adjustments
    top=6pt, bottom=6pt,
    title=\textbf{Design Recommendation:}\\\textnormal{#2},
    #1
}


%%
%% end of the preamble, start of the body of the document source.
\begin{document}

%%
%% The "title" command has an optional parameter,
%% allowing the author to define a "short title" to be used in page headers.
\title{The Design Space of Recent AI-assisted Research Tools for Ideation, Sensemaking, and Scientific Creativity}

%%
%% The "author" command and its associated commands are used to define
%% the authors and their affiliations.
%% Of note is the shared affiliation of the first two authors, and the
%% "authornote" and "authornotemark" commands
%% used to denote shared contribution to the research.
\author{Runlong (Harry) Ye}
\orcid{0000-0003-1064-2333}
\email{harryye@cs.utoronto.ca}
\affiliation{
  \institution{Computer Science, \\University of Toronto}
  \city{Toronto}
  \state{Ontario}
  \country{Canada}
}

\author{Matthew Varona}
\orcid{0009-0005-6201-973X}
\email{varona@cs.toronto.edu}
\affiliation{
  \institution{Computer Science, \\University of Toronto}
  \city{Toronto}
  \state{Ontario}
  \country{Canada}
}

\author{Oliver Huang}
\orcid{0009-0007-1585-1229}
\email{oliver@cs.toronto.edu}
\affiliation{
  \institution{Computer Science, \\University of Toronto}
  \city{Toronto}
  \state{Ontario}
  \country{Canada}
}

\author{Patrick Yung Kang Lee}
\orcid{0000-0002-3385-5756}
\email{patricklee@cs.toronto.edu}
\affiliation{
  \institution{Computer Science, \\University of Toronto}
  \city{Toronto}
  \state{Ontario}
  \country{Canada}
}

\author{Michael Liut}
\orcid{0000-0003-2965-5302}
\email{michael.liut@utoronto.ca}
\affiliation{
  \institution{Mathematical and Computational Sciences, \\University of Toronto Mississauga}
  \city{Mississauga}
  \state{Ontario}
  \country{Canada}
}

\author{Carolina Nobre}
\orcid{0000-0002-2892-0509}
\email{cnobre@cs.toronto.edu}
\affiliation{
  \institution{Computer Science, \\University of Toronto}
  \city{Toronto}
  \state{Ontario}
  \country{Canada}
}

%%
%% By default, the full list of authors will be used in the page
%% headers. Often, this list is too long, and will overlap
%% other information printed in the page headers. This command allows
%% the author to define a more concise list
%% of authors' names for this purpose.
\renewcommand{\shortauthors}{Trovato et al.}

%%
%% The abstract is a short summary of the work to be presented in the
%% article.
\begin{abstract}
Large language model (LLM)-based agents have shown promise in tackling complex tasks by interacting dynamically with the environment. 
Existing work primarily focuses on behavior cloning from expert demonstrations and preference learning through exploratory trajectory sampling. However, these methods often struggle in long-horizon tasks, where suboptimal actions accumulate step by step, causing agents to deviate from correct task trajectories.
To address this, we highlight the importance of \textit{timely calibration} and the need to automatically construct calibration trajectories for training agents. We propose \textbf{S}tep-Level \textbf{T}raj\textbf{e}ctory \textbf{Ca}libration (\textbf{\model}), a novel framework for LLM agent learning. 
Specifically, \model identifies suboptimal actions through a step-level reward comparison during exploration. It constructs calibrated trajectories using LLM-driven reflection, enabling agents to learn from improved decision-making processes. These calibrated trajectories, together with successful trajectory data, are utilized for reinforced training.
Extensive experiments demonstrate that \model significantly outperforms existing methods. Further analysis highlights that step-level calibration enables agents to complete tasks with greater robustness. 
Our code and data are available at \url{https://github.com/WangHanLinHenry/STeCa}.
\end{abstract}

%%
%% The code below is generated by the tool at http://dl.acm.org/ccs.cfm.
%% Please copy and paste the code instead of the example below.
%%
\begin{CCSXML}
<ccs2012>
   <concept>
       <concept_id>10003120.10003121</concept_id>
       <concept_desc>Human-centered computing~Interactive systems and tools</concept_desc>
       <concept_significance>500</concept_significance>
   </concept>
   <concept>
       <concept_id>10003120.10003123</concept_id>
       <concept_desc>Human-centered computing~Interaction techniques</concept_desc>
       <concept_significance>500</concept_significance>
   </concept>
   <concept>
       <concept_id>10003120.10003122</concept_id>
       <concept_desc>Human-centered computing~Empirical studies in HCI</concept_desc>
       <concept_significance>500</concept_significance>
   </concept>
</ccs2012>
\end{CCSXML}

\ccsdesc[500]{Human-centered computing~Interactive systems and tools}
\ccsdesc[500]{Human-centered computing~Interaction techniques}
\ccsdesc[500]{Human-centered computing~Empirical studies in HCI}

%%
%% Keywords. The author(s) should pick words that accurately describe
%% the work being presented. Separate the keywords with commas.
\keywords{information seeking; multilevel exploration; sensemaking; levels of abstraction; abstraction hierarchy; large language models; systems thinking; human-AI interaction}
%% A "teaser" image appears between the author and affiliation
%% information and the body of the document, and typically spans the
%% page.


%%
%% This command processes the author and affiliation and title
%% information and builds the first part of the formatted document.
\maketitle

\section{Introduction}

Despite the remarkable capabilities of large language models (LLMs)~\cite{DBLP:conf/emnlp/QinZ0CYY23,DBLP:journals/corr/abs-2307-09288}, they often inevitably exhibit hallucinations due to incorrect or outdated knowledge embedded in their parameters~\cite{DBLP:journals/corr/abs-2309-01219, DBLP:journals/corr/abs-2302-12813, DBLP:journals/csur/JiLFYSXIBMF23}.
Given the significant time and expense required to retrain LLMs, there has been growing interest in \emph{model editing} (a.k.a., \emph{knowledge editing})~\cite{DBLP:conf/iclr/SinitsinPPPB20, DBLP:journals/corr/abs-2012-00363, DBLP:conf/acl/DaiDHSCW22, DBLP:conf/icml/MitchellLBMF22, DBLP:conf/nips/MengBAB22, DBLP:conf/iclr/MengSABB23, DBLP:conf/emnlp/YaoWT0LDC023, DBLP:conf/emnlp/ZhongWMPC23, DBLP:conf/icml/MaL0G24, DBLP:journals/corr/abs-2401-04700}, 
which aims to update the knowledge of LLMs cost-effectively.
Some existing methods of model editing achieve this by modifying model parameters, which can be generally divided into two categories~\cite{DBLP:journals/corr/abs-2308-07269, DBLP:conf/emnlp/YaoWT0LDC023}.
Specifically, one type is based on \emph{Meta-Learning}~\cite{DBLP:conf/emnlp/CaoAT21, DBLP:conf/acl/DaiDHSCW22}, while the other is based on \emph{Locate-then-Edit}~\cite{DBLP:conf/acl/DaiDHSCW22, DBLP:conf/nips/MengBAB22, DBLP:conf/iclr/MengSABB23}. This paper primarily focuses on the latter.

\begin{figure}[t]
  \centering
  \includegraphics[width=0.48\textwidth]{figures/demonstration.pdf}
  \vspace{-4mm}
  \caption{(a) Comparison of regular model editing and EAC. EAC compresses the editing information into the dimensions where the editing anchors are located. Here, we utilize the gradients generated during training and the magnitude of the updated knowledge vector to identify anchors. (b) Comparison of general downstream task performance before editing, after regular editing, and after constrained editing by EAC.}
  \vspace{-3mm}
  \label{demo}
\end{figure}

\emph{Sequential} model editing~\cite{DBLP:conf/emnlp/YaoWT0LDC023} can expedite the continual learning of LLMs where a series of consecutive edits are conducted.
This is very important in real-world scenarios because new knowledge continually appears, requiring the model to retain previous knowledge while conducting new edits. 
Some studies have experimentally revealed that in sequential editing, existing methods lead to a decrease in the general abilities of the model across downstream tasks~\cite{DBLP:journals/corr/abs-2401-04700, DBLP:conf/acl/GuptaRA24, DBLP:conf/acl/Yang0MLYC24, DBLP:conf/acl/HuC00024}. 
Besides, \citet{ma2024perturbation} have performed a theoretical analysis to elucidate the bottleneck of the general abilities during sequential editing.
However, previous work has not introduced an effective method that maintains editing performance while preserving general abilities in sequential editing.
This impacts model scalability and presents major challenges for continuous learning in LLMs.

In this paper, a statistical analysis is first conducted to help understand how the model is affected during sequential editing using two popular editing methods, including ROME~\cite{DBLP:conf/nips/MengBAB22} and MEMIT~\cite{DBLP:conf/iclr/MengSABB23}.
Matrix norms, particularly the L1 norm, have been shown to be effective indicators of matrix properties such as sparsity, stability, and conditioning, as evidenced by several theoretical works~\cite{kahan2013tutorial}. In our analysis of matrix norms, we observe significant deviations in the parameter matrix after sequential editing.
Besides, the semantic differences between the facts before and after editing are also visualized, and we find that the differences become larger as the deviation of the parameter matrix after editing increases.
Therefore, we assume that each edit during sequential editing not only updates the editing fact as expected but also unintentionally introduces non-trivial noise that can cause the edited model to deviate from its original semantics space.
Furthermore, the accumulation of non-trivial noise can amplify the negative impact on the general abilities of LLMs.

Inspired by these findings, a framework termed \textbf{E}diting \textbf{A}nchor \textbf{C}ompression (EAC) is proposed to constrain the deviation of the parameter matrix during sequential editing by reducing the norm of the update matrix at each step. 
As shown in Figure~\ref{demo}, EAC first selects a subset of dimension with a high product of gradient and magnitude values, namely editing anchors, that are considered crucial for encoding the new relation through a weighted gradient saliency map.
Retraining is then performed on the dimensions where these important editing anchors are located, effectively compressing the editing information.
By compressing information only in certain dimensions and leaving other dimensions unmodified, the deviation of the parameter matrix after editing is constrained. 
To further regulate changes in the L1 norm of the edited matrix to constrain the deviation, we incorporate a scored elastic net ~\cite{zou2005regularization} into the retraining process, optimizing the previously selected editing anchors.

To validate the effectiveness of the proposed EAC, experiments of applying EAC to \textbf{two popular editing methods} including ROME and MEMIT are conducted.
In addition, \textbf{three LLMs of varying sizes} including GPT2-XL~\cite{radford2019language}, LLaMA-3 (8B)~\cite{llama3} and LLaMA-2 (13B)~\cite{DBLP:journals/corr/abs-2307-09288} and \textbf{four representative tasks} including 
natural language inference~\cite{DBLP:conf/mlcw/DaganGM05}, 
summarization~\cite{gliwa-etal-2019-samsum},
open-domain question-answering~\cite{DBLP:journals/tacl/KwiatkowskiPRCP19},  
and sentiment analysis~\cite{DBLP:conf/emnlp/SocherPWCMNP13} are selected to extensively demonstrate the impact of model editing on the general abilities of LLMs. 
Experimental results demonstrate that in sequential editing, EAC can effectively preserve over 70\% of the general abilities of the model across downstream tasks and better retain the edited knowledge.

In summary, our contributions to this paper are three-fold:
(1) This paper statistically elucidates how deviations in the parameter matrix after editing are responsible for the decreased general abilities of the model across downstream tasks after sequential editing.
(2) A framework termed EAC is proposed, which ultimately aims to constrain the deviation of the parameter matrix after editing by compressing the editing information into editing anchors. 
(3) It is discovered that on models like GPT2-XL and LLaMA-3 (8B), EAC significantly preserves over 70\% of the general abilities across downstream tasks and retains the edited knowledge better.
\section{Method}

Our literature search focused on AI-assisted creative research tools in contrast to AI-assisted writing tools, creative research tools assist with the co-creation of concepts and ideas in the research process rather than merely improving stylistic or rhetorical choices in written research. In total, we surveyed 11 systems papers published in top HCI venues (i.e., CHI, CSCW, UIST, and ToCHI) over the last three years (2022-2024)\footnote{We include a relevant CHI 2025 paper made available on arXiv.}; details can be found in Table~\ref{tab:tool_classification}. 

As LLMs became widely used in 2022 with the release of ChatGPT \cite{openai2022chatgpt}, this timeframe was chosen to reflect the period of significant growth in LLM popularity and adoption, allowing us to capture the most relevant and impactful developments in GenAI-driven research tools. Of the systems surveyed, six of the systems integrate LLM-based functionalities, while the other five represent a more traditional AI approach and employ machine learning techniques (e.g., Seq2Seq, BERT, RNN). By examining both GenAI and traditional AI approaches, we aim to understand to what extent GenAI tools represent a fundamental shift in capabilities and design considerations compared to more established AI approaches.

The thematic dimensions presented in Section~\ref{sec:design-space} resulted from an iterative process among the authors. We engaged in extensive internal discussions and consulted with an external expert specializing in knowledge spaces and the role of AI in fostering creativity and sensemaking. This collaborative and iterative approach resulted in the four dimensions presented in the next section.
\section{Design Space}
\label{sec:design-space}

% \subsection{Is the User Actively Engaged with the Tool?}

A fundamental challenge in integrating AI-powered research tools is ensuring that users remain cognitively ``in the loop'' rather than passively accepting AI-generated output. While automation promises efficiency and rapid idea generation, it also risks encouraging users to rely too heavily on system outputs, which can diminish human critical thinking and decision-making processes. We explore four design dimensions critical to navigating this tension.

% We explore 4 design dimensions that inform cognitive engagement when developing AI-assisted research tools: User Agency \& Control, Divergent \& Convergent Thinking, Adaptability, and Accuracy. In doing so, we highlight design patterns that promote meaningful human-AI collaboration while identifying potential pitfalls that could limit researcher agency, cognitive engagement, or trust in AI-driven insights.

% We expect a tool with good user agency that allows users to intervene, override, or refine AI outputs. They take an active role in evaluating and steering the research process. This active involvement forces researchers to check the validity of automated suggestions. It helps to strengthen their critical thinking and decision-making. For example, systems like CoQuest require users to constantly evaluate AI outputs and guide the research process \cite{CoQuest}

% The balance of AI contribution in divergent and convergent thinking affects cognition by guiding how researchers balance searching for new ideas with refining those ideas. The tools should support both creative exploration and systematic analysis. This dual approach keeps users mentally agile and engaged. In brainwriting sessions, participants must justify their choices and debate various ideas, which reinforces both divergent and convergent modes of thought \cite{brainwriting}. Similarly, IdeaSynth uses clarifying questions to prompt deeper idea refinement \cite{IdeaSynth}.

% We define adaptability as a tool’s flexibility to support different research tasks and user preferences. A tool that adapts to diverse approaches aligns with a user’s natural cognitive strategies. This design enhances various problem-solving styles. It lets users work in a way that suits their thinking patterns.

% Lastly, accuracy is a crucial consideration for building trust in a system. A tool must deliver reliable and contextually relevant information. When researchers trust the tool, they integrate its insights with confidence. For example, Paperweaver directly provide the transparency on all relevant information that researchers can depend on \cite{paperweaver}. This trust leads to more profound results and focuses cognitive effort on meaningful analysis rather than refining LLM-generated outputs. 

% These dimensions create a framework that continuously shapes cognitive engagement through design and functionality. LLM tool designs must balance these factors to enhance human cognition. Their goal is to keep users active and critical throughout the research process. The following subsections will examine all existing tools under each dimension and discuss how they contribute to overall cognitive engagement.


% "is user actively engaged in the work"
% We want to discuss how current tools engage users, passively or actively. The design challenge is to ensure that users remain cognitively “in the loop,” regardless of the automation level, especially on critical/high-stake intersections.

% TODO: we don't need this many example; a couple of high-level ones would be enough; we will mainly use the rest of the dimensions to articulate the appropriate amount of cognitive engagement.


% \subsubsection{Example of good use cases}
% \begin{itemize}
%     \item \cite{AnalogicalSearchEngine} The core design principle of the system is to surface articles that are \textit{analogically} relevant, potentially from very different domains. The fact that user need to be cognitively active to search for relavant article. And engaged in search results and refine the search query again.
    
%     \item \cite{brainwriting} Convergence Stage: The discussion and selection process necessitates active cognitive engagement. Participants must justify their choices, debate the merits of different ideas, and synthesize information from both human and AI sources.
    
%     \item \cite{CoQuest} RQ Flow Editor: The user is constantly required to evaluate the AI's output, provide feedback, and make decisions about the direction of the RQ development. This is not a passive consumption of AI-generated content; it's an active, intellectually demanding process. The user must remain "in the loop" to guide the AI.

%     \item \cite{CoQuest} AI Thoughts Panel: is specifically design for explaining LLM's reasoning and rationale. This is to engage user actively, and make sense of system's decision making process. Also, the wait time introduced, indirectly force user to engage with other component/interaction, further increase Cognitive Engagement.

%     \item \cite{IdeaSynth} Node-Based Canvas Visualization (Fig. 2, 4): The canvas interface with interconnected nodes provides a visual representation of the research idea's structure and development. his externalizes the user's thought process, making it easier to see connections, identify gaps, and explore different variations. It promotes active cognitive engagement by requiring users to organize and structure their ideas. The visual representation can reduce cognitive load compared to a purely text-based approach.

%     \item \cite{IdeaSynth} Clarifying Questions from AI (Fig 4, 5): The AI prompts users to clarify aspects of their ideas. This forces users to actively think about the details and nuances of their ideas, promoting deeper cognitive engagement. It's not just passive reception of suggestions; it's an interactive process of refinement.

%     \item \cite{paperweaver} This directly addresses cognitive engagement by providing concise, relevant information. It helps users quickly assess relevance without reading the full abstract, addressing the problem of information overload highlighted in the paper's introduction and formative study. It moves beyond simple keyword matching to provide a more semantically meaningful connection.

%     \item \cite{scholastic} Document Map: The Document Map encourages active cognitive engagement. Users are not passively presented with results; they must actively explore the clusters, hover for information, and make decisions about which documents to sample. The visual metaphor of a map encourages exploration and the formation of mental models. (Section 6.1)

%     \item \cite{scholastic} Document Reader: The Document Reader promotes active engagement through the coding process. Highlighting text and selecting keywords requires careful reading and interpretation. The ability to add memos further encourages reflection and documentation of the user's reasoning. (Section 6.2)

%     \item \cite{scholastic} Code Examiner: The Code Examiner encourages active engagement by requiring users to navigate the hierarchical word clusters and make decisions about which documents to sample based on their codes. The ability to refine codes and categories further promotes reflection and iterative analysis. (Section 6.3)

%     \item \cite{synergi} Combined PDF Viewer and Highlighter. Requires active reading and selection of relevant text. (Section 3.1 and shown in Figure 3A). Highlighting is an active reading strategy. The user must constantly evaluate the text for relevance to their research question.

%     \item \cite{synergi} Drag-and-Drop Outline Editor: Requires the user to actively synthesize and organize information. (Section 4.4 and Figure 5) This is not a passive activity. The user must make decisions about the structure of their outline, the relevance of different threads, and the overall narrative of their literature review.

%     \item \cite{threddy} Active Reading and Highlighting: The core interaction of Threddy – highlighting text and extracting references – requires active reading and engagement with the source material. Users are not passively consuming; they are actively selecting and curating.

%     \item \cite{threddy} Thread Organization: The drag-and-drop interface for organizing threads forces users to actively consider the relationships between different pieces of information and construct a mental model. This promotes deeper understanding.


% \end{itemize}

% \subsubsection{Mixed}
% \begin{itemize}
%     \item \cite{CoQuest}Paper Graph Visualizer: provides the tools for cognitive engagement (access to papers, summaries, citation links), but it doesn't actively enforce or guide users towards critical thinking. It relies on the user's own initiative and research skills to avoid the pitfalls you've mentioned. This is a significant difference. It's a good tool, but it's not inherently good for cognitive engagement if used passively. Fail to consider alternative perspectives: The tool might present a biased or incomplete view of the literature. Users might not actively seek out papers outside the Tool's suggestions. Skip deep reading because it's readily accessible: User might engage superficially, and think they have fully understood the main idea, while skipping deeper cognitive engagement.

%     \item \cite{disciplink} By presenting EQs rather than definitive answers, the system encourages users to think critically about different perspectives. The theme extraction and information scent features help users process information efficiently without getting lost in details. But passively present papers may be risk. 


%     \item \cite{IdeaSynth} Edge-Based Connection Validation (Fig.4): User can expand the edge between two nodes, and see how the connection can be improved. 

%     \cite{threddy} Recommendation: While recommendations are presented, the user must actively evaluate their relevance. This requires cognitive engagement, but the limited control over the recommendation criteria means this engagement might be less efficient or focused than it could be. The "FOMO" expressed by one participant (Section 6.2) suggests that the recommendations, while engaging, might also be overwhelming.
% \end{itemize}

% \subsubsection{Example of bad use cases}
% \begin{itemize}
%     \item \cite{brainwriting} Divergence stage: risk of over-reliance on the LLM, leading to passive acceptance of its suggestions. If users become "prompt engineers" rather than "idea generators," their cognitive engagement shifts from understanding the problem domain to understanding the LLM's quirks. 
%     \item 
% \end{itemize}

\subsection{User Agency and Control}
\label{sec:agency}

Human-AI collaboration systems aim to improve productivity and creativity by offloading certain tasks from humans to AI systems while keeping the user in the driver's seat. Providing user agency and control via source material participation, the ability to refine AI output, and the ability to reject and override automated actions are crucial to maintaining cognitive engagement from the user.

\subsubsection{Engagement with source material.} For tools aimed at making sense of existing text content (such as qualitative coding, thematic analysis, or literature reviews), there is a delicate balance to be struck between AI assistance and user agency. While recent advancements in LLM capabilities show promise in processing and summarizing large amounts of text, users still have to engage with the source text to build their own understanding and avoid model overreliance. One way to foster engagement is for user-selected text highlights to drive content generation. Systems such as Synergi~\cite{synergi} and Threddy~\cite{threddy} integrate PDF readers that transform user-selected highlights into seeds for AI-driven research thread generation, while Relatedly~\cite{Relatedly} further refine this process by organizing and highlighting overlapping research themes. Meanwhile, platforms like Scholastic~\cite{scholastic} and SenseMate~\cite{sensemate} scaffold existing analysis methods with cluster suggestions and strategic sampling. The role of AI in these systems (such as Scholastic~\cite{scholastic} and SenseMate~\cite{sensemate}) is to scaffold existing analysis methods with cluster suggestions and strategic sampling.

\subsubsection{Refining AI output.} Human-centric AI tool design assumes the user is the expert, giving them the power to modify the output provided by the AI system. Most tools we reviewed allow users to edit and refine AI-generated artifacts to ensure that they meet the user's goals. In Threddy~\cite{threddy}, for example, users can manually clean up errors in references and links extracted from a paper snippet. Beyond fixing errors, user editing can be designed into human-AI sensemaking systems to varying degrees. Arranging AI outputs into a more interpretable structure (such as node-link diagrams~\cite{IdeaSynth, CoQuest} or outlines~\cite{synergi}) can foster deeper engagement with suggestions. Finally, iterating on prompts and queries can help users gradually incorporate new ideas and discoveries into AI output. This is notably useful in cases such as the Analogical Search Engine~\cite{AnalogicalSearchEngine}, where users may not know exactly how to initially prompt but can gain and apply new information with each re-prompt. 

% Systems can provide the user with multiple ways of engaging with and refining AI suggestions. In Synergi~\cite{synergi}, for instance, users extract clips from papers which are then used to recommend research threads that map to the broader literature. The resulting threads can be curated and reorganized into new research outlines in an iterative process.

\subsubsection{Rejection and overriding.} One aspect of control that merits further consideration is the ability of users to reject, override, or ignore model output. Rejection can manifest in systems implicitly. For example, editing or curating AI output implies rejection of the original content in part or whole. However, most of the research papers we reviewed had AI assistance embedded in the system, with minimal ability to ``turn off'' AI suggestions. One notable exception is SenseMate, which explicitly aims to provide AI theme suggestions on demand rather than by default~\cite{sensemate}. In SenseMate, AI suggestions are hidden by default; users can also see the reasoning for the theme suggestions and explicitly reject them. Similarly, Scholastic's text clustering algorithm does not impose keywords on clusters. Instead, users have the option of either developing internal mental models of the meaning of the cluster or providing the algorithm with explicit codes that can be iterated on~\cite{scholastic}.

\begin{designRecom}{Provide User Agency and Control}
\textit{Design Insight:} Users should have meaningful control and agency within AI-assisted systems to ensure that they remain the primary decision-makers. Users should be able to engage with the source text and edit, refine, or reject AI output.
\smallskip

\textit{Implications:} 
\begin{itemize} 
\item \textbf{Customizable AI Assistance:} Users should have control over when and how AI assistance is applied, including the ability to disable suggestions when not needed.
\item \textbf{Iterative Refinement:} Enabling users to edit and refine AI output can lead to more relevant and intentional output.
\item \textbf{Engage the User:} Support user understanding and agency by giving users control over the structure of collaboratively generated insights and providing access to source text where relevant.
\end{itemize}

% \textit{Recommendation example:} Disciplink~\cite{disciplink} highlights a limitation where users can guide the generation of new exploratory research questions but cannot directly edit existing ones. Expanding user control to include full-text editing would enhance agency, foster iterative refinement, and enable LLM suggestions to evolve alongside the user's understanding and mental model.
\end{designRecom}

% \begin{itemize}
%     \item \cite{AnalogicalSearchEngine} The search system gives the user full freedom to continue progressing, user does not need to rely only on this search result.
%     \item \cite{brainwriting} The freeform and plug-and-play LLM usage gives full control to the user. Also, in the Convergence Stage: The selection process is entirely human-driven. The LLM is used as a tool for further development after humans have made the key decisions about which ideas are worth pursuing. This maintains a high level of user agency, as the humans are "calling the shots" on which direction the ideation process takes.
%     \item \cite{IdeaSynth} Facet Node Creation and Editing (Fig. 5): Users can create, title, and edit the content of idea facet nodes (Problem Description, Solution, Evaluation, Contribution). Users are in full control of expressing their ideas within the defined facets. This allows for personalized input and avoids the feeling of being dictated to by the AI.
%     \item \cite{IdeaSynth} Manual Node Linking (Fig. 5)Users can manually link nodes to create relationships between idea facets. This is a good use case. This empowers users to define the structure of their research idea, representing their mental model. It's not forced by the AI, giving the user ultimate control over the conceptual relationships.

%     \item \cite{IdeaSynth} Adopting or Rejecting AI Suggestions (Fig. 4, 5): Users can choose to accept, modify, or ignore AI-generated suggestions for node content, alternatives, and expansions. User is not forced to accept AI output. The system acts as an assistant, offering options, but the user makes the final decision. This prevents overreliance on the AI and maintains user control over the ideation process.

%     \item \cite{IdeaSynth} Literature Search and Collection Management (Fig 3): User can search for relevant scientific papers, add to personal collection, and remove papers from collection, or get AI recommendation if needed. User decide what literature background is relevant to the current ideation context.

%     \item \cite{paperweaver} Suggested Topic Description (Section 4.2.1): The system allow user's direct input for folder description, or generate an default one using LLM. They can override the suggestion, refining it to better reflect their interests.

%     \item \cite{scholastic} Document Sampling (Breadth-First - Document Map): Users can choose to sample randomly (initial exploration) or strategically based on the visualization of document clusters (geographical treemap). They control the granularity of the clusters using a slider. The ability to hover for titles and right-click for previews gives users control over which documents to examine further. The system does not force a particular sampling method. This aligns well with interpretive research, where the researcher's judgment is important.

%     \item \cite{scholastic} Document Sampling (Depth-First - Code Examiner): The Code Examiner allows depth-first sampling based on user-defined codes and categories. Users select codes, and the system presents documents ranked by relevance to those codes (using the underlying model). This provides high agency, as the sampling is directly driven by the researcher's evolving understanding. (Section 6.3, Figure 6) This reinforces user control by allowing them to directly leverage their coding schema to find relevant documents. It's a form of querying the corpus based on the user's interpretive framework.

%     \item \cite{sensemate} Theme Recommendations (On-Demand): AI suggests themes for each unit of analysis (snippet of text). These are not automatically applied. Users must explicitly choose to view and then accept, reject, or mark them as unsure. The system presents themes one at a time to encourage thoughtful review. Note: SenseMate is designed to help users apply this existing codebook more efficiently and consistently, not to develop a new one from scratch. The focus is on deductive coding, where a pre-defined codebook is used, rather than inductive coding, where codes are developed from the data. The authors state: "High rates of intercoder reliability are especially helpful during deductive coding where sensemakers start with a predefined set of thematic codes, which are assigned to qualitative data. The AI support in SenseMate can help users stay grounded in the codebook and minimize coding errors, which can speed up the analysis." (Page 12)

%     \item \cite{sensemate} Rationale Explanations: For each theme recommendation, users can click a "View Reason" button to see why the AI suggested that theme. The rationale is presented as bolded words within the original text snippet (Figure 3). 

%     \item \cite{sensemate}  Feedback Mechanisms (Quick Questions \& Highlighting): If a user rejects a recommendation, they are presented with up to five yes/no questions about whether specific phrases in the rationale are relevant (Figure 4). Users can select or deselect words within the rationale to indicate which parts are relevant or irrelevant (Figure 5). This allows for more nuanced feedback than simple acceptance/rejection. (page 8). feedback system is excellent for several reasons. It increases user agency by allowing them to directly influence the model's behavior (although retraining wasn't implemented in this version). It's adaptable because it provides different levels of feedback granularity (quick questions vs. highlighting). It fosters cognitive engagement by making users actively participate in refining the AI's understanding. \textbf{also in cog engagement (user engage in refining / enhance models), adaptability (provides different levels of feedback granularity (quick questions vs. highlighting), accuracy (refine existing coding to enhance model accuracy))}

    
%     \item \cite{synergi} Outline Editor: The user can drag and drop threads, edit labels, remove irrelevant information, and add their own insights. This allows for direct manipulation and customization, restoring agency after the relatively low-agency thread generation step.

%     \item \cite{threddy} Highlighting and Selection: Threddy allows users to highlight text and areas within PDFs, triggering the extraction of references. This gives users direct control over the starting point of thread creation. The user actively chooses what content is relevant. (Section 4, "Highlighting and Selection")

%     \item \cite{threddy} Reference Deselection: Within the holding tank, users can "deselect any reference they do not want to include or to fix any extraction error." (Section 4) This provides fine-grained control over the automatically extracted information, allowing users to curate the input to the thread.

%     \item \cite{threddy} Thread Creation and Organization: Users can create new threads, add content to existing threads, and (re-)organize threads using drag-and-drop. This offers significant control over the structure and content of their knowledge organization. (Section 4, "Organization")
    

% \end{itemize}

% \subsubsection{Mixed use cases}
% \begin{itemize}
%     \item \cite{CoQuest} The user has direct input at multiple stages. They initiate the process with keywords, provide feedback to refine AI suggestions, and can directly edit the structure of the RQ flow by adding, deleting, or moving nodes. The concern of artificially limiting the expansion of related RQ can be mitigated by re-prompting and re-generating. However, users may still experience unintentional bias from LLM-recommended RQs, and users do not have the option to override manually.
    
%     \item \cite{disciplink} The lack of direct EQ editing, addition, and deletion is a limitation. The user's control is primarily through influencing the generation of new EQs, not manipulating existing ones. This is less agency than a system that would allow full text editing of EQs.

%     \item \cite{scholastic} Coding (Document Reader): Users have high agency in applying codes. They highlight text, apply existing codes, or create new ones. They can also add memos, which is crucial for interpretive research. The "in vivo" keyword selection, while offering some automation, is still under user control (they choose which keywords to select). But user cannot apply multiple code at the same time on a single selection, it is acknowledge as a limitation by original author.

%     \item \cite{synergi} Loopy Belief Propagation (LBP) for Retrieval: The user sets the starting point, but the algorithm determines the expansion. There's no mention of parameters the user can adjust to tune the LBP process (e.g., weighting factors, stopping criteria). This lack of transparency and fine-grained control reduces agency. (Fig 1-B, Section 4.1)

%     \item \cite{synergi} Hierarchical Thread Generation:  The user receives a pre-structured hierarchy generated by GPT-4. While useful for overview, the user has limited input on how this hierarchy is created. The user doesn't choose the clustering method, similarity thresholds, or summarization prompts. This is a "black box" from the user's perspective. It's efficient, but the lack of control could lead to issues if the generated hierarchy doesn't align with the user's mental model. (Fig 1-C, Section 4.2)

%     \item \cite{threddy} Thread Selector Suggestions: The thread selector uses an algorithm to suggest the most relevant thread for new content. (Appendix A). While helpful, this is a recommendation, not a forced action. The user retains the ultimate choice. However, the algorithm's reliance on similarity metrics (and the potential for "high dispersion, low cohesion" as noted in Appendix A) could lead to less-than-ideal suggestions, subtly nudging users in certain directions.


% \end{itemize}
% \subsubsection{Example of bad use cases}
% \begin{itemize}
%     \item \cite{brainwriting} the LLM only suggests relevant diverging topics, without clear reason on the process. If it produces redundant, irrelevant, or biased output (as some students reported), the users can only try different prompts – they can't "debug" the AI's reasoning. Prone to automation bias.
%     \item \cite{disciplink} While users have control over EQs, the process of query generation is somewhat opaque. Section 4.2.2 describes query expansion using LLMs, but the user doesn't directly see or modify these queries. This is a potential limitation, as users might want more transparency or control over the search terms used, especially in unfamiliar disciplines.

%     \item \cite{threddy} Recommendation Control: The "Overview and Discovery" panel provides recommendations based on citation coverage. While users can click "Refresh" to regenerate recommendations, they have limited direct control over the recommendation criteria. The paper mentions a desire for "additional mechanisms for specifying which context is personally more important" (Section 7.1), acknowledging this limitation. The user can only accept or reject recommendations, not shape how they are generated. This is a significant weakness. They can't, for example, prioritize recency, specific authors, or keywords.
% \end{itemize}

\subsection{Divergent and Convergent Thinking}
\label{sec:divergent-convergent}
% Talk about when AI tool helps in divergent thinking (to expand space) and convergent thinking (to filter and narrow down). We try to show that AI can be useful to take over more work in divergent stages, so users can be exposed to more previously unknown knowledge, but it is less okay if we are in convergent stages where the user should be taking charge in terms of direction and criteria.

% CoQuest: divergent thinking: balanced cognitive engagement, effective for both divergent and convergent tasks

% Alalogical Search engine: more cognitive engagement, user are asked to actively adpat and apply ideas from analogical articles, the ai are maining focused on divergent thinking 

% brainswriting: the system fosters collaboration and critical engagement with AI, it is designed to help users to generate many idea, which is AI partcipated in a divergent thinking

% Disciplink: divergent thinking, DiscipLink generates diverse exploratory questions (EQs) from various disciplines, and crafts queries to broaden the search, users can create their own EQs based on findings.

% ideasynth: ai contribute into divergent thinking, it helps users generate and connect research ideas by using AI to suggest alternative ideas and connections, AI gives suggestions on improving the idea. It also helps users refine their research questions. 
% also provides some convergent thinking, provides literature analysis to summarize and connect papers with the nodes.

% luminate: divergent thinking, users diverge by generating multiple responses from a single prompt, the system is designed to help with the generation of the design space rather than individual artifacts

% paperweaver: mixed approach, PaperWeaver surfaces connections between papers that may be relevant to the user, which helps to expand their view of the literature (divergent), and then the system provides contextualized descriptions and helps users decide which papers are worth saving (convergent)

% personalFlow: divergent thinking from ai, uses ai generated personas with different expertise, which helps users explore diverse perspectives and direction. then user can refine their research questions after evaluating personas. The ability to customize personas promotes better engagement with the system and helps users feel in control.

% scholastic: 

% sensemate: convergent thinking from ai, AI is used for semi-automating qualitative coding and supports transparency and modification. 

% synergi: divergent thinking, increase cursitisity from engagement, very good for broadening perspectives with citation graphs and LLM to epand and structure research threads.

% threddy: supports divergent thinking by helping users to expand their understanding of a research area through the discovery of new connections and threads of research, human are the ones that do convergent thinking by organizing information into threads, consolidate their knowledge and identity key themes.

% sensecape: mixed thinking, in the divergent phase, the AI generates a wide range of subtopics and questions that allows users to expand their understanding of a topic, in the convergent phase, the AI organizes these ideas into a hierarchical structure, which helps users filter and make sense of the information
\subsubsection{Divergent Thinking}
 Divergent thinking is essential for expanding the horizons of research. AI tools support this by generating exploratory questions and novel insights that researchers may not have thought about. For example, the Analogical Search Engine~\cite{AnalogicalSearchEngine}, DiscipLink~\cite{disciplink}, and IdeaSynth~\cite{IdeaSynth} provide functionalities to explore creative connections in seemingly unrelated fields. However, when using an AI tool for divergent thinking is used in isolation, this approach can overwhelm users with many novel ideas without providing the guidance necessary to narrow them down or refine them effectively.
\subsubsection{Convergent Thinking}
In convergent thinking, AI tools transform raw data into clear and structured insights, acting as intelligent partners that help researchers distill complex information into coherent narratives. These systems guide scholars in filtering and organizing diverse inputs so that essential themes emerge with clarity and precision. 

For example, SenseMate~\cite{sensemate} leverages rationale extraction models to generate transparent theme recommendations and human-interpretable explanations for qualitative coding. By grounding its suggestions in data-driven rationale rather than relying on large language models, SenseMate empowers even novice users to engage deeply with the source material while retaining full control over their coding decisions. Focusing on the same task, CollabCoder~\cite{CollabCoder} utilizes LLMs to automatically generate qualitative code suggestions and facilitate structured group discussions, thereby bridging individual insights into a collective consensus. Complementing these approaches, Scholastic~\cite{scholastic} employs advanced visual analytics to help teams organize and interpret complex datasets, while PaperWeaver~\cite{paperweaver} presents contextualized links that highlight the most relevant insights. These convergent functionalities make thematic grouping and filtering more efficient and transparent. 

\subsubsection{Mixed-Thinking}
Some tools strike a balance by supporting both divergent and convergent thinking. Threddy~\cite{threddy}, for example, allows users to input various requests so that LLM systems can organize ideas into coherent themes. They also leverage hierarchical structures to discover new connections based on user input. Similarly, Synergi~\cite{synergi} uses citation graphs and language models to expand research threads and consolidate them. However, these mixed-thinking approaches often leave users with limited control over the balance between exploration and refinement, which may hinder effective sensemaking. So while too much convergence might stifle creative exploration, too much divergence can cause cognitive overload. A well-designed research tool should support either or both modalities, but its user-facing capabilities should be made transparent. In the design space of research tools, allowing users to control the levels of divergent and convergent functionality is key to researcher autonomy.


\begin{designRecom}{Enabling Divergent and Convergent Thinking}
\textit{Design Insight:} Consider how users may need different types of support during the research process. Users may need more creative support from GenAI when they are discovering a new research area to help them understand the space. Following this, GenAI should provide guidance and structure when users are converging their existing research material.
\smallskip

\textit{Implications:} Future systems should be engineered with a design space that considers:

\begin{itemize}
\item \textbf{Transparency of AI Contributions:} Users should clearly understand whether the AI is supporting divergent exploration, convergent refinement, or both.  
\item \textbf{User Control Over AI Assistance:} Users should be able to adjust the level of AI support for idea generation and refinement based on their needs.  
\end{itemize}

% \textit{Recommendation example:} 
% CoQuest\cite{CoQuest} guides users through both expansive idea generation and systematic refinement.  The tool provides functionality for divergent thinking by generating exploratory questions and provides analogical search outputs to expose users to diverse research perspectives. CoQuest aggregates and refines these outputs in a convergent manner to help researchers critically evaluate and select the most relevant information.

\end{designRecom}


\subsection{Adaptability}
\label{sec:adaptability}
% "is there enough variation/ways  this tool can be used for?"
% We want to discuss how the research tools are adopting a flexible design philosophy, aiming to create systems that support different types of tasks, different workflows, and different personal preferences. It asks whether the tool can adjust its mode of operation depending on the context—say, leaning more on machine initiative when the task is exploratory and the user lacks domain expertise, but shifting to a mode that grants more manual control when the task requires careful evaluation or detailed refinement.
% \subsubsection{Example of good use cases}
% \begin{itemize}
%     \item \cite{AnalogicalSearchEngine} The fact of a search-based system makes it adaptable for a lot of unknown scenarios, and using "purposes" and "mechanisms" as a goal to prioritize make it even more adaptable, as the goal and way of doing will be captured for different research projects.
%     \item \cite{brainwriting} Since the process are entirely driven by user, including main topics, key diverging and converging steps. And it is not subject/process specific, user can plug-in and get additional AI assistance when needed.
%     \item \cite{CoQuest} RQ Flow Editor: The system doesn't restrict the user to predefined categories or specific input formats. This allows it to be used for a variety of research topics and at different stages of the research process. The user can start with a vague idea and gradually refine it, demonstrating adaptability to different user needs and workflows.

%     \item \cite{IdeaSynth} Literature-Based vs. Idea-Based Exploration: The system supports both literature-driven exploration (through paper recommendations and summaries) and idea-driven exploration (through facet generation and refinement) (section 4.1). 
%     \item \cite{IdeaSynth} Prompt Customization (Fig 5): The ability to customize prompts for AI suggestions allows for some adaptation to user needs.

%     \item \cite{scholastic} Support for different tasks and free workflow:  The tool is designed to support a complete qualitative analysis workflow, from initial exploration to the development of a refined coding schema. The tool allows for flexible workflows. Users can move freely between the Document Map, Document Reader, and Code Examiner. They can start with random sampling and then move to strategic sampling, or vice versa. They can code and categorize iteratively.

%     \item \cite{synergi} Mixed-Initiative Workflow: The system supports both bottom-up (starting from specific papers) and top-down (exploring the generated hierarchy) approaches.

%     \item \cite{synergi} Input Flexibility: user can focus and tag different part of the paper, depending on the interest and research goal. And thus can be expanded to facilitate different research need.

%     \item \cite{threddy} Multiple Entry Points: Users can start from any PDF and build threads from there. They are not forced into a specific starting point or workflow. This supports diverse research approaches.

%     \item \cite{threddy} Flexible Thread Structure: The nested thread structure, combined with the ability to reorganize threads, allows for a high degree of flexibility in how users organize their knowledge. This can adapt to different research styles and evolving mental models.
%  \end{itemize}

% \subsubsection{Mixed use cases}
% \begin{itemize}
%     \item \cite{disciplink} The system's ability to generate EQs from multiple disciplines is a core strength, directly addressing the needs of interdisciplinary information seeking (IIS). The ability to refine the exploration based on user input shows responsiveness to the user's evolving understanding. But the paper acknowledges limitations in adaptability. Section 7.3 discusses the need for more nuanced understanding of user contexts, including research phase and preferences (e.g., "why" vs. "what" questions). While DiscipLink adapts to user input during a session, it doesn't seem to learn user preferences across sessions. Also, it does not allow users to set a specific range of discipline to start with.
%     \item \cite{IdeaSynth} The system's core functionality is built around a specific model of research. While prompt customization offers some flexibility, it doesn't fundamentally change the underlying structure. A truly adaptable system would need to be able to handle different types of research questions, different forms of evidence, and different ways of structuring arguments. The lack of explicit support for different research paradigms limits its adaptability.
% \end{itemize}
    
    

% \subsubsection{Example of bad use cases}
% \begin{itemize}
%     \item  \cite{AnalogicalSearchEngine} The system treats all users and all search tasks the same way. It does not adapt to individual research styles, levels of expertise, or specific project goals. The lack of personalization and customization makes the system a "one-size-fits-all" tool, limiting its ability to support diverse.
% \end{itemize}

Adaptability in AI-assisted research tools refers to a system’s capacity to support the diversity of tasks, workflows, and preferences of researchers. Ways this could be supported in tooling include flexible input mechanisms; fluid, nonlinear workflows; and context-sensitive design.

\subsubsection{Flexible Input \& Customization} Systems that prioritize flexible input mechanisms empower users to tailor the tool’s behavior from the outset. For example, the Analogical Search Engine~\cite{AnalogicalSearchEngine} leverages a custom ranking algorithm that focuses on user-defined “purposes” and “mechanisms” to modulate search results according to varying research objectives, allowing researchers to specify the kind of analogical relationships they are seeking. Similarly, the Brainwriting tool~\cite{brainwriting} allows users to dictate main topics in addition to divergent and convergent steps through a user-agnostic approach. These approaches underscore the importance of customization as a means of preserving user agency and aligning system outputs with specific investigative goals.

\subsubsection{Fluid, Non-Linear Workflows} A second facet of adaptability is found in tools that support non-linear, iterative workflows. The RQ Flow Editor in CoQuest~\cite{CoQuest} exemplifies this by eschewing predefined categories and instead promoting continuous refinement of ideas through a mixed-initiative interaction where AI suggests new research questions and users can provide feedback. Likewise, IdeaSynth~\cite{IdeaSynth} supports both literature-driven and idea-driven explorations through dynamic facet generation and prompt customization, enabling users to decompose an initial idea into finer-grained aspects and explore variations of them. Tools like Scholastic~\cite{scholastic} further demonstrate adaptability by allowing researchers to shift seamlessly between exploration, strategic sampling, and coding — highlighting the value of fluid transitions in non-linear research processes via its interactive document and word clustering.

\subsubsection{Mixed-Initiative \& Context-Sensitive Design} 
Adaptability also manifests in systems that accommodate varied research approaches through mixed-initiative interactions and context-sensitive features. For instance, Synergi~\cite{synergi} offers a mixed-initiative workflow that caters to both bottom-up and top-down strategies, allowing users to engage with content according to their preferred mode of inquiry and seamlessly combine machine-generated summaries and user-curated threads. Interdisciplinary platforms like DiscipLink~\cite{disciplink} contribute to this design space by facilitating the generation and refinement of research questions across multiple disciplines, albeit with some limitations regarding persistent user customization.

% Collectively, these examples emphasize a design space in which adaptability is achieved through varied strategies that maintain user agency while supporting a wide range of research tasks. By enabling flexible inputs, fluid workflows, and context-sensitive interactions, these tools foster environments where researchers can tailor system behavior to their evolving needs. This design philosophy is essential not only for accommodating diverse research paradigms but also for promoting deep cognitive engagement and innovative inquiry.

\begin{designRecom}{Ensure Adaptability and Workflow Flexibility} \textit{Design Insight:}
Research tools must provide flexible workflows and customizable interfaces to address varied researcher needs.

\smallskip

\textit{Implications:}
Future systems should be engineered with a design space that considers:
\begin{itemize}
\item \textbf{Flexible Input \& Customization:}
Enable users to tailor key parameters and input formats to align system behavior with their individual research goals.
\item \textbf{Fluid, Non-Linear Workflows:} Provide multiple entry points and modular interfaces that allow researchers to navigate, reorganize, and iteratively refine their inquiry as new insights emerge.
\item \textbf{Mixed-Initiative \& Context-Sensitive Design:}  Implement context-aware features that respond to shifts in research focus, ensuring a balanced mix of automated support and user control.
\end{itemize}

% \textit{Recommendation Example:}
% The Analogical Search Engine~\cite{AnalogicalSearchEngine} currently relies on a token-level ranking algorithm that identifies `purposes' (the problems addressed) and `mechanisms' (the solutions offered) across research articles. While effective for large-scale analogical reasoning, this uniform approach limits search flexibility. By enabling configurable search options -- such as toggling between analogical ranking, concept-based similarity matching, and methodological pattern exploration -- the system can better accommodate diverse research interests and methodologies.
\end{designRecom}

\subsection{Accuracy} 
\label{sec:accuracy} 
Ensuring that users receive accurate, unbiased, and contextually relevant information is paramount in AI-assisted research tools. To address challenges such as hallucination and contextual drift inherent to large language models, researchers have developed multifaceted strategies that combine human oversight, contextual grounding, and carefully managed efficiency-accuracy trade-offs.

\subsubsection{Interactive Interfaces for Accuracy Validation} A key design pattern embeds interactive mechanisms that enable real-time verification of AI outputs by linking inferences directly to their original sources. For instance, SenseMate \cite{sensemate} employs a ``View Reason'' feature that highlights source phrases underlying a theme recommendation, thus promoting local explainability and inviting critical evaluation rather than passive acceptance. Similarly, Synergi~\cite{synergi} and PaperWeaver~\cite{paperweaver} enhance their LLM-generated summaries by providing citation contexts and contextualized descriptions. Synergi groups related information to offer clear reference, while PaperWeaver uses aspect-based summaries (e.g., problem, method, findings) alongside paper comparisons to help researchers quickly assess the relevance of new publications.

\subsubsection{Iterative Refinement and Human-in-the-Loop Strategies} Complementing interactive validation, iterative refinement processes further emphasize human oversight. Scholastic~\cite{scholastic} demonstrates this approach by utilizing a machine-in-the-loop framework for qualitative text coding, wherein user feedback continuously refines coding decisions through rationale extraction models that enhance transparency and trust. Likewise, Threddy~\cite{threddy} and IdeaSynth~\cite{IdeaSynth} empower researchers to actively shape AI outputs. Threddy facilitates the extraction and iterative organization of research threads, while IdeaSynth leverages LLMs to propose new ideas that users can further refine. Together, these strategies enable researchers to combine automated suggestions with their own expert judgment.

% Collectively, these design strategies both ensure output accuracy and facilitate a reflective, engaged research practice. By integrating interactive validation and iterative human-centered refinement, AI-assisted research tools can minimize misinformation while promoting critical interrogation of outputs, which further reinforces the core principles of human-centered AI design \cite{shneiderman2020human}. In doing so, these systems encourage researchers to actively integrate practices that promote accuracy through ongoing validation and refinement, ensuring that their work remains reliable and contextually grounded throughout the ideation and sensemaking process.


\begin{designRecom}{Prioritize Transparency and Accuracy} \textit{Design Insight:} In scientific research, accuracy is non-negotiable. Systems must balance automation with robust mechanisms that empower users to verify AI outputs.

\textit{Implications:} 
\begin{itemize} 
\item \textbf{Transparent Rationale Explanations:} Allow users to understand the underlying reasoning behind AI outputs. 
\item \textbf{Citation Contexts:} Provide direct access to source materials, enabling quick cross-referencing and validation. 
\item \textbf{Human-in-the-Loop Reviews:} Facilitate iterative feedback that mitigates risks such as hallucinations and inadvertent biases. 
\end{itemize}

% \textit{Recommendation:} Avoid relying solely on fully automated LLM summarization for mission-critical tasks. As illustrated by PaperWeaver~\cite{paperweaver}, while summarization can quickly surface key insights, it also introduces uncertainty that may lead to misinformation or automation bias. A preferable approach, as demonstrated in Synergi~\cite{synergi}, employs recursive summarization techniques that preserve meaning across layers and present exact text snippets from original sources. This strategy enhances transparency and facilitates easier, more reliable validation by researchers. 
\end{designRecom}
\section{Result}

\subsection{Quantative Metrices}
In this paper, we use the distribution of drawn object categories and approaches to material utilization to illustrate the narrow creativity of human and GenAI. The object categories were determined based on a coding framework developed from an initial analysis of common design features and functional elements observed in the dataset. Each drawing was manually coded into a category according to its primary function and physical characteristics, following a structured coding process. This process involved two coders who categorized designs independently and resolved disagreements through discussion to ensure consistency and reliability in the coding scheme. The material utilization approaches were categorized by identifying and coding how materials were incorporated into the designs (e.g., direct use of materials and complex composition).

According to research on creativity evaluation \cite{tromp2024creativity, knight2015managing,li2008exploration}, creativity on generating new ideas are best understood by evaluating the exploration and exploitation of design space. 
To investigate the narrow creativity of human and GenAI by analyzing their exploration and exploitation of the explored design space, we developed the following quantitative metrics. 

\begin{itemize}
  \item \textbf{Number of used categories (\# of used cat.)} This metric represents the average number of distinct categories used by each participant during the task.
  \item \textbf{Number of frequent categories (\# of the equation cat.)} To identify the frequently used categories, we calculate the average and standard deviation of the number of circles within each category for an individual. A category is classified as 'frequent' if the number of circles in that category exceeds the average of the individual.
  \item \textbf{Number of highly frequent categories (\# of highly freq. cat.)} This metric identifies categories that are "highly frequent." A category is classified as such if the number of circles it contains exceeds the average by more than one standard deviation.

  \item \textbf{Proportations of drawings in the frequent categories (\% of the frequency category)} This metric counts the proportions of drawings made by an individual within their frequent categories.
  \item \textbf{Proportations of drawings in the highly frequent categories (\% of freq cat.)} Similar to the previous metric, this counts the total number of drawings within an individual’s highly frequent categories.
\end{itemize}

Exploration activity can be measured by the number of categories that an individual has used, while exploitation activity can be assessed by the number of frequent and highly frequent categories that an individual has. An individual who possesses a strong exploration mindset will have a relatively high number of categories used. In addition, the gap between the number of categories used and the number of frequent categories reveals the balance between exploration and exploitation.
When individuals explore multiple categories, a small gap indicates a good balance.

In order to further investigate the inclination between exploration and exploitation, we analyze the distribution of the circles among categories, which reveals how concentratedly an individual focuses on the frequent categories when doing the assignment. If a person has a large portion of circles that fall into the frequent or highly frequent categories, we can conclude that they lean toward exploitation and have narrow creativity issues.
Furthermore, conveying newly generated ideas is a part of the creativity process. In this work, we analyze the categories of artistic expression techniques that humans and GenAI frequently adopt. 

\subsection{Insights into Human Narrow Creativity from the Circles Excercise}

Human creative fixation often reflects a tendency to operate within familiar and constrained boundaries when engaging in creative tasks.
This tendency is evident in the results of the circle test, where individuals demonstrated preferences in object categories, artistic expression, and approaches to material utilization.
By clustering their creative output into key thematic aspects, we gain insight into the strategies humans used to interpret the task. Structurally encoding these strategies further allows us to pinpoint how fixation manifests in human creativity.
Furthermore, analysis of human creativity serves as a valuable baseline for interpreting similar patterns in the results generated by GenAI, enabling a deeper understanding of its capability and limitations.

\subsubsection{Distribution of Drawn Object Categories}

Human creativity tends to cluster around familiar categories of objects. 
Through analysis of students' drawings in the 28 Circles test, we observe that \textbf{daily objects} is the category most frequently used, while the \textbf{ mechanical} is the least adopted category. 
% We intentionally created this category because the students are mostly in the Department of Mechanical Engineering. 
The distribution suggests a tendency to draw inspiration from familiar, easily recognizable elements (daily objects), and reveals a clear preference for relatable, concrete objects over abstract forms or highly imaginative constructs. We provide the frequency of the average number of used categories in Figure \ref{humanobject}.
It shows that in the creativity task, humans exhibit a narrow and skewed bandwidth of creativity, with a significant inclination toward certain categories.



\begin{figure*}
    \includegraphics[width=0.9\textwidth]{figures/Object.pdf} % Adjust width as needed
    \captionof{figure}{The frequency distribution of categories of objects drawn by humans and GenAI, with a more even distribution across different categories indicating better performance.}
    \label{humanobject}
\end{figure*}

\begin{table*}
  \caption{Quantative analysis of human and GenAI narrow creativity based on the distribution of drawn object categories. For the number of used categories, frequent categories, and highly frequent categories, a higher value indicates better performance. Conversely, for the percentage of objects within frequent and highly frequent categories, a lower value reflects better performance.}
  \label{expression_table_1}
  \begin{tabular}{ccccccccc}
    \toprule
    Object Categories                    &  \multicolumn{2}{c}{Human}          & \multicolumn{2}{c}{Zero-shot} & \multicolumn{2}{c}{Few-shot} &\multicolumn{2}{c}{CoT}\\
    \midrule
                                               & Mean & Std.          & Mean & Std.     & Mean & Std.  & Mean & Std. \\
    \texttt{\# of used cat.}             & 5.6 & 2.1            & 6.6 & 1.6       & 5.6 & 2.0     & 7.5 & 0.97 \\
    \texttt{\# of freq. cat.}            & 2.5 & 1.2            & 2.5 & 0.66       & 2.8 & 1.1     & 3.2 & 0.78 \\
    \texttt{\# of highly freq. cat.}     & 1.4 & 0.96            & 1.4 & 0.65      & 1.2 & 0.28     & 1.6 & 0.69\\ \hline
    \texttt{\% of freq cat.}             & \multicolumn{2}{c}{70}             & \multicolumn{2}{c}{81}        & \multicolumn{2}{c}{77}      & \multicolumn{2}{c}{70}   \\
    \texttt{\% of highly freq. cat. }    & \multicolumn{2}{c}{47}             & \multicolumn{2}{c}{48}        & \multicolumn{2}{c}{43}      &\multicolumn{2}{c}{45}  \\
    \bottomrule
  \end{tabular}
\end{table*}

In this creativity exercise, we observe that human creativity only explores a narrow range when participating in the circle exercise. 
Specifically, an individual likely produces circles concentrated within only a limited number of categories. 
The average number of categories used, frequent categories, and highly frequent categories in all individuals is reported in Table \ref{expression_table_1}. 
Compared with the total number of object categories (10), these results suggest that humans tend to explore a limited subset of categories and frequently narrow their focus even further. 
Humans not only explore narrow perspectives, but also exploit even narrower ones. 
The percentage of objects belonging to frequent categories reveals that the majority of the drawn objects (70\%) fall into frequent categories. 
Thus, the result indicates that the creativity of individuals is strongly inclined towards a limited set of frequent categories.

\begin{figure*}
    \includegraphics[width=0.9\textwidth]{figures/Drawn_Object_Categories.pdf} % Adjust width as needed
    \captionof{figure}{Example of Drawn Object Categories: A) Human Sketched; B) GenAI-Generated, categorized into: 1) Animals, 2) Sport Equipment, 3) Foods, 4) Icons, 5) Daily Objects, and 6) Natural Elements.}
\end{figure*}

\subsubsection{Approaches to Material Utilization}
Students demonstrate diverse strategies for utilizing the circles for their ideas. These include \textbf{direct use}, where students transform the circles into recognizable objects such as clocks, wheels, or buttons by drawing directly within them; \textbf{personification}, where features of human faces are added to anthropomorphize the circles; \textbf{circle-based abstraction}, where the circles are used as references for similar shapes existing in other objects, such as a tennis racket, lollipop, and gear; \textbf{complex compositions}, where multiple circles were combined to form intricate objects, such as bicycles, ice-cream, and glasses; and \textbf{use as background}, where objects are draw within the circles. The frequency distribution of approaches to material utilization is shown in Figure \ref{humanutilization}. It indicates that humans also present narrow creativity and a skewed preference in terms of material utilization approaches.

\begin{figure*}
    \includegraphics[width=0.9\textwidth]{figures/Material_Utilization.pdf} % Adjust width as needed
    \captionof{figure}{Example of Approaches to Material Utilization: A) Human Sketched; B) GenAI Generated, categorized into: 1) Complex Compositions 2) Circle-based Abstraction 3) Use as Background.}
\end{figure*}


% \begin{center}
%     \includegraphics[width=0.6\textwidth]{figures/Distribution of Drawn Object Categories.png} % Adjust width as needed
%     \captionof{figure}{Place holder for example of Distribution of Drawn Object Categories}
% \end{center}

\begin{figure*}
    \includegraphics[width=0.7\textwidth]{figures/Utilization.pdf} % Adjust width as needed
    \captionof{figure}{The frequency distribution of approaches to material utilization is analyzed, with a more even distribution across different approaches indicating better performance.}
    \label{humanutilization}
\end{figure*}

Similarl to the analysis in the above subsection, Table \ref{expression_table_2} provides the average number of approaches used, frequent approaches, and highly frequent approaches. 
Based on the percentage of the frequent approaches, 80\% of the object are proposed based on one frequent approach of using the circles.

\begin{table*}
  \caption{Quantative analysis of human and GenAI narrow creativity based on material utilization approaches. For the number of used approaches, frequent approaches, and highly frequent approaches, a higher value indicates better performance. Conversely, for the percentage of objects within frequent and highly frequent approaches, a lower value reflects better performance.}
  \label{expression_table_2}
  \begin{tabular}{ccccccccc}
    \toprule
    Utilization Approaches                     &  \multicolumn{2}{c}{Human}          & \multicolumn{2}{c}{Zero-shot} & \multicolumn{2}{c}{Few-shot} &\multicolumn{2}{c}{CoT}\\
    \midrule
                                               & Mean & Std.          & Mean & Std.     & Mean & Std.  & Mean & Std. \\
    \texttt{\# of used apch.}                  & 3.0 & 1.2            & 3.5 & 0.96       & 3.4 & 1.2     & 3.6 & 0.69 \\
    \texttt{\# of freq. apch.}                 & 1.5 & 0.61           & 1.9 & 0.86       & 1.4 & 0.50    & 1.7 & 0.67 \\
    \texttt{\# of highly freq. apch.}          & 1.1 & 0.40           & 1.1 & 0.27       & - & -     & - & - \\ \hline
    \texttt{\% of freq apch.}                  & \multicolumn{2}{c}{80}             & \multicolumn{2}{c}{81}        & \multicolumn{2}{c}{63}      & \multicolumn{2}{c}{68}  \\
    \texttt{\% of highly freq. apch. }         & \multicolumn{2}{c}{68}             & \multicolumn{2}{c}{48}        & \multicolumn{2}{c}{51}      & \multicolumn{2}{c}{45}  \\
    \bottomrule
  \end{tabular}
\end{table*}

\subsubsection{Variation in Artistic Expression}

Although students' object categories and artistic expressions exhibit considerable diversity, the artistic styles and techniques they employ show limited variation, reflecting a monotonic and uniform approach to conveying their ideas. The approaches include \textbf{simple sketches}, where many students opt for minimalistic, black-and-white drawings focusing on the core concept; \textbf{detailed illustrations}, with some students enhancing their designs through intricate details that add depth and character. The portion of the two is presented in Table \ref{expression_table_2}, which suggests that humans almost adopt simple sketches. A possible reason might be that humans emphasize task efficiency and believe simple sketches are effective enough to convey their ideas.

Besides sketching, some students incorporate additional elements to facilitate their expression. The elements are summarized as follows: \textbf{use of color}, where some students incorporate vibrant colors to enrich their visual representations; and \textbf{annotations and labels}, where some drawings include textual annotation to explain or narrate their drawings, adding an interpretive layer to their visual output. 
The portion of drawings that use the additional elements is reported in Table \ref{expression_table_3}. 
The relatively low portion of drawings with additional elements (5\%) suggests that humans have limited capability to convey their ideas with detailed expressions. It might be because adding additional elements is time-consuming and requires extra resources. 
Figure \ref{fig: example of art} shows that while the GenAI outputs demonstrate enhanced refinement, they require explicit instructions to incorporate annotations or labels.


\begin{figure*}
    \includegraphics[width=0.9\textwidth]{figures/Artistic_Expression.pdf} % Adjust width as needed
    \captionof{figure}{Example of Artistic Expression: A) Human Sketched; B) GenAI-Generated, categorized into: 1) Simple Sketches 2) Detailed Illustrations 3) Use of Color 4) Annotations and Labels. }
    \label{fig: example of art}
\end{figure*}

\begin{table*}
  \caption{The portion of the frequent artistic expression of humans and GenAI.}
  \label{expression_table_3}
  \begin{tabular}{ccccc}
    \toprule
    Artistic Expressions                     & Human          & Zero-shot & Few-shot & CoT\\
    \midrule
    \texttt{\% of simple sketches}           & 95             & 2.9        & 10      & 3.2\\
    \texttt{\% of detailed illustrations}    & 5              & 97       & 89      & 97\\ \hline
    \texttt{\% of use of colors}             & 16             & 52        & 20      & 70\\
    \texttt{\% of use of annotations}        & 13             & 0.0        & 0.0     & 0.0\\
    \bottomrule
  \end{tabular}
\end{table*}

\subsection{Understanding GenAI Narrow Creativity with Different Prompting Strategies}

We present the results of different prompting strategies that are frequently adopted in GenAI-augmented creativity support tools.
We clustered the GenAI result based on the same aspects of narrow creativity as we used for human result.
This analysis offers insights into how AI-augmented creativity can complement or challenge human tendencies, revealing both the limitations and opportunities of prompting strategies in addressing narrow creative issues.

\subsubsection{Naive Prompting}

We conducted a pilot study to evaluate the results of GenAI under the condition of naive prompting.
We adopted the same quantitative metrics that were used to evaluate human creativity.
The zero-shot prompts provided to GenAI were based on a pre-articulated template, as described in the appendix.

For few-shot prompting, we provide each prompt with an examples from the results.

The statistics for zero-shot and few-shot prompting align closely with observations from human data (object categories: human=5.6, zero-shot=6.6, few-shot=5.6; utilization approaches: human=3.0, zero-shot=3.5, few-shot=3.4). 
The distribution of object categories suggests that GenAI, under naive prompting strategies, exhibits a similar pattern of narrow creativity as humans in this task. Interestingly, compared to zero-shot prompting, few-shot prompting produces GenAI results that are more quantitatively aligned with human performance. This observation implies that providing human examples may lead to a similar representation of narrow creativity in GenAI, particularly if the prompts are not further refined or articulated to encourage more diverse outputs.


In terms of artistic expression, humans and GenAI demonstrate differing preferences. Most humans (95\%) prefer to express their ideas through simple sketches. By contrast, GenAI models tend to favor detailed illustrations (zero-shot=97\%, few-shot=89\%), likely due to their stronger image-generation capabilities.
A similar pattern is observed in the use of color. Few-shot prompted GenAI exhibits preferences more aligned with human behavior (human=16\%, few-shot=20\%) due to exposure to human examples during training. This suggests that the inclusion of human examples in few-shot prompting can guide GenAI to mimic certain human tendencies, albeit within the constraints of its learned patterns.


\subsubsection{Chain-of-Thought Prompting}

To better understand the capability of GenAI, we adopted the Chain-of-Thought (CoT) prompting strategy to perform a circle test with GenAI. 
While CoT is considered an advanced technique to enhance GenAI's reasoning capabilities, the experiment results reveal GenAI under CoT prompting demonstrate similiar pattern of narrow creativity as human does.

The result demonstrate that a significant proportion of the objects generated by GenAI under CoT prompting belong to frequently used categories (70\%). This mirrors the behavior seen in other strategies (e.g., zero-shot=81\% and few-shot=70\%), showing that CoT does not substantially expand the variety of generated object categories.
Moreover, 45\% of the objects belong to highly frequent categories, reinforcing the observation that GenAI under CoT also tend to explore on narrowed regions of design space, rather than exploring more diverse ideas.
In terms of material utilization approaches, CoT-generated ideas also exhibit constrained diversity. Approximately 68\% of the approaches employed by GenAI under CoT prompting narrow to the most frequently used methods.
This suggests that CoT does not effectively overcome the bias toward relying on dominant patterns of material utilization.

These results highlight a persistent reliance on frequent object categories and approaches. While CoT improves reasoning capabilities, it does not necessarily enhance the creative breadth of GenAI, as it struggles to generate ideas that break away from the narrow design space.

\section{Discussion}

% Shift from findings to discussion
This study on robotic art explores human-machine relationships in creative processes.
It first contributes as an empirical description of artistic creativity in robotic art practice---an unconventional use of robots---examined through the artists' perspectives on their creative experiences. Our analysis reveals three facets of creativity in robotic art practices: the \textit{social}, \textit{material}, and \textit{temporal}. Creativity emerges from the co-constitution between artists, robots, audience, and environment in spatial-temporal dimensions, as illustrated in \autoref{PracticeDiagram}. Acknowledging the audience as an important actor reflects the social dimension in that creativity does not stem from the artists but from their interactions with the audience. Robots are the major material and technological actants characterizing creative practices, shaping the conditions for creativity to emerge. The axis of the temporal process signifies that the practice is situated within a time continuum, and the actors/actants and their relations shift over time. In this way, temporality constitutes another dimension of creativity in robotic art.

Accordingly, as the second contribution, this study outlines implications for \textit{socially informed}, \textit{material-attentive}, and \textit{process-oriented} creation with computing systems\footnote{For the sake of clarity, we mean ``creation with computing systems'' by three types of scenarios: human creator(s) create computing system(s) as the final artifact(s) (e.g., robots are artworks themselves); human creator(s) use computing system(s) to create the artifact(s) (e.g., robots create artworks as human planned); and human creator(s) and system(s) work in tandem to produce the artifact(s) (e.g., human-robot co-creation).} to facilitate creation practices. These insights can inform related HCI research on media/art creation, crafting, digital fabrication, and tangible computing.
In each following subsection, we present each implication with a grounding in corresponding findings from our study and relevant literature in HCI and adjacent fields on art, creativity, and creation.

\begin{figure*}[htbp]
    \centering
    \includegraphics[width=0.88\textwidth]{Writings/figure/PracticeDiagram.pdf}
    \caption{Actors/actants in robotic art practice and their interactive relations. Robotic art practice unfolds primarily in two spaces: the creation space where interactions happen mainly between artists and robots, and the exhibition space where interactions mostly involve audiences and robots. The two spaces constitute the ENVIRONMENT plane. Within the plane, directed arrows between the actors indicate the types of interaction. For example, the \textit{Design} arrow indicates that the artist designs the robot(s), and the \textit{Revise} arrow indicates that the robot(s) make the artist revise artistic ideas. All the actors/actants may also intra-act with the ENVIRONMENT. The actors/actants and their interactive relations may differ at different times along the axis of TEMPORAL PROCESS that is orthogonal to the plane.}
    \Description{This figure shows the actors/actants in robotic art practice and their interactive relations.}
    \label{PracticeDiagram}
\end{figure*}

\subsection{Socially Informed Creation}

% Introduce social aspect of distributed creativity
The sociality of creativity means that creativity is distributed among different human actors, commonly within the creators or between the creators and the audience. Glăveanu’s ethnographic study on Easter egg decoration in northern Romania~\cite{glaveanu_distributed_2014} showed that artisans anticipate how others might appreciate their work and adjust their creative decisions accordingly. Even in the absence of direct interaction, the audience’s potential responses become part of the creative process, as artisans imagine feedback and predict reactions. In this sense, the sociologist Katherine Giuffre argues that ``\textit{creative individuals are embedded within specific network contexts so that creativity itself, rather than being an individual personality characteristic is, instead, a collective phenomenon}''~\cite[p. 1]{giuffre2012collective}.

% Recall findings about audience feedback
We found that the practice of robotic art manifests this sociality as it involves, particularly artists and audiences. 
Our artists take audiences' reactions to their artwork as feedback and then revise the robots' functions and aesthetics accordingly. 
For example, as shown earlier, Robert added a protective fuse onto his robot because he expected that children would squeeze the springs together and cause a short circuit; Alex's enthusiasm and attention to the audience's imagination about his robots led him to new aesthetic designs of both the robots and the scene layouts. The artists may directly ask about the audience's judgment of quality but they often receive feedback just by observing the audience's reactions or sometimes by learning from the audience's imagination about the robots.
% Recall findings about audience's sociocultural expectations and codes
Meanwhile, our findings reveal that audience reception is not an individual matter but is often associated with their sociocultural codes, including shared cultural norms, beliefs, expectations, and aesthetic values. The audience can be seen as representatives of these broader cultural codes. For example, Mark and Robert observed that the animist tendency in some East Asian societies is associated with higher acceptance of and interest among the audience in intelligence and agency of robots and non-human entities. Such sociocultural contexts influence not only how audiences interpret the work but also how artists anticipate and respond to these perspectives in their creative process.

% Situate in HCI literature
A creative process, including creation and reception, is essentially a social activity. The second wave of creativity research in psychology has argued for creativity's dependency on sociocultural settings and group dynamics~\cite{sawyer2024explaining}. Recent discussions from creativity-support and social computing researchers also called for more attention to the social aspect of creativity~\cite{kato2023special, fischer2005beyond, fischer2009creativity}. There is a clear need to consider the audience when producing creative content. For instance, researchers studying video-creation support have examined audience preferences to inform system designs that align with these preferences~\cite{wang2024podreels}. Such work highlights how creative activities extend beyond individual creators to co-creators and heterogeneous audiences. Some HCI researchers conceptualize creativity as by large a socially constructed concept, perceived and determined by social groups~\cite{fischer2009creativity}. 
Prior HCI work examined the social aspects between art creators. For example, creators and performers in music and dance form social relationships through artifacts, making the final work a collaborative outcome~\cite{hsueh2019deconstructing}. There is also a system designed to support collaborative creation between artists~\cite{striner2022co}. However, the social creative process between creators and audience is less articulated in HCI. Jeon et al.'s work~\cite{jeon2019rituals} stands as an exception, suggesting that professional interactive art can involve evaluation with the audience in the creation stage. 
Another relevant approach in HCI involves enabling the general public to participate in co-creation alongside professional creators. ~\citet{matarasso2019restless}, for instance, promoted ``participatory art'' as ``\textit{the creation of an artwork by professional artists and non-professional artists working together}'' with non-professional artists referring to the general public engaged in the art-making process. Similarly, socially inclusive community-based art also considers target communities' perception of the artwork during creation~\cite{clark2016situated, clarke2014socially}. But like participatory design~\cite{schuler1993participatory}, these art projects aim for social justice more than creativity in the work~\cite{murray2024designing}, let alone that direct participation in art creation is not always feasible. Our findings suggest that feedback from the audience can lead to creative ideas, as well as that the feedback can be generative and remain low-effort for the audience.

Unlike conventional design feedback---which is typically expected to be specific, justified, and actionable~\cite{yen2024give, krishna2021ready}---the feedback that resonates with our artists is often implicit, creative, and generative. Such feedback may include audiences' imaginations stimulated by the work, personal and societal reflections, and even emotions, facial expressions, micro-actions, and observable behaviors following the art experience. Our artists gathered this implicit feedback not by posing evaluative questions, as commonly done in typical design processes (e.g., usability testing, think-aloud protocols), which seek to elicit clear, relatively structured responses. Instead, they closely observe the audience's reactions and interpret their subjective perceptions. This form of implicit feedback, while indirect, can lead to more creative ideas by embracing open, multifaceted interpretations of the work~\cite{sengers2006staying}. Computing systems for creation should better incorporate implicit feedback in addition to explicit ones from the audience into the creation process. Implicit feedback can be indirect, creative, inspirational, and heuristic about functions and aesthetics. A hypothetical instance of such design can be a system that helps creators perceive audiences' implicit reactions and perceptions and variously interpret them, for further iteration.

% Recall findings about audience interacting with robots as a performative art
Moreover, as seen in Robert and Daniel's experiences, the audience may participate in robotic live performances by interacting with the robots, who may change actions accordingly, triggering a loop of simultaneous mutual influence that makes the work performative and improvisational.
% Situate in HCI
HCI researchers explored performative and improvisational creation with machines, focusing on developing and evaluating systems with performative capabilities, including music improvisation with robots~\cite{hoffman2010shimon}, dance with virtual agents~\cite{jacob2015viewpoints, triebus2023precious}, and narrative theatre~\cite{magerko2011employing, piplica2012full}. \citet{kang2018intermodulation} discussed the improvisational nature of interactions between humans and computers and argued that an HCI researcher-designers' improvisation with the environment facilitates the emergence of creativity and knowledge. Designs of computing systems for creation can leverage performativity in service of creative experience. One possible direction could be to allow the audience to embed themselves in and interact with elements of static artwork in a virtual space, turning the exhibition into an improvisational on-site creation~\cite{zhou2023painterly}.
% Our new implication different from current discussion on perf and impr
While interactions with machines during performance are mostly physical or embodied, we posit that they can also be a \textit{symbolic engagement}. Alex's audience projected themselves and their personalities onto his robots, which established a symbolic relevance, generating creative imaginations. During exhibitions, East Asian audiences carried the animist views shaped by their sociocultural backgrounds, and robots, through the performance, were successful in symbolically matching the views, stimulating aesthetic satisfaction. Symbolic engagement resonates with what ~\citet{nam2014interactive} called the ``reference'' of the interactive installation performance to participants' sociocultural conditions.
As such, we propose that designers of computing systems for creation may consider establishing symbolic engagement between the produced artifacts and the audience as a way to enhance perceived creativity or enrich the creative experience. One example is an interactive installation, \textit{Boundary Functions}~\cite{snibbe1998}, which encourages viewers to reflect on their personal spaces while interacting with the installation and others. Another example is \textit{Blendie}, a voice-controlled blender that requires a user to ``speak'' the machine's language to use it. This interaction builds a symbolic connection between the user and the device, transforming the act of blending into a novel experience~\cite{dobson2004blendie}.


\subsection{Material-Attentive Creation}

% Intro paragraph to the importance of materiality for creative activities with machines and the end goal of this discussion--- design suggestions
The theory of distributed creativity by Glaveanu claims that creativity distributes across humans and materials, so the creation practice itself is inevitably shaped by objects~\cite{glaveanu_distributed_2014}. In his case of Easter egg decoration, materials are not passive objects but active participants in artistic creation; e.g., the egg decorators face challenges from color pigments not matching the shell, wax not melted at the desired temperature, to eggs that break at the last step of decoration; hence, materials often go against the decorators' intentions and influence future creative pathways~\cite{glaveanu_distributed_2014}.
Materials manifest specific properties, which afford certain uses of the materials while constraining others~\cite{leonardi2012materiality}. Our findings highlight the critical role of materiality in artistic practice, showing that artists intentionally arrange materials to enhance the creative values of their work.

% Materiality aspect One: physicality and embodiment
% Embodiment or physicality fascilitates creative interaction with machines
Robotic art relies on the material properties of robots and other objects. An apparent property of most materials is their physicality~\cite{leonardi2012materiality}, meaning they possess a tangible presence that enables interaction with other physical entities. Here, we consider physicality and embodiment interchangeable as computational creativity researchers have conceptualized~\cite{guckelsberger2021embodiment}.
% Recall findings on embodiment's value in making art
Our findings support both the conceptual and operational contributions of embodiment for creative activities. For the conceptual aspect, the embodied presence of robotic systems supports creative thinking for our artists, exemplary in Linda's case where she found new art ideas around the difference between human and robot bodies through bodily engagement with robots. 
For the operational aspect, the embodied nature of robotic artworks and their creation processes exhibit original aesthetics that are based on physics much different from disembodied works, e.g., embodied drawings by David's non-industrial robotic arms are dynamic due to physical movements and thus artistically pleasant, which is hard to replicate in simulated programs.

% References: embodied interaction, embodied cognition theories, tangible computing
These findings on embodiment of robotic art (Section \ref{f:emb}) closely relate to HCI's attention on embodied interaction as a way to leverage human bodies and environmental objects to expand disembodied user experiences. 
For example, as~\citet{hollan2000distributed} explained, a blind person's cane and a cell biologist's microscope as embodied materials are part of the distributed system of cognitive control, showing that cognition is distributed and embodied. 
Similarly, theories of embodied interaction in HCI explicate how bodily interactions shape perception, experience, and cognition~\cite{marshall2013introduction, antle2011workshop, antle2009body}, backed up by the framework of 4E cognition (embodied, embedded, enactive, and extended)~\cite{wheeler2005reconstructing, newen20184E}. 
Prior works suggest that creative activities with interactive machines rely on similar embodied cognitive mechanisms ~\cite{guckelsberger2021embodiment, malinin2019radical}, which are operationalized by tangible computing~\cite{hornecker2011role}. 
% References: embodiment's consequence in creation
As related to robots in creation, HCI researchers show that physicality or embodiment of robots in creation may lead to some beneficial outcomes, such as curiosity from the audience, feelings of co-presence, body engagement, and mutuality, which are hard to simulate through computer programs~\cite{dell2022ah, hoggenmueller2020woodie}. Embodied robotic motions convey emotional expressions and social cues that potentially enrich and facilitate creation activities like drawings~\cite{ariccia2022make, grinberg2023implicit, dietz2017human, santos2021motions}. Guckelsberger et al.~\cite{guckelsberger2021embodiment} showed in their review that embodiment-related constraints (e.g., the physical limitations of a moving robotic arm) can also stimulate creativity. These constraints push creators to develop new and useful movements, echoing the broader principle that encountering obstacles in forms or materials can lead to generative processes. This phenomenon is similarly observed in activities such as art and digital fabrication~\cite{devendorf2015being, hirsch2023nothing}. In co-drawing with robots, physical touch and textures of drawing materials made the artists prefer tangible mediums (e.g., pencils) than digital tools (e.g., tablets) that fall short in these respects~\cite{jansen2021exploring}.

% Transit to materiality aspect two
% Materiality aspect Two: malfunction as manifestation of unique materiality of robots
% Intro to materials of robots
Materiality plays a crucial role in the embodiment of robots, as the choice of materials fundamentally shapes the physical forms and properties. This focus on materials extends to art practices, where robots made with soft materials introduce new aesthetics and sensory experiences~\cite{jorgensen2019constructing, belling2021rhythm}, and the use of plants and soil in robotic printing creates unique visual effects~\cite{harmon2022living}. Following Leonardi's ~\cite{leonardi2012materiality} conceptualization of materiality, we refer to the materials of robots as encompassing physical and digital components---including the shell, hardware, mechanical parts, software, programs, data, and controllers---each significant to the artist's intent. ~\citet{nam2023dreams} found that the material constraints of robots can limit creative expression but simultaneously stimulate creativity when artists push the boundaries.

%-----maybe here the real "malfuction" start ------------------
% Move to introduce malfunctions as unique materiality

Even carefully designed, digital and mechanical components in robots are prone to errors or bugs in everyday runs, causing malfunctions or unexpected consequences. This reflects the unique materiality of robots as complex computing systems. From an engineering perspective, errors signal unreliability and must be eliminated, driving advancements in robotics---where error detection and recovery are central~\cite{gini1987monitoring}---as well as in digital fabrication, which prioritizes precision over creative exploration~\cite{yildirim2020digital}. % Recall findings on embracing malfunctions
However, material failures and accidents are inevitable, exemplifying what has been called the `craftsmanship of risk'~\cite{glaveanu_distributed_2014} in material art. For our artists, these risks are often creatively utilized and incorporated into their work: these moments of breakdown---whether physical or digital---become resources for new creative expression. Errors are anticipated and intentionally designed into the process and work of our artists. In some cases, such as for Alex, the entire concept of one of his works is machine errors.

% Situate in literature
Reports on how artists view errors within engineering and creation processes are dispersed throughout HCI literature. ~\citet{nam2023dreams} showed that the accumulation of ``contingency'' and ``accidents''---unexpected, serendipitous, and emergent events during art creation like errors---meaningfully constituted the final presentation of the artwork. Song and Paulos's concept of ``unmaking'' highlighted the values of material failures in enabling new aesthetics and creativity~\cite{song2021unmaking}. Kang et al.~\cite{kang2022electronicists, kang2023lady} introduced the notion of an ``error-engaged studio'' for design research in which errors in creative processes are identified, accommodated, and leveraged for their creative potential. Collectively, these works advocate for reframing errors from something to avoid to something to embrace and recognize. We want to push this further by arguing that errors can be intended and be part or sometimes entire of the design. Several artists, including participants from our study, have been deliberately seeking errors to formulate their designs. Roboticist Damith Herath recounted when he mistakenly programmed a motion sequence of a robotic arm, his collaborator, robotic artist Stelac responded with ``[W]e need to make more mistakes;'' as many mistakes were made, the initial pointless movements became beautiful, rendering the robot ``alive'' and ``seductive'' \cite{herath2016robots}. Similarly, AI artists sometimes look for program glitches to generate unusual styles and content~\cite{chang2023prompt}. Therefore, creators may not only passively accept errors but can actively seek and utilize them. Errors can be integral to the design itself---errors can \textit{be designed into} an artifact, and the design/idea of the artifact can be all about errors.

Thus, to focus on material-attentive creation---considering the creative arrangement of materials---we suggest exploring the embodiment and materiality of creation materials, objects, and environments to recognize their creative potential. %This perspective aligns with insights from professional digital fabrication practitioners, who advocate for systems that integrate support for machine settings and material properties~\cite{hirsch2023nothing}.
Specifically, we propose using a design method/probe that enables creators to realize both the conceptual and operational contributions of materiality. This approach may build on the material probe developed by~\citet{jung2010material}, which calls for exploring the materiality of digital artifacts. A material-attentive probe would enable creators to engage with diverse materials, objects, and environments through embodied interaction, encouraging them to speculate on material preferences and limitations, and to compare and contrast material qualities---insights that can inform creative decisions.
To accommodate, seek, and actively harness the creative potential of errors, we propose embracing failures, glitches, randomness, and malfunctions in computing systems as critical design materials---elements that creators can intentionally control and manipulate. By doing so, we can begin to systematically approach errors. For instance, as part of the design process, we may document how to replicate these errors and changes, allowing creators to explore them further at their discretion. This could include intentionally inducing errors or random changes to influence the creative process or outcomes.

\subsection{Process-Oriented Creation}

% Introduce the key idea: process itself embeds creative value and can be pursued as the goal of creation
As shown in our findings, the creation process itself embeds creative values and meanings, and experiencing the process can be pursued as the goal of creation with computing systems.
% Recall findings
For the robotic artists in our study, artistic values were often placed on the creation process rather than the outcome.  For example, in Alex's robotic live drawing performance, the drawing process is more important than the drawn pattern on canvas. Techniques used, decisions made, or stimuli received by robots during creation or exhibition reflect artistic ideas and nuanced thinking, as seen in Sophie's exploration of interactive decision-making in robotic drawing.

% Situate in HCI lit
Previous HCI work has touched on the value of the process of creation. ~\citet{bremers2024designing} shared a vignette where a robotic pen plotter simultaneously imitates the creator's drawing, serving as a material presence rather than a pragmatic co-creator; here the focus of the work is no longer the outcome but the process of drawing itself. ~\citet{devendorf2015reimagining} concluded that performative actions of digital fabrication systems, rather than the fabricated products themselves, convey artistic meanings tied to histories, public spaces, time, environments, audiences, and gestures. This emphasis on process is particularly significant for media such as improvisational theatre, where the creation itself is an integral part of the final work~\cite{o2011knowledge}. ~\citet{davis2016empirically} named their improvisational co-drawing robotic agents as ``casual creators,'' who are meant to creatively engage users and provide enjoyable creative experiences rather than necessarily helping users make a higher quality product. Shifting the focus from product to process and experiences \textit{in} creation may generate alternative creative meanings.

% Findings about process extends beyond creation
Our artists pointed out that even a ``finished'' artwork in an exhibition is not truly finished. A crack in Daniel's robotic artwork introduced a new artistic meaning, ultimately subverting the entire work. As the properties of the work change over time---whether due to the artist's intent, material characteristics, or environmental factors---the artwork evolves, revealing new aesthetics and meanings. % Situate in HCI lit
Based on these observations, we argue that creation processes should not be regarded as one-shot transactions, as creative artifacts, particularly physical ones, continue to change and generate artistic values. For instance, material wear and destruction bring unique aesthetics, often contrasting with the original form ~\cite{zoran2013hybrid}, and are seen as signs of mature use~\cite{giaccardi2014growing}.
Changes such as material failure, destruction, decay, and deformation---what~\citet{song2021unmaking} referred to as ``unmaking,'' a process that occurs after making---meaningfully transforms the original objects. Similarly, through Broken Probes, a process of assembling fractured objects, ~\citet{ikemiya2014broken} demonstrated that personal connections, reminiscence, and reflections related to material wear and breakage add new values to the objects. Drawing from Japanese philosophy Wabi-Sabi, ~\citet{tsaknaki2016expanding} reflected on the creeds of `Nothing lasts,' `Nothing is finished,' and `Nothing is perfect' and pointed to the impermanence, incompleteness, and imperfection of artifacts as a resource that designers, producers, and users can utilize to achieve long-term, improving, and richer interactive experience~\cite{tsaknaki2016things}. Insights from this study contribute to this line of thought by showing how robotic artists appreciate the aesthetics and meanings of temporal changes after the creation phase.

The findings underscore the need to reconceptualize creation as encompassing more than just the process aimed at producing a final product; it also includes what we term \textit{post-creation}. Distinct from repair, maintenance, or recycle, \textit{post-creation} entails anticipating and managing how an artifact evolves after its ``completion'' in the conventional sense. Specifically, we encourage creators to anticipate and strategically engage with the post-creation phase, considering potential changes to the artifact and their consequences for interactions with human users. For instance, during the creation process, creators may focus on possible material changes the artifact might undergo post-creation, allowing them to either mitigate or creatively exploit these potential changes. This expanded view of creation invites us to trace post-creation developments and to plan how our creative intentions can be embedded in its potential degradation, transformation, or evolution over time.

% A conclusion paragraph
We categorize the design implications into three aspects, but we do not suggest that a computing system must implement all simultaneously, nor that each aspect should be considered in isolation. Social interactions, such as those between artists and audiences, already presume the presence of material actants like robots, and these interactions inform future arrangements of materials. Thus the social and material aspects can be entangled and mutually constitutive as seen in sociomaterial practices~\cite{orlikowski2007sociomaterial, cheatle2015digital, rosner2012material}. The temporal aspect is orthogonal to the other aspects because both social interactions and material manifestations unfold and shift in a temporal continuum.


%%
%% The acknowledgments section is defined using the "acks" environment
%% (and NOT an unnumbered section). This ensures the proper
%% identification of the section in the article metadata, and the
%% consistent spelling of the heading.
% \begin{acks}
% We thank Dr. Sangho Suh for the insightful discussion and constructive feedback during our theme ideation phase. 

We acknowledge the support of the Natural Sciences and Engineering Research Council of Canada (NSERC), [funding reference number RGPIN-2024-04348 and RGPIN-2024-06005].
% \end{acks}

%%
%% The next two lines define the bibliography style to be used, and
%% the bibliography file.
\bibliographystyle{ACM-Reference-Format}
\bibliography{reference}


%%
%% If your work has an appendix, this is the place to put it.
\appendix

\section{Hard Threshold of EAC}\label{threshhold}
In constructing a weighted-gradient saliency map, the value of \(\gamma\) determines the number of the dimensions we select where important feature anchors are located. As the value of \(\gamma\) increases, the number of selected dimensions decreases, requiring the editing information to be compressed into a smaller space during the compression process. 
During compression, it is desired for the compression space to be as small as possible to preserve the general abilities of the model. However, reducing the compression space inevitably increases the loss of editing information, which reduces the editing performance of the model.
Therefore, to ensure editing performance in a single editing scenario, different values of \(\gamma\) are determined for various models, methods, and datasets. Fifty pieces of knowledge were randomly selected from the dataset, and reliability, generalization, and locality were measured after editing. The averages of these metrics were then taken as a measure of the editing performance of the model.
Table~\ref{value} presents the details of \(\gamma\), while Table~\ref{s} illustrates the corresponding editing performance before and after the introduction of EAC. $P_{x}$ denotes the value below which x\% of the values in the dataset.


\begin{table}[!htb]
\caption{The value of $\gamma$.}
\centering
\resizebox{0.45\textwidth}{!}{
\begin{tabular}{lcccc}
\toprule
\textbf{Datasets} & \textbf{Model} & \textbf{ROME} & \textbf{MEMIT} \\
\midrule
\multirow{2}{*}{\textbf{ZSRE}} & GPT-2 XL & $P_{80}$ & $P_{80}$ \\
 & LLaMA-3 (8B) & $P_{90}$ & $P_{95}$ \\
\midrule
\multirow{2}{*}{\textbf{COUNTERFACT}} & GPT-2 XL & $P_{85}$ & $P_{85}$ \\
 & LLaMA-3 (8B) & $P_{95}$ & $P_{95}$ \\
\bottomrule
\end{tabular}}
\label{value}
\end{table}


\begin{table}[!htb]
\caption{The value of $\gamma$.}
\centering
\resizebox{\textwidth}{!}{%
\begin{tabular}{lccccccccccccc}
\toprule
\multirow{1}{*}{Dataset} & \multirow{1}{*}{Method} & \multicolumn{3}{c}{\textbf{GPT-2 XL}} & \multicolumn{3}{c}{\textbf{LLaMA-3 (8B)}} \\
\cmidrule(lr){3-5} \cmidrule(lr){6-8}
& & \multicolumn{1}{c}{Reliability} & \multicolumn{1}{c}{Generalization} & \multicolumn{1}{c}{Locality} & \multicolumn{1}{c}{Reliability} & \multicolumn{1}{c}{Generalization} & \multicolumn{1}{c}{Locality} \\
\midrule
\multirow{1}{*}{ZsRE} & ROME & 1.0000 & 0.9112 & 0.9661 & 1.0000 & 0.9883 & 0.9600  \\
& ROME-EAC & 1.0000 & 0.8923 & 0.9560 & 0.9933 & 0.9733 & 0.9742  \\
\cmidrule(lr){2-8}
& MEMIT & 0.6928 & 0.5208 & 1.0000 & 0.9507 & 0.9333 & 0.9688  \\
& MEMIT-EAC & 0.6614 & 0.4968 & 0.9971 & 0.9503 & 0.9390 & 0.9767  \\
\midrule
\multirow{1}{*}{CounterFact} & ROME & 1.0000 & 0.4200 & 0.9600 & 1.0000 & 0.3600 & 0.7800  \\
& ROME-EAC & 0.9800 & 0.3800 & 0.9600 & 1.0000 & 0.3200 & 0.8800  \\
\cmidrule(lr){2-8}
& MEMIT & 0.9000 & 0.2200 & 1.0000 & 1.0000 & 0.3800 & 0.9500  \\
& MEMIT-EAC & 0.8000 & 0.1800 & 1.0000 & 1.0000 & 0.3200 & 0.9800  \\
\bottomrule
\end{tabular}%
}
\label{s}
\end{table}

\section{Optimization Details}\label{b}
ROME derives a closed-form solution to achieve the optimization:
\begin{equation}
\text{minimize} \ \| \widehat{W}K - V \| \ \text{such that} \ \widehat{W}\mathbf{k}_* = \mathbf{v}_* \ \text{by setting} \ \widehat{W} = W + \Lambda (C^{-1}\mathbf{k}_*)^T.
\end{equation}

Here \( W \) is the original matrix, \( C = KK^T \) is a constant that is pre-cached by estimating the uncentered covariance of \( \mathbf{k} \) from a sample of Wikipedia text, and \( \Lambda = (\mathbf{v}_* - W\mathbf{k}_*) / ( (C^{-1}\mathbf{k}_*)^T \mathbf{k}_*) \) is a vector proportional to the residual error of the new key-value pair on the original memory matrix.

In ROME, \(\mathbf{k}_*\) is derived from the following equation:
\begin{equation}
\mathbf{k}_* = \frac{1}{N} \sum_{j=1}^{N} \mathbf{k}(x_j + s), \quad \text{where} \quad \mathbf{k}(x) = \sigma \left( W_{f_c}^{(l^*)} \gamma \left( a_{[x],i}^{(l^*)} + h_{[x],i}^{(l^*-1)} \right) \right).
\end{equation}

ROME set $\mathbf{v}_* = \arg\min_z \mathcal{L}(z)$, where the objective $\mathcal{L}(z)$ is:
\begin{equation}
\frac{1}{N} \sum_{j=1}^{N} -\log \mathbb{P}_{G(m_{i}^{l^*}:=z))} \left[ o^* \mid x_j + p \right] + D_{KL} \left( \mathbb{P}_{G(m_{i}^{l^*}:=z)} \left[ x \mid p' \right] \parallel \mathbb{P}_{G} \left[ x \mid p' \right] \right).
\end{equation}

\section{Experimental Setup} \label{detail}

\subsection{Editing Methods}\label{EM}

In our experiments, Two popular editing methods including ROME and MEMIT were selected as baselines.

\textbf{ROME} \cite{DBLP:conf/nips/MengBAB22}: it first localized the factual knowledge at a specific layer in the transformer MLP modules, and then updated the knowledge by directly writing new key-value pairs in the MLP module.

\textbf{MEMIT} \cite{DBLP:conf/iclr/MengSABB23}: it extended ROME to edit a large set of facts and updated a set of MLP layers to update knowledge.

The ability of these methods was assessed based on EasyEdit~\cite{DBLP:journals/corr/abs-2308-07269}, an easy-to-use knowledge editing framework which integrates the released codes and hyperparameters from previous methods.

\subsection{Editing Datasets}\label{dat}
In our experiment, two popular model editing datasets \textsc{ZsRE}~\cite{DBLP:conf/conll/LevySCZ17} and \textsc{CounterFact}~\cite{DBLP:conf/nips/MengBAB22} were adopted.

\textbf{\textsc{ZsRE}} is a QA dataset using question rephrasings generated by back-translation as the equivalence neighborhood.
Each input is a question about an entity, and plausible alternative edit labels are sampled from the top-ranked predictions of a BART-base model trained on \textsc{ZsRE}.

\textbf{\textsc{CounterFact}} accounts for counterfacts that start with low scores in comparison to correct facts. It constructs out-of-scope data by substituting the subject entity for a proximate subject entity sharing a predicate. This alteration enables us to differentiate between superficial wording changes and more significant modifications that correspond to a meaningful shift in a fact. 

\subsection{Metrics for Evaluating Editing Performance}\label{Mediting performance}
\paragraph{Reliability} means that given an editing factual knowledge, the edited model should produce the expected predictions. The reliability is measured as the average accuracy on the edit case:
\begin{equation}
\mathbb{E}_{(x'_{ei}, y'_{ei}) \sim \{(x_{ei}, y_{ei})\}} \mathbf{1} \left\{ \arg\max_y f_{\theta_{i}} \left( y \mid x'_{ei} \right) = y'_{ei} \right\}.
\label{rel}
\end{equation}

\paragraph{Generalization} means that edited models should be able to recall the updated knowledge when prompted within the editing scope. The generalization is assessed by the average accuracy of the model on examples uniformly sampled from the equivalence neighborhood:
\begin{equation}
\mathbb{E}_{(x'_{ei}, y'_{ei}) \sim N(x_{ei}, y_{ei})} \mathbf{1} \left\{ \arg\max_y f_{\theta_{i}} \left( y \mid x'_{ei} \right) = y'_{ei} \right\}.
\label{gen}
\end{equation}

\paragraph{Locality} means that the edited model should remain unchanged in response to prompts that are irrelevant or the out-of-scope. The locality is evaluated by the rate at which the edited model's predictions remain unchanged compared to the pre-edit model.
\begin{equation}
\mathbb{E}_{(x'_{ei}, y'_{ei}) \sim O(x_{ei}, y_{ei})} \mathbf{1} \left\{ f_{\theta_{i}} \left( y \mid x'_{ei} \right) = f_{\theta_{i-1}} \left( y \mid x'_{ei} \right) \right\}.
\label{loc}
\end{equation}

\subsection{Downstream Tasks}\label{pro}

Four downstream tasks were selected to measure the general abilities of models before and after editing:
\textbf{Natural language inference (NLI)} on the RTE~\cite{DBLP:conf/mlcw/DaganGM05}, and the results were measured by accuracy of two-way classification.
\textbf{Open-domain QA} on the Natural Question~\cite{DBLP:journals/tacl/KwiatkowskiPRCP19}, and the results were measured by exact match (EM) with the reference answer after minor normalization as in \citet{DBLP:conf/acl/ChenFWB17} and \citet{DBLP:conf/acl/LeeCT19}.
\textbf{Summarization} on the SAMSum~\cite{gliwa-etal-2019-samsum}, and the results were measured by the average of ROUGE-1, ROUGE-2 and ROUGE-L as in \citet{lin-2004-rouge}.
\textbf{Sentiment analysis} on the SST2~\cite{DBLP:conf/emnlp/SocherPWCMNP13}, and the results were measured by accuracy of two-way classification.

The prompts for each task were illustrated in Table~\ref{tab-prompt}.

\begin{table*}[!htb]
% \small
\centering
\begin{tabular}{p{0.95\linewidth}}
\toprule

NLI:\\
\{\texttt{SENTENCE1}\} entails the \{\texttt{SENTENCE2}\}. True or False? answer:\\

\midrule

Open-domain QA:\\
Refer to the passage below and answer the following question. Passage: \{\texttt{DOCUMENT}\} Question: \{\texttt{QUESTION}\}\\

\midrule

Summarization:\\
\{\texttt{DIALOGUE}\} TL;DR:\\

\midrule


Sentiment analysis:\\
For each snippet of text, label the sentiment of the text as positive or negative. The answer should be exact 'positive' or 'negative'. text: \{\texttt{TEXT}\} answer:\\

\bottomrule
\end{tabular}
\caption{The prompts to LLMs for evaluating their zero-shot performance on these general tasks.}
\label{tab-prompt}
\end{table*}

\subsection{Hyper-parameters for Elastic Net}\label{hy}

In our experiment, we set \(\lambda = 5 \times 10^{-7} \), \(\mu = 5 \times 10^{-1} \) for GPT2-XL\cite{radford2019language} and \(\lambda = 5 \times 10^{-7} \), \(\mu = 1 \times 10^{-3} \) for LLaMA-3 (8B)\cite{llama3}.

\begin{figure*}[!hbt]
  \centering
  \includegraphics[width=0.5\textwidth]{figures/legend_edit.pdf}
  \vspace{-4mm}
\end{figure*}

\begin{figure*}[!hbt]
  \centering
  \subfigure{
  \includegraphics[width=0.23\textwidth]{figures/ROME-GPT2XL-CF-edit.pdf}}
  \subfigure{
  \includegraphics[width=0.23\textwidth]{figures/ROME-LLaMA3-8B-CF-edit.pdf}}
  \subfigure{
  \includegraphics[width=0.23\textwidth]{figures/MEMIT-GPT2XL-CF-edit.pdf}}
  \subfigure{
  \includegraphics[width=0.23\textwidth]{figures/MEMIT-LLaMA3-8B-CF-edit.pdf}}
  \caption{Edited on CounterFact, editing performance of edited models using the ROME~\cite{DBLP:conf/nips/MengBAB22} and MEMIT~\cite{DBLP:conf/iclr/MengSABB23} on GPT2-XL~\cite{radford2019language} and LLaMA-3 (8B)~\cite{llama3}, as the number of edits increases before and after the introduction of EAC.}
  \vspace{-4mm}
  \label{edit-performance-cf}
\end{figure*}

\begin{figure*}[!hbt]
  \centering
  \includegraphics[width=0.75\textwidth]{figures/legend.pdf}
  \vspace{-4mm}
\end{figure*}

\begin{figure*}[!htb]
  \centering
  \subfigure{
  \includegraphics[width=0.23\textwidth]{figures/ROME-GPT2XL-CounterFact.pdf}}
  \subfigure{
  \includegraphics[width=0.23\textwidth]{figures/ROME-LLaMA3-8B-CounterFact.pdf}}
  \subfigure{
  \includegraphics[width=0.23\textwidth]{figures/MEMIT-GPT2XL-CounterFact.pdf}}
  \subfigure{
  \includegraphics[width=0.23\textwidth]{figures/MEMIT-LLaMA3-8B-CounterFact.pdf}}
  \caption{Edited on CounterFact, performance on general tasks using the ROME~\cite{DBLP:conf/nips/MengBAB22} and MEMIT~\cite{DBLP:conf/iclr/MengSABB23} on GPT2-XL~\cite{radford2019language} and LLaMA-3 (8B)~\cite{llama3}, as the number of edits increases before and after the introduction of EAC.}
  \vspace{-4mm}
  \label{task-performance-cf}
\end{figure*}

\section{Experimental Results}\label{app}

\subsection{Results of Editing Performance}\label{cf-performance}
Applying CounterFact as the editing dataset, Figure~\ref{edit-performance-cf} presents the editing performance of the ROME~\cite{DBLP:conf/nips/MengBAB22} and MEMIT~\cite{DBLP:conf/iclr/MengSABB23} methods on GPT2-XL~\cite{radford2019language} and LLaMA-3 (8B)~\cite{llama3}, respectively, as the number of edits increases before and after the introduction of EAC.
The dashed line represents the ROME or MEMIT, while the solid line represents the ROME or MEMIT applying the EAC.


\subsection{Results of General Abilities}\label{cf-ability}
Applying CounterFact as the editing dataset, Figure~\ref{task-performance-cf} presents the performance on general tasks of edited models using the ROME~\cite{DBLP:conf/nips/MengBAB22} and MEMIT~\cite{DBLP:conf/iclr/MengSABB23} methods on GPT2-XL~\cite{radford2019language} and LLaMA-3 (8B)~\cite{llama3}, respectively, as the number of edits increases before and after the introduction of EAC. 
The dashed line represents the ROME or MEMIT, while the solid line represents the ROME or MEMIT applying the EAC.

\subsection{Results of Larger Model}\label{13 B}
To better demonstrate the scalability and efficiency of our approach, we conducted experiments using the LLaMA-2 (13B)~\cite{DBLP:journals/corr/abs-2307-09288}.
Figure~\ref{13B-edit} presents the editing performance of the ROME~\cite{DBLP:conf/nips/MengBAB22} and MEMIT~\cite{DBLP:conf/iclr/MengSABB23} methods on LLaMA-2 (13B)
~\cite{DBLP:journals/corr/abs-2307-09288}, as the number of edits increases before and after the introduction of EAC.
Figure~\ref{13B} presents the performance on general tasks of edited models using the ROME and MEMIT methods on LLaMA-2 (13B), as the number of edits increases before and after the introduction of EAC.
The dashed line represents the ROME or MEMIT, while the solid line represents the ROME or MEMIT applying the EAC.

\begin{figure*}[!hbt]
  \centering
  \includegraphics[width=0.5\textwidth]{figures/legend_edit.pdf}
  \vspace{-4mm}
\end{figure*}

\begin{figure*}[!hbt]
  \centering
  \subfigure{
  \includegraphics[width=0.23\textwidth]{figures/ROME-LLaMA2-13B-ZsRE-edit.pdf}}
  \subfigure{
  \includegraphics[width=0.23\textwidth]{figures/MEMIT-LLaMA2-13B-ZsRE-edit.pdf}}
  \subfigure{
  \includegraphics[width=0.23\textwidth]{figures/ROME-LLaMA2-13B-CF-edit.pdf}}
  \subfigure{
  \includegraphics[width=0.23\textwidth]{figures/MEMIT-LLaMA2-13B-CF-edit.pdf}}
  \caption{Editing performance of edited models using the ROME~\cite{DBLP:conf/nips/MengBAB22} and MEMIT~\cite{DBLP:conf/iclr/MengSABB23} on LLaMA-2 (13B)~\cite{DBLP:journals/corr/abs-2307-09288}, as the number of edits increases before and after the introduction of EAC.}
  \vspace{-4mm}
  \label{13B-edit}
\end{figure*}

\begin{figure*}[!hbt]
  \centering
  \includegraphics[width=0.75\textwidth]{figures/legend.pdf}
  \vspace{-4mm}
\end{figure*}

\begin{figure*}[!htb]
  \centering
  \subfigure{
  \includegraphics[width=0.23\textwidth]{figures/ROME-LLaMA2-13B-ZsRE.pdf}}
  \subfigure{
  \includegraphics[width=0.23\textwidth]{figures/MEMIT-LLaMA2-13B-ZsRE.pdf}}
  \subfigure{
  \includegraphics[width=0.23\textwidth]{figures/ROME-LLaMA2-13B-CounterFact.pdf}}
  \subfigure{
  \includegraphics[width=0.23\textwidth]{figures/MEMIT-LLaMA2-13B-CounterFact.pdf}}
  \caption{Performance on general tasks using the ROME~\cite{DBLP:conf/nips/MengBAB22} and MEMIT~\cite{DBLP:conf/iclr/MengSABB23} on LLaMA-2 (13B)~\cite{DBLP:journals/corr/abs-2307-09288}, as the number of edits increases before and after the introduction of EAC.}
  \vspace{-4mm}
  \label{13B}
\end{figure*}

\section{Analysis of Elastic Net}
\label{FT}
It is worth noting that the elastic net introduced in EAC can be applied to methods beyond ROME and MEMIT, such as FT~\cite{DBLP:conf/emnlp/CaoAT21}, to preserve the general abilities of the model.
Unlike the previously mentioned fine-tuning, FT is a model editing approach. It utilized the gradient to gather information about the knowledge to be updated and applied this information directly to the model parameters for updates.
Similar to the approaches of ROME and MEMIT, which involve locating parameters and modifying them, the FT method utilizes gradient information to directly update the model parameters for editing. Therefore, we incorporate an elastic net during the training process to constrain the deviation of the edited matrix.
Figure~\ref{ft} shows the sequential editing performance of FT on GPT2-XL and LLaMA-3 (8B) before and after the introduction of elastic net.
The dashed line represents the FT, while the solid line represents the FT applying the elastic net.
The experimental results indicate that when using the FT method to edit the model, the direct use of gradient information to modify the parameters destroys the general ability of the model. By constraining the deviation of the edited matrix, the incorporation of the elastic net effectively preserves the general abilities of the model.

\begin{figure*}[t]
  \centering
  \subfigure{
  \includegraphics[width=0.43\textwidth]{figures/legend_FT.pdf}}
\end{figure*}

\begin{figure*}[t]%[!ht]
  \centering
  \subfigure{
  \includegraphics[width=0.22\textwidth]{figures/FT-GPT2XL-ZsRE.pdf}}
  \subfigure{
  \includegraphics[width=0.22\textwidth]{figures/FT-GPT2XL-CounterFact.pdf}}
  \vspace{-2mm}
  \caption{Edited on the ZsRE or CounterFact datasets, the sequential editing performance of FT~\cite{DBLP:conf/emnlp/CaoAT21} and FT with elastic net on GPT2-XL before and after the introduction of elastic net.}
  \vspace{-2mm}
  \label{ft}
\end{figure*}


\end{document}
\endinput
%%
%% End of file `sample-sigconf-authordraft.tex'.
