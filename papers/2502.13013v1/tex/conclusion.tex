\section{Conclusion and Limitations}
In this paper, we introduce \ourshort, a novel humanoid teleoperation cockpit for humanoid loco-manipulation. With a low-cost isomorphic exoskeleton hardware system and a humanoid loco-manipulation policy trained by our RL training framework, ~\ourshort enables a single operator to teleoperate the whole body of humanoid robots and perform diverse loco-manipulation tasks either in the real world or in the simulation. Owing to the incorporation of an upper-body pose curriculum, a height-tracking reward, and symmetry-based techniques, Our training framework enables the development of robust loco-manipulation policies, ensuring stable walking and squatting capabilities across diverse robotic platforms, even under dynamically changing upper-body poses. Leveraging isomorphic exoskeleton arms, \ourshort enables significantly faster task execution than other systems, and our gloves are compatible with multiple kinds of dexterous hands. We present several ablation studies and real-world experiments to validate the robustness and accuracy of our system. In addition, we show the usability of collected data for IL.

\textbf{Limitations} Although we have developed robust loco-manipulation policies, these policies still fall short of ensuring reliable traversal over diverse terrains. Additionally, the 15-DoF design of the motion-sensing gloves for the thumb does not fully align with human anatomy, resulting in less intuitive and smooth operation when controlling certain dexterous robotic hands. Furthermore, the current system lacks force feedback, which limits its effectiveness in applications requiring precise haptic interaction. Addressing these limitations will be a central focus of our future research efforts.