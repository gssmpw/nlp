\begin{abstract}

Although vision-language models (VLMs) have achieved significant success in various applications such as visual question answering, their resilience to prompt distractions remains as an under-explored area. Understanding how distractions affect VLMs is crucial for improving their real-world applicability, as inputs could have noisy and irrelevant information in many practical scenarios. This paper aims to assess the robustness of VLMs against both visual and textual distractions in the context of science question answering. Built on the \emph{ScienceQA} dataset, we developed a new benchmark that introduces distractions in both the visual and textual contexts. To evaluate the reasoning capacity of VLMs amidst these distractions, we analyzed the performance of ten state-of-the-art VLMs, including GPT-4o. Our findings reveal that most VLMs are vulnerable to various types of distractions, experiencing noticeable degradation in reasoning capabilities when confronted with distractions. Notably, models such as InternVL2 demonstrates a higher degree of robustness to these distractions. We also found that models exhibit greater sensitivity to textual distractions than visual ones. Additionally, we explored various mitigation strategies, such as prompt engineering, to counteract the impact of distractions. While these strategies improved model resilience, our analysis shows that there remain significant opportunities for improvement. 

\end{abstract}