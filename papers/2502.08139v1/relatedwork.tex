\section{Related works}
\label{sec:related-work}

Putting it down in words, the next release problem can be defined as \emph{how to select a set of features or requirements to be covered by the application (or its the new release) that is being developed when they face contradictory goals}. One of the first works about \ensuremath{\mathsf{NRP}} were done by Karlsson~\cite{karlsson1996} in the 1990s. It's was not until the next decade that the term Next Release Problem \ensuremath{\mathsf{NRP}} was coined by the works of Bagnall et al.~\cite{bagnall2001} and Van den Akker et al.~\cite{van2005}. These two initial formulations intertwined offering to the software engineering community a solid set of different strategies that have been in development for the last twenty years to solve this problem  using a search based approach~\cite{pitangueira2015,hujainah2018}.

Since then,  several optimization schemes (single-objective, bi-objective, multi-objective) have been deeply studied. The original single-objective formulation~\cite{bagnall2001,van2005} deals with the search of the set of requirements or customers that maximizes the stakeholders' satisfaction, without violating the restriction defined for the development effort bound ($B$). That is, the implementation cost cannot exceed the available resources. According to this formulation, the \ensuremath{\mathsf{NRP}} is a single objective optimization problem based on the satisfaction concept~\cite{jung1998}. The bi-objective formulation of the problem~\cite{zhang2007} was obtained by incorporating cost/effort as the second objective to be optimized, giving solutions in the Pareto optimal sense. Other papers use more than two objectives~\cite{ramirez2019}. Some of them include as objectives the number of requirements and the fairness~\cite{finkelstein2009}, one objective for each customer considering their coverage~\cite{zhang2011}, dissatisfaction~\cite{nayebi2018, sangwan2020}, coverage and fairness~\cite{geng2018}, urgency~\cite{agarwal2014}, or risks~\cite{pitangueira2017}.

In the case of the single objective formulation the main solving techniques applied are  hill climbing~\cite{bagnall2001}, simulated annealing ~\cite{bagnall2001,sagrado2011}, integer linear programming~\cite{bagnall2001,van2005}, genetic algorithms~\cite{greer2004}, ant colony optimization~\cite{sagrado2010,sagrado2011}, or the approximate backbone based multilevel algorithm~\cite{Xuan2012}. Further, when several objectives are considered other algorithms have been applied, such as  algorithms based on genetic inspiration~\cite{zhang2007,durillo2011}, the application of differential evolution (i.e., evolutionary algorithms)~\cite{chaves2015}, the use of nature inspired optimization  (e.g., ant colony~\cite{sagrado2015}, bee colony~\cite{alrezaamiri2020}, whale and grey wolf optimization~\cite{ghasemi2021} or, algae algorithm~\cite{piroz2021}),  linear programming~\cite{dominguez2019}, clustering approaches~\cite{sagrado2021,HUJAINAH2021} or even,  exact methods to find the whole Pareto front~\cite{DONG2022}. Other studies go one step beyond  looking for more complex techniques applying hyper-heuristics as a search methodology that  automatize the process of selecting or combining simpler heuristics~\cite{zhang2018}, or proposing methods that are able to automatically learn from programs that solve optimization problems (Virtual Savant)~\cite{massobrio2021}.
 
Since requirements often have dependencies or interactions that force some requirements to be implemented before others or to be dropped due to mutual exclusions, another point addressed  in the literature is the management of the requirements interactions. First works that defined the different types of interactions~\cite{karlsson1996, Carlshamre2001} were followed by the use of graphs to represent implication dependencies~\cite{ngo2004} which was later extended to include all the functional dependencies between requirements~\cite{sagrado2015}. 
These functional interactions are problem constraints, being precedence/implication ones those that limit the search space~\cite{sagrado2011}. Other kinds of interactions, which are related to the increment/decrement of some attributes values assigned to the requirements (i.e., revenue-based and cost-based~\cite{Carlshamre2001}), have been addressed  using additional elements such as preference matrices~\cite{mougouei2021} or pre- and post-processing actions~\cite{sagrado2015}.

Nonetheless, the research on these interactions reveals that few of them capture valuable relationships between requirements~\cite{zhang2014}. Approaches to embed interactions in NRP vary from linear programming ~\cite{li2010,veerapen2015}, the use of a layered metamodel~\cite{aydemir2018}, graphs and matrices ~\cite{sagrado2015,sangwan2020,mougouei2021}, or the use of interactions to define an underlying probabilistic graphical model ~\cite{sagrado2023}. For instance, some of them (e.g., combination) can be addressed prior to the execution of the optimization algorithm, while others (i.e., implication, revenue-based, and cost-based) need to be embedded in the optimization algorithm~\cite{sagrado2015}.

Finally, the analysis of the set of solutions (i.e., the Pareto front) obtained for a \ensuremath{\mathsf{NRP}}, when more than one objective is included in its formulation, has been addressed in several works,  because the \ensuremath{\mathsf{NRP}} is not finished when the Pareto front is available~\cite{ferreira2007}, as the solutions have not yet been compared. Some authors suggest explaining the results using quality indicators closer to the problem domain~\cite{aguila2016} or using tools to support optimization~\cite{zhang2020,aguila2016}.

In summary, \ensuremath{\mathsf{NRP}} is an open research area, as it involves the use of different optimization schemes, solving techniques, management of requirement interactions, and also there is a need of analyzing the set of solutions obtained. None of the current formulations covers all of these points. The new \ensuremath{\mathsf{extNRP}} is intended to be a general framework that accommodates, in a single formulation, all aspects that have been considered so far, and even be open enough to incorporate new aspects that may appear in the future.  The next section will therefore present and review the requirements selection issues that influence the extension of \ensuremath{\mathsf{NRP}}, specifically the relevant requirements data that we need to consider when selecting requirements in a software engineering project.