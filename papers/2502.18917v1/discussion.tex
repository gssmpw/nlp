
\tech uses generated tests for invariant pruning, but the test suite may include spurious tests that can incorrectly prune valid invariants. The generated tests might not represent valid sequences of method calls; for example, invoking a \CodeIn{pop()} method before a \CodeIn{push()} method could fail certain assertions, leading to improper pruning.

Another limitation is the LLM's context window, which restricts the amount of code that can be processed in a single call. This limitation makes it challenging to handle large codebases. \tech partially addresses this issue through compositional generation, breaking down the code into manageable parts. Ongoing advancements in LLMs, as highlighted in recent work~\cite{liu2024lost,gao2023retrieval}, are also expected to mitigate this limitation.

For future work, we plan to integrate invariant generation with the generation of formal specifications for member functions, enabling LLM a more comprehensive understanding of program behavior. Additionally, we aim to evaluate \tech on larger and more complex systems beyond Z3, demonstrating its scalability to diverse codebases.





