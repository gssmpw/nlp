\PassOptionsToPackage{dvipsnames}{xcolor}
%\documentclass[acmsmall,screen,review,anonymous,nonacm]{acmart}
\documentclass[acmsmall,screen,nonacm]{acmart}


\acmJournal{JACM}
\acmVolume{37}
\acmNumber{4}
\acmArticle{111}
\acmMonth{8}

\usepackage{longtable}
\usepackage{tikz}
\usepackage[ruled, vlined, linesnumbered, commentsnumbered, longend]{algorithm2e}

\usepackage{xcolor}
\usepackage{xspace}
\usepackage{booktabs, multirow}
\usepackage{soul}
\usepackage{listings}
\usepackage{subcaption}
\usepackage{wrapfig}
\usepackage{enumitem}
\usepackage{wrapfig}

\renewcommand{\ttdefault}{txtt}

\definecolor{codegreen}{rgb}{0,0.6,0}
\definecolor{codegray}{rgb}{0.5,0.5,0.5}
\definecolor{codepurple}{rgb}{0.58,0,0.82}
\definecolor{backcolour}{rgb}{0.97,0.97,0.95}
\definecolor{forestgreen}{rgb}{0.28,0.62,0.37}
\definecolor{codeblue}{rgb}{0,0.5,1}

\lstdefinelanguage{markdown}{
    morekeywords={*, \#, \_, `},
    sensitive=false,
    morecomment=[l]{//},   % to allow comments in markdown syntax
    morestring=[b]",        % double quotes for strings
    postbreak={},
breakindent=0pt,
breakautoindent=false,
}


\lstdefinestyle{mystyle}{
    backgroundcolor=\color{backcolour},   
    commentstyle=\color{codepurple},
    keywordstyle=\color{codepurple},
    numberstyle=\tiny\color{codegray},
    stringstyle=\color{blue},
    basicstyle=\ttfamily\scriptsize,
    breakatwhitespace=false,         
    breaklines=true,                 
    captionpos=b,                    
    keepspaces=true,                 
    numbers=left,                    
    numbersep=5pt,                  
    showspaces=false,                
    showstringspaces=false,
    showtabs=false,                  
    tabsize=4
}

\lstset{style=mystyle}
\newcommand{\HighlightBG}{\makebox[0pt][l]{\color{yellow!50}\rule[-0.45em]{\linewidth}{1.3em}}}

\usetikzlibrary{shapes, arrows}

\begin{document}
%%%%%%%%%%%%%%%%%%%%%%%%%%%%%%%%%%%%%%%%%%%%%%%%%%%%%%%
%%%%%%%%%%%%%%%    theorems %%%%%%%%%%%%%%%%%%%%%%%%%%%
%%%%%%%%%%%%%%%%%%%%%%%%%%%%%%%%%%%%%%%%%%%%%%%%%%%%%%%
% \usepackage{mdframed}
\usepackage{kantlipsum}

%%%%%%%%%%%%%%%%%%%%%%%%%%%%%%%%%%%%%%%%%%%%%%%%%%%%%%%
%%%%%%%%%%%%%%%    theorems %%%%%%%%%%%%%%%%%%%%%%%%%%%
%%%%%%%%%%%%%%%%%%%%%%%%%%%%%%%%%%%%%%%%%%%%%%%%%%%%%%%
\theoremstyle{plain}
\newtheorem{theorem}{Theorem}[section]
\newtheorem{proposition}[theorem]{Proposition}
\newtheorem{lemma}[theorem]{Lemma}
\newtheorem{example}[theorem]{Example}
\newtheorem{corollary}[theorem]{Corollary}
\theoremstyle{definition}
\newtheorem{definition}[theorem]{Definition}
\newtheorem{assumption}[theorem]{Assumption}
\theoremstyle{remark}
\newtheorem{remark}[theorem]{Remark}


% \titleformat{\subsection}[runin]% runin puts it in the same paragraph
%        {\normalfont\bfseries}% formatting commands to apply to the whole heading
%        {\thesubsection}% the label and number
%        {0.5em}% space between label/number and subsection title
%        {}% formatting commands applied just to subsection title
%        [.]% punctuation or other commands following subsection title


%%%%%%%%%%%%%%%%%%%%%%%%%%%%%%%%%%%%%%%%%%%%%%%%%%%%%%%
%%%%%%%%%%%%%%%  mathematical notations%%%%%%%%%%%%%%%%
% \usepackage[english]{babel}
% \usepackage[utf8]{inputenc}
% \usepackage[T1]{fontenc}

%% Figures, tables and lists
\usepackage[dvipsnames]{xcolor}
\usepackage{paralist}
\usepackage{graphicx}
\usepackage{subcaption}
\usepackage{longtable} 
\usepackage{multirow}
\usepackage{listings}
\usepackage{makecell}
\usepackage{array}
\usepackage{float}
\usepackage{dsfont}
\usepackage{rotating}
\usepackage{booktabs}
\usepackage{enumerate}
\usepackage{tikz}
\usepackage{pgf}
\usepackage{enumitem}
\usepackage{lipsum} % for generating filler text
\usepackage{titlesec}

%% Math
% \usepackage{amssymb, amsthm,bbm}
\usepackage{mathtools}
\usepackage{mathrsfs}
%% References and author info 
\mathtoolsset{showonlyrefs}
\usepackage{natbib}
\usepackage{authblk}
\usepackage{todonotes}
\usepackage{xr-hyper}


%%%%%%%%%%%%%%%%%%%%%%%%%%%%%%%%%%%%%%%%%%%%%%%%%%%%%%%
\newcommand{\R}{\mathbb R}
\newcommand{\EE}{\mathbb{E}}

\DeclareMathOperator{\Tr}{Tr}
\DeclareMathOperator*{\argmin}{argmin}
\DeclareMathOperator*{\argmax}{argmax}

\newcommand{\bs}[1]{\ensuremath{\boldsymbol{#1}}}
\newcommand{\mc}{\mathcal}
\newcommand{\opt}{^\star}


\newcommand{\diff}{\textnormal{d}}


\def \iid {\stackrel{\textnormal{i.i.d.}}{\sim}}
\def \iidtext {\textnormal{i.i.d.}}





%%%%%%%%%%%%%%%%%%%%%%%%%%%%%%%%%%%%%%%%%%%%%%%%%%%%%%%
%%%%%%%%%%%%%%%%%%%%% colors     %%%%%%%%%%%%%%%%%%%%%%
%%%%%%%%%%%%%%%%%%%%%%%%%%%%%%%%%%%%%%%%%%%%%%%%%%%%%%%
\definecolor{myblue}{rgb}{.8, .8, 1}
\definecolor{mathblue}{rgb}{0.2472, 0.24, 0.6} % mathematica's Color[1, 1--3]
\definecolor{mathred}{rgb}{0.6, 0.24, 0.442893}
\definecolor{mathyellow}{rgb}{0.6, 0.547014, 0.24}


% May add more in future.







\title{\tech: Class Invariant Synthesis using Large Language Models}

\author{Chuyue Sun}
\authornote{Work done while interning at Microsoft.}
\affiliation{%
   \institution{Stanford University}
   \country{USA}
}
\email{chuyues@stanford.edu}
\author{Viraj Agashe}
\affiliation{%
   \institution{Microsoft Research}
   \country{India}
}
\email{t-vagashe@microsoft.com}
\author{Saikat Chakraborty}
\affiliation{%
   \institution{Microsoft Research}
   \country{USA}
}
\email{saikatc@microsoft.com}
\author{Jubi Taneja}
\affiliation{%
   \institution{Microsoft Research}
   \country{USA}
}
\email{jubitaneja@microsoft.com}
\author{Clark Barrett}
\affiliation{%
   \institution{Stanford University}
   \country{USA}
}
\email{barrettc@stanford.edu}
\author{David Dill}
\affiliation{%
   \institution{Stanford University}
   \country{USA}
}
\email{dill@cs.stanford.edu}
\author{Xiaokang Qiu}
\authornote{Work done while visiting Microsoft.}
\affiliation{%
  \institution{Purdue University}
   \country{USA}
}
\email{xkqiu@purdue.edu}
\author{Shuvendu K. Lahiri}
\affiliation{%
   \institution{Microsoft Research}
   \country{USA}
}
\email{shuvendu@microsoft.com}


\begin{abstract}
Formal program specifications in the form of preconditions, postconditions, and class invariants have several benefits for the construction and maintenance of programs. They not only aid in program understanding due to their unambiguous semantics but can also be enforced dynamically (or even statically when the language supports a formal verifier).  However, synthesizing high-quality specifications in an underlying programming language is limited by the expressivity of the specifications or the need to express them in a declarative manner. Prior work has demonstrated the potential of large language models (LLMs) for synthesizing high-quality method pre/postconditions for Python and Java, but does not consider class invariants.

In this work, we describe \tech, a method for co-generating executable class invariants and test inputs to produce high-quality class invariants for a mainstream language such as C++, leveraging LLMs' ability to synthesize pure functions. We show that \tech outperforms a pure LLM-based technique to generate specifications (from code) as well as prior data-driven invariant inference techniques such as Daikon. We contribute a benchmark of standard C++ data structures along with a harness that can help measure both the correctness and completeness of generated specifications using tests and mutants. We also demonstrate its applicability to real-world code by performing a case study on several classes within a widely used and high-integrity C++ codebase.
\end{abstract}


\maketitle

\section{Introduction}
\section{Introduction}
\label{sec:intro}

\begin{figure*}[tb]
    \centering
    \includegraphics[width=0.848\linewidth]{figs/circuitnn.pdf} 
    \caption{Illustration of differentiable CircuitNN. CircuitNN is designed based on differentiable NAND gates. After DAS is guided by PI and PO pairs of the truth table, CircuitNN can get the precise circuit architecture logic equivalent to the truth table.}
    \label{fig:circuitnn}
\end{figure*}

% 1. Describe the importance of logic synthesis
% 2. Existing Problems
% (a) Neural Architecture Search: Unstable, Predefined Setting, etc.
% (b) Circuit Generation: Probabilistic Model, Logic Equivalence

With the rapid advancement of technology, the scale of integrated circuits (ICs) has expanded exponentially. 
This expansion has introduced significant challenges in chip manufacturing, particularly concerning power and area metrics.
A primary objective in IC design is achieving the same circuit function with fewer transistors, thereby reducing power usage and area occupancy.

Logic synthesis~\cite{hachtel2005logicsynth}, a critical step in electronic design automation (EDA), transforms behavioral-level circuit designs into optimized gate-level circuits, ultimately yielding the final IC layout. 
The primary goal of logic synthesis is to identify the physical implementation with the fewest gates for a given circuit function. 
This task constitutes a challenging NP-hard combinatorial optimization problem. 
Current logic synthesis tools~\cite{brayton2010abc, wolf2013yosys} rely on human-designed heuristics, often leading to sub-optimal outcomes.

Differentiable architecture search (DAS) techniques~\cite{liu2018darts, chu2020darts} offer novel perspectives on addressing challenges in this problem.
Circuit functions can be represented through truth tables, which map binary inputs to their corresponding outputs. 
Truth tables provide a precise representation of input-output relationships, ensuring the design of functionally equivalent circuits.
Inspired by this, researchers~\cite{deepmind2024ai4sys, wang2024tnet} have begun exploring the application of DAS to synthesize circuits directly from truth tables.
Specifically, \citet{deepmind2024ai4sys} proposed CircuitNN, a framework that learns differentiable connection structures with logic gates, enabling the automatic generation of logic circuits from truth tables.
This approach significantly reduces the complexity of traditional circuit generation. 
Building on this, \citet{wang2024tnet} introduced T-Net, a triangle-shaped variant of CircuitNN, incorporating regularization techniques to enhance the efficiency of DAS.

Despite these advancements, several challenges remain. 
The computational complexity of DAS grows quadratically with the number of gates, posing scalability issues.
Although triangle-shaped architecture~\cite{wang2024tnet} partially mitigates this problem, redundancy persists. 
%Additionally, DAS is susceptible to converging to local optima, limiting the ability to search architectures that satisfy the given truth tables~\cite{liu2018darts}. 
%Furthermore, hyperparameters (network depth and layer width) require extensive searches, introducing complexity and prolonging the synthesis process. 
Additionally, DAS is susceptible to converging to local optima~\cite{liu2018darts} and hyperparameters (network depth and layer width) require extensive searches. 
The challenges arise from the vast search space in DAS. 
% Even with predefined settings for CircuitNN, finding a configuration that meets the truth table requires extensive trial and error during the DAS process. 
Intuitively, limiting the search space through predefined parameters (network depth, gates per layer, and connection probabilities) can significantly reduce the complexity.

Recent advances~\cite{openai2023gpt4, abramson2024alphafold3, esser2024sd3, li2024mar} in conditional generative models have demonstrated remarkable performance across language, vision, and graph generation tasks. 
Motivated by these developments, we propose a novel approach to circuit generation that generates preliminary circuit structures to guide DAS in generating refined circuits matching specified truth tables. 
Firstly, we introduce CircuitVQ, a tokenizer with a discrete codebook for circuit tokenization. 
Built upon our Circuit AutoEncoder framework~\cite{hou2022graphmae,li2023maskgae,wu2025mgvga}, CircuitVQ is trained through a circuit reconstruction task. 
Specifically, the CircuitVQ encoder encodes input circuits into discrete tokens using a learnable codebook, while the decoder reconstructs the circuit adjacency matrix based on these tokens.
Subsequently, the CircuitVQ encoder serves as a circuit tokenizer for CircuitAR pretraining, which employs a masked autoregressive modeling paradigm~\cite{chang2022maskgit, li2023mage}. 
In this process, the discrete codes function as supervision signals. 
After training, CircuitAR can generate discrete tokens progressively, which can be decoded into initial circuit structures by the decoder of the CircuitVQ. 
These prior insights can guide DAS in producing refined circuits that match the target truth tables precisely.

Our key contributions can be summarized as follows:
\begin{itemize}
\item We introduce CircuitVQ, a circuit tokenizer that facilitates graph autoregressive modeling for circuit generation, based on our Circuit AutoEncoder framework;
\item Develop CircuitAR, a model trained using masked autoregressive modeling, which generates initial circuit structures conditioned on given truth tables;
\item Propose a refinement framework that integrates differentiable architecture search to produce functionally equivalent circuits guided by target truth tables;
\item Comprehensive experiments demonstrating the scalability and capability emergence of our CircuitAR and the superior performance of the proposed circuit generation approach.
\end{itemize}

% Motivation
% (a) Diffusion (Vision, Graph), Autoregressive (Language, Vision)
% (b) Circuit Generation for Predefined Setting
% (c) Neural Architecture Search for Strict Logic Equivalence

% Contribution
% (a) Circuit Tokenizer (new transformer arch, training strategy)
% (b) CircuitAR (train and gen strategies, post-ar strategy)
% (c) Extensive Evaluation including BitD (Bit Distance) for Scalability




\section{Running Example: AVL Tree}
\label{sec:avl_intro}

\section{Motivation}
\label{sec:motivation}



% In LLM inference, not only does weight matter, but the memory requirements of the KV Cache are also considerable.
In this section, we first demonstrate that the emerging paradigm of group quantization demands a high level of adaptivity, which current adaptive methods lack.
We then discuss how adapting these methods to group quantization could compromise their efficiency.
Given that LLMs generate KV caches during runtime, real-time quantization capability is crucial.
These challenges lead to our proposal of a mathematical adaptive numerical type (\texttt{MANT}), which we will detail later.



\begin{figure}[t]
    \centering
    \begin{minipage}[t]{0.48\columnwidth}
      \centering
      \includegraphics[width=\columnwidth]{fig/moti_group_ppl.pdf}
      \caption{LLM accuracy with different quantization granularities. We report the perplexity (PPL) metric (lower is better).}\label{fig:moti_group_ppl} 
    \end{minipage}
    \hspace{2pt}
    \begin{minipage}[t]{0.48\columnwidth}
      \centering
      \includegraphics[width=\columnwidth]{fig/motivation_adaptive_ppl.pdf}
      \caption{Accuracy loss for \texttt{INT}, \texttt{ANT}, and Ideal (clustering algorithm K-Means) adaptive methods in group quantization. }\label{fig:moti_ppl} 
    \end{minipage}
    % \vspace*{-0.3cm}
\end{figure}




\subsection{Group Quantization Accuracy Analysis}
\label{sec:acc_analysis}

In this subsection, we begin by comparing the accuracy of traditional channel-wise quantization with group-wise quantization~\cite{shao2024omniquant,zhao2023atom,liu2024kivi,sheng2023flexgen,lin2023awq,zhao2023atom}, establishing the baseline for group-wise quantization in this study.
We then delve into the use of various adaptive data types in group quantization, emphasizing the necessity for full adaptivity.



\Fig{fig:moti_group_ppl} illustrates the perplexity when quantizing the LLaMA-7B model~\cite{touvron2023llama} with various granularities using the \texttt{INT4}-based symmetric quantization.
Channel-wise quantization significantly worsens the perplexity of the examined LLM, increasing it from 5.68 to 6.85.
Conversely, group-wise quantization mitigates this loss in perplexity with a group size of 128, corresponding to an average of 4.125 bits per element (16-bit scaling factor).
Additionally, we observe that a smaller group size of 32 offers only a slight improvement in perplexity, but the scaling factor overhead increases by $4\times$.



Given this analysis, we adopt a group size of 128 as our standard configuration for the remainder of this section.
Previous research indicates that the \texttt{INT} data type is not optimal for accuracy since tensors or channels exhibit varied distributions, leading to the proposal of various adaptive data types~\cite{guo2022ant, guo2023olive, zadeh2020gobo, zadeh2022mokey}.
We evaluate their efficacy in the context of group quantization, which falls into two main categories: data-type-based and clustering-based.



\textbf{Data-type-based adaptive methods} select data types from discrete sets based on tensor data distribution.
ANT~\cite{guo2022ant} is a representative example of the data-type-based method.
ANT packages several different data types for selection, including \texttt{INT} for the uniform distribution, \texttt{PoT} (Power of Two) for the Laplace distribution, and \texttt{flint} for the Gaussian distribution.
%ANT designed \texttt{flint} for Gaussian distributions.

\textbf{Clustering-based adaptive methods} utilize clustering algorithms to generate centroids that align with the data distribution and provide considerable adaptivity. 
Mokey~\cite{zadeh2022mokey} and GOBO~\cite{zadeh2020gobo} exemplify this approach, though they focus on tensor- or channel-wise quantization. In our study, we adapt them to group quantization through per-group clustering.

%Clustering-based methods employ clustering algorithms to generate centroids that fit the data distribution, demonstrating sufficient adaptivity.
%Mokey~\cite{zadeh2022mokey} and GOBO~\cite{zadeh2020gobo} are such presentative works, but only target tensor- or channel-wise quantization.
%In our work, we modify those works to support group quantization by performing per-group clustering.
\Fig{fig:moti_ppl} compares the accuracy of the methods described above for the LLaMA-7B model under 4-bit group-wise quantization. 
The group-wise \texttt{ANT} method outperforms the \texttt{INT} type by dynamically selecting from three data types to better match the value distribution, resulting in reduced perplexity (PPL) loss. 
Moreover, per-group clustering adjusts more effectively to the value distribution of each group, establishing itself as the accuracy-optimal and ideal adaptive method. 
This approach achieves nearly lossless 4-bit quantization, equivalent to 16 centroids per group. 
However, this ideal scenario is impractical due to the significant overhead associated with storing per-group centroids, effectively rendering it a 6-bit quantization.

\begin{figure}[t] 
    \centering 
    \includegraphics[width=1.0\linewidth]{fig/intro_cdf.pdf}  
    \caption{The cumulative distribution function (CDF) of the tensor, channel, and group, respectively. The tensor data were taken from layers 8 to 23, while the 16 channel and group data were sampled from one tensor with specific strides.}\label{fig:moti_dist} 
    %  \vspace*{-0.3cm}
\end{figure}

To illustrate the group-wise diversity in data distribution, we sampled the weights of the Q and V tensors in LLaMA-7B model. 
We normalized all sampled data to their absolute maximum values, which ranged from -1 to 1. \Fig{fig:moti_dist} displays the cumulative distribution function (CDF) for the tensor, channel, and group levels, respectively. 
We observed that the diversity at the group level is significantly higher than at the tensor level. 
In simpler terms, while different tensors exhibit similar distributions, groups can have markedly different distributions. This finding underscores the necessity for full adaptivity in group quantization to fully realize its potential.
\paragraph{Takeaway 1.} The group quantization is an emerging paradigm to accelerate LLMs, and the significant group-level diversity requires a high level of adaptivity to fully unleash its potential.

\subsection{Group Quantization Efficiency Analysis}
\label{subsec:efficiency}


In this subsection, we provide a detailed efficiency analysis for the above adaptive quantization methods.
In \Tbl{intro:dtype}, we compare OliVe~\cite{guo2023olive}, ANT~\cite{guo2022ant}, GOBO~\cite{zadeh2020gobo}, and Mokey~\cite{zadeh2022mokey} with \texttt{INT} regarding the efficiency of computation, encoding, and decoding. 
In this paper, we use the term encoding (decoding) interchangeably with quantization (dequantization).
 

Data-type-based adaptive methods such as ANT~\cite{guo2022ant} and Olive~\cite{guo2023olive} achieve computational efficiency comparable to \texttt{INT}. 
Both utilize specialized decoders that decode these data types prior to computation, resulting in high decoding efficiency. 
However, as previously demonstrated, these methods suffer from limited adaptivity in the group quantization paradigm. 
A straightforward approach to enhance adaptivity is to expand their set of data types. 
However, incorporating new data types necessitates additional decoders, escalating hardware design costs. 
Additionally, compatibility issues between new and existing data types may reduce computational efficiency. 
For instance, the \texttt{NF4} data type~\cite{dettmers2023qlora} requires an FP16 MAC unit, which is incompatible with existing \texttt{ANT} data types.


\paragraph{Takeaway 2.} Enhancing the data-type-based adaptive method for group quantization is challenging and requires a careful balance for the computation and decoding efficiency.

Clustering-based adaptive methods like GOBO~\cite{zadeh2020gobo} and Mokey~\cite{zadeh2022mokey} can sufficiently adapt to various distributions at the group level. 
However, they require codebooks for quantization and dequantization, leading to high adaptivity at the expense of encoding and computational efficiency. 
For instance, a 16-entry codebook with 8 bits per entry requires 128 bits per group, creating an inevitable trade-off between adaptivity and memory overhead. GOBO~\cite{zadeh2020gobo} employs the K-means algorithm to quantize weights and requires dequantization to \texttt{FP16} using a codebook lookup table before computation, resulting in high adaptivity but low computational efficiency. 
Conversely, Mokey~\cite{zadeh2022mokey} enhances the computation of clustering-based methods by using indices for centroid values via approximate calculations, though matrix multiplication still relies on floating-point units, increasing overhead compared to integer units. 
Furthermore, Mokey creates one \texttt{golden dictionary} for all activations and weights, akin to using a single data type in quantization, thus reducing adaptivity.


\paragraph{Takeaway 3.} Deploying the clustering-based adaptive methods under group quantization is challenging owing to the low encoding and computation efficiency. 


\begin{table}[t]
    \centering
    \small
    \renewcommand{\arraystretch}{1.2}
    \caption[]{Features of DNN accelerators with adaptive and flexible data types are summarized. Here, `Effi.' stands for efficiency, `Med.' for medium, and `LUT' for lookup table.}
  
    \resizebox{1.0\columnwidth}{!}{
      \begin{tabular}{c|cc|ccc|cc|c}
        \Xhline{1.2pt}
        \multirow{2}{*}{Architecture} & \multicolumn{2}{c|}{Encode} & \multicolumn{3}{c|}{Computation} & \multicolumn{2}{c|}{Decode} & \multirow{2}{*}{Adaptivity} \\ \cline{2-8}
        & Method & Effi. & Method & Bit & Effi. & Method & Effi. \\
        \Xhline{1.2pt}
        \texttt{INT} & Round & High & INT & 4 \& 8 & High & Calculation & High & Low \\ 
        OliVe~\cite{guo2023olive} & Search & Med. & INT & 4 \& 8 & High & Decoder & High & Med. \\ 
        ANT~\cite{guo2022ant} & Search & Med. & INT & 4 \& 8 & High & Decoder & High & Med. \\ 
        Mokey~\cite{zadeh2022mokey} & Cluster & Med. & Float & 4 \& 8 & Med. & Calculation & Med. & Low \\ 
        GOBO~\cite{zadeh2020gobo} & Cluster & Low & Float & 16 & Low & LUT & Med. & High \\ 
        \hline
        \multirow{2}{*}{\proj}  & Search  & Med.  & \multirow{2}{*}{INT} & \multirow{2}{*}{4 \& 8} & \multirow{2}{*}{High} & \multirow{2}{*}{Calculation} & \multirow{2}{*}{High} & \multirow{2}{*}{High} \\ \cline{2-3}
        &  Map &  High &  &&&\\ 
        \Xhline{1.2pt}
    \end{tabular}
    }
    \vspace*{0.1cm}
    \label{intro:dtype}
    \vspace*{-0.2cm}
  \end{table}

\subsection{Support for Real-time Quantization}
\label{sec:moti_kvcache}

The above group-wise diversity presents a challenge for both weights and KV cache.
In addition, KV cache faces challenges in real-time group-wise quantization because the KV cache is generated dynamically during LLM inference.


To facilitate low-precision computation in group-wise quantization, it is necessary to quantize K and V along the inner dimension. 
This requirement stems from the support for matrix inner product operations in most GPUs and TPUs. 
During these operations, the group-wise scaling factor can be extracted from the multiply-accumulate process. 
\Fig{fig:kv_process} depicts the computation process of K and V during the decode stage. We define the dimension used for matrix inner product operations as the inner dimension. 
The inner dimensions of the K and V caches differ; the K cache requires a transpose operation, whereas the V cache does not, complicating the situation.


In the prefill stage, K and V can easily compute the scaling factor for each group. 
During the decode stage, the newly generated K vector is concatenated along the inner dimension of the K cache, enabling immediate quantization. 
However, the newly generated V vector is associated with different groups, with only one element per group produced per iteration. This process prevents the scaling factor for the entire group from being obtained in a single iteration, posing a significant challenge for the real-time quantization of the V cache.


\begin{figure}[t] 
  \centering 
  % \includegraphics[width=1.0\linewidth]{fig/dse_kv_process.pdf}  
  \includegraphics[width=0.9\linewidth]{fig/moti_kv_dimension.pdf}  
  \caption{\small Comparison of group-wise K and V cache quantization. They have different inner dimensions due to the transposition of K (key).}

  \label{fig:kv_process}
  % \vspace*{-0.4cm}
\end{figure}


Given those challenges, we propose \proj with a mathematical encoding format that can fuse with integer computation and enhance the decoding efficiency.
In addition, this encoding format provides sufficient adaptivity for group-wise quantization.
Regarding the challenge in KV cache, \proj employs a real-time quantization engine that ensures efficient encoding and decoding for KV cache.
By addressing these challenges, \proj enables efficient low-bit group-wise quantization.



\section{Approach}
\label{sec:approach}
\section{Methodology}
\label{sec:approach}

\begin{figure}[!t]
\centering
\includegraphics[width=0.5\textwidth]{Pipeline.png}
\caption{Workflow. For each synthesis or sketching task, we create an input query for the LLM such that the query contains the target property in natural language or Alloy (depending on the kind of task), run the query, get the LLM's output, and use the Alloy analyzer to validate it with respect to a reference (ground truth) formula.}
\label{fig:workflow}
\end{figure}

We consider the following three methods for employing large language models (LLMs) to create Alloy formulas to investigate the capabilities and limitations of LLMs in writing Alloy:

\begin{enumerate}
\item
{\bf English to Alloy}. We employ LLMs to write complete Alloy formulas in multiple different ways from given natural language descriptions (in English);
\item
{\bf Alloy to Alloy}. We employ LLMs to create multiple alternative but equivalent formulas in Alloy with respect to given formulas in Alloy; and
\item
{\bf Sketch to Alloy}. We employ LLMs to complete sketches~\cite{SolarLazemaPhD2008,WangETALABZ2018ASketch} of Alloy
formulas and populate the holes in the sketches by synthesizing Alloy
expressions and operators so that the completed formulas accurately
represent the desired properties (that are given in natural language).  \end{enumerate}

\begin{table}[!t]
\begin{tabular}{r@{\hskip 0.2cm}|l|p{4cm}|p{5cm}}
& \multicolumn{1}{c|}{\Intro{Property}} & \multicolumn{1}{c|}{\Intro{Natural language desc.}} & \multicolumn{1}{c}{\Intro{Reference Alloy formula}}\\
\hline
1 & DAG & Directed acyclic graph &
\begin{lstlisting}[style=AlloyTable]
all n: Node | n !in n.^link
\end{lstlisting} \\
\hline
2 & Cycle & Graph with directed cycle &
\begin{lstlisting}[style=AlloyTable]
some n: Node | n in n.^link
\end{lstlisting} \\
\hline
3 & Circular & The number of nodes is equal to the number of edges and the graph has a directed cycle that visits all nodes &
\begin{lstlisting}[style=AlloyTable]
#Node = #link
all n: Node | one n.link
all m, n: Node | m in n.^link
\end{lstlisting} \\
\hline
4 & Connex & For every pair of elements in S, either the first is related to the second or vice versa &
\begin{lstlisting}[style=AlloyTable]
all s, t: S |
  s->t in r or t->s in r
\end{lstlisting} \\
\hline
5 & Reflexive & Every element in S is related to itself &
\begin{lstlisting}[style=AlloyTable]
all s: S | s->s in r
\end{lstlisting} \\
\hline
6 & Symmetric & If element x in S is related to y, then y is also related to x &
\begin{lstlisting}[style=AlloyTable]
all s, t: S |
  s->t in r implies t->s in r
\end{lstlisting} \\
\hline
7 & Transitive & If element x in S is related to y and y is related to z, then x is also related to z &
\begin{lstlisting}[style=AlloyTable]
all s, t, u: S |
  s->t in r and t->u in r
    implies s->u in r
\end{lstlisting} \\
\hline
8 & Antisymmetric & If element x in S is related to y and y is related to x, then x and y are the same element &
\begin{lstlisting}[style=AlloyTable]
all s, t: S |
  s->t in r and t->s in r
    implies s = t
\end{lstlisting} \\
\hline
9 & Irreflexive & No element in S is related to itself &
\begin{lstlisting}[style=AlloyTable]
all s, t: S |
  s->t in r implies s != t
\end{lstlisting} \\
\hline
10 & Functional & Every element in S is related to at most one element (making r a partial function) &
\begin{lstlisting}[style=AlloyTable]
all s: S | lone s.r
\end{lstlisting} \\
\hline
11 & Function & Every element in S is related to exactly one element (making r a total function) &
\begin{lstlisting}[style=AlloyTable]
all s: S | one s.r
\end{lstlisting} \\
\hline
\end{tabular}
\vspace*{2ex}
\caption{Subject properties. The table lists for each property, its
  natural language description that defines the corresponding natural
  language to Alloy task, and its reference formulation in Alloy that
  defines the corresponding Alloy to Alloy
  task.}\label{tab:subjects-synthesis}
\vspace*{-4ex}
\end{table}


\begin{table}[!h]
\centering
\begin{tabular}{p{12cm}}
\hline
\begin{lstlisting}[style=AlloyTable]
pred DAG {
  // Directed acyclic graph
  all n: Node | \E,e\ \CO,co\ \E,e\
}
co := {| =|in|!=|!in |}
e := {| Node|n|((Node|n).(*|^)link) |}
\end{lstlisting} \\ \hline

\begin{lstlisting}[style=AlloyTable]
pred Cycle {
  // Graph with directed cycle
  some n: Node | \E,e\ \CO,co\ \E,e\
}
co := {| =|in|!=|!in |}
e := {| Node|n|((Node|n).(*|^)link) |}
\end{lstlisting} \\ \hline

\begin{lstlisting}[style=AlloyTable]
pred Circular {
  // The number of nodes is equal to the number of edges and the graph has a directed cycle that visits all nodes
#Node = #link
  all n: Node | one n.link
  all m, n: Node | \E,e\ \CO,co\ \E,e\
}
co := {| =|in|!=|!in |}
e := {| (Node|m|n).(*|^)link |}
\end{lstlisting} \\ \hline

\end{tabular}
\vspace*{2ex}
\caption{Sketches for Alloy specifications for Properties 1--3.}
\vspace*{-8ex}
\label{tab:sketches-1-3}
\end{table}

Figure~\ref{fig:workflow} graphically illustrates our approach.
For each synthesis or sketching task, we create an input query for the LLM such that the query contains the target property in natural language or Alloy (depending on the kind of task), run the query, get the LLM's output, and run the Alloy analyzer to validate it with respect to a ground truth formula, which we provide to the analyzer. There are three possible outcomes of running the Alloy analyzer: (1) the LLM's answer is correct (when the analyzer does not find a counterexample to the equivalence of the LLM's answer and ground truth); (2) the LLM's answer has a syntax error (when the analyzer fails to compile the LLM's answer); and (3) the LLM's answer is wrong (when the analyzer finds a counterexample to the equivalence of the LLM's answer and ground truth). Note for "Alloy to Alloy" synthesis tasks, the ground truth formula is the reference formula given as input to the LLM. Note also that for any "English to Alloy" synthesis task and for any "Sketch to Alloy" sketching task, the input to the LLM does not include the ground truth formula.

We employ the LLMs directly as available for public use.  Specifically, we do not fine-tune them.  Moreover, the queries we write are minimalistic in their description of the problem domain and do not provide instructions to the LLM on how to approach solving any given task.

\subsection{Subject tasks}

We use \NumSubjects~well-known properties of graphs and binary relations to create \NumTotalTasks~tasks for the LLMs to answer.  Three of the properties (DAG, Cycle, and Circular) are regarding edge-labeled graphs, and the remaining eight properties (Connex, Reflexive, Symmetric, Transitive, Antisymmetric, Irreflexive, Functional, and Function) are regarding binary relations.  In Alloy, in general, we can use one signature $S$ and one binary relation $r: S\times S$ to represent either an edge-labeled graph or a binary relation. However, in view of the specific domain of graphs, we name the signature `\CodeIn{Node}' and the binary relation `\CodeIn{link}' when creating the tasks relating graph properties. For the tasks relating properties of binary relations, we name the signature `\CodeIn{S}' and the relation `\CodeIn{r}'.

For each property, we create 2~kinds of synthesis tasks: (1) create 20~unique Alloy formulas that represent the given natural language description of the property; and (2) create 20~unique Alloy formulas that are equivalent to the given Alloy formula that captures the property, which is also included as a natural language comment in the prompt.  In addition, for each property, we create one sketching task: complete the given sketch of the property with respect to its natural language description that is included as a comment in the prompt.  Thus, for each property, we have a total of 3~tasks for the LLM to answer.

Table~\ref{tab:subjects-synthesis} lists each property, its natural language description, and a reference (ground truth) formula that characterizes it in Alloy. Moreover, Tables~\ref{tab:sketches-1-3}, \ref{tab:sketches-4-8} (Appendix), and \ref{tab:sketches-9-11} (Appendix) list each property, its sketch that defines the corresponding sketching problem. Together these four tables summarize the key elements of our tasks for the LLMs. To illustrate, consider the DAG property.  Figure~\ref{fig:three-tasks-for-DAG} describes the actual prompts we run against each LLM for this property.

\begin{figure}[!p]
\centering
\begin{tcolorbox}[mytextbox]
Give me 20 unique solutions to the problem of synthesizing the body of the following Alloy predicate (without markdown or comments) with respect to the property described in the comments:
\begin{lstlisting}
sig Node {
  link: set Node
}
pred DAG{
  // Directed acyclic graph
  // your code go here
}
\end{lstlisting}
\end{tcolorbox}
(a) "English to Alloy" task\\
\begin{tcolorbox}[mytextbox]
Give me 20 unique solutions to the problem of synthesizing the body of the following Alloy predicate (without markdown or comments) with respect to the property described in the comments:
\begin{lstlisting}
sig Node {
  link: set Node
}
pred DAG{
  // Directed acyclic graph
  all n: Node | n !in n.^link
}
\end{lstlisting}
\end{tcolorbox}
(b) "Alloy to Alloy" task\\
\begin{tcolorbox}[mytextbox]
Complete the following sketch of the Alloy predicate (without markdown or comments) by selecting values for the holes with respect to the given constraints such that the predicate is correct with respect to the property described in the comments:

\begin{lstlisting}
sig Node {
  link: set Node
}
pred DAG {
  // Directed acyclic graph
  all n: Node | \E,e\ \CO,co\ \E,e\
}

co := {| =|in|!=|!in |}
e := {| Node|n|((Node|n).(*|^)link) |}
\end{lstlisting}
\end{tcolorbox}
(c) "Sketch to Alloy" task
\caption{Three tasks for the LLMs with respect to the DAG property.}
\label{fig:three-tasks-for-DAG}
\end{figure}

In a predicate sketch, certain components of the predicate are placeholder holes~\cite{WangETALABZ2018ASketch}. These holes can be of different forms, e.g., comparison operator holes, expression holes, and quantifier holes.  For all our sketching tasks, we only use two kinds of holes: comparison operator holes and expression holes. A predicate sketch includes a definition of the sets of possible values that each hole can be completed with.  These sets are typically defined using regular expressions~\cite{SolarLazemaPhD2008}.  For our DAG sketching task, the comparison operator hole may be completed with one of four possible values from the set \{ `\CodeIn{=}', `\CodeIn{in}', `\CodeIn{!=}', `\CodeIn{!in}'\}, and each expression hole may be completed with one of six possible values from the set \{ `\CodeIn{Node}', `\CodeIn{n}', `\CodeIn{Node.*link}', `\CodeIn{Node.\^{}link}', `\CodeIn{n.*link}', `\CodeIn{n.\^{}link}' \}.






\section{Results}
\label{sec:results}
\subsection{Benchmark}
\label{sec:benchmark}
Frontier language models demonstrate a remarkable mismatch between their problem-solving capabilities and poor out-of-box verification capabilities.
These limitations have largely been attributed to the inability of current language models to self-diagnose hallucinations or enforce rigour \citep{zhang_how_2023,orgad_llms_2024,snyder_early_2024,kamoi_evaluating_2024, tyen_llms_2024, DBLP:conf/iclr/0009CMZYSZ24}.
However, our findings that models can be directed to accurately perform verifications at scale suggest that these out-of-box limitations can be addressed with standard methods like instruction tuning.
We compiled a set of challenging reasoning problems and candidate solutions to provide a benchmark for these deficits.

Each entry in this benchmark consists of a question, a correct candidate response, and an incorrect candidate response, and is manually curated from the residuals of our sampling-based search experiments (Section~\ref{section:pipeline}).
An example entry from this benchmark can be found below (see Appendix~\ref{app:examplebenchmark} for more).

\vspace{0.4cm}
\begin{tcolorbox}[title=Question from LiveBench Reasoning (Web-of-Lies Puzzle), breakable]
In this question, assume each person either always tells the truth or always lies. The person at the campground thinks their friend is lying. Mateo is at the aquarium. The person at the restaurant says the person at the hotel lies. Farid is at the movie theater. The person at the movie theater says the person at the campground lies. Ryan is at the shopping mall. The person at the cafe says the person at the campground lies. The person at the observatory says the person at the museum lies. The person at the museum says the person at the restaurant tells the truth. The person at the campground says the person at the airport tells the truth. The person at the campground tells the truth. Luna is at the restaurant. The person at the movie theater saw a firetruck. Isabella is at the observatory. The person at the amusement park says the person at the museum tells the truth. The person at the restaurant says the person at the shopping mall tells the truth. Liam is at the beach. Ayaan is at the airport. The person at the restaurant says the person at the beach tells the truth. Maya is at the amusement park. The person at the restaurant says the person at the aquarium lies. The person at the restaurant says the person at the movie theater lies. The person at the library says the person at the campground tells the truth. Elowen is at the hotel. Priya is at the library. The person at the gym says the person at the restaurant tells the truth. Kehinde is at the cafe. Anika is at the museum. Quan is at the campground. Soren is at the gym. Does the person at the museum tell the truth? Does the person at the restaurant tell the truth? Does the person at the movie theater tell the truth? Think step by step, and then put your answer in **bold** as a list of three words, yes or no (for example, **yes, no, yes**). If you don't know, guess.
\end{tcolorbox}
\definecolor{mydarkgreen}{rgb}{0.0, 0.5, 0.0} %


\begin{paracol}{2}

\switchcolumn[0]
\begin{tcolorbox}[breakable, colframe=mydarkgreen!70, colback=mydarkgreen!10, title=Correct response]
Here's how we can break down this logic puzzle:

1. **Quan is at the campground and tells the truth.** This is a given fact.

...

4. **The person at the restaurant (Luna) says the person at the hotel (Elowen) lies.**  We don't know yet if Luna is lying or telling the truth.

...

10. **Since Luna tells the truth about Farid, Luna *always* tells the truth.**  This unlocks several other pieces of information.

...

13. **The person at the cafe (Kehinde) says the person at the campground lies.** Since Quan tells the truth, Kehinde is lying.

Therefore:
* **Anika (museum):** Tells the truth.
* **Luna (restaurant):** Tells the truth.
* **Farid (movie theater):** Lies.

**yes, yes, no**
\end{tcolorbox}

\switchcolumn
\begin{tcolorbox}[breakable, colframe=red!70, colback=red!10, title=Incorrect Response Due to Hallucination]
Here's how we can solve this logic puzzle:

1. **Quan is at the campground and tells the truth.** This is given.

2. **The person at the airport (Ayaan) tells the truth.** Quan (campground) says the person at the airport tells the truth, and Quan tells the truth.

...

4. **Luna (restaurant) lies.**  The person at the restaurant says the person at the hotel (Elowen) lies. \textbf{\color{red} Later, the person at the restaurant says the person at the campground (Quan) lies.} Since Quan tells the truth, Luna (restaurant) must be lying.

...

Therefore:
* **Anika (museum): lies**
* **Luna (restaurant): lies**
* **Farid (movie theater): lies**

**Answer: no, no, no** 
\end{tcolorbox}
\end{paracol}

\noindent\begin{minipage}{\textwidth}
\captionof{figure}{Example of an entry in our verification benchmark. The question is sourced from the LiveBench Reasoning benchmark, and the two responses are generated by Gemini v1.5 Pro. The green response has the correct final answer; the red response has the wrong final answer due to hallucinating a non-existent clause.}
\label{fig:example}
\vspace{0.4cm}
\end{minipage}

\noindent
On each entry, our benchmark studies verification accuracy on two tasks:
\begin{enumerate}
    \item \textbf{Scoring task.} When given only the question and one of the responses, is the model able to discern the correctness of the response?
    \item \textbf{Comparison task.} When provided the whole tuple with the correctness labels of the responses masked and a guarantee that at least one response is correct, is the model able to discern which response is correct and which is incorrect?
\end{enumerate}

\noindent
The scoring task is also evaluated over a separate set of (question, response) pairs where the response reaches the correct final answer by coincidence but contains fatal errors and should be labeled by a reasonable verifier as being incorrect; an example can be found in Appendix~\ref{app:examplebenchmark}.
In the scoring task, models are provided only with the task description; in the comparison task, models are provided only with the task description and a suggestion to identify disagreements between responses in its reasoning.

Table~\ref{tab:benchmark} lists the baseline performances of current commercial model offerings on this benchmark.
Gemini v1.5 Pro is omitted from the benchmark as the entries in the benchmark are curated from the residuals of Gemini v1.5 Pro.
The prompts used in Table~\ref{tab:benchmark} are provided in Appendix~\ref{app:benchmarkprompts}.

As we previously observed, and has been noted in prior works \citep{tyen_llms_2024, kamoi_evaluating_2024}, verification errors are typically due to low recall.
Even the easier comparison task, models perform only marginally better---and often worse---than random chance.
In many cases, Consistency@5 performs worse than one-shot inference because Consistency simply averages out noise from an output distribution, meaning that a model biased towards producing an incorrect answer will do so with higher probability under Consistency.
Addressing these deficits in verification capabilities---which we see as low-hanging fruit for post-training---would enable not only better sampling-based search, but also other downstream applications of verification including reinforcement learning \citep[e.g.][]{o1-preview,deepseekai2025deepseekr1incentivizingreasoningcapability}, data flywheeling \citep[e.g.,][]{welleck_generating_2022}, and end-user experience (see Section~\ref{sec:related} for further discussion).


\begin{table}[htbp]
\centering
\begin{tabular}{llcccccc}
\toprule
\textbf{Model} & \textbf{Metric} & \multicolumn{3}{c}{\textbf{Scoring Accuracy}} & \multicolumn{1}{c}{\textbf{Comparison Accuracy}} \\
\cmidrule(lr){3-5} \cmidrule(lr){6-6}
 &  & \textbf{Correct} & \textbf{Wrong} & \textbf{Flawed} &  \\
\midrule
\multirow{2}{*}{GPT-4o} & Pass@1    & 76.5\%  & 31.0\% & 22.2\% & 43.2\%\\
 & Consistency@5 & 77.4\% & 30.0\% & 11.1\% & 35.4\% \\
\midrule
\multirow{2}{*}{Claude 3.5 Sonnet} & Pass@1 & 89.6\% & 22.5\% & 33.3\% & 56.1\% \\
 & Consistency@5 & 90.3\% & 17.5\% & 33.3\% & 61.2\% \\
\midrule
\multirow{2}{*}{o1-preview} & Pass@1 & 100\% & 68.8\% & 80.0\% & 84.5\% \\
 & Consistency@5 & 100\% & 79.4\% & 88.8\% & 92\% \\
\midrule
\multirow{2}{*}{Gemini 2.0 Flash} & Pass@1 & 73.5\% & 44.5\% & 60\% & 58\%  \\
 & Consistency@5 & 77.4\% & 42.5\% & 66.6\% & 58.7\% \\
\midrule
\multirow{2}{*}{Gemini 2.0 Thinking Flash} & Pass@1 & 75.4\% & 56.5\% & 53.3\%  & 80\%  \\
 & Consistency@5 & 77.4\%  & 55\% & 55.5\%  & 89.1\% \\
\midrule
\multicolumn{2}{c}{Random guessing}  & 80\% & 20\% & 20\% & 50\% \\
\bottomrule
\end{tabular}
\caption{Accuracy rates of commercial language models on our verification benchmark. For the task of response scoring (Scoring Accuracy), accuracy rates are broken down for entries that require identifying a correct response as being correct (Correct), entries that require identifying a wrong response as being wrong (Wrong), and entries that require identifying a wrong response that coincidentally reaches the correct answer as being wrong (Flawed).
GPT-4o and Claude 3.5 Sonnet only perform marginally better than random guessing across all tasks. o1-Preview performs better, but still fails to identify 20-30\% of wrong responses.
}
\label{tab:benchmark}
\end{table}

\subsection{Evaluation}
\section{Experiments: Planning outperforms Heuristics}
\label{sec:experiment}

We begin our empirical demonstrations by showcasing the effectiveness of our planning framework on both synthetic and real datasets. We focus on the simplest planning algorithm, 1-step lookaheads (Algorithm~\ref{alg:complete}), and show that even basic planning can hold great promise. 
We illustrate our framework using two uncertainty quantification modules---GPs and 
\ensembles/ \ensembleplus. 

Throughout this section, we focus on evaluating the mean squared error of 
a regression model $\model$,  and develop adaptive policies that minimize uncertainty on $g(f)$ defined in~\eqref{eqn:l2-g-f}.
When GPs provide a valid model of uncertainty, 
our experiments show that our planning framework significantly outperforms other baselines. 
We further demonstrate that our conceptual framework extends to deep learning-based uncertainty quantification methods such as  \ensembleplus while highlighting computational challenges that need to be resolved in order to scale our ideas. 
For simplicity, we assume a naive predictor, i.e., $\psi(\cdot) \equiv 0$. However, we emphasize that this problem is just as complex as if we were using a sophisticated model $\psi(.)$. The performance gap between the algorithms 
primarily depends
on the level  of uncertainty in our prior beliefs.

To evaluate the performance of our algorithm, we benchmark it against several baselines. 
%Active learning baselines use an acquisition function $\ac$ to select points that have the highest   function value: $X\opt_t \in \argmax_{X \in \xpoolj{t}} \ac({X})$ at every step $t$. These methods may also need an UQ module, which we simply use the same UQ module as in our algorithm, and it  outputs $V(X)$ that measures the the uncertainty of each point $X \in \xpoolj{t}$.
Our first set of baselines are from active learning~\citep{AggarwalKoGuHaPh14}:
\\ % \noindent\textbf{Active Learning Heuristics:} 
\textbf{(1)} 
\textsf{Uncertainty Sampling (Static):}  In this approach, we query the samples for which the model is least certain about. Specifically, we estimate the variance of the latent output $f(X)$ for each $X \in \xpool$ using the UQ module and select the top-$K$ points with the highest uncertainty. \\
\textbf{(2)} \textsf{Uncertainty Sampling (Sequential):} This is a greedy heuristic that sequentially selects the points with the highest uncertainty within a batch, while updating the posterior beliefs using pseudo labels from the current posterior state. Unlike \textsf{Uncertainty Sampling (Static)}, this method takes into account the information gained from each point within batch, and hence tries to diversify the selected points within a batch. 

 
We also compare our approach to the  \textbf{(3)} \textsf{Random Sampling}, which selects each batch uniformly at random from the pool. Additionally, we compare solving the planning problem using  \textsf{REINFORCE}-based policy gradients with   $\mathsf{Smoothed\text{-}Autodiff}$ policy gradients.\footnote{Our code repository is available at
  \url{https://github.com/namkoong-lab/adaptive-labeling}.}
%Detailed experimental setups are provided in Section \ref{sec:details-experiments}.

%We repeat all experiments with 10 random seeds.




\begin{figure}[t]
\centering
\begin{minipage}[b]{0.49\textwidth}
\centering
\includegraphics[width=\textwidth, height=5cm]{figures/original_scale/Var_of_l_2_loss.pdf}
\caption{(Synthetic data) Variance of mean squared loss evaluated through the posterior belief $\mu_t$ at each horizon $t$. This is the objective that policy gradient methods like \textsf{REINFORCE} and $\ouralgo$ optimizes. 1-step lookaheads are surprisingly effective even in long horizons.}
\label{fig:var-l2-sim}
\end{minipage}
\hfill
\begin{minipage}[b]{0.49\textwidth}
\centering \includegraphics[width=\textwidth, height=5cm]{figures/original_scale/Error_of_estimated_model_l_2_loss.pdf}
\caption{(Synthetic data) Error between MSE calculated based on collected data $\mc{D}^{0:T}$ vs. population oracle MSE over $\mc{D}_{\rm eval} \sim P_X$. Reducing uncertainty over posteriors directly leads to better OOD evaluations. 1-step lookaheads significantly outperform active learning heuristics in small horizons.}
\label{fig:mean-l2-sim}
\end{minipage}
%\caption{Simulated data for GPs}
%\label{fig:both_plots}
\end{figure}

\subsection{Planning with Gaussian processes}
\label{sec:experiment-plan-GP}
We now briefly describe the data generation process for the GP experiments,  deferring a more detailed discussion of the dataset generation to Section~\ref{sec:details-experiments}. 
We use both the synthetic data and the real data to test our methodology.
For the \emph{simulated data},  we construct a setting where the general population is distributed across \emph{51 non-overlapping clusters} while the initial labeled data $\dtrain$ just comes from one cluster. In contrast, both $\dpool \defeq (\xpool,\ypool),\deval \defeq (\xeval,\yeval)$ are generated   from all the clusters. 
We begin with a low-dimensional scenario, generating a one-dimensional regression setting using a GP. %Gaussian Process (GP).
Although the data-generating process is not known to the algorithms,  we assume that the GP hyperparameters are known to all the algorithms
to ensure fair comparisons. This can be viewed as a setting where our prior is well-specified, allowing us to isolate the effects
of different policy optimization approaches
 without any concerns about the misspecified priors. We select $10$ batches, each of size $K=5$ across $T = 10$ time horizons.

To examine the robustness of our method against the distributional assumptions made  in the simulated case, we then move to a real dataset where the correct prior is not known. We simulate selection bias from the eICU dataset~\citep{PollardJoRaCeMaBa18}, which contains real-world patient data with in-hospital mortality outcomes. 
We conduct a $k$-means clustering to generate 51 clusters and then select data from those clusters. We view this to be a credible replication of practice, as severe distribution shifts are common due to selection bias in clinical labels.  To convert the binary mortality labels into a regression setting, we train a  random forest classifier and fit a GP on predicted scores, which serves as the UQ module for all the algorithms. As before, the task is to select 10 batches, each consisting of 5 samples, across 10 time horizons.

 In Figures~\ref{fig:var-l2-sim} and~\ref{fig:mean-l2-sim}, we present results for the simulated data. 
Figure~\ref{fig:var-l2-sim} shows the variance of $\ell_2$ loss, and Figure~\ref{fig:mean-l2-sim} presents the error in the estimated $\ell_2$ loss using $\mu_t$ (relative to true $\ell_2$ loss, that is unknown to the algorithm). 
As we can see from these plots, our method one-step lookahead  gives substantial improvements  over active learning baselines and random sampling. In addition,
compared to the one-step lookahead planning approach using \textsf{REINFORCE}-based policy gradients, 
we observe that $\mathsf{Smoothed\text{-}Autodiff}$-based policy gradients provide significantly more robust performance over all horizons.

In Figures~\ref{fig:var-l2-real}~and~\ref{fig:mean-l2-real}, we observe similar findings on the eICU data. We see that planning policies (\textsf{REINFORCE} and $\mathsf{Smoothed\text{-}Autodiff}$) consistently outperform other heuristics by a large margin.  Active learning baselines perform poorly in these small-horizon batched problems and can sometimes be even worse than the random search baselines.  Overall, our results show the importance of careful planning in adaptive labeling for reliable model evaluation. 

We offer some intuition as to why one-step lookahead planning may outperform other heuristic algorithms. 
 First,  \textsf{Uncertainty sampling (Static)} while myopically selects the
 top-$K$ inputs with the highest uncertainty, it fails to consider 
the overlap in information content among the ``best” instances; see \citep{AggarwalKoGuHaPh14} for more details. 
In other words,  it might acquire points from the same region with high uncertainty while failing to induce diversity among the batch.
Although \textsf{Uncertainty Sampling (Sequential)} somewhat addresses the issue of information overlap, a significant drawback of 
this algorithm
is the disconnect between the objective we aim to optimize and the algorithm. For example, it might sample from a region with high uncertainty but very low density. 

\begin{figure}[t]
\centering
\begin{minipage}[b]{0.48\textwidth}
\centering
\includegraphics[width=\textwidth, height=5cm]{figures/original_scale/Var_of_l_2_loss_real.pdf}
\caption{(Real-world eICU data) Variance of mean squared loss evaluated through the posterior belief $\mu_t$ at each horizon $t$. Even 1-step lookaheads are extremely effective planners, and auto-differentiation-based pathwise policy gradients provide a reliable optimization algorithm based on low-variance gradient estimates.}
\label{fig:var-l2-real}
\end{minipage}
\hfill
\begin{minipage}[b]{0.48\textwidth}
\centering \includegraphics[width=\textwidth, height=5cm]{figures/original_scale/Error_of_estimated_model_l_2_loss_real.pdf}
\caption{(Real-world eICU data) Error between MSE calculated based on collected data $\mc{D}^{0:T}$ vs. population oracle MSE over $\mc{D}_{\rm eval} \sim P_X$. Reducing uncertainty over posteriors directly leads to better OOD evaluations. Our method significantly outperforms active learning-based heuristics, and random sampling.}
\label{fig:mean-l2-real}
\end{minipage}
%\caption{Real data for GPs}
\end{figure}
 
%\vspace{-1.5cm}
% \begin{wrapfigure}{r}{.32\columnwidth}
%   \vspace{-.5cm} 
%   \centering
% \includegraphics[scale=.29]{figures/Var of l2l_2 loss.pdf}
%   \vspace{-0.2cm}
%   \caption{Results of GP}
% \label{fig:var-l2-gp}
%   \vspace{-0.1cm}
% \end{wrapfigure}


% Attempts have been made  in the past to address these  drawbacks heuristically  (see \citep{AggarwalKoGuHaPh14}). We give a unified computational framework while approaching the problem in a more principled manner and solving it more optimally.




\subsection{Planning with  neural network-based uncertainty quantification methods ($\ensembleplus$)}


We now provide a proof-of-concept that shows the generalizability of our conceptual framework  to the deep learning-based UQ modules, specifically focusing on $\ensembleplus$ due to their previously observed superior performance~\citep{OsbandWenAsDwIbLuRo23}. Recall that implementing our framework with deep learning-based UQ modules  requires us to retrain the model across multiple possible random actions $\bm{a}(\theta)$ sampled from the current policy $\pi_\theta$.
This requires significant computational resources, in sharp contrast to the GPs where the posteriors are in closed form and can be readily updated and differentiated. 

Due to the computational constraints, we test $\ensembleplus$ on a toy setting to demonstrate the generalizability of our framework. We consider a setting where the general population consists of four clusters, while the initial labeled data only comes from one cluster. Again we generate data using GPs.  The task is to select a batch of 2 points in one horizon. We detail the $\ensembleplus$ architecture in Section \ref{sec:details-experiments}, and we assume prior uncertainty to be large (depends on the scaling of the prior generating functions). 
The results are summarized in the Table~\ref{tab:UQ_ensemble}.

% \begin{table}[H]
% \vspace{-10pt}
% \caption{Performance under \ensembleplus as UQ module}
%     \centering
%     \begin{tabular}{|m{3cm}|m{2.5cm}|m{2cm}|} 
%     \hline
%       Algorithm   & Variance of $\loss_2$ loss estimate & Error of $\loss_2$ loss estimate  \\ \hline Random Sampling 
%          & $1710.9 \pm 1352.1$ & $8.67\pm6.62$ 
%       \\ \hline \ouralgo & $1.30 \pm 0.68$ & $0.91\pm0.25$ \\ \hline
%     \end{tabular}
%     \label{tab:UQ_ensemble}
%     %\vspace{-10pt}
% \end{table}




\begin{table}[h]
\vspace{-10pt}
\caption{Performance under \ensembleplus as the UQ module}
\centering
\begin{tabular}{|l|l|l|}
\hline
Algorithm   & Variance of $\loss_2$ loss estimate & Error of $\loss_2$ loss estimate  \\
\hline
\textsf{Random sampling} & 7129.8 $\pm$ 1027.0 & 136.2 $\pm$ 8.28 \\ \hline
\textsf{Uncertainty sampling (Static)} & 10852 $\pm$ 0.0 & 162.156 $\pm$ 0.0 \\ \hline
\textsf{Uncertainty sampling (Sequential)} & 8585.5 $\pm$ 898.9 & 144 $\pm$ 6.93 \\ \hline
\textsf{REINFORCE} & 1697.1 $\pm$ 0.0 & 45.27 $\pm$ 0.0 \\ \hline
\ouralgo & 1697.1 $\pm$ 0.0 & 45.27 $\pm$ 0.0 \\ \hline
\end{tabular}
%\caption{Comparison of different algorithms based on variance   and   error in $\ell_2$ loss estimation with Ensemble $+$ as the UQ module. Our results demonstrate that {\ouralgo} and REINFORCE outperformthe other active learning based heuristics, confirming the benefits of our MDP formulation for the adaptive labeling problem, as also demonstrated in Section 4.\\
%\footnotesize{Experimental details: We use Gaussian Processes as our data generating process, GP parameters are the same as in Section D.3.  The task is to select a batch of 2 points along one horizon.The marginal distribution $p_X$ has 4 \textit{non-overlapping} clusters. Initial data comes from one cluster, while pool and evaluation points comes from all the clusters. We have $20$ initial labeled data points, $10$ pool points, and $252$ evaluation points.  Training procedures are similar to the one in Section D.3.} }
\label{tab:UQ_ensemble}
\end{table}



% We faced  issues in scaling up these experiments which will be our focus in the future. 





% \begin{itemize}
%     \item Posteriors should be consistent. Two dimensions: even with less training,  
%     \item the inference should be  fast enough
% \end{itemize}


% Potential research directions for uncertainty quantification

% In this section we consider a simple setting We consider a simpler setting and 


% For synthetic dataset generation, we use ...... For real datasets, we use ...... We compare our methodolgy to several baselines ()    This Section is structured as follows:
% \begin{itemize}
%     \item \textbf{GPs, square loss objective} (Section \ref{}): 
%     %the broad aim of the experiments  in this section is to isolate the performance of our methodology without any concerns for the inefficiencies induced due to a mis-specified prior or imperfect posterior inference. To accomplish this we generate synthetic datasets using GPs (detailed later). We use the well specified prior (GPs - with same hyperparameter setting) as our UQ module.   
%      As GPs provide differentaible posterior inference - any errors induced due to imperfect posterior updates are also isolated. We note that under this setting
%      \item In Section\ref{} we demonstrate why our methodology performs better than other baselines - by devising various synthetic experiments ()
%     \item  \textbf{UQ Benchmarking }(Section \ref{}): Before diving into the experiments using $\ensembleplus$ and ENNs,  we showcase our benchmarking experiments in Section \ref{}. We use real datasets We observe that ENNs perform better
%      \item \textbf{Ensemble $+$}, objective: recall, accuracy
%     \item \textbf{ENN}, objective: recall, accuracy
% \end{itemize}




% In Section {}, we test 
% \subsection{Experimental details}

% \begin{itemize}
%     \item UQ methodologies - GPs, ENNs
%     \item Objectives - Recall,  ATE
%     \item Datasets - ATE-synthetic datasets, Recall-synthetic, real datasets
%     \item Baselines - 
%     \begin{itemize}
%         \item Random sampling
%         \item Active learning - Uncertainty based sampling - In regression setting almost all of the 
%         \item Myopic greedy - Greedy Batch based sampling
%         \item Policy Gradient
%     \end{itemize}
    
% \end{itemize}

% \subsection{Experiments}
%     \begin{itemize}
%     \item GPs with square loss
%     \item Benchmarking ENN
%         \item ENNs with ATE
%         \item ENNs with Recall
%     \end{itemize}

% \subsection{Benefits over other algorithms - intuition and experiments}

%Active learning - Myopic greedy / Don't rely on the objective rather some entropy version.


%%% Local Variables:
%%% mode: latex
%%% TeX-master: "main"
%%% End:

\begin{table}[ht!]
\centering
\caption{\textbf{Super Resolution Performance Results.} Our proposed WGAN EEG Spatial Upsampling method significantly outperforms a baseline of Bicubic Interpolation commonly used in EEG upsampling pipelines.}
\label{tab:results}
\resizebox{0.8\linewidth}{!}{%
\begin{tabular}{@{}cccccc@{}}
\toprule
\multirow{2}{*}{\textbf{Dataset}} & \multirow{2}{*}{\textbf{Scale}} & \multicolumn{2}{c}{\textbf{Bicubic}} & \multicolumn{2}{c}{\textbf{WGAN}} \\ \cmidrule(l){3-6} 
                      &   & \textbf{MSE} & \textbf{MAE} & \textbf{MSE}    & \textbf{MAE}   \\
\toprule
\multirow{2}{*}{Val}  & 2 & 3.71E7       & 3.89E3       & \textbf{2.01E3} & \textbf{24.38} \\
                      & 4 & 7.23E7       & 6.42E3       & \textbf{8.53E3} & \textbf{63.83} \\
\midrule
\multirow{2}{*}{Test} & 2 & 3.75E7       & 3.91E3       & \textbf{2.06E3} & \textbf{24.66} \\
                      & 4 & 7.30E7       & 6.45E3       & \textbf{8.68E3} & \textbf{64.39} \\
\bottomrule
\end{tabular}%
}
\end{table}

\section{Case study: Z3 \CodeIn{bdd\_manager} class}
\label{sec:z3}
\begin{table}[!htp]\centering
\caption{Statistics of the studied data structures in Z3}\label{tab:z3-data}
\small
\begin{tabular}{l|rrrrrrrr}\toprule
&ema &dlist &heap &hashtable &permutation &scoped\_vector &bdd\_manager \\\midrule
\# LoC &57 &243 &309 &761 &177 &220 &1635 \\
\# dependencies &0 &2 &1 &9 &2 &2 &13 \\
\bottomrule
\end{tabular}
\end{table}

As a real-world case study, we apply \tech to synthesize invariants for 7 core data structures from Z3~\cite{z3}, ranging from the simple 57-line \CodeIn{ema} class to the complex 1635-line \CodeIn{bdd\_manager}. The complete set includes \CodeIn{dlist}, \CodeIn{heap}, \CodeIn{hashtable}, \CodeIn{permutation}, and \CodeIn{scoped\_vector}, with varying implementation complexity and the number of dependent classes as shown in Table~\ref{tab:z3-data}. 
Our results were validated by one of the Z3 authors, who confirmed at least one \emph{correct and useful} invariant for each studied class, with the \CodeIn{bdd\_manager} class yielding 11 valuable invariants including the 2 already present from 8 generations.

Z3 is a widely adopted SMT solver used in a variety of high-stakes applications requiring rigorous correctness, such as formal verification, program analysis, and automated reasoning. It is integrated into tools like LLVM~\cite{llvm}, KLEE~\cite{klee}, Dafny~\cite{leino2010dafny} and Frama-C~\cite{kirchner2015frama}.
We selected the Z3 codebase due to its stringent correctness requirements; as an SMT solver, Z3 is employed in applications demanding high reliability. This high-stakes environment makes Z3 an ideal testbed for assessing the effectiveness of synthesized invariants.


The \CodeIn{bdd\_manager} class\footnote{\url{https://github.com/Z3Prover/z3/blob/master/src/math/dd/dd_bdd.h}} is particularly noteworthy. 
It was chosen because it is a self-contained example with developer-written unit tests for validation, presenting a realistic yet manageable challenge. 
Note that the existing developer tests were used after invariants were generated, not as input to the LLM. 
The \CodeIn{bdd\_manager} class in Z3 is a utility for managing Binary Decision Diagrams (BDDs), which are data structures used to represent Boolean functions efficiently. In BDDs, Boolean functions are represented as directed acyclic graphs, where each non-terminal node corresponds to a Boolean variable, and edges represent the truth values of these variables (\textit{true} or \textit{false}). This representation simplifies complex Boolean expressions and enables efficient operations on Boolean functions.


With 382 lines of code in its header and 1253 lines in the implementation file, \CodeIn{bdd\_manager} surpasses standard data structure complexity, offering an opportunity to evaluate \tech's capability to generate meaningful invariants relevant to real-world scenarios. 
\tech achieves this by compositional generation, recursively traversing the source program's AST (Section~\ref{sec:llm}). 
Recursive generation became crucial when handling large classes like \CodeIn{bdd\_manager}, which exceeded the LLM’s context window. Decomposing and processing its components separately allowed us to fit relevant parts into the model’s input, demonstrating the utility of recursive invariant generation for large codebases. This supports its relevance in real-world applications beyond the benchmarks.

The  \CodeIn{bdd\_manager} class includes a developer-written member function for checking its well-formedness, as shown in Figure~\ref{fig:bdd_well_formed}, which we removed during \tech generation. Of the 56 invariants generated by \tech, one of Z3 main authors identified 11 distinct \textit{correct and useful} invariants (e.g., Figure~\ref{fig:bdd_correct_useful}) including the 2 developer-written invariants; these invariants could potentially be integrated into the codebase.
An additional 5 distinct \textit{ok} invariants (e.g., Figure~\ref{fig:bdd_ok})  are labeled correct but have limited utility, 16 distinct \textit{correct but useless} invariants (e.g., those already checked during compilation, such as type checks and constants, Figure~\ref{fig:bdd_correct_useless}), and 2 \textit{incorrect} invariants (e.g., Figure~\ref{fig:bdd_incorrect}). The remaining invariants were repetitions within these categories. This evaluation aligns with \tech's validation results, as our validation pipeline also identified 2 incorrect invariants that failed \CodeIn{bdd\_manager} unit tests.
\shuvendu{Why did you choose to report invariants that already failed the unit test?}

Overall, the Z3 authors' evaluation results further confirm \tech's potential utility in real-world, large-scale codebases.

\begin{figure}[htp]
    \centering
\begin{lstlisting}[language=c++]
bool bdd_manager::well_formed() {
    bool ok = true;
    for (unsigned n : m_free_nodes) {
        ok &= (lo(n) == 0 && hi(n) == 0 && m_nodes[n].m_refcount == 0);
        if (!ok) {
            IF_VERBOSE(0, verbose_stream() << "free node is not internal " << n << " " << lo(n) << " " << hi(n) << " " << m_nodes[n].m_refcount << "\n";
            display(verbose_stream()););
            UNREACHABLE();
            return false;
        }
    }
    
    for (bdd_node const& n : m_nodes) {
        if (n.is_internal()) continue;
        unsigned lvl = n.m_level;
        BDD lo = n.m_lo;
        BDD hi = n.m_hi;
        ok &= is_const(lo) || level(lo) < lvl;
        ok &= is_const(hi) || level(hi) < lvl;
        ok &= is_const(lo) || !m_nodes[lo].is_internal();
        ok &= is_const(hi) || !m_nodes[hi].is_internal();
        if (!ok) {
            IF_VERBOSE(0, display(verbose_stream() << n.m_index << " lo " << lo << " hi " << hi << "\n"););
            UNREACHABLE();
            return false;
        }
    }
    return ok;
}
\end{lstlisting}
    \caption{Z3 developer-written class invariants for \CodeIn{bdd\_manager} class}
    \label{fig:bdd_well_formed}
\end{figure}

\begin{figure}[htp]
    \centering
\begin{lstlisting}[language=c++]
// Node consistency: Each node's index should match its position in m_nodes
for (unsigned i = 0; i < m_nodes.size(); ++i) {
    assert(m_nodes[i].m_index == i);
}
\end{lstlisting}
    \caption{\textit{Correct and useful} invariant for \CodeIn{bdd\_manager} class}
    \label{fig:bdd_correct_useful}
\end{figure}

\begin{figure}[htp]
    \centering
\begin{lstlisting}[language=c++]
// Cache consistency: Entries in the operation cache should be valid
for (const auto* e : m_op_cache) {
    assert(e != nullptr);
    assert(e->m_result != null_bdd);
}
\end{lstlisting}
    \caption{\textit{Ok} invariant for \CodeIn{bdd\_manager} class}
    \label{fig:bdd_ok}
\end{figure}

\begin{figure}[htp]
    \centering
\begin{lstlisting}[language=c++]
// m_is_new_node is a boolean
assert(m_is_new_node == true || m_is_new_node == false);
\end{lstlisting}
    \caption{\textit{Correct and useless} invariant for \CodeIn{bdd\_manager} class}
    \label{fig:bdd_correct_useless}
\end{figure}


\begin{figure}[htp]
    \centering
\begin{lstlisting}[language=c++]
// The number of nodes should not exceed the maximum number of BDD nodes
assert(m_nodes.size() <= m_max_num_bdd_nodes);
\end{lstlisting}
    \caption{\textit{Incorrect} invariant for \CodeIn{bdd\_manager} class}
    \label{fig:bdd_incorrect}
\end{figure}



\section{Related Work}
\section{Related Work}
\label{sec:related-work}
%We now contextualize our work with related literature so that our contributions are highlighted. We cover FMTS, perturbations in time-series, 
% robustness testing of FMs, 
%and rating of AI systems. 

\noindent \textbf{Foundation Models Supporting Time Series} 
The use of FMs for time series forecasting has advanced significantly. 
% \cite{lu2022frozen} first demonstrated that transformers pre-trained on text data (LLMs) can effectively solve sequence modeling tasks in other modalities, paving the way for leveraging language pre-trained transformers for time series analysis. Recent studies have focused on reprogramming LLMs for time series tasks through parameter-efficient fine-tuning and suitable tokenization strategies \cite{zhou2023one, gruver2024large, jin2023time, cao2023tempo, ekambaram2024tiny}. These methods have successfully adapted transformers to the unique challenges of time series forecasting. \cite{zhou2023one} and \cite{jin2023time} further illustrate the versatility and robustness of fine-tuned language pre-trained transformers for diverse time series tasks.
\cite{lu2022frozen} showed that transformers pre-trained on text data can solve sequence modeling tasks in other modalities, enabling their application to time series analysis. Recent studies have reprogrammed LLMs for time series tasks through parameter-efficient fine-tuning and tokenization strategies \cite{zhou2023one, gruver2024large, jin2023time, cao2023tempo, ekambaram2024tiny}. 
% These methods have successfully adapted transformers to the unique challenges of time series forecasting. 
\cite{zhou2023one} and \cite{jin2023time} further illustrate the versatility and robustness of fine-tuned language pre-trained transformers for diverse time series tasks.
% Several models have contributed to the advancement of time series forecasting. \cite{ansari2024chronos} and \cite{woo2024unified} have improved forecasting accuracy and model generalization.  
% % \cite{ansari2024chronos} and \cite{woo2024unified} have pushed the boundaries of forecasting accuracy and model generalization. 
% \cite{rasul2023lag} and \cite{das2023decoder} have explored new tokenization strategies and fine-tuning methods to improve model performance. Additionally, \cite{garza2023timegpt} and \cite{ekambaram2024tiny} have focused on creating lightweight and efficient models for real-time applications. \cite{talukder2024totem} stands out with its unique approach to integrating multiple temporal patterns, enhancing forecasting precision.
% FMs trained from scratch have achieved SOTA on time series tasks. Zero-shot forecasting, exemplified by \cite{gruver2024large}, showcases the ability of these models to make accurate predictions without domain-specific training. \cite{cao2023tempo} and \cite{goswami2024moment} have introduced approaches to enhance the performance and efficiency of time series models, leveraging transformer architectures to capture temporal dependencies more effectively. In our experiments, we select Gemini-V and Phi-3 as the GP models and Chronos and MOMENT as TS models due to their SOTA performance in their respective categories.
Several models have advanced time series forecasting. \cite{ansari2024chronos} and \cite{woo2024unified} have improved forecasting accuracy and model generalization, while
% \cite{ansari2024chronos} and \cite{woo2024unified} have pushed the boundaries of forecasting accuracy and model generalization. 
\cite{rasul2023lag} and \cite{das2023decoder} have explored new tokenization strategies and fine-tuning methods. \cite{garza2023timegpt} and \cite{ekambaram2024tiny} developed lightweight models for real-time applications, and \cite{talukder2024totem} integrated multiple temporal patterns to improve precision. FMs trained from scratch, like \cite{gruver2024large}, achieved SOTA in zero-shot forecasting, with \cite{cao2023tempo} and \cite{goswami2024moment} further improving model performance. 
%In our experiments, we select Gemini-V and Phi-3 as the GP models and Chronos and MOMENT as TS models due to their SOTA performance in their respective categories.
Please see Section~\ref{sec:exp_app} for the FMTS we selected due to their SOTA performance in their respective categories.

%The use of FMs for time series forecasting has seen significant advancements in recent years. \cite{lu2022frozen} first demonstrated that transformers pre-trained on text data (LLMs) can effectively solve sequence modeling tasks in other modalities. This work opened the door to leveraging language pre-trained transformers for time series analysis. Recent studies have built on this foundation, focusing on reprogramming LLMs for time series tasks through parameter-efficient fine-tuning and suitable tokenization strategies \cite{zhou2023one, gruver2024large, jin2023time, cao2023tempo, ekambaram2024tiny}. These methods have proven successful in adapting the powerful capabilities of transformers to the unique challenges of time series forecasting. OneFitsAll \cite{zhou2023one} and Time-LLM \cite{jin2023time} further illustrate how language pre-trained transformers can be fine-tuned for diverse time series tasks, demonstrating their versatility and robustness. 
% \zhen{reason why we didn't include these models in our study, weights not available? or other justification, to prevent that naturally raised question from readers.}\kl{Good point. We need to discuss. I added 2 sentences at the bottom but they are probably not very convincing.}
%Several other models have contributed to the advancement of time series forecasting. Chronos \cite{ansari2024chronos} and Moirai \cite{woo2024unified} have pushed the boundaries of forecasting accuracy and model generalization. Lag-llama \cite{rasul2023lag} and TimesFM \cite{das2023decoder} have explored new tokenization strategies and fine-tuning methods to improve model performance. Additionally, Time-GPT1 \cite{garza2023timegpt} and Tiny-Time Mixers \cite{ekambaram2024tiny} have focused on creating lightweight and efficient models suitable for real-time applications. TOTEM \cite{talukder2024totem} stands out with its unique approach to integrating multiple temporal patterns, further enhancing forecasting precision.
%Aside from reprogramming LLMs for time series, FMs trained from scratch have achieved SOTA on times series tasks. 
%Zero-shot forecasting, exemplified by \cite{gruver2024large}, showcases the ability of these models to make accurate predictions without domain-specific training.  TEMPO \cite{cao2023tempo} and MOMENT \cite{goswami2024moment} have introduced approaches to enhance the performance and efficiency of time series models, leveraging transformer architectures to capture temporal dependencies more effectively.
% \zhen{and these are on various time series tasks including time series forecasting?}
% \zhen{These are models that are specifically trained for time series forecasting, I'd suggest mentioning them first after the LLM reprogramming, and then expanding to the models that are trained across time series tasks instead. The flow of this subsection feels a bit odd as of now.} \kl{Done.}
%In our experiments, we select Gemini-V and Phi-3 as the GP models and Chronos and MOMENT as TS models due to their SOTA performance in their respective categories. 

%\vspace{-0.3em}
\noindent \textbf{Perturbations in Time Series Data} TS data is commonly stored in spreadsheets and databases, which are prone to changes due to acts of omission (e.g., negligence, data-entry errors) or commission (e.g., adversarial attacks, sabotage). Omission errors are most common \cite{spreadsheets-errors-risks-survey}. Tools like Microsoft Excel and Google Sheets are widely used for data collection and analysis, allowing end-user programming \cite{spreadsheets-future-workshop}. However, over 90\% of spreadsheets contain errors due to issues like incorrect formulae, leading to multi-billion dollar losses \cite{spreadsheet-qa-survey}.
%\cite{spreadsheet-qa-survey,spreadsheets-errors-risks-survey}.
Adversarial attacks are also increasing in data stores and AI models for tasks like forecasting.
% \cite{papernot2016transferability} introduced a black-box attack method using a substitute model to generate adversarial examples, demonstrating transferability across tasks. \cite{baluja2017adversarial} focused on white-box attacks using gradient information. 
\cite{karim2019adversarial} adapted these concepts to time series, exploring both black-box and white-box attacks. \cite{oregi2018adversarial} revealed the vulnerability of distance-based classifiers. \cite{rathore2020untargeted} examined various adversarial attacks on time series classifiers. TSFool \cite{li2022tsfool} introduced a multi-objective black-box attack to craft imperceptible adversarial time series to fool RNN classifiers.
%Time series (TS) data is widely stored and manipulated in spreadsheets and databases. These are also the tools which see considerable changes or perturbations due to acts of omission that are unintended (e.g., negligence, data-entry errors) or commission which are deliberate (e.g., adversarial attacks, sabotage). 
%Among these, changes due to omission are most common \cite{spreadsheets-errors-risks-survey}.
%For example, a spreadsheet, implemented in tools like Microsoft Excel and Google Sheets, is a common data collection and analysis environment that also allows end-user programming \cite{spreadsheets-future-workshop}. They are used widely at the workplace and are often a door opener to more advanced scientific tools. But gaining expertise in them needs practice since a large proportion of spreadsheets ($\succ$ 90\%) are known to have errors due to issues like incorrect formulae caused by improper understanding of behavior during routine operations like copy-paste and end-user programming, which have caused losses of multi-billion dollars \cite{spreadsheet-qa-survey,spreadsheets-errors-risks-survey}.
% \zhen{do we need to relate our perturbations to these attacks? otherwise, we must manage the readers' expectations on what types of perturbations we focus on other than adversarial attacks, and motivate it properly}
%\zhen{Play down this a bit, and emphasize and justify why we focus on the type of perturbations we consider in the paper, to mimic operational errors in practice apart from adversarial attacks, citing the 2024 and 1996 papers Biplav added.} 
%Furthermore, adversarial attacks are also increasing both in data stores and in AI models created to solve tasks like forecasting.
%Foundational work by ~\cite{papernot2016transferability} introduced a black-box attack method that involved training a substitute model to generate adversarial examples capable of misleading the target model, demonstrating the transferability property across similar tasks. In contrast, research by ~\cite{baluja2017adversarial} focused on white-box attacks, using gradient information and probabilistic outputs to craft adversarial examples. Researchers~\cite{karim2019adversarial} have adapted these concepts to the time series domain, exploring both black-box and white-box attacks on time series classification models. In addition, ~\cite{oregi2018adversarial} revealed the susceptibility of distance-based time-series classifiers to adversarial examples. ~\cite{rathore2020untargeted} examined untargeted, targeted, and universal adversarial attacks on time series classifiers, demonstrating the effectiveness of these attacks across various datasets. TSFool~\cite{li2022tsfool} introduced a multi-objective black-box attack to craft highly imperceptible adversarial time series to fool RNN classifiers.
%Adversarial attacks on time-series data are initially focused on time-series classification tasks, leveraging concepts adapted from adversarial attacks in other domains.
%explored adversarial sample crafting for time series classification using elastic similarity measures,  %These works collectively underscore the ongoing efforts to understand and mitigate the risks posed by adversarial attacks on time series classification models.
% More recently, research into adversarial attacks on time series forecasting models has revealed distinct challenges and novel attack strategies. One primary challenge is targeted attacks. While targeted adversarial attacks on time series classification aim to misclassify specific instances, achieving similar precision in time series forecasting is more complex due to the sequential nature of the data. Perturbations must be designed to influence specific aspects of the forecast (e.g., directional shifts or amplitude changes) without disrupting the overall temporal dependencies, making precise control more challenging~\cite{govindarajulu2023targeted}. Another challenge is attacks on multivariate forecasting. Adversarial attacks could exploit the inter-dependences between variables. ~\cite{liu2022robust} introduced sparse and indirect cross-time-series attacks in multivariate settings, which are more effective and realistic than direct attacks in univariate cases.
% \zhen{Biplav, could we make a quick comment here as well that we focus more on data error side in practice, other than attacks? and cite the paper that you mentioned on data errors? Otherwise this section of adversarial attacks feel a bit standalone to other sections}
%These challenges underscore the need for ongoing research to develop effective adversarial attack strategies and robust defense mechanisms tailored to the unique characteristics of time series forecasting models.
% -----


\noindent \textbf{Rating AI Systems} Several works have assessed and rated AI systems for trustworthiness from a third-party perspective without access to training data. \cite{srivastava2020rating} proposed a method to rate AI systems for bias, specifically targeting gender bias in machine translators \cite{srivastava2018towards}, and used visualizations to communicate these ratings \cite{bernagozzi2021vega}. They conducted user studies on trust perception through visualizations \cite{vega-userstudy-translatorbias}, but these lacked causal interpretation. \cite{kausik2024rating} introduced a causal analysis approach to rate bias in sentiment analysis systems, extending it to assess their impact when used with translators \cite{kausik2023the}. We extend their method to rate MM-TSFM for robustness against perturbations. Causal analysis offers advantages over statistical analysis by determining accountability, aligning with humanistic values, and quantifying the direct influence of various attributes on forecasting accuracy.



\section{Limitation}
This work identifies signal collapse as a critical bottleneck in one-shot neural network pruning. Performance loss in pruned networks is due to \textbf{signal collapse} in addition to the removal of critical parameters. We propose \textbf{REFLOW} (\textbf{Re}storing \textbf{F}low of \textbf{Low}-variance signals), a simple yet effective method that mitigates signal collapse without computationally expensive weight updates. By focusing on signal preservation, REFLOW highlights the importance of mitigating signal collapse in sparse networks and enables magnitude pruning to match or surpass state-of-the-art one-shot pruning methods such as CHITA, CBS, and WF.

REFLOW consistently achieves state-of-the-art accuracy across diverse architectures, restoring ResNeXt-101 from under 4.1\% to 78.9\% top-1 accuracy at 80\% sparsity on ImageNet. Its lightweight design makes it a practical solution for both research and deployment, delivering high-quality sparse models without the overhead of traditional approaches. These findings challenge the traditional emphasis on weight selection strategies and underscore the critical role of signal propagation for achieving high-quality sparse networks in the context of one-shot pruning.




\section{Conclusion}
\label{sec:concl}


In this paper we have described our efforts in mechanizing the strand spaces framework~\cite{FHG98} in Coq.
To assess the flexibility of the approach and the usability of the library and of the proofs we have analyzed a variety of examples: a basic authentication protocol and some of its variants, the classical Needham-Schroeder-Lowe authentication protocol, and a recent key management API equipped with a key management policy.

Wherever possible, our mechanization remains faithful to the original pen-and-paper development of strand spaces.
At the same time, we put a lot of engineering effort to make the code and the proofs reusable.
For that, we have made the framework modular and parametric in the terms and the penetrator.
Additionally, we have developed a number of strands-specific tactics whose goal is to make the life of the protocol's analyst easier by removing some of the burden of these kinds of proofs.
Indeed, the tactics automate a number of intermediate steps enabling, in some cases, easy proof reuse.
For instance, the proof of the NSL responder's nonce secrecy
 required just one hour of work using the initiator's nonce secrecy.
The mechanization
gives the freedom to experiment with protocols and their properties, while retaining the unique ability of strand spaces-based analyses to give interesting insights on the inner workings of protocols.
With our experiments, we uncovered
and fixed issues, discarded
redundant or unused requirements, and significantly improved previous results on the analysis of key management policies, making it possible to formally prove the security of the \emph{secure templates} policy from \cite{BCFS-ccs10} (\cref{sec:casestudies}).

\cref{tab:simpleauth,tab:nsl} in \cref{sec:summary}  summarize the premises for each security property across the analyzed protocol variants. These premises are essential for our security proofs and offer important insights into the assumptions required to make a security protocol correct. The strand spaces model highlights this aspect, and the use of Coq and the \easystrands{} library further clarifies the minimal and necessary nature of these assumptions, reinforcing the model's ability to accurately capture security requirements.
With the insights from these experiments we also developed a new proof technique which we call \emph{protected predicate} technique that, in certain situations, simplifies the proofs making some previously challenging cases trivial.


Another advantage of having this mechanized platform is that it opens up new and interesting avenues of research.
\ifdefined\COLORDIFF
    \color{cbred}
\else
\fi
For instance, an intriguing enhancement to our framework would be the inclusion of algebraic intruders. We believe they can be implemented using at least two approaches, which we briefly outline below.

Given an equational theory $E$ over a signature $\mathit{FS}$, the first approach requires implementing $E$ as a (terminating and confluent) rewriting system \lstinline{rew_E}, and allow penetrators to use \lstinline{rew_E} to manipulate terms containing symbols of $\mathit{FS}$.
More concretely, we first need to create an instance of \easystrands{} terms with support for function symbols in $\mathit{FS}$, then we can extend the penetrator as:
\begin{lstlisting}
Inductive penetrator_strand : Σ -> Prop := ...
| PT_Eqn : forall (g h : 𝔸) i, replace g h rew_E  -> penetrator_strand (i, [⊖ g; ⊕ h]).
\end{lstlisting}
where \lstinline{replace g h rew_E} holds iff \lstinline{g} can be rewritten as \lstinline{h} under \lstinline{rew_E}.
This approach is inspired by that of Tamarin \cite{MSCB13}.

The second approach aligns  with the method used in DY*~\cite{DY}, where cryptographic primitives are modeled as functions that symbolically represent the actual primitives, e.g., \lstinline{dec (c, k) = (if c = enc (m, k) then m else Error)}.
With these definitions, the equational theory $E$ could be defined using Coq Setoids and used for terms in place of Leibniz equality.
This has the advantage to allow both honest parties and the intruder to transparently use the equational theory.
However, as observed by~\citet{DY}, this approach requires proving (at least) that $E$ is an equivalence relation respected by all functions, predicates, and protocol specifications which can be lengthy and tedious.
\ifdefined\COLORDIFF
    \color{black}
\else
\fi

Despite their age, strand spaces have been a catalyst for extensive research, leading to notable extensions that include authentication tests~\cite{guttman2000authentication}, process algebraic-style choice operators~\cite{YEMMS16},
 compositionality \cite{StrandComposition,StrandIndependence,StrandMixed}, and stateful protocols \cite{J12}.
Many of these advancements are crucial for enhancing the expressiveness and usability of the model.
Our plan is to enhance \easystrands{} by integrating these extensions, thereby enabling scalability to more realistic protocols.
Ultimately, this will help narrow the gap with state-of-the-art tools such as DY* \cite{DY}.
In terms of foundational research, an intriguing avenue involves closely examining the relationship between Paulson's inductive method \cite{Paulson94} and strand spaces. We plan to mechanize Paulson's method in Coq and conduct a comparative analysis to assess the relative merits of these two inductive methods.

Finally, we defined a maximal penetrator as the set of strands that do not violate sensitive cryptographic operations required for protocol security. This method is inspired by the approach in \cite{banaSymbolic} to achieve computational soundness and, to our knowledge, has not been explored in a purely symbolic context before. It allows for proving injective agreement without explicitly defining the Dolev-Yao attacker, which we showed to be \diff{strictly} subsumed by the maximal penetrator. Notably, this approach facilitates the composition of protocols proven secure under their respective maximal penetrators, provided they adhere to each other's constraints. We are currently extending this technique to protocols like NSL, where security relies on decryption capabilities.



\section*{Acknowledgments}
We would like to thank Nikolaj Bjørner for his encouragement and help in inspecting the several generated invariants for the Z3 source code and provide feedback about their utility. 


\bibliographystyle{ACM-Reference-Format}
\bibliography{custom}

% \appendix
\appendix

\section{Prompt}
\label{appendix:prompt}
\begin{figure*}[htp]
\begin{lstlisting}[language=markdown]
You are an expert in creating program invariants from code and natural language.
Invariants are assertions on the variables in scope that hold true at different program points
We are interested in finding invariants that hold at both start and end of a function within a data structure. Such an invariant is commonly known as an object invariant.  

The invariants can usually be expressed as a check on the state at the particular program point. The check should be expressed as a check in the same underlying programming language which evaluates to true or false. To express these, you can use:
- An assertion in the programming language
- A pure method (which does not have any side effect on the variables in scope) that checks one or more assertion
- For a collection, you can use a loop to iterate over elements of the collection and assert something on each element or a pair of elements.  

Task Description: 
Task 1: Given a module, in the form of a class definition, your task is to infer object invariants about the class. For doing so, you may examine how the methods of the class read and modify the various fields of the class. 
For coming up with invariants, you may use the provided code and any comments in the code. You may also use world knowledge to guide the search for invariants. 
 
Task 2: Generate unit tests for the class based on the class definition and public API methods. The test cases should simulate a series of public method calls to verify the behavior of the class, but do not use any testing framework like gtest. Do not add `assert` or any form of assertions.
\end{lstlisting}
    \caption{\tech Generation system prompt: instruction and task description.}
    \label{fig:prompt_generation_system_task_app}
\end{figure*}

\begin{figure}
    \centering
    \begin{lstlisting}[language=markdown,firstnumber=15]
Input Format:
You will be given the name of a class or typedef, and a section of code containing the definition of the class. You will also be given the definitions of functions that read and modify the fields of the class. 

Output Format:
The output should be in the following format:

The first paragraph should begin with "REASONING:". From the next line onwards, it should contain the detailed reasoning and analysis used for the inference of the object invariants. The entire text should be enclosed in $$$. For example,
``$$$
REASONING: 
explanation
$$$

The next paragraph should begin with "INVARIANTS:". From the next line onwards, it should contain a list of the various invariants inferred. The invariants should be in the form of code in the same underlying programming language, enclosed by ```. Each invariant should start from a new line, and be separated by "---". Use lambda if necessary. If lambda is recursive, explicitly specify the type of the lambda function and use `std::function` for recursion. Do not use helper functions. For example, 
INVARIANTS: 
```/* Invariant 1 */```
---
```/* Multi line Invariant 2  */
    assert(...); ```
---
```/* Invariant 3 */```

The next paragraph should begin with "TESTS:". From the next line onwards, it should contain a list of a API call sequence in the form of code enclosed by ```. Each test should start from a new line, and be separated by "---". For example, 
TESTS: 
```/* Test 1 */
   this->method1();
   this->method2();```
---
```/* Test 2 */
    this.method3(...); ```
---
```/* Test 3 */```

Important:
1. Follow the output format strictly, particularly enclosing each invariant in triple-ticks (```), and enclosing the reasoning in $$$.
2. Only find object invariants for the target class provided to you, do not infer invariants for any other class. 
3. Make sure the invariant is a statement in the same underlying programming language as the source program.
4. If you can decompose a single invariant into smaller ones, try to output multiple invariants.
\end{lstlisting}
\vspace{-0.1in}
    \caption{\tech Generation system prompt: input-output format.}
    \label{fig:prompt_generation_system_inputoutput}
\end{figure}

Figure~\ref{fig:prompt_generation_system_task_app} and Figure~\ref{fig:prompt_generation_system_inputoutput} show the system prompt used by \tech for invariant-test co-generation. The former presents the instruction and task description, while the latter illustrates the input-output format.

\begin{figure*}[htp]
\begin{lstlisting}[language=markdown]
Name of Data Structure to Annotate: {struct}
Code:
```
{code}
```
\end{lstlisting}
\vspace{-0.1in}
    \caption{\tech Generation user prompt template.}
    \label{appendix:prompt_generation_user}
\end{figure*}

\begin{figure*}[htp]
\begin{lstlisting}[language=markdown]
You are an expert in repairing program invariants from code and natural language.
Invariants are assertions on the variables in scope that hold true at different program points. 

We are interested in finding invariants that hold at both start and end of a function within a data structure. Such an invariant is commonly known as an object invariant.  

The invariants can usually be expressed as a check on the state at the particular program point. The check should be expressed as a check in the same underlying programming language which evaluates to true or false. To express these, you can use:
- An assertion in the programming language
- A pure method (which does not have any side effect on the variables in scope) that checks one or more assertion
- For a collection, you can use a loop to iterate over elements of the collection and assert something on each element or a pair of elements.  

Task Description: 
Given a module, in the form of a class definition, your task is to infer object invariants about the class. For doing so, you may examine how the methods of the class read and modify the various fields of the class. 

For coming up with invariants, you may use the provided code and any comments in the code. You may also use world knowledge to guide the search for invariants. 

Input Format:
You will be given the name of a class or typedef, and a section of code containing the definition of the class. You will also be given the definitions of functions which read and modify the fields of the class. 

Output Format:
The output should be in the following format:

The first paragraph should begin with "REASONING:". From the next line onwards, it should contain the detailed reasoning and analysis used for the inference of the object invariants. The entire text should be enclosed in $$$. For example,
``$$$
REASONING: 
explanation
$$$``

The next paragraph should begin with "INVARIANTS:". From the next line onwards, it should contain a list of the various invariants inferred. The invariants should be in the form of code in the same underlying programming language, enclosed by ```. Each invariant should start from a new line, and be separated by "---". For example, 
INVARIANTS: 
```/* Invariant 1 */```
---
```/* Multi line Invariant 2  */
    assert(...);```
---
```/* Invariant 3 */```

Important:
1. Follow the output format strictly, particularly enclosing each invariant in triple-ticks (```), and enclosing the reasoning in $$$.
2. Only find object invariants for the target class provided to you, do not infer invariants for any other class. 
3. Make sure the invariant is a statement in the same underlying programming language as the source program.
4. If you can decompose a single invariant into smaller ones, try to output multiple invariants.
\end{lstlisting}
    \caption{\tech Refinement system prompt: instruction and task description.}
    \label{appendix:prompt_refinement_system}
\end{figure*}



\begin{figure*}[htp]
\begin{lstlisting}[language=markdown]
Please fix the failed invariants given the feedback, tests and the original source code.

Failed Invariant:
```
{invariant}
```

Error message for the failed invariant:
```
{feedback}
```

Name of Data Structure to Annotate: {struct}

Original Code:
```
{code}
```

Gold Tests that Fail the Invariant:
{tests}

\end{lstlisting}
\vspace{-0.1in}
    \caption{\tech Refinement user prompt template.}
    \label{appendix:prompt_refinement_user}
\end{figure*}
%\section{Daikon Invariants Frequency Tables}
\begin{table}[ht]
\centering
\scriptsize
\caption{Invariants for avl\_tree, 11 public methods.}
\label{avl_tree_daikon}
\begin{tabular}{|l|c|}
\hline
Invariant & Count \\
\hline
this->root has only one value & 4 \\
this->root.\_M\_t has only one value & 4 \\
this->root.\_M\_t.\_\_uniq\_ptr\_impl<AvlTree::Node, std::default\_delete<AvlTree::Node> >.\_M\_t has only one value & 4 \\
this[0] has only one value & 3 \\
this->n one of \{ 3, 4 \} & 3 \\
this[0] != null & 2 \\
this->root != null & 2 \\
this->root.\_M\_t != null & 2 \\
this->root.\_M\_t.\_\_uniq\_ptr\_impl<AvlTree::Node, std::default\_delete<AvlTree::Node> >.\_M\_t != null & 2 \\
this->n one of \{ 1, 2, 3 \} & 1 \\
this->n one of \{ 0, 3 \} & 1 \\
this->n >= 1 & 1 \\
this->n == 3 & 1 \\
t.root has only one value & 1 \\
t.root.\_M\_t has only one value & 1 \\
t.root.\_M\_t.\_\_uniq\_ptr\_impl<AvlTree::Node, std::default\_delete<AvlTree::Node> >.\_M\_t has only one value & 1 \\
t.n == 3 & 1 \\
this->n == return & 1 \\
\hline
\end{tabular}
\end{table}

\begin{table}[ht]
\centering
\scriptsize
\caption{Invariants for red\_black\_tree, 11 public methods.}
\label{red_black_tree_daikon}
\begin{tabular}{|l|c|}
\hline
Invariant & Count \\
\hline
this[0] != null & 3 \\
this->root != null & 3 \\
this->root.\_M\_t != null & 3 \\
this->root.\_M\_t.\_\_uniq\_ptr\_impl<RedBlackTree::Node, std::default\_delete<RedBlackTree::Node> >.\_M\_t != null & 3 \\
this->n one of \{ 3, 4, 6 \} & 3 \\
this->root has only one value & 3 \\
this->root.\_M\_t has only one value & 3 \\
this->root.\_M\_t.\_\_uniq\_ptr\_impl<RedBlackTree::Node, std::default\_delete<RedBlackTree::Node> >.\_M\_t has only one value & 3 \\
this[0] has only one value & 2 \\
this->n >= 0 & 1 \\
(No intersection exists) & 1 \\
t.root has only one value & 1 \\
t.root.\_M\_t has only one value & 1 \\
t.root.\_M\_t.\_\_uniq\_ptr\_impl<RedBlackTree::Node, std::default\_delete<RedBlackTree::Node> >.\_M\_t has only one value & 1 \\
t.n == 3 & 1 \\
No intersection & 1 \\
\hline
\end{tabular}
\end{table}

\begin{table}[ht]
\centering
\scriptsize
\caption{Invariants for linked\_list, 8 public methods.}
\label{linked_list_daikon}
\begin{tabular}{|l|c|}
\hline
Invariant & Count \\
\hline
this[0] has only one value & 5 \\
this->head has only one value & 5 \\
this->head.\_M\_t has only one value & 5 \\
this->head.\_M\_t.\_\_uniq\_ptr\_impl<LinkedList::Node, std::default\_delete<LinkedList::Node> >.\_M\_t has only one value & 5 \\
this[0] != null & 4 \\
this->head != null & 4 \\
this->head.\_M\_t != null & 4 \\
this->head.\_M\_t.\_\_uniq\_ptr\_impl<LinkedList::Node, std::default\_delete<LinkedList::Node> >.\_M\_t != null & 4 \\
this->tail[] elements != null & 2 \\
this->tail[].next elements != null & 2 \\
this->tail[].next.\_M\_t elements != null & 2 \\
this->n >= 0 & 2 \\
this->n == 0 & 1 \\
this->tail[].data elements one of \{ 1 \} & 1 \\
this->tail[].data one of \{ [1] \} & 1 \\
this->n one of \{ 1 \} & 1 \\
this->n one of \{ 1, 2 \} & 1 \\
this->tail[].data elements >= 1 & 1 \\
this->tail[].data elements <= this->n & 1 \\
this->tail[].data elements one of \{ 1, 2, 4 \} & 1 \\
this->tail[].data one of \{ [1], [2], [4] \} & 1 \\
this->n one of \{ 2, 3 \} & 1 \\
this->tail[].data == [3] & 1 \\
this->tail[].data elements == 3 & 1 \\
\hline
\end{tabular}
\end{table}

\begin{table}[ht]
\centering
\scriptsize
\caption{Invariants for binary\_search\_tree, 11 public methods.}
\label{binary_search_tree_daikon}
\begin{tabular}{|l|c|}
\hline
Invariant & Count \\
\hline
this->root has only one value & 4 \\
this->root.\_M\_t has only one value & 4 \\
this->root.\_M\_t.\_\_uniq\_ptr\_impl<BinarySearchTree::Node, std::default\_delete<BinarySearchTree::Node> >.\_M\_t has only one value & 4 \\
this->n one of \{ 0, 2, 3 \} & 3 \\
this->n one of \{ 0, 3 \} & 3 \\
this[0] has only one value & 3 \\
this[0] != null & 2 \\
this->root != null & 2 \\
this->root.\_M\_t != null & 2 \\
this->root.\_M\_t.\_\_uniq\_ptr\_impl<BinarySearchTree::Node, std::default\_delete<BinarySearchTree::Node> >.\_M\_t != null & 2 \\
t.root has only one value & 1 \\
t.root.\_M\_t has only one value & 1 \\
t.root.\_M\_t.\_\_uniq\_ptr\_impl<BinarySearchTree::Node, std::default\_delete<BinarySearchTree::Node> >.\_M\_t has only one value & 1 \\
t.n == 3 & 1 \\
(this->n == return) == (return == orig(this->n)) & 1 \\
\hline
\end{tabular}
\end{table}

\begin{table}[ht]
\centering
\scriptsize
\caption{Invariants for heap, 7 public methods.}
\label{heap_daikon}
\begin{tabular}{|l|c|}
\hline
Invariant & Count \\
\hline
this->comp.\_M\_invoker has only one value & 7 \\
this->comp.\_Function\_base.\_M\_manager has only one value & 7 \\
this[0] has only one value & 6 \\
this->data has only one value & 6 \\
this->data.\_Vector\_base<int, std::allocator<int> >.\_M\_impl has only one value & 6 \\
this->comp has only one value & 6 \\
this->comp.\_Function\_base.\_M\_functor has only one value & 6 \\
this->comp.\_Function\_base.\_M\_functor.\_M\_unused has only one value & 6 \\
this[0] != null & 3 \\
this->data != null & 3 \\
this->data.\_Vector\_base<int, std::allocator<int> >.\_M\_impl != null & 3 \\
this->comp != null & 3 \\
this->comp.\_M\_invoker != null & 3 \\
this->comp.\_Function\_base.\_M\_functor != null & 3 \\
this->comp.\_Function\_base.\_M\_functor.\_M\_unused != null & 3 \\
this->comp.\_Function\_base.\_M\_manager != null & 3 \\
this->data.\_Vector\_base<int, std::allocator<int> >.\_M\_impl.\_Vector\_impl\_data.\_M\_start[] elements >= 1 & 2 \\
this->data.\_Vector\_base<int, std::allocator<int> >.\_M\_impl.\_Vector\_impl\_data.\_M\_start != null & 1 \\
this->data.\_Vector\_base<int, std::allocator<int> >.\_M\_impl.\_Vector\_impl\_data.\_M\_finish != null & 1 \\
this->data.\_Vector\_base<int, std::allocator<int> >.\_M\_impl.\_Vector\_impl\_data.\_M\_end\_of\_storage != null & 1 \\
\hline
\end{tabular}
\end{table}

\begin{table}[ht]
\centering
\scriptsize
\caption{Invariants for hash\_table, 7 public methods.}
\label{hash_table_daikon}
\begin{tabular}{|l|c|}
\hline
Invariant & Count \\
\hline
::\_\_digits == "000102...6979899" & 3 \\
::\_\_tag == "" & 3 \\
this[0] has only one value & 3 \\
this->hash\_function has only one value & 3 \\
this->hash\_function.\_Function\_base.\_M\_functor has only one value & 3 \\
this->hash\_function.\_Function\_base.\_M\_functor.\_M\_unused has only one value & 3 \\
this->hash\_function.\_Function\_base.\_M\_functor.\_M\_pod\_data == "" & 3 \\
this->load\_factor == 0.75 & 3 \\
this->table has only one value & 3 \\
this->table.\_Vector\_base<... >.\_M\_impl has only one value & 2 \\
this->table.\_Vector\_base<... >.\_M\_impl has only one value & 1 \\
this->\_num\_elements one of \{ 0, 1 \} & 1 \\
this->\_size == 10 & 1 \\
this->table.\_Vector\_base<... > > > > >.\_M\_impl.\_Vector\_impl\_data.\_M\_end\_of\_storage & 1 \\
key.\_M\_dataplus has only one value & 1 \\
key.\_M\_dataplus.\_M\_p one of \{ "key1", "key2" \} & 1 \\
key.\_M\_string\_length == 4 & 1 \\
this->\_num\_elements >= 0 & 1 \\
this->\_size >= 0 & 1 \\
this->\_num\_elements <= this->\_size & 1 \\
\hline
\end{tabular}
\end{table}

\begin{table}[ht]
\centering
\scriptsize
\caption{Invariants for vector, 11 public methods. }
\label{vector_daikon}
\begin{tabular}{|l|c|}
\hline
Invariant & Count \\
\hline
this[0] has only one value & 9 \\
this->capacity == 5 & 7 \\
this->data has only one value & 5 \\
this->data[] elements one of \{ 1, 2 \} & 5 \\
this->n == 2 & 4 \\
this->data[] == [1, 2] & 4 \\
this->n in this->data[] & 4 \\
this->data[] == [1] & 3 \\
this->n == 0 & 2 \\
this->n one of \{ 1 \} & 2 \\
this->n in return[] & 1 \\
return[] == [1, 2] & 1 \\
return[] elements one of \{ 1, 2 \} & 1 \\
this->capacity == 0 & 1 \\
this->data == null & 1 \\
this->n == 1 & 1 \\
this->n == v.n & 1 \\
this->capacity == v.capacity & 1 \\
v.data has only one value & 1 \\
v one of \{ 1, 2 \} & 1 \\
this->capacity one of \{ 5 \} & 1 \\
this->data[] sorted by < & 1 \\
this->n <= this->capacity & 1 \\
this->n < this->capacity & 1 \\
this->n one of \{ 2, 5 \} & 1 \\
this->data[] elements == 1 & 1 \\
this->data[] one of \{ [1], [1, 2] \} & 1 \\
return one of \{ 1, 2 \} & 1 \\
\hline
\end{tabular}
\end{table}

\begin{table}[ht]
\centering
\scriptsize
\caption{Invariants for queue, 7 public methods.}
\label{queue_daikon}
\begin{tabular}{|l|c|}
\hline
Invariant & Count \\
\hline
this->data has only one value & 5 \\
this->maxSize == 10 & 5 \\
this[0] has only one value & 4 \\
this->data[] == [10, 20, 30] & 2 \\
this->data[] elements one of \{ 10, 20, 30 \} & 2 \\
this->head one of \{ 0, 1 \} & 2 \\
this->tail == 3 & 2 \\
this->n one of \{ 2, 3 \} & 2 \\
this->maxSize in this->data[] & 2 \\
this[0] != null & 2 \\
this->data != null & 2 \\
this->data[] elements >= 0 & 2 \\
this->data[] sorted by < & 2 \\
this->head < this->maxSize & 2 \\
this->tail < this->maxSize & 2 \\
this->data[] elements one of \{ 1, 2 \} & 2 \\
this->data[] one of \{ [1], [1, 2] \} & 2 \\
this->tail one of \{ 0, 1, 2 \} & 2 \\
this->tail in this->data[] & 2 \\
this->head - this->tail + this->n == 0 & 1 \\
this->tail one of \{ 2, 3, 100 \} & 1 \\
this->maxSize one of \{ 10, 160 \} & 1 \\
this->n < this->maxSize & 1 \\
this->head one of \{ 0, 50 \} & 1 \\
this->tail >= 0 & 1 \\
this->n <= this->maxSize & 1 \\
this->head <= this->tail & 1 \\
this->head one of \{ 0, 2 \} & 1 \\
this->n one of \{ 0, 1 \} & 1 \\
this->head == other.head & 1 \\
this->maxSize == other.maxSize & 1 \\
this->data[] == [7, 14] & 1 \\
this->data[] elements one of \{ 7, 14 \} & 1 \\
this->head == 0 & 1 \\
other.data has only one value & 1 \\
other.tail == 2 & 1 \\
this->head one of \{ 0, 1, 2 \} & 1 \\
this->head - this->tail + return == 0 & 1 \\
return one of \{ 0, 1, 2 \} & 1 \\
\hline
\end{tabular}
\end{table}

\begin{table}[ht]
\centering
\scriptsize
\caption{Invariants for stack, 6 public methods.}
\label{stack_daikon}
\begin{tabular}{|l|c|}
\hline
Invariant & Count \\
\hline
this->maxSize one of \{ 10, 160, 1280 \} & 4 \\
this->data[] sorted by < & 4 \\
this->data[] elements < this->maxSize & 4 \\
this->n < this->maxSize & 3 \\
this[0] != null & 3 \\
this->data != null & 3 \\
this[0] has only one value & 2 \\
this->data[] elements >= 0 & 2 \\
this->n one of \{ 0, 1, 2 \} & 1 \\
this->maxSize == other.maxSize & 1 \\
this has only one value & 1 \\
this->data has only one value & 1 \\
this->n == 2 & 1 \\
this->maxSize == 10 & 1 \\
other.data has only one value & 1 \\
other.data[] == [1, 2] & 1 \\
other.data[] elements one of \{ 1, 2 \} & 1 \\
this->n <= this->maxSize & 1 \\
\hline
\end{tabular}
\end{table}



\end{document}
\endinput
%%
%% End of file `sample-acmcp.tex'.
