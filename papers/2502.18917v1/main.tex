\PassOptionsToPackage{dvipsnames}{xcolor}
%\documentclass[acmsmall,screen,review,anonymous,nonacm]{acmart}
\documentclass[acmsmall,screen,nonacm]{acmart}


\acmJournal{JACM}
\acmVolume{37}
\acmNumber{4}
\acmArticle{111}
\acmMonth{8}

\usepackage{longtable}
\usepackage{tikz}
\usepackage[ruled, vlined, linesnumbered, commentsnumbered, longend]{algorithm2e}

\usepackage{xcolor}
\usepackage{xspace}
\usepackage{booktabs, multirow}
\usepackage{soul}
\usepackage{listings}
\usepackage{subcaption}
\usepackage{wrapfig}
\usepackage{enumitem}
\usepackage{wrapfig}

\renewcommand{\ttdefault}{txtt}

\definecolor{codegreen}{rgb}{0,0.6,0}
\definecolor{codegray}{rgb}{0.5,0.5,0.5}
\definecolor{codepurple}{rgb}{0.58,0,0.82}
\definecolor{backcolour}{rgb}{0.97,0.97,0.95}
\definecolor{forestgreen}{rgb}{0.28,0.62,0.37}
\definecolor{codeblue}{rgb}{0,0.5,1}

\lstdefinelanguage{markdown}{
    morekeywords={*, \#, \_, `},
    sensitive=false,
    morecomment=[l]{//},   % to allow comments in markdown syntax
    morestring=[b]",        % double quotes for strings
    postbreak={},
breakindent=0pt,
breakautoindent=false,
}


\lstdefinestyle{mystyle}{
    backgroundcolor=\color{backcolour},   
    commentstyle=\color{codepurple},
    keywordstyle=\color{codepurple},
    numberstyle=\tiny\color{codegray},
    stringstyle=\color{blue},
    basicstyle=\ttfamily\scriptsize,
    breakatwhitespace=false,         
    breaklines=true,                 
    captionpos=b,                    
    keepspaces=true,                 
    numbers=left,                    
    numbersep=5pt,                  
    showspaces=false,                
    showstringspaces=false,
    showtabs=false,                  
    tabsize=4
}

\lstset{style=mystyle}
\newcommand{\HighlightBG}{\makebox[0pt][l]{\color{yellow!50}\rule[-0.45em]{\linewidth}{1.3em}}}

\usetikzlibrary{shapes, arrows}

\begin{document}
\def\method{\text MixMin~}
\def\methodnospace{\text MixMin}
\def\genmethod{$\mathbb{R}$\text Min~}
\def\genmethodnospace{ $\mathbb{R}$\text Min}


\title{\tech: Class Invariant Synthesis using Large Language Models}

\author{Chuyue Sun}
\authornote{Work done while interning at Microsoft.}
\affiliation{%
   \institution{Stanford University}
   \country{USA}
}
\email{chuyues@stanford.edu}
\author{Viraj Agashe}
\affiliation{%
   \institution{Microsoft Research}
   \country{India}
}
\email{t-vagashe@microsoft.com}
\author{Saikat Chakraborty}
\affiliation{%
   \institution{Microsoft Research}
   \country{USA}
}
\email{saikatc@microsoft.com}
\author{Jubi Taneja}
\affiliation{%
   \institution{Microsoft Research}
   \country{USA}
}
\email{jubitaneja@microsoft.com}
\author{Clark Barrett}
\affiliation{%
   \institution{Stanford University}
   \country{USA}
}
\email{barrettc@stanford.edu}
\author{David Dill}
\affiliation{%
   \institution{Stanford University}
   \country{USA}
}
\email{dill@cs.stanford.edu}
\author{Xiaokang Qiu}
\authornote{Work done while visiting Microsoft.}
\affiliation{%
  \institution{Purdue University}
   \country{USA}
}
\email{xkqiu@purdue.edu}
\author{Shuvendu K. Lahiri}
\affiliation{%
   \institution{Microsoft Research}
   \country{USA}
}
\email{shuvendu@microsoft.com}


\begin{abstract}
Formal program specifications in the form of preconditions, postconditions, and class invariants have several benefits for the construction and maintenance of programs. They not only aid in program understanding due to their unambiguous semantics but can also be enforced dynamically (or even statically when the language supports a formal verifier).  However, synthesizing high-quality specifications in an underlying programming language is limited by the expressivity of the specifications or the need to express them in a declarative manner. Prior work has demonstrated the potential of large language models (LLMs) for synthesizing high-quality method pre/postconditions for Python and Java, but does not consider class invariants.

In this work, we describe \tech, a method for co-generating executable class invariants and test inputs to produce high-quality class invariants for a mainstream language such as C++, leveraging LLMs' ability to synthesize pure functions. We show that \tech outperforms a pure LLM-based technique to generate specifications (from code) as well as prior data-driven invariant inference techniques such as Daikon. We contribute a benchmark of standard C++ data structures along with a harness that can help measure both the correctness and completeness of generated specifications using tests and mutants. We also demonstrate its applicability to real-world code by performing a case study on several classes within a widely used and high-integrity C++ codebase.
\end{abstract}


\maketitle

\section{Introduction}
\section{Introduction}


\begin{figure}[t]
\centering
\includegraphics[width=0.6\columnwidth]{figures/evaluation_desiderata_V5.pdf}
\vspace{-0.5cm}
\caption{\systemName is a platform for conducting realistic evaluations of code LLMs, collecting human preferences of coding models with real users, real tasks, and in realistic environments, aimed at addressing the limitations of existing evaluations.
}
\label{fig:motivation}
\end{figure}

\begin{figure*}[t]
\centering
\includegraphics[width=\textwidth]{figures/system_design_v2.png}
\caption{We introduce \systemName, a VSCode extension to collect human preferences of code directly in a developer's IDE. \systemName enables developers to use code completions from various models. The system comprises a) the interface in the user's IDE which presents paired completions to users (left), b) a sampling strategy that picks model pairs to reduce latency (right, top), and c) a prompting scheme that allows diverse LLMs to perform code completions with high fidelity.
Users can select between the top completion (green box) using \texttt{tab} or the bottom completion (blue box) using \texttt{shift+tab}.}
\label{fig:overview}
\end{figure*}

As model capabilities improve, large language models (LLMs) are increasingly integrated into user environments and workflows.
For example, software developers code with AI in integrated developer environments (IDEs)~\citep{peng2023impact}, doctors rely on notes generated through ambient listening~\citep{oberst2024science}, and lawyers consider case evidence identified by electronic discovery systems~\citep{yang2024beyond}.
Increasing deployment of models in productivity tools demands evaluation that more closely reflects real-world circumstances~\citep{hutchinson2022evaluation, saxon2024benchmarks, kapoor2024ai}.
While newer benchmarks and live platforms incorporate human feedback to capture real-world usage, they almost exclusively focus on evaluating LLMs in chat conversations~\citep{zheng2023judging,dubois2023alpacafarm,chiang2024chatbot, kirk2024the}.
Model evaluation must move beyond chat-based interactions and into specialized user environments.



 

In this work, we focus on evaluating LLM-based coding assistants. 
Despite the popularity of these tools---millions of developers use Github Copilot~\citep{Copilot}---existing
evaluations of the coding capabilities of new models exhibit multiple limitations (Figure~\ref{fig:motivation}, bottom).
Traditional ML benchmarks evaluate LLM capabilities by measuring how well a model can complete static, interview-style coding tasks~\citep{chen2021evaluating,austin2021program,jain2024livecodebench, white2024livebench} and lack \emph{real users}. 
User studies recruit real users to evaluate the effectiveness of LLMs as coding assistants, but are often limited to simple programming tasks as opposed to \emph{real tasks}~\citep{vaithilingam2022expectation,ross2023programmer, mozannar2024realhumaneval}.
Recent efforts to collect human feedback such as Chatbot Arena~\citep{chiang2024chatbot} are still removed from a \emph{realistic environment}, resulting in users and data that deviate from typical software development processes.
We introduce \systemName to address these limitations (Figure~\ref{fig:motivation}, top), and we describe our three main contributions below.


\textbf{We deploy \systemName in-the-wild to collect human preferences on code.} 
\systemName is a Visual Studio Code extension, collecting preferences directly in a developer's IDE within their actual workflow (Figure~\ref{fig:overview}).
\systemName provides developers with code completions, akin to the type of support provided by Github Copilot~\citep{Copilot}. 
Over the past 3 months, \systemName has served over~\completions suggestions from 10 state-of-the-art LLMs, 
gathering \sampleCount~votes from \userCount~users.
To collect user preferences,
\systemName presents a novel interface that shows users paired code completions from two different LLMs, which are determined based on a sampling strategy that aims to 
mitigate latency while preserving coverage across model comparisons.
Additionally, we devise a prompting scheme that allows a diverse set of models to perform code completions with high fidelity.
See Section~\ref{sec:system} and Section~\ref{sec:deployment} for details about system design and deployment respectively.



\textbf{We construct a leaderboard of user preferences and find notable differences from existing static benchmarks and human preference leaderboards.}
In general, we observe that smaller models seem to overperform in static benchmarks compared to our leaderboard, while performance among larger models is mixed (Section~\ref{sec:leaderboard_calculation}).
We attribute these differences to the fact that \systemName is exposed to users and tasks that differ drastically from code evaluations in the past. 
Our data spans 103 programming languages and 24 natural languages as well as a variety of real-world applications and code structures, while static benchmarks tend to focus on a specific programming and natural language and task (e.g. coding competition problems).
Additionally, while all of \systemName interactions contain code contexts and the majority involve infilling tasks, a much smaller fraction of Chatbot Arena's coding tasks contain code context, with infilling tasks appearing even more rarely. 
We analyze our data in depth in Section~\ref{subsec:comparison}.



\textbf{We derive new insights into user preferences of code by analyzing \systemName's diverse and distinct data distribution.}
We compare user preferences across different stratifications of input data (e.g., common versus rare languages) and observe which affect observed preferences most (Section~\ref{sec:analysis}).
For example, while user preferences stay relatively consistent across various programming languages, they differ drastically between different task categories (e.g. frontend/backend versus algorithm design).
We also observe variations in user preference due to different features related to code structure 
(e.g., context length and completion patterns).
We open-source \systemName and release a curated subset of code contexts.
Altogether, our results highlight the necessity of model evaluation in realistic and domain-specific settings.








\section{Running Example: AVL Tree}
\label{sec:avl_intro}

\section{Bellman Error Centering}

Centering operator $\mathcal{C}$ for a variable $x(s)$ is defined as follows:
\begin{equation}
\mathcal{C}x(s)\dot{=} x(s)-\mathbb{E}[x(s)]=x(s)-\sum_s{d_{s}x(s)},
\end{equation} 
where $d_s$ is the probability of $s$.
In vector form,
\begin{equation}
\begin{split}
\mathcal{C}\bm{x} &= \bm{x}-\mathbb{E}[x]\bm{1}\\
&=\bm{x}-\bm{x}^{\top}\bm{d}\bm{1},
\end{split}
\end{equation} 
where $\bm{1}$ is an all-ones vector.
For any vector $\bm{x}$ and $\bm{y}$ with a same distribution $\bm{d}$,
we have
\begin{equation}
\begin{split}
\mathcal{C}(\bm{x}+\bm{y})&=(\bm{x}+\bm{y})-(\bm{x}+\bm{y})^{\top}\bm{d}\bm{1}\\
&=\bm{x}-\bm{x}^{\top}\bm{d}\bm{1}+\bm{y}-\bm{y}^{\top}\bm{d}\bm{1}\\
&=\mathcal{C}\bm{x}+\mathcal{C}\bm{y}.
\end{split}
\end{equation}
\subsection{Revisit Reward Centering}


The update (\ref{src3}) is an unbiased estimate of the average reward
with  appropriate learning rate $\beta_t$ conditions.
\begin{equation}
\bar{r}_{t}\approx \lim_{n\rightarrow\infty}\frac{1}{n}\sum_{t=1}^n\mathbb{E}_{\pi}[r_t].
\end{equation}
That is 
\begin{equation}
r_t-\bar{r}_{t}\approx r_t-\lim_{n\rightarrow\infty}\frac{1}{n}\sum_{t=1}^n\mathbb{E}_{\pi}[r_t]= \mathcal{C}r_t.
\end{equation}
Then, the simple reward centering can be rewrited as:
\begin{equation}
V_{t+1}(s_t)=V_{t}(s_t)+\alpha_t [\mathcal{C}r_{t+1}+\gamma V_{t}(s_{t+1})-V_t(s_t)].
\end{equation}
Therefore, the simple reward centering is, in a strict sense, reward centering.

By definition of $\bar{\delta}_t=\delta_t-\bar{r}_{t}$,
let rewrite the update rule of the value-based reward centering as follows:
\begin{equation}
V_{t+1}(s_t)=V_{t}(s_t)+\alpha_t \rho_t (\delta_t-\bar{r}_{t}),
\end{equation}
where $\bar{r}_{t}$ is updated as:
\begin{equation}
\bar{r}_{t+1}=\bar{r}_{t}+\beta_t \rho_t(\delta_t-\bar{r}_{t}).
\label{vrc3}
\end{equation}
The update (\ref{vrc3}) is an unbiased estimate of the TD error
with  appropriate learning rate $\beta_t$ conditions.
\begin{equation}
\bar{r}_{t}\approx \mathbb{E}_{\pi}[\delta_t].
\end{equation}
That is 
\begin{equation}
\delta_t-\bar{r}_{t}\approx \mathcal{C}\delta_t.
\end{equation}
Then, the value-based reward centering can be rewrited as:
\begin{equation}
V_{t+1}(s_t)=V_{t}(s_t)+\alpha_t \rho_t \mathcal{C}\delta_t.
\label{tdcentering}
\end{equation}
Therefore, the value-based reward centering is no more,
 in a strict sense, reward centering.
It is, in a strict sense, \textbf{Bellman error centering}.

It is worth noting that this understanding is crucial, 
as designing new algorithms requires leveraging this concept.


\subsection{On the Fixpoint Solution}

The update rule (\ref{tdcentering}) is a stochastic approximation
of the following update:
\begin{equation}
\begin{split}
V_{t+1}&=V_{t}+\alpha_t [\bm{\mathcal{T}}^{\pi}\bm{V}-\bm{V}-\mathbb{E}[\delta]\bm{1}]\\
&=V_{t}+\alpha_t [\bm{\mathcal{T}}^{\pi}\bm{V}-\bm{V}-(\bm{\mathcal{T}}^{\pi}\bm{V}-\bm{V})^{\top}\bm{d}_{\pi}\bm{1}]\\
&=V_{t}+\alpha_t [\mathcal{C}(\bm{\mathcal{T}}^{\pi}\bm{V}-\bm{V})].
\end{split}
\label{tdcenteringVector}
\end{equation}
If update rule (\ref{tdcenteringVector}) converges, it is expected that
$\mathcal{C}(\mathcal{T}^{\pi}V-V)=\bm{0}$.
That is 
\begin{equation}
    \begin{split}
    \mathcal{C}\bm{V} &= \mathcal{C}\bm{\mathcal{T}}^{\pi}\bm{V} \\
    &= \mathcal{C}(\bm{R}^{\pi} + \gamma \mathbb{P}^{\pi} \bm{V}) \\
    &= \mathcal{C}\bm{R}^{\pi} + \gamma \mathcal{C}\mathbb{P}^{\pi} \bm{V} \\
    &= \mathcal{C}\bm{R}^{\pi} + \gamma (\mathbb{P}^{\pi} \bm{V} - (\mathbb{P}^{\pi} \bm{V})^{\top} \bm{d_{\pi}} \bm{1}) \\
    &= \mathcal{C}\bm{R}^{\pi} + \gamma (\mathbb{P}^{\pi} \bm{V} - \bm{V}^{\top} (\mathbb{P}^{\pi})^{\top} \bm{d_{\pi}} \bm{1}) \\  % 修正双重上标
    &= \mathcal{C}\bm{R}^{\pi} + \gamma (\mathbb{P}^{\pi} \bm{V} - \bm{V}^{\top} \bm{d_{\pi}} \bm{1}) \\
    &= \mathcal{C}\bm{R}^{\pi} + \gamma (\mathbb{P}^{\pi} \bm{V} - \bm{V}^{\top} \bm{d_{\pi}} \mathbb{P}^{\pi} \bm{1}) \\
    &= \mathcal{C}\bm{R}^{\pi} + \gamma (\mathbb{P}^{\pi} \bm{V} - \mathbb{P}^{\pi} \bm{V}^{\top} \bm{d_{\pi}} \bm{1}) \\
    &= \mathcal{C}\bm{R}^{\pi} + \gamma \mathbb{P}^{\pi} (\bm{V} - \bm{V}^{\top} \bm{d_{\pi}} \bm{1}) \\
    &= \mathcal{C}\bm{R}^{\pi} + \gamma \mathbb{P}^{\pi} \mathcal{C}\bm{V} \\
    &\dot{=} \bm{\mathcal{T}}_c^{\pi} \mathcal{C}\bm{V},
    \end{split}
    \label{centeredfixpoint}
    \end{equation}
where we defined $\bm{\mathcal{T}}_c^{\pi}$ as a centered Bellman operator.
We call equation (\ref{centeredfixpoint}) as centered Bellman equation.
And it is \textbf{centered fixpoint}.

For linear value function approximation, let define
\begin{equation}
\mathcal{C}\bm{V}_{\bm{\theta}}=\bm{\Pi}\bm{\mathcal{T}}_c^{\pi}\mathcal{C}\bm{V}_{\bm{\theta}}.
\label{centeredTDfixpoint}
\end{equation}
We call equation (\ref{centeredTDfixpoint}) as \textbf{centered TD fixpoint}.

\subsection{On-policy and Off-policy Centered TD Algorithms
with Linear Value Function Approximation}
Given the above centered TD fixpoint,
 mean squared centered Bellman error (MSCBE), is proposed as follows:
\begin{align*}
    \label{argminMSBEC}
 &\arg \min_{{\bm{\theta}}}\text{MSCBE}({\bm{\theta}}) \\
 &= \arg \min_{{\bm{\theta}}} \|\bm{\mathcal{T}}_c^{\pi}\mathcal{C}\bm{V}_{\bm{{\bm{\theta}}}}-\mathcal{C}\bm{V}_{\bm{{\bm{\theta}}}}\|_{\bm{D}}^2\notag\\
 &=\arg \min_{{\bm{\theta}}} \|\bm{\mathcal{T}}^{\pi}\bm{V}_{\bm{{\bm{\theta}}}} - \bm{V}_{\bm{{\bm{\theta}}}}-(\bm{\mathcal{T}}^{\pi}\bm{V}_{\bm{{\bm{\theta}}}} - \bm{V}_{\bm{{\bm{\theta}}}})^{\top}\bm{d}\bm{1}\|_{\bm{D}}^2\notag\\
 &=\arg \min_{{\bm{\theta}},\omega} \| \bm{\mathcal{T}}^{\pi}\bm{V}_{\bm{{\bm{\theta}}}} - \bm{V}_{\bm{{\bm{\theta}}}}-\omega\bm{1} \|_{\bm{D}}^2\notag,
\end{align*}
where $\omega$ is is used to estimate the expected value of the Bellman error.
% where $\omega$ is used to estimate $\mathbb{E}[\delta]$, $\omega \doteq \mathbb{E}[\mathbb{E}[\delta_t|S_t]]=\mathbb{E}[\delta]$ and $\delta_t$ is the TD error as follows:
% \begin{equation}
% \delta_t = r_{t+1}+\gamma
% {\bm{\theta}}_t^{\top}\bm{{\bm{\phi}}}_{t+1}-{\bm{\theta}}_t^{\top}\bm{{\bm{\phi}}}_t.
% \label{delta}
% \end{equation}
% $\mathbb{E}[\delta_t|S_t]$ is the Bellman error, and $\mathbb{E}[\mathbb{E}[\delta_t|S_t]]$ represents the expected value of the Bellman error.
% If $X$ is a random variable and $\mathbb{E}[X]$ is its expected value, then $X-\mathbb{E}[X]$ represents the centered form of $X$. 
% Therefore, we refer to $\mathbb{E}[\delta_t|S_t]-\mathbb{E}[\mathbb{E}[\delta_t|S_t]]$ as Bellman error centering and 
% $\mathbb{E}[(\mathbb{E}[\delta_t|S_t]-\mathbb{E}[\mathbb{E}[\delta_t|S_t]])^2]$ represents the the mean squared centered Bellman error, namely MSCBE.
% The meaning of (\ref{argminMSBEC}) is to minimize the mean squared centered Bellman error.
%The derivation of CTD is as follows.

First, the parameter  $\omega$ is derived directly based on
stochastic gradient descent:
\begin{equation}
\omega_{t+1}= \omega_{t}+\beta_t(\delta_t-\omega_t).
\label{omega}
\end{equation}

Then, based on stochastic semi-gradient descent, the update of 
the parameter ${\bm{\theta}}$ is as follows:
\begin{equation}
{\bm{\theta}}_{t+1}=
{\bm{\theta}}_{t}+\alpha_t(\delta_t-\omega_t)\bm{{\bm{\phi}}}_t.
\label{theta}
\end{equation}

We call (\ref{omega}) and (\ref{theta}) the on-policy centered
TD (CTD) algorithm. The convergence analysis with be given in
the following section.

In off-policy learning, we can simply multiply by the importance sampling
 $\rho$.
\begin{equation}
    \omega_{t+1}=\omega_{t}+\beta_t\rho_t(\delta_t-\omega_t),
    \label{omegawithrho}
\end{equation}
\begin{equation}
    {\bm{\theta}}_{t+1}=
    {\bm{\theta}}_{t}+\alpha_t\rho_t(\delta_t-\omega_t)\bm{{\bm{\phi}}}_t.
    \label{thetawithrho}
\end{equation}

We call (\ref{omegawithrho}) and (\ref{thetawithrho}) the off-policy centered
TD (CTD) algorithm.

% By substituting $\delta_t$ into Equations (\ref{omegawithrho}) and (\ref{thetawithrho}), 
% we can see that Equations (\ref{thetawithrho}) and (\ref{omegawithrho}) are formally identical 
% to the linear expressions of Equations (\ref{rewardcentering1}) and (\ref{rewardcentering2}), respectively. However, the meanings 
% of the corresponding parameters are entirely different.
% ${\bm{\theta}}_t$ is for approximating the discounted value function.
% $\bar{r_t}$ is an estimate of the average reward, while $\omega_t$ 
% is an estimate of the expected value of the Bellman error.
% $\bar{\delta_t}$ is the TD error for value-based reward centering, 
% whereas $\delta_t$ is the traditional TD error.

% This study posits that the CTD is equivalent to value-based reward 
% centering. However, CTD can be unified under a single framework 
% through an objective function, MSCBE, which also lays the 
% foundation for proving the algorithm's convergence. 
% Section 4 demonstrates that the CTD algorithm guarantees 
% convergence in the on-policy setting.

\subsection{Off-policy Centered TDC Algorithm with Linear Value Function Approximation}
The convergence of the  off-policy centered TD algorithm
may not be guaranteed.

To deal with this problem, we propose another new objective function, 
called mean squared projected centered Bellman error (MSPCBE), 
and derive Centered TDC algorithm (CTDC).

% We first establish some relationships between
%  the vector-matrix quantities and the relevant statistical expectation terms:
% \begin{align*}
%     &\mathbb{E}[(\delta({\bm{\theta}})-\mathbb{E}[\delta({\bm{\theta}})]){\bm{\phi}}] \\
%     &= \sum_s \mu(s) {\bm{\phi}}(s) \big( R(s) + \gamma \sum_{s'} P_{ss'} V_{\bm{\theta}}(s') - V_{\bm{\theta}}(s)  \\
%     &\quad \quad-\sum_s \mu(s)(R(s) + \gamma \sum_{s'} P_{ss'} V_{\bm{\theta}}(s') - V_{\bm{\theta}}(s))\big)\\
%     &= \bm{\Phi}^\top \mathbf{D} (\bm{TV}_{\bm{{\bm{\theta}}}} - \bm{V}_{\bm{{\bm{\theta}}}}-\omega\bm{1}),
% \end{align*}
% where $\omega$ is the expected value of the Bellman error and $\bm{1}$ is all-ones vector.

The specific expression of the objective function 
MSPCBE is as follows:
\begin{align}
    \label{MSPBECwithomega}
    &\arg \min_{{\bm{\theta}}}\text{MSPCBE}({\bm{\theta}})\notag\\ 
    % &= \arg \min_{{\bm{\theta}}}\big(\mathbb{E}[(\delta({\bm{\theta}}) - \mathbb{E}[\delta({\bm{\theta}})]) \bm{{\bm{\phi}}}]^\top \notag\\
    % &\quad \quad \quad\mathbb{E}[\bm{{\bm{\phi}}} \bm{{\bm{\phi}}}^\top]^{-1} \mathbb{E}[(\delta({\bm{\theta}}) - \mathbb{E}[\delta({\bm{\theta}})]) \bm{{\bm{\phi}}}]\big) \notag\\
    % &=\arg \min_{{\bm{\theta}},\omega}\mathbb{E}[(\delta({\bm{\theta}})-\omega) \bm{\bm{{\bm{\phi}}}}]^{\top} \mathbb{E}[\bm{\bm{{\bm{\phi}}}} \bm{\bm{{\bm{\phi}}}}^{\top}]^{-1}\mathbb{E}[(\delta({\bm{\theta}}) -\omega)\bm{\bm{{\bm{\phi}}}}]\\
    % &= \big(\bm{\Phi}^\top \mathbf{D} (\bm{TV}_{\bm{{\bm{\theta}}}} - \bm{V}_{\bm{{\bm{\theta}}}}-\omega\bm{1})\big)^\top (\bm{\Phi}^\top \mathbf{D} \bm{\Phi})^{-1} \notag\\
    % & \quad \quad \quad \bm{\Phi}^\top \mathbf{D} (\bm{TV}_{\bm{{\bm{\theta}}}} - \bm{V}_{\bm{{\bm{\theta}}}}-\omega\bm{1}) \notag\\
    % &= (\bm{TV}_{\bm{{\bm{\theta}}}} - \bm{V}_{\bm{{\bm{\theta}}}}-\omega\bm{1})^\top \mathbf{D} \bm{\Phi} (\bm{\Phi}^\top \mathbf{D} \bm{\Phi})^{-1} \notag\\
    % &\quad \quad \quad \bm{\Phi}^\top \mathbf{D} (\bm{TV}_{\bm{{\bm{\theta}}}} - \bm{V}_{\bm{{\bm{\theta}}}}-\omega\bm{1})\notag\\
    % &= (\bm{TV}_{\bm{{\bm{\theta}}}} - \bm{V}_{\bm{{\bm{\theta}}}}-\omega\bm{1})^\top {\bm{\Pi}}^\top \mathbf{D} {\bm{\Pi}} (\bm{TV}_{\bm{{\bm{\theta}}}} - \bm{V}_{\bm{{\bm{\theta}}}}-\omega\bm{1}) \notag\\
    &= \arg \min_{{\bm{\theta}}} \|\bm{\Pi}\bm{\mathcal{T}}_c^{\pi}\mathcal{C}\bm{V}_{\bm{{\bm{\theta}}}}-\mathcal{C}\bm{V}_{\bm{{\bm{\theta}}}}\|_{\bm{D}}^2\notag\\
    &= \arg \min_{{\bm{\theta}}} \|\bm{\Pi}(\bm{\mathcal{T}}_c^{\pi}\mathcal{C}\bm{V}_{\bm{{\bm{\theta}}}}-\mathcal{C}\bm{V}_{\bm{{\bm{\theta}}}})\|_{\bm{D}}^2\notag\\
    &= \arg \min_{{\bm{\theta}},\omega}\| {\bm{\Pi}} (\bm{\mathcal{T}}^{\pi}\bm{V}_{\bm{{\bm{\theta}}}} - \bm{V}_{\bm{{\bm{\theta}}}}-\omega\bm{1}) \|_{\bm{D}}^2\notag.
\end{align}
In the process of computing the gradient of the MSPCBE with respect to ${\bm{\theta}}$, 
$\omega$ is treated as a constant.
So, the derivation process of CTDC is the same 
as for the TDC algorithm \cite{sutton2009fast}, the only difference is that the original $\delta$ is replaced by $\delta-\omega$.
Therefore, the updated formulas of the centered TDC  algorithm are as follows:
\begin{equation}
 \bm{{\bm{\theta}}}_{k+1}=\bm{{\bm{\theta}}}_{k}+\alpha_{k}[(\delta_{k}- \omega_k) \bm{\bm{{\bm{\phi}}}}_k\\
 - \gamma\bm{\bm{{\bm{\phi}}}}_{k+1}(\bm{\bm{{\bm{\phi}}}}^{\top}_k \bm{u}_{k})],
\label{thetavmtdc}
\end{equation}
\begin{equation}
 \bm{u}_{k+1}= \bm{u}_{k}+\zeta_{k}[\delta_{k}-\omega_k - \bm{\bm{{\bm{\phi}}}}^{\top}_k \bm{u}_{k}]\bm{\bm{{\bm{\phi}}}}_k,
\label{uvmtdc}
\end{equation}
and
\begin{equation}
 \omega_{k+1}= \omega_{k}+\beta_k (\delta_k- \omega_k).
 \label{omegavmtdc}
\end{equation}
This algorithm is derived to work 
with a given set of sub-samples—in the form of 
triples $(S_k, R_k, S'_k)$ that match transitions 
from both the behavior and target policies. 

% \subsection{Variance Minimization ETD Learning: VMETD}
% Based on the off-policy TD algorithm, a scalar, $F$,  
% is introduced to obtain the ETD algorithm, 
% which ensures convergence under off-policy 
% conditions. This paper further introduces a scalar, 
% $\omega$, based on the ETD algorithm to obtain VMETD.
% VMETD by the following update:
% \begin{equation}
% \label{fvmetd}
%  F_t \leftarrow \gamma \rho_{t-1}F_{t-1}+1,
% \end{equation}
% \begin{equation}
%  \label{thetavmetd}
%  {{\bm{\theta}}}_{t+1}\leftarrow {{\bm{\theta}}}_t+\alpha_t (F_t \rho_t\delta_t - \omega_{t}){\bm{{\bm{\phi}}}}_t,
% \end{equation}
% \begin{equation}
%  \label{omegavmetd}
%  \omega_{t+1} \leftarrow \omega_t+\beta_t(F_t  \rho_t \delta_t - \omega_t),
% \end{equation}
% where $\rho_t =\frac{\pi(A_t | S_t)}{\mu(A_t | S_t)}$ and $\omega$ is used to estimate $\mathbb{E}[F \rho\delta]$, i.e., $\omega \doteq \mathbb{E}[F \rho\delta]$.

% (\ref{thetavmetd}) can be rewritten as
% \begin{equation*}
%  \begin{array}{ccl}
%  {{\bm{\theta}}}_{t+1}&\leftarrow& {{\bm{\theta}}}_t+\alpha_t (F_t \rho_t\delta_t - \omega_t){\bm{{\bm{\phi}}}}_t -\alpha_t \omega_{t+1}{\bm{{\bm{\phi}}}}_t\\
%   &=&{{\bm{\theta}}}_{t}+\alpha_t(F_t\rho_t\delta_t-\mathbb{E}_{\mu}[F_t\rho_t\delta_t|{{\bm{\theta}}}_t]){\bm{{\bm{\phi}}}}_t\\
%  &=&{{\bm{\theta}}}_t+\alpha_t F_t \rho_t (r_{t+1}+\gamma {{\bm{\theta}}}_t^{\top}{\bm{{\bm{\phi}}}}_{t+1}-{{\bm{\theta}}}_t^{\top}{\bm{{\bm{\phi}}}}_t){\bm{{\bm{\phi}}}}_t\\
%  & & \hspace{2em} -\alpha_t \mathbb{E}_{\mu}[F_t \rho_t \delta_t]{\bm{{\bm{\phi}}}}_t\\
%  &=& {{\bm{\theta}}}_t+\alpha_t \{\underbrace{(F_t\rho_tr_{t+1}-\mathbb{E}_{\mu}[F_t\rho_t r_{t+1}]){\bm{{\bm{\phi}}}}_t}_{{b}_{\text{VMETD},t}}\\
%  &&\hspace{-7em}- \underbrace{(F_t\rho_t{\bm{{\bm{\phi}}}}_t({\bm{{\bm{\phi}}}}_t-\gamma{\bm{{\bm{\phi}}}}_{t+1})^{\top}-{\bm{{\bm{\phi}}}}_t\mathbb{E}_{\mu}[F_t\rho_t ({\bm{{\bm{\phi}}}}_t-\gamma{\bm{{\bm{\phi}}}}_{t+1})]^{\top})}_{\textbf{A}_{\text{VMETD},t}}{{\bm{\theta}}}_t\}.
%  \end{array}
% \end{equation*}
% Therefore, 
% \begin{equation*}
%  \begin{array}{ccl}
%   &&\textbf{A}_{\text{VMETD}}\\
%   &=&\lim_{t \rightarrow \infty} \mathbb{E}[\textbf{A}_{\text{VMETD},t}]\\
%   &=& \lim_{t \rightarrow \infty} \mathbb{E}_{\mu}[F_t \rho_t {\bm{{\bm{\phi}}}}_t ({\bm{{\bm{\phi}}}}_t - \gamma {\bm{{\bm{\phi}}}}_{t+1})^{\top}]\\  
%   &&\hspace{1em}- \lim_{t\rightarrow \infty} \mathbb{E}_{\mu}[  {\bm{{\bm{\phi}}}}_t]\mathbb{E}_{\mu}[F_t \rho_t ({\bm{{\bm{\phi}}}}_t - \gamma {\bm{{\bm{\phi}}}}_{t+1})]^{\top}\\
%   &=& \lim_{t \rightarrow \infty} \mathbb{E}_{\mu}[{\bm{{\bm{\phi}}}}_tF_t \rho_t ({\bm{{\bm{\phi}}}}_t - \gamma {\bm{{\bm{\phi}}}}_{t+1})^{\top}]\\   
%   &&\hspace{1em}-\lim_{t \rightarrow \infty} \mathbb{E}_{\mu}[ {\bm{{\bm{\phi}}}}_t]\lim_{t \rightarrow \infty}\mathbb{E}_{\mu}[F_t \rho_t ({\bm{{\bm{\phi}}}}_t - \gamma {\bm{{\bm{\phi}}}}_{t+1})]^{\top}\\
%   && \hspace{-2em}=\sum_{s} d_{\mu}(s)\lim_{t \rightarrow \infty}\mathbb{E}_{\mu}[F_t|S_t = s]\mathbb{E}_{\mu}[\rho_t\bm{{\bm{\phi}}}_t(\bm{{\bm{\phi}}}_t - \gamma \bm{{\bm{\phi}}}_{t+1})^{\top}|S_t= s]\\   
%   &&\hspace{1em}-\sum_{s} d_{\mu}(s)\bm{{\bm{\phi}}}(s)\sum_{s} d_{\mu}(s)\lim_{t \rightarrow \infty}\mathbb{E}_{\mu}[F_t|S_t = s]\\
%   &&\hspace{7em}\mathbb{E}_{\mu}[\rho_t(\bm{{\bm{\phi}}}_t - \gamma \bm{{\bm{\phi}}}_{t+1})^{\top}|S_t = s]\\
%   &=& \sum_{s} f(s)\mathbb{E}_{\pi}[\bm{{\bm{\phi}}}_t(\bm{{\bm{\phi}}}_t- \gamma \bm{{\bm{\phi}}}_{t+1})^{\top}|S_t = s]\\   
%   &&\hspace{1em}-\sum_{s} d_{\mu}(s)\bm{{\bm{\phi}}}(s)\sum_{s} f(s)\mathbb{E}_{\pi}[(\bm{{\bm{\phi}}}_t- \gamma \bm{{\bm{\phi}}}_{t+1})^{\top}|S_t = s]\\
%   &=&\sum_{s} f(s) \bm{\bm{{\bm{\phi}}}}(s)(\bm{\bm{{\bm{\phi}}}}(s) - \gamma \sum_{s'}[\textbf{P}_{\pi}]_{ss'}\bm{\bm{{\bm{\phi}}}}(s'))^{\top}  \\
%   &&-\sum_{s} d_{\mu}(s) {\bm{{\bm{\phi}}}}(s) * \sum_{s} f(s)({\bm{{\bm{\phi}}}}(s) - \gamma \sum_{s'}[\textbf{P}_{\pi}]_{ss'}{\bm{{\bm{\phi}}}}(s'))^{\top}\\
%   &=&{\bm{\bm{\Phi}}}^{\top} \textbf{F} (\textbf{I} - \gamma \textbf{P}_{\pi}) \bm{\bm{\Phi}} - {\bm{\bm{\Phi}}}^{\top} {d}_{\mu} {f}^{\top} (\textbf{I} - \gamma \textbf{P}_{\pi}) \bm{\bm{\Phi}}  \\
%   &=&{\bm{\bm{\Phi}}}^{\top} (\textbf{F} - {d}_{\mu} {f}^{\top}) (\textbf{I} - \gamma \textbf{P}_{\pi}){\bm{\bm{\Phi}}} \\
%   &=&{\bm{\bm{\Phi}}}^{\top} (\textbf{F} (\textbf{I} - \gamma \textbf{P}_{\pi})-{d}_{\mu} {f}^{\top} (\textbf{I} - \gamma \textbf{P}_{\pi})){\bm{\bm{\Phi}}} \\
%   &=&{\bm{\bm{\Phi}}}^{\top} (\textbf{F} (\textbf{I} - \gamma \textbf{P}_{\pi})-{d}_{\mu} {d}_{\mu}^{\top} ){\bm{\bm{\Phi}}},
%  \end{array}
% \end{equation*}
% \begin{equation*}
%  \begin{array}{ccl}
%   &&{b}_{\text{VMETD}}\\
%   &=&\lim_{t \rightarrow \infty} \mathbb{E}[{b}_{\text{VMETD},t}]\\
%   &=& \lim_{t \rightarrow \infty} \mathbb{E}_{\mu}[F_t\rho_tR_{t+1}{\bm{{\bm{\phi}}}}_t]\\
%   &&\hspace{2em} - \lim_{t\rightarrow \infty} \mathbb{E}_{\mu}[{\bm{{\bm{\phi}}}}_t]\mathbb{E}_{\mu}[F_t\rho_kR_{k+1}]\\  
%   &=& \lim_{t \rightarrow \infty} \mathbb{E}_{\mu}[{\bm{{\bm{\phi}}}}_tF_t\rho_tr_{t+1}]\\
%   &&\hspace{2em} - \lim_{t\rightarrow \infty} \mathbb{E}_{\mu}[  {\bm{{\bm{\phi}}}}_t]\mathbb{E}_{\mu}[{\bm{{\bm{\phi}}}}_t]\mathbb{E}_{\mu}[F_t\rho_tr_{t+1}]\\ 
%   &=& \lim_{t \rightarrow \infty} \mathbb{E}_{\mu}[{\bm{{\bm{\phi}}}}_tF_t\rho_tr_{t+1}]\\
%   &&\hspace{2em} - \lim_{t \rightarrow \infty} \mathbb{E}_{\mu}[ {\bm{{\bm{\phi}}}}_t]\lim_{t \rightarrow \infty}\mathbb{E}_{\mu}[F_t\rho_tr_{t+1}]\\  
%   &=&\sum_{s} f(s) {\bm{{\bm{\phi}}}}(s)r_{\pi} - \sum_{s} d_{\mu}(s) {\bm{{\bm{\phi}}}}(s) * \sum_{s} f(s)r_{\pi}  \\
%   &=&\bm{\bm{\bm{\Phi}}}^{\top}(\textbf{F}-{d}_{\mu} {f}^{\top}){r}_{\pi}.
%  \end{array}
% \end{equation*}



\section{Approach}
\label{sec:approach}
\section{Temporal Representation Alignment}
\label{sec:approach}

When training a series of short-horizon goal-reaching and instruction-following tasks, our goal is to learn a representation space such that our policy can generalize to a new (long-horizon) task that can be viewed as a sequence of known subtasks.
We propose to structure this representation space by aligning the representations of states, goals, and language in a way that is more amenable to compositional generalization.

\paragraph{Notation.}
We take the setting of a goal- and language-conditioned MDP $\cM$ with state space $\cS$, continuous action space $\cA \subseteq (0,1)^{d_{\cA}}$, initial state distribution $p_0$, dynamics $\p(s'\mid s,a)$, discount factor $\gamma$, and language task distribution $p_{\ell}$.
A policy $\pi(a\mid s)$ maps states to a distribution over actions. We inductively define the $k$-step (action-conditioned) policy visitation distribution as:
\begin{align*}
    p^{\pi}_{1}(s_{1} \mid s_1, a_{1})
    &\triangleq p(s_1 \mid s_1, a_1),\\
    p^{\pi}_{k+1}(s_{k+1} \mid s_1, a_1)
    &\triangleq \nonumber\\*
      &\mspace{-120mu} \int_{\cA}\int_{\cS} p(s_{k+1} \mid s,a) \dd p^{\pi}_{k}(s \mid s_{1},a_1) \dd
        \pi(a \mid s)\\
    p^{\pi}_{k+t}(s_{k+t} \mid s_t,a_t)
    &\triangleq p^{\pi}(s_{k} \mid s_1, a_1) . \eqmark
        \label{eq:successor_distribution}
\end{align*}
Then, the discounted state visitation distribution can be defined as the distribution over $s^{+}$\llap, the state reached after $K\sim \operatorname{Geom}(1-\gamma)$ steps:
\begin{equation}
    p^{\pi}_{\gamma}(s^{+}  \mid  s,a) \triangleq \sum_{k=0}^{\infty} \gamma^{k} p^{\pi}_{k}(s^{+} \mid s,a).
    \label{eq:discounted_state_visitation}
\end{equation}

We assume access to a dataset of expert demonstrations $\cD = \{\tau_{i},\ell_i\}_{i=1}^{K}$, where each trajectory
\begin{equation}
    \tau_{i} = \{s_{t,i},a_{t,i}\}_{t=1}^{H} \in \cS \times \cA
    \label{eq:trajectory}
\end{equation}
is gathered by an expert policy $\expert$, and is then annotated with $p_{\ell}(\ell_{i} \mid s_{1,i}, s_{H,i})$.
Our aim is to learn a policy $\pi$ that can select actions conditioned on a new language instruction $\ell$.
As in prior work~\citep{walke2023bridgedata}, we handle the continuous action space by representing both our policy and the expert policy as an isotropic Gaussian with fixed variance; we will equivalently write $\pi(a\mid s, \varphi)$ or denote the mode as $\hat{a} = \pi(s,\varphi)$ for a task $\varphi$.

\begin{rebuttal}
    \subsection{Representations for Reaching Distant Goals}
    \label{sec:reaching_goals}

    We learn a goal-conditioned policy $\pi(a\mid s,g)$ that selects actions to reach a goal $g$ from expert demonstrations with behavioral cloning.
    Suppose we directly selected actions to imitate the expert on two trajectories in $\cD$:
    
    \begin{equation}
        \mspace{-100mu}\begin{tikzcd}[remember picture,sep=small]
            s_1 \rar & s_2 \rar  & \ldots \rar & s_{H} \rar & w      \quad \\
            w \rar   & s_1' \rar & \ldots \rar & s_{H}' \rar & g\quad
        \end{tikzcd}
        \begin{tikzpicture}[remember picture,overlay] \coordinate (a) at (\tikzcdmatrixname-1-5.north east);
            \coordinate (b) at (\tikzcdmatrixname-2-5.south east);
            \coordinate (c) at (a|-b);
            \draw[decorate,line width=1.5pt,decoration={brace,raise=3pt,amplitude=5pt}]
        (a) -- node[right=1.5em] {$\tau_{i}\in \cD$} (c); \end{tikzpicture}
        \label{eq:trajectory_diagram}
    \end{equation}
    When conditioned with the composed goal $g$, we would be unable to imitate effectively
        as the composed state-goal $(s,g)$ is jointly out of the training distribution.

    What \emph{would} work for reaching $g$ is to first condition the policy on the intermediate waypoint $w$, then upon reaching $w$, condition on the goal $g$, as the state-goal pairs $(s_{i},w)$, $(w,g)$, and $(s_{i}',g)$ are all in the training distribution.
    If we condition the policy on some intermediate waypoint distribution $p(w)$ (or sufficient statistics thereof) that captures all of these cases, we can stitch together the expert behaviors to reach the goal $g$.

    Our approach is to learn a representation space that captures this ability, so that a GCBC objective used in this space can effectively imitate the expert on the composed task.
     We begin with the goal-conditioned behavioral cloning~\citep{kaelbling1993learning}
        loss $\cL_{\textsc{bc}}^{\phi,\psi,\xi}$ conditioned with waypoints $w$.
    \begin{equation}
        \cL_{\textsc{bc}}\bigl(\{s_{i},a_{i},s_{i}^{+},g_{i}\}_{i=1}^{K}\bigr) = \sum_{{i=1}}^{K} \log \pi\bigl(a_{i} \mid s_{i},\psi(g_{i})\bigr).
        \label{eq:goal_conditioned_bc}
    \end{equation}
    Enforcing the invariance needed to stitch \cref{eq:trajectory_diagram} then reduces to aligning \mbox{$\psi(g) \leftrightarrow \psi(w).$}
    The temporal alignment objective $\phi(s)\leftrightarrow \phi(s^{+})$ accomplishes this indirectly by aligning both $\psi(w)$ and $\psi(g)$ to the shared waypoint representation $\phi(w)$:

    \csuse{color indices}
    \begin{align}
        &\cL_{\textsc{nce}}\bigl(\{s_{i},s_{i}^{+}\}_{i=1}^{K};\phi,\psi\bigr) =
        \log \biggl( {\frac{e^{\phi(s^+_{\i})^{T}\psi(s_{\i})}}{\sum_{{\j=1}}^{K}
                e^{\phi(s^+_{\i})^{T}\psi(s_{\j})}}} \biggr)  \nonumber\\*
                &\mspace{100mu} +
        \sum_{{\j=1}}^{K} \log \biggl( {\frac{e^{\phi(s^+_{\i})^{T}\psi(s_{\i})}}{\sum_{{\i=1}}^{K}
                e^{\phi(s^+_{\i})^{T}\psi(s_{\j})}}} \biggr)
        \label{eq:goal_alignment}
    \end{align}

        
\end{rebuttal}
\subsection{Interfacing with Language Instructions}
\label{sec:language_instructions}

To extend the representations from \cref{sec:reaching_goals} to compositional instruction following with language tasks, we need some way to ground language into the $\psi$ (future state)
representation space.
We use a similar approach to GRIF~\citep{myers2023goal}, which uses an additional CLIP-style \citep{radford2021learning} contrastive alignment loss with an additional pretrained language encoder $\xi$:
\csuse{no color indices}
\begin{align}
    &\cL_{\textsc{nce}}\bigl(\{g_{i},\ell_{i}\}_{i=1}^{K};\psi,\xi\bigr)
    = \sum_{{i=1}}^{K} \log \biggl( {\frac{e^{\psi(g_{\i})^{T}\xi(\ell_{\i})}}{\sum_{{\j=1}}^{K}
            e^{\psi(g_{\i})^{T}\xi(\ell_{\j})}}} \biggr)  \nonumber\\*
            &\mspace{100mu} +
    \sum_{{\j=1}}^{K} \log \biggl( {\frac{e^{\psi(g_{\i})^{T}\xi(\ell_{\i})}}{\sum_{{\i=1}}^{K}
            e^{\psi(g_{\i})^{T}\xi(\ell_{\j})}}} \biggr)
    \label{eq:task_alignment}
\end{align}

\subsection{Temporal Alignment}
\label{sec:temporal_alignment}

Putting together the objectives from \cref{sec:reaching_goals,sec:language_instructions} yields the Temporal Representation Alignment (\Method) approach.
\Method{} structures the representation space of goals and language instructions to better enable compositional generalization.
We learn encoders $\phi, \psi ,$ and $\xi$ to map states, goals, and language instructions to a shared representation space.

\csuse{color indices}
\begin{align}
    \cL_{\textsc{nce}} \label{eq:NCE}
    &(\{x_{i}, y_{i}\}_{i=1}^{K};f,h) =
        \sum_{{\i=1}}^{K} \log \biggl( {\frac{e^{f(y_{\i})^{T}h(x_{\i})}}{\sum_{{\j=1}}^{K}
        e^{f(y_{\i})^{T}h(x_{\j})}}} \biggr) \nonumber\\*
      &\mspace{100mu} +
        \sum_{{\j=1}}^{K} \log \biggl( {\frac{e^{f(y_{\i})^{T}h(x_{\i})}}{\sum_{{\i=1}}^{K}
        e^{f(y_{\i})^{T}h(x_{\j})}}} \biggr) \\
    \cL_{\textsc{bc}} \label{eq:BC}
    &\bigl(\{s_{i},a_{i},s^{+}_{i},\ell_{i}\}_{i=1}^{K};\pi,\psi,\xi\bigr) = \nonumber\\*
      &\mspace{-10mu} \sum_{{i=1}}^{K} \log
        \pi\bigl(a_{i} \mid s_{i},\xi(\ell_{i})\bigr) + \log \pi\bigl(a_{i} \mid
        s_{i},\psi(s^{+}_{i})\bigr) \\
    \cL_{\textsc{tra}}
    &\label{eq:TRA} \bigl( \{s_{i},a_{i},s_{i}^{+},g_{i},\ell_{i}\}_{i=1}^{K}; \pi,\phi,\psi,\xi\bigr)
        \\
    &= \underbrace{\cL_{\textsc{bc}}\bigl(\{s_{i},a_{i},s_{i}^{+},\ell_{i}\}_{i=1}^{K};\pi,\psi,\xi\bigr)}_{\text{behavioral
    cloning}} \nonumber\\*
    &+
        \underbrace{\cL_{\textsc{nce}}\bigl(\{s_{i},s_{i}^{+}\}_{i=1}^{K};\phi,\psi\bigr)}_{\text{temporal alignment}}
        + \underbrace{\cL_{\textsc{nce}}\bigl(\{g_{i},\ell_{i}\}_{i=1}^{K};\psi,\xi\bigr)}_{\text{task alignment}} \nonumber
\end{align}Note that the NCE alignment loss uses a CLIP-style symmetric contrastive objective~\citep{radford2021learning,eysenbach2024inference} \-- we highlight the indices in the NCE alignment loss~\eqref{eq:NCE} for clarity.

Our overall objective is to minimize \cref{eq:TRA} across states, actions, future states, goals, and language tasks within the training data:
\begin{align}
    &\min_{\pi,\phi,\psi,\xi} \mathbb{E}_{\substack{(s_{1,i},a_{1,i},\ldots,s_{H,i},a_{H,i},\ell) \sim \mathcal{D} \\
    i\sim\operatorname{Unif}(1\ldots H) \\
    k\sim\operatorname{Geom}(1-\gamma)}} \\*
    &\mspace{10mu}
    \Bigl[\cL_{\text{TRA}}\bigl(\{s_{t,i},a_{t,i},s_{\min(t+k,H),i},s_{H,i},\ell\}_{i=1}^{K};\pi,\phi,\psi,\xi\bigr)\Bigr].
    \label{eq:overall_objective}
\end{align}

\begin{algorithm}
    \caption{Temporal Representation Alignment}
    \label{alg:tra}
    \begin{algorithmic}[1]
        \State \textbf{input:} dataset $\mathcal{D} = (\{s_{t,i},a_{t,i}\}_{t=1}^{H},\ell_i)_{i=1}^N$
        \State initialize networks $\Theta \triangleq (\pi,\phi,\psi,\xi)$
        \While{training}
        \State sample batch $\bigl\{(s_{t,i},a_{t,i},s_{t+k,i},\ell_i)\bigr\}_{i=1}^K\sim\mathcal{D}$ \\
        \hspace*{2ex} for $k\sim\operatorname{Geom}(1-\gamma)$
        \State $\Theta \gets \Theta - \alpha \nabla_{\Theta} \cL_{\text{TRA}}\bigl(\{s_{t,i},a_{t,i},s_{t+k,i},\ell_i\}_{i=1}^K; \Theta\bigr)$
        \EndWhile
        \smallskip
        \State \textbf{output:} \parbox[t]{\linewidth}{language-conditioned policy $\pi(a_{t} | s_{t}, \xi(\ell))$ \\
            goal-conditioned policy $\pi(a_{t} | s_{t}, \psi(g))$
        }
    \end{algorithmic}
\end{algorithm}

\subsection{Implementation}
\label{sec:implementation}

A summary of our approach is shown in \cref{alg:tra}.
In essence, TRA learns three encoders: $\phi$, which encodes states, $\psi$ which encodes future goals, and $\xi$ which encodes language instructions.
Contrastive losses are used to align state representations $\phi(s_{t})$ with future goal representations $\psi(s_{t+k})$, which are in turn aligned with equivalent language task specifications $\xi(\ell)$ when available.
We then learn a behavior cloning policy $\pi$ that can be conditioned on either the goal or language instruction through the representation $\psi(g)$ or $\xi(\ell)$, respectively.

\begin{rebuttal}
    \subsection{Temporal Alignment and Compositionality}
    \label{sec:compositionality}

    We will formalize the intuition from \cref{sec:reaching_goals} that \Method{} enables compositional generalization by considering the error on a ``compositional'' version of $\cD,$ denoted $\cD^{*}$.
    Using the notation from \cref{eq:trajectory}, we can say $\cD$ is distributed according to:
    \begin{align}
        &\cD \triangleq \cD^{H} \sim \prod_{i=1}^{K} p_0(s_{1,i}) p_{\ell}(\ell_{i} \mid s_{1,i}, s_{H,i})
            \nonumber\\*
          &\mspace{60mu} \prod_{t=1}^{H} \expert(a_{t,i} \mid s_{t,i}) \p(s_{t+1,i} \mid s_{t,i}, a_{t,i}) ,
            \label{eq:dataset_distribution}
    \end{align}
    or equivalently
    \begin{align}
            &\cD^{H} \sim \prod_{i=1}^{K} p_0(s_{1,i}) p_{\ell}(\ell_{i} \mid s_{1,i}, s_{H,i}) \nonumber\\*
            &\mspace{60mu} \prod_{t=1}^{H}
            e^{\sigma^2\|\expert(s_{t,i}) - a_{t,i}\|^2}\p(s_{t+1,i} \mid s_{t,i}, a_{t,i}) ,
            \label{eq:dataset_distribution_2}
    \end{align}
    by the isotropic Gaussian assumption.
    We will define $\cD^{*} \triangleq \cD^{H'}$ to be a longer-horizon version of $\cD$ extending the behaviors gathered under $\expert$ across a horizon $\alpha H \ge H' \ge H$ that additionally satisfies a ``time-isotropy'' property: the marginal distribution of the states is uniform across the horizon, i.e., $p_0(s_{1,i}) = p_0(s_{t,i})$ for all $t \in \{1\ldots H'\}$.

    We will relate the in-distribution imitation error $\textsc{Err}(\bullet; \cD)$ to the compositional out-of-distribution imitation error $\textsc{Err}(\bullet;\cD^{*})$.
    We define
    \begin{align}
        \textsc{Err}(\hat{\pi}; \tilde{\cD})
        &= \E_{\tilde{\cD}}\Bigl[\frac{1}{H}\sum_{t=1}^{H} \mathbb{E}_{\hat{\pi}}\left[\|\tilde{a}_{t,i} -
        \hat{\pi}(\tilde{s}_{t,i}, \tilde{s}_{H, i})\|^{2}/d_{\cA}\right]\Bigr] \nonumber\\
        &\quad \text{for} \quad \{\tilde{s}_{t,i},\tilde{a}_{t,i},\tilde{\ell}_{i}\}_{t=1}^{H} \sim
            \tilde{\cD}.
            \label{eq:imitation_error}
    \end{align}
    On the training dataset this is equivalent to the expected behavioral cloning loss from \cref{eq:BC}.

                            
    \begin{assumption}
        \label{asm:policy_factorization}
        The policy factorizes through inferred waypoints as:
\begin{align}
    &\textrm{goals: }\pi(a \mid s, g)
        = \nonumber\\*
      &\mspace{50mu} \int \pi(a\mid s, w) \p(s_{t}=w \mid s_{t+k}=g) \dd{w}
        \label{eq:goal-conditioned} \\
    &\textrm{language: } \pi(a \mid s, \ell)
        = \int \pi(a\mid s, w) \nonumber\\*
      &\mspace{20mu} \p(s_{t}=w \mid s_{t+k}=g) \p(s_{t+k}=g \mid \ell) \dd{w} \dd{g} ,
        \label{eq:language-conditioned}
        \end{align}
        where denote by $\pi(s,g)$ the MLE estimate of the action $a$.

    \end{assumption}

    \makerestatable
    \begin{theorem}
        \label{thm:compositionality}
        Suppose $\cD$ is distributed according to \cref{eq:dataset_distribution} and $\cD^{*}$ is distributed according to \cref{eq:dataset_distribution}.
        When $\gamma > 1-1/H$ and $\alpha > 1$, for optimal features $\phi$ and $\psi$ under \cref{eq:overall_objective}, we have
        \begin{gather}
            \textsc{Err}(\pi; \cD^{*}) \le \textsc{Err}(\pi; \cD) +  \frac{\alpha -1}{2 \alpha }+\Bigl(\frac{ \alpha - 2 }{2\alpha}\Bigr) \1 \{\alpha >2\}  .
            \label{eq:compositionality}
        \end{gather}
    \end{theorem}

    We can also define a notion of the language-conditioned compositional generalization error:
    \begin{equation*}
        \errl(\pi; \cD^{*}) \triangleq \E_{\cD^{*}}\Bigl[\frac{1}{H}\sum_{t=1}^{H}
            \mathbb{E}_{\pi}\bigl[\|\tilde{a}_{t,i} - \pi(\tilde{s}_{t,i}, \tilde{\ell}_{i})\|^{2}\bigr]\Bigr].
            \label{eq:language_error}
    \end{equation*}

    \makerestatable
    \begin{corollary}
        \label{thm:language}
        Under the same conditions as \cref{thm:compositionality},
        \begin{equation*}
            \errl(\pi; \cD^{*}) \le \errl(\pi; \cD) +  \frac{\alpha -1}{2 \alpha }+\Bigl(\frac{ \alpha - 2 }{2\alpha}\Bigr) \1 \{\alpha >2\}  .
            \label{eq:compositionality_language}
        \end{equation*}

    \end{corollary}

    The proofs as well as a visualization of the bound are in \cref{app:compositionality}. Policy implementation details can be found in \cref{app:tra_impl}

    
                
        
    \end{rebuttal}




\section{Results}
\label{sec:results}
\subsection{Benchmark}
\label{sec:benchmark}
\begin{table}[t]
    \centering
    \caption{The performance of different pre-trained models on ImageNet and infrared semantic segmentation datasets. The \textit{Scratch} means the performance of randomly initialized models. The \textit{PT Epochs} denotes the pre-training epochs while the \textit{IN1K FT epochs} represents the fine-tuning epochs on ImageNet \citep{imagenet}. $^\dag$ denotes models reproduced using official codes. $^\star$ refers to the effective epochs used in \citet{iBOT}. The top two results are marked in \textbf{bold} and \underline{underlined} format. Supervised and CL methods, MIM methods, and UNIP models are colored in \colorbox{orange!15}{\rule[-0.2ex]{0pt}{1.5ex}orange}, \colorbox{gray!15}{\rule[-0.2ex]{0pt}{1.5ex}gray}, and \colorbox{cyan!15}{\rule[-0.2ex]{0pt}{1.5ex}cyan}, respectively.}
    \label{tab:benchmark}
    \centering
    \scriptsize
    \setlength{\tabcolsep}{1.0mm}{
    \scalebox{1.0}{
    \begin{tabular}{l c c c c  c c c c c c c c}
        \toprule
         \multirow{2}{*}{Methods} & \multirow{2}{*}{\makecell[c]{PT \\ Epochs}} & \multicolumn{2}{c}{IN1K FT} & \multicolumn{4}{c}{Fine-tuning (FT)} & \multicolumn{4}{c}{Linear Probing (LP)} \\
         \cmidrule{3-4} \cmidrule(lr){5-8} \cmidrule(lr){9-12} 
         & & Epochs & Acc & SODA & MFNet-T & SCUT-Seg & Mean & SODA & MFNet-T & SCUT-Seg & Mean \\
         \midrule
         \textcolor{gray}{ViT-Tiny/16} & & &  & & & & & & & & \\
         Scratch & - & - & - & 31.34 & 19.50 & 41.09 & 30.64 & - & - & - & - \\
         \rowcolor{gray!15} MAE$^\dag$ \citep{mae} & 800 & 200 & \underline{71.8} & 52.85 & 35.93 & 51.31 & 46.70 & 23.75 & 15.79 & 27.18 & 22.24 \\
         \rowcolor{orange!15} DeiT \citep{deit} & 300 & - & \textbf{72.2} & 63.14 & 44.60 & 61.36 & 56.37 & 42.29 & 21.78 & 31.96 & 32.01 \\
         \rowcolor{cyan!15} UNIP (MAE-L) & 100 & - & - & \underline{64.83} & \textbf{48.77} & \underline{67.22} & \underline{60.27} & \underline{44.12} & \underline{28.26} & \underline{35.09} & \underline{35.82} \\
         \rowcolor{cyan!15} UNIP (iBOT-L) & 100 & - & - & \textbf{65.54} & \underline{48.45} & \textbf{67.73} & \textbf{60.57} & \textbf{52.95} & \textbf{30.10} & \textbf{40.12} & \textbf{41.06}  \\
         \midrule
         \textcolor{gray}{ViT-Small/16} & & & & & & & & & & & \\
         Scratch & - & - & - & 41.70 & 22.49 & 46.28 & 36.82 & - & - & - & - \\
         \rowcolor{gray!15} MAE$^\dag$ \citep{mae} & 800 & 200 & 80.0 & 63.36 & 42.44 & 60.38 & 55.39 & 38.17 & 21.14 & 34.15 & 31.15 \\
         \rowcolor{gray!15} CrossMAE \citep{crossmae} & 800 & 200 & 80.5 & 63.95 & 43.99 & 63.53 & 57.16 & 39.40 & 23.87 & 34.01 & 32.43 \\
         \rowcolor{orange!15} DeiT \citep{deit} & 300 & - & 79.9 & 68.08 & 45.91 & 66.17 & 60.05 & 44.88 & 28.53 & 38.92 & 37.44 \\
         \rowcolor{orange!15} DeiT III \citep{deit3} & 800 & - & 81.4 & 69.35 & 47.73 & 67.32 & 61.47 & 54.17 & 32.01 & 43.54 & 43.24 \\
         \rowcolor{orange!15} DINO \citep{dino} & 3200$^\star$ & 200 & \underline{82.0} & 68.56 & 47.98 & 68.74 & 61.76 & 56.02 & 32.94 & 45.94 & 44.97 \\
         \rowcolor{orange!15} iBOT \citep{iBOT} & 3200$^\star$ & 200 & \textbf{82.3} & 69.33 & 47.15 & 69.80 & 62.09 & 57.10 & 33.87 & 45.82 & 45.60 \\
         \rowcolor{cyan!15} UNIP (DINO-B) & 100 & - & - & 69.35 & 49.95 & 69.70 & 63.00 & \underline{57.76} & \underline{34.15} & \underline{46.37} & \underline{46.09} \\
         \rowcolor{cyan!15} UNIP (MAE-L) & 100 & - & - & \textbf{70.99} & \underline{51.32} & \underline{70.79} & \underline{64.37} & 55.25 & 33.49 & 43.37 & 44.04 \\
         \rowcolor{cyan!15} UNIP (iBOT-L) & 100 & - & - & \underline{70.75} & \textbf{51.81} & \textbf{71.55} & \textbf{64.70} & \textbf{60.28} & \textbf{37.16} & \textbf{47.68} & \textbf{48.37} \\ 
        \midrule
        \textcolor{gray}{ViT-Base/16} & & & & & & & & & & & \\
        Scratch & - & - & - & 44.25 & 23.72 & 49.44 & 39.14 & - & - & - & - \\
        \rowcolor{gray!15} MAE \citep{mae} & 1600 & 100 & 83.6 & 68.18 & 46.78 & 67.86 & 60.94 & 43.01 & 23.42 & 37.48 & 34.64 \\
        \rowcolor{gray!15} CrossMAE \citep{crossmae} & 800 & 100 & 83.7 & 68.29 & 47.85 & 68.39 & 61.51 & 43.35 & 26.03 & 38.36 & 35.91 \\
        \rowcolor{orange!15} DeiT \citep{deit} & 300 & - & 81.8 & 69.73 & 48.59 & 69.35 & 62.56 & 57.40 & 34.82 & 46.44 & 46.22 \\
        \rowcolor{orange!15} DeiT III \citep{deit3} & 800 & 20 & \underline{83.8} & 71.09 & 49.62 & 70.19 & 63.63 & 59.01 & \underline{35.34} & 48.01 & 47.45 \\
        \rowcolor{orange!15} DINO \citep{dino} & 1600$^\star$ & 100 & 83.6 & 69.79 & 48.54 & 69.82 & 62.72 & 59.33 & 34.86 & 47.23 & 47.14 \\
        \rowcolor{orange!15} iBOT \citep{iBOT} & 1600$^\star$ & 100 & \textbf{84.0} & 71.15 & 48.98 & 71.26 & 63.80 & \underline{60.05} & 34.34 & \underline{49.12} & \underline{47.84} \\
        \rowcolor{cyan!15} UNIP (MAE-L) & 100 & - & - & \underline{71.47} & \textbf{52.55} & \underline{71.82} & \textbf{65.28} & 58.82 & 34.75 & 48.74 & 47.43 \\
        \rowcolor{cyan!15} UNIP (iBOT-L) & 100 & - & - & \textbf{71.75} & \underline{51.46} & \textbf{72.00} & \underline{65.07} & \textbf{63.14} & \textbf{39.08} & \textbf{52.53} & \textbf{51.58} \\
        \midrule
        \textcolor{gray}{ViT-Large/16} & & & & & & & & & & & \\
        Scratch & - & - & - & 44.70 & 23.68 & 49.55 & 39.31 & - & - & - & - \\
        \rowcolor{gray!15} MAE \citep{mae} & 1600 & 50 & \textbf{85.9} & 71.04 & \underline{51.17} & 70.83 & 64.35 & 52.20 & 31.21 & 43.71 & 42.37 \\
        \rowcolor{gray!15} CrossMAE \citep{crossmae} & 800 & 50 & 85.4 & 70.48 & 50.97 & 70.24 & 63.90 & 53.29 & 33.09 & 45.01 & 43.80 \\
        \rowcolor{orange!15} DeiT3 \citep{deit3} & 800 & 20 & \underline{84.9} & \underline{71.67} & 50.78 & \textbf{71.54} & \underline{64.66} & \underline{59.42} & \textbf{37.57} & \textbf{50.27} & \underline{49.09} \\
        \rowcolor{orange!15} iBOT \citep{iBOT} & 1000$^\star$ & 50 & 84.8 & \textbf{71.75} & \textbf{51.66} & \underline{71.49} & \textbf{64.97} & \textbf{61.73} & \underline{36.68} & \underline{50.12} & \textbf{49.51} \\
        \bottomrule
    \end{tabular}}}
    \vspace{-2mm}
\end{table}
\subsection{Evaluation}
\section{Experiments}
\label{sec:experiments}
The experiments are designed to address two key research questions.
First, \textbf{RQ1} evaluates whether the average $L_2$-norm of the counterfactual perturbation vectors ($\overline{||\perturb||}$) decreases as the model overfits the data, thereby providing further empirical validation for our hypothesis.
Second, \textbf{RQ2} evaluates the ability of the proposed counterfactual regularized loss, as defined in (\ref{eq:regularized_loss2}), to mitigate overfitting when compared to existing regularization techniques.

% The experiments are designed to address three key research questions. First, \textbf{RQ1} investigates whether the mean perturbation vector norm decreases as the model overfits the data, aiming to further validate our intuition. Second, \textbf{RQ2} explores whether the mean perturbation vector norm can be effectively leveraged as a regularization term during training, offering insights into its potential role in mitigating overfitting. Finally, \textbf{RQ3} examines whether our counterfactual regularizer enables the model to achieve superior performance compared to existing regularization methods, thus highlighting its practical advantage.

\subsection{Experimental Setup}
\textbf{\textit{Datasets, Models, and Tasks.}}
The experiments are conducted on three datasets: \textit{Water Potability}~\cite{kadiwal2020waterpotability}, \textit{Phomene}~\cite{phomene}, and \textit{CIFAR-10}~\cite{krizhevsky2009learning}. For \textit{Water Potability} and \textit{Phomene}, we randomly select $80\%$ of the samples for the training set, and the remaining $20\%$ for the test set, \textit{CIFAR-10} comes already split. Furthermore, we consider the following models: Logistic Regression, Multi-Layer Perceptron (MLP) with 100 and 30 neurons on each hidden layer, and PreactResNet-18~\cite{he2016cvecvv} as a Convolutional Neural Network (CNN) architecture.
We focus on binary classification tasks and leave the extension to multiclass scenarios for future work. However, for datasets that are inherently multiclass, we transform the problem into a binary classification task by selecting two classes, aligning with our assumption.

\smallskip
\noindent\textbf{\textit{Evaluation Measures.}} To characterize the degree of overfitting, we use the test loss, as it serves as a reliable indicator of the model's generalization capability to unseen data. Additionally, we evaluate the predictive performance of each model using the test accuracy.

\smallskip
\noindent\textbf{\textit{Baselines.}} We compare CF-Reg with the following regularization techniques: L1 (``Lasso''), L2 (``Ridge''), and Dropout.

\smallskip
\noindent\textbf{\textit{Configurations.}}
For each model, we adopt specific configurations as follows.
\begin{itemize}
\item \textit{Logistic Regression:} To induce overfitting in the model, we artificially increase the dimensionality of the data beyond the number of training samples by applying a polynomial feature expansion. This approach ensures that the model has enough capacity to overfit the training data, allowing us to analyze the impact of our counterfactual regularizer. The degree of the polynomial is chosen as the smallest degree that makes the number of features greater than the number of data.
\item \textit{Neural Networks (MLP and CNN):} To take advantage of the closed-form solution for computing the optimal perturbation vector as defined in (\ref{eq:opt-delta}), we use a local linear approximation of the neural network models. Hence, given an instance $\inst_i$, we consider the (optimal) counterfactual not with respect to $\model$ but with respect to:
\begin{equation}
\label{eq:taylor}
    \model^{lin}(\inst) = \model(\inst_i) + \nabla_{\inst}\model(\inst_i)(\inst - \inst_i),
\end{equation}
where $\model^{lin}$ represents the first-order Taylor approximation of $\model$ at $\inst_i$.
Note that this step is unnecessary for Logistic Regression, as it is inherently a linear model.
\end{itemize}

\smallskip
\noindent \textbf{\textit{Implementation Details.}} We run all experiments on a machine equipped with an AMD Ryzen 9 7900 12-Core Processor and an NVIDIA GeForce RTX 4090 GPU. Our implementation is based on the PyTorch Lightning framework. We use stochastic gradient descent as the optimizer with a learning rate of $\eta = 0.001$ and no weight decay. We use a batch size of $128$. The training and test steps are conducted for $6000$ epochs on the \textit{Water Potability} and \textit{Phoneme} datasets, while for the \textit{CIFAR-10} dataset, they are performed for $200$ epochs.
Finally, the contribution $w_i^{\varepsilon}$ of each training point $\inst_i$ is uniformly set as $w_i^{\varepsilon} = 1~\forall i\in \{1,\ldots,m\}$.

The source code implementation for our experiments is available at the following GitHub repository: \url{https://anonymous.4open.science/r/COCE-80B4/README.md} 

\subsection{RQ1: Counterfactual Perturbation vs. Overfitting}
To address \textbf{RQ1}, we analyze the relationship between the test loss and the average $L_2$-norm of the counterfactual perturbation vectors ($\overline{||\perturb||}$) over training epochs.

In particular, Figure~\ref{fig:delta_loss_epochs} depicts the evolution of $\overline{||\perturb||}$ alongside the test loss for an MLP trained \textit{without} regularization on the \textit{Water Potability} dataset. 
\begin{figure}[ht]
    \centering
    \includegraphics[width=0.85\linewidth]{img/delta_loss_epochs.png}
    \caption{The average counterfactual perturbation vector $\overline{||\perturb||}$ (left $y$-axis) and the cross-entropy test loss (right $y$-axis) over training epochs ($x$-axis) for an MLP trained on the \textit{Water Potability} dataset \textit{without} regularization.}
    \label{fig:delta_loss_epochs}
\end{figure}

The plot shows a clear trend as the model starts to overfit the data (evidenced by an increase in test loss). 
Notably, $\overline{||\perturb||}$ begins to decrease, which aligns with the hypothesis that the average distance to the optimal counterfactual example gets smaller as the model's decision boundary becomes increasingly adherent to the training data.

It is worth noting that this trend is heavily influenced by the choice of the counterfactual generator model. In particular, the relationship between $\overline{||\perturb||}$ and the degree of overfitting may become even more pronounced when leveraging more accurate counterfactual generators. However, these models often come at the cost of higher computational complexity, and their exploration is left to future work.

Nonetheless, we expect that $\overline{||\perturb||}$ will eventually stabilize at a plateau, as the average $L_2$-norm of the optimal counterfactual perturbations cannot vanish to zero.

% Additionally, the choice of employing the score-based counterfactual explanation framework to generate counterfactuals was driven to promote computational efficiency.

% Future enhancements to the framework may involve adopting models capable of generating more precise counterfactuals. While such approaches may yield to performance improvements, they are likely to come at the cost of increased computational complexity.


\subsection{RQ2: Counterfactual Regularization Performance}
To answer \textbf{RQ2}, we evaluate the effectiveness of the proposed counterfactual regularization (CF-Reg) by comparing its performance against existing baselines: unregularized training loss (No-Reg), L1 regularization (L1-Reg), L2 regularization (L2-Reg), and Dropout.
Specifically, for each model and dataset combination, Table~\ref{tab:regularization_comparison} presents the mean value and standard deviation of test accuracy achieved by each method across 5 random initialization. 

The table illustrates that our regularization technique consistently delivers better results than existing methods across all evaluated scenarios, except for one case -- i.e., Logistic Regression on the \textit{Phomene} dataset. 
However, this setting exhibits an unusual pattern, as the highest model accuracy is achieved without any regularization. Even in this case, CF-Reg still surpasses other regularization baselines.

From the results above, we derive the following key insights. First, CF-Reg proves to be effective across various model types, ranging from simple linear models (Logistic Regression) to deep architectures like MLPs and CNNs, and across diverse datasets, including both tabular and image data. 
Second, CF-Reg's strong performance on the \textit{Water} dataset with Logistic Regression suggests that its benefits may be more pronounced when applied to simpler models. However, the unexpected outcome on the \textit{Phoneme} dataset calls for further investigation into this phenomenon.


\begin{table*}[h!]
    \centering
    \caption{Mean value and standard deviation of test accuracy across 5 random initializations for different model, dataset, and regularization method. The best results are highlighted in \textbf{bold}.}
    \label{tab:regularization_comparison}
    \begin{tabular}{|c|c|c|c|c|c|c|}
        \hline
        \textbf{Model} & \textbf{Dataset} & \textbf{No-Reg} & \textbf{L1-Reg} & \textbf{L2-Reg} & \textbf{Dropout} & \textbf{CF-Reg (ours)} \\ \hline
        Logistic Regression   & \textit{Water}   & $0.6595 \pm 0.0038$   & $0.6729 \pm 0.0056$   & $0.6756 \pm 0.0046$  & N/A    & $\mathbf{0.6918 \pm 0.0036}$                     \\ \hline
        MLP   & \textit{Water}   & $0.6756 \pm 0.0042$   & $0.6790 \pm 0.0058$   & $0.6790 \pm 0.0023$  & $0.6750 \pm 0.0036$    & $\mathbf{0.6802 \pm 0.0046}$                    \\ \hline
%        MLP   & \textit{Adult}   & $0.8404 \pm 0.0010$   & $\mathbf{0.8495 \pm 0.0007}$   & $0.8489 \pm 0.0014$  & $\mathbf{0.8495 \pm 0.0016}$     & $0.8449 \pm 0.0019$                    \\ \hline
        Logistic Regression   & \textit{Phomene}   & $\mathbf{0.8148 \pm 0.0020}$   & $0.8041 \pm 0.0028$   & $0.7835 \pm 0.0176$  & N/A    & $0.8098 \pm 0.0055$                     \\ \hline
        MLP   & \textit{Phomene}   & $0.8677 \pm 0.0033$   & $0.8374 \pm 0.0080$   & $0.8673 \pm 0.0045$  & $0.8672 \pm 0.0042$     & $\mathbf{0.8718 \pm 0.0040}$                    \\ \hline
        CNN   & \textit{CIFAR-10} & $0.6670 \pm 0.0233$   & $0.6229 \pm 0.0850$   & $0.7348 \pm 0.0365$   & N/A    & $\mathbf{0.7427 \pm 0.0571}$                     \\ \hline
    \end{tabular}
\end{table*}

\begin{table*}[htb!]
    \centering
    \caption{Hyperparameter configurations utilized for the generation of Table \ref{tab:regularization_comparison}. For our regularization the hyperparameters are reported as $\mathbf{\alpha/\beta}$.}
    \label{tab:performance_parameters}
    \begin{tabular}{|c|c|c|c|c|c|c|}
        \hline
        \textbf{Model} & \textbf{Dataset} & \textbf{No-Reg} & \textbf{L1-Reg} & \textbf{L2-Reg} & \textbf{Dropout} & \textbf{CF-Reg (ours)} \\ \hline
        Logistic Regression   & \textit{Water}   & N/A   & $0.0093$   & $0.6927$  & N/A    & $0.3791/1.0355$                     \\ \hline
        MLP   & \textit{Water}   & N/A   & $0.0007$   & $0.0022$  & $0.0002$    & $0.2567/1.9775$                    \\ \hline
        Logistic Regression   &
        \textit{Phomene}   & N/A   & $0.0097$   & $0.7979$  & N/A    & $0.0571/1.8516$                     \\ \hline
        MLP   & \textit{Phomene}   & N/A   & $0.0007$   & $4.24\cdot10^{-5}$  & $0.0015$    & $0.0516/2.2700$                    \\ \hline
       % MLP   & \textit{Adult}   & N/A   & $0.0018$   & $0.0018$  & $0.0601$     & $0.0764/2.2068$                    \\ \hline
        CNN   & \textit{CIFAR-10} & N/A   & $0.0050$   & $0.0864$ & N/A    & $0.3018/
        2.1502$                     \\ \hline
    \end{tabular}
\end{table*}

\begin{table*}[htb!]
    \centering
    \caption{Mean value and standard deviation of training time across 5 different runs. The reported time (in seconds) corresponds to the generation of each entry in Table \ref{tab:regularization_comparison}. Times are }
    \label{tab:times}
    \begin{tabular}{|c|c|c|c|c|c|c|}
        \hline
        \textbf{Model} & \textbf{Dataset} & \textbf{No-Reg} & \textbf{L1-Reg} & \textbf{L2-Reg} & \textbf{Dropout} & \textbf{CF-Reg (ours)} \\ \hline
        Logistic Regression   & \textit{Water}   & $222.98 \pm 1.07$   & $239.94 \pm 2.59$   & $241.60 \pm 1.88$  & N/A    & $251.50 \pm 1.93$                     \\ \hline
        MLP   & \textit{Water}   & $225.71 \pm 3.85$   & $250.13 \pm 4.44$   & $255.78 \pm 2.38$  & $237.83 \pm 3.45$    & $266.48 \pm 3.46$                    \\ \hline
        Logistic Regression   & \textit{Phomene}   & $266.39 \pm 0.82$ & $367.52 \pm 6.85$   & $361.69 \pm 4.04$  & N/A   & $310.48 \pm 0.76$                    \\ \hline
        MLP   &
        \textit{Phomene} & $335.62 \pm 1.77$   & $390.86 \pm 2.11$   & $393.96 \pm 1.95$ & $363.51 \pm 5.07$    & $403.14 \pm 1.92$                     \\ \hline
       % MLP   & \textit{Adult}   & N/A   & $0.0018$   & $0.0018$  & $0.0601$     & $0.0764/2.2068$                    \\ \hline
        CNN   & \textit{CIFAR-10} & $370.09 \pm 0.18$   & $395.71 \pm 0.55$   & $401.38 \pm 0.16$ & N/A    & $1287.8 \pm 0.26$                     \\ \hline
    \end{tabular}
\end{table*}

\subsection{Feasibility of our Method}
A crucial requirement for any regularization technique is that it should impose minimal impact on the overall training process.
In this respect, CF-Reg introduces an overhead that depends on the time required to find the optimal counterfactual example for each training instance. 
As such, the more sophisticated the counterfactual generator model probed during training the higher would be the time required. However, a more advanced counterfactual generator might provide a more effective regularization. We discuss this trade-off in more details in Section~\ref{sec:discussion}.

Table~\ref{tab:times} presents the average training time ($\pm$ standard deviation) for each model and dataset combination listed in Table~\ref{tab:regularization_comparison}.
We can observe that the higher accuracy achieved by CF-Reg using the score-based counterfactual generator comes with only minimal overhead. However, when applied to deep neural networks with many hidden layers, such as \textit{PreactResNet-18}, the forward derivative computation required for the linearization of the network introduces a more noticeable computational cost, explaining the longer training times in the table.

\subsection{Hyperparameter Sensitivity Analysis}
The proposed counterfactual regularization technique relies on two key hyperparameters: $\alpha$ and $\beta$. The former is intrinsic to the loss formulation defined in (\ref{eq:cf-train}), while the latter is closely tied to the choice of the score-based counterfactual explanation method used.

Figure~\ref{fig:test_alpha_beta} illustrates how the test accuracy of an MLP trained on the \textit{Water Potability} dataset changes for different combinations of $\alpha$ and $\beta$.

\begin{figure}[ht]
    \centering
    \includegraphics[width=0.85\linewidth]{img/test_acc_alpha_beta.png}
    \caption{The test accuracy of an MLP trained on the \textit{Water Potability} dataset, evaluated while varying the weight of our counterfactual regularizer ($\alpha$) for different values of $\beta$.}
    \label{fig:test_alpha_beta}
\end{figure}

We observe that, for a fixed $\beta$, increasing the weight of our counterfactual regularizer ($\alpha$) can slightly improve test accuracy until a sudden drop is noticed for $\alpha > 0.1$.
This behavior was expected, as the impact of our penalty, like any regularization term, can be disruptive if not properly controlled.

Moreover, this finding further demonstrates that our regularization method, CF-Reg, is inherently data-driven. Therefore, it requires specific fine-tuning based on the combination of the model and dataset at hand.

\begin{table*}[t]
\centering
\fontsize{11pt}{11pt}\selectfont
\begin{tabular}{lllllllllllll}
\toprule
\multicolumn{1}{c}{\textbf{task}} & \multicolumn{2}{c}{\textbf{Mir}} & \multicolumn{2}{c}{\textbf{Lai}} & \multicolumn{2}{c}{\textbf{Ziegen.}} & \multicolumn{2}{c}{\textbf{Cao}} & \multicolumn{2}{c}{\textbf{Alva-Man.}} & \multicolumn{1}{c}{\textbf{avg.}} & \textbf{\begin{tabular}[c]{@{}l@{}}avg.\\ rank\end{tabular}} \\
\multicolumn{1}{c}{\textbf{metrics}} & \multicolumn{1}{c}{\textbf{cor.}} & \multicolumn{1}{c}{\textbf{p-v.}} & \multicolumn{1}{c}{\textbf{cor.}} & \multicolumn{1}{c}{\textbf{p-v.}} & \multicolumn{1}{c}{\textbf{cor.}} & \multicolumn{1}{c}{\textbf{p-v.}} & \multicolumn{1}{c}{\textbf{cor.}} & \multicolumn{1}{c}{\textbf{p-v.}} & \multicolumn{1}{c}{\textbf{cor.}} & \multicolumn{1}{c}{\textbf{p-v.}} &  &  \\ \midrule
\textbf{S-Bleu} & 0.50 & 0.0 & 0.47 & 0.0 & 0.59 & 0.0 & 0.58 & 0.0 & 0.68 & 0.0 & 0.57 & 5.8 \\
\textbf{R-Bleu} & -- & -- & 0.27 & 0.0 & 0.30 & 0.0 & -- & -- & -- & -- & - &  \\
\textbf{S-Meteor} & 0.49 & 0.0 & 0.48 & 0.0 & 0.61 & 0.0 & 0.57 & 0.0 & 0.64 & 0.0 & 0.56 & 6.1 \\
\textbf{R-Meteor} & -- & -- & 0.34 & 0.0 & 0.26 & 0.0 & -- & -- & -- & -- & - &  \\
\textbf{S-Bertscore} & \textbf{0.53} & 0.0 & {\ul 0.80} & 0.0 & \textbf{0.70} & 0.0 & {\ul 0.66} & 0.0 & {\ul0.78} & 0.0 & \textbf{0.69} & \textbf{1.7} \\
\textbf{R-Bertscore} & -- & -- & 0.51 & 0.0 & 0.38 & 0.0 & -- & -- & -- & -- & - &  \\
\textbf{S-Bleurt} & {\ul 0.52} & 0.0 & {\ul 0.80} & 0.0 & 0.60 & 0.0 & \textbf{0.70} & 0.0 & \textbf{0.80} & 0.0 & {\ul 0.68} & {\ul 2.3} \\
\textbf{R-Bleurt} & -- & -- & 0.59 & 0.0 & -0.05 & 0.13 & -- & -- & -- & -- & - &  \\
\textbf{S-Cosine} & 0.51 & 0.0 & 0.69 & 0.0 & {\ul 0.62} & 0.0 & 0.61 & 0.0 & 0.65 & 0.0 & 0.62 & 4.4 \\
\textbf{R-Cosine} & -- & -- & 0.40 & 0.0 & 0.29 & 0.0 & -- & -- & -- & -- & - & \\ \midrule
\textbf{QuestEval} & 0.23 & 0.0 & 0.25 & 0.0 & 0.49 & 0.0 & 0.47 & 0.0 & 0.62 & 0.0 & 0.41 & 9.0 \\
\textbf{LLaMa3} & 0.36 & 0.0 & \textbf{0.84} & 0.0 & {\ul{0.62}} & 0.0 & 0.61 & 0.0 &  0.76 & 0.0 & 0.64 & 3.6 \\
\textbf{our (3b)} & 0.49 & 0.0 & 0.73 & 0.0 & 0.54 & 0.0 & 0.53 & 0.0 & 0.7 & 0.0 & 0.60 & 5.8 \\
\textbf{our (8b)} & 0.48 & 0.0 & 0.73 & 0.0 & 0.52 & 0.0 & 0.53 & 0.0 & 0.7 & 0.0 & 0.59 & 6.3 \\  \bottomrule
\end{tabular}
\caption{Pearson correlation on human evaluation on system output. `R-': reference-based. `S-': source-based.}
\label{tab:sys}
\end{table*}



\begin{table}%[]
\centering
\fontsize{11pt}{11pt}\selectfont
\begin{tabular}{llllll}
\toprule
\multicolumn{1}{c}{\textbf{task}} & \multicolumn{1}{c}{\textbf{Lai}} & \multicolumn{1}{c}{\textbf{Zei.}} & \multicolumn{1}{c}{\textbf{Scia.}} & \textbf{} & \textbf{} \\ 
\multicolumn{1}{c}{\textbf{metrics}} & \multicolumn{1}{c}{\textbf{cor.}} & \multicolumn{1}{c}{\textbf{cor.}} & \multicolumn{1}{c}{\textbf{cor.}} & \textbf{avg.} & \textbf{\begin{tabular}[c]{@{}l@{}}avg.\\ rank\end{tabular}} \\ \midrule
\textbf{S-Bleu} & 0.40 & 0.40 & 0.19* & 0.33 & 7.67 \\
\textbf{S-Meteor} & 0.41 & 0.42 & 0.16* & 0.33 & 7.33 \\
\textbf{S-BertS.} & {\ul0.58} & 0.47 & 0.31 & 0.45 & 3.67 \\
\textbf{S-Bleurt} & 0.45 & {\ul 0.54} & {\ul 0.37} & 0.45 & {\ul 3.33} \\
\textbf{S-Cosine} & 0.56 & 0.52 & 0.3 & {\ul 0.46} & {\ul 3.33} \\ \midrule
\textbf{QuestE.} & 0.27 & 0.35 & 0.06* & 0.23 & 9.00 \\
\textbf{LlaMA3} & \textbf{0.6} & \textbf{0.67} & \textbf{0.51} & \textbf{0.59} & \textbf{1.0} \\
\textbf{Our (3b)} & 0.51 & 0.49 & 0.23* & 0.39 & 4.83 \\
\textbf{Our (8b)} & 0.52 & 0.49 & 0.22* & 0.43 & 4.83 \\ \bottomrule
\end{tabular}
\caption{Pearson correlation on human ratings on reference output. *not significant; we cannot reject the null hypothesis of zero correlation}
\label{tab:ref}
\end{table}


\begin{table*}%[]
\centering
\fontsize{11pt}{11pt}\selectfont
\begin{tabular}{lllllllll}
\toprule
\textbf{task} & \multicolumn{1}{c}{\textbf{ALL}} & \multicolumn{1}{c}{\textbf{sentiment}} & \multicolumn{1}{c}{\textbf{detoxify}} & \multicolumn{1}{c}{\textbf{catchy}} & \multicolumn{1}{c}{\textbf{polite}} & \multicolumn{1}{c}{\textbf{persuasive}} & \multicolumn{1}{c}{\textbf{formal}} & \textbf{\begin{tabular}[c]{@{}l@{}}avg. \\ rank\end{tabular}} \\
\textbf{metrics} & \multicolumn{1}{c}{\textbf{cor.}} & \multicolumn{1}{c}{\textbf{cor.}} & \multicolumn{1}{c}{\textbf{cor.}} & \multicolumn{1}{c}{\textbf{cor.}} & \multicolumn{1}{c}{\textbf{cor.}} & \multicolumn{1}{c}{\textbf{cor.}} & \multicolumn{1}{c}{\textbf{cor.}} &  \\ \midrule
\textbf{S-Bleu} & -0.17 & -0.82 & -0.45 & -0.12* & -0.1* & -0.05 & -0.21 & 8.42 \\
\textbf{R-Bleu} & - & -0.5 & -0.45 &  &  &  &  &  \\
\textbf{S-Meteor} & -0.07* & -0.55 & -0.4 & -0.01* & 0.1* & -0.16 & -0.04* & 7.67 \\
\textbf{R-Meteor} & - & -0.17* & -0.39 & - & - & - & - & - \\
\textbf{S-BertScore} & 0.11 & -0.38 & -0.07* & -0.17* & 0.28 & 0.12 & 0.25 & 6.0 \\
\textbf{R-BertScore} & - & -0.02* & -0.21* & - & - & - & - & - \\
\textbf{S-Bleurt} & 0.29 & 0.05* & 0.45 & 0.06* & 0.29 & 0.23 & 0.46 & 4.2 \\
\textbf{R-Bleurt} & - &  0.21 & 0.38 & - & - & - & - & - \\
\textbf{S-Cosine} & 0.01* & -0.5 & -0.13* & -0.19* & 0.05* & -0.05* & 0.15* & 7.42 \\
\textbf{R-Cosine} & - & -0.11* & -0.16* & - & - & - & - & - \\ \midrule
\textbf{QuestEval} & 0.21 & {\ul{0.29}} & 0.23 & 0.37 & 0.19* & 0.35 & 0.14* & 4.67 \\
\textbf{LlaMA3} & \textbf{0.82} & \textbf{0.80} & \textbf{0.72} & \textbf{0.84} & \textbf{0.84} & \textbf{0.90} & \textbf{0.88} & \textbf{1.00} \\
\textbf{Our (3b)} & 0.47 & -0.11* & 0.37 & 0.61 & 0.53 & 0.54 & 0.66 & 3.5 \\
\textbf{Our (8b)} & {\ul{0.57}} & 0.09* & {\ul 0.49} & {\ul 0.72} & {\ul 0.64} & {\ul 0.62} & {\ul 0.67} & {\ul 2.17} \\ \bottomrule
\end{tabular}
\caption{Pearson correlation on human ratings on our constructed test set. 'R-': reference-based. 'S-': source-based. *not significant; we cannot reject the null hypothesis of zero correlation}
\label{tab:con}
\end{table*}

\section{Results}
We benchmark the different metrics on the different datasets using correlation to human judgement. For content preservation, we show results split on data with system output, reference output and our constructed test set: we show that the data source for evaluation leads to different conclusions on the metrics. In addition, we examine whether the metrics can rank style transfer systems similar to humans. On style strength, we likewise show correlations between human judgment and zero-shot evaluation approaches. When applicable, we summarize results by reporting the average correlation. And the average ranking of the metric per dataset (by ranking which metric obtains the highest correlation to human judgement per dataset). 

\subsection{Content preservation}
\paragraph{How do data sources affect the conclusion on best metric?}
The conclusions about the metrics' performance change radically depending on whether we use system output data, reference output, or our constructed test set. Ideally, a good metric correlates highly with humans on any data source. Ideally, for meta-evaluation, a metric should correlate consistently across all data sources, but the following shows that the correlations indicate different things, and the conclusion on the best metric should be drawn carefully.

Looking at the metrics correlations with humans on the data source with system output (Table~\ref{tab:sys}), we see a relatively high correlation for many of the metrics on many tasks. The overall best metrics are S-BertScore and S-BLEURT (avg+avg rank). We see no notable difference in our method of using the 3B or 8B model as the backbone.

Examining the average correlations based on data with reference output (Table~\ref{tab:ref}), now the zero-shoot prompting with LlaMA3 70B is the best-performing approach ($0.59$ avg). Tied for second place are source-based cosine embedding ($0.46$ avg), BLEURT ($0.45$ avg) and BertScore ($0.45$ avg). Our method follows on a 5. place: here, the 8b version (($0.43$ avg)) shows a bit stronger results than 3b ($0.39$ avg). The fact that the conclusions change, whether looking at reference or system output, confirms the observations made by \citet{scialom-etal-2021-questeval} on simplicity transfer.   

Now consider the results on our test set (Table~\ref{tab:con}): Several metrics show low or no correlation; we even see a significantly negative correlation for some metrics on ALL (BLEU) and for specific subparts of our test set for BLEU, Meteor, BertScore, Cosine. On the other end, LlaMA3 70B is again performing best, showing strong results ($0.82$ in ALL). The runner-up is now our 8B method, with a gap to the 3B version ($0.57$ vs $0.47$ in ALL). Note our method still shows zero correlation for the sentiment task. After, ranks BLEURT ($0.29$), QuestEval ($0.21$), BertScore ($0.11$), Cosine ($0.01$).  

On our test set, we find that some metrics that correlate relatively well on the other datasets, now exhibit low correlation. Hence, with our test set, we can now support the logical reasoning with data evidence: Evaluation of content preservation for style transfer needs to take the style shift into account. This conclusion could not be drawn using the existing data sources: We hypothesise that for the data with system-based output, successful output happens to be very similar to the source sentence and vice versa, and reference-based output might not contain server mistakes as they are gold references. Thus, none of the existing data sources tests the limits of the metrics.  


\paragraph{How do reference-based metrics compare to source-based ones?} Reference-based metrics show a lower correlation than the source-based counterpart for all metrics on both datasets with ratings on references (Table~\ref{tab:sys}). As discussed previously, reference-based metrics for style transfer have the drawback that many different good solutions on a rewrite might exist and not only one similar to a reference.


\paragraph{How well can the metrics rank the performance of style transfer methods?}
We compare the metrics' ability to judge the best style transfer methods w.r.t. the human annotations: Several of the data sources contain samples from different style transfer systems. In order to use metrics to assess the quality of the style transfer system, metrics should correctly find the best-performing system. Hence, we evaluate whether the metrics for content preservation provide the same system ranking as human evaluators. We take the mean of the score for every output on each system and the mean of the human annotations; we compare the systems using the Kendall's Tau correlation. 

We find only the evaluation using the dataset Mir, Lai, and Ziegen to result in significant correlations, probably because of sparsity in a number of system tests (App.~\ref{app:dataset}). Our method (8b) is the only metric providing a perfect ranking of the style transfer system on the Lai data, and Llama3 70B the only one on the Ziegen data. Results in App.~\ref{app:results}. 


\subsection{Style strength results}
%Evaluating style strengths is a challenging task. 
Llama3 70B shows better overall results than our method. However, our method scores higher than Llama3 70B on 2 out of 6 datasets, but it also exhibits zero correlation on one task (Table~\ref{tab:styleresults}).%More work i s needed on evaluating style strengths. 
 
\begin{table}%[]
\fontsize{11pt}{11pt}\selectfont
\begin{tabular}{lccc}
\toprule
\multicolumn{1}{c}{\textbf{}} & \textbf{LlaMA3} & \textbf{Our (3b)} & \textbf{Our (8b)} \\ \midrule
\textbf{Mir} & 0.46 & 0.54 & \textbf{0.57} \\
\textbf{Lai} & \textbf{0.57} & 0.18 & 0.19 \\
\textbf{Ziegen.} & 0.25 & 0.27 & \textbf{0.32} \\
\textbf{Alva-M.} & \textbf{0.59} & 0.03* & 0.02* \\
\textbf{Scialom} & \textbf{0.62} & 0.45 & 0.44 \\
\textbf{\begin{tabular}[c]{@{}l@{}}Our Test\end{tabular}} & \textbf{0.63} & 0.46 & 0.48 \\ \bottomrule
\end{tabular}
\caption{Style strength: Pearson correlation to human ratings. *not significant; we cannot reject the null hypothesis of zero corelation}
\label{tab:styleresults}
\end{table}

\subsection{Ablation}
We conduct several runs of the methods using LLMs with variations in instructions/prompts (App.~\ref{app:method}). We observe that the lower the correlation on a task, the higher the variation between the different runs. For our method, we only observe low variance between the runs.
None of the variations leads to different conclusions of the meta-evaluation. Results in App.~\ref{app:results}.

\section{Case study: Z3 \CodeIn{bdd\_manager} class}
\label{sec:z3}
\begin{table}[!htp]\centering
\caption{Statistics of the studied data structures in Z3}\label{tab:z3-data}
\small
\begin{tabular}{l|rrrrrrrr}\toprule
&ema &dlist &heap &hashtable &permutation &scoped\_vector &bdd\_manager \\\midrule
\# LoC &57 &243 &309 &761 &177 &220 &1635 \\
\# dependencies &0 &2 &1 &9 &2 &2 &13 \\
\bottomrule
\end{tabular}
\end{table}

As a real-world case study, we apply \tech to synthesize invariants for 7 core data structures from Z3~\cite{z3}, ranging from the simple 57-line \CodeIn{ema} class to the complex 1635-line \CodeIn{bdd\_manager}. The complete set includes \CodeIn{dlist}, \CodeIn{heap}, \CodeIn{hashtable}, \CodeIn{permutation}, and \CodeIn{scoped\_vector}, with varying implementation complexity and the number of dependent classes as shown in Table~\ref{tab:z3-data}. 
Our results were validated by one of the Z3 authors, who confirmed at least one \emph{correct and useful} invariant for each studied class, with the \CodeIn{bdd\_manager} class yielding 11 valuable invariants including the 2 already present from 8 generations.

Z3 is a widely adopted SMT solver used in a variety of high-stakes applications requiring rigorous correctness, such as formal verification, program analysis, and automated reasoning. It is integrated into tools like LLVM~\cite{llvm}, KLEE~\cite{klee}, Dafny~\cite{leino2010dafny} and Frama-C~\cite{kirchner2015frama}.
We selected the Z3 codebase due to its stringent correctness requirements; as an SMT solver, Z3 is employed in applications demanding high reliability. This high-stakes environment makes Z3 an ideal testbed for assessing the effectiveness of synthesized invariants.


The \CodeIn{bdd\_manager} class\footnote{\url{https://github.com/Z3Prover/z3/blob/master/src/math/dd/dd_bdd.h}} is particularly noteworthy. 
It was chosen because it is a self-contained example with developer-written unit tests for validation, presenting a realistic yet manageable challenge. 
Note that the existing developer tests were used after invariants were generated, not as input to the LLM. 
The \CodeIn{bdd\_manager} class in Z3 is a utility for managing Binary Decision Diagrams (BDDs), which are data structures used to represent Boolean functions efficiently. In BDDs, Boolean functions are represented as directed acyclic graphs, where each non-terminal node corresponds to a Boolean variable, and edges represent the truth values of these variables (\textit{true} or \textit{false}). This representation simplifies complex Boolean expressions and enables efficient operations on Boolean functions.


With 382 lines of code in its header and 1253 lines in the implementation file, \CodeIn{bdd\_manager} surpasses standard data structure complexity, offering an opportunity to evaluate \tech's capability to generate meaningful invariants relevant to real-world scenarios. 
\tech achieves this by compositional generation, recursively traversing the source program's AST (Section~\ref{sec:llm}). 
Recursive generation became crucial when handling large classes like \CodeIn{bdd\_manager}, which exceeded the LLM’s context window. Decomposing and processing its components separately allowed us to fit relevant parts into the model’s input, demonstrating the utility of recursive invariant generation for large codebases. This supports its relevance in real-world applications beyond the benchmarks.

The  \CodeIn{bdd\_manager} class includes a developer-written member function for checking its well-formedness, as shown in Figure~\ref{fig:bdd_well_formed}, which we removed during \tech generation. Of the 56 invariants generated by \tech, one of Z3 main authors identified 11 distinct \textit{correct and useful} invariants (e.g., Figure~\ref{fig:bdd_correct_useful}) including the 2 developer-written invariants; these invariants could potentially be integrated into the codebase.
An additional 5 distinct \textit{ok} invariants (e.g., Figure~\ref{fig:bdd_ok})  are labeled correct but have limited utility, 16 distinct \textit{correct but useless} invariants (e.g., those already checked during compilation, such as type checks and constants, Figure~\ref{fig:bdd_correct_useless}), and 2 \textit{incorrect} invariants (e.g., Figure~\ref{fig:bdd_incorrect}). The remaining invariants were repetitions within these categories. This evaluation aligns with \tech's validation results, as our validation pipeline also identified 2 incorrect invariants that failed \CodeIn{bdd\_manager} unit tests.
\shuvendu{Why did you choose to report invariants that already failed the unit test?}

Overall, the Z3 authors' evaluation results further confirm \tech's potential utility in real-world, large-scale codebases.

\begin{figure}[htp]
    \centering
\begin{lstlisting}[language=c++]
bool bdd_manager::well_formed() {
    bool ok = true;
    for (unsigned n : m_free_nodes) {
        ok &= (lo(n) == 0 && hi(n) == 0 && m_nodes[n].m_refcount == 0);
        if (!ok) {
            IF_VERBOSE(0, verbose_stream() << "free node is not internal " << n << " " << lo(n) << " " << hi(n) << " " << m_nodes[n].m_refcount << "\n";
            display(verbose_stream()););
            UNREACHABLE();
            return false;
        }
    }
    
    for (bdd_node const& n : m_nodes) {
        if (n.is_internal()) continue;
        unsigned lvl = n.m_level;
        BDD lo = n.m_lo;
        BDD hi = n.m_hi;
        ok &= is_const(lo) || level(lo) < lvl;
        ok &= is_const(hi) || level(hi) < lvl;
        ok &= is_const(lo) || !m_nodes[lo].is_internal();
        ok &= is_const(hi) || !m_nodes[hi].is_internal();
        if (!ok) {
            IF_VERBOSE(0, display(verbose_stream() << n.m_index << " lo " << lo << " hi " << hi << "\n"););
            UNREACHABLE();
            return false;
        }
    }
    return ok;
}
\end{lstlisting}
    \caption{Z3 developer-written class invariants for \CodeIn{bdd\_manager} class}
    \label{fig:bdd_well_formed}
\end{figure}

\begin{figure}[htp]
    \centering
\begin{lstlisting}[language=c++]
// Node consistency: Each node's index should match its position in m_nodes
for (unsigned i = 0; i < m_nodes.size(); ++i) {
    assert(m_nodes[i].m_index == i);
}
\end{lstlisting}
    \caption{\textit{Correct and useful} invariant for \CodeIn{bdd\_manager} class}
    \label{fig:bdd_correct_useful}
\end{figure}

\begin{figure}[htp]
    \centering
\begin{lstlisting}[language=c++]
// Cache consistency: Entries in the operation cache should be valid
for (const auto* e : m_op_cache) {
    assert(e != nullptr);
    assert(e->m_result != null_bdd);
}
\end{lstlisting}
    \caption{\textit{Ok} invariant for \CodeIn{bdd\_manager} class}
    \label{fig:bdd_ok}
\end{figure}

\begin{figure}[htp]
    \centering
\begin{lstlisting}[language=c++]
// m_is_new_node is a boolean
assert(m_is_new_node == true || m_is_new_node == false);
\end{lstlisting}
    \caption{\textit{Correct and useless} invariant for \CodeIn{bdd\_manager} class}
    \label{fig:bdd_correct_useless}
\end{figure}


\begin{figure}[htp]
    \centering
\begin{lstlisting}[language=c++]
// The number of nodes should not exceed the maximum number of BDD nodes
assert(m_nodes.size() <= m_max_num_bdd_nodes);
\end{lstlisting}
    \caption{\textit{Incorrect} invariant for \CodeIn{bdd\_manager} class}
    \label{fig:bdd_incorrect}
\end{figure}



\section{Related Work}

%\section{Related Work}
%\label{sec:related-work}

%\subsection{Background}

%Defect detection is critical to ensure the yield of integrated circuit manufacturing lines and reduce faults. Previous research has primarily focused on wafer map data, which engineers produce by marking faulty chips with different colors based on test results. The specific spatial distribution of defects on a wafer can provide insights into the causes, thereby helping to determine which stage of the manufacturing process is responsible for the issues. Although such research is relatively mature, the continual miniaturization of integrated circuits and the increasing complexity and density of chip components have made chip-level detection more challenging, leading to potential risks\cite{ma2023review}. Consequently, there is a need to combine this approach with magnified imaging of the wafer surface using scanning electron microscopes (SEMs) to detect, classify, and analyze specific microscopic defects, thus helping to identify the particular process steps where defects originate.

%Previously, wafer surface defect classification and detection were primarily conducted by experienced engineers. However, this method relies heavily on the engineers' expertise and involves significant time expenditure and subjectivity, lacking uniform standards. With the ongoing development of artificial intelligence, deep learning methods using multi-layer neural networks to extract and learn target features have proven highly effective for this task\cite{gao2022review}.

%In the task of defect classification, it is typical to use a model structure that initially extracts features through convolutional and pooling layers, followed by classification via fully connected layers. Researchers have recently developed numerous classification model structures tailored to specific problems. These models primarily focus on how to extract defect features effectively. For instance, Chen et al. presented a defect recognition and classification algorithm rooted in PCA and classification SVM\cite{chen2008defect}. Chang et al. utilized SVM, drawing on features like smoothness and texture intricacy, for classifying high-intensity defect images\cite{chang2013hybrid}. The classification of defect images requires the formulation of numerous classifiers tailored for myriad inspection steps and an Abundance of accurately labeled data, making data acquisition challenging. Cheon et al. proposed a single CNN model adept at feature extraction\cite{cheon2019convolutional}. They achieved a granular classification of wafer surface defects by recognizing misclassified images and employing a k-nearest neighbors (k-NN) classifier algorithm to gauge the aggregate squared distance between each image feature vector and its k-neighbors within the same category. However, when applied to new or unseen defects, such models necessitate retraining, incurring computational overheads. Moreover, with escalating CNN complexity, the computational demands surge.

%Segmentation of defects is necessary to locate defect positions and gather information such as the size of defects. Unlike classification networks, segmentation networks often use classic encoder-decoder structures such as UNet\cite{ronneberger2015u} and SegNet\cite{badrinarayanan2017segnet}, which focus on effectively leveraging both local and global feature information. Han Hui et al. proposed integrating a Region Proposal Network (RPN) with a UNet architecture to suggest defect areas before conducting defect segmentation \cite{han2020polycrystalline}. This approach enables the segmentation of various defects in wafers with only a limited set of roughly labeled images, enhancing the efficiency of training and application in environments where detailed annotations are scarce. Subhrajit Nag et al. introduced a new network structure, WaferSegClassNet, which extracts multi-scale local features in the encoder and performs classification and segmentation tasks in the decoder \cite{nag2022wafersegclassnet}. This model represents the first detection system capable of simultaneously classifying and segmenting surface defects on wafers. However, it relies on extensive data training and annotation for high accuracy and reliability. 

%Recently, Vic De Ridder et al. introduced a novel approach for defect segmentation using diffusion models\cite{de2023semi}. This approach treats the instance segmentation task as a denoising process from noise to a filter, utilizing diffusion models to predict and reconstruct instance masks for semiconductor defects. This method achieves high precision and improved defect classification and segmentation detection performance. However, the complex network structure and the computational process of the diffusion model require substantial computational resources. Moreover, the performance of this model heavily relies on high-quality and large amounts of training data. These issues make it less suitable for industrial applications. Additionally, the model has only been applied to detecting and segmenting a single type of defect(bridges) following a specific manufacturing process step, limiting its practical utility in diverse industrial scenarios.

%\subsection{Few-shot Anomaly Detection}
%Traditional anomaly detection techniques typically rely on extensive training data to train models for identifying and locating anomalies. However, these methods often face limitations in rapidly changing production environments and diverse anomaly types. Recent research has started exploring effective anomaly detection using few or zero samples to address these challenges.

%Huang et al. developed the anomaly detection method RegAD, based on image registration technology. This method pre-trains an object-agnostic registration network with various images to establish the normality of unseen objects. It identifies anomalies by aligning image features and has achieved promising results. Despite these advancements, implementing few-shot settings in anomaly detection remains an area ripe for further exploration. Recent studies show that pre-trained vision-language models such as CLIP and MiniGPT can significantly enhance performance in anomaly detection tasks.

%Dong et al. introduced the MaskCLIP framework, which employs masked self-distillation to enhance contrastive language-image pretraining\cite{zhou2022maskclip}. This approach strengthens the visual encoder's learning of local image patches and uses indirect language supervision to enhance semantic understanding. It significantly improves transferability and pretraining outcomes across various visual tasks, although it requires substantial computational resources.
%Jeong et al. crafted the WinCLIP framework by integrating state words and prompt templates to characterize normal and anomalous states more accurately\cite{Jeong_2023_CVPR}. This framework introduces a novel window-based technique for extracting and aggregating multi-scale spatial features, significantly boosting the anomaly detection performance of the pre-trained CLIP model.
%Subsequently, Li et al. have further contributed to the field by creating a new expansive multimodal model named Myriad\cite{li2023myriad}. This model, which incorporates a pre-trained Industrial Anomaly Detection (IAD) model to act as a vision expert, embeds anomaly images as tokens interpretable by the language model, thus providing both detailed descriptions and accurate anomaly detection capabilities.
%Recently, Chen et al. introduced CLIP-AD\cite{chen2023clip}, and Li et al. proposed PromptAD\cite{li2024promptad}, both employing language-guided, tiered dual-path model structures and feature manipulation strategies. These approaches effectively address issues encountered when directly calculating anomaly maps using the CLIP model, such as reversed predictions and highlighting irrelevant areas. Specifically, CLIP-AD optimizes the utilization of multi-layer features, corrects feature misalignment, and enhances model performance through additional linear layer fine-tuning. PromptAD connects normal prompts with anomaly suffixes to form anomaly prompts, enabling contrastive learning in a single-class setting.

%These studies extend the boundaries of traditional anomaly detection techniques and demonstrate how to effectively address rapidly changing and sample-scarce production environments through the synergy of few-shot learning and deep learning models. Building on this foundation, our research further explores wafer surface defect detection based on the CLIP model, especially focusing on achieving efficient and accurate anomaly detection in the highly specialized and variable semiconductor manufacturing process using a minimal amount of labeled data.


\section{Limitation}
\section{Discussion of Assumptions}\label{sec:discussion}
In this paper, we have made several assumptions for the sake of clarity and simplicity. In this section, we discuss the rationale behind these assumptions, the extent to which these assumptions hold in practice, and the consequences for our protocol when these assumptions hold.

\subsection{Assumptions on the Demand}

There are two simplifying assumptions we make about the demand. First, we assume the demand at any time is relatively small compared to the channel capacities. Second, we take the demand to be constant over time. We elaborate upon both these points below.

\paragraph{Small demands} The assumption that demands are small relative to channel capacities is made precise in \eqref{eq:large_capacity_assumption}. This assumption simplifies two major aspects of our protocol. First, it largely removes congestion from consideration. In \eqref{eq:primal_problem}, there is no constraint ensuring that total flow in both directions stays below capacity--this is always met. Consequently, there is no Lagrange multiplier for congestion and no congestion pricing; only imbalance penalties apply. In contrast, protocols in \cite{sivaraman2020high, varma2021throughput, wang2024fence} include congestion fees due to explicit congestion constraints. Second, the bound \eqref{eq:large_capacity_assumption} ensures that as long as channels remain balanced, the network can always meet demand, no matter how the demand is routed. Since channels can rebalance when necessary, they never drop transactions. This allows prices and flows to adjust as per the equations in \eqref{eq:algorithm}, which makes it easier to prove the protocol's convergence guarantees. This also preserves the key property that a channel's price remains proportional to net money flow through it.

In practice, payment channel networks are used most often for micro-payments, for which on-chain transactions are prohibitively expensive; large transactions typically take place directly on the blockchain. For example, according to \cite{river2023lightning}, the average channel capacity is roughly $0.1$ BTC ($5,000$ BTC distributed over $50,000$ channels), while the average transaction amount is less than $0.0004$ BTC ($44.7k$ satoshis). Thus, the small demand assumption is not too unrealistic. Additionally, the occasional large transaction can be treated as a sequence of smaller transactions by breaking it into packets and executing each packet serially (as done by \cite{sivaraman2020high}).
Lastly, a good path discovery process that favors large capacity channels over small capacity ones can help ensure that the bound in \eqref{eq:large_capacity_assumption} holds.

\paragraph{Constant demands} 
In this work, we assume that any transacting pair of nodes have a steady transaction demand between them (see Section \ref{sec:transaction_requests}). Making this assumption is necessary to obtain the kind of guarantees that we have presented in this paper. Unless the demand is steady, it is unreasonable to expect that the flows converge to a steady value. Weaker assumptions on the demand lead to weaker guarantees. For example, with the more general setting of stochastic, but i.i.d. demand between any two nodes, \cite{varma2021throughput} shows that the channel queue lengths are bounded in expectation. If the demand can be arbitrary, then it is very hard to get any meaningful performance guarantees; \cite{wang2024fence} shows that even for a single bidirectional channel, the competitive ratio is infinite. Indeed, because a PCN is a decentralized system and decisions must be made based on local information alone, it is difficult for the network to find the optimal detailed balance flow at every time step with a time-varying demand.  With a steady demand, the network can discover the optimal flows in a reasonably short time, as our work shows.

We view the constant demand assumption as an approximation for a more general demand process that could be piece-wise constant, stochastic, or both (see simulations in Figure \ref{fig:five_nodes_variable_demand}).
We believe it should be possible to merge ideas from our work and \cite{varma2021throughput} to provide guarantees in a setting with random demands with arbitrary means. We leave this for future work. In addition, our work suggests that a reasonable method of handling stochastic demands is to queue the transaction requests \textit{at the source node} itself. This queuing action should be viewed in conjunction with flow-control. Indeed, a temporarily high unidirectional demand would raise prices for the sender, incentivizing the sender to stop sending the transactions. If the sender queues the transactions, they can send them later when prices drop. This form of queuing does not require any overhaul of the basic PCN infrastructure and is therefore simpler to implement than per-channel queues as suggested by \cite{sivaraman2020high} and \cite{varma2021throughput}.

\subsection{The Incentive of Channels}
The actions of the channels as prescribed by the DEBT control protocol can be summarized as follows. Channels adjust their prices in proportion to the net flow through them. They rebalance themselves whenever necessary and execute any transaction request that has been made of them. We discuss both these aspects below.

\paragraph{On Prices}
In this work, the exclusive role of channel prices is to ensure that the flows through each channel remains balanced. In practice, it would be important to include other components in a channel's price/fee as well: a congestion price  and an incentive price. The congestion price, as suggested by \cite{varma2021throughput}, would depend on the total flow of transactions through the channel, and would incentivize nodes to balance the load over different paths. The incentive price, which is commonly used in practice \cite{river2023lightning}, is necessary to provide channels with an incentive to serve as an intermediary for different channels. In practice, we expect both these components to be smaller than the imbalance price. Consequently, we expect the behavior of our protocol to be similar to our theoretical results even with these additional prices.

A key aspect of our protocol is that channel fees are allowed to be negative. Although the original Lightning network whitepaper \cite{poon2016bitcoin} suggests that negative channel prices may be a good solution to promote rebalancing, the idea of negative prices in not very popular in the literature. To our knowledge, the only prior work with this feature is \cite{varma2021throughput}. Indeed, in papers such as \cite{van2021merchant} and \cite{wang2024fence}, the price function is explicitly modified such that the channel price is never negative. The results of our paper show the benefits of negative prices. For one, in steady state, equal flows in both directions ensure that a channel doesn't loose any money (the other price components mentioned above ensure that the channel will only gain money). More importantly, negative prices are important to ensure that the protocol selectively stifles acyclic flows while allowing circulations to flow. Indeed, in the example of Section \ref{sec:flow_control_example}, the flows between nodes $A$ and $C$ are left on only because the large positive price over one channel is canceled by the corresponding negative price over the other channel, leading to a net zero price.

Lastly, observe that in the DEBT control protocol, the price charged by a channel does not depend on its capacity. This is a natural consequence of the price being the Lagrange multiplier for the net-zero flow constraint, which also does not depend on the channel capacity. In contrast, in many other works, the imbalance price is normalized by the channel capacity \cite{ren2018optimal, lin2020funds, wang2024fence}; this is shown to work well in practice. The rationale for such a price structure is explained well in \cite{wang2024fence}, where this fee is derived with the aim of always maintaining some balance (liquidity) at each end of every channel. This is a reasonable aim if a channel is to never rebalance itself; the experiments of the aforementioned papers are conducted in such a regime. In this work, however, we allow the channels to rebalance themselves a few times in order to settle on a detailed balance flow. This is because our focus is on the long-term steady state performance of the protocol. This difference in perspective also shows up in how the price depends on the channel imbalance. \cite{lin2020funds} and \cite{wang2024fence} advocate for strictly convex prices whereas this work and \cite{varma2021throughput} propose linear prices.

\paragraph{On Rebalancing} 
Recall that the DEBT control protocol ensures that the flows in the network converge to a detailed balance flow, which can be sustained perpetually without any rebalancing. However, during the transient phase (before convergence), channels may have to perform on-chain rebalancing a few times. Since rebalancing is an expensive operation, it is worthwhile discussing methods by which channels can reduce the extent of rebalancing. One option for the channels to reduce the extent of rebalancing is to increase their capacity; however, this comes at the cost of locking in more capital. Each channel can decide for itself the optimum amount of capital to lock in. Another option, which we discuss in Section \ref{sec:five_node}, is for channels to increase the rate $\gamma$ at which they adjust prices. 

Ultimately, whether or not it is beneficial for a channel to rebalance depends on the time-horizon under consideration. Our protocol is based on the assumption that the demand remains steady for a long period of time. If this is indeed the case, it would be worthwhile for a channel to rebalance itself as it can make up this cost through the incentive fees gained from the flow of transactions through it in steady state. If a channel chooses not to rebalance itself, however, there is a risk of being trapped in a deadlock, which is suboptimal for not only the nodes but also the channel.

\section{Conclusion}
This work presents DEBT control: a protocol for payment channel networks that uses source routing and flow control based on channel prices. The protocol is derived by posing a network utility maximization problem and analyzing its dual minimization. It is shown that under steady demands, the protocol guides the network to an optimal, sustainable point. Simulations show its robustness to demand variations. The work demonstrates that simple protocols with strong theoretical guarantees are possible for PCNs and we hope it inspires further theoretical research in this direction.

\section{Conclusion}
\label{sec:concl}

\section{CONCLUSION}
\label{sec:concl}
The rapid rise of AI-generated media challenges information authenticity and societal trust, necessitating robust detection mechanisms. This survey examines the evolution of AI-generated media detection, focusing on the shift from Non-MLLM-based domain-specific detectors to MLLM-based general-purpose approaches. We compare these methods across authenticity, explainability, and localization tasks from both single-modal and multi-modal perspectives. Additionally, we review datasets, methodologies, and evaluation metrics, identifying key limitations and research challenges.
Beyond technical concerns, MLLM-based detection raises ethical and security issues. As GenAI sees broader deployment, regulatory frameworks vary significantly across jurisdictions, complicating governance. By summarizing these regulations, we provide insights for researchers navigating legal and ethical challenges.
While many challenges remain, We hope this survey sparks further discussion, informs future research, and contributes to a more secure and trustworthy AI ecosystem.

\section*{Acknowledgments}
We would like to thank Nikolaj Bjørner for his encouragement and help in inspecting the several generated invariants for the Z3 source code and provide feedback about their utility. 


\bibliographystyle{ACM-Reference-Format}
\bibliography{custom}

% \appendix
\appendix

\section{Prompt}
\label{appendix:prompt}
\begin{figure*}[htp]
\begin{lstlisting}[language=markdown]
You are an expert in creating program invariants from code and natural language.
Invariants are assertions on the variables in scope that hold true at different program points
We are interested in finding invariants that hold at both start and end of a function within a data structure. Such an invariant is commonly known as an object invariant.  

The invariants can usually be expressed as a check on the state at the particular program point. The check should be expressed as a check in the same underlying programming language which evaluates to true or false. To express these, you can use:
- An assertion in the programming language
- A pure method (which does not have any side effect on the variables in scope) that checks one or more assertion
- For a collection, you can use a loop to iterate over elements of the collection and assert something on each element or a pair of elements.  

Task Description: 
Task 1: Given a module, in the form of a class definition, your task is to infer object invariants about the class. For doing so, you may examine how the methods of the class read and modify the various fields of the class. 
For coming up with invariants, you may use the provided code and any comments in the code. You may also use world knowledge to guide the search for invariants. 
 
Task 2: Generate unit tests for the class based on the class definition and public API methods. The test cases should simulate a series of public method calls to verify the behavior of the class, but do not use any testing framework like gtest. Do not add `assert` or any form of assertions.
\end{lstlisting}
    \caption{\tech Generation system prompt: instruction and task description.}
    \label{fig:prompt_generation_system_task_app}
\end{figure*}

\begin{figure}
    \centering
    \begin{lstlisting}[language=markdown,firstnumber=15]
Input Format:
You will be given the name of a class or typedef, and a section of code containing the definition of the class. You will also be given the definitions of functions that read and modify the fields of the class. 

Output Format:
The output should be in the following format:

The first paragraph should begin with "REASONING:". From the next line onwards, it should contain the detailed reasoning and analysis used for the inference of the object invariants. The entire text should be enclosed in $$$. For example,
``$$$
REASONING: 
explanation
$$$

The next paragraph should begin with "INVARIANTS:". From the next line onwards, it should contain a list of the various invariants inferred. The invariants should be in the form of code in the same underlying programming language, enclosed by ```. Each invariant should start from a new line, and be separated by "---". Use lambda if necessary. If lambda is recursive, explicitly specify the type of the lambda function and use `std::function` for recursion. Do not use helper functions. For example, 
INVARIANTS: 
```/* Invariant 1 */```
---
```/* Multi line Invariant 2  */
    assert(...); ```
---
```/* Invariant 3 */```

The next paragraph should begin with "TESTS:". From the next line onwards, it should contain a list of a API call sequence in the form of code enclosed by ```. Each test should start from a new line, and be separated by "---". For example, 
TESTS: 
```/* Test 1 */
   this->method1();
   this->method2();```
---
```/* Test 2 */
    this.method3(...); ```
---
```/* Test 3 */```

Important:
1. Follow the output format strictly, particularly enclosing each invariant in triple-ticks (```), and enclosing the reasoning in $$$.
2. Only find object invariants for the target class provided to you, do not infer invariants for any other class. 
3. Make sure the invariant is a statement in the same underlying programming language as the source program.
4. If you can decompose a single invariant into smaller ones, try to output multiple invariants.
\end{lstlisting}
\vspace{-0.1in}
    \caption{\tech Generation system prompt: input-output format.}
    \label{fig:prompt_generation_system_inputoutput}
\end{figure}

Figure~\ref{fig:prompt_generation_system_task_app} and Figure~\ref{fig:prompt_generation_system_inputoutput} show the system prompt used by \tech for invariant-test co-generation. The former presents the instruction and task description, while the latter illustrates the input-output format.

\begin{figure*}[htp]
\begin{lstlisting}[language=markdown]
Name of Data Structure to Annotate: {struct}
Code:
```
{code}
```
\end{lstlisting}
\vspace{-0.1in}
    \caption{\tech Generation user prompt template.}
    \label{appendix:prompt_generation_user}
\end{figure*}

\begin{figure*}[htp]
\begin{lstlisting}[language=markdown]
You are an expert in repairing program invariants from code and natural language.
Invariants are assertions on the variables in scope that hold true at different program points. 

We are interested in finding invariants that hold at both start and end of a function within a data structure. Such an invariant is commonly known as an object invariant.  

The invariants can usually be expressed as a check on the state at the particular program point. The check should be expressed as a check in the same underlying programming language which evaluates to true or false. To express these, you can use:
- An assertion in the programming language
- A pure method (which does not have any side effect on the variables in scope) that checks one or more assertion
- For a collection, you can use a loop to iterate over elements of the collection and assert something on each element or a pair of elements.  

Task Description: 
Given a module, in the form of a class definition, your task is to infer object invariants about the class. For doing so, you may examine how the methods of the class read and modify the various fields of the class. 

For coming up with invariants, you may use the provided code and any comments in the code. You may also use world knowledge to guide the search for invariants. 

Input Format:
You will be given the name of a class or typedef, and a section of code containing the definition of the class. You will also be given the definitions of functions which read and modify the fields of the class. 

Output Format:
The output should be in the following format:

The first paragraph should begin with "REASONING:". From the next line onwards, it should contain the detailed reasoning and analysis used for the inference of the object invariants. The entire text should be enclosed in $$$. For example,
``$$$
REASONING: 
explanation
$$$``

The next paragraph should begin with "INVARIANTS:". From the next line onwards, it should contain a list of the various invariants inferred. The invariants should be in the form of code in the same underlying programming language, enclosed by ```. Each invariant should start from a new line, and be separated by "---". For example, 
INVARIANTS: 
```/* Invariant 1 */```
---
```/* Multi line Invariant 2  */
    assert(...);```
---
```/* Invariant 3 */```

Important:
1. Follow the output format strictly, particularly enclosing each invariant in triple-ticks (```), and enclosing the reasoning in $$$.
2. Only find object invariants for the target class provided to you, do not infer invariants for any other class. 
3. Make sure the invariant is a statement in the same underlying programming language as the source program.
4. If you can decompose a single invariant into smaller ones, try to output multiple invariants.
\end{lstlisting}
    \caption{\tech Refinement system prompt: instruction and task description.}
    \label{appendix:prompt_refinement_system}
\end{figure*}



\begin{figure*}[htp]
\begin{lstlisting}[language=markdown]
Please fix the failed invariants given the feedback, tests and the original source code.

Failed Invariant:
```
{invariant}
```

Error message for the failed invariant:
```
{feedback}
```

Name of Data Structure to Annotate: {struct}

Original Code:
```
{code}
```

Gold Tests that Fail the Invariant:
{tests}

\end{lstlisting}
\vspace{-0.1in}
    \caption{\tech Refinement user prompt template.}
    \label{appendix:prompt_refinement_user}
\end{figure*}
%\section{Daikon Invariants Frequency Tables}
\begin{table}[ht]
\centering
\scriptsize
\caption{Invariants for avl\_tree, 11 public methods.}
\label{avl_tree_daikon}
\begin{tabular}{|l|c|}
\hline
Invariant & Count \\
\hline
this->root has only one value & 4 \\
this->root.\_M\_t has only one value & 4 \\
this->root.\_M\_t.\_\_uniq\_ptr\_impl<AvlTree::Node, std::default\_delete<AvlTree::Node> >.\_M\_t has only one value & 4 \\
this[0] has only one value & 3 \\
this->n one of \{ 3, 4 \} & 3 \\
this[0] != null & 2 \\
this->root != null & 2 \\
this->root.\_M\_t != null & 2 \\
this->root.\_M\_t.\_\_uniq\_ptr\_impl<AvlTree::Node, std::default\_delete<AvlTree::Node> >.\_M\_t != null & 2 \\
this->n one of \{ 1, 2, 3 \} & 1 \\
this->n one of \{ 0, 3 \} & 1 \\
this->n >= 1 & 1 \\
this->n == 3 & 1 \\
t.root has only one value & 1 \\
t.root.\_M\_t has only one value & 1 \\
t.root.\_M\_t.\_\_uniq\_ptr\_impl<AvlTree::Node, std::default\_delete<AvlTree::Node> >.\_M\_t has only one value & 1 \\
t.n == 3 & 1 \\
this->n == return & 1 \\
\hline
\end{tabular}
\end{table}

\begin{table}[ht]
\centering
\scriptsize
\caption{Invariants for red\_black\_tree, 11 public methods.}
\label{red_black_tree_daikon}
\begin{tabular}{|l|c|}
\hline
Invariant & Count \\
\hline
this[0] != null & 3 \\
this->root != null & 3 \\
this->root.\_M\_t != null & 3 \\
this->root.\_M\_t.\_\_uniq\_ptr\_impl<RedBlackTree::Node, std::default\_delete<RedBlackTree::Node> >.\_M\_t != null & 3 \\
this->n one of \{ 3, 4, 6 \} & 3 \\
this->root has only one value & 3 \\
this->root.\_M\_t has only one value & 3 \\
this->root.\_M\_t.\_\_uniq\_ptr\_impl<RedBlackTree::Node, std::default\_delete<RedBlackTree::Node> >.\_M\_t has only one value & 3 \\
this[0] has only one value & 2 \\
this->n >= 0 & 1 \\
(No intersection exists) & 1 \\
t.root has only one value & 1 \\
t.root.\_M\_t has only one value & 1 \\
t.root.\_M\_t.\_\_uniq\_ptr\_impl<RedBlackTree::Node, std::default\_delete<RedBlackTree::Node> >.\_M\_t has only one value & 1 \\
t.n == 3 & 1 \\
No intersection & 1 \\
\hline
\end{tabular}
\end{table}

\begin{table}[ht]
\centering
\scriptsize
\caption{Invariants for linked\_list, 8 public methods.}
\label{linked_list_daikon}
\begin{tabular}{|l|c|}
\hline
Invariant & Count \\
\hline
this[0] has only one value & 5 \\
this->head has only one value & 5 \\
this->head.\_M\_t has only one value & 5 \\
this->head.\_M\_t.\_\_uniq\_ptr\_impl<LinkedList::Node, std::default\_delete<LinkedList::Node> >.\_M\_t has only one value & 5 \\
this[0] != null & 4 \\
this->head != null & 4 \\
this->head.\_M\_t != null & 4 \\
this->head.\_M\_t.\_\_uniq\_ptr\_impl<LinkedList::Node, std::default\_delete<LinkedList::Node> >.\_M\_t != null & 4 \\
this->tail[] elements != null & 2 \\
this->tail[].next elements != null & 2 \\
this->tail[].next.\_M\_t elements != null & 2 \\
this->n >= 0 & 2 \\
this->n == 0 & 1 \\
this->tail[].data elements one of \{ 1 \} & 1 \\
this->tail[].data one of \{ [1] \} & 1 \\
this->n one of \{ 1 \} & 1 \\
this->n one of \{ 1, 2 \} & 1 \\
this->tail[].data elements >= 1 & 1 \\
this->tail[].data elements <= this->n & 1 \\
this->tail[].data elements one of \{ 1, 2, 4 \} & 1 \\
this->tail[].data one of \{ [1], [2], [4] \} & 1 \\
this->n one of \{ 2, 3 \} & 1 \\
this->tail[].data == [3] & 1 \\
this->tail[].data elements == 3 & 1 \\
\hline
\end{tabular}
\end{table}

\begin{table}[ht]
\centering
\scriptsize
\caption{Invariants for binary\_search\_tree, 11 public methods.}
\label{binary_search_tree_daikon}
\begin{tabular}{|l|c|}
\hline
Invariant & Count \\
\hline
this->root has only one value & 4 \\
this->root.\_M\_t has only one value & 4 \\
this->root.\_M\_t.\_\_uniq\_ptr\_impl<BinarySearchTree::Node, std::default\_delete<BinarySearchTree::Node> >.\_M\_t has only one value & 4 \\
this->n one of \{ 0, 2, 3 \} & 3 \\
this->n one of \{ 0, 3 \} & 3 \\
this[0] has only one value & 3 \\
this[0] != null & 2 \\
this->root != null & 2 \\
this->root.\_M\_t != null & 2 \\
this->root.\_M\_t.\_\_uniq\_ptr\_impl<BinarySearchTree::Node, std::default\_delete<BinarySearchTree::Node> >.\_M\_t != null & 2 \\
t.root has only one value & 1 \\
t.root.\_M\_t has only one value & 1 \\
t.root.\_M\_t.\_\_uniq\_ptr\_impl<BinarySearchTree::Node, std::default\_delete<BinarySearchTree::Node> >.\_M\_t has only one value & 1 \\
t.n == 3 & 1 \\
(this->n == return) == (return == orig(this->n)) & 1 \\
\hline
\end{tabular}
\end{table}

\begin{table}[ht]
\centering
\scriptsize
\caption{Invariants for heap, 7 public methods.}
\label{heap_daikon}
\begin{tabular}{|l|c|}
\hline
Invariant & Count \\
\hline
this->comp.\_M\_invoker has only one value & 7 \\
this->comp.\_Function\_base.\_M\_manager has only one value & 7 \\
this[0] has only one value & 6 \\
this->data has only one value & 6 \\
this->data.\_Vector\_base<int, std::allocator<int> >.\_M\_impl has only one value & 6 \\
this->comp has only one value & 6 \\
this->comp.\_Function\_base.\_M\_functor has only one value & 6 \\
this->comp.\_Function\_base.\_M\_functor.\_M\_unused has only one value & 6 \\
this[0] != null & 3 \\
this->data != null & 3 \\
this->data.\_Vector\_base<int, std::allocator<int> >.\_M\_impl != null & 3 \\
this->comp != null & 3 \\
this->comp.\_M\_invoker != null & 3 \\
this->comp.\_Function\_base.\_M\_functor != null & 3 \\
this->comp.\_Function\_base.\_M\_functor.\_M\_unused != null & 3 \\
this->comp.\_Function\_base.\_M\_manager != null & 3 \\
this->data.\_Vector\_base<int, std::allocator<int> >.\_M\_impl.\_Vector\_impl\_data.\_M\_start[] elements >= 1 & 2 \\
this->data.\_Vector\_base<int, std::allocator<int> >.\_M\_impl.\_Vector\_impl\_data.\_M\_start != null & 1 \\
this->data.\_Vector\_base<int, std::allocator<int> >.\_M\_impl.\_Vector\_impl\_data.\_M\_finish != null & 1 \\
this->data.\_Vector\_base<int, std::allocator<int> >.\_M\_impl.\_Vector\_impl\_data.\_M\_end\_of\_storage != null & 1 \\
\hline
\end{tabular}
\end{table}

\begin{table}[ht]
\centering
\scriptsize
\caption{Invariants for hash\_table, 7 public methods.}
\label{hash_table_daikon}
\begin{tabular}{|l|c|}
\hline
Invariant & Count \\
\hline
::\_\_digits == "000102...6979899" & 3 \\
::\_\_tag == "" & 3 \\
this[0] has only one value & 3 \\
this->hash\_function has only one value & 3 \\
this->hash\_function.\_Function\_base.\_M\_functor has only one value & 3 \\
this->hash\_function.\_Function\_base.\_M\_functor.\_M\_unused has only one value & 3 \\
this->hash\_function.\_Function\_base.\_M\_functor.\_M\_pod\_data == "" & 3 \\
this->load\_factor == 0.75 & 3 \\
this->table has only one value & 3 \\
this->table.\_Vector\_base<... >.\_M\_impl has only one value & 2 \\
this->table.\_Vector\_base<... >.\_M\_impl has only one value & 1 \\
this->\_num\_elements one of \{ 0, 1 \} & 1 \\
this->\_size == 10 & 1 \\
this->table.\_Vector\_base<... > > > > >.\_M\_impl.\_Vector\_impl\_data.\_M\_end\_of\_storage & 1 \\
key.\_M\_dataplus has only one value & 1 \\
key.\_M\_dataplus.\_M\_p one of \{ "key1", "key2" \} & 1 \\
key.\_M\_string\_length == 4 & 1 \\
this->\_num\_elements >= 0 & 1 \\
this->\_size >= 0 & 1 \\
this->\_num\_elements <= this->\_size & 1 \\
\hline
\end{tabular}
\end{table}

\begin{table}[ht]
\centering
\scriptsize
\caption{Invariants for vector, 11 public methods. }
\label{vector_daikon}
\begin{tabular}{|l|c|}
\hline
Invariant & Count \\
\hline
this[0] has only one value & 9 \\
this->capacity == 5 & 7 \\
this->data has only one value & 5 \\
this->data[] elements one of \{ 1, 2 \} & 5 \\
this->n == 2 & 4 \\
this->data[] == [1, 2] & 4 \\
this->n in this->data[] & 4 \\
this->data[] == [1] & 3 \\
this->n == 0 & 2 \\
this->n one of \{ 1 \} & 2 \\
this->n in return[] & 1 \\
return[] == [1, 2] & 1 \\
return[] elements one of \{ 1, 2 \} & 1 \\
this->capacity == 0 & 1 \\
this->data == null & 1 \\
this->n == 1 & 1 \\
this->n == v.n & 1 \\
this->capacity == v.capacity & 1 \\
v.data has only one value & 1 \\
v one of \{ 1, 2 \} & 1 \\
this->capacity one of \{ 5 \} & 1 \\
this->data[] sorted by < & 1 \\
this->n <= this->capacity & 1 \\
this->n < this->capacity & 1 \\
this->n one of \{ 2, 5 \} & 1 \\
this->data[] elements == 1 & 1 \\
this->data[] one of \{ [1], [1, 2] \} & 1 \\
return one of \{ 1, 2 \} & 1 \\
\hline
\end{tabular}
\end{table}

\begin{table}[ht]
\centering
\scriptsize
\caption{Invariants for queue, 7 public methods.}
\label{queue_daikon}
\begin{tabular}{|l|c|}
\hline
Invariant & Count \\
\hline
this->data has only one value & 5 \\
this->maxSize == 10 & 5 \\
this[0] has only one value & 4 \\
this->data[] == [10, 20, 30] & 2 \\
this->data[] elements one of \{ 10, 20, 30 \} & 2 \\
this->head one of \{ 0, 1 \} & 2 \\
this->tail == 3 & 2 \\
this->n one of \{ 2, 3 \} & 2 \\
this->maxSize in this->data[] & 2 \\
this[0] != null & 2 \\
this->data != null & 2 \\
this->data[] elements >= 0 & 2 \\
this->data[] sorted by < & 2 \\
this->head < this->maxSize & 2 \\
this->tail < this->maxSize & 2 \\
this->data[] elements one of \{ 1, 2 \} & 2 \\
this->data[] one of \{ [1], [1, 2] \} & 2 \\
this->tail one of \{ 0, 1, 2 \} & 2 \\
this->tail in this->data[] & 2 \\
this->head - this->tail + this->n == 0 & 1 \\
this->tail one of \{ 2, 3, 100 \} & 1 \\
this->maxSize one of \{ 10, 160 \} & 1 \\
this->n < this->maxSize & 1 \\
this->head one of \{ 0, 50 \} & 1 \\
this->tail >= 0 & 1 \\
this->n <= this->maxSize & 1 \\
this->head <= this->tail & 1 \\
this->head one of \{ 0, 2 \} & 1 \\
this->n one of \{ 0, 1 \} & 1 \\
this->head == other.head & 1 \\
this->maxSize == other.maxSize & 1 \\
this->data[] == [7, 14] & 1 \\
this->data[] elements one of \{ 7, 14 \} & 1 \\
this->head == 0 & 1 \\
other.data has only one value & 1 \\
other.tail == 2 & 1 \\
this->head one of \{ 0, 1, 2 \} & 1 \\
this->head - this->tail + return == 0 & 1 \\
return one of \{ 0, 1, 2 \} & 1 \\
\hline
\end{tabular}
\end{table}

\begin{table}[ht]
\centering
\scriptsize
\caption{Invariants for stack, 6 public methods.}
\label{stack_daikon}
\begin{tabular}{|l|c|}
\hline
Invariant & Count \\
\hline
this->maxSize one of \{ 10, 160, 1280 \} & 4 \\
this->data[] sorted by < & 4 \\
this->data[] elements < this->maxSize & 4 \\
this->n < this->maxSize & 3 \\
this[0] != null & 3 \\
this->data != null & 3 \\
this[0] has only one value & 2 \\
this->data[] elements >= 0 & 2 \\
this->n one of \{ 0, 1, 2 \} & 1 \\
this->maxSize == other.maxSize & 1 \\
this has only one value & 1 \\
this->data has only one value & 1 \\
this->n == 2 & 1 \\
this->maxSize == 10 & 1 \\
other.data has only one value & 1 \\
other.data[] == [1, 2] & 1 \\
other.data[] elements one of \{ 1, 2 \} & 1 \\
this->n <= this->maxSize & 1 \\
\hline
\end{tabular}
\end{table}



\end{document}
\endinput
%%
%% End of file `sample-acmcp.tex'.
