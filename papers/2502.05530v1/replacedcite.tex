\section{Related Work and Positioning}
\label{Sec:RelatedWork}

Numerous works have focused on security ceremonies involving formal modeling and analysis of human errors for existing security protocol formalizations. Basin et al. in____, take a rule-based approach to formalize errors from various human behaviors, such as revealing secret information due to social-engineering attacks or entering passwords on rogue platforms, accounting for untrained, infallible, or fallible humans. They used the Tamarin Prover to analyze these formalizations against properties like device/message authentication and message secrecy in `MP-Auth' authentication protocol and phone-based protocols like `OTP over SMS'. Jacomme and Kremerin in____, defined a threat model for protocols involving multi-factor authentication over Transport Layer Security channels. They used Pi calculus to formalize threat scenarios, including human errors like the omission of actions, and used ProVerif to analyze the `Google 2-Step' and `U2F' protocols' versions for threat combinations. Unlike these approaches, which consider a Dolev-Yao adversary exploiting human weaknesses or protocol deviations, our analysis remains effective without such an adversary. Instead, we focus on how human errors can affect other participants' behavior and propagate throughout the ceremony. While Basin et al.____ defined rules to model human errors like sending/receiving unintended messages and improper comparisons, we consider a different, yet richer, set of mutations in this work.

Statistical analysis techniques have been used to detect application-level attacks, like brute force attacks, in Virtual Reality Learning Environments____, by considering potential user errors when typing a password for access. However, these techniques require collecting application-specific data to analyze error patterns, which are inherently probabilistic and may not offer definitive guarantees. In contrast, our work does not rely on empirical data collection. Instead, we use mathematical models and proofs to formally analyze the ceremony regarding security goals in all modeled scenarios. 

Some works have used the Tamarin Prover to analyze threats arising from human interactions in distributed settings. Bella et al. in____ take an epistemic modal logic approach to formalize both intentional and erroneous human-level threats in a security ceremony. They analyzed security properties against combinations of threat models forming a lattice, using examples of Deposit-Return Systems. Their approach assumes that all technical agents behave according to the ceremony specification. In contrast, we consider the impact of human mistakes on other ceremony participants by automatically generating matching mutations with propagation rules in X-Men, producing executable traces that can be analyzed for security attacks. We adapted Sempreboni and Viganò's____ approach to model user identification procedures as security ceremonies, capturing human interactions with systems and agents to identify vulnerabilities arising from human errors, even without an active adversary. We also extended this approach by introducing a new mutation rule, \emph{disorder}, which permits human users to perform actions out of sequence, deviating from the original ceremony's prescribed order.

More recently, Fila and Radomirovic presented in____ a formal modeling and analysis framework for ceremonies with synchronous agent interactions. The framework is generic in that it considers agents of arbitrary types, including humans, albeit in a finite number, and unifies existing symbolic models for specifying cryptographic protocols and security ceremonies. They considered the impact of execution contexts on human perception and behavior, defining them as ceremonies running parallel to the ceremony of interest. While they differentiate between normal and distracted human behavior, e.g., losing objects or forgetting information, by treating them as separate ceremonies, they offer a preliminary formalization without in-depth analysis.