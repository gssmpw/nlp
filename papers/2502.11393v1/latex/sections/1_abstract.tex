\begin{abstract}
%Despite the impressive performance, it remains unclear whether large language models (LLMs) truly understand the commonsense knowledge in commonsense reasoning tasks, or merely memorize superficial patterns. 
%In fact, LLMs are non-robust to the question variants with the same commonsense knowledge but different reasoning forms, a phenomenon not fully investigated by existing studies. 
%As an essential component of intelligence, commonsense reasoning has shown remarkable performance in large language models (LLMs). However, minor changes to questions can lead to incorrect responses. 
Large language models (LLMs) have shown remarkable capabilities in commonsense reasoning; however, some variations in questions can trigger incorrect responses.
\textit{Do these models truly understand commonsense knowledge, or just memorize expression patterns?}
To investigate this question, we present the first extensive robustness evaluation of LLMs in commonsense reasoning.
%This work presents the first comprehensive evaluation of the robustness of LLMs in commonsense reasoning. 
We introduce HellaSwag-Pro, a large-scale bilingual benchmark consisting of 11,200 cases, by designing and compiling seven types of question variants. 
To construct this benchmark, we propose a two-stage method to develop Chinese HellaSwag, a finely annotated dataset comprising 12,000 instances across 56 categories. 
We conduct extensive experiments on 41 representative LLMs, revealing that these LLMs are far from robust in commonsense reasoning. 
% We also find that such robustness is influenced by the specific language that the LLM is evaluated on.
Furthermore, this robustness varies depending on the language in which the LLM is tested.
% We hope our work can provide some insights to the research community in the domain of commonsense reasoning on LLMs. 
This work establishes a high-quality evaluation benchmark, with extensive experiments offering valuable insights to the community in commonsense reasoning for LLMs.

% Large language models (LLMs) have shown impressive capabilities in commonsense reasoning. However, it remains unclear whether LLMs truly understand the commonsense knowledge or merely memorize superficial patterns. 
% This work presents the first comprehensive assessment of LLMs' robustness in commonsense reasoning.
% We introduce HellaSwag-Pro, a large-scale bilingual benchmark designed to evaluate this robustness through seven variants, consisting of 11,200 cases. 
% To construct this benchmark, we also propose a two-stage method to develop Chinese HellaSwag, a fine-grained annotated dataset comprising 12,000 instances across 56 categories and three lengths.
% Our extensive experiments with diverse prompts on 41 representative models reveal that while LLMs demonstrate varying levels of commonsense reasoning ability, they are far from robust. Notably, models often fail to solve variations of problems they can initially solve correctly. Moreover, we also find that prompt performance is influenced by the model's native language as well as the type of task involved. 
% Our work points to promising directions for improving LLMs in the domain of commonsense reasoning, and we hope to provide some valuable insights to the community.



% Please add the following required packages to your document preamble:
% \usepackage{multirow}
\begin{table*}[!h]
\centering
\setlength{\abovecaptionskip}{0.05cm}
\setlength{\belowcaptionskip}{0cm}
%\begin{adjustbox}{width=\textwidth}
% Please add the following required packages to your document preamble:
% \usepackage{multirow}
% Please add the following required packages to your document preamble:
\scalebox{0.72}{
\begin{tabular}{lll}
\hline
\multicolumn{1}{l}{Variant Type} & Context& Choices\\ \hline
\multicolumn{1}{l}{\multirow{4}{*}{Initial data}} & \multirow{4}{*}{\begin{tabular}[c]{@{}l@{}}A lady walks to a barbell. She bends down \\ and grabs the pole. The lady\end{tabular}} & A. stands and lifts the weight over her head. \\ 
\multicolumn{1}{l}{}&& B. swings and lands in her arms.\\ 
\multicolumn{1}{l}{}&& C. pulls the barbell forward. \\ 
\multicolumn{1}{l}{}&& D. pulls a rope attached to the barbell.\\ \hline
\multicolumn{1}{l}{\multirow{4}{*}{\textcolor{bleudefrance}{Problem restatement}}} & \multirow{4}{*}{\begin{tabular}[c]{@{}l@{}}\textcolor{bleudefrance}{A woman approaches a weightlifting bar.} \\ \textcolor{bleudefrance}{She lowers her body and grasps the metal} \\ \textcolor{bleudefrance}{rod. The woman}\end{tabular}} &\textcolor{bleudefrance}{A. rises and hoists the barbell above her head.} \\ 
\multicolumn{1}{l}{}&&B. swings and lands in her arms.\\ 
\multicolumn{1}{l}{}&&C. pulls the barbell forward. \\ 
\multicolumn{1}{l}{}&&D. pulls a rope attached to the barbell.\\ \hline
\multicolumn{1}{l}{\multirow{4}{*}{\textcolor{cadmiumgreen}{Reverse conversion}}} & \multirow{4}{*}{\begin{tabular}[c]{@{}l@{}}The lady stands and lifts the weight over \\ her head. \textcolor{cadmiumgreen}{Which could be the most} \\ \textcolor{cadmiumgreen}{possible context for this action?}\end{tabular}}&\textcolor{cadmiumgreen}{A. A lady walks to a barbell. She bends down and grabs the pole.}\\
\multicolumn{1}{l}{}&&\textcolor{cadmiumgreen}{B. A lady positions herself at the squat rack. She lowers her body } \\ &&\textcolor{cadmiumgreen}{ \quad before rising steadily.} \\ 
\multicolumn{1}{l}{}&&\textcolor{cadmiumgreen}{C. A lady approaches the kettlebell set. She swings the weight} \\ && \textcolor{cadmiumgreen}{\quad forcefully between her legs.
}\\ 
\multicolumn{1}{l}{}&&\textcolor{cadmiumgreen}{D. A lady stands beside the bench press station. She lies down} \\&& \textcolor{cadmiumgreen}{\quad and lifts the barbell from her chest. }\\ \hline
\multirow{4}{*}{\textcolor{chromeyellow}{Causal inference}} & \multirow{4}{*}{\begin{tabular}[c]{@{}l@{}}A lady walks to a barbell. She bends down \\ and grabs the pole. The lady stands and \\ lifts the weight over her head. \textcolor{chromeyellow}{Which could} \\ \textcolor{chromeyellow}{be the most possible reason for this action?}\end{tabular}} &\textcolor{chromeyellow}{A. She is performing a weightlifting exercise.}\\ 
&&\textcolor{chromeyellow}{B. She is using the barbell as a decoration for an event.}\\ 
&&\textcolor{chromeyellow}{C. She is moving the barbell to a different location in the gym.}\\ 
&&\textcolor{chromeyellow}{D. She is cleaning the barbell after a workout session.} \\ \hline
\multirow{4}{*}{\textcolor{darkpastelpurple}{Sentence ordering}}& \multirow{4}{*}{\begin{tabular}[c]{@{}l@{}}1. She bends down and grabs the pole. \\ 2. A lady walks to a barbell. \\ 3. The lady stands and lifts the weight over \\ her head. \textcolor{darkpastelpurple}{Which is the correct order?}\end{tabular}}&\textcolor{darkpastelpurple}{A. 2-1-3}\\ 
&&\textcolor{darkpastelpurple}{B. 3-1-2}\\ 
&&\textcolor{darkpastelpurple}{C. 2-3-1}\\ 
&&\textcolor{darkpastelpurple}{D. 1-3-2}\\ \hline
\multirow{4}{*}{\textcolor{mauvelous}{Scenario refinement}}& \multirow{4}{*}{\begin{tabular}[c]{@{}l@{}}A lady walks to a barbell. She bends down \\ and grabs the pole. The lady \textcolor{mauvelous}{hesitates for} \\ \textcolor{mauvelous}{a moment, then changes her mind. Instead} \\ \textcolor{mauvelous}{of lifting the barbell, she}\end{tabular}}&A. swings and lands in her arms.\\ 
&&B. stands and lifts the weight over her head. \\ 
&&C. pulls the barbell forward.\\ 
&&D. pulls a rope attached to the barbell.\\ \hline
\multirow{4}{*}{\textcolor{lightseagreen}{Negative transformation}} & \multirow{4}{*}{\begin{tabular}[c]{@{}l@{}}A woman approaches a weightlifting bar. \\ She lowers her body and grasps the metal \\ rod. The lady \textcolor{lightseagreen}{will not}\end{tabular}} &A. swings and lands in her arms.\\ 
&&B. stands and lifts the weight over her head.\\ 
&&\textcolor{lightseagreen}{C. bend her knees and lift the barbell.} \\ 
&&\textcolor{lightseagreen}{D. adjust her grip and lift the weight.} \\ \hline
\multirow{5}{*}{\textcolor{peru}{Critical testing}} & \multirow{5}{*}{\begin{tabular}[c]{@{}l@{}}A lady walks to a barbell. She bends down\\ and grabs the pole. The lady \textcolor{peru}{suddenly} \\ \textcolor{peru}{realizes she forgot her weightlifting gloves}\\ \textcolor{peru}{and decides to postpone her workout.} \\ \textcolor{peru}{The lady}\end{tabular}} &A. stands and lifts the weight over her head. \\ 
&&B. swings and lands in her arms.\\ 
&&C. pulls the barbell forward. \\ 
&&D. pulls a rope attached to the barbell.\\ 
&&\textcolor{peru}{E. None of the above four options are suitable.} \\ \hline
\end{tabular}
}
%\end{adjustbox}
\caption{Examples of the seven variants we adopt for an initial question, with the correct answer unchanged as (A). Modifications are highlighted in different colors for clarity.}
\label{variant type}
\vspace{-10pt}
\end{table*}

\end{abstract}