%!TEX root = Article.tex

% Begin of file 3-SHACL.tex

\section{SHACL on common graphs}
\label{sec:shacl}

\begin{table}[t]
  \caption{Evaluation of a path expressions.}
  \label{tab:seme2}
  \centering
  \begin{tabular}{cl}
    \toprule
    $\pathExpr$ & $\iexpr{\pathExpr}{\graph} \subseteq (\Nodes\cup\Values)\times(\Nodes \times \Values)\ $ \\ % for $\graph = (E, \rho)$\\
    \midrule
    $\id$ & $\{ (v,v) \mid v \in \Nodes  \cup \Values \}$\\[2pt]
    $q$ & $\{(v,u)\mid (v,q,u)\in \graph \}$ \\[2pt]
    %$k$ & $\{(v,u)\mid \rho(v,k)=u\}$ \\[2pt]
    $\pathExpr^{-}$ & $\{(v,u)\mid (u,v) \in \iexpr{\pathExpr}{\graph}\}$ \\[2pt]
    $\pathExpr \cdot \pathExpr'$ & $\{(v,u) \mid \exists v': (v,v')\in\iexpr{\pathExpr}{\graph}\land (v',u)\in\iexpr{\pathExpr'}{\graph}\}$ \\[2pt]
    $\pathExpr\cup \pathExpr'$ & $\iexpr{\pathExpr}{\graph}\cup\iexpr{\pathExpr'}{\graph}$\\[2pt]
   $\pathExpr^{*}$ & $ \iexpr{\id}{\graph} \cup \iexpr{\pathExpr}{\graph}  \cup \iexpr{\pathExpr \cdot \pathExpr}{\graph} \cup \ldots $ \\[2pt]
    %$\pathExpr^{*}$ & $ \iexpr{\id}{\graph} \cup \{ (a,c) \mid (a,b) \in \iexpr{\pathExpr}{\graph} \land (b,c) \in \iexpr{\pathExpr^{*}}{\graph} \} $\\
       % the transitive reflexive closure of $\iexpr{\pathExpr}{\graph}$\\
    \bottomrule
  \end{tabular}
\end{table}

\newcommand{\defs}{\mathit{def}}
\renewcommand{\models}{\vDash}
\newcommand{\nmodels}{\nvDash}

We first treat SHACL, because it is conceptually the simplest of the three
languages.
It is essentially a logic---some call it a \emph{description logic in
disguise}~\cite{BJB22}.
Our abstraction is inspired by~\cite{MJPHD}.
We focus on the standard, non-recursive SHACL, leaving recursive extensions
\cite{CRS18,ACORSS20,BJ21,PKM22,OS24} for the future.
Some features of SHACL are incompatible with common graphs, and are therefore
omitted (see Appendix~\ref{app:standard-shacl}).
\todo{Remove reference to the appendix.}

%It defines shapes with formulas and relies on a generative formalism only to specify paths in the graph.

% \todo[inline]{Maybe we should mention SHACL core and also mention briefly the features not considered here and refer to the appendix for a more extensive discussion?}

%\todo[inline]{Filip: I do not know what SHACL core is. If we talk about it, we should explain. Do we have to?}

\begin{definition}[Path Expression]
  A \emph{path expression} $\pathExpr$ is given by the following grammar:
  \[
    \pathExpr
  \gDef
        \id
  \gMid q
  \gMid \pathExpr^{-}
  \gMid \pathExpr \cdot \pathExpr
  \gMid \pathExpr \cup \pathExpr
  \gMid \pathExpr^{*}
  \gEnd
  \]
  with $q \in \Predicates \cup \Keys$ and $\id$ the identity relation (or empty
  word).
%\todo[inline]{JH: The font for $\id$ is different from other keywords such as $\hasvalue$ and $\test$. Is there a reason for this?}
\end{definition}

\begin{definition}[SHACL Shape]
  \label{def:shacl-shape}
  A \emph{SHACL shape} $\varphi$ is given by the following grammar:
  \begin{align*}
    \varphi
  \gDef \
  & \top
  \gMid \hasvalue(c)
  \gMid \test(\vtype)
  \gMid \closed(Q)
  \gMid \eq(\pathExpr, p)
  \gMid \\
  & \disj(\pathExpr, p)
  \gMid \neg \varphi
  \gMid \varphi \land \varphi
  \gMid \varphi \lor \varphi
  \gMid \geqn{n}{\pathExpr}{\varphi}
  \gMid \leqn{n}{\pathExpr}{\varphi}
  \gEnd
  \end{align*}
  %with $c\in\Nodes\cup\Values$,
  with $c \in \Values$, $\vtype \in \ValueTypes$, $Q \subseteq_{\mathit{fin}}
  \Predicates \cup \Keys$, $p \in \Predicates$, and $n$ a natural number.
  We may use $\exists \pathExpr \ldotp \varphi$ as syntactic sugar for
  $\exists^{\geq 1} \pathExpr \ldotp \varphi$.
\end{definition}

\begin{definition}[SHACL Selector]
  A \selTerm $\mathit{sel}$ is a SHACL shape of a restricted form, given by the
  following grammar:
%\[ \mathit{sel} \gDef \geqn{1}{p}{\top}  \gMid \geqn{1}{k}{\top}  \gMid \geqn{1}{p^{-}}{\top} \gMid  \geqn{1}{k^{-}}{\top} \gMid  \hasvalue(c)\gEnd \]
  \[
    \mathit{sel}
  \gDef
        \exists\, q \ldotp \top
  \gMid \exists\, q^{-} \ldotp \top
  \gMid \hasvalue(c)
  \gEnd
  \]
  with $q \in \Predicates \cup \Keys$, and
%$c \in \Nodes \cup \Values$.
  $c \in \Values$.
\end{definition}

Putting it together, a \emph{SHACL Schema} $\SHACLSchema$ is a finite set of
pairs $(\mathit{sel}, \varphi)$, where $\mathit{sel}$ is a \selTerm and
$\varphi$ is a \shapeTerm.

To define the semantics of SHACL schemas, we first define in
Table~\ref{tab:seme2} the semantics of a SHACL path expression $\pi$ on a graph
$\graph$ as a binary relation $\iexpr{\pathExpr}{\graph}$ over $\Nodes \cup
\Values$.
The semantics of SHACL shapes is defined in Table~\ref{tab:semphi2}, which
specifies when a node or value $v$ \emph{satisfies} a \shapeTerm  $\varphi$ \wrt
a $\graph$, written $\graph, v \models \varphi$.
Note that both $\iexpr{\pathExpr}{\graph}$ and $\{ v \in \Nodes \cup \Values
\mid \graph, v \models \varphi\}$ may be infinite: for example,
$\sem{\text{id}}^\graph$ is the identity relation over the infinite set $\Nodes
\cup \Values$.

The semantics of SHACL schemas then follows \Cref{ssec:shapes}.
Importantly, SHACL selectors always select a finite subset of $\Nodes \cup
\Values$: the selected nodes or values come either from the selector itself, in
the case of $\hasvalue(c)$, or from $\graph$, in the remaining four cases.
For example, $\exists p \ldotp \top$  selects those nodes of $\graph$ that have
an outgoing $p$-edge in $\graph$---it is grounded to $\graph$ in the second line
of Table~\ref{tab:seme2}.
In consequence, each pair $(\mathit{sel}, \varphi)$ in a SHACL schema tests the
inclusion of a finite set of nodes or values in a possibly infinite set.
%\todo[inline]{Added explanation about finiteness vs infinity. Before, we said nothing at all about it, which I think was bad. Maybe my explanation can still be improved though!}

\begin{table}[t]
  \caption{Semantics of a \shapeTerm $\varphi$ .}
  \label{tab:semphi2}
  \centering
  \begin{tabular}{cl}
    \toprule
    $\varphi$ & $\graph, v \models \varphi$ if: \\
    \midrule

    $\top$ & trivially satisfied \\[2pt]
        $\hasvalue(c)$ & $v = c$\\[2pt]
    $\test(\vtype)$ & $v \in \sem{\vtype}$\\[2pt]

    $\closed(Q)$ &
                 % \begin{aligned}[t]
                    $  \forall p \in (\Predicates\cup\Keys) \setminus Q:$ not $\graph, v \models \exists^{\geq 1} p. \top$
                      %\notexists b \Rightarrow (y\in Q \; \land \\
                 %     & \qquad \text{if $\exists k: \rho(a,k)$ is defined, then $k\in Q)$}
                 %   \end{aligned}
                 %   $
                 \\[2pt]

    


    % $\hasshape(s)$ & $\SHACLShapeDef, \graph, v \models \SHACLShapeDef(s)$ \\
    % $\closed(Q)$ & $
    %                \begin{aligned}[t]
    %                  &\forall (x,y,z)\in E: x = v \Rightarrow (y\in Q \; \land \\
    %                  & \qquad \text{if $\exists k: \rho(a,k)$ is defined, then $k\in Q)$}
    %                \end{aligned}
    %                $\\

    $\eq(\pathExpr,p)$ & $\{ u \mid (v,u) \in \iexpr{\pathExpr}{\graph} \} = \{ u \mid (v,u) \in \iexpr{p}{\graph} \} $ \\[2pt]
    $\disj(\pathExpr,p)$ & $\{ u \mid (v,u) \in \iexpr{\pathExpr}{\graph} \} \cap \{ u \mid (v,u) \in \iexpr{p}{\graph} \} = \emptyset$\\[2pt]
    $\neg \varphi $ & not $\graph, v \models \varphi$  \\[2pt]
    $\varphi \land \varphi' $ & $\graph, v \models \varphi$ and $\graph, v  \models \varphi'$  \\[2pt]
    $\varphi \lor \varphi' $  & $\graph, v \models \varphi$ or $\graph, v  \models \varphi'$\\[2pt]

    
    % $\lessthan(F,E)$ & $b<c$ for all $b\in\iexpr{E}{\graph}(v)$ and $c\in\iexpr{F}{\graph}(v)$\\
    % $\lessthaneq(F,E)$ & $b\leq c$ for all $b\in\iexpr{E}{\graph}(v)$ and $c\in\iexpr{F}{\graph}(v)$\\
    % $\uniquelang(E)$ & $b$ and $c$ have different language tags \\ & for $b,c\in\iexpr{E}{\graph}(v)$ \\
    % $\forall E.\varphi$ & every $b\in\iexpr{E}{\graph}(v)$ satisfies $\graph,b\models\varphi$\\
    $\geqn{n}{\pathExpr}{\varphi}$ & $\#\{u\mid (v,u) \in \iexpr{\pathExpr}{\graph} \land \graph,u\models\varphi
    \}\geq n$\\[2pt]
    $\leqn{n}{\pathExpr}{\varphi}$ & $\#\{u\mid (v,u) \in \iexpr{\pathExpr}{\graph} \land\graph,u\models\varphi \}\leq n$\\
    \bottomrule
  \end{tabular}
\end{table}

% \paragraph{\textbf{Examples}}

% We now provide some examples using the vocabulary introduced in~\Cref{ex:sharedScenario} to illustrate some of the main features of SHACL and the notion of validation.

%\todo[inline]{O: for now we keep it a separate section, later integrate within the text}
%\begin{example}[Examples for SHACL schemas]
%\small
% inspired by \\ (https://w3c.github.io/data-shapes/data-shapes-test-suite/) and \\
% (https://www.w3.org/TR/shacl)/ \\
%{ \color{blue} (Re)written examples to match the scenario described in the Introduction.}
%We use $\exists^{=n}\pi.\varphi$ as syntactic sugar for $\exists^{\leq n}\pi.\varphi \land \exists^{\geq n}\pi.\varphi$.

% \OMIT {
%
% \noindent
% $\ 1) \ $ Every account has to have a primary credit card, i.e. there must exist a path from the account to the owner to a credit card, such that its key $\Exkey{primary}$ is set to ``true''.
% \[ \exists \Exprop{ownsAccount}^{-}\Rightarrow \exists ( \Exprop{ownsAccount}^{-} \cdot \Exkey{primary} \c ).( \hasvalue(\text{``true''})  ) \]
% \noindent
% $\ 2) \ $ Values of an accounts' $\Exkey{validity}$ key are of value type $\textbf{date}$.
% \[    \exists\Exprop{ownsAccount}^{-} \Rightarrow \exists \Exkey{validity}.\test(\Exvt{date}) \]
% \noindent
% $\ 3) \ $ Family sharing accounts (i.e. accounts with the key $\Exkey{familySharing}$ reaching the value ``true'' can have up to five users accessing it.
% \begin{align*}
%     \exists\Exprop{ownsAccount}^{-}  &\Rightarrow \\
%     &\neg\exists \Exkey{familySharing}.( \hasvalue(\text{``true''})) \lor \exists^{\leq 5} \Exprop{hasAccess}^{-}
% \end{align*}
% \noindent
% $\ 4) \ $ Every owner of an account has a unique email address (and no two users share an email).
% \begin{align*}
%     \exists\Exprop{ownsAccount} &\Rightarrow \exists \Exkey{email} . (\exists^{\leq 1} \Exkey{email}^{-})\\  \exists\Exkey{email}^{-}  &\Rightarrow \exists^{\leq 1} \Exkey{email}^{-}
% \end{align*}
% \noindent
% $\ 5) \ $ Owners of accounts may not have accounts in countries ($\Exkey{inCountry}$) for which they do not have at least one credit card issued in the same country.
% \begin{align*}
%     \exists \Exprop{ownsAccount} \Rightarrow \eq(& \Exprop{ownsAccount} \cdot \Exkey{inCountry}, \\ & \Exprop{hasCreditCard} \cdot \Exkey{inCountry}     )
% \end{align*}
%
% }

\begin{example}

For better readability we write $\exists \pi$ instead of $\exists^{\ge 1} \pi
\ldotp \top$ (that is, we omit $\top$) and $\forall \pathExpr \ldotp \varphi$
instead of $\leqn{0}{\pathExpr}{\lnot \varphi}$.
% $\exists^{=n}\pi.\varphi$ instead of $\exists^{\leq n}\pi.\varphi \land \exists^{\geq n}\pi.\varphi$
%$\exists \pi.\varphi$ for $\exists^{\ge 1}\pi.\varphi$ and
% $\ge_n \pi.\varphi$ instead of $\exists^{\ge n}\pi.\varphi$ (and similarly for $\leq$) and
% $\exists \pi$ instead of $\exists^{\ge 1}\pi.\top$ (that is, we omit $\top$) and $\forall \pathExpr. \varphi$ instead of $\leqn{0}{\pathExpr}{\lnot \varphi}$. %We shall also often omit $\top$ to improve readability, so we write $\exists \pi$ to express $\exists^{\ge 1}\pi.\top$.
% \noindent
Let us see how the constraints from Example~\ref{ex:constraint-desc} can be
handled in SHACL.
For (C1), we assume the value type $\mathbbm{int}$ with the obvious meaning.
The  following SHACL constraints express the constraints (C1--C5):
\begin{align*}
  & \exists \excard^{-} \Rightarrow \test(\mathbbm{int})
    & \mbox{(C1)} \\
  & \exists \exowns     \Rightarrow \exists \exemail
    & \mbox{(C2)} \\
  & \exists\exemail^{-} \Rightarrow \exists^{\leq 1} \exemail^{-}
    & \mbox{(C3)} \\
  & \exists \excard      \Rightarrow (\exists \exprivileged \ldotp \neg
      \hasvalue(\mathit{true})) \,\lor
    & \\
  & \qquad  \forall \exaccess^{-} \ldotp (\exists \exprivileged \ldotp
      \hasvalue(\mathit{true}))
    & \mbox{(C4)} \\
% & \exists\Exprop{ownsAccount} \Rightarrow \exists \Exkey{email}  & \\
  & \exists \exemail \Rightarrow \leqn{5}{\exaccess}
    & \mbox{(C5)}
\end{align*}
Concerning constraint (C3), notice that by using inverse \textit{email} edges,
the constraint indeed states that the email addresses uniquely identify users.
\end{example}

%\todo[inline]{Wim: Rule 3 seems to be saying ``Each account owner has a unique email address.'' In Sec 2 we ask: Each email address uniquely identifies the account owner.}
%\todo[inline]{Mantas: Rule uses the "inverse" of email. The rule then says that a given email (the value of some email attribute) can only have one receiving edge. Wim: AHA!}
% \todo[inline]{Wim: Rule 4 additionally needs to say that the object being referred to is an account. Currently it can also be a user.
% Cem: fixed it by checking for presence of key card.
% }
% \todo[inline]{Mantas: I did think about this originally, I am not sure this is really an issue, because a user cannot have incoming edges via hasAcess. }
% \todo[inline]{@Mantas: You're right about the hasAccess. It looks really strange from a modeling poing of view though. What we want to do here is usually done using labels in property graphs. We don't have these anymore, so we said that we were going to do such tests using incoming or ougoing edges; or the presence or absence of keys. We still need to do this (and explain it).
% \\
% Mantas: I know I know this was not a robust approach, but also putting $\neg \exists card$ is not a robust solution: it works mathematically in this example, but it is not a great way to identify users vs accounts
% \\
% Wim: But it's better, no? How would one really do it? By having an isUser key? Or an isUser incoming edge? This feels exactly the same to me than saying that a node had an email. (I changed 'not card' with email, because 'not card' is indeed bad.}
% \todo[inline]{There was a discussion here about C4. The outcome is that now  C4 was changed and now has an additional $\exists email$ to distinguish user from account. TODO explain this modelling approach of users vs accounts in Section 2.
% }

The constructs $\eq(\pathExpr, p)$ and $\disj(\pathExpr, p)$ are unique to
SHACL.
Let us see them in use.

%\todo[inline]{ Mantas: why not put $\exists hasAccess^{-}$ as a selector in C4 ?
%We select a node to which somebody has access to (it is indeed a modeling decision that such a node is inferred to be an account). Then we check that either it is not priviliged or all people who access it are priviliged. \\ Cem: done. this sidesteps the issue \\ SA: this will not select ACC2 \\
%Cem: ACC2 cannot violate the constraint. C4 implicitly assumes that paths of a certain kind (using hasAccess) must exist. If they don't it is trivially satisfied. I agree that it would be nicer to some way of addressing the ``class'' of accounts in the selector instead of the current hack. }
%\todo[inline]{SA: maybe one can use $\exists ownsAccount^-$ to distinguish since we somehow encode the meaning in the predicate.\\ Cem: later on in the paper it talks about encoding classes into keys. Not important whether we pick properties or keys, but we should be consistent \\ SA: I see, but the ownsAccount was specifically designed to carry the meaning of the class account. anyway, this should be relevant also for the next example}

%\todo[inline]{Wim: Maybe can be done. But why would I need to select the ``privileged accounts'' using an edge that does something different? I'd like to ``check that it's an account'' \textbf{and} ``check that it's privileged''.
%\\ Mantas: the part "and" is a big problem in SHACL the way I see it, for no good reason\\
%Moreover: as it is the selector $\exists \exprivileged$ does not say the selected account is in fact privileged, it only says that the privileged key is defined!
%\\ %Wim: I have to go to a meeting in the next 2 hrs :-(. Be back around 12. (I was also thinking about the "and" in selectors btw and agree with you wrt SHACL.) Maybe we can have it as syntactic sugar in the examples?
% Still have 30 mins :-)
%Maxime: I agree with Mantas, isn't the point to select all nodes that have `privileged' set to true?\\
%Wim: Not all nodes; only the accounts, right?}

\begin{example}\label{ex:fancy-shacl-eq}
  Using $\eq(\pathExpr, p)$, we can say, for instance, that an owner of an
  account also has access to it:
  \[
    \exists \exowns \Rightarrow  \eq(\exaccess \cup \exowns, \exaccess)\,.
  \]
  Note how we use $\eq$ and $\cup$ to express that the existence of one path
  ($\exowns$) implies the existence of another path ($\exaccess$) with the same
  endpoints.
\end{example}

A key feature in SHACL that is not available in ShEx is the ability to use
regular expressions to talk about complex paths.
%\todo[inline]{ShEx has recursion in general, so for paths with existentials (without counting) it may be possible to express them by nesting of shapes.}
%\todo[inline]{Filip: Yup. We talk about it after defining the core. But the sentence above is still true: there are no regular expressions for paths in ShEx. I rephrased a bit.}
This provides a limited, still non-trivial, form of recursive navigation in the
graph, even though the standard SHACL does not support recursive constraints
(in contrast to standard ShEx).
% See below for an example.

\begin{example}
  \label{ex:fancy-shacl-paths}
  Suppose that in~\Cref{fig:common-graph}, we impose that for every node with a
  $\exprivileged$ key, either its value is $\mathit{false}$ or, along inverse
  $\exinvited$ edges there is a unique, privileged ``ancestor'', which has no
  further inverse  $\exinvited$ edges.
% we wanted to express that a privileged user may only invite other privileged users, who in turn can also only invite other privileged users.
  This is expressible as follows:
  \begin{align*}
    & \exists \exprivileged \Rightarrow \exists \exprivileged \ldotp
      \hasvalue(\mathit{false}) \lor \\
    & \quad \exists^{\leq 1} {\exinvited^{-}}^{*} \ldotp \big( \exists
      \exprivileged \ldotp \hasvalue(\mathit{true}) \land \exists^{\leq 0}
      \exinvited^{-}  \big) \,.
  \end{align*}
\end{example}

% \todo[inline]{Filip: I was actually thinking about "at least" one.}
% \todo[inline]{JH: This can also be expressed without the Kleene star: just removing it gives an equivalent constraint. So it is not a strong example of something where the Kleene star is needed. }
% \todo[inline]{Cem: Here the draft for the fixed version, that needs reachability to be expressed. Problem: how to express its meaning in textual form, without being too technical }

% \noindent

% \noindent
% $\ \bullet \ $ Users owning an account must have a unique email address.
%     \[    \exists^{\geq1}\Exprop{isHandOf} \ \Rightarrow \exists^{=1}\Exprop{hasThumb} \land  \exists^{=4}\Exprop{hasFinger}  \land \closed (\Exprop{hasThumb},\Exprop{hasFinger})    \]
% \noindent
% $\ \bullet \ $ As the first one, but now we do not allow that the same object to be reachable by $\Exprop{hasThumb}$ and $\Exprop{hasFinger}$.
%     \[    \exists^{\geq1}\Exprop{isHandOf} \ \Rightarrow \exists^{=1}\Exprop{hasThumb} \land  \exists^{=4}\Exprop{hasFinger}  \land \disj(\Exprop{hasThumb},\Exprop{hasFinger})    \]
% \noindent
% $\ \bullet \ $ For node $\mathit{bob}$, all node reachable by $\Exprop{firstName}$ and $\Exprop{givenName}$ coincide.
% \[    \hasvalue(\mathit{bob}) \ \Rightarrow \eq(\Exprop{firstName},\Exprop{givenName})    \]
% \noindent
% $\ \bullet \ $  For node $\mathit{alice}$, no nodes reachable by $\Exprop{friendOf}$ must overlap with nodes reachable by $\Exprop{enemyOf}$.
% \[    \hasvalue(\mathit{alice}) \ \Rightarrow \disj(\Exprop{friendOf},\Exprop{enemyOf})    \]
% \noindent
% $\ \bullet \ $ For every subject of $\Exprop{isStarringIn}$, there must exist at path (as given) to the node $\Exprop{KevinBacon}$.
% \[    \exists^{\geq1}\Exprop{isStarringIn} \ \Rightarrow \exists^{\geq1}( \Exprop{isStarringIn} \cdot \Exprop{isStarringIn}^{-}  )^{*}.\hasvalue(\mathit{KevinBacon}).    \]
% \noindent
% $\ \bullet \ $ For every subject of $\Exprop{isFrenemy}$, there must exist at least one node, such that it can be reached both via $\Exprop{friendOf}$ and $\Exprop{enemyOf}$.
%     \[ \exists^{\geq1}\Exprop{isFrenemy} \ \Rightarrow \neg \disj(\Exprop{friendOf},\Exprop{enemyOf})    \]

% \begin{itemize}
% \itemsep=0pt
%     % Every subject of $\mathit{isHandOf}$ must have exactly one object reached by following $\mathit{hasThumb}$ and exactly four objects reachable by $\mathit{hasFinger}$.
%     \item
%     \item
% \end{itemize}

% \todo[inline]{Wim: I rewrote the 1st one to fit to our syntax. What about replacing it by, say, $\exists^{\geq 1}.timestamp.\top \Rightarrow \forall timestamp.test(date)$. In the 2nd one, I'd say that there are 5 fingers and explain the difference that you'd have by adding $... \land \textit{closed}(\{hasThumb,hasFinger\})$. The 5th one indeed doesn't seem to work (but the example may be nice to illustrate the capability in other languages). If we want to illustrate far navigation, we could also do something with transitive closure. We could say that every actor needs to have a co-starred path to Kevin Bacon.
% %Or an example about syntactical structures: every ``variable'' needs to have a finite ``grounded'' path to ``ground''.
% A wikidata-like example could be that every ontological class has some instance-of path to something that's not an ontological entity. (This something would be the actual object that's in the ontological class.)}

% \todo[inline]{Ognjen: Few comments. i) Should we have a different syntax for keys and properties... hard to understand which is which unless we say beforehand;
% ii) $\mathit{hasValue}(\mathit{bob})$, shouldnt it be
% $\exists^{\geq 1}.\mathit{firstName}.\mathit{hasValue}(\mathit{bob})$, otherwise
% we are testing it on a value but can we have a path from a value?
% Also $\mathit{bob}$ is a string so either so it would be good to distinguish them somehow, either by quotes or a special font, my 2cents;
% iii)  In the last example adding condition: $\mathit{eq}(\mathit{friendOf},\mathit{enemyOf})$ is improving the fact
% that every friend is also an enemy but I guess that's too strong for your example?}

% \OMIT{

% \todo[inline]{Maxime: I have some other example shapes that I used in the past to show that SHACL is quite powerful:}
% \begin{itemize}
%     \item ``All subjects of $\mathit{worksAt}$ must have at least one $\mathit{colleague}$ who is also a $\mathit{friend}$''.
%     \[\exists^{\geq1}\mathit{worksAt}.\top\Rightarrow \neg\disj(\mathit{colleague}, \mathit{friend})\]
%     \item ``Two different nodes never have the same email''. This is like a primary key constraint.
%     \[\exists^{\geq1}\mathit{email}^{-} .\top\Rightarrow
%     \exists^{\leq1} \mathit{email}^{-} .\top\]
%     \item ``Everyone who knows someone, knows at least themselves''.
%     \[\exists^{\geq1}\mathit{knows}\Rightarrow\eq(\mathit{knows}, \mathit{knows} \cup \id)\]
%     \todo[inline]{Mantas: this is not equivalent to the following?
%     \[\exists^{\geq1}\mathit{knows}\Rightarrow \neg \disj(\id,\mathit{knows}) \]
%     Cem: it is equivalent in this case, though in general $\eq$ and $\disj$ are not inter-expressible, this is a special case due to id.
%     }
%     \item ``Every researcher has a finite Erd\"os number''.
%     \[\exists^{\geq1}\mathit{hasPaper}.\top\Rightarrow
%     \exists^{\geq1}(\mathit{author}^{-}\cdot\mathit{author})^{*}.\hasvalue(\text{Erd\"os})\]
% \end{itemize}

% }
%\todo[inline]{Maxime: normally, the last one is ``every mathematician has ...'' but I cannot write class targets. Maybe we can change it somewhat to capture the same idea?}
%\todo[inline]{Maxime: about the knows example: this can be expressed with the build-in ``zero-or-one'' paths in SHACL, so there is no ``trickery'' with the identity.}
%\todo[inline]{SA: PG-schema and ShEx allow negation of predicates, which SHACL does not allow. We can surely come up with simple examples that are not expressible in SHACL. E.g., the domain elements of a predicate must participate in at most one outgoing edge. Also, PG-schema seems to not allow specific nodes, also in the selector, which SHACL and ShEx do allow. to continue...}

% \footnotetext{This example is in some sense problematic, since it hits at an ability we do not have: SHACL cannot say the same node must be reachable via 2 predicates (not the same as an alternative path that can follow either).}

% \todo[inline]{Since we can't have classes, this has to encode such information awkwardly into the property }

%\end{example}

% \todo[inline]{need to fix the language for $\mathit{sel}$, coordinated with ShEx part}

% A SHACL-Schema  is a tuple $\SHACLSchema = (\SHACLShapeDef, \SHACLShapeMap)$, where $\SHACLShapeDef$ is a shapes definition and $\SHACLShapeMap$ is a shape map. We require that the tuples in $\SHACLShapeMap$ may only pick shape names $s \in \Shapes$ from the domain of $\SHACLShapeDef$.

%%%  Without Shape names, there is no recursion

% We say a SHACL-Schema $\SHACLSchema$ is recursive when there exists some shape $s$ with shape expression $\varphi = \SHACLShapeDef(s)$, where this shape expression (in)directly refers to its own shape $s$ in $\varphi$ when one follows all $\hasshape(s)$ constructs across all shape expressions in the shapes definition of the schema. For now, we only define non-recursive SHACL-Schema, where shapes never refer back to themselves.

% Note that in case when $\SHACLShapeDef{}$ is clear from context, we may, by slight abuse of notation, associate a shape $s$ with its shape expression $\varphi = \SHACLShapeDef{}(s)$, and simply write $\graph, v \models s $, where  for  any occurrences of $\hasshape(s')$ we directly unfold and insert $\SHACLShapeDef{}(s')$ instead.

% This is seen in Table~\ref{tab:semphi2} where we use the notation $\defs(\SHACLShapeDef,s)$ for the shape expression $\phi$ when $(s,\phi)\in \SHACLShapeDef$.
% We note that the $\mathit{test}$-feature abstracts some of
% the SHACL constraint components used to perform tests on nodes. The
% possible tests are listed in \Cref{sec:possible_node_tests_in_shacl}.

% \todo[inline]{ The test(t) part should be unified with the value types in PG Schema semantically, maybe even notationally.  }

% \subsection{Validation of a Graph}
% \todo[inline]{Wim: This is now already taken care of, no?}

% We are now ready to define when a given SHACL-Schema  $\SHACLSchema $ satisfies a given graph $\graph$.  In order to simplify the definition, we introduce a some new notation. For a given graph $\graph$ and shape $\varphi$, we write $\iexpr{\varphi}{\graph}$ to capture the set $\{ v \mid v \in \Nodes \cup \Values \land \graph,v \models \varphi \}$.
% % Before we proceed with this, we need some notation to speak about which nodes are ``selected'' by a SHACL triple pattern.  \todo[inline]{formal definition of the nodes selected is probably needed}
% % For a given triple pattern $t$,

% % clarify the sets of nodes in a graph ``captured'' by target declarations, i.e. the set of nodes which it refers to. This definition can be seen in Table~\ref{tab:semTarget}.

% % For a given graph $\graph$ and triple pattern $\mathit{sel}$, we use the notation $\mathit{sel}(G)$ to identify the set of its selected nodes.
% % Analogously, for a set of target declarations $Q$, we have that $Q(G) = \bigcup_{t \in Q} t(G)$.
% % we note that we not yet clearly specified the exact sets of nodes that are selected by target declarations used in shape definition. We provide such a translation in Table . For a given graph $G$ and a set of target declaration

% \begin{definition}[SHACL-Schema Validity]
%     Given a SHACL-Schema $\SHACLSchema $ and a graph $\graph$, we say that $\graph$ is valid w.r.t. $\SHACLSchema$, denoted by $\graph \models \SHACLSchema$,  if it holds that for every $(sel,\varphi) \in \SHACLSchema$ we have:
% \[
%    \iexpr{\mathit{sel}}{\graph} \subseteq   \iexpr{\varphi}{\graph}
% \]
%     % \todo[inline]{current definition with nested universal quantification is very cumbersome. Introduce a denotational semantics short-hand and express that the nodes satisfying sel are subset of those satisfying $\varphi$}
%     % for every pair $(\mathit{sel},\varphi) \in \SHACLSchema$ and every node $v \in \mathit{sel}(\graph)$, we have that $\graph, v \models  \varphi$.

%     % Formally:
% % \begin{flalign*}
% % 	\graph \models \SHACLSchema   \ \Leftrightarrow
% % 	\ \forall (\mathit{sel},\varphi) \in \SHACLSchema  \big (  \forall v \in \mathit{sel}(\graph) . (  \graph, v \models \varphi )  \big )
% % \end{flalign*}

% \end{definition}

% \newcommand{\redSHACL}{SHACL-1S\xspace}

% \subsection{\redSHACL} To find a common intersection of all schema languages, we also consider a restriction of SHACL that limits allowed path expressions.
%  A \emph{one-step path expression} (OSPE) $E_1$ is defined as follows:
% \[
% E_1 \gDef \id \gMid p \gMid k \gMid E_1^{-} \gMid E_1\cup E_1 \gEnd
% \]
% with $p\in\Predicates$, $k\in\Keys$ and $\id$ the identity relation (or empty word).
% Informally, OSPEs allow paths of length one, following exactly one key or predicate, but none more. Complex expressions, such as the union of one-step paths, are allowed, as are inverses of OSPEs.
% The restriction of SHACL to OSPEs shall be referred to as \redSHACL.

% \todo[inline]{With the explanations in the Preliminaries on shapes and shape maps, most of the content of this section can be reduced to simply filling gaps in the `framework`: For SHACL you get this shape language, this selector language, and you are done}

% End of file 3-SHACL.tex
