\section{Common Graph Schema Language}
\label{sec:core}

We now present the Common Graph Schema Language (CoGSL), which combines the core functionalities shared by SHACL, ShEx, and PG-Schema (over common graphs). 

Let us begin by examining the restrictions that need to be imposed. We shall refer to shapes and selectors used in CoGSL as \emph{common shapes} and \emph{common selectors}. 
%
Common shapes cannot be closed under disjunction and negation, because PG-Schema shapes are purely conjunctive. For the same reason common shapes cannot be nested. 
%
Kleene star ${}^{*}$ cannot be allowed in path expressions because we consider ShEx without recursion. 
% \todo{Drop the following sentence?}
% By switching to ShEx with recursion, we would be able to support arbitrary SHACL path expressions in shapes of the form $\exists \pi$, but not arbitrary PG-path expressions as these are too expressive for SHACL. 
%
%\todo[inline]{also reading this sentence, the last part is not clear whether should be SHACL or ShEx. Do we also want to say something about not allowing nesting of shapes in the common shapes? It is mentioned in Section 5, but wonder whether it should be mentioned here as well. } 
%%E.g., can one express in PG-Schema the SHACL constaint $\exists p \Rightarrow \leq_2 p_1.=_3 p_2$?}
%%% FILIP: Rephrased. Added a sentence about nesting in the previous paragraph.  
%
Supporting path expressions traversing more than one edge under counting quantifiers is impossible as this is not expressible in ShEx. Supporting disjunctions of labels of the form $p_1 \cup p_2$ is also impossible, due to a mismatch in the approach to counting: while SHACL and PG-Schema count nodes and values, ShEx counts triples, as illustrated in \Cref{ex:shex-counts-edges}. 

Closed content types and $\lnot P$ cannot be used freely, because neither SHACL nor ShEx are capable of closing only properties or only predicate edges: both must be closed at the same time. 

Finally, selectors are restricted because SHACL and ShEx do not support $\top$ as a selector; that is, one cannot say that each node (or value) in the graph satisfies a given shape. This means that SHACL and ShEx schemas always allow a disconnected part of the graph that uses only predicates and keys not mentioned in the schema, whereas PG-Schema can disallow it (see Example~\ref{ex:closedgraph}). 

Putting these restrictions together we obtain the Common Graph Schema Language. We define it below as a fragment of PG-Schema. 

\begin{definition}[common shape]
\label{def:simple-shape}
    A \emph{common shape} $\varphi$ is an expression given by the grammar
\begin{align*}
\varphi  \gDef  & 
 \exists\,\pexpr
\gMid \exists^{\leq n} \, \pexpr_1
\gMid \exists^{\geq n} \, \pexpr_1 \gMid 
\exists\, \ntype \land \not\exists\, \lnot P
\gMid \varphi \land \varphi \gEnd\\
\ntype \gDef &\closedRT{}\  \gMid\  \closedRT{k : \vtype} \ \gMid\   \ntype \tAnd \ntype  \ \gMid \  \ntype \tOr \ntype  \gEnd\\
\pexpr_0 \gDef & \keyIsVal{k}{c} \gMid 
\lnot \keyIsVal{k}{c} \gMid \ntype\tAnd\top\gMid \lnot (\ntype\tAnd\top) \gMid \pexpr_0 \cdot \pexpr_0 \gEnd\\
\pexpr_1 \gDef &  \pexpr_0  \cdot p \cdot
\pexpr_0 
\gMid  \pexpr_0  \cdot p^{-} \cdot
\pexpr_0  \gMid \pexpr_0\cdot k \gMid k^{-}\cdot \pexpr_0 \gEnd\\
\ppexpr \gDef & \pexpr_0  \gMid p 
\gMid {\ppexpr}^{-} \gMid \ppexpr \cdot \ppexpr 
\gMid \ppexpr \cup \ppexpr \gEnd \\
\pexpr \gDef & \ppexpr \gMid \ppexpr \cdot k \gMid k^{-}\cdot \ppexpr \gMid k^{-}\cdot \ppexpr\cdot k' \gEnd
\end{align*}
where $n \in \mathbb{N}$, $P\subseteq_{\mathit{fin}} \Predicates$, $k,k'\in\Keys$, $c\in\Values$, and $p\in\Predicates$. 
\end{definition}

That is, $\ntype$ is a content type that does not use $\top$ (a \emph{closed} content type),  $\pi_0$ is a PG-path expression that always stays in the same node (a \emph{filter}), $\pi_1$ is a PG-path expression that traverses a single edge or property (forward or backwards), and $\pi$ is a PG-path expression that uses neither ${}^{*}$ nor $\lnot P$. 
Moreover, $\pi_0$, $\pi_1$, and $\pi$ can only use \emph{open} content types; that is, content types of the form $\ntype \tAnd \top$.
The use of $\lnot P$ is limited to closing the neighbourhood of a node (this is the only way PG-Schema can do it). 


% \begin{definition}[Simple selector]
% \label{def:simple-selector}
% A \emph{simple selector} is a common shape of one of the following forms 
% \[ 
% \exists\, p \,, \quad 
% \exists\, p^{-} \,,\quad 
% \exists\, k \,,\quad
% \exists\, \{k:w\} \,, \quad \exists\, k^{-}\,,
% \]
% where $k\in\Keys$,  $p\in\Predicates$, and $w\in \Values$. 
% \end{definition}

\begin{definition}[common selector]
\label{def:common-selector}
A \emph{common selector} is a common shape of one of the following forms
\[ 
\exists\, k \,,\;
\exists\, p \cdot \pexpr\,, \;
\exists\, p^{-}\!\cdot \pexpr \,,\;
\exists\, \keyIsVal{k}{c}\cdot \pexpr \,, \;
\exists\, \big(\{k:\vtype\}\tAnd\top\big)\cdot \pexpr \,, \;
\exists\, k^{-}\!\cdot \pexpr\,,\!\!
\]
where $k\in\Keys$,  $p\in\Predicates$, $c\in \Values$, $\vtype\in\ValueTypes$ and $\pexpr = \ppexpr$ or $\pexpr = \ppexpr\cdot k'$ for some PG-path expression $\ppexpr$ generated by the grammar in Definition~\ref{def:simple-shape} and some $k'\in\Keys$.
\end{definition}

That is, a common selector is a common shape of the form $\exists\, \pexpr$ such that  the PG-path expression $\pexpr$ requires the focus node or value to occur in a triple with a specified predicate or key. 


A \emph{common schema} is a finite set of  pairs $(\textit{sel}, \varphi)$ where $\textit{sel}$ is a common selector and $\varphi$ is a common shape. The semantics is inherited from PG-Schema. 

%We illustrate common schemas in \Cref{ex:sharedExamplesPGS}, which describes PG-Schemas, but all of which are also valid common schemas.
We note that we showed that the constraints (C1)-(C5) from our running example can be expressed in all three formalisms. Specifically, the PG-Schema representation from \Cref{ex:sharedExamplesPGS} is also a common schema. 

\begin{proposition}
\label{prop:core}
For every common schema there exist equivalent SHACL and ShEx schemas. 
\end{proposition}

The translation is relatively straightforward (see Appendix~\ref{sec:appendix-core}). The two main observations are that star-free PG-path expressions can be simulated by nested SHACL and ShEx shapes, and that closure of SHACL and ShEx shapes under Boolean connectives  allows encoding complex selectors in the shape (as the antecedent of an implication). We illustrate the latter in~\Cref{ex:PathSelector}.

\begin{example}[Complex paths in selectors] \label{ex:PathSelector}
We want to express that all users who have invited a user who has invited someone (so there is a path following two $\exinvited$ edges) must have a key $\exemail$ of type $\Exvt{str}$.
In PG-schema we express this as:
\[ \exists \exinvited \cdot \exinvited \Rightarrow \{ \exemail : \Exvt{str}  \} \tAnd \top     \]

At first glance, it seems unclear how to express this in the other formalisms, since they do not permit paths in the selector. However, we can see that paths in selectors can be encoded into the shape: 

\noindent In SHACL, using the same example, we do this by 
\begin{align*}
\exists \exinvited  \Rightarrow &\  \neg (\exists \exinvited \cdot \exinvited ) \lor  \exists \exemail . \test(\Exvt{str})      
\end{align*}
And in  ShEx for this example would be: 
\begin{align*}
\shexneigh{\exinvited.\shextop \shexeach \shexallte}  \Rightarrow &\ 
\neg \se_2
\lor \shexneighopen{\exemail.\test(\Exvt{str})}  
\end{align*}
where $\se_2 = \shexneighopen{\;
    \left(
    \exinvited. \se_1
    \right)^{\geq 1}
\;}$ and $\se_1 = \shexneighopen{\,
        \exinvited . \shextop^{\geq 1}
    \,}$.
That is, $\se_1$ is satisfied by nodes that have an outgoing path $\exinvited$, and $\se_2$ by nodes that have an outgoing path $\exinvited\cdot\exinvited$.
For paths of unbounded length, it is not apparent how such a translation would proceed for ShEx schemas in the absence of recursion. 
\end{example}


