\section{Shape-based PG-Schema}
\label{sec:pgschema-simplified}

Shape-based PG-Schema is a non-recursive combination of a logic and two generative formalisms. It uses path expressions to specify paths (as in SHACL), and \emph{content types} to specify node contents. Both path expressions and content types are then used in formulas defining shapes.
%% \todo[inline]{Fabio: At the beginning of SHACL section we say that SHACL is also based on a combination of a logic and a generative formalism. I think we need to make this distinction clearer or to remove it tout court.}
%% \todo[inline]{Filip: Good point! Rewrote this. Hope it's fine now.}
Content types in PG-Schema play a role similar to triple expressions in ShEx, but they are only used for properties. Because all properties of a node must have different keys, they are much simpler than triple expressions (in fact, they can be translated into a fragment of SHACL). Unlike for SHACL and ShEx, the abstraction of shape-based PG-Schema departs significantly from the original design. Original PG-Schema uses queries written in an external query language,  which is left unspecified aside from some basic assumptions about the expressive power. Here we use a specific query language (PG-path expressions). Importantly, up to the choice of the query language, the abstraction we present here faithfully captures the expressive power of the original PG-Schema. A detailed comparison can be found in Appendix~\ref{sec:standard-pg-schema}.


\begin{definition}[Content type]
\label{def:contentType}
A \emph{content type} is an expression $\ntype$ of the form defined by the grammar
 $$\ntype \gDef \top \ \gMid \ \closedRT{}\  \gMid\  \closedRT{k : \vtype} \ \gMid\   \ntype \tAnd \ntype  \ \gMid \  \ntype \tOr \ntype  \gEnd$$
where $k \in \Keys$ and $\vtype \in \ValueTypes$. 
\end{definition}

Recall that $\Records$ is the set of all records (finite-domain partial functions $r : \Keys \pto \Values$). We write $\emptyRec$ for the empty record. 
% For records $r_1$ and $r_2$, we let $r_1 \cup r_2$ be the function that behaves as $r_1$ on $\dom{r_1}$ and as $r_2$ on $\dom{r_2}$. 
% We require that $r_1(k) = r_2(k)$ for every $k \in \dom{r_1} \cap \dom{r_2}$. 
The semantics of content types is defined in Table~\ref{tab:semPG-content-types}. Note that $\sem{\ntype}$ is independent from $\graph$ and can be infinite.

\begin{table}
  \caption{Semantics of content types.}
  \label{tab:semPG-content-types}  
  \centering
  \begin{tabular}{cl}
    \toprule
    $\ntype$ & $\sem{\ntype} \subseteq \Records$ \\
    \midrule    
     $\sem{\top}$ & $\Records$ \\[2pt]
     $\sem{\closedRT{}}$ & $\{ \emptyRec \}$\\[2pt]
     $\sem{\closedRT{k : \vtype}}$ & $\big\{ \{(k, w)\} \ \big|\ w \in \sem{\vtype} \big\}$\\[2pt]
    $\sem{\,\ntype_1 \tAnd \ntype_2\,}$ & $\{ (r_1 \cup r_2) \in \Records \mid r_1 \in \sem{\ntype_1} \wedge r_2  \in \sem{\ntype_2} \}$\\[2pt]
    $\sem{\,\ntype_1 \tOr \ntype_2\,}$ & $\sem{\ntype_1} \cup \sem{\ntype_2}$\\[2pt]
    \bottomrule
  \end{tabular}
\end{table}

\begin{example}
We assume integers and strings are represented via $\mathbbm{int},\mathbbm{str}\in \ValueTypes$.  Suppose we want to create a content type for nodes that have a string value for the $\exemail$ key and \emph{optionally} have an integer value for the $\excard$ key. No other key-value pairs are allowed. We should then use $\closedRT{\exemail : \mathbbm{str}} \tAnd (\closedRT{\excard : \mathbbm{int}} \tOr \closedRT{})  $.
\end{example}

%\todo[inline]{ Give more insight about the choice of the query language.
%based on a variant of path expressions, that is natural from the perspective of property graphs and
%} 

\begin{definition}[PG-path expressions] 
\label{def:pgpaths-syntax}
A PG-path expression is an expression $\pexpr$ of the form defined by the  grammar
\begin{align*} 
& \pexpr \gDef  \ppexpr \gMid \ppexpr \cdot k \gMid k^{-} \cdot \ppexpr \gMid k^{-} \cdot \ppexpr \cdot  k' \gEnd \\
& \ppexpr \gDef \keyIsVal{k}{c} \gMid \neg \keyIsVal{k}{c} \gMid \ntype \gMid \lnot \ntype \gMid p \gMid \lnot P \gMid  
{\ppexpr}^{-} \gMid \ppexpr \cdot \ppexpr \gMid \ppexpr \cup \ppexpr \gMid {\ppexpr}^{*} \gEnd
\end{align*}
where $k,k' \in \Keys$, $c \in \Values$, $\ntype$ is a content type, $p \in \Predicates$, and $P \subseteq_{\mathit{fin}} \Predicates$. We use $k$, $k^{-}$, and $k^{-}\cdot k'$ as short-hands for PG-path expressions $\top\cdot k$, $k^{-}\cdot \top$, and $k^{-}\cdot \top\cdot  k'$, respectively. 
\end{definition}

% \todo[inline]{JH: I would prefer notation $[k = c]$ over $\keyIsVal{k}{c}$. That makes it more clear we are filtering and not navigating away by following edges or keys, and it avoids better the impression that this is similar to a closed type. TODO: Jan does this.}

Unlike in SHACL, PG-path expressions cannot navigate freely through values. In the property graph world, this would correspond to a join, which is a costly operation. Indeed, existing query languages for property graphs do not allow joins under ${}^{*}$. However, PG-path expressions can start in a value and finish in a value. This leads to \emph{node-to-node}, \emph{node-to-value}, \emph{value-to-node}, and \emph{value-to-value} PG-path expressions, reflected in the four cases in the first rule of the grammar. 


The semantics of PG-path expression $\pexpr$ for graph $\graph$ is a binary relation over
$\nodes(\graph) \cup \values(\graph)$, defined in Table~\ref{tab:semPGtypes}. In the table, $k$ is treated as any other subexpressions, eventhough it can only be used at the end of a PG-path expression, or in the beginning as $k^{-}$.
Notice that $\lnot \ntype$ matches nodes whose content is not of type $\ntype$,  
$\lnot P$ matches edges with a label that is not in $P$ (in particular, $\lnot\emptyset$ matches all edges). 
Also, 
$\gsem{\pexpr}$ is always a subset of $\Nodes \times \Nodes$, $\Nodes \times \Values$, $\Values \times \Nodes$, or $\Values \times \Values$, corresponding to the four kinds of PG-path expressions discussed above.
 

%\todo[inline]{SA: Shouldn't it be $\gsem{\pexpr}\subseteq \nodes(\graph) \cup \values(\graph)$ in Table 4?}
%\todo[inline]{Filip: I guess you mean: \[\gsem{\pexpr}\subseteq (\nodes(\graph) \cup \values(\graph))\times (\nodes(\graph) \cup \values(\graph))\]
%It would be more precise, but it's really long... I am not even sure it is necessary at all in the table. It was meant as a recollection of the "type" of the returned objects, at Fabio's request. }

\begin{table}[tb]
  \caption{Semantics of PG-path expressions.}
  \label{tab:semPGtypes}  
  \centering
  \begin{tabular}{cl}
    \toprule
    $\pexpr$ & $\gsem{\pexpr}\subseteq (\Nodes\cup\Values)\times (\Nodes\cup\Values)\ $  for  $\graph = (E, \rho)$ \\[2pt]
    \midrule    
    $\keyIsVal{k}{c}$ & $\left\{ (u, u) \mid u \in \nodes(\graph) \land (k,c) \in \rho(u) \right\}$ \\[2pt]
        $\neg \keyIsVal{k}{c}$ & $\left\{ (u, u) \mid u \in \nodes(\graph) \land (k,c) \notin \rho(u) \right\}$ \\[2pt]
    $\ntype$ & $\left\{ (u, u) \mid u\in\nodes(\graph)\land \rho(u) \in \sem{\ntype} \right\}$ \\[2pt]
    $\lnot \ntype$ & $\left\{ (u, u) \mid u \in \nodes(\graph)\land \rho(u)\notin \sem{\ntype} \right\}$ \\[2pt]
    $k$ & $\{(u,w)\mid \rho(u,k)=w\}$ \\[2pt]
    $p$ & $\left\{ (u, v) \mid (u, p, v) \in E \right\}$  \\[2pt]
    $\lnot P$ & $\left\{ (u, v) \mid \exists p : (u, p, v) \in E \wedge p \notin P \right\}$ \\[2pt]
    $\pathExpr^{-}$ & $\left\{(u,v)\mid (v,u) \in \iexpr{\pathExpr}{\graph}\right\}$ \\[2pt]
    $\pathExpr \cdot \pathExpr'$ & $\left\{(u,v) \mid \exists w: (u,w)\in\iexpr{\pathExpr}{\graph}\land (w,v)\in\iexpr{\pathExpr'}{\graph}\right\}$ \\[2pt]
    $\pathExpr\cup \pathExpr'$ & $\iexpr{\pathExpr}{\graph}\cup\iexpr{\pathExpr'}{\graph}$\\[2pt]
   $\pathExpr^{*}$ & $ \{(u,u)\mid u\in\nodes(\graph)\}\cup \iexpr{\pathExpr}{\graph}  \cup \iexpr{\pathExpr \cdot \pathExpr}{\graph} \cup \ldots $ \\
    \bottomrule
  \end{tabular}
\end{table}


\begin{definition}[PG-Shapes]
A PG-Shape is an expression $\varphi$ defined by the following grammar:
\[ 
\varphi \gDef \exists ^{\leq n} \, \pexpr \gMid \exists^{\geq n} \,\pexpr \gMid \varphi \land \varphi \gEnd
\]
where $\pexpr$ is a PG-path expression. We use $\exists$ and $\nexists$ as short-hands for $\exists^{\geq 1}$ and $\exists^{\leq 0}$.
\end{definition}


The semantics of PG-shapes  is defined in Table~\ref{tab:semPGshapes}. We say $v\in\Nodes\cup\Values$ \emph{satisfies} a PG-shape $\varphi$ in a graph $\graph$ if $\graph, v \models \varphi$. Every PG-shape is satisfied by nodes only or by values only. 

\begin{table}[tb]
  \caption{Satisfaction of PG-shapes}
  \label{tab:semPGshapes}  
  \centering
  \begin{tabular}{cl}
    \toprule
    $\varphi$ & $\graph, v \models\varphi\ $ for  $v\in\Nodes\cup\Values$\\
    \midrule   
    $\exists^{\leq n} \, \pexpr$ &
    $\#\left\{v' \mid (v, v') \in \gsem{\pexpr}\right\} \leq n$  \\[2pt]
    $\exists^{\geq n} \, \pexpr$ &
    $\#\left\{v' \mid (v, v') \in \gsem{\pexpr}\right\} \geq n$  \\[1pt]
    $\varphi_1 \land \varphi_2$ &
    $\graph, v \models\varphi_1$ and $\graph, v \models\varphi_2$ \\
    \bottomrule
  \end{tabular}
\end{table}

%\subsection{PG-Schemas}

\begin{definition}[PG-Selectors]
A \emph{PG-selector} is a PG-shape of the form $\exists\, \pexpr$. 
\end{definition}



A \emph{PG-Schema} $\schema$ is 
a finite set of pairs $(\textit{sel}, \varphi)$ where $\textit{sel}$ is a PG-selector and $\varphi$ is a PG-shape.
The semantics of PG-Schemas is defned just like in Section ~\ref{ssec:shapes}.

% \paragraph{\textbf{Examples}} We now provide some examples using the vocabulary introduced in~\Cref{ex:sharedScenario} to illustrate some of the main features of PG-Schema.

% We introduce the following short-hands:
% \begin{itemize}
% 	\item $\closedRT{k_1 : \textbf{bool}, k_2 : \textbf{date}}$ for $\closedRT{k_1 : \textbf{bool}} \tAnd \closedRT{k_2 : \textbf{date}}$
% 	\item $\openRT{k_1 : \textbf{bool}, k_2 : \textbf{date}}$ for $\closedRT{k_1 : \textbf{bool}, k_2 : \textbf{date}} \tAnd \top$
% \end{itemize}
% %\todo[inline]{Wim: I guess that the shorthands will mainly (only?) be useful in examples?}


% % \begin{example}[Examples for PG-schemas]
% % \todo[inline]{Sandbox for examples. Working in progress. }
% %\todo[inline]{Wim: I can take care of syntax here.}
% %\todo[inline]{Yes, go ahead}

% \begin{enumerate}
% \item Every timestamp key has datatype ``date'':
% \[\exists \; \closedRT{\mathit{timestamp} : \top} \  \Rightarrow \  \nexists \; \lnot \closedRT{\mathit{timestamp} : \textbf{date}} \]

% \item The value ``Aretha'' appears in the graph.

% \hfill $\leadsto$ Cannot be said?
% \todo[inline]{SA: I do not see how it can be done in SHACL either. However, maybe the following is enough to capture it in ShEx $\hasvalue(\textit{Aretha})  \  \Rightarrow \ \shexneigh{\neg\emptyset^{*} \shexeach {\neg\shexinverse{\emptyset}}^{*}}$?\\
% Cem: I believe the negated property constraint can \emph{also} express it: 
% $\hasvalue(\textit{Aretha})  \  \Rightarrow \ \neg(\emptyset)^{-}.(\hasvalue(\textit{Aretha}))$ (this assumes ``Aretha'' is indeed a value, not a node).\\
% }
% \todo[inline]{Wim: FYI, I think that it's fine (and possibly insightful!) to give examples of things that cannot be expressed.}
% \todo[inline]{Maxime: I think it cannot be done in SHACL. However, as a not-so-useful-but-fun remark, you can in SHACL write ``Aretha appears as a subject of a triple in the graph.'' This is not really what we want, there is --- to my knowledge --- no way to say that Aretha must appear as an object in the graph.}

% \item The graph has an edge with property ``subclassOf'':
% \[\exists \lnot \emptyset \ \Rightarrow \  ... \]

% \hfill $\leadsto$ Cannot be said?

% \item Each node with an outgoing ``isHandOf'' edge has one outgoing ``hasThumb'' edge, five outgoing ``hasFinger'' edges, and arbitrarily many other outgoing edges.
% \[\exists \mathit{isHandOf} \ \Rightarrow \ \exists^{=1} \mathit{hasThumb} \land \exists^{=5} \mathit{hasFinger} \land \exists^{\geq 0} \star\]
% Here, we use $\star$ as a shorthand for $\lnot \emptyset$ (the construct $\lnot P$ in PG-path expressions $\bar{\pi}$).

% \item Each node with an outgoing ``isHandOf'' edge has one outgoing ``hasThumb'' edge and five outgoing ``hasFinger'' edges, and no other outgoing edges.
% \[\exists \mathit{isHandOf} \ \Rightarrow \ \exists^{=1} \mathit{hasThumb} \land \exists^{=5} \mathit{hasFinger}\]

% \item Each node with an outgoing ``isHandOf'' edge has one ``hasThumb'' key.
% \[\exists \mathit{isHandOf} \ \Rightarrow \ \exists^{=1} \{\mathit{hasThumb} : \top\} \]

% \item Each node with an outgoing ``isHandOf'' edge has five ``hasFinger'' keys.

% Cannot be said? One may think that it is
% \[\exists \mathit{isHandOf} \ \Rightarrow \ \exists^{=5} \{\mathit{hasThumb} : \top\} \]
% but the data model doesn't allow it.


% \end{enumerate}



% \begin{align*}
% \textit{hasValue}(c_1) &\Rightarrow \exists^{\geq 1} \, (l^{-} \cdot l^{-}) \land \top \\
% \exists^{\geq 1} \, \{ \mathit{isHandOf} \} &\Rightarrow \exists^{=1} \, \{ \mathit{hasThumb} \} \land \exists^{=4} \, \{ \mathit{hasFinger} \} \\
% \textit{hasValue}(\mathit{bob}) &\Rightarrow \{ \mathit{firstName} : \mathit{givenName} \} = \{ \mathit{givenName} : \mathit{firstName} \}  \\
% \textit{hasValue}(\mathit{alice}) &\Rightarrow \{ \mathit{friendOf} \} \land \lnot \{ \mathit{enemyOf} \} \\
% \exists^{\geq 1} \, \{ \mathit{isFrenemy} \} &\Rightarrow \exists^{\geq 1} \, \{ \mathit{friendOf} \} \land \exists^{\geq 1} \, \{ \mathit{enemyOf} \} \\
% \end{align*}


% \todo[inline]{Ognjen: 
% 1) $\top$ in datatypes is a bit confusing. For example, the most general dataype in XML is called $\texttt{anyType}$;
% 2) How would you say this in SHACL or SheX, I am not sure one can do it unless listing all keys?
% 3) the same as for 2);
% 5) I think we dont need TODO part since PG-schema is closed world?
% 6) Here we need TODO (i guess you want to allow to have more than five fingers)

% Overall, I think we should gather all the short hands that we use for the examples to make them more readable, otherwise it looks like a bash scripting :)}


% \todo[inline]{Mantas: I am not sure what will happen to the above material on examples. Below is the example related to the example common graph. }
\begin{example}\label{ex:sharedExamplesPGS} The constraints (C1-C5) from Example~\ref{ex:constraint-desc} can be handled in PG-Schema as follows: 
\begin{align*}
& \exists \excard \Rightarrow  \exists \big ( \closedRT{\excard : \mathbbm{int}}\tAnd \top \big )  & \mbox{(C1)} \\
&  \exists \exowns\Rightarrow \exists \exemail  & \mbox{(C2)} \\
&  \exists\exemail^{-}  \Rightarrow \exists^{\leq 1} \exemail^{-} & \mbox{(C3)} \\
& \exists\, (\{ \excard : \any  \} \tAnd \top) \cdot  \{\exprivileged :\mathit{true} \} \Rightarrow  & \\ &   \qquad  
\nexists \,\exaccess^{-} \cdot \neg \{\exprivileged :\mathit{true}\} & \mbox{(C4)} \\
% & \exists\exowns \Rightarrow \exists \exemail  & \\
& \exists \exemail \Rightarrow \exists ^{\leq 5} \, \exaccess & \mbox{(C5)} 
\end{align*}
Notice that in rule (C1), we indeed need $\exists \excard,$ rather than $\exists \excard^{-}$, because there is no PG-Shape to state that the selected value is of type $\mathbbm{int}$, and so we formulate C1 as a statement about nodes.
%\todo[inline]{JH: this is just to flag that the previous remark was added. It was originally suggested as a footnote, but I feel it is important enough to be in the text. Originally the formulation was that PG-Selectors cannot select values (which is not really true) or that PG-Shapes cannot make statements about values (which is also not really true).}
% \todo[inline]{JH: the solution for C1 should have selector $\exists \excard$, and not $\exists \excard^{-}$ as it does now. TODO: Fixed, but maybe add a foot note about this. @Jan}
\end{example}
% \todo[inline]{Cem: $\not \exists \pi$ not directly expressible in PG-Shapes, so C4 should be rewritten}
% \todo[inline]{Mantas: hm, but there is $\nexists$ defined as a shorthand for $\exists^{\leq 0}$. Am I missing something here?}

% \end{example}


A characteristic feature of PG-Schema, revealing its database provenience, is that it can close the whole graph by imposing restrictions on all nodes.

\begin{example}
\label{ex:closedgraph}
    Given a common graph such as the one in~\Cref{fig:common-graph}, we might want to express that each node has a key $\exprivileged$ with a boolean value and  either a key $\excard$ with an integer value or a key $\exemail$ with a string value, and no other keys are allowed. In PG-Schema this  can be expressed as follows:
    \[ \exists \top \Rightarrow \exists \{\exprivileged: \mathbbm{bool}\} \tAnd\big(\{ \excard : \mathbbm{int} \} \tOr \{ \exemail : \mathbbm{str} \}\big) \,.\]
    We can also forbid  any predicates except those mentioned in the running example:
    \[ \exists \top \Rightarrow \not\exists \lnot \{\exowns,\exaccess,\exinvited\} \,.\]
\end{example}