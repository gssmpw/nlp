%!TEX root =  Article.tex

\section{Standard PG-Schema}
\label{sec:standard-pg-schema}


The version of PG-Schema presented in the body of the paper is a variant of PG-Schema that is constructed to preserve the essence of the original PG-Schema as presented in \cite{ABDF23} but also to fit a paradigm of a shape-based schema language like SHCAL and ShEx, and in particular to follow the paradigm where a schema consists of a set of selector-shape pairs. 
However, the original PG-Schema follows a different paradigm, namely one where a schema consists of a set of node types, a set of edge types, and a set of constraints.
To show that nevertheless the version of PG-Schema in the body of the paper preserves its core functionality, we will present here an intermediate version that we will refer to as \emph{PG-Schema on Common Graphs} while we refer to the version of PG-Schema in the body of the paper as \emph{shape-based PG-Schema}, and to the PG-Schema defined in \cite{ABDF23} as \emph{original PG-Schema}.

\subsection{PG-Schema on Common Graphs}

The central idea of the original PG-Schema in \cite{ABDF23} 
is that a schema (called \emph{graph type} 
in this context) consists of three parts: (1) a set of node types, 
(2) a set of edge types, and (3) a set of graph constraints
that represents logical statements about the property graph that must hold for it to be valid.  
A particular property graph is then said to be valid wrt.\ such a graph type if (1) every node 
in the property graph is in the semantics of at least one node type, (2) every edge in the property graph 
is in the semantics of at least one edge type, and (3) the property graph satisfies all specified graph constraints.

The organisation of this section is as follows. We first discuss the notions of \emph{node types} and \emph{edge types}.
After that we discuss how path expressions are defined, after which we discuss what graph constraints look like in this setting.
In the final two subsections we discuss how this version of PG-Schema relates the original defined in
\cite{ABDF23}, and how it relates to the one define in this paper.

\subsubsection{Node types}

The purpose of node types in the original PG-Schema is to describe nodes, their properties and their labels. 
Since in the common graph model nodes there are no labels, node types become simply record types
where the record fields describe the allowed keys.
Therefore node types are here defined to be the same as the content types defined in Definition~\ref{def:contentType}.
In the original PG-Schema it was possible to indicate if these record types are \emph{closed} or \emph{open},
where the former indicates that only the indicated keys are allowed, and the latter that additional keys
are allowed. This is easily expressed with such node types, and for example a node type that requires the presence of a key with name $\excard$ and a value of type $\mathbbm{int}$, and allows in addition other keys, is represented by $\closedRT{\excard : \mathbbm{int}} \tAnd \top$.

\subsubsection{Edge types}

In the original PG-Schema there is a notion of edge type, which consists of three parts: 
(1) a type describing the source node, (2) a type describing describing the contents of the edge itself, and (3) a type describing the the target node.
Since in common graphs the content of an edge is just a label, a type describing this content can be simply an expression of the form $\pwc$ (a wild-card indicating that any label is possible) or a finite set $P$ of labels (indicating that only these labels are allowed). 
So we get the following definition for edge types.

\begin{definition}[Edge type]
An \emph{edge type} is an expression $\etype$ of the form defined by the grammar
$$\etype \gDef \et{\ntype}{\pwc}{\ntype} \gMid \et{\ntype}{P}{\ntype} \gMid  \etype \tAnd \etype  \gMid  \etype \tOr \etype  \gEnd $$
where $P$ is a finite subset of $\Predicates$.
\end{definition}

As for node types, we define for edge types a value semantics, which in this case defines which combinations of (1) source node content, (2) edge content, and (3) target node content are allowed. 

\begin{definition}[Value semantics of edge types]
 With an \emph{edge type} $\etype$ we associate a \emph{value semantics} $\sem{\etype} \subseteq \Records \times \Predicates \times \Records$ which is defined with induction on the structure of $\etype$ as follows:
 \begin{enumerate}
   \item $\sem{\et{\ntype_1}{\pwc}{\ntype_2}} = \sem{\ntype_1} \times \Predicates \times \sem{\ntype_2}$
   \item $\sem{\et{\ntype_1}{P}{\ntype_2}} = \sem{\ntype_1} \times P \times \sem{\ntype_2}$   
   \item \begin{tabbing}
         $\sem{ \etype_1 \tAnd \etype_2 } = \{ $ \= $ ( (r_1 \cup s_1), p, (r_2 \cup s_2) ) \in \Records \times \Predicates \times \Records \mid $ \\
         \> $( r_1, p, r_2 ) \in \sem{\etype_1} \land ( s_1, p, s_2 ) \in \sem{\etype_2} \}$
      \end{tabbing}
   	\item $\sem{ \etype_1 \tOr \etype_2 } = \sem{\etype_1} \cup \sem{\etype_2}$
 \end{enumerate}
\end{definition}

\subsubsection{Path expressions}

We define here a notion of path expression that we call \emph{extended PG-path expression} and that is similar to the notion of PG-path expression of Definition~\ref{def:pgpaths-syntax}, except that in the positions where a content type $\ntype$ is allowed, 
we also allow an edge type $\etype$.

\begin{definition}[Extended PG-path expressions] 
\label{def:ext-pgpaths-syntax}
An extended  PG-path expression is an expression $\pexpr$ of the form defined by the  grammar
\begin{align*} 
& \pexpr \gDef  \ppexpr \gMid \ppexpr \cdot k \gMid k^{-} \cdot \ppexpr \gMid k^{-} \cdot \ppexpr \cdot  k' \gEnd \\
& \ppexpr \gDef \keyIsVal{k}{c} \gMid \neg \keyIsVal{k}{c} \gMid \ntype \gMid \lnot \ntype \gMid \etype \gMid \lnot \etype \gMid \\
& \quad \quad p \gMid \lnot P \gMid  
{\ppexpr}^{-} \gMid \ppexpr \cdot \ppexpr \gMid \ppexpr \cup \ppexpr \gMid {\ppexpr}^{*} \gEnd
\end{align*}
where $k,k' \in \Keys$, $c \in \Values$, $\ntype$ is a content type, $p \in \Predicates$, and $P \subseteq_{\mathit{fin}} \Predicates$. 
% We use $k$, $k^{-}$, and $k^{-}\cdot k'$ as short-hands for PG-path expressions $\top\cdot k$, $k^{-}\cdot \top$, and $k^{-}\cdot \top\cdot  k'$, respectively. 
\end{definition}

The semantics of extended PG-path expressions is identical to that of PG-path expressions for the expressions they have in common, and for the additional parts, the edge types $\etype$ and $\lnot \etype$, the semantics is given in Table~\ref{tab:semSPGtypes}.

\begin{table}[tb]
  \caption{Semantics extended PG-path expressions.}
  \label{tab:semSPGtypes}  
  \centering
  \begin{tabular}{cl}
    \toprule
    $\pexpr$ & $\gsem{\pexpr}\subseteq (\Nodes\cup\Values)\times (\Nodes\cup\Values)\ $  for  $\graph = (E, \rho)$ \\[2pt]
    \midrule    
    $\etype$ & $\left\{ (u, v) \mid \exists p : (u, p, v) \in E \land (\rho(u), p, \rho(v)) \in \sem{\etype} \right\}$ \\
    $\lnot \etype$ & $\left\{ (u, v) \mid \exists p : (u, p, v) \in E \land (\rho(u), p, \rho(v)) \notin \sem{\etype} \right\}$ \\
    \bottomrule
  \end{tabular}
\end{table}

\subsubsection{Graph constraints}

The graph constraints in the original PG-Schema are based on the constraints discussed in PG-Keys \cite{ABDF21}. 
Although the latter paper focuses on key constraints, it also discusses other closely related cardinality constraints. 
We capture these constraints here in the context of the common graph data model with the following formal definition.

\begin{definition}[PG-constraint]
 A \emph{PG-constraint} is a formula of one of the following three forms:
\begin{description}
    \item[Key:] $\forall x : \varphi(x) \Lleftarrow \Key \, \bar{y} : \psi(x, \bar{y})$
    \item[Upb:] $\forall x : \varphi(x) \to \exists^{\leq n} \, \bar{y} : \psi(x, \bar{y})$
    \item[Lwb:] $\forall x : \varphi(x) \to \exists^{\geq n} \, \bar{y} : \psi(x, \bar{y})$
\end{description}
where $x$ is a variable that ranges over nodes and values, $\varphi(x)$ and $\psi(x, \bar{y})$ are formulas of the form $\exists \bar{z} : \xi$ with $\bar{z}$ a vector of node and value variables and $\xi$ a conjunction of atoms of the form $\pathExpr(z_i, z_j)$ with $z_i$ and $z_j$ either equal to $x$, or in $\bar{y},$ or in $\bar{z}$, and $\pathExpr$ an extended PG-path expression, such that the free variables in $\varphi(x)$ are just $x$ and those in $\psi(x, \bar{y})$ are $x$ and the variables in $\bar{y}$. 
 \end{definition}

The semantics of the constraints of the form \textbf{Key} is the logical formula $\forall \bar{y} : \exists^{\leq 1} x : \varphi(x) \land \psi(x, \bar{y})$. This corresponds to the intuition that it defines a key constraint for all values or nodes $x$ that are selected by $\varphi(x)$ and for those it specifies that that the vector $\bar{y}$ for which $\psi(x, \bar{y})$ holds identifies at most one such $x$.
So the symbol $\Lleftarrow$ should be read here as stating that the left-hand side is functionally determined by the right-hand side.

For the constraints of the forms \textbf{Upb} and \textbf{Lwb} the interpretation is simply the usual one in first-order logic.

\subsection{Comparison with the original PG-Schema }

% \todo[inline]{Let's see how much of it we really want to keep here. Goal: keep it roughly analogous to the discussion for RDF-based formalisms. @Jan @Filip}
% \todo[inline]{JH: This has not been done yet.}

The PG-Schema on Common Graphs defined here introduces two important simplification w.r.t.\ original PG-Schema: (1) It is defined over common graphs which simplifies the property graph data model in several ways and (2) it assumes what is called the STRICT semantics of a graph type in  \cite{ABDF23} and ignores the LOOSE semantics. We briefly discuss here why these simplification preserve the essential characteristics of the original schema language.

\subsubsection{Concerning the simplification of the data model}

As discussed in Section~\ref{sect:PGCGComparison} common graphs simplify property graphs in three ways: (1) nodes only have properties and no labels, (2) edges only have one label and no properties, and (3) edges have no independent identity. 
However, these features can be readily simulated in the common graph data model.
For example, edges with identity can be simulated by nodes that have an outgoing edge with label $\Exkey{source}$ and an outgoing edge with label $\Exkey{target}$ to respectively the source node and the target node of the simulated edge.
Moreover, labels can be simulated by introducing a special dummy value $\Lambda$ that is used for keys that represent labels. 
For example, a node $n$ where $\rho(n)$ contains the pairs $(\Exkey{Person}, \Lambda)$, $(\Exkey{Employee}, \Lambda)$, $(\Exkey{hiringDate}, \textit{12-Dec-2023})$, and $(\Exkey{fulltime}, \Exkey{true})$, simulates a node with labels $\Exkey{Person}$ and $\Exkey{Employee}$, and properties $\Exkey{hiringDate}$ and $\Exkey{fulltime}$. 

It is not hard to see how under such a simulation PG-Schema on Common Graph could simulate a more powerful schema language where we could use tests in path expressions for the presence (or absence) of (combinations of) labels in path expressions and
tests for presence (or absence) of (combinations of) properties of edges.
Moreover, we could navigate via simulated edges and test for certain properties with a path expression of the form $\Exkey{source}^{-} \cdot \pathExpr \cdot \Exkey{target}$ where $\pathExpr$ simulates any test over the content of the edge.
Finally, we could straightforwardly simulate key and cardinality constraints for edges.

\subsubsection{Concerning the STRICT and LOOSE semantics} 

In the original PG-Schema there is a separate LOOSE semantics defined for graph types. 
In that case the set of node types and the set of edge types in the graph type are ignored and 
a property graph is said to be already valid wrt.\ a graph type if it satisfies all graph constraints in the graph type.
The LOOSE interpretation can be easily simulated in PG-Schema on Common Graphs
by letting the set of node types contain only $\top$, the trivial node type, 
and the set of edge types contain only $\et{\top}{\pwc}{\top}$, the trivial edge type.

\subsection{Comparison with Shape-based PG-Schema}

The constraints of the forms \textbf{Upb} and \textbf{Lwb} are very similar to the selector-shape pairs presented for 
PG-Schema in Section~\ref{sec:pgschema-simplified}. 
Indeed, the selector is in this case the formula $\varphi(x)$ and the shape is 
the formulas of the forms $\exists^{\leq n} \bar{y} : \psi(x, \bar{y})$ and $\exists^{\geq n} \bar{y} : \psi(x, \bar{y})$.
However, there are also several notable differences:
(1) The schema in Shape-based PG-Schema only consists of constraints and does not separately define sets of allowed node and edge types.
(2) There are no edge types in path expressions. 
(3) All constraints are restricted so that $\bar{y}$ is just a single variable.
(4) There are no constraints of the form \textbf{Key}.
(5) The constraints are syntactically restricted such that (a)
$\varphi(x)$ is restricted to just one atom, so the form $\exists z : \pathExpr(x, z)$, and (b)
$\psi(x, \bar{y})$ is restricted to just one atom, so the form $\pathExpr(x, y)$. It is this last restriction that allows a notation in description-logics style without variables.
Apart from these restrictions, there is also a generalisation, namely in Section~\ref{sec:pgschema-simplified} the shapes are closed under intersection.
That this does not change the expressive power is easy to see, since a selector-shape pair of the form $(\sel, (\varphi_1 \land \varphi_2))$ 
can always be replaced with the combination of the pairs  $(\sel, \varphi_1)$ and $(\sel, \varphi_2)$ without changing the semantics of the schema.

In the following subsections we discuss the previously mentioned restrictions.

\subsubsection{No separate sets of allowed node types and edge types}

It is not hard to show that this can be simulated. 
Assume for example we have a graph type with a set of node types $\{ \ntype_1, \ntype_2, \ntype_3 \}$.
The check that each node must be in the semantics of at least one of these node types can be simulated in PG-Schema on Common Graphs by 
the \textbf{Lwb} constraint $$\forall x : \top(x, x) \to \exists y : (\ntype_1 \tOr \ntype_2 \tOr \ntype_3)(x, y)$$ 
Note that node types are closed under the $\tOr$ operator, and so $(\ntype_1 \tOr \ntype_2 \tOr \ntype_3)$ is indeed a node type, and therefore an extended PG-Path expression in PG-Schema on Common Graphs.
Recall that a node type acts in a path expression as the identity relation restricted to nodes that are in the semantics of that type.

Similarly, if the set of edge types of a graph type is $\{ \etype_1, \etype_2, \etype_3 \}$, we can ensure that each edge is in the semantics of at least one of these edge types using the following \textbf{Upb} constraint in PG-Schema on Common Graphs: 
$$\forall x : \top(x, x) \to \exists^{\leq 0} y : \lnot(\etype_1 \tOr \etype_2 \tOr \etype_3)(x, y)\,.$$
Note that edge types are closed under union, and so $\etype_1 \tOr \etype_2 \tOr \etype_3$ is also an edge type, and in addition edge type can appear negated and extended PG-Path expressions, and so $\lnot(\etype_1 \tOr \etype_2 \tOr \etype_3)$ is indeed a valid path expression in PG-Schema on Common Graphs.


\subsubsection{No edge types in path expressions}

It is not hard to show that path expressions that contain tests involving edge types 
can be rewritten to equivalent path expressions that do not use edge types.

We first consider the non-negated edge types in path expressions. We start with the observation that we can normalise edge types to a union of edge types that do not contain the $\tOr$ operator.
This is based on the following equivalences for path semantics that allow us to push down the $\tOr$ operator:
\begin{itemize}
    \item $\et{(\ntype_1 \tOr \ntype_2)}{\alpha}{\ntype_3} \equiv ( \et{\ntype_1}{\alpha}{\ntype_3} \tOr \et{\ntype_2}{\alpha}{\ntype_3} )$
    \item $\et{\ntype_1}{\alpha}{(\ntype_2 \tOr \ntype_3)} \equiv ( \et{\ntype_1}{\alpha}{\ntype_2} \tOr \et{\ntype_1}{\alpha}{\ntype_3} )$
\end{itemize}
In a next normalisation step we can remove bottom-up the $\tAnd$ operator using the following rules, where we use the symbol $\etype_{\emptyset}$ to denote the empty edge type:
\begin{itemize}
    \item $(\et{\ntype_1}{\alpha}{\ntype_2}) \tAnd (\et{\ntype_3}{\beta}{\ntype_4}) \equiv \et{(\ntype_1 \tAnd \ntype_3)}{\alpha \sqcap \beta}{(\ntype_3 \tAnd \ntype_4)}$
\end{itemize}
where $\sqcap$ is defined such that (1) $\pwc \sqcap P = P \sqcap \pwc = P$ for $P \subseteq \Predicates$, and (2) $P \sqcap Q = P \cap Q$ for $P, Q \subseteq \Predicates$.

As a final normalisation step we get rid of edge types $\et{\ntype_1}{P}{\ntype_2}$ where $P$ contains two or more predicates, by applying the rule:
\begin{itemize}
    \item $\et{\ntype_1}{\{ p_1, \ldots, p_k \}}{\ntype_2} \equiv (\et{\ntype_1}{\{ p_1 \}}{\ntype_2} \tOr \ldots \tOr \et{\ntype_1}{\{ p_k \}}{\ntype_2})$
\end{itemize}

After these normalisation steps we will have rewritten the edge type to the form $(\etype_1 \tOr \ldots \tOr \etype_k)$ with each $\etype_i$ a \emph{primitive edge type} in the sense that it cannot be normalised further and therefore one of the following forms:
(1) $\et{\ntype_1}{\pwc}{\ntype_2}$, 
(2) $\et{\ntype_1}{\{ p \}}{\ntype_2}$, and 
(3) $\et{\ntype_1}{\emptyset}{\ntype_2}$.
We can express such an edge type $(\etype_1 \tOr \ldots \tOr \etype_k)$ as a path expression $(\pathExpr_1 \cup \ldots \cup \pathExpr_k)$, where each $\pathExpr_i$ is constructed as follows:
\begin{itemize}
    \item $\et{\ntype_1}{\pwc}{\ntype_2} \equiv \ntype_1 \cdot \neg \emptyset \cdot \ntype_2$
    \item $\et{\ntype_1}{\{ p \}}{\ntype_2} \equiv \ntype_1 \cdot p \cdot \ntype_2$
    \item $\et{\ntype_1}{\emptyset}{\ntype_2} \equiv \neg \top$
\end{itemize}
Recall that $\neg\top$ is the negation of the trivial node type and so in a path expression represents the empty binary relation.

We now turn our attention to negated edge types. The part under the negation can be normalised as before,
and so we end up with an edge type of the form $\neg (\etype_1 \tOr \ldots \tOr \etype_k)$ with each $\etype_i$ a primitive edge type.
This can be represented as a path expression $(\pathExpr_1 \cup \ldots \cup \pathExpr_{m})$ where each $\pathExpr_j$ is a path expression the expresses a particular reason that an edge might not conform to any of the types in $\etype_1, \ldots, \etype_k$.
To illustrate this consider as an example the following negated edge type:
\[\lnot \big( \et{\ntype_1}{\{ p \}}{\ntype_2} \tOr \et{\ntype'_1}{\{ p' \}}{\ntype'_2} \big)\]
This can be simulated in a path expression by replacing it with the following path expression:
\[\begin{array}{c@{\hspace{6pt}}c@{\hspace{6pt}}c@{\hspace{6pt}}c@{\hspace{6pt}}c@{\hspace{6pt}}c@{\hspace{6pt}}c}
& \lnot\ntype_1\cdot \lnot\ntype'_1 \cdot \lnot\emptyset & \cup &
 \lnot\ntype_1 \cdot \lnot\{p'\} & \cup &
 \lnot\ntype_1 \cdot \lnot\emptyset \cdot \lnot\ntype'_2 & \cup\\
\cup\;  & \lnot\ntype'_1 \cdot \lnot\{p\} & \cup &
\lnot\{p,p'\} & \cup &
  \lnot\{p\} \cdot \lnot\ntype'_2 & \cup\\
\cup \; & \lnot\ntype'_1\cdot  \lnot\emptyset \cdot \lnot\ntype_2 &\cup&
\lnot\{p'\} \cdot \lnot\ntype_2  &\cup &\;
\lnot\emptyset \cdot \lnot\ntype_2 \cdot \lnot\ntype'_2\;
\end{array}\]
Note that this indeed enumerates all the ways that an edge could not be in the semantics of $\big( \et{\ntype_1}{\{ p \}}{\ntype_2} \tOr \et{\ntype'_1}{\{ p' \}}{\ntype'_2} \big)$. Basically we pick for each of the primitive edge types whether the edge is not in the semantics because of (1) the source node, (2) the label, or (3) the target node.


\subsubsection{Only single variable counting}

The restriction to allow only one variable in $\bar{y}$ is introduced because in SHACL and ShEx all the counting is also restricted to single values and nodes, rather than tuples of values and nodes. 
Although this is often useful in real-world data modelling, e.g., to represent composite keys, this restriction is introduced to make PG-Schema more comparable to SHACL and ShEx.

\subsubsection{No \textbf{Key} constraints}
If \textbf{Key} constraint are restricted to single-variable counting, \textbf{Key} constraints are of the form \[\forall x : \varphi(x) \Lleftarrow \Key \, y : \psi(x, y)\,.\] 
Recall that its semantics is defined by the formula $\forall y : \exists^{\leq 1} x : \varphi(x) \land \psi(x, y)$. If $y$ matches nodes (which can be detected based on path expressions used in the atoms involving $y$), we can equivalently express this constraint in PG-Schema for Common Graphs as
\[
\forall y : \top(y, y) \to \exists^{\leq 1} x: \varphi(x)\land\psi(x,y)
\]

If $y$ matches values, then it is used in the first position of an atom whose path expressions begins from $k^{-}$ or in the second position of an atom whose path expression ends with $k$. In either case, we can equivalently express this constraint in PG-Schema for Common Graphs as
\[
\forall y : (k^{-}\!\cdot k) (y', y) \to \exists^{\leq 1} x: \varphi(x)\land\psi(x,y)
\]


\subsubsection{Only one atom in formulas}

This restriction of the query language underlying PG-Schema for Common Graphs limits the expressive power of PG-Schema, but similar restrictions are present in SHACL and ShEx. Some additional expressive power could be gained by allowing tree-shaped conjunctions of atoms with at most 2 free  variables, but this would further complicate the formal development.




% \todo[inline]{JH: To be discussed: (1) We remove edge types. Because we can simulate them. (2) We restrict ourselves to just one atom in the formulas and just one variable in $\bar{y}$. This is because we restrict ourselves to unary keys, to be comparable with SHACL. Also, a single path expression as selector, existentially quantified, seems the most straightforward way to define unary selectors. (3) We remove constraints of type \textbf{Key}. Because we can simulate them. (4) conjunction in shapes is added in the version in the body of the paper, but can be simulated.}



% \subsection{\cgkeys}


% \todo[inline]{JH: Note to self. CPRQs, but for edge labels we also allow edge types, and we can also use node types as filters.}

% \todo[inline]{Filip: I think it would be better to use the intermediate formalisation, with at-least, at-most, and key constraints. Otherwise we are not going to be able to provide the power we have in the body. One can still argue that this is PG-Schema, as defined in the PG-Schema paper, even if at-least and at-most constraints are only proposed as an extension. }



% We assume a pattern matching language over graphs where pattern can contain zero or more variables, and these variables can range over nodes and values.

% \begin{remark}
%     A difference with \pgkeys is that there variables can range over edges as well. Since edges do not have an identity of their own in common graphs this is logical restriction. As will be discussed later, adding them does not add expressive power. 
% \end{remark}

% \begin{definition}[Key constraint]
%     A \emph{key constraint} is a formula of the form $\forall x : \varphi(x) \to \Key^{C} \bar{y} : \psi(x, \bar{y})$ where $\varphi(x)$ is a graph pattern with free variable $x$, $\psi(x, \bar{y})$ a graph pattern with free variables $x$ and those in the vector of variables $\bar{y}$, and $C$ is a subset of $\{ \mand, \sing \}$ that qualifies the key.
% \end{definition}

% In the preceding definition $\mand$ indicates a key is \emph{mandatory}, i.e., must exist, and $\sing$ indicates that it is a \emph{singleton}, i.e., there is at most one key.

% \begin{definition}[Key semantics]
%     We say that a graph $G$ \emph{satisfies a key constraint} $\forall x : \varphi(x) \to \Key^{C} \bar{y} : \psi(x, \bar{y})$ if
%     \begin{enumerate}
%         \item graph $G$ satisfies the formula $\forall \bar{y} : \exists^{\leq 1} x : \varphi(x) \land \psi(x, \bar{y})$,
%          \item if $\mand \in C$ then $G$ satisfies $\forall x : \varphi(x) \to \exists \bar{y} : \psi(x, \bar{y})$, and
%          \item if $\sing \in C$ then $G$ satisfies $\forall x : \varphi(x) \to \exists^{\leq 1} \bar{y} : \psi(x, \bar{y})$.
%     \end{enumerate}
% \end{definition}

% \begin{remark}
%     It can be shown that under fairly weak assumptions for the graph pattern language it will not add expressive power if we allow key expressions of the form  $\forall \bar{x} : \varphi(\bar{x}) \to \Key^{C} \bar{y} : \psi(\bar{x}, \bar{y})$, so where $x$ is replaced with $\bar{x}$, a vector of variables.

%     To understand why this is, consider the following:
%     \begin{enumerate}
%         \item A key of the form $\forall \bar{x} : \varphi(\bar{x}) \to \Key^{\emptyset} \bar{y} : \psi(\bar{x}, \bar{y})$ is equivalent to the key $\forall \bar{x} : \textbf{true} \to \Key^{\emptyset} \bar{y} : \varphi(\bar{x}) \land \psi(\bar{x}, \bar{y})$ 
%         \item The previous formula can be simulated with formulas of the form $\forall x_i : \textbf{true} \to \Key^{\emptyset} \bar{y} : \exists \bar{x}_{\neq i} : \varphi(\bar{x}) \land \psi(\bar{x}, \bar{y})$ where $\bar{x}_{\neq i}$ denotes the vector $\bar{x}$ minus the variable $x_i$. A conjunction of such formulas that contains such a formula for each $x_i$ in $\bar{x}$ would be equivalent.
%         \item To simulate the general key $\forall \bar{x} : \varphi(\bar{x}) \to \Key^{C} \bar{y} : \psi(\bar{x}, \bar{y})$ we introduce the key that simulates $\forall \bar{x} : \varphi(\bar{x}) \to \Key^{\emptyset} \bar{y} : \psi(\bar{x}, \bar{y})$. Depending on $C$ we then add the keys described in the following items.
%         \item If $\mand \in C$ we add to the simulation for every variable $x_i$ in $\bar{x}$ a key $\forall \bar{x} : (\exists \bar{x}_{\neq i} : \varphi(\bar{x})) \to \Key^{\{\mand\}} \bar{y} : (\exists \bar{x}_{\neq i} : \varphi(\bar{x}) \land \psi(\bar{x}, \bar{y}))$. Note that this works because we already have a key to simulate $\forall \bar{x} : \varphi(\bar{x}) \to \Key^{\emptyset} \bar{y} : \psi(\bar{x}, \bar{y})$. So the fact that the mandated pattern is not necessarily connected to the same nodes that the variables in $\bar{x}_{\neq i}$ were bound to in $(\exists \bar{x}_{\neq i} : \varphi(\bar{x}))$ is fine since the earlier key constraint forces them to be the same if they exist.
%         \item If $\sing \in C$ we add the keys that simulate $\forall \bar{y} : \textbf{true} \to \Key^{\emptyset} \bar{x} : \varphi(\bar{x}) \land \psi(\bar{x}, \bar{y})$.
%     \end{enumerate}
%   Summarising, we can see that the only assumptions for the graph pattern language we have used are that (1) it is closed under conjunctions and (2) it is closed under existential quantification.
% \end{remark}

% \begin{remark}
%   It follow from the observation in the previous remark that removing edge variables indeed does not remove expressive power. After all, since edge have no independent identity, allowing them can be simulated if we allow $x$ in the key to be replaced with a vector of two variables. However, as the previous remark showed, this does not add expressive power.
% \end{remark}
