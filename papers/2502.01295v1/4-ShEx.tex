\section{ShEx on common graphs}
\label{sec:shex}

While SHACL is conceptually the simplest of the three languages, ShEx lies at the opposite end of the spectrum. It is an intricate, nested combination of a simple logic for shapes and a powerful formalism (triple expressions) for generating the allowed neighbourhoods. In this work we focus on non-recursive ShEx, where shapes and triple expressions can be nested multiple times, but cannot be recursive.
%\todo[inline]{Ognjen: i believe this is another kind of recursion, and at this point is a bit confusing.}
%\todo[inline]{Filip: Good catch. I replaced "mutually recursive" with "nested". Now it is about the same kind of recursion. }
This allows us to simplify the abstraction without compromising our primary goal of understanding the common features, as neither PG-Schema nor standard SHACL support such a general recursion mechanism.
The abstraction of ShEx over common graphs is based on the treatment of ShEx on RDF triples~\cite{BGP17}. Deviations from standard ShEx are discussed in Appendix~\ref{app:shex-appendix}.

\begin{definition}[shapes and triple expressions]
\label{def:shex-syntax}
ShEx \emph{shapes} $\se$ and
\emph{closed triple expressions} $e$
are defined by the  grammar
\begin{align*}
\se  \gDef & \shexhasvalue(c)
    \gMid
    \ \shextest(\text{$\vtype$})
    \gMid\! \shexneigh{\tte \shexeach \ttclosed }
    \gMid\! \shexneigh{\tte \shexeach \ttopen }
    \gMid\! \se \land \se
    \gMid\! \se \lor \se
    \gMid\! \lnot \se
    \gEnd \\
\tte \gDef &
    \ \varepsilon
    \ \ \gMid \ \
    q.\se
    \ \ \gMid \ \
    \shexinverse q.\se
    \ \ \gMid \ \
    \tte \shexeach \tte
    \ \ \gMid \ \
    \tte \shexone \tte
    \ \ \gMid \ \
    \tte^{*}
  \gEnd \\
  % pushinng op at the end to simplify the parsing
  \ttclosed \gDef &\ (\shexneginv{R})^{*} \gEnd \\
\ttopen \gDef &\  (\shexneginv{R})^{*} \shexeach (\shexneg{Q})^{*}  \gEnd
\end{align*}
where $c \in\Values$,
$\vtype\in\ValueTypes$, $q \in \Predicates \cup \Keys$, and $R,Q\subseteq_{\mathit{fin}}\Predicates \cup \Keys$. We refer to expressions derived from $\tte \shexeach \ttclosed$ and $\tte \shexeach \ttopen$ as \emph{half-open} and \emph{open triple expressions}, respectively.
\end{definition}

The notion of satisfaction for ShEx shapes and the semantics of triple expressions are defined by mutual recursion in Table~\ref{tab:semantics-shex-shape-expr} and Table~\ref{tab:semantics-shex-triple-expr-denotational}.
Triple expressions are used to specify neighbourhoods of nodes and values. They require to consider incoming and outgoing edges separately. For this purpose we decorate incoming edges with ${}^{-}$. Formally, we introduce a fresh predicate $p^{-}$ for each $p\in\Predicates$ and a fresh key $k^{-}$ for each $k\in\Keys$. We  let $\Predicates^{-}= \left\{p^{-} \mid p\in \Predicates\right\}$,
$\Keys^{-}= \left\{k^{-} \mid k\in \Keys\right\}$,
$\Triples^{-} = \Nodes \times \Predicates^{-} \times \Nodes \cup \Values \times \Keys^{-} \times \Nodes$, and define $\neigh^{\pm}_\graph(v) \subseteq \Triples \cup\Triples^{-}$ as
\[ \left \{ (v, p, v') \mid (v, p, v') \in \graph\right\} \cup \left\{(v, p^{-}, v') \mid (v', p, v) \in \graph \right\}.\] Compared to $\neigh_\graph(v)$, apart from flipping the incoming edges and marking them with ${}^{-}$, we also represent each loop $(v,p,v)$ twice: once as an outgoing edge $(v,p,v)$ and once as an incoming edge $(v,p^{-},v)$.
In Table~\ref{tab:semantics-shex-triple-expr-denotational}, we treat $\shexneg{Q}$ and  $\shexneginv{R}$ as triple expressions. So, the rule for $e^{*}$ gives semantics to $(\shexneg{Q})^{*}$ and  $(\shexneginv{R})^{*}$, and the rule for $e_1\shexeach e_2$ gives semantics to open and half-open triple expressions. In Table~\ref{tab:semantics-shex-shape-expr}, $f$ is an  open or half-open triple expression.

Closed triple expressions $\tte$ define neighbourhoods that use only a finite number of predicates and keys (also called \emph{closed} in ShEx terminology) and cannot be directly used in shape expressions.
Half-open triple expressions $\te\shexeach (\shexneginv{R})^{*}$ allow any \emph{incoming} triples whose predicate or key is not in  $R$. Open triple expressions $\te\shexeach (\shexneginv{R})^{*} \shexeach (\shexneg{Q})^{*}$ additionally allow any \emph{outgoing} triples whose predicate or key is not in $Q$.
%
Let $\shexallte = \varepsilon \shexeach (\shexneginv{\emptyset})^{*} \shexeach {(\shexneg{\emptyset})}^{*}$. Then $\shexallte$ describes all possible neighbourhoods, and $\shextop$ is satisfied in every node and in every value of every graph.

\begin{example}
The ShEx shape $\shexneigh{p.\varphi_1 \shexeach p.\varphi_2;\shexallte}$ specifies nodes with at least two different $p$-successors, one satisfying $\varphi_1$ and one satisfying $\varphi_2$. Note that this is different from SHACL shape $\exists p. \varphi_1 \land \exists p.\varphi_2$ which says that the node has a $p$-successor satisfying $\varphi_1$ and a $p$-successor satisfying $\varphi_2$, but they might not be different.
\end{example}


\begin{example}
    Assume that integers and strings are represented by $\mathbbm{int},\mathbbm{str}\in \ValueTypes$.
    The ShEx shape
    \[ \shexneigh{\exemail.\shextest(\mathbbm{str}) \,\shexeach\, (\excard.\shextest(\mathbbm{int}) \,\shexone\, \varepsilon) \,\shexeach\, (\shexneginv{\emptyset})^{*}}
    \]
    specifies nodes with an $\exemail$ property with a string value, an optional $\excard$ property with an integer value, arbitrary incoming edges, and no other properties or outgoing edges.
    %
    To allow additional properties and outgoing edges, we replace $ (\shexneginv{\emptyset})^{*}$ with $\shexallte$.
    The modified shape can be rewritten using $\land$ as
    \[ \shexneigh{\exemail.\shextest(\mathbbm{str}) \,\shexeach\,\shexallte} \land \shexneigh{(\excard.\shextest(\mathbbm{int}) \,\shexone\, \varepsilon) \,\shexeach\, \shexallte}
    \] but the original shape cannot be rewritten in a similar way.
\end{example}

%Note also that the $\shexeach$ and $\shexone$ operators are associative and commutative.

\begin{definition}[ShEx Selectors]
\label{def:shex-selector}
A ShEx selector is a ShEx shape  of a restricted form, defined by the grammar
\begin{align*}
\shexsel \gDef & \shexhasvalue(c)
    \gMid \shexneigh{q.\shexhasvalue(c) \shexeach \shexallte}
    %\gMid \shexneigh{\shexinverse p.\shexhasvalue(c) \shexeach \shexallte}
    \gMid %\\
    \shexneigh{q.\shextop \shexeach \shexallte}
    \gMid \shexneigh{\shexinverse q.\shextop \shexeach \shexallte}
    \gEnd
\end{align*}
where $q \in \Predicates \cup \Keys$ and
%$c \in \Nodes \cup \Values$.
$c \in \Values$.
\end{definition}

Following Section~\ref{ssec:shapes}, a \emph{ShEx schema} $\shexschema$ is a set of pairs of the form $(\shexsel, \se)$ where $\se$ is a ShEx shape and $\shexsel$ is a ShEx selector.


\begin{table}
  \caption{Satisfaction of ShEx shapes.}
  \label{tab:semantics-shex-shape-expr}
  \centering
  \begin{tabular}{cl}
    \toprule
    $\se$ & $\graph, v \models \se\ $ for $v\in\Nodes\cup\Values$ \\
    \midrule
    $\shexhasvalue(c)$ & $v = c$ \\
    $\shextest(\vtype)$ & $v \in \sem{\vtype}$ \\
    $\shexneigh{f}$ & $\neigh^\pm_\graph(v) \in \sem{f}_v^\graph$ \\
    $\se_1 \land \se_2$ & $\graph, v \models \se_1$ and $\graph, v \models \se_2$ \\
    $\se_1 \lor \se_2$ & $\graph, v \models \se_1$ or $\graph, v \models \se_2$ \\
    $\lnot \se$ & not $\graph, v \models \se$ \\
    \bottomrule
  \end{tabular}
\end{table}


\begin{table}
  \caption{Semantics of triple expressions.}
  \label{tab:semantics-shex-triple-expr-denotational}
  \centering
  \begin{tabular}{cl}
    \toprule
    $\te$ & $\sem{\te}_v^\graph \subseteq 2^{\Triples \cup \Triples^{-}}$ \\[2pt]
    \midrule
    $\varepsilon$ & $\{\emptyset\}$ \\[2pt]
    $q.\se$ & $\big\{\{(v, q, v')\} \subseteq \Triples \ \big|\  \graph, v' \models \varphi\big\}$ \\[2pt]
    $\shexinverse q.\se$ & $\big\{\{(v, \shexinverse q, v')\} \subseteq \shexinverse\Triples\ \big|\ \graph, v' \models \varphi\big\}$  \\[2pt]
    $\te_1 \shexeach \te_2$ & $\left\{ T_1 \cup T_2 \ \middle|\  T_1\in \sem{\te_1}_v^\graph\,,\  T_2\in \sem{\te_2}_v^\graph\,,\ T_1\cap T_2 = \emptyset \right\}$ \\[2pt]
    $\te_1 \shexone \te_2$ & $\sem{\te_1}_v^\graph \cup \sem{\te_2}_v^\graph$ \\[2pt]
    $\te^{*}$
    & $ \{\emptyset\} \cup \bigcup_{n=1}^\infty \left \{\, T_1 \cup \dots \cup T_n \ \bigg| \begin{array}{l}
    T_1, \dots, T_n \in \sem{\te}_v^\graph \text{ and } \\
    %T_1, \dots, T_n \text{ are pairwise disjoint}
    T_i\cap T_j = \emptyset \text{ for all } i\neq j
    \end{array} \!
    \right\}$ \\[2pt]
    $\shexneg{Q}$ & $\big\{ \{(v, q , v')\} \subseteq  \Triples\ \big|\  q\notin Q\big\}$ \\[2pt]
    $\shexneginv{R}$ & $\big\{ \{(v, \shexinverse q , v')\} \subseteq  \shexinverse\Triples\ \big|\  q\notin R\big\}$
    \\
    \bottomrule
  \end{tabular}
\end{table}
%\ognjen{In table 4, why not just $(v, q, v')\in \Triples$?}
%%%% Because we want sets of neighbourhoods, not sets of edges.
%\ognjen{In table 4, and tables later I find $\subseteq 2^{\Triples \cup \Triples^{-}}$ confusing a bit, since it may give an impression that is imposing a restriction but actually it holds the definition... maybe its just me, so feel free to ignore my comment}

In what follows, for a positive integer $n$, we write $\te^n$ for $ \te \shexeach \ldots \shexeach \te$ where  $\te$ is repeated $n$ times,
$\te^{\leq n}$ for $\varepsilon \shexone \te^{1} \shexone \ldots \shexone \te^{n}$, and $\te^{\geq n}$ for $\te^{n} \shexeach \te^{*}$.
For a closed triple expression $\tte$, we let $\shexneighopen{\tte} = \shexneigh{\tte \shexeach (\shexneginv{R})^{*} \shexeach (\shexneg{Q})^{*}}$ where $Q$ is the set of predicates and keys that appear \emph{directly} in $\tte$ (as opposed to appearing in $\varphi$ for a sub-expression $q.\se$ of $\tte$) and $R$ is the set of predicates and keys whose inversions appear directly in $\tte$. For instance, if $\tte = p.\shexneigh{q.\shexneigh{\top}\shexeach\shexinverse{p}.\shexneigh{\top}}$, then $Q = \{p\}$ and $R=\emptyset$.

% \begin{itemize}
% % \item $\te^n$ for some positive integer $n$ denotes the triple expression $ \te \shexeach \ldots \shexeach \te$ where  $\te$ is used $n$ times,
% % \item $\te^{\leq n}$ as a short-hand for $\varepsilon \shexone \te^{1} \shexone \ldots \shexone \te^{n}$ and $\te^{\geq n}$ as a short-hand for $\te^{n} \shexeach \te^{*}$,
% \item
% \end{itemize}
\begin{example}
Let us now see how the concrete constraints from Example~\ref{ex:constraint-desc} can be handled in ShEx.
\begin{align*}
& \shexneigh{
\shexinverse{\excard}.\shextop \shexeach \shexallte} \Rightarrow   \test(\mathbbm{int})  & \mbox{(C1)} \\
& \shexneigh{\exowns.\shextop \shexeach \shexallte} \Rightarrow \shexneigh{ \exemail.\shextop%^{\geq 1}
 \shexeach \shexallte} & \mbox{(C2)} \\
&    \shexneigh{\shexinverse{\exemail}.\shextop \shexeach \shexallte} \Rightarrow \shexneighopen{ (\exemail^{-}.\shextop)^{\leq 1}   } & \mbox{(C3)} \\
& \shexneigh{\excard . \shextop \shexeach \shexallte}  \Rightarrow   \shexneighopen{\exprivileged.\neg \shexhasvalue(\mathit{true})} \lor  &  \\
& \qquad \shexneighopen{(\exaccess^{-}.\shexneighopen{\exprivileged.\shexhasvalue(\mathit{true})})^{*} }  & \mbox{(C4)} \\
&  \shexneigh{\exemail.\shextop \shexeach \shexallte}  \Rightarrow \shexneighopen{(\exaccess.\shextop)^{\leq 5}} & \mbox{(C5)}
\end{align*}
\end{example}

We next show a more complex example, which illustrates the power of ShEx that is not readily available in SHACL or PG-Schema.

\begin{example} \label{ex:SheXCounting}
    Suppose that we want to express the following constraint on each user who owns an account: the number of accounts to which the user has access is greater or equal to the number of accounts that the user owns. We can do this in ShEx as follows:
    \begin{flalign*}
    & \shexneigh{\exowns.\shextop \shexeach \shexallte} \Rightarrow  \\ & \shexneighopen{ (\exaccess.\shextop )^{*}\shexeach (\exowns.\shextop \shexeach \exaccess.\shextop)^{*}}
    \end{flalign*}

    Similarly to the above (yet more abstractly) consider the following requirement:  for the node $c$, the number of outgoing $p$-edges  is \emph{equal} to the number of outgoing $q$-edges. %This can be expressed via a ShEx schema $\shexschema^{eq}=\{\shexhasvalue(c)\Rightarrow \shexneighopen{  (p.\shextop \shexeach q.\shextop)^{*}}\}$.In Appendix \ref{app:indistinguishabilitySHACL} we prove that $\shexschema^{eq}$ cannot be expressed in SHACL.
This can be expressed in ShEx using $\shexhasvalue(c)\Rightarrow \shexneighopen{  (p.\shextop \shexeach q.\shextop)^{*}}$ but  cannot be expressed in SHACL (see Appendix \ref{app:indistinguishabilitySHACL})
   \end{example}


Finally, let us see why ShEx and SHACL count differently.

\begin{example}%[ShEx counts edges]
\label{ex:shex-counts-edges}
    The following SHACL schema ensures that from every node with an outgoing $\exaccess$-edge, exactly two nodes are accessible via a $\exaccess$-edge or an $\exowns$-edge:
    \[
    \exists \exaccess \Rightarrow \exists^{= 2} (\exaccess \cup
    \exowns).\top
    \]
    Here $\exists^{=n} \pexpr.\varphi$ is a shorthand for $\exists^{\leq n} \pexpr.\varphi \land \exists^{\geq n} \pexpr.\varphi$.
    For instance, in Figure~\ref{fig:example-shex-counts-edges}, the graph on the right is valid, whereas the one on the left is not.
    The same constraint cannot be expressed in ShEx because ShEx cannot distinguish these two graphs (see Appendix~\ref{app:indistinguishabilityShEx}).
    The reason is that ShEx triple expressions count triples adjacent to a node, whereas SHACL and PG-Schema count nodes on the opposite end of such triples.
    This makes counting edges simpler in ShEx: the ShEx shape $\shexneigh{(p. \shextop \shexone q.\shextop)^{2} \shexeach (\shexneginv{\emptyset})^{*}}$ allows exactly two outgoing edges labelled $p$ or $q$. In SHACL this is written as $(\exists^{=2} p. \top \land \exists^{=0} q.\top) \lor (\exists^{=2} q.\top \land \exists^{=0} p.\top) \lor (\exists^{=1} p.\top \wedge \exists^{=1} q.\top)$.
    %%% THIS PROPOSAL IS INCORRECT, BECAUSE IT OPENS OUTGOING EDGES.
    %\textcolor{red}{As a consequence, counting edges is simpler in ShEx: the triple expression $(p. \shextop \shexone q.\shextop)^{2} \shexeach \shexallte$ says that a node should have exactly two outgoing edges labelled $p$ or $q$. The same constraint in SHACL gives $(\exists^{=2} p. \top \land \exists^{=0} q.\top) \lor (\exists^{=2} q.\top \land \exists^{=0} p.\top) \lor (\exists^{=1} p.\top \wedge \exists^{=1} q.\top)$}.
\end{example}

\begin{figure}[t]
\resizebox{.9\linewidth}{!}{
  \includegraphics{shex-indisting-example.pdf}
}
\Description{A diagram showing two graphs indistinguishable by ShEx}
\caption{Two graphs indistinguishable by ShEx}
\label{fig:example-shex-counts-edges}
\end{figure}



%%% Local Variables:
%%% mode: latex
%%% TeX-master: "Article"
%%% End:
