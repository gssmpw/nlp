\section{Distilling the common data model}
\label{sec:appendix-foundations}

\todo[inline]{Remember to add what we promised to the reviewers. @Filip}

In this section we discuss the relationship between common graphs and the standard data models of the three schema formalisms formalisms---RDF and property graphs. 


\subsection{Comparison with RDF}
\label{app:sec-foundations-comparison-rdf}

As explained in Section~\ref{sec:prl}, common graphs can be naturally seen as finite sets of triples from 
$\Triples = \left(\Nodes\times\Predicates\times\Nodes\right) \;\cup\; \left(\Nodes\times\Keys\times\Values\right)$, with  $(E,\rho)$ corresponding to 
$E \;\cup\; \{(u,k,v) \mid \rho(u,k) = v\}$. 

Unlike in RDF, a common graph may contain at most one tuple of the form $(u,k,v)$ for each $u\in \Nodes$ and $k\in\Keys$. This reflects the assumption that  properties are single-valued, which is present in the property graph data model. 

In the RDF context, one would assume the following:
\begin{itemize}
\item $\Nodes \subseteq \IRIs \cup \Blanks$,
\item $\Predicates \subseteq \IRIs$, 
\item $\Keys \subseteq \IRIs$,
\item $\Values = \Literals$.
\end{itemize}
However, the common graph data model does not refer to $\IRIs$, $\Blanks$, and $\Literals$ at all, because these are not part of the property graph data model. 

In contrast to the RDF model, but in accordance with the perspective commonly taken in databases, both values and nodes are atomic. For nodes we completely abstract away from the actual representation of their identities. We do not even distinguish between $\IRIs$ and $\Blanks$. An immediate consequence of this is that schemas do not have access to any information about the node other than the triples in which it participates. In particular, they cannot compare nodes with constants. This is a significant restriction with respect to the RDF data model, but it follows immediately from the same assumption made in the property graph data model. On the positive side, this aspect is entirely orthogonal to the main discussion in this paper, so eliminating it from the common data model does not oversimplify the picture. 

For values we take a more subtle approach: we assume a set $\ValueTypes$ of value types, with each $\vtype\in\ValueTypes$ representing a set $\sem{\vtype}\subseteq\Values$. This captures uniformly data types, such as \texttt{integer} or \texttt{string}, and user-defined checks, such as interval bounds for numeric values or regular expressions for strings.
On the other hand, the common graph data model does not include any binary relations over values, such as an order.  

\todo[inline]{Iovka: The section Comparison with Proprety graphs includes a hint on how a common graph could encode a general property graph. Similar encoding could be considered also for RDF: keys \texttt{iri} and \texttt{blank} for node identities, nodes with special key \texttt{value} to represent literal values (which would also lift the constraint of not being able to give to different literal values for the same predicate). We haven't considered such encoding approach in the present paper, but it seems to be a valid question that might be worth investigating.}

\todo[inline]{After discussion, we decided to write a small paragraph about the possibility to encode RDF in common graphs, in the spirit of the similar paragraph for property graphs.}
%\todo[inline]{Maxime: Discussing these ``encoding'' or ``translation'' approaches might be an interesting Appendix onto itself. 

%One issue I currently see is that while property graphs might be ``easily'' encoded into the common data model, RDF graphs might not be. For example, how would we handle multiple values for a given key?

%Another subtle point is that our common graph datamodel treats values are a ``navigational'' object, a little bit like nodes. This is important for RDF/SHACL to express for example the key constraint. You could argue that this is more of a ``schema thing'' than it is a ``data model thing''. I am unsure. It might come down to the idea of a ``node'' in a graph. In RDF speak, values are nodes, while in LPGs they are not.}

%We note that in common graphs, there are no simple mechanisms to encode classes of objects and that this is a difference to the three formalisms that we compare, which do expose mechanisms to easily expressing classes in some form (in the form of node labels in Property Graphs, and classes in RDF). We believe our decision to omit this is justified because ... 



\subsection{Comparison with property graphs}
\label{sect:PGCGComparison}

Let us recall the standard definition of property graphs \cite{ABDF23}. 

\begin{definition}[Property graph]
A \emph{property graph} is a tuple $(N, E, \pi, \lambda, \rho)$ such that 
\begin{itemize}
\item $N$ is a finite set of nodes;
\item $E$ is a finite set of edges, disjoint from $N$; 
\item $\pi : E \to (N \times N)$ maps edges to their source and target;
\item $\lambda : (N \cup E) \to 2^{\Predicates}$ maps nodes and edges to finite sets of labels;
\item $\rho : (N \cup E)\times \Keys \pto \Values$ is a finite-domain partial function mapping element-key pairs to values.
\end{itemize}
\end{definition}

A common graph $G = (E', \rho')$ can be easily represented as a property graph by letting
\begin{itemize}
    \item $N = \nodes(G)$,
    \item $E = E'$,
    \item $\pi = \{ (e, (v_1, v_2)) \mid e = (v_1, p, v_2) \in E \}$,
    \item $\lambda = \{ (e, \{ p \}) \mid e = (v_1, p, v_2) \in E \} \cup \{ (v, \emptyset) \mid v \in N \}$, and
    \item $\rho = \rho'$.
\end{itemize}

It is possible to characterise exactly the property graphs that are such representations of common graphs. These are the property graphs $(N, E, \pi, \lambda, \rho)$ for which it holds that:
\begin{enumerate}
    \item $\lambda(v) = \emptyset$ for all $v \in N$, and $\lambda(e)$ is a singleton for all $e \in E$,
    \item there cannot be two distinct edges $e_1, e_2 \in E$ such that $\pi(e_1) = \pi(e_2)$ and $\lambda(e_1) = \lambda(e_2)$, and
    \item $\rho(e, k)$ is undefined for all $e \in E$, $k \in \Keys$.
\end{enumerate}



So, common graphs can be interpreted as restricted property graphs: no labels on nodes, single labels on edges, no parallel edges with the same label, and no properties on edges. All these restrictions are direct consequences of the nature of the RDF data model. 

While these restrictions seem severe at a first glance, the resulting data model can actually easily simulate unrestricted property graphs: labels on nodes can be simulated with the presence of corresponding keys, edges can be materialised as nodes if we need properties over edges or parallel edges with the same label. This means not only that common graphs can be used without loss of generality in expressiveness and complexity studies, but also that the corresponding restricted property graphs are flexible enough to be usable in practice, while additionally guaranteeing interoperability with the RDF data model. 

\subsection{Class information}
\todo[inline]{Dominik: it might be helpful to include examples or more detailed explanations on how to use designated predicates and keys to simulate class memberships and hierarchies. Providing these examples would clarify how common graphs can indirectly support class information despite not having direct mechanisms for it.}
The common graph data model does not have direct support for class information. The reason for this is that RDF and property graphs handle class information rather differently. In RDF, both class and instance information is part of the graph data itself:  classes are elements of the graph, subclass-superclass relationships are represented as edges between classes, and membership relationships are represented as edges between elements and classes. In property graphs, the membership of a node in a class is indicated by a label put on the node. A node can belong to many classes, but the only way to say that class $A$ is a subclass of class $B$ is to ensure in the schema that each node with label $A$ also has label $B$. That is, 
\begin{itemize}
\item in property graphs class membership information is available locally in a node, but consistency must be ensured by the schema, 
\item in RDF, obtaining class membership information requires navigating in the graph, but consistency is for free.
\end{itemize}
Clearly, both approaches have their merits, but when passing from one to the other data needs to be translated. This means that we cannot pick one of these approaches for the common data model while keeping it a natural submodel of both RDF and property graphs. Therefore, to reduce the complexity of this study, we do not include any dedicated features for supporting class information in our common data model. 
Note, however, that common graphs can support both these approaches indirectly: designated predicates can be used to represent membership and subclass relationships, and keys with a dummy value can simulate node labels. 
