\section{Conclusions}

We provided a formal and comprehensive comparison of the three most prominent schema languages in the Semantic Web and Graph Database communities: SHACL, ShEx, and PG-Schema. Through painstaking discussions within our working group, we managed to (1) agree on a common data model that captures features of both RDF and Property Graphs and (2) extract, for each of the languages, a core that we mutually agree on, which we define formally. Moreover, the definitions of (the cores of) each of the schema languages on a common formal framework allows readers to maximally leverage their understanding of one schema language in order to understand the others. Furthermore, this common framework allowed us to extract the Common Graph Schema Language, which is a cleanly defined set of functionalities shared by SHACL, ShEx, and PG-Schema. This commonality can serve as a basis for future efforts in integrating or translating between the languages, promoting interoperability in applications that rely on heterogeneous data models. For example, we want to investigate recursive ShEx and more expressive query languages for PG-Schema more deeply.



% Through the introduction of a uniform framework, we offered precise definitions of the core components of these languages, allowing for a structured comparison of their capabilities and limitations. The introduction of a Common Data Model has facilitated the alignment of RDF and Property Graph models, enabling us to systematically embed the key features of both graph paradigms into a unified representation.

% Our comparison has identified both the shared functionalities and distinctive aspects of each language. We have shown that while all three languages serve the purpose of expressing constraints and validating graph-structured data, they achieve these objectives through different mechanisms. SHACL, for instance, provides an extensive set of validation rules for RDF data, whereas ShEx focuses on shape-based expressions, and PG-Schema applies to property graphs, introducing specific constraints suited for that model. The mathematical abstractions we have presented ensure a clean and unambiguous understanding of these languages, and the wide array of examples we provided throughout the paper has illustrated the types of constraints and conditions that each language can express.

% Furthermore, our work has established a common core of functionalities shared by SHACL, ShEx, and PG-Schema, a set of features that underscores their compatibility in certain contexts. This commonality can serve as a basis for future efforts in integrating or translating between the languages, promoting interoperability in applications that rely on heterogeneous data models.