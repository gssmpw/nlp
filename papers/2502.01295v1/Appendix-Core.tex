\section{More on the core}
\label{sec:appendix-core}

In this section we prove Proposition~\ref{prop:core}. Recall that common shapes are defined by the grammar 
\begin{align*}
\varphi  \gDef  & 
 \exists\,\pexpr
\gMid \exists^{\leq n} \, \pexpr_1
\gMid \exists^{\geq n} \, \pexpr_1 \gMid 
\exists\, \ntype \land \not\exists\, \lnot P
\gMid \varphi \land \varphi \gEnd\\
\ntype \gDef &\closedRT{}\  \gMid\  \closedRT{k : \vtype} \ \gMid\   \ntype \tAnd \ntype  \ \gMid \  \ntype \tOr \ntype  \gEnd\\
\pexpr_0 \gDef & 
\keyIsVal{k}{c} \gMid 
\lnot \keyIsVal{k}{c} \gMid 
\ntype\tAnd\top\gMid \lnot (\ntype\tAnd\top) \gMid \pexpr_0 \cdot \pexpr_0 \gEnd\\
\pexpr_1 \gDef &  \pexpr_0  \cdot p \cdot
\pexpr_0 
\gMid  \pexpr_0  \cdot p^{-} \cdot
\pexpr_0  \gMid \pexpr_0\cdot k \gMid k^{-}\cdot \pexpr_0 \gEnd\\
\ppexpr \gDef & \pexpr_0  \gMid p 
\gMid {\ppexpr}^{-} \gMid \ppexpr \cdot \ppexpr 
\gMid \ppexpr \cup \ppexpr \gEnd \\
\pexpr \gDef & \ppexpr \gMid \ppexpr \cdot k \gMid k^{-}\cdot \ppexpr \gMid k^{-}\cdot \ppexpr\cdot k' \gEnd
\end{align*}
where $n \in \mathbb{N}$, $P\subseteq_{\mathit{fin}} \Predicates$, $k,k'\in\Keys$, $c\in\Values$, and $p\in\Predicates$. We will refer to  PG-path expressions defined by the nonterminal $\pexpr_0$ in the grammar as  \emph{filters}.


%First, note that  $\exists^{\geq 0} \pexpr_0\cdot k$ is trivially satisfied in each node and  $\exists^{\geq n} \pexpr_0\cdot k$ is equivalent to $\exists\, \pexpr_0\cdot k$ for all $n>0$. Hence, we can assume that core shapes do not contain subexpressions of the form $\exists^{\geq n} \pexpr_0\cdot k$.

% Second, we can assume without loss of generality that in each core shape of the form $\exists\, \pexpr$, the PG-path expression $\ppexpr$ underlying $\pexpr$ is a union of concatenations of filters and atomic path expressions of the form $p$ or $p^{-}$.

%Third, we can assume without loss of generality that each closed content type has the form $\ntype_1 \tOr \ntype_2 \tOr \dots \tOr \ntype_n$, where each $\ntype_i$ has the form  $\closedRT$ or $\closedRT{k_1:\vtype_1}\tAnd\closedRT{k_2:\vtype_2}\tAnd \dots \tAnd \closedRT{k_m :\vtype_m}$.

%From now on we only consider core schemas normalized as described above. 

The following two subsections describe the translations of common schemas to SHACL and ShEx. The translations are very similar but we include them both for the convenience of the reader.

\subsection{Translation to SHACL}

\todo[inline]{Cem: Explain here that open type is just a short-hand for $\ntype \&  \top$. It could also be mentioned directly in the Core section, that the core language only allows for open types (or arbitrary content types with a ``guard'' to ensure closedness). }

\begin{lemma}
\label{lem:contents-shacl}
For each open content type $\ntype$ there is a SHACL shape $\varphi_{\ntype}$ such that  $\graph, v\models \varphi_{\ntype}$ iff $\rho(v) \in \sem{\ntype}$ for all  $\graph=(E,\rho)$ and $v\in\nodes(\graph)$.
\end{lemma}

\begin{proof}
For the content type $\top$ the corresponding SHACL shape is $\top$.
For a content type of the form \[\closedRT{k_1:\vtype_1}\tAnd\closedRT{k_2:\vtype_2}\tAnd \dots \tAnd \closedRT{k_m :\vtype_m}\tAnd \top\,,\] the corresponding SHACL shape is \[\exists\, k_1. \test(\vtype_1) \land \exists\, k_2. \test(\vtype_2) \land \dots \land \exists\, k_m. \test(\vtype_m)\,.\]

\todo[inline]{Cem: maybe a technical lemma that shows why/how every content type can be represented in the needed \emph{normal form} would make the proofs of this section easier to follow. Feels strange to just leave a (mostly, but not entirely trivial) claim open in a proof.}

Finally, every open content type different from $\top$ can be expressed as 
\[(\ntype_1 \tOr \dots \tOr \ntype_\ell)\tAnd \top\,,\]
where each $\ntype_i$ is a  content type of the form $\closedRT{k_1:\vtype_1}\tAnd\closedRT{k_2:\vtype_2}\tAnd \dots \tAnd \closedRT{k_m :\vtype_m}$ for some $m$.
The corresponding SHACL shape is 
\[\varphi_1 \lor \dots \lor \varphi_\ell\,,\]
where $\varphi_i$ is the SHACL shape corresponding to the content type $\ntype_i \tAnd \top$.
\end{proof}


\begin{lemma}
\label{lem:filter-shacl} For each filter $\pexpr_0$ there is a SHACL shape $\varphi_{\pexpr_0}$ such that  $\graph, v\models \varphi_{\pexpr_0}$ iff $(v,v) \in \sem{\pexpr_0}^\graph$ for all  $\graph$ and $v\in\nodes(\graph)$. 
\end{lemma}

\begin{proof} By Lemma~\ref{lem:contents-shacl}, the claim holds for $\pexpr_0=\ntype\tAnd \top$. For $\{k:c\}$ the corresponding SHACL shape is $\exists k.\hasvalue(c)$. 
As SHACL shapes are closed under negation, the claim holds for $\lnot \{k:c\}$ and $\lnot(\ntype\tAnd \top)$.
Finally, concatenations of filters correspond to conjunctions of shapes, so the claim follows because SHACL shapes are closed under conjunction. 
\end{proof}

\begin{lemma}
\label{lem:paths-shacl}
For each common shape of the form $\exists\,\pexpr$ there is a SHACL shape $\varphi_{\exists\pexpr}$ such that  $\graph, v\models \varphi_{\exists\pexpr}$ iff $\graph,v\models \exists\, \pexpr$ for all $\graph$ and $v\in\nodes(\graph) \cup \values(\graph)$. 
\end{lemma}

\begin{proof}
Let us first look at common shapes of the form $\exists\,\pexpr$ where $\pexpr$ is a concatenation of filters and atomic path expressions of the form  $p$, $p^{-}$, $k$, or $k^{-}$. Without loss of generality we can assume that the concatenation ends with a filter or with $k$.
We proceed by induction on the length of the concatenation. The base cases are $\exists \pi_0$ and $\exists k$, which correspond to $\varphi_{\pi_0}$ (Lemma~\ref{lem:filter-shacl}) and $\exists k.\top$. 
For $\exists\, \pexpr_0\cdot\pexpr$ we can take $\varphi_{\pexpr_0} \land \varphi_{\exists \pi}$. For $\exists\, p\cdot\pexpr$ we can take $\exists p.\varphi_{\exists \pi}$, and similarly for $\exists\, p^{-}\cdot\pexpr$ and $\exists\, k^{-}\cdot\pexpr$. 

The general case follows because SHACL shapes are closed under union. Indeed, because our PG-path expressions are star-free, we can assume without loss of generality that in each common shape of the form $\exists\, \pexpr$, the PG-path expression $\ppexpr$ underlying $\pexpr$ is a union of concatenations of filters and atomic path expressions of the form $p$ or $p^{-}$. Then, for \[\exists\, k^{-}\cdot(\pexpr^1 \cup \dots \cup\pexpr^m)\cdot k'\] we can take \[\varphi_{\exists k^{-}\cdot\pexpr^1\cdot k'} \lor \dots \lor \varphi_{\exists k^{-}\cdot\pexpr^m\cdot k}\,.\] Simiarly for $\exists\, k^{-}\cdot(\pexpr^1 \cup \dots \cup\pexpr^m)$, $\exists\, (\pexpr^1 \cup \dots \cup\pexpr^m)\cdot k'$, and $\exists\, (\pexpr^1 \cup \dots \cup\pexpr^m)$.
\end{proof}

\todo[inline]{Cem: in \Cref{lem:paths-shacl}, it seems strange to me to use natural induction on the length of paths (where the use of IH in the step case is left very implicit) and not opt for the more obvious choice of a structural induction on paths. Space saving measure? }


\begin{lemma}
\label{lem:shapes-shacl}
For each common shape $\varphi$ there is a SHACL shape $\hat \varphi$ such that  $\graph, v\models \varphi$ iff $\graph,v\models \hat \varphi$ for all $\graph$ and $v\in\nodes(\graph)  \cup \values(\graph)$. 
\end{lemma}

\begin{proof} 
Because SHACL shapes are closed under conjunction, it suffices to prove the claim for the atomic common shapes of the forms $\exists\, \pexpr$, $\exists^{\leq n}\pexpr_1$, $\exists^{\geq n}\pexpr_1$, and $\exists \ntype \land \not\exists\lnot P$. The first case follows from Lemma~\ref{lem:paths-shacl}. 

Let us look at common shapes of the form
$\exists^{\geq n}\pexpr_1$. If $n=0$ we can simply take $\top$. Suppose $n>0$. Then, for \[\exists^{\geq n} \pexpr_0\cdot p\cdot \pexpr'_0\] we can take \[\varphi_{\pexpr_0} \land \exists^{\geq n}p. \varphi_{\pexpr'_0},\] and similarly for $\exists^{\geq n}\pexpr_0\cdot p^{-}\cdot \pexpr'_0$, $\exists^{\geq n}\pexpr_0\cdot k^{-}\cdot \pexpr'_0$, and $\exists^{\geq n}\pexpr_0\cdot k$ (using $\top$ instead of $\varphi_{\pexpr'_0}$). 

Next, we consider common shapes of the form 
$\exists^{\leq n}\pexpr_1$. For \[\exists^{\leq n} \pexpr_0\cdot p\cdot \pexpr'_0\] we can take \[\lnot\varphi_{\pexpr_0} \lor \exists^{\leq n}p. \varphi_{\pexpr'_0},\] and similarly for $\exists^{\leq n}\pexpr_0\cdot p^{-}\cdot \pexpr'_0$, $\exists^{\leq n}\pexpr_0\cdot k^{-}\cdot \pexpr'_0$, and $\exists^{\leq n}\pexpr_0\cdot k$ (again, using $\top$ instead of $\varphi_{\pexpr'_0}$).

Finally, let us consider a common shape of the form $\exists\, \ntype \land \not\exists\,\lnot P$. Suppose first that \[\ntype = \closedRT{}\,.\] Then, the corresponding SHACL shape is simply \[\closed(P)\,.\] Next, suppose that \[\ntype = \closedRT{k_1:\vtype_1}\tAnd \dots \tAnd \closedRT{k_m :\vtype_m}\,.\] 
Then, the corresponding SHACL shape is 
\[
\exists k_1.\test(\vtype_1) \land \dots \land \exists k_m.\test(\vtype_m) \land \closed\big(\{k_1, \dots,k_m\} \cup P\big)\,.\]
In general, as in Lemma~\ref{lem:contents-shacl}, we can assume that 
\[\ntype = \ntype_1 \tOr \dots \tOr \ntype_m\] where each $\ntype_i$ is of one of the two forms considered above. The corresponding SHACL shape is then 
\[\varphi_1 \lor \dots \lor \varphi_m\] where $\varphi_i$ is the SHACL shape corresponding to $\exists \ntype_i \land \not\exists\lnot P$, obtained as described above. 
\end{proof}

\begin{lemma}
\label{lem:schemas-shacl}
For every common schema there is an equivalent SHACL schema.  
\end{lemma}

\begin{proof} 
Let $\schema$ be a common schema. We obtain an equivalent SHACL schema $\schema'$ by translating each  $(\sel, \varphi) \in \schema$ to $(\sel', \varphi')$ such that for all $\graph$ and $v\in \Nodes\cup\Values$, 

\begin{center}
 $\graph, v \models \sel$ implies $\graph, v \models \varphi$ 
 
 iff 
 
 $\graph, v \models \sel'$ implies $\graph, v \models \varphi'$.
 \end{center}
 Recall that $\sel$ is a common shape of one of the following forms:
\[ 
\exists\, k \,,\;
\exists\, p \cdot \pexpr\,, \;
\exists\, p^{-}\!\cdot \pexpr \,,\;
\exists\, \{k:c\}\cdot \pexpr \,, \;
\exists\, \big(\{k:\vtype\}\tAnd\top\big)\cdot \pexpr \,, \;
\exists\, k^{-}\!\cdot \pexpr\,.
\] 
For $\sel'$ we take, respectively, 
\[ 
\exists\, k.\top \,,\quad
\exists\, p.\top\,, \quad 
\exists\, p^{-}.\top \,,\quad 
\exists\, k.\top\,, \quad 
\exists\, k.\top\,, \quad 
\exists\, k^{-}.\top\,.
\]
For $\varphi'$ we take $\lnot \varphi_{\sel} \lor \hat\varphi$ where $\varphi_{\sel}$ is obtained using Lemma~\ref{lem:paths-shacl}
 and $\hat\varphi$ is obtained using Lemma~\ref{lem:shapes-shacl}. 
 \end{proof}
\todo[inline]{Cem: suggestion last line of proof of \Cref{lem:schemas-shacl},

For $\varphi'$ we take $\lnot \varphi_{\sel} \lor \hat\varphi$ where $\varphi_{\sel}$ is obtained using Lemma~\ref{lem:paths-shacl} {\color{red} on $\sel$}
 and $\hat\varphi$ is obtained using Lemma~\ref{lem:shapes-shacl} {\color{red} on $\varphi$}. 

}

\subsection{Translation to ShEx}


\begin{lemma}
\label{lem:contents-shex}
For each open content type $\ntype$ there is a ShEx shape $\varphi_{\ntype}$ such that  $\graph, v\models \varphi_{\ntype}$ iff $\rho(v) \in \sem{\ntype}$ for all  $\graph=(E,\rho)$ and $v\in\nodes(\graph)$.
\end{lemma}

\begin{proof}
For the content type $\top$ the corresponding ShEx shape is $\shexneigh{\shexallte}$.

% For a content type of the form \[\closedRT{k_1:\vtype_1}\tAnd \dots \tAnd \closedRT{k_m :\vtype_m}\tAnd \top\,,\] where $k_1, \dots, k_m$ are pairwise different, 
% the corresponding ShEx shape is \[\shexneigh{k_1. \test(\vtype_1)  \shexeach \dots \shexeach k_m.\test(\vtype_m)\shexeach \shexallte}\,.\] If there are repetitions among $k_1, \dots, k_m$, we need to collect assertions about each key in a single atomic triple expression.  For example, for \[\closedRT{k_1:\vtype_1}\tAnd \closedRT{k_1:\vtype'_1}\tAnd \closedRT{k_2 :\vtype_2}\tAnd \top\,,\] we take 
% \[\shexneigh{k_1. (\test(\vtype_1)\land\test(\vtype'_1))  \shexeach k_m.\test(\vtype_2)\shexeach\shexallte}\,.\]

For a content type of the form \[\closedRT{k_1:\vtype_1}\tAnd \dots \tAnd \closedRT{k_m :\vtype_m}\tAnd \top\,,\] 
the corresponding ShEx shape is \[\shexneigh{k_1. \test(\vtype_1)\shexeach \shexallte} \land  \dots \land \shexneigh{k_m.\test(\vtype_m)\shexeach \shexallte}\,.\] 

Finally, every other  open content type can be expressed as 
\[(\ntype_1 \tOr \dots \tOr \ntype_\ell)\tAnd \top\,,\]
where each $\ntype_i$ has the form $\closedRT{k_1:\vtype_1}\tAnd \dots \tAnd \closedRT{k_m :\vtype_m}$ for some $m$.
The corresponding ShEx shape is 
\[\varphi_1 \lor \dots \lor \varphi_\ell\,,\]
where $\varphi_i$ is the ShEx shape corresponding to the content type $\ntype_i \tAnd \top$.
\end{proof}

\begin{lemma}
\label{lem:filter-shex} For each filter $\pexpr_0$ there is a ShEx shape $\varphi_{\pexpr_0}$ such that  $\graph, v\models \varphi_{\pexpr_0}$ iff $(v,v) \in \sem{\pexpr_0}^\graph$ for all  $\graph$ and $v\in\nodes(\graph)$. 
\end{lemma}

\begin{proof} By Lemma~\ref{lem:contents-shex}, the claim holds for $\pi_0=\ntype\tAnd \top$. 
For $\{k:c\}$ the corresponding ShEx shape is $\shexneigh{k.\hasvalue(c);\shexallte}$. Because ShEx shapes are closed under negation, the claim also holds for $\lnot\{k:c\}$ and $\lnot(\ntype\tAnd \top)$.  Finally, concatenations of filters correspond to conjunctions of shapes, so the claim follows because ShEx shapes are closed under conjunction. 
\end{proof}

\begin{lemma}
\label{lem:paths-shex}
For each common shape of the form $\exists\,\pexpr$ there is a ShEx shape $\varphi_{\exists\pexpr}$ such that $\graph, v\models \varphi_{\exists\pexpr}$ iff $\graph,v\models \exists\, \pexpr$ for all $\graph$ and $v\in\nodes(\graph) \cup \values(\graph)$. 
\end{lemma}

\begin{proof}
Let us first look at common shapes of the form $\exists\,\pexpr$ where $\pexpr$ is a concatenation of filters and atomic path expressions of the form  $p$, $p^{-}$, $k$, or $k^{-}$. Without loss of generality we can assume that the concatenation ends with a filter or with $k$.
We proceed by induction on the length of the concatenation. The base cases are $\exists \pi_0$ and $\exists k$, which correspond to $\varphi_{\pi_0}$ (Lemma~\ref{lem:filter-shex}) and $\shexneigh{k.\shextop \shexeach \shexallte} $, respectively.
For $\exists\, \pexpr_0\cdot\pexpr$ we can take $\varphi_{\pexpr_0} \land \varphi_{\exists \pi}$. 
For $\exists\, p\cdot\pexpr$ we can take $\shexneigh{p.\varphi_{\exists \pi} \shexeach \shexallte}$, and similarly for $\exists\, p^{-}\cdot\pexpr$ and $\exists\, k^{-}\cdot\pexpr$. 

The general case follows because ShEx shapes are closed under union. Indeed, because our PG-path expressions are star-free, we can assume without loss of generality that in each common shape of the form $\exists\, \pexpr$, the PG-path expression $\ppexpr$ underlying $\pexpr$ is a union of concatenations of filters and atomic path expressions of the form $p$ or $p^{-}$. Then, for \[\exists\, k^{-}\cdot(\pexpr^1 \cup \dots \cup\pexpr^m)\cdot k'\] we can take \[\varphi_{\exists k^{-}\cdot\pexpr^1\cdot k'} \lor \dots \lor \varphi_{\exists k^{-}\cdot\pexpr^m\cdot k}\,.\] Simiarly for $\exists\, k^{-}\cdot(\pexpr^1 \cup \dots \cup\pexpr^m)$, $\exists\, (\pexpr^1 \cup \dots \cup\pexpr^m)\cdot k'$, and $\exists\, (\pexpr^1 \cup \dots \cup\pexpr^m)$.
\end{proof}


\begin{lemma}
\label{lem:shapes-shex}
For each common shape $\varphi$ there is a ShEx shape $\hat \varphi$ such that  $\graph, v\models \varphi$ iff $\graph,v\models \hat \varphi$ for all $\graph$ and $v\in\nodes(\graph)  \cup \values(\graph)$. 
\end{lemma}

\begin{proof} 
Because ShEx shapes are closed under conjunction, it suffices to prove the claim for the atomic common shapes of the forms $\exists\, \pexpr$, $\exists^{\leq n}\pexpr_1$, $\exists^{\geq n}\pexpr_1$, and $\exists \ntype \land \not\exists\lnot P$. The first case follows from Lemma~\ref{lem:paths-shex}. 

Let us look at common shapes of the form $\exists^{\geq n}\pexpr_1$. If $n=0$ we can simply take $\shextop$. Suppose $n>0$. Then, for \[\exists^{\geq n} \pexpr_0\cdot p\cdot \pexpr'_0\] we can take 
\[\varphi_{\pexpr_0} \land \big\{\big(p. \varphi_{\pexpr'_0}\big)^{n}\shexeach \shexallte\big\},\] and similarly for $\exists^{\geq n}\pexpr_0\cdot p^{-}\cdot \pexpr'_0$, $\exists^{\geq n}\pexpr_0\cdot k^{-}\cdot \pexpr'_0$, and $\exists^{\geq n}\pexpr_0\cdot k$ (using $\shextop$ instead of $\varphi_{\pexpr'_0}$).
 
Next, we consider common shapes of the form 
$\exists^{\leq n}\pexpr_1$. For \[\exists^{\leq n} \pexpr_0\cdot p\cdot \pexpr'_0\] we can take \[\lnot\varphi_{\pexpr_0} \lor \lnot 
\big\{\big(p.\varphi_{\pexpr'_0}\big)^{n+1}\shexeach \shexallte\big\}\] and similarly for $\exists^{\leq n}\pexpr_0\cdot p^{-}\cdot \pexpr'_0$, $\exists^{\leq n}\pexpr_0\cdot k^{-}\cdot \pexpr'_0$, and $\exists^{\leq n}\pexpr_0\cdot k$ (again, using $\shextop$ instead of $\varphi_{\pexpr'_0}$). 

Before we move on, let us introduce a bit of syntactic sugar. For a set $Q = \{q_1, q_2, \dots, q_n\}\subseteq \Predicates \cup \Keys$ we write $Q^{*}$ for the triple expression
$\big(q_1.\shextop \shexone q_2.\shextop \shexone \dots \shexone q_n.\shextop\big)^{*}$. 

We are now ready to consider a common shape of the form $\exists\, \ntype \land \not\exists\,\lnot P$. Suppose first that 
\[\ntype = \closedRT{}\,.\] 
Then, the corresponding ShEx shape is simply 
\[\shexneigh{P^{*} \shexeach \big(\lnot\emptyset^{-}\big)^{*}}\,.\] 
Next, suppose that \[\ntype = \closedRT{k_1:\vtype_1}\tAnd \dots \tAnd \closedRT{k_m :\vtype_m}\,.\] 
Then, the corresponding ShEx shape is 
\[
\varphi_{\ntype\tAnd\top} \land \shexneigh{\{k_1, \dots,k_m\}^{*}\shexeach P^{*} \shexeach \big(\lnot\emptyset^{-}\big)^{*}} \,,\]
where $\varphi_{\ntype\tAnd\top}$ is obtained from Lemma~\ref{lem:contents-shex}.
In general, as in Lemma~\ref{lem:contents-shex}, we can assume that 
\[\ntype = \ntype_1 \tOr \dots \tOr \ntype_m\] where each $\ntype_i$ is of one of the two forms considered above. The corresponding ShEx shape is then 
\[\varphi_1 \lor \dots \lor \varphi_m\] where $\varphi_i$ is the ShEx shape corresponding to $\exists \ntype_i \land \not\exists\lnot P$, obtained as described above. 
\end{proof}


\begin{lemma}
\label{lem:schemas-shex}
For every common schema there is an equivalent ShEx schema.  
\end{lemma}

\begin{proof} 
Let $\schema$ be a common schema. We obtain an equivalent ShEx schema $\schema'$ by translating each  $(\sel, \varphi) \in \schema$ to $(\sel', \varphi')$ such that for all $\graph$ and $v\in \Nodes\cup\Values$, 

\begin{center}
 $\graph, v \models \sel$ implies $\graph, v \models \varphi$ 
 
 iff 
 
 $\graph, v \models \sel'$ implies $\graph, v \models \varphi'$.
 \end{center}
 Recall that $\sel$ is a common shape of one of the following forms:
\[ 
\exists\, k \,,\;
\exists\, p \cdot \pexpr\,, \;
\exists\, p^{-}\cdot \pexpr \,,\;
\exists\, \{k:c\}\cdot \pexpr \,, \;
\exists\, \big(\{k:\vtype\}\tAnd\top\big)\cdot \pexpr \,, \;
\exists\, k^{-}\cdot \pexpr\,.
\]
If $\sel$ is of the form  \[\exists\, k\,,\quad \exists\, \{k:c\}\cdot \pi\,,\quad \text{or}\quad \exists\, \big(\{k:\vtype\}\tAnd \top\big)\cdot \pi\,,\] for $\sel'$ we take $\shexneigh{ k.\!\shextop \shexeach \shexallte}$. In the remaining cases, we take, respectively, 
\[ 
\shexneigh{ p.\!\shextop \shexeach \shexallte},\quad
\shexneigh{ p^{-}\!.\!\shextop \shexeach \shexallte},\quad
\shexneigh{ k^{-}\!.\!\shextop \shexeach \shexallte}.
\]
For $\varphi'$ we take $\lnot \varphi_{\sel} \lor \hat\varphi$ where $\varphi_{\sel}$ is obtained using Lemma~\ref{lem:paths-shex}
 and $\hat\varphi$ is obtained using Lemma~\ref{lem:shapes-shex}. 
 \end{proof}


