\section{Related Work}
\paragraph{SHACL literature.} The authoritative source for SHACL is the W3C recommendation____.
%that first introduced SHACL
Further literature on SHACL following its standardisation can be roughly divided into two groups. The first group studies the formal properties and expressiveness of the non-recursive fragment____. Notable examples in this category are: the work by Delva et al. on data provenance____, the work of Pareti et al. on satisfiability and (shape) containment____, and the work of Leinberger et al. connecting the containment problem to description logics____.
The second group of papers is concerned with proposing a suitable semantics for recursive SHACL____ or studying the complexity of certain problems for recursive SHACL under a chosen semantics____. First reports on practical applications and use-cases for SHACL include the study of expressivity of property constraints, as well as  mining and extracting constraints in the context of large knowledge graphs such as Wikidata and DBpedia____. Finally, the underlying ideas of SHACL where transposed to the setting of Property Graphs in a formalism called ProGS ____.

\paragraph{ShEx literature}

ShEx was initially proposed in 2014 as a concise and human-readable language to describe, validate, and transform RDF data____.
Its formal semantics was formally defined in____.
The semantics of ShEx schemas combining recursion and negation was later presented in____.
The current semantic specification of the ShEx language has been published as a W3C Community group report____ and a new language version is currently being defined as part of the IEEE Working group on Shape Expressions\footnote{\url{https://shex.io/shex-next/}}. As for practical applications, ShEx has been applied as a descriptive schema language through the
Wikidata Schemas project\footnote{\url{https://www.wikidata.org/wiki/Wikidata:WikiProject_Schemas}}.
%Additional work went into extending ShEx to handle graph models that go beyond RDF, like WShEx to validate Wikibase graphs____, ShEx-Star to handle RDF-Star and PShEx to handle property graphs____; while these wokrs extend ShEx to different types of property graphs and does not look for a common graph model, so it has a very different focusneither of ths and does not look for a common graph model, so it has a very different focus
%Wikidata Schemas project\footnote{\url{https://www.wikidata.org/wiki/Wikidata:WikiProject_Schemas}}.
Additional work went into extending ShEx to handle graph models that go beyond
RDF, like WShEx to validate Wikibase graphs____, ShEx-Star to handle
RDF-Star and PShEx to handle property graphs____. While these works
extend ShEx to (different types of) property graphs, they do not provide a
common graph data model \todo{nor compare schema languages, as we do} that
allows comparing schema languages, as we do.


\paragraph{PG-Schema literature.}

PG-Schema, as introduced in____, builds upon an earlier proposal of PG-Keys____ to enhance schema support for property graphs, in the light of limited schema support in existing systems and the current version of the GQL standard____. It is currently being used in the GQL standardization process as a basis for a standard for property graph schemas.


\paragraph{Comparing RDF schema formalisms.}

In Chapter 7 of ____, the authors compare common features and differences between ShEx and SHACL and____ presents a simplified language called S, which captures the essence of ShEx and SHACL. Tomaszuk____ analyzes advances in RDF validation, highlighting key requirements for validation languages and comparing the strengths and weaknesses of various approaches.
% The following sentence is about comparing schemas for RDF and property graphs

\paragraph{Interoperability between schema graph formalisms.}

Interoperability between schema graph formalisms like RDF and Property Graphs remains challenging due to differences in structure and semantics. RDF focuses on triple-based modeling with formal semantics, while Property Graphs allow flexible annotation of relationships with properties. RDF-star ____ and RDF 1.2 ____ extend RDF 1.1 by enabling statements about triples, aligning more closely with LPG: for instance, RDF-star allows triples to function as subjects or objects, similar to how LPG edges carry properties.

By adopting \emph{named graphs}____, already RDF 1.1 provided a mechanism for making statements about (sub-)graphs. Likewise, different \emph{reification} mechanisms have been proposed in the literature for RDF in order to ``embed'' meta-statements about triples (and graphs) in ``vanilla'' RDF graphs, ranging from the relatively verbose original W3C reification vocabulary as part of the original RDF specification, to more subtle approaches such as singleton property reification____, which is close to the unique identifiers used for edges in most LPG models. Custom reification models are used, for instance, in Wikidata, to map Wikibase's property graph schema to RDF, cf. e.g.____.
There is also work on  schema-independent and schema-dependent methods for transforming RDF into Property Graphs, providing formal foundations for preserving information and semantics ____.
All these approaches, in principle, facilitate general or specific mappings between RDF and LPGs, which is what the present paper tries to avoid by focusing on a common submodel.

There have been several prior proposals for uniying graph data models, rather then providing mappings between them.  The OneGraph initiative ____ aims to bridge the different graph data models by promoting a unified graph data model for seamless interaction. Similarly, MilleniumDB's Domain Graph model____ aims at covering RDF, RDF-star, and property graphs. These works seek a common \emph{supermodel}, aiming to support a both RDF and LPGs via more general solutions. In contrast, we aim at understanding the existing schema languages by studying them over a common submodel of RDF and LPGs.

\paragraph{Schemas for tree-structured data.}

The principle of defining (parts of) schemas as a set of pairs $(sel,\varphi)$ is also used in schema languages for XML. A DTD____ is essentially such a set of pairs in which $sel$ selects nodes with a certain label, and $\varphi$ describes the structure of their children. In XML Schema, the principle was used for defining key constraints (using \emph{selectors} and \emph{fields}) \cite[Section~3.11.1]{xsd}. The equally expressive language BonXai____ is based on writing the entire schema using such rules. Schematron ____ is another XML schema language that differs from grammar-based languages by defining patterns of assertions using XPath expressions____. It excels in specifying constraints across different branches of a document tree, where traditional schema paradigms often fall short. Schematron's rule-based structure, composed of phases, patterns, rules, and assertions, allows for the validation of documents.

\paragraph{RDF validation}

Last, but not least, it should be noted that the requirement for (constraining) schema languages—besides ontology languages such as OWL and RDF Schema—in the Semantic Web community is much older than the more recent additions of SHACL and ShEx. Earlier proposals in a similar direction include efforts to add constraint readings of Description Logic axioms to OWL, such as OWL Flight ____ or OWL IC ____. Another approach is Resource Shapes (ReSh) ____, a vocabulary for specifying RDF shapes. The authors of ReSh recognize that RDF terms originate from various vocabularies, and the ReSh shape defines the integrity constraints that RDF graphs are required to satisfy. Similarly, Description Set Profiles (DSP) ____ and SPARQL Inferencing Notation (SPIN) ____ are notable alternatives. While SHACL, ShEx, and ReSh share declarative, high-level descriptions of RDF graph content, DSP and SPIN offer additional mechanisms for validating and constraining RDF data, each with its own strengths and applications.

\paragraph{Implementations}
Dozens of tools support graph data validation, including ShEx and SHACL. A comprehensive collaborative list of resources is available at:
\url{https://github.com/w3c-cg/awesome-semantic-shapes}.