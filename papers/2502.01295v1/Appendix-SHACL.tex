%!TEX root =  Article.tex

\section{Standard SHACL}

\label{app:standard-shacl}

Standard SHACL defines shapes as a conjunction of \emph{constraint components}. The different constructs from our formalization correspond to fundamental building blocks of these constraint components. Next to that, the formalization of SHACL presented in this paper is less expressive than standard SHACL. First, because we define it here for the common data model (which corresponds to a strict subset of RDF, see \Cref{app:sec-foundations-comparison-rdf}), and second, because we want to simplify our narrative: we leave out the comparison of RDF terms using \texttt{sh:lessThan} for this reason. Furthermore, because the common data model omits language tags, the corresponding constraint components from standard SHACL are omitted as well.

Our formalization is closely tied to the ones found in the literature. There, correspondence between the formalization of the literature and standard SHACL has been shown in detail \cite{MJPHD}. This section highlights and discusses some relevant details.

\paragraph{Class targets and constraint component.} As a consequence of the common data model not directly supporting the modelling of classes, some class-based features are not adapted in our formalization. Specifically, there are no selectors (``target declarations'' in standard SHACL) that involve classes. Furthermore, the value type constraint component \texttt{sh:class} is not covered by our formalization.

%The common data model does not support defining instances of a certain class. Even though both property graphs and RDF have support for this, they both tackle it very differently: in RDF, class and instance information is part of the graph data itself, while in property graphs it is more of a modelling aspect. A concrete issue in this regard is that RDF allows for navigating the graph in order to determine class information, while this is not possible in property graphs where nodes are labelled directly. This makes it difficult to compare the two approaches, as a common data model is then no longer a subset of RDF and property graphs --- it would require a form of translation. Therefore, to reduce the complexity of our approach, we decided not to encode class information in our common data model. A consequence of this is that we leave out all class related features of SHACL. Specifically, the class constraint component, and the ability to target nodes of a certain SHACL class. 

%Technically, you could encode class information, as it exists in RDF, in this datamodel because it is part of the data. However, this is then difficult to reconcile with PG-Schema, because ...

%\todo[inline]{Filip: I copied the paragraph above to the first appendix. Here it will probably be sufficient to repeat briefly what disappears from the schema language. }


\paragraph{Closedness.} In standard SHACL syntax, closedness is a property that takes a true or false value. The semantics of closedness is based on a list of predicates that are allowed for a given focus node. This list can be inferred based on the predicates used in property shapes, or this list can be explicitly given using the \texttt{sh:ignoredProperties} keyword. Our formalization effectively adopts the latter approach: $\closed(Q)$ means that the properties mentioned in the set $Q$ are the ignored properties.

\paragraph{Path expressions.} The path expressions used in our formalization deviate from the standard in three obvious ways. First, we make a distinction between `keys' and `predicates'. This is simply a consequence of using our common data model. Second, we leave out some of the immediately available path constructs from standard SHACL: one-or-more paths and zero-or-one path. However, these are expressible using the building blocks of our formalization: one-or-more paths are expressed as $\pi\cdot\pi^{*}$, and zero-or-one paths are expressed as $\pi\cup\id$. Lastly, our path expression allow for writing the identity relation explicitly. This cannot be done in literal standard SHACL syntax, but its addition to the formalization serves to highlight its hidden presence in the language. Writing the identity relation directly in a counting construct, e.g., $\exists^{\geq n}\id.\top$, never adds expressive power. In the case of $n=1$, the shape is always satisfied (and thus equivalent to $\top$), and it is easy to see that for any $n > 1$, it is never satisfied. The situation with complex path expressions in counting constructs is less clear from the outset. However, it has been shown \cite{BJVdB24} (Lemma 3.3), that the only case where $\id$ adds expressiveness is with complex path expressions of the form $\pi\cup\id$. This is exactly the definition of zero-or-one paths and is therefore covered by standard SHACL. Another place where $\id$ can occur in our formalization is in the equality and disjointness constraints, e.g., $\eq(\id,p)$. According to the standard SHACL recommendation, you cannot write this shape. However, in the SHACL Test Suite \cite{LKK24} test \texttt{core/node/equals-001}, the following shape is tested for:
\begin{verbatim}
ex:TestShape
  rdf:type sh:NodeShape ;
  sh:equals ex:property ;
  sh:targetNode ex:ValidResource1 .
\end{verbatim}
on the following data:
\begin{verbatim}
ex:ValidResource1
  ex:property ex:ValidResource1 .
\end{verbatim}
The intended meaning of this test is, in natural language: ``The targetnode \texttt{ex:ValidResource1} has a \texttt{ex:property} self-loop and no other \texttt{ex:property} properties''. Effectively, this is the semantics for our $\eq(\id,p)$ construct. The situation with $\disj(\id,p)$ is similar.

We therefore have an ambiguous situation: the standard description of SHACL does not allow for shapes of the form $\eq(\id,p)$, but the test suite, and therefore all implementations that pass it completely, do\footnote{Incidentally, all implementations currently mentioned in the implementation report handle these cases correctly.}. It then seems fair to include this powerful construct in the formalization. 

\paragraph{Comparisons with constants.}
A direct consequence of the assumption that node identities in the common data model are hidden from the user, our abstraction of SHACL on common graphs does not support comparisons with constants from $\Nodes$. Comparisons with constants from $\Values$ are allowed.

\paragraph{Node tests.} 

Our formalization uses the $\test(\vtype)$ construct to denote many of the node tests available in SHACL. We list the tests from standard SHACL that are covered by this construct.

\smallskip
\noindent
$\bullet$
{\bf \texttt{DatatypeConstraintComponent}}  \\ Tests whether a node has a certain datatype. \\
\noindent
$\bullet$ {\bf\texttt{MinExclusiveConstraintComponent}} or \\ \noindent \phantom{$\bullet$} {\bf\texttt{MinInclusiveConstraintComponent}} or \\ \noindent \phantom{$\bullet$} {\bf\texttt{MaxExclusiveConstraintComponent} }or \\ \noindent \phantom{$\bullet$} {\bf\texttt{MaxInclusiveConstraintComponent}} \\
These four constraints cover can check whether a node is larger (\texttt{\textbf{Max}}) or smaller (\texttt{\textbf{Min}}) than some value, and whether this forms a partial order (\texttt{\textbf{Inclusive}}) or a strict, or total, order (\texttt{\textbf{Exclusive}}).
Based on the SPARQL $<$ or $\leq$ operator mapping. \\
\noindent
$\bullet$ {\bf\texttt{MaxLengthConstraintComponent}} or \\ \noindent \phantom{$\bullet$}   {\bf\texttt{MinLengthConstraintComponent} } \\
  These two constraints test whether the
  length of the lexical form of the node is ``larger'' or equal (resp. ``smaller'' or equal) than
  some provided integer value. Strictly speaking, the recommendation defines these constraint components also on IRIs. However, we limit their use to Literals. \\
\noindent
$\bullet$ {\bf\texttt{PatternConstraintComponent}} \\ Tests whether the length of the lexical form of the node satisfies some regular expression. Strictly speaking, the recommendation defines these constraint components also on IRIs. However, we limit their use to Literals.

\smallskip 

Then there are two tests not covered by our formalization:

\smallskip

\noindent
$\bullet$ {\bf\texttt{NodeKindConstraintComponent}} \\
Tests whether a node is an IRI, Blank Node, or Literal. Our tests apply only to RDF Literals.\\
\noindent
$\bullet$ {\bf\texttt{LanguageInConstraintComponent}} \\
 Test whether the language tag of the node is one of the specified language tags. This feature is not supported by our data model, since it lacks language tags.




% Given a common graph $G = (E,\rho)$, the idea of a ``neighbourhood'' is to identify the nodes that share some ``local'' properties in $E$ and $\rho$. 

% \begin{definition}[Neighbourhood]
%  We are given a data graph $G = (E,\rho)$. A graph suffix, is a pair $\mathcal{P} \times \mathcal{N}$ or $\mathcal{K} \times \mathcal{V}$. We call a neighboorhood $N$ in $G$ a given subset of graph suffixes, where 
%  \[
%    N \subseteq 2 ^ {\{  \mathcal{P} \times \mathcal{N} \} \cup \{ \mathcal{K} \times \mathcal{V} \} }
%  \]
% Given a neighbourhood $N$, we say \emph{a node $u \in \mathcal{N}$ is captured by N}, if there exists $(p,n) \in N$ s.t. $(u,p,n) \in E$ or $(k,v) \in N$ and $\rho(u,k) = v$.

% %Given a neighbourhood $N$ and node $u$, we say \emph{a node $u$ has neighbourhood $N$ in graph $G$}, if $(p,n) \in N$ iff $(u,p,n) \in E^G$ and $(k,v) \in N$ iff $\rho(n,k)=v$ for all $n\in \Nodes$, $k\in\Keys$, and $v\in\Values$.
% \end{definition}

% Each constraint in SHACL can be seen (in a roundabout way) as identifying a very specific kind of neighboorhood. 

% \begin{flalign*}
%   \mathit{eq}(p_1,p_2)  =& \ \{ (p_i,u) \mid \exists n \in \mathcal{N}.\forall u \in \mathcal{N}( (n,p_1,u) \in E \land (n,p_2,u) \in E  )   \}  \\ 
%    \mathit{eq}(k_1,k_2)  =& \  \{ (k_i,v) \mid \exists n \in \mathcal{N}.\forall v \in \mathcal{V}( \rho(n,k_1) =  \rho(n,k_2) = v )   \}   \\
%   \mathit{disj}(p_1,p_2)  =& \  \{ (p_i,u) \mid \exists n \in \mathcal{N}.\forall u \in \mathcal{N}( \neg \big ( (n,p_1,u) \in E \land (n,p_2,u) \in E \big ) )   \} \\
%   \mathit{test}(t) =& \  \{ (p,n) \mid u \in \mathcal{N} \land \mathit{test}(u) \land (u,p,n) \in E  \} \cup \\ 
%   & \  \{ (k,v) \mid u \in \mathcal{N} \land \mathit{test}(u) \land \rho(u,k) = v  \}
% \end{flalign*}

% Then one can think of SHACL shape definitions as identifying the intersection of all these neighbourhods. 



%% Maxime rant 
%\section{Difficulties with defining SHACL for the common data model}

%In the context of the common data model (CDM), a \textit{shape expression} $\phi$ is a unary query over $\Nodes$. Here, we go over every SHACL feature and try to interpret their possible meaning when considering the CDM.

%When $\phi$ is $\top$, the meaning is clear. Unlike in RDF, the meaning of $\top$ is restricted to $\Nodes$ (and not $\Values$).

%When $\phi$ is $\hasvalue(n)$, with $n\in \Nodes$, the meaning is clear. In RDF, this feature is also used to identify literals. That is not the case here, because we look at shape expressions as unary queries over $\Nodes$.

%When $\phi$ is $\test(t)$, the task is to specify which tests are applicable to elements from $\Nodes$. There are two tests that seem directly applicable here: nodekind IRI, and nodekind Blank. However, if we want to be very precise, all tests involving strings are also applicable to IRIs (which we consider as elements from $\Nodes$) in their string representation. We may want to ignore this for clarity.

%When $\phi$ uses one of the boolean connectives, the meaning is also clear.

%Now, we need to still consider all features involving path expressions. Take as the simplest case the shape expression $\exists E.\top$. What does this kind of shape expression express over the CDM? Basically: ``There exists a node or value reachable by $E$''. Indeed, in RDF there is no difference between nodes and values, so here we need to be explicit about it. To clarify this situation, we need to know what exactly path expressions $E$ are. Are they expressions using only elements from $\Predicates$? Or can they also involve elements from $\Keys$? I (maxime) would argue that the most natural interpretation is when $E$ is seen as an expression over $\Predicates$. When this is the case, $\exists E.\phi$ is nicely defined (and analogously, $\geqn{n}{E}{\psi}$ is well defined). 

%There is one problem, of course. In RDF, these kind of shape expressions are also used to check for the existence of what we in the CDM would call \emph{values}. When we restrict $E$ to only predicates, we lose this functionality (and this seems not reasonable). I see two solutions to this. Either we allow for keys in path expressions, or we add specific constructs to the language that add our desired functionality. Both directions seem messy to me. The former was actually the inspiration for the SHACL presented in Appendix~\ref{sec:weirdshacl}. The latter would require us to possibly add many constructs to shape expressions. The task then becomes to basically create a new language that simulates SHACL but for the CDM. One way to approach this is to look at the language from Appendix~\ref{sec:weirdshacl} and go over every feature, analysing its meaning, and constructing a more ``clean'' language from it. For example: looking at an expression $\phi$ of the form $\geqn{n}{E}{\psi}$, we know $E$ can reach:
%\textcolor{red}{ognjen: I am thinking that if we have a normal form from the beginning, 
%then there it would be easier to define paths that end with nodes and path that end with values}
%\begin{enumerate}
%    \item only nodes from $\Nodes$;
%    \item only values from $\Values$; or
%    \item both nodes and values
%\end{enumerate}
%In the first case, the intended meaning of $\phi$ is clear, and $\psi$ is simply another shape expression.

%In the second case, the intended meaning is slightly different. Here, $\psi$ cannot be a shape expression, because it would apply to \emph{values}, not nodes. However, SHACL has some constraints that are applicable to values: $\top$, tests and $\hasvalue$. So it makes sense to add a construct here.

%The last case, only makes sense for me when $\psi$ is $\top$, i.e., counting of values and nodes. Although even that is slightly strange in the context of the CDM.

%One last remark is that viewing shapes as unary queries over $\Nodes$ limits the expressive power of constraints. One example is the simple key constraint...

%\subsection{Why it will not work in the naive way}
%This is a starting point for a non-problematic definition of SHACL over the common data model (CDM), leading to a dead end (in my opinion).

%(We assume an infinite set of \emph{shape names} $\Shapes$.)

%A \emph{path expression} $E$ is given by the following grammar:
%\[
%E ::= \id \mid p \mid E^{-}\mid E\comp E %\mid E\cup E\mid E^{*}
%\]
%with $p\in\Predicates$ and $\id$ represents the identity relation (or empty word).

%\newcommand{\nodekind}{\mathit{nodeKind}}
%\newcommand{\blank}{\mathit{Blank}}
%\newcommand{\iri}{\mathit{IRI}}
%A \emph{shape expression} $\phi$ is given by the following grammar:
%\begin{align*}
%F & ::= \id \mid E \\
%\phi & ::= \top \mid \hasvalue(c) \mid \nodekind(k) \mid \hasshape(s) \mid \neg\phi \\
%& \mid \phi\land\phi \mid \phi\lor\phi \mid \forall E.\phi \mid \geqn{n}{E}{\phi} \mid \leqn{n}{E}{\phi} \\
%& \mid \eq(F,p) \mid \disj(F,p) \mid \closed(Q) \end{align*}
%with $c\in\Nodes$, $k\in\{\blank, \iri\}$, $s\in\Shapes$, $n$ a natural number, and $Q\subseteq \Predicates$.

%A \emph{shape definition} is a triple $(s,\phi,q)$ with $s\in\Shapes$, $\phi$ a shape expression and $q$ a target declaration (given by the SHACL recommendation, but also allowing targeting nothing).

%A \emph{shape schema} is a finite set of shape definitions. We say a shape schema is recursive when a shape (in)directly references itself by following $\hasshape(s)$ constructs. For now, we only define non-recursive shape schemas, i.e., where shapes never reference themselves.

%Before going on to the semantics of the above expressions, we should highlight why this is called ``restricted SHACL''. First of all, it is impossible to talk about the keys and their associated values in this language. Intuitively, this is because path expressions are only over predicates. Second, we have omitted the features $\lessthan$ and $\lessthaneq$ because they can have no possible meaning. Intuitively, this is because we cannot sensibly compare elements from $\Nodes$. Lastly, the $\closed$ feature is only applicable to predicates --- any key is still allowed, even if not mentioned in closed. This is problematic, as we now strongly diverge from real SHACL semantics. This is simply corrected, however, by stating that $Q\subseteq \Predicates\cup \Keys$.








%\section{Proposal: SHACL for the Common Data Model}\label{sec:propsalSHACL}
%(We assume an infinite set of \emph{shape names} $\Shapes$.)

%A \emph{path expression} $E$ is given by the following grammar:
%\[
%E ::= \id \mid p \mid k \mid E^{-} \mid %E\comp E \mid E\cup E\mid E^{*}
%\]
%with $p\in\Predicates$, $k\in\Keys$ and $\id$ represents the identity relation (or empty word).

%In the context of the common data model (CDM), a \textit{shape expression} $\phi$ is a unary query over $\Nodes\cup\Values$, and is given by the following grammar:
%\begin{align*}
%F & ::= \id \mid E \\
%\phi & ::= \top 
%\mid \hasvalue(c) 
%\mid \test(t) 
%\mid \hasshape(s) 
%\mid \closed(Q) 
%\mid \eq(F,p) \\
%& \mid \disj(F,p) 
%\mid \lessthan(F,p) 
%\mid \lessthaneq(F,p) 
%\mid \uniquelang(E)  \\
%& \mid \neg\phi 
%\mid \phi\land\phi 
%\mid \phi\lor\phi 
%\mid \forall E.\phi 
%\mid \geqn{n}{E}{\phi} 
%\mid \leqn{n}{E}{\phi} 
%\end{align*}
%with $c\in\Nodes\cup\Values$, $t$ a test identifying sets of nodes or values, $s\in\Shapes$, $n$ a natural number, and $Q\subseteq \Predicates\cup\Keys$.

%A \emph{shape definition} is a triple $(s,\phi,q)$ with $s\in\Shapes$, $\phi$ a shape expression and $q$ a target declaration (given by the SHACL recommendation, but also allowing targeting nothing). The target declaration identifies nodes or values from the graph. The allowed target declarations are parameterized with $l$ and are given next:
%\begin{itemize}
%    \item Target nothing (equivalent to having no target declaration in real SHACL)
%    \item Target all nodes that have an outgoing edge labelled $l$ (\texttt{sh:targetSubjectsOf})
%    \item Target all nodes that have a value for key $l$ (\texttt{sh:targetObjectsOf})
%    \item Target all values used for keys $l$ (\texttt{sh:targetObjectsOf})
%    \item Target a specific node or value $l$ (\texttt{sh:targetNode})
%\end{itemize}
%We have to leave out the target declarations involving classes because of the restrictions of the CDM.

%A \emph{shape schema} is a finite set of shape definitions. We say a shape schema is recursive when a shape (in)directly references itself by following $\hasshape(s)$ constructs. For now, we only define non-recursive shape schemas, i.e., where shapes never reference themselves.

%In order to define the semantics of shape expressions, we first define the evaluation of a path expression $E$ on a graph $G$ as a binary relation $\iexpr{E}{G}$ on $\Nodes\cup\Values$. This evaluation is given in Table~\ref{tab:seme2}.
%\begin{table}
%  \caption{Evaluation of a path expression %$E$ on a graph $G$.}
%  \label{tab:seme2}
%  \centering
%  \begin{tabular}{l|l}
%    \toprule
%    $E$ & $\iexpr{E}{G}$ \\
%    \midrule
%    $\id$ & the identity relation over $\Nodes\cup\Values$\\
%    $p$ & $\{(a,b)\mid (a,p,b)\in E\}$\\
%    $k$ & $\{(a,b)\mid \rho(a,k)=b\}$ \\
%    $E_1^{-}$ & $\{(a,b)\mid (b,a) \in \iexpr{E_1}{G}\}$ \\
%    $E_1\comp E_2$ & $\{(a,b) \mid \exists c: (a,c)\in\iexpr{E_1}{G}\enne (c,b)\in\iexpr{E_2}{G}\}$ \\
%    $E_1\cup E_2$ & $\iexpr{E_1}{G}\cup\iexpr{E_2}{G}$\\
%    $E_1^{*}$ & the transitive reflexive closure of $\iexpr{E_1}{G}$\\
%    \bottomrule
%  \end{tabular}
%\end{table}

% \newcommand{\defs}{\mathit{def}}
% \renewcommand{\models}{\vDash}
% \newcommand{\nmodels}{\nvDash}

%We can now define when a node or value $v$ from graph $G$ satisfies the shape expression $\phi$ in context of a shape schema $H$, denoted $H, G, v \models \phi$. This is defined in Table~\ref{tab:semphi2} where we use the notation $\defs(s,H)$ to denote the shape expression $\phi$ when $(s,\phi,q)\in H$.

%\begin{table}
%\small
%  \caption{Semantics of a shape expression $\phi$.}
%  \label{tab:semphi2}
%  \centering
%  \begin{tabular}{l|l}
%    \toprule
%    $\phi$ & $H, G, v \models \phi$ if: \\
%    \midrule
%    $\hasvalue(c)$ & $v = c$\\
%    $\test(t)$ & $v$ satisfies the test $t$\\
%    $\hasshape(s)$ & $H, G, v \models \defs(H,s)$ \\
%    $\closed(Q)$ & $
%                   \begin{aligned}[t]
%                     &\forall (x,y,z)\in E: x = v \Rightarrow y\in Q \enne \\
%                     & \text{if $\exists k: \rho(a,k)$ is defined, then $k\in Q$}
%                   \end{aligned}
 %                  $\\
 %   $\eq(F,E)$ & the sets $\iexpr{F}{G}(v)$ and $\iexpr{E}{G}(v)$ are equal\\
  %  $\disj(F,E)$ & the sets $\iexpr{F}{G}(v)$ and $\iexpr{E}{G}(v)$ are disjoint\\
   % $\lessthan(F,E)$ & $b<c$ for all $b\in\iexpr{E}{G}(v)$ and $c\in\iexpr{F}{G}(v)$\\
   % $\lessthaneq(F,E)$ & $b\leq c$ for all $b\in\iexpr{E}{G}(v)$ and $c\in\iexpr{F}{G}(v)$\\
    %$\uniquelang(E)$ & $b$ and $c$ have different language tags for $b,c\in\iexpr{E}{G}(v)$ \\
    %$\forall E.\phi$ & every $b\in\iexpr{E}{G}(v)$ satisfies $H,G,b\models\phi$\\
    %$\geqn{n}{E}{\phi}$ & $\#\{b\mid (a,b)\in \iexpr{E}{G} \enne b\in\iexpr{\phi_1}{G}\}\geq n$\\
    %$\leqn{n}{E}{\phi}$ & $\#\{b\mid (a,b)\in \iexpr{E}{G} \enne b\in\iexpr{\phi_1}{G}\}\leq n$\\
    %\bottomrule
  %\end{tabular}
%\end{table}

%\paragraph{Node tests.} The $\mathit{test}$-feature abstracts some of the SHACL constraint components used to perform tests on nodes. The possible tests will be listed here.

%\begin{description}
%\item[\texttt{DatatypeConstraintComponent}] can test for any allowed datatype. Based on the SPARQL datatype function\footnote{\url{https://www.w3.org/TR/sparql11-query/\#func-datatype}}.
%\item[\texttt{NodeKindConstraintComponent}] can test whether a node
%  is an IRI, Blank Node, or Literal (or any subset of these).
%\item[\texttt{MinExclusiveConstraintComponent}] can test whether
%  the node is strictly ``larger'' than some value. Again, based on the
%  SPARQL $<$-operator.
%\item[\texttt{MinInclusiveConstraintComponent}] similar to previous, but not strict.
%\item[\texttt{MaxExclusiveConstraintComponent}] similar to
%  previous, but strictly ``smaller'' than some value.
%\item[\texttt{MaxInclusiveConstraintComponent}] similar to previous, but not strict.
%\item[\texttt{MinLengthConstraintComponent}] can test whether the
%  length of the lexical form of the node is ``larger'' or equal than
%  some provided integer value.
%\item[\texttt{MaxLengthConstraintComponent}] similar to previous,
%  but ``smaller'' or equal.
%\item[\texttt{PatternConstraintComponent}] can test whether the
%  length of the lexical form of the node satisfies some regular
%  expression.
%\item[\texttt{LanguageInConstraintComponent}] can test whether the
%  language tag of the node is one of the specified language tags.
%\end{description}