%!TEX root = Article.tex

% Begin of file 2-Preliminaries.tex

\section{Foundations}
\label{sec:prl}

In this section we present some material that we will need in the subsequent
sections, and define a data model that consists of common aspects of RDF and
Property Graphs.


\subsection{A Common Data Model}

When developing a common framework for SHACL, ShEx, and PG-Schema, the first
challenge is establishing  a \emph{common data model}, since SHACL and ShEx work
on RDF, whereas PG-Schema works on Property Graphs.
Rather than using a model that generalises  both RDF and Property Graphs, we
propose a simple model, called \emph{common graphs}, which we obtained by asking
what, fundamentally, are the \emph{common aspects} of RDF and Property Graphs
(Appendix~\ref{sec:appendix-foundations} gives more details on the distilling of
common graphs).

Let us assume disjoint countable sets of nodes $\Nodes$, values $\Values$,
predicates $\Predicates$, and keys $\Keys$ (sometimes called properties).

% We sometimes say \emph{element} for a node or a value, and \emph{label} for a predicate or key. \todo{Drop if not used.}

\begin{definition}
  A \emph{common graph} is a pair $\graph = (E, \rho)$ where
  \begin{itemize}[\textbullet]
  \item
    $E \subseteq_{\mathit{fin}} \Nodes \times \Predicates \times \Nodes$ is its
    set of edges (which carry predicates), and
  \item
    $\rho \colon \Nodes \times \Keys \pto \Values$ is a finite-domain partial
    function mapping node-key pairs to values.
  \end{itemize}
  The set of nodes of a common graph $\graph$, written $\nodes(\graph)$,
  consists of all elements of $\Nodes$ that occur in $E$ or in the domain of
  $\rho$.
  Similarly, $\keys(\graph)$ is the subset of $\Keys$ that is used in $\rho$,
  and $\values(\graph)$ is the subset of $\Values$ that is used in $\rho$ (that
  is, the range of $\rho$).
\end{definition}

% \begin{example}[Media Service Common Graph] \label{ex:sharedScenario}
% To illustrate the common graphs, we introduce the following scenario. We assume a data model that has users, who can access and own accounts and invite other users to their accounts. Users have keys, such as email and credit-card. An example for this can be seen in~\Cref{fig}.
% % The nodes correspond to  conceptual classes, which will be identified by their available properties and keys. Properties are depicted as directed arrows, and keys are shown inside the conceptual classes.
% % The boxes inform about the available categories of nodes, with the keys they may have available (such as the key $\Exkey{plan}$ for nodes of category ``Account''), and properties connect nodes via directed arrows (such as $\Exprop{buyer}$, which connects nodes of category ``Sale'' and ``Account'').
% \end{example}

\begin{example}
  \label{ex:common-graph}
  Consider Figure~\ref{fig:common-graph}, containing a graph to store
  information about \emph{users} who may have access to (possibly multiple)
  \emph{accounts} in, \eg, a media streaming service.
  In this example, we have six nodes describing four persons ($u_1,...,u_4$) and
  two accounts ($a_1$, $a_2$).
  As a common graph $\graph = (E, \rho)$, the nodes are $a_1$, $u_1$, etc.
  Examples of edges in $E$ are $(u_2, \exaccess,a_1)$ and $(u_3, \exinvited,
  u_2)$.
  Furthermore, we have $\rho(u_2, \exemail) =$ d@d.d and $\rho(a_1,card) =
  1234$.
  So, $E$ captures the arrows in the figure (labelled with predicates) and
  $\rho$ captures the key/value information for each node.
  %
% Moreover, 3 predicates are used, appearing in Figure~\ref{fig:common-graph} as labels on links between nodes, representing the relation~$E$. Nodes are further associated with some key-value pairs, representing the function $\rho$.
  %
  Notice that a person may be the owner of an account, and may potentially have
  access to other accounts.
  This is captured using the predicates $\exowns$ and $\exaccess$, respectively.
  In addition, the system implements an invitation functionality, where users
  may invite other people to join the platform.
  The previous invitations are recorded using the predicate $\exinvited$.
  Both accounts and users may be privileged, which is stored via a Boolean value
  of the key~$\exprivileged$.
  We note that the presence of the key $\exemail$ (\resp, of the key (credit)
  $\excard$) is associated with, and indeed identifies users (\resp, accounts).
\end{example}

% \todo[inline]{In the example, worth noting that the graph node names are names, and not identities. Maybe it would be better to name them A, B, C, D to avoid misunderstanding?}

\begin{figure}[t]
\resizebox{1\linewidth}{!}{
  \includegraphics{example-common.pdf}
}
\Description{A diagram of the user common graph.}
\caption{The media service common graph. }
\label{fig:common-graph}
\end{figure}

It is easy to see that every common graph is a property graph (as per the formal
definition of property graphs~\cite{ABDF23}).
A common graph can also be seen as a set of triples, as in RDF.
Let
\[
  \Triples
=
  \left( \Nodes \times \Predicates \times \Nodes \right)
\;\cup\;
  \left( \Nodes \times \Keys \times \Values \right)\,.
\]
Then, a common graph can be seen as a finite set $\graph \subseteq \Triples$
such that for each $u \in \Nodes$ and $k \in \Keys$ there is at most one
$v \in \Values$ such that $(u, k, v) \in \graph$.
Indeed, a common graph $(E, \rho)$ corresponds to
\[
  E \;\cup\; \{ (u, k, v) \mid \rho(u,k) = v\}\;.
\]
When we write $\rho(u, k) = v$ we assume that $\rho$ is defined on $(u, k)$.

\medskip

\noindent\emph{Throughout the paper we see property graph $\graph$
simultaneously as a pair $(E, \rho)$ and as a set of triples from $\Triples$,
switching between these perspectives depending on what is most convenient at a
given moment.}


\subsection{Node Contents and  Neighbourhoods}

Let $\Records$ be the set of all \emph{records}, \ie, finite-domain partial
functions $r \colon \Keys \pto \Values$.
We write records as sets of pairs $\left\{ (k_1, w_1), \dots (k_n, w_n)
\right\}$ where $k_1, \dots, k_n$ are all different, meaning that $k_i$ is
mapped to $w_i$.

For a common graph $\graph = (E,\rho)$ and node $v$ in $\graph$, by a slight
abuse of notation we write $\rho(v)$ for the record $\left\{ (k, w) \mid
\rho(v,k) = w \right\}$ that collects all key-value pairs associated with node
$v$ in $\graph$.
We call $\rho(v)$ the \emph{content} of node $v$ in $\graph$.
This is how PG-Schema interprets common graphs: it views key-value pairs in
$\rho(v)$ as \emph{properties} of the node $v$, rather than independent,
navigable objects in the graph.

SHACL and ShEx, on the other hand, view common graphs as sets of triples and
make little distinction between keys and predicates.
The following notion---when applied to a node---uniformly captures the local
context of this node from that perspective: the content of the node and all
edges incident with the node.

%\begin{definition}[Neighbourhood]
%Given a common graph $\graph = (E,\rho)$ and a node $v\in\Nodes$, we write $\neigh_\graph(v)$ for the common graph $(E',\rho')$ where $E' = \left \{ (u_1,p,u_2) \in  E \mid u_1 = v \text{ or } u_2 = v\right\}$ and $\rho'$ is obtained by restricting $\rho$ so that $\rho'(v) = \rho(v)$ and $\rho'(u)$ is empty for all $u\neq v$. Similarly, for $w\in\Values$, we let $\neigh_\graph(w)$ be the common graph $(\emptyset,\rho')$ where $\rho'(u) = \left\{(k,w')\in\rho(u)\mid w'=w\right\}$ for all $u\in\Nodes$.
%Given a common graph $\graph$ and a node or value $v\in\Nodes\cup\Values$, the \emph{neighbourhood of $v$ in $\graph$}, written $\neigh_\graph(v)$, is the common graph consisting of triples $(u_1, p, u_2)$ from $\graph$ such that $p\in\Predicates\cup\Keys$ and either $u_1=v$ or $u_2=v$.
%\end{definition}

%That is, for $v\in\Nodes$,  $\neigh_\graph(v)$ is a star-shaped graph where only the central node has non-empty content.  For $w\in\Values$, $\neigh_\graph(w)$ is a graph with no edges and only a single value occurring in the contents of nodes.

%If we view common graphs as sets of triples, $\neigh_\graph(v)$ for $v\in\Nodes\cup\Values$ is simply the set of all triples from $\graph$ that mention $v$.

%We will also use the notion of \emph{partial neighbourhoods}, where only specified subsets of keys and predicates are taken into account.

%It is easiest to define it seeing common graphs as sets of triples.

\begin{definition}[Neighbourhood]
  Given a common graph $\graph$ and a node or value $v \in \Nodes \cup \Values$,
  the \emph{neighbourhood} of $v$ in $\graph$ is $\neigh_\graph(v) = \left\{
  (u_1, p, u_2) \in \graph \mid u_1 = v \text{ or } u_2 = v \right\}$.
  %
% \todo[inline]{Wim: This is ill-defined. We do say before that a common graph can be viewed as a set of triples if we want to think about it as RDF. But this definition should also apply to the PG view. We should be clearer about what we mean with the key/value pairs and only use ingredients from Def 1. In fact, if we take the RDF view, the definition is inconsistent with text below that says that, if $v$ is a value, then the neighborhood has no edges.}
% \todo[inline]{Suggestion to rephrase: introduce $\graph = (E,\rho)$ and say $\neigh_\graph(v) = \{(u_1,p,u_2) \in E \mid ... \} \cup \{???\}$ (Actually I don't understand yet what we want wrt $\rho$.)}
% \todo[inline]{Filip: In many places in the paper we treat $\graph$ as a pair $(E,\rho)$ or as a subset of $\Triples$, whatever is more convenient. It should suffice to warn the reader that we do this. We could write the definition in terms of $(E,\rho)$, but it would be clumsy. I really think it is fine as written.  On the other hand, if this is not helping, we can probably just skip this definition entirely and introduce only the $\pm$ variant of neighbourhoods in the section on ShEx.}
% \todo[inline]{Wim: OK, I understand better now what's intended and clarified below.}
\end{definition}

\todo{JH: Is this actually used anywhere?}

When $v \in \Nodes$, then $\neigh_\graph(v)$ is a star-shaped graph
where only the central node has non-empty content.
When $v \in \Values$, then $\neigh_\graph(v)$ consists of all the nodes in
$\graph$ that have some key with value $v$, which is a common graph with no
edges and a restricted function $\rho$.

%\todo[inline]{Maybe move to respective sections. Could also save space.}


\subsection{Value Types}

We assume an enumerable set of \emph{value types} $\ValueTypes$.
The reader should think of value types as \texttt{integer}, \texttt{boolean},
\texttt{date}, \etc
Formally, for each value type $\vtype \in \ValueTypes$, we assume that there is
a set $\sem{\vtype} \subseteq \Values$ of all values of that type and that each
value $v \in \Values$ belongs to some type, \ie, there is at least one $\vtype
\in \ValueTypes$ such that $v \in \sem{\vtype}$.
Finally, we assume that there is a type $\any \in \ValueTypes$ such that
$\sem{\any} = \Values$.


\subsection{Shapes and Schemas}
\label{ssec:shapes}

We formulate all three schema languages using \emph{shapes}, which are unary
formulas describing the graph's structure around a \emph{focus} node or a value.
Shapes will be expressed in different formalisms, specific to the schema
language; for each of these formalisms we will define when a focus node or value
$v \in \Nodes \cup \Values$ \emph{satisfies} shape $\varphi$ in a common graph
$\graph$, written $\graph, v \models \varphi$.

Inspired by ShEx \emph{shape maps}, we abstract a schema $\schema$ as a set of
pairs $(\sel,\varphi)$, where $\varphi$ is a shape and $\sel$ is a
\emph{selector}.
A selector is also a shape, but usually a very simple one, typically checking
the presence of an incident edge with a given predicate, or a property with a
given key.
A graph $\graph$ is \emph{valid} \wrt $\schema$, in symbols $\graph \models
\schema$, if
\[
  \graph, v \models \sel
\quad \text{implies} \quad
  \graph, v \models \varphi,
\]
for all $v \in \Nodes \cup \Values$ and $(\mathit{sel}, \varphi) \in \schema$.
That is, for each focus node or value satisfying the selector, the graph around
it looks as specified by the shape.
We call schemas $\schema$ and $\schema'$ \emph{equivalent} if $\graph \models
\schema$ \iff $\graph \models \schema'$, for all $\graph$.
In what follows, we may use $\mathit{sel} \Rightarrow \varphi$ to indicate a
pair $(\mathit{sel}, \varphi)$ from a schema $\SHACLSchema$.

% \begin{example}[Schemas over Media Service Common Graph]
%     \label{ex:ShapeExample}

% We stay in the same scenario introduced in \Cref{ex:sharedScenario}. We list here illustrative examples for requirements on common graphs that can be imposed via schemas.  To give an intuitive idea about the selector and the shape, we indicate this informally by splitting the sentences into an initial part that selects nodes or values, and the second part which must hold for these elements:\\
% \noindent
% \emph{For every account}, there must exist a primary credit card ; \\
% \noindent \emph{For every account}, there are  five users of it or less;\\
% \emph{Every owner of an account}, has a unique email address.
% \end{example}

\begin{example}
  \label{ex:constraint-desc}
  We next describe some constraints one may want to express in the domain of
  Example~\ref{ex:common-graph}.
  \begin{enumerate}[(C1)]
  \item
    We may want the values associated to certain keys to belong to concrete
    datatypes, like strings or Boolean values.
    In our example, we want to state that the value of the key $\excard$ is
    always an integer.
  \item
    We may expect the existence of a value associated to a key, an outgoing
    edge, or even a complex path for a given source node.
    For our example, we require that all owners of an account have an email
    address defined.
  \item
    We may want to express database-like uniqueness constraints.
    For instance, we may wish to ensure that the email address of an account
    owner uniquely identifies them.
  \item
    We may want to ensure that all paths of a certain kind end in nodes with
    some desired properties. For example, if an account is privileged, then all
    users that have access to it should also be privileged.
  \item
    We may want to put an upper bound on the number of nodes reached from a
    given node by certain paths. For instance, every user may have access to at
    most 5 accounts.
\end{enumerate}

% \todo[inline]{Wim: Reminder to self. I'd like to illustrate some open/closed things here. (There's no time anymore for this.)}
% \todo[inline]{Wim: More urgently though, we should explain better about how we model things. Let's say that ``users'' are those nodes that have an email key and ``accounts'' are those that have a card key?}
% \todo[inline]{Iovka: I support the need to make this precise. Then, should we use these two selectors in all examples?\\
% Also, we might say that we need this trick because we do not have rdf:type nor labels on nodes.}
% \todo[inline]{Cem: After discussion with Filip, I fixed the setting such that it is keys that identify users and accounts. Problem: this makes C2 awkward. }

\end{example}

% End of file 2-Preliminaries.tex
