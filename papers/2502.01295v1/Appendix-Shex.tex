\section{Standard ShEx}
\label{app:shex-appendix}

\newcommand{\stshex}{s-ShEx\xspace}

The Shape Expressions Language (ShEx)~\cite{PBGK19} and the ShapeMaps
language~\cite{PB19} have been defined by the Shape Expressions Community
group\footnote{\url{https://www.w3.org/community/shex/}} at W3C.
%
% (already in related work)
%The semantics of standard ShEx language has been formalised first in
%\cite{SBG15} for triple expressions, then in \cite{BGP17} for the full
%language, but without inverse predicates.
%
Hereafter, we use \emph{standard ShEx} or \emph{\stshex} to refer to the
language defined in~\cite{PBGK19} and formalised in~\cite{SBG15,BGP17}, while
\emph{ShEx} designates the language presented in the current work.

In this appendix, we support the following
\begin{claim}
\label{claim:app-shex}
  On common graphs, the expressive power of ShEx schemas is equivalent to the
  expressive power of non-recursive \stshex schemas.
\end{claim}

Section~\ref{app:sec-shex-standard-shex-on-common-graphs} explains \stshex on
common graphs, while Section~\ref{app:sec-shex-recursion} explains non-recursive
\stshex.


\subsection{\stshex schema and the validation problem}

A standard ShEx schema is a set of named shape expressions, and it is usually
formalised as a pair $(L, \mathit{def})$, where $L$ is a finite set of shape
names (in practice, these are IRIs) and $\mathit{def}$ is a function that
associates a shape expression with every shape name.
In \stshex, the validation problem $\graph \models \schema$ from
Section~\ref{ssec:shapes} is defined in a different way.
In fact, the ShEx specification~\cite{PBGK19} does not specify what it means for
a graph to be valid \wrt a \stshex schema; it only defines what it means for a
node in a graph to satisfy a shape expression.

However, the problem considered in practice is whether some selected nodes in
the graph satisfy some prescribed shape expressions from the schema.
This is specified by a shape map~\cite{PB19}.
A shape map can be formalised as a set of pairs of the form $(\mathit{sel},
l)$, where $l \in L$ and $\mathit{sel}$ is a unary query.
While the shape map specification~\cite{PB19} allows the selectors from
Definition~\ref{def:shex-selector}, most implementations allow general SPARQL
queries as selectors.

In the current paper, we integrate the shape map into the schema itself, which
allows us to specify the validation problem in a uniform way for the three graph
schema formalisms considered.
Additionally, in shape maps we do not use shape names, but shape expressions
directly; the next section argues why this is not a problem from the point of
view of expressive power for standard ShEx without recursion.


\subsection{Shape names and recursion}
\label{app:sec-shex-recursion}

Recursion is an important mechanism in standard ShEx.
% In this work, however, we consider only ShEx schemas without recursion (to be
% defined shortly).
% This restriction has been made because neither standard SHACL\footnote{There
% are SHACL formalisations that introduce recursion, but the semantics of this
% construct is not specified in the standard.} nor PG-Schema allow for recursion.
% In standard ShEx, 
The fact that shape expressions are named permits to refer to
them using their name.
In particular, these references allow for circular recursive definitions.
As an example, consider the standard ShEx schema in
Figure~\ref{fig:app-shex-schema-example}.
It contains the single shape name \texttt{ex:User}, whose definition is given by
the shape expression inside the curly braces.
The latter shape expression refers to itself: \texttt{@ex:User} indicates a
reference to the shape expression named \texttt{ex:User}.
Concretely, the shape expression requires from an RDF node to have an
\texttt{ex:email} predicate whose value is a string, as well as any number of
\texttt{ex:invited} predicates whose values are nodes that satisfy the shape
expression named \texttt{ex:User}.

\begin{figure}[h]
\centering
{\small
\begin{verbatim}
  PREFIX ex: <http://ex.example/#>
  PREFIX xsd: <http://www.w3.org/2001/XMLSchema#>
  ex:User {
    ex:email   xsd:string ;
    ex:invited @ex:User *
  }
\end{verbatim}}
\vspace{-2em}
\caption{\label{fig:app-shex-schema-example}%
  A standard ShEx schema.}
\end{figure}

A \stshex schema $(L, \mathit{def})$ is called recursive, when there is a shape
name $l \in L$ whose definition $\mathit{def}(l)$ uses a reference \texttt{@}$l$
to itself, either directly or transitively through references to other shape
names.
Every standard non-recursive ShEx schema can be rewritten to an equivalent
schema without references, simply by replacing every reference with its
definition.
In other words, references in standard ShEx do not add expressive power for
non-recursive schemas.
Therefore, here after we will present standard ShEx without references.


\subsection{Syntax of standard ShEx on common graphs}
\label{app:sec-shex-standard-shex-on-common-graphs}

We present here a version of \stshex restricted on common graphs.
The principle difference between \stshex described here and the ShEx
recommendation~\cite{PBGK19} resides in so called \emph{node
constraints}\footnote{\url{http://shex.io/shex-semantics/\#node- constraints}}.
These are constraints to be verified on the actual node of an RDF graph (which
is an IRI, a literal or a blank node) without considering its neighbourhood.
As pointed out in Section~\ref{app:sec-foundations-comparison-rdf}, such
constraints are irrelevant for nodes (\ie, elements of $\Nodes$) in common
graphs.
Therefore, we restrict \stshex node constraints on values (elements of
$\Values$) only.
\stshex node constraints on values correspond to the atomic shape expressions
$\shextest(\vtype)$ and $\shexhasvalue(c)$ in ShEx.
Their expressive power can be entirely captured by selecting for $\ValueTypes$ a
language equivalent to node constraints on values in \stshex.
Note finally that $\shextest(\any)$ in ShEx allows to distinguish nodes from
values.

\newcommand{\stse}{\mathit{se}}
\newcommand{\stte}{\mathit{te}}
\newcommand{\stclosed}{\mathsf{closed}}
\newcommand{\stextra}{\mathsf{extra}}
\newcommand{\stshape}{\mathit{sh}}
\newcommand{\sttc}{\mathit{tc}}
\newcommand{\stcard}{\mathit{card}}
\newcommand{\normalised}[1]{\tilde{#1}}
\newcommand{\trtefromst}[1]{\tau_{\text{e}}(#1)}
\newcommand{\trfromst}[1]{\tau(#1)}
\newcommand{\trtetost}[1]{\sigma_{\text{e}}(#1)}
\newcommand{\trtost}[1]{\sigma(#1)}
\newcommand{\stses}[2]{\mathit{SE}_{#2}(#1)}
\newcommand{\stextraconstr}[2]{\eta^{#2}_{#1}}
\newcommand{\preds}{\mathit{preds}}

Figure~\ref{fig:app-standard-shex-syntax} gives an abstract syntax for \stshex
\emph{shape expressions} $\stse$ and \emph{triple expressions} $\stte$
restricted on common graphs \wrt node constraints, as discussed above.
The non-terminal $\mathit{sh}$ corresponds to \emph{Shape}s, while the
non-terminal $\mathit{tc}$ is for
\emph{TripleExpression}s\footnote{\url{http://shex.io/shex-semantics/\#shapes-
and-TEs}}.
This abstract syntax is intended to be understandable by those familiar with
standard ShEx after taking into account the following purely syntactic
differences:
\begin{itemize}
\item
  we use the mathematical notation $\land$, $\lor$ and $\neg$ for the
  \stshex operators $\mathsf{and}$, $\mathsf{or}$ and $\mathsf{not}$;
\item
  according to the ShEx specification, the $\stextra\ Q$ modifier is optional
  for shapes; however, an absent extra set is equivalent to $\stextra \
  \emptyset$, therefore we will consider that it is always present.
\end{itemize}

\begin{figure}[h]
\begin{align*}
  \stse
\gDef
  &     \shexhasvalue(c)
  \gMid \shextest(\text{$\vtype$})
  \gMid \stshape
  \gMid \stse \land \stse
  \gMid \stse \lor \stse
  \gMid \neg \stse
\gEnd \\
  \stshape
\gDef
  &     \stextra\ Q\ \shexneigh{\stte}
  \gMid \stclosed\ \stextra\ Q\ \shexneigh{\stte}
\gEnd \\
  \stte
\gDef
  &     \sttc
  \gMid \stte \shexeach \stte
  \gMid \stte \!\shexone\! \stte
  \gMid \stte {[\mathit{min}; \mathit{max}]}
\gEnd \\
  \sttc
\gDef
  &     q\ \stse
  \gMid q\ .
\gEnd
\end{align*}
with $c \in \Nodes \cup \Values$, $\vtype \in \ValueTypes$, $q \in \Predicates
\cup \Keys \cup \Predicates^{-} \cup \Keys^{-}$, $Q \subseteq_{\mathit{fin}}
\Predicates \cup \Keys \cup \Predicates^{-} \cup \Keys^{-}$, $\mathit{min} \in
\mathbb{N}$ and $\mathit{max} \in \mathbb{N} \cup \{*\}$.
\caption{\label{fig:app-standard-shex-syntax}%
  Abstract syntax for \stshex.}
\end{figure}


\subsection{Translations between ShEx and \stshex}

In this section, we introduce a back and forth translation between non-recursive
\stshex and ShEx.
We claim that these translations preserve the semantics \wrt the validity of a
graph.
The claim is presented without a correctness proof, as it
would require to introduce here a formal semantics for \stshex.
However, it is not difficult to write a proof making use of the formal semantics
from~\cite{BGP17}.


\subsubsection{Differences between \stshex and ShEx}

We now list the syntactic differences between the two languages, and describe
how they are handled by the translation:
\begin{itemize}[\textbullet]
\item
  Triple constraints in \stshex (non-terminal $\sttc$) allow to use a $.$
  (dot) instead of the shape expression, which is in fact equivalent to the
  ShEx shape expression $\shexneigh{\shexallte}$.
\item
  Triple expressions in ShEx contain the atomic expression $\varepsilon$,
  while \stshex does not allow it directly.
  On the other hand \stshex allows us to use intervals of the form
  $[\mathit{min};\mathit{max}]$ to define bounded or unbounded repetition,
  while ShEx allows only the unbounded repetition $*$.
  We show in Section~\ref{app:sec-shex-syntax-triple-expr} that the two
  variants have equivalent expressive power.
\item
  In \stshex, the atomic shape expression that defines the neighbourhood of a
  node (non-terminal $\stshape$) is parametrised by a set $Q$ of \emph{extra}
  (possibly inverse) predicates and keys.
  In Section~\ref{app:sec-shex-eliminate-extra}, we show that $\stextra$ is
  just syntactic sugar in \stshex.
\item
  In \stshex, the atomic shape expression derived from the non-terminal
  $\stshape$ can have an optional $\stclosed$ modifier.
  On the other hand, ShEx introduces the triple expressions $\shexneg{P}$ and
  $\shexneginv{P}$ (for $P \subseteq \Predicates \cup \Keys$).
  As we will see, the latter are used when translating \stshex to ShEx in
  order to distinguish between $\stclosed$ and non-$\stclosed$ \stshex
  shape expressions.
\end{itemize}


\subsubsection{Normalised triple expressions}
\label{app:sec-shex-syntax-triple-expr}

We now show how both \stshex and ShEx triple expressions can be normalised so as
to use a limited number of operators; this shall be useful for the translation
between \stshex and ShEx.


\paragraph{Normalisation of \stshex triple expressions}

A \stshex triple expression is called \emph{normalised} if it uses only the
intervals $[0; 1]$ and $[0; *]$; these can be normalised using rewriting rules
based on the following equivalences:
\begin{align*}
  \stte[\mathit{min};*]
& =
  \stte[0;*] \shexeach \underbrace{\stte \shexeach \cdots \shexeach
  \stte}_{\mathit{min} \text{ times}}
\\
  \stte[\mathit{min};\mathit{max}]
& =
  \underbrace{\stte \shexeach \cdots \shexeach \stte}_{\mathit{min} \text{
  times}} \shexeach \underbrace{\stte[0;1] \shexeach \cdots \shexeach
  \stte[0;1]}_{\mathit{max}-\mathit{min} \text{ times}} \quad \text{ when }
  \mathit{max} \not= *
\end{align*}


\paragraph{Normalisation of ShEx triple expressions}

Here after, $\tte$ designates a \todo{cosed!}closed ShEx triple expression derivable from the
rule $f$ of the grammar in Definition~\ref{def:shex-syntax}.
For every $\tte$, we define $\tte^{?} = \tte \shexone \varepsilon$.
A triple expression $\tte$ is \emph{normalised} if either $\tte = \varepsilon$,
or $\tte$ does not use $\varepsilon$ as sub-expression, but can use the $?$
operator defined above.
Every triple expression can be normalised by eliminating occurrences of
$\varepsilon$ using the $?$ operator and the following two properties:
\begin{itemize}
\item
  $\varepsilon$ is a neutral element for the $\shexeach$ operator, \ie, $\tte
  \shexeach \varepsilon = \varepsilon \shexeach \tte = \tte$ for every ShEx
  triple expression $\tte$,
\item
  $\varepsilon^{*} = \varepsilon$.
\end{itemize}
\Wlogx, \textbf{from now on we only consider normalised triple expressions}.


\subsubsection{Direct predicates of triple expressions}
\label{app:sec-shex-define-preds-of-triple-expr}

This section is devoted to the introduction of two technical definitions.
For every triple expression we define the set of (possibly inverted) predicates
and keys that appear directly in the expression.
Formally, if $\tte$ is a ShEx triple expression derived by the \todo{check with grammar} third rule of
the grammar in Definition~\ref{def:shex-syntax}, then we define the set
$\preds(\tte) \subseteq \Predicates \cup \Keys \cup \Predicates^{-} \cup
\Keys^{-}$ inductively on the structure of $\tte$ by:
\begin{align*}
  \preds(\varepsilon)
& =
  \emptyset
\\
  \preds(p.\se)
& =
  \{ p \}
\\
  \preds(\shexinverse{p}.\se)
& =
  \{ \shexinverse{p}\}
\\
  \preds(\se \shexeach \se')
& =
  \preds(\se) \cup \preds(\se')
\\
  \preds(\se \shexone \se')
& =
  \preds(\se) \cup \preds(\se')
\\
  \preds(\se^{*})
& =
  \preds(\se)
\end{align*}
For a \stshex triple expression $\stte$, the set $\preds(\stte)$ is defined
similarly (recall that $q \in \Predicates \cup \Keys \cup \Predicates^{-} \cup
\Keys^{-}$):
\begin{align*}
  \preds(q\ \stse)
& =
  \{ q \}
\\
  \preds(q\ .)
& =
  \{ q \}
\\
  \preds(\stse \shexeach \stse')
& =
  \preds(\stse) \cup \preds(\stse')
\\
  \preds(\stse \shexone \stse')
& =
  \preds(\stse) \cup \preds(\stse')
\\
  \preds(\stte[\mathit{min}; \mathit{max}])
& =
  \preds(\stse)
\end{align*}

\subsubsection{Eliminating $\stextra$ from \stshex}
\label{app:sec-shex-eliminate-extra}

We show by means of an example how the extra construct can be eliminated in
\stshex.
%The same example will be used later on for the translation from standard ShEx
%to ShEx, therefore it is described in detail.

\begin{example}
\label{ex:app-shex-extra}
Consider the \stshex shape expression
\begin{align*}
  \stse
& =
  \stextra\ \{p_1, p_2\}\ \ \shexneigh{\stte}
\\
  \text{with} \qquad \stte
& =
  p_1 \shexneigh{p\ .} \,\shexeach\, p_1 \shexneigh{p'\ .} \,\shexeach\, p_3\ .
\\ %\,\shexeach\, \shexinverse{p}_4\ . \\
  \text{and}\qquad \quad
&
  \quad p_1, p_2, p_3, p, p' \in \Predicates \cup \Keys
\end{align*}
that has a set of extra predicates and keys $\{ p_1, p_2 \}$.
It is satisfied by nodes whose neighbourhood can have any incoming triples and
has the following outgoing triples:
\begin{enumerate}
\item
  one $p_1$-triple leading to a node that satisfies $\shexneigh{p\ .}$,
\item
  another $p_1$-triple leading to a node that satisfies $\shexneigh{p'\ .}$,
\item
  a $p_3$-triple leading to an unconstrained node,
\item
  \label{req:extrap1}
  because $p_1$ appears in the $\stextra$ set, other $p_1$-triples are also
  allowed as long as they satisfy \textbf{none} of the constraints present for
  $p_1$ in $\stte$, that is, they satisfy neither $\shexneigh{p\ .}$ nor
  $\shexneigh{p'\ .}$,
\item
  \label{req:extrap2}
  because $p_2$ appears in the $\stextra$ set too, $p_2$-triples are allowed and
  their target is not constrained because $p_2$ does not appear in the triple
  expression $\stte$,
\item
  \label{req:closed}
  finally, since the shape is not $\stclosed$, all outgoing triples whose
  predicate is not in $\{ p_1, p_3 \}$ are allowed, noting that $\{ p_1, p_3 \} =
  \preds(\stte) \cap \Predicates \cap \Keys$.
\end{enumerate}
% The node has also the following incoming triples:
% \begin{enumerate}
% \addtocounter{enumi}{6}
% \item one incoming $p_4$-triple coming from an unconstrained node,
% \item because standard ShEx does not close the incoming triples, all incoming triples whose predicate is different from $p_4$ are allowed.
% \end{enumerate}
\end{example}
The shape expression $\stse$ from Example~\ref{ex:app-shex-extra} is equivalent
to the following shape expression without extra:
\[
  \shexneigh{ \stte \;\shexeach\; \stte_{p_1}^{*} \;\shexeach\; \stte_{p_2}^{*}}
\]
where
\[
  \stte_{p_1}
=
  p_1\ \left( \neg \shexneigh{p\ .} \land \neg \shexneigh{p'\ .} \right)
\qquad \text{ and } \qquad
  \stte_{p_2}
=
  p_2\ .
\]
The sub-expression $\stte_{p_1}^{*}$ allows to satisfy the
requirement~(\ref{req:extrap1}) from Example~\ref{ex:app-shex-extra}, while the
sub-expression $\stte_{p_2}^{*}$ allows to satisfy the
requirement~(\ref{req:extrap2}).

The construction for eliminating $\stextra$ just discussed can be generalised
to arbitrary shape expressions.
The idea is to combine (with the $\shexeach$ operator) the initial triple
expression with a sub-expression of the form $q\ \stse_q^{*}$ for every
(possibly inverse) $\stextra$ predicate $q$, where $\stse_q$ is the conjunction
of the negated shape expressions $\stse'$ such that $q\ \stse'$ appears directly
in $\stte$ (without traversing any shape expressions $\stse$).

\Wlogx, \textbf{from now on, we consider only \stshex shape expressions without
$\stextra$}.


\subsubsection{Translation from \stshex to ShEx}

With every \stshex shape expression $\stse$ we associate the ShEx shape
expression $\trfromst{\stse}$ as defined in
Table~\ref{tab:app-shex-translation-from-standard}.
It is defined by mutual recursion with the corresponding translation function
$\trtefromst{\stte}$ for standard ShEx triple expressions $\stte$ presented in
Table~\ref{tab:app-shex-translation-from-standard-te}.


\begin{table}[ht]
\caption{\label{tab:app-shex-translation-from-standard-te}%
  Translation from \stshex to ShEx for normalised triple expressions $\stte$,
  with $q \in \Predicates \cup \Keys \cup \Predicates^{-} \cup \Keys^{-}$.}
\centering
\begin{tabular}{ll}
\toprule
  $\stte$                   & $\trtefromst{\stte}$ \\
\midrule
  $q\ \stse$                & $p. \,\trfromst{\stse}$ \\
  $q\ .$                    & $p.\shextop$ \\
  $\stte \shexeach \stte'$  & $\trtefromst{\stte} \shexeach
                              \trtefromst{\stte'}$ \\
  $\stte \shexone \stte'$   & $\trtefromst{\stte} \shexone
                              \trtefromst{\stte'}$ \\
  $\stte[0;*]$              & $\trtefromst{\stte}^{*}$ \\
  $\stte[0;1]$              & $\trtefromst{\stte} \shexone \varepsilon$ \\
\bottomrule
\end{tabular}
\end{table}

\begin{table}[ht]
\caption{\label{tab:app-shex-translation-from-standard}%
  Translation from \stshex to ShEx for shape expressions.}
\centering
\begin{tabular}{ll}
\toprule
  $\stse$                         & $\trfromst{\stse}$ \\
\midrule
  $\shexhasvalue(c)$              & $\shexhasvalue(c)$ \\
  $\trfromst{\shextest(\vtype)}$  & $\shextest(\vtype)$ \\
  $\stse \land \stse'$            & $\trfromst{\stse} \land
                                    \trfromst{\stse'}$ \\
  $\stse \lor \stse'$             & $\trfromst{\stse} \lor \trfromst{\stse'}$\\
  $\neg \stse$                    & $\neg \trfromst{\stse}$ \\
  $\stclosed\ \shexneigh{\stte}$  & $\shexneigh{\,\trtefromst{\stte}
                                    \,\shexeach\, ({\shexneginv{R}})^{*}\,}$ \\
                                  & \qquad with $R = \preds(\stte) \cap
                                    (\Predicates^{-} \cup \Keys^{-})$ \\
  $\shexneigh{\stte}$             & $\shexneigh{\,\trtefromst{\stte}
                                    \,\shexeach\, ({\shexneginv{R}})^{*}
                                    \,\shexeach\, (\shexneg{Q})^{*}\,}$ \\
                                  & \qquad with $R = \preds(\stte) \cap
                                    (\Predicates^{-} \cup \Keys^{-})$ \\
                                  & \qquad and $Q = \preds(\stte) \cap
                                    (\Predicates \cup \Keys)$ \\
\bottomrule
\end{tabular}
\end{table}


\OMIT{
% The example is too complex, not sure that this is a clarification
\begin{example}
The standard ShEx shape expression $\stse$ from Example~\ref{ex:app-shex-extra}
gives the following in ShEx:
$$
\shexneigh{\; \te \;\shexeach\; \te_{p_1}^{*} \;\shexeach\; \te_{p_2}^{*}
\;\shexeach\; (\shexneginv{Q})^{*} \;\shexeach\; (\shexneg{P})^{*}}
$$
where $\te = \trtefromst{\stte}$, $\te_{p_1}^{*}$ accounts for the extra $p_1$,
$\te_{p_2}^{*}$ accounts for the extra $p_2$, $\neg Q$ are the predicates open
for incoming edges, and $\neg P$ are the predicates open for outgoing edges.
\begin{align*}
\te = &\ p_1. \se_1 \shexeach p_1. \se'_1 \shexeach p_3.\shextop \shexeach
\shexinverse{p_4}.\shextop\\
\te_{p_1} = &\ \\
\te_{p_2} = &\ \\
Q = &\ \\
P = &\ \\
%   \se_1 = &\ \shexneigh{
%   p.\shextop \shexeach (\shexneginv{\emptyset}^{*} \shexeach (\shexneg{\{p\}})^{*}
%   }
\end{align*}
\begin{align*}
\stse &= \stextra\ \{p_1, p_2\}\ \ \shexneigh{\stte} \\
\text{with}\qquad \stte &= p_1 \shexneigh{p\ .} \,\shexeach\, p_1 \shexneigh{p'\ .} \,\shexeach\, p_3\ . \,\shexeach\, \shexinverse{p}_4\ . \\
\text{and}\qquad \quad & \quad p_1, p_2, p_3, p_4, p, p' \in \Predicates \cup \Keys
\end{align*}

$$
\shexneigh{ \stte \;\shexeach\; \stte_{p_1}^{*} \;\shexeach\; \stte_{p_2}^{*}}
$$
where
$$\stte_{p_1} = p_1\ \left(\neg \shexneigh{p\ .} \land \neg \shexneigh{p'\ .}\right)
\qquad \text{ and } \qquad\stte_{p_2} = (p_2\ .)$$

\end{example}
}%end OMIT


\subsubsection{Translation from ShEx to \stshex}

Unless otherwise specified, in the sequel, $\tte$ designates a closed \todo{closed!} ShEx
triple expression produced by the non-terminal $\tte$ of the grammar in
Definition~\ref{def:shex-syntax}.
In Table~\ref{tab:app-shex-translation-to-standard} we present a function that
with every normalised ShEx triple expression $\tte$ associates the standard ShEx
triple expression $\trtetost{\tte}$.
It is defined by mutual recursion with the translation function that with every
ShEx shape expression $\se$ associates a standard ShEx shape expression
$\trtost{\se}$, and that will be presented shortly.
Note that the case $\tte = \varepsilon$ is omitted in
Table~\ref{tab:app-shex-translation-to-standard}: recall that in normalised ShEx
triple expressions, $\varepsilon$ can only appear standalone (not in
sub-expressions), therefore the case $\tte = \varepsilon$ will be treated with
shape expressions.

\begin{table}[ht]
\caption{\label{tab:app-shex-translation-to-standard}%
  Translation from ShEx to \stshex for normalised triple expressions.}
\centering
\begin{tabular}{ll}
\toprule
  $\tte$                  & $\trtetost{\tte}$ \\
\midrule
  $p.\se$                 & $p\ \trtost{\se}$\\
  $\shexinverse{p}.\se$   & $\shexinverse{p}\ \trtost{\se}$\\
  $\tte \shexeach \tte'$  & $\trtetost{\tte} \shexeach \trtetost{\tte'}$\\
  $\tte \shexone \tte'$   & $\trtetost{\tte} \shexone \trtetost{\tte'}$\\
  $\tte^{*}$              & $\trtetost{\tte}[0;*]$\\
  $\tte^{?}$              & $\trtetost{\tte}[0;1]$\\
\bottomrule
\end{tabular}
\end{table}

The definition of $\trtost{\se}$ is straightforward for the following cases:
\begin{align*}
  \trtost{\shexhasvalue(c)}
& =
  \shexhasvalue(c) \\
  \trtost{\shextest(\vtype)}
& =
  \shextest(\vtype) \\
  \trtost{\se \land \se'}
& =
  \trtost{\se} \land \trtost{\se'} \\
  \trtost{\se \lor \se'}
& =
  \trtost{\se} \lor \trtost{\se'} \\
  \trtost{\neg \se}
& =
  \neg \trtost{\se}
\end{align*}
\todo{check with new terminology}
The remaining case is for a shape expression of the form $\shexneigh{\te} =
\shexneigh{\tte \shexeach \cdots}$ where $\tte$ is normalised.
Consider the most general case
\[
  \te
=
  \tte \shexeach (\shexneginv{P})^{*} \shexeach (\shexneg{Q})^{*}
\]
Let also
\begin{align*}
  \{ p_1, \ldots, p_m \}
& =
  (\preds(\tte) \cap \Predicates^{-} \cap \Keys^{-}) \setminus P \\
  \{ q_1, \ldots, q_n \}
& =
  (\preds(\tte) \cap \Predicates \cap \Keys) \setminus Q.
\end{align*}
Intuitively, $\{ p_1, \ldots, p_m \}$ is the set of predicates that are not
allowed to appear on incoming edges in the neighbourhoods defined by $\te$, and
similarly $\{q_1, \ldots, q_n\}$ are the forbidden predicates for outgoing
edges.
Then
\[
  \trtost{\shexneigh{\te}}
=
  \left\{
  \begin{array}{l}
    \trtetost{\tte} \;\shexeach \\
    \shexinverse{p_1}\ .[0;0] \shexeach \cdots \shexeach \shexinverse{p_m}\
    .[0;0] \;\shexeach \\
    q_1\ .[0;0] \shexeach \cdots q_n\ .[0;0]
  \end{array}
  \right\}
\]
If $\tte = \varepsilon$, then the term $\trtetost{\tte}$ on the first line of
the definition of $\trtost{\shexneigh{\te}}$ must be omitted.

The remaining case for the definition of $\trtost{\shexneigh{\te}}$ is for
\[
\te = \tte \shexeach (\shexneginv{P})^{*}
\]
Let $\{p_1, \dots, p_m\}$ be as before.
Then
\[
  \trtost{\shexneigh{\te}}
=
  \stclosed \
  \left\{
  \begin{array}{l}
    \trtetost{\tte} \;\shexeach \\
    \shexinverse{p_1}\ .[0;0] \shexeach \cdots \shexeach \shexinverse{p_m}\
    .[0;0]
  \end{array}
  \right\}
\]
As before, if $\tte = \varepsilon$, then the term $\trtetost{\tte}$ must be
omitted.

This concludes the demonstration of Claim~\ref{claim:app-shex}.

\subsection{Comparative expressiveness of ShEx and SHACL}

In Section~\ref{app:indistinguishabilitySHACL} we show a property expressible in ShEx but not in SHACL, while in Section~\ref{app:indistinguishabilityShEx} we show a property expressible in SHACL but not in ShEx.


\subsubsection{Indistinguishability by SHACL}
\label{app:indistinguishabilitySHACL}

\todo[inline]{Iovka: Introductory sentence for the section.}

\ognjen{Overall, it appears correct.
Most of the cases by induction are claimed to be straightforward, without proving. Perhaps, it can be simplified since it proves a slightly bigger property here than needed... or we define it as an auxiliary lemma.}
\begin{proposition}\label{prop:shacl-inexp}
  The ShEx schema $\shexschema^{eq}$ from Example~\ref{ex:SheXCounting}
  cannot be expressed in SHACL, i.e.\,there is no  SHACL schema $\SHACLSchema'$ such that $\graph$ is valid w.r.t.\,$\SHACLSchema'$ iff $\graph$ is valid w.r.t.\, $\shexschema^{eq}$, for any graph $\graph$. 
\end{proposition}
\begin{proof}
\ognjen{I would provide more guidance wrt to the proof structure, and the idea in general}
To prove this proposition, we first need some preparations. For a node $c$ and an integer $n>0$, a \emph{$(c,n)$-neighbourhood} is a graph $\graph=\{(c,p_1,v_1),\ldots,(c,p_k,v_k)\}$ such that 
\begin{enumerate}
    \item $c,v_1,\ldots,v_k$ are distinct nodes,
    \item for every property $p$, either $p$ does not occur in $\graph$ or $p$ occurs in at least $n$ triples of $\graph$. 
  \end{enumerate}
We say two $(c,n)$-neighbourhoods $\graph_1,\graph_2$ are \emph{similar}, if exactly the same properties appear in $\graph_1$ and $\graph_2$. Intuitively, if $\graph_1,\graph_2$ are similar, then for all properties $p$  we have that either (1) $p$ does not occur in $\graph_1$ nor in  $\graph_2$, or (2) in both $\graph_1$ and $\graph_2$ the node $c$ has at least $n$ outgoing $p$-edges.

Assume a SHACL schema $\SHACLSchema$. We assume that $\SHACLSchema$ does not use expressions of the form $\leqn{n}{\pathExpr}{\varphi}$. This can be assumed w.l.o.g.\,since $\leqn{n}{\pathExpr}{\varphi}$  can be written as $\neg \geqn{n+1}{\pathExpr}{\varphi}$. Let $k$ be the maximal integer that appears among the numeric restrictions of the form  $\geqn{n}{\pathExpr}{\varphi}$ and $\leqn{n}{\pathExpr}{\varphi}$ in $\SHACLSchema$\todo{$\le n$ not in $\SHACLSchema$?}. Assume we have two 
$(c,k+1)$-neighbourhoods $\graph_1,\graph_2$ such that (a) $\graph_1,\graph_2$ are similar, and (b) 
for all nodes $d$ that appear in $\varphi$,\ognjen{fir any node $d$?}  we have that \ognjen{either?} (i) $d=c$, or (ii)  $d$ does not occur in $\graph_1$ or in  $\graph_2$. Then we have ($\dagger$) $\graph_1$ validates w.r.t.\,$\SHACLSchema$ iff $\graph_2$ validates w.r.t.\,$\SHACLSchema$. 
\ognjen{maybe I missed, validates wrt to a node $c$ or some target} 
\todo[inline]{Iovka: "all nodes $d$ that appear in $\varphi$ ?}

To see the above claim, take the binary relation as follows:
\[R=\{(c,c)\}\cup \{(u,v)\mid \exists p: (c,p,u)\in \graph_1,(c,p,v)\in \graph_2 \}.\] 
\ognjen{Not important, but 
I would use something else than $R$, maybe $\sim$ or $\approx$}
We can show  that the following holds for all shape expressions $\varphi$ and property paths $\pi$ that appear in $\SHACLSchema$:
\begin{enumerate}
    \item $\graph_1,d_1\models \varphi$ iff $\graph_2,d_2\models \varphi$, for all $(d_1,d_2)\in R$, and 
    \item $(d_1,d_1')\in \sem{\pi}^{\graph_1}$ iff $(d_2,d_2')\in \sem{\pi}^{\graph_2}$, for all $(d_1,d_2),(d_1',d_2')\in R$.
\end{enumerate}
\ognjen{This is a more general property than we need for the proof. I am wondering if we focus on the exact graph $\G$ defined below we could simplify a bit}
Note that the claim ($\dagger$) follows from (1) above as a special case: since $(c,c)\in R$, we get that $\graph_1,c\models \varphi$ iff $\graph_2,c\models \varphi$. The claims (1-2) are shown by induction on the structure of $\varphi$ and $\pi$. 

We start with the claim (1). Assume arbitrary $(d_1,d_2),(d_1',d_2')\in R$, and consider the possible cases for $\pi$:
\begin{enumerate}[(i)]
    \item $\pi=p$ for some property $p$. Assume  $(d_1,d_1')\in \sem{p}^{\graph_1}$. Then $d_1=c$ and $(c,p,d_1')\in \graph_1$. Since $(d_1',d_2')\in R$, we have that $(c,p,d_2')\in \graph_2$. By the definition of $R$, $d_2=c$ and thus $(d_2,d_2')\in\sem{p}^{\graph_2}$. The other direction is symmetric.
    \item $\pi=k$ for some key $k$. Then trivially $\sem{k}^{\graph_1}=\sem{k}^{\graph_2}=\emptyset$  by the definition of $(c,n)$-neighborhoods and the claim follows.
   
   \item The remaining cases for $\pi=\pi_1^{-}$, $\pi=\pi_1\cdot \pi_2$, $\pi=\pi_1\cup \pi_2$, and $\pi=\pi_1^{*}$ are shown by a straightforward application of the induction hypothesis and the semantics of the operators four operators. 
\end{enumerate}

We can now proceed to prove claim (2). \ognjen{(1)?} Assume arbitrary $(d_1,d_2)\in R$. We only show that  $\graph_1,d_1\models \varphi$ implies $\graph_2,d_2\models \varphi$. The other direction is symmetric. The proof is by structural induction on $\varphi$. Assume $\graph_1,d_1\models \varphi$, and consider the possible cases for $\varphi$:
\begin{enumerate}[(a)]
  \item $\varphi = \geqn{n}{\pathExpr}{\varphi_1}$. Assume  $\graph_1,d_1\models \varphi$. Take the set $F=\{b\mid (d_1,b) \in \iexpr{\pathExpr}{\graph} \land\graph_1,b\models\varphi_1 \}$. There can be two cases: $F=\{c\}$ and $F\neq \{c\}$.

  \medskip
  Suppose $F=\{c\}$. Thus $n=1$ and $\graph_1,c\models\varphi_1$. Since $(c,c)\in R$ and by the induction hypothesis, we get $\graph_2,c\models\varphi_1 $. Moreover, given $(d_1,d_2)\in R$, from  $(d_1,c)\in \sem{\pi}^{\graph_1}$ we infer $(d_2,c)\in \sem{\pi}^{\graph_2}$. Thus we get   
   $\graph_2,d_2\models \geqn{n}{\pathExpr}{\varphi_1}$.

\medskip
  Suppose $F\neq \{c\}$. Since $|F|>0$, there is some $e\in F$ and a unique property $p$ such that $(d_1,p,e)\in \graph_1$. Since $\graph_2$ is a $(c,k+1)$-neigborhood similar to $\graph_1$, we have that $\graph_2$ has $k+1$ distinct edges $(c,p,e_1),\ldots,(c,p,e_{k+1})$ with $(e,e_1),\ldots,(e,e_{k+1})\in R$. Note that $n<k+1$. Since $\graph_1,e\models\varphi_1$, using the induction hypothesis we get that $\graph_2,e_j\models\varphi_1$ for all $1 \leq j \leq k+1$. Moreover, from $(d_1,e)\in \sem{\pi}^{\graph_1}$ we get that  
$(d_2,e_j)\in \sem{\pi}^{\graph_2}$ for all  $1 \leq j \leq k+1$. Thus we get   
   $\graph_2,d_2\models \geqn{n}{\pathExpr}{\varphi_1}$.  
 \item The remaining cases are straightforward.  
\end{enumerate}

%The only interesting aspect is that for all $(d_1,d_2)\in R$ and all property paths $\pi$ in $\SHACLSchema$ we have $\{(d_1',d_2')\mid (d_1,d_1')\in \sem{\pi}^{\graph_1},(d_2,d_2')\in \sem{\pi}^{\graph_2}\} \subseteq R$.

%$\{u\mid (d_1,u)\in \sem{\pi}^{\graph_1}\}$

We can now come back to the main claim of the proposition. %  Take the ShEx schema  $\shexschema$ consisting of the following statement: 
% \[ \shexhasvalue(c) \Rightarrow \shexneigh{(p\shexeach q)^{+}}\]
Towards a contradiction, suppose that there exists a SHACL schema $\SHACLSchema'$ such that $\graph$ is valid w.r.t.\,$\SHACLSchema'$ iff $\graph$ is valid w.r.t.\,$\shexschema^{eq}$, for any graph $\graph$. 

Let $k$ be the maximal integer that appears among the numeric restrictions of the form  $\geqn{n}{\pathExpr}{\varphi}$ and $\leqn{n}{\pathExpr}{\varphi}$ in $\SHACLSchema'$.

Take the graph
\[\graph=\{(c,p,v_j),(c,q,w_j)\mid 1 \leq j \leq k+1\},\]
where none of $v_j$ and $w_j$ appear in $\SHACLSchema'$.
Note that here $\graph$ is such that  $c$ has exactly the same number (i.e.,\,$k+1$) $p$-edges and $q$-edges. Clearly, $\graph$ validates w.r.t.\,$\shexschema$, and hence also $\graph$ validates w.r.t.\,$\SHACLSchema'$

Consider a new graph $\graph'=\graph\cup \{(c,p,u) \}$, where $u$ does not appear in $\graph $. We have that the node $c$ in  $\graph'$ has more outgoing $p$-edges that the number of outgoing $q$-edges, and thus $\graph'$ does not validate w.r.t.\,$\shexschema$. Observe that $\graph$ and $\graph'$ are $(c,k+1)$-neighbourhoods that are similar in the sense defined above, and thus due to ($\dagger$), we have that $\graph'$ does validate w.r.t.\,$\SHACLSchema'$. Contradiction. 
   \end{proof}




\subsubsection{Indistinguishably by ShEx}
\label{app:indistinguishabilityShEx}

%For a graph $\graph$ and edge $e$ in $\graph$, let $\copyswap(\graph, e)$ be the graph obtained by copying $\graph$, then swapping the targets of edge $e$ and its copy $e'$. For instance, in Fig.~\ref{fig:example-shex-counts-edges}, the graph on the right is the copy-swap of the graph on the left and its $\exaccess$ edge (where each node and its copy are horizontally aligned). We can then show that $\graph$ and $\copyswap(\graph, e)$ satisfy the same ShEx schemas (see Appendix~\ref{app:indistinguishabilityShEx}). 

The two graphs in Figure~\ref{fig:example-shex-counts-edges} cannot be distinguished by a ShEx schema.
\todo[inline]{Iovka: adapt to how this is treated in the paper.}
In fact, we show a stronger property. 
Let $\graph = (E, \rho)$ be a graph and $e = (u, p, v) \in E$.
%
A \emph{double} of $\graph$ is a graph of the form $\graph \cup \graph'$ together with a bijection $d: \nodes(\graph) \to \nodes(\graph')$, where $\graph' = (E', \rho')$ is a disjoint copy of $\graph$.
%
Now, let $\graph \cup \graph'$ as above be a double of $\graph$, with bijection $d$.
Then $\copyswap(\graph, e)$ is the graph $(E'', \rho \cup \rho')$ such that 
$$E'' = E \cup E' \setminus \{e, d(e)\} \cup \left\{(u, p, d(v)), (d(u), p, v)\right\}.$$

Back to the graphs in Fig.~\ref{fig:example-shex-counts-edges}, we have $\graph_{\text{right}} = \copyswap(\graph_{\text{left}}, e)$, where $e$ is the unique edge in $\graph_{\text{left}}$ labelled $\exaccess$, and $\graph_{\text{left}}$, resp. $\graph_{\text{right}}$, are the graphs on the left, resp. on the right, in Fig.~\ref{fig:example-shex-counts-edges}.

\begin{lemma}
  For every ShEx schema $\shexschema$, every graph $\graph$ and every edge $e$ in $\graph$, if $\graph \models \shexschema$, then
  $\copyswap(\graph, e) \models \shexschema$.
\end{lemma}
%
\begin{proof}
Let $e = (u, p, v)$ and $\graph' = \copyswap(\graph, e)$.
The proof easily follows by structural induction on shape expression
$\varphi$, where the induction base $\graph, u \models \se$ \iff
$\graph', u \models \se$ \iff $\graph', d(u) \models \se$  is
immediate due to the definition of the $\copyswap$ function.
% \begin{itemize}[\textbullet]
% \item
%   consider a derivations for an atomic statement $\graph, v \models \se$;
% \item
%   show that for every node $u$ in $\graph$ and every shape expression $\se$, if
%   $\graph, u \models \se$ then there exist derivations for $\graph',
%   d_1(u) \models \se$ and for $\graph', d_2(u) \models \se$. Such derivations
%   are basically copies of the derivation for $\graph, u \models \se$ in which,
%   if $e$ is part of the derivation, we need to switch from $d_1(x)$ to $d_2(x)$
%   and vice versa.
% \end{itemize}
\end{proof}
%\input{Appendix-Shex-garbage-collector}




%%% Local Variables:
%%% mode: latex
%%% TeX-master: "Article"  
%%% End:                                                                                                                         
