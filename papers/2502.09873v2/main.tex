% ICCV 2025 Paper Template; see https://github.com/iccv-org/author-kit

\documentclass[10pt,twocolumn,letterpaper]{article}

%%%%%%%%% PAPER TYPE  - PLEASE UPDATE FOR FINAL VERSION
% \usepackage{iccv}              % To produce the CAMERA-READY version
% \usepackage[review]{iccv}      % To produce the REVIEW version
\usepackage[pagenumbers]{iccv} % To force page numbers, e.g. for an arXiv version
\usepackage{array}      % 用于 \newcolumntype 和 >{\centering\arraybackslash}
\usepackage{tabularx}   % 用于 tabularx 环境
\usepackage{xcolor}     % 用于 \rowcolor
\usepackage{colortbl}   % 用于表格颜色
\usepackage{booktabs}   % 用于 \toprule
\usepackage{multirow}
\usepackage{subcaption}


\definecolor{gray}{rgb}{0.95,0.95,0.95}
% Import additional packages in the preamble file, before hyperref
%
% --- inline annotations
%
\newcommand{\red}[1]{{\color{red}#1}}
\newcommand{\todo}[1]{{\color{red}#1}}
\newcommand{\TODO}[1]{\textbf{\color{red}[TODO: #1]}}
% --- disable by uncommenting  
% \renewcommand{\TODO}[1]{}
% \renewcommand{\todo}[1]{#1}



\newcommand{\VLM}{LVLM\xspace} 
\newcommand{\ours}{PeKit\xspace}
\newcommand{\yollava}{Yo’LLaVA\xspace}

\newcommand{\thisismy}{This-Is-My-Img\xspace}
\newcommand{\myparagraph}[1]{\noindent\textbf{#1}}
\newcommand{\vdoro}[1]{{\color[rgb]{0.4, 0.18, 0.78} {[V] #1}}}
% --- disable by uncommenting  
% \renewcommand{\TODO}[1]{}
% \renewcommand{\todo}[1]{#1}
\usepackage{slashbox}
% Vectors
\newcommand{\bB}{\mathcal{B}}
\newcommand{\bw}{\mathbf{w}}
\newcommand{\bs}{\mathbf{s}}
\newcommand{\bo}{\mathbf{o}}
\newcommand{\bn}{\mathbf{n}}
\newcommand{\bc}{\mathbf{c}}
\newcommand{\bp}{\mathbf{p}}
\newcommand{\bS}{\mathbf{S}}
\newcommand{\bk}{\mathbf{k}}
\newcommand{\bmu}{\boldsymbol{\mu}}
\newcommand{\bx}{\mathbf{x}}
\newcommand{\bg}{\mathbf{g}}
\newcommand{\be}{\mathbf{e}}
\newcommand{\bX}{\mathbf{X}}
\newcommand{\by}{\mathbf{y}}
\newcommand{\bv}{\mathbf{v}}
\newcommand{\bz}{\mathbf{z}}
\newcommand{\bq}{\mathbf{q}}
\newcommand{\bff}{\mathbf{f}}
\newcommand{\bu}{\mathbf{u}}
\newcommand{\bh}{\mathbf{h}}
\newcommand{\bb}{\mathbf{b}}

\newcommand{\rone}{\textcolor{green}{R1}}
\newcommand{\rtwo}{\textcolor{orange}{R2}}
\newcommand{\rthree}{\textcolor{red}{R3}}
\usepackage{amsmath}
%\usepackage{arydshln}
\DeclareMathOperator{\similarity}{sim}
\DeclareMathOperator{\AvgPool}{AvgPool}

\newcommand{\argmax}{\mathop{\mathrm{argmax}}}     



% It is strongly recommended to use hyperref, especially for the review version.
% hyperref with option pagebackref eases the reviewers' job.
% Please disable hyperref *only* if you encounter grave issues, 
% e.g. with the file validation for the camera-ready version.
%
% If you comment hyperref and then uncomment it, you should delete *.aux before re-running LaTeX.
% (Or just hit 'q' on the first LaTeX run, let it finish, and you should be clear).
\definecolor{iccvblue}{rgb}{0.21,0.49,0.74}
\usepackage[pagebackref,breaklinks,colorlinks,allcolors=iccvblue]{hyperref}

\newcommand{\mymodel}{CODiff}

\providecommand{\yulun}[1]{\textcolor{red}{[{\bf #1}]}}

%%%%%%%%% PAPER ID  - PLEASE UPDATE
\def\paperID{824} % *** Enter the Paper ID here
\def\confName{ICCV}
\def\confYear{2025}

%%%%%%%%% TITLE - PLEASE UPDATE
\title{Compression-Aware One-Step Diffusion Model for JPEG Artifact Removal}

%%%%%%%%% AUTHORS - PLEASE UPDATE
\author{Jinpei Guo$^{1,2}$,\; Zheng Chen$^{2}$,\; Wenbo Li$^{3}$,\; Yong Guo$^{4}$,\; Yulun Zhang$^{2,\ast}$\\
$^{1}$Carnegie Mellon University,\; $^{2}$Shanghai Jiao Tong University, \\$^{3}$Huawei Noah’s Ark Lab,\; $^{4}$South China University of Technology
}

\begin{document}
\maketitle

% \clearpage
\pagenumbering{gobble}
\maketitlesupplementary

\section{Additional Results on Embodied Tasks}

To evaluate the broader applicability of our EgoAgent's learned representation beyond video-conditioned 3D human motion prediction, we test its ability to improve visual policy learning for embodiments other than the human skeleton.
Following the methodology in~\cite{majumdar2023we}, we conduct experiments on the TriFinger benchmark~\cite{wuthrich2020trifinger}, which involves a three-finger robot performing two tasks: reach cube and move cube. 
We freeze the pretrained representations and use a 3-layer MLP as the policy network, training each task with 100 demonstrations.

\begin{table}[h]
\centering
\caption{Success rate (\%) on the TriFinger benchmark, where each model's pretrained representation is fixed, and additional linear layers are trained as the policy network.}
\label{tab:trifinger}
\resizebox{\linewidth}{!}{%
\begin{tabular}{llcc}
\toprule
Methods       & Training Dataset & Reach Cube & Move Cube \\
\midrule
DINO~\cite{caron2021emerging}         & WT Venice        & 78.03     & 47.42     \\
DoRA~\cite{venkataramanan2023imagenet}          & WT Venice        & 81.62     & 53.76     \\
DoRA~\cite{venkataramanan2023imagenet}          & WT All           & 82.40     & 48.13     \\
\midrule
EgoAgent-300M & WT+Ego-Exo4D      & 82.61    & 54.21      \\
EgoAgent-1B   & WT+Ego-Exo4D      & \textbf{85.72}      & \textbf{57.66}   \\
\bottomrule
\end{tabular}%
}
\end{table}

As shown in Table~\ref{tab:trifinger}, EgoAgent achieves the highest success rates on both tasks, outperforming the best models from DoRA~\cite{venkataramanan2023imagenet} with increases of +3.32\% and +3.9\% respectively.
This result shows that by incorporating human action prediction into the learning process, EgoAgent demonstrates the ability to learn more effective representations that benefit both image classification and embodied manipulation tasks.
This highlights the potential of leveraging human-centric motion data to bridge the gap between visual understanding and actionable policy learning.



\section{Additional Results on Egocentric Future State Prediction}

In this section, we provide additional qualitative results on the egocentric future state prediction task. Additionally, we describe our approach to finetune video diffusion model on the Ego-Exo4D dataset~\cite{grauman2024ego} and generate future video frames conditioned on initial frames as shown in Figure~\ref{fig:opensora_finetune}.

\begin{figure}[b]
    \centering
    \includegraphics[width=\linewidth]{figures/opensora_finetune.pdf}
    \caption{Comparison of OpenSora V1.1 first-frame-conditioned video generation results before and after finetuning on Ego-Exo4D. Fine-tuning enhances temporal consistency, but the predicted pixel-space future states still exhibit errors, such as inaccuracies in the basketball's trajectory.}
    \label{fig:opensora_finetune}
\end{figure}

\subsection{Visualizations and Comparisons}

More visualizations of our method, DoRA, and OpenSora in different scenes (as shown in Figure~\ref{fig:supp pred}). For OpenSora, when predicting the states of $t_k$, we use all the ground truth frames from $t_{0}$ to $t_{k-1}$ as conditions. As OpenSora takes only past observations as input and neglects human motion, it performs well only when the human has relatively small motions (see top cases in Figure~\ref{fig:supp pred}), but can not adjust to large movements of the human body or quick viewpoint changes (see bottom cases in Figure~\ref{fig:supp pred}).

\begin{figure*}
    \centering
    \includegraphics[width=\linewidth]{figures/supp_pred.pdf}
    \caption{Retrieval and generation results for egocentric future state prediction. Correct and wrong retrieval images are marked with green and red boundaries, respectively.}
    \label{fig:supp pred}
\end{figure*}

\begin{figure*}[t]
    \centering
    \includegraphics[width=0.9\linewidth]{figures/motion_prediction.pdf}
    \vspace{-0.5mm}
    \caption{Motion prediction results in scenes with minor changes in observation.}
    \vspace{-1.5mm}
    \label{fig:motion_prediction}
\end{figure*}

\subsection{Finetuning OpenSora on Ego-Exo4D}

OpenSora V1.1~\cite{opensora}, initially trained on internet videos and images, produces severely inconsistent results when directly applied to infer future videos on the Ego-Exo4D dataset, as illustrated in Figure~\ref{fig:opensora_finetune}.
To address the gap between general internet content and egocentric video data, we fine-tune the official checkpoint on the Ego-Exo4D training set for 50 epochs.
OpenSora V1.1 proposed a random mask strategy during training to enable video generation by image and video conditioning. We adopted the default masking rate, which applies: 75\% with no masking, 2.5\% with random masking of 1 frame to 1/4 of the total frames, 2.5\% with masking at either the beginning or the end for 1 frame to 1/4 of the total frames, and 5\% with random masking spanning 1 frame to 1/4 of the total frames at both the beginning and the end.

As shown in Fig.~\ref{fig:opensora_finetune}, despite being trained on a large dataset, OpenSora struggles to generalize to the Ego-Exo4D dataset, producing future video frames with minimal consistency relative to the conditioning frame. While fine-tuning improves temporal consistency, the moving trajectories of objects like the basketball and soccer ball still deviate from realistic physical laws. Compared with our feature space prediction results, this suggests that training world models in a reconstructive latent space is more challenging than training them in a feature space.


\section{Additional Results on 3D Human Motion Prediction}

We present additional qualitative results for the 3D human motion prediction task, highlighting a particularly challenging scenario where egocentric observations exhibit minimal variation. This scenario poses significant difficulties for video-conditioned motion prediction, as the model must effectively capture and interpret subtle changes. As demonstrated in Fig.~\ref{fig:motion_prediction}, EgoAgent successfully generates accurate predictions that closely align with the ground truth motion, showcasing its ability to handle fine-grained temporal dynamics and nuanced contextual cues.

\section{OpenSora for Image Classification}

In this section, we detail the process of extracting features from OpenSora V1.1~\cite{opensora} (without fine-tuning) for an image classification task. Following the approach of~\cite{xiang2023denoising}, we leverage the insight that diffusion models can be interpreted as multi-level denoising autoencoders. These models inherently learn linearly separable representations within their intermediate layers, without relying on auxiliary encoders. The quality of the extracted features depends on both the layer depth and the noise level applied during extraction.


\begin{table}[h]
\centering
\caption{$k$-NN evaluation results of OpenSora V1.1 features from different layer depths and noising scales on ImageNet-100. Top1 and Top5 accuracy (\%) are reported.}
\label{tab:opensora-knn}
\resizebox{0.95\linewidth}{!}{%
\begin{tabular}{lcccccc}
\toprule
\multirow{2}{*}{Timesteps} & \multicolumn{2}{c}{First Layer} & \multicolumn{2}{c}{Middle Layer} & \multicolumn{2}{c}{Last Layer} \\
\cmidrule(r){2-3}   \cmidrule(r){4-5}  \cmidrule(r){6-7}  & Top1           & Top5           & Top1            & Top5           & Top1           & Top5          \\
\midrule
32        &  6.10           & 18.20             & 34.04               & 59.50             & 30.40             & 55.74             \\
64        & 6.12              & 18.48              & 36.04               & 61.84              & 31.80         & 57.06         \\
128       & 5.84             & 18.14             & 38.08               & 64.16              & 33.44       & 58.42 \\
256       & 5.60             & 16.58              & 30.34               & 56.38              &28.14          & 52.32        \\
512       & 3.66              & 11.70            & 6.24              & 17.62              & 7.24              & 19.44  \\ 
\bottomrule
\end{tabular}%
}
\end{table}

As shown in Table~\ref{tab:opensora-knn}, we first evaluate $k$-NN classification performance on the ImageNet-100 dataset using three intermediate layers and five different noise scales. We find that a noise timestep of 128 yields the best results, with the middle and last layers performing significantly better than the first layer.
We then test this optimal configuration on ImageNet-1K and find that the last layer with 128 noising timesteps achieves the best classification accuracy.

\section{Data Preprocess}
For egocentric video sequences, we utilize videos from the Ego-Exo4D~\cite{grauman2024ego} and WT~\cite{venkataramanan2023imagenet} datasets.
The original resolution of Ego-Exo4D videos is 1408×1408, captured at 30 fps. We sample one frame every five frames and use the original resolution to crop local views (224×224) for computing the self-supervised representation loss. For computing the prediction and action loss, the videos are downsampled to 224×224 resolution.
WT primarily consists of 4K videos (3840×2160) recorded at 60 or 30 fps. Similar to Ego-Exo4D, we use the original resolution and downsample the frame rate to 6 fps for representation loss computation.
As Ego-Exo4D employs fisheye cameras, we undistort the images to a pinhole camera model using the official Project Aria Tools to align them with the WT videos.

For motion sequences, the Ego-Exo4D dataset provides synchronized 3D motion annotations and camera extrinsic parameters for various tasks and scenes. While some annotations are manually labeled, others are automatically generated using 3D motion estimation algorithms from multiple exocentric views. To maximize data utility and maintain high-quality annotations, manual labels are prioritized wherever available, and automated annotations are used only when manual labels are absent.
Each pose is converted into the egocentric camera's coordinate system using transformation matrices derived from the camera extrinsics. These matrices also enable the computation of trajectory vectors for each frame in a sequence. Beyond the x, y, z coordinates, a visibility dimension is appended to account for keypoints invisible to all exocentric views. Finally, a sliding window approach segments sequences into fixed-size windows to serve as input for the model. Note that we do not downsample the frame rate of 3D motions.

\section{Training Details}
\subsection{Architecture Configurations}
In Table~\ref{tab:arch}, we provide detailed architecture configurations for EgoAgent following the scaling-up strategy of InternLM~\cite{team2023internlm}. To ensure the generalization, we do not modify the internal modules in InternML, \emph{i.e.}, we adopt the RMSNorm and 1D RoPE. We show that, without specific modules designed for vision tasks, EgoAgent can perform well on vision and action tasks.

\begin{table}[ht]
  \centering
  \caption{Architecture configurations of EgoAgent.}
  \resizebox{0.8\linewidth}{!}{%
    \begin{tabular}{lcc}
    \toprule
          & EgoAgent-300M & EgoAgent-1B \\
          \midrule
    Depth & 22    & 22 \\
    Embedding dim & 1024  & 2048 \\
    Number of heads & 8     & 16 \\
    MLP ratio &    8/3   & 8/3 \\
    $\#$param.  & 284M & 1.13B \\
    \bottomrule
    \end{tabular}%
    }
  \label{tab:arch}%
\end{table}%

Table~\ref{tab:io_structure} presents the detailed configuration of the embedding and prediction modules in EgoAgent, including the image projector ($\text{Proj}_i$), representation head/state prediction head ($\text{MLP}_i$), action projector ($\text{Proj}_a$) and action prediction head ($\text{MLP}_a$).
Note that the representation head and the state prediction head share the same architecture but have distinct weights.

\begin{table}[t]
\centering
\caption{Architecture of the embedding ($\text{Proj}_i$, $\text{Proj}_a$) and prediction ($\text{MLP}_i$, $\text{MLP}_a$) modules in EgoAgent. For details on module connections and functions, please refer to Fig.~2 in the main paper.}
\label{tab:io_structure}
\resizebox{\linewidth}{!}{%
\begin{tabular}{lcl}
\toprule
       & \multicolumn{1}{c}{Norm \& Activation} & \multicolumn{1}{c}{Output Shape}  \\
\midrule
\multicolumn{3}{l}{$\text{Proj}_i$ (\textit{Image projector})} \\
\midrule
Input image  & -          & 3$\times$224$\times$224 \\
Conv 2D (16$\times$16) & -       & Embedding dim$\times$14$\times$14    \\
\midrule
\multicolumn{3}{l}{$\text{MLP}_i$ (\textit{State prediction head} \& \textit{Representation head)}} \\
\midrule
Input embedding  & -          & Embedding dim \\
Linear & GELU       & 2048          \\
Linear & GELU       & 2048          \\
Linear & -          & 256           \\
Linear & -          & 65536     \\
\midrule
\multicolumn{3}{l}{$\text{Proj}_a$ (\textit{Action projector})} \\
\midrule
Input pose sequence  & -          & 4$\times$5$\times$17 \\
Conv 2D (5$\times$17) & LN, GELU   & Embedding dim$\times$1$\times$1    \\
\midrule
\multicolumn{3}{l}{$\text{MLP}_a$ (\textit{Action prediction head})} \\
\midrule
Input embedding  & -          & Embedding dim$\times$1$\times$1 \\
Linear & -          & 4$\times$5$\times$17     \\
\bottomrule
\end{tabular}%
}
\end{table}


\subsection{Training Configurations}
In Table~\ref{tab:training hyper}, we provide the detailed training hyper-parameters for experiments in the main manuscripts.

\begin{table}[ht]
  \centering
  \caption{Hyper-parameters for training EgoAgent.}
  \resizebox{0.86\linewidth}{!}{%
    \begin{tabular}{lc}
    \toprule
    Training Configuration & EgoAgent-300M/1B \\
    \midrule
    Training recipe: &  \\
    optimizer & AdamW~\cite{loshchilov2017decoupled} \\
    optimizer momentum & $\beta_1=0.9, \beta_2=0.999$ \\
    \midrule
    Learning hyper-parameters: &  \\
    base learning rate & 6.0E-04 \\
    learning rate schedule & cosine \\
    base weight decay & 0.04 \\
    end weight decay & 0.4 \\
    batch size & 1920 \\
    training iters & 72,000 \\
    lr warmup iters & 1,800 \\
    warmup schedule & linear \\
    gradient clip & 1.0 \\
    data type & float16 \\
    norm epsilon & 1.0E-06 \\
    \midrule
    EMA hyper-parameters: &  \\
    momentum & 0.996 \\
    \bottomrule
    \end{tabular}%
    }
  \label{tab:training hyper}%
\end{table}%

\clearpage


\begin{abstract}
Diffusion models have demonstrated remarkable success in image restoration tasks. However, their multi-step denoising process introduces significant computational overhead, limiting their practical deployment. Furthermore, existing methods struggle to effectively remove severe JPEG artifact, especially in highly compressed images. To address these challenges, we propose \mymodel, a \textbf{c}ompression-aware \textbf{o}ne-step \textbf{diff}usion model for JPEG artifact removal. The core of \mymodel~is the compression-aware visual embedder (CaVE), which extracts and leverages JPEG compression priors to guide the diffusion model. Moreover, We propose a dual learning strategy for CaVE, which combines explicit and implicit learning. Specifically, explicit learning enforces a quality prediction objective to differentiate low-quality images with different compression levels. Implicit learning employs a reconstruction objective that enhances the model's generalization. This dual learning allows for a deeper understanding of JPEG compression. Experimental results demonstrate that \mymodel~surpasses recent leading methods in both quantitative and visual quality metrics. The code and models will be publicly available
\end{abstract}
\vspace{-4mm}


%% narrow the gap between equations and sentences
\setlength{\abovedisplayskip}{2pt}
\setlength{\belowdisplayskip}{2pt}

\section{Introduction}
\label{intro}

% JPEG~\cite{wallace1991jpeg} is one of the most widely used image compression algorithms due to its simplicity and fast encoding/decoding speeds. The JPEG algorithm applies the discrete cosine transform (DCT) to convert the image into the frequency domain. These frequency domain representation coefficients are then divided by a quantization table and rounded to the nearest integer. The quantization table elements control the compression ratio, and the rounding operation is the only lossy step in the entire process. The quantization table is typically represented by an integer known as the quality factor (QF), which ranges from 0 to 100. A lower QF results in a smaller storage size but greater information loss. 


JPEG~\cite{wallace1991jpeg} artifact removal task aims to remove the artifact caused by the compression algorithm and recover the lost information from the compressed images. Recent advancements have focused on leveraging CNN-based and Transformer-based methods~\cite{dong2015compression,zhang2019residual,liang2021swinir,jiang2021towards,li2024promptcir} to remove JPEG artifact for compressed images with different quality factors (QFs), achieving remarkable performance. However, these methods (see Fig.~\ref{fig:head}) face significant challenges at high compression levels, due to substantial compression artifact and severe visual information loss.% in compressed images.

Recent advances in diffusion models~\cite{song2020score,ho2020denoising,song2020denoising,dhariwal2021diffusion}, particularly large-scale pre-trained text-to-image (T2I) models~\cite{rombach2022high,podell2023sdxl,saharia2022photorealistic}, have strong image generation priors. These priors make diffusion models a promising solution for compression artifact removal, especially in cases of severe degradation. However, their multi-step denoising process introduces substantial computational overhead. To address this, one-step diffusion (OSD) models~\cite{wu2024one,li2024distillation} have emerged as efficient alternatives. By leveraging large-scale pre-trained multi-step T2I diffusion models~\cite{rombach2022high,podell2023sdxl}, these OSD models strike a balance between strong restoration capabilities and significantly faster inference, making them a compelling choice for JPEG artifact removal.


\begin{figure}[t]
\scriptsize
\centering
\begin{tabular}{cc}
% % one row
\hspace{-0.55cm}
\begin{tabular}{c}
\includegraphics[width=0.172\textwidth, height=0.185\textwidth]{figs/img_014/HQ.png}
\\
Urban100: img\_014 \\
MACs (T) / Time (s)
\end{tabular}
\hspace{-6mm}
\begin{tabular}{cccccc}
\includegraphics[width=0.105\textwidth]{figs/img_014/GT.png} \hspace{-5mm} &
\includegraphics[width=0.105\textwidth]{figs/img_014/JDEC.png} \hspace{-5mm} &
\includegraphics[width=0.105\textwidth]{figs/img_014/PromptCIR.png} \hspace{-5mm} 
\\
HQ \hspace{-5mm}
 &
JDEC~\cite{han2024jdec} \hspace{-5mm}  &
PromptCIR~\cite{li2024promptcir} \hspace{-5mm} &
\\
 \hspace{-5mm} &
3.40 / 0.86 \hspace{-5mm}  &
30.27 / 27.97 \hspace{-5mm} &
\\
\includegraphics[width=0.105\textwidth]{figs/img_014/DiffBIR.png} \hspace{-5mm} &
\includegraphics[width=0.105\textwidth]{figs/img_014/OSEDiff.png} \hspace{-5mm} &
\includegraphics[width=0.105\textwidth]{figs/img_014/ours.png} \hspace{-5mm}  
\\ 
DiffBIR*~\cite{lin2024diffbir} \hspace{-5mm}  &
OSEDiff*~\cite{wu2024one} \hspace{-5mm} &
\mymodel~(ours) \hspace{-5mm}
\\
188.24 / 50.81 \hspace{-5mm}  &
10.39 / 0.65 \hspace{-5mm} &
9.46 / 0.57 \hspace{-5mm}
\\
\end{tabular}
\vspace{-1mm}
%\\


\end{tabular}
\vspace{-2.5mm}
\caption{Visual comparison of JPEG artifact removal (QF=5). We provide multiply accumulate operations (MACs) and time during inference. DiffBIR*~\cite{lin2024diffbir}, OSEDiff*~\cite{wu2024one} are retrained on the same dataset as our method. All models are tested with an input image size of 1,024$\times$1,024. Our \mymodel~ reconstructs a more realistic and faithful visual result than competing methods.}
\label{fig:head}
\vspace{-7mm}
\end{figure}
As illustrated in Fig.~\ref{fig:head}, for highly compressed images (\eg, QF=5), diffusion-based models generally outperform traditional CNN-based and Transformer-based methods in reducing JPEG artifact such as blocky patterns and color banding. However, the restored images remain inharmonious, with certain artifact still visible. The primary issue lies in the insufficient integration of JPEG compression priors. Without this information, models struggle to differentiate between compression-induced distortions and natural image features, leading to suboptimal restoration quality. Notably, most existing diffusion-based methods~\cite{lin2024diffbir,yu2024scaling,wu2024seesr,wu2024one} for image restoration tasks neglect such compression-related information.

Thus, it is worthwhile to investigate how to extract JPEG compression priors and integrate them into the diffusion model. A major challenge is how to design effective JPEG compression representations. Previous works~\cite{ehrlich2020quantization,han2024jdec} use the quantization matrix~\cite{wallace1991jpeg} to represent the compression priors. However, since it only depends on the quality factor (QF) with static numerical values, the information it provides is insufficient. To dynamically utilize the JPEG priors, QF learning approaches~\cite{jiang2021towards,wang2021jpeg} attempt to learn compression representations. They train models to predict QF from low-quality (LQ) images and use intermediate visual embeddings as priors. However, as QF is merely a single integer, taking it as the sole learning objective constrains the model's ability to capture the comprehensive JPEG priors. Furthermore, the models may encounter generalization challenges when exposed to unseen compression levels. 

\begin{figure}[t]
    \begin{center}
    \begin{tabular}{cc}
        \hspace{-3mm}
        \includegraphics[width=0.49\columnwidth]{figs/clusters_3d/urban100_qf.pdf}  &
        \hspace{-3mm}
        \includegraphics[width=0.49\columnwidth]{figs/clusters_3d/urban100_qf+recon.pdf}
        % \label{fig:motivation-qf-leanring}
        \vspace{-2mm}
        \\
        \hspace{-3mm}
       \scriptsize{(a) Explicit learning }& \hspace{-3mm} \scriptsize{(b) Dual learning} 
       \vspace{1mm}
       \\
       \multicolumn{2}{c}{\includegraphics[width=0.9\columnwidth]{figs/motivation.pdf}} 
       \vspace{-2mm}
       \\
       
       \multicolumn{2}{c}{\scriptsize{(c) Overview of our dual learning strategy.}}\\
       
       % \includegraphics[width=0.49\columnwidth]{figs/motivation_qf.pdf} &
       % \includegraphics[width=0.49\columnwidth]{figs/motivation_qf+recon.pdf} 
       % % \label{fig:motivation-dual-leanring}
       %  \\
       
    \end{tabular}
    \end{center}
    \vspace{-6mm}
    \caption{An illustration of the explicit learning approach with our dual learning strategy. We employ the t-SNE~\cite{van2008visualizing} method to visualize the clustering of output compression quality representations on the Urban100~\cite{huang2015single} dataset, where QF=1 and QF=5 are unseen compression levels. Notably, the explicit learning method struggles to distinguish between these unseen QFs, whereas our dual learning strategy effectively differentiates them.}
    \label{fig:motivation}
    \vspace{-6mm}
\end{figure}

To alleviate these challenges, we propose \mymodel, a \textbf{c}ompression-aware \textbf{o}ne-step \textbf{diff}usion model for JPEG artifact removal. \mymodel~leverages the powerful pre-trained knowledge of multi-step diffusion (MSD) models while significantly improving inference speed. A key component of \mymodel~is the compression-aware visual embedder (CaVE), which extracts JPEG compression priors to guide the denoising process. To provide more informative JPEG compression priors and improve generalization, CaVE learns the JPEG compression process through a dual-learning strategy (see Fig.~\ref{fig:motivation}). In the explicit learning phase, it is trained to predict the quality factor (QF) from a low-quality image. In the implicit learning phase, CaVE learns to restore high-quality (HQ) images from compressed inputs. This joint learning approach enhances CaVE's ability to model the JPEG compression process, thereby enhancing its generalization ability to differentiate unseen compression levels. Comparison in Fig.~\ref{fig:head} highlights the superior restoration quality and efficiency of our~\mymodel. Our main contributions are summarized as follows:
% \vspace{-1mm}
\begin{itemize}
    \item We propose \mymodel, a novel and effective OSD model for JPEG artifact removal. To the best of our knowledge, this is the first attempt to design an OSD model specifically for this task with compression priors.
    % \vspace{-1mm}
    \item We design compression-aware visual embedder (CaVE), which generates embeddings that capture rich JPEG compression information. CaVE provides effective compression priors for the OSD model to restore LQ images.
    \item We propose dual learning strategy for CaVE. Specifically, explicit learning allows CaVE to distinguish LQ images across various compression levels. While implicit learning enhances CaVE's generalization capabilities.
    % \vspace{-1mm}
    \item Our \mymodel~achieves SOTA performance in JPEG artifact removal, excelling in both quantitative metrics and visual quality. \mymodel~ reduces computational costs compared with both MSD and OSD models.
\end{itemize}


\vspace{-2mm}
\section{Related Work}
\vspace{-1mm}
\subsection{JPEG Artifact Removal}
\vspace{-1mm}
In recent years, learning-based methods have significantly advanced JPEG artifact removal. The pioneering work ARCNN~\cite{dong2015compression} first introduces deep learning into this task, leveraging a super-resolution network~\cite{dong2014learning} to reduce compression artifact. Transformer-based methods ~\cite{zhang2019residual,liang2021swinir} incorporate attention mechanisms to enhance feature representation, achieving strong performance across image artifact removal tasks. Inspired by the success of GANs in image restoration, several GAN-based methods~\cite{galteri2017deep,galteri2019deep} have been proposed to enhance the perceptual quality of compressed images. Meanwhile, dual-domain convolutional network approaches~\cite{guo2016building,kim2020agarnet,zhang2018dmcnn,zheng2019implicit,han2024jdec} have been developed to exploit redundancies in both the pixel and frequency domains. 

To further integrate the JPEG compression priors, the ranker-guided framework~\cite{wang2021jpeg} utilizes compression quality ranking as auxiliary information.
Quantization tables~\cite{li2020learning,ehrlich2020quantization} are also utilized as prior knowledge, enabling a single network to handle artifact across different quality factors (QFs). To enable blind JPEG artifact removal, FBCNN~\cite{jiang2021towards} predicts an adjustable QF to balance artifact removal and detail preservation. Additionally, PromptCIR~\cite{li2024promptcir} explores prompt learning for blind compressed image restoration. However, most existing methods struggle to recover highly compressed images due to severe information loss. With the advancement of diffusion models, incorporating JPEG compression priors into the pre-trained large scale T2I diffusion models offers a promising solution to mitigate this information loss. 

\begin{figure*}[t]
    \begin{center}
        \includegraphics[width=\textwidth]{figs/pipeline.pdf}
    \end{center}
    \vspace{-20pt}
    \caption{Overview of our proposed \mymodel. In the first stage, we train our compression-aware visual embedder (CaVE) via a dual learning strategy. In implicit learning, compression prior embeddings are fed into a UNet decoder to reconstruct high-quality (HQ) images. In explicit learning, they are input into a lightweight quality factor (QF) predictor. In the second stage, the priors from CaVE are then used by the generator $\mathcal{G}_\theta$ to restore the HQ images: $\hat{\mathbf{I}}_H=\mathcal{G}_\theta(\mathbf{I}_L;\mathbf{c}_L)$. The generator $\mathcal{G}_\theta$ integrates a pre-trained VAE and UNet from StableDiffusion~\cite{rombach2022high}, with the VAE encoder and UNet fine-tuned via LoRA~\cite{hu2021lora}.} 
    \label{fig:pipeline}
    \vspace{-15pt}
\end{figure*}



\subsection{Diffusion Models}
\vspace{-2mm}
Diffusion models, known for their powerful generative capabilities, progressively transform random noise into structured data through iterative denoising. Recently, they have shown strong performance in image-to-image tasks, particularly in image restoration~\cite{wang2024exploiting,lin2024diffbir,wu2024seesr,yang2024pixel,yue2024resshift,jiang2024autodir,yu2024scaling}. By leveraging their ability to capture fine-grained details and produce high-quality outputs, diffusion models have outperformed traditional image restoration methods. However, their complex architectures~\cite{lin2024diffbir,yu2024scaling,yang2024pixel} and reliance on numerous iterative steps hinder their real-world deployment due to high computational costs.

Accelerating diffusion models by reducing inference steps is crucial for practical applications. However, excessive reduction often degrades performance. Therefore, most one-step diffusion (OSD) methods use distillation to learn from a teacher model, preserving image fidelity~\cite{yin2024dmd,yin2024dmd2,wang2024sinsr}. Notably, OSEDiff~\cite{wu2024one} has obtained promising results using variational score distillation. Despite these advances, existing diffusion-based restoration models often overlook degradation-related priors. Consequently, it is difficult to distinguish compression artifact from natural image features. Thus, exploring how to extract prior knowledge specific to JPEG compression is essential to guide diffusion models for JPEG artifact removal tasks.

\begin{figure*}[t]
    \begin{center}
    \resizebox{1\textwidth}{!}{
        \begin{tabular}{cccc}
            \hspace{-2mm}
            \includegraphics[width=0.25\textwidth]{figs/clusters/LIVE1_color_qf_cluster_box.pdf} &
            \includegraphics[width=0.25\textwidth]{figs/clusters/LIVE1_color_cluster_box.pdf} &
            \includegraphics[width=0.25\textwidth]{figs/clusters/DIV2K_valid_qf_cluster_box.pdf}  &
            \includegraphics[width=0.25\textwidth]{figs/clusters/DIV2K_valid_cluster_box.pdf}\\
            
           (a) LIVE-1; Explicit learning & (b) LIVE-1; Dual learning & (c) DIV2K-val; Explicit learning & (d) DIV2K-val; Dual learning
        \end{tabular}
    }
    \end{center}
    \vspace{-6.5mm}
    \caption{Visualization of JPEG prior embeddings from CaVE under different training objectives. In (a) and (c), CaVE is trained using only explicit learning, whereas in (b) and (d), it incorporates both explicit and implicit learning. The clusters enclosed in the \textcolor{red}{red box} correspond to unseen compression levels (QF=1,5). Notably, clusters from CaVE with dual learning separate from each other more clearly.}
    \vspace{-6mm}
    \label{fig:clusters}
\end{figure*}


% \vspace{-4mm}
\section{Methodology}
\label{sec:method}
\vspace{-2mm}
\subsection{Preliminaries: One-Step Diffusion Model}
\label{sec:diffusion_models}
\vspace{-2mm}
Latent diffusion models \cite{rombach2022high} operate through a two-stage process: forward diffusion and reverse denoising. During the forward phase, Gaussian noise with variance $\beta_t \in (0, 1)$ is progressively added to the latent representation $\mathbf{z}$: $\mathbf{z}_t = \sqrt{\bar{\alpha}_t} \mathbf{z} + \sqrt{1 - \bar{\alpha}_t} \mathbf{\varepsilon}$, 
where $\mathbf{\varepsilon} \sim \mathcal{N}(0, \mathbf{I})$. Here, the term $\bar{\alpha}_t$ is defined as $\bar{\alpha}_t = \prod_{i=1}^t(1 - \beta_i)$. In the reverse process, the clean latent vector $\hat{z}_0$ is estimated from predicted noise $\hat{\varepsilon}$:
$
\hat{\mathbf{z}}_0 = \frac{\mathbf{z}_t - \sqrt{1 - \bar{\alpha}_t}\hat{\mathbf{\varepsilon}}}{\sqrt{\bar{\alpha}_t}}
$
, where $\mathbf{\hat{\varepsilon}}$ is the prediction of the model $\epsilon_\theta$ given $\mathbf{z_t}$, $t$, and the prior $\mathbf{c}$: $\hat{\varepsilon}=\varepsilon_{\theta}(\mathbf{z}_t; \mathbf{c}, t)$.

Unlike conventional text-to-image (T2I) diffusion models \cite{rombach2022high,podell2023sdxl,saharia2022photorealistic}, \mymodel~feeds the latent representation of the low-quality (LQ) image to the UNet instead of the Gaussian noise. Specifically, our \mymodel~first encodes the LQ image $\mathbf{I}_L$ into the latent space via an encoder $E_\theta$, resulting in $\mathbf{z}_L = E_\theta(\mathbf{I}_L)$. A single-step denoising operation is then performed to predict the noise $\mathbf{\hat{\varepsilon}}$, which allows us to derive the estimated high-quality (HQ) latent representation $\hat{\mathbf{z}}_H$:
\begin{equation}
\hat{\mathbf{z}}_H = \frac{\mathbf{z}_L - \sqrt{1 - \bar{\alpha}_{T_L}} \varepsilon_{\theta} (\mathbf{z}_L; \mathbf{c}_L, T_L)}{\sqrt{\bar{\alpha}_{T_L}}},
\label{eq:generate_zh}
\end{equation}
where $\varepsilon_\theta$ is the denoising network with learnable parameters $\theta$, $\mathbf{c}_L$ is the compression prior, and $T_L\in [0, T]$ is the predefined diffusion timestep. 

Finally, the decoder $D_\theta$ reconstructs HQ image $\hat{\mathbf{I}}_H$ from the latent representation $\hat{\mathbf{z}}_H$: $\hat{\mathbf{I}}_H = D_\theta(\hat{\mathbf{z}}_H).$ Defining the generator $\mathcal{G}_\theta$ and the compression prior $\mathbf{c}_L$, the overall reconstruction from $\mathbf{I}_L$ to $\hat{\mathbf{I}}_H$ can be compactly written as:
\begin{equation}
\hat{\mathbf{I}}_H = \mathcal{G}_\theta(\mathbf{I}_L; \mathbf{c}_L).
\end{equation}
\subsection{Stage 1: Compression-aware Visual Embedder}
\label{sec:cave}
Previous studies have shown that incorporating compression information into the reconstruction process can enhance performance. To incorporate JPEG compression priors into diffusion models, we first train a compression-aware visual embedder (CaVE) to capture these priors effectively. In the second stage, the pre-trained CaVE is leveraged to guide the diffusion model, enabling it to utilize compression-related features from low-quality (LQ) images to improve reconstruction.

\subsubsection{Overall Architecture}
Our CaVE consists of a UNet~\cite{ronneberger2015u} architecture equipped with a lightweight quality factor (QF) predictor designed specifically for extracting compression priors. The UNet decoder and the QF predictor are utilized exclusively for the CaVE training. As illustrated in Fig.~\ref{fig:pipeline}, in explicit learning, the QF predictor guides CaVE to explicitly learn the JPEG compression priors. Simultaneously, implicit learning helps CaVE further capture the relationship between compression and reconstruction, enhancing its ability to model the JPEG compression process.

As illustrated in Fig.~\ref{fig:pipeline}, given a low-quality image $ \mathbf{I}_L \in \mathbb{R}^{H \times W \times 3} $, the CaVE encodes it into a set of feature vectors:
\begin{equation}
    \mathbf{c}_L=\{ \mathbf{c}_{L_k}\in\mathbb{R}^{d} \}_{k=1}^K.
\end{equation}
The QF predictor will then utilize the average feature representation, $\bar{\mathbf{c}}_L=\frac{1}{K}\sum_{k=1}^K \mathbf{c}_{L_k}$, to estimate QF.
Meanwhile, the UNet decoder leverages the extracted feature vectors $\mathbf{c}_L$ to reconstruct the corresponding high-quality image $\hat{\mathbf{I}}_H$.

\subsubsection{Explicit Learning}
\label{subsubsec:explicit}
To explicitly incorporate JPEG compression priors, we train CaVE to produce embeddings that directly facilitate QF prediction~\cite{jiang2021towards,wang2022jpeg}. Specifically, given a low-quality image, we employ the UNet encoder to encode the image into latent representations $\mathbf{c}_L=E(\mathbf{I}_L; \phi)$. The UNet encoder contains four scales, each of which involves residual blocks. Each residual block consists of two 3$\times$3 convolutional layers with Batch Normalization layer and ReLU activation function. 4$\times$4 stride convolutions are employed for downsampling operations. The output dimensions of each scale are set to 64, 128, 256, and 512, respectively. 

We then use the embeddings $\mathbf{z}_L$ from the UNet encoder to explicitly predict the JPEG compression level QF as $\mathit{QF}_{pred}=P(\bar{\mathbf{c}}_L; \phi)$, where $\bar{\mathbf{z}}_L$ is the average of latent embeddings $\mathbf{z}_L$. The QF predictor $P(\cdot;\phi)$ is a lightweight convolutional neural network (CNN) cascaded to a 3-layer multilayer perceptron (MLP). We set the channels in CNN and MLP hidden layers as 512 for a more accurate prediction. 

During training, small patches may contain limited compression-related information (\eg, background regions in compressed images could correspond to multiple QFs). This ambiguity can introduce instability in the training process and hinder CaVE’s ability to effectively capture the underlying JPEG compression patterns~\cite{jiang2021towards}. Therefore, we use the $\mathcal{L}_1$ loss function to mitigate the negative impact of such outliers. Let $B$ be the batch size during training, the QF prediction loss can be expressed as:
\begin{equation}
    \mathcal{L}_{\mathit{QF}}=\frac{1}{B}\sum_{i=1}^B\| \mathit{QF}_{pred}^i-\mathit{QF}_{gt}^i\|_1.
\end{equation}
To illustrate that our CaVE learns discriminative representations for images with different QFs, we visualize $\mathbf{z_L}$ from the UNet encoder using the t-SNE~\cite{van2008visualizing} method, as shown in Fig.~\ref{fig:clusters}. Our results show that CaVE, when trained with the QF loss, effectively distinguishes compression qualities present in the training set. However, it struggles to differentiate unseen compression levels (\eg, QF = 1, 5), indicating that relying solely on the QF loss fails to capture the full spectrum of compression-related information.

\subsubsection{Implicit Learning}
To more comprehensively capture JPEG compression priors, we train CaVE to produce embeddings that not only predict the quality factor (QF) (explicit learning) but also reconstruct high-quality images (implicit learning). By incorporating a reconstruction objective, CaVE implicitly learns the intricate details of the JPEG compression process.

Specifically, given a low-quality image $\mathbf{I}_L$, we employ the UNet-based architecture~\cite{ronneberger2015u} to restore the corresponding high-quality image: $\hat{\mathbf{I}}_{H}= D(\mathbf{z}_L; \phi)$. We use the same encoder in Sec.~\ref{subsubsec:explicit}. The decoder, on the other hand, includes three scales, each of which contains residual blocks that consist of two 3$\times$3 convolutional layers with Batch Normalization layer and ReLU activation function. Each scale receives the corresponding intermediate image features from the encoder to generate the restored image. 

Given a batch of $B$ training samples, the objective of implicit learning is to minimize the $\mathcal{L}_1$ loss between the reconstructed image $\hat{\mathbf{I}}_{H}$ and the ground-truth $\mathbf{I}_{H}$:
\begin{equation}
    \mathcal{L}_{\mathit{rec}}=\frac{1}{B}\sum_{i=1}^B\| \hat{\mathbf{I}}_{H}^i-\mathbf{I}_{H}^i\|_1.
\end{equation}
The overall CaVE training objective is formulated as:
\begin{equation}
    \mathcal{L}_{\mathit{CaVE}}=\mathcal{L}_{\mathit{QF}}+\lambda \cdot \mathcal{L}_{\mathit{rec}}.
\end{equation}

To demonstrate how implicit learning enhances CaVE’s understanding of JPEG compression, we visualize its learned embeddings in Fig.~\ref{fig:clusters}. Our results reveal that the dual learning allows CaVE to capture more nuanced variations in compression artifact, enabling it to differentiate even previously unseen compression levels. 

\subsection{Stage 2: JPEG Artifact Removal}
\label{sec:diff}
After extracting compression prior embeddings, we integrate them into our OSD model and fine-tune the pre-trained diffusion model using LoRA~\cite{hu2021lora}. The overall training objective combines perceptual and GAN losses. Specifically, perceptual loss directly aligns with the reconstruction process to ensure high-fidelity restoration. The GAN loss~\cite{goodfellow2020generative} enhances the realism of the generated images by refining textures and preserving natural image characteristics. By jointly optimizing these loss functions, our model effectively achieve high-quality, photorealistic reconstructions.

\noindent\textbf{Perceptual Loss.} While $\mathcal{L}_2$ loss is widely used for image reconstruction due to its simplicity and pixel-wise precision, it often struggles to capture perceptual quality. Minimizing $\mathcal{L}_2$ loss tends to produce overly smooth outputs, as it treats all pixel differences equally, ignoring structural and textural information critical for human perception.

To address this issue, we additionally incorporate DISTS (Deep Image Structure and Texture Similarity) loss~\cite{ding2020image}, which goes beyond pixel-wise comparisons by measuring perceptual similarity between images. Unlike $\mathcal{L}_2$ loss, DISTS is designed to align with human visual perception, capturing both structural details and textural characteristics. The perceptual loss can be written as:
\begin{equation}
    \mathcal{L}_{per}=\mathcal{L}_2(\hat{\mathbf{I}}_H, \mathbf{I}_H) + \lambda_{D} \mathcal{L}_{DISTS}(\hat{\mathbf{I}}_H, \mathbf{I}_H).
\end{equation}

\noindent\textbf{GAN Loss.} Generating stable images remains a significant challenge for OSD models, primarily due to their constrained computational resources. Prior research~\cite{yin2024dmd,yin2024dmd2,wang2024sinsr,wu2024one} has largely relied on distillation techniques to transfer knowledge from multi-step diffusion (MSD) models. However, these approaches are inherently limited by the performance of their teacher models, restricting their potential for further improvement. 

To overcome these limitations, we adopt an alternative strategy by integrating a discriminative network to enhance the realism of restoration. The GAN losses~\cite{goodfellow2020generative} used to train the generator $G_\theta$ and discriminator $D_\theta$ are defined as:
\begin{equation}
    \mathcal{L}_\mathcal{G} = -\mathbb{E}_t \left[\log \mathcal{D}_{\theta} \left( \hat{\mathbf{z}}_H\right)\right],
\end{equation}
\begin{equation}
\begin{aligned}
\mathcal{L}_\mathcal{D} = &-\mathbb{E}_t \left[\log \left( 1 - \mathcal{D}_{\theta} \left( \hat{\mathbf{z}}_H \right) \right)\right] \\
&- \mathbb{E}_t \left[\log \mathcal{D}_{\theta} \left(\mathbf{z}_H \right)\right], 
\end{aligned}
\end{equation}
where $\hat{\mathbf{z}}_H$ and $\mathbf{z}_H$ are the latent embeddings of the reconstructed image and the ground-truth image, respectively.

In our JPEG artifact removal pipeline, we first train CaVE in stage 1, then we incorporate the perceptual and GAN losses to fine-tune our OSD with JPEG compression priors from CaVE. The overall training objective for the generator is formulated as:
\begin{equation}
    \mathcal{L}=\mathcal{L}_{per} + \lambda_{G} \mathcal{L}_\mathcal{G}.
\end{equation}
\section{Experiments}
\subsection{Experimental Settings}
\noindent\textbf{Training and Testing Datasets.} Following the previous work~\cite{wu2024one,li2024promptcir}, We use DF2K which comprises 800 images from DIV2K~\cite{agustsson2017ntire} and 2,650 images from Flickr2K~\cite{timofte2017ntire}, and LSDIR~\cite{li2023lsdir} as our training datasets, totaling 88,441 different high-quality images. During training, we randomly crop 256$\times$256 patches and dynamically synthesize low-quality and high-quality image pairs by sampling the quality factor from a range of 8 to 95. We evaluate \mymodel~and compare its performance against competing methods using LIVE-1~\cite{sheikh2005live}, Urban100~\cite{huang2015single}, and DIV2K-Val~\cite{agustsson2017ntire} datasets. 



\noindent\textbf{Evaluation Metrics.}
To ensure a comprehensive and holistic evaluation of different methods, we employ a diverse set of both full-reference and no-reference image quality metrics. For perceptual quality assessment with a reference image, we utilize LPIPS~\cite{zhang2018unreasonable} and DISTS~\cite{ding2020image}, both of which measure structural and perceptual similarity by leveraging deep feature representations. In addition, to assess image quality without requiring a reference, we adopt MANIQA~\cite{yang2022maniqa}, MUSIQ~\cite{ke2021musiq}, and CLIPIQA~\cite{wang2023exploring}, which are designed to predict human-perceived image quality. 

\begin{table*}[t]
\scriptsize
\setlength{\tabcolsep}{0.4mm}

\newcolumntype{?}{!{\vrule width 1pt}}
\newcolumntype{C}{>{\centering\arraybackslash}X}

\begin{center}

\begin{tabularx}{\textwidth}{l|*{5}{C}|*{5}{C}|*{5}{C}}
\toprule[0.15em]
\rowcolor{gray} & \multicolumn{5}{c|}{QF=5} & \multicolumn{5}{c|}{QF=10} & \multicolumn{5}{c}{QF=20} \\
\rowcolor{gray}
\multirow{-2}{*}{Methods}  & LPIPS$\downarrow$ & DISTS$\downarrow$ & MUSIQ$\uparrow$ & M-IQA$\uparrow$& C-IQA$\uparrow$ & LPIPS$\downarrow$ & DISTS$\downarrow$ & MUSIQ$\uparrow$ & M-IQA$\uparrow$ & C-IQA$\uparrow$ & LPIPS$\downarrow$ & DISTS$\downarrow$ & MUSIQ$\uparrow$ & M-IQA$\uparrow$ & C-IQA$\uparrow$ \\
\midrule[0.15em]
JPEG~\cite{wallace1991jpeg} & 0.4384 & 0.3242 & 40.33 & 0.2294 & 0.1716 & 0.3013 & 0.2387 & 53.88 & 0.3509 & 0.2737 & 0.1799 & 0.1653 & 64.12 & 0.4411 & 0.5542 \\
FBCNN~\cite{jiang2021towards} & 0.3736 & 0.2353 & 63.56 & \textcolor{iccvblue}{0.3425} & 0.2763 & 0.2503 & 0.1785 & 71.00 & 0.4207 & 0.4767 & 0.1583 & 0.1319 & 73.96 & 0.4551 & 0.5535 \\
JDEC~\cite{han2024jdec} & 0.4113 & 0.2364 & 55.66 & 0.2002 & 0.1539 & 0.2450 & 0.1740 & 70.80 & 0.4065 & 0.4811 & 0.1555 & 0.1282 & 73.81 & 0.4433 & 0.5512 \\
PromptCIR~\cite{li2024promptcir} & 0.3797 & 0.2334 & 60.34 & 0.2790 & 0.2655 & 0.2290 & 0.1658 & \textcolor{iccvblue}{72.39} & 0.4500 & 0.5176 & 0.1450 & 0.1223 & \textcolor{iccvblue}{74.12} & 0.4713 & 0.5847 \\
\midrule
DiffBIR*~\cite{lin2024diffbir} (s=50) & 0.3509 & 0.2035 & 58.09 & 0.2812 & 0.3776 & 0.2160 & 0.1319 & 67.38 & 0.3789 & 0.5789 & 0.1500 & 0.0988 & 71.08 & 0.4371 & 0.6814 \\
SUPIR~\cite{yu2024scaling} (s=50) & 0.4856 & 0.2720 & 52.69 & 0.3229 & 0.3149 & 0.2770 & 0.1558 & 68.77 & \textcolor{iccvblue}{0.5183} & 0.6115 & 0.1683 & 0.1121 & 73.02 & \textcolor{red}{0.6237} & \textcolor{iccvblue}{0.7364} \\
OSEDiff*~\cite{wu2024one} (s=1) & \textcolor{iccvblue}{0.2675} & \textcolor{iccvblue}{0.1653} & \textcolor{iccvblue}{65.51} & 0.3417 & \textcolor{iccvblue}{0.5623} & \textcolor{iccvblue}{0.1749} & \textcolor{iccvblue}{0.1164} & 71.23 & 0.3963 & \textcolor{iccvblue}{0.7022} & \textcolor{iccvblue}{0.1270} & \textcolor{iccvblue}{0.0856} & 72.70 & 0.4219 & 0.7260 \\
\textbf{\mymodel}~(ours, s=1) & \textcolor{red}{0.2062} & \textcolor{red}{0.1121} & \textcolor{red}{73.16} & \textcolor{red}{0.5321} & \textcolor{red}{0.7212} & \textcolor{red}{0.1428} & \textcolor{red}{0.0867} & \textcolor{red}{74.39} & \textcolor{red}{0.5438} & \textcolor{red}{0.7559} & \textcolor{red}{0.1101} & \textcolor{red}{0.0692} & \textcolor{red}{74.34} & \textcolor{iccvblue}{0.5323} & \textcolor{red}{0.7565} \\

\bottomrule[0.15em]
\end{tabularx}

\vspace{1mm}
% \scriptsize
% \footnotesize
\small
% \normalsize
(a) Quantitative comparison on the LIVE-1 dataset
\vspace{1mm}
\scriptsize

\begin{tabularx}{\textwidth}{l|*{5}{C}|*{5}{C}|*{5}{C}}
\toprule[0.15em]
\rowcolor{gray} & \multicolumn{5}{c|}{QF=5} & \multicolumn{5}{c|}{QF=10} & \multicolumn{5}{c}{QF=20} \\
\rowcolor{gray}
\multirow{-2}{*}{Methods}  & LPIPS$\downarrow$ & DISTS$\downarrow$ & MUSIQ$\uparrow$ & M-IQA$\uparrow$& C-IQA$\uparrow$ & LPIPS$\downarrow$ & DISTS$\downarrow$ & MUSIQ$\uparrow$ & M-IQA$\uparrow$ & C-IQA$\uparrow$ & LPIPS$\downarrow$ & DISTS$\downarrow$ & MUSIQ$\uparrow$ & M-IQA$\uparrow$ & C-IQA$\uparrow$ \\
\midrule[0.15em]
JPEG~\cite{wallace1991jpeg} & 0.3481 & 0.2834 & 50.46 & 0.3656 & 0.2806 & 0.2254 & 0.2145 & 60.87 & 0.4401 & 0.3517 & 0.1244 & 0.1521 & 67.60 & 0.4967 & 0.5343 \\
FBCNN~\cite{jiang2021towards} & 0.2341 & 0.2162 & 69.03 & 0.4263 & 0.3800 & 0.1462 & 0.1648 & 72.55 & 0.5033 & 0.5014 & 0.0896 & 0.1249 & \textcolor{iccvblue}{73.39} & 0.5288 & 0.5437 \\
JDEC~\cite{han2024jdec} & 0.2794 & 0.2309 & 62.97 & 0.3386 & 0.2518 & 0.1382 & 0.1570 & 72.52 & 0.5001 & 0.4959 & 0.0846 & 0.1175 & 73.30 & 0.5230 & 0.5369 \\
PromptCIR~\cite{li2024promptcir} & 0.2389 & 0.2037 & 66.08 & 0.3946 & 0.3619 & \textcolor{iccvblue}{0.1183} & 0.1431 & \textcolor{red}{73.01} & 0.5380 & 0.5337 & \textcolor{iccvblue}{0.0739} & 0.1083 & \textcolor{red}{73.47} & 0.5489 & 0.5662 \\
\midrule
DiffBIR*~\cite{lin2024diffbir} (s=50) & 0.2018 & \textcolor{iccvblue}{0.1657} & 69.63 & 0.4285 & 0.5470 & 0.1344 & 0.1207 & 71.77 & 0.4813 & 0.5966 & 0.1005 & 0.0939 & 72.51 & 0.5105 & 0.6306 \\
SUPIR~\cite{yu2024scaling} (s=50) & 0.3279 & 0.2018 & \textcolor{iccvblue}{69.94} & \textcolor{iccvblue}{0.5546} & 0.5536 & 0.2489 & 0.1659 & 72.57 & \textcolor{red}{0.5995} & 0.6178 & 0.2125 & 0.1518 & 73.01 & \textcolor{red}{0.6105} & 0.6397 \\
OSEDiff*~\cite{wu2024one} (s=1) & \textcolor{iccvblue}{0.1959} & 0.1690 & 68.60 & 0.4491 & \textcolor{iccvblue}{0.5710} & 0.1262 & \textcolor{iccvblue}{0.1168} & 71.55 & 0.4927 & \textcolor{iccvblue}{0.6364} & 0.0911 & \textcolor{iccvblue}{0.0860} & 72.43 & 0.5082 & \textcolor{iccvblue}{0.6591} \\
\textbf{\mymodel}~(ours, s=1) & \textcolor{red}{0.1407} & \textcolor{red}{0.1101} & \textcolor{red}{72.16} & \textcolor{red}{0.5693} & \textcolor{red}{0.6741} & \textcolor{red}{0.0974} & \textcolor{red}{0.0842} & \textcolor{iccvblue}{72.61} & \textcolor{iccvblue}{0.5725} & \textcolor{red}{0.6824} & \textcolor{red}{0.0753} & \textcolor{red}{0.0667} & 72.63 & \textcolor{iccvblue}{0.5694} & \textcolor{red}{0.6830} \\

\bottomrule[0.15em]
\end{tabularx}

\vspace{1mm}
% \scriptsize
% \footnotesize
\small
% \normalsize
(b) Quantitative comparison on the Urban100 dataset
\vspace{1mm}
\scriptsize

\begin{tabularx}{\textwidth}{l|*{5}{C}|*{5}{C}|*{5}{C}}
\toprule[0.15em]
\rowcolor{gray} & \multicolumn{5}{c|}{QF=5} & \multicolumn{5}{c|}{QF=10} & \multicolumn{5}{c}{QF=20} \\
\rowcolor{gray}
\multirow{-2}{*}{Methods}  & LPIPS$\downarrow$ & DISTS$\downarrow$ & MUSIQ$\uparrow$ & M-IQA$\uparrow$& C-IQA$\uparrow$ & LPIPS$\downarrow$ & DISTS$\downarrow$ & MUSIQ$\uparrow$ & M-IQA$\uparrow$ & C-IQA$\uparrow$ & LPIPS$\downarrow$ & DISTS$\downarrow$ & MUSIQ$\uparrow$ & M-IQA$\uparrow$ & C-IQA$\uparrow$ \\
\midrule[0.15em]
JPEG~\cite{wallace1991jpeg} & 0.4466 & 0.3183 & 34.59 & 0.2570 & 0.2595 & 0.3234 & 0.2255 & 47.53 & 0.3120 & 0.3303 & 0.2072 & 0.1465 & 57.45 & 0.3557 & 0.5072 \\
FBCNN~\cite{jiang2021towards} & 0.3445 & 0.2078 & 56.52 & 0.3025 & 0.3004 & 0.2448 & 0.1581 & 61.79 & 0.3593 & 0.4561 & 0.1733 & 0.1168 & 65.20 & 0.3775 & 0.5221 \\
JDEC~\cite{han2024jdec} & 0.3811 & 0.2234 & 53.88 & 0.2118 & 0.1841 & 0.2313 & 0.1574 & \textcolor{red}{67.48} & 0.3689 & 0.4675 & 0.1565 & 0.1152 & \textcolor{red}{69.90} & 0.3927 & 0.5319 \\
PromptCIR~\cite{li2024promptcir} & 0.3549 & 0.2067 & 52.21 & 0.2705 & 0.3041 & 0.2240 & 0.1459 & 62.63 & \textcolor{iccvblue}{0.3758} & 0.4956 & 0.1581 & 0.1061 & 65.62 & 0.3871 & 0.5483 \\
\midrule
DiffBIR*~\cite{lin2024diffbir} (s=50) & 0.2788 & \textcolor{iccvblue}{0.1533} & 60.21 & 0.3220 & \textcolor{iccvblue}{0.4975} & 0.1953 & 0.1072 & 65.22 & 0.3754 & 0.5912 & 0.1542 & 0.0856 & \textcolor{iccvblue}{67.06} & \textcolor{iccvblue}{0.4033} & \textcolor{iccvblue}{0.6355} \\
SUPIR~\cite{yu2024scaling} (s=50) & 0.4372 & 0.2148 & 54.07 & \textcolor{iccvblue}{0.3438} & 0.4219 & 0.3121 & 0.1410 & 61.93 & 0.3570 & 0.5186 & 0.2295 & 0.1161 & 64.87 & 0.3723 & 0.5535 \\
OSEDiff*~\cite{wu2024one} (s=1) & \textcolor{iccvblue}{0.2624} & 0.1474 & \textcolor{iccvblue}{60.83} & 0.3252 & 0.4974 & \textcolor{iccvblue}{0.1823} & \textcolor{iccvblue}{0.0996} & 64.88 & 0.3640 & \textcolor{iccvblue}{0.6208} & \textcolor{iccvblue}{0.1341} & \textcolor{iccvblue}{0.0721} & 66.10 & 0.3689 & 0.6323 \\
\textbf{\mymodel}~(ours, s=1) & \textcolor{red}{0.2086} & \textcolor{red}{0.0994} & \textcolor{red}{66.28} & \textcolor{red}{0.4069} & \textcolor{red}{0.6413} & \textcolor{red}{0.1436} & \textcolor{red}{0.0714} &\textcolor{iccvblue}{66.97} & \textcolor{red}{0.4113} & \textcolor{red}{0.6498} & \textcolor{red}{0.1020} & \textcolor{red}{0.0511} & 66.66 & \textcolor{red}{0.4042} & \textcolor{red}{0.6443} \\

\bottomrule[0.15em]
\end{tabularx}

\vspace{1mm}
% \scriptsize
% \footnotesize
\small
% \normalsize
(c) Quantitative comparison on the DIV2K-val dataset
% \vspace{1mm}

\vspace{-2.5mm}
\caption{Quantitative comparison on LIVE-1, Urban100, and DIV2K-Val datasets with non-diffusion, MSD, and OSD methods. M-IQA stands for MANIQA, and C-IQA stands for CLIPIQA. The best and second best results are colored with \textcolor{red}{red} and \textcolor{iccvblue}{blue}. DiffBIR* and OSEDiff* are retrained on the same dataset as our method for reference.}
\label{table:main_results}
\end{center}
\vspace{-8mm}
\end{table*}


\noindent\textbf{Implementation Details.} We use PyTorch~\citep{paszke2019pytorch} to implement our model. In the first stage (Sec.~\ref{sec:cave}), we optimize CaVE using the Adam optimizer~\cite{kingma2014adam} with a learning rate of 2$\times$10$^{-5}$ and a batch size of $B$=128. The weight for implicit compression prior learning, $\lambda$, is set to 1,000. We train CaVE for 200K iterations with 4 NVIDIA A6000 GPUs.

In the second stage (Sec.~\ref{sec:diff}), we train \mymodel~using the AdamW optimizer~\cite{loshchilov2017decoupled} with a learning rate of 5$\times$10$^{-5}$, weight decay of 1$\times$10$^{-5}$, and a batch size of 32 for both the generator and discriminator. The discriminator employs a pre-trained StableDiffusion UNet encoder and a lightweight MLP. We set the LoRA~\cite{hu2021lora} rank to 16 for fine-tuning. We utilize CaVE to construct the prompt embeddings. The discriminator follows the same training setup as \mymodel. The weight for DISTS loss $\lambda_{D}$ and GAN loss $\lambda_{G}$ is set to 1 and 5$\times$10$^{-3}$, respectively. This phase is trained for 100K iterations on 4 NVIDIA A6000 GPUs. 

\begin{figure*}[t]
%\newlength-4mm
%\setlength{-4mm}{-0.4cm}
\scriptsize
\centering
\begin{tabular}{cc}

\resizebox{1.\textwidth}{!}{
    \begin{tabular}{c}
    
    \hspace{-0.4cm}
    \includegraphics[width=0.2\textwidth, height=0.174\textwidth]{figs//img_075_1/HQ.png}
    \\
    Urban100: img\_091
    \end{tabular}
    \hspace{-0.6cm}
    \begin{tabular}{cccc}
    \includegraphics[width=0.143\textwidth]{figs//img_075_1/GT.png} \hspace{-5mm} &
    \includegraphics[width=0.143\textwidth]{figs//img_075_1/JPEG.png} \hspace{-5mm} &
    \includegraphics[width=0.143\textwidth]{figs//img_075_1/FBCNN.png} \hspace{-5mm} &
    \includegraphics[width=0.143\textwidth]{figs//img_075_1/JDEC.png}
    \\
    HQ \hspace{-5mm} &
    JPEG (QF=1) \hspace{-5mm} &
    FBCNN~\cite{jiang2021towards} \hspace{-5mm} &
    JDEC~\cite{han2024jdec}
    \\
    \includegraphics[width=0.143\textwidth]{figs//img_075_1/PromptCIR.png} \hspace{-5mm} &
    \includegraphics[width=0.143\textwidth]{figs//img_075_1/DiffBIR.png} \hspace{-5mm} &
    \includegraphics[width=0.143\textwidth]{figs//img_075_1/OSEDiff.png} \hspace{-5mm} &
    \includegraphics[width=0.143\textwidth]{figs//img_075_1/ours.png}  
    \\ 
    PromptCIR~\cite{li2024promptcir} \hspace{-5mm} &
    DiffBIR*~\cite{lin2024diffbir} \hspace{-5mm} &
    OSEDiff*~\cite{wu2024one} \hspace{-5mm} &
    CODiff (ours)
    \\
    \end{tabular}
}
\\


% \hspace{-0.4cm}
\resizebox{1.\textwidth}{!}{
    \begin{tabular}{c}
    
    \hspace{-0.4cm}
    \includegraphics[width=0.2\textwidth]{figs/0862/HQ.png}
    \\
    DIV2K-Val: 0862
    \end{tabular}
    
    \hspace{-0.6cm}
    \begin{tabular}{cccc}
    \includegraphics[width=0.143\textwidth]{figs/0862/GT.png} \hspace{-5mm} &
    \includegraphics[width=0.143\textwidth]{figs/0862/JPEG.png} \hspace{-5mm} &
    \includegraphics[width=0.143\textwidth]{figs/0862/FBCNN.png} \hspace{-5mm} &
    \includegraphics[width=0.143\textwidth]{figs/0862/JDEC.png}
    \\
    HQ \hspace{-5mm} &
    JPEG (QF=5) \hspace{-5mm} &
    FBCNN~\cite{jiang2021towards} \hspace{-5mm} &
    JDEC~\cite{han2024jdec}
    \\
    \includegraphics[width=0.143\textwidth]{figs/0862/PromptCIR.png} \hspace{-5mm} &
    \includegraphics[width=0.143\textwidth]{figs/0862/DiffBIR.png} \hspace{-5mm} &
    \includegraphics[width=0.143\textwidth]{figs/0862/OSEDiff.png} \hspace{-5mm} &
    \includegraphics[width=0.143\textwidth]{figs/0862/ours.png}  
    \\ 
    PromptCIR~\cite{li2024promptcir} \hspace{-5mm} &
    DiffBIR*~\cite{lin2024diffbir} \hspace{-5mm} &
    OSEDiff*~\cite{wu2024one} \hspace{-5mm} &
    \mymodel~(ours)
    \\
    \end{tabular}
} \\

\resizebox{1.\textwidth}{!}{
    \begin{tabular}{c}
    
    \hspace{-0.4cm}
    \includegraphics[width=0.2\textwidth, height=0.171\textwidth]{figs//img_001_10/HQ.png}
    \\
    Urban100: img\_080
    \end{tabular}
    \hspace{-0.6cm}
    \begin{tabular}{cccc}
    \includegraphics[width=0.143\textwidth]{figs//img_001_10/GT.png} \hspace{-5mm} &
    \includegraphics[width=0.143\textwidth]{figs//img_001_10/JPEG.png} \hspace{-5mm} &
    \includegraphics[width=0.143\textwidth]{figs//img_001_10/FBCNN.png} \hspace{-5mm} &
    \includegraphics[width=0.143\textwidth]{figs//img_001_10/JDEC.png}
    \\
    HQ \hspace{-5mm} &
    JPEG (QF=10) \hspace{-5mm} &
    FBCNN~\cite{jiang2021towards} \hspace{-5mm} &
    JDEC~\cite{han2024jdec}
    \\
    \includegraphics[width=0.143\textwidth]{figs//img_001_10/PromptCIR.png} \hspace{-5mm} &
    \includegraphics[width=0.143\textwidth]{figs//img_001_10/DiffBIR.png} \hspace{-5mm} &
    \includegraphics[width=0.143\textwidth]{figs//img_001_10/OSEDiff.png} \hspace{-5mm} &
    \includegraphics[width=0.143\textwidth]{figs//img_001_10/ours.png}  
    \\ 
    PromptCIR~\cite{li2024promptcir} \hspace{-5mm} &
    DiffBIR*~\cite{lin2024diffbir} \hspace{-5mm} &
    OSEDiff*~\cite{wu2024one} \hspace{-5mm} &
    CODiff (ours)
    \\
    \end{tabular}
}

\end{tabular}
\vspace{-4mm}
\caption{Visual comparison on different quality factors (QF=1, 5 and 10). DiffBIR* and OSEDiff* are retrained as references.}
\label{fig:main_comp}
\vspace{-5mm}
\end{figure*}

\subsection{Comparison with SOTA Methods}
\noindent\textbf{Compared Methods.} We compare \mymodel~with state-of-the-art CNN and Transformer-based methods specifically designed for JPEG artifact removal, including FBCNN~\cite{jiang2021towards}, JDEC~\cite{han2024jdec}, and PromptCIR~\cite{li2024promptcir}. Additionally, we evaluate \mymodel~against recent leading diffusion-based image restoration methods, including DiffBIR~\cite{lin2024diffbir}, SUPIR~\cite{yu2024scaling}, and OSEDiff~\cite{wu2024one}. Among them, DiffBIR and OSEDiff are built upon the pre-trained StableDiffusion~\cite{rombach2022high} model. While SUPIR leverages StableDiffusion-XL~\cite{podell2023sdxl}, a larger model with 2.6 billion parameters. We re-train DiffBIR and OSEDiff under the same experimental settings as ours and denote them as DiffBIR* and OSEDiff* respectively.

\noindent\textbf{Quantitative Results.} 
The quantitative comparisons on LIVE-1, Urban100, and DIV2K-Val datasets are summarized in Tab.~\ref{table:main_results}. \mymodel~consistently outperforms competing methods across a diverse range of evaluation metrics. Specifically, it achieves notable improvements in both full-reference metrics (LPIPS and DISTS) and no-reference metrics (MUSIQ, MANIQA, and CLIPIQA). Those comparisons demonstrate its superior ability to recover visually pleasing and perceptually accurate high-quality images.

Compared to existing OSD models, \mymodel~significantly surpasses OSEDiff across all evaluation criteria, highlighting the effectiveness of incorporating compression-aware visual embeddings into the diffusion process, as well as the elaborately designed training objectives. Additionally, \mymodel~outperforms MSD models, such as DiffBIR and SUPIR, in most metrics, despite requiring much fewer sampling steps. This suggests that \mymodel~can achieve high-quality restoration with greater efficiency.

Moreover, diffusion-based methods generally outperform CNN and Transformer-based approaches, particularly when handling highly compressed images (\eg, QF=5). This advantage primarily stems from the rich image generation priors of large-scale T2I diffusion models (\eg, StableDiffusion~\cite{rombach2022high}). These priors enable the models to compensate for the severe loss of visual information in highly compressed images, allowing for more faithful and perceptually realistic reconstructions.


\vspace{-0.1mm}
\noindent\textbf{Qualitative Results.} 
% 先分析再给结论
% 先别的方法不足,再说自己好,再分析自己的好在哪里
As shown in Fig.~\ref{fig:main_comp}, previous methods, including FBCNN~\cite{jiang2021towards}, JDEC~\cite{han2024jdec}, PromptCIR~\cite{li2024promptcir}, and OSEDiff~\cite{wu2024one}, can only partially mitigate JPEG artifact. However, noticeable compression artifact, such as blocky patterns, color banding, and grid-like distortions, still persist in their outputs. While multi-step DiffBIR is more effective in suppressing artifact, it tends to produce overly smooth results, sacrificing texture details in the process. In contrast, our \mymodel~effectively eliminates JPEG artifact while preserving fine-grained contents, such as fur patterns and architectural details. This highlights the model’s ability to recover complex details that are often lost due to heavy compression. Overall, our method strikes a balance between artifact removal and texture preservation, ensuring that the restored images retain both high perceptual quality and structural integrity. 

\vspace{-2mm}
\subsection{Complexity Analyses}
\vspace{-2mm}
Table~\ref{table:complexity} presents a comparison of model complexity, considering key factors such as the number of sampling steps (\#Step), parameter number (Params), multiply-accumulate operations (MACs), and inference time (Time). To ensure a fair assessment, all models are evaluated on a single NVIDIA A6000 GPU, except for SUPIR, which requires two A6000 GPUs due to its extensive model size. Notably, \mymodel~achieves a significant reduction in computational cost, outperforming MSD models by a substantial margin. This highlights its remarkable efficiency without compromising image quality. Moreover, \mymodel~requires fewer parameters compared to other diffusion based approaches, because we avoid complex auxiliary modules. Specifically, OSEDiff~\cite{wu2024one} employs DAPE~\cite{wu2024seesr} for textual prompt extraction. Meanwhile, DiffBIR~\cite{lin2024diffbir} and SUPIR~\cite{yu2024scaling} utilize ControlNet~\cite{zhang2023adding} to integrate low-quality (LQ) image information as the denosing condition. These auxiliary modules substantially increasing parameter counts. In contrast, our CaVE is a lightweight module that efficiently extracts compression priors with low computational overhead.

\begin{table}[t]
%\scriptsize
\footnotesize
%\small
%\normalsize
\centering
\begin{center}
\vspace{.10mm}
%\vspace{-5mm}
%\begin{tabular*}{84.5mm}{@{\extracolsep{-0.99mm}}|c|c|c|c|c|c|c|c|c|c|c|c|c|c|c|c|c|}
\resizebox{1\columnwidth}{!}{
%\setlength{\tabcolsep}{2.1mm}
\begin{tabular}{l|c|c|c|c}
\toprule[0.15em]
\rowcolor{gray}
Method &  {\#Step} & Params (G) &  MACs (T) & Time (s)
\\
\midrule[0.15em]
DiffBIR~\cite{lin2024diffbir} & 50 & 1.52 & 188.24 & 50.81     
\\
SUPIR~\cite{yu2024scaling} & 50 & 4.49 & 464.29 & 24.33
\\
OSEDiff~\cite{wu2024one} & 1 & 1.40 & 10.39 & 0.65        
\\
%\hline
\mymodel~(ours) & 1 & 1.00 & 9.46 & 0.57    
\\
\bottomrule[0.15em]
\end{tabular}}
\vspace{-3mm}
\caption{Complexity comparison among diffusion-based methods. Input image size is 1,024$\times$1,024 for inference.} 

\label{table:complexity}
\end{center}
\vspace{-6mm}
\end{table}

\subsection{Ablation Studies}
\vspace{-0.5mm}
\noindent\textbf{Compression-aware Visual Embedder (CaVE).} We assess the effectiveness of different prompt methods, including empty strings, learnable embeddings, Degradation-Aware Prompt Extractor (DAPE)~\cite{wu2024seesr}, and CaVE. Table~\ref{table:embed} provides the results on LIVE-1 and DIV2K-Val test sets, focusing on challenging scenarios (\eg, QF=1), where extreme compression leads to severe visual information loss. Our findings demonstrate that CaVE consistently outperforms other approaches across most evaluation metrics. In Fig.~\ref{fig:embed}, utilizing CaVE significantly reduces JPEG artifact, yielding noticeably more clear image reconstructions. Such visualization further shows the effect of CaVE.

\begin{table}[t]
% \scriptsize
% \footnotesize
%\small
% \normalsize
\Large
\centering
\begin{center}
\vspace{.10mm}

\resizebox{1\columnwidth}{!}{
\begin{tabular}{l|ccc|ccc}
\toprule[0.15em]
\rowcolor{gray} & \multicolumn{3}{c|}{LIVE-1} & \multicolumn{3}{c}{DIV2K-Val} \\

\rowcolor{gray}
\multirow{-2}{*}{Method} & LPIPS$\downarrow$ & MUSIQ$\uparrow$ & M-IQA$\uparrow$ & LPIPS$\downarrow$ & MUSIQ$\uparrow$ & M-IQA$\uparrow$ \\
\midrule[0.15em]
Empty & 0.3485 & 62.56 & 0.3793 & 0.3241 & 56.68 & 0.3142 \\
Learnable & 0.3471 & \textcolor{iccvblue}{63.39} & \textcolor{iccvblue}{0.3900} & 0.3235 & 56.92 & \textcolor{iccvblue}{0.3262} \\
DAPE~\cite{wu2024seesr} & \textcolor{iccvblue}{0.3463} & 62.54 & 0.3793 & \textcolor{iccvblue}{0.3230} & \textcolor{iccvblue}{57.29} & 0.3240 \\
CaVE (ours) & \textcolor{red}{0.3426} & \textcolor{red}{67.13} & \textcolor{red}{0.4584} & \textcolor{red}{0.3179} & \textcolor{red}{61.83} & \textcolor{red}{0.3709} \\
\bottomrule[0.15em]
\end{tabular}}
\vspace{-3mm}
\caption{Ablation study of prompt embedding generation on LIVE-1 and DIV2K-val datasets. The best and second best results are colored with \textcolor{red}{red} and \textcolor{iccvblue}{blue}, respectively.} 
\label{table:embed}
\end{center}
\vspace{-7mm}
\end{table}

\begin{figure}[t]
\scriptsize
\centering
\begin{tabular}{cc}
% % one row
\hspace{-0.56cm}

\begin{tabular}{c}
\includegraphics[width=0.14\textwidth, height=0.181\textwidth]{figs/embed/img_023/HQ.png}
\\
Urban100: img\_023 \\
\end{tabular}

\hspace{-5.6mm}

\begin{tabular}{ccc}
\includegraphics[width=0.114\textwidth]{figs/embed/img_023/GT.png} \hspace{-5mm} &
\includegraphics[width=0.114\textwidth]{figs/embed/img_023/JPEG.png} \hspace{-5mm} &
\includegraphics[width=0.114\textwidth]{figs/embed/img_023/empty_text.png} \hspace{-5mm}
\\
HQ \hspace{-5mm}
 &
JPEG (QF=5) \hspace{-5mm}  &
Empty String \hspace{-5mm}
\\

\includegraphics[width=0.114\textwidth]{figs/embed/img_023/learnable.png} \hspace{-5mm} &
\includegraphics[width=0.114\textwidth]{figs/embed/img_023/text.png} \hspace{-5mm} &
\includegraphics[width=0.114\textwidth]{figs/embed/img_023/ours.png} \hspace{-5mm}  
\\ 
Learnable \hspace{-5mm}  &
DAPE~\cite{wu2024seesr} \hspace{-5mm} &
CaVE~(ours) \hspace{-5mm}
\\

\end{tabular}

\end{tabular}
\vspace{-3mm}
\caption{Visual comparison of prompt embedding generation. We use Urban100 img\_023 as an example.}
\label{fig:embed}
\vspace{-5.5mm}

\end{figure}

\noindent\textbf{Dual Learning.} To validate the effectiveness of our dual learning, we train CaVE with different learning strategies. Then we utilize CaVE to guide the reconstruction of \mymodel. As shown in Tab.~\ref{table:cavelearning}, the model trained with dual learning consistently achieves the best performance across all evaluation metrics. This suggests that dual learning paradigms enables CaVE to capture richer compression priors.

Moreover, we evaluate CaVE's generalization ability by predicting QFs from low-quality images, specifically testing on QFs that are absent from the training set. As shown in Fig.~\ref{fig:qf_pred}, CaVE trained with explicit learning struggles to generalize to unseen QFs, whereas the dual learning approach significantly enhances its generalization capability.

\begin{table}[t]
% \scriptsize
% \footnotesize
%\small
% \normalsize
\Large
\centering
\begin{center}
\vspace{.10mm}

\resizebox{1\columnwidth}{!}{
\begin{tabular}{l|ccc|ccc}
\toprule[0.15em]
\rowcolor{gray} & \multicolumn{3}{c|}{LIVE-1} & \multicolumn{3}{c}{DIV2K-Val} \\

\rowcolor{gray}
\multirow{-2}{*}{Type} & LPIPS$\downarrow$ & MUSIQ$\uparrow$ & M-IQA$\uparrow$ & LPIPS$\downarrow$ & MUSIQ$\uparrow$ & M-IQA$\uparrow$ \\
\midrule[0.15em]
Explicit & 0.3733 & 62.39 & 0.3828 & 0.3398 & 57.26 & 0.3202 \\
Implicit & \textcolor{iccvblue}{0.3436} & \textcolor{iccvblue}{64.74} & \textcolor{iccvblue}{0.4192} & \textcolor{iccvblue}{0.3208} & \textcolor{iccvblue}{58.74} & \textcolor{iccvblue}{0.3403} \\
Dual & \textcolor{red}{0.3426} & \textcolor{red}{67.13} & \textcolor{red}{0.4584} & \textcolor{red}{0.3179} & \textcolor{red}{61.83} & \textcolor{red}{0.3709} \\
\bottomrule[0.15em]
\end{tabular}}
\vspace{-3mm}
\caption{Ablation study of the dual learning strategy. The best and second best results are colored with \textcolor{red}{red} and \textcolor{iccvblue}{blue}, respectively.} 

\label{table:cavelearning}
\end{center}
\vspace{-7mm}
\end{table}



\noindent\textbf{\mymodel~Training Loss Functions.}
\mymodel~incorporates multiple loss functions during training, including MSE loss, DISTS loss, and GAN loss, to enhance overall model performance. To systematically evaluate the contribution of each component, we conduct extensive ablation studies. As shown in Tab.~\ref{table:ablation_loss}, training \mymodel~solely with $\mathcal{L}_{2}$ results in suboptimal perceptual metrics. However, incorporating $\mathcal{L}_{DISTS}$ significantly enhances \mymodel's performance. This highlights the pivotal role of DISTS loss in enhancing model's ability to achieve high-quality reconstructions. Furthermore, adding the GAN loss provides an additional performance boost, indicating its complementary effect in refining the reconstruction quality. 

\begin{table}[t]
    \centering
    % \scriptsize
    \footnotesize
    % \small
    \setlength{\tabcolsep}{0.5mm} 
    \newcolumntype{C}{>{\centering\arraybackslash}X}

    \newcolumntype{S}{>{\centering\arraybackslash}c}

    \begin{tabularx}{\columnwidth}{SSS|*{5}{C}} 
        \toprule[0.15em]
        \rowcolor{gray} $\mathcal{L}_2$ & $\mathcal{L}_{\text{DISTS}}$ & $\mathcal{L}_\mathcal{G}$ & LPIPS$\downarrow$ & DISTS$\downarrow$ & MUSIQ$\uparrow$ & M-IQA$\uparrow$& C-IQA$\uparrow$ \\
        \midrule[0.15em]
        \checkmark & &  & 0.3976 & 0.2509 & 59.79 & 0.3212 & 0.3085  \\
        \checkmark & \checkmark & & \textcolor{red}{0.2109} & \textcolor{iccvblue}{0.1229} & \textcolor{iccvblue}{72.69} & \textcolor{iccvblue}{0.5147} & \textcolor{iccvblue}{0.7119} \\ \checkmark & & \checkmark &  0.3899 & 0.2411 & 59.98 & 0.3301 & 0.3130 \\
        \checkmark & \checkmark & \checkmark & \textcolor{iccvblue}{0.2113} & \textcolor{red}{0.1147} & \textcolor{red}{73.34} & \textcolor{red}{0.5221} & \textcolor{red}{0.7317}  \\
        \bottomrule[0.15em]
    \end{tabularx}

    \vspace{-3mm}
    \caption{Ablation studies on different loss functions. The best and second best results are colored with \textcolor{red}{red} and \textcolor{iccvblue}{blue}, respectively.}
    \label{table:ablation_loss}
    \vspace{-4mm}
\end{table}





\begin{figure}[t]
    \begin{center}
    \begin{tabular}{cc}
    \hspace{-4mm}
        \includegraphics[width=0.5\columnwidth]{figs/LIVE1_color_qf_err.pdf}  &
        \hspace{-3.5mm}
        \includegraphics[width=0.5\columnwidth]{figs/DIV2K_valid_qf_err.pdf} %\\
       % \scriptsize{(a) LIVE-1} & \scriptsize{(b) DIV2K-val} \\
    \end{tabular}
    \end{center}
    \vspace{-8.5mm}
    \caption{Comparison of CaVE's QF prediction mean squared error (MSE) between explicit learning and dual learning on LIVE-1 and DIV2K-val datasets.}
    \label{fig:qf_pred}
    \vspace{-6mm} 
\end{figure}

\vspace{-1mm}
\section{Conclusion}
\vspace{-1mm}
We propose \mymodel, a one-step diffusion model designed for efficient JPEG artifact removal. Our approach leverages the generative capabilities of text-to-image diffusion models to restore heavily JPEG-compressed images. By integrating a compression-aware visual embedder, we effectively utilize JPEG compression priors to guide the denoising process. This enables more accurate and visually coherent restorations. Beyond addressing the limitations of existing methods, \mymodel~establishes a novel pathway for using compression priors to guide diffusion models for JPEG artifact removal. Extensive experiments show the superiority of our \mymodel~over recent leading methods.

{
    \small
    \bibliographystyle{ieeenat_fullname}
    \bibliography{main}
}

\end{document}
