% Standard package includes
\usepackage{times}
\usepackage{latexsym}
\usepackage{booktabs}

% For proper rendering and hyphenation of words containing Latin characters (including in bib files)
\usepackage[T1]{fontenc}
\usepackage{multirow}
\usepackage{colortbl}
\usepackage{pifont}
\usepackage[ruled,linesnumbered]{algorithm2e}


% For Vietnamese characters
% \usepackage[T5]{fontenc}
% See https://www.latex-project.org/help/documentation/encguide.pdf for other character sets

% This assumes your files are encoded as UTF8
\usepackage[utf8]{inputenc}
\usepackage{amsmath}

% This is not strictly necessary, and may be commented out,
% but it will improve the layout of the manuscript,
% and will typically save some space.
\usepackage{microtype}
\usepackage{hyperref}

% This is also not strictly necessary, and may be commented out.
% However, it will improve the aesthetics of text in
% the typewriter font.
\usepackage{inconsolata}
\usepackage{threeparttable}

%Including images in your LaTeX document requires adding
%additional package(s)
\usepackage{graphicx}
\usepackage{enumitem}
\usepackage{tcolorbox}
\usepackage{amsmath}
\usepackage{amsfonts}
\usepackage{mathtools}

% If the title and author information does not fit in the area allocated, uncomment the following
%
%\setlength\titlebox{<dim>}
%
% and set <dim> to something 5cm or larger.

\newcommand{\tool}{\textsc{AutoCMD}}

\newcommand{\qing}[1]{{\color{green}[wq:#1]}}
%\newcommand{\lin}[1]{{\color{blue}\{Lin: #1\}}}
\newcommand{\yang}[1]{{\color{blue}\textbf{[Mingyang: #1]}}}
\newcommand{\junjie}[1]{{\color{orange}\textbf{[Junjie: #1]}}}
\newcommand{\ziyou}[1]{{\color{red}\textbf{[#1]}}}

\newcommand{\guowei}[1]{{\color{cyan}\textbf{[Guowei: #1]}}}
