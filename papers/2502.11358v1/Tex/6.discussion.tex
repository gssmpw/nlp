\section{Case Analysis}\label{sec:case_study}

\begin{figure}[b]
\centering
\includegraphics[width=\columnwidth]{Fig/case_study_v2.pdf}
\vspace{-0.6cm}
\caption{The case study of {\tool}.}
\label{fig:case_study}
\vspace{-0.6cm}
\end{figure}

To intuitively illustrate the benefits of {\tool}, we apply FixedCMD, FixedDBCMD, and {\tool} to Figure \ref{fig:motivation_tool_learning}'s example and observe the attack results from the output of ToolBench.
Figure \ref{fig:case_study} shows the results of the case study.
We can see that, the command generated by FixedCMD is defended by the LLM, so the frontend output and the backend toolchain are not affected.
FixedDBCMD can generate the command that successfully calls \textit{Book\_Flight} again and steals the \textit{Book\_Hotel}'s input. However, this abnormal toolchain is shown in the frontend, which will be observed by the users.
Compared with them, The command generated by {\tool} can not only achieve information theft but also have stealthiness, which means the attack is not exposed in the frontend.
In conclusion, {\tool} is applicable to generate effective commands that can applied to information theft attacks. 

