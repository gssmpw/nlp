\section{Related Work}
In this section, we illustrate how gaze behavior is related to cognitive processes in reading and examine the reading behavior analysis enabled by gaze tracking. Then, we explain the limitations of the gaze-based unknown word detection method and how the previous works improved their performance. After that, we analyze the gaps in detecting unknown words accurately.

\subsection{Gaze in Reading}
\label{sec:related_word_gaze}
Reading as a cognitive process affects eye movement~\cite{just1980theory}. When people are reading, the eyes follow the text through small amplitudes, ballistic motions called saccades. The pauses between two saccades are called a fixation~\cite{idict2006hyrskykari}. According to the eye-mind hypothesis, the fixation on a word persists during its processing phase~\cite{just1980theory}. Consequently, fixation duration can be a metric for identifying difficult words and measuring cognitive processes in reading. For this reason, previous research combines fixation duration with other gaze features to predict reading comprehension~\cite{cheng_gaze-based_2015, okoso_towards_2015,sanches_using_2017}, detecting mind wandering~\cite{bixler_automatic_2016}, identifying interest~\cite{wikigaze_2020_dubey}, and detecting attention~\cite{li_multimodal_2016, zermiani-etal-2024-interead, hollenstein-etal-2020-zuco} in reading. Other works facilitate natural language processing tasks such as named entity recognition and sequence classification by integrating gaze into neural network~\cite{sentiment_long_21,barrett2020sequence, barrett-etal-2018-sequence, ner_Hollenstein_2019}. Besides, some researches also improve the prediction of gaze behavior such as scanpath and calibration process based on the connection between gaze and text~\cite{TSM_NEURIPS2020_460191c7,eyettention_2023, CalibRead_2024_liu}. However, most of these works focus on paragraph-level and sentence-level tasks rather than word detection.

Extended periods of fixation on the focal word suggest difficulties in word identification~\cite{idict2006hyrskykari}, forming the theoretical basis for the detection of unknown words via gaze. However, it is hard to achieve high accuracy only based on the fixation because of the inaccuracy of gaze-tracking hardware, algorithms, and the ambiguous relationship between gaze patterns and the cognitive processes of understanding words. The highest accuracy of eye tracking is around $0.3^\circ$ (2.6 mm when the distance is 50 cm) under optimal conditions, but the accuracy can easily be affected by the calibration performance and user posture~\cite{eye_tracking2022liu}. Thus, how to accurately match the gaze point to the focused text is a problem. Additionally, other researchers pointed out that the processing of words can occur when they are not held in fixation~\cite{rayner1998eye}. 

Although there is a strong correlation between gaze behavior and word difficulty, the above limitations make it difficult to achieve high accuracy in detecting unknown words based only on gaze.
Previous gaze-based unknown word detection methods improve their accuracy by combining multiple gaze features and leveraging text information such as word length and word rarity~\cite{unknown-word_hiraoka_2016, gaze-text_garain_2017}. 

\subsection{Unknown Word Detection}

According to the importance of gaze in analyzing reading behavior as explained above, most of the related works are gaze-based. iDict~\cite{idict2006hyrskykari} detects unknown words based on gaze duration and word frequency and sets a threshold to trigger a gloss or margin note on unknown words. It successfully detects 36.5\% of unknown words. Later works extend iDict by replacing the threshold function with machine learning. Hiroka et al.~\cite{unknown-word_hiraoka_2016} uses several gaze features such as first gaze duration, number of fixations, and number of regressions and feeds them into support vector machines (SVM) to classify the unknown word. Their model performs best (F1-score of 55.6\%) when adding linguistics features including word length and word rarity. Similarly, Garain et al.~\cite{gaze-text_garain_2017} also take both gaze and linguistics features into consideration to achieve the best F1-score of 86\% on a single user using an SVM. However, each of these methods rely on a professional eye tracker to obtain sufficiently accurate gaze data to extract  gaze features and match these features to the words. Even with a dedicated eye tracker, it is still difficult to accurately assign each gaze point to the corresponding line due to the vertical error~\cite{yamaya2015dynamic, Sanches2016Vertical}. Therefore, in their experiment, the line spacing was set between 3.0 to 6.0 to better distinguish lines based on the y-value. The large line spacing makes it difficult for their method to be applied in real-world scenarios, considering that the line spacing of text we usually read is mostly between 1.0 and 2.0. Moreover, their data preprocessing that relies on global gaze coordination makes them even harder to be applied in a real-time application.

Apart from gaze, other reading behaviors such as mouse clicks and hand motion can be used to detect unknown words. Ehara et al.~\cite{web_ehara_2010} gives users potential highlights in advance and analyzes users' feedback based on their clicks on the web page. Predictions are improved based on this feedback, reaching an accuracy of up to 80.01\%. Higashimura et al.~\cite{imu_higa_2022} targets at vocabulary acquisition on smartphones and identifies unknown words utilizing the motion data obtained from the inertia sensors on smartphones. The estimation improves through the reading and the AUPR is about 0.3. Both methods are device-specific and only applicable to a single user as they need to be optimized based on personalized iterative feedback for higher accuracy.

In summary, the current best-performing detection methods combine gaze and text data. However, these current methods still rely on heavy gaze data preprocessing and special experimental settings. A real-time method that is robust to relatively inaccurate gaze data can promote the availability of unknown-word detection. Since word difficulty is highly dependent on linguistic features\cite{More_than_frequency}, we seek to build upon the rapid development of natural language processing (NLP) technology~\cite{attention_is_all_you_need} in recent years to yield a solution. Pre-trained language models such as ~\cite{devlin2019bert,liu2019roberta} encompass extensive text information for downstream applications. We harness their capabilities in a new architecture to increase the tolerance of inaccuracy in gaze tracking. We propose a transformer-based model that requires linguistic information provided by PLM from a region of interest around the target words and learns the gaze pattern automatically from gaze trajectory and text position.

