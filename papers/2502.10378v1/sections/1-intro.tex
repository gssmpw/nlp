
\section{Introduction}

\begin{figure*}
  \centering
  % \includegraphics[scale = 0.4]{figures/intro.png}
  \includegraphics[width = 0.9\linewidth]{figures/intro.png}
  %\setlength{\abovecaptionskip}{0.1cm}
  \caption{Our method locates the content the user is reading in real-time through gaze, and inputs the gaze data and text data to the transform-based model to detect unknown words.}
  \label{fig:intro} 
\end{figure*}

Unknown words can greatly reduce reading fluency and worsen the reading experience of non-native speakers~\cite{rigg1991whole,mokhtar2012guessing}. By automatically detecting these unknown words as users read, computing systems can assist users in reading and language comprehension, and provide just-in-time word explanations for learning vocabulary. Because unknown words vary among users, many previous works were based on the explicit expression of intention by users, such as mouse clicks~\cite{web_ehara_2010} or looking at words intentionally~\cite{dwell_time, eye_tracking_dussias}. To facilitate a more natural reading experience, other methods use gaze characteristics such as fixation duration, number of fixations, and saccade length~\cite{sibert_reading_2000, Reading_Assistanc_Guo} to detect unknown words automatically, since there is a correlation between gaze and word difficulty~\cite{just1980theory}. Previous research shows that fixation, the maintaining of the gaze on a single location, happens when people are processing phrases~\cite{just1980theory}. However, these gaze-based methods have two major challenges affecting their accuracy and ease of deployment. First, these methods require dedicated and costly eye-tracking hardware to obtain accurate eye movement data for these features. Moreover, even with professional eye trackers, measuring gaze is inherently inaccurate due to complicated eye motions, making it hard to precisely map a gaze point to the word in the text being read~\cite{eye_mice_bates_2003}. To reduce reliance on gaze information, other works seek to compensate or replace inaccurate eye-tracking data on commercial devices by incorporating text~\cite{gaze-text_garain_2017, unknown-word_hiraoka_2016, idict2006hyrskykari, du_using_2024}, click~\cite{web_ehara_2010} and motion data~\cite{imu_higa_2022}. 

Besides the cognitive process reflected by gaze, inherent linguistic information about words is also crucial for identifying unknown words. With the development of Natural Language Processing (NLP) technology, pre-trained language models (PLMs) demonstrate a powerful ability to capture rich linguistic information~\cite{devlin2019bert, liu2019roberta} which is strongly related to word difficulty~\cite{More_than_frequency}.
%recent work attempts to make full use of text information by introducing language models to reduce the method's dependence on high-accuracy eye tracking~\cite{ding_gazereader_2023}. However, it requires taking the gaze data and the text data on the whole page as the input, which makes it unsuitable for real-time detection. Based on this work, 
We explore how to take advantage of the language model in addition to gaze to make unknown word detection accurate, easily accessible, and more applicable. We present a real-time unknown word detection method that locates a region of interest based on gaze and then integrates linguistic information provided by PLM and gaze trajectory to predict unknown words using a transformer-based model. In this way the inaccuracy of the gaze-based method can be compensated by the probabilities distributed on words derived from the language model.

As shown in Fig.~\ref{fig:intro}, we tackle the problem of unknown word detection with two modalities of information, gaze and text. To process gaze patterns, we utilize the cutting-edge transformer-based encoder-decoder model with cross-attention modules to learn the positional information based on gaze trajectory and text layout. For linguistic information, we apply RoBERTa to the text in the region of interest with several crucial word-level knowledge embeddings (term frequency, part-of-speech, etc.). By jointly training the models above on our collected dataset, our approach surpasses existing methods with 71.1\% F1-scores and 97.6\% accuracy for unknown word detection.
With our method, real-time language learning assistance and just-in-time vocabulary acquisition tools can be enabled.

In this paper, we first verified the feasibility of our approach on the dataset collected by a professional eye tracker. We also applied our unknown word detection method to the relatively noisy webcam-based gaze data to show the robustness of our method with commodity deployments. Then, we conducted experiments to analyze the contribution of gaze and PLM. Finally, we implemented a reading assistance prototype to evaluate our method in a real world setting where data is processed in real time and investigate its usefulness and users' subjective experience.

Contributions in this work are summarized as follows:
\begin{enumerate}
    \item We propose an unknown word detection method (\name{}) that leverages gaze to locate a region of interest and then classifies wrods based on linguistic characteristics provided by a pre-trained language model and gaze encoding derived from a transformer-based model. It achieves an accuracy of 97.6\% and an F1-score of 71.1\%.
    \item We analyze the contributions of gaze and the pre-trained language model. Unknown word detection is mainly based on linguistic characteristics provided by a pre-trained language model. Gaze provides user-dependent and timely information to detect different unknown words for different users in a real-time manner.
    \item We build a reading assistance prototype and conduct a real-time evaluation. The result shows that the F1-score is 56.5\%. The user study shows improvement in willingness to use and usefulness compared to the traditional method.
\end{enumerate}

