\section{Applications}
Many applications can be enabled by our unknown word detection method by acquiring use's unknown words during reading in real time. Foreign language reading can occur in two scenarios. One is literature reading and language learning on 2D interfaces such as laptops or pads. The other is getting information in the surrounding environment through a 3D display such as AR glasses and head-mounted devices. We then discuss the potential applications of our method in these two scenarios.


\subsection{Language Learning Assistance}

We can divide the functions of language learning assistance into two categories: real-time and non-real-time. By supporting real-time unknown word detection, our method can make translation less obtrusive and help users read more fluently. Our method can track gaze to locate text areas (the sliding window in Fig.~\ref{fig:application1}) and detect unknown words in the area. Then, the application is able to translate these unknown words automatically. It can save users the time of copying and pasting words into the dictionary or retrieving words through the cursor, as well as reducing interruptions to users' reading. At the same time, unknown words can also be automatically added to the user's word list, allowing users to view them at any time. 

If the overall reading performance is considered and the real-time feedback is not necessary, many applications can be enabled by summarizing and analyzing the unknown words encountered during the reading process. Potential applications include generating flashcards to facilitate users' memory, counting the user's vocabulary mastery to provide users with learning reports and assessing the forgetting rate to offer users a personalized word learning plan. Combined with generative AI, it is also possible to generate new documents based on recently encountered unknown words. This can help users consolidate vocabulary in an intriguing way.

In summary, fluent reading and efficient word learning are the most urgent needs of second language learners. The highly accurate unknown word detection provided by our method can assist users' language learning in either real-time or summarized manner.

\begin{figure}[htbp]
  \includegraphics[width=0.49\columnwidth]{figures/app_1.png}
  \includegraphics[width=0.48\columnwidth]{figures/app_2.png}

  %\setlength{\abovecaptionskip}{0.1cm}
  \caption{Applications in 2D language learning scenario: (Left) Real-time auto translation. (Right) Unknown-word summary and word learning analysis.}
  \label{fig:application1} 
\end{figure}



\subsection{Reading Assistant in Foreign Language Environment}
With the development of augmented reality technology, head-mounted display devices such as Apple Vision Pro\footnote{https://www.apple.com/apple-vision-pro/} will gradually be integrated into daily life in the future. This will allow reading behavior in three-dimensional space to be captured as well. Therefore, we envision that in addition to reading 2D materials, our unknown word detection technology will also be used to assist reading in the three-dimensional world. Moreover, AR headsets are generally equipped with eye trackers, which will enable our method to be easily applied.

Three-dimensional application scenarios include but are not limited to obtaining key information from menus, manuals, and street signs when traveling or living in a foreign language environment and reading commentaries in foreign language exhibitions. Compared with directly displaying large sections of translated text in front of users, providing only key information based on unknown words can reduce the interference to the user's view and reduce the user's burden on extracting key information from a large amount of text. It will provide users with more precise and less intrusive reading assistance in foreign language environments.

\begin{figure}[htbp]
  \includegraphics[width=0.47\columnwidth]{figures/app_3.png}
  \includegraphics[width=0.48\columnwidth]{figures/app_4.png}

  %\setlength{\abovecaptionskip}{0.1cm}
  \caption{Applications in 3D AR scenario: (Left) The user wearing the AR headset encounters an unknown word when reading the manual. (Right) The translation of the unknown word pops out automatically.}
  \label{fig:application2} 
\end{figure}

