%%
%% This is file `sample-acmsmall.tex',
%% generated with the docstrip utility.
%%
%% The original source files were:
%%
%% samples.dtx  (with options: `acmsmall')
%% 
%% IMPORTANT NOTICE:
%% 
%% For the copyright see the source file.
%% 
%% Any modified versions of this file must be renamed
%% with new filenames distinct from sample-acmsmall.tex.
%% 
%% For distribution of the original source see the terms
%% for copying and modification in the file samples.dtx.
%% 
%% This generated file may be distributed as long as the
%% original source files, as listed above, are part of the
%% same distribution. (The sources need not necessarily be
%% in the same archive or directory.)
%%
%% Commands for TeXCount
%TC:macro \cite [option:text,text]
%TC:macro \citep [option:text,text]
%TC:macro \citet [option:text,text]
%TC:envir table 0 1
%TC:envir table* 0 1
%TC:envir tabular [ignore] word
%TC:envir displaymath 0 word
%TC:envir math 0 word
%TC:envir comment 0 0
%%
%%
%% The first command in your LaTeX source must be the \documentclass command.
%\documentclass[acmsmall,screen,anonymous]{acmart}

%% NOTE that a single column version is required for 
%% submission and peer review. This can be done by changing
%% the \doucmentclass[...]{acmart} in this template to 
%% \documentclass[manuscript,screen]{acmart}
\documentclass[sigconf]{acmart}

%% 
%% To ensure 100% compatibility, please check the white list of
%% approved LaTeX packages to be used with the Master Article Template at
%% https://www.acm.org/publications/taps/whitelist-of-latex-packages 
%% before creating your document. The white list page provides 
%% information on how to submit additional LaTeX packages for 
%% review and adoption.
%% Fonts used in the template cannot be substituted; margin 
%% adjustments are not allowed.
%%
%% \BibTeX command to typeset BibTeX logo in the docs
\AtBeginDocument{%
  \providecommand\BibTeX{{%
    \normalfont B\kern-0.5em{\scshape i\kern-0.25em b}\kern-0.8em\TeX}}}

\usepackage{graphics}
\usepackage{stfloats}
\usepackage{url}
% \usepackage{hyperref}
% \usepackage{booktabs}
%% Rights management information.  This information is sent to you
%% when you complete the rights form.  These commands have SAMPLE
%% values in them; it is your responsibility as an author to replace
%% the commands and values with those provided to you when you
%% complete the rights form.
\copyrightyear{2025}
\acmYear{2025}
\setcopyright{cc}
\setcctype{by}
\acmConference[CHI '25]{CHI Conference on Human Factors in Computing Systems}{April 26-May 1, 2025}{Yokohama, Japan}
\acmBooktitle{CHI Conference on Human Factors in Computing Systems (CHI '25), April 26-May 1, 2025, Yokohama, Japan}\acmDOI{10.1145/3706598.3714181}
\acmISBN{979-8-4007-1394-1/25/04}

\newcommand{\todo}[1]{\textcolor{teal}{\emph{{#1}}}}
% \newcommand \changed[1]{\textcolor{blue}{#1}}
\newcommand \changed[1]{\textcolor{black}{#1}}
\newcommand{\yuntao}[1]{\textcolor{green}{\emph{{#1}}}}
\newcommand{\YOURNAME}[1]{\textcolor{orange}{\emph{{#1}}}}
\newcommand{\incomplete}[1]{\textcolor{red}{\emph{{#1}}}}

\newcommand \change[1]{{\textcolor{black}{#1}}}
\newcommand \del[1]{}
\newcommand{\reviewcomment}[1]{\vspace{0.0cm}\begin{mdframed}[backgroundcolor=gray!20]#1\end{mdframed}\vspace{0.4cm}}

\newcommand{\name}{EyeLingo}

%%
%% These commands are for a JOURNAL article.
% \acmJournal{JACM}
% \acmVolume{37}
% \acmNumber{4}
% \acmArticle{111}
% \acmMonth{8}

%%
%% Submission ID.
%% Use this when submitting an article to a sponsored event. You'll
%% receive a unique submission ID from the organizers
%% of the event, and this ID should be used as the parameter to this command.
%%\acmSubmissionID{123-A56-BU3}

%%
%% For managing citations, it is recommended to use bibliography
%% files in BibTeX format.
%%
%% You can then either use BibTeX with the ACM-Reference-Format style,
%% or BibLaTeX with the acmnumeric or acmauthoryear sytles, that include
%% support for advanced citation of software artefact from the
%% biblatex-software package, also separately available on CTAN.
%%
%% Look at the sample-*-biblatex.tex files for templates showcasing
%% the biblatex styles.
%%

%%
%% The majority of ACM publications use numbered citations and
%% references.  The command \citestyle{authoryear} switches to the
%% "author year" style.
%%
%% If you are preparing content for an event
%% sponsored by ACM SIGGRAPH, you must use the "author year" style of
%% citations and references.
%% Uncommenting
%% the next command will enable that style.
%%\citestyle{acmauthoryear}

%%
%% end of the preamble, start of the body of the document source.
\begin{document}

%%
%% The "title" command has an optional parameter,
%% allowing the author to define a "short title" to be used in page headers.
\title[Unknown Word Detection for English as a Second Language (ESL) Learners ...]{Unknown Word Detection for English as a Second Language (ESL) Learners Using Gaze and Pre-trained Language Models}

%%
%% The "author" command and its associated commands are used to define
%% the authors and their affiliations.
%% Of note is the shared affiliation of the first two authors, and the
%% "authornote" and "authornotemark" commands
%% used to denote shared contribution to the research.

\author{Jiexin Ding}
\email{jxding17@gmail.com}
\affiliation{%
  \department{Department of Computer Science and Technology, Global Innovation Exchange (GIX) Institute}
  \institution{Tsinghua University}
  \city{Beijing}
  \country{China}
}
\affiliation{%
  \department{Electrical \& Computer Engineering}
  \institution{University of Washington}
  \city{Seattle}
  \state{WA}
  \country{USA}
}

\author{Bowen Zhao}
\email{bowen@groundlight.ai}
\affiliation{%
  \institution{Groundlight AI}
  \city{Seattle}
  \state{WA}
  \country{USA}
}

\author{Yuntao Wang}
\email{yuntaowang@tsinghua.edu.cn}
\affiliation{%
  \department{Department of Computer Science and Technology}
  \institution{Tsinghua University}
  \city{Beijing}
  \country{China}
  % \postcode{100084}
}
\authornote{denotes as the corresponding author.}

\author{Xinyun Liu}
\email{liuxinyun6@yahoo.com}
\affiliation{%
  \institution{Rice University}
  \city{Houston}
  \state{TX}
  \country{USA}
}

\author{Rui Hao}
\email{haorui24@mails.ucas.ac.cn}
\affiliation{%
  \department{School of Artificial Intelligence}
  \institution{University of Chinese Academy of Sciences}
  \city{Beijing}
  \country{China}
}

\author{Ishan Chatterjee}
\email{ichat@cs.washington.edu}
\affiliation{%
    \department{Paul G. Allen School of Computer Science and Engineering}
   \institution{University of Washington}
  \city{Seattle}
  \state{WA}
  \country{USA}
}

\author{Yuanchun Shi}
\email{shiyc@tsinghua.edu.cn}
\affiliation{%
  \department{Department of Computer Science and Technology}
  \institution{Tsinghua University}
  \city{Beijing}
  \country{China}
}
\affiliation{%
  \institution{Qinghai University}
  \city{Xining}
  \country{China}
}

%%
%% By default, the full list of authors will be used in the page
%% headers. Often, this list is too long, and will overlap
%% other information printed in the page headers. This command allows
%% the author to define a more concise list
%% of authors' names for this purpose.
\renewcommand{\shortauthors}{XXX, et al.}

%%
%% The abstract is a short summary of the work to be presented in the
%% article.
% \begin{abstract}
% Automatically detecting unknown words can help drive interactive methods to support reading for English as a second language (ESL) learners. 
% Previous methods of unknown word detection methods rely on gaze features obtained through eye trackers, with their detection accuracy greatly affected by the accuracy of eye tracking devices. 
% In this work, we present a real-time and high-accuracy unknown word detection method by combining information about the text with gaze data. We locate a target area using gaze and utilize pre-trained language models and knowledge grounding to analyze text for unknown word probabilities in this area. We then combine this text data with gaze information through a transformer-based model. The accuracy of our unknown word detection method is 97.6\%, and the F1-score is 71.1\%. To demonstrate the robustness of our method, we applied our method to another dataset collected with a relatively inaccurate webcam-based eye-tracking system. Our model can achieve the accuracy of 97.3\% and the F1-score of 65.1\% opening the potential for more accessible solutions than previously possible. 
% We also implemented a real-time reading assistance prototype to evaluate our model in real-time setting and show the usability of our method. The explanations of unknown words detected by our model are presented automatically. The user study shows that the precision is xx.x\% and xx.x\% of users think our prototype is useful.
% \end{abstract}

%%% for 9.5 ddl
\begin{abstract}
% Automatically detecting unknown words can help drive interactive methods to support reading for English as a second language (ESL) learners. 
% Previous methods of unknown word detection methods rely on gaze features obtained through eye trackers, with their detection accuracy greatly affected by the accuracy of eye tracking devices. 
%real-time and high-accuracy unknown word detection method by combining text information with gaze data. 
%We first utilized  the word difficulty as 
%We locate a target area using gaze and utilize pre-trained language models to analyze unknown word probabilities in this area. 
%We then combine this text data with gaze information through a transformer-based model. 
English as a Second Language (ESL) learners often encounter unknown words that hinder their text comprehension. Automatically detecting these words as users read can enable computing systems to provide just-in-time definitions, synonyms, or contextual explanations, thereby helping users learn vocabulary in a natural and seamless manner.
This paper presents \name, a transformer-based machine learning method that predicts the probability of unknown words based on text content and eye gaze trajectory in real time with high accuracy. 
A 20-participant user study revealed that our method can achieve an accuracy of 97.6\%, and an F1-score of 71.1\%.
We implemented a real-time reading assistance prototype to show the effectiveness of \name. The user study shows improvement in willingness to use and usefulness compared to baseline methods.
\end{abstract}

%%
%% The code below is generated by the tool at http://dl.acm.org/ccs.cfm.
%% Please copy and paste the code instead of the example below.
%% 

\begin{CCSXML}
<ccs2012>
   <concept>
       <concept_id>10003120.10003121.10003128</concept_id>
       <concept_desc>Human-centered computing~Interaction techniques</concept_desc>
       <concept_significance>500</concept_significance>
       </concept>
 </ccs2012>
\end{CCSXML}


\ccsdesc[500]{Human-centered computing~Interaction techniques}


%%
%% Keywords. The author(s) should pick words that accurately describe
%% the work being presented. Separate the keywords with commas.
\keywords{Unknown word detection, gaze, pre-trained language model.}

% \received{20 February 2007}
% \received[revised]{12 March 2009}
% \received[accepted]{5 June 2009}

%%
%% This command processes the author and affiliation and title
%% information and builds the first part of the formatted document.
\maketitle

%!TEX root = gcn.tex
\section{Introduction}
Graphs, representing structural data and topology, are widely used across various domains, such as social networks and merchandising transactions.
Graph convolutional networks (GCN)~\cite{iclr/KipfW17} have significantly enhanced model training on these interconnected nodes.
However, these graphs often contain sensitive information that should not be leaked to untrusted parties.
For example, companies may analyze sensitive demographic and behavioral data about users for applications ranging from targeted advertising to personalized medicine.
Given the data-centric nature and analytical power of GCN training, addressing these privacy concerns is imperative.

Secure multi-party computation (MPC)~\cite{crypto/ChaumDG87,crypto/ChenC06,eurocrypt/CiampiRSW22} is a critical tool for privacy-preserving machine learning, enabling mutually distrustful parties to collaboratively train models with privacy protection over inputs and (intermediate) computations.
While research advances (\eg,~\cite{ccs/RatheeRKCGRS20,uss/NgC21,sp21/TanKTW,uss/WatsonWP22,icml/Keller022,ccs/ABY318,folkerts2023redsec}) support secure training on convolutional neural networks (CNNs) efficiently, private GCN training with MPC over graphs remains challenging.

Graph convolutional layers in GCNs involve multiplications with a (normalized) adjacency matrix containing $\numedge$ non-zero values in a $\numnode \times \numnode$ matrix for a graph with $\numnode$ nodes and $\numedge$ edges.
The graphs are typically sparse but large.
One could use the standard Beaver-triple-based protocol to securely perform these sparse matrix multiplications by treating graph convolution as ordinary dense matrix multiplication.
However, this approach incurs $O(\numnode^2)$ communication and memory costs due to computations on irrelevant nodes.
%
Integrating existing cryptographic advances, the initial effort of SecGNN~\cite{tsc/WangZJ23,nips/RanXLWQW23} requires heavy communication or computational overhead.
Recently, CoGNN~\cite{ccs/ZouLSLXX24} optimizes the overhead in terms of  horizontal data partitioning, proposing a semi-honest secure framework.
Research for secure GCN over vertical data  remains nascent.

Current MPC studies, for GCN or not, have primarily targeted settings where participants own different data samples, \ie, horizontally partitioned data~\cite{ccs/ZouLSLXX24}.
MPC specialized for scenarios where parties hold different types of features~\cite{tkde/LiuKZPHYOZY24,icml/CastigliaZ0KBP23,nips/Wang0ZLWL23} is rare.
This paper studies $2$-party secure GCN training for these vertical partition cases, where one party holds private graph topology (\eg, edges) while the other owns private node features.
For instance, LinkedIn holds private social relationships between users, while banks own users' private bank statements.
Such real-world graph structures underpin the relevance of our focus.
To our knowledge, no prior work tackles secure GCN training in this context, which is crucial for cross-silo collaboration.


To realize secure GCN over vertically split data, we tailor MPC protocols for sparse graph convolution, which fundamentally involves sparse (adjacency) matrix multiplication.
Recent studies have begun exploring MPC protocols for sparse matrix multiplication (SMM).
ROOM~\cite{ccs/SchoppmannG0P19}, a seminal work on SMM, requires foreknowledge of sparsity types: whether the input matrices are row-sparse or column-sparse.
Unfortunately, GCN typically trains on graphs with arbitrary sparsity, where nodes have varying degrees and no specific sparsity constraints.
Moreover, the adjacency matrix in GCN often contains a self-loop operation represented by adding the identity matrix, which is neither row- nor column-sparse.
Araki~\etal~\cite{ccs/Araki0OPRT21} avoid this limitation in their scalable, secure graph analysis work, yet it does not cover vertical partition.

% and related primitives
To bridge this gap, we propose a secure sparse matrix multiplication protocol, \osmm, achieving \emph{accurate, efficient, and secure GCN training over vertical data} for the first time.

\subsection{New Techniques for Sparse Matrices}
The cost of evaluating a GCN layer is dominated by SMM in the form of $\adjmat\feamat$, where $\adjmat$ is a sparse adjacency matrix of a (directed) graph $\graph$ and $\feamat$ is a dense matrix of node features.
For unrelated nodes, which often constitute a substantial portion, the element-wise products $0\cdot x$ are always zero.
Our efficient MPC design 
avoids unnecessary secure computation over unrelated nodes by focusing on computing non-zero results while concealing the sparse topology.
We achieve this~by:
1) decomposing the sparse matrix $\adjmat$ into a product of matrices (\S\ref{sec::sgc}), including permutation and binary diagonal matrices, that can \emph{faithfully} represent the original graph topology;
2) devising specialized protocols (\S\ref{sec::smm_protocol}) for efficiently multiplying the structured matrices while hiding sparsity topology.


 
\subsubsection{Sparse Matrix Decomposition}
We decompose adjacency matrix $\adjmat$ of $\graph$ into two bipartite graphs: one represented by sparse matrix $\adjout$, linking the out-degree nodes to edges, the other 
by sparse matrix $\adjin$,
linking edges to in-degree nodes.

%\ie, we decompose $\adjmat$ into $\adjout \adjin$, where $\adjout$ and $\adjin$ are sparse matrices representing these connections.
%linking out-degree nodes to edges and edges to in-degree nodes of $\graph$, respectively.

We then permute the columns of $\adjout$ and the rows of $\adjin$ so that the permuted matrices $\adjout'$ and $\adjin'$ have non-zero positions with \emph{monotonically non-decreasing} row and column indices.
A permutation $\sigma$ is used to preserve the edge topology, leading to an initial decomposition of $\adjmat = \adjout'\sigma \adjin'$.
This is further refined into a sequence of \emph{linear transformations}, 
which can be efficiently computed by our MPC protocols for 
\emph{oblivious permutation}
%($\Pi_{\ssp}$) 
and \emph{oblivious selection-multiplication}.
% ($\Pi_\SM$)
\iffalse
Our approach leverages bipartite graph representation and the monotonicity of non-zero positions to decompose a general sparse matrix into linear transformations, enhancing the efficiency of our MPC protocols.
\fi
Our decomposition approach is not limited to GCNs but also general~SMM 
by 
%simply 
treating them 
as adjacency matrices.
%of a graph.
%Since any sparse matrix can be viewed 

%allowing the same technique to be applied.

 
\subsubsection{New Protocols for Linear Transformations}
\emph{Oblivious permutation} (OP) is a two-party protocol taking a private permutation $\sigma$ and a private vector $\xvec$ from the two parties, respectively, and generating a secret share $\l\sigma \xvec\r$ between them.
Our OP protocol employs correlated randomnesses generated in an input-independent offline phase to mask $\sigma$ and $\xvec$ for secure computations on intermediate results, requiring only $1$ round in the online phase (\cf, $\ge 2$ in previous works~\cite{ccs/AsharovHIKNPTT22, ccs/Araki0OPRT21}).

Another crucial two-party protocol in our work is \emph{oblivious selection-multiplication} (OSM).
It takes a private bit~$s$ from a party and secret share $\l x\r$ of an arithmetic number~$x$ owned by the two parties as input and generates secret share $\l sx\r$.
%between them.
%Like our OP protocol, o
Our $1$-round OSM protocol also uses pre-computed randomnesses to mask $s$ and $x$.
%for secure computations.
Compared to the Beaver-triple-based~\cite{crypto/Beaver91a} and oblivious-transfer (OT)-based approaches~\cite{pkc/Tzeng02}, our protocol saves ${\sim}50\%$ of online communication while having the same offline communication and round complexities.

By decomposing the sparse matrix into linear transformations and applying our specialized protocols, our \osmm protocol
%($\prosmm$) 
reduces the complexity of evaluating $\numnode \times \numnode$ sparse matrices with $\numedge$ non-zero values from $O(\numnode^2)$ to $O(\numedge)$.

%(\S\ref{sec::secgcn})
\subsection{\cgnn: Secure GCN made Efficient}
Supported by our new sparsity techniques, we build \cgnn, 
a two-party computation (2PC) framework for GCN inference and training over vertical
%ly split
data.
Our contributions include:

1) We are the first to explore sparsity over vertically split, secret-shared data in MPC, enabling decompositions of sparse matrices with arbitrary sparsity and isolating computations that can be performed in plaintext without sacrificing privacy.

2) We propose two efficient $2$PC primitives for OP and OSM, both optimally single-round.
Combined with our sparse matrix decomposition approach, our \osmm protocol ($\prosmm$) achieves constant-round communication costs of $O(\numedge)$, reducing memory requirements and avoiding out-of-memory errors for large matrices.
In practice, it saves $99\%+$ communication
%(Table~\ref{table:comm_smm}) 
and reduces ${\sim}72\%$ memory usage over large $(5000\times5000)$ matrices compared with using Beaver triples.
%(Table~\ref{table:mem_smm_sparse}) ${\sim}16\%$-

3) We build an end-to-end secure GCN framework for inference and training over vertically split data, maintaining accuracy on par with plaintext computations.
We will open-source our evaluation code for research and deployment.

To evaluate the performance of $\cgnn$, we conducted extensive experiments over three standard graph datasets (Cora~\cite{aim/SenNBGGE08}, Citeseer~\cite{dl/GilesBL98}, and Pubmed~\cite{ijcnlp/DernoncourtL17}),
reporting communication, memory usage, accuracy, and running time under varying network conditions, along with an ablation study with or without \osmm.
Below, we highlight our key achievements.

\textit{Communication (\S\ref{sec::comm_compare_gcn}).}
$\cgnn$ saves communication by $50$-$80\%$.
(\cf,~CoGNN~\cite{ccs/KotiKPG24}, OblivGNN~\cite{uss/XuL0AYY24}).

\textit{Memory usage (\S\ref{sec::smmmemory}).}
\cgnn alleviates out-of-memory problems of using %the standard 
Beaver-triples~\cite{crypto/Beaver91a} for large datasets.

\textit{Accuracy (\S\ref{sec::acc_compare_gcn}).}
$\cgnn$ achieves inference and training accuracy comparable to plaintext counterparts.
%training accuracy $\{76\%$, $65.1\%$, $75.2\%\}$ comparable to $\{75.7\%$, $65.4\%$, $74.5\%\}$ in plaintext.

{\textit{Computational efficiency (\S\ref{sec::time_net}).}} 
%If the network is worse in bandwidth and better in latency, $\cgnn$ shows more benefits.
$\cgnn$ is faster by $6$-$45\%$ in inference and $28$-$95\%$ in training across various networks and excels in narrow-bandwidth and low-latency~ones.

{\textit{Impact of \osmm (\S\ref{sec:ablation}).}}
Our \osmm protocol shows a $10$-$42\times$ speed-up for $5000\times 5000$ matrices and saves $10$-2$1\%$ memory for ``small'' datasets and up to $90\%$+ for larger ones.

\section{Related Work}

\subsection{Personalization and Role-Playing}
Recent works have introduced benchmark datasets for personalizing LLM outputs in tasks like email, abstract, and news writing, focusing on shorter outputs (e.g., 300 tokens for product reviews \citep{kumar2024longlamp} and 850 for news writing \citep{shashidhar-etal-2024-unsupervised}). These methods infer user traits from history for task-specific personalization \citep{sun-etal-2024-revealing, sun-etal-2025-persona, pal2024beyond, li2023teach, salemi2025reasoning}. In contrast, we tackle the more subjective problem of long-form story writing, with author stories averaging 1500 tokens. Unlike prior role-playing approaches that use predefined personas (e.g., Tony Stark, Confucius) \citep{wang-etal-2024-rolellm, sadeq-etal-2024-mitigating, tu2023characterchat, xu2023expertprompting}, we propose a novel method to infer story-writing personas from an author’s history to guide role-playing.


\subsection{Story Understanding and Generation}  
Prior work on persona-aware story generation \citep{yunusov-etal-2024-mirrorstories, bae-kim-2024-collective, zhang-etal-2022-persona, chandu-etal-2019-way} defines personas using discrete attributes like personality traits, demographics, or hobbies. Similarly, \citep{zhu-etal-2023-storytrans} explore story style transfer across pre-defined domains (e.g., fairy tales, martial arts, Shakespearean plays). In contrast, we mimic an individual author's writing style based on their history. Our approach differs by (1) inferring long-form author personas—descriptions of an author’s style from their past works, rather than relying on demographics, and (2) handling long-form story generation, averaging 1500 tokens per output, exceeding typical story lengths in prior research.
\vspace{-5pt}
\section{Method}
\label{sec:method}
\section{Overview}

\revision{In this section, we first explain the foundational concept of Hausdorff distance-based penetration depth algorithms, which are essential for understanding our method (Sec.~\ref{sec:preliminary}).
We then provide a brief overview of our proposed RT-based penetration depth algorithm (Sec.~\ref{subsec:algo_overview}).}



\section{Preliminaries }
\label{sec:Preliminaries}

% Before we introduce our method, we first overview the important basics of 3D dynamic human modeling with Gaussian splatting. Then, we discuss the diffusion-based 3d generation techniques, and how they can be applied to human modeling.
% \ZY{I stopp here. TBC.}
% \subsection{Dynamic human modeling with Gaussian splatting}
\subsection{3D Gaussian Splatting}
3D Gaussian splatting~\cite{kerbl3Dgaussians} is an explicit scene representation that allows high-quality real-time rendering. The given scene is represented by a set of static 3D Gaussians, which are parameterized as follows: Gaussian center $x\in {\mathbb{R}^3}$, color $c\in {\mathbb{R}^3}$, opacity $\alpha\in {\mathbb{R}}$, spatial rotation in the form of quaternion $q\in {\mathbb{R}^4}$, and scaling factor $s\in {\mathbb{R}^3}$. Given these properties, the rendering process is represented as:
\begin{equation}
  I = Splatting(x, c, s, \alpha, q, r),
  \label{eq:splattingGA}
\end{equation}
where $I$ is the rendered image, $r$ is a set of query rays crossing the scene, and $Splatting(\cdot)$ is a differentiable rendering process. We refer readers to Kerbl et al.'s paper~\cite{kerbl3Dgaussians} for the details of Gaussian splatting. 



% \ZY{I would suggest move this part to the method part.}
% GaissianAvatar is a dynamic human generation model based on Gaussian splitting. Given a sequence of RGB images, this method utilizes fitted SMPLs and sampled points on its surface to obtain a pose-dependent feature map by a pose encoder. The pose-dependent features and a geometry feature are fed in a Gaussian decoder, which is employed to establish a functional mapping from the underlying geometry of the human form to diverse attributes of 3D Gaussians on the canonical surfaces. The parameter prediction process is articulated as follows:
% \begin{equation}
%   (\Delta x,c,s)=G_{\theta}(S+P),
%   \label{eq:gaussiandecoder}
% \end{equation}
%  where $G_{\theta}$ represents the Gaussian decoder, and $(S+P)$ is the multiplication of geometry feature S and pose feature P. Instead of optimizing all attributes of Gaussian, this decoder predicts 3D positional offset $\Delta{x} \in {\mathbb{R}^3}$, color $c\in\mathbb{R}^3$, and 3D scaling factor $ s\in\mathbb{R}^3$. To enhance geometry reconstruction accuracy, the opacity $\alpha$ and 3D rotation $q$ are set to fixed values of $1$ and $(1,0,0,0)$ respectively.
 
%  To render the canonical avatar in observation space, we seamlessly combine the Linear Blend Skinning function with the Gaussian Splatting~\cite{kerbl3Dgaussians} rendering process: 
% \begin{equation}
%   I_{\theta}=Splatting(x_o,Q,d),
%   \label{eq:splatting}
% \end{equation}
% \begin{equation}
%   x_o = T_{lbs}(x_c,p,w),
%   \label{eq:LBS}
% \end{equation}
% where $I_{\theta}$ represents the final rendered image, and the canonical Gaussian position $x_c$ is the sum of the initial position $x$ and the predicted offset $\Delta x$. The LBS function $T_{lbs}$ applies the SMPL skeleton pose $p$ and blending weights $w$ to deform $x_c$ into observation space as $x_o$. $Q$ denotes the remaining attributes of the Gaussians. With the rendering process, they can now reposition these canonical 3D Gaussians into the observation space.



\subsection{Score Distillation Sampling}
Score Distillation Sampling (SDS)~\cite{poole2022dreamfusion} builds a bridge between diffusion models and 3D representations. In SDS, the noised input is denoised in one time-step, and the difference between added noise and predicted noise is considered SDS loss, expressed as:

% \begin{equation}
%   \mathcal{L}_{SDS}(I_{\Phi}) \triangleq E_{t,\epsilon}[w(t)(\epsilon_{\phi}(z_t,y,t)-\epsilon)\frac{\partial I_{\Phi}}{\partial\Phi}],
%   \label{eq:SDSObserv}
% \end{equation}
\begin{equation}
    \mathcal{L}_{\text{SDS}}(I_{\Phi}) \triangleq \mathbb{E}_{t,\epsilon} \left[ w(t) \left( \epsilon_{\phi}(z_t, y, t) - \epsilon \right) \frac{\partial I_{\Phi}}{\partial \Phi} \right],
  \label{eq:SDSObservGA}
\end{equation}
where the input $I_{\Phi}$ represents a rendered image from a 3D representation, such as 3D Gaussians, with optimizable parameters $\Phi$. $\epsilon_{\phi}$ corresponds to the predicted noise of diffusion networks, which is produced by incorporating the noise image $z_t$ as input and conditioning it with a text or image $y$ at timestep $t$. The noise image $z_t$ is derived by introducing noise $\epsilon$ into $I_{\Phi}$ at timestep $t$. The loss is weighted by the diffusion scheduler $w(t)$. 
% \vspace{-3mm}

\subsection{Overview of the RTPD Algorithm}\label{subsec:algo_overview}
Fig.~\ref{fig:Overview} presents an overview of our RTPD algorithm.
It is grounded in the Hausdorff distance-based penetration depth calculation method (Sec.~\ref{sec:preliminary}).
%, similar to that of Tang et al.~\shortcite{SIG09HIST}.
The process consists of two primary phases: penetration surface extraction and Hausdorff distance calculation.
We leverage the RTX platform's capabilities to accelerate both of these steps.

\begin{figure*}[t]
    \centering
    \includegraphics[width=0.8\textwidth]{Image/overview.pdf}
    \caption{The overview of RT-based penetration depth calculation algorithm overview}
    \label{fig:Overview}
\end{figure*}

The penetration surface extraction phase focuses on identifying the overlapped region between two objects.
\revision{The penetration surface is defined as a set of polygons from one object, where at least one of its vertices lies within the other object. 
Note that in our work, we focus on triangles rather than general polygons, as they are processed most efficiently on the RTX platform.}
To facilitate this extraction, we introduce a ray-tracing-based \revision{Point-in-Polyhedron} test (RT-PIP), significantly accelerated through the use of RT cores (Sec.~\ref{sec:RT-PIP}).
This test capitalizes on the ray-surface intersection capabilities of the RTX platform.
%
Initially, a Geometry Acceleration Structure (GAS) is generated for each object, as required by the RTX platform.
The RT-PIP module takes the GAS of one object (e.g., $GAS_{A}$) and the point set of the other object (e.g., $P_{B}$).
It outputs a set of points (e.g., $P_{\partial B}$) representing the penetration region, indicating their location inside the opposing object.
Subsequently, a penetration surface (e.g., $\partial B$) is constructed using this point set (e.g., $P_{\partial B}$) (Sec.~\ref{subsec:surfaceGen}).
%
The generated penetration surfaces (e.g., $\partial A$ and $\partial B$) are then forwarded to the next step. 

The Hausdorff distance calculation phase utilizes the ray-surface intersection test of the RTX platform (Sec.~\ref{sec:RT-Hausdorff}) to compute the Hausdorff distance between two objects.
We introduce a novel Ray-Tracing-based Hausdorff DISTance algorithm, RT-HDIST.
It begins by generating GAS for the two penetration surfaces, $P_{\partial A}$ and $P_{\partial B}$, derived from the preceding step.
RT-HDIST processes the GAS of a penetration surface (e.g., $GAS_{\partial A}$) alongside the point set of the other penetration surface (e.g., $P_{\partial B}$) to compute the penetration depth between them.
The algorithm operates bidirectionally, considering both directions ($\partial A \to \partial B$ and $\partial B \to \partial A$).
The final penetration depth between the two objects, A and B, is determined by selecting the larger value from these two directional computations.

%In the Hausdorff distance calculation step, we compute the Hausdorff distance between given two objects using a ray-surface-intersection test. (Sec.~\ref{sec:RT-Hausdorff}) Initially, we construct the GAS for both $\partial A$ and $\partial B$ to utilize the RT-core effectively. The RT-based Hausdorff distance algorithms then determine the Hausdorff distance by processing the GAS of one object (e.g. $GAS_{\partial A}$) and set of the vertices of the other (e.g. $P_{\partial B}$). Following the Hausdorff distance definition (Eq.~\ref{equation:hausdorff_definition}), we compute the Hausdorff distance to both directions ($\partial A \to \partial B$) and ($\partial B \to \partial A$). As a result, the bigger one is the final Hausdorff distance, and also it is the penetration depth between input object $A$ and $B$.


%the proposed RT-based penetration depth calculation pipeline.
%Our proposed methods adopt Tang's Hausdorff-based penetration depth methods~\cite{SIG09HIST}. The pipeline is divided into the penetration surface extraction step and the Hausdorff distance calculation between the penetration surface steps. However, since Tang's approach is not suitable for the RT platform in detail, we modified and applied it with appropriate methods.

%The penetration surface extraction step is extracting overlapped surfaces on other objects. To utilize the RT core, we use the ray-intersection-based PIP(Point-In-Polygon) algorithms instead of collision detection between two objects which Tang et al.~\cite{SIG09HIST} used. (Sec.~\ref{sec:RT-PIP})
%RT core-based PIP test uses a ray-surface intersection test. For purpose this, we generate the GAS(Geometry Acceleration Structure) for each object. RT core-based PIP test takes the GAS of one object (e.g. $GAS_{A}$) and a set of vertex of another one (e.g. $P_{B}$). Then this computes the penetrated vertex set of another one (e.g. $P_{\partial B}$). To calculate the Hausdorff distance, these vertex sets change to objects constructed by penetrated surface (e.g. $\partial B$). Finally, the two generated overlapped surface objects $\partial A$ and $\partial B$ are used in the Hausdorff distance calculation step.

Our goal is to increase the robustness of T2I models, particularly with rare or unseen concepts, which they struggle to generate. To do so, we investigate a retrieval-augmented generation approach, through which we dynamically select images that can provide the model with missing visual cues. Importantly, we focus on models that were not trained for RAG, and show that existing image conditioning tools can be leveraged to support RAG post-hoc.
As depicted in \cref{fig:overview}, given a text prompt and a T2I generative model, we start by generating an image with the given prompt. Then, we query a VLM with the image, and ask it to decide if the image matches the prompt. If it does not, we aim to retrieve images representing the concepts that are missing from the image, and provide them as additional context to the model to guide it toward better alignment with the prompt.
In the following sections, we describe our method by answering key questions:
(1) How do we know which images to retrieve? 
(2) How can we retrieve the required images? 
and (3) How can we use the retrieved images for unknown concept generation?
By answering these questions, we achieve our goal of generating new concepts that the model struggles to generate on its own.

\vspace{-3pt}
\subsection{Which images to retrieve?}
The amount of images we can pass to a model is limited, hence we need to decide which images to pass as references to guide the generation of a base model. As T2I models are already capable of generating many concepts successfully, an efficient strategy would be passing only concepts they struggle to generate as references, and not all the concepts in a prompt.
To find the challenging concepts,
we utilize a VLM and apply a step-by-step method, as depicted in the bottom part of \cref{fig:overview}. First, we generate an initial image with a T2I model. Then, we provide the VLM with the initial prompt and image, and ask it if they match. If not, we ask the VLM to identify missing concepts and
focus on content and style, since these are easy to convey through visual cues.
As demonstrated in \cref{tab:ablations}, empirical experiments show that image retrieval from detailed image captions yields better results than retrieval from brief, generic concept descriptions.
Therefore, after identifying the missing concepts, we ask the VLM to suggest detailed image captions for images that describe each of the concepts. 

\vspace{-4pt}
\subsubsection{Error Handling}
\label{subsec:err_hand}

The VLM may sometimes fail to identify the missing concepts in an image, and will respond that it is ``unable to respond''. In these rare cases, we allow up to 3 query repetitions, while increasing the query temperature in each repetition. Increasing the temperature allows for more diverse responses by encouraging the model to sample less probable words.
In most cases, using our suggested step-by-step method yields better results than retrieving images directly from the given prompt (see 
\cref{subsec:ablations}).
However, if the VLM still fails to identify the missing concepts after multiple attempts, we fall back to retrieving images directly from the prompt, as it usually means the VLM does not know what is the meaning of the prompt.

The used prompts can be found in \cref{app:prompts}.
Next, we turn to retrieve images based on the acquired image captions.

\vspace{-3pt}
\subsection{How to retrieve the required images?}

Given $n$ image captions, our goal is to retrieve the images that are most similar to these captions from a dataset. 
To retrieve images matching a given image caption, we compare the caption to all the images in the dataset using a text-image similarity metric and retrieve the top $k$ most similar images.
Text-to-image retrieval is an active research field~\cite{radford2021learning, zhai2023sigmoid, ray2024cola, vendrowinquire}, where no single method is perfect.
Retrieval is especially hard when the dataset does not contain an exact match to the query \cite{biswas2024efficient} or when the task is fine-grained retrieval, that depends on subtle details~\cite{wei2022fine}.
Hence, a common retrieval workflow is to first retrieve image candidates using pre-computed embeddings, and then re-rank the retrieved candidates using a different, often more expensive but accurate, method \cite{vendrowinquire}.
Following this workflow, we experimented with cosine similarity over different embeddings, and with multiple re-ranking methods of reference candidates.
Although re-ranking sometimes yields better results compared to simply using cosine similarity between CLIP~\cite{radford2021learning} embeddings, the difference was not significant in most of our experiments. Therefore, for simplicity, we use cosine similarity between CLIP embeddings as our similarity metric (see \cref{tab:sim_metrics}, \cref{subsec:ablations} for more details about our experiments with different similarity metrics).

\vspace{-3pt}
\subsection{How to use the retrieved images?}
Putting it all together, after retrieving relevant images, all that is left to do is to use them as context so they are beneficial for the model.
We experimented with two types of models; models that are trained to receive images as input in addition to text and have ICL capabilities (e.g., OmniGen~\cite{xiao2024omnigen}), and T2I models augmented with an image encoder in post-training (e.g., SDXL~\cite{podellsdxl} with IP-adapter~\cite{ye2023ip}).
As the first model type has ICL capabilities, we can supply the retrieved images as examples that it can learn from, by adjusting the original prompt.
Although the second model type lacks true ICL capabilities, it offers image-based control functionalities, which we can leverage for applying RAG over it with our method.
Hence, for both model types, we augment the input prompt to contain a reference of the retrieved images as examples.
Formally, given a prompt $p$, $n$ concepts, and $k$ compatible images for each concept, we use the following template to create a new prompt:
``According to these examples of 
$\mathord{<}c_1\mathord{>:<}img_{1,1}\mathord{>}, ... , \mathord{<}img_{1,k}\mathord{>}, ... , \mathord{<}c_n\mathord{>:<}img_{n,1}\mathord{>}, ... , $
$\mathord{<}img_{n,k}\mathord{>}$,
generate $\mathord{<}p\mathord{>}$'', 
where $c_i$ for $i\in{[1,n]}$ is a compatible image caption of the image $\mathord{<}img_{i,j}\mathord{>},  j\in{[1,k]}$. 

This prompt allows models to learn missing concepts from the images, guiding them to generate the required result. 

\textbf{Personalized Generation}: 
For models that support multiple input images, we can apply our method for personalized generation as well, to generate rare concept combinations with personal concepts. In this case, we use one image for personal content, and 1+ other reference images for missing concepts. For example, given an image of a specific cat, we can generate diverse images of it, ranging from a mug featuring the cat to a lego of it or atypical situations like the cat writing code or teaching a classroom of dogs (\cref{fig:personalization}).
\vspace{-2pt}
\begin{figure}[htp]
  \centering
   \includegraphics[width=\linewidth]{Assets/personalization.pdf}
   \caption{\textbf{Personalized generation example.}
   \emph{ImageRAG} can work in parallel with personalization methods and enhance their capabilities. For example, although OmniGen can generate images of a subject based on an image, it struggles to generate some concepts. Using references retrieved by our method, it can generate the required result.
}
   \label{fig:personalization}\vspace{-10pt}
\end{figure}
\section{Simulations and Experiment}
In this section, we conduct comprehensive experiments in both simulation and the real-world robot to address the following questions:
\begin{itemize}[leftmargin=*]
    \item \textbf{Q1(Sim)}: How does the \our policy perform in tracking across different commands?
    \item  \textbf{Q2(Sim)}: How to reasonably combine various commands in the general command space? % Command Analysis
    \item \textbf{Q3(Sim)}: How does large-scale noise intervention training help in policy robustness? % Ablation Study
    \item \textbf{Q4(Real)}: How does \our behave in the real world? % Real World Demo
\end{itemize}

\noindent\textbf{Robot and Simulator.} 
Our main experiments in this paper are conducted on the Unitree H1 robot, which has 19 Degrees of Freedom (DOF) in total, including 
two 3-DOF shoulder joints, two elbow joints, one waist joint, two 3-DOF hip joints, two knee joints, and two ankle joints.
The simulation training is based on the NVIDIA IsaacGym simulator~\citep{makoviychuk2021isaac}. It takes 16 hours on a single RTX 4090 GPU to train one policy.

\noindent\textbf{Command analysis principle and metric.}
One of the main contributions of this paper is an extended and general command space for humanoid robots. Therefore, we pay much attention to command analysis (regarding Q1 and Q2). This includes analysis of single command tracking errors, along with the combination of different commands under different gaits.
% we categorize the commands into three groups: \emph{movement}, \emph{foot}, and \emph{posture}. The \emph{movement} commands include the linear velocity and angular velocity, forming the foundational locomotion commands and are considered the most critical aspect of the tasks. The \emph{foot} commands include the gait frequency and foot swing height, representing the mode of leg movement. The \emph{posture} commands include body height, body pitch and waist yaw, which determine the desired body posture.
For analysis, we evaluate the averaged episodic command tracking error (denoted as $E_\text{cmd}$), which measures the discrepancy between the actual robot states and the command space using $L_1$ norm.
% The tracking error is measured in units of $m/s$, $rad/s$, $Hz$, $m$, and $rad$, corresponding to linear velocity, angular velocity, frequency, position, and rotation, respectively.
All commands are uniformly sampled within a pre-defined command range, as shown in \tb{tab:commands}\footnote{Note that the hopping gait keeps a different command range, due to its asymmetric type of motion. More details can be referred to \ap{ap:Hopping}.}.

%%%%%%%%%%%---SETME-----%%%%%%%%%%%%%
%replace @@ with the submission number submission site.
\newcommand{\thiswork}{INF$^2$\xspace}
%%%%%%%%%%%%%%%%%%%%%%%%%%%%%%%%%%%%


%\newcommand{\rev}[1]{{\color{olivegreen}#1}}
\newcommand{\rev}[1]{{#1}}


\newcommand{\JL}[1]{{\color{cyan}[\textbf{\sc JLee}: \textit{#1}]}}
\newcommand{\JW}[1]{{\color{orange}[\textbf{\sc JJung}: \textit{#1}]}}
\newcommand{\JY}[1]{{\color{blue(ncs)}[\textbf{\sc JSong}: \textit{#1}]}}
\newcommand{\HS}[1]{{\color{magenta}[\textbf{\sc HJang}: \textit{#1}]}}
\newcommand{\CS}[1]{{\color{navy}[\textbf{\sc CShin}: \textit{#1}]}}
\newcommand{\SN}[1]{{\color{olive}[\textbf{\sc SNoh}: \textit{#1}]}}

%\def\final{}   % uncomment this for the submission version
\ifdefined\final
\renewcommand{\JL}[1]{}
\renewcommand{\JW}[1]{}
\renewcommand{\JY}[1]{}
\renewcommand{\HS}[1]{}
\renewcommand{\CS}[1]{}
\renewcommand{\SN}[1]{}
\fi

%%% Notion for baseline approaches %%% 
\newcommand{\baseline}{offloading-based batched inference\xspace}
\newcommand{\Baseline}{Offloading-based batched inference\xspace}


\newcommand{\ans}{attention-near storage\xspace}
\newcommand{\Ans}{Attention-near storage\xspace}
\newcommand{\ANS}{Attention-Near Storage\xspace}

\newcommand{\wb}{delayed KV cache writeback\xspace}
\newcommand{\Wb}{Delayed KV cache writeback\xspace}
\newcommand{\WB}{Delayed KV Cache Writeback\xspace}

\newcommand{\xcache}{X-cache\xspace}
\newcommand{\XCACHE}{X-Cache\xspace}


%%% Notions for our methods %%%
\newcommand{\schemea}{\textbf{Expanding supported maximum sequence length with optimized performance}\xspace}
\newcommand{\Schemea}{\textbf{Expanding supported maximum sequence length with optimized performance}\xspace}

\newcommand{\schemeb}{\textbf{Optimizing the storage device performance}\xspace}
\newcommand{\Schemeb}{\textbf{Optimizing the storage device performance}\xspace}

\newcommand{\schemec}{\textbf{Orthogonally supporting Compression Techniques}\xspace}
\newcommand{\Schemec}{\textbf{Orthogonally supporting Compression Techniques}\xspace}



% Circular numbers
\usepackage{tikz}
\newcommand*\circled[1]{\tikz[baseline=(char.base)]{
            \node[shape=circle,draw,inner sep=0.4pt] (char) {#1};}}

\newcommand*\bcircled[1]{\tikz[baseline=(char.base)]{
            \node[shape=circle,draw,inner sep=0.4pt, fill=black, text=white] (char) {#1};}}

\subsection{Single Command Tracking}
We first analyze each command separately while keeping all other commands held at their default values. The results are shown in \tb{tab:Single commands}.
It is easily observed that the tracking errors in the walking and standing gaits are significantly lower than those in the jumping and hopping, with hopping exhibiting the largest tracking errors.
For hopping gaits, the robot may fall during the tracking of specific commands, like high-speed tracking, body pitch, and waist-yaw control.
This can be attributed to the fact that hopping requires rather high stability. Moreover, the complex postures and motions further exacerbate the risk of instability. Consequently, the policy prioritizes learning to maintain the balance, which, to some extent, compromises the accuracy of command tracking.

We conclude that the tracking accuracy of each gait aligns with the training difficulty of that gait in simulation. For example, the walking and standing patterns can be learned first during training, while the jumping and hopping gaits appear later and require an extended training period for the robot to acquire proficiency.
Similarly, the tracking accuracy of robots under low velocity is significantly better than those under high velocity, since 1) the locomotion skills under low velocity are much easier to master, and 2) the dynamic stability of the robot decreases at high speeds, leading to a trade-off with tracking accuracy.

We also found that the tracking accuracy for longitudinal velocity commands $v_x$ surpasses that of horizontal velocity commands $v_y$, which is due to the limitation of the hardware configuration of the selected Unitree H1 robots. In addition, the {foot swing height} $l$ is the least accurately tracked.
Furthermore, the tracking reward related to foot placement outperforms the tracking performance associated with posture control, since adjusting posture introduces greater challenges to stability. In response, the policy adopts more conservative actions to mitigate balance-threatening postural changes.
% In contrast, the influence of foot placement on stability is comparatively less pronounced, allowing for more precise tracking.

\begin{table}[t]
\setlength{\abovecaptionskip}{0.cm}
\setlength{\belowcaptionskip}{-0.cm}
\centering
\caption{\small \textbf{Single command tracking error.} The tracking errors for foot commands are calculated over a complete gait cycle, and the remaining ones are over one environmental step. For standing gait, we only tested the body height, body pitch, and waist yaw tracking error. $E^\text{high}$ and $E^\text{low}$ represents high-speed ($v_x > 1m/s$) and low-speed ($v_x \le 1m/s$) modes categorized by the linear velocity $v$. 
The tracking error is computed by sampling each command in a predefined range (\tb{tab:commands}) while keeping all other commands held at their default values.}
\label{tab:Single commands}
\resizebox{\columnwidth}{!}{
\begin{tabular}{@{}c|cccc|cc|ccc@{}}
\toprule
\multirow{3}{*}{Gait} & \multicolumn{4}{c|}{Movement} & \multicolumn{2}{c|}{Foot} & \multicolumn{3}{c}{Posture} \\
\cmidrule(l){2-5} \cmidrule{6-7} \cmidrule{8-10} 
& \multirow{2}{*}{\makecell{$E_{v_x}^\text{low}$\\($m/s$)}} & \multirow{2}{*}{\makecell{$E_{v_x}^\text{high}$\\($m/s$)}} & \multirow{2}{*}{\makecell{$E_{v_y}$\\($m/s$)}} & \multirow{2}{*}{\makecell{$E_{\omega}$\\$rad/s$}} & \multirow{2}{*}{\makecell{$E_{f}$\\($HZ$)}} & \multirow{2}{*}{\makecell{$E_{l}$\\($m$)}} & \multirow{2}{*}{\makecell{$E_{h}$\\($m$)}}  & \multirow{2}{*}{\makecell{$E_{p}$\\($rad$)}} & \multirow{2}{*}{\makecell{$E_{w}$\\($rad$)}}   \\ 
&  &  &  &  &  &  &  &  &    \\ 
\midrule
Standing  & - & - & - & - & - & - & 0.035 & 0.047 & 0.022  \\
Walking   & 0.030 & 0.216 & 0.085 & 0.054 & 0.028 & 0.011 & 0.064 & 0.038 & 0.075  \\
Jumping  & 0.090 & 0.532 & 0.069 & 0.077 & 0.027 & 0.012 & 0.058 & 0.048 & 0.022 \\
Hopping   & 0.033 & - & 0.046 & 0.078 & - & - & 0.103 & - & - \\
\bottomrule
\end{tabular}}
\end{table}



\begin{table*}[t]
\setlength{\abovecaptionskip}{0.cm}
\setlength{\belowcaptionskip}{-0.cm}
\centering
\caption{\small \textbf{Tracking errors with different intervention strategies under the walking gait}. We evaluate three upper-body intervention training strategies: Noise (\our), the AMASS dataset, and no intervention at all. The tracking errors across various task and behavior commands reflect the intervention tolerance, \textit{i.e.}, the ability of precise locomotion control under external intervention.}
\label{tab:Intervetion Tracking Error}
\begin{tabular}{c|c|ccc|cc|ccc}
\toprule
\multirow{3}{*}{Training Strategy} & \multirow{3}{*}{Intervention Task} & \multicolumn{3}{c|}{Task Commands}                        & \multicolumn{5}{c}{Behavior Commands}\\ \cmidrule{3-10}
 & & \multicolumn{3}{c|}{Movement}                        & \multicolumn{2}{c|}{Foot}          & \multicolumn{3}{c}{Posture}                         \\ \cmidrule{3-10}
                                      &                                      &$E_{v_x}$ ($m/s$)     & $E_{v_y}$ ($m/s$)   & $E_{\omega}$ ($rad/s$)    & $E_{f}$ ($Hz$)         & $E_{l}$ ($m$)         & $E_{h}$ ($m$)        & $E_{p}$ ($rad$)     & $E_{w}$ ($rad$)         \\ \midrule
\multirow{3}{*}{\makecell{Noise Curriculum\\(\our)}}        & Noise                        & \textbf{0.0483} & \textbf{0.0962} & \textbf{0.1879} & \textbf{0.0471} & \textbf{0.0542} & \textbf{0.0402} & \textbf{0.0432} & \textbf{0.0552} \\
                                      & AMASS                                & \textbf{0.0391} & \textbf{0.0920} & \textbf{0.1039} & \textbf{0.0464} & \textbf{0.0543} & \textbf{0.0387} & \textbf{0.0364} & \textbf{0.0540} \\
                                      & None                                 & \textbf{0.0264} & \textbf{0.0863} & \textbf{0.0543} & \textbf{0.0447} & \textbf{0.0522} & 0.0372          & 0.0375          & 0.0475          \\ \cmidrule{1-10}
\multirow{3}{*}{AMASS}                & Noise                        & 0.1697          & 0.1055          & 0.2156          & 0.0621          & 0.0542          & 0.0620          & 0.0812          & 0.0694          \\
                                      & AMASS                                & 0.0567          & 0.0965          & 0.1593          & 0.0466          & 0.0555          & 0.0579          & 0.0458          & 0.0554          \\
                                      & None                                 & 0.0645          & 0.0916          & 0.0802          & 0.0460          & 0.0531          & 0.0577          & 0.0455          & 0.0568          \\ \cmidrule{1-10}
\multirow{3}{*}{No Intervention}                 & Noise                        & 0.8658          & 0.7511          & 0.9116          & 0.1930          & 0.1913          & 0.1658          & 0.3622          & 0.2241          \\
                                      & AMASS                                & 0.6299          & 0.4026          & 0.5758          & 0.2245          & 0.2527          & 0.1305          & 0.2367          & 0.1112          \\
                                      & None                                 & 0.0755          & 0.1076          & 0.1151          & 0.0450          & 0.0678          & \textbf{0.0255} & \textbf{0.0211} & \textbf{0.0380} \\ \bottomrule
\end{tabular}
\end{table*}



\begin{table}[t]
\setlength{\abovecaptionskip}{0.cm}
\setlength{\belowcaptionskip}{-0.cm}
\centering
\caption{ \small
\textbf{Averaged foot displacement under intervention}. We compare foot displacement $D_\text{cmd}$ of different training strategies under various intervention tasks, which computes the total movement of both feet in one episode with sampled posture behavior commands.
}
\label{tab:Intervention Mean Foot Movement}
\resizebox{\linewidth}{!}{
\begin{tabular}{ccccc}
\toprule
Training Strategy                 & Intervention Task     & $D_{h}$ ($m/s$)                  & $D_{p}$ ($m/s$)      & $D_{w}$ ($m/s$)       \\ \midrule
\multirow{3}{*}{\makecell{Noise Curriculum\\(\our)}}  & Noise & \textbf{0.0339}             & \textbf{0.0892} & \textbf{0.0199} \\
                       & AMASS         & \textbf{0.0454}             & \textbf{0.0728} & \textbf{0.0196} \\
                       & None          & \textbf{0.0003}             & \textbf{0.0016} & \textbf{0.0007} \\ \midrule
\multirow{3}{*}{AMASS only} & Noise         & 2.0815                      & 2.8978          & 3.2630          \\
                       & AMASS         & 0.0536                      & 0.1743          & 0.0396          \\
                       & None          & 0.0139                      & 0.0160          & 0.0013          \\ \midrule
\multirow{3}{*}{No Intervention}  & Noise         & 17.5358                     & 17.9732         & 25.7132         \\
                       & AMASS         & 25.3802 & 26.3496         & 21.3078         \\
                       & None          & 0.0159  & 1.7065          & 1.7152          \\ \bottomrule
\end{tabular}}
\end{table}

\subsection{Command Combination Analysis}
To provide an in-depth analysis of the command space and to 
reveal the underlying interaction of various commands under different gaits.
Here, we aim to analyze the \emph{orthogonality} of commands based on the interference or conflict between the tracking errors of these commands across their reasonable ranges. For instance, when we say that a set of commands are \emph{orthogonal}, each command does not significantly affect the tracking performance of each other in its range. To this end, we plot the tracking error $E_\text{cmd}$ as heat maps, generated by systematically scanning the command values for each pair of parameters, revealing the correlation of each command.
We leave the full heat maps at \ap{ap:heatmaps}, and conclude our main observation for all gaits.

\noindent\textbf{Walking.} Walking is the most basic gait, which preserves the best performance of the robot hardware.
\begin{itemize}[leftmargin=*]
    \item The {linear velocity} $v_x$, the {angular velocity yaw} $\omega$, the {body height} $h$, and the {waist yaw} $w$ are orthogonal during walking.
    \item When the {linear velocity} $v_x$ exceeds $1.5m/s$, the orthogonality between $v_x$ and other commands decreases due to reduced dynamic stability and the robot's need to maintain body stability over tracking accuracy.
    \item The {gait frequency} $f$ shows discrete orthogonality, with optimal tracking performance at frequencies of 1.5 or 2. High-frequency gait conditions reduce tracking accuracy.
    \item The {linear velocity} $v_y$, the {foot swing height} $l$, and the {body pitch} $p$ are orthogonal to other commands only within a narrow range.
\end{itemize}

\noindent\textbf{Jumping.} The command orthogonality in jumping is similar to walking, but the overall orthogonal range is smaller, due to the increased challenge of the jumping gait, especially in high-speed movement modes.
During each gait cycle, the robot must leap forward significantly to maintain its speed. To execute this complex jumping action continuously, the robot must adopt an optimal posture at the beginning of each cycle. Both legs exert substantial torque to propel the body forward. Upon landing, the robot must quickly readjust its posture to maintain stability and repeat the actions. Consequently, during movement, the robot can only execute other commands within a relatively narrow range.

\noindent\textbf{Hopping.}
The hopping gait introduces more instability, and the robot's control system must focus more on maintaining balance, making it difficult to simultaneously handle complex, multi-dimensional commands.
\begin{itemize}[leftmargin=*]
    \item Hopping gait commands lack clear orthogonal relationships.
    \item Effective tracking is limited to the x-axis {linear velocity} $v_x$, the y-axis {linear velocity} $v_y$, the {angular velocity yaw} $\omega$, and the {body height} $h$.
    \item Adjustments to $h$ can be understood that a lower body height improves dynamic stability, therefore, it plays a positive role in maintaining the target body posture.
    % enhancing the robot's hopping performance.
\end{itemize}

\noindent\textbf{Standing.} As for the standing gait, we tested the tracking errors of commands related to posture. The results showed that the tracking errors were similar to those observed during walking with zero velocity.

\begin{itemize}[leftmargin=*]
    \item The {waist yaw} $w$ command is almost orthogonal to the other two commands.
    \item As the range of commands increases, orthogonality between the {body height} $h$ and the {body pitch} $p$ decreases. This is because the H1 robot has only one degree of freedom at the waist, limiting posture adjustments to the hip pitch joint.
    \item A 0.3 m decrease of the body height relative to the default height reduces the range of motion of the hip pitch joint to almost zero, hindering precise tracking of body pitch.
\end{itemize}

Furthermore, we conclude that {gait frequency} $f$ highly affects the tracking accuracy of \emph{movement} commands when it is excessively high and low; the \emph{posture} commands can significantly impact the tracking errors of other commands, especially when they are near the range limits.
% We categorize the commands into three groups: \emph{movement}, \emph{foot}, and \emph{posture}. 1) The \emph{movement} commands include the linear velocity $v_x, v_y$ and angular velocity $\omega$, forming the foundational locomotion commands, and are considered the most critical aspect of the tasks. 2) The \emph{foot} commands include the {foot swing height} $l$, which is the least accurately tracked; and the {gait frequency} $f$, which can affect the tracking accuracy of \emph{movement} commands when it is excessively high and low. 3) The \emph{posture} commands, which include body height $h$, the body pitch $p$, and waist yaw $w$, determine the desired body posture, and can significantly impact the tracking errors of other commands, especially when the command is challenging. 
For different gaits, the orthogonality range between commands is greatest in the walking gait and smallest in the hopping gait.

\subsection{Ablation on Intervention Training Strategy}
\label{sec:InterventionExp}
% The three policies use the same random seeds and training time.
To validate the effectiveness of the intervention training strategy on the policy robustness when external upper-body intervention is involved, we compare the policies trained with different strategies, including noise curriculum (\our), filtered AMASS data~\citep{he2024omnih2o}, and no intervention. We test the tracking errors under two different intervention tasks, \textit{i.e.}, uniform noise, AAMAS dataset, along with a no-intervention setup. The results under the walking gait are shown in \tb{tab:Intervetion Tracking Error}, and we leave other gaits in \ap{ap:SingleCommandsTracking-REMAIN}. 
It is obvious that the noise curriculum strategy of \our achieved the best performance under almost all test cases, except the posture-related tracking with no intervention. 
In particular, \our showed less of a decrease in tracking accuracy with various interventions, indicating our noise curriculum intervention strategy enables the control policy to handle a large range of arm movements, making it very useful and supportive for loco-manipulation tasks.
In comparison, the policy trained with AMASS data shows a significant decrease in the tracking accuracy when intervening with uniform noise, due to the limited motion in the training data. The policy trained without any intervention only performs well without external upper-body control.

It is worth noting that when intervention training is involved, the tracking error related to the movement and foot is also better than those of the policy trained without intervention, and \our provides the most accurate tracking. This shows that intervention training also contributes to the robustness of the policy. During our real robot experiments, we further observed that the robot behaves with a harder force when in contact with the floor, indicating a possible trade-off between motion regularization and tracking accuracy when involving intervention.

\noindent\textbf{Stability under standing gait.}
Adjusting posture in the standing state introduces additional requirements for stability, since the robot pacing to maintain balance may increase the difficulty of achieving manipulation tasks that require stand still. To investigate the necessity of noise curriculum for manipulation, we further measured the averaged foot displacement (in meters) under the standing gait, which computes the total movement of both feet in one episode (20 seconds) while tracking the posture behavior commands. Results in \tb{tab:Intervention Mean Foot Movement} show that \our exhibits minimal foot displacement. On the contrary, the strategy trained on AMASS data requires frequent small steps to adjust the posture and maintain stability for noise interventions. 
Without intervention training, the policy tends to tip over when involving intervention, leading to failure of the entire task.

%  鲁棒性测试的结果分析
\begin{figure}[t]
    \centering
    \includegraphics[width=\linewidth]{imgs/radar_chart_V2.pdf}
    \vspace{-13pt}
    \caption{\small \textbf{External disturbance tolerance}. Left: A constant and continuous force is applied to the robot. Right: A one-second force is exerted on the robot. The experiment is conducted under a standing gait with default commands. If the robot's survival ratio exceeds $98\%$, it is deemed capable of tolerating such external disturbance. 
    The survival ratio computes the trajectory ratio of non-termination (ends of timeout) during 4096 rollouts.}
    \label{fig:Robust}
    \vspace{-12pt}
\end{figure}
\noindent\textbf{Robustness for external disturbance.}
Finally, we test the contribution of intervention training and noise curriculum to the robustness of external disturbance. In particular, we evaluated the robot's maximum tolerance to external disturbance forces in eight directions and compared the policy trained without intervention. Results illustrated in \fig{fig:Robust} demonstrate that \our preserves greater tolerance for external disturbances in both pushing and loading scenarios across most of the directions. The reason behind this is that the intervention brings the robot exposed to various disturbances originating from its upper body, and thereby enhances the overall stability by dynamically adjusting leg strength.

% \our has a significantly higher tolerance for external disturbance forces in almost all directions compared to the strategy without intervention training.
% This is attributed to the fact that, during large-scale noise intervention training, the robot effectively explored a wide range of extreme scenarios and learned to enhance body stability by adjusting leg movements.

\subsection{Real-World Experiments}
We deploy \our on a real-world robot to verify its effectiveness. In \fig{fig:teaser}, we illustrate the humanoid capabilities supported by \our, showing the versatile behavior of the Unitree H1 robot. In particular, we demonstrate the intriguing potential of the comprehensive task range that \our is able to achieve, with a flexible combination of commands in high dynamics. To qualitatively analyze the performance of \our, we estimate the tracking error of two pose parameters (body pitch $p$ and waist rotation $w$ from the motor readings) on real robots, since other commands are hard to measure without a highly accurate motion capture system. The results are shown in \tb{tb:track-real}, where $E^{\text{real}}_{\text{cmd}}$ illustrates the tracking error of the posture command.
We observe that the tracking error in real-world experiments is slightly higher than in simulation environments, primarily due to sensor noise and the wear of the robot's hardware. Among different gaits, the tracking error for the waist rotation $w$ is smaller compared to that for the body pitch $p$, as waist control has less impact on the robot’s overall stability. In both error tests, the jumping gait exhibited the smallest $E_{cmd}$, while the walking gait showed slightly higher errors, consistent with the findings observed in the simulation environment.

\begin{table}[t]
\centering
\caption{\small \textbf{Tracking error in real world.} We conducted five tests to measure the tracking error for each command under three gaits. The tracking error for each command was calculated during each control step. The tested commands gradually increased from the minimum to the maximum values within a predefined range, while the remaining commands were kept at their default values.} % To account for the impact of communication delays on the actual tracking error, we introduced a 0.1-second delay in the command execution.
\label{tb:track-real}
\begin{tabular}{c|cc} \toprule
Gait     & $E_p^{\text{real}}$ & $E_w^{\text{real}}$ \\ \midrule
Standing & 0.0712 $\pm$ 0.0425 & 0.0718 $\pm$ 0.0614 \\
Walking  & 0.1006 $\pm$ 0.0581  & 0.0571 $\pm$ 0.0489 \\
Jumping  & 0.0674 $\pm$ 0.0569  & 0.0552 $\pm$ 0.0469 \\ \bottomrule
\end{tabular}
\end{table}

\begin{table*}[tb]
\centering
\caption{Demographics of Participant Clients: Previous Art Therapy Sessions indicates the number of times the client has previously participated in art therapy; Familiarity with Traditional Drawing reflects the client's level of experience with traditional drawing techniques (0-not familiar; 1-very familiar); Familiarity with Digital Drawing reflects the client's level of experience with digital drawing techniques (0-not familiar; 1-very familiar); Participation Purposes reflects the reasons clients choose to engage in the activity.}
\vspace{-3mm}
\label{tab:clients}
\small
\resizebox{1\linewidth}{!}{
\begin{tabular}{cccccccccc}
\toprule
\textbf{ID} & \textbf{Gender} & \textbf{Age} & \textbf{Education} & \textbf{Region} & \parbox[t]{2.5cm}{\centering\textbf{Previous Art Therapy Sessions}} & \parbox[t]{3cm}{\centering\textbf{Familiarity with Traditional Drawing}} & \parbox[t]{2cm}{\centering\textbf{Familiarity with Digital Drawing}} & \parbox[t]{2cm}{\centering\textbf{Therapist Assignment}} & \parbox[t]{2.5cm}{\centering\textbf{Participation Purposes}} \\
\midrule
C1  & Female & 37  & Bachelor's & China/Shanghai & 0                            & 1                                   & 0.25  &T3 & Personal Growth                   \\
C2  & Female & 35  & Bachelor's & China/Shenzhen & 3                            & 0.5                                   & 0.5   &T3 & Career Development and Family                 \\
C3  & Female & 28  & Master's   & China/Hebei    & 2                            & 0.75                                  & 0.75   &T3  & Family and Emotional Management                \\
C4  & Female & 36  & Bachelor's & China/Beijing  & 10                           & 0.75                                   & 0   &T3  &Career Development                \\
C5  & Male   & 28  & Master's   & Germany       & 0                            & 1                                   & 0.75   &T3   &  Emotional Management and Personal Growth                       \\
C6  & Other  & 26  & Associate's & China/Heilongjiang & 1                            & 0.5                                   & 0.25  &T5  & Emotional Exploration and Intimate Relationships                           \\
C7  & Female & 23  & Master's   & China/Shanghai & 0                            & 1                                   & 1     &T5     &  Intimate Relationships                    \\
C8  & Female & 20  & Bachelor's & China/Shenzhen & 0                            & 0.5                                   & 0.5    &T5   &  Emotional Management and Intimate Relationships                       \\
C9  & Female & 25  & Bachelor's & China/Guangxi  & 4                            & 0                                   & 0.5    &T5    &  Self-Expression and Emotional Exploration                      \\
C10 & Male   & 23  & Master's   & China/Shenzhen & 0                            & 0.75                                   & 0.5   &T5   &             Self-Expression and Social Skills             \\
C11 & Female & 26  & Master's   & China/Hangzhou & 0                            & 0.5                                   & 0.25    &T4  &        Emotional Management, Social Skills and Intimate Relationships                 \\
C12 & Female & 26  & Master's   & China/Shanghai & 2                            & 0.75                                   & 0.5    &T4   &                   Stress Relieving and Intimate Relationships  \\
C13 & Female & 30  & Master's   & China/Dalian   & 0                            & 0.5                                   & 0.25   &T4    &             Family and Emotional Management            \\
C14 & Female & 19  & Bachelor's & China/Chongqing & 0                            & 0.25                                   & 0.25   &T4  &                Personal Growth and Self-Exploration           \\
C15 & Male   & 27  & Bachelor's & China/Beijing  & 0                            & 0.25                                  & 0.25   &T4    &                 Stress Relieving and Personal Growth        \\
C16 & Female & 22  & Bachelor's & China/Shandong & 0                            & 0.5                                   & 0.25   &T1     &              Emotional Management and Social Skills       \\
C17 & Male   & 38  & Master's   & China/Sichuan  & 0                            & 0.75                                   & 0.75   &T1     &                    Personal Growth      \\
C18 & Female & 40  & Master's   & China/Beijing  & 20                           & 1                                   & 0.75    &T1      &               Stress Relieving and Emotional Management          \\
C19 & Female & 28  & Bachelor's & China/Guangzhou & 0                            & 0.5                                   & 0   &T1       &                 Future Career Planning and Personal Growth      \\
C20 & Male   & 25  & Master's   & China/Guangzhou & 0                            & 1                                   & 1   &T1        &                    Academic Pressure Relieving   \\
C21 & Male   & 24  & Master's   & China/Hubei    & 0                            & 0                                   & 0   &T2        &                Childhood Family and Dreams Exploration  \\
C22 & Female & 24  & Master's   & China/Shenzhen & 0                            & 0.25                                   & 0.25    &T2  &                Emotional Management and Personal Growth     \\
C23 & Male   & 25  & Master's   & China/Zhejiang & 10                           & 0.5                                   & 0.5    &T2   &                  Emotional Development and Self-Expression        \\
C24 & Male & 55  & Bachelor's & Dubai& 0 & 0.5& 0.5&T2 &                           Emotional Management \\
\bottomrule

\end{tabular}}
\Description{The table 2 describes 24 participants in art therapy sessions. The participants are from diverse locations, including China (Shanghai, Shenzhen, Hebei, Beijing, Heilongjiang, Guangxi, Hangzhou, Chongqing, Shandong, Sichuan, Hubei, and Zhejiang), Germany, and Dubai. The ages range from 19 to 55 years old, with varying levels of education from associate degrees to master's degrees and bachelor's degrees. Their familiarity with traditional drawing techniques ranges from no familiarity to very familiar, while their familiarity with digital drawing techniques also varies across the spectrum. The participants have attended between 0 and 20 previous art therapy sessions and are assigned to different therapists identified by codes T1 to T5.Participation Purposes reflects the reasons clients choose to engage in the activity}
\end{table*}

\section{Field study}
Using \name{} as both a novel system to study and a research tool to study with, we aim to explore how a human-AI system support clients' art therapy homework in their daily settings (\textbf{RQ1}) and how such a system could mediate therapist-client collaboration surrounding art therapy homework (\textbf{RQ2}). To this end, we conducted a field deployment involving 24 recruited clients and five therapists over the course of one month.



%参与者与实验的setup
    %参与者招募
        % 我们招募的途径:To recruit our clients, we distributed digital recruitment flyers through social media platforms.
        % 海报上描述了什么:The recruitment flyer described the art therapy activities as "promoting self-exploration using a digital software".
        % 我首先要求参与者填写pre-问卷,这个问卷主要包括descriptions of the art therapy activities, demographic information, the number of art therapy sessions they attended, familiarity with digital drawing, and specific needs for the art therapy activities.
        % Participants were included in this study with the aim of reducing stress and anxiety, fostering personal growth, improving emotional regulation, and strengthening social skills.
        % 此外,we tried to selection of participants based on their regions, occupations, the types of devices they used, and the number of times they participated in art therapy.
        % finally, 有27名参与者开始使用这个系统,其中有3名参与者drop out因为缺乏时间
\subsection{Participants and Study Procedure}
\subsubsection{Participants}

The five therapists who participated in the field evaluation were the same ones from our contextual study (see \autoref{tab:expert}). Each therapist was compensated at their regular hourly rate.
For client recruitment, we distributed digital flyers through social media platforms, describing the art therapy activities as an "online art therapy experience promoting self-exploration using a digital software." This aligns with the common goal of art therapy sessions, which are widely used to promote self-exploration for all clients, beyond treating mental illness~\cite{kahn1999art, riley2003family}.

Participants first completed a pre-questionnaire, which provided an overview of the activities and collected demographics, and prior experiences with art therapy experience and with digital drawing---to ensure that we include both novices and experienced user---and their personal goals for participation. 
The therapists guided the recruitment and screening of the the clients, and included individuals seeking for reducing stress, fostering personal growth, enhancing emotional regulation, and strengthening social skills. The therapists excluded individuals with serious mental health conditions to minimize ethical risks.
%Based on the therapists' advice, clients with goals such as reducing stress and anxiety, fostering personal growth, enhancing emotional regulation, and strengthening social skills were included, avoiding ethical concerns related to clinically diagnosed mental health conditions. 
%We also considered participants' regions, device types, drawing familiarity, and prior art therapy experience to create a balanced selection.

In total, 27 clients began using \name{}, but 3 withdrew early due to scheduling conflicts. The final group of 24 clients (C1-C24; 8 self-identified males, 15 self-identified females, 1 identifying as other; aged 19-55) completed the study (client demographics are detailed in the~\autoref{tab:clients}). Clients who completed the full process were compensated with \$37, others were compensated with a prorated fee.
Our study protocol was approved by the institutional research ethics board, and all participant names in this paper have been changed to pseudonyms. Participants reviewed and signed informed consent forms before taking part, acknowledging their understanding of the study.

% The five therapists participated in the field evaluation were the ones who also participated in our contextual study (see \autoref{tab:expert}).
% Five art therapists were compensated with their regular hourly rate.
% For the clients recruitment, we distributed digital recruitment flyers through social media platforms. 
% The recruitment flyer described the art therapy activities as ``online art therapy experience promoting self-exploration using a digital software''.
% This is due to that this is a common goal for art therapy sessions, since Art therapy activities are not only effective in treating mental illness but also widely promote self-exploration for every clients, as commonly integrated into practice~\cite{kahn1999art,riley2003family}.
% First, participants completed a pre-questionnaire that provided an overview of the art therapy activities and gathered details such as their demographics, the number of art therapy sessions they've attended, familiarity with digital drawing, and any specific needs they hoped to address.
% Following that, based on the advices from the therapists, clients were included with the goal of reducing stress and anxiety, fostering personal growth, enhancing emotional regulation, and strengthening social skills.
% The therapists suggest so since they agree that these therapeutic goals would be beneficial for eavery day therapy clients and would could It might avoid the potential ethical and safety risks associated with clinically diagnosed mental health issue.
% Further, we selected participants based on a balance of their regions, the types of devices they used, the familiarity with drawing and their prior experience with art therapy. 

% In total, 27 clients began using \name{}, but 3 withdrew from the study at the early stage due to scheduling conflicts.
% Finally, 24 clients (C1-C24; 8 self-identified males, 15 self-identified females, 1 identifying as other; aged 19-55) completed our field study. 
% APPENDIX shows the specific client demographics.
% We compensated clients based on their level of involvement, with those who completed the full one-month study receiving 200 RMB as a bonus, and clients who dropped out receiving a prorated fee according to the duration of their participation.

% Our protocol was approved by the institutional research ethics board, and all names in this paper have been changed to pseudonyms.
% Also, before participating in the activity, participants carefully reviewed and signed the informed consent form, acknowledging their understanding.

%在与治疗师协商讨论下,这些用户被分到5位治疗师(see Table),其中T2有4位来访者,其余治疗师有5位来访者。
%这个研究. .
%在活动开始前,我们邀请每位参与者开展了一场介绍session. 主要是目的是介绍活动目的与流程,并且演示如何使用\name{},并且为每位来访者可以接触到系统的URL的链接;
%介绍活动结束后,来访者被鼓励有规律地去自行探索使用\name{};
%每隔一周,我们会安排治疗师与来访者进行线上一对一的session。我们会鼓励治疗师在线上一对一session之前提前review来访者的使用数据,并通过即时通讯软件与我们交流review之后的洞见与想法。
%在线上一对一session时,在不干扰治疗师艺术治疗实践的基础上,我们鼓励治疗师在线上一对一session时利用这些数据。在艺术创作阶段,来访者可以通过分享屏幕的方式使用系统的第一个阶段进行创作并与治疗师进行讨论交流,在session快结束前治疗师会给来访者推荐家庭作业。
%在session结束后,治疗师会在治疗师系统上安排家庭作业并给予来访者的个人赠言。此外,来访者在结束线上session后可以按照治疗师的推荐完成家庭作业或者自行探索使用系统。
\subsubsection{Procedures}

Clients were distributed in coordination with the five therapists, as shown in \autoref{tab:expert}. T2 was assigned four clients, while the other therapists each had five clients. The field study consisted of two main activities: (1) three online in-session activities, where clients had one-on-one conversations and collaborated with the therapist, and (2) unstructured between-session activities, where clients practiced therapy homework using \name{} following the therapist’s recommendations.
Before the study, we held online introductory sessions to familiarize the clients with \name{}, and provided both demonstrations and hands-on exploration on their preferred devices. Similarly, we offered online training for therapists on customizing and reviewing homework, while allowing them to explore both the therapist-facing and client-facing applications. After the session, clients were encouraged to regularly explore \name{}.
Two weeks into the study, we scheduled weekly one-on-one online sessions between therapists and clients, each lasting approximately 60 minutes. Therapists were encouraged to review the clients' homework history using \autoref{fig:ui}(c) before each session. During the online session, therapists used this data to inform their practices without interrupting the flow of therapy. We encouraged clients in advance, to create artworks during the Art-making Phase~(\autoref{fig:qual_results}(a)), sharing screens and discussing their creations with the therapist, but did not interfere with the therapeutic process.

%Clients also used \autoref{fig:qual_results}(a) to create artwork, sharing their screens and discussing their creations with the therapist.

At the end of each session, therapists recommended homework tasks based on insights gained during the conversation. After the session, therapists might customize homework agents, including customizing conversational principles, assigning homework tasks, and providing personal messages through \autoref{fig:ui}~(d). Clients could then either complete the assigned homework or engage in self-exploration using \name{} between sessions.

% Clients were distributed In coordination with the five therapists, as shown by \autoref{tab:clients}: T2 was assigned with four clients, while each of the other therapists was assigned with five clients.
% The procedure for the field study consisted of two activities: (1) three online in-session activities where they have one-on-one conversation and collaboration with the therapist and (2) unstructured between-session activities where they perform therapy homework practices either upon recommendations of usage from the therapist or volunteerily use it in their daily lives.
% Before the study, we conducted an introductory session for each client to explain the activities, demonstrate how to use \name{}, and provide access to \name{} via a URL on their preferred devices.
% After the introductory session, the clients were encouraged to explore the use of \name{} on a regular basis.

% After two weeks of self-exploration, we started scheduling weekly one one-on-one online sessions between the therapists and the clients.
% Therapists were encouraged to review clients' homework history using \autoref{fig:ui}~(c) before the online session.
% During the online one-on-one session, we encouraged therapists to use these history data without interfering with their art therapy practices. 
% Also, they would utilize \autoref{fig:ui}~(a) to create their artwork by sharing their screens and discussing their artworks with therapists. 
% Before the end of the session, the therapist would recommend the homework tasks for the client based on the insights gained from the one-on-one session.
% After the online session ends, therapists would customize homework agents, including modifying or updating the conversational agent principles, assigning homework tasks and providing therapist's messages to the client through \autoref{fig:system}~(d). 
% Correspondingly, clients could either complete the homework or engage in self-exploration using \name{} between sessions.

% 对于异步session场景数据收集下,所有来访者使用系统的图像以及对话记录等日志数据以及治疗师在治疗师系统中使用定制功能的日志数据在保存在数据库中。
% 此外,我们鼓励来访者和治疗师通过即时通讯软件发送给我们images以及comments关于使用系统的实践以及感受。
% 对于线上session的场景数据收集,首先,online sessions were audio- and video-recorded.
% 此外,at the end of each online session, we conducted a 5-minute interview with therapists, mainly to collect their practices and experiences about the session.
% Upon concluding all the sessions,我们与治疗师以及来访者开展了约为30分钟的semi-structured interview to 探索ai agents如何支持艺术治疗场景的家庭作业(RQ1)以及AI agents如何mediate 治疗师与来访者合作(RQ2). We used 治疗师与来访者在 the trial period使用系统的log 数据以及他们的反馈作为stimuli 去问特定的使用实践的问题。
% With participants' consent, we recorded the interviews and transcribed them for thematic analysis.
% First, two researchers conducted collaborative inductive coding. They initially annotated the transcript to identify relevant quotes, key concepts, and recurring patterns in the data. These findings were further developed through regular discussions, leading to a detailed coding scheme aligned with the research questions. Quotes were then coded and clustered into a hierarchy of emerging themes, continually reviewed, and refined in recurrent meetings, where exemplar quotes were also selected for presenting each theme and sub-theme. 
% Also, we collected the log data from 治疗师和来访者 作为证据以及examples for the thematic analysis results.

\subsection{Data Gathering Methods} 

For between-sessions, we stored all homework-related data in a database, including artwork, dialogue, usage logs, as well as information on homework customization such as conversational principles, tasks, and personal messages.
We encouraged participants to use personal messaging (WeChat) to share pictures and comments about on-the-spot experience and feelings after homework with \name{} to compensate for semi-structured interviews.
During online sessions, we recorded audio and video. 
The researchers did not observe the therapy session in live, but reviewed post hoc, as the therapists believed a third party's presence could affect a client's emotional expression and the therapist-client dynamic.
After each session, we conducted a brief 5-minute interview with the therapists to gather their insights and feelings.

Upon the completion of the final one-on-one sessions, we conducted 30-minute semi-structured interviews with both therapists and clients. These interviews aimed to explore how \name{} supported art therapy homework in clients' daily lives (\textbf{RQ1}) and how therapists and clients collaborated surrounding art therapy homework (\textbf{RQ2}). We used feedback and homework outcomes from the trial period to ask targeted questions about their practices.
With participants' consent, we recorded and transcribed the brief 5-minute interviews and the 30-minute interviews for thematic analysis~\cite{braun2006using}. This analysis also included the personal messages shared by the participants about their on-the-spot experiences.
%we recorded and transcribed the interviews for thematic analysis. 
Two researchers then engaged in inductive coding, annotating transcripts to identify relevant quotes, key concepts, and patterns. They developed a detailed coding scheme through regular discussions, grouping quotes into a hierarchical structure of themes and sub-themes. Exemplar quotes were selected to represent each theme. We also used homework history (e.g., images or conversation data) and customization data (e.g., homework dialogue principle data) as evidences or examples to back up the findings in our thematic analysis.



% In between sessions, all homework history data~(e.g., artwork, creative process data and dialogue data) and history data on homework customization~(e.g., principles of conversational agents, homework tasks and personal messages) were stored in the database.
% In addition, we encouraged clients and therapists to send us images and comments about their experiences and feelings when using \name{} via an instant messaging app.
% For online in-sessions, the sessions were first audio- and video-recorded.
% At the end of each in-session, we conducted a brief 5-minute interview with the therapists to gather insights into their practices and feelings during the session.
% Upon concluding all the sessions, we conducted approximately 30-minute semi-structured interviews with both the therapists and the clients to explore how \name{} support art therapy homework in clients' daily settings~(\textbf{RQ1}), and how therapists tailored the homework and tracked the homework history surrounding art therapy homework~(\textbf{RQ2}). 
% Further, we employed the homework outcomes and feedback from both therapists and clients during the trial period as stimuli to ask specific questions about their practices. 

% With participants' consent, we recorded the interviews and transcribed them for thematic analysis~\cite{braun2006using}.
% Initially, two researchers engaged in collaborative inductive coding. They began by annotating the transcript to highlight relevant quotes, key concepts, and recurring patterns in the data. Through regular discussions, they expanded these insights into a detailed coding scheme that aligned with their research questions. The quotes were then systematically coded and grouped into a hierarchical structure of emerging themes, which were continuously reviewed and refined during recurring meetings. During these discussions, exemplar quotes were also chosen to represent each theme and sub-theme.
% We also gathered homework history and customization data, including artworks and conversation records from both therapists and clients, as evidence and examples to support the results of the thematic analysis.

\begin{figure*}[tb]
  \centering
  \includegraphics[width=\linewidth]{images/findings_1.png}
  \vspace{-7mm}
  \caption{Overview of The Homework Engagement of Clients with \name{}: (a) Homework Activity Date Distribution; (b) Accumulated Homework Activity Hourly Distribution of the Day; (c) Usage of AI Brushes in Artworks; 
  }
  \Description{Figure 5 contains three sub-figures. Figure 5a shows the Homework Activity Date Distribution for 24 clients over a four-week period, using seven different shades of purple to represent varying levels of participation in the homework sessions. Figure 5b illustrates the frequency of AI brush usage during clients' homework art-making, with the top 20 most frequently used brushes highlighted in larger font. Figure 5c depicts the distribution of homework sessions across different times of the day, revealing that clients tend to engage in homework sessions more frequently in the afternoon and evening.}
  \label{fig:quan_results}
\end{figure*}




% \section{Applications}
Many applications can be enabled by our unknown word detection method by acquiring use's unknown words during reading in real time. Foreign language reading can occur in two scenarios. One is literature reading and language learning on 2D interfaces such as laptops or pads. The other is getting information in the surrounding environment through a 3D display such as AR glasses and head-mounted devices. We then discuss the potential applications of our method in these two scenarios.


\subsection{Language Learning Assistance}

We can divide the functions of language learning assistance into two categories: real-time and non-real-time. By supporting real-time unknown word detection, our method can make translation less obtrusive and help users read more fluently. Our method can track gaze to locate text areas (the sliding window in Fig.~\ref{fig:application1}) and detect unknown words in the area. Then, the application is able to translate these unknown words automatically. It can save users the time of copying and pasting words into the dictionary or retrieving words through the cursor, as well as reducing interruptions to users' reading. At the same time, unknown words can also be automatically added to the user's word list, allowing users to view them at any time. 

If the overall reading performance is considered and the real-time feedback is not necessary, many applications can be enabled by summarizing and analyzing the unknown words encountered during the reading process. Potential applications include generating flashcards to facilitate users' memory, counting the user's vocabulary mastery to provide users with learning reports and assessing the forgetting rate to offer users a personalized word learning plan. Combined with generative AI, it is also possible to generate new documents based on recently encountered unknown words. This can help users consolidate vocabulary in an intriguing way.

In summary, fluent reading and efficient word learning are the most urgent needs of second language learners. The highly accurate unknown word detection provided by our method can assist users' language learning in either real-time or summarized manner.

\begin{figure}[htbp]
  \includegraphics[width=0.49\columnwidth]{figures/app_1.png}
  \includegraphics[width=0.48\columnwidth]{figures/app_2.png}

  %\setlength{\abovecaptionskip}{0.1cm}
  \caption{Applications in 2D language learning scenario: (Left) Real-time auto translation. (Right) Unknown-word summary and word learning analysis.}
  \label{fig:application1} 
\end{figure}



\subsection{Reading Assistant in Foreign Language Environment}
With the development of augmented reality technology, head-mounted display devices such as Apple Vision Pro\footnote{https://www.apple.com/apple-vision-pro/} will gradually be integrated into daily life in the future. This will allow reading behavior in three-dimensional space to be captured as well. Therefore, we envision that in addition to reading 2D materials, our unknown word detection technology will also be used to assist reading in the three-dimensional world. Moreover, AR headsets are generally equipped with eye trackers, which will enable our method to be easily applied.

Three-dimensional application scenarios include but are not limited to obtaining key information from menus, manuals, and street signs when traveling or living in a foreign language environment and reading commentaries in foreign language exhibitions. Compared with directly displaying large sections of translated text in front of users, providing only key information based on unknown words can reduce the interference to the user's view and reduce the user's burden on extracting key information from a large amount of text. It will provide users with more precise and less intrusive reading assistance in foreign language environments.

\begin{figure}[htbp]
  \includegraphics[width=0.47\columnwidth]{figures/app_3.png}
  \includegraphics[width=0.48\columnwidth]{figures/app_4.png}

  %\setlength{\abovecaptionskip}{0.1cm}
  \caption{Applications in 3D AR scenario: (Left) The user wearing the AR headset encounters an unknown word when reading the manual. (Right) The translation of the unknown word pops out automatically.}
  \label{fig:application2} 
\end{figure}


\section{Discussion}
The development of foundation models has increasingly relied on accessible data support to address complex tasks~\cite{zhang2024data}. Yet major challenges remain in collecting scalable clinical data in the healthcare system, such as data silos and privacy concerns. To overcome these challenges, MedForge integrates multi-center clinical knowledge sources into a cohesive medical foundation model via a collaborative scheme. MedForge offers a collaborative path to asynchronously integrate multi-center knowledge while maintaining strong flexibility for individual contributors.
This key design allows a cost-effective collaboration among clinical centers to build comprehensive medical models, enhancing private resource utilization across healthcare systems.

Inspired by collaborative open-source software development~\cite{raffel2023building, github}, our study allows individual clinical institutions to independently develop branch modules with their data locally. These branch modules are asynchronously integrated into a comprehensive model without the need to share original data, avoiding potential patient raw data leakage. Conceptually similar to the open-source collaborative system, iterative module merging development ensures the aggregation of model knowledge over time while incorporating diverse data insights from distributed institutions. In particular, this asynchronous scheme alleviates the demand for all users to synchronize module updates as required by conventional methods (e.g., LoRAHub~\cite{huang2023lorahub}).


MedForge's framework addresses multiple data challenges in the cycle of medical foundation model development, including data storage, transmission, and leakage. As the data collection process requires a large amount of distributed data, we show that dataset distillation contributes greatly to reducing data storage capacity. In MedForge, individual contributors can simply upload a lightweight version of the dataset to the central model developer. As a result, the remarkable reduction in data volume (e.g., 175 times less in LC25000) alleviates the burden of data transfer among multiple medical centers. For example, we distilled a 10,500 image training set into 60 representative distilled data while maintaining a strong model performance. We choose to use a lightweight dataset as a transformed representation of raw data to avoid the leakage of sensitive raw information.
Second, the asynchronous collaboration mode in MedForge allows flexible model merging, particularly for users from various local medical centers to participate in model knowledge integration. 
Third, MedForge reformulates the conventional top-down workflow of building foundational models by adopting a bottom-up approach. Instead of solely relying on upstream builders to predefine model functionalities, MedForge allows medical centers to actively contribute to model knowledge integration by providing plugin modules (i.e., LoRA) and distilled datasets. This approach supports flexible knowledge integration and allows models to be applicable to wide-ranging clinical tasks, addressing the key limitation of fixed functionalities in traditional workflows.

We demonstrate the strong capacity of MedForge via the asynchronous merging of three image classification tasks. MedForge offered an incremental merging strategy that is highly flexible compared to plain parameter average~\cite{wortsman2022model} and LoRAHub~\cite{huang2023lorahub}. Specifically, plain parameter averaging merges module parameters directly and ignores the contribution differences of each module. Although LoRAHub allows for flexible distribution of coefficients among modules, it lacks the ability to continuously update, limiting its capacity to incorporate new knowledge during the merging process. In contrast, MedForge shows its strong flexibility of continuous updates while considering the contribution differences among center modules. The robustness of MedForge has been demonstrated by shuffling merging order (Tab~\ref{tab:order}), which shows that merging new-coming modules will not hurt the model ability of previous tasks in various orders, mitigating the model catastrophic forgetting. 
MedForge also reveals a strong generality on various choices of component modules. Our experiments on dataset distillation settings (such as DC and without DSA technique) and PEFT techniques (such as DoRA) emphasize the extensible ability of MedForge's module settings. 

To fully exploit multi-scale clinical data, it will be necessary to include broader data modalities (e.g., electronic health records and radiological images). Managing these diverse data formats and standards among numerous contributors can be challenging due to the potential conflict between collaborators. 
Moreover, since MedForge integrates multiple clinical tasks that involve varying numbers of classification categories, conventional classifier heads with fixed class sizes are not applicable. However, the projection head of the CLIP model, designed to calculate similarities between image and text, is well-suited for this scenario. It allows MedForge to flexibly handle medical datasets with different category numbers, thus overcoming the challenge of multi-task classification. That said, this design choice also limits the variety of model architectures that can be utilized, as it depends specifically on the CLIP framework. Future investigations will explore extensive solutions to make the overall architecture more flexible. Additionally, incorporating more sophisticated data anonymization, such as synthetic data generation~\cite{ding2023large}, and encryption methods can also be considerable. To improve data privacy protection, test-time adaptation technique~\cite{wang2020tent, liang2024comprehensive} without substantial training data can be considered to alleviate the burden of data sharing in the healthcare system.



             

\section{Conclusion}
We reveal a tradeoff in robust watermarks: Improved redundancy of watermark information enhances robustness, but increased redundancy raises the risk of watermark leakage. We propose DAPAO attack, a framework that requires only one image for watermark extraction, effectively achieving both watermark removal and spoofing attacks against cutting-edge robust watermarking methods. Our attack reaches an average success rate of 87\% in detection evasion (about 60\% higher than existing evasion attacks) and an average success rate of 85\% in forgery (approximately 51\% higher than current forgery studies). 

\begin{acks}
This work is supported by the Natural Science Foundation of China under Grant No. 62132010, 62472244, 62102221, the Tsinghua University Initiative Scientific Research Program, and the Undergraduate Education Innovation Grants, Tsinghua University.
\end{acks}

%%
%% The next two lines define the bibliography style to be used, and
%% the bibliography file.
\bibliographystyle{ACM-Reference-Format}
\bibliography{ref}

\end{document}
\endinput
%%
%% End of file `sample-acmsmall.tex'.
