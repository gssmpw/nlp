\begin{longtable}{p{3cm}p{6cm}p{6cm}}
\caption{\textbf{Qualitative Examples of Synthetic Intents}: We show randomly selected examples of the top 3 closest synthetic intents by embedding distance (using OpenAI's \texttt{text-embedding-3-large}) to a query intent. Each synthetic intent is generated conditioned on a symbol from the codebase (using \texttt{GPT-4o-mini}). To ground plan steps during plan search, the intent from the plan step is matched to synthetic intents and therefore to the symbols corresponding to the synthetic intents.} \\
\label{tab:synthetic-intents-example} \\
\toprule
Repository & Intent & Top-3 Closest Synthetic Intents (\texttt{Symbol}) \\
\midrule
\endfirsthead

\multicolumn{3}{l}{\emph{Continued from previous page}} \\
\toprule
Repository & Intent & Top-3 Closest Synthetic Intents (\texttt{Symbol}) \\
\midrule
\endhead

\midrule
\multicolumn{3}{r}{\emph{Continued on next page}} \\
\endfoot

\bottomrule
\endlastfoot

\multirow{3}{*}{pybamm} & Add a no SEI submodel. & \textit{I need to create a new discretisation instance without providing a mesh.} (\texttt{Discretisation}) \newline \textit{I want to create an isothermal thermal submodel for my simulation.} (\texttt{Isothermal}) \newline \textit{I want to assign parameter values to a specific model.} (\texttt{process\_model}) \\
\cmidrule(l){2-3}
& Add a constant porosity submodel. & \textit{I want to create an instance of the constant concentration diffusion model.} (\texttt{ConstantConcentration}) \newline \textit{I want to create an isothermal thermal submodel for my simulation.} (\texttt{Isothermal}) \newline \textit{I want to initialize a constant concentration model for the diffusion process with specific parameters.} (\texttt{ConstantConcentration}) \\
\cmidrule(l){2-3}
& Add an isothermal thermal submodel. & \textit{I want to create an isothermal thermal submodel for my simulation.} (\texttt{Isothermal}) \newline \textit{I need to set up the Isothermal model for my thermal simulations.} (\texttt{Isothermal}) \newline \textit{I need to gather all temperature variables associated with the isothermal submodel.} (\texttt{get\_fundamental\_variables}) \\
\midrule
\multirow{3}{*}{dd4hep} & Set up a particle gun with specified parameters. & \textit{I want to set up a particle gun in the simulation to start generating particles.} (\texttt{setupGun}) \newline \textit{I need to configure the particle gun with specific parameters such as name, particle type, and energy level.} (\texttt{setupGun}) \newline \textit{I want to customize the position and multiplicity settings for the particle gun in the simulation.} (\texttt{setupGun}) \\
\cmidrule(l){2-3}
& Set up a tracker for the simulation. & \textit{I want to set up a tracking field for my particle simulation using a specific configuration.} (\texttt{setupTrackingFieldMT}) \newline \textit{I want to set up the construction of the detector in the simulation.} (\texttt{detectorConstruction}) \newline \textit{I want to configure the tracking field setup for my Geant4 simulation.} (\texttt{setupTrackingField}) \\
\cmidrule(l){2-3}
& Set up event actions for particle printing. & \textit{I am looking to set up a generator action for particle generation in my application.} (\texttt{GeneratorAction}) \newline \textit{I want to set up a particle gun in the simulation to start generating particles.} (\texttt{setupGun}) \newline \textit{I would like to use the tracking action functionality to monitor particle tracks in my experiment.} (\texttt{TrackingAction}) \\
\midrule
\multirow{3}{*}{fealpy} & Create a uniform time mesh for the simulation. & \textit{I want to generate the initial mesh for my 2D time harmonic solver.} (\texttt{init\_mesh}) \newline \textit{I want to ensure that the mesh is refined uniformly to improve simulation accuracy.} (\texttt{init\_mesh}) \newline \textit{I want to create a uniform triangular mesh to use in my analysis.} (\texttt{init\_mesh}) \\
\cmidrule(l){2-3}
& Solve the linear system to update the solution at the current time step. & \textit{I need to update my solution by solving the linear system after applying Dirichlet boundary conditions.} (\texttt{solve}) \newline \textit{I need to iterate through time steps and update my model's solutions.} (\texttt{time\_integration}) \newline \textit{I need to update the state of my model variables after solving the system.} (\texttt{solve}) \\
\cmidrule(l){2-3}
& Advance to the next time level in the time mesh. & \textit{I want to progress the time in my algorithm by moving to the next time level.} (\texttt{next\_time\_level}) \newline \textit{I want to advance to the next time level in the simulation.} (\texttt{next\_time\_level}) \newline \textit{I want to progress to the subsequent time level in the timeline.} (\texttt{next\_time\_level}) \\
\midrule
\multirow{3}{*}{nplab} & Create an experiment class that involves a shutter and a spectrometer. & \textit{I want to initialize a new shutter instance in my experiment setup.} (\texttt{Shutter}) \newline \textit{I want to prepare a shutter for my nanophotonics experiments.} (\texttt{Shutter}) \newline \textit{I need to construct a shutter object to manage exposure times.} (\texttt{Shutter}) \\
\cmidrule(l){2-3}
& Initialize and display the GUI application. & \textit{I want to initialize a new GUI widget that will display a plot.} (\texttt{Widget}) \newline \textit{I want to initialize the user interface for the spectrometers in the application.} (\texttt{\_init\_ui}) \newline \textit{I need to initialize a GUI component that displays spectrometer controls.} (\texttt{SpectrometersUI}) \\
\cmidrule(l){2-3}
& Define properties for irradiation time and wait time. & \textit{I need to configure the integration time and delay settings for my spectrometer to ensure accurate time series measurements.} (\texttt{update\_time\_series\_params}) \newline \textit{I need to expose the instrument for a set amount of time and ensure it blocks until the exposure completes.} (\texttt{expose}) \newline \textit{I want to specify the duration for which the spectrometer should collect data during a measurement.} (\texttt{set\_integration\_time}) \\
\midrule
\multirow{3}{*}{python-sc2} & Manage drones to gather minerals if vespene gas is above a certain threshold or Zergling speed upgrade is pending. & \textit{I want to check if my unit is currently gathering resources from a mineral field or vespene geyser.} (\texttt{is\_gathering}) \newline \textit{I need to identify the units that are engaged in gathering minerals or vespene.} (\texttt{gathering}) \newline \textit{I need to direct a unit to gather either minerals or gas for my economy.} (\texttt{gather}) \\
\cmidrule(l){2-3}
& Research Zergling speed upgrade if conditions are met. & \textit{I need to queue an upgrade research for my unit.} (\texttt{research}) \newline \textit{I need to determine how fast my unit can move considering the effects of any active upgrades.} (\texttt{calculate\_speed}) \newline \textit{I want to start researching an upgrade if the necessary tech building is ready.} (\texttt{research}) \\
\cmidrule(l){2-3}
& Draw a creep pixelmap for debugging purposes. & \textit{I want to check if there is creep on a specific grid point in the game.} (\texttt{has\_creep}) \newline \textit{I want to output debug information by drawing a box around a game unit.} (\texttt{debug\_box2\_out}) \newline \textit{I want to draw a visual line between two points in my game for debugging purposes.} (\texttt{debug\_line\_out}) \\
\midrule
\multirow{3}{*}{basilisk} & Initialize and execute the simulation within the scenario execution function. & \textit{I need to initialize the simulation and prepare all modules for execution.} (\texttt{SimBaseClass}) \newline \textit{I want to execute a simulation by assigning the appropriate execution function.} (\texttt{setExecutionFunction}) \newline \textit{I need to prepare my simulation for execution by initializing all required data structures and parameters.} (\texttt{SimBaseClass}) \\
\cmidrule(l){2-3}
& Import necessary modules and set up file paths for the simulation. & \textit{I need to initialize the simulation and prepare all modules for execution.} (\texttt{SimBaseClass}) \newline \textit{I want to configure my simulation environment with the correct paths and logger setup on initialization.} (\texttt{SimBaseClass}) \newline \textit{I need to ensure that all modules in the simulation are properly self-initialized.} (\texttt{InitializeSimulation}) \\
\cmidrule(l){2-3}
& Define a function to execute the simulation scenario, including configuring stop time and initializing the simulation. & \textit{I want to define an execution function that will run my simulation instance.} (\texttt{setExecutionFunction}) \newline \textit{I want to define the parameters for running a simulation, including the creation and execution functions.} (\texttt{SimulationParameters}) \newline \textit{I want to define how long my simulation should run by setting the stop time.} (\texttt{ConfigureStopTime}) \\

\end{longtable}