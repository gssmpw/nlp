
\documentclass{article} % For LaTeX2e
\usepackage{iclr2025_conference,times}

% Optional math commands from https://github.com/goodfeli/dlbook_notation.
%%%%% NEW MATH DEFINITIONS %%%%%

\usepackage{amsmath,amsfonts,bm}
\usepackage{derivative}
% Mark sections of captions for referring to divisions of figures
\newcommand{\figleft}{{\em (Left)}}
\newcommand{\figcenter}{{\em (Center)}}
\newcommand{\figright}{{\em (Right)}}
\newcommand{\figtop}{{\em (Top)}}
\newcommand{\figbottom}{{\em (Bottom)}}
\newcommand{\captiona}{{\em (a)}}
\newcommand{\captionb}{{\em (b)}}
\newcommand{\captionc}{{\em (c)}}
\newcommand{\captiond}{{\em (d)}}

% Highlight a newly defined term
\newcommand{\newterm}[1]{{\bf #1}}

% Derivative d 
\newcommand{\deriv}{{\mathrm{d}}}

% Figure reference, lower-case.
\def\figref#1{figure~\ref{#1}}
% Figure reference, capital. For start of sentence
\def\Figref#1{Figure~\ref{#1}}
\def\twofigref#1#2{figures \ref{#1} and \ref{#2}}
\def\quadfigref#1#2#3#4{figures \ref{#1}, \ref{#2}, \ref{#3} and \ref{#4}}
% Section reference, lower-case.
\def\secref#1{section~\ref{#1}}
% Section reference, capital.
\def\Secref#1{Section~\ref{#1}}
% Reference to two sections.
\def\twosecrefs#1#2{sections \ref{#1} and \ref{#2}}
% Reference to three sections.
\def\secrefs#1#2#3{sections \ref{#1}, \ref{#2} and \ref{#3}}
% Reference to an equation, lower-case.
\def\eqref#1{equation~\ref{#1}}
% Reference to an equation, upper case
\def\Eqref#1{Equation~\ref{#1}}
% A raw reference to an equation---avoid using if possible
\def\plaineqref#1{\ref{#1}}
% Reference to a chapter, lower-case.
\def\chapref#1{chapter~\ref{#1}}
% Reference to an equation, upper case.
\def\Chapref#1{Chapter~\ref{#1}}
% Reference to a range of chapters
\def\rangechapref#1#2{chapters\ref{#1}--\ref{#2}}
% Reference to an algorithm, lower-case.
\def\algref#1{algorithm~\ref{#1}}
% Reference to an algorithm, upper case.
\def\Algref#1{Algorithm~\ref{#1}}
\def\twoalgref#1#2{algorithms \ref{#1} and \ref{#2}}
\def\Twoalgref#1#2{Algorithms \ref{#1} and \ref{#2}}
% Reference to a part, lower case
\def\partref#1{part~\ref{#1}}
% Reference to a part, upper case
\def\Partref#1{Part~\ref{#1}}
\def\twopartref#1#2{parts \ref{#1} and \ref{#2}}

\def\ceil#1{\lceil #1 \rceil}
\def\floor#1{\lfloor #1 \rfloor}
\def\1{\bm{1}}
\newcommand{\train}{\mathcal{D}}
\newcommand{\valid}{\mathcal{D_{\mathrm{valid}}}}
\newcommand{\test}{\mathcal{D_{\mathrm{test}}}}

\def\eps{{\epsilon}}


% Random variables
\def\reta{{\textnormal{$\eta$}}}
\def\ra{{\textnormal{a}}}
\def\rb{{\textnormal{b}}}
\def\rc{{\textnormal{c}}}
\def\rd{{\textnormal{d}}}
\def\re{{\textnormal{e}}}
\def\rf{{\textnormal{f}}}
\def\rg{{\textnormal{g}}}
\def\rh{{\textnormal{h}}}
\def\ri{{\textnormal{i}}}
\def\rj{{\textnormal{j}}}
\def\rk{{\textnormal{k}}}
\def\rl{{\textnormal{l}}}
% rm is already a command, just don't name any random variables m
\def\rn{{\textnormal{n}}}
\def\ro{{\textnormal{o}}}
\def\rp{{\textnormal{p}}}
\def\rq{{\textnormal{q}}}
\def\rr{{\textnormal{r}}}
\def\rs{{\textnormal{s}}}
\def\rt{{\textnormal{t}}}
\def\ru{{\textnormal{u}}}
\def\rv{{\textnormal{v}}}
\def\rw{{\textnormal{w}}}
\def\rx{{\textnormal{x}}}
\def\ry{{\textnormal{y}}}
\def\rz{{\textnormal{z}}}

% Random vectors
\def\rvepsilon{{\mathbf{\epsilon}}}
\def\rvphi{{\mathbf{\phi}}}
\def\rvtheta{{\mathbf{\theta}}}
\def\rva{{\mathbf{a}}}
\def\rvb{{\mathbf{b}}}
\def\rvc{{\mathbf{c}}}
\def\rvd{{\mathbf{d}}}
\def\rve{{\mathbf{e}}}
\def\rvf{{\mathbf{f}}}
\def\rvg{{\mathbf{g}}}
\def\rvh{{\mathbf{h}}}
\def\rvu{{\mathbf{i}}}
\def\rvj{{\mathbf{j}}}
\def\rvk{{\mathbf{k}}}
\def\rvl{{\mathbf{l}}}
\def\rvm{{\mathbf{m}}}
\def\rvn{{\mathbf{n}}}
\def\rvo{{\mathbf{o}}}
\def\rvp{{\mathbf{p}}}
\def\rvq{{\mathbf{q}}}
\def\rvr{{\mathbf{r}}}
\def\rvs{{\mathbf{s}}}
\def\rvt{{\mathbf{t}}}
\def\rvu{{\mathbf{u}}}
\def\rvv{{\mathbf{v}}}
\def\rvw{{\mathbf{w}}}
\def\rvx{{\mathbf{x}}}
\def\rvy{{\mathbf{y}}}
\def\rvz{{\mathbf{z}}}

% Elements of random vectors
\def\erva{{\textnormal{a}}}
\def\ervb{{\textnormal{b}}}
\def\ervc{{\textnormal{c}}}
\def\ervd{{\textnormal{d}}}
\def\erve{{\textnormal{e}}}
\def\ervf{{\textnormal{f}}}
\def\ervg{{\textnormal{g}}}
\def\ervh{{\textnormal{h}}}
\def\ervi{{\textnormal{i}}}
\def\ervj{{\textnormal{j}}}
\def\ervk{{\textnormal{k}}}
\def\ervl{{\textnormal{l}}}
\def\ervm{{\textnormal{m}}}
\def\ervn{{\textnormal{n}}}
\def\ervo{{\textnormal{o}}}
\def\ervp{{\textnormal{p}}}
\def\ervq{{\textnormal{q}}}
\def\ervr{{\textnormal{r}}}
\def\ervs{{\textnormal{s}}}
\def\ervt{{\textnormal{t}}}
\def\ervu{{\textnormal{u}}}
\def\ervv{{\textnormal{v}}}
\def\ervw{{\textnormal{w}}}
\def\ervx{{\textnormal{x}}}
\def\ervy{{\textnormal{y}}}
\def\ervz{{\textnormal{z}}}

% Random matrices
\def\rmA{{\mathbf{A}}}
\def\rmB{{\mathbf{B}}}
\def\rmC{{\mathbf{C}}}
\def\rmD{{\mathbf{D}}}
\def\rmE{{\mathbf{E}}}
\def\rmF{{\mathbf{F}}}
\def\rmG{{\mathbf{G}}}
\def\rmH{{\mathbf{H}}}
\def\rmI{{\mathbf{I}}}
\def\rmJ{{\mathbf{J}}}
\def\rmK{{\mathbf{K}}}
\def\rmL{{\mathbf{L}}}
\def\rmM{{\mathbf{M}}}
\def\rmN{{\mathbf{N}}}
\def\rmO{{\mathbf{O}}}
\def\rmP{{\mathbf{P}}}
\def\rmQ{{\mathbf{Q}}}
\def\rmR{{\mathbf{R}}}
\def\rmS{{\mathbf{S}}}
\def\rmT{{\mathbf{T}}}
\def\rmU{{\mathbf{U}}}
\def\rmV{{\mathbf{V}}}
\def\rmW{{\mathbf{W}}}
\def\rmX{{\mathbf{X}}}
\def\rmY{{\mathbf{Y}}}
\def\rmZ{{\mathbf{Z}}}

% Elements of random matrices
\def\ermA{{\textnormal{A}}}
\def\ermB{{\textnormal{B}}}
\def\ermC{{\textnormal{C}}}
\def\ermD{{\textnormal{D}}}
\def\ermE{{\textnormal{E}}}
\def\ermF{{\textnormal{F}}}
\def\ermG{{\textnormal{G}}}
\def\ermH{{\textnormal{H}}}
\def\ermI{{\textnormal{I}}}
\def\ermJ{{\textnormal{J}}}
\def\ermK{{\textnormal{K}}}
\def\ermL{{\textnormal{L}}}
\def\ermM{{\textnormal{M}}}
\def\ermN{{\textnormal{N}}}
\def\ermO{{\textnormal{O}}}
\def\ermP{{\textnormal{P}}}
\def\ermQ{{\textnormal{Q}}}
\def\ermR{{\textnormal{R}}}
\def\ermS{{\textnormal{S}}}
\def\ermT{{\textnormal{T}}}
\def\ermU{{\textnormal{U}}}
\def\ermV{{\textnormal{V}}}
\def\ermW{{\textnormal{W}}}
\def\ermX{{\textnormal{X}}}
\def\ermY{{\textnormal{Y}}}
\def\ermZ{{\textnormal{Z}}}

% Vectors
\def\vzero{{\bm{0}}}
\def\vone{{\bm{1}}}
\def\vmu{{\bm{\mu}}}
\def\vtheta{{\bm{\theta}}}
\def\vphi{{\bm{\phi}}}
\def\va{{\bm{a}}}
\def\vb{{\bm{b}}}
\def\vc{{\bm{c}}}
\def\vd{{\bm{d}}}
\def\ve{{\bm{e}}}
\def\vf{{\bm{f}}}
\def\vg{{\bm{g}}}
\def\vh{{\bm{h}}}
\def\vi{{\bm{i}}}
\def\vj{{\bm{j}}}
\def\vk{{\bm{k}}}
\def\vl{{\bm{l}}}
\def\vm{{\bm{m}}}
\def\vn{{\bm{n}}}
\def\vo{{\bm{o}}}
\def\vp{{\bm{p}}}
\def\vq{{\bm{q}}}
\def\vr{{\bm{r}}}
\def\vs{{\bm{s}}}
\def\vt{{\bm{t}}}
\def\vu{{\bm{u}}}
\def\vv{{\bm{v}}}
\def\vw{{\bm{w}}}
\def\vx{{\bm{x}}}
\def\vy{{\bm{y}}}
\def\vz{{\bm{z}}}

% Elements of vectors
\def\evalpha{{\alpha}}
\def\evbeta{{\beta}}
\def\evepsilon{{\epsilon}}
\def\evlambda{{\lambda}}
\def\evomega{{\omega}}
\def\evmu{{\mu}}
\def\evpsi{{\psi}}
\def\evsigma{{\sigma}}
\def\evtheta{{\theta}}
\def\eva{{a}}
\def\evb{{b}}
\def\evc{{c}}
\def\evd{{d}}
\def\eve{{e}}
\def\evf{{f}}
\def\evg{{g}}
\def\evh{{h}}
\def\evi{{i}}
\def\evj{{j}}
\def\evk{{k}}
\def\evl{{l}}
\def\evm{{m}}
\def\evn{{n}}
\def\evo{{o}}
\def\evp{{p}}
\def\evq{{q}}
\def\evr{{r}}
\def\evs{{s}}
\def\evt{{t}}
\def\evu{{u}}
\def\evv{{v}}
\def\evw{{w}}
\def\evx{{x}}
\def\evy{{y}}
\def\evz{{z}}

% Matrix
\def\mA{{\bm{A}}}
\def\mB{{\bm{B}}}
\def\mC{{\bm{C}}}
\def\mD{{\bm{D}}}
\def\mE{{\bm{E}}}
\def\mF{{\bm{F}}}
\def\mG{{\bm{G}}}
\def\mH{{\bm{H}}}
\def\mI{{\bm{I}}}
\def\mJ{{\bm{J}}}
\def\mK{{\bm{K}}}
\def\mL{{\bm{L}}}
\def\mM{{\bm{M}}}
\def\mN{{\bm{N}}}
\def\mO{{\bm{O}}}
\def\mP{{\bm{P}}}
\def\mQ{{\bm{Q}}}
\def\mR{{\bm{R}}}
\def\mS{{\bm{S}}}
\def\mT{{\bm{T}}}
\def\mU{{\bm{U}}}
\def\mV{{\bm{V}}}
\def\mW{{\bm{W}}}
\def\mX{{\bm{X}}}
\def\mY{{\bm{Y}}}
\def\mZ{{\bm{Z}}}
\def\mBeta{{\bm{\beta}}}
\def\mPhi{{\bm{\Phi}}}
\def\mLambda{{\bm{\Lambda}}}
\def\mSigma{{\bm{\Sigma}}}

% Tensor
\DeclareMathAlphabet{\mathsfit}{\encodingdefault}{\sfdefault}{m}{sl}
\SetMathAlphabet{\mathsfit}{bold}{\encodingdefault}{\sfdefault}{bx}{n}
\newcommand{\tens}[1]{\bm{\mathsfit{#1}}}
\def\tA{{\tens{A}}}
\def\tB{{\tens{B}}}
\def\tC{{\tens{C}}}
\def\tD{{\tens{D}}}
\def\tE{{\tens{E}}}
\def\tF{{\tens{F}}}
\def\tG{{\tens{G}}}
\def\tH{{\tens{H}}}
\def\tI{{\tens{I}}}
\def\tJ{{\tens{J}}}
\def\tK{{\tens{K}}}
\def\tL{{\tens{L}}}
\def\tM{{\tens{M}}}
\def\tN{{\tens{N}}}
\def\tO{{\tens{O}}}
\def\tP{{\tens{P}}}
\def\tQ{{\tens{Q}}}
\def\tR{{\tens{R}}}
\def\tS{{\tens{S}}}
\def\tT{{\tens{T}}}
\def\tU{{\tens{U}}}
\def\tV{{\tens{V}}}
\def\tW{{\tens{W}}}
\def\tX{{\tens{X}}}
\def\tY{{\tens{Y}}}
\def\tZ{{\tens{Z}}}


% Graph
\def\gA{{\mathcal{A}}}
\def\gB{{\mathcal{B}}}
\def\gC{{\mathcal{C}}}
\def\gD{{\mathcal{D}}}
\def\gE{{\mathcal{E}}}
\def\gF{{\mathcal{F}}}
\def\gG{{\mathcal{G}}}
\def\gH{{\mathcal{H}}}
\def\gI{{\mathcal{I}}}
\def\gJ{{\mathcal{J}}}
\def\gK{{\mathcal{K}}}
\def\gL{{\mathcal{L}}}
\def\gM{{\mathcal{M}}}
\def\gN{{\mathcal{N}}}
\def\gO{{\mathcal{O}}}
\def\gP{{\mathcal{P}}}
\def\gQ{{\mathcal{Q}}}
\def\gR{{\mathcal{R}}}
\def\gS{{\mathcal{S}}}
\def\gT{{\mathcal{T}}}
\def\gU{{\mathcal{U}}}
\def\gV{{\mathcal{V}}}
\def\gW{{\mathcal{W}}}
\def\gX{{\mathcal{X}}}
\def\gY{{\mathcal{Y}}}
\def\gZ{{\mathcal{Z}}}

% Sets
\def\sA{{\mathbb{A}}}
\def\sB{{\mathbb{B}}}
\def\sC{{\mathbb{C}}}
\def\sD{{\mathbb{D}}}
% Don't use a set called E, because this would be the same as our symbol
% for expectation.
\def\sF{{\mathbb{F}}}
\def\sG{{\mathbb{G}}}
\def\sH{{\mathbb{H}}}
\def\sI{{\mathbb{I}}}
\def\sJ{{\mathbb{J}}}
\def\sK{{\mathbb{K}}}
\def\sL{{\mathbb{L}}}
\def\sM{{\mathbb{M}}}
\def\sN{{\mathbb{N}}}
\def\sO{{\mathbb{O}}}
\def\sP{{\mathbb{P}}}
\def\sQ{{\mathbb{Q}}}
\def\sR{{\mathbb{R}}}
\def\sS{{\mathbb{S}}}
\def\sT{{\mathbb{T}}}
\def\sU{{\mathbb{U}}}
\def\sV{{\mathbb{V}}}
\def\sW{{\mathbb{W}}}
\def\sX{{\mathbb{X}}}
\def\sY{{\mathbb{Y}}}
\def\sZ{{\mathbb{Z}}}

% Entries of a matrix
\def\emLambda{{\Lambda}}
\def\emA{{A}}
\def\emB{{B}}
\def\emC{{C}}
\def\emD{{D}}
\def\emE{{E}}
\def\emF{{F}}
\def\emG{{G}}
\def\emH{{H}}
\def\emI{{I}}
\def\emJ{{J}}
\def\emK{{K}}
\def\emL{{L}}
\def\emM{{M}}
\def\emN{{N}}
\def\emO{{O}}
\def\emP{{P}}
\def\emQ{{Q}}
\def\emR{{R}}
\def\emS{{S}}
\def\emT{{T}}
\def\emU{{U}}
\def\emV{{V}}
\def\emW{{W}}
\def\emX{{X}}
\def\emY{{Y}}
\def\emZ{{Z}}
\def\emSigma{{\Sigma}}

% entries of a tensor
% Same font as tensor, without \bm wrapper
\newcommand{\etens}[1]{\mathsfit{#1}}
\def\etLambda{{\etens{\Lambda}}}
\def\etA{{\etens{A}}}
\def\etB{{\etens{B}}}
\def\etC{{\etens{C}}}
\def\etD{{\etens{D}}}
\def\etE{{\etens{E}}}
\def\etF{{\etens{F}}}
\def\etG{{\etens{G}}}
\def\etH{{\etens{H}}}
\def\etI{{\etens{I}}}
\def\etJ{{\etens{J}}}
\def\etK{{\etens{K}}}
\def\etL{{\etens{L}}}
\def\etM{{\etens{M}}}
\def\etN{{\etens{N}}}
\def\etO{{\etens{O}}}
\def\etP{{\etens{P}}}
\def\etQ{{\etens{Q}}}
\def\etR{{\etens{R}}}
\def\etS{{\etens{S}}}
\def\etT{{\etens{T}}}
\def\etU{{\etens{U}}}
\def\etV{{\etens{V}}}
\def\etW{{\etens{W}}}
\def\etX{{\etens{X}}}
\def\etY{{\etens{Y}}}
\def\etZ{{\etens{Z}}}

% The true underlying data generating distribution
\newcommand{\pdata}{p_{\rm{data}}}
\newcommand{\ptarget}{p_{\rm{target}}}
\newcommand{\pprior}{p_{\rm{prior}}}
\newcommand{\pbase}{p_{\rm{base}}}
\newcommand{\pref}{p_{\rm{ref}}}

% The empirical distribution defined by the training set
\newcommand{\ptrain}{\hat{p}_{\rm{data}}}
\newcommand{\Ptrain}{\hat{P}_{\rm{data}}}
% The model distribution
\newcommand{\pmodel}{p_{\rm{model}}}
\newcommand{\Pmodel}{P_{\rm{model}}}
\newcommand{\ptildemodel}{\tilde{p}_{\rm{model}}}
% Stochastic autoencoder distributions
\newcommand{\pencode}{p_{\rm{encoder}}}
\newcommand{\pdecode}{p_{\rm{decoder}}}
\newcommand{\precons}{p_{\rm{reconstruct}}}

\newcommand{\laplace}{\mathrm{Laplace}} % Laplace distribution

\newcommand{\E}{\mathbb{E}}
\newcommand{\Ls}{\mathcal{L}}
\newcommand{\R}{\mathbb{R}}
\newcommand{\emp}{\tilde{p}}
\newcommand{\lr}{\alpha}
\newcommand{\reg}{\lambda}
\newcommand{\rect}{\mathrm{rectifier}}
\newcommand{\softmax}{\mathrm{softmax}}
\newcommand{\sigmoid}{\sigma}
\newcommand{\softplus}{\zeta}
\newcommand{\KL}{D_{\mathrm{KL}}}
\newcommand{\Var}{\mathrm{Var}}
\newcommand{\standarderror}{\mathrm{SE}}
\newcommand{\Cov}{\mathrm{Cov}}
% Wolfram Mathworld says $L^2$ is for function spaces and $\ell^2$ is for vectors
% But then they seem to use $L^2$ for vectors throughout the site, and so does
% wikipedia.
\newcommand{\normlzero}{L^0}
\newcommand{\normlone}{L^1}
\newcommand{\normltwo}{L^2}
\newcommand{\normlp}{L^p}
\newcommand{\normmax}{L^\infty}

\newcommand{\parents}{Pa} % See usage in notation.tex. Chosen to match Daphne's book.

\DeclareMathOperator*{\argmax}{arg\,max}
\DeclareMathOperator*{\argmin}{arg\,min}

\DeclareMathOperator{\sign}{sign}
\DeclareMathOperator{\Tr}{Tr}
\let\ab\allowbreak


% \iclrfinalcopy

\usepackage{hyperref}
\usepackage{url}

% for figures
\usepackage{graphicx}
\usepackage{subfigure}
\usepackage{wrapfig}

% for equation
\usepackage{amsmath}
\usepackage{amstext}
\usepackage{amsfonts}
\usepackage{bm}

\usepackage{bbm}
% for table
\usepackage{multirow}
\usepackage{booktabs}
\usepackage{array}
\usepackage{caption}
\usepackage{multirow}
\usepackage{array}
\usepackage{caption}
\usepackage{color}
\usepackage{colortbl}
\usepackage{tablefootnote}
\usepackage{adjustbox}

% for algorithm
% \usepackage{caption}
\usepackage{algorithm}
\usepackage{algpseudocode}
\newcommand{\mycolor}[1]{\textcolor[RGB]{64,101,149}{#1}}
\newcommand{\mydarkcolor}[1]{\textcolor[RGB]{64,101,149}{#1}}
\algnewcommand{\LineComment}[1]{\Statex ~~~~~~\textsc{//}~\textit{#1}}

%for itemize
\usepackage{enumitem}
\setenumerate[1]{itemsep=0pt,partopsep=0pt,parsep=\parskip,topsep=5pt}
\setitemize[1]{itemsep=0pt,partopsep=0pt,parsep=\parskip,topsep=5pt}
\setdescription{itemsep=0pt,partopsep=0pt,parsep=\parskip,topsep=5pt}

% for highlight
\usepackage{soul}

% for notation
% \usepackage[mathscr]{euscript}
% \newcommand{\M}{\mathscr{M}}
% \newcommand{\M}{\mathcal{M}}

%for itemize
\usepackage{amssymb}  
\usepackage{pifont}
\newcommand{\cmark}{\ding{51}}
\newcommand{\xmark}{\ding{55}}
\newcommand{\greenyes}{\textcolor{green}{\ding{51}}}
\newcommand{\redno}{\textcolor{red}{\ding{55}}}
\usepackage[capitalize]{cleveref}
% for table color box
\definecolor{c0}{cmyk}{1,0.3968,0,0.2588} 
\definecolor{LightCyan}{rgb}{0.88,1,1}
\newcommand{\gray}{\cellcolor{gray!10}} % LightCyanc0!5
\usepackage{scalerel} 

%% === commands for comments ===
% \usepackage{ulem}
\usepackage[normalem]{ulem}
\usepackage{makecell} %zj
\usepackage{colortbl} %zj

\usepackage{scalerel} % for logo

%citation color
\definecolor{uclablue}{rgb}{0.15, 0.45, 0.68}
\definecolor{custommagenta}{rgb}{0.1, 0.90, 1}
% {0.85, 0.1, 0.85}
\newcommand{\blueone}{\cellcolor{uclablue!10}} 
\newcommand{\bluetwo}{\cellcolor{uclablue!20}}
\newcommand{\bluethree}{\cellcolor{uclablue!30}}
\newcommand{\bluefour}{\cellcolor{uclablue!40}}

\usepackage{xcolor}
\usepackage{hyperref}
\usepackage{tikz}

\newcommand{\rainbowurl}{%
    \textbf{\texttt{\textcolor{red!50}{h}\textcolor{orange!50}{t}\textcolor{yellow!50}{t}\textcolor{green!50}{p}\textcolor{blue!50}{s}\textcolor{purple!50}{:}//%
    \textcolor{red!50}{g}\textcolor{orange!50}{i}\textcolor{yellow!50}{t}\textcolor{green!50}{h}\textcolor{blue!50}{u}\textcolor{purple!50}{b}%
    \textcolor{red!50}{.}\textcolor{orange!50}{c}\textcolor{yellow!50}{o}\textcolor{green!50}{m}/%
    \textcolor{blue!50}{j}\textcolor{purple!50}{u}\textcolor{red!50}{n}\textcolor{orange!50}{z}\textcolor{yellow!50}{h}\textcolor{green!50}{a}\textcolor{blue!50}{n}\textcolor{purple!50}{g}%
    \textcolor{red!50}{-}\textcolor{orange!50}{z}\textcolor{yellow!50}{j}/%
    \textcolor{green!50}{L}\textcolor{blue!50}{o}\textcolor{purple!50}{R}\textcolor{red!50}{A}\textcolor{orange!50}{M}}%
}}

\hypersetup{
    breaklinks,
    citecolor=uclablue,
    colorlinks=true,
    linkcolor=uclablue
}


\title{
\scalerel*{\includegraphics{fig/loram_logo.png}}{{\rule{1.6ex}{1.6ex}}}
Train Small, Infer Large: Memory-Efficient LoRA Training for Large Language Models
}


\author{Jun Zhang$^{1,3} \thanks{\ Work done during an internship at OPPO AI Center.}$, Jue Wang$^{1,3}$, Huan Li$^{1,2}\thanks{\ Huan Li and Lidan Shou are the corresponding authors.}$ , Lidan Shou$^{1,2 {\dagger}}$, Ke Chen$^{1,2}$, \\ \textbf{Yang You$^{4}$, Guiming Xie$^{5}$, Xuejian Gong$^{5}$, and Kunlong Zhou$^{5}$} \\
  $^1$The State Key Laboratory of Blockchain and Data Security, Zhejiang University \\
  $^2$Hangzhou High-Tech Zone (Binjiang) Institute of Blockchain and Data Security \\
  $^3$College of Computer Science and Technology, Zhejiang University \\
  $^4$Department of Computer Science, National University of Singapore \\
  $^5$AI Center, Guangdong OPPO Mobile Telecommunications Corp., Ltd. \\
  \texttt{\{zj.cs,zjuwangjue,lihuan.cs,should,chenk\}@zju.edu.cn},\\ \texttt{youy@comp.nus.edu.sg}, \texttt{\{xieguiming,gongxuejian,zhoukunlong\}@oppo.com}
}
% \thanks{ Use footnote for providing further information
% about author (webpage, alternative address)---\emph{not} for acknowledging
% funding agencies.  Funding acknowledgements go at the end of the paper.} \\
% The State Key Laboratory of Blockchain and Data Security\\
% Zhejiang University\\
% Pittsburgh, PA 15213, USA \\
% \texttt{\{zj.cs,brain,jen\}@cs.cranberry-lemon.edu} \\
% \And
% Ji Q. Ren \& Yevgeny LeNet \\
% Department of Computational Neuroscience \\
% University of the Witwatersrand \\
% Joburg, South Africa \\
% \texttt{\{robot,net\}@wits.ac.za} \\
% \AND
% Coauthor \\
% Affiliation \\
% Address \\
% \texttt{email}

% The \author macro works with any number of authors. There are two commands
% used to separate the names and addresses of multiple authors: \And and \AND.
%
% Using \And between authors leaves it to \LaTeX{} to determine where to break
% the lines. Using \AND forces a linebreak at that point. So, if \LaTeX{}
% puts 3 of 4 authors names on the first line, and the last on the second
% line, try using \AND instead of \And before the third author name.

\usepackage{xspace}
\newcommand{\method}{\textsc{LoRAM}\xspace}
\newcommand{\methodrand}{\textsc{LoRAM-Rand}\xspace}
\newcommand{\methodstru}{\textsc{LoRAM-Stru}\xspace}
\newcommand{\methodsemi}{\textsc{LoRAM-Semi}\xspace}
\newcommand{\methodunst}{\textsc{LoRAM-Unst}\xspace}
\newcommand{\Qmethod}{\textsc{QLoRAM}\xspace}
\newcommand{\Qmethodrand}{\textsc{QLoRAM-Rand}\xspace}
\newcommand{\Qmethodstru}{\textsc{QLoRAM-Stru}\xspace}

\iclrfinalcopy % Uncomment for camera-ready version, but NOT for submission.
\begin{document}

\maketitle

\begin{abstract}
Large Language Models (LLMs) have significantly advanced natural language processing with exceptional task generalization capabilities. 
% Parameter-Efficient Fine-Tuning (PEFT), exemplified by LoRA, 
Low-Rank Adaption (LoRA) offers a cost-effective fine-tuning solution, freezing the original model parameters and training only lightweight, low-rank adapter matrices.
However, the memory footprint of LoRA is largely dominated by the original model parameters.
% \textcolor{custommagenta}{
% To further reduce the fine-tuning memory footprint, 
To mitigate this, 
we propose \method, a memory-efficient LoRA training scheme
founded on the intuition that many neurons in over-parameterized LLMs have low training utility but are essential for inference. 
\method presents a unique twist: it trains on a pruned (small) model to obtain pruned low-rank matrices, 
which are then
% dimensionally 
recovered and utilized with the original (large) model for inference.
Additionally, minimal-cost continual pre-training,
% }
performed by the model publishers in advance, aligns the knowledge discrepancy between pruned and original models.
Our extensive experiments demonstrate the efficacy of \method across various pruning strategies and downstream tasks. For a model with 70 billion parameters, \method enables training on a GPU with only 20G HBM, replacing an A100-80G GPU for LoRA training and 15 GPUs for full fine-tuning. Specifically, \Qmethod implemented by structured pruning combined with 4-bit quantization, for LLaMA-3.1-70B (LLaMA-2-70B), reduces the parameter storage cost that dominates the memory usage in low-rank matrix training by 15.81×
% \footnote{The reported QLoRAM parameter reduction is based on Hugging Face’s implementation, which underestimates the impact of \texttt{NF4} quantization, implying a higher actual reduction.}
(16.95×), while achieving dominant performance gains over both the original LLaMA-3.1-70B (LLaMA-2-70B) and LoRA-trained LLaMA-3.1-8B (LLaMA-2-13B).
Code is available at \href{https://github.com/junzhang-zj/LoRAM}{\rainbowurl}.



\end{abstract}

\section{Introduction}

Deep Reinforcement Learning (DRL) has emerged as a transformative paradigm for solving complex sequential decision-making problems. By enabling autonomous agents to interact with an environment, receive feedback in the form of rewards, and iteratively refine their policies, DRL has demonstrated remarkable success across a diverse range of domains including games (\eg Atari~\citep{mnih2013playing,kaiser2020model}, Go~\citep{silver2018general,silver2017mastering}, and StarCraft II~\citep{vinyals2019grandmaster,vinyals2017starcraft}), robotics~\citep{kalashnikov2018scalable}, communication networks~\citep{feriani2021single}, and finance~\citep{liu2024dynamic}. These successes underscore DRL's capability to surpass traditional rule-based systems, particularly in high-dimensional and dynamically evolving environments.

Despite these advances, a fundamental challenge remains: DRL agents typically rely on deep neural networks, which operate as black-box models, obscuring the rationale behind their decision-making processes. This opacity poses significant barriers to adoption in safety-critical and high-stakes applications, where interpretability is crucial for trust, compliance, and debugging. The lack of transparency in DRL can lead to unreliable decision-making, rendering it unsuitable for domains where explainability is a prerequisite, such as healthcare, autonomous driving, and financial risk assessment.

To address these concerns, the field of Explainable Deep Reinforcement Learning (XRL) has emerged, aiming to develop techniques that enhance the interpretability of DRL policies. XRL seeks to provide insights into an agent’s decision-making process, enabling researchers, practitioners, and end-users to understand, validate, and refine learned policies. By facilitating greater transparency, XRL contributes to the development of safer, more robust, and ethically aligned AI systems.

Furthermore, the increasing integration of Reinforcement Learning (RL) with Large Language Models (LLMs) has placed RL at the forefront of natural language processing (NLP) advancements. Methods such as Reinforcement Learning from Human Feedback (RLHF)~\citep{bai2022training,ouyang2022training} have become essential for aligning LLM outputs with human preferences and ethical guidelines. By treating language generation as a sequential decision-making process, RL-based fine-tuning enables LLMs to optimize for attributes such as factual accuracy, coherence, and user satisfaction, surpassing conventional supervised learning techniques. However, the application of RL in LLM alignment further amplifies the explainability challenge, as the complex interactions between RL updates and neural representations remain poorly understood.

This survey provides a systematic review of explainability methods in DRL, with a particular focus on their integration with LLMs and human-in-the-loop systems. We first introduce fundamental RL concepts and highlight key advances in DRL. We then categorize and analyze existing explanation techniques, encompassing feature-level, state-level, dataset-level, and model-level approaches. Additionally, we discuss methods for evaluating XRL techniques, considering both qualitative and quantitative assessment criteria. Finally, we explore real-world applications of XRL, including policy refinement, adversarial attack mitigation, and emerging challenges in ensuring interpretability in modern AI systems. Through this survey, we aim to provide a comprehensive perspective on the current state of XRL and outline future research directions to advance the development of interpretable and trustworthy DRL models.
% \section{Related Work}

\subsection{Large Language Models in Biosciences}
Large language models (LLMs) have emerged as powerful tools for natural language comprehension and generation~\cite{llms-survey}. Beyond their application in traditional natural language tasks, there is a growing interest in leveraging LLMs to accelerate scientific research. Early studies revealed that general-purpose LLMs, owing to their rich pre-training data, exhibit promise across various research domains~\cite{ai4science}. Subsequent efforts have focused on directly training LLMs using domain-specific data, aiming to extend the transfer learning paradigm from natural language processing (NLP) to biosciences. This body of work primarily falls into three categories: molecular LLMs, protein LLMs, and genomic LLMs.

For molecular modeling, extensive work has been conducted on training with various molecular string representations, such as SMILES~\cite{Smiles-bert,space-of-chemical,large-scale-chemical}, SELFIES~\cite{SELFIES,chemberta,chemberta2}, and InChI~\cite{inchi}. Additionally, several studies address the modeling of molecular 2D~\cite{mol-2d} and 3D structures~\cite{uni-mol} to capture more detailed molecular characteristics. In the realm of protein LLMs, related work~\cite{msa-transformer,esm2,Prottrans} mainly concentrates on modeling the primary structure of proteins (amino acid sequences), providing a solid foundation for protein structure prediction~\cite{AlphaFold2,AlphaFold3}. For genomic sequences, numerous studies have attempted to leverage the power of LLMs for improved genomic analysis and understanding. These efforts predominantly involve training models on DNA~\cite{BPNet,DNABERT,enformer,nucleotide-transformer,DNABERT-2,GROVER,gena-lm,Caduceus,dnagpt,megaDNA,HyenaDNA,Evo} and RNA~\cite{RNAErnie,uni-rna,Rinalmo} sequences. In the following section, we delve deeper into genomic LLMs specifically designed for DNA sequence modeling.

\subsection{DNA Language Models}
In the early stages, \citeauthor{BPNet} introduced the BPNet convolutional architecture to learn transcription factor binding patterns and their syntax in a supervised manner. Prior to the emergence of large-scale pre-training, BPNet was widely used in genomics for supervised learning on relatively small datasets. With the advent of BERT~\cite{BERT}, DNABERT~\cite{DNABERT} pioneered the application of pre-training on the human genome using K-mer tokenizers. To effectively capture long-range interactions, Enformer~\cite{enformer} advanced human genome modeling by incorporating convolutional downsampling into transformer architectures.

Following these foundational works, numerous models based on the transformer encoder architecture have emerged. A notable example is the Nucleotide Transformer (NT)~\cite{nucleotide-transformer}, which scales model parameters from 100 million to 2.5 billion and includes a diverse set of multispecies genomes. Recent studies, DNABERT-2~\cite{DNABERT-2} and GROVER~\cite{GROVER}, have investigated optimal tokenizer settings for masked language modeling, concluding that Byte Pair Encoding (BPE) is better suited for masked DNA LLMs. The majority of these models face the limitation of insufficient context length, primarily due to the high computational cost associated with extending the context length in the transformer architecture. To address this limitation, GENA-LM~\cite{gena-lm} employs sparse attention, and Caduceus~\cite{Caduceus} uses the more lightweight BiMamba architecture~\cite{Mamba}, both trained on the human genome.

Although these masked DNA LLMs effectively understand and predict DNA sequences, they lack generative capabilities, and generative DNA LLMs remain in the early stages of development. An early preprint~\cite{dnagpt} introduced DNAGPT, which learns mammalian genomic structures through three pre-training tasks, including next token prediction. Recent works, such as HyenaDNA~\cite{HyenaDNA} and megaDNA~\cite{megaDNA}, achieve longer context lengths by employing the Hyena~\cite{Hyena} and multiscale transformer architectures respectively, though they are significantly limited by their data and model scales. A more recent influential study, Evo~\cite{Evo}, trained on an extensive dataset of prokaryotic and viral genomes, has garnered widespread attention for its success in designing CRISPR-Cas molecular complexes, thus demonstrating the practical utility of generative DNA LLMs in the genomic field.

\section{\method}
\label{sec:tokenskip}
\begin{figure*}[t]
\centering
\includegraphics[width=0.95\textwidth]{fig/tokenskip.pdf}
\caption{Illustration of \method. During the training phase, \method first generates CoT trajectories from the target LLM. These CoTs are then compressed to a specified ratio, $\gamma$, based on the semantic importance of tokens. \method fine-tunes the target LLM using compressed CoTs, enabling controllable CoT inference at the desired $\gamma$.}
\label{fig:tokenskip}
\end{figure*}

We introduce \method, a simple yet effective approach that enables LLMs to skip less important tokens, enabling controllable CoT compression with adjustable ratios. This section demonstrates the details of our methodology, including token pruning~(\S\ref{sec:token-pruning}), training~(\S\ref{sec:training}), and inference~(\S\ref{sec:inference}).

\subsection{Token Pruning}
\label{sec:token-pruning}
The key insight behind \method is that ``\textit{each reasoning token contributes differently to deriving the answer.}'' To enhance CoT efficiency, we propose to trim redundant tokens from LLM CoT outputs and fine-tune LLMs using these trimmed CoT trajectories. The token pruning process is guided by the concept of \textit{token importance}, as detailed in Section~\ref{sec:token-importance}. 

Specifically, given a target LLM $\M$, one of its CoT trajectories $\boldsymbol{c}=\left\{c_i\right\}_{i=1}^{m}$, and a desired compression ratio $\gamma \in \left[0,1\right]$, \method first calculates the semantic importance of each CoT token $I\left(c\right)$, as defined in Eq~(\ref{eq:llmlingua2}). The tokens are then ranked in descending order based on their importance values. Next, the $\gamma$-th percentile of these importance values is computed, representing the threshold for token pruning:
\begin{equation}
I_\gamma=\mathrm{np.percentile}\left(\left[I\left(c_1\right), . ., I\left(c_m\right)\right], \gamma\right).
\end{equation}
Finally, CoT tokens with an importance value greater than or equal to $I_\gamma$ are retained in the compressed CoT trajectory:
\begin{equation}
\widetilde{\boldsymbol{c}}=\left\{c_i \mid I\left(c_i\right) \geq I_\gamma\right\}, 1 \leq i \leq m.
\end{equation}

\subsection{Training}
\label{sec:training}
Given a training dataset $\mathcal{D}$ with $N$ samples and a target LLM $\M$, we first obtain $N$ CoT trajectories with $\M$. Then, we filter out trajectories with incorrect answers to ensure the high quality of training data. For the remaining CoT trajectories, we prune each CoT with a randomly selected compression ratio $\gamma$, as demonstrated in Section~\ref{sec:token-pruning}. For each $\langle\text{question}, \text{compressed CoT}, \text{answer}\rangle$, we inserted the compression ratio $\gamma$ after the question. Finally, each training sample is formatted as follows: 
\begin{equation}
\nonumber
    \mathcal{Q} \ \mathrm{[EOS]} \ \gamma \ \mathrm{[EOS]} \ \mathrm{Compressed\ CoT} \ \mathcal{A},
\end{equation}
where $\langle\mathcal{Q}, \mathcal{A}\rangle$ indicates the $\langle\text{question}, \text{answer}\rangle$ pair. Formally, given a question $\boldsymbol{x}$, compression ratio $\gamma$, and the output sequence $\boldsymbol{y}=\left\{y_i\right\}_{i=1}^{l}$, which includes the compressed CoT $\widetilde{\boldsymbol{c}}$ and the answer $\boldsymbol{a}$, we fine-tunes the target LLM $\M$, enabling it to perform chain-of-thought in a compressed pattern by minimizing
\begin{equation}
\mathcal{L}=\sum_{i=1}^{l} \log P\left(y_{i} \mid \bm{x}, \gamma, \bm{y}_{<i}; \bm{\theta}_{\M}\right),
\end{equation}
where $\bm{y} =\left\{\widetilde{c}_1, \cdots,\widetilde{c}_{m^{\prime}}, a_1, \cdots, a_t  \right\}$. Note that the compression is performed solely on CoT sequences, and we keep the answer $\boldsymbol{a}=\left\{a_i\right\}_{i=1}^{t}$ unchanged. To preserve LLMs' reasoning capabilities, we also include a portion of the original CoT trajectories in the training data, with $\gamma$ set to 1.

\subsection{Inference}
\label{sec:inference}
The inference of \method follows autoregressive decoding. Compared to original CoT outputs that may contain redundancy, \method facilitates LLMs to skip \textit{unimportant} tokens during the chain-of-thought process, thereby enhancing reasoning efficiency. Formally, given a question $\boldsymbol{x}$ and the compression ratio $\gamma$, the input prompt of \method follows the same format adopted in fine-tuning, which is $\mathcal{Q} \ \mathrm{[EOS]} \ \gamma \ \mathrm{[EOS]}$. The LLM $\M$ sequentially predicts the output sequence $\hat{\bm{y}}$:
\begin{equation}
\nonumber
\hat{\boldsymbol{y}}=\arg \max _{\boldsymbol{y}^*} \sum_{j=1}^{l^{\prime}} \log P\left(y_j \mid \boldsymbol{x}, \gamma, \boldsymbol{y}_{<j}; \bm{\theta}_{\M}\right),
\end{equation}
where $\hat{\bm{y}} =\left\{\hat{c}_1, \cdots,\hat{c}_{m^{\prime\prime}}, \hat{a}_1, \cdots, \hat{a}_{t^{\prime}}  \right\}$ denotes the output sequence, which includes CoT tokens $\hat{\bm{c}}$ and the answer $\bm{\hat{a}}$. We illustrate the training and inference process of \method in Figure~\ref{fig:tokenskip}. 

\section{Experiments}
\label{sec:experiments}

\subsection{Next K-mer Prediction}
\label{sec:kmer_predition}
\begin{figure}[t]
    \centering
    \includegraphics[width=0.5\textwidth]{figures/pdf/kmer_prediction_main_text.pdf}
    \caption{Evaluation of next K-mer prediction. (A) Accuracy of the next K-mer prediction task across various tokenizers and input token lengths. (B) Comparison of the \textbf{Gener}\textit{ator} against baseline models on a dataset comprised exclusively mammalian DNA.}
    \label{fig:kmer_main}
\end{figure}

As mentioned in \textit{Sec.} \ref{sec:tokenization}, we conducted extensive experiments to explore the most suitable tokenizer for training causal DNA language models. This was achieved by training multiple models on identical datasets, each employing a different tokenizer. All models share the same architecture as the \textbf{Gener}\textit{ator} and are uniformly compared at 32,000 training steps. We employed the accuracy of the next K-mer prediction task as our evaluation metric. This zero-shot task facilitates a direct assessment of the pre-trained model quality, ensuring equitable comparisons across various tokenizers. As depicted in \textit{Fig.} \ref{fig:kmer_main}A, the tested tokenizers include BPE tokenizers with vocabulary sizes ranging from 512 to 8192, and K-mer tokenizers with K values from 1 to 8 (noting that the single nucleotide tokenizer corresponds to a K-mer tokenizer with K=1). Overall, K-mer tokenizers demonstrate superior performance compared to BPE tokenizers. Among the K-mer tokenizers, the 6-mer tokenizer is selected for its robust performance with limited input tokens and its ability to maintain top-tier performance as the number of input tokens increases.

Moreover, we evaluated the performance of Mamba \cite{Mamba,Mamba-2}, recognized for its capacity in handling long-context pre-training. To adequately assess its capabilities, we configured a Mamba model utilizing the single nucleotide tokenizer with 1.2B parameters and a context length of 98k bp. The Mamba model is compared to the 1-mer and 6-mer models under varied configurations. The comparison with the 1-mer model is straightforward; the Mamba model (denoted as Mamba\texttimes1 in \textit{Fig.} \ref{fig:kmer_main}A) exhibits slightly better performance with fewer input tokens but underperforms as the token count increases. Despite Mamba's context length being six times that of the 1-mer model, this feature does not translate into improved performance. This might suggest that Mamba's renowned ability to handle long-context pre-training primarily refers to cost-effective training rather than enhanced model performance \cite{Empirical, DeciMamba}. To compare against the 6-mer model, we adjust the input token count for the Mamba model by a factor of six (denoted as Mamba\texttimes6) to compare the models on the same base-pair basis. In this context, Mamba\texttimes6 shows slightly better performance with fewer input tokens; however, it rapidly lags as the token count increases. These findings collectively indicate that a transformer decoder architecture paired with a 6-mer tokenizer provides the most effective approach for training causal DNA language models, aligning with the configuration of the \textbf{Gener}\textit{ator}.

We further compared the \textbf{Gener}\textit{ator} model with other baseline models to evaluate their generative capabilities. As illustrated in \textit{Fig.} \ref{fig:kmer_main}B, we assess model performance using a dataset composed exclusively of mammalian DNA, given that HyenaDNA and GROVER are trained solely on human genomes. The \textbf{Gener}\textit{ator} significantly outperforms other baseline models, including its variant, \textbf{Gener}\textit{ator}-All, which incorporates pre-training on non-gene regions. This suggests that the gene sequence training strategy, which emphasizes semantically rich regions, provides a more effective training scheme compared to the conventional whole sequence training. This effectiveness is likely due to the sparsity of gene segments in the whole genome (less than 10\%) and the disproportionate importance of these segments. Among the other baseline models, NT-multi demonstrates the best performance, likely attributable to its extensive model scale (2.5B parameters), yet it still lags significantly behind the \textbf{Gener}\textit{ator}. This result aligns with expectations, as the MLM training paradigm is recognized for its limitations in generative capabilities. Meanwhile, HyenaDNA, despite utilizing the NTP training paradigm, does not show improved performance compared to other masked language models, likely due to its overly small model size (55M parameters), insufficient for exhibiting robust generative abilities. This comparison underscores the critical role of the \textbf{Gener}\textit{ator} in bridging the gap for large-scale generative DNA language models within the eukaryotic domain.

Due to space constraints, we have chosen only to demonstrate specific examples with mammalian DNA data and a fixed K-mer prediction length of 16 bp in \textit{Fig.} \ref{fig:kmer_main}. A more comprehensive analysis across various taxonomic groups and K-mer lengths is provided in the appendix.

\subsection{Benchmark Evaluations}
In this section, we compare the \textbf{Gener}\textit{ator} with state-of-the-art genomic foundation models: Enformer~\cite{enformer}, DNABERT-2, HyenaDNA, Nucleotide Transformer, Caduceus, and GROVER, across various benchmark tasks. To ensure a fair comparison, we uniformly fine-tune each model and perform a 10-fold cross-validation on all datasets. For each model on each dataset, we conduct a hyperparameter search, exhaustively tuning learning rates in $\{1e^{-5}, 2e^{-5}, 5e^{-5}, \ldots, 1e^{-3}, 2e^{-3}, 5e^{-3}\}$ and batch sizes in $\{64, 128, 256, 512\}$. Detailed hyperparameter settings and implementation specifics are provided in the appendix.

\paragraph{Nucleotide Transformer Tasks}
Since the NT task dataset was revised recently~\cite{nucleotide-transformer}, we conducted experiments on both the original and revised datasets. The results for the revised NT tasks are provided in Table~\ref{tab:nucleotide_transformer_tasks_revised}, and the results for the original NT tasks are provided in Table~\ref{tab:nucleotide_transformer_tasks}. Overall, the \textbf{Gener}\textit{ator} outperforms other baseline models. However, the \textbf{Gener}\textit{ator}-All variant shows some performance decline. Notably, despite its earlier release, Enformer continues to deliver competitive results in chromatin profile and regulatory element tasks. This performance could be attributed to its original training in a supervised manner specifically for chromatin and gene expression tasks. The latest release of Nucleotide Transformer, NT-v2, although smaller in size (500M), demonstrates enhanced performance compared to NT-multi (2.5B). In contrast, DNABERT-2 and GROVER, which utilize BPE tokenizers, along with HyenaDNA and Caduceus, which employ the finer-grained single nucleotide tokenizer, do not show distinct performance advantages, likely due to the limited model scope and data scale.

\paragraph{Genomic Benchmarks}
We also conducted a comparative analysis on the Genomic Benchmarks~\cite{genomic-benchmarks}, which primarily focus on the human genome. The evaluation results are provided in Table~\ref{tab:genomic_benchmarks}. Overall, the \textbf{Gener}\textit{ator} still outperforms other models. However, it is worth noting that the Caduceus models also exhibit comparable performance while being significantly smaller (8M). This is likely due to the fact that Caduceus models are trained exclusively on the human genome, making them efficient and compact. Nevertheless, this exclusivity may limit their generalizability to other genomic contexts.

\paragraph{Gener Tasks} 
Lastly, we evaluated the newly proposed Gener tasks, which focus on assessing genomic context comprehension across various sequence lengths and organisms. As shown in Table~\ref{tab:gener_tasks}, the \textbf{Gener}\textit{ator} achieves the best performance on both gene and taxonomic classification tasks, with NT-v2 also demonstrating similar results. Further details on the evaluation of Gener tasks, including visualizations of confusion matrices, are provided in the appendix. The superior performance of the \textbf{Gener}\textit{ator} and NT-v2 is likely due to their pre-training on multispecies datasets. In contrast, despite also being trained on multispecies data, DNABERT-2 exhibits noticeable performance degradation. This may be attributed to its limited model size (117M for DNABERT-2, 500M for NT-v2, and 1.2B for \textbf{Gener}\textit{ator}) and shorter context length (3k for DNABERT-2, 12k for NT-v2, and 98k for \textbf{Gener}\textit{ator}). Other models, such as HyenaDNA and Caduceus, although trained exclusively on the human genome, still exhibit relevant generalizability on both tasks after fine-tuning, attributable to their long-context capacity (\textgreater 100k). GROVER, on the other hand, significantly lags behind in taxonomic classification due to its limited context length (3k).

\begin{table*}[!htb]
\small
\renewcommand{\arraystretch}{1}
\centering
\caption{Evaluation of the revised Nucleotide Transformer tasks. The reported values represent the Matthews correlation coefficient (MCC) averaged over 10-fold cross-validation, with the standard error in parentheses.}
\resizebox{\textwidth}{!}{
\begin{tabular}{lcccccccccc}
\toprule
& Enformer & DNABERT-2 & HyenaDNA & NT-multi & NT-v2 & Caduceus-Ph & Caduceus-PS & GROVER & \textbf{Gener}\textit{ator} & \textbf{Gener}\textit{ator}-All \\
& (252M) & (117M) & (55M) & (2.5B) & (500M) & (8M) & (8M) & (87M) & (1.2B) & (1.2B) \\
\midrule
H2AFZ          & 0.522 (0.019) & 0.490 (0.013) & 0.455 (0.015) & 0.503 (0.010) & \underline{0.524 (0.008)} & 0.417 (0.016) & 0.501 (0.013) & 0.509 (0.013) & \textbf{0.529 (0.009)} & 0.506 (0.019) \\
H3K27ac        & \underline{0.520 (0.015)} & 0.491 (0.010) & 0.423 (0.017) & 0.481 (0.020) & 0.488 (0.013) & 0.464 (0.018) & 0.464 (0.022) & 0.489 (0.023) & \textbf{0.546 (0.015)} & 0.496 (0.014) \\
H3K27me3       & 0.552 (0.007) & 0.599 (0.010) & 0.541 (0.018) & 0.593 (0.016) & \underline{0.610 (0.006)} & 0.547 (0.010) & 0.561 (0.036) & 0.600 (0.008) & \textbf{0.619 (0.008)} & 0.590 (0.014) \\
H3K36me3       & 0.567 (0.017) & \underline{0.637 (0.007)} & 0.543 (0.010) & 0.635 (0.016) & 0.633 (0.015) & 0.543 (0.009) & 0.602 (0.008) & 0.585 (0.008) & \textbf{0.650 (0.006)} & 0.621 (0.013) \\
H3K4me1        & \textbf{0.504 (0.021)} & \underline{0.490 (0.008)} & 0.430 (0.014) & 0.481 (0.012) & \underline{0.490 (0.017)} & 0.411 (0.012) & 0.434 (0.030) & 0.468 (0.011) & \textbf{0.504 (0.010)} & \underline{0.490 (0.016)} \\
H3K4me2        & \textbf{0.626 (0.015)} & 0.558 (0.013) & 0.521 (0.024) & 0.552 (0.022) & 0.552 (0.013) & 0.480 (0.013) & 0.526 (0.035) & 0.558 (0.012) & \underline{0.607 (0.010)} & 0.569 (0.012) \\
H3K4me3        & 0.635 (0.019) & \underline{0.646 (0.008)} & 0.596 (0.015) & 0.618 (0.015) & 0.627 (0.020) & 0.588 (0.020) & 0.611 (0.015) & 0.634 (0.011) & \textbf{0.653 (0.008)} & 0.628 (0.018) \\
H3K9ac         & \textbf{0.593 (0.020)} & 0.564 (0.013) & 0.484 (0.022) & 0.527 (0.017) & 0.551 (0.016) & 0.514 (0.014) & 0.518 (0.018) & 0.531 (0.014) & \underline{0.570 (0.017)} & 0.556 (0.018) \\
H3K9me3        & 0.453 (0.016) & 0.443 (0.025) & 0.375 (0.026) & 0.447 (0.018) & 0.467 (0.044) & 0.435 (0.019) & 0.455 (0.019) & 0.441 (0.017) & \textbf{0.509 (0.013)} & \underline{0.480 (0.037)} \\
H4K20me1       & 0.606 (0.016) & \underline{0.655 (0.011)} & 0.580 (0.009) & 0.650 (0.014) & 0.654 (0.011) & 0.572 (0.012) & 0.590 (0.020) & 0.634 (0.006) & \textbf{0.670 (0.006)} & 0.652 (0.010) \\
Enhancer       & \textbf{0.614 (0.010)} & 0.517 (0.011) & 0.475 (0.006) & 0.527 (0.012) & 0.575 (0.023) & 0.480 (0.008) & 0.490 (0.009) & 0.519 (0.009) & \underline{0.594 (0.013)} & 0.553 (0.020) \\
Enhancer type & \textbf{0.573 (0.013)} & 0.476 (0.009) & 0.441 (0.010) & 0.484 (0.012) & 0.541 (0.013) & 0.461 (0.009) & 0.459 (0.011) & 0.481 (0.009) & \underline{0.547 (0.017)} & 0.510 (0.022) \\
Promoter all   & 0.745 (0.012) & 0.754 (0.009) & 0.693 (0.016) & 0.761 (0.009) & \underline{0.780 (0.012)} & 0.707 (0.017) & 0.722 (0.014) & 0.721 (0.011) & \textbf{0.795 (0.005)} & 0.765 (0.009) \\
Promoter non-TATA & 0.763 (0.012) & 0.769 (0.009) & 0.723 (0.013) & 0.773 (0.010) & 0.785 (0.009) & 0.740 (0.012) & 0.746 (0.009) & 0.739 (0.018) & \textbf{0.801 (0.005)} & \underline{0.786 (0.007)} \\
Promoter TATA  & 0.793 (0.026) & 0.784 (0.036) & 0.648 (0.044) & \underline{0.944 (0.016)} & 0.919 (0.028) & 0.868 (0.023) & 0.853 (0.034) & 0.891 (0.041) & \textbf{0.950 (0.009)} & 0.862 (0.024) \\
Splice acceptor & 0.749 (0.007) & 0.837 (0.006) & 0.815 (0.049) & 0.958 (0.003) & \textbf{0.965 (0.004)} & 0.906 (0.015) & 0.939 (0.012) & 0.812 (0.012) & \underline{0.964 (0.003)} & 0.951 (0.006) \\
Splice site all & 0.739 (0.011) & 0.855 (0.005) & 0.854 (0.053) & 0.964 (0.003) & \textbf{0.968 (0.003)} & 0.941 (0.006) & 0.942 (0.012) & 0.849 (0.015) & \underline{0.966 (0.003)} & 0.959 (0.003) \\
Splice donor   & 0.780 (0.007) & 0.861 (0.004) & 0.943 (0.024) & 0.970 (0.002) & \underline{0.976 (0.003)} & 0.944 (0.026) & 0.964 (0.010) & 0.842 (0.009) & \textbf{0.977 (0.002)} & 0.971 (0.002) \\
\bottomrule
\end{tabular}
}
\label{tab:nucleotide_transformer_tasks_revised}
\end{table*}
\begin{table*}[!htb]
\small
\renewcommand{\arraystretch}{1.2}
\centering
\caption{Evaluation of the original Nucleotide Transformer tasks. The reported values represent the Matthews correlation coefficient (MCC) averaged over 10-fold cross-validation, with the standard error in parentheses.}
\resizebox{\textwidth}{!}{%
\begin{tabular}{lcccccccccc}
\toprule
& Enformer & DNABERT-2 & HyenaDNA & NT-multi & NT-v2 & Caduceus-Ph & Caduceus-PS & GROVER & \textbf{Gener}\textit{ator} & \textbf{Gener}\textit{ator}-All \\
& (252M) & (117M) & (55M) & (2.5B) & (500M) & (8M) & (8M) & (87M) & (1.2B) & (1.2B) \\
\midrule
H3 & 0.724 (0.018) & 0.785 (0.012) & 0.781 (0.015) & 0.793 (0.013) & 0.788 (0.010) & 0.794 (0.012) & 0.772 (0.022) & 0.768 (0.008) & \textbf{0.806 (0.005)} & \underline{0.803 (0.007)} \\
H3K14ac & 0.284 (0.024) & 0.515 (0.009) & \textbf{0.608 (0.020)} & 0.538 (0.009) & 0.538 (0.015) & 0.564 (0.033) & 0.596 (0.038) & 0.548 (0.020) & \underline{0.605 (0.008)} & 0.580 (0.038) \\
H3K36me3 & 0.345 (0.019) & 0.591 (0.005) & 0.614 (0.014) & 0.618 (0.011) & 0.618 (0.015) & 0.590 (0.018) & 0.611 (0.048) & 0.563 (0.017) & \textbf{0.657 (0.007)} & \underline{0.631 (0.013)} \\
H3K4me1 & 0.291 (0.016) & 0.512 (0.008) & 0.512 (0.008) & 0.541 (0.005) & 0.544 (0.009) & 0.468 (0.015) & 0.487 (0.029) & 0.461 (0.018) & \textbf{0.553 (0.009)} & \underline{0.549 (0.018)} \\
H3K4me2 & 0.207 (0.021) & 0.333 (0.013) & \textbf{0.455 (0.028)} & 0.324 (0.014) & 0.302 (0.020) & 0.332 (0.034) & \underline{0.431 (0.016)} & 0.403 (0.042) & 0.424 (0.013) & 0.400 (0.015) \\
H3K4me3 & 0.156 (0.022) & 0.353 (0.021) & \textbf{0.550 (0.015)} & 0.408 (0.011) & 0.437 (0.028) & 0.490 (0.042) & \underline{0.528 (0.033)} & 0.458 (0.022) & 0.512 (0.009) & 0.473 (0.047) \\
H3K79me3 & 0.498 (0.013) & 0.615 (0.010) & 0.669 (0.014) & 0.623 (0.010) & 0.621 (0.012) & 0.641 (0.028) & \textbf{0.682 (0.018)} & 0.626 (0.026) & \underline{0.670 (0.011)} & 0.631 (0.021) \\
H3K9ac & 0.415 (0.020) & 0.545 (0.009) & 0.586 (0.021) & 0.547 (0.011) & 0.567 (0.020) & 0.575 (0.024) & 0.564 (0.018) & 0.581 (0.015) & \textbf{0.612 (0.006)} & \underline{0.603 (0.019)} \\
H4 & 0.735 (0.023) & 0.797 (0.008) & 0.763 (0.012) & \underline{0.808 (0.007)} & 0.795 (0.008) & 0.788 (0.010) & 0.799 (0.010) & 0.769 (0.017) & \textbf{0.815 (0.008)} & \underline{0.808 (0.010)} \\
H4ac & 0.275 (0.022) & 0.465 (0.013) & 0.564 (0.011) & 0.492 (0.014) & 0.502 (0.025) & 0.548 (0.027) & \underline{0.585 (0.018)} & 0.530 (0.017) & \textbf{0.592 (0.015)} & 0.565 (0.035) \\
Enhancer & 0.454 (0.029) & 0.525 (0.026) & 0.520 (0.031) & 0.545 (0.028) & \underline{0.561 (0.029)} & 0.522 (0.024) & 0.511 (0.026) & 0.516 (0.018) & \textbf{0.580 (0.015)} & 0.540 (0.026) \\
Enhancer type & 0.312 (0.043) & 0.423 (0.018) & 0.403 (0.056) & 0.444 (0.022) & 0.444 (0.036) & 0.403 (0.028) & 0.410 (0.026) & 0.433 (0.029) & \textbf{0.477 (0.017)} & \underline{0.463 (0.023)} \\
Promoter all & 0.910 (0.004) & 0.945 (0.003) & 0.919 (0.003) & 0.951 (0.004) & 0.952 (0.002) & 0.937 (0.002) & 0.941 (0.003) & 0.926 (0.004) & \textbf{0.962 (0.002)} & \underline{0.955 (0.002)} \\
Promoter non-TATA & 0.910 (0.006) & 0.944 (0.003) & 0.919 (0.004) & \underline{0.955 (0.003)} & 0.952 (0.003) & 0.935 (0.007) & 0.940 (0.002) & 0.925 (0.006) & \textbf{0.962 (0.001)} & \underline{0.955 (0.002)} \\
Promoter TATA & 0.920 (0.012) & 0.911 (0.011) & 0.881 (0.020) & 0.919 (0.008) & \underline{0.933 (0.009)} & 0.895 (0.010) & 0.903 (0.010) & 0.891 (0.009) & \textbf{0.948 (0.008)} & 0.931 (0.007) \\
Splice acceptor & 0.772 (0.007) & 0.909 (0.004) & 0.935 (0.005) & \underline{0.973 (0.002)} & \underline{0.973 (0.004)} & 0.918 (0.017) & 0.907 (0.015) & 0.912 (0.010) & \textbf{0.981 (0.002)} & 0.957 (0.009) \\
Splice site all & 0.831 (0.012) & 0.950 (0.003) & 0.917 (0.006) & 0.974 (0.004) & \underline{0.975 (0.002)} & 0.935 (0.011) & 0.953 (0.005) & 0.919 (0.005) & \textbf{0.978 (0.001)} & 0.973 (0.002) \\
Splice donor & 0.813 (0.015) & 0.927 (0.003) & 0.894 (0.013) & 0.974 (0.002) & \underline{0.977 (0.007)} & 0.912 (0.009) & 0.930 (0.010) & 0.888 (0.012) & \textbf{0.978 (0.002)} & 0.967 (0.005) \\
\bottomrule
\end{tabular}
}
\label{tab:nucleotide_transformer_tasks}
\end{table*}
\begin{table*}[!htb]
\small
\renewcommand{\arraystretch}{1.2}
\centering
\caption{Evaluation of the Genomic Benchmarks. The reported values represent the accuracy averaged over 10-fold cross-validation, with the standard error in parentheses.}
\resizebox{\textwidth}{!}{
\begin{tabular}{lcccccccc}
\toprule
& DNABERT-2 & HyenaDNA & NT-v2 & Caduceus-Ph & Caduceus-PS & GROVER & \textbf{Gener}\textit{ator} & \textbf{Gener}\textit{ator}-All \\
& (117M) & (55M) & (500M) & (8M) & (8M) & (87M) & (1.2B) & (1.2B) \\
\midrule
Coding vs. Intergenomic & 0.951 (0.002) & 0.902 (0.004) & 0.955 (0.001) & 0.933 (0.001) & 0.944 (0.002) & 0.919 (0.002) & \textbf{0.963 (0.000)} & \underline{0.959 (0.001)} \\
Drosophila Enhancers Stark & 0.774 (0.011) & 0.770 (0.016) & 0.797 (0.009) & \textbf{0.827 (0.010)} & 0.816 (0.015) & 0.761 (0.011) & \underline{0.821 (0.005)} & 0.768 (0.015) \\
Human Enhancers Cohn & \underline{0.758 (0.005)} & 0.725 (0.009) & 0.756 (0.006) & 0.747 (0.003) & 0.749 (0.003) & 0.738 (0.003) & \textbf{0.763 (0.002)} & 0.754 (0.006) \\
Human Enhancers Ensembl & 0.918 (0.003) & 0.901 (0.003) & 0.921 (0.004) & \textbf{0.924 (0.002)} & \underline{0.923 (0.002)} & 0.911 (0.004) & 0.917 (0.002) & 0.912 (0.002) \\
Human Ensembl Regulatory & 0.874 (0.007) & 0.932 (0.001) & \textbf{0.941 (0.001)} & \underline{0.938 (0.004)} & \textbf{0.941 (0.002)} & 0.897 (0.001) & 0.928 (0.001) & 0.926 (0.001) \\
Human non-TATA Promoters & 0.957 (0.008) & 0.894 (0.023) & 0.932 (0.006) & \textbf{0.961 (0.003)} & \textbf{0.961 (0.002)} & 0.950 (0.005) & \underline{0.958 (0.001)} & 0.955 (0.005) \\
Human OCR Ensembl & 0.806 (0.003) & 0.774 (0.004) & 0.813 (0.001) & \underline{0.825 (0.004)} & \textbf{0.826 (0.003)} & 0.789 (0.002) & 0.823 (0.002) & 0.812 (0.003) \\
Human vs. Worm & 0.977 (0.001) & 0.958 (0.004) & 0.976 (0.001) & 0.975 (0.001) & 0.976 (0.001) & 0.966 (0.001) & \textbf{0.980 (0.000)} & \underline{0.978 (0.001)} \\
Mouse Enhancers Ensembl & \underline{0.865 (0.014)} & 0.756 (0.030) & 0.855 (0.018) & 0.788 (0.028) & 0.826 (0.021) & 0.742 (0.025) & \textbf{0.871 (0.015)} & 0.784 (0.027) \\
\bottomrule
\end{tabular}
}
\label{tab:genomic_benchmarks}
\end{table*}
\begin{table*}[!htb]
\small
\renewcommand{\arraystretch}{1}
\centering
\caption{Evaluation of the Gener tasks. The reported values represent the weighted F1 score averaged over 10-fold cross-validation, with the standard error in parentheses.}
\resizebox{\textwidth}{!}{
\begin{tabular}{lcccccccc}
\toprule
& DNABERT-2 & HyenaDNA & NT-v2 & Caduceus-Ph & Caduceus-PS & GROVER & \textbf{Gener}\textit{ator} & \textbf{Gener}\textit{ator}-All \\
& (117M) & (55M) & (500M) & (8M) & (8M) & (87M) & (1.2B) & (1.2B) \\
\midrule
Gene & 0.660 (0.002) & 0.610 (0.007) & \underline{0.692 (0.005)} & 0.629 (0.005) & 0.644 (0.007) & 0.630 (0.003) & \textbf{0.700 (0.002)} & 0.687 (0.003) \\
Taxonomic & 0.922 (0.003) & 0.970 (0.024) & 0.981 (0.001) & 0.958 (0.021) & 0.968 (0.006) & 0.843 (0.006) & \textbf{0.999 (0.000)} & \underline{0.998 (0.001)} \\
\bottomrule
\end{tabular}
}
\label{tab:gener_tasks}
\end{table*}

\subsection{Central Dogma}

In our experimental setup, we selected two target protein families from the UniProt~\cite{UniProt} database: the Histone and Cytochrome P450 families. By cross-referencing gene IDs and protein IDs, we extracted the corresponding protein-coding DNA sequences from RefSeq~\cite{RefSeq}. These sequences served as training data for fine-tuning the \textbf{Gener}\textit{ator} model, directing it to generate analogous protein-coding DNA sequences.

To assess the quality of generation, we compared several summary statistics. The results for the Histone family are provided in \textit{Fig.} \ref{fig:histone_generation}, while the evaluation results for the Cytochrome P450 family are provided in \textit{Fig.} \ref{fig:cytochrome_generation}.  After deduplication, the lengths of the generated DNA sequences and their translated protein sequences, using a codon table, closely resemble the distributions observed in the target families. This preliminary validation suggests that our generated DNA sequences maintain stable codon structures that are translatable into proteins. We conducted a more in-depth structural and functional analysis of these translated protein sequences. First, we assessed whether protein language models `recognize' these generated protein sequences by calculating their perplexity (PPL) using Progen2~\cite{progen2}. The results show that the PPL distribution of generated sequences closely matches that of the natural families and significantly differs from the shuffled sequences.

Furthermore, we used AlphaFold3 to predict the folded structures of the generated protein sequences and employed Foldseek~\cite{Foldseek} to find analogous proteins in the Protein Data Bank (PDB)~\cite{RCSBPDB}. Remarkably, we identified numerous instances where the conformations of the generated sequences exhibited high similarity to established structures in the PDB ($\text{TM-score}>0.8$). This structural congruence is observed despite substantial divergence in sequence composition, as indicated by sequence identities less than $0.3$. This low sequence identity positively suggests that the model is not merely replicating existing protein sequences but has learned the underlying principles to design new molecules with similar structures. This highlights the capability of the \textbf{Gener}\textit{ator} in generating biologically relevant sequences. 

\subsection{Enhancer Design}
We employed the enhancer activity data from DeepSTARR~\cite{DeepSTARR}, following the dataset split initially proposed by DeepSTARR and later adopted by NT. Using this data, we developed an enhancer activity predictor by fine-tuning the \textbf{Gener}\textit{ator}. This predictor surpasses the accuracy of DeepSTARR and NT-multi (Table \ref{tab:enhancer_benchmark}), establishing itself as the current state-of-the-art predictor. By applying our refined SFT approach as outlined in \textit{Sec.} \ref{sec:sequence_design}, we generated a collection of candidate enhancer sequences with specific activity profiles. As illustrated in \textit{Fig.} \ref{fig:enhancer_design}, the predicted activities of these candidates exhibit significant differentiation between the generated high/low-activity enhancers and natural samples.

To our knowledge, this represents one of the first attempts to use LLMs for prompt-guided design of DNA sequences, highlighting the capability of the \textbf{Gener}\textit{ator} in this domain. These generated sequences, and more broadly, this sequence design paradigm using the \textbf{Gener}\textit{ator}, merit further exploration. Our approach underscores the potential of the \textbf{Gener}\textit{ator} model to transform DNA sequence design methodologies, providing a novel pathway for the conditional design of DNA sequences using LLMs. In our subsequent research, we plan to extend our evaluations through further validations in wet lab conditions to explore the real-world applicability of these designed sequences.

\begin{figure}[!htb]
    \centering
    \includegraphics[width=0.6\textwidth]{figures/pdf/histone_generation.pdf}
    \caption{Histone generation. (A) Distribution densities of the protein sequence lengths for generated and natural samples. (B) Distribution densities of Progen2 PPL for generated and natural samples, along with randomly shuffled sequences. (C) Scatter plot of TM-score and AlphaFold3 prediction confidence (pLDDT) with marginal distributions. (D) Two folded structures of generated samples displaying structural congruence with natural samples.}
    \label{fig:histone_generation}
\end{figure}
\begin{figure}[!htb]
    \centering
    \includegraphics[width=0.6\textwidth]{figures/pdf/cytochrome_generation.pdf}
    \caption{Cytochrome P450 generation. (A) Distribution densities of the protein sequence lengths for generated and natural samples. (B) Distribution densities of Progen2 PPL for generated and natural samples, along with randomly shuffled sequences. (C) Scatter plot of TM-score and AlphaFold3 prediction confidence (pLDDT) with marginal distributions. (D) Two folded structures of generated samples displaying structural congruence with natural samples.}
    \label{fig:cytochrome_generation}
\end{figure}

\begin{table}[!htb]
\small
\renewcommand{\arraystretch}{1}
\centering
\caption{Evaluation of the DeepSTARR dataset. The reported values represent the Pearson correlation coefficient.}
\begin{tabular}{lccc}
\toprule
 & DeepSTARR & NT-multi & \textbf{Gener}\textit{ator} \\
\midrule
Developmental & \underline{0.68} & 0.64 & \textbf{0.70} \\
Housekeeping & 0.74 & \underline{0.75} & \textbf{0.79} \\
\bottomrule
\end{tabular}
\label{tab:enhancer_benchmark}
\end{table}
\begin{figure}[!htb]
    \centering
    \includegraphics[width=0.6\textwidth]{figures/pdf/enhancer_design.pdf}
    \caption{Enhancer design. (A-B) Pearson correlation between the predicted enhancer activity and the measured activity. (C-D) Distribution densities of the predicted activity of generated enhancer sequences with distinct activity profiles.}
    \label{fig:enhancer_design}
\end{figure}

% \input{Sections/7-Analysis}
\section{Conclusion}

We justify that a flow matching generative model can produce dense and reliable rewards for training LLMs to explain the decisions of RL agents and other LLMs. 
Looking into the future, we envision extending this method to a general LLM training approach, automatically generating high-quality dense rewards, and ultimately reducing the reliance on human feedback. 

% Our method has the potential to facilitate human-AI collaboration applications, such as transportation, education, and security defense.

\newpage
\section{Impact Statements}
This paper presents work whose goal is to advance the field of machine learning by developing a model-agnostic explanation generator for intelligent agents, enhancing transparency and interpretability in agent decision prediction. The ability to generate effective and interpretable explanations has the potential to foster trust in AI systems, improving effectiveness in high-stakes applications such as healthcare, finance, and autonomous systems. Overall, we believe our work contributes positively to the broader AI ecosystem by promoting more explainable and trustworthy AI.
% Reproducibility Statement
\subsubsection*{Acknowledgments}
This work is supported by the Pioneer R\&D Program of Zhejiang (No.~2024C01021), ``Leading Talent of Technological Innovation Program'' of Zhejiang Province (No.~2023R5214), OPPO Research Fund,
the Major Research Program of Zhejiang Provincial Natural Science Foundation (No.~LD24F020015), and NSFC Grant No.~62402420.

\clearpage
\bibliography{loram,custom}
\bibliographystyle{iclr2025_conference}
\clearpage
\appendix
\section*{Appendix}
\section{Related Work}

\subsection{Large Language Models in Biosciences}
Large language models (LLMs) have emerged as powerful tools for natural language comprehension and generation~\cite{llms-survey}. Beyond their application in traditional natural language tasks, there is a growing interest in leveraging LLMs to accelerate scientific research. Early studies revealed that general-purpose LLMs, owing to their rich pre-training data, exhibit promise across various research domains~\cite{ai4science}. Subsequent efforts have focused on directly training LLMs using domain-specific data, aiming to extend the transfer learning paradigm from natural language processing (NLP) to biosciences. This body of work primarily falls into three categories: molecular LLMs, protein LLMs, and genomic LLMs.

For molecular modeling, extensive work has been conducted on training with various molecular string representations, such as SMILES~\cite{Smiles-bert,space-of-chemical,large-scale-chemical}, SELFIES~\cite{SELFIES,chemberta,chemberta2}, and InChI~\cite{inchi}. Additionally, several studies address the modeling of molecular 2D~\cite{mol-2d} and 3D structures~\cite{uni-mol} to capture more detailed molecular characteristics. In the realm of protein LLMs, related work~\cite{msa-transformer,esm2,Prottrans} mainly concentrates on modeling the primary structure of proteins (amino acid sequences), providing a solid foundation for protein structure prediction~\cite{AlphaFold2,AlphaFold3}. For genomic sequences, numerous studies have attempted to leverage the power of LLMs for improved genomic analysis and understanding. These efforts predominantly involve training models on DNA~\cite{BPNet,DNABERT,enformer,nucleotide-transformer,DNABERT-2,GROVER,gena-lm,Caduceus,dnagpt,megaDNA,HyenaDNA,Evo} and RNA~\cite{RNAErnie,uni-rna,Rinalmo} sequences. In the following section, we delve deeper into genomic LLMs specifically designed for DNA sequence modeling.

\subsection{DNA Language Models}
In the early stages, \citeauthor{BPNet} introduced the BPNet convolutional architecture to learn transcription factor binding patterns and their syntax in a supervised manner. Prior to the emergence of large-scale pre-training, BPNet was widely used in genomics for supervised learning on relatively small datasets. With the advent of BERT~\cite{BERT}, DNABERT~\cite{DNABERT} pioneered the application of pre-training on the human genome using K-mer tokenizers. To effectively capture long-range interactions, Enformer~\cite{enformer} advanced human genome modeling by incorporating convolutional downsampling into transformer architectures.

Following these foundational works, numerous models based on the transformer encoder architecture have emerged. A notable example is the Nucleotide Transformer (NT)~\cite{nucleotide-transformer}, which scales model parameters from 100 million to 2.5 billion and includes a diverse set of multispecies genomes. Recent studies, DNABERT-2~\cite{DNABERT-2} and GROVER~\cite{GROVER}, have investigated optimal tokenizer settings for masked language modeling, concluding that Byte Pair Encoding (BPE) is better suited for masked DNA LLMs. The majority of these models face the limitation of insufficient context length, primarily due to the high computational cost associated with extending the context length in the transformer architecture. To address this limitation, GENA-LM~\cite{gena-lm} employs sparse attention, and Caduceus~\cite{Caduceus} uses the more lightweight BiMamba architecture~\cite{Mamba}, both trained on the human genome.

Although these masked DNA LLMs effectively understand and predict DNA sequences, they lack generative capabilities, and generative DNA LLMs remain in the early stages of development. An early preprint~\cite{dnagpt} introduced DNAGPT, which learns mammalian genomic structures through three pre-training tasks, including next token prediction. Recent works, such as HyenaDNA~\cite{HyenaDNA} and megaDNA~\cite{megaDNA}, achieve longer context lengths by employing the Hyena~\cite{Hyena} and multiscale transformer architectures respectively, though they are significantly limited by their data and model scales. A more recent influential study, Evo~\cite{Evo}, trained on an extensive dataset of prokaryotic and viral genomes, has garnered widespread attention for its success in designing CRISPR-Cas molecular complexes, thus demonstrating the practical utility of generative DNA LLMs in the genomic field.

\section{Experimental Details}
\label{apd:detail_setup}
\paragraph{Pre-train Corpus.}
To align the inconsistent knowledge between the pruned model during training and the original model during inference, we apply \method to continual pre-training 
LLMs in a teacher-forcing manner~\citep{bachmann2024teacherforce}
on a mixed corpus of FineWeb~\citep{penedo2024fineweb} and OpenWebMath~\citep{paster2023openwebmath}.
FineWeb, containing over 15TB of cleaned and deduplicated English web data from Common Crawl. 
OpenWebMath, extracted from over 200 billion HTML files on Common Crawl, provides high-quality mathematical text. Mixing these datasets enhances the pruned model's capabilities in both general and mathematical domains.

Unless specified otherwise, we randomly sample 102,400 instances from both FineWeb and OpenWebMath to construct a mixed dataset with a sequence length of 512, yielding approximately 105 million tokens. The default training batch size is 128, allowing up to 1,600 update steps. We train without data repetition over a sufficiently large corpus to simulate a realistic pre-training scenario. 
Notably, this alignment process is a one-time, offline operation that model publishers can execute.

\paragraph{Fine-tuning Data.}
Following the fine-tuning scenario of LoRA~\citep{Edw:2022lora}, we primarily conduct supervised fine-tuning (SFT) on the OpenHermes-2.5~\citep{OpenHermes} (referred to as OpenHermes). OpenHermes is a large-scale dataset constructed from synthetically generated instructions and chat samples, encompassing diverse sources such as Airoboros 2.2~\citep{wang2023selfinstructaligning}, CamelAI Domain Expert Dataset~\citep{li2023camel}, ChatBot Arena (GPT-4 Only)~\citep{zheng2023lmsyschat1m}, and more.
To further demonstrate the general effectiveness of the \method alignment process, we also evaluate \method on the OpenOrca~\citep{OpenOrca} dataset. OpenOrca is a widely used instruction fine-tuning dataset where each data instance represents entries from the FLAN collection~\citep{longpre2023flan}, augmented by submitting the listed questions to either GPT-4 or GPT-3.5.

By default, we train SFT on the instruction dataset with a batch size of 128 and a sequence length of 512 for 400 steps, totaling approximately 26.2 million tokens. 
To effectively evaluate the overall fine-tuning performance, we assess the perplexity of the fine-tuned model on an out-of-domain test set. This out-of-domain test set is constructed by randomly sampling 2,000 instances from the Alpaca~\citep{alpaca} test set, truncated to a sequence length of 512.

\paragraph{Downstream Task.}
We focus on the performance of \method in various downstream tasks, including mathematical reasoning, common sense reasoning, and code generation. All our downstream task evaluations are performed on lm-evaluation-harness\footnote{~\url{https://github.com/EleutherAI/lm-evaluation-harness} (MIT License).} and code-eval~\footnote{~\url{https://github.com/abacaj/code-eval} (MIT License).} with VLLM~\footnote{~\url{https://github.com/vllm-project/vllm} (Apache-2.0 license).}.

For mathematical reasoning, we benchmark the accuracy of baseline models using greedy decoding on MathQA~\citep{amini-etal-2019-mathqa} with a 1-shot setting and GSM8K (Grade School Math 8K)~\citep{cobbe2021gsm8k} with 8-shots, Chain of Thought (CoT) prompting and strict match
% \footnote{
% % The evaluation script 
% Follow~\url{https://github.com/EleutherAI/lm-evaluation-harness} (MIT License).}.
MathQA is a large-scale dataset comprising 37k English multiple-choice math word problems, covering diverse math domains. It extends the AQuA-RAT dataset~\citep{ling2017program} by annotating problems with fully specified operational programs using a new representation language, building on the questions, options, rationale, and correct answers provided by AQuA-RAT.
The GSM8K is a dataset of 8.5K high-quality, linguistically diverse grade school math word problems, designed to evaluate multi-step reasoning in basic arithmetic operations (+-×÷). We conduct evaluations on its 1.3K test set with \textit{strict-match} to assess logical and mathematical reasoning in language models.

For commonsense reasoning (CSR), we report the average accuracy across six tasks—Arc Challenge \& Easy~\citep{clark2018arc}, HellaSwag~\citep{zellers-etal-2019-hellaswag}, OpenBookQA~\citep{OpenBookQA2018}, PIQA~\citep{Bisk2020}, and WinoGrande~\citep{WinoGrande2021}—under 1-shot and greedy decoding settings. These benchmarks comprehensively assess the model’s ability to apply ``commonsense" or world knowledge for reasoning, rather than relying on pattern recognition.

For code generation, we compare two pass rates, \textsc{Pass@1} and \textsc{Pass@10}~\citep{kulal2019spoc}, on HumanEval~\citep{chen2021evaluating} of each baseline in a zero-shot setting with sampling parameters of  $\textsc{temperature}=\{0.0,0.2,0.4,0.6,0.8\}$, and $\textsc{top}_\textsc{p} =0.95$.
The HumanEval dataset released by OpenAI consists of 164 handwritten Python programming problems, each with a function signature, docstring, body, and unit tests. Serving as a benchmark, HumanEval assesses models on a range of Python coding skills, from basic syntax to complex problem-solving, offering insights into their programming capabilities alongside language-focused tasks.

\paragraph{Sparsification \& Quantization.}
\method incorporates two model compression techniques: sparsification, which generates a pruned model for low-rank matrix updates, and quantization, which forms \Qmethod further to reduce the memory footprint of the pruned model.
For sparsification, 
to validate the general effectiveness of \method, we benchmark its performance across various pruning strategies $\mathtt{P}(\cdot)$. 
Specifically, we first establish a variant using randomly structured pruning and adapt \method to another three variants based on leading approaches: the structured pruning LLM-Pruner\footnote{\url{https://github.com/horseee/LLM-Pruner} (Apache-2.0 license)}~\citep{ma2023llmpruner} and the non-structured (semi-structured \& unstructured) pruning SparseGPT\footnote{\url{https://github.com/IST-DASLab/sparsegpt} (Apache-2.0 license)}~\citep{FrantarA23spasegpt}. 
These baselines are summarized below, with the corresponding configurations presented in~\cref{tab:llama2_13b,tab:llama2_70b,tab:llama2_70b_q}.
\begin{itemize}[leftmargin=20pt]
    \item \textbf{\methodrand}: 
    We adhere to the pruning settings of \methodstru, modifying only by randomly removing weights instead of the original gradient-based pruning criterion.
    \item \textbf{\methodstru}: 
    We follow LLM-Pruner and employ a block-wise strategy for local structured pruning. Attention and MLP layers are treated as separate blocks, with non-critical coupling weights pruned based on gradient information at a uniform ratio. We retain the first four and last two layers of both blocks, focusing pruning on the intermediate layers.
    \item \textbf{\methodsemi}: 
    We utilize SparseGPT with a 4:8 semi-structured sparsity pattern to prune pre-trained weights across all model layers.
    \item \textbf{\methodunst}:  
    We prune individual weights uniformly across layers using a predefined pruning ratio based on an unstructured version of SparseGPT.
\end{itemize}

For quantization $\mathtt{Q}(\cdot)$, to further reduce memory usage during training, especially when dealing with models exceeding 70 billion parameters, we achieve \Qmethod by combining \method with the LoRA-tailored quantization algorithm QLoRA~\citep{Tim:2023qlora}. While \method is compatible with the quantization of other customized LoRA methods~\citep{Xu:2023QALoRA,li2024loftq,guo2024lqlora,OPTQ2023,chai2023int21}, this falls outside the scope of this article.

\paragraph{Architecture \& Hyperparameters.}
We adopt a LLaMA architecture with RMSNorm~\citep{ZhangS19a} and SwiGLU
activations~\citep{Noglu,ZhaoSA22}. 
We run all experiments with BF16 format to reduce memory usage.
For all configurations, we default to a learning rate of 1e-3. However, the downstream performance of models fine-tuned on OpenOrca is relatively sensitive to the learning rate. Therefore, in this evaluation, we tune the learning rates for each baseline within the range of [1e-5, 1e-3] and report their respective optimal downstream scores. Specifically, we use 1e-5 for the 7B LoRA and 13B \& 70B LoRAM models, and 1e-4 for the 13B LoRA model.
All experiments run on NVIDIA A100-80GB GPUs with environments of CUDA 12.2, PyTorch 2.4.0, and Transformer 4.45.1.
For LLaMA-2 herds, we set low-rank matrices $\mathbf{B}$ and $\mathbf{A}$ of rank $r=8$ for $\mathbf{W}_\text{q}$, $\mathbf{W}_\text{k}$, $\mathbf{W}_\text{v}$, and $\mathbf{W}_\text{o}$ in the attention layer, $\mathbf{W}_\text{up}$, $\mathbf{W}_\text{gate}$, and $\mathbf{W}_\text{down}$ in the MLP layer, and the head embedding matrix $\mathbf{W}_\text{lm\_head}$;
for LLaMA-3 herds, we exclude the injection of the low-rank matrix of $\mathbf{W}_\text{lm\_head}$.
\begin{table*}[h]
    \centering
    \renewcommand{\arraystretch}{1.1}
    \setlength{\tabcolsep}{3pt}

    \caption{LoRAM configures on LLaMA-2-13B. Comparison of different pruning methods in terms of parameter reduction ratio (Reduction) and HBM footprint (GB) of pruned parameters (HBM), ignoring low-rank matrix overhead.}
    \label{tab:llama2_13b}
    \begin{tabular}{@{}lccccc@{}}
        \toprule
        Method & \#Orig. Params & Pruning Ratio & \#Pruned Params & Reduction & HBM  \\ 
        \midrule
        LoRAM-Semi & 13015864320 & 0.50 & 6738415616 & 1.93$\times$ & 12.55 \\
        LoRAM-Unst & 13015864320 & 0.55 & 6037628912 & 2.16$\times$ & 11.25 \\
        LoRAM-Rand \& Stru & 13015864320 & 0.65 & 6005662720 & 2.17$\times$ & 11.19 \\
        \bottomrule
    \end{tabular}
\end{table*}

\begin{table*}[h]
    \centering
    \renewcommand{\arraystretch}{1.1}
    \setlength{\tabcolsep}{3pt}

    \caption{LoRAM configures on LLaMA-2-70B and LLaMA-3.1-70B with different pruning ratios.}
    \label{tab:llama2_70b}
    \begin{tabular}{@{}lccccc@{}}
        \toprule
        Method & \#Orig. Params & Pruning Ratio & \#Pruned Params & Reduction & HBM \\ 
        \midrule
        LoRAM-Rand \& Stru & 68976648192 & 0.65 & 28099436544 & 2.45$\times$ & 52.34 \\
        LoRAM-Rand \& Stru & 68976648192 & 0.75 & 21488738304 & 3.21$\times$ & 40.03 \\
        LoRAM-Rand \& Stru & 68976648192 & 0.85 & 16272924672 & 4.24$\times$ & 30.31 \\
        LoRAM-Rand \& Stru & 68976648192 & 0.95 & 9662226432 & 7.14$\times$ & 18.00 \\
        LoRAM-Rand \& Stru & 70553706496 & 0.85 & 17849982976 & 3.95$\times$ & 33.25 \\
        \bottomrule
    \end{tabular}
\end{table*}

\begin{table*}[h]
    \centering
    \renewcommand{\arraystretch}{1.1}
    \setlength{\tabcolsep}{3pt}

    \caption{QLoRAM configures on LLaMA-2-70B and LLaMA-3.1-70B with , demonstrating more aggressive parameter compression.}
    \label{tab:llama2_70b_q}
    \begin{tabular}{@{}lccccc@{}}
        \toprule
        Method & \#Orig. Params & Pruning Ratio & \#Pruned Params & Reduction & HBM \\ 
        \midrule
        QLoRAM-Rand \& Stru & 68976648192 & 0.65 & 7024859136 & 9.82$\times$ & 13.08 \\
        QLoRAM-Rand \& Stru & 68976648192 & 0.75 & 5372184576 & 12.84$\times$ & 10.01 \\
        QLoRAM-Rand \& Stru & 68976648192 & 0.85 & 4068231168 & 16.95$\times$ & 7.58 \\
        QLoRAM-Rand \& Stru & 68976648192 & 0.95 & 2415556608 & 28.56$\times$ & 4.50 \\
        QLoRAM-Rand \& Stru & 70553706496 & 0.85 &  4462495744 & 15.81$\times$ & 8.31 \\
        \bottomrule
    \end{tabular}
\end{table*}










\clearpage
\section{Visualization of Dimension Evolution}
\label{sec:dimension_vis}
To clearly illustrate the evolution of weight matrix dimensions across the multiple stages in the proposed scheme, we take LLM-Pruner~\citep{ma2023llmpruner} as an example in (e.g.,~\methodstru) in~\cref{fig:dimension_vis}, visualizing the transformation from $\mathbf{W}_{0} \Rightarrow \mathbf{W}_{0}^\mathtt{P}$, $\mathbf{W}_{\Delta} \Rightarrow \mathbf{W}_{\Delta}^\mathtt{P}$, and $\mathbf{W}_{\Delta}^{\mathtt{P}^{\star}}\Rightarrow \mathbf{W}_{\Delta}^{\mathtt{R}^{\star}}$ under \method with structured pruning. 
For \method variants employing non-structured pruning, the parameter dimensionality remains unchanged during training due to the use of a mask matrix. Therefore, these visualizations are omitted.
\begin{figure*}[ht]
\begin{center}
\includegraphics[width=\textwidth]{fig/dimension_evaluation_final.pdf}
\caption{
Dimensional evolution of the weight matrices: $\mathbf{W}_{0} \Rightarrow \mathbf{W}_{0}^\mathtt{P}$ (a), $\mathbf{W}_{\Delta} \Rightarrow \mathbf{W}_{\Delta}^\mathtt{P}$ (b), and $\mathbf{W}_{\Delta}^{\mathtt{P}^{\star}} \Rightarrow \mathbf{W}_{\Delta}^{\mathtt{R}^{\star}}$ (c) during \methodstru training. This includes updates for $\mathbf{W}_\text{q}$, $\mathbf{W}_\text{k}$, $\mathbf{W}_\text{v}$, and $\mathbf{W}_\text{o}$ in the attention layer, as well as $\mathbf{W}_\text{up}$, $\mathbf{W}_\text{gate}$, and $\mathbf{W}_\text{down}$ in the MLP layer.
}
\label{fig:dimension_vis}
\end{center}
\end{figure*}


\clearpage
\section{Visualization of Low-rank Matrices}
\label{apd:vis_matrix}
In this section, we utilize the \(L_{2}\)-norm to evaluate variations in low-rank matrices trained with different \method variants. This metric facilitates the visualization of captured features and allows for an analysis of \method's effectiveness. Specifically, we examine the updated low-rank matrices in the self-attention and MLP layers of LLaMA-2-13B and LLaMA-2-70B, trained with \method variants on OpenHermes.

\subsection{Head-wise Norm of Attention}
For the low-rank matrices in the attention layer, denoted as \(\mathbf{W}_{\Delta^{*}}\) where \({*} \in \{\text{q}, \text{k}, \text{v}, \text{o}\}\), we compute the \(L_{2}\) norms for each attention head. Let \(\text{H}^{*}\) represent the number of heads. The \(L_{2}\) norms for each head \(\text{h}\) (where \(\text{h} = 0, 1, \ldots, \text{H}^{*} - 1\)) are defined as follows:

\begin{equation}
\| \mathbf{W}_{\Delta^{*}}^{(h)} \|_2 = 
\begin{cases}
\left\| \mathbf{W}_{\Delta^{*}}[h, :] \right\|_2 & \text{if } {*} \in \{\text{q}, \text{k}, \text{v}\} \\
\left\| \mathbf{W}_{\Delta^{*}}[:, h] \right\|_2 & \text{if } {*} = \text{o}
\end{cases}.
\end{equation}

The results are visualized through heatmaps in ~\cref{fig:attn_lora_vis_13B,fig:attn_lora_vis_70B}, effectively illustrating the distribution of features captured by different attention heads.

\subsection{Layer-wise Norm of MLP}
For the low-rank matrices in the MLP layers, denoted as \(\mathbf{W}_{\Delta^{*}}\) where \(\Delta^{*} \in \{\text{up}, \text{gate}, \text{down}\}\), we denote the number of layers as \(\text{L}\). The average \(L_{2}\) norm for a specific layer \(l\) (where \(l = 0, 1, \ldots, \text{L} - 1\)) is computed as follows, excluding elements equal to zero using a mask, ensuring that only active parameters contribute to the average:

\begin{equation}
\| \mathbf{W}_{\Delta^{*}}^{(l)} \|_2 = 
\begin{cases}
\frac{1}{m} \sum_{i=0}^{m-1} \left\| \mathbf{W}_{\Delta^{*}}^{(l)}[i, :] \right\|_2 \cdot \mathbb{I}(\mathbf{W}_{\Delta^{*}}^{(l)}[i, :] \neq 0) & \text{if } \Delta^{*} \in \{\text{up}, \text{gate}\} \\
\frac{1}{n} \sum_{j=0}^{n-1} \left\| \mathbf{W}_{\Delta^{*}}^{(l)}[:, j] \right\|_2 \cdot \mathbb{I}(\mathbf{W}_{\Delta^{*}}^{(l)}[:, j] \neq 0) & \text{if } \Delta^{*} = \text{down}
\end{cases}.
\end{equation}

Here, \(\mathbb{I}(\cdot)\) denotes the indicator function, which returns 1 only when the corresponding element is non-zero, effectively excluding zero elements from the average calculation. The average norms for the MLP layers are visualized in ~\cref{fig:mlp_lora_vis_13B,fig:mlp_lora_vis_70B}, clearly depicting the trends in updating amplitudes across the various projections.

\subsection{Attention Update Patterns}

\paragraph{Layer Update Patterns in \method and LoRA.} \cref{fig:attn_lora_vis_13B,fig:attn_lora_vis_70B} reveal that both LoRA and \method display similar layer update behaviors. In any low-rank matrix \(\mathbf{W}_{\Delta^{*}}\) where \({*} \in \{\text{q}, \text{k}, \text{v}, \text{o}\}\), deeper colors predominantly concentrate in either shallow or deep layers, while middle layers receive relatively few updates. This suggests that training primarily focuses on optimizing the shallow layers to capture semantic information, with deeper layers refining this knowledge, rendering middle layers somewhat redundant.

\paragraph{More Uniform Projection Updates in \method.} \cref{fig:attn_lora_vis_13B,fig:attn_lora_vis_70B} further indicates that updates in the LoRA-trained low-rank matrices, particularly for \(\mathbf{W}_{\Delta^{\text{v}}}\), are relatively uniform, exhibiting substantial deep colors across multiple heads. In contrast, other matrices emphasize specific rows and heads. For instance, in the 70B model's \(\mathbf{W}_{\Delta^{\text{k}}}\), only the heads in the uppermost layers experience significant updates, while lower layers show minimal changes. This suggests that the unpruned model retains rich knowledge, requiring only minor adjustments to a few heads in certain layers for task adaptation. Conversely, \method demonstrates a more uniform distribution of deep colors across each low-rank matrix, indicating that the pruned model must effectively utilize every limited neuron to capture knowledge, thereby enhancing downstream performance.

\begin{figure*}[!ht]
\begin{center}
\includegraphics[width=\textwidth]{fig/l2_norms_attn_13B.pdf}
    \caption{Visualization of low-rank matrices in the attention layers of LLaMA-2-13B.}
    \label{fig:attn_lora_vis_13B}
\includegraphics[width=\textwidth]{fig/l2_norms_mlp_13B.pdf}
    \caption{Average \(L_{2}\) norms of low-rank matrices in the MLP layers of LLaMA-2-70B.}
    \label{fig:mlp_lora_vis_13B}
\end{center}
\end{figure*}

\begin{figure*}[!t]
\begin{center}
\includegraphics[width=\textwidth]{fig/l2_norms_attn_70B.pdf}
    \caption{Visualization of low-rank matrices in the attention layers of LLaMA-2-70B.}
    \label{fig:attn_lora_vis_70B}
\includegraphics[width=\textwidth]{fig/l2_norms_mlp_70b.pdf}
    \caption{Average \(L_{2}\) norms of low-rank matrices in the MLP layers of LLaMA-2-70B.}
    \label{fig:mlp_lora_vis_70B}
\end{center}
\end{figure*}

\subsection{MLP Update Patterns}

\paragraph{\method Exhibits Greater Update Amplitude than LoRA.} For both the 13B and 70B models, \method consistently exhibits a greater update amplitude across each layer compared to LoRA, as shown in~\cref{fig:mlp_lora_vis_13B,fig:mlp_lora_vis_70B}. This increased amplitude indicates that \method is more effective in adjusting the weights in all layers, thus enhancing the adaptability and overall performance.
% \clearpage
\paragraph{Distinct Update Trends in Layer Amplitudes.} The amplitude changes reveal a distinct pattern in~\cref{fig:mlp_lora_vis_13B,fig:mlp_lora_vis_70B}: first decreasing, then increasing, and finally decreasing again. Shallow layers (0-3) and deeper layers (25-35 for the 13B model and 50–75 for the 70B model) undergo intensive updates. This behavior indicates that model prioritizes foundational feature extraction in shallow layers and the refinement of complex representations in deeper layers. Such a strategic update distribution optimizes the learning process, ensuring effective capture of basic and advanced features.

\subsection{Analysis of Unchanged Weights}
Here, we try to analyze the unchanged weights to support the motivation of LoRAM.
\paragraph{Fine-Grained Visualizations.} 
As the above visualization, we conducted detailed visualizations comparing the updated magnitudes of pruned and unpruned weights across layers. The results demonstrate that unpruned weights in both attention and MLP layers exhibit consistently smaller updates during fine-tuning as shown in \cref{fig:attn_lora_vis_70B}, indicating their critical role in preserving the model's capacity for inference.{\paragraph{Theoretical Perspective.} The phenomenon can be explained by the gradient-based importance of these weights, which prioritize parameters with minimal updates but high sensitivity during recovery. These weights stabilize inference outputs, making them indispensable despite their limited fine-tuning updates.}

{\paragraph{Quantitative Evidence} Our analysis reveals a strong correlation between weight update magnitudes and downstream performance. Pruning weights with smaller updates significantly degrades performance, highlighting their importance for inference and validating our intuition.}
{\paragraph{Impact on Large Models} The selective pruning strategy shows notable benefits in larger models such as LLaMA-2-70B, where it outperforms random pruning by a substantial margin. Retaining critical parameters ensures effective task adaptation and generalization across diverse domains.}

% \section{Pruning with Oracle}

% \section{Alignment with Distillation}
\clearpage
\section{Performance of Sub-Tasks in CSR}
\label{apd:detail_csr}
We report the performance of six sub-tasks in CSR, with~\cref{fig:csr_comparison_13B,fig:csr_comparison_70B} showcasing the results for \method-trained LLaMA-2-13B and LLaMA-2-70B, respectively. Our findings indicate that various \method variants outperform core competitive benchmarks: for the 13B model, \method surpasses both the untrained 13B and the LoRA-trained 7B, while for the 70B model, it exceeds the untrained 70B and the LoRA-trained 13B. This demonstrates that \method consistently achieves performance gains across models of different scales while effectively reducing memory usage. Furthermore, selective weight contributions in the 70B model significantly enhance performance, as evidenced by \methodstru's marked improvement, particularly in the challenging Arc Challenge multi-choice question-answering task. This suggests that \methodstru effectively identifies and leverages weight differences, focusing on the most trainable weights compared to \methodrand.
\begin{figure}[ht]
    \centering
    \begin{minipage}[b]{0.5\textwidth}
        \centering
        \includegraphics[width=\textwidth]{fig/CSR_LLaMA-2-13B_OpenHermes.pdf}
\label{fig:csr_detail_13B_hermes}
    \end{minipage}%
    \begin{minipage}[b]{0.5\textwidth}
        \centering
        \includegraphics[width=\textwidth]{fig/CSR_LLaMA-2-13B_OpenOrca.pdf}     \label{fig:csr_detail_13B_orca}
    \end{minipage}
    \caption{Performance of six CSR sub-tasks on the trained LLaMA-2-13B using \method.}
    \label{fig:csr_comparison_13B}
\begin{minipage}[b]{0.5\textwidth}
        \centering
        \includegraphics[width=\textwidth]{fig/CSR_LLaMA-2-70B_OpenHermes.pdf}
\label{fig:csr_detail_70B_hermes}
    \end{minipage}%
    \begin{minipage}[b]{0.5\textwidth}
        \centering
        \includegraphics[width=\textwidth]{fig/CSR_LLaMA-2-70B_OpenOrca.pdf}     \label{fig:csr_detail_70B_orca}
    \end{minipage}
    \caption{Performance of six CSR sub-tasks on the trained LLaMA-2-70B using \method.}
    \label{fig:csr_comparison_70B}
\end{figure}

\clearpage
\section{Algorithm of \method}
\label{alg:loram}
Here, we present the complete algorithm of \method in \cref{algo:loram}.

\begin{algorithm}[ht]
\small
\caption{\method (Memory-Efficient LoRA Training)}
\label{algo:loram}
\begin{algorithmic}[1]
\Require 
original full-rank pre-trained weight $\mathbf{W}_0$, 
alignment corpus $\mathcal{D}_\mathtt{A}$, 
and flags $\mathcal{F}^\mathtt{P}, \mathcal{F}^\mathtt{A}$, $\mathcal{F}^\mathtt{Q}, \mathcal{F}^\mathtt{R}$.

\State \mydarkcolor{\textbf{Offline $\mathbf{W}_{0}^{*}$ Process Stage:}}
\If{$\mathcal{F}^\mathtt{P}$}
    \State $\mathbf{W}_{0}^\mathtt{P} = \mathtt{P}(\mathbf{W}_{0}) =  \mathbf{W}_0 \circ \mathbf{M}^\mathtt{P}$  \Comment{\mydarkcolor{Pruned Full-Rank Weight Generation.}}
    \If{$\mathcal{F}^\mathtt{A}$} 
        \State $\mathbf{W}_{0}^\mathtt{P,A} \gets \text{argmin} {\ } \mathcal{L}_{\mathtt{A}}(\mathcal{D}_\mathtt{A};\mathbf{W}_{0}^\mathtt{P})$ 
        \Comment{\mydarkcolor{Pruned Full-Rank Weight Alignment.}}
        \If{$\mathcal{F}^\mathtt{Q}$}
        \State $\mathbf{W}_{0}^\mathtt{P,A,Q} = \mathtt{Q}(\mathbf{W}_{0}^\mathtt{P,A})$
        \Comment{\mydarkcolor{Pruned Full-Rank Weight Quantization.}}
        \EndIf
        \ElsIf{$\mathcal{F}^\mathtt{Q}$}
        \State $\mathbf{W}_{0}^\mathtt{P,Q} = \mathtt{Q}(\mathbf{W}_{0}^\mathtt{P})$
    \EndIf
    \ElsIf{$\mathcal{F}^\mathtt{Q}$} 
    \State $\mathbf{W}_{0}^\mathtt{Q} = \mathtt{Q}(\mathbf{W}_{0})$ \Comment{\mydarkcolor{Standard Quantization for LoRA}}
\EndIf

\State{Record the processing result of $\mathbf{W}_{0}$ as $\mathbf{W}_{0}^{*}$, $* \in \{\mathtt{NULL},\mathtt{P},\mathtt{Q},\mathtt{(P,Q)},\mathtt{(P,A)},\mathtt{(P,A,Q)}\}$.}
\\
\State \mydarkcolor{\textbf{Online $\mathbf{W}_{\Delta}^{*}$ Training Stage:}} 

\If{$\mathcal{F}^\mathtt{P}$} \Comment{\mydarkcolor{Pruned Low-Rank Matrix Generation.}}
    \State $\mathbf{W}_{\Delta}^\mathtt{P} = \
    \mathbf{B}^\mathtt{P}\mathbf{A}^\mathtt{P} = \
    \mathtt{P}(\mathbf{W}_{\Delta}) =  \ 
    \mathbf{W}_{\Delta} \circ \mathbf{M}^\mathtt{P} = \
    \mathbf{B}\mathbf{A} \circ \mathbf{M}^\mathtt{P}$
    \While  {$\textsc{Training}$} \Comment{\mydarkcolor{Pruned Low-Rank Matrix Training.}}
    \State Update low-rank matrix via objective $\mathcal{L}_{\mathtt{SFT}}$ with the forward pass $\mathbf{h} = \mathbf{x} \mathbf{W}_{0}^\mathtt{*} + \mathbf{x}\mathbf{W}_{\Delta}^\mathtt{P}$.
    \State Return trained low-rank matrix $\mathbf{W}_{\Delta}^{\mathtt{P}^{\star}}=\mathbf{B}^{\mathtt{P}^{\star}}\mathbf{A}^{\mathtt{P}^{\star}}$.
    \EndWhile
    \If{$\mathcal{F}^\mathtt{R}$} \Comment{\mydarkcolor{Recovered Low-Rank Matrix Generation.}}
    \State $\mathbf{W}_{\Delta}^{\mathtt{R}^{\star}} = \ 
    \mathbf{B}^{\mathtt{R}^{\star}}\mathbf{A}^{\mathtt{R}^{\star}} = \ 
    \mathtt{R}(\mathbf{W}_{\Delta}^{\mathtt{P}^{\star}}) = \ 
    \mathbf{W}_{\Delta}^{\mathtt{P}^{\star}} \circ (1-\mathbf{M}^\mathtt{P})$
    \Comment{\mydarkcolor{Structured \method}}
    \Else
    \State $\mathbf{W}_{\Delta}^{\mathtt{R}^{\star}} = \ 
    \mathbf{B}^{\mathtt{R}^{\star}}\mathbf{A}^{\mathtt{R}^{\star}} = \ 
    \mathbf{B}^{\mathtt{P}^{\star}}\mathbf{A}^{\mathtt{P}^{\star}}$
    \Comment{\mydarkcolor{Non-structured \method}}
    \EndIf
\Else
    \While  {$\textsc{Training}$} \Comment{\mydarkcolor{Standard LoRA Training.}}
    \State Update low-rank matrix via objective $\mathcal{L}_{\mathtt{SFT}}$ with the forward pass $\mathbf{h} = \mathbf{x} \mathbf{W}_{0}^\mathtt{*} + \mathbf{x}\mathbf{W}_{\Delta}$.
    \State Return trained low-rank matrix $\mathbf{W}_{\Delta}^{\star}=\mathbf{B}^{\star}\mathbf{A}^{\star}$.
    \EndWhile
\EndIf

\State{Record the trained low-rank matrix as $\mathbf{W}_{\Delta}^{*}$, $* \in \{{\mathtt{R}^{\star},\star}\}$.}
\\
\State \mydarkcolor{\textbf{Online $\mathbf{W}_{0},\mathbf{W}_{\Delta}^{*}$ Inference Stage:}} 
\While {$\textsc{Inference}$ with {$*$ is $\mathtt{R}^{\star}$}} \Comment{\mydarkcolor{Recovered Low-Rank Matrix Inference.}}
\State Perform inference with the forward pass $\mathbf{h} 
= \mathbf{x} (\mathbf{W}_0 + \mathbf{W}_{\Delta}^{\mathtt{R}^{\star}}) = \mathbf{x} (\mathbf{W}_0 + \mathbf{B}^{\mathtt{R}^{\star}}\mathbf{A}^{\mathtt{R}^{\star}})$.
\EndWhile

\While {$\textsc{Inference}$ with {$*$ is ${\star}$}} \Comment{\mydarkcolor{Standard LoRA Inference.}}
\State Perform Inference with the forward pass $\mathbf{h} 
= \mathbf{x} (\mathbf{W}_0 + \mathbf{W}_{\Delta}^{\star}) = \mathbf{x} (\mathbf{W}_0 + \mathbf{B}^{\star}\mathbf{A}^{\star})$.
\EndWhile

\end{algorithmic}
\end{algorithm}







\clearpage

\section{{Tuning of Learning Rate}}
\label{apd:detail_lr}
{We provide additional details on the learning rate tuning process for full LoRA applied to LLaMA-2-7B and LLaMA-2-13B models, trained on the OpenHermes dataset. These experiments in~\cref{fig:LR-Tuning} demonstrate that a learning rate of 1e-3 consistently achieves the best perplexity across both in-domain and out-of-domain datasets, further validating the reliability of our comparison.}

\begin{figure*}[ph]
\begin{center}
\includegraphics[width=\textwidth]{fig/LR-Tuning.pdf}
\caption{
{Learning rate tuning for LLaMA-2-7B and LLaMA-2-13B on OpenHermes using LoRA.}
}
\label{fig:LR-Tuning}
\end{center}
\end{figure*}

\section{{Performance of Domain-Specific Task}}

{To assess the effectiveness of LoRAM in domain-specific tasks, we conducted experiments on GSM8K (using the training set for tuning and the test set for evaluation), a mathematical reasoning benchmark known for its sensitivity to sparsification. Specifically, we trained LLaMA-3.1-70B using QLoRAM under various configurations.}

{The results, summarized in~\cref{tab:gsm8k_results}, highlight that LoRAM achieves excellent performance in this domain-specific setting. Notably, LoRAM-based models maintain high accuracy with substantial parameter reduction ratios, showcasing their robustness and efficiency in domain-specific tasks. These findings emphasize LoRAM's broad applicability beyond general-purpose instruction fine-tuning.}

\begin{table}[ht]
    \centering
    \caption{{Evaluation of LoRAM on the GSM8K dataset for domain-specific fine-tuning. Results show accuracy (\%) and parameter reduction ratios for different configurations.}}
    \label{tab:gsm8k_results}
    {
    \begin{tabular}{@{}lcc@{}}
        \toprule
        \textbf{LLaMA-3.1} & \textbf{GSM8K} & \textbf{Parameter Reduction Ratio} \\
        \midrule
        8B w/o Fine-Tuning            & 55.27  & 8.79× \\
        8B LoRA (OpenHermes 400)      & 55.80  & 8.79× \\
        70B w/o Fine-Tuning           & 75.28  & 1.00× \\
        70B QLoRAM-Stru 400 (OpenHermes 400) & 80.36 & \textbf{15.81×} \\
        70B QLoRAM-Stru 400 (GSM8K 100) & 77.18 & \textbf{15.81×} \\
        70B QLoRAM-Stru 400 (GSM8K 200) & 79.15 & \textbf{15.81×} \\
        70B LoRA (OpenHermes 400)     & \textbf{80.74} & 1.00× \\
        \bottomrule
    \end{tabular}}
\end{table}

\clearpage
\section{{Analysis of LoRAM Cost}}
\label{apd:diverse_metrics}

{Identifying the costs of LoRAM is indeed important, which is why we report both the number of training tokens used during the alignment phase and the parameter reduction ratios in the low-rank training phase. Below, we clarify the two stages of LoRAM:}
{\paragraph{Offline Knowledge Alignment Phase.} 
The offline phase is task-agnostic and can be conducted by the model publisher prior to deployment, making its cost negligible for end users. To quantify the offline cost, we measured the number of training tokens (as in \citet{xia2024sheared}) rather than end-to-end latency, which can vary based on hardware configurations. As shown in Figure~5, LoRAM achieves significant performance gains using only 13 million tokens, demonstrating the efficiency of the alignment phase.}
{\paragraph{Online Low-Rank Matrix Training Phase.} 
For the online phase, the memory and latency costs are primarily determined by the size of the base model parameters, which dominate resource consumption during training. To avoid redundancy in reporting, we focused on parameter reduction ratios instead of absolute time or memory usage.}
{\paragraph{Comparative Metrics for Online Training.}
Here, we provide additional metrics, including memory and latency comparisons for the online training phase. We conducted experiments using a workload of 1024 samples (batch size 128, micro-batch size 4, sequence length 512) randomly selected from OpenHermes. The results in \cref{tab:lora_comparison} demonstrate that LoRAM with a structured pruning ratio of $2.17\times$ (13B $\rightarrow$ 6B) achieves comparable peak memory, latency, and throughput to 7B LoRA, with only minor trade-offs. These differences arise due to the larger layer count in 13B LoRAM, introducing more non-GEMM operations, slightly affecting latency and throughput.}

{These results underscore the advantages of LoRAM's design in achieving substantial resource efficiency without significant trade-offs in memory or latency.}

\begin{table*}[ht]
\centering
\caption{{Comparison of peak memory (MiB), latency (s), and throughput (samples/s) during the online training phase for LoRAM and LoRA models. Results are based on a workload of 1024 samples (batch size 128, micro-batch size 4, sequence length 512).}}
{
\label{tab:lora_comparison}
\begin{tabular}{@{}lccccc@{}}
\toprule
\textbf{LLaMA-2}            & \textbf{\#Model Params} & \textbf{Reduction Ratio} & \textbf{Memory} & \textbf{Latency} & \textbf{Throughput} \\
\midrule
7B LoRA           & 6.73B                & 1.93$\times$             & 30,517                     & \textbf{134.27}      & \textbf{7.626}                 \\
13B LoRA          & 13.02B               & 1.00$\times$             & 51,661                     & 206.07               & 4.969                          \\
13B LoRAM-Stru    & \textbf{6.01B}       & \textbf{2.17$\times$}    & \textbf{29,799}            & 147.86               & 6.925                          \\
\bottomrule
\end{tabular}}
\end{table*}

\clearpage
\section{{Analysis of Changes in Performance Trends}}
{We analyze performance at two stages: after fine-tuning but before recovery, and after both fine-tuning and recovery.}
{\paragraph{After Fine-Tuning but Before Recovery.}
At this stage, the results of LoRAM align with prior work (e.g., SparseGPT, Wanda, and LLM-Pruner). Unstructured and semi-structured pruning consistently outperform structured pruning (see \cref{fig:pruning-methods}, solid lines). This trend holds true across both aligned and unaligned settings, with the performance order as follows: \methodsemi $<$ \methodunst $<$ \methodstru $<$ \methodrand
The slight advantage of \methodsemi over \methodunst can be attributed to its smaller pruning ratio, which retains more parameters and mitigates performance degradation.}
{\paragraph{After Fine-Tuning and Recovery.}
Post-recovery results show that structured pruning outperforms unstructured pruning. This can be explained by two factors:}

\begin{itemize}
    \item {\textbf{Preserved Structure for Recovery:} Structured pruning maintains the organization of the pruned weights into coherent structures (e.g., rows and columns in MLP layers, attention heads in attention layers), ensuring that activations after recovery are aligned with those of the original model. This alignment improves the recovery process.}
    \item {\textbf{Pruned Weight Quality:} The quality of pruned weights influences the recovery effectiveness. Structured pruning tends to remove less critical weights, leaving more recoverable parameters. In contrast, unstructured pruning can remove weights that are more difficult to recover, which negatively impacts performance post-recovery.}
\end{itemize}

{These results highlight the interplay between pruning and recovery, suggesting that structured pruning, despite initial performance disadvantages, facilitates more effective recovery.}




% \section{Exploration of Diverse Tasks}

% \section{Effect of Alignment Data Size}

% \section{Exploration of Learning Rate}

% \subsection{Effect of Model Scale}




\end{document}
