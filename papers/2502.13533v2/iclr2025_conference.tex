
\documentclass{article} % For LaTeX2e
\usepackage{iclr2025_conference,times}

% Optional math commands from https://github.com/goodfeli/dlbook_notation.
%%%%% NEW MATH DEFINITIONS %%%%%

% \usepackage{amsmath,amsfonts,bm}
\usepackage{amsmath,amsfonts}

\usepackage{pifont}


\newcommand{\R}{\mathbb{R}}


\def\va{{\mathbf{a}}}
\def\vg{{\mathbf{g}}}

% Sets
\def\sR{\mathbb{R}}
\def\sC{\mathbb{C}}
\def\sZ{\mathbb{Z}}
\def\sN{\mathbb{N}}
\def\sQ{\mathbb{Q}}

\def\sS{\mathcal{S}}



% Vectors
\def\vzero{{\mathbf{0}}}
\def\vone{{\mathbf{1}}}
\def\vmu{{\mathbf{\mu}}}
\def\vtheta{{\mathbf{\theta}}}
\def\va{{\mathbf{a}}}
\def\vb{{\mathbf{b}}}
\def\vc{{\mathbf{c}}}
\def\vd{{\mathbf{d}}}
\def\ve{{\mathbf{e}}}
\def\vf{{\mathbf{f}}}
\def\vg{{\mathbf{g}}}
\def\vh{{\mathbf{h}}}
\def\vi{{\mathbf{i}}}
\def\vj{{\mathbf{j}}}
\def\vk{{\mathbf{k}}}
\def\vl{{\mathbf{l}}}
\def\vm{{\mathbf{m}}}
\def\vn{{\mathbf{n}}}
\def\vo{{\mathbf{o}}}
\def\vp{{\mathbf{p}}}
\def\vq{{\mathbf{q}}}
\def\vr{{\mathbf{r}}}
\def\vs{{\mathbf{s}}}
\def\vt{{\mathbf{t}}}
\def\vu{{\mathbf{u}}}
\def\vv{{\mathbf{v}}}
\def\vw{{\mathbf{w}}}
\def\vx{{\mathbf{x}}}
\def\vy{{\mathbf{y}}}
\def\vz{{\mathbf{z}}}
\def\vzeta{{\mathbf{\zeta}}}

% Matrix
\def\mA{{\mathbf{A}}}
\def\mB{{\mathbf{B}}}
\def\mC{{\mathbf{C}}}
\def\mD{{\mathbf{D}}}
\def\mE{{\mathbf{E}}}
\def\mF{{\mathbf{F}}}
\def\mG{{\mathbf{G}}}
\def\mH{{\mathbf{H}}}
\def\mI{{\mathbf{I}}}
\def\mJ{{\mathbf{J}}}
\def\mK{{\mathbf{K}}}
\def\mL{{\mathbf{L}}}
\def\mM{{\mathbf{M}}}
\def\mN{{\mathbf{N}}}
\def\mO{{\mathbf{O}}}
\def\mP{{\mathbf{P}}}
\def\mQ{{\mathbf{Q}}}
\def\mR{{\mathbf{R}}}
\def\mS{{\mathbf{S}}}
\def\mT{{\mathbf{T}}}
\def\mU{{\mathbf{U}}}
\def\mV{{\mathbf{V}}}
\def\mW{{\mathbf{W}}}
\def\mX{{\mathbf{X}}}
\def\mY{{\mathbf{Y}}}
\def\mZ{{\mathbf{Z}}}
\def\mBeta{{\mathbf{\beta}}}
\def\mPhi{{\mathbf{\Phi}}}
\def\mLambda{{\mathbf{\Lambda}}}
\def\mSigma{{\mathbf{\Sigma}}}


% Expectation
% \def\eE{\mathop{\mathbb{E}}\limits}
\def\eE{\mathbb{E}}

% Probability
\def\pP{\mathbb{P}}

% Tilde
\def\tf{\tilde{f}}
\def\tS{\tilde{S}}
\def\wtF{\widetilde{\mathcal{F}}}
\def\whR{\widehat{R}}
\def\tvx{\tilde{\mathbf{x}}}
\def\ty{\tilde{y}}


\def\defeq{\overset{\textup{def}}{=}}
% \def\defeq{\overset{.}{=}}
\def\defone{\overset{\text{\ding{172}}}{=}}
\def\deftwo{\overset{\text{\ding{173}}}{=}}
\def\leqone{\overset{\text{\ding{172}}}{\leq}}
\def\leqtwo{\overset{\text{\ding{173}}}{\leq}}
\def\leqthree{\overset{\text{\ding{174}}}{\leq}}
\def\leqfour{\overset{\text{\ding{175}}}{\leq}}
\def\eqone{\overset{\text{\ding{172}}}{=}}
\def\eqtwo{\overset{\text{\ding{173}}}{=}}
\def\eqthree{\overset{\text{\ding{174}}}{=}}
\def\eqfour{\overset{\text{\ding{175}}}{=}}
\def\geqfive{\overset{\text{\ding{176}}}{\geq}}

% \iclrfinalcopy

\usepackage{hyperref}
\usepackage{url}

% for figures
\usepackage{graphicx}
\usepackage{subfigure}
\usepackage{wrapfig}

% for equation
\usepackage{amsmath}
\usepackage{amstext}
\usepackage{amsfonts}
\usepackage{bm}

\usepackage{bbm}
% for table
\usepackage{multirow}
\usepackage{booktabs}
\usepackage{array}
\usepackage{caption}
\usepackage{multirow}
\usepackage{array}
\usepackage{caption}
\usepackage{color}
\usepackage{colortbl}
\usepackage{tablefootnote}
\usepackage{adjustbox}

% for algorithm
% \usepackage{caption}
\usepackage{algorithm}
\usepackage{algpseudocode}
\newcommand{\mycolor}[1]{\textcolor[RGB]{64,101,149}{#1}}
\newcommand{\mydarkcolor}[1]{\textcolor[RGB]{64,101,149}{#1}}
\algnewcommand{\LineComment}[1]{\Statex ~~~~~~\textsc{//}~\textit{#1}}

%for itemize
\usepackage{enumitem}
\setenumerate[1]{itemsep=0pt,partopsep=0pt,parsep=\parskip,topsep=5pt}
\setitemize[1]{itemsep=0pt,partopsep=0pt,parsep=\parskip,topsep=5pt}
\setdescription{itemsep=0pt,partopsep=0pt,parsep=\parskip,topsep=5pt}

% for highlight
\usepackage{soul}

% for notation
% \usepackage[mathscr]{euscript}
% \newcommand{\M}{\mathscr{M}}
% \newcommand{\M}{\mathcal{M}}

%for itemize
\usepackage{amssymb}  
\usepackage{pifont}
\newcommand{\cmark}{\ding{51}}
\newcommand{\xmark}{\ding{55}}
\newcommand{\greenyes}{\textcolor{green}{\ding{51}}}
\newcommand{\redno}{\textcolor{red}{\ding{55}}}
\usepackage[capitalize]{cleveref}
% for table color box
\definecolor{c0}{cmyk}{1,0.3968,0,0.2588} 
\definecolor{LightCyan}{rgb}{0.88,1,1}
\newcommand{\gray}{\cellcolor{gray!10}} % LightCyanc0!5
\usepackage{scalerel} 

%% === commands for comments ===
% \usepackage{ulem}
\usepackage[normalem]{ulem}
\usepackage{makecell} %zj
\usepackage{colortbl} %zj

\usepackage{scalerel} % for logo

%citation color
\definecolor{uclablue}{rgb}{0.15, 0.45, 0.68}
\definecolor{custommagenta}{rgb}{0.1, 0.90, 1}
% {0.85, 0.1, 0.85}
\newcommand{\blueone}{\cellcolor{uclablue!10}} 
\newcommand{\bluetwo}{\cellcolor{uclablue!20}}
\newcommand{\bluethree}{\cellcolor{uclablue!30}}
\newcommand{\bluefour}{\cellcolor{uclablue!40}}

\usepackage{xcolor}
\usepackage{hyperref}
\usepackage{tikz}

\newcommand{\rainbowurl}{%
    \textbf{\texttt{\textcolor{red!50}{h}\textcolor{orange!50}{t}\textcolor{yellow!50}{t}\textcolor{green!50}{p}\textcolor{blue!50}{s}\textcolor{purple!50}{:}//%
    \textcolor{red!50}{g}\textcolor{orange!50}{i}\textcolor{yellow!50}{t}\textcolor{green!50}{h}\textcolor{blue!50}{u}\textcolor{purple!50}{b}%
    \textcolor{red!50}{.}\textcolor{orange!50}{c}\textcolor{yellow!50}{o}\textcolor{green!50}{m}/%
    \textcolor{blue!50}{j}\textcolor{purple!50}{u}\textcolor{red!50}{n}\textcolor{orange!50}{z}\textcolor{yellow!50}{h}\textcolor{green!50}{a}\textcolor{blue!50}{n}\textcolor{purple!50}{g}%
    \textcolor{red!50}{-}\textcolor{orange!50}{z}\textcolor{yellow!50}{j}/%
    \textcolor{green!50}{L}\textcolor{blue!50}{o}\textcolor{purple!50}{R}\textcolor{red!50}{A}\textcolor{orange!50}{M}}%
}}

\hypersetup{
    breaklinks,
    citecolor=uclablue,
    colorlinks=true,
    linkcolor=uclablue
}


\title{
\scalerel*{\includegraphics{fig/loram_logo.png}}{{\rule{1.6ex}{1.6ex}}}
Train Small, Infer Large: Memory-Efficient LoRA Training for Large Language Models
}


\author{Jun Zhang$^{1,3} \thanks{\ Work done during an internship at OPPO AI Center.}$, Jue Wang$^{1,3}$, Huan Li$^{1,2}\thanks{\ Huan Li and Lidan Shou are the corresponding authors.}$ , Lidan Shou$^{1,2 {\dagger}}$, Ke Chen$^{1,2}$, \\ \textbf{Yang You$^{4}$, Guiming Xie$^{5}$, Xuejian Gong$^{5}$, and Kunlong Zhou$^{5}$} \\
  $^1$The State Key Laboratory of Blockchain and Data Security, Zhejiang University \\
  $^2$Hangzhou High-Tech Zone (Binjiang) Institute of Blockchain and Data Security \\
  $^3$College of Computer Science and Technology, Zhejiang University \\
  $^4$Department of Computer Science, National University of Singapore \\
  $^5$AI Center, Guangdong OPPO Mobile Telecommunications Corp., Ltd. \\
  \texttt{\{zj.cs,zjuwangjue,lihuan.cs,should,chenk\}@zju.edu.cn},\\ \texttt{youy@comp.nus.edu.sg}, \texttt{\{xieguiming,gongxuejian,zhoukunlong\}@oppo.com}
}
% \thanks{ Use footnote for providing further information
% about author (webpage, alternative address)---\emph{not} for acknowledging
% funding agencies.  Funding acknowledgements go at the end of the paper.} \\
% The State Key Laboratory of Blockchain and Data Security\\
% Zhejiang University\\
% Pittsburgh, PA 15213, USA \\
% \texttt{\{zj.cs,brain,jen\}@cs.cranberry-lemon.edu} \\
% \And
% Ji Q. Ren \& Yevgeny LeNet \\
% Department of Computational Neuroscience \\
% University of the Witwatersrand \\
% Joburg, South Africa \\
% \texttt{\{robot,net\}@wits.ac.za} \\
% \AND
% Coauthor \\
% Affiliation \\
% Address \\
% \texttt{email}

% The \author macro works with any number of authors. There are two commands
% used to separate the names and addresses of multiple authors: \And and \AND.
%
% Using \And between authors leaves it to \LaTeX{} to determine where to break
% the lines. Using \AND forces a linebreak at that point. So, if \LaTeX{}
% puts 3 of 4 authors names on the first line, and the last on the second
% line, try using \AND instead of \And before the third author name.

\usepackage{xspace}
\newcommand{\method}{\textsc{LoRAM}\xspace}
\newcommand{\methodrand}{\textsc{LoRAM-Rand}\xspace}
\newcommand{\methodstru}{\textsc{LoRAM-Stru}\xspace}
\newcommand{\methodsemi}{\textsc{LoRAM-Semi}\xspace}
\newcommand{\methodunst}{\textsc{LoRAM-Unst}\xspace}
\newcommand{\Qmethod}{\textsc{QLoRAM}\xspace}
\newcommand{\Qmethodrand}{\textsc{QLoRAM-Rand}\xspace}
\newcommand{\Qmethodstru}{\textsc{QLoRAM-Stru}\xspace}

\iclrfinalcopy % Uncomment for camera-ready version, but NOT for submission.
\begin{document}

\maketitle

\begin{abstract}
Large Language Models (LLMs) have significantly advanced natural language processing with exceptional task generalization capabilities. 
% Parameter-Efficient Fine-Tuning (PEFT), exemplified by LoRA, 
Low-Rank Adaption (LoRA) offers a cost-effective fine-tuning solution, freezing the original model parameters and training only lightweight, low-rank adapter matrices.
However, the memory footprint of LoRA is largely dominated by the original model parameters.
% \textcolor{custommagenta}{
% To further reduce the fine-tuning memory footprint, 
To mitigate this, 
we propose \method, a memory-efficient LoRA training scheme
founded on the intuition that many neurons in over-parameterized LLMs have low training utility but are essential for inference. 
\method presents a unique twist: it trains on a pruned (small) model to obtain pruned low-rank matrices, 
which are then
% dimensionally 
recovered and utilized with the original (large) model for inference.
Additionally, minimal-cost continual pre-training,
% }
performed by the model publishers in advance, aligns the knowledge discrepancy between pruned and original models.
Our extensive experiments demonstrate the efficacy of \method across various pruning strategies and downstream tasks. For a model with 70 billion parameters, \method enables training on a GPU with only 20G HBM, replacing an A100-80G GPU for LoRA training and 15 GPUs for full fine-tuning. Specifically, \Qmethod implemented by structured pruning combined with 4-bit quantization, for LLaMA-3.1-70B (LLaMA-2-70B), reduces the parameter storage cost that dominates the memory usage in low-rank matrix training by 15.81×
% \footnote{The reported QLoRAM parameter reduction is based on Hugging Face’s implementation, which underestimates the impact of \texttt{NF4} quantization, implying a higher actual reduction.}
(16.95×), while achieving dominant performance gains over both the original LLaMA-3.1-70B (LLaMA-2-70B) and LoRA-trained LLaMA-3.1-8B (LLaMA-2-13B).
Code is available at \href{https://github.com/junzhang-zj/LoRAM}{\rainbowurl}.



\end{abstract}

\section{Introduction}

Chain-of-Thought (CoT) prompting~\cite{Nye:2021, cot, Kojima:2022cotzero} has emerged as a cornerstone strategy for enhancing Large Language Models (LLMs) in complex reasoning tasks. By eliciting step-by-step inference, CoT enables LLMs to decompose intricate problems into manageable subtasks, thereby improving their problem-solving performance~\cite{Yao:2023tot, Wang:2023self-consistency, Zhou:2023least, Shinn:2023Reflexion}. Recent advancements, such as OpenAI's o1~\cite{o1} and DeepSeek-R1~\cite{deepseekr1}, further demonstrate that scaling up CoT lengths from hundreds to thousands of reasoning steps could continuously improve LLM reasoning. These breakthroughs have underscored CoT’s potential to advance LLM capabilities, expanding the boundaries of AI-driven problem-solving.

\begin{figure}[t]
\centering
    \includegraphics[width=0.95\columnwidth]{fig/intro.pdf}
    \caption{In contrast to vanilla CoT that generates all reasoning tokens sequentially, \method enables LLMs to \textit{skip} tokens with less semantic importance (\textit{e.g.,} \includegraphics[width=7pt]{fig/token.pdf}~) and learn shortcuts between critical reasoning tokens, facilitating controllable CoT compression.}
    \label{fig:intro}
\end{figure}

Despite its effectiveness, the increased length of CoT sequences introduces substantial computational overhead. Due to the autoregressive nature of LLM decoding, longer CoT outputs lead to proportional increases in both inference latency and memory footprints of key-value cache. Additionally, the quadratic computational cost of attention layers further exacerbates this burden. These issues become particularly pronounced when CoT sequences extend into thousands of reasoning steps, resulting in significant computational costs and prolonged response times. While prior research has explored methods for selectively skipping reasoning steps~\cite{Ding:2024cotshortcut, liu2024skipstep}, recent findings~\cite{jin:2024cotlength, Merrill:2024cotlength} suggest that such reductions may conflict with test-time scaling~\cite{o1-blog, snell2025scaling}, ultimately impairing LLM reasoning performance. Therefore, striking an optimal balance between CoT efficiency and reasoning accuracy remains a critical open challenge.

In this work, we delve into CoT efficiency and seek the answer to an important question: \textit{``Does every token in the CoT output contribute equally to deriving the answer?''} We empirically analyze the semantic importance of tokens within CoT outputs and reveal that their contributions to the reasoning performance vary, as depicted in Figure 2. Building on this insight, we introduce \method, a simple yet effective approach that enables LLMs to \textit{skip} less important tokens within CoT sequences and learn shortcuts between critical reasoning tokens, thereby allowing for controllable CoT compression with adjustable ratios. Specifically, as shown in Figure~\ref{fig:intro}, \method constructs compressed CoT training data with various compression ratios, by pruning unimportance tokens from original LLM CoT trajectories. Then, it conducts a general supervised fine-tuning process on target LLMs with this training data, facilitating LLMs to automatically trim redundant tokens during reasoning.

We conduct extensive experiments across various models, including LLaMA-3.1-8B-Instruct and the Qwen2.5-Instruct series, using two widely recognized math reasoning benchmarks: GSM8K and MATH-500. The results validate the effectiveness of \method in compressing CoT outputs while maintaining robust reasoning performance. Notably, Qwen2.5-14B-Instruct exhibits almost \textbf{NO} performance drop (less than $0.4\%$) with a $\bm{40\%}$ reduction in token usage on GSM8K. On the challenging MATH-500 dataset, LLaMA-3.1-8B-Instruct effectively reduces CoT token usage by $\bm{30}\%$ with a performance decline of less than $4\%$, resulting in a $\bm{1.4}\times$ inference speedup. Further analysis underscores the coherence of \method in specified compression ratios and its potential scalability with stronger compression techniques.

\method is distinguished by its low training cost. For Qwen2.5-14B-Instruct, \method fine-tunes only 0.2\% of the model's parameters using LoRA. The size of the compressed CoT training data is no larger than that of the original training set, with 7,473 examples in GSM8K and 7,500 in MATH. The training is completed in approximately 2 hours for the 7B model and 2.5 hours for the 14B model on two 3090 GPUs. These characteristics make \method an efficient and reproducible approach, suitable for use in efficient and cost-effective LLM deployment.

To sum up, our key contributions are:
\begin{enumerate}
    \item To the best of our knowledge, this work is the \textit{first} to investigate the potential of enhancing CoT efficiency through \textit{token skipping}, inspired by the varying semantic importance of tokens in CoT trajectories of LLMs.
    \item We introduce \method, a simple yet effective approach that enables LLMs to skip redundant tokens within CoTs and learn shortcuts between critical tokens, facilitating CoT compression with adjustable ratios.
    \item Our experiments validate the effectiveness of \method. When applied to Qwen2.5-14B-Instruct, \method reduces reasoning tokens by $40\%$ (from 313 to 181) on GSM8K, with less than a $0.4\%$ performance drop.
\end{enumerate}

% \section{Related Work}
Researchers have been leveraging eye tracking methodologies from human perception research to model how people perceive images~\cite{shanmuga2015eye, bonhage2015combined, conklin2016using}.
These models help assess the appearance and salience of visual representations, enabling eye movement tracking to understand the perceptual and cognitive mechanisms of scene perception~\cite{itti1998model} and object detection~\cite{borji2015salient}.
The existing saliency models perform well in naturalistic scenes
%and real-world object detection
; however, there are unique perception rules and cognitive biases in the artificial world of data visualization 
%does not always follow the rules of perception in the natural world
~\cite{franconeri2021science, correll2012comparing, polatsek2018exploring, knittel2024gridlines}, and, thus, these models do not accurately predict where people would look in visualizations. 
Visualization researchers have been building visual saliency models geared to visualizations~\cite{DVSaliencyModel2017Matzen, bylinskii2016should}. %and adopting them for predicting eye gaze on visualizations. %enabling the prediction of visual saliency across design styles~\cite{fosco2020predicting}.
However, these models rely on handcrafted features, making it difficult to generalize to complex visualizations. Additionally, these models cannot incorporate textual information to generate task-specific saliency maps since the prediction is solely based on visual inputs.

With the advent of deep learning, gaze data were used as the ground truth of saliency models~\cite{fosco2020predicting, scannerDeeply, scanpath}, leading to higher performance in saliency prediction while enabling task-specific saliency~\cite{salchartQA}. 
These models usually need large-scale datasets to learn complex patterns. However, gathering precise gaze data is 
%challenging and requires specialized eye-tracking devices. While these devices provide accurate results, they tend to be 
costly and cumbersome, which limits large-scale data collection efforts. 
Many researchers, therefore, proposed several proxies for eye gaze. WebGaze~\cite{webgaze} uses a webcam for cheap and easy deployment in online studies yet suffers from data quality issues due to low-resolution cameras and uncontrolled calibration.
Therefore, mouse-(cursor-)based annotation tools~\cite{jiang2015salicon,bubbleView,importAnnot} were proposed to improve data quality. Among these methods, BubbleView~\cite{bubbleView} was the most used tool for capturing visual saliency and importance~\cite{graphicDesignImportance, salchartQA}.
However, BubbleView is primarily designed for exploring images and gathering information, which differs slightly from the goal of capturing perceived importance. As a result, while BubbleView is well-suited for measuring visual saliency, it may not be the best tool for capturing %instruction-tuned \yao{I would keep it consistent saying task-specific}
task-specific importance~\cite{turkeyes}. Built upon these prior approaches' limitations, our Grid Labeling aims to collect responses that cover all essential areas of the visualization with minimum noise, leading to more efficient data collection.



% One key motivation to our grid-based approach is to help people 
% We also demonstrate that the grid-based approaches can minimize biases in annotation to disproportionally emphasize text elements~\cite{DVSaliencyModel2017Matzen}
% % blurring the visualization can disproportionately emphasize text elements~\cite{DVSaliencyModel2017Matzen}, potentially misrepresenting a user's true areas of interest.
% More recently, 
% % \ms{Changed a bit using Yao's work (task-dependent saliency), but not sure whether it looks ok}
% Yao et al.~\cite{salchartQA} collect task-dependent saliency using the BubbleView method, % but their approach had some limitations. 
% and made a significant improvement on existing saliency models.
% First, the blurred visualization allowed users to perceive the overall structure of the chart, which prevented the system from capturing the specific action of identifying the maximum value. However, increasing the blur to address this issue introduced another challenge. As the structure became less visible, users had to explore the entire image, leading to the consideration of irrelevant regions as salient.
\begin{figure*}[t!]
    \centering
    \includegraphics[width=\linewidth]{Figure/TokenSwift.pdf}
    % \vskip -0.1 in
    \caption{\textbf{Illustration of \ours Framework.} First, target model (LLM) with partial KV cache and three linear layers outputs 4 logits in a single forward pass. Tree-based attention is then applied to construct candidate tokens. Secondly, top-$k$ candidate $4$-grams are retrieved accordingly. These candidates compose draft tokens, which are fed into the LLM with full KV cache to generate target tokens. The verification is performed by checking if draft tokens match exactly with target tokens (\cref{alg:algorithm}). Finally, we randomly select one of the longest valid draft tokens, and update $n$-gram table and KV cache accordingly.}
    \label{fig:frame}
    % \vskip -0.15 in
\end{figure*}

\section{\ours}
\label{sec:method}
To achieve \textbf{lossless acceleration in generating ultra-long sequences}, we propose tailored solutions for each challenge inherent to this process. These solutions are seamlessly integrated into a unified framework, \ie \ours.

\subsection{Overview}
\label{sec:overall}
The overall framework is depicted in \cref{fig:frame}. \ours 
% is highly lightweight and conceptually similar to \ac{sd}. It 
generate a sequence of draft tokens with self-drafting, which are then passed to the target (full) model for validation using a tree-based attention mechanism (See \cref{app:tree_attn} for more tree-based attention details). This process ensures that the final generated output aligns with the target model’s predictions, effectively achieving lossless acceleration.

\ours is lightweight because the draft model is the target model itself with a partial KV cache. This eliminates the need to train a separate draft \ac{llm}; instead, only $\gamma$ linear layers need to be trained, where $\gamma + 1$\footnote{The target model itself can also predict one logit, making the total number of logits $\gamma+1$. We take $\gamma=3$.} represents the number of logits predicted in a single forward pass. In addition, during the verification process, once we obtain the target tokens from the target model with full KV cache, we directly compare draft tokens with target tokens sequentially to ensure that the process is lossless~\citep{rest}.

\subsection{Multi-token Generation and Token Reutilization}
\label{sec:multi_token}
\paragraph{Multi-token Self-Drafting} 
% Inspired by Medusa~\citep{medusa}, we propose a modification where the final output of \ac{llm} is used as input to train $3$ linear layers, enabling the model to generate multiple draft tokens in a single forward pass. However, we argue that the generated draft tokens should not be independent of each other. Unlike Medusa, where the linear layers operate entirely independently, we introduce a simple adjustment to this structure. 
Inspired by Medusa~\citep{medusa}, we enable the \ac{llm} to generate multiple draft tokens in a single forward pass by incorporating $\gamma$ additional linear layers. However, we empirically note that \textbf{the additional linear layers should not be independent of each other}. Specifically, we propose the following structure:
\begin{equation}
\label{equ:ours}
% \small
% \resizebox{.9\hsize}{!}{
% $
    \begin{aligned}
    h_1=f_1(h_0) + h_0,\quad{}h_2=&f_2(h_1) + h_1,\quad{}h_3=f_3(h_2) + h_2,\\
l_0,~l_{1},~l_{2},~l_{3}=&~g(h_0),~g(h_1),~g(h_2),~g(h_3),
    \end{aligned}
% $
% }
\end{equation}
where $h_0$ denotes the last hidden state of \ac{llm}, $f_i(\cdot)$ represents the $i$-th linear layer, $h_i$ refers to the $i$-th hidden representation, $g(\cdot)$ represents the LM Head of target model, and $l_i$ denotes output logits.
% By comparing \cref{equ:medsua} (Medusa) and \cref{equ:ours} (\ours), it is evident that in \ours, the generation of each token depends on the previously generated token, which aligns more closely with the \ac{ar} nature of the model. Moreover, this adjustment incurs no additional computational cost.
This structure aligns more closely with the \ac{ar} nature of the model. Moreover, this adjustment incurs no additional computational cost.
\vspace{-0.05 in}
\paragraph{Token Reutilization} 
% Given the relatively low acceptance rate of using linear to approximate the entire \ac{llm} for generating draft tokens, we propose a method named \textbf{token reutilization}  to further reduce the frequency of model reloads. 
Given the relatively low acceptance rate of using linear layers to generate draft tokens, we propose a method named \textbf{token reutilization} to further reduce the frequency of model reloads. The idea behind token reutilization is that some phrases could appear frequently, and they are likely to reappear in subsequent generations.

% Specifically, we define $(\mathcal{G}, \mathcal{F})$, where $\mathcal{G}=\{x_{i+1}, ..., x_{i+n}\}$ represents an $n$-gram and $\mathcal{F}$ denotes its corresponding frequency $\mathcal{F}$ within the generated token sequence $S=\{x_0, x_1, ..., x_{t-1}\}$ by time step $t$ ($t \geq n$). At subsequent time steps, we use the token generated by target model as the first token to select top-$k$ most frequent $n$-grams $\{\mathcal{G}_1, \mathcal{G}_2,...,\mathcal{G}_k\}$ and incorporate them as additional draft tokens. These selected draft tokens, along with the newly generated ones, are then fed to the \ac{llm} for parallel validation. 
Specifically, we maintain a set of tuples $\{(\mathcal{G}, \mathcal{F})\}$, where $\mathcal{G}=\{x_{i+1}, ..., x_{i+n}\}$ represents an $n$-gram and $\mathcal{F}$ denotes its corresponding frequency $\mathcal{F}$ within the generated token sequence $S=\{x_0, x_1, ..., x_{t-1}\}$ by time step $t$ ($t \geq n$). After obtaining $\{p_0,\ldots, p_3\}$ as described in \S \ref{sec:penalty}, we retrieve the top-$k$ most frequent $n$-grams beginning with token $\arg\max p_0$ to serve as additional draft tokens.

Although this method can be applied to tasks with long prefixes, its efficacy is constrained by the limited decoding steps, which reduces the opportunities for accepting $n$-gram candidates. Additionally, since the long prefix text is not generated by the \ac{llm} itself, a distributional discrepancy exists between the generated text and the authentic text~\citep{detectgpt}. As a result, this method is particularly suitable for generating ultra-long sequences.
 
% \subsection{Dynamic and Memory-Saving KV Pruning}
\subsection{Dynamic KV Cache Management}
\label{sec:kv_update}
\paragraph{Dynamic KV Cache Updates}
Building upon the findings of~\citet{stram_llm}, we preserve the initial $|S|$ KV pairs within the cache during the drafting process, while progressively evicting less important KV pairs. Specifically, we enforce a fixed budget size $|B|$, ensuring that the KV cache at any given time can be represented as:
\begin{equation}
    \nonumber
    % \resizebox{\hsize}{!}{$
    \mathbf{KV}=\{(\mathbf{K}_0,\mathbf{V}_0), ..., (\mathbf{K}_{|S|},\mathbf{V}_{|S|}), (\mathbf{K}_{|S|+1},\mathbf{V}_{|S|+1}),..., (\mathbf{K}_{|B|-1},\mathbf{V}_{|B|-1})\},
   % $},
\end{equation}
where the first $|S|$ pairs remain fixed, and the pairs from position $|S|$ to $|B|-1$ are ordered by decreasing importance. 
As new tokens are generated, less important KV pairs are gradually replaced, starting from the least important ones at position $|B|-1$ and moving towards position $|S|$. Once replacements extend beyond the $|S|$ position, we recalculate the \textit{importance scores} of all preceding tokens and select the most relevant $|B|-|S|$ pairs to reconstruct the cache. 
This process consistently preserves the critical information required for ultra-long sequence generation. 
\vspace{-0.05 in}
% \paragraph{Memory-Saving Top-K Pruning} 
% To implement dynamic updates efficiently, we employ a simple yet effective Top-K pruning strategy. Specifically, we rank the KV pairs based on the importance scores derived from the dot product between the query ($\mathbf{Q}$) and key ($\mathbf{K}$), \ie $\mathbf{Q}\mathbf{K}^T$. 
\paragraph{Importance Score of KV pairs} 
We rank the KV pairs based on the \textit{importance scores} derived from the dot product between the query ($\mathbf{Q}$) and key ($\mathbf{K}$), \ie $\mathbf{Q}\mathbf{K}^T$. 

In the case of Group Query Attention (GQA), since each $\mathbf{K}$ corresponds to a group of $\mathcal{Q}=\{\mathbf{Q}_0, ..., \mathbf{Q}_{g-1}\}$, direct dot-product computation is not feasible. Unlike methods such as SnapKV~\citep{snapkv}, we do not replicate the $\mathbf{K}$. Instead, we partition the $\mathcal{Q}$, as shown in \cref{equ:gqa}:
\begin{equation}
    \label{equ:gqa}
    \vspace{-2mm}
    \text{importance score}_i = \sum_{j=i\cdot g}^{((i+1)\cdot g)-1}\mathbf{Q}_j \cdot \mathbf{K}_i,
        % \vspace{-2mm}
\end{equation}
where for position $i$, $\mathbf{Q}_j$ in the group $\mathcal{Q}_i$ are dot-product with the same $\mathbf{K}_i$, and their results are aggregated to obtain the final \textit{importance score}. This approach enhances memory saving while preserving the quality of the attention mechanism, ensuring that each query is effectively utilized without introducing unnecessary redundancy.

\subsection{Contextual Penalty and Random N-gram Selection}
\label{sec:penalty}
% \paragraph{Contextual Length Penalty} 
\paragraph{Contextual Penalty} 
% To mitigate repetition in generated text, we have explored various sampling strategies. However, with the significantly larger sequence length, the likelihood of repetition increases compared to generating shorter texts (\cref{sec:repeat}). As a result, we decided to apply an additional penalty to the generated tokens to further mitigate repetition.
To mitigate repetition in generated text, we have explored various sampling strategies. However, with the significantly larger sequence length, the likelihood of repetition increases significantly (\S \ref{sec:repeat}). As a result, we decided to apply an additional penalty to the generated tokens to further mitigate repetition.

The penalized sampling approach proposed in \citep{penalty} suggests applying a penalty to all generated tokens. However, when generating ultra-long sequences, the set of generated tokens may cover nearly all common words, which limits the ability to sample appropriate tokens. Therefore, we propose an improvement to this method. 

Specifically, we introduce a fixed \emph{penalty window} $W$ and apply \emph{penalty value} $\theta$ to the most recent $W$ tokens, denoted as $\mathbb{W}$, generated up to the current position, as illustrated in \cref{equ:repeat}: 
\begin{equation}
% \small
    \label{equ:repeat}
% \vspace{-3mm}
    \begin{aligned}
        p_i &= \frac{\exp \big(l_i/(t\cdot I(l_i))\big)}{\sum_j \exp \big(l_j/(t\cdot I(l_j))\big)},\\
    I(l)=\theta\,\,&\text{if}\,\,l \in \mathbb{W}\,\text{else}\,\,1.0,\quad \theta \in (1, \infty),
    \end{aligned}
    % \vspace{-1mm}
\end{equation}
where $t$ denotes temperature, $l_i$ and $p_i$ represent the logit and probability of $i$-th token. This adjustment aims to maintain diversity while still mitigating repetitive generation.

\section{\thename}
\subsection{End-to-End Driving Policy}
The overall framework of \thename{} is depicted in Fig.~\ref{fig:framework}. 
\thename{} takes multi-view image sequences as input, transforms the sensor data into scene token embeddings, outputs the probabilistic distribution of actions, and samples an action to control the vehicle. 

\boldparagraph{BEV Encoder.} 
We first employ a BEV encoder~\cite{li2022bevformer} to transform multi-view image features from the perspective view to the Bird's Eye View (BEV), obtaining a feature map in the BEV space. This feature map is then used to learn instance-level map features and agent features.

\boldparagraph{Map Head.} 
Then we utilize a group of map tokens~\cite{maptrv2, liao2022maptr, lanegap} to learn the vectorized map elements of the driving scene from the BEV feature map, including lane centerlines, lane dividers, road boundaries, arrows, traffic signals, \etc.

\boldparagraph{Agent Head.} 
Besides, a group of agent tokens~\cite{jiang2022pip} is adopted to predict the motion information of other traffic participants, including location, orientation, size, speed, and multi-mode future trajectories.

\boldparagraph{Image Encoder.} 
Apart from the above instance-level map and agent tokens, we also use an individual image encoder~\cite{vit,he2016resnet} to transform the original images into image tokens. These image tokens provide dense and rich scene information for planning, complementary to the instance-level tokens.

\begin{figure}[t]
\centering
\includegraphics[width=0.98\linewidth]{fig/post-training-2.pdf} 
\caption{\textbf{Post-training.}  $N$  workers parallelly run. The generated rollout data $(s_t,a_t, r_{t+1},s_{t+1},...)$ are recorded in a rollout buffer. Rollout data and human driving demonstrations are used in RL- and IL-training steps to fine-tune the AD policy synergistically.
}
\label{fig:post-training}
\end{figure}

\boldparagraph{Action Space.} 
To accelerate the convergence of RL training, we design a decoupled discrete action representation. 
We divide the action into two independent components: lateral action and longitudinal action. 
The action space is constructed over a short $0.5$-second time horizon, during which the vehicle's motion is approximated by assuming constant linear and angular velocities. 
Under this assumption, the lateral action $a^x$ and longitudinal action $a^y$ can be directly computed based on the current linear and angular velocities.
By combining decoupling with a limited temporal scope and simplified motion model, our approach effectively reduces the dimensionality of the action space, accelerating training convergence.


\boldparagraph{Planning Head.} 
We use $E_\text{scene}$ to denote the scene representation, which consists of map tokens, agent tokens, and image tokens. We initialize a planning embedding denoted as $E_\text{plan}$. A cascaded Transformer decoder $\phi$ takes the planning embedding $E_\text{plan}$ as the query and the scene representation $E_\text{scene}$ as both key and value.

The output of the decoder $\phi$ is then combined with navigation information $E_\text{navi}$ and ego state $E_\text{state}$ to output the probabilistic distributions of the lateral action $a^x$ and the longitudinal action $a^y$:
\begin{equation}
\begin{aligned}
     \pi(a^x\mid s) = & \text{softmax}(\text{MLP}(\phi(E_\text{plan}, E_\text{scene}) \\
    & + E_\text{navi} + E_\text{state})), \\
     \pi(a^y\mid s) = & \text{softmax}(\text{MLP}(\phi(E_\text{plan}, E_\text{scene}) \\
     & + E_\text{navi} + E_\text{state})),
\label{eq:action distribution}
\end{aligned}
\end{equation}
where $E_\text{plan}$, $E_\text{navi}$, $E_\text{state}$, and the output of $\text{MLP}$ are all of the same dimension ($1 \times D$).

The planning head also outputs the value functions $V_x(s)$ and $V_y(s)$, which estimate the expected cumulative rewards for the lateral and longitudinal actions, respectively: 
\begin{equation}
\begin{aligned}
    & V_x(s) = \text{MLP}(\phi(E_\text{plan}, E_\text{scene}) + E_\text{navi} + E_\text{state}), \\
    & V_y(s) = \text{MLP}(\phi(E_\text{plan}, E_\text{scene}) + E_\text{navi} + E_\text{state}).
\end{aligned}
\end{equation}
The value functions are used in RL training (Sec.~\ref{sec:optimization}).

\subsection{Training Paradigm}
We adopt a three-stage training paradigm: perception pre-training, planning pre-training, and reinforced post-training, as shown in Fig.~\ref{fig:framework}.

\boldparagraph{Perception Pre-Training.} 
Information in the image is sparse and low-level. In the first stage,  
the map head and the agent head explicitly output map elements and agent motion information, which are supervised with ground-truth labels. Consequently,  
map tokens and agent tokens implicitly encode the corresponding high-level information.  
In this stage, we only update the parameters of the BEV encoder, the map head, and the agent head.



\boldparagraph{Planning Pre-Training.} 
In the second stage, to prevent the unstable cold start of RL training, IL is first performed to initialize the probabilistic distribution of actions based on large-scale real-world driving demonstrations from expert drivers. In this stage, we only update the parameters of the image encoder and the planning head, while the parameters of the BEV encoder, map head, and agent head are frozen. The optimization objectives of perception tasks and planning tasks may conflict with each other. However, with the training stage and parameters decoupled, such conflicts are mostly avoided.

\boldparagraph{Reinforced Post-Training.} 
In the reinforced post-training, RL and IL synergistically fine-tune the distribution. RL aims to guide the policy to be sensitive to critical risky events and adaptive to out-of-distribution situations. IL serves as the regularization term to keep the policy's behavior similar to that of humans.

We select a large amount of risky dense-traffic clips from collected driving demonstrations. For each clip, we train an independent 3DGS model that reconstructs the clip and serves as a digital driving environment.  
As shown in Fig.~\ref{fig:post-training}, we set $N$ parallel workers.  
Each worker randomly samples a 3DGS environment and begins rollout, i.e., the AD policy controls the ego vehicle to move and iteratively interacts with the 3DGS environment. After the rollout process of this 3DGS environment ends, the generated rollout data $(s_t,a_t, r_{t+1},s_{t+1},...)$ are recorded in a rollout buffer, and the worker will sample a new 3DGS environment for another round of rollout.

As for policy optimization, we iteratively perform RL-training steps and IL-training steps. For RL-training steps, we sample data from the rollout buffer and follow the Proximal Policy Optimization (PPO) framework~\cite{PPO} to update the AD policy. For IL-training steps, we use real-world driving demonstrations to update the policy. After a fixed number of training steps, the updated AD policy is sent to every worker to replace the old one, to avoid a distribution shift between data collection and optimization.
We only update the parameters of the image encoder and the planning head. The parameters of the BEV encoder, the map head, and the agent head are frozen.  
The detailed RL design is presented below.

\subsection{Interaction Mechanism between AD Policy and 3DGS Environment}
In the 3DGS environment, the ego vehicle acts according to the AD policy. Other traffic participants act according to real-world data in a log-replay manner.  
A simplified kinematic bicycle model is employed to iteratively update the ego vehicle's pose at every $\Delta t$ seconds as follows:  
\begin{equation}
\begin{aligned}
x_{t+1}^{w} & = x_{t}^w + v_t \cos \left(\psi_{t}^w\right) \Delta t, \\
y_{t+1}^{w} & = y_{t}^w + v_t \sin \left(\psi_{t}^w\right) \Delta t, \\
\psi_{t+1}^{w} & = \psi_{t}^w + \frac{v_t}{L} \tan \left(\delta_t\right) \Delta t,
\label{equation:kinematic_model}
\end{aligned}
\end{equation}  
where $x_t^{w}$ and $y_t^{w}$ denote the position of the ego vehicle relative to the world coordinate; $\psi_t^w$ is the heading angle that defines the vehicle's orientation with respect to the world $x$-coordinate; $v_t$ is the linear velocity of the ego vehicle; $\delta_t$ is the steering angle of the front wheels; and $L$ is the wheelbase, i.e., the distance between the front and rear axles.

During the rollout process, the AD policy outputs actions $(a_t^x, a_t^y)$ for a $0.5$-second time horizon at time step $t$. We derive the linear velocity $v_t$ and steering angle $\delta_t$ based on $(a_t^x, a_t^y)$.  
Based on the kinematic model in Eq.~\ref{equation:kinematic_model},  
the pose of the ego vehicle in the world coordinate system is updated from ${p}_t = (x_{t}^w, y_{t}^w, \psi_{t}^w)$ to ${p}_{t+1} = (x_{t+1}^{w}, y_{t+1}^{w}, \psi_{t+1}^{w})$.  

Based on the updated ${p}_{t+1}$, the 3DGS environment computes the new ego vehicle's state $s_{t+1}$. The updated pose ${p}_{t+1}$ and state $s_{t+1}$ serve as the input for the next iteration of the inference process.

The 3DGS environment also generates rewards $\mathcal{R}$ (Sec.~\ref{sec:reward}) according to multi-source information (including trajectories of other agents, map information, the expert trajectory of the ego vehicle, and the parameters of Gaussians), which are used to optimize the AD policy (Sec.~\ref{sec:optimization}).

\begin{figure}[t]
\centering
\includegraphics[width=1.0\linewidth]{fig/reward.pdf} 
\caption{\textbf{Example diagram of four types of reward sources.}  (1): Collision with a dynamic obstacle ahead triggers a reward $r_{\text{dc}}$. (2): Hitting a static roadside obstacle incurs a reward $r_{\text{sc}}$. (3): Moving onto the curb exceeds the positional deviation threshold $d_{\text{max}}$, triggering a reward $r_{\text{pd}}$. (4): Drifting toward the adjacent lane exceeds the heading deviation threshold $\psi_{\text{max}}$, triggering a reward $r_{\text{hd}}$.
}
\label{fig: reward source}
\end{figure}
\subsection{Reward Modeling}
\label{sec:reward}
The reward is the source of the training signal, which determines the optimization direction of RL. The reward function is designed to guide the ego vehicle's behavior by penalizing unsafe actions and encouraging alignment with the expert trajectory. It is composed of four reward components: (1) collision with dynamic obstacles, (2) collision with static obstacles, (3) positional deviation from the expert trajectory, and (4) heading deviation from the expert trajectory:
\begin{equation}
\begin{aligned}
\mathcal{R} = \{r_{\text{dc}}, r_{\text{sc}}, r_{\text{pd}}, r_{\text{hd}}  \}. 
\end{aligned}
\end{equation}

As illustrated in Fig.~\ref{fig: reward source}, these reward components are triggered under specific conditions.  
In the 3DGS environment, dynamic collision is detected if the ego vehicle's bounding box overlaps with the annotated bounding boxes of dynamic obstacles, triggering a negative reward $r_{\text{dc}}$. Similarly, static collision is identified when the ego vehicle's bounding box overlaps with the Gaussians of static obstacles, resulting in a negative reward $r_{\text{sc}}$.  
Positional deviation is measured as the Euclidean distance between the ego vehicle's current position and the closest point on the expert trajectory. A deviation beyond a predefined threshold $d_{\text{max}}$ incurs a negative reward $r_{\text{pd}}$.  
Heading deviation is calculated as the angular difference between the ego vehicle's current heading angle $ \psi_t $ and the expert trajectory's matched heading angle $\psi_{\text{expert}}$. A deviation beyond a threshold $ \psi_{\text{max}}$ results in a negative reward $r_{\text{hd}}$.

Any of these events, including dynamic collision, static collision, excessive positional deviation, or excessive heading deviation, triggers immediate episode termination. Because after such events occur, the 3DGS environment typically generates noisy sensor data, which is detrimental to RL training.

\subsection{Policy Optimization}
\label{sec:optimization}
In the closed-loop environment, the error in each single step accumulates over time. The aforementioned rewards are not only caused by the current action but also by the actions of the preceding steps.  
The rewards are propagated forward with Generalized Advantage Estimation (GAE)~\cite{gae} to optimize the action distribution of the preceding steps.

Specifically, for each time step $t$, we store the current state $s_t$, action $a_t$, reward $r_t$, and the estimate of the value $V(s_t)$.  
Based on the decoupled action space, and considering that different rewards have different correlations to lateral and longitudinal actions, the reward $r_t$ is divided into lateral reward $r_t^x$ and longitudinal reward $r_t^y$:
\begin{equation}
\begin{aligned}
r_t^x &= r_t^{\text{sc}} + r_t^{\text{pd}} + r_t^{\text{hd}}, \\
r_t^y &= r_t^{\text{dc}}.
\label{eq:reward-decouple}
\end{aligned}
\end{equation}
Similarly, the value function $V(s_t)$ is decoupled into two components: $V_x(s_t)$ for the lateral dimension and $V_y(s_t)$ for the longitudinal dimension. These value functions estimate the expected cumulative rewards for the lateral and longitudinal actions, respectively. The advantage estimates $\hat{A}_t^x$ and $\hat{A}_t^y$ are then computed as follows:
\begin{equation}
\begin{aligned}
\delta_t^x &= r_t^x + \gamma V_x(s_{t+1}) - V_x(s_t), \\
\delta_t^y &= r_t^y + \gamma V_y(s_{t+1}) - V_y(s_t), \\
\hat{A}_t^x &= \sum_{l=0}^{\infty}(\gamma \lambda)^l \delta_{t+l}^x, \\
\hat{A}_t^y &= \sum_{l=0}^{\infty}(\gamma \lambda)^l \delta_{t+l}^y,
\label{eq:advantage}
\end{aligned}
\end{equation}
where $\delta_t^x$ and $\delta_t^y$ are the temporal difference errors for the lateral and longitudinal dimensions, $\gamma$ is the discount factor, and $\lambda$ is the GAE parameter that controls the trade-off between bias and variance.

To further clarify the relationship between the advantage estimates and the reward components, we decompose $\hat{A}_t^x$ and $\hat{A}_t^y$ based on the reward decomposition in Eq.~\ref{eq:reward-decouple} and the advantage estimation in Eq.~\ref{eq:advantage}. Specifically, we derive the following decomposition:
\begin{equation}
\begin{aligned}
\hat{A}_t^x &= \hat{A}_t^{\text{sc}} + \hat{A}_t^{\text{pd}} + \hat{A}_t^{\text{hd}}, \\
\hat{A}_t^y &= \hat{A}_t^{\text{dc}},
\end{aligned}
\end{equation}
where $\hat{A}_t^{\text{sc}}$ is the advantage estimate for avoiding static collisions, $\hat{A}_t^{\text{pd}}$ is the advantage estimate for minimizing positional deviations, $\hat{A}_t^{\text{hd}}$ is the advantage estimate for minimizing heading deviations, and $\hat{A}_t^{\text{dc}}$ is the advantage estimate for avoiding dynamic collisions.

These advantage estimates are used to guide the update of the AD policy $\pi_{\theta}$, following the PPO framework~\cite{PPO}. By leveraging the decomposed advantage estimates $\hat{A}_t^x$ and $\hat{A}_t^y$, we can independently optimize the lateral and longitudinal dimensions of the policy. This is achieved by defining separate objective functions $\mathcal{L}_x^{\text{CLIP}}(\theta)$ and $\mathcal{L}_y^{\text{CLIP}}(\theta)$ for each dimension,  as follows:
\begin{equation}
\begin{aligned}
\mathcal{L}_x^{\text{PPO}}(\theta) &= \mathbb{E}_t \left[ \min \left( \rho_t^x \hat{A}_t^x, \ \text{clip}(\rho_t^x, 1-\epsilon_x, 1+\epsilon_x) \hat{A}_t^x \right) \right], \\
\mathcal{L}_y^{\text{PPO}}(\theta) &= \mathbb{E}_t \left[ \min \left( \rho_t^y \hat{A}_t^y, \ \text{clip}(\rho_t^y, 1-\epsilon_y, 1+\epsilon_y) \hat{A}_t^y \right) \right], \\
\mathcal{L}^{\text{PPO}}(\theta) &= \mathcal{L}_x^{\text{PPO}}(\theta) + \mathcal{L}_y^{\text{PPO}}(\theta),
\end{aligned}
\end{equation}
where $\rho_t^x = \frac{\pi_{\theta}(a_t^x \mid s_t)}{\pi_{\theta_{\text{old}}}(a_t^x \mid s_t)}$ is the importance sampling ratio for the lateral dimension, $\rho_t^y = \frac{\pi_{\theta}(a_t^y \mid s_t)}{\pi_{\theta_{\text{old}}}(a_t^y \mid s_t)}$ is the importance sampling ratio for the longitudinal dimension, $\epsilon_x$ and $\epsilon_y$ are small constants that control the clipping range for the lateral and longitudinal dimensions, ensuring stable policy updates.

The clipped objective function $\mathcal{L}^{\text{PPO}}(\theta)$ prevents excessively large updates to the policy parameters $\theta$, thereby maintaining training stability.

\begin{table*}[ht]
    \centering
{
\begin{tabular}{lccccccccc}
    \toprule
    RL:IL & CR$\downarrow$ & DCR$\downarrow$ & SCR$\downarrow$ & DR$\downarrow$ & PDR$\downarrow$ & HDR$\downarrow$ &ADD$\downarrow$ & Long. Jerk$\downarrow$ & Lat. Jerk$\downarrow$ \\
    \midrule
     0:1  & 0.229 & 0.211 & 0.018 & 0.066 & 0.039 & 0.027  & 0.238 & 3.928 & 0.103\\
     1:0  & 0.143 & 0.128 & 0.015 &0.080 &0.065 &0.015 &0.345 &4.204 &0.085\\
     2:1 & 0.137 & 0.125 & 0.012 & 0.059 & 0.050 & 0.009  & 0.274 & 4.538 & 0.092\\
     4:1 & 0.089 & 0.080 & 0.009 & 0.063 & 0.042 & 0.021  & 0.257 & 4.495 & 0.082 \\
     8:1 & 0.125 & 0.116 & 0.009 & 0.084 & 0.045 & 0.039  & 0.323 & 5.285 & 0.115\\
    \bottomrule
\end{tabular}
}
    \caption{\textbf{Ablation on RL-to-IL step mixing ratios in the reinforced post-training stage.}}
    \label{tab:ratio}
\end{table*}

\subsection{Auxiliary Objective}
RL usually faces the challenge of sparse rewards, which makes the convergence process unstable and slow. To speed up convergence, we introduce auxiliary objectives that provide dense guidance to the entire action distribution.

The auxiliary objectives are designed to penalize undesirable behaviors by incorporating specific reward sources, including dynamic collisions, static collisions, positional deviations, and heading deviations. These objectives are computed based on the actions \( a_t^{x, \text{old}} \) and \( a_t^{y, \text{old}} \) selected by the old AD policy \( \pi_{\theta_{\text{old}}} \) at time step \( t \). To facilitate the evaluation of these actions, we separate the probability distribution of the action into four parts:
\begin{equation}
\begin{aligned}
\Delta \pi_y^{\text{dec}} &= \sum_{a_t^y < a_t^{y, \text{old}}} \pi_\theta(a_t^y \mid s_t), \\
\Delta \pi_y^{\text{acc}} &= \sum_{a_t^y > a_t^{y, \text{old}}} \pi_\theta(a_t^y \mid s_t), \\
\Delta \pi_x^{\text{left}} &= \sum_{a_t^x < a_t^{x, \text{old}}} \pi_\theta(a_t^x \mid s_t), \\
\Delta \pi_x^{\text{right}} &= \sum_{a_t^x > a_t^{x, \text{old}}} \pi_\theta(a_t^x \mid s_t).
\end{aligned}
\end{equation}
Here, \( \Delta \pi_y^{\text{dec}} \) represents the total probability of deceleration actions, \( \Delta \pi_y^{\text{acc}} \) represents the total probability of acceleration actions, \( \Delta \pi_x^{\text{left}} \) represents the total probability of leftward steering actions, and \( \Delta \pi_x^{\text{right}} \) represents the total probability of rightward steering actions.

\boldparagraph{Dynamic Collision Auxiliary Objective.}  
The dynamic collision auxiliary objective adjusts the longitudinal control action \(a_t^y\) based on the location of potential collisions relative to the ego vehicle. If a collision is detected ahead, the policy prioritizes deceleration actions (\(a_t^y < a_t^{y, \text{old}}\)); if a collision is detected behind, it encourages acceleration actions (\(a_t^y > a_t^{y, \text{old}}\)). To formalize this behavior, we define a directional factor \(f_\text{dc}\):
\begin{equation}
\begin{aligned}
f_\text{dc} = \begin{cases} 
1 & \text{if the collision is ahead}, \\
-1 & \text{if the collision is behind}.
\end{cases} 
\end{aligned}
\end{equation}

The auxiliary objective for dynamic collision avoidance is defined as:
\begin{equation}
\begin{aligned}
\mathcal{L}_\text{dc}(\theta_y) = \mathbb{E}_t \left[ 
    \hat{A}_t^\text{dc} \cdot f_\text{dc} \cdot (\Delta \pi_y^{\text{dec}} - \Delta \pi_y^{\text{acc}})
\right],
\end{aligned}
\end{equation}
where \(\hat{A}_t^\text{dc}\) is the advantage estimate for dynamic collision avoidance.

\boldparagraph{Static Collision Auxiliary Objective.}  
The static collision auxiliary objective adjusts the steering control action $a_t^x$ based on the proximity to static obstacles. If the static obstacle is detected on the left side, the policy promotes rightward steering actions ($a_t^x > a_t^{x,\text{old}}$); if the static obstacle is detected on the right side, it promotes leftward steering actions ($a_t^x < a_t^{x,\text{old}}$). To formalize this behavior, we define a directional factor $f_\text{sc}$:  
\begin{equation}
\begin{aligned}
f_\text{sc} = \begin{cases} 
1 & \text{if static obstacle is on the left}, \\
-1 & \text{if static obstacle is on the right}.
\end{cases} 
\end{aligned}
\end{equation}

The auxiliary objective for static collision avoidance is defined as:  
\begin{equation}
\begin{aligned}
\mathcal{L}_\text{sc}(\theta_x) = \mathbb{E}_t \left[ 
    \hat{A}_t^\text{sc} \cdot f_\text{sc} \cdot (\Delta \pi_x^{\text{right}} - \Delta \pi_x^{\text{left}})
\right],
\end{aligned}
\end{equation}  
where $\hat{A}_t^\text{sc}$ is the advantage estimate for static collision avoidance.  

\boldparagraph{Positional Deviation Auxiliary Objective.}  
The positional deviation auxiliary objective adjusts the steering control action $a_t^x$ based on the ego vehicle's lateral deviation from the expert trajectory. If the ego vehicle deviates leftward, the policy promotes rightward corrections ($a_t^x > a_t^{x,\text{old}}$); if it deviates rightward, it promotes leftward corrections ($a_t^x < a_t^{x,\text{old}}$). We formalize this with a directional factor $f_\text{pd}$:  
\begin{equation}
\begin{aligned}
f_\text{pd} = \begin{cases} 
1 & \text{if ego vehicle deviates leftward}, \\
-1 & \text{if ego vehicle deviates rightward}.
\end{cases} 
\end{aligned}
\end{equation}

The auxiliary objective for positional deviation correction is:
\begin{equation}
\begin{aligned}
\mathcal{L}_\text{pd}(\theta_x) = \mathbb{E}_t \left[ 
    \hat{A}_t^\text{pd} \cdot f_\text{pd} \cdot (\Delta \pi_x^{\text{right}} - \Delta \pi_x^{\text{left}})
\right],
\end{aligned}
\end{equation}  
where $\hat{A}_t^\text{pd}$ estimates the advantage of trajectory alignment.

\boldparagraph{Heading Deviation Auxiliary Objective.}  
The heading deviation auxiliary objective adjusts the steering control action $a_t^x$ based on the angular difference between the ego vehicle’s current heading and the expert’s reference heading. If the ego vehicle deviates counterclockwise, the policy promotes clockwise corrections ($a_t^x > a_t^{x,\text{old}}$); if it deviates clockwise, it promotes counterclockwise corrections ($a_t^x < a_t^{x,\text{old}}$). To formalize this behavior, we define a directional factor $f_\text{hd}$:  
\begin{equation}
\begin{aligned}
f_\text{hd} = \begin{cases} 
1 & \text{if ego vehicle deviates clockwise}, \\
-1 & \text{if ego vehicle deviates counterclockwise}.
\end{cases} 
\end{aligned}
\end{equation}

The auxiliary objective for heading deviation correction is then defined as:  
\begin{equation}
\begin{aligned}
\mathcal{L}_\text{hd}(\theta_x) = \mathbb{E}_t \left[ 
    \hat{A}_t^\text{hd} \cdot f_\text{hd} \cdot (\Delta \pi_x^{\text{right}} - \Delta \pi_x^{\text{left}})
\right],
\end{aligned}
\end{equation}  
where $\hat{A}_t^\text{hd}$ is the advantage estimate for heading alignment.  

\begin{table*}[ht]
\begin{center}
\centering
\resizebox{0.98\textwidth}{!}{
\begin{tabular}{cccccccccccccc}
\toprule
\multirow{2}{*}{ID} & Dynamic & Static & Position & Heading & \multirow{2}{*}{CR$\downarrow$} &\multirow{2}{*}{DCR$\downarrow$} &\multirow{2}{*}{SCR$\downarrow$} &\multirow{2}{*}{DR$\downarrow$} &\multirow{2}{*}{PDR$\downarrow$} &\multirow{2}{*}{HDR$\downarrow$} &\multirow{2}{*}{ADD$\downarrow$} &\multirow{2}{*}{Long. Jerk$\downarrow$} &\multirow{2}{*}{Lat. Jerk$\downarrow$}\\
& Collision & Collision & Deviation & Deviation & & & & & & & & & \\
\midrule
1 & \cmark  &  &  &  & 0.172 & 0.154 & 0.018 & 0.092 & 0.033 & 0.059  & 0.259 & 4.211 & 0.095 \\
2 &  & \cmark & \cmark & \cmark & 0.238 & 0.217 & 0.021 & 0.090 & 0.045 & 0.045  & 0.241 & 3.937 & 0.098 \\
3 & \cmark &  & \cmark & \cmark & 0.146 & 0.128 & 0.018 & 0.060 & 0.030 & 0.030  & 0.263 & 3.729 & 0.083\\
4 & \cmark & \cmark &  & \cmark & 0.151 & 0.142 & 0.009 & 0.069 & 0.042 & 0.027 & 0.303 & 3.938 & 0.079\\
5 & \cmark & \cmark & \cmark &  & 0.166 & 0.157 & 0.009 & 0.048 & 0.036 & 0.012 & 0.243 & 3.334 & 0.067\\
6 & \cmark & \cmark & \cmark & \cmark & 0.089 & 0.080 & 0.009 & 0.063 & 0.042 & 0.021 & 0.257 & 4.495 & 0.082 \\
\bottomrule
\end{tabular}
}
\end{center}
\vspace{-2mm}
\caption{\textbf{Ablation on reward sources.} The table shows the impact of different reward components on performance.}
\label{tab:reward_ablation}
\end{table*}

\begin{table*}[ht]
\begin{center}
\centering
\resizebox{0.98\textwidth}{!}{
\begin{tabular}{ccccccccccccccc}
\toprule
\multirow{2}{*}{ID} & \multirow{2}{*}{PPO Obj.}  & Dynamic Col. & Static Col. & Position Dev. & Heading Dev. & \multirow{2}{*}{CR$\downarrow$} & \multirow{2}{*}{DCR$\downarrow$}  & \multirow{2}{*}{SCR$\downarrow$} & \multirow{2}{*}{DR$\downarrow$} & \multirow{2}{*}{PDR$\downarrow$} & \multirow{2}{*}{HDR$\downarrow$} & \multirow{2}{*}{ADD$\downarrow$} & \multirow{2}{*}{Long. Jerk$\downarrow$} & \multirow{2}{*}{Lat. Jerk$\downarrow$} \\
& & Auxiliary Obj. & Auxiliary Obj. & Auxiliary Obj. & Auxiliary Obj. & & & & & & & & & \\
\midrule
1 &\cmark&  &  &  &  & 0.249 & 0.223 & 0.026 & 0.077 & 0.047 & 0.030  & 0.266 & 4.209 & 0.104 \\
2 &\cmark& \cmark &  &  &  & 0.178 & 0.163 & 0.015 & 0.151 & 0.101 & 0.050 & 0.301 & 3.906 & 0.085 \\
3 &\cmark&  & \cmark & \cmark & \cmark & 0.137 & 0.125 & 0.012 & 0.157 & 0.145 & 0.012 & 0.296 & 3.419 & 0.071 \\
4 &\cmark& \cmark &  & \cmark & \cmark & 0.169 & 0.151 & 0.018 & 0.075 & 0.042 & 0.033 & 0.254 & 4.450 & 0.098 \\
5 &\cmark& \cmark & \cmark &  & \cmark & 0.149 & 0.134 & 0.015 & 0.063 & 0.057 & 0.006 & 0.324 & 3.980 & 0.086 \\
6 &\cmark& \cmark & \cmark & \cmark & & 0.128 & 0.119  & 0.009 & 0.066 & 0.030 & 0.036  & 0.254 & 4.102 & 0.092 \\
7 &&\cmark  &\cmark  &\cmark  &\cmark  & 0.187 &0.175  &0.012 &0.077 &0.056  &0.021  &0.309  &5.014  &0.112  \\
8 &\cmark& \cmark & \cmark & \cmark & \cmark & 0.089 & 0.080 & 0.009 & 0.063 & 0.042 & 0.021  & 0.257 & 4.495 & 0.082 \\
\bottomrule
\end{tabular}
}
\end{center}
\vspace{-2mm}
\caption{\textbf{Ablation on auxiliary objectives.} The table shows the impact of different auxiliary objectives on performance.}
\label{tab:auxiliary_ablation}
\end{table*}

\boldparagraph{Overall Auxiliary Objectives.}  
The overall auxiliary objectives are a weighted sum of the individual objectives:
\begin{equation}
\begin{aligned}
\mathcal{L}_\text{aux}(\theta) = &\lambda_1 \mathcal{L}_\text{dc}(\theta_y) + \lambda_2 \mathcal{L}_\text{sc}(\theta_x)  + \\ 
&\lambda_3 \mathcal{L}_\text{pd}(\theta_x) +\lambda_4 \mathcal{L}_\text{hd}(\theta_x),
\end{aligned}
\end{equation}
where $\lambda_1$, $\lambda_2$, $\lambda_3$, and $\lambda_4$ are weighting coefficients that balance the contributions of each auxiliary objective.

\boldparagraph{Optimization Objective.}  
The final optimization objective combines the clipped PPO objective with the auxiliary objective:
\begin{equation}
\mathcal{L}(\theta) = \mathcal{L}^{\text{PPO}}(\theta) + \mathcal{L}_\text{aux}(\theta).
\end{equation}

\vspace{-0.05 in}
\paragraph{Random $n$-gram Selection}
% In the process of reutilizing generated $n$-grams as draft tokens and applying repetition penalty, there exists an inherent trade-off. Meanwhile, 

In our experiments, we observe that the draft tokens provided to the target model for parallel validation often yield multiple valid groups. Building on this observation, we randomly select one valid $n$-gram to serve as the final output. By leveraging the fact that multiple valid $n$-grams emerge during verification, we ensure that the final output is both diverse and accurate.

% we observe that the draft tokens provided to the target model for parallel validation can yield multiple valid groups.

In summary, the overall flow of our framework is presented in \cref{alg:algorithm}. 

\section{Experiment}\label{sec: exp}
In this section, we assess the efficacy of our algorithm by addressing the following key questions. 
(1) Can offline RL algorithms achieve stronger performance on the reduced datasets selected by~\name?
(2) How does \name~perform compare to other offline data selection methods? 
(3) What are the factors that contribute to \name's effectiveness?

\begin{figure}[t]
    \centering
    \subfigure{\includegraphics[scale=0.24]{d4rl-hard/walker2d-medium-v0-hard.pdf}}
    \hspace{0.2cm}
    \subfigure{\includegraphics[scale=0.24]{d4rl-hard/walker2d-expert-v0-hard.pdf}}
    \hspace{0.2cm}
    \subfigure{\includegraphics[scale=0.24]{d4rl-hard/walker2d-medium-replay-v0-hard.pdf}}
    % \subfigure{\includegraphics[scale=0.20]{d4rl-hard/walker2d-medium-expert-v0-hard.pdf}}
    \subfigure{\includegraphics[scale=0.24]{d4rl-hard/hopper-medium-v0-hard.pdf}}
    \hspace{0.2cm}
    \subfigure{\includegraphics[scale=0.24]{d4rl-hard/hopper-expert-v0-hard.pdf}}
    \hspace{0.2cm}
    \subfigure{\includegraphics[scale=0.24]{d4rl-hard/hopper-medium-replay-v0-hard.pdf}}
    % \subfigure{\includegraphics[scale=0.20]{d4rl-hard/hopper-medium-expert-v0-hard.pdf}}
    \subfigure{\includegraphics[scale=0.24]{d4rl-hard/halfcheetah-medium-expert-v0-hard.pdf}}
    \hspace{0.2cm}
    \subfigure{\includegraphics[scale=0.24]{d4rl-hard/halfcheetah-expert-v0-hard.pdf}}
    \hspace{0.2cm}
    \subfigure{\includegraphics[scale=0.24]{d4rl-hard/halfcheetah-medium-replay-v0-hard.pdf}}
    % \subfigure{\includegraphics[scale=0.20]{d4rl-hard/halfcheetah-medium-v0-hard.pdf}}
    \caption{Experimental results on the D4RL (Hard) offline datasets. All experiment results were averaged over five random seeds. Our method achieves better or
    comparable results than the baselines with lower computational complexity.}
    \label{fig: d4rl hard}
    \vspace{-0.5cm}
\end{figure}

% \begin{figure*}[t]
%     \centering
%     \includegraphics[width=\linewidth]{mujoco/fig1.pdf}
%     \vspace{-2em}
%     \caption{Sample-based selection performance of several baselines and \name~with different selected subset sizes~($x\%$).
%     The horizontal line is the performance of TD3+BC trained with the original dataset.}
%     \label{fig: d4rl minimal ratio}
%     \vspace{-1em}
% \end{figure*}

% \begin{figure}[t]
%     \centering
%     \includegraphics[width=\linewidth]{mujoco/traj.pdf}
%     \caption{In trajectory-based selection, \name~outperforms behavior cloning (\nameh) using trajectories with the highest accumulative returns, presenting a robust method for selecting the most useful data from training sets of compromised quality.}
%     \label{fig: d4rl topbc}
%     \vspace{-1em}
% \end{figure}

\subsection{Setup}
We evaluate algorithms on the offline RL benchmark D4RL~\citep{fu2020d4rl} to answer the aforementioned questions.
In addition, we consider a more challenging scenario where we add additional low-quality data to the dataset to simulate noise in real-world tasks, named D4RL~(hard).
The evaluation process commences with the selection of offline data, followed by the training of a widely recognized offline RL algorithm, TD3+BC~\citep{fujimoto2021minimalist}, on this reduced dataset for 1 million time steps.
To ensure a fair comparison, we apply the same offline RL algorithm to data subsets obtained by different algorithms. 
Evaluation points are set at every 5,000 training time steps and involve calculating the return of 10 episodes per point.
The results, comprising averages and standard deviations, are computed with five independent random seeds.
On the other hand, we can also incorporate our method into offline model-based approaches, such as MOPO~\citep{yu2020mopo} and MoERL~\citep{kidambi2020morel}.
Similarly, we only need to replace the current offline loss with the corresponding policy and model loss.

\textbf{Baselines}. 
We compare \name~with data selection methods in RL.
Specifically, previous work on prioritized experience replay for online RL~\citep{schaul2015prioritized} aligns closely with our objective. 
We make this a baseline \namep~where samples with the highest TD losses form the reduced dataset. 
Baseline \nameo~presents the performance by training TD3+BC with the original, complete dataset. 
Baseline \namer~randomly selects subsets from the D4RL dataset that are of the same size as \name.
We also compare our method with general dataset reduction techniques from supervised learning.
Specifically, we adopt the coherence criterion from Kernel recursive least squares~($\mathtt{KRLS}$)~\citep{engel2004kernel}, the log det criterion by forward selection in informative vector machines~($\mathtt{LogDet}$)~\citep{seeger2004greedy} and the adapting kernel representation~($\mathtt{BlockGreedy}$)~\citep{schlegel2017adapting} as our baselines.

%Specifically, we consider randomly selecting offline coreset as our baseline algorithms.
% In addition, we consider separately selecting high-reward offline datasets and low-reward offline datasets as our baseline algorithms.

\subsection{Experimental Results}
\label{sec:exp_perf}
% To compare the performance of different algorithms, we adopt two data selection schemes: sample-based selection and trajectory-based selection. They differ in the smallest unit of selection: the first selects samples in each iteration, while the second selects trajectories.

% As for the trajectory-based selection, prioritized sampling is no loner applicable. As an alternative, we compare with \nameh, which selects trajectories with the highest accumulative reward from the complete dataset. We again compare with the \nameo~as the reference to an upper limit of performance.

\begin{table*}[t]
    \centering
    \begin{tabular}{c|cccc}
    \toprule
    & KRLS & Log-Det & BlockGreedy & \name \\
    \midrule
    Hopper-medium-v0 & 69.4$\pm$2.5 & 58.4$\pm$3.6 & 83.7$\pm$2.2 & \textbf{94.3$\pm$4.6}\\
    Hopper-expert-v0 & 91.0$\pm$1.1 & 90.7$\pm$1.3 & 98.7$\pm$0.5 & \textbf{110.0$\pm$0.5}\\
    Hopper-medium-replay-v0 & 28.5$\pm$3.2 & 29.4$\pm$1.2 & 30.5$\pm$2.4 & \textbf{35.3$\pm$3.2}\\
    Walker2d-medium-v0 & 49.1$\pm$2.8 & 47.5$\pm$3.4 & 53.3$\pm$3.6 & \textbf{80.5$\pm$2.9}\\
    Walker2d-expert-v0 & 68.4$\pm$3.2 & 67.5$\pm$5.6 & 74.8$\pm$3.4 & \textbf{104.6$\pm$2.5}\\
    Walker2d-medium-replay-v0 & 14.3$\pm$1.2 & 15.2$\pm$2.2 & 16.7$\pm$1.3 & \textbf{21.1$\pm$1.8}\\
    Halfcheetah-medium-v0 & 23.4$\pm$0.5 & 21.9$\pm$0.9 & 27.5$\pm$0.7 & \textbf{41.0$\pm$0.2}\\
    Halfcheetah-expert-v0 & 73.9$\pm$1.4 & 72.1$\pm$2.2 & 79.2$\pm$1.8 & \textbf{88.5$\pm$2.4}\\
    Halfcheetah-medium-replay-v0 & 39.5$\pm$0.3 &39.9$\pm$0.5 & 40.5$\pm$1.0 & \textbf{41.1$\pm$0.4}\\
    \bottomrule
    \end{tabular}
    \caption{Experimental results on the D4RL~(Hard) offline datasets. All experiment results were averaged over five random seeds. Our method performs better than the dataset reduction baselines.}
    \label{tab: varied performance}
\end{table*}

\begin{figure}[t]
    \centering
    \subfigure{\includegraphics[scale=0.20]{d4rl/halfcheetah-medium-expert-v0.pdf}}
    \subfigure{\includegraphics[scale=0.20]{d4rl/hopper-medium-v0.pdf}}
    \subfigure{\includegraphics[scale=0.20]{d4rl/hopper-medium-expert-v0.pdf}}
    \subfigure{\includegraphics[scale=0.20]{d4rl/walker2d-medium-expert-v0.pdf}}
    \caption{Experimental results on the D4RL offline datasets. All experiment results were averaged over five random seeds. Our method achieves better or comparable results than the baselines consistently.}
    \label{fig: d4rl original}
\end{figure}

\paragraph{Answer of Question 1:}
To show that \name~can improve offline RL algorithms, we compare \name~with Complete Dataset, Prioritized, and Random in the Mujoco domain.
The experimental results in Figure~\ref{fig: d4rl hard} show that our method achieves superior performance than baselines.
By leveraging the reduced dataset generated from \name, the agent can learn much faster than learning from the complete dataset.
Further, the results in Figure~\ref{fig: d4rl original} show that \name~also performs better than the complete dataset and data selection RL baselines in the standard D4RL datasets. 
This is because prior methods select data in a random or loss-priority manner, which lacks guidance for subset selection and leads to degraded performance for downstream tasks.

In addition, to test \name's generality across various offline RL algorithms on various domains, we also conduct experiments on Antmaze tasks.
We use IQL~\citep{kostrikov2021offline} as the backbone of offline RL algorithms.
The experimental results in Table~\ref{tab: other domain2} show that our method achieves stronger performance than baselines.
In the antmaze tasks, the agent is required to stitch together various trajectories to reach the target location.
In this scenario, randomly removing data could result in the loss of critical data, thereby preventing complete the task.
Differently, \name~extracts valuable subset by balancing data quantity with performance, achieving a stronger performance than the complete dataset.

% In Figure~\ref{fig: d4rl minimal ratio}, we show the performance of different algorithms with the sample-based selection scheme. The experimental results show that \name~can achieve performance close to \nameo~with a small amount of data. For example, we use only $3\%$ of the original dataset in the Hopper tasks. \namer~and \namep, on the other hand, present a stark contrast, even not showing a stable learning trend with the same amount of training data. 
% In addition, we also evaluate the performance on the trajectory-based selection setting. Please refer to Appendix~\ref{appendix: trajectory} for the detailed experimental results.
% For the trajectory-based selection, experimental results in Figure~\ref{fig: d4rl topbc} show that \name~maintains its superiority in this setting with suboptimal (e.g., \texttt{medium}) datasets. This evidence suggests that \name~provides a valuable strategy for selecting data conducive to effective training under conditions of compromised data quality.

\paragraph{Answer of Question 2:}
To test whether \name~can select more valuable data than the data selection algorithms in supervised learning, we compare our method with KRLS~\citep{engel2004kernel}, Log-Det~\citep{seeger2004greedy} and BlockGreedy~\citep{schlegel2017adapting} in the D4RL~(Hard) datasets.
The experimental results in Table~\ref{tab: varied performance} show that our method generally outperforms baselines.
We hypothesize that supervised learning is static with fixed learning objectives, while offline RL's dynamic nature makes the target values evolve with policy updates, complicating the data selection process.
Therefore, the data selection methods in supervised learning cannot be directly applied to offline RL scenarios.

% Additionally, we observe that  $\texttt{Random}$ performs better than $\texttt{Q-diff}$.
% We attribute this phenomenon to the broader data coverage of $\texttt{Random}$, while the data coverage of $\texttt{Q-diff}$ is narrow.
% However, we also note that in some tasks, such as $\texttt{Hopper-medium-expert-v0}$, $\texttt{Hopper-expert-v0}$ and $\texttt{Walker2d-expert-v0}$, $\texttt{Random}$ initially performs well, but as training progresses, its performance starts to decline.
% We find that this coincides with unstable Q-values, which can be attributed to the increased extrapolation error caused by the reduced training dataset.
% In contrast, \name~performs better since it closely approximates the original gradients, thus preventing Q-values from diverging.


% For this reason, when the dataset quality is high~(e.g., \texttt{medium-expert} dataset), TopBC performs comparably to \name.

% \begin{table*}[t]
%     \centering
%     \caption{\name~with varying dataset sizes~($x\%$). Highlighted is the performance comparable to training TD3+BC with the complete dataset. \name~typically achieves good results with 5\%-15\% data, indicating that existing offline RL datasets contain a high degree of redundancy.
%     We adopt the normalized score metric proposed by the D4RL benchmark. Scores roughly range from 0 to 100, where 0 corresponds to the performance of a random policy and 100 indicates the performance of an expert.} 
%     \label{tab: varied performance}
%     \begin{tabular}{c|cccc}
%     \toprule
%         & 5\% & 10\% & 15\% & 20\% \\
%         \midrule
%         Hopper-medium-v0 & 91.8$\pm$3.6 & 92.6$\pm$3.0 & 94.0$\pm$4.8 & 95.2$\pm$1.6\\
%         Walker2d-medium-v0 & 14.8$\pm$7.3 & 57.9$\pm$3.6 & 69.3$\pm$4.0 & 71.7$\pm$1.9 \\
%         Halfcheetah-medium-v0 & 40.5$\pm$0.0 & 40.9$\pm$0.1 & 41.3$\pm$0.1 & 41.2$\pm$0.5 \\
%         Hopper-expert-v0 & 111.6$\pm$0.9 & 110.6$\pm$1.9 & 112.7$\pm$0.1 & 112.4$\pm$0.1 \\
%         Walker2d-expert-v0 & 74.5$\pm$6.4 & 84.4$\pm$5.0 & 97.6$\pm$3.1 & 100.2$\pm$1.0 \\
%         Halfcheetah-expert-v0 & 57.5$\pm$6.4 & 84.3$\pm$2.7 & 97.8$\pm$0.8 & 100.1$\pm$3.0 \\
%         Hopper-medium-expert-v0 & 108.1$\pm$1.1 & 112.4$\pm$0.3 & 112.3$\pm$0.05 & 112.8$\pm$0.1\\
%         Walker2d-medium-expert-v0 & 79.3$\pm$2.1 & 85.4$\pm$5.3 & 96.2$\pm$6.7 & 101.4$\pm$3.6 \\
%         Halfcheetah-medium-expert-v0 & 67.5$\pm$0.5 & 86.2$\pm$5.0 & 85.8$\pm$1.5 & 92.4$\pm$1.3\\
%     \bottomrule
%     \end{tabular}
% \end{table*}


% \subsection{Ablation Study}\label{sec:exp_ab}
% \textbf{Varying dataset size}.\ \ In Table~\ref{tab: varied performance}, we show the performance of \name~with varying dataset sizes ranging from $5\%$ to $20\%$.
% The results demonstrate that \name~requires only $5\%$ or $10\%$ of the original dataset to obtain good performance.
% Further, \name~can achieve similar performance with \nameo~with $20\%$ data of the original dataset.
% This indicates that existing offline RL datasets are characterized by a high degree of redundancy.

\begin{figure}[t]
    \centering
    \includegraphics[width=0.97\linewidth]{visual.jpg}
    \caption{Visualization of the \textcolor{blue}{complete dataset} and the \textcolor{orange}{reduced dataset} in \texttt{halfcheetah} task. The higher opacity of a point represents a large time step towards the end of an episode. The dataset embedding is characterized by its division into different components. 
    % In \texttt{walker2d} (upper), components vary with time steps.
     Samples selected by \name~connect different components by focusing on the data related to the task.}
    \label{fig: t-sne}
\end{figure}

\begin{table}[t]
    \centering 
    \begin{tabular}{c|cccc}
    \toprule
        Env & Random & Prioritized & Complete Dataset & \name\\
        \midrule
        Antmaze-umaze-v0 & 75.1$\pm$2.5 & 70.2$\pm$3.6 & 87.5$\pm$1.3 & \textbf{90.7$\pm$3.3}\\
        Antmaze-umaze-diverse-v0 & 46.3$\pm$1.9 & 44.7$\pm$2.7 & 62.2$\pm$2.0 & \textbf{76.7$\pm$2.2} \\
        Antmaze-medium-play-v0 & 59.3$\pm$1.6 & 60.3$\pm$2.9 & 71.2$\pm$2.2 & \textbf{80.3$\pm$2.9}\\
        Antmaze-medium-diverse-v0 & 43.6$\pm$2.7 & 46.9$\pm$3.8 & 70.0$\pm$1.6 & \textbf{84.9$\pm$3.8}\\
        Antmaze-large-play-v0 &	3.7$\pm$0.7 & 15.0$\pm$3.5 & 39.6$\pm$3.6 & \textbf{46.0$\pm$3.5}\\
        Antmaze-large-diverse-v0 & 16.0$\pm$3.6 & 20.5$\pm$3.7 & 47.5$\pm$1.1 & \textbf{52.0$\pm$3.7}\\
    \bottomrule
    \end{tabular}
    \caption{Experimental results on the Antmaze offline datasets. All experiment results were averaged over five random seeds. Our method performs better than baselines. }
    \label{tab: other domain2}
\end{table}

% \begin{figure*}[t]
%     \centering
%     \subfigure{\includegraphics[scale=0.27]{ablation_moduler1.pdf}}
%     \hspace{0.3cm}\subfigure{\includegraphics[scale=0.27]{ablation_moduler2.pdf}}
%     \caption{Ablation results on D4RL~(Hard) tasks with the normalized score metric.}
%     \label{fig: modular ablation}
% \end{figure*}

% In this subsection, we conduct ablation studies to study the effect of different modules and import hyper-parameters.


\paragraph{Answer of Question 3:}
To study the contribution of each component in our learning framework, we conduct the following ablation study. 
\nameq: We replace the empirical returns used to update Q functions with the standard target Q function in the TD loss function. 
\namei: We set the number of data selection rounds to 1 and study the function of multi-round data selection.
The experimental results in Figure~\ref{fig: modular ablation} in Appendix~\ref{sec: ablation} show that removing any of these two modules will worsen the performance of \name. In case like $\texttt{walker2d-medium}$, ablation \namei~even decrease the performance by over 80\%, and ablation \nameq~results in a 95\% performance drop in $\texttt{walker2d-expert}$. Furthermore, we also find that in the $\texttt{halfcheetah}$ tasks, the impact of removing the two modules is relatively small. This result can be attributable to the fact that this task has a limited state space, and we can directly apply OMP to the entire dataset and identify important and diverse data.

We visualize the selected data by \name~to better understand how it works. 
Figure~\ref{fig: t-sne} displays the t-SNE low-dimensional embeddings, with the complete dataset in blue and the selected data in orange. 
The higher opacity of a point indicates a larger time step. The dataset's structure is revealed by its segmentation into diverse components: 
In \texttt{halfcheetah}, each component reflects a distinct skill of the agent.
For example, from 1 to 7, they represent falling, leg lifting, jumping, landing, leg swapping, stepping, and starting, respectively.
We can observe that the selected samples by \name~ not only cover each component of the dataset but also effectively bridge the gaps between them, enhancing the dataset's versatility and coherence. 
Moreover, we find that \name~is less concerned with the falling data and instead focuses on the data related to the task.
This observation can explain the improved performance of \name. For additional visualizations, please refer to Appendix~\ref{appendix: visual}.

% \textbf{Generalizability of \name}. \ \
% We evaluate the generalizability of \name~from two perspectives.
% First, we add IQL~\cite{kostrikov2021offline} as a baseline and apply \name~to IQL by using the gradient of the training loss of the V-function in IQL as the criterion.
% On the other hand, we evaluate \name~on the other domains, such as robotic manipulation (Adroit) and sparse reward (Antmaze) tasks.
% The experiments in Appendix~\ref{appendix: other domain} and Appendix~\ref{appendix: other algorithm} show that \name~is not only applicable to other algorithms, such as IQL~\cite{kostrikov2021offline}, but also to other domains.

% \textbf{Generalizability of subset}. \ \
% To test the generalizability of the dataset selected by~\name, we select subset by applying~\name~to TD3+BC.
% Then we evaluate the performance of IQL on the selected subset. 
% The experimental results in Table~\ref{tab: td3bc2iql} in Appendix~\ref{appendix: tb3bc2iql} demonstrate that the selected subset based on TD3+BC is effectively applicable to IQL.

% \textbf{Sensitivity for hyperparameter}. \ \
% We evaluate the performance of \name~with various cluster numbers~(from 1 to 50) and approximation bounds~(from 0.0001 to 0.05).
% The experimental results in Appendix~\ref{appendix: cluster number} and Appendix~\ref{appendix: approx bound} show that the suitable cluster number is between 25 and 50.
% Too few clusters (e.g., less than 5) are detrimental to the algorithm.
% In addition, a smaller approximation bound represents a larger reduced dataset.
% Similar to the ablation of the size of the reduced dataset in Table~\ref{tab: varied performance}, \name~requires only a 0.01 approximation bound to obtain good performance.

\subsection{Computational complexity}
We report the computational overhead of \name~on various datasets. 
All experiments are conducted on the same computational device (GeForce RTX 3090 GPU). 
The results in Appendix~\ref{appendix: computation complexity} indicate that even on datasets containing millions of data points, the computational overhead of our method remains low~(e.g., several minutes).
This low computational complexity can be attributed to the trajectory-based selection technique in Sec.~\ref{sec: offline omp}~(II) and the regularized constraint technique in Sec.~\ref{sec:method:outer}, making our method easily scalable to large-scale datasets. 

% This low computational complexity can be attributed to the batch mechanism designed in section 3.2 (IV), which reduces the computational complexity from $O(MN)$ to $O(|\mathcal{B}|N)$, making our method easily scalable to large-scale datasets. $M, N, |\mathcal{B}|$ are the size of the full dataset, reduced dataset, and batch respectively.

% We conduct t-SNE based dimensionality reduction to the cluster centroids and these five trajectories.
% The experimental results are shown in the , where darker colors indicate moving towards the end of the trajectory.

% From the experimental results, we find that in the walker2d task, \name~ tends to select more low-reward but more diverse data points ~(upper right) while selecting a few high-reward data points~(left and bottom).
% We attribute this phenomenon to the narrow distribution of the high-reward points, allowing us to approximate the original gradients with only a few points. 
% In the halfcheetah task, \name~ connects useful information while ignoring low-quality data~(e.g., data point \texttt{1}).

% \input{Sections/7-Analysis}
\section{Conclusion and Discussion}
In this paper, we introduce 3DMolFormer for structure-based drug discovery, a dual-channel transformer-based framework designed to process parallel sequences of tokens and numerical values representing pocket-ligand complexes. Through self-supervised large-scale pre-training and supervised fine-tuning, 3DMolFormer can accurately and efficiently predict the binding poses of ligands to protein pockets. Furthermore, through reinforcement learning fine-tuning, 3DMolFormer can generate drug candidates that exhibit high binding affinities for a given protein target, along with favorable drug-likeness and synthesizability. Above all, 3DMolFormer is the first machine learning framework that can simultaneously address both protein-ligand docking and pocket-aware 3D drug design, and it outperforms previous baselines in both tasks.

It is noteworthy that many recent deep learning models for 3D molecules, such as Uni-Mol, Pocket2Mol, TargetDiff, and DecompDiff, which serve as baselines in our experiments, adhere to the concept of "equivariance" introduced by geometric deep learning~\citep{Equivariance,Equivariance2}. However, the 3DMolFormer model does not explicitly enforce SE(3)-symmetry. It appears that through the normalization of 3D coordinates and random rotations during data augmentation, 3DMolFormer has acquired the SE(3)-equivariance by training on a sufficiently large and diverse dataset. This approach aligns with recent successful methods in the field, including AlphaFold3~\citep{AlphaFold3}, which also does not rely on SE(3)-equivariant architectures.

Admittedly, our approach still has some limitations. First, 3DMolFormer does not account for the flexibility of proteins during ligand binding, which may affect the accuracy of subsequent binding affinity prediction. Second, protein-ligand binding is a dynamic process, but 3DMolFormer struggles to capture this dynamism effectively. Finally, 3DMolFormer does not consider environmental factors such as temperature and pH, which can significantly influence the 3D conformation of the binding complex. These issues represent core challenges in current computational methods for structure-based drug discovery, and we look forward to future work addressing these limitations. Furthermore, the implementation details in 3DMolFormer have the potential to be further optimized, for example, advanced methods of multi-objective reinforcement learning~\citep{MORL} may be introduced into the drug design process.
% Reproducibility Statement
\subsubsection*{Acknowledgments}
This work is supported by the Pioneer R\&D Program of Zhejiang (No.~2024C01021), ``Leading Talent of Technological Innovation Program'' of Zhejiang Province (No.~2023R5214), OPPO Research Fund,
the Major Research Program of Zhejiang Provincial Natural Science Foundation (No.~LD24F020015), and NSFC Grant No.~62402420.

\clearpage
\bibliography{loram,custom}
\bibliographystyle{iclr2025_conference}
\clearpage
\appendix
\section*{Appendix}
\section{Related Work}
Researchers have been leveraging eye tracking methodologies from human perception research to model how people perceive images~\cite{shanmuga2015eye, bonhage2015combined, conklin2016using}.
These models help assess the appearance and salience of visual representations, enabling eye movement tracking to understand the perceptual and cognitive mechanisms of scene perception~\cite{itti1998model} and object detection~\cite{borji2015salient}.
The existing saliency models perform well in naturalistic scenes
%and real-world object detection
; however, there are unique perception rules and cognitive biases in the artificial world of data visualization 
%does not always follow the rules of perception in the natural world
~\cite{franconeri2021science, correll2012comparing, polatsek2018exploring, knittel2024gridlines}, and, thus, these models do not accurately predict where people would look in visualizations. 
Visualization researchers have been building visual saliency models geared to visualizations~\cite{DVSaliencyModel2017Matzen, bylinskii2016should}. %and adopting them for predicting eye gaze on visualizations. %enabling the prediction of visual saliency across design styles~\cite{fosco2020predicting}.
However, these models rely on handcrafted features, making it difficult to generalize to complex visualizations. Additionally, these models cannot incorporate textual information to generate task-specific saliency maps since the prediction is solely based on visual inputs.

With the advent of deep learning, gaze data were used as the ground truth of saliency models~\cite{fosco2020predicting, scannerDeeply, scanpath}, leading to higher performance in saliency prediction while enabling task-specific saliency~\cite{salchartQA}. 
These models usually need large-scale datasets to learn complex patterns. However, gathering precise gaze data is 
%challenging and requires specialized eye-tracking devices. While these devices provide accurate results, they tend to be 
costly and cumbersome, which limits large-scale data collection efforts. 
Many researchers, therefore, proposed several proxies for eye gaze. WebGaze~\cite{webgaze} uses a webcam for cheap and easy deployment in online studies yet suffers from data quality issues due to low-resolution cameras and uncontrolled calibration.
Therefore, mouse-(cursor-)based annotation tools~\cite{jiang2015salicon,bubbleView,importAnnot} were proposed to improve data quality. Among these methods, BubbleView~\cite{bubbleView} was the most used tool for capturing visual saliency and importance~\cite{graphicDesignImportance, salchartQA}.
However, BubbleView is primarily designed for exploring images and gathering information, which differs slightly from the goal of capturing perceived importance. As a result, while BubbleView is well-suited for measuring visual saliency, it may not be the best tool for capturing %instruction-tuned \yao{I would keep it consistent saying task-specific}
task-specific importance~\cite{turkeyes}. Built upon these prior approaches' limitations, our Grid Labeling aims to collect responses that cover all essential areas of the visualization with minimum noise, leading to more efficient data collection.



% One key motivation to our grid-based approach is to help people 
% We also demonstrate that the grid-based approaches can minimize biases in annotation to disproportionally emphasize text elements~\cite{DVSaliencyModel2017Matzen}
% % blurring the visualization can disproportionately emphasize text elements~\cite{DVSaliencyModel2017Matzen}, potentially misrepresenting a user's true areas of interest.
% More recently, 
% % \ms{Changed a bit using Yao's work (task-dependent saliency), but not sure whether it looks ok}
% Yao et al.~\cite{salchartQA} collect task-dependent saliency using the BubbleView method, % but their approach had some limitations. 
% and made a significant improvement on existing saliency models.
% First, the blurred visualization allowed users to perceive the overall structure of the chart, which prevented the system from capturing the specific action of identifying the maximum value. However, increasing the blur to address this issue introduced another challenge. As the structure became less visible, users had to explore the entire image, leading to the consideration of irrelevant regions as salient.
\section{Experimental Details}
\label{apd:detail_setup}
\paragraph{Pre-train Corpus.}
To align the inconsistent knowledge between the pruned model during training and the original model during inference, we apply \method to continual pre-training 
LLMs in a teacher-forcing manner~\citep{bachmann2024teacherforce}
on a mixed corpus of FineWeb~\citep{penedo2024fineweb} and OpenWebMath~\citep{paster2023openwebmath}.
FineWeb, containing over 15TB of cleaned and deduplicated English web data from Common Crawl. 
OpenWebMath, extracted from over 200 billion HTML files on Common Crawl, provides high-quality mathematical text. Mixing these datasets enhances the pruned model's capabilities in both general and mathematical domains.

Unless specified otherwise, we randomly sample 102,400 instances from both FineWeb and OpenWebMath to construct a mixed dataset with a sequence length of 512, yielding approximately 105 million tokens. The default training batch size is 128, allowing up to 1,600 update steps. We train without data repetition over a sufficiently large corpus to simulate a realistic pre-training scenario. 
Notably, this alignment process is a one-time, offline operation that model publishers can execute.

\paragraph{Fine-tuning Data.}
Following the fine-tuning scenario of LoRA~\citep{Edw:2022lora}, we primarily conduct supervised fine-tuning (SFT) on the OpenHermes-2.5~\citep{OpenHermes} (referred to as OpenHermes). OpenHermes is a large-scale dataset constructed from synthetically generated instructions and chat samples, encompassing diverse sources such as Airoboros 2.2~\citep{wang2023selfinstructaligning}, CamelAI Domain Expert Dataset~\citep{li2023camel}, ChatBot Arena (GPT-4 Only)~\citep{zheng2023lmsyschat1m}, and more.
To further demonstrate the general effectiveness of the \method alignment process, we also evaluate \method on the OpenOrca~\citep{OpenOrca} dataset. OpenOrca is a widely used instruction fine-tuning dataset where each data instance represents entries from the FLAN collection~\citep{longpre2023flan}, augmented by submitting the listed questions to either GPT-4 or GPT-3.5.

By default, we train SFT on the instruction dataset with a batch size of 128 and a sequence length of 512 for 400 steps, totaling approximately 26.2 million tokens. 
To effectively evaluate the overall fine-tuning performance, we assess the perplexity of the fine-tuned model on an out-of-domain test set. This out-of-domain test set is constructed by randomly sampling 2,000 instances from the Alpaca~\citep{alpaca} test set, truncated to a sequence length of 512.

\paragraph{Downstream Task.}
We focus on the performance of \method in various downstream tasks, including mathematical reasoning, common sense reasoning, and code generation. All our downstream task evaluations are performed on lm-evaluation-harness\footnote{~\url{https://github.com/EleutherAI/lm-evaluation-harness} (MIT License).} and code-eval~\footnote{~\url{https://github.com/abacaj/code-eval} (MIT License).} with VLLM~\footnote{~\url{https://github.com/vllm-project/vllm} (Apache-2.0 license).}.

For mathematical reasoning, we benchmark the accuracy of baseline models using greedy decoding on MathQA~\citep{amini-etal-2019-mathqa} with a 1-shot setting and GSM8K (Grade School Math 8K)~\citep{cobbe2021gsm8k} with 8-shots, Chain of Thought (CoT) prompting and strict match
% \footnote{
% % The evaluation script 
% Follow~\url{https://github.com/EleutherAI/lm-evaluation-harness} (MIT License).}.
MathQA is a large-scale dataset comprising 37k English multiple-choice math word problems, covering diverse math domains. It extends the AQuA-RAT dataset~\citep{ling2017program} by annotating problems with fully specified operational programs using a new representation language, building on the questions, options, rationale, and correct answers provided by AQuA-RAT.
The GSM8K is a dataset of 8.5K high-quality, linguistically diverse grade school math word problems, designed to evaluate multi-step reasoning in basic arithmetic operations (+-×÷). We conduct evaluations on its 1.3K test set with \textit{strict-match} to assess logical and mathematical reasoning in language models.

For commonsense reasoning (CSR), we report the average accuracy across six tasks—Arc Challenge \& Easy~\citep{clark2018arc}, HellaSwag~\citep{zellers-etal-2019-hellaswag}, OpenBookQA~\citep{OpenBookQA2018}, PIQA~\citep{Bisk2020}, and WinoGrande~\citep{WinoGrande2021}—under 1-shot and greedy decoding settings. These benchmarks comprehensively assess the model’s ability to apply ``commonsense" or world knowledge for reasoning, rather than relying on pattern recognition.

For code generation, we compare two pass rates, \textsc{Pass@1} and \textsc{Pass@10}~\citep{kulal2019spoc}, on HumanEval~\citep{chen2021evaluating} of each baseline in a zero-shot setting with sampling parameters of  $\textsc{temperature}=\{0.0,0.2,0.4,0.6,0.8\}$, and $\textsc{top}_\textsc{p} =0.95$.
The HumanEval dataset released by OpenAI consists of 164 handwritten Python programming problems, each with a function signature, docstring, body, and unit tests. Serving as a benchmark, HumanEval assesses models on a range of Python coding skills, from basic syntax to complex problem-solving, offering insights into their programming capabilities alongside language-focused tasks.

\paragraph{Sparsification \& Quantization.}
\method incorporates two model compression techniques: sparsification, which generates a pruned model for low-rank matrix updates, and quantization, which forms \Qmethod further to reduce the memory footprint of the pruned model.
For sparsification, 
to validate the general effectiveness of \method, we benchmark its performance across various pruning strategies $\mathtt{P}(\cdot)$. 
Specifically, we first establish a variant using randomly structured pruning and adapt \method to another three variants based on leading approaches: the structured pruning LLM-Pruner\footnote{\url{https://github.com/horseee/LLM-Pruner} (Apache-2.0 license)}~\citep{ma2023llmpruner} and the non-structured (semi-structured \& unstructured) pruning SparseGPT\footnote{\url{https://github.com/IST-DASLab/sparsegpt} (Apache-2.0 license)}~\citep{FrantarA23spasegpt}. 
These baselines are summarized below, with the corresponding configurations presented in~\cref{tab:llama2_13b,tab:llama2_70b,tab:llama2_70b_q}.
\begin{itemize}[leftmargin=20pt]
    \item \textbf{\methodrand}: 
    We adhere to the pruning settings of \methodstru, modifying only by randomly removing weights instead of the original gradient-based pruning criterion.
    \item \textbf{\methodstru}: 
    We follow LLM-Pruner and employ a block-wise strategy for local structured pruning. Attention and MLP layers are treated as separate blocks, with non-critical coupling weights pruned based on gradient information at a uniform ratio. We retain the first four and last two layers of both blocks, focusing pruning on the intermediate layers.
    \item \textbf{\methodsemi}: 
    We utilize SparseGPT with a 4:8 semi-structured sparsity pattern to prune pre-trained weights across all model layers.
    \item \textbf{\methodunst}:  
    We prune individual weights uniformly across layers using a predefined pruning ratio based on an unstructured version of SparseGPT.
\end{itemize}

For quantization $\mathtt{Q}(\cdot)$, to further reduce memory usage during training, especially when dealing with models exceeding 70 billion parameters, we achieve \Qmethod by combining \method with the LoRA-tailored quantization algorithm QLoRA~\citep{Tim:2023qlora}. While \method is compatible with the quantization of other customized LoRA methods~\citep{Xu:2023QALoRA,li2024loftq,guo2024lqlora,OPTQ2023,chai2023int21}, this falls outside the scope of this article.

\paragraph{Architecture \& Hyperparameters.}
We adopt a LLaMA architecture with RMSNorm~\citep{ZhangS19a} and SwiGLU
activations~\citep{Noglu,ZhaoSA22}. 
We run all experiments with BF16 format to reduce memory usage.
For all configurations, we default to a learning rate of 1e-3. However, the downstream performance of models fine-tuned on OpenOrca is relatively sensitive to the learning rate. Therefore, in this evaluation, we tune the learning rates for each baseline within the range of [1e-5, 1e-3] and report their respective optimal downstream scores. Specifically, we use 1e-5 for the 7B LoRA and 13B \& 70B LoRAM models, and 1e-4 for the 13B LoRA model.
All experiments run on NVIDIA A100-80GB GPUs with environments of CUDA 12.2, PyTorch 2.4.0, and Transformer 4.45.1.
For LLaMA-2 herds, we set low-rank matrices $\mathbf{B}$ and $\mathbf{A}$ of rank $r=8$ for $\mathbf{W}_\text{q}$, $\mathbf{W}_\text{k}$, $\mathbf{W}_\text{v}$, and $\mathbf{W}_\text{o}$ in the attention layer, $\mathbf{W}_\text{up}$, $\mathbf{W}_\text{gate}$, and $\mathbf{W}_\text{down}$ in the MLP layer, and the head embedding matrix $\mathbf{W}_\text{lm\_head}$;
for LLaMA-3 herds, we exclude the injection of the low-rank matrix of $\mathbf{W}_\text{lm\_head}$.
\begin{table*}[h]
    \centering
    \renewcommand{\arraystretch}{1.1}
    \setlength{\tabcolsep}{3pt}

    \caption{LoRAM configures on LLaMA-2-13B. Comparison of different pruning methods in terms of parameter reduction ratio (Reduction) and HBM footprint (GB) of pruned parameters (HBM), ignoring low-rank matrix overhead.}
    \label{tab:llama2_13b}
    \begin{tabular}{@{}lccccc@{}}
        \toprule
        Method & \#Orig. Params & Pruning Ratio & \#Pruned Params & Reduction & HBM  \\ 
        \midrule
        LoRAM-Semi & 13015864320 & 0.50 & 6738415616 & 1.93$\times$ & 12.55 \\
        LoRAM-Unst & 13015864320 & 0.55 & 6037628912 & 2.16$\times$ & 11.25 \\
        LoRAM-Rand \& Stru & 13015864320 & 0.65 & 6005662720 & 2.17$\times$ & 11.19 \\
        \bottomrule
    \end{tabular}
\end{table*}

\begin{table*}[h]
    \centering
    \renewcommand{\arraystretch}{1.1}
    \setlength{\tabcolsep}{3pt}

    \caption{LoRAM configures on LLaMA-2-70B and LLaMA-3.1-70B with different pruning ratios.}
    \label{tab:llama2_70b}
    \begin{tabular}{@{}lccccc@{}}
        \toprule
        Method & \#Orig. Params & Pruning Ratio & \#Pruned Params & Reduction & HBM \\ 
        \midrule
        LoRAM-Rand \& Stru & 68976648192 & 0.65 & 28099436544 & 2.45$\times$ & 52.34 \\
        LoRAM-Rand \& Stru & 68976648192 & 0.75 & 21488738304 & 3.21$\times$ & 40.03 \\
        LoRAM-Rand \& Stru & 68976648192 & 0.85 & 16272924672 & 4.24$\times$ & 30.31 \\
        LoRAM-Rand \& Stru & 68976648192 & 0.95 & 9662226432 & 7.14$\times$ & 18.00 \\
        LoRAM-Rand \& Stru & 70553706496 & 0.85 & 17849982976 & 3.95$\times$ & 33.25 \\
        \bottomrule
    \end{tabular}
\end{table*}

\begin{table*}[h]
    \centering
    \renewcommand{\arraystretch}{1.1}
    \setlength{\tabcolsep}{3pt}

    \caption{QLoRAM configures on LLaMA-2-70B and LLaMA-3.1-70B with , demonstrating more aggressive parameter compression.}
    \label{tab:llama2_70b_q}
    \begin{tabular}{@{}lccccc@{}}
        \toprule
        Method & \#Orig. Params & Pruning Ratio & \#Pruned Params & Reduction & HBM \\ 
        \midrule
        QLoRAM-Rand \& Stru & 68976648192 & 0.65 & 7024859136 & 9.82$\times$ & 13.08 \\
        QLoRAM-Rand \& Stru & 68976648192 & 0.75 & 5372184576 & 12.84$\times$ & 10.01 \\
        QLoRAM-Rand \& Stru & 68976648192 & 0.85 & 4068231168 & 16.95$\times$ & 7.58 \\
        QLoRAM-Rand \& Stru & 68976648192 & 0.95 & 2415556608 & 28.56$\times$ & 4.50 \\
        QLoRAM-Rand \& Stru & 70553706496 & 0.85 &  4462495744 & 15.81$\times$ & 8.31 \\
        \bottomrule
    \end{tabular}
\end{table*}










\clearpage
\section{Visualization of Dimension Evolution}
\label{sec:dimension_vis}
To clearly illustrate the evolution of weight matrix dimensions across the multiple stages in the proposed scheme, we take LLM-Pruner~\citep{ma2023llmpruner} as an example in (e.g.,~\methodstru) in~\cref{fig:dimension_vis}, visualizing the transformation from $\mathbf{W}_{0} \Rightarrow \mathbf{W}_{0}^\mathtt{P}$, $\mathbf{W}_{\Delta} \Rightarrow \mathbf{W}_{\Delta}^\mathtt{P}$, and $\mathbf{W}_{\Delta}^{\mathtt{P}^{\star}}\Rightarrow \mathbf{W}_{\Delta}^{\mathtt{R}^{\star}}$ under \method with structured pruning. 
For \method variants employing non-structured pruning, the parameter dimensionality remains unchanged during training due to the use of a mask matrix. Therefore, these visualizations are omitted.
\begin{figure*}[ht]
\begin{center}
\includegraphics[width=\textwidth]{fig/dimension_evaluation_final.pdf}
\caption{
Dimensional evolution of the weight matrices: $\mathbf{W}_{0} \Rightarrow \mathbf{W}_{0}^\mathtt{P}$ (a), $\mathbf{W}_{\Delta} \Rightarrow \mathbf{W}_{\Delta}^\mathtt{P}$ (b), and $\mathbf{W}_{\Delta}^{\mathtt{P}^{\star}} \Rightarrow \mathbf{W}_{\Delta}^{\mathtt{R}^{\star}}$ (c) during \methodstru training. This includes updates for $\mathbf{W}_\text{q}$, $\mathbf{W}_\text{k}$, $\mathbf{W}_\text{v}$, and $\mathbf{W}_\text{o}$ in the attention layer, as well as $\mathbf{W}_\text{up}$, $\mathbf{W}_\text{gate}$, and $\mathbf{W}_\text{down}$ in the MLP layer.
}
\label{fig:dimension_vis}
\end{center}
\end{figure*}


\clearpage
\section{Visualization of Low-rank Matrices}
\label{apd:vis_matrix}
In this section, we utilize the \(L_{2}\)-norm to evaluate variations in low-rank matrices trained with different \method variants. This metric facilitates the visualization of captured features and allows for an analysis of \method's effectiveness. Specifically, we examine the updated low-rank matrices in the self-attention and MLP layers of LLaMA-2-13B and LLaMA-2-70B, trained with \method variants on OpenHermes.

\subsection{Head-wise Norm of Attention}
For the low-rank matrices in the attention layer, denoted as \(\mathbf{W}_{\Delta^{*}}\) where \({*} \in \{\text{q}, \text{k}, \text{v}, \text{o}\}\), we compute the \(L_{2}\) norms for each attention head. Let \(\text{H}^{*}\) represent the number of heads. The \(L_{2}\) norms for each head \(\text{h}\) (where \(\text{h} = 0, 1, \ldots, \text{H}^{*} - 1\)) are defined as follows:

\begin{equation}
\| \mathbf{W}_{\Delta^{*}}^{(h)} \|_2 = 
\begin{cases}
\left\| \mathbf{W}_{\Delta^{*}}[h, :] \right\|_2 & \text{if } {*} \in \{\text{q}, \text{k}, \text{v}\} \\
\left\| \mathbf{W}_{\Delta^{*}}[:, h] \right\|_2 & \text{if } {*} = \text{o}
\end{cases}.
\end{equation}

The results are visualized through heatmaps in ~\cref{fig:attn_lora_vis_13B,fig:attn_lora_vis_70B}, effectively illustrating the distribution of features captured by different attention heads.

\subsection{Layer-wise Norm of MLP}
For the low-rank matrices in the MLP layers, denoted as \(\mathbf{W}_{\Delta^{*}}\) where \(\Delta^{*} \in \{\text{up}, \text{gate}, \text{down}\}\), we denote the number of layers as \(\text{L}\). The average \(L_{2}\) norm for a specific layer \(l\) (where \(l = 0, 1, \ldots, \text{L} - 1\)) is computed as follows, excluding elements equal to zero using a mask, ensuring that only active parameters contribute to the average:

\begin{equation}
\| \mathbf{W}_{\Delta^{*}}^{(l)} \|_2 = 
\begin{cases}
\frac{1}{m} \sum_{i=0}^{m-1} \left\| \mathbf{W}_{\Delta^{*}}^{(l)}[i, :] \right\|_2 \cdot \mathbb{I}(\mathbf{W}_{\Delta^{*}}^{(l)}[i, :] \neq 0) & \text{if } \Delta^{*} \in \{\text{up}, \text{gate}\} \\
\frac{1}{n} \sum_{j=0}^{n-1} \left\| \mathbf{W}_{\Delta^{*}}^{(l)}[:, j] \right\|_2 \cdot \mathbb{I}(\mathbf{W}_{\Delta^{*}}^{(l)}[:, j] \neq 0) & \text{if } \Delta^{*} = \text{down}
\end{cases}.
\end{equation}

Here, \(\mathbb{I}(\cdot)\) denotes the indicator function, which returns 1 only when the corresponding element is non-zero, effectively excluding zero elements from the average calculation. The average norms for the MLP layers are visualized in ~\cref{fig:mlp_lora_vis_13B,fig:mlp_lora_vis_70B}, clearly depicting the trends in updating amplitudes across the various projections.

\subsection{Attention Update Patterns}

\paragraph{Layer Update Patterns in \method and LoRA.} \cref{fig:attn_lora_vis_13B,fig:attn_lora_vis_70B} reveal that both LoRA and \method display similar layer update behaviors. In any low-rank matrix \(\mathbf{W}_{\Delta^{*}}\) where \({*} \in \{\text{q}, \text{k}, \text{v}, \text{o}\}\), deeper colors predominantly concentrate in either shallow or deep layers, while middle layers receive relatively few updates. This suggests that training primarily focuses on optimizing the shallow layers to capture semantic information, with deeper layers refining this knowledge, rendering middle layers somewhat redundant.

\paragraph{More Uniform Projection Updates in \method.} \cref{fig:attn_lora_vis_13B,fig:attn_lora_vis_70B} further indicates that updates in the LoRA-trained low-rank matrices, particularly for \(\mathbf{W}_{\Delta^{\text{v}}}\), are relatively uniform, exhibiting substantial deep colors across multiple heads. In contrast, other matrices emphasize specific rows and heads. For instance, in the 70B model's \(\mathbf{W}_{\Delta^{\text{k}}}\), only the heads in the uppermost layers experience significant updates, while lower layers show minimal changes. This suggests that the unpruned model retains rich knowledge, requiring only minor adjustments to a few heads in certain layers for task adaptation. Conversely, \method demonstrates a more uniform distribution of deep colors across each low-rank matrix, indicating that the pruned model must effectively utilize every limited neuron to capture knowledge, thereby enhancing downstream performance.

\begin{figure*}[!ht]
\begin{center}
\includegraphics[width=\textwidth]{fig/l2_norms_attn_13B.pdf}
    \caption{Visualization of low-rank matrices in the attention layers of LLaMA-2-13B.}
    \label{fig:attn_lora_vis_13B}
\includegraphics[width=\textwidth]{fig/l2_norms_mlp_13B.pdf}
    \caption{Average \(L_{2}\) norms of low-rank matrices in the MLP layers of LLaMA-2-70B.}
    \label{fig:mlp_lora_vis_13B}
\end{center}
\end{figure*}

\begin{figure*}[!t]
\begin{center}
\includegraphics[width=\textwidth]{fig/l2_norms_attn_70B.pdf}
    \caption{Visualization of low-rank matrices in the attention layers of LLaMA-2-70B.}
    \label{fig:attn_lora_vis_70B}
\includegraphics[width=\textwidth]{fig/l2_norms_mlp_70b.pdf}
    \caption{Average \(L_{2}\) norms of low-rank matrices in the MLP layers of LLaMA-2-70B.}
    \label{fig:mlp_lora_vis_70B}
\end{center}
\end{figure*}

\subsection{MLP Update Patterns}

\paragraph{\method Exhibits Greater Update Amplitude than LoRA.} For both the 13B and 70B models, \method consistently exhibits a greater update amplitude across each layer compared to LoRA, as shown in~\cref{fig:mlp_lora_vis_13B,fig:mlp_lora_vis_70B}. This increased amplitude indicates that \method is more effective in adjusting the weights in all layers, thus enhancing the adaptability and overall performance.
% \clearpage
\paragraph{Distinct Update Trends in Layer Amplitudes.} The amplitude changes reveal a distinct pattern in~\cref{fig:mlp_lora_vis_13B,fig:mlp_lora_vis_70B}: first decreasing, then increasing, and finally decreasing again. Shallow layers (0-3) and deeper layers (25-35 for the 13B model and 50–75 for the 70B model) undergo intensive updates. This behavior indicates that model prioritizes foundational feature extraction in shallow layers and the refinement of complex representations in deeper layers. Such a strategic update distribution optimizes the learning process, ensuring effective capture of basic and advanced features.

\subsection{Analysis of Unchanged Weights}
Here, we try to analyze the unchanged weights to support the motivation of LoRAM.
\paragraph{Fine-Grained Visualizations.} 
As the above visualization, we conducted detailed visualizations comparing the updated magnitudes of pruned and unpruned weights across layers. The results demonstrate that unpruned weights in both attention and MLP layers exhibit consistently smaller updates during fine-tuning as shown in \cref{fig:attn_lora_vis_70B}, indicating their critical role in preserving the model's capacity for inference.{\paragraph{Theoretical Perspective.} The phenomenon can be explained by the gradient-based importance of these weights, which prioritize parameters with minimal updates but high sensitivity during recovery. These weights stabilize inference outputs, making them indispensable despite their limited fine-tuning updates.}

{\paragraph{Quantitative Evidence} Our analysis reveals a strong correlation between weight update magnitudes and downstream performance. Pruning weights with smaller updates significantly degrades performance, highlighting their importance for inference and validating our intuition.}
{\paragraph{Impact on Large Models} The selective pruning strategy shows notable benefits in larger models such as LLaMA-2-70B, where it outperforms random pruning by a substantial margin. Retaining critical parameters ensures effective task adaptation and generalization across diverse domains.}

% \section{Pruning with Oracle}

% \section{Alignment with Distillation}
\clearpage
\section{Performance of Sub-Tasks in CSR}
\label{apd:detail_csr}
We report the performance of six sub-tasks in CSR, with~\cref{fig:csr_comparison_13B,fig:csr_comparison_70B} showcasing the results for \method-trained LLaMA-2-13B and LLaMA-2-70B, respectively. Our findings indicate that various \method variants outperform core competitive benchmarks: for the 13B model, \method surpasses both the untrained 13B and the LoRA-trained 7B, while for the 70B model, it exceeds the untrained 70B and the LoRA-trained 13B. This demonstrates that \method consistently achieves performance gains across models of different scales while effectively reducing memory usage. Furthermore, selective weight contributions in the 70B model significantly enhance performance, as evidenced by \methodstru's marked improvement, particularly in the challenging Arc Challenge multi-choice question-answering task. This suggests that \methodstru effectively identifies and leverages weight differences, focusing on the most trainable weights compared to \methodrand.
\begin{figure}[ht]
    \centering
    \begin{minipage}[b]{0.5\textwidth}
        \centering
        \includegraphics[width=\textwidth]{fig/CSR_LLaMA-2-13B_OpenHermes.pdf}
\label{fig:csr_detail_13B_hermes}
    \end{minipage}%
    \begin{minipage}[b]{0.5\textwidth}
        \centering
        \includegraphics[width=\textwidth]{fig/CSR_LLaMA-2-13B_OpenOrca.pdf}     \label{fig:csr_detail_13B_orca}
    \end{minipage}
    \caption{Performance of six CSR sub-tasks on the trained LLaMA-2-13B using \method.}
    \label{fig:csr_comparison_13B}
\begin{minipage}[b]{0.5\textwidth}
        \centering
        \includegraphics[width=\textwidth]{fig/CSR_LLaMA-2-70B_OpenHermes.pdf}
\label{fig:csr_detail_70B_hermes}
    \end{minipage}%
    \begin{minipage}[b]{0.5\textwidth}
        \centering
        \includegraphics[width=\textwidth]{fig/CSR_LLaMA-2-70B_OpenOrca.pdf}     \label{fig:csr_detail_70B_orca}
    \end{minipage}
    \caption{Performance of six CSR sub-tasks on the trained LLaMA-2-70B using \method.}
    \label{fig:csr_comparison_70B}
\end{figure}

\clearpage
\section{Algorithm of \method}
\label{alg:loram}
Here, we present the complete algorithm of \method in \cref{algo:loram}.

\begin{algorithm}[ht]
\small
\caption{\method (Memory-Efficient LoRA Training)}
\label{algo:loram}
\begin{algorithmic}[1]
\Require 
original full-rank pre-trained weight $\mathbf{W}_0$, 
alignment corpus $\mathcal{D}_\mathtt{A}$, 
and flags $\mathcal{F}^\mathtt{P}, \mathcal{F}^\mathtt{A}$, $\mathcal{F}^\mathtt{Q}, \mathcal{F}^\mathtt{R}$.

\State \mydarkcolor{\textbf{Offline $\mathbf{W}_{0}^{*}$ Process Stage:}}
\If{$\mathcal{F}^\mathtt{P}$}
    \State $\mathbf{W}_{0}^\mathtt{P} = \mathtt{P}(\mathbf{W}_{0}) =  \mathbf{W}_0 \circ \mathbf{M}^\mathtt{P}$  \Comment{\mydarkcolor{Pruned Full-Rank Weight Generation.}}
    \If{$\mathcal{F}^\mathtt{A}$} 
        \State $\mathbf{W}_{0}^\mathtt{P,A} \gets \text{argmin} {\ } \mathcal{L}_{\mathtt{A}}(\mathcal{D}_\mathtt{A};\mathbf{W}_{0}^\mathtt{P})$ 
        \Comment{\mydarkcolor{Pruned Full-Rank Weight Alignment.}}
        \If{$\mathcal{F}^\mathtt{Q}$}
        \State $\mathbf{W}_{0}^\mathtt{P,A,Q} = \mathtt{Q}(\mathbf{W}_{0}^\mathtt{P,A})$
        \Comment{\mydarkcolor{Pruned Full-Rank Weight Quantization.}}
        \EndIf
        \ElsIf{$\mathcal{F}^\mathtt{Q}$}
        \State $\mathbf{W}_{0}^\mathtt{P,Q} = \mathtt{Q}(\mathbf{W}_{0}^\mathtt{P})$
    \EndIf
    \ElsIf{$\mathcal{F}^\mathtt{Q}$} 
    \State $\mathbf{W}_{0}^\mathtt{Q} = \mathtt{Q}(\mathbf{W}_{0})$ \Comment{\mydarkcolor{Standard Quantization for LoRA}}
\EndIf

\State{Record the processing result of $\mathbf{W}_{0}$ as $\mathbf{W}_{0}^{*}$, $* \in \{\mathtt{NULL},\mathtt{P},\mathtt{Q},\mathtt{(P,Q)},\mathtt{(P,A)},\mathtt{(P,A,Q)}\}$.}
\\
\State \mydarkcolor{\textbf{Online $\mathbf{W}_{\Delta}^{*}$ Training Stage:}} 

\If{$\mathcal{F}^\mathtt{P}$} \Comment{\mydarkcolor{Pruned Low-Rank Matrix Generation.}}
    \State $\mathbf{W}_{\Delta}^\mathtt{P} = \
    \mathbf{B}^\mathtt{P}\mathbf{A}^\mathtt{P} = \
    \mathtt{P}(\mathbf{W}_{\Delta}) =  \ 
    \mathbf{W}_{\Delta} \circ \mathbf{M}^\mathtt{P} = \
    \mathbf{B}\mathbf{A} \circ \mathbf{M}^\mathtt{P}$
    \While  {$\textsc{Training}$} \Comment{\mydarkcolor{Pruned Low-Rank Matrix Training.}}
    \State Update low-rank matrix via objective $\mathcal{L}_{\mathtt{SFT}}$ with the forward pass $\mathbf{h} = \mathbf{x} \mathbf{W}_{0}^\mathtt{*} + \mathbf{x}\mathbf{W}_{\Delta}^\mathtt{P}$.
    \State Return trained low-rank matrix $\mathbf{W}_{\Delta}^{\mathtt{P}^{\star}}=\mathbf{B}^{\mathtt{P}^{\star}}\mathbf{A}^{\mathtt{P}^{\star}}$.
    \EndWhile
    \If{$\mathcal{F}^\mathtt{R}$} \Comment{\mydarkcolor{Recovered Low-Rank Matrix Generation.}}
    \State $\mathbf{W}_{\Delta}^{\mathtt{R}^{\star}} = \ 
    \mathbf{B}^{\mathtt{R}^{\star}}\mathbf{A}^{\mathtt{R}^{\star}} = \ 
    \mathtt{R}(\mathbf{W}_{\Delta}^{\mathtt{P}^{\star}}) = \ 
    \mathbf{W}_{\Delta}^{\mathtt{P}^{\star}} \circ (1-\mathbf{M}^\mathtt{P})$
    \Comment{\mydarkcolor{Structured \method}}
    \Else
    \State $\mathbf{W}_{\Delta}^{\mathtt{R}^{\star}} = \ 
    \mathbf{B}^{\mathtt{R}^{\star}}\mathbf{A}^{\mathtt{R}^{\star}} = \ 
    \mathbf{B}^{\mathtt{P}^{\star}}\mathbf{A}^{\mathtt{P}^{\star}}$
    \Comment{\mydarkcolor{Non-structured \method}}
    \EndIf
\Else
    \While  {$\textsc{Training}$} \Comment{\mydarkcolor{Standard LoRA Training.}}
    \State Update low-rank matrix via objective $\mathcal{L}_{\mathtt{SFT}}$ with the forward pass $\mathbf{h} = \mathbf{x} \mathbf{W}_{0}^\mathtt{*} + \mathbf{x}\mathbf{W}_{\Delta}$.
    \State Return trained low-rank matrix $\mathbf{W}_{\Delta}^{\star}=\mathbf{B}^{\star}\mathbf{A}^{\star}$.
    \EndWhile
\EndIf

\State{Record the trained low-rank matrix as $\mathbf{W}_{\Delta}^{*}$, $* \in \{{\mathtt{R}^{\star},\star}\}$.}
\\
\State \mydarkcolor{\textbf{Online $\mathbf{W}_{0},\mathbf{W}_{\Delta}^{*}$ Inference Stage:}} 
\While {$\textsc{Inference}$ with {$*$ is $\mathtt{R}^{\star}$}} \Comment{\mydarkcolor{Recovered Low-Rank Matrix Inference.}}
\State Perform inference with the forward pass $\mathbf{h} 
= \mathbf{x} (\mathbf{W}_0 + \mathbf{W}_{\Delta}^{\mathtt{R}^{\star}}) = \mathbf{x} (\mathbf{W}_0 + \mathbf{B}^{\mathtt{R}^{\star}}\mathbf{A}^{\mathtt{R}^{\star}})$.
\EndWhile

\While {$\textsc{Inference}$ with {$*$ is ${\star}$}} \Comment{\mydarkcolor{Standard LoRA Inference.}}
\State Perform Inference with the forward pass $\mathbf{h} 
= \mathbf{x} (\mathbf{W}_0 + \mathbf{W}_{\Delta}^{\star}) = \mathbf{x} (\mathbf{W}_0 + \mathbf{B}^{\star}\mathbf{A}^{\star})$.
\EndWhile

\end{algorithmic}
\end{algorithm}







\clearpage

\section{{Tuning of Learning Rate}}
\label{apd:detail_lr}
{We provide additional details on the learning rate tuning process for full LoRA applied to LLaMA-2-7B and LLaMA-2-13B models, trained on the OpenHermes dataset. These experiments in~\cref{fig:LR-Tuning} demonstrate that a learning rate of 1e-3 consistently achieves the best perplexity across both in-domain and out-of-domain datasets, further validating the reliability of our comparison.}

\begin{figure*}[ph]
\begin{center}
\includegraphics[width=\textwidth]{fig/LR-Tuning.pdf}
\caption{
{Learning rate tuning for LLaMA-2-7B and LLaMA-2-13B on OpenHermes using LoRA.}
}
\label{fig:LR-Tuning}
\end{center}
\end{figure*}

\section{{Performance of Domain-Specific Task}}

{To assess the effectiveness of LoRAM in domain-specific tasks, we conducted experiments on GSM8K (using the training set for tuning and the test set for evaluation), a mathematical reasoning benchmark known for its sensitivity to sparsification. Specifically, we trained LLaMA-3.1-70B using QLoRAM under various configurations.}

{The results, summarized in~\cref{tab:gsm8k_results}, highlight that LoRAM achieves excellent performance in this domain-specific setting. Notably, LoRAM-based models maintain high accuracy with substantial parameter reduction ratios, showcasing their robustness and efficiency in domain-specific tasks. These findings emphasize LoRAM's broad applicability beyond general-purpose instruction fine-tuning.}

\begin{table}[ht]
    \centering
    \caption{{Evaluation of LoRAM on the GSM8K dataset for domain-specific fine-tuning. Results show accuracy (\%) and parameter reduction ratios for different configurations.}}
    \label{tab:gsm8k_results}
    {
    \begin{tabular}{@{}lcc@{}}
        \toprule
        \textbf{LLaMA-3.1} & \textbf{GSM8K} & \textbf{Parameter Reduction Ratio} \\
        \midrule
        8B w/o Fine-Tuning            & 55.27  & 8.79× \\
        8B LoRA (OpenHermes 400)      & 55.80  & 8.79× \\
        70B w/o Fine-Tuning           & 75.28  & 1.00× \\
        70B QLoRAM-Stru 400 (OpenHermes 400) & 80.36 & \textbf{15.81×} \\
        70B QLoRAM-Stru 400 (GSM8K 100) & 77.18 & \textbf{15.81×} \\
        70B QLoRAM-Stru 400 (GSM8K 200) & 79.15 & \textbf{15.81×} \\
        70B LoRA (OpenHermes 400)     & \textbf{80.74} & 1.00× \\
        \bottomrule
    \end{tabular}}
\end{table}

\clearpage
\section{{Analysis of LoRAM Cost}}
\label{apd:diverse_metrics}

{Identifying the costs of LoRAM is indeed important, which is why we report both the number of training tokens used during the alignment phase and the parameter reduction ratios in the low-rank training phase. Below, we clarify the two stages of LoRAM:}
{\paragraph{Offline Knowledge Alignment Phase.} 
The offline phase is task-agnostic and can be conducted by the model publisher prior to deployment, making its cost negligible for end users. To quantify the offline cost, we measured the number of training tokens (as in \citet{xia2024sheared}) rather than end-to-end latency, which can vary based on hardware configurations. As shown in Figure~5, LoRAM achieves significant performance gains using only 13 million tokens, demonstrating the efficiency of the alignment phase.}
{\paragraph{Online Low-Rank Matrix Training Phase.} 
For the online phase, the memory and latency costs are primarily determined by the size of the base model parameters, which dominate resource consumption during training. To avoid redundancy in reporting, we focused on parameter reduction ratios instead of absolute time or memory usage.}
{\paragraph{Comparative Metrics for Online Training.}
Here, we provide additional metrics, including memory and latency comparisons for the online training phase. We conducted experiments using a workload of 1024 samples (batch size 128, micro-batch size 4, sequence length 512) randomly selected from OpenHermes. The results in \cref{tab:lora_comparison} demonstrate that LoRAM with a structured pruning ratio of $2.17\times$ (13B $\rightarrow$ 6B) achieves comparable peak memory, latency, and throughput to 7B LoRA, with only minor trade-offs. These differences arise due to the larger layer count in 13B LoRAM, introducing more non-GEMM operations, slightly affecting latency and throughput.}

{These results underscore the advantages of LoRAM's design in achieving substantial resource efficiency without significant trade-offs in memory or latency.}

\begin{table*}[ht]
\centering
\caption{{Comparison of peak memory (MiB), latency (s), and throughput (samples/s) during the online training phase for LoRAM and LoRA models. Results are based on a workload of 1024 samples (batch size 128, micro-batch size 4, sequence length 512).}}
{
\label{tab:lora_comparison}
\begin{tabular}{@{}lccccc@{}}
\toprule
\textbf{LLaMA-2}            & \textbf{\#Model Params} & \textbf{Reduction Ratio} & \textbf{Memory} & \textbf{Latency} & \textbf{Throughput} \\
\midrule
7B LoRA           & 6.73B                & 1.93$\times$             & 30,517                     & \textbf{134.27}      & \textbf{7.626}                 \\
13B LoRA          & 13.02B               & 1.00$\times$             & 51,661                     & 206.07               & 4.969                          \\
13B LoRAM-Stru    & \textbf{6.01B}       & \textbf{2.17$\times$}    & \textbf{29,799}            & 147.86               & 6.925                          \\
\bottomrule
\end{tabular}}
\end{table*}

\clearpage
\section{{Analysis of Changes in Performance Trends}}
{We analyze performance at two stages: after fine-tuning but before recovery, and after both fine-tuning and recovery.}
{\paragraph{After Fine-Tuning but Before Recovery.}
At this stage, the results of LoRAM align with prior work (e.g., SparseGPT, Wanda, and LLM-Pruner). Unstructured and semi-structured pruning consistently outperform structured pruning (see \cref{fig:pruning-methods}, solid lines). This trend holds true across both aligned and unaligned settings, with the performance order as follows: \methodsemi $<$ \methodunst $<$ \methodstru $<$ \methodrand
The slight advantage of \methodsemi over \methodunst can be attributed to its smaller pruning ratio, which retains more parameters and mitigates performance degradation.}
{\paragraph{After Fine-Tuning and Recovery.}
Post-recovery results show that structured pruning outperforms unstructured pruning. This can be explained by two factors:}

\begin{itemize}
    \item {\textbf{Preserved Structure for Recovery:} Structured pruning maintains the organization of the pruned weights into coherent structures (e.g., rows and columns in MLP layers, attention heads in attention layers), ensuring that activations after recovery are aligned with those of the original model. This alignment improves the recovery process.}
    \item {\textbf{Pruned Weight Quality:} The quality of pruned weights influences the recovery effectiveness. Structured pruning tends to remove less critical weights, leaving more recoverable parameters. In contrast, unstructured pruning can remove weights that are more difficult to recover, which negatively impacts performance post-recovery.}
\end{itemize}

{These results highlight the interplay between pruning and recovery, suggesting that structured pruning, despite initial performance disadvantages, facilitates more effective recovery.}




% \section{Exploration of Diverse Tasks}

% \section{Effect of Alignment Data Size}

% \section{Exploration of Learning Rate}

% \subsection{Effect of Model Scale}




\end{document}
