\newpage
\appendix

%structure of paper:
%1 intro
%2 backgroud: systolic array: mix size regularity, replace flow: not good at PE array placement, without using regularity info( 1 page)
%3 preliminary and motivation: formulation of mixed size placement problem, (better quality) -> ILP, runtime-> regularity -> reduce searching space( 1 page)
%4 method (1.5 page)
%5 experiment(1 page) 


\begin{figure*}[ht]
    \centering
    \includegraphics[width=0.9\linewidth]{figure/workfolw15.pdf}
    \caption{Overview of the Lorecast methodology.}
    \label{fig:workfolw1}
\end{figure*}

% \thispagestyle{empty}
\section{Background}
\label{sec:background}

\subsection{Large Language Models }
\label{sec:graph_clustering}
Large Language Models (LLMs) are advanced AI systems designed to understand and generate human-like text by processing vast amounts of language data\cite{Radford2019LLM}. These models are typically based on Transformer architectures\cite{Vaswani2017Transformer} and employ billions of parameters to capture complex patterns and relationships within language, enabling them to perform tasks such as text generation, translation, summarization, and question answering\cite{Brown2020LLMLearners}. Prominent examples include OpenAI’s GPT-4 \cite{OpenAI2023GPT-4} and Google’s Gemini 1.5 Pro \cite{Google2024Gemini}, showcasing the immense capabilities of LLMs. With 175 billion parameters and a context capacity of 1 million tokens, these models achieve near-human performance across a variety of Natural Language Processing (NLP) tasks. The primary strength of LLMs lies in their pretraining on extensive text corpora, which gives them a general understanding of language that can be fine-tuned for specific tasks or domains \cite{Bommasani2022finite}\cite{Liu2024RTLCoder}\cite{Thakur2023Benchmarking}. As LLMs continue to evolve, their applications are expanding across various fields, including EDA, where they are increasingly employed for tasks such as code generation\cite{Wang2024ChatCPU}\cite{Wong2024VGV}\cite{Pei2024BetterV}.
\subsection{Verilog and Code Correctness}
Verilog is a hardware description language (HDL) widely used in digital design and circuit development for specifying and modeling electronic systems at various abstraction levels, from high-level functional descriptions to detailed structural representations \cite{Palnitkar2003Verilog}. Ensuring code correctness in Verilog is crucial because errors at the HDL level can lead to significant functional failures, performance inefficiencies, and increased costs when translated to physical hardware. Code correctness in Verilog typically includes both syntax correctness, ensuring code is free from syntax errors, and functional correctness, validating that the code’s behavior aligns with design specifications \cite{Harris2010CMOS}. Recent advances in machine learning and automated code generation, including the use of LLMs, have opened new avenues for improving Verilog code correctness through automated error detection and code synthesis\cite{Wang2024ChatCPU,Pei2024BetterV}.

\iffalse
\subsection{systolic array}
describe the pipeline and parallel computing function 
due to the repeating PE structure and its dataflow.
IO and control logic unmatch to the regular schematic of the PE array
\fi


\section{Formal Proofs Regarding DCPO}
\label{sec:appendix_proof_dcpo}

\subsection{Optimizing the DCPO Loss is Optimizing the DPO Loss}
Inspired by \citet{bansal2024comparing},  we can intuitively assume that the DCPO objective is to learn an aligned model $p_\theta$ by weighting the joint probability of preferred images $p_\theta(x^w_0, z^w)$ over less preferred images $p_\theta(x^l_0, z^l)$. We set the optimization objective of DCPO is to minimize the following:

\begin{equation}
\begin{split}
    \mathcal{L}_{\text{DCPO}}(\theta) = -\mathit{\mathbb{E}}_{(x^{w}_0, x^{l}_0, z^l, z^w) \sim \mathcal{D'}} \log \sigma( 
    \beta \mathit{\mathbb{E}}_{x^{w}_{1:T}\sim p_\theta(x^{w}_{1:T}|x^{w}_0, z^w),x^{l}_{1:T} \sim p_\theta (x^{l}_{1:T}, x^{l}_0,z^l)} \\
    [\log \frac{p_{\theta} (x^{w}_{0:T},z^w)}{p_{\text{ref}}(x^{w}_{0:T},z^w)} - \log \frac{p_\theta (x^{l}_{0:T},z^l)}{p_{\text{ref}}(x^{l}_{0:T}|z^l)}])
\label{dcpo_loss}
\end{split}
\end{equation}
\\
Here, we highlight that reducing $\mathcal{L}_{\text{DCPO}}(\theta)$ is equivalently reducing $\mathcal{L}_{\text{DPO}}(\theta)$ when the captions are the same for the preferred and less preferred images.

\paragraph{Lemma 1.} \textit{Under the case where $\mathcal{D}_{\text{define}}=\{x^w_0, c, x^l_0, c\}$, that is, the image captions are identical for the given pair of preferred and less preferred images $(x^w_0, x^l_0)$, we have $L_{\text{DPO}}(\theta; \mathcal{D}_{\text{DPO}}; \beta; p_{\text{ref}}) = L_{\text{DCPO}}(\theta; \mathcal{D}_{\text{define}}; \beta; p_{\text{ref}})$, in which $\mathcal{D}_{\text{DPO}}=\{c, x^w_0, x^l_0\}.$}

\paragraph{\textbf{Proof of Lemma 1.}}

\begin{equation}
    \begin{split}
        \mathcal{L}_{\text{DCPO}}(\theta;\mathcal{D'}, \beta, p_{\text{ref}}) & = \mathit{\mathbb{E}}_{(x^w_0, x^l_0, z^w, z^l) \sim \mathcal{D'}}
        \\ &
        \left[\log \left(\sigma \left(\beta \log \frac{p_\theta(x^w_0, z^w)}{p_{\text{ref}}(x^w_0, z^w)} - 
        \beta \log \frac{p_\theta(x^l_0, z^l)}{p_{\text{ref}}(x^l_0, z^l)} \right) \right)\right] \\
        & = \mathit{\mathbb{E}}_{(x^w_0, x^l_0, z^w, z^l) \sim \mathcal{D'}}
        \\ &
        \left[\log \left( \sigma \left(\beta \log 
        \frac{p_\theta (x^w_0|z^w) p_\theta(z^w)}{p_{\text{ref}}(x^w_0|z^w)p_{\text{ref}}(z^w)} \right. \right. \right.  \left. \left. \left. - \beta \log \frac{p_\theta(x^l_0|z^l) p_\theta(z^l)}{p_{\text{ref}}(x^l_0|z^l)p_{\text{ref}}(z^l)} \right) \right) \right] \\
    \end{split}
\end{equation}

\begin{equation}
\label{eq:dcpo-lemma1}
    \begin{split}
         \mathcal{L}_{\text{DCPO}}(\theta;\mathcal{D}_{\text{define}}, \beta, p_{\text{ref}}) & \stackrel{z^w=z^l=c}{=} 
         \mathit{\mathbb{E}}_{(x^w_0, c, x^l_0, c) \sim \mathcal{D}_\text{define}} 
         \\ &
         \left[ \log \left(\sigma \left(\frac{p_\theta(c)}{p_{\text{ref}}(c)} \left ( \beta \log \frac{p_\theta(x^w_0|c)}{p_{\text{ref}}(x^w_0|c)} - \beta \log \frac{p_\theta (x^l_0|c)}{p_\text{ref}(x^l_0|c)} \right) \right) \right) \right] \\
         & \stackrel{\frac{p_\theta(c)}{p_{\text{ref}}(c)}=C}{=} \mathit{\mathbb{E}}_{(x^w_0, x^l_0, c) \sim \mathcal{D}_\text{DPO}} 
         \\ &
         \left[ \log \left(\sigma \left(C \cdot \beta \log \frac{p_\theta(x^w_0|c)}{p_{\text{ref}}(x^w_0|c)} - C \cdot  \beta \log \frac{p_\theta (x^l_0|c)}{p_\text{ref}(x^l_0|c)} \right) \right) \right] \\
         & = \mathcal{L}_{\text{DPO}}(\theta; \mathcal{D}_\text{DPO}, \beta, p_{\text{ref}} )
    \end{split}
\end{equation}
In Equation \ref{eq:dcpo-lemma1}, $C$ is a constant value that equates 
 to $\frac{p_\theta(c)}{p_{\text{ref}}(c)}$. The proof above follows the Bayes rule by substituting $c$ according to $z^w=z^l=c$.


\subsection{Analyses of DCPO's Effectiveness}
In this section, we present the formal proofs of why our DCPO leads to a more optimized $L(\theta)$ of a Diffusion-based model and, consequently, better performance in preference alignment tasks.

\paragraph{Proof 1.} \textit{Increasing the difference between $\Delta_{\text{preferred}}$ and $ \Delta_{\text{less-preferred}}$ improves the optimization of $L(\theta)$.}
\\
For better clarity, the loss function $L(\theta)$ can be written as:
$$
L(\theta) = -\mathbb{E} \left[ \log \sigma \big( -\beta T \omega(\lambda_t) \cdot M \big) \right]
$$
where $\sigma(x) $ is the sigmoid function that squashes its input $x$ into the output range $ (0, 1) $, and $\ M = \Delta_{\text{preferred}} - \Delta_{\text{less-preferred}} $, i.e., the margin between the respective importance of the preferred and less preferred predictions. 
\\
\\
Characteristically, the gradient of $ \sigma(x) $ is at its maximum near $ x = 0 $ and decreases as $ |x| $ increases. 
A larger margin in terms of $M$ makes it easier for the optimization to drive the sigmoid function towards its asymptotes, reducing loss.

\begin{itemize}
    \item When $ M $ is small ($ |M| \approx 0 $): The sigmoid $ \sigma(-\beta T \omega(\lambda_t) \cdot M) $ is near 0.5 (its midpoint). Also, the gradient of $ \log \sigma(x) $ is the largest near this point, meaning the model struggles to differentiate between preferred and less preferred predictions effectively.

    \item When $ M $ is large ($ |M| \gg 0 $): The sigmoid $ \sigma(-\beta T \omega(\lambda_t) \cdot M) $ moves closer to 0 or 1, depending on the sign of $ M $. For a well-aligned model, if the preferred predictions are correct, $ M > 0 $ and $ \sigma(-\beta T \omega(\lambda_t) \cdot M) $ approach 1, thus minimizing the loss.
    
\end{itemize}
Intuitively, an ideally large $ M $ represents a clear distinction between the preferred image-caption versus the less preferred image-caption. Thus, by maximizing $ M $, we may push the loss $ L(\theta) $ towards its minimum, leading to better soft-margin optimization. 
\\
\\
\paragraph{Proof 2.} \textit{Replacing caption $ c $ with the specifically generated caption $ z^l $ for the less-preferred image $ \mathbf{x}_0^l $ decreases $\Delta_{\text{less-preferred}}$.}
\\
\\
To analyze how replacing $ \mathbf{c} $ with $ \mathbf{z}^l $, where $ \mathbf{c} \subset \mathbf{z}^l $ and  $\mathbf{z}^l \sim Q(\mathbf{z}^l | x^l, c)$, for the less-preferred image $ \mathbf{x}_0^l $ improves the optimization, we delve into how the loss function is affected by this substitution.
\\
The term relevant to the less-preferred image $ \mathbf{x}_t^l $ in the loss is:
$$
\Delta_{\text{less-preferred}} = \| \boldsymbol{\epsilon}^l - \boldsymbol{\epsilon}_\theta(\mathbf{x}_t^l, t, \mathbf{c}) \|_2^2 - \| \boldsymbol{\epsilon}^l - \boldsymbol{\epsilon}_{\text{ref}}(\mathbf{x}_t^l, t, \mathbf{c}) \|_2^2.
$$

Replacing $ \mathbf{c} $ with $ \mathbf{z}^l $ modifies the predicted noise term $ \boldsymbol{\epsilon}_\theta(\mathbf{x}_t^l, t, \mathbf{c}) $ to $ \boldsymbol{\epsilon}_\theta(\mathbf{x}_t^l, t, \mathbf{z}^l) $. Since $ \mathbf{z}^l $ better represents $ \mathbf{x}_t^l $, we have:

\begin{equation}
\label{equ:appendix_1}
\| \boldsymbol{\epsilon}^l - \boldsymbol{\epsilon}_\theta(\mathbf{x}_t^l, t, \mathbf{z}^l) \|_2^2 < \| \boldsymbol{\epsilon}^l - \boldsymbol{\epsilon}_\theta(\mathbf{x}_t^l, t, \mathbf{c}) \|_2^2 
\end{equation}

When $ \| \boldsymbol{\epsilon}^l - \boldsymbol{\epsilon}_\theta(\mathbf{x}_t^l, t, \mathbf{z}^l) \|_2^2 $ becomes smaller, the term $ \Delta_{\text{less-preferred}} $ decreases. This leads to $\Delta_{\text{preferred}} - \Delta_{\text{less-preferred}}$ becoming larger, which improves the soft-margin optimization in the loss function $ L(\theta) $ that we have shown in Proof 1.


% Proof 3: Let's assume that replacing the caption $ c $ with a more detailed caption $ z^l $ (where $ c \subset z^l $) helps the neural network $ \boldsymbol{\epsilon}_\theta $ predict the noise better for the less preferred sample $ \mathbf{x}_t^l $.


We further elaborate on why Equation \ref{equ:appendix_1} is true. In the context of mean squared error (MSE) minimization, the optimal predictor of $ \boldsymbol{\epsilon}^l $ given some information is the conditional expectation:

\begin{itemize}
\item When conditioned on $ (\mathbf{x}_t^l, t, c) $:
$$
\boldsymbol{\epsilon}_\theta^\ast(\mathbf{x}_t^l, t, c) = \mathbb{E}\left[ \boldsymbol{\epsilon}^l \mid \mathbf{x}_t^l, t, c \right]
$$
\item When conditioned on $ (\mathbf{x}_t^l, t, z^l) $:
$$
\boldsymbol{\epsilon}_\theta^\ast(\mathbf{x}_t^l, t, z^l) = \mathbb{E}\left[ \boldsymbol{\epsilon}^l \mid \mathbf{x}_t^l, t, z^l \right]
$$
\end{itemize}
The total variance of $ \boldsymbol{\epsilon}^l $ can be decomposed as by the Law of Total Variance (conditional variance formula) \citep{ross2014introduction}:

$$
\operatorname{Var}\left( \boldsymbol{\epsilon}^l \right) = \mathbb{E}\left[ \operatorname{Var}\left( \boldsymbol{\epsilon}^l \mid \mathbf{x}_t^l, t, c \right) \right] + \operatorname{Var}\left( \mathbb{E}\left[ \boldsymbol{\epsilon}^l \mid \mathbf{x}_t^l, t, c \right] \right)
$$
Similarly, when conditioning on $ z^l $:
$$
\operatorname{Var}\left( \boldsymbol{\epsilon}^l \right) = \mathbb{E}\left[ \operatorname{Var}\left( \boldsymbol{\epsilon}^l \mid \mathbf{x}_t^l, t, z^l \right) \right] + \operatorname{Var}\left( \mathbb{E}\left[ \boldsymbol{\epsilon}^l \mid \mathbf{x}_t^l, t, z^l \right] \right)
$$

Since $ c \subset z^l $, the information provided by $ z^l $ is richer than that of $ c $. In probability theory, conditioning on more information does not increase the conditional variance:

\begin{equation}
\label{equ:appendix_2}
\operatorname{Var}\left( \boldsymbol{\epsilon}^l \mid \mathbf{x}_t^l, t, z^l \right) \leq \operatorname{Var}\left( \boldsymbol{\epsilon}^l \mid \mathbf{x}_t^l, t, c \right)
\end{equation}
This inequality holds because conditioning on additional information ($ z^l $) can only reduce or leave unchanged the uncertainty (variance) about $ \boldsymbol{\epsilon}^l $.


The expected squared error when using the optimal predictor is equal to the conditional variance:

$$
\mathbb{E}\left[ \left\| \boldsymbol{\epsilon}^l - \boldsymbol{\epsilon}_\theta^\ast(\mathbf{x}_t^l, t, c) \right\|_2^2 \right] = \mathbb{E}\left[ \operatorname{Var}\left( \boldsymbol{\epsilon}^l \mid \mathbf{x}_t^l, t, c \right) \right]
$$

Similarly,

$$
\mathbb{E}\left[ \left\| \boldsymbol{\epsilon}^l - \boldsymbol{\epsilon}_\theta^\ast(\mathbf{x}_t^l, t, z^l) \right\|_2^2 \right] = \mathbb{E}\left[ \operatorname{Var}\left( \boldsymbol{\epsilon}^l \mid \mathbf{x}_t^l, t, z^l \right) \right]
$$

From \ref{equ:appendix_2}, we have:

$$
\operatorname{Var}\left( \boldsymbol{\epsilon}^l \mid \mathbf{x}_t^l, t, z^l \right) \leq \operatorname{Var}\left( \boldsymbol{\epsilon}^l \mid \mathbf{x}_t^l, t, c \right)
$$

Taking expectations on both sides:

$$
\mathbb{E}\left[ \operatorname{Var}\left( \boldsymbol{\epsilon}^l \mid \mathbf{x}_t^l, t, z^l \right) \right] \leq \mathbb{E}\left[ \operatorname{Var}\left( \boldsymbol{\epsilon}^l \mid \mathbf{x}_t^l, t, c \right) \right]
$$

Therefore,

$$
\mathbb{E}\left[ \left\| \boldsymbol{\epsilon}^l - \boldsymbol{\epsilon}_\theta^\ast(\mathbf{x}_t^l, t, z^l) \right\|_2^2 \right] \leq \mathbb{E}\left[ \left\| \boldsymbol{\epsilon}^l - \boldsymbol{\epsilon}_\theta^\ast(\mathbf{x}_t^l, t, c) \right\|_2^2 \right]
$$

Assuming that the neural network $ \boldsymbol{\epsilon}_\theta $ is capable of approximating the optimal predictor $ \boldsymbol{\epsilon}_\theta^\ast $, especially as training progresses and the model capacity is sufficient, we can write:

$$
\left\| \boldsymbol{\epsilon}^l - \boldsymbol{\epsilon}_\theta(\mathbf{x}_t^l, t, z^l) \right\|_2^2 \approx \left\| \boldsymbol{\epsilon}^l - \boldsymbol{\epsilon}_\theta^\ast(\mathbf{x}_t^l, t, z^l) \right\|_2^2 
$$

Similarly for $ c $

$$
\left\| \boldsymbol{\epsilon}^l - \boldsymbol{\epsilon}_\theta(\mathbf{x}_t^l, t, c) \right\|_2^2 \approx \left\| \boldsymbol{\epsilon}^l - \boldsymbol{\epsilon}_\theta^\ast(\mathbf{x}_t^l, t, c) \right\|_2^2 .
$$

Therefore, the expected squared error satisfies:

$$
\mathbb{E}\left[ \left\| \boldsymbol{\epsilon}^l - \boldsymbol{\epsilon}_\theta(\mathbf{x}_t^l, t, z^l) \right\|_2^2 \right] \leq \mathbb{E}\left[ \left\| \boldsymbol{\epsilon}^l - \boldsymbol{\epsilon}_\theta(\mathbf{x}_t^l, t, c) \right\|_2^2 \right]
$$


Since the term of $ \Delta_{\text{less-preferred}} $ in the loss function involves the difference of squared errors, using $ z^l $ instead of $ c $ for the less preferred sample results in a lower error term:

$$
\Delta_{\text{less-preferred}}^{(z^l)} = \left\| \boldsymbol{\epsilon}^l - \boldsymbol{\epsilon}_\theta(\mathbf{x}_t^l, t, z^l) \right\|_2^2 - \left\| \boldsymbol{\epsilon}^l - \boldsymbol{\epsilon}_{\text{ref}}(\mathbf{x}_t^l, t, z^l) \right\|_2^2
$$

Comparing with the original:

$$
\Delta_{\text{less-preferred}}^{(c)} = \left\| \boldsymbol{\epsilon}^l - \boldsymbol{\epsilon}_\theta(\mathbf{x}_t^l, t, c) \right\|_2^2 - \left\| \boldsymbol{\epsilon}^l - \boldsymbol{\epsilon}_{\text{ref}}(\mathbf{x}_t^l, t, c) \right\|_2^2
$$

Assuming the reference model $ \boldsymbol{\epsilon}_{\text{ref}} $ remains the same or also benefits similarly from the additional information in $ z^l $, the net effect is that the first term decreases more than the second term, leading to a reduced $ \Delta_{\text{less-preferred}} $.
% Law of Total Variance (conditional variance formula): Ross, S. M. (2014). Introduction to Probability Models (11th ed.). Academic Press. 
\\
\\
\paragraph{Proof 3} \textit{Replacing caption $ c $ with the specifically generated caption $ z^w $ for the preferred image $ \mathbf{x}_0^w $ increases $\Delta_{\text{preferred}}$.}
\\
\\
To prove that replacing $ \mathbf{c} $ with $ \mathbf{z}^w  \sim Q(z^w|x^w, c) $, where $ \mathbf{c} \subset \mathbf{z}^w $, for $ \mathbf{x}_0^w $ also contributes to a better optimized loss $L(\theta)$, we examine how this particular substitution affects the loss function.
\\
We let
$$
R_\theta(\mathbf{c}) = \| \boldsymbol{\epsilon}^w - \boldsymbol{\epsilon}_\theta(\mathbf{x}_t^w, t, \mathbf{c}) \|_2^2,
$$
$$
R_{\text{ref}}(\mathbf{c}) = \| \boldsymbol{\epsilon}^w - \boldsymbol{\epsilon}_{\text{ref}}(\mathbf{x}_t^w, t, \mathbf{c}) \|_2^2.
$$
\\
The rate of decrease in $ R_\theta $ due to $ \mathbf{z}^w $ is proportional to the model's ability to exploit the additional conditioning. Since $ \boldsymbol{\epsilon}_\theta $ is learnable, it can more effectively leverage $ \mathbf{z}^w $ than $ \boldsymbol{\epsilon}_{\text{ref}} $, yielding:
$$
\Delta R_\theta = R_\theta(\mathbf{c}) - R_\theta(\mathbf{z}^w) \gg \Delta R_{\text{ref}} = R_{\text{ref}}(\mathbf{c}) - R_{\text{ref}}(\mathbf{z}^w).
$$

% Thus:
% $$
% \Delta_{\text{preferred}} = R_\theta - R_{\text{ref}}
% $$
% increases as $ R_\theta $ decreases faster than $ R_{\text{ref}} $ we will show in Proof 5.



% In conclusion, replacing $ \mathbf{c} $ with $ \mathbf{z}^w $ for the preferred image $ \mathbf{x}_t^w $ improves optimization by increasing $ \Delta_{\text{preferred}} $, leading to a larger margin $ \Delta_{\text{preferred}} - \Delta_{\text{less-preferred}} $. This enhances soft-margin optimization, resulting in faster convergence and better differentiation between preferred and less preferred predictions.


% Proof 5: 
We further elaborate on why the learnable model's noise prediction residual ($ R_\theta $) decreases faster than the reference model's residual ($ R_{\text{ref}} $) when $ \mathbf{c} $ is replaced by $ \mathbf{z}^w $. The residuals for the learnable and reference models are defined as:

$$
R_\theta(\mathbf{c}) = \| \boldsymbol{\epsilon}^w - \boldsymbol{\epsilon}_\theta(\mathbf{x}_t^w, t, \mathbf{c}) \|_2^2,
$$
$$
R_{\text{ref}}(\mathbf{c}) = \| \boldsymbol{\epsilon}^w - \boldsymbol{\epsilon}_{\text{ref}}(\mathbf{x}_t^w, t, \mathbf{c}) \|_2^2.
$$

When $ \mathbf{c} $ is replaced with $ \mathbf{z}^w $ (where $ \mathbf{c} \subset \mathbf{z}^w $), the residuals become:

$$
R_\theta(\mathbf{z}^w) = \| \boldsymbol{\epsilon}^w - \boldsymbol{\epsilon}_\theta(\mathbf{x}_t^w, t, \mathbf{z}^w) \|_2^2,
$$
$$
R_{\text{ref}}(\mathbf{z}^w) = \| \boldsymbol{\epsilon}^w - \boldsymbol{\epsilon}_{\text{ref}}(\mathbf{x}_t^w, t, \mathbf{z}^w) \|_2^2.
$$

The rate of decrease for each residual is defined as:
$$
\Delta R_\theta = R_\theta(\mathbf{c}) - R_\theta(\mathbf{z}^w),
$$
$$
\Delta R_{\text{ref}} = R_{\text{ref}}(\mathbf{c}) - R_{\text{ref}}(\mathbf{z}^w).
$$

The quality of conditioning, $ Q(\mathbf{c}) $, represents how well the conditioning $ \mathbf{c} $ aligns with the true noise~$ \boldsymbol{\epsilon}^w $. We assume that
$$
Q(\mathbf{z}^w) > Q(\mathbf{c}),
$$

where the improvement in conditioning quality $ \Delta Q $ is defined as
$$
\Delta Q = Q(\mathbf{z}^w) - Q(\mathbf{c}).
$$


The residual for $ R_\theta $ is proportional to the misalignment between $ Q(\mathbf{c}) $ and $ \boldsymbol{\epsilon}^w $:
 $$
 R_\theta(\mathbf{c}) \propto \frac{1}{Q(\mathbf{c})}.
 $$

Replacing $ \mathbf{c} $ with $ \mathbf{z}^w $ (higher $ Q $) results in a larger proportional reduction:
 $$
 R_\theta(\mathbf{z}^w) \propto \frac{1}{Q(\mathbf{z}^w)} \quad \text{with} \quad \Delta R_\theta \propto \Delta Q.
 $$


The reference model's residual $ R_{\text{ref}} $ depends weakly on $ Q(\mathbf{c}) $, as it is fixed or less adaptable:
$$
R_{\text{ref}}(\mathbf{c}) \propto \frac{1}{Q_{\text{ref}}(\mathbf{c})},
$$

where $ Q_{\text{ref}}(\mathbf{c}) $ is less sensitive to changes in $ \mathbf{c} $.

Thus, the proportional improvement in $ R_\theta $ due to $ \Delta Q $ is significantly larger than for $ R_{\text{ref}} $.



The preferred difference term is:
$$
\Delta_{\text{preferred}} = R_\theta - R_{\text{ref}}.
$$

As $ R_\theta $ decreases significantly more than $ R_{\text{ref}} $, the gap $ R_\theta - R_{\text{ref}} $ becomes larger, increasing $ \Delta_{\text{preferred}} $:
$$
\Delta R_\theta \gg \Delta R_{\text{ref}} \implies \Delta_{\text{preferred}} \text{ increases.}
$$


The learnable model $ \boldsymbol{\epsilon}_\theta $ benefits more from the improved conditioning $ \mathbf{z}^w $ because of its adaptability and training dynamics. This results in a larger reduction in $ R_\theta $ compared to $ R_{\text{ref}} $. Mathematically, the relative rate of decrease:
$$
\text{Relative Rate} = \frac{\Delta R_\theta}{\Delta R_{\text{ref}}} \gg 1,
$$

which ensures that $ \Delta_{\text{preferred}} $ also increases, hence improving the optimization process in $L(\theta)$ and helping the model distinguish predictions on preferred and less preferred image-captions more effectively.


\section{More Insights on DCPO}
\label{sec:appendix_more_insights}



A preference alignment dataset, such as Pick-a-Pic \citep{kirstain2023pickapic}, is defined as $ D = \{c, x^w, x^l\} $, where $ x^w $ and $ x^l $ represent the preferred and less preferred images for the prompt $ c $. Diffusion-KTO \citep{li2024aligning} hypothesizes the optimization of a diffusion model using only a single preference label based on whether an image $ x $ is suitable or not for a given prompt $ c $.  Diffusion-KTO uses a differently formatted input dataset $ D = \{c, x\} $, where $ x $ is a generated image corresponding to the prompt $ c $.

\begin{table*}[h]
\centering
\small
\begin{tabular}{c|ccccc}
\toprule
\textbf{Method} & \textbf{GenEval ($\uparrow$)}         & \textbf{Pickscore ($\uparrow$)}      & \textbf{HPSv2.1 ($\uparrow$)} & \textbf{ImageReward ($\uparrow$)} & \textbf{CLIPscore ($\uparrow$)} \\ \midrule
Diffusion-KTO  &   0.5008   &  20.41   &   24.80    &   55.5   &    26.95  \\ 
DCPO-h & \textbf{0.5100}   & \textbf{20.57}  &   \textbf{25.62}   &    \textbf{58.2}    &   \textbf{27.13}   \\ \bottomrule
\end{tabular}
\caption{Comparison of DCPO-h and Diffusion-KTO across various benchmarks.}
\label{tab:comparsion_dcpo_kto}
\end{table*}



Diffusion-KTO's hypothesis is fundamentally different from our DCPO's. While Diffusion-KTO focuses on binary preferences (like/dislike) for individual image-prompt pairs, our approach involves paired preferences. We observe that using the same prompt $ c $ for both preferred and less preferred images may not be ideal. To address this, we propose optimizing a diffusion model using a dataset in terms of $ D = \{z^w, z^l, x^w, x^l\} $, where $ z^w $ and $ z^l $ are the captions generated by a static captioning model $ Q_\phi $ for the preferred and less preferred images, respectively, referring to the original prompt.

% 
\begin{table}[t]
    \caption{Performance comparison of DCPO-h and DCPO-p across different perturbation levels. The perturbation method has a strong impact on captions that are more closely correlated with images.}
    \centering
    \resizebox{\linewidth}{!}{
    \begin{tabular}{l|ccccccc}
        \toprule
        \textbf{Method} & \textbf{Pair Caption} & \textbf{Perturbed Level} & \textbf{Pickscore ($\uparrow$)} & \textbf{HPSv2.1 ($\uparrow$)} & \textbf{ImageReward ($\uparrow$)} & \textbf{CLIPscore ($\uparrow$)} & \textbf{GenEval ($\uparrow$)} \\
        \midrule
        
        DCPO-p & ($c, c_{p}$) & weak  & 20.28 & 25.42 & 54.20 & 26.98 & 0.4906 \\
        DCPO-h & ($z^w, z^w_p$) & weak  & 20.55 & 25.61 & 57.70 & 27.07 & \textbf{0.5070} \\
        DCPO-h & ($z^w, z^l_p$) & weak  & \textbf{20.58} & \textbf{25.70} & \textbf{58.10} & \textbf{27.15} & 0.5060 \\
        \midrule
        DCPO-p & ($c, c_{p}$) & medium  & 20.21 & 25.34 & 53.10 & 26.87 & 0.4852 \\
        DCPO-h & ($z^w, z^w_p$) & medium  & \textbf{20.59} & \textbf{25.73} & \textbf{58.47} & 27.12 & 0.5008\\
        DCPO-h & ($z^w, z^l_p$) & medium  & 20.57 & 25.62 & 58.20 & \textbf{27.13} & \textbf{0.5100}\\
        \midrule
        DCPO-p & ($c, c_{p}$) & strong  & 20.31 & 25.06 & 54.60 & 27.03 & 0.4868 \\
        DCPO-h & ($z^w, z^w_p$) & strong  & 20.57 & 25.27 & 57.43 & 27.18 & \textbf{0.5110}\\
        DCPO-h & ($z^w, z^l_p$) & strong  & \textbf{20.58} & \textbf{25.43} & \textbf{57.90} & \textbf{27.21} & 0.4993\\

        \bottomrule
    \end{tabular}
    }
    \label{tab:hyphotesis2-results}
\end{table}
% 
\begin{table}[t]
    \caption{Performance comparison of DCPO and Diffusion-DPO fine-tuned on the \textit{Pick-Double Caption} dataset. While larger captions improve the performance of Diffusion-DPO, DCPO-h still significantly outperforms Diffusion-DPO.}
    \centering
    \resizebox{\linewidth}{!}{
    
    \begin{tabular}{l|c|c|cccccc}
        \toprule
        \textbf{Method} & \textbf{Input Prompt} & \textbf{Token Length (Avg)} & \textbf{Pickscore ($\uparrow$)} & \textbf{HPSv2.1 ($\uparrow$)} & \textbf{ImageReward ($\uparrow$)} & \textbf{CLIPscore ($\uparrow$)} & \textbf{GenEval ($\uparrow$)} \\
        \midrule
        
        Diffusion-DPO & prompt $c$ & 15.95 & 20.36 & 25.10 & 56.4 & 26.98 & 0.4857 \\
        Diffusion-DPO & caption $z^w$ (LLaVA)  & 32.32 & \underline{20.40} & 25.19 & 56.6 & 27.10 & 0.4958 \\
        Diffusion-DPO & caption $z^w$ (Emu2)  & 7.75 & 20.36 & 25.08 & 56.3 & 26.98 & 0.4960 \\
        \midrule
        DCPO-h (LLaVA) & Pair ($z^w$,$z^l_p$) & (32.32, 31.17) & \textbf{20.57} & \textbf{25.62} & \textbf{58.2} & \underline{27.13} & \underline{0.5100}\\
        DCPO-h (LLaVA) & Pair ($z^w$,$z^w_p$) & (32.32, 27.01) & \textbf{20.57} & \underline{25.27} & \underline{57.4} & \textbf{27.18} & \textbf{0.5110}\\

        \bottomrule
    \end{tabular}
    }
    \label{tab:compared_dpo}
\end{table}
We nonetheless conduct comparisons between Diffusion-KTO and DCPO on various preference alignment benchmarks. The results in Table \ref{tab:comparsion_dcpo_kto} show that our DCPO-h consistently outperforms Diffusion-KTO on all benchmarks, demonstrating the effectiveness of our DCPO method.


% \label{sec:appendix_bachground}

% \section{Proof for DCPO reducing DPO}
% \label{sec:app_proof}
% Here, we highlight a result that reduces DCPO to DPO when the prompts are the same.
% \paragraph{Lemma 1.} \textit{Under the case where $\mathcal{D}_{\text{define}}=\{x^w_0, c, x^l_0, c\}$, that is, captions are the same for the preferred and less preferred images pairs, $L_{\text{DPO}}(\theta; \mathcal{D}_{\text{DPO}}; \beta; p_{\text{ref}}) = L_{\text{DCPO}}(\theta; \mathcal{D}_{\text{define}}; \beta; p_{\text{ref}})$, where $\mathcal{D}_{\text{DPO}}=\{c, x^w_0 x^l_0\}.$}

% \textit{Proof.}

% \begin{equation}
%     \begin{split}
%         \mathcal{L}_{\text{DCPO}}(\theta;\mathcal{D'}, \beta, p_{\text{ref}}) & = \mathit{\mathbb{E}}_{(x^w_0, x^l_0, z^w, z^l) \sim \mathcal{D'}}
%         \left[\log \left(\sigma \left(\beta \log \frac{p_\theta(x^w_0, z^w)}{p_{\text{ref}}(x^w_0, z^w)} - 
%         \beta \log \frac{p_\theta(x^l_0, z^l)}{p_{\text{ref}}(x^l_0, z^l)} \right) \right)\right] \\
%         & = \mathit{\mathbb{E}}_{(x^w_0, x^l_0, z^w, z^l) \sim \mathcal{D'}}
%         \left[\log \left( \sigma \left(\beta \log 
%         \frac{p_\theta (x^w_0|z^w) p_\theta(z^w)}{p_{\text{ref}}(x^w_0|z^w)p_{\text{ref}}(z^w)} \right. \right. \right.  \left. \left. \left. - \beta \log \frac{p_\theta(x^l_0|z^l) p_\theta(z^l)}{p_{\text{ref}}(x^l_0|z^l)p_{\text{ref}}(z^l)} \right) \right) \right] \\
%     \end{split}
% \end{equation}

% \begin{equation}
%     \begin{split}
%          \mathcal{L}_{\text{DCPO}}(\theta;\mathcal{D}_{\text{define}}, \beta, p_{\text{ref}}) & \stackrel{z^w=z^l=c}{=} \mathit{\mathbb{E}}_{(x^w_0, c, x^l_0, c) \sim \mathcal{D}_\text{define}} \left[ \log \left(\sigma \left(\frac{p_\theta(c)}{p_{\text{ref}}(c)} \left ( \beta \log \frac{p_\theta(x^w_0|c)}{p_{\text{ref}}(x^w_0|c)} - \beta \log \frac{p_\theta (x^l_0|c)}{p_\text{ref}(x^l_0|c)} \right) \right) \right) \right] \\
%          & \stackrel{\frac{p_\theta(c)}{p_{\text{ref}}(c)}=C}{=} \mathit{\mathbb{E}}_{(x^w_0, x^l_0, c) \sim \mathcal{D}_\text{DPO}} \left[ \log \left(\sigma \left(C. \beta \log \frac{p_\theta(x^w_0|c)}{p_{\text{ref}}(x^w_0|c)} - C. \beta \log \frac{p_\theta (x^l_0|c)}{p_\text{ref}(x^l_0|c)} \right) \right) \right] \\
%          & = \mathcal{L}_{\text{DPO}}(\theta; \mathcal{D}_\text{DPO}, \beta, p_{\text{ref}} )
%     \end{split}
% \end{equation}

% Where $C$ is a constant value, the proof follows from applying the Bayes rule and substituting $z^w=z^l=c$.

\section{Pick-Double Caption Dataset}
\label{sec:appendix_double_caption_dataset}
In this section, we provide details about the \textit{Pick-Double Caption} dataset. As discussed in Section \ref{sec:pick-double-caption}, we sampled 20,000 instances from the Pick-a-Pic v2 dataset and excluded those with equal preference scores. We plot the distribution of the original prompts, as shown in Figure \ref{fig:token-distribution}.

\begin{figure}[h]
    \centering
    \includegraphics[width=1\textwidth]{images/distribution_tokens.pdf}
    \vspace{-1em}
    \caption{Token distribution of original prompt.}
    \label{fig:token-distribution}
\end{figure}

We observed that some prompts contained only one or two words, while others were excessively long. To ensure a fair comparison, we removed prompts that were too short or too long, leaving us with approximately 17,000 instances. We then generated captions using two state-of-the-art models, LLaVA-1.6-34B, and Emu2-32B. Figure \ref{fig:pick_double_caption_samples} provides examples from the dataset.

As explained in Section \ref{sec:pick-double-caption}, we utilized two types of prompts to generate captions: 1) Conditional prompt and 2) Non-conditional prompt. Below, we outline the specific prompts used for each captioning method.

\begin{tcolorbox}[colback=black!5, colframe=black, title= Example of Conditional Prompt]
Using one sentence, describe the image based on the following prompt: \textit{playing chess tournament on the moon.}
\end{tcolorbox}

\begin{tcolorbox}[colback=black!5, colframe=black, title= Example of Non-Conditional Prompt]
Using one sentence, describe the image.
\end{tcolorbox}


Table \ref{tab:tokens-detail} presents a statistical analysis of the \textit{Pick-Double Caption} dataset. With the non-conditional prompt method, we found that the average token length of captions generated by LLaVA is similar to that of the original prompts. However, captions generated by LLaVA using conditional prompts are twice as long as the original prompts. Additionally, Emu2 generated captions that, on average, are half the length of the original prompts for both methods.



% \begin{wraptable}{r}{0.5\linewidth}
% \small
% \centering
% \caption{Statistical information on the Pick-Double Caption dataset, including the CLIPscore of in-distribution data and average token count of captions generated by LLaVA and Emu2 for both in-distribution and out-of-distribution data. [SHOULD GO TO APPENDIX]}
% \label{tab:tokens-detail}
% \resizebox{\linewidth}{!}{%
% \begin{tabular}{@{}lccccc@{}}
% \toprule
% \multicolumn{1}{c}{\textbf{Text}} & \stackanchor{\textbf{Token Len.}}{\textbf{(Avg-in)}}& \stackanchor{\textbf{Token Len.}}{\textbf{(Avg-out)}} &  \textbf{\stackanchor{CLIP}{score (in)}} &  \textbf{\stackanchor{CLIP}{score (out)}} \\ 
% \midrule
% prompt $c$ & 15.95 & 15.95 & (26.74, 25.41)\\
% \midrule
% caption $z^w$ (LLaVA) & 32.32 & 17.69 & 30.85 \\
% caption $z^l$ (LLaVA) & 32.83 & 17.91 & 26.48 \\
% caption $z^w$ (Emu2) & 7.75 & 8.40 & 25.44 \\
% caption $z^l$ (Emu2) & 7.84 & 8.44 & 22.64 \\

% \bottomrule
% \end{tabular}%
% }
% \end{wraptable}

\begin{table}[ht]
    \caption{Statistical information on the Pick-Double Caption dataset, including the CLIPscore of in-distribution data and average token count of captions generated by LLaVA and Emu2 for both in-distribution and out-of-distribution data. \\}
    \small
    \centering
    
\begin{tabular}{@{}lccccc@{}}
\toprule
\multicolumn{1}{c}{\textbf{Text}} & \stackanchor{\textbf{Token Len.}}{\textbf{(Avg-in)}}& \stackanchor{\textbf{Token Len.}}{\textbf{(Avg-out)}} &  \textbf{\stackanchor{CLIP}{score (in)}} &  \textbf{\stackanchor{CLIP}{score (out)}} \\ 
\midrule
prompt $c$ & 15.95 & 15.95 & (26.74, 25.41) & (26.74, 25.41)\\
\midrule
caption $z^w$ (LLaVA) & 32.32 & 17.69 & 30.85 & 29.04 \\
caption $z^l$ (LLaVA) & 32.83 & 17.91 & 26.48 & 28.29 \\
caption $z^w$ (Emu2) & 7.75 & 8.40 & 25.44 & 25.18\\
caption $z^l$ (Emu2) & 7.84 & 8.44 & 22.64 & 24.88\\

\bottomrule
\end{tabular}%
    
    \label{tab:tokens-detail}
\end{table}



\begin{figure}[h]
    \centering
    
    \includegraphics[width=1\linewidth]{images/dataset_preview.pdf}
    \vspace{-2em}
    \caption{Examples of Pick-Double Caption dataset.}
    \label{fig:pick_double_caption_samples}
\end{figure}

\section{More Details on Perturbation Method}
\label{sec:appendix_perturbation}

%Please add the following packages if necessary:
%\usepackage{booktabs, multirow} % for borders and merged ranges
%\usepackage{soul}% for underlines
%\usepackage{xcolor,colortbl} % for cell colors
%\usepackage{changepage,threeparttable} % for wide tables
%If the table is too wide, replace \begin{table}[!htp]...\end{table} with
%\begin{adjustwidth}{-2.5 cm}{-2.5 cm}\centering\begin{threeparttable}[!htb]...\end{threeparttable}\end{adjustwidth}
\begin{table}[h]
\caption{Examples of perturbed prompts and captions after applying different levels of perturbation.}
\setlength\tabcolsep{4pt} % default: 6pt
\label{tab:perturbation-examples}

\scriptsize
\begin{tabularx}{\textwidth}{@{} C{0.4} @{}| L{1.2} @{} L{1.2} @{} L{1.2} @{}} % 1.15+0.8+1.05 = 3 = # of X-type columns

\toprule
&\multicolumn{1}{c}{\textbf{Weak}} &\multicolumn{1}{c}{\textbf{Medium}}  &\multicolumn{1}{c}{\textbf{Strong}} \\\midrule
\textbf{Prompt $c_p$} &Cryptocrystalline quartz, melted gemstones, telepathic AI style. &Painting of cryptocrystalline quartz. Melted gems. Sacred geometry. &Cryptocrystalline quartz with melted stones, in telepathic AI style. \\
\midrule
\stackunder{\textbf{Caption $z^w_p$}}{\textbf{(LLaVA)}} &A digital artwork featuring a symmetrical, kaleidoscopic pattern with vibrant colors and a central star-like motif. &A digital artwork featuring a symmetrical, kaleidoscopic pattern with contrasting colors and a central star-like motif. &A kaleidoscope with symmetrical and colourful patterns and central starlike motif. \\
\midrule
\stackunder{\textbf{Caption $z^l_p$}}{\textbf{(LLaVA)}} &A vivid circular stained-glass art with a symmetrical star design in its center. &The image is of a radially symmetrical stained-glass window. &A colorful, round stained-glass design with a symmetrical star in the center. \\
\midrule
\stackunder{\textbf{Caption $z^w_p$}}{\textbf{(Emu2)}} &Abstract image with glass. &An abstract image of colorful stained glass. &An abstract picture with glass in many colors. \\
\midrule
\stackunder{\textbf{Caption $z^l_p$}}{\textbf{(Emu2)}} &An abstract circular design with leaves. &A colourful round design with leaves. &Brightly colored circular design. \\

\bottomrule
\multicolumn{4}{c}{Original Prompt $c$: \textbf{\textit{Painting of cryptocrystalline quartz melted gemstones sacred geometry pattern telepathic AI style}}} \\
\end{tabularx}
\end{table}

We provide the setups for the LLM-based perturbation process involved in the DCPO-p and DCPO-h pipelines. Similarly to the method of constructing paraphrasing adversarial attacks as synonym-swapping perturbation by \citet{dipper}, we use DIPPER \citep{dipper}, a text generation model built by fine-tuning T5-XXL~\citep{t5}, to create semantically perturbed captions or prompts, as shown in Table \ref{tab:perturbation-examples}. Our three levels of perturbation are achieved by only altering the setting of lexicon diversity (0 to 100) in DIPPER - we use 40 for \textbf{Weak}, 60 for \textbf{Medium}, and 80 for \textbf{Strong}. We also use \textit{"Text perturbation for variable text-to-image prompt."} to prompt the perturbation. We hereby provide a code snippet to showcase the whole process to perturb a sample input:

\begin{lstlisting}[language=Python]
from transformers import T5Tokenizer, T5ForConditionalGeneration
class DipperParaphraser(object):
    # As defined in https://huggingface.co/kalpeshk2011/dipper-paraphraser-xxl
    
prompt = "Text perturbation for variable text-to-image prompt."
input_text = "playing chess tournament on the moon."

dp = DipperParaphraser()

cap_weak = dp.paraphrase(input_text, lex_diversity=40, prefix=prompt, do_sample=True, top_p=0.75, top_k=None, max_length=256)
cap_medium = dp.paraphrase(input_text, lex_diversity=60, prefix=prompt, do_sample=True, top_p=0.75, top_k=None, max_length=256)
cap_strong = dp.paraphrase(input_text, lex_diversity=80, prefix=prompt, do_sample=True, top_p=0.75, top_k=None, max_length=256)
\end{lstlisting}


\section{More Details about Training of Diffusion Models}
\label{sec:appendix_details_train}
In this section, we provide a detailed explanation of the fine-tuning methods used. We fine-tuned SD 2.1 with the best hyperparameters reported in the original papers for \( \text{SFT}_{\text{Chosen}} \), Diffusion-DPO, and MaPO, using 8 A100×80 GB GPUs for all models. To fine-tune SD 2.1 with Diffusion and MaPO methods, we used a dataset \(D=\{c,x^w,x^l\}\) where \(c,x^w,x^l\) represent the prompt, preferred image, and less preferred image. To optimize a SD2.1 with \( \text{SFT}_{\text{Chosen}} \) we utilized a dataset \(D=\{c,x^w\}\) where \(c,x^w\) represent the prompt, preferred image and image. In this paper, dataset $D$ represents the sampled and cleaned version of the Pick-a-Pic v2 dataset. Additionally, we clarify the DCPO models DCPO-c, DCPO-p, and DCPO-h. In this paper, DCPO-c and DCPO-p refer to SD 2.1 models fine-tuned with the DCPO method, using LLaVA and Emu2 for captioning and perturbation methods at three distinct levels, respectively. The main results for DCPO-p in the text are based on weak perturbation applied to the original prompt. In Table \ref{tab:hyphotesis2-results}, we also report DCPO-p's performance across other perturbation levels.


For DCPO-h, we applied perturbations to both the preferred and less preferred captions generated by LLaVA. The reported results for DCPO-h reflect a medium level of perturbation applied to the less preferred caption. In Table \ref{tab:hyphotesis2-results}, we present the performance of DCPO-h across various perturbation levels, including perturbations to the preferred captions. Additionally, in Table \ref{tab:emu2-perturbed-caption-results}, we show the results for DCPO-h using captions generated by Emu2.


\begin{table}[h]
    \caption{Results of the perturbation method applied to Emu2 captions across different levels.}
    \centering
    \resizebox{\linewidth}{!}{
    \begin{tabular}{l|ccccccc}
        \toprule
        \textbf{Method} & \textbf{Pair Caption} & \textbf{Perturbed Level} & \textbf{Pickscore ($\uparrow$)} & \textbf{HPSv2.1 ($\uparrow$)} & \textbf{ImageReward ($\uparrow$)} & \textbf{CLIPscore ($\uparrow$)} & \textbf{GenEval ($\uparrow$)} \\
        \midrule
        DCPO-h & ($z^w, z^w_p$) & weak  & 20.10 & 21.23 & 49.7 & 26.87 & 0.5003 \\
        DCPO-h & ($z^w, z^l_p$) & weak  & 20.32 & 23.4 & 53.8 & 27.06 & 0.5070 \\
        \midrule
        DCPO-h & ($z^w, z^w_p$) & medium  & 20.31 & 23.08 & 53.2 & 27.01 & 0.4895 \\
        DCPO-h & ($z^w, z^l_p$) & medium  & 20.33 & 23.22 & 53.8 & 27.09 & 0.5009 \\
        \midrule
        DCPO-h & ($z^w, z^w_p$) & strong  & 20.31 & 22.95 & 53.1 & 27.11 & 0.4878 \\
        DCPO-h & ($z^w, z^l_p$) & strong  & 20.35 & 23.24 & 53.63 & 27.08 & 0.5050 \\

        \bottomrule
    \end{tabular}
    }
    \label{tab:emu2-perturbed-caption-results}
\end{table}


The key findings indicate that perturbation on short captions not only fails to improve performance but also produces worse outcomes compared to DCPO-c (Emu2).

Additionally, we conducted more experiments on in-distribution and out-of-distribution data. For this, we generated out-of-distribution data using LLaVA and Emu2 in the captioning setup. As shown in Figure \ref{fig:in-vs-out-dcpo-c}, in-distribution data generally outperformed out-of-distribution data. However, the most significant improvement was observed with the hybrid method, as reported in Figure \ref{fig:in-vs-out}.


\begin{figure}[h]
    \centering
    \includegraphics[width=1\textwidth]{images/in_vs_out_plot_dcpo_c.pdf}
    \vspace{-2em}
    \caption{Comparison of DCPO-c performance on in-distribution and out-of-distribution data.}
    \label{fig:in-vs-out-dcpo-c}
\end{figure}


Table \ref{tab:beta-hyperparameter-results} presents the performance details for different values of \( \beta \), conducted using the medium level of DCPO-h. The results indicate that while lower values of \( \beta \) significantly improve GenEval and HPSv2.1 on average, the optimal value for \( \beta \) is 5000. We suggest that this hyperparameter may vary based on the dataset and task.


\begin{table}[h]
    \caption{Results of DCPO-h across different $\beta$.}
    \centering
    \resizebox{\linewidth}{!}{
    \begin{tabular}{l|c|ccccc}
        \toprule
        \textbf{Method} & \textbf{$\beta$} & \textbf{Pickscore ($\uparrow$)} & \textbf{HPSv2.1 ($\uparrow$)} & \textbf{ImageReward ($\uparrow$)} & \textbf{CLIPscore ($\uparrow$)} & \textbf{GenEval ($\uparrow$)} \\
        \midrule
        
        DCPO-h  & 500 & 20.43 & \textbf{26.42} & 58.1 & 27.02 & \textbf{0.5208} \\
        DCPO-h  & 1000 & 20.51 & \underline{26.12} & \textbf{58.2} & \underline{27.10} & 0.4900 \\
        DCPO-h  & 2500 & \underline{20.53} & 25.81 & 58.0 & 27.02 & 0.5036 \\
        \midrule
        DCPO-h  & 5000  & \textbf{20.57} & 25.62 & \textbf{58.2} & \textbf{27.13} & \underline{0.5100} \\


        \bottomrule
    \end{tabular}
    }
    \label{tab:beta-hyperparameter-results}
\end{table}

\section{GPT-4o as an Evaluator}
\label{gpt4o_evaluator}
To obtain binary preferences from the API evaluator, we followed the approach outlined in the MaPO paper \citep{hong2024marginawarepreferenceoptimizationaligning}. Similar to Diffusion-DPO, we used three distinct questions to evaluate the images generated by the DCPO-h and Diffusion-DPO models, both utilizing SD 2.1 as the backbone. These questions were presented to the GPT-4o model to identify the preferred image. Below, we provide details of the prompts used.

\begin{tcolorbox}[colback=black!5, colframe=black, title= GPT-4o Evaluation Prompt for Q1: General Preference]
Select the output (a) or (b) that best matches the given prompt. Choose your preferred output, which can be subjective. Your answer should ONLY contain: Output (a) or Output (b).
\\

\#\# Prompt:\\
\texttt{\{prompt\}}
\\

\#\# Output (a):\\
The first image attached.
\\

\#\# Output (b):\\
The second image attached.
\\
\\

\#\# Which image do you prefer given the prompt?
\end{tcolorbox}

\begin{tcolorbox}[colback=black!5, colframe=black, title= GPT-4o Evaluation Prompt for Q2: Visual Appeal]
Select the output (a) or (b) that best matches the given prompt. Choose your preferred output, which can be subjective. Your answer should ONLY contain: Output (a) or Output (b).
\\

\#\# Prompt:\\
\texttt{\{prompt\}}
\\

\#\# Output (a):\\
The first image attached.
\\

\#\# Output (b):\\
The second image attached.
\\
\\

\#\# Which image is more visually appealing?
\end{tcolorbox}


\begin{tcolorbox}[colback=black!5, colframe=black, title= GPT-4o Evaluation Prompt for Q3: Prompt Alignment]
Select the output (a) or (b) that best matches the given prompt. Choose your preferred output, which can be subjective. Your answer should ONLY contain: Output (a) or Output (b).
\\

\#\# Prompt:\\
\texttt{\{prompt\}}
\\

\#\# Output (a):\\
The first image attached.
\\

\#\# Output (b):\\
The second image attached.
\\
\\

\#\# Which image better fits the text description?
\end{tcolorbox}

\newpage
\section{Additional Generation Samples}
\label{sec:app_additional_examples}
We also present additional samples for qualitative comparison generated by SD 2.1, \( \text{SFT}_{\text{Chosen}} \), Diffusion-DPO, MaPO, and DCPO-h from prompts on Pickscore, HPSv2, and GenEval benchmarks.

\begin{figure}[h]
    \centering
    \includegraphics[width=1\linewidth]{images/appendix_more_samples_hps.pdf}
    \caption{Additional generated outcomes using prompts from HPSv2 benchmark.}
    \label{fig:enter-label}
\end{figure}

\begin{figure}[h]
    \centering
    \includegraphics[width=1\linewidth]{images/appendix_more_samples_pickscore2.pdf}
    \caption{Additional generated outcomes using prompts from Pickscore benchmark.}
    \label{fig:enter-label}
\end{figure}

\begin{figure}[h]
    \centering
    \includegraphics[width=1\linewidth]{images/appendix_more_samples_geneval2.pdf}
    \caption{Additional generated outcomes using prompts from GenEval benchmark.}
    \label{fig:enter-label}
\end{figure}

