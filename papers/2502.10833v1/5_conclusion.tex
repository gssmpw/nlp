\section{Conclusion}\label{sec:conclusion}
In this work, we revealed inherent issues of existing identifiers for LLM-based generative recommendation, \ie inadequate information, local optima, and generation inefficiency.    
We then summarized two principles for identifier design, \ie 1) integration of both CF and semantic information, and 2) order-agnostic identifiers. 
Meeting the two principles, we introduced a novel set identifier paradigm, which employs order-agnostic set identifiers to encode user history and generate the set identifier simultaneously. 
To implement this paradigm, we proposed SETRec, which uses CF and semantic tokenizers to obtain a set of CF and semantic tokens. 
To remove token dependencies, we introduced a sparse attention mask for user history encoding and a query-guided generation mechanism for simultaneous generation. 
Empirical results on four datasets across various scenarios demonstrated the effectiveness, efficiency, generalization ability, and scalability of SETRec. 

% future work
This work underscores the order agnosticism and multi-dimensional information utilization for identifier design, paving the way for numerous promising avenues for future research. 
% 探索离散化的set-based identifier,这可能和llm之前预训练的更能对齐的上
% 探索set-based identifier 在open-domain上的泛化能力,发挥它不会陷入local optima的优势
% 
1) To better align with the pre-training tasks and fully utilize the knowledge within LLMs, it is worth exploring how discrete set identifiers (\ie a set of order-agnostic discrete tokens) perform on generative recommendation. 
% 2) As empirical results demonstrate the promising scalability of SETRec on cold-start recommendation, it is worthwhile to 
2) While SETRec shows strong generalization ability in challenging scenarios such as unpopular item groups, it is worthwhile to apply SETRec for open-ended recommendation with open-domain user behaviors. 
