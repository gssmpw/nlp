\section{Related Work}\label{sec:related_work}

% \subsection{LLM-based Recommendation}
\noindent$\bullet\quad$\textbf{LLM-based Recommendation.} Harnessing LLMs for recommendations has garnered substantial attention across both academia and industry~\cite{sun2024large,zhang2024gpt4rec,fu2024iisan,zhang2024gpt4rec,shi2024large,zhao2024let,zheng2024harnessing,li2024large}. 
% 随着llm的涌现能力,extensive effort has been made to investigate how LLM can be leveraged for recommender systems. 
Existing studies on LLM-based recommendations can be grouped into two research lines. 
% Existing research can be categorized into two groups, LLM as enhancer, and LLM as recommender. 
1) LLMs for discriminative recommendation, which typically aims to leverage LLMs to assist the conventional recommender models~\cite{wang2024reinforcement,ren2024enhancing,chen2024enhancing,wu2024coral}.                     
% 其中enhancer主要关注如何,来服务于下游的传统推荐模型,比如feature augmentation, user behavior summary, etc. 
Harnessing LLMs' strong reasoning ability, this line of work usually involves LLMs in different steps of recommendation pipelines~\cite{lin2023can,zhang2024agentcf,zhang2024generative,wang2024automated} such as feature engineering~\cite{xi2024towards,ren2024representation,xi2024towards} and feature encoder~\cite{chen2024hllm}. 
% 另一个line是llm as recommender, which直接让LLM生成下一个item。我们这篇工作就fall under 这一类分支。 
2) LLMs for generative recommendation, which regards LLMs as recommenders to directly generate items~\cite{liao2024llara,kim2024large,lin2024bridging,wang2024eager,li2024large}. 
% 而在llm做generative rec分支中,其中一个关键步骤就是item tokenization, 为每个item assign an identifier for LLMs to encode user historical interactions and decode next item. 
To build LLMs for generative recommendation, a key step is item tokenization, where each item is assigned an identifier for LLMs to encode user history and generate the next item. 
% 在这个工作中我们讨论了现有的identifier方法并且揭露了现有方法存在的问题
In this work, we critically analyze the fundamental principles of identifier design to achieve effective and efficient LLM-based generative recommendation. 

% \subsection{Identifier for LLM-based Recommendation}
\vspace{2pt}
\noindent$\bullet\quad$\textbf{Identifier for LLM-based Recommendation.} 
% 现有的identifier设计主要分为两大类:
% token-sequence-based identifier。这类方法主要为了搞成LLM-compatible format~\cite{} (how2index)的形式,which aligns well with the format in pre-training tasks.
% 有一部分工作利用human vocabulary来构造identifier。这些通常是human readabl的,并且token是存在于原来预训练的词表中的。这样就能比较好的利用上pre-training学习到的知识。
% 比如BIGRec,GenRec,TransRec就用title;idgenrec用tags,M6-Rec用description。
Existing identifier designs can be broadly categorized into two types: 
1) token-sequence identifiers represent each item with a discrete token sequence. 
% human-vocabulary-based identifiers.
Under this group of work, 
prior effort has been made to utilize tokens in human vocabulary (\ie tokens that are included in the LLM vocabulary). 
% Using tokens in human vocabulary, existing work adopt item's textual meta information such as titles, descriptions, and tags, as seen in BIGRec~\cite{bao2023bi}, M6-Rec~\cite{cui2022m6}, IDGenRec~\cite{tan2024idgenrec}, and so on, aiming to utilize knowledge encoded in LLMs. 
Previous work leverages items' textual information such as titles~\cite{bao2023bi}, descriptions~\cite{cui2022m6}, and tags~\cite{tan2024idgenrec}, aiming to utilize knowledge encoded in LLMs. 
% More recently, utilizing external tokens for identifiers has attracted extensive attention due to its hierarchical characteristics. 
More recently, utilizing external tokens for identifiers has attracted extensive attention due to its potential to include hierarchical information~\cite{zhu2024cost,zheng2024adapting,wang2024content}. 
% Given the item semantic information or user-item interactions, existing work usually adopts hierarchical clustering or RQ-VAE~\cite{lee2022autoregressive} to obtain tokens in different granularity. 
% single-embedding-based identifier. 
To achieve this, existing work usually adopts hierarchical clustering or RQ-VAE~\cite{lee2022autoregressive} to obtain tokens in different granularity. 
Despite the effectiveness, token-sequence identifiers suffer from local optima issue and inference inefficiency. 
To improve efficiency, 
2) single-token identifiers are proposed to represent each item with ID or semantic embedding~\cite{li2023e4srec,wang2024rethinking}. 
Nonetheless, existing work neither has poor generalization ability nor fails to capture CF information, thus leading to suboptimal results. 
% In this work, we propose a novel set identifier paradigm, which meets two principles: 1) incorporation of CF and semantic information and 2) order-agnostic identifier. 
In this work, we propose a novel set identifier paradigm, which employs a set of order-agnostic CF and semantic tokens.  
Two concurrent studies explore set identifiers for generative retrieval~\cite{zeng2024planning,zhang2024generative}, yet they still preserve token dependencies and heavily rely on autoregressive generation. Differently, our proposed paradigm achieves simultaneous generation without token dependencies, significantly enhancing generation efficiency. 

