\documentclass[sigconf,natbib=true]{acmart}
\AtBeginDocument{%
  \providecommand\BibTeX{{%
    \normalfont B\kern-0.5em{\scshape i\kern-0.25em b}\kern-0.8em\TeX}}}

% 解决图片浮动问题
\usepackage{float}


\usepackage{booktabs}
\usepackage{array}
\usepackage{balance} 
% \usepackage{lipsum}
\usepackage{multirow}
\usepackage[normalem]{ulem}
\usepackage{color}
\definecolor{lightgray}{RGB}{215,215,215}
\definecolor{myred}{RGB}{210,109,91}
\usepackage{colortbl}  %彩色表格需要加载的宏包
\usepackage{xcolor}
\useunder{\uline}{\ul}{}
\usepackage{subfigure}
% \usepackage{subcaption}
\usepackage{algorithm}  
\usepackage{algorithmicx}  
\usepackage[noend]{algpseudocode}  
% \usepackage[noend]{algorithmic}

\usepackage{amsmath}  
\usepackage{enumitem}
\usepackage{tabularx}
\usepackage[utf8]{inputenc}
\usepackage[english]{babel}
\usepackage{amsthm}
\usepackage{bm}
\newcommand{\ie}{\emph{i.e., }}
\newcommand{\eg}{\emph{e.g., }}
\newcommand{\etal}{\emph{et al. }}
\newcommand{\st}{\emph{s.t. }}
\newcommand{\etc}{\emph{etc.}}
\newcommand{\wrt}{\emph{w.r.t. }}
\newcommand{\cf}{\emph{cf. }}
\newcommand{\aka}{\emph{a.k.a. }}
\newcommand{\todo}[1]{\textcolor{red}{todo: #1}}

% for algorithm
\newlength\myindent
\setlength\myindent{2em}
\newcommand\bindent{%
    \begingroup
    \setlength{\itemindent}{\myindent}
    \addtolength{\algorithmicindent}{\myindent}
}
\newcommand\eindent{\endgroup}

\newtheorem{theorem}{Theorem}[section]
\newtheorem{proposition}{Proposition}[section]
% \newtheorem{lemma}[theorem]{Lemma}

\floatname{algorithm}{Algorithm}
\renewcommand{\algorithmicrequire}{\textbf{Input:}} 
\renewcommand{\algorithmicensure}{\textbf{Output:}}

\clubpenalty=10000
\widowpenalty = 10000
\hyphenpenalty=2000
\tolerance=7000

% \copyrightyear{2022}
% \acmYear{2022}
% \setcopyright{acmcopyright}\acmConference[WWW '22]{Proceedings of the ACM Web Conference 2022}{April 25--29, 2022}{Virtual Event, Lyon, France}
% \acmBooktitle{Proceedings of the ACM Web Conference 2022 (WWW '22), April 25--29, 2022, Virtual Event, Lyon, France}
% \acmPrice{15.00}
% \acmDOI{10.1145/3485447.3512251}
% \acmISBN{978-1-4503-9096-5/22/04}


\begin{document}

\title{Order-agnostic Identifier for Large Language Model-based Generative Recommendation}


\author{Xinyu Lin}
\email{xylin1028@gmail.com}
\affiliation{
\institution{National University of Singapore}
\city{}
\country{Singapore}
}

\author{Haihan Shi}
\email{shh924@mail.ustc.edu.cn}
\affiliation{
\institution{University of Science and Technology of China}
\city{Hefei}
\country{China}
}

\author{Wenjie Wang}
\email{wenjiewang96@gmail.com}
\affiliation{
\institution{University of Science and Technology of China}
\city{Hefei}
\country{China}
}

\author{Fuli Feng}
\email{fulifeng93@gmail.com}
% \authornote{Corresponding author. This work is supported by the CCCD Key Lab of Ministry of Culture and Tourism.}
\affiliation{
\institution{University of Science and Technology of China}
\city{Hefei}
\country{China}
}

\author{Qifan Wang}
\email{wqfcr@meta.com}
\affiliation{
\institution{Meta AI}
\city{Menlo Park}
\country{USA}
}

\author{See-Kiong Ng}
\email{seekiong@nus.edu.sg}
\affiliation{
\institution{National University of Singapore}
\city{}
\country{Singapore}
}


\author{Tat-Seng Chua}
\email{dcscts@nus.edu.sg}
\affiliation{
\institution{National University of Singapore}
\city{}
\country{Singapore}
}
% \def\thefootnote{*}\footnotetext{Corresponding author. This work is supported by the CCCD Key Lab of Ministry of Culture and Tourism.}

% \thanks{$*$ }
\renewcommand{\shortauthors}{Xinyu Lin et al.}


\begin{abstract}
% Leveraging Large Language Models (LLMs) for generative recommendation has garnered significant research interest. 
% A pivotal step is item tokenization, which assigns identifiers to items for user history encoding and item decoding. 
% Existing identifiers broadly fall into two categories: token-sequence-based identifiers, which represent items as discrete token sequences, and single-token-based identifiers, which use either ID or semantic embedding. 
% However, token-sequence-based identifiers suffer from the local optima issue in beam search and low generation efficiency due to step-by-step generation. 
% Conversely, single-token-based identifiers neither overlook the rich semantics beneficial for long-tailed users/items, nor the collaborative information crucial to recommendation, thus leading to suboptimal performance. 
Leveraging Large Language Models (LLMs) for generative recommendation has attracted significant research interest, where item tokenization is a critical step. 
It involves assigning item identifiers for LLMs to encode user history and generate the next item. 
Existing approaches leverage either token-sequence identifiers, representing items as discrete token sequences, or single-token identifiers, using ID or semantic embeddings. Token-sequence identifiers face issues such as the local optima problem in beam search and low generation efficiency due to step-by-step generation.
In contrast, single-token identifiers fail to capture rich semantics or encode Collaborative Filtering (CF) information, resulting in suboptimal performance. 


% To address these issues, we highlight two fundamental principles for identifier design. 
% 1) Integration of both collaborative and semantic information aims to fully leverage multidimensional item information, and 
% 2) order-agnostic identifier emphasizes eliminating dependencies of tokens within items. 
% In this work, we introduce a novel set identifier paradigm for LLM-based generative recommendation, which leverages a set of order-agnostic tokens to represent each item and has simultaneous token generation ability. 
% To achieve this, 
% we propose a method called SETRec, 
% which utilizes CF and semantic tokenizers to obtain order-agnostic multidimensional tokens. 
% For next-item generation, SETRec introduces a sparse attention mask to disregard dependencies within items. 
% Moreover, to ensure LLMs generate tokens aligning with each dimension accurately, SETRec introduces a query-guided simultaneous generation mechanism to guide LLMs for token generation.
% % To enhance efficiency, we design a sparse attention mask that mitigates redundant computations during decoding. 
% Lastly, we instantiate SETRec on T5 and Qwen (from 1.5B to 7B). 
% Results on four real-world datasets demonstrate the superiority of SETRec under various settings (\eg warm- and cold-start settings, and item groups with different popularity) with significantly improved efficiency, and show promising scalability on cold-start items by expanding model sizes. 


To address these issues, we propose two fundamental principles for item identifier design:
1) integrating both CF and semantic information to fully capture multi-dimensional item information, and
2) designing order-agnostic identifiers without token dependency, mitigating the local optima issue and achieving simultaneous generation for generation efficiency. 
Accordingly, we introduce a novel \textit{set identifier} paradigm for LLM-based generative recommendation, representing each item as a set of order-agnostic tokens. 
To implement this paradigm, we propose SETRec, which leverages CF and semantic tokenizers to obtain order-agnostic multi-dimensional tokens. 
To eliminate token dependency, SETRec uses a sparse attention mask for user history encoding and a query-guided generation mechanism for simultaneous token generation.
We instantiate SETRec on T5 and Qwen (from 1.5B to 7B). 
Extensive experiments on four datasets demonstrate its effectiveness across various scenarios (\eg full ranking, warm- and cold-start ranking, and various item popularity groups). 
Moreover, results validate SETRec's superior efficiency and show promising scalability on cold-start items as model sizes increase. 



\end{abstract}

%%
%% The code below is generated by the tool at http://dl.acm.org/ccs.cfm.
%% Please copy and paste the code instead of the example below.
%%
% \vspace{-2pt}
\begin{CCSXML}
% <ccs2012>
% <concept>
% <concept_id>10002951.10003260.10003261.10003271</concept_id>
% <concept_desc>Information systems~Personalization</concept_desc>
% <concept_significance>500</concept_significance>
% </concept>
<concept>
<concept_id>10002951.10003317.10003347.10003350</concept_id>
<concept_desc>Information systems~Recommender systems</concept_desc>
<concept_significance>500</concept_significance>
</concept>
</ccs2012>
\end{CCSXML}
% \ccsdesc[500]{Information systems~Personalization}
\ccsdesc[500]{Information systems~Recommender systems}
% \vspace{-2pt}
\keywords{Item Tokenization, Set Identifier, LLM-based Recommendation}



\maketitle

% \def\thefootnote{*}\footnotetext{Corresponding author. This work is supported by the CCCD Key Lab of Ministry of Culture and Tourism.}

\section{Introduction}
Backdoor attacks pose a concealed yet profound security risk to machine learning (ML) models, for which the adversaries can inject a stealth backdoor into the model during training, enabling them to illicitly control the model's output upon encountering predefined inputs. These attacks can even occur without the knowledge of developers or end-users, thereby undermining the trust in ML systems. As ML becomes more deeply embedded in critical sectors like finance, healthcare, and autonomous driving \citep{he2016deep, liu2020computing, tournier2019mrtrix3, adjabi2020past}, the potential damage from backdoor attacks grows, underscoring the emergency for developing robust defense mechanisms against backdoor attacks.

To address the threat of backdoor attacks, researchers have developed a variety of strategies \cite{liu2018fine,wu2021adversarial,wang2019neural,zeng2022adversarial,zhu2023neural,Zhu_2023_ICCV, wei2024shared,wei2024d3}, aimed at purifying backdoors within victim models. These methods are designed to integrate with current deployment workflows seamlessly and have demonstrated significant success in mitigating the effects of backdoor triggers \cite{wubackdoorbench, wu2023defenses, wu2024backdoorbench,dunnett2024countering}.  However, most state-of-the-art (SOTA) backdoor purification methods operate under the assumption that a small clean dataset, often referred to as \textbf{auxiliary dataset}, is available for purification. Such an assumption poses practical challenges, especially in scenarios where data is scarce. To tackle this challenge, efforts have been made to reduce the size of the required auxiliary dataset~\cite{chai2022oneshot,li2023reconstructive, Zhu_2023_ICCV} and even explore dataset-free purification techniques~\cite{zheng2022data,hong2023revisiting,lin2024fusing}. Although these approaches offer some improvements, recent evaluations \cite{dunnett2024countering, wu2024backdoorbench} continue to highlight the importance of sufficient auxiliary data for achieving robust defenses against backdoor attacks.

While significant progress has been made in reducing the size of auxiliary datasets, an equally critical yet underexplored question remains: \emph{how does the nature of the auxiliary dataset affect purification effectiveness?} In  real-world  applications, auxiliary datasets can vary widely, encompassing in-distribution data, synthetic data, or external data from different sources. Understanding how each type of auxiliary dataset influences the purification effectiveness is vital for selecting or constructing the most suitable auxiliary dataset and the corresponding technique. For instance, when multiple datasets are available, understanding how different datasets contribute to purification can guide defenders in selecting or crafting the most appropriate dataset. Conversely, when only limited auxiliary data is accessible, knowing which purification technique works best under those constraints is critical. Therefore, there is an urgent need for a thorough investigation into the impact of auxiliary datasets on purification effectiveness to guide defenders in  enhancing the security of ML systems. 

In this paper, we systematically investigate the critical role of auxiliary datasets in backdoor purification, aiming to bridge the gap between idealized and practical purification scenarios.  Specifically, we first construct a diverse set of auxiliary datasets to emulate real-world conditions, as summarized in Table~\ref{overall}. These datasets include in-distribution data, synthetic data, and external data from other sources. Through an evaluation of SOTA backdoor purification methods across these datasets, we uncover several critical insights: \textbf{1)} In-distribution datasets, particularly those carefully filtered from the original training data of the victim model, effectively preserve the model’s utility for its intended tasks but may fall short in eliminating backdoors. \textbf{2)} Incorporating OOD datasets can help the model forget backdoors but also bring the risk of forgetting critical learned knowledge, significantly degrading its overall performance. Building on these findings, we propose Guided Input Calibration (GIC), a novel technique that enhances backdoor purification by adaptively transforming auxiliary data to better align with the victim model’s learned representations. By leveraging the victim model itself to guide this transformation, GIC optimizes the purification process, striking a balance between preserving model utility and mitigating backdoor threats. Extensive experiments demonstrate that GIC significantly improves the effectiveness of backdoor purification across diverse auxiliary datasets, providing a practical and robust defense solution.

Our main contributions are threefold:
\textbf{1) Impact analysis of auxiliary datasets:} We take the \textbf{first step}  in systematically investigating how different types of auxiliary datasets influence backdoor purification effectiveness. Our findings provide novel insights and serve as a foundation for future research on optimizing dataset selection and construction for enhanced backdoor defense.
%
\textbf{2) Compilation and evaluation of diverse auxiliary datasets:}  We have compiled and rigorously evaluated a diverse set of auxiliary datasets using SOTA purification methods, making our datasets and code publicly available to facilitate and support future research on practical backdoor defense strategies.
%
\textbf{3) Introduction of GIC:} We introduce GIC, the \textbf{first} dedicated solution designed to align auxiliary datasets with the model’s learned representations, significantly enhancing backdoor mitigation across various dataset types. Our approach sets a new benchmark for practical and effective backdoor defense.



\section{Preliminaries}\label{sec:task_formulation}
% In this section, we first introduce the task of LLM-based recommendation, and then retrospect existing item identifier designs and reveal their inherent critical issues. 

\vspace{2pt}
\noindent\textbf{LLM-based Generative Recommendation.} 
Harnessing LLMs' strong capabilities, LLM-based generative recommendation aims to use LLMs as recommenders to directly generate personalized recommendations. 
Formally, given the recommendation data 
$\mathcal{D}=\{\mathcal{S}_u|u\in\mathcal{U}, i\in\mathcal{I}\}$, where $\mathcal{S}_u = [i_1^{u}, i_2^{u}, \dots, i_L^{u}]$ is the user's historical interactions in chronological order and $L=|\mathcal{S}_u|$, 
% the target is to utilize a tokenizer $f(\cdot)$ and an LLM-based recommender model $\mathcal{M}(\cdot)$, to tokenize items into item identifiers $\tilde{\mathcal{I}}$, and encode the transformed user history (\ie identifier sequence) to generate recommended items. 
the target is to utilize a tokenizer $f(\cdot)$ to tokenize items into item identifiers $\tilde{\mathcal{I}}$, 
and an LLM-based recommender model $\mathcal{M}(\cdot)$ to encode the transformed user history $\bm{x}= [f(i_{1}), f(i_{2}), \dots, f(i_{L})]$ and generate next item identifier. 
% $\hat{\mathcal{I}}=\{\hat{i}\}$. 

% item tokenization起到了至关重要的作用, bridging the xxx

\vspace{2pt}
Bridging the language space and the item space, item identifier is a fundamental component for LLMs to encode user history and generate items. 
Existing identifiers can be divided into two groups:

% 先前的工作设计了不同的f来得到token sequence,为了更好的把pretraining的语言模型adapt到文本形式的推荐任务上。
\vspace{2pt}
\noindent$\bullet\quad$\textbf{\textit{Token-sequence identifier}} assigns each item with a discrete token sequence, \ie $\tilde{i} = [z_1, z_2, \dots, z_N]$, where $z_i$ is the discrete token. 
Given the user history $\mathcal{S}_u$, it is transformed to an identifier sequence $\bm{x}= [\tilde{i}_1, \tilde{i}_2, \dots, \tilde{i}_L]$, which is then encoded by LLMs to generate the next identifier via autoregressive generation: 
\begin{equation}
% \left\{
\begin{aligned}
    &\hat{y}_t = \mathop{\arg\max}_{v\in\mathcal{V}} \mathcal{M}(v|\hat{y}_{<t},\bm{x}), \\
\end{aligned}
% \right.
\end{equation} 
where $\mathcal{V}$ is the LLM vocabulary. 
Despite the effectiveness, generating token sequences would result in the local optima issue and inference inefficiency. 
As shown in Figure~\ref{fig:beam_size}, continuously increasing the beam size slightly improves recommendation accuracy, but remains inferior to globally optimal results. 
Worse still, the token-by-token generation requires multiple serial LLM calls, which significantly lowers the inference speed and hinders real-world applications.

\begin{figure}[t]
% \vspace{-0.2cm}
\setlength{\abovecaptionskip}{0.0cm}
\setlength{\belowcaptionskip}{-0.3cm}
\centering
\includegraphics[scale=0.4]{figures/beam_size.pdf}
\caption{Performance comparison between beam search and global search of LETTER on Toys. The global search is implemented by computing sequence probability for every item and ranking them based on the probabilities. }
\label{fig:beam_size}
\end{figure}


% of LLM-based generative recommendation. 



% 2. 而single embedding则是利用cf recommender 或者 feature extractor之类的 -> 获取到每个item的
\vspace{2pt}
\noindent$\bullet\quad$\textbf{\textit{Single-token identifier}} assigns each item with an ID or semantic embedding, \ie $\tilde{i}=\bm{z}$, 
% Existing work typically leverages a conventional CF recommender model as the tokenizer to obtain ID embedding 
which is usually obtained by a conventional CF recommender model (\eg SASRec~\cite{kang2018self})
or a pre-trained semantic extractor (\eg SentenceT5~\cite{ni2021sentence}). 
% Escaping from token-by-token generation, this line of work facilitates efficient item generation~\cite{wang2024rethinking}. 
Given the transformed user history $\bm{x}= [f(i_{1}), f(i_{2}), \dots, f(i_{L})]$, it first generates the embedding:
\begin{equation}
\begin{aligned}
% &x= [f(i_{1}), f(i_{2}), \dots, f(i_{L})] \\
&\hat{i} = \text{LLM\_Layers}(\bm{x}), \\
\end{aligned}
\end{equation}
where $\text{LLM\_Layers}(\cdot)$ is the attention layers from the LLM $\mathcal{M}(\cdot)$. 
Based on the generated item embedding $\hat{i}$, 
an additional grounding head is added on top of the LLM layers to obtain the scores for all items for ranking. 
% 虽然这类方法不会有beam search的问题,因为它只有一个token。
% 但是 - 只有单一维度的信息。只有CF会导致泛化能力弱,只有sem会导致与交互信息不对齐。
% 2) single-embedding-based(只有cf的难以泛化到只有sem的,而只有sem又可能和cf有所冲突)(-这个可以考虑加一下图)
Although it improves inference efficiency by bypassing the token-by-token autoregressive generation, 
representing items with a single ID embedding struggles with items with fewer interactions while a single semantic embedding overlooks the crucial CF information, thus leading to suboptimal results. 
% As shown in Figure~\ref{}, items with similar semantics might have different interactions. 

% \noindent$\bullet\quad${\textbf{\textit{Set identifier}}}. 
Based on the above insights, we
summarize two fundamental principles for identifier designs: 
1) integration of semantic and CF information, to leverage rich multi-dimensional item information, and  
2) order-agnostic identifier, to eliminate the unnecessary dependencies between tokens associated with an identifier, which can alleviate the local optima issue and improve generation efficiency.  
In this light, we introduce a novel set identifier paradigm, which employs a set of order-agnostic tokens to represent multi-dimensional item information. 
% facilitating globally optimal results and achieving simultaneous generation to boost efficiency. 
% It encodes the user history (\ie identifier sequence) and decodes the tokens in set identifiers simultaneously. 
% which removes the dependency from token to token within identifiers, thus alleviating the local optima issue and boost the efficiency. 


\section{SETRec}\label{sec:method}
% To meet the two principles for identifier design, we propose a set-based-identifier for effective and efficient LLM-based generative recommendation. 
% In this section, we first elaborate the design of set-based item tokenization, and then delve into the decoding of set-based item identifier. 
% Lastly, we detail the instantiation of our proposed SETRec, including training and inference.  
To implement the set identifier paradigm, we propose a framework called SETRec for effective and efficient LLM-based generative recommendation, including order-agnostic item tokenization and simultaneous item generation as illustrated in Figure~\ref{fig:method_tokenizer}.  
% This section first elaborates on order-agnostic item tokenization and then delves into the simultaneous item generation. 
% Lastly, we detail the instantiation of our proposed SETRec, including training and inference.  


\begin{figure}[t]
% \vspace{-0.2cm}
\setlength{\abovecaptionskip}{0.02cm}
\setlength{\belowcaptionskip}{-0.3cm}
\centering
\includegraphics[scale=0.88]{figures/method_tokenizer.pdf}
\caption{(a) demonstrates SETRec framework, including order-agnostic item tokenization, and simultaneous item generation. The dependencies within identifiers and query vectors are eliminated by the sparse attention mask (see Figure~\ref{fig:sparse_attn} for details). (b) illustrates order-agnostic item tokenization via CF and semantic tokenizers.}
\label{fig:method_tokenizer}
\end{figure}


\subsection{Order-agnostic Item Tokenization} 
% In SETRec, we integrate both semantic and CF information, 
% To assign each item with a set of tokens representing both CF and semantic information
% In SETRec, we introduce a set-based item tokenization strategy, which assigns each item with a set of tokens representing both CF and semantic information. 
Meeting the two principles, SETRec leverages a CF and a semantic tokenizer to endow multi-dimensional information into a set of order-agnostic continuous tokens\footnote{We do not use discrete tokens in SETRec because discretization inevitably suffers from information loss~\cite{lazebnik2008supervised}, potentially leading to suboptimal results.} as illustrated in Figure~\ref{fig:method_tokenizer}(b). 
% To integrate semantic and CF information into identifiers, 
% a possible solution is using a set of discrete tokens. 
% 为什么用连续的不用离散的?
% Using a set of discrete tokens is a possible solution. 
% Therefore, we alternatively adopt a set of continuous tokens for SETRec, aiming to harness the item information as much as possible. 
% 我们通过cf tokenizer和semantic tokenizer获取 CF token 和 semantic token
% To endow multi-dimensional features into the token set, we leverage a CF tokenizer and a semantic tokenizer to obtain CF tokens and semantic tokens, respectively. 

% \noindent$\bullet\quad$\textbf{Multi-dimensional Tokenizer}
\noindent$\bullet\quad$\textbf{CF Tokenizer.}
% CF tokenizer 
As shown in Figure~\ref{fig:method_tokenizer}(b), we utilize a pre-trained conventional recommender model (\eg SASRec~\cite{lazebnik2008supervised}) with a linear projection layer to obtain item CF embedding $\bm{z}_{\text{CF}}\in\mathbb{R}^{d}$, where $d$ is the hidden dimension of LLMs. 
Incorporating CF embeddings encourages LLM-based recommenders to facilitate recommendations for users/items with rich interactions. 
% The CF tokenizer ensures that items with similar interactions have similar identifiers. 
% facilitating recommendations for users and items with rich interactions
% As such, representing item with CF embeddings encourages LLM-based recommender to leverage collaborative information that is crucial for recommendation. 
% merely using CF embedding would struggle on cold-start items, thus leading to poor generalization. 

\noindent$\bullet\quad$\textbf{Semantic Tokenizer.} 
To fully utilize rich item semantic information, SETRec introduces a semantic tokenizer to obtain a set of semantic embeddings. 
Specifically, 
given the item semantic information $c$ such as title and categories, we first extract the item semantic representations $\bm{s}$ with a pre-trained semantic extractor (\eg SentenceT5~\cite{ni2021sentence}). 

To obtain the semantic embeddings, a straightforward approach is to compress semantic representation $\bm{s}$ into a single latent semantic embedding. 
% Nonetheless, single embedding for semantic tokens might suffer from the embedding collapse issue~\cite{}, potentially undermining the rich semantic content that distinguishes between items. 
Nonetheless, compressing multi-dimensional semantic information (\eg ``brand'' and ``price'') might suffer from the embedding collapse issue~\cite{guoembedding,pan2024ads}, potentially undermining the rich semantic content that distinguishes between items. 
To prevent this issue, as depicted in Figure~\ref{fig:method_tokenizer}(b), we tokenize each item into $N$ order-agnostic semantic embeddings via an AE: 
\begin{equation}
% \left\{
\begin{aligned}
&\bm{z} = \text{Encoder}(\bm{s}), 
\end{aligned}
% \right.
\end{equation}
where 
$\bm{z}=[\bm{z}_{S_{1}}, \bm{z}_{S_{2}},\dots,\bm{z}_{S_{N}}]\in\mathbb{R}^{Nd}$ denotes the concatenated semantic embeddings representing different latent semantic dimensions, and ${z}_{S_{n}}\in\mathbb{R}^{d}$ is the $n$-th semantic embedding. 
% where $d$ is the latent dimension of LLMs. 
% $N$ is the number of semantic embeddings, and $n\in[1,\dots, N]$. 
Notably, we utilize a unified AE instead of multiple independent AEs for two considerations: 
1) employing a single AE reduces the parameters with an approximate ratio of $\frac{1}{N}$, which is highly practical; 
2) alleviating the training instability that might be caused by multiple encoders' training~\cite{tang2023improving}. 
In addition, to encourage the semantic embeddings to preserve useful information as much as possible, 
a reconstruction loss is used to train the semantic tokenizer: 
\begin{equation}
    \mathcal{L}_{AE} = \|\bm{s}-\hat{\bm{s}}\|_2^{2}, 
\end{equation}
where 
$\hat{\bm{s}} = \text{Decoder}(\bm{z})$ is the reconstructed semantic representation. 

\noindent$\bullet\quad$\textbf{Token Corpus.} 
% 最终基于两个tokenizer,一个item的set-based identifier是什么,怎么表示的。
Based on the CF and the semantic tokenizer, we can obtain the set identifier for each item
$\tilde{i} = \{\bm{z}_{\text{CF}},\bm{z}_{S_{1}},\dots,\bm{z}_{S_{N}}\}$, consisting of a CF embedding and $N$ semantic embeddings. 
% 基于这两个之后,加一个token corpus。
We then can collect tokens from all items and obtain the token corpus for each information dimension, \ie $\mathcal{Z}_\text{CF}, \mathcal{Z}_{S_1}, \dots, \mathcal{Z}_{S_N}$. 
% where $\mathcal{Z}_k=\{\bm{z}_{k,j}|j\in\{1,\dots, |I|\}\}$. 
The collected token corpus is used as the grounding head for effective item grounding (\cf Section~\ref{sec:query-based_decoding}). 

\subsection{Simultaneous Item Generation} 
% To achieve item generation with set-based identifier, the key lies in generating the set 
% Given the user's historical interactions, SETRec gene
% Based on the set-based item tokenization, the user 
% without token dependencies, 
To efficiently and effectively generate set identifiers, it is crucial for SETRec to 
% the key lies in enabling simultaneous token generation in multiple dimensions.  
% To achieve this, 
1) guide LLMs to distinguish different dimensions and generate tokens aligning well with each dimension simultaneously (Section~\ref{sec:query-based_decoding}); 
2) ground the generated token set to existing items effectively (Section~\ref{sec:query-based_decoding}); 
3) eliminate the unnecessary dependencies introduced in user history (Section~\ref{sec:sparse_attention_mask}); 

\subsubsection{\textbf{Query-guided Generation}.}\label{sec:query-based_decoding}
% \noindent$\bullet\quad$\textbf{Query-based Simultaneous Decoding.}  
As shown in Figure~\ref{fig:method_tokenizer}(a), to guide LLMs to generate tokens that align well with the information dimensions, 
we introduce a set of learnable query vectors $\bm{q}\in\mathbb{R}^{d}$, where $d$ is the latent dimension of the LLMs, to guide the LLMs to distinguish between information dimensions (\eg CF and semantic) for token generation. 
Formally, the generated token $\hat{\bm{z}}_k$ for each dimension $k\in\{\text{CF}, S_1, S_2, \dots, S_N\}$ is obtained via:
\begin{equation}\label{eqn:query_decoding}
\left\{
\begin{aligned}
    &\bm{x} = [\{\bm{z}_{\text{CF}}, \bm{z}_{S_1}, 
     \dots, \bm{z}_{S_N}\}^{1}, \dots, \{\bm{z}_{\text{CF}}, \bm{z}_{S_1}, \dots,\bm{z}_{S_N}\}^{L}], \\
    &\hat{\bm{z}}_{k} = \text{LLM\_Layers}(\bm{x}, \bm{q}_k),
\end{aligned}
\right.
\end{equation}
where $\bm{q}_k$ is the learnable query vector to guide LLM generation for the information dimension $k$. 
Based on Eq. (\ref{eqn:query_decoding}), we can collect the generated token for all dimensions and obtain the generated set identifier  $\hat{i}=\{\hat{\bm{z}}_{\text{CF}}, \hat{\bm{z}}_{S_1}, \dots, 
\hat{\bm{z}}_{S_N}
\}$. 

\vspace{2pt}
\textbf{\textit{Token Generation Optimization.}} 
% 1. 为了教会LLM在生成的时候能做准确的item reocmmendation,我们鼓励llm能够生成target item的token。
To achieve accurate item recommendations, 
% we train SETRec to generate the set identifier of the target item. 
% 2. 介绍我们怎么做的 - specifically, 对于每一个feature 维度,我们encourage生成的token要和target token尽可能的相似. 
% Specifically, 
we encourage the generated token to align with the target token for every dimension: 
% 3. 公式 - xxx 介绍每个变量是什么
\begin{equation}\label{eqn:loss_gen}
    \mathcal{L}_{\text{Gen}} = - \frac{1}{|\mathcal{D}|} \sum_{\mathcal{D}} \sum_{k\in\mathcal{F}}\frac{\exp (sim(\hat{\bm{z}}_k,\bm{z}_{k}))}{\sum_{\bm{z}\in\mathcal{Z}_k}\exp(sim(\hat{\bm{z}}_k,\bm{z}))},
\end{equation}
where $\mathcal{F}=\{\text{CF}, S_1, \dots, S_N\}$, $sim(\cdot)$ is the similarity function (\eg inner product), and $\bm{z}_k$ is the target item token for the information dimension $k$. 
% 4. intuitive解释这样能够拉近正确的距离,拉远不相似的距离,鼓励也能鼓励item之间有所区分。
Intuitively, Eq. (\ref{eqn:loss_gen}) pushes the generated embedding closer to the target embedding and pulls away from other embeddings within the specific information dimension.  
% Besides, by pulling embeddings away from each other is also encouraged to distinguish between items for every feature dimensions, which further improves the embedding diversity. 


% 标题这里考虑换成两个黑体,token generatinon training, 和 token grounding for inference
% \noindent$\bullet\quad$\textbf{Token Set Grounding.}
\vspace{2pt}
\textit{\textbf{Token Generation Grounding.}} 
% 提出需要grounding的问题 - 解释在inference的时候生成token,得到上一步的set之后,直接使用是无法match到现有item上的。
Based on generated tokens obtained via Eq. (\ref{eqn:query_decoding}), the next step is to ground them to the existing items. 
% 顺应解释按照token去ground仍然会ooc - 一种简单的方法是用l2distance等等去match到token。但即便如此,match后得到的set,仍然很可能会out-of-corpus. 
% We introduce a two-step grounding strategy, which utilizes a 
However, this can be challenging since the possible combinations of the tokens from different information dimensions are much larger than the existing item corpus, \ie $\prod_{k\in\mathcal{F}}|\mathcal{Z}_k|\gg |\mathcal{I}|$. 
% 提出我们的解决方法:因此我们考虑借助token corpus来获取所有item上的logits. 
To solve this issue, we introduce a token set grounding strategy, which leverages the token corpus as grounding heads to obtain the item score. 
% specifically,(1. 根据token来获取每个feature维度的score代表item的score,2. 合并多维度之间的结果,获得全局的score)
% Specifically, we first obtain the item scores for each dimension $k$: 
% \begin{equation}\label{eqn:single_logits}
%     s_k=W_{\text{proj}}^{k}\bm{\hat{z}}_k,
% \end{equation}
% where $W_{\text{proj}}^{k}\in\mathbb{R}^{|I|\times d}$ is adopted from the token corpus $\mathcal{Z}_k$. 
% We then obtain the final matching scores with a linear combination of CF dimension and semantic dimension via 
% \begin{equation}\label{eqn:final_logits}
%     s = (1-\beta) s_\text{CF} + \beta \sum_{i=1}^{N} s_i,
% \end{equation}
% where $\beta$ is a hyper-parameter to balance the strength between CF and semantic dimensions. 
Formally, we have 
\begin{equation}\label{eqn:single_logits}\small
\left\{
\begin{aligned}    &s_k=W_{k}\bm{\hat{z}}_k, \\
&s = (1-\beta) s_\text{CF} + \beta \sum\nolimits_{{k\in \mathcal{F}\setminus{\text{CF}}}} s_{k}, \\
\end{aligned}
\right.
\end{equation}
% Specifically, we first obtain the item scores for each dimension $k$: 
% \begin{equation}\label{eqn:single_logits}
%     s_k=W_{\text{proj}}^{k}\bm{\hat{z}}_k,
% \end{equation}
where $W_{k}\in\mathbb{R}^{|I|\times d}$ is adopted from the token corpus $\mathcal{Z}_k$. 
The final item scores are obtained via a linear combination of CF and semantic dimensions, where $\beta$ is a hyper-parameter to balance the strength between CF and semantic dimensions. 
It is highlighted that the grounding heads for semantic dimensions are extendable to new items, leading to strong generalization ability (\cf Section~\ref{sec:overall_performance}). 


\begin{figure}[t]
% \vspace{-0.2cm}
\setlength{\abovecaptionskip}{0.02cm}
\setlength{\belowcaptionskip}{-0.3cm}
\centering
\includegraphics[scale=1.1]{figures/sparse_attn.pdf}
\setlength{\fboxrule}{1pt}
\caption{Comparison between original attention and sparse attention ($N=1$). The sparse attention 1) eliminates the dependency over other tokens within the same item ({\color{myred}\fbox{\phantom{\rule{0.06cm}{0.06cm}}}}), and 2) boosts the efficiency with the flattened input, \ie query vectors are in the same sequence.}
\label{fig:sparse_attn}
\end{figure}

\vspace{2pt}
\subsubsection{\textbf{Sparse Attention Mask.}}\label{sec:sparse_attention_mask}
% \noindent$\bullet\quad$\textbf{Sparse Attention Mask.} 
% 提出问题 - 可能需要画一个图说明一下什么意思 。
% 两个问题 1.第一个问题是query在生成的时候batch会很浪费计算资源
% Based on the simultaneous decoding, each token can be generated independently at a single LLM call via batch decoding (Eq. (\ref{eqn:query_decoding})). 
% Nonetheless, this will cause repetitive self-attention computations for the user's historical interactions $\bm{x}$, thus leading to inefficiency. 
% % 1.第二个是transformer仍然需要序列送进去,如何瓦解item内部token的序列关系?
% Moreover, while simultaneous decoding bypasses the sequential generation of item identifier, the user's historical interactions are still sequentially encoded via causal attention mask\footnote{Most of LLMs adopt decoder-only architecture, which employs the causal attention mask. For the encoder-deocder LLMs, the bi-directional attention mask will not introduce noisy dependencies.}, potentially introducing dependencies between tokens within each identifier. 

% \vspace{2pt}
% To address these two challenges, we introduce a sparse attention mask, as illustrated in Figure~\ref{fig:sparse_attn}. 
% % 我们提出一个sparse attention mask 同时来解决这两个问题,获得efficient的set-based identifier generation的方法。
% Specifically, for the user's historical interactions, tokens associated with an identifier are treated as independent from each other (\eg CF embedding cannot attend to semantic embeddings for item 1). 
% However, these tokens can still attend to all tokens in previously interacted items (\eg a fully attended mask is applied to item 1 when calculating self-attention for tokens in item 2).
% This sparse attention mask ensures the order invariance of our proposed set-based identifier.
% %Meanwhile, they can attend all tokens in previously interacted items (\eg fully attended mask on item 1 when calculating self-attention for tokens of item 2). 
% %Based on the sparse mask, we can ensure the order invariance of our proposed set-based identifier. 

% % todo: 加一个证明,可以放到appendix里面

% 1. 先讲implicit order的事情,呼应intro以及方法部分。接着介绍我们的方法。
% 介绍完之后,说这样的sparse attention不仅移除了dependency,还能够有效的boost inference efficiency。
% 加一句解释说,eq 5 支持independent generation,但是利用original attention的话会重复计算。使用我们的sparse能够显著降低计算
% 跟上complexity analysis
% 1.第二个是transformer仍然需要序列送进去,如何瓦解item内部token的序列关系?
While simultaneous generation bypasses the sequential generation of item identifier, the flattened user's historical interactions are still sequentially encoded, inevitably introducing order information of tokens within each identifier (Figure~\ref{fig:sparse_attn}(a)). 
To combat this issue, we introduce a sparse attention mask as illustrated in Figure~\ref{fig:sparse_attn}(b). 
% 我们提出一个sparse attention mask 同时来解决这两个问题,获得efficient的set-based identifier generation的方法。
Specifically, for the user's historical interactions, tokens associated with an identifier are treated as independent from each other (\eg CF embedding cannot attend to semantic embeddings). 
However, these tokens can still attend to all tokens in previously interacted items (\eg a fully attended mask is applied to football when calculating self-attention for tokens in basketball).
Therefore, the sparse attention mask ensures the order agnosticism of the set identifier. 
% In addition, it also improves inference efficiency by reducing the duplicate self-attention calculations for the user's historical interactions $\bm{x}$ via original attention mask (Figure~\ref{fig:sparse_attn}(a)). 
% Based on the simultaneous decoding, each token can be generated independently at a single LLM call via batch decoding (Eq. (\ref{eqn:query_decoding})). 
% Nonetheless, this will cause repetitive self-attention computations for the user's historical interactions $\bm{x}$, thus leading to inefficiency. 

\vspace{2pt}
\noindent$\bullet\quad$\textbf{Time Complexity Analysis.} 
Moreover, the sparse attention mask can improve the generation efficiency by reducing the duplicate computations of the shared prefix via original attention mask (Figure~\ref{fig:sparse_attn}(a)). 
With $M$ information dimensions and $L$ historically interacted items, the time complexity for batch generation with the original attention mask is $M^3L^2d$. 
Remarkably, based on the flattened input with our proposed sparse attention mask, the time complexity reduces to $M^2L^2d$. 

% \noindent$\bullet\quad$\textbf{Identifier Order Invariance.} 

% The sparse attention mask not only accelerate the simultaneous decoding, 
% but also ensures the order invariance

% \noindent$\bullet\quad$\textbf{Order Invariance Analysis}

% 这里主要主张两件事 1. 提高效率。2. 这样子弄完了之后,是order-agnostic的。
% (可能写一个order-agnostic的分析) - 得参考一下这种equivariance怎么写



\subsection{Instantiation}
% 把SETRec instantiate到LLM上,我们通过一个overall loss来训练LLM和tokenizer:
To instantiate SETRec on LLMs, we optimize the CF and semantic tokenizers, learnable query vectors, and LLMs by minimizing: 
% 公式 - overall loss = L_AE + L_Gen
\begin{equation}\label{eqn:overall_loss}
    \mathcal{L} = \mathcal{L}_{\text{Gen}} + \alpha \mathcal{L}_{\text{AE}},
\end{equation}
where $\alpha$ is a hyper-parameter to control the strength of the tokenizer training. 
% 然后是inference的时候。我们利用tokenizer将所有item 进行tokenize,然后
During inference, SETRec first tokenizes all items into set identifiers and obtain token corpus $\mathcal{Z}$ for each information dimension. 
Then, to recommend item, SETRec transforms user history into identifier sequence and performs query-guided simultaneous generation with sparse attention mask via Eq. (\ref{eqn:query_decoding}) to generate tokens for all information dimensions.  
Finally, SETRec leverages token corpus as extendable grounding heads to ground the generated token set to the valid items via Eq. (\ref{eqn:single_logits}). 

% In this work, we instantiate SETRec on two representative generative language models with different architectures, \ie T5 and Qwen (refer to Section~\ref{sec:experiment}). 

\section{Experiments}\label{sec_exp}
\vspace{-0.2cm}
Our experiments investigate three key research questions:

\noindent\emph{Q1: Method Effectiveness.} How does our approach enhance performance across both in-domain and out-of-domain mathematical benchmarks compared to existing math LLMs?

\noindent\emph{Q2: Baseline Comparisons.} How does our method compare to standard RL and SFT baselines in terms of training efficiency and exploration patterns?

\noindent\emph{Q3: AutoCode Analysis.} What strategies does the model learn for code integration, and how do these strategies contribute to performance gains?

\noindent\textbf{Datasets and Benchmarks.} Our method only requires a query set for training. We collect public available queries from MATH~\citep{math} and Numina~\cite{numina}, and sample \(7K\) queries based on difficulties. We upload the collected data to the annonymous repo. For evaluation, we employ: GSM8k~\citep{gsm8k}, MATH500~\citep{math}, GaokaoMath2023~\citep{mario}, OlympiadBench~\citep{olympiad}, the American Invitational Mathematics Examination (AIME24), and the American
Mathematics Competitions (AMC23). This benchmark suite spans elementary to Olympiad-level mathematics. We adopt Pass@1 accuracy~\citep{pass1, dsr1} as our primary metric, using evaluation scripts from DeepseekMath~\citep{dsmath} and Qwen2Math~\citep{yang2024qwen2}. For competition-level benchmarks (AIME/AMC), we use 64 samples with temperature 0.6 following Deepseek R1 protocols.

% Please add the following required packages to your document preamble:

% Beamer presentation requires \usepackage{colortbl} instead of \usepackage[table,xcdraw]{xcolor}
\begin{table*}[t]
\centering
\caption{Main Results. Eurus-2-7B-PRIME demonstrates the best reasoning ability.}
\label{tab:main_results}
\resizebox{\textwidth}{!}{
\begin{tabular}{lcccccc}
\toprule
\textbf{Model}                     & \textbf{AIME 2024}                           & \textbf{MATH-500} & \textbf{AMC}          & \textbf{Minerva Math} & \textbf{OlympiadBench} & \textbf{Avg.}          \\ \midrule
\textbf{GPT-4o}                    & 9.3                                          & 76.4              & 45.8                  & 36.8                  & \textbf{43.3}          & 43.3                   \\
\textbf{Llama-3.1-70B-Instruct}    & 16.7                                         & 64.6              & 30.1                  & 35.3                  & 31.9                   & 35.7                   \\
\textbf{Qwen-2.5-Math-7B-Instruct} & 13.3                                         & \textbf{79.8}     & 50.6                  & 34.6                  & 40.7                   & 43.8                   \\
\textbf{Eurus-2-7B-SFT}            & 3.3                                          & 65.1              & 30.1                  & 32.7                  & 29.8                   & 32.2                   \\
\textbf{Eurus-2-7B-PRIME}          & \textbf{26.7 {\color[HTML]{009901} (+23.3)}} & 79.2 {\color[HTML]{009901}(+14.1)}      & \textbf{57.8 {\color[HTML]{009901}(+27.7)}} & \textbf{38.6 {\color[HTML]{009901}(+5.9)}}  & 42.1 {\color[HTML]{009901}(+12.3) }          & \textbf{48.9 {\color[HTML]{009901}(+ 16.7)}} \\ \bottomrule
\end{tabular}
}
\end{table*}
\noindent\textbf{Baselines and Implementation.} 
We compare against three model categories: \begin{itemize}[leftmargin=0.5cm,itemsep=0pt,parsep=0pt]
\item Proprietary models: o1~\cite{o1}, GPT-4~\citep{gpt4} and Claude~\citep{claude}
\item Recent math-specialized LMs: NuminaMath~\citep{numina}, Mathstral~\citep{mathstral}, Mammoth~\citep{mammoth}, ToRA~\citep{tora}, DartMath~\cite{tong2024dartmath}. We do not compare with models that rely on test-time scaling, such as MCTS or long CoT. 
\item Foundation models enhanced with our method: Qwen2Math~\citep{yang2024qwen2}, DeepseekMath~\citep{dsmath} and Qwen-2.5~\cite{qwen25}.
\end{itemize}

Our implementation uses \( K = 8 \) rollouts per query (temperature=1.0, top-p=0.9). Training completes in about 10 hours on \(8\times\) A100 (80GB) GPUs across three epochs of 7K queries. We release code, models and data via an \href{https://anonymous.4open.science/r/AnnonySubmission-0C62}{anonymous repository}.

% \vspace{-0.1cm}
\subsection{Main Results}\label{sec_main}
Notably, we observe a minimum performance gain of 11\% on the MATH500 benchmark, escalating to an impressive 9.4\% absolute improvement on the highly challenging AIME benchmark.  Across in-domain benchmarks, our method yields an average improvement of 8.9\%, and for out-of-domain benchmarks, we achieve a substantial average gain of 6.98\%. These results  validate the effectiveness of our approach across model families and problem difficulty levels.  

\subsection{Ablation Study}\label{sec_ablation}
We conduct three primary analyses: (a) comparison with standard RL and SFT baselines to validate our method's effectiveness in facilitating exploration, (b) visualization of exploration patterns to reveal limitations in the standard RL paradigam, and (c) behavioral analysis of code integration strategies. These analyses collectively demonstrate our method's benefits in facilitating guided exploration and explains how it improves performance.

\begin{figure*}[t]
    \centering
    \includegraphics[width=0.95\linewidth]{figs/rl_curves.pdf}
    \caption{ \small \textbf{Training Efficiency and Convergence.} We benchmark the learning dynamics of our approach against three two training paradigms: supervised fine-tuning and reinforcement learning (RL). The Pass@1 accuracy is evaluated on an held-out dev-set. We use Qwen-2.5-Base as the base model. SFT is conducted using collected public data~\cite{openmath, mammoth}. The dashed lines indicate asymptotic performance. }\label{fig_training_efficiency}
    % \begin{minipage}{0.47\textwidth}
    %     \centering
    %     \begin{subfigure}[b]{1.0\textwidth}
    %         \centering
    %         \includegraphics[width=0.95\linewidth]{figs/abl_qwen_curve.pdf}
    %         % \caption{Top right image}
    %     \end{subfigure}
    %     % \vskip -0.3\baselineskip % Add vertical space between subfigures
    %     \begin{subfigure}[b]{1.0\textwidth}
    %         \centering
    %         \includegraphics[width=1.\linewidth]{figs/abl_deepseek_curve.pdf}
    %         % \caption{Bottom right image}
    %     \end{subfigure}
    %     \caption{ \small \textbf{Performance Convergence. } Experiments are conducted based on Qwen2Math (Top) and DeepseekMath (Bottom). AutoCode achieves higher accuracy with sustained improvement, while standard RL converge to sub-optimal solutions. }\label{fig_training_effici}
    % \end{minipage}
    % \hfill
    % \begin{minipage}{0.47\textwidth}
    %     \centering
    %     % \vspace{-0.3cm}
    %     % \hspace{-1.6cm}
    %     \includegraphics[width=1.\linewidth]{figs/abl_strategies.pdf}
    %     \caption{\small \textbf{Analysis of the Learned Strategies.} Correct Responses are classified based on their alignment to the oracle selection, namely, \emph{StrictAlign}, \emph{AllowCode} and \emph{MisAlign}. We show how different categories of alignment contribute to the accuracy in the stacked bars, and include the overall StrictAlign rate in the separate orange bar.} \label{fig_learned_strategies}
    % \end{minipage}
% \vspace{-0.3cm}
\end{figure*}
\noindent\textbf{Training Efficiency.} We evaluated the learning dynamics of our approach in direct comparison to three established training paradigms:
\begin{itemize}[leftmargin=0.5cm,itemsep=0pt,parsep=0pt]
\item \emph{Base+RL}:  On-policy Reinforcement Learning (RL) initialized from a base model without Supervised Fine-Tuning (SFT). This follows the methodology of DeepSeek R1, designed to isolate and assess the pure effects of RL training.
\item \emph{SFT}: Supervised Fine-Tuning, the prevailing training paradigm widely adopted in current tool-integrated math Language Models (LMs).
\item \emph{SFT+RL}: Standard RL applied after SFT, serving as a conventional baseline for evaluating our EM-based RL method.
\end{itemize}

From the figure, we make the following key observations: 


\begin{itemize}[leftmargin=0.5cm,itemsep=0pt,parsep=0pt]
   \item  While Reinforcement Learning directly from the base model (\emph{Base+RL}) exhibits consistent performance improvement, its training efficiency is lower than training paradigms incorporating SFT.  In addition, the model rarely explores code-integrated solutions, with the code invocation rate below 5\%. This strongly suggest that \emph{reinforcement learning tool-usage behavior from scratch is inherently inefficient}.
    \item SFT effectively provides a strong initialization point, but \emph{SFT alone exhibits limited asymptotic performance}. This suggests that SFT lacks the capacity to adapt and optimize beyond the scope of the expert demonstrations, thereby limiting further improvement. 
    \item Standard RL applied after SFT shows initial further improvement but subsequently plateaus, \emph{even after an extended training stage}.  This suggests \emph{the exploration-exploitation dilemma when applying RL for LLM post-training}: standard RL with vanilla rollout exploration tends to exploit local optima and insufficiently explores the combinatorial code-integrated trajectories.
\end{itemize}

To further substantiate the exploration limitations inherent in the conventional \emph{SFT+RL} paradigm, we present a visualization of the exploration patterns. We partitioned the model-generated responses during self-exploration into three distinct training phases and analyzed the statistical distribution of code invocation rates across queries as the model's policy evolved throughout training. As depicted in Figure~\ref{fig_visualize_explore}, the distribution of code invocation progressively concentrates towards the extremes – either minimal or maximal code use – indicating the model's growing tendency to exploit its local policy neighborhood. This exploitation manifests as a focus on refining established code-triggering decisions, rather than engaging in broader exploration of alternative approaches.



\begin{figure}[t]
    \centering % Center the figure
    \resizebox{1.\linewidth}{!}{\includegraphics[width=\linewidth]{figs/visualize_explore.pdf} }% Include the figure
    \vspace{-0.2cm}
    \caption{\small \textbf{Visualization of Exploration in the SFT+RL paradigm.} \small The distribution of code invocation rates \emph{across queries} to visualize policy's exploration of code-integrated trajectories. Without external guidance, LLM tends to exploit its local policy neighborhood, concentrating code usage toward extremes as training phase evolves. } \vspace{-0.2cm}
    \label{fig_visualize_explore} 
\end{figure}
These empirical observations lend strong support to our assertion that standard RL methods are susceptible to premature exploitation of the local policy space when learning AutoCode strategies. In sharp contrast, our proposed EM method facilitates a more guided exploration by sub-sampling trajectories according to the reference strategy (Sec.~\ref{sec_impl}). This enables continuous performance (evidenced in Sec.~\ref{sec_main}) and mitigating the risk of converging to suboptimal local optima (Fig.~\ref{fig_training_efficiency}).



\begin{figure}[t]
    \centering % Center the figure
    \resizebox{1.\linewidth}{!}{\includegraphics[width=\linewidth]{figs/learned_behavior.pdf}} % Include the figure
    \caption{\small \textbf{Analysis of AutoCode Strategies. }\small We compare AutoCode performance against scenarios where models explicitly prompted to utilize code or CoT, and consider the union of solved queries as the bound for AutoCode performance. Existing models show inferior AutoCode performance than explicit instructed, with their AutoCode strategies close to random (50\%). Our approach consistently improves AutoCode performance, with AutoCode selection accuracy near 90\%.  } 
    \label{fig_learned_behavior} 
\end{figure}
\noindent\textbf{Analysis on Code Integration Behaviors.}
We investigated the properties of the learned code integration strategies to gain deeper insights into the mechanisms behind our method's performance gains. Our central hypothesis posits that optimal code integration unlocks synergistic performance benefits by effectively combining the strengths of CoT and code executions.  This synergy presents a "free lunch" scenario: a well-learned metacognitive tool-usage strategy can elevate overall performance, provided the model demonstrates competence in solving \emph{distinct} subsets of queries using either CoT or code execution.

To empirically validate this "free lunch" principle and demonstrate the superiority of our approach in realizing it, we benchmarked our model against baselines that inherently support both code execution and Chain-of-Thought (CoT) reasoning: GPT-4, Mammoth-70B, and DeepseekMath-Instruct-7B. Our analysis evaluated the model's autonomous decision to invoke code when not explicitly instructed on which strategy to employ. We compared this "AutoCode" performance against scenarios where models were explicitly prompted to utilize either code or CoT reasoning. We also considered the theoretical "free lunch" upper bound – the accuracy achieved by combining the successful predictions from either strategy (i.e., taking the union of queries solved by CoT or code).

As visually presented in Figure~\ref{fig_learned_behavior}, existing baseline models exhibit inferior performance in AutoCode mode compared to scenarios where code invocation is explicitly prompted, e.g., DeepseekMath-Instruct-7B shows a degradation of 11.54\% in AutoCode mode. This suggests that their AutoCode strategies are often suboptimal, performing closer to random selection between CoT and code (selection accuracy near 50\%), resulting in AutoCode falling between the performance of explicitly triggered CoT and code. In contrast, our models learn more effective code integration strategies.  AutoCode4Math-Qwen2.5, for example, improves upon explicitly code-triggered performance by 7\%, indicating a true synergistic integration of reasoning and code execution.


To quantify the effectiveness of these learned "AutoCode" strategies, we calculated the CoT/code selection accuracy. We used the outcome of explicit instruction (i.e., performance when explicitly prompted for CoT or code) as a proxy for the ground-truth optimal method selection.  Our model achieves a selection accuracy of 89.53\%, showcasing the high efficacy of the learned code integration strategy.
\paragraph{Errors in machine learning benchmarks} Previous works have studied the identification of errors in machine learning benchmarks, as well as the resulting impact of these label errors on the quality of model evaluations. \citet{tsipras2020from} investigate the original ImageNet labeling process and %
release a refined, multi-label re-labeling of the ImageNet validation set. \citet{northcutt2021pervasive} identify errors in commonly used machine learning benchmarks, and then find that evaluating on benchmarks with significant rates of errors can lead practitioners to incorrectly select less performant models. \citet{bowman2021will} raise concerns similar to ours over issues in benchmarking for NLP tasks, and lay out a set of criteria that good benchmarks should satisfy.
However, they focus on overall design and social impact of benchmarks, whereas we focus specifically on better assessing model reliability. Additionally, while their work identifies similar flaws in NLP benchmarks, we propose an approach that can address some of these flaws (in particular, erroneous examples and ambiguity). %

Recently, \citet{gema2024we} released MMLU-Redux, a re-annotated subset of the MMLU benchmark~\cite{hendrycks2020measuring} created through manual assessment by 14 human experts. Our re-labeling of the MMLU high school mathematics subset actually intersects with MMLU-Redux on 100 examples. Our revised annotations align with MMLU-Redux on all but one example, on which we find that one of their human experts accidentally re-annotated a correct solution to make it incorrect\footnote{See the question marked ``wrong\_groundtruth'' here: \url{https://huggingface.co/datasets/edinburgh-dawg/mmlu-redux/viewer/high_school_mathematics?row=52}}. This slight remaining inconsistency highlights the difficulty of avoiding errors when creating and revising benchmarks.

\paragraph{LLM failures on simple tasks} It is generally known and often discussed that LLMs fail in surprising and unintuitive ways even on simple tasks. For instance, the common example of LLMs failing on the query ``how many r's are there in the word strawberry'' has circulated both social media and news outlets \cite{silberling2024strawberry}. Previous works have investigated specific instantiations of such failures. \citet{yang2024can} find that models frequently fail on simple problems even when they can solve harder versions of these same problems, suggesting inconsistency in their reasoning abilities. 
\citet{nezhurina2024alice} raise similar concerns over breakdowns in LLM reasoning behavior by identifying a specific category of logic tasks on which current frontier LLMs fail consistently.


\paragraph{Adversarial examples}
Adversarial attacks are small, sometimes imperceptible perturbations to model inputs that can drastically change their behavior---these lightly perturbed inputs are referred to as \textit{adversarial examples}. There has been significant work studying adversarial attacks and defenses against them in computer vision (and other) domains \cite{szegedy2014intriguing,carlini2017towards,madry2018towards,papernot16jsma}, and recent work has demonstrated successful adversarial attacks on LLMs, especially in the context of breaking safety alignment~\cite{zou2023universal,xu2023llm}. As a result, one possible approach to identifying LLM failures on simple queries might be to adversarially optimize for queries that result in model failures. However, we aim for our benchmark to assess whether models can be deployed reliably on real-world tasks, and adversarial examples generally do not align with the ``corner cases'' that models might face in the real world (unless the users themselves adversarially optimize their own inputs).


\section{Conclusion}
We introduce a novel approach, \algo, to reduce human feedback requirements in preference-based reinforcement learning by leveraging vision-language models. While VLMs encode rich world knowledge, their direct application as reward models is hindered by alignment issues and noisy predictions. To address this, we develop a synergistic framework where limited human feedback is used to adapt VLMs, improving their reliability in preference labeling. Further, we incorporate a selective sampling strategy to mitigate noise and prioritize informative human annotations.

Our experiments demonstrate that this method significantly improves feedback efficiency, achieving comparable or superior task performance with up to 50\% fewer human annotations. Moreover, we show that an adapted VLM can generalize across similar tasks, further reducing the need for new human feedback by 75\%. These results highlight the potential of integrating VLMs into preference-based RL, offering a scalable solution to reducing human supervision while maintaining high task success rates. 

\section*{Impact Statement}
This work advances embodied AI by significantly reducing the human feedback required for training agents. This reduction is particularly valuable in robotic applications where obtaining human demonstrations and feedback is challenging or impractical, such as assistive robotic arms for individuals with mobility impairments. By minimizing the feedback requirements, our approach enables users to more efficiently customize and teach new skills to robotic agents based on their specific needs and preferences. The broader impact of this work extends to healthcare, assistive technology, and human-robot interaction. One possible risk is that the bias from human feedback can propagate to the VLM and subsequently to the policy. This can be mitigated by personalization of agents in case of household application or standardization of feedback for industrial applications. 

% \clearpage
% \appendix
% \section{Appendix}\label{sec:appendix}


{
\tiny
\bibliographystyle{ACM-Reference-Format}
\balance
\bibliography{bibfile}
}

\newpage
% \appendix
% \section{Appendix}\label{sec:appendix}


\end{document}
\endinput

