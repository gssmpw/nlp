\documentclass[sigconf,natbib=true]{acmart}
\AtBeginDocument{%
  \providecommand\BibTeX{{%
    \normalfont B\kern-0.5em{\scshape i\kern-0.25em b}\kern-0.8em\TeX}}}

% 解决图片浮动问题
\usepackage{float}


\usepackage{booktabs}
\usepackage{array}
\usepackage{balance} 
% \usepackage{lipsum}
\usepackage{multirow}
\usepackage[normalem]{ulem}
\usepackage{color}
\definecolor{lightgray}{RGB}{215,215,215}
\definecolor{myred}{RGB}{210,109,91}
\usepackage{colortbl}  %彩色表格需要加载的宏包
\usepackage{xcolor}
\useunder{\uline}{\ul}{}
\usepackage{subfigure}
% \usepackage{subcaption}
\usepackage{algorithm}  
\usepackage{algorithmicx}  
\usepackage[noend]{algpseudocode}  
% \usepackage[noend]{algorithmic}

\usepackage{amsmath}  
\usepackage{enumitem}
\usepackage{tabularx}
\usepackage[utf8]{inputenc}
\usepackage[english]{babel}
\usepackage{amsthm}
\usepackage{bm}
\newcommand{\ie}{\emph{i.e., }}
\newcommand{\eg}{\emph{e.g., }}
\newcommand{\etal}{\emph{et al. }}
\newcommand{\st}{\emph{s.t. }}
\newcommand{\etc}{\emph{etc.}}
\newcommand{\wrt}{\emph{w.r.t. }}
\newcommand{\cf}{\emph{cf. }}
\newcommand{\aka}{\emph{a.k.a. }}
\newcommand{\todo}[1]{\textcolor{red}{todo: #1}}

% for algorithm
\newlength\myindent
\setlength\myindent{2em}
\newcommand\bindent{%
    \begingroup
    \setlength{\itemindent}{\myindent}
    \addtolength{\algorithmicindent}{\myindent}
}
\newcommand\eindent{\endgroup}

\newtheorem{theorem}{Theorem}[section]
\newtheorem{proposition}{Proposition}[section]
% \newtheorem{lemma}[theorem]{Lemma}

\floatname{algorithm}{Algorithm}
\renewcommand{\algorithmicrequire}{\textbf{Input:}} 
\renewcommand{\algorithmicensure}{\textbf{Output:}}

\clubpenalty=10000
\widowpenalty = 10000
\hyphenpenalty=2000
\tolerance=7000

% \copyrightyear{2022}
% \acmYear{2022}
% \setcopyright{acmcopyright}\acmConference[WWW '22]{Proceedings of the ACM Web Conference 2022}{April 25--29, 2022}{Virtual Event, Lyon, France}
% \acmBooktitle{Proceedings of the ACM Web Conference 2022 (WWW '22), April 25--29, 2022, Virtual Event, Lyon, France}
% \acmPrice{15.00}
% \acmDOI{10.1145/3485447.3512251}
% \acmISBN{978-1-4503-9096-5/22/04}


\begin{document}

\title{Order-agnostic Identifier for Large Language Model-based Generative Recommendation}


\author{Xinyu Lin}
\email{xylin1028@gmail.com}
\affiliation{
\institution{National University of Singapore}
\city{}
\country{Singapore}
}

\author{Haihan Shi}
\email{shh924@mail.ustc.edu.cn}
\affiliation{
\institution{University of Science and Technology of China}
\city{Hefei}
\country{China}
}

\author{Wenjie Wang}
\email{wenjiewang96@gmail.com}
\affiliation{
\institution{University of Science and Technology of China}
\city{Hefei}
\country{China}
}

\author{Fuli Feng}
\email{fulifeng93@gmail.com}
% \authornote{Corresponding author. This work is supported by the CCCD Key Lab of Ministry of Culture and Tourism.}
\affiliation{
\institution{University of Science and Technology of China}
\city{Hefei}
\country{China}
}

\author{Qifan Wang}
\email{wqfcr@meta.com}
\affiliation{
\institution{Meta AI}
\city{Menlo Park}
\country{USA}
}

\author{See-Kiong Ng}
\email{seekiong@nus.edu.sg}
\affiliation{
\institution{National University of Singapore}
\city{}
\country{Singapore}
}


\author{Tat-Seng Chua}
\email{dcscts@nus.edu.sg}
\affiliation{
\institution{National University of Singapore}
\city{}
\country{Singapore}
}
% \def\thefootnote{*}\footnotetext{Corresponding author. This work is supported by the CCCD Key Lab of Ministry of Culture and Tourism.}

% \thanks{$*$ }
\renewcommand{\shortauthors}{Xinyu Lin et al.}


\begin{abstract}
% Leveraging Large Language Models (LLMs) for generative recommendation has garnered significant research interest. 
% A pivotal step is item tokenization, which assigns identifiers to items for user history encoding and item decoding. 
% Existing identifiers broadly fall into two categories: token-sequence-based identifiers, which represent items as discrete token sequences, and single-token-based identifiers, which use either ID or semantic embedding. 
% However, token-sequence-based identifiers suffer from the local optima issue in beam search and low generation efficiency due to step-by-step generation. 
% Conversely, single-token-based identifiers neither overlook the rich semantics beneficial for long-tailed users/items, nor the collaborative information crucial to recommendation, thus leading to suboptimal performance. 
Leveraging Large Language Models (LLMs) for generative recommendation has attracted significant research interest, where item tokenization is a critical step. 
It involves assigning item identifiers for LLMs to encode user history and generate the next item. 
Existing approaches leverage either token-sequence identifiers, representing items as discrete token sequences, or single-token identifiers, using ID or semantic embeddings. Token-sequence identifiers face issues such as the local optima problem in beam search and low generation efficiency due to step-by-step generation.
In contrast, single-token identifiers fail to capture rich semantics or encode Collaborative Filtering (CF) information, resulting in suboptimal performance. 


% To address these issues, we highlight two fundamental principles for identifier design. 
% 1) Integration of both collaborative and semantic information aims to fully leverage multidimensional item information, and 
% 2) order-agnostic identifier emphasizes eliminating dependencies of tokens within items. 
% In this work, we introduce a novel set identifier paradigm for LLM-based generative recommendation, which leverages a set of order-agnostic tokens to represent each item and has simultaneous token generation ability. 
% To achieve this, 
% we propose a method called SETRec, 
% which utilizes CF and semantic tokenizers to obtain order-agnostic multidimensional tokens. 
% For next-item generation, SETRec introduces a sparse attention mask to disregard dependencies within items. 
% Moreover, to ensure LLMs generate tokens aligning with each dimension accurately, SETRec introduces a query-guided simultaneous generation mechanism to guide LLMs for token generation.
% % To enhance efficiency, we design a sparse attention mask that mitigates redundant computations during decoding. 
% Lastly, we instantiate SETRec on T5 and Qwen (from 1.5B to 7B). 
% Results on four real-world datasets demonstrate the superiority of SETRec under various settings (\eg warm- and cold-start settings, and item groups with different popularity) with significantly improved efficiency, and show promising scalability on cold-start items by expanding model sizes. 


To address these issues, we propose two fundamental principles for item identifier design:
1) integrating both CF and semantic information to fully capture multi-dimensional item information, and
2) designing order-agnostic identifiers without token dependency, mitigating the local optima issue and achieving simultaneous generation for generation efficiency. 
Accordingly, we introduce a novel \textit{set identifier} paradigm for LLM-based generative recommendation, representing each item as a set of order-agnostic tokens. 
To implement this paradigm, we propose SETRec, which leverages CF and semantic tokenizers to obtain order-agnostic multi-dimensional tokens. 
To eliminate token dependency, SETRec uses a sparse attention mask for user history encoding and a query-guided generation mechanism for simultaneous token generation.
We instantiate SETRec on T5 and Qwen (from 1.5B to 7B). 
Extensive experiments on four datasets demonstrate its effectiveness across various scenarios (\eg full ranking, warm- and cold-start ranking, and various item popularity groups). 
Moreover, results validate SETRec's superior efficiency and show promising scalability on cold-start items as model sizes increase. 



\end{abstract}

%%
%% The code below is generated by the tool at http://dl.acm.org/ccs.cfm.
%% Please copy and paste the code instead of the example below.
%%
% \vspace{-2pt}
\begin{CCSXML}
% <ccs2012>
% <concept>
% <concept_id>10002951.10003260.10003261.10003271</concept_id>
% <concept_desc>Information systems~Personalization</concept_desc>
% <concept_significance>500</concept_significance>
% </concept>
<concept>
<concept_id>10002951.10003317.10003347.10003350</concept_id>
<concept_desc>Information systems~Recommender systems</concept_desc>
<concept_significance>500</concept_significance>
</concept>
</ccs2012>
\end{CCSXML}
% \ccsdesc[500]{Information systems~Personalization}
\ccsdesc[500]{Information systems~Recommender systems}
% \vspace{-2pt}
\keywords{Item Tokenization, Set Identifier, LLM-based Recommendation}



\maketitle

% \def\thefootnote{*}\footnotetext{Corresponding author. This work is supported by the CCCD Key Lab of Ministry of Culture and Tourism.}

\section{Introduction}
\label{sec:intro}
% Image editing methods in diffusion models depend on user-defined control directions - users can unlock their creativity using these methods by specifying the desired manipulation through prompts~\cite{gandikota2023concept}, reference images~\cite{ruiz2022dreambooth, kumari2022customdiffusion, gal2022image, chen2024trainingfreeregionalpromptingdiffusion}, or attribute vectors~\cite{parmar2023zero,hertz2022prompt}. In this work, we ask a fundamentally different question: \emph{Can we automatically discover the underlying visual structure of a concept within diffusion model's knowledge?} %Rather than requiring user-specified controls, we aim to decompose the model's internal knowledge into meaningful directions.

% This question touches on a fundamental limitation in how we interact with diffusion models. Current control methods ~\cite{zhang2023addingconditionalcontroltexttoimage, gandikota2023concept, ye2023ipadaptertextcompatibleimage,ye2023ipadaptertextcompatibleimage, hertz2024stylealignedimagegeneration, li2023photomaker, shi2024instantbooth, chen2024trainingfreeregionalpromptingdiffusion} require users to specify their desired manipulations in advance, limiting interactive creativity. This contrasts with natural human artistic workflows, where creators dynamically explore creative ideas while jointly refining them toward meaningful artistic outcomes~\cite{hoffmann2016modeling}. This synergy between specification and exploration is not new to generative models. Early GAN architectures naturally developed disentangled latent spaces that enabled continuous\cite{harkonen2020ganspace,radford2015unsupervised, wu2021stylespace, shen2020interfacegan}, compositional control over generated images. Users could explore these spaces to discover interesting variations that would be difficult to describe in words~\cite{wu2021stylespace}, then combine them to achieve their creative goals~\cite{grabe2022towards}. 


% While diffusion models have largely superseded GANs in conditional image synthesis~\cite{dhariwal2021diffusion},  their underlying structure remains less understood. Diffusion models achieve remarkable diversity through high-dimensional latents, unlike GANs' compact latent spaces.  With a single prompt, diffusion models can generate radically different variations through different random initializations of input noise. We ask - Is it possible to discover interpretable structure within this vast space of variations?

Text-to-image diffusion models are capable of generating remarkable visual variations from a single prompt through different random initializations. However, this vast creative potential remains largely opaque to users---while we can generate diverse images, we lack understanding of the underlying structure of these variations. This presents a fundamental challenge: how can we discover and expose the latent visual capabilities encoded within these models?

\let\thefootnote\relax \footnote{$^{*}$Correspondence to \texttt{gandikota.ro@northeastern.edu}}

The challenge touches on a key limitation in how we interact with diffusion models today. Current control methods require users to explicitly specify their desired edits in advance through prompts~\cite{gandikota2023concept}, reference images~\cite{zhang2023addingconditionalcontroltexttoimage, chen2024trainingfreeregionalpromptingdiffusion, ruiz2022dreambooth,kumari2022customdiffusion, Ryu_lora, hu2021lora}, or attribute vectors~\cite{ye2023ipadaptertextcompatibleimage, hertz2024stylealignedimagegeneration, li2023photomaker, shi2024instantbooth,parmar2023zero,hertz2022prompt}. That contrasts sharply with natural human creative workflows, where artists dynamically explore creative ideas and jointly refine them toward meaningful artistic outcomes~\cite{hoffmann2016modeling}. The need for pre-specified controls creates a barrier between users and the full creative potential of these models.

Interestingly, earlier generative models like GANs~\cite{gans,karras2019style,brock2018large} naturally developed more interpretable internal structures. Their compact latent spaces often exhibited emergent disentanglement~\cite{harkonen2020ganspace,radford2015unsupervised, wu2021stylespace, shen2020interfacegan}, enabling continuous and compositional control over generated images. Users could explore these spaces to discover interesting variations that would be difficult to describe in words~\cite{wu2021stylespace}, then combine them to achieve their creative goals~\cite{grabe2022towards}.

Diffusion models have largely superseded GANs in conditional image synthesis~\cite{dhariwal2021diffusion}, achieving greater diversity through much higher-dimensional latents. And yet an understanding of the underlying structure of these larger latent spaces has remained elusive. In this work, we ask a fundamental question: \emph{Can we automatically discover the visual structure within a diffusion model's knowledge of a concept?} Rather than requiring user-specified controls, we aim to decompose the model's internal representations into expressive directions that users can explore and combine.

To address these needs, we present \textbf{SliderSpace}, a framework that brings systematic explorability to diffusion models. Given just a text prompt, SliderSpace discovers a canonical set of meaningful, diverse, and controllable directions within the model's knowledge of that concept. Each direction is implemented as a low-rank adapter~\cite{hu2021lora} that can be scaled and composed with others, allowing users to explore and smoothly combine different aspects of variation, as shown in Figure~\ref{fig:intro}.

We ground SliderSpace discovery in three key requirements for meaningful decomposition of a diffusion model's visual manifold: 
\begin{enumerate}
    \item \textbf{Unsupervised Discovery:} The decomposition process should emerge from the intrinsic structure of the model's learned representation, rather than being guided by predefined attributes. This ensures we capture the true topology of the model's knowledge space rather than projecting our assumptions onto it.
    
    \item \textbf{Semantic Orthogonality:} Each discovered control must represent a distinct semantic direction. This is enforced in a semantic feature space, like CLIP, where every slider has an orthogonal effect in embeddings. This prevents discovering multiple controls that create similar semantic effects, making the system more efficient and easier.
    
    \item \textbf{Distribution Consistency:} Directions must induce consistent transformations across both random seeds and prompt variations. 
\end{enumerate}

These requirements naturally lead to our proposed framework, which we formalize in Section~\ref{sec:method}. As we show in our experiments, SliderSpace is architecture-agnostic, working with both conventional U-Net based models like Stable Diffusion~\cite{rombach2022high, rombach2022sd20, podell2023sdxl, turbo, dmd} and recent transformer-based architectures like Flux~\cite{flux}.

We demonstrate the expressiveness of SliderSpace through three applications: First, we show how SliderSpace can decompose high-level concepts into diverse and expressive components, revealing the natural axes of variation in the model's understanding. Second, we explore artistic style variation, where SliderSpace discovers directions that match or exceed the diversity of manually curated artist lists while being judged more useful by human evaluators. Finally, we show how SliderSpace can help reverse the mode collapse commonly observed in distilled diffusion models, restoring diversity while maintaining generation speed.

Beyond providing practical creative control, SliderSpace opens new avenues for understanding and utilizing the latent capabilities of diffusion models. By mapping these models' visual potential into intuitive, composable directions, we take a step toward making their creative possibilities more accessible and interpretable to users.

% Image editing methods in diffusion models unlock the creativity of users. In this work we ask an alternate question: \emph{Can we organize and expose what of the diffusion model is already capable of?}.
% Existing methods for controlling image generation typically require users to manually specify edit directions for desired changes. This process is time-consuming, requires technical expertise, and limits the spontaneity of the creative process. For instance, if a user wants to adjust the smile of a generated person, they must explicitly request this edit, often through imprecise prompt engineering or model fine-tuning. This approach of predefined controls or manual specifications restricts users from fully exploring the latent capabilities of the model. There may be interesting stylistic variations or attributes that the model can generate, but users have no easy way to discover or utilize these.

% Natural visual disentanglement was an emergent property in the latent space of Generative Adversarial Models (GANs) \cite{harkonen2020ganspace,radford2015unsupervised, wu2021stylespace, shen2020interfacegan}. In particular, it has been observed that StyleGAN~\cite{karras2019style} stylespace neurons offer detailed control over many meaningful aspects of images that would be difficult to describe in words~\cite{wu2021stylespace}. However, diffusion models do not share such a compact latent space~\cite{park2023unsupervised}; and efforts to uncover such a space in the semantic embeddings of the text conditioning have met with limited success \nik{Nick - is there a specific citation you were thinking about?}.

% In this work we introduce \textbf{SliderSpace}, which takes a step towards uncovering an analogous low dimensional representation of diffusion models' visual breadth; in essence treating the diffusion model as many generators sharing parameters, where a particular generator is defined by a specific prompt. For a given prompt we sample many random seeds (and optionally prompt expansions using an LLM), generate the corresponding images, and apply an off the shelf feature extractor (in this work CLIP, but our method can be applied to any differentiable feature extractor). We use PCA to analyze these features, and for each of the leading $k$ principal components we train a LoRA \cite{} which causes the diffusion model to produces images which increase the feature magnitude along that component when passed back through the same feature extractor. This leads to a 'Slider' for each principal component, because each LoRA can be scaled and applied to the original diffusion model, continuously varying those visual features in the generated results (as measured, in our case, by CLIP).

% There are many other works that enhance the controllability of diffusion models. One common approach is enabling users to add spatial constraints to a generation either manually, or via a reference image \cite{zhang2023addingconditionalcontroltexttoimage, chen2024trainingfreeregionalpromptingdiffusion}, a second is leveraging more abstract embeddings (e.g. identity, style) extracted from a reference image \cite{ye2023ipadaptertextcompatibleimage, hertz2024stylealignedimagegeneration, li2023photomaker, shi2024instantbooth}, a third is finetuning a foundation model to better generate a concept important to the user \cite{ruiz2022dreambooth, kumari2022customdiffusion, Ryu_lora, hu2021lora}, and a fourth (most relevant to this work) is finding low-rank adaptors of the model based on a prompt or small training set which can be scaled to provide continous control over one aspect of generated image (e.g. night vs day, basic vs luxury, etc.) \cite{gandikota2023concept}. SliderSpace is complementary to all of these methods and offers something distinct. All of the other methods we are aware require the user (and / or model designer) to know in advance what type of control they want. In contrast SliderSpace assists users in discovering and controlling hidden capabilities present in the diffusion model's distribution of possible generations.

%We propose that truly intuitive creative control in a text-to-image model should meet three key criteria: \emph{discoverability}, \emph{intuitiveness}, and \emph{specificity}. The model should reveal controllable attributes that may not be immediately obvious, offer controls that are easy to understand and manipulate, and ensure each control affects a distinct attribute of the generated image.

% We demonstrate the utility and power of SliderSpace using three applications built on top of SDXL-DMD \cite{dmd}, because its fast generation speed lends itself well to the continuous control offered by SliderSpace.

% First, we study concept decomposition (Section \ref{sec:concept_exp}), where we learn sliders for a specific concept (e.g. 'monster', 'waterfall', 'car'). Through quantitative metrics of diversity and text alignment we demonstrate that the learned sliders dramatically boost the diversity of generations when randomly applied without harming text alignment; we also ask humans to qualitatively judge these results in a user study where they find the SliderSpace results to be more 'Diverse', 'Useful', and 'Creative' than our baselines.

% Second, we attempt to compare the automatic discoveries of SliderSpace to a large scale manual study of artistic styles (Section \ref{sec:art_exp}), open-sourced by ParrotZone \cite{parrotzone}. In this study SDXL was prompted with over 4300 artist names,  and based on visual inspection the cases of successful stylistic mimicry recorded. Quantitatively SliderSpace more closely matches the distribution of artistic variation discovered by ParrotZone than other baselines, and in our user studies was judged to be significantly more 'Diverse' and 'Useful' than the baselines. To our surprise humans even judged SliderSpace results to be slightly more 'Diverse' than the results generated by the manually discovered artist names of \cite{parrotzone}.

% Third, we attempt to use SliderSpace to reverse the mode collapse commonly observed in distilled few-step diffusion models relative to the original teacher model (Section \ref{sec:diverse_exp}). We quantitatively demonstrate that applying SliderSpace to SDXL-DMD leads to more closely matching the distribution of images by the original teacher, SDXL.

%Through extensive experiments on various state-of-the-art text-to-image models, we demonstrate that SliderSpace significantly enhances user control and creative expression in AI-assisted image generation tasks. Our method enables a range of applications, including concept decomposition and control, diversity improvement in generated images, customization dissection and edits, and the exploration of artistic styles inherent in the model.

% SliderSpace goes beyond providing a practical tool for enhanced creative control. By mapping the visual potential of diffusion models it can open new avenues for generative creativity and deepens our understanding of each model's hidden potential.
\section{Preliminaries}\label{sec:task_formulation}
% In this section, we first introduce the task of LLM-based recommendation, and then retrospect existing item identifier designs and reveal their inherent critical issues. 

\vspace{2pt}
\noindent\textbf{LLM-based Generative Recommendation.} 
Harnessing LLMs' strong capabilities, LLM-based generative recommendation aims to use LLMs as recommenders to directly generate personalized recommendations. 
Formally, given the recommendation data 
$\mathcal{D}=\{\mathcal{S}_u|u\in\mathcal{U}, i\in\mathcal{I}\}$, where $\mathcal{S}_u = [i_1^{u}, i_2^{u}, \dots, i_L^{u}]$ is the user's historical interactions in chronological order and $L=|\mathcal{S}_u|$, 
% the target is to utilize a tokenizer $f(\cdot)$ and an LLM-based recommender model $\mathcal{M}(\cdot)$, to tokenize items into item identifiers $\tilde{\mathcal{I}}$, and encode the transformed user history (\ie identifier sequence) to generate recommended items. 
the target is to utilize a tokenizer $f(\cdot)$ to tokenize items into item identifiers $\tilde{\mathcal{I}}$, 
and an LLM-based recommender model $\mathcal{M}(\cdot)$ to encode the transformed user history $\bm{x}= [f(i_{1}), f(i_{2}), \dots, f(i_{L})]$ and generate next item identifier. 
% $\hat{\mathcal{I}}=\{\hat{i}\}$. 

% item tokenization起到了至关重要的作用, bridging the xxx

\vspace{2pt}
Bridging the language space and the item space, item identifier is a fundamental component for LLMs to encode user history and generate items. 
Existing identifiers can be divided into two groups:

% 先前的工作设计了不同的f来得到token sequence,为了更好的把pretraining的语言模型adapt到文本形式的推荐任务上。
\vspace{2pt}
\noindent$\bullet\quad$\textbf{\textit{Token-sequence identifier}} assigns each item with a discrete token sequence, \ie $\tilde{i} = [z_1, z_2, \dots, z_N]$, where $z_i$ is the discrete token. 
Given the user history $\mathcal{S}_u$, it is transformed to an identifier sequence $\bm{x}= [\tilde{i}_1, \tilde{i}_2, \dots, \tilde{i}_L]$, which is then encoded by LLMs to generate the next identifier via autoregressive generation: 
\begin{equation}
% \left\{
\begin{aligned}
    &\hat{y}_t = \mathop{\arg\max}_{v\in\mathcal{V}} \mathcal{M}(v|\hat{y}_{<t},\bm{x}), \\
\end{aligned}
% \right.
\end{equation} 
where $\mathcal{V}$ is the LLM vocabulary. 
Despite the effectiveness, generating token sequences would result in the local optima issue and inference inefficiency. 
As shown in Figure~\ref{fig:beam_size}, continuously increasing the beam size slightly improves recommendation accuracy, but remains inferior to globally optimal results. 
Worse still, the token-by-token generation requires multiple serial LLM calls, which significantly lowers the inference speed and hinders real-world applications.

\begin{figure}[t]
% \vspace{-0.2cm}
\setlength{\abovecaptionskip}{0.0cm}
\setlength{\belowcaptionskip}{-0.3cm}
\centering
\includegraphics[scale=0.4]{figures/beam_size.pdf}
\caption{Performance comparison between beam search and global search of LETTER on Toys. The global search is implemented by computing sequence probability for every item and ranking them based on the probabilities. }
\label{fig:beam_size}
\end{figure}


% of LLM-based generative recommendation. 



% 2. 而single embedding则是利用cf recommender 或者 feature extractor之类的 -> 获取到每个item的
\vspace{2pt}
\noindent$\bullet\quad$\textbf{\textit{Single-token identifier}} assigns each item with an ID or semantic embedding, \ie $\tilde{i}=\bm{z}$, 
% Existing work typically leverages a conventional CF recommender model as the tokenizer to obtain ID embedding 
which is usually obtained by a conventional CF recommender model (\eg SASRec~\cite{kang2018self})
or a pre-trained semantic extractor (\eg SentenceT5~\cite{ni2021sentence}). 
% Escaping from token-by-token generation, this line of work facilitates efficient item generation~\cite{wang2024rethinking}. 
Given the transformed user history $\bm{x}= [f(i_{1}), f(i_{2}), \dots, f(i_{L})]$, it first generates the embedding:
\begin{equation}
\begin{aligned}
% &x= [f(i_{1}), f(i_{2}), \dots, f(i_{L})] \\
&\hat{i} = \text{LLM\_Layers}(\bm{x}), \\
\end{aligned}
\end{equation}
where $\text{LLM\_Layers}(\cdot)$ is the attention layers from the LLM $\mathcal{M}(\cdot)$. 
Based on the generated item embedding $\hat{i}$, 
an additional grounding head is added on top of the LLM layers to obtain the scores for all items for ranking. 
% 虽然这类方法不会有beam search的问题,因为它只有一个token。
% 但是 - 只有单一维度的信息。只有CF会导致泛化能力弱,只有sem会导致与交互信息不对齐。
% 2) single-embedding-based(只有cf的难以泛化到只有sem的,而只有sem又可能和cf有所冲突)(-这个可以考虑加一下图)
Although it improves inference efficiency by bypassing the token-by-token autoregressive generation, 
representing items with a single ID embedding struggles with items with fewer interactions while a single semantic embedding overlooks the crucial CF information, thus leading to suboptimal results. 
% As shown in Figure~\ref{}, items with similar semantics might have different interactions. 

% \noindent$\bullet\quad${\textbf{\textit{Set identifier}}}. 
Based on the above insights, we
summarize two fundamental principles for identifier designs: 
1) integration of semantic and CF information, to leverage rich multi-dimensional item information, and  
2) order-agnostic identifier, to eliminate the unnecessary dependencies between tokens associated with an identifier, which can alleviate the local optima issue and improve generation efficiency.  
In this light, we introduce a novel set identifier paradigm, which employs a set of order-agnostic tokens to represent multi-dimensional item information. 
% facilitating globally optimal results and achieving simultaneous generation to boost efficiency. 
% It encodes the user history (\ie identifier sequence) and decodes the tokens in set identifiers simultaneously. 
% which removes the dependency from token to token within identifiers, thus alleviating the local optima issue and boost the efficiency. 


\section{SETRec}\label{sec:method}
% To meet the two principles for identifier design, we propose a set-based-identifier for effective and efficient LLM-based generative recommendation. 
% In this section, we first elaborate the design of set-based item tokenization, and then delve into the decoding of set-based item identifier. 
% Lastly, we detail the instantiation of our proposed SETRec, including training and inference.  
To implement the set identifier paradigm, we propose a framework called SETRec for effective and efficient LLM-based generative recommendation, including order-agnostic item tokenization and simultaneous item generation as illustrated in Figure~\ref{fig:method_tokenizer}.  
% This section first elaborates on order-agnostic item tokenization and then delves into the simultaneous item generation. 
% Lastly, we detail the instantiation of our proposed SETRec, including training and inference.  


\begin{figure}[t]
% \vspace{-0.2cm}
\setlength{\abovecaptionskip}{0.02cm}
\setlength{\belowcaptionskip}{-0.3cm}
\centering
\includegraphics[scale=0.88]{figures/method_tokenizer.pdf}
\caption{(a) demonstrates SETRec framework, including order-agnostic item tokenization, and simultaneous item generation. The dependencies within identifiers and query vectors are eliminated by the sparse attention mask (see Figure~\ref{fig:sparse_attn} for details). (b) illustrates order-agnostic item tokenization via CF and semantic tokenizers.}
\label{fig:method_tokenizer}
\end{figure}


\subsection{Order-agnostic Item Tokenization} 
% In SETRec, we integrate both semantic and CF information, 
% To assign each item with a set of tokens representing both CF and semantic information
% In SETRec, we introduce a set-based item tokenization strategy, which assigns each item with a set of tokens representing both CF and semantic information. 
Meeting the two principles, SETRec leverages a CF and a semantic tokenizer to endow multi-dimensional information into a set of order-agnostic continuous tokens\footnote{We do not use discrete tokens in SETRec because discretization inevitably suffers from information loss~\cite{lazebnik2008supervised}, potentially leading to suboptimal results.} as illustrated in Figure~\ref{fig:method_tokenizer}(b). 
% To integrate semantic and CF information into identifiers, 
% a possible solution is using a set of discrete tokens. 
% 为什么用连续的不用离散的?
% Using a set of discrete tokens is a possible solution. 
% Therefore, we alternatively adopt a set of continuous tokens for SETRec, aiming to harness the item information as much as possible. 
% 我们通过cf tokenizer和semantic tokenizer获取 CF token 和 semantic token
% To endow multi-dimensional features into the token set, we leverage a CF tokenizer and a semantic tokenizer to obtain CF tokens and semantic tokens, respectively. 

% \noindent$\bullet\quad$\textbf{Multi-dimensional Tokenizer}
\noindent$\bullet\quad$\textbf{CF Tokenizer.}
% CF tokenizer 
As shown in Figure~\ref{fig:method_tokenizer}(b), we utilize a pre-trained conventional recommender model (\eg SASRec~\cite{lazebnik2008supervised}) with a linear projection layer to obtain item CF embedding $\bm{z}_{\text{CF}}\in\mathbb{R}^{d}$, where $d$ is the hidden dimension of LLMs. 
Incorporating CF embeddings encourages LLM-based recommenders to facilitate recommendations for users/items with rich interactions. 
% The CF tokenizer ensures that items with similar interactions have similar identifiers. 
% facilitating recommendations for users and items with rich interactions
% As such, representing item with CF embeddings encourages LLM-based recommender to leverage collaborative information that is crucial for recommendation. 
% merely using CF embedding would struggle on cold-start items, thus leading to poor generalization. 

\noindent$\bullet\quad$\textbf{Semantic Tokenizer.} 
To fully utilize rich item semantic information, SETRec introduces a semantic tokenizer to obtain a set of semantic embeddings. 
Specifically, 
given the item semantic information $c$ such as title and categories, we first extract the item semantic representations $\bm{s}$ with a pre-trained semantic extractor (\eg SentenceT5~\cite{ni2021sentence}). 

To obtain the semantic embeddings, a straightforward approach is to compress semantic representation $\bm{s}$ into a single latent semantic embedding. 
% Nonetheless, single embedding for semantic tokens might suffer from the embedding collapse issue~\cite{}, potentially undermining the rich semantic content that distinguishes between items. 
Nonetheless, compressing multi-dimensional semantic information (\eg ``brand'' and ``price'') might suffer from the embedding collapse issue~\cite{guoembedding,pan2024ads}, potentially undermining the rich semantic content that distinguishes between items. 
To prevent this issue, as depicted in Figure~\ref{fig:method_tokenizer}(b), we tokenize each item into $N$ order-agnostic semantic embeddings via an AE: 
\begin{equation}
% \left\{
\begin{aligned}
&\bm{z} = \text{Encoder}(\bm{s}), 
\end{aligned}
% \right.
\end{equation}
where 
$\bm{z}=[\bm{z}_{S_{1}}, \bm{z}_{S_{2}},\dots,\bm{z}_{S_{N}}]\in\mathbb{R}^{Nd}$ denotes the concatenated semantic embeddings representing different latent semantic dimensions, and ${z}_{S_{n}}\in\mathbb{R}^{d}$ is the $n$-th semantic embedding. 
% where $d$ is the latent dimension of LLMs. 
% $N$ is the number of semantic embeddings, and $n\in[1,\dots, N]$. 
Notably, we utilize a unified AE instead of multiple independent AEs for two considerations: 
1) employing a single AE reduces the parameters with an approximate ratio of $\frac{1}{N}$, which is highly practical; 
2) alleviating the training instability that might be caused by multiple encoders' training~\cite{tang2023improving}. 
In addition, to encourage the semantic embeddings to preserve useful information as much as possible, 
a reconstruction loss is used to train the semantic tokenizer: 
\begin{equation}
    \mathcal{L}_{AE} = \|\bm{s}-\hat{\bm{s}}\|_2^{2}, 
\end{equation}
where 
$\hat{\bm{s}} = \text{Decoder}(\bm{z})$ is the reconstructed semantic representation. 

\noindent$\bullet\quad$\textbf{Token Corpus.} 
% 最终基于两个tokenizer,一个item的set-based identifier是什么,怎么表示的。
Based on the CF and the semantic tokenizer, we can obtain the set identifier for each item
$\tilde{i} = \{\bm{z}_{\text{CF}},\bm{z}_{S_{1}},\dots,\bm{z}_{S_{N}}\}$, consisting of a CF embedding and $N$ semantic embeddings. 
% 基于这两个之后,加一个token corpus。
We then can collect tokens from all items and obtain the token corpus for each information dimension, \ie $\mathcal{Z}_\text{CF}, \mathcal{Z}_{S_1}, \dots, \mathcal{Z}_{S_N}$. 
% where $\mathcal{Z}_k=\{\bm{z}_{k,j}|j\in\{1,\dots, |I|\}\}$. 
The collected token corpus is used as the grounding head for effective item grounding (\cf Section~\ref{sec:query-based_decoding}). 

\subsection{Simultaneous Item Generation} 
% To achieve item generation with set-based identifier, the key lies in generating the set 
% Given the user's historical interactions, SETRec gene
% Based on the set-based item tokenization, the user 
% without token dependencies, 
To efficiently and effectively generate set identifiers, it is crucial for SETRec to 
% the key lies in enabling simultaneous token generation in multiple dimensions.  
% To achieve this, 
1) guide LLMs to distinguish different dimensions and generate tokens aligning well with each dimension simultaneously (Section~\ref{sec:query-based_decoding}); 
2) ground the generated token set to existing items effectively (Section~\ref{sec:query-based_decoding}); 
3) eliminate the unnecessary dependencies introduced in user history (Section~\ref{sec:sparse_attention_mask}); 

\subsubsection{\textbf{Query-guided Generation}.}\label{sec:query-based_decoding}
% \noindent$\bullet\quad$\textbf{Query-based Simultaneous Decoding.}  
As shown in Figure~\ref{fig:method_tokenizer}(a), to guide LLMs to generate tokens that align well with the information dimensions, 
we introduce a set of learnable query vectors $\bm{q}\in\mathbb{R}^{d}$, where $d$ is the latent dimension of the LLMs, to guide the LLMs to distinguish between information dimensions (\eg CF and semantic) for token generation. 
Formally, the generated token $\hat{\bm{z}}_k$ for each dimension $k\in\{\text{CF}, S_1, S_2, \dots, S_N\}$ is obtained via:
\begin{equation}\label{eqn:query_decoding}
\left\{
\begin{aligned}
    &\bm{x} = [\{\bm{z}_{\text{CF}}, \bm{z}_{S_1}, 
     \dots, \bm{z}_{S_N}\}^{1}, \dots, \{\bm{z}_{\text{CF}}, \bm{z}_{S_1}, \dots,\bm{z}_{S_N}\}^{L}], \\
    &\hat{\bm{z}}_{k} = \text{LLM\_Layers}(\bm{x}, \bm{q}_k),
\end{aligned}
\right.
\end{equation}
where $\bm{q}_k$ is the learnable query vector to guide LLM generation for the information dimension $k$. 
Based on Eq. (\ref{eqn:query_decoding}), we can collect the generated token for all dimensions and obtain the generated set identifier  $\hat{i}=\{\hat{\bm{z}}_{\text{CF}}, \hat{\bm{z}}_{S_1}, \dots, 
\hat{\bm{z}}_{S_N}
\}$. 

\vspace{2pt}
\textbf{\textit{Token Generation Optimization.}} 
% 1. 为了教会LLM在生成的时候能做准确的item reocmmendation,我们鼓励llm能够生成target item的token。
To achieve accurate item recommendations, 
% we train SETRec to generate the set identifier of the target item. 
% 2. 介绍我们怎么做的 - specifically, 对于每一个feature 维度,我们encourage生成的token要和target token尽可能的相似. 
% Specifically, 
we encourage the generated token to align with the target token for every dimension: 
% 3. 公式 - xxx 介绍每个变量是什么
\begin{equation}\label{eqn:loss_gen}
    \mathcal{L}_{\text{Gen}} = - \frac{1}{|\mathcal{D}|} \sum_{\mathcal{D}} \sum_{k\in\mathcal{F}}\frac{\exp (sim(\hat{\bm{z}}_k,\bm{z}_{k}))}{\sum_{\bm{z}\in\mathcal{Z}_k}\exp(sim(\hat{\bm{z}}_k,\bm{z}))},
\end{equation}
where $\mathcal{F}=\{\text{CF}, S_1, \dots, S_N\}$, $sim(\cdot)$ is the similarity function (\eg inner product), and $\bm{z}_k$ is the target item token for the information dimension $k$. 
% 4. intuitive解释这样能够拉近正确的距离,拉远不相似的距离,鼓励也能鼓励item之间有所区分。
Intuitively, Eq. (\ref{eqn:loss_gen}) pushes the generated embedding closer to the target embedding and pulls away from other embeddings within the specific information dimension.  
% Besides, by pulling embeddings away from each other is also encouraged to distinguish between items for every feature dimensions, which further improves the embedding diversity. 


% 标题这里考虑换成两个黑体,token generatinon training, 和 token grounding for inference
% \noindent$\bullet\quad$\textbf{Token Set Grounding.}
\vspace{2pt}
\textit{\textbf{Token Generation Grounding.}} 
% 提出需要grounding的问题 - 解释在inference的时候生成token,得到上一步的set之后,直接使用是无法match到现有item上的。
Based on generated tokens obtained via Eq. (\ref{eqn:query_decoding}), the next step is to ground them to the existing items. 
% 顺应解释按照token去ground仍然会ooc - 一种简单的方法是用l2distance等等去match到token。但即便如此,match后得到的set,仍然很可能会out-of-corpus. 
% We introduce a two-step grounding strategy, which utilizes a 
However, this can be challenging since the possible combinations of the tokens from different information dimensions are much larger than the existing item corpus, \ie $\prod_{k\in\mathcal{F}}|\mathcal{Z}_k|\gg |\mathcal{I}|$. 
% 提出我们的解决方法:因此我们考虑借助token corpus来获取所有item上的logits. 
To solve this issue, we introduce a token set grounding strategy, which leverages the token corpus as grounding heads to obtain the item score. 
% specifically,(1. 根据token来获取每个feature维度的score代表item的score,2. 合并多维度之间的结果,获得全局的score)
% Specifically, we first obtain the item scores for each dimension $k$: 
% \begin{equation}\label{eqn:single_logits}
%     s_k=W_{\text{proj}}^{k}\bm{\hat{z}}_k,
% \end{equation}
% where $W_{\text{proj}}^{k}\in\mathbb{R}^{|I|\times d}$ is adopted from the token corpus $\mathcal{Z}_k$. 
% We then obtain the final matching scores with a linear combination of CF dimension and semantic dimension via 
% \begin{equation}\label{eqn:final_logits}
%     s = (1-\beta) s_\text{CF} + \beta \sum_{i=1}^{N} s_i,
% \end{equation}
% where $\beta$ is a hyper-parameter to balance the strength between CF and semantic dimensions. 
Formally, we have 
\begin{equation}\label{eqn:single_logits}\small
\left\{
\begin{aligned}    &s_k=W_{k}\bm{\hat{z}}_k, \\
&s = (1-\beta) s_\text{CF} + \beta \sum\nolimits_{{k\in \mathcal{F}\setminus{\text{CF}}}} s_{k}, \\
\end{aligned}
\right.
\end{equation}
% Specifically, we first obtain the item scores for each dimension $k$: 
% \begin{equation}\label{eqn:single_logits}
%     s_k=W_{\text{proj}}^{k}\bm{\hat{z}}_k,
% \end{equation}
where $W_{k}\in\mathbb{R}^{|I|\times d}$ is adopted from the token corpus $\mathcal{Z}_k$. 
The final item scores are obtained via a linear combination of CF and semantic dimensions, where $\beta$ is a hyper-parameter to balance the strength between CF and semantic dimensions. 
It is highlighted that the grounding heads for semantic dimensions are extendable to new items, leading to strong generalization ability (\cf Section~\ref{sec:overall_performance}). 


\begin{figure}[t]
% \vspace{-0.2cm}
\setlength{\abovecaptionskip}{0.02cm}
\setlength{\belowcaptionskip}{-0.3cm}
\centering
\includegraphics[scale=1.1]{figures/sparse_attn.pdf}
\setlength{\fboxrule}{1pt}
\caption{Comparison between original attention and sparse attention ($N=1$). The sparse attention 1) eliminates the dependency over other tokens within the same item ({\color{myred}\fbox{\phantom{\rule{0.06cm}{0.06cm}}}}), and 2) boosts the efficiency with the flattened input, \ie query vectors are in the same sequence.}
\label{fig:sparse_attn}
\end{figure}

\vspace{2pt}
\subsubsection{\textbf{Sparse Attention Mask.}}\label{sec:sparse_attention_mask}
% \noindent$\bullet\quad$\textbf{Sparse Attention Mask.} 
% 提出问题 - 可能需要画一个图说明一下什么意思 。
% 两个问题 1.第一个问题是query在生成的时候batch会很浪费计算资源
% Based on the simultaneous decoding, each token can be generated independently at a single LLM call via batch decoding (Eq. (\ref{eqn:query_decoding})). 
% Nonetheless, this will cause repetitive self-attention computations for the user's historical interactions $\bm{x}$, thus leading to inefficiency. 
% % 1.第二个是transformer仍然需要序列送进去,如何瓦解item内部token的序列关系?
% Moreover, while simultaneous decoding bypasses the sequential generation of item identifier, the user's historical interactions are still sequentially encoded via causal attention mask\footnote{Most of LLMs adopt decoder-only architecture, which employs the causal attention mask. For the encoder-deocder LLMs, the bi-directional attention mask will not introduce noisy dependencies.}, potentially introducing dependencies between tokens within each identifier. 

% \vspace{2pt}
% To address these two challenges, we introduce a sparse attention mask, as illustrated in Figure~\ref{fig:sparse_attn}. 
% % 我们提出一个sparse attention mask 同时来解决这两个问题,获得efficient的set-based identifier generation的方法。
% Specifically, for the user's historical interactions, tokens associated with an identifier are treated as independent from each other (\eg CF embedding cannot attend to semantic embeddings for item 1). 
% However, these tokens can still attend to all tokens in previously interacted items (\eg a fully attended mask is applied to item 1 when calculating self-attention for tokens in item 2).
% This sparse attention mask ensures the order invariance of our proposed set-based identifier.
% %Meanwhile, they can attend all tokens in previously interacted items (\eg fully attended mask on item 1 when calculating self-attention for tokens of item 2). 
% %Based on the sparse mask, we can ensure the order invariance of our proposed set-based identifier. 

% % todo: 加一个证明,可以放到appendix里面

% 1. 先讲implicit order的事情,呼应intro以及方法部分。接着介绍我们的方法。
% 介绍完之后,说这样的sparse attention不仅移除了dependency,还能够有效的boost inference efficiency。
% 加一句解释说,eq 5 支持independent generation,但是利用original attention的话会重复计算。使用我们的sparse能够显著降低计算
% 跟上complexity analysis
% 1.第二个是transformer仍然需要序列送进去,如何瓦解item内部token的序列关系?
While simultaneous generation bypasses the sequential generation of item identifier, the flattened user's historical interactions are still sequentially encoded, inevitably introducing order information of tokens within each identifier (Figure~\ref{fig:sparse_attn}(a)). 
To combat this issue, we introduce a sparse attention mask as illustrated in Figure~\ref{fig:sparse_attn}(b). 
% 我们提出一个sparse attention mask 同时来解决这两个问题,获得efficient的set-based identifier generation的方法。
Specifically, for the user's historical interactions, tokens associated with an identifier are treated as independent from each other (\eg CF embedding cannot attend to semantic embeddings). 
However, these tokens can still attend to all tokens in previously interacted items (\eg a fully attended mask is applied to football when calculating self-attention for tokens in basketball).
Therefore, the sparse attention mask ensures the order agnosticism of the set identifier. 
% In addition, it also improves inference efficiency by reducing the duplicate self-attention calculations for the user's historical interactions $\bm{x}$ via original attention mask (Figure~\ref{fig:sparse_attn}(a)). 
% Based on the simultaneous decoding, each token can be generated independently at a single LLM call via batch decoding (Eq. (\ref{eqn:query_decoding})). 
% Nonetheless, this will cause repetitive self-attention computations for the user's historical interactions $\bm{x}$, thus leading to inefficiency. 

\vspace{2pt}
\noindent$\bullet\quad$\textbf{Time Complexity Analysis.} 
Moreover, the sparse attention mask can improve the generation efficiency by reducing the duplicate computations of the shared prefix via original attention mask (Figure~\ref{fig:sparse_attn}(a)). 
With $M$ information dimensions and $L$ historically interacted items, the time complexity for batch generation with the original attention mask is $M^3L^2d$. 
Remarkably, based on the flattened input with our proposed sparse attention mask, the time complexity reduces to $M^2L^2d$. 

% \noindent$\bullet\quad$\textbf{Identifier Order Invariance.} 

% The sparse attention mask not only accelerate the simultaneous decoding, 
% but also ensures the order invariance

% \noindent$\bullet\quad$\textbf{Order Invariance Analysis}

% 这里主要主张两件事 1. 提高效率。2. 这样子弄完了之后,是order-agnostic的。
% (可能写一个order-agnostic的分析) - 得参考一下这种equivariance怎么写



\subsection{Instantiation}
% 把SETRec instantiate到LLM上,我们通过一个overall loss来训练LLM和tokenizer:
To instantiate SETRec on LLMs, we optimize the CF and semantic tokenizers, learnable query vectors, and LLMs by minimizing: 
% 公式 - overall loss = L_AE + L_Gen
\begin{equation}\label{eqn:overall_loss}
    \mathcal{L} = \mathcal{L}_{\text{Gen}} + \alpha \mathcal{L}_{\text{AE}},
\end{equation}
where $\alpha$ is a hyper-parameter to control the strength of the tokenizer training. 
% 然后是inference的时候。我们利用tokenizer将所有item 进行tokenize,然后
During inference, SETRec first tokenizes all items into set identifiers and obtain token corpus $\mathcal{Z}$ for each information dimension. 
Then, to recommend item, SETRec transforms user history into identifier sequence and performs query-guided simultaneous generation with sparse attention mask via Eq. (\ref{eqn:query_decoding}) to generate tokens for all information dimensions.  
Finally, SETRec leverages token corpus as extendable grounding heads to ground the generated token set to the valid items via Eq. (\ref{eqn:single_logits}). 

% In this work, we instantiate SETRec on two representative generative language models with different architectures, \ie T5 and Qwen (refer to Section~\ref{sec:experiment}). 
\section{Experiment}\label{sec:experiment}
We carry out extensive experiments on four real-world datasets to answer the following research questions: 
\begin{itemize}[leftmargin=*]
    % \item \textbf{RQ1:} How does our proposed DEALRec perform compared to the coreset selection baselines for LLM-based recommendation and the models trained with full data? 
    \item \textbf{RQ1:} How does our proposed SETRec perform compared to different identifier baselines on different architectures of LLMs? 
    % \item \textbf{RQ2:} How do the different components of DEALRec (\ie influence score, gap regularization, and stratified sampling) affect the performance, and is DEALRec generalizable to different surrogate models? 
    \item \textbf{RQ2:} How do the different components of SETRec (\ie CF embeddings, semantic embeddings, query vectors, and sparse attention) affect the performance?
    \item \textbf{RQ3:} How does SETRec perform when scaling up the model size and how does SETRec improve the overall performance? 
    \item \textbf{RQ4:} How does SETRec perform with different number of semantic embeddings, tokenizer training strength, and semantic strength for inference? 
\end{itemize}
\subsection{Experimental Settings}
\subsubsection{\textbf{Datasets}}
We conduct experiments on four real-world datasets across various domains. 
From Amazon review datasets\footnote{\url{https://jmcauley.ucsd.edu/data/amazon/}.}, we adopt three widely used benchmarks 
1)\textbf{Toys}, 2) \textbf{Beauty}, and 3) \textbf{Sports}. 
The three Amazon datasets contain rich user interactions over a specific category of e-commerce products, where each item is associated with rich textual meta information such as title, description, category, and brand. 
In addition, we use a video games dataset 4) \textbf{Steam}\footnote{\url{https://github.com/kang205/SASRec}.} proposed in~\cite{kang2018self}, which contains substantial user interactions on video games with abundant textual semantic information. 
For all datasets, we follow previous work~\cite{wang2023causal} to sort user interactions chronologically according to the timestamps and divide them into training, validation, and testing sets with a ratio of 8:1:1. 
In addition, we divide the items into warm and cold items\footnote{We denote warm- and cold-start items as warm and cold items for brevity.}, where the items that appear in the training set are warm items, otherwise cold items. 


\noindent$\bullet\quad$\textbf{Evaluation.} 
We adopt the widely used metrics Recall@$K$ and NDCG@$K$, where $K=5$ and $10$ to evaluate all methods. 
Additionally, 
we introduce three different settings that evaluate over 1) all items, 2) warm items only, and 3) cold items only, respectively.  
% todo: 这里数据集可能要解释一下xxx为了保证cold数量能多一点,切割的比例是xxx


% Please add the following required packages to your document preamble:
% \usepackage{multirow}
% \usepackage[normalem]{ulem}
% \useunder{\uline}{\ul}{}
\begin{table*}[t]
\setlength{\abovecaptionskip}{0.05cm}
\setlength{\belowcaptionskip}{0.2cm}
\caption{Overall performance of baselines and SETRec instantiated on T5. The best results are in bold and the second-best results are underlined. $*$ implies the improvements over the second-best results are statistically significant ($p$-value < 0.01) under one-sample t-tests. ``Inf. Time'' denotes the inference time over all test users tested on a single NVIDIA RTX A5000 GPU.}
\setlength{\tabcolsep}{2mm}{
\resizebox{\textwidth}{!}{
\begin{tabular}{l|l|cccc|cccc|cccc|c}
\toprule
 &  & \multicolumn{4}{c|}{\textbf{All}} & \multicolumn{4}{c|}{\textbf{Warm}} & \multicolumn{4}{c|}{\textbf{Cold}} & \multicolumn{1}{l}{\textbf{Inf. Time (s)}} \\ \hline
\textbf{Dataset} & \textbf{Method} & \textbf{R@5} & \textbf{R@10} & \textbf{N@5} & \textbf{N@10} & \textbf{R@5} & \textbf{R@10} & \textbf{N@5} & \textbf{N@10} & \textbf{R@5} & \textbf{R@10} & \textbf{N@5} & \textbf{N@10} & \textbf{All Users} \\ \midrule
\multirow{9}{*}{\textbf{Toys}} & \textbf{DreamRec} & 0.0020 & 0.0027 & 0.0015 & 0.0018 & 0.0027 & 0.0039 & 0.0020 & 0.0024 & 0.0066 & 0.0168 & 0.0045 & 0.0082 & 912 \\
 & \textbf{E4SRec} & 0.0061 & 0.0098 & 0.0051 & 0.0064 & 0.0081 & 0.0128 & 0.0065 & 0.0082 & 0.0065 & 0.0122 & 0.0056 & 0.0078 & \textbf{55} \\ \cmidrule{2-15}
 & \textbf{BIGRec} & 0.0008 & 0.0013 & 0.0007 & 0.0009 & 0.0014 & 0.0019 & 0.0011 & 0.0013 & 0.0278 & 0.0360 & 0.0196 & 0.0223 & 2,079 \\
 & \textbf{IDGenRec} & 0.0063 & 0.0110 & 0.0052 & 0.0069 & 0.0109 & {\ul 0.0161} & 0.0081 & {0.0102} & {\ul 0.0318} & {\ul 0.0589} & {\ul 0.0236} & {\ul 0.0335} & 658 \\
 & \textbf{CID} & 0.0044 & 0.0082 & 0.0040 & 0.0053 & 0.0065 & 0.0128 & 0.0049 & 0.0071 & 0.0059 & 0.0111 & 0.0047 & 0.0066 & 810 \\
 & \textbf{SemID} & 0.0071 & 0.0108 & 0.0061 & 0.0074 & 0.0086 & 0.0153 & 0.0075 & 0.0100 & 0.0307 & 0.0507 & 0.0220 & 0.0292 & 1,215 \\
 & \textbf{TIGER} & 0.0064 & 0.0106 & 0.0060 & 0.0076 & 0.0091 & 0.0147 & 0.0080 & {\ul 0.0102} & 0.0315 & 0.0555 & 0.0228 & 0.0314 & 448 \\
 & \textbf{LETTER} & {\ul 0.0081} & {\ul 0.0117} & {\ul 0.0064} & {\ul 0.0077} & {\ul 0.0109} & 0.0155 & {\ul 0.0083} & 0.0101 & 0.0183 & 0.0395 & 0.0115 & 0.0190 & 448 \\  \cmidrule{2-15}
 & \cellcolor{gray!16}\textbf{SETRec} & \cellcolor{gray!16}\textbf{0.0110*} & \cellcolor{gray!16}\textbf{0.0189*} & \cellcolor{gray!16}\textbf{0.0089*} & \cellcolor{gray!16}\textbf{0.0118*} & \cellcolor{gray!16}\textbf{0.0139*} & \cellcolor{gray!16}\textbf{0.0236*} & \cellcolor{gray!16}\textbf{0.0112*} & \cellcolor{gray!16}\textbf{0.0147*} & \cellcolor{gray!16}\textbf{0.0443*} & \cellcolor{gray!16}\textbf{0.0812*} & \cellcolor{gray!16}\textbf{0.0310*} & \cellcolor{gray!16}\textbf{0.0445*} & \cellcolor{gray!16}{\ul 60} \\ \midrule\midrule
\multirow{9}{*}{\textbf{Beauty}} & \textbf{DreamRec} & 0.0012 & 0.0025 & 0.0013 & 0.0017 & 0.0016 & 0.0028 & 0.0016 & 0.0019 & 0.0078 & 0.0161 & 0.0065 & 0.0094 & 1,102 \\
 & \textbf{E4SRec} & 0.0061 & 0.0092 & 0.0052 & 0.0063 & 0.0080 & 0.0121 & 0.0067 & 0.0082 & 0.0072 & 0.0118 & 0.0065 & 0.0077 & \textbf{120} \\ \cmidrule{2-15}
 & \textbf{BIGRec} & 0.0054 & 0.0064 & 0.0051 & 0.0054 & 0.0008 & 0.0009 & 0.0006 & 0.0008 & 0.0106 & 0.0251 & 0.0095 & 0.0151 & 4,544 \\
 & \textbf{IDGenRec} & {\ul 0.0080} & 0.0115 & {\ul 0.0066} & {0.0078} & {\ul 0.0106} & 0.0165 & 0.0078 & 0.0099 & 0.0187 & 0.0350 & 0.0186 & 0.0224 & 840 \\
 & \textbf{CID} & 0.0071 & 0.0125 & 0.0060 & {\ul 0.0080} & 0.0098 & {0.0166} & 0.0077 & 0.0101 & 0.0087 & 0.0183 & 0.0071 & 0.0104 & 815 \\
 & \textbf{SemID} & 0.0071 & {\ul 0.0131} & 0.0056 & {0.0078} & 0.0098 & {\ul 0.0174} & 0.0074 & {\ul 0.0103} & {\ul 0.0260} & {\ul 0.0465} & 0.0178 & 0.0255 & 1,310 \\
 & \textbf{TIGER} & 0.0063 & 0.0098 & 0.0050 & 0.0062 & 0.0086 & 0.0131 & 0.0065 & 0.0082 & 0.0190 & 0.0325 & 0.0130 & 0.0178 & 430 \\ 
 & \textbf{LETTER} & 0.0071 & 0.0103 & 0.0061 & 0.0070 & 0.0094 & 0.0135 & {\ul 0.0079} & 0.0091 & 0.0251 & 0.0410 & {\ul 0.0241} & {\ul 0.0285} & 430 \\ \cmidrule{2-15}
 & \cellcolor{gray!16}\textbf{SETRec} & \cellcolor{gray!16}\textbf{0.0106*} & \cellcolor{gray!16}\textbf{0.0161*} & \cellcolor{gray!16}\textbf{0.0083*} & \cellcolor{gray!16}\textbf{0.0103*} & \cellcolor{gray!16}\textbf{0.0139*} & \cellcolor{gray!16}\textbf{0.0212*} & \cellcolor{gray!16}\textbf{0.0108*} & \cellcolor{gray!16}\textbf{0.0134*} & \cellcolor{gray!16}\textbf{0.0384*} & \cellcolor{gray!16}\textbf{0.0761*} & \cellcolor{gray!16}\textbf{0.0280*} & \cellcolor{gray!16}\textbf{0.0413*} & \cellcolor{gray!16}{\ul 126} \\ \midrule\midrule
\multirow{9}{*}{\textbf{Sports}} & \textbf{DreamRec} & 0.0027 & 0.0044 & 0.0025 & 0.0031 & 0.0032 & 0.0052 & 0.0028 & 0.0035 & 0.0045 & 0.0108 & 0.0026 & 0.0049 & 2,100 \\ 
 & \textbf{E4SRec} & 0.0079 & 0.0131 & 0.0075 & 0.0094 & 0.0092 & 0.0154 & 0.0085 & 0.0107 & 0.0031 & 0.0093 & 0.0019 & 0.0039 & \textbf{117} \\ \cmidrule{2-15}
 & \textbf{BIGRec} & 0.0033 & 0.0042 & 0.0030 & 0.0033 & 0.0001 & 0.0002 & 0.0001 & 0.0001 & 0.0059 & 0.0104 & 0.0043 & 0.0061 & 7,822 \\
 & \textbf{IDGenRec} & 0.0087 & 0.0127 & 0.0079 & 0.0092 & 0.0101 & 0.0149 & 0.0091 & 0.0107 & 0.0181 & 0.0302 & 0.0134 & 0.0179 & 1,724 \\
 & \textbf{CID} & 0.0077 & 0.0131 & 0.0073 & 0.0092 & 0.0074 & 0.0119 & 0.0045 & 0.0061 & 0.0082 & 0.0149 & 0.0075 & 0.0099 & 2,135 \\
 & \textbf{SemID} & {\ul 0.0094} & {\ul 0.0167} & {\ul 0.0088} & {\ul 0.0114} & {\ul 0.0119} & {\ul 0.0201} & {\ul 0.0104} & {\ul 0.0135} & {\ul 0.0254} & {\ul 0.0495} & {\ul 0.0175} & {\ul 0.0256} & 2,367 \\
 & \textbf{TIGER} & 0.0085 & 0.0129 & 0.0080 & 0.0095 & 0.0100 & 0.0151 & 0.0091 & 0.0109 & 0.0190 & 0.0310 & 0.0120 & 0.0159 & 481 \\
 & \textbf{LETTER} & 0.0077 & 0.0131 & 0.0073 & 0.0092 & 0.0074 & 0.0119 & 0.0045 & 0.0061 & 0.0082 & 0.0149 & 0.0075 & 0.0099 & 481 \\ \cmidrule{2-15}
 & \cellcolor{gray!16}\textbf{SETRec} & \cellcolor{gray!16}\textbf{0.0114*} & \cellcolor{gray!16}\textbf{0.0185*} & \cellcolor{gray!16}\textbf{0.0101*} & \cellcolor{gray!16}\textbf{0.0126*} & \cellcolor{gray!16}\textbf{0.0134*} & \cellcolor{gray!16}\textbf{0.0216*} & \cellcolor{gray!16}\textbf{0.0115*} & \cellcolor{gray!16}\textbf{0.0144*} & \cellcolor{gray!16}\textbf{0.0341*} & \cellcolor{gray!16}\textbf{0.0595*} & \cellcolor{gray!16}\textbf{0.0233*} & \cellcolor{gray!16}\textbf{0.0323*} & \cellcolor{gray!16}{\ul 136} \\ \midrule\midrule
\multirow{9}{*}{\textbf{Steam}} & \textbf{DreamRec} & 0.0029 & 0.0057 & 0.0037 & 0.0046 & 0.0042 & 0.0080 & 0.0045 & 0.0059 & 0.0017 & 0.0029 & 0.0013 & 0.0018 & 4,620 \\
 & \textbf{E4SRec} & 0.0194 & 0.0351 & 0.0220 & 0.0270 & 0.0312 & 0.0558 & 0.0283 & 0.0370 & 0.0006 & 0.0010 & 0.0006 & 0.0007 & \textbf{328} \\ \cmidrule{2-15}
 & \textbf{BIGRec} & 0.0030 & 0.0049 & 0.0046 & 0.0049 & 0.0048 & 0.0053 & 0.0061 & 0.0053 & 0.0099 & 0.0107 & {\ul 0.0129} & 0.0127 & 5,167 \\
 & \textbf{IDGenRec} & 0.0199 & 0.0307 & 0.0241 & 0.0265 & 0.0309 & 0.0479 & 0.0311 & 0.0363 & 0.0047 & 0.0151 & 0.0039 & 0.0078 & 2,846 \\
 & \textbf{CID} & 0.0200 & {\ul 0.0360} & {\ul 0.0249} & {\ul 0.0295} & 0.0314 & {\ul 0.0566} & {\ul 0.0315} & {\ul 0.0400} & 0.0008 & 0.0021 & 0.0006 & 0.0011 & 3,194 \\
 & \textbf{SemID} & 0.0155 & 0.0278 & 0.0192 & 0.0229 & 0.0248 & 0.0443 & 0.0246 & 0.0313 & 0.0017 & 0.0027 & 0.0015 & 0.0018 & 3,605 \\
 & \textbf{TIGER} & {\ul 0.0202} & 0.0348 & 0.0244 & 0.0287 & {\ul 0.0320} & 0.0552 & 0.0314 & 0.0393 & 0.0060 & {0.0152} & 0.0044 & 0.0078 & 1,747 \\
 & \textbf{LETTER} & 0.0164 & 0.0312 & 0.0195 & 0.0244 & 0.0268 & 0.0500 & 0.0253 & 0.0336 & {\ul 0.0115} & {\ul 0.0317} & {0.0077} & {\ul 0.0157} & 1,747 \\ \cmidrule{2-15}
 & \cellcolor{gray!16}\textbf{SETRec} & \cellcolor{gray!16}\textbf{0.0216*} & \cellcolor{gray!16}\textbf{0.0383*} & \cellcolor{gray!16}\textbf{0.0254*} & \cellcolor{gray!16}\textbf{0.0308*} & \cellcolor{gray!16}\textbf{0.0339*} & \cellcolor{gray!16}\textbf{0.0591*} & \cellcolor{gray!16}\textbf{0.0326*} & \cellcolor{gray!16}\textbf{0.0414*} & \cellcolor{gray!16}\textbf{0.0313*} & \cellcolor{gray!16}\textbf{0.0572*} & \cellcolor{gray!16}\textbf{0.0248*} & \cellcolor{gray!16}\textbf{0.0342*} & \cellcolor{gray!16}{\ul 347} \\ \hline
\end{tabular}
}}
\label{tab:overall_performance}
\end{table*}


\subsubsection{\textbf{Baselines}}
We compare SETRec with competitive baselines, including single-token identifiers (DreamRec, E4SRec) and token-sequence identifiers (BIGRec, IDGenRec, CID, SemID, TIGER, LETTER). 
1) \textbf{DreamRec}~\cite{yang2024generate} is a closely related method that leverages ID embedding to represent each item and adopts a diffusion model to refine the generated ID embedding from LLMs.  
2) \textbf{E4SRec}~\cite{li2023e4srec} utilizes a pre-trained CF model to obtain ID embedding, and uses a linear projection layer to obtain the item scores efficiently. 
3) \textbf{BIGRec}~\cite{bao2023bi} adopts item titles as identifiers, where the tokens are from human vocabulary. 
4) \textbf{IDGenRec}~\cite{tan2024idgenrec} is a learnable ID generator, which aims to generate concise but informative tags from human vocabulary to represent each item. 
5) \textbf{CID}~\cite{hua2023index} leverages hierarchical clustering to obtain token sequence, which utilizes item co-occurrence matrix to obtain identifiers to ensure items with similar interactions share similar tokens. 
6) \textbf{SemID}~\cite{hua2023index} also represents items with external token sequence, which is obtained based on the hierarchical item category. 
7) \textbf{TIGER}~\cite{rajput2023recommender} leverages RQ-VAE with codebooks to quantize item semantic information into token sequence with external tokens. The identifier sequentially contains coarse-grained to fine-grained information. 
8) \textbf{LETTER}~\cite{wang2024learnable} is one of the SOTA item tokenization methods, which incorporates both semantic and CF information into the training of RQ-VAE, achieving identifiers with multi-dimensional information and improved diversity. 

\subsubsection{\textbf{Implementation Details}} 
% 我们把所有的identifier方法都instantiate到了两个不同的LLMs上,T5-small 和 Qwen上,其中我们用1.5B来测试overall performance,然后还扩展到3B和7B上去验证scalability。
% 针对tokenizer的训练,对于使用到AE的方法(TIGER, LETTER,还有我们的方法),我们统一了隐藏层在"512,256,128". 
% 对于LLM的训练,我们为所有方法设置一样的prompt as "xxx"
% 对于T5-small模型,我们是全量微调。对于Qwen模型,我们采用parameter-efficeint tuning technique LoRA~\cite{}. 并且所有实验在4块A5000上跑。
% 针对我们的方法,N的数量在{1,2,3,4,5,6}里面选,alpha在0.1,0.3,0.5,0.7,0.9里选。而inference阶段的beta则是从0-1选。
We instantiate all methods on two LLMs with different architectures, \ie T5-small~\cite{raffel2020exploring} (encoder-decoder) and Qwen2.5~\cite{yang2024qwen2} (decoder-only). 
Specifically, we adopt Qwen\footnote{We denote T5-small and Qwen2.5 as T5 and Qwen for brevity.} with different sizes, including 1.5B, 3B, and 7B, for a comprehensive evaluation. 
To ensure a fair comparison, we set the hidden layer dimensions at 512, 256, and 128 with ReLU activation for methods that adopt AE in tokenizer training, including TIGER, LETTER, and our proposed SETRec. 
For LLM training, 
we adopt the same prompt for all methods as ``What would the user be likely to purchase next after buying items {history}?;'' for a fair comparison. 
We fully fine-tune the T5 model and perform parameter-efficient fine-tuning technique LoRA~\cite{hu2021lora} for Qwen. 
All experiments are conducted on four NVIDIA RTX A5000 GPUs. 
% For SETRec, 
% we use SASRec~\cite{kang2018self} as pre-trained CF model, and utilize 
% SentenceT5 and Qwen as semantic extractors for T5 and Qwen backend LLMs, respectively. 
For SETRec, 
we select $N$, $\alpha$, and $\beta$ from $\{1,2,3,4,5,6\}$, $\{0.1,0.3,0.5,0.7,0.9\}$, and $\{0, 0.1, 0.2, 0.3, 0.4, 0.5, 0.6, 0.7, 0.8, 0.9,1.0\}$, respectively. 


\begin{table*}[t]
\setlength{\abovecaptionskip}{0.05cm}
\setlength{\belowcaptionskip}{0.2cm}
\caption{Overall performance on Qwen-1.5B over Toys and Beauty. The best results are in bold and the second-best results are underlined. ``Inf. Time'' denotes the inference time over all test users tested on a single NVIDIA RTX A5000 GPU.}
\setlength{\tabcolsep}{2mm}{
\resizebox{\textwidth}{!}{
\begin{tabular}{l|l|cccc|cccc|cccc|c}
\toprule
 &  & \multicolumn{4}{c}{\textbf{All}} & \multicolumn{4}{c}{\textbf{Warm}} & \multicolumn{4}{c}{\textbf{Cold}} & \textbf{Inf. Time(s)} \\ \hline
\textbf{Dataset} & \textbf{Method} & \textbf{R@5} & \textbf{R@10} & \textbf{N@5} & \textbf{N@10} & \textbf{R@5} & \textbf{R@10} & \textbf{N@5} & \textbf{N@10} & \textbf{R@5} & \textbf{R@10} & \textbf{N@5} & \textbf{N@10} & \textbf{All Users} \\ \midrule
\multirow{9}{*}{\textbf{Toys}} & \textbf{DreamRec} & 0.0006 & 0.0013 & 0.0005 & 0.0008 & 0.0008 & 0.0019 & 0.0007 & 0.0012 & 0.0076 & 0.0137 & 0.0052 & 0.0074 & 1,093 \\
 & \textbf{E4SRec} & 0.0065 & 0.0108 & {\ul 0.0056} & 0.0072 & 0.0089 & 0.0144 & {\ul 0.0075} & {\ul 0.0096} & 0.0084 & 0.0235 & 0.0055 & 0.0111 & \textbf{905} \\ \cmidrule{2-15} 
 & \textbf{BIGRec} & 0.0009 & 0.0016 & 0.0009 & 0.0012 & 0.0011 & 0.0013 & 0.0010 & 0.0011 & 0.0194 & 0.0311 & 0.0147 & 0.0191 & 43,304 \\
 & \textbf{IDGenRec} & 0.0030 & 0.0053 & 0.0022 & 0.0031 & 0.0043 & 0.0086 & 0.0032 & 0.0048 & 0.0189 & 0.0364 & 0.0161 & 0.0224 & 30,720 \\
 & \textbf{CID} & 0.0027 & 0.0047 & 0.0025 & 0.0033 & 0.0055 & 0.0084 & 0.0044 & 0.0056 & 0.0055 & 0.0156 & 0.0044 & 0.0081 & {27,248} \\
 & \textbf{SemID} & 0.0024 & 0.0042 & 0.0018 & 0.0024 & 0.0034 & 0.0055 & 0.0026 & 0.0034 & 0.0140 & 0.0275 & 0.0095 & 0.0143 & 32,288 \\
 & \textbf{TIGER} & {\ul 0.0068} & {\ul 0.0117} & 0.0054 & {\ul 0.0072} & {\ul 0.0094} & {\ul 0.0159} & 0.0070 & 0.0095 & {\ul 0.0384} & {\ul 0.0715} & {\ul 0.0291} & {\ul 0.0408} & {13,800} \\
 & \textbf{LETTER} & 0.0057 & 0.0093 & 0.0050 & 0.0064 & 0.0080 & 0.0126 & 0.0066 & 0.0085 & 0.0217 & 0.0416 & 0.0170 & 0.0239 & 13,800 \\ \cmidrule{2-15} 
 & \cellcolor[HTML]{ECF4FF}\textbf{SETRec} & \cellcolor[HTML]{ECF4FF}\textbf{0.0116*} & \cellcolor[HTML]{ECF4FF}\textbf{0.0188*} & \cellcolor[HTML]{ECF4FF}\textbf{0.0095*} & \cellcolor[HTML]{ECF4FF}\textbf{0.0120*} & \cellcolor[HTML]{ECF4FF}\textbf{0.0144*} & \cellcolor[HTML]{ECF4FF}\textbf{0.0236*} & \cellcolor[HTML]{ECF4FF}\textbf{0.0118*} & \cellcolor[HTML]{ECF4FF}\textbf{0.0151*} & \cellcolor[HTML]{ECF4FF}\textbf{0.0531*} & \cellcolor[HTML]{ECF4FF}\textbf{0.0883*} & \cellcolor[HTML]{ECF4FF}\textbf{0.0382*} & \cellcolor[HTML]{ECF4FF}\textbf{0.0507*} & \cellcolor[HTML]{ECF4FF}{\ul 926} \\ \midrule\midrule
\multirow{9}{*}{\textbf{Beauty}} & \textbf{DreamRec} & 0.0007 & 0.0009 & 0.0005 & 0.0005 & 0.0010 & 0.0011 & 0.0007 & 0.0007 & 0.0090 & 0.0167 & 0.0075 & 0.0103 & 1,326 \\
 & \textbf{E4SRec} & {\ul 0.0067} & {\ul 0.0109} & {\ul 0.0056} & {\ul 0.0072} & {\ul 0.0088} & {\ul 0.0146} & {\ul 0.0072} & {\ul 0.0094} & 0.0017 & 0.0071 & 0.0010 & 0.0029 & \textbf{910} \\ \cmidrule{2-15} 
 & \textbf{BIGRec} & 0.0006 & 0.0010 & 0.0006 & 0.0007 & 0.0010 & 0.0010 & 0.0008 & 0.0008 & 0.0141 & 0.0246 & 0.0094 & 0.0135 & 29,500 \\
 & \textbf{IDGenRec}  & 0.0042 & 0.0078 & 0.0030 & 0.0043 & 0.0045 & 0.0104 & 0.0033 & 0.0054 & {\ul 0.0254} & {\ul 0.0471} & {\ul 0.0207} & {\ul 0.0292} & 35,040 \\
 & \textbf{CID} & 0.0046 & 0.0077 & 0.0040 & 0.0052 & 0.0059 & 0.0107 & 0.0051 & 0.0068 & 0.0075 & 0.0155 & 0.0071 & 0.0096 & {27,792} \\
 & \textbf{SemID} & 0.0030 & 0.0045 & 0.0027 & 0.0033 & 0.0050 & 0.0076 & 0.0042 & 0.0052 & 0.0159 & 0.0227 & 0.0116 & 0.0159 & 45,160 \\
 & \textbf{TIGER} & 0.0041 & 0.0065 & 0.0032 & 0.0041 & 0.0054 & 0.0085 & 0.0042 & 0.0054 & 0.0083 & 0.0167 & 0.0064 & 0.0091 & {12,600} \\
 & \textbf{LETTER} & 0.0040 & 0.0069 & 0.0031 & 0.0042 & 0.0051 & 0.0088 & 0.0039 & 0.0054 & 0.0043 & 0.0129 & 0.0043 & 0.0071 & 12,600 \\ \cmidrule{2-15} 
 & \cellcolor[HTML]{ECF4FF}\textbf{SETRec} & \cellcolor[HTML]{ECF4FF}\textbf{0.0104*} & \cellcolor[HTML]{ECF4FF}\textbf{0.0167*} & \cellcolor[HTML]{ECF4FF}\textbf{0.0085*} & \cellcolor[HTML]{ECF4FF}\textbf{0.0108*} & \cellcolor[HTML]{ECF4FF}\textbf{0.0140*} & \cellcolor[HTML]{ECF4FF}\textbf{0.0221*} & \cellcolor[HTML]{ECF4FF}\textbf{0.0109*} & \cellcolor[HTML]{ECF4FF}\textbf{0.0141*} & \cellcolor[HTML]{ECF4FF}\textbf{0.0477*} & \cellcolor[HTML]{ECF4FF}\textbf{0.0748*} & \cellcolor[HTML]{ECF4FF}\textbf{0.0370*} & \cellcolor[HTML]{ECF4FF}\textbf{0.0464*} & \cellcolor[HTML]{ECF4FF}{\ul 1,050} \\ \bottomrule
\end{tabular}
}}
\label{tab:Overall_performance_on_Qwen}
\end{table*}



\subsection{Overall Performance (RQ1)}\label{sec:overall_performance}


\subsubsection{\textbf{Performance on T5.}} 
The performance comparison between baselines and SETRec instantiated on T5 are shown in Table~\ref{tab:overall_performance}, from which we have the following observations: 
\begin{itemize}[leftmargin=*]
    % 1. token-seq-based 整体会比单一的embedding表示好。这是因为他们利用了多token来表示丰富的item信息。针对用human vocab来表示的方法,他们能够利用上语言模型内部的pre-training知识;针对那些词表的方法,他们将信息压缩到了多个token里,让item的表示更加具有层次化。
    \item Token-sequence identifier (BIGRec, IDGenRec, CID, SemID, TIGER, LETTER) generally performs better than single-token identifier under ``all'', ``warm'', and ``cold'' settings. This is reasonable because token-sequence identifier represent each item with multiple tokens, which explicitly encode rich item information into different dimensions.   
    % 2. token-seq-based中,用codebook的比用human vocab的大部分情况要好一些。这主要是因为他们利用了hierarchy的信息,从粗粒度到细粒度,一定程度上缓解了local optima的问题。
    \item Among the token-sequence identifiers, methods with external tokens (CID, SemID, TIGER, LETTER) generally outperform those relying on human vocabulary (\eg BIGRec) under ``all'' and ``warm'' settings. 
    This is attributed to their hierarchically structured identifier, where the initial tokens represent coarse-grained semantics while subsequent tokens contain fine-grained semantics. 
    This aligns better with the autoregressive generation process, potentially alleviating the local optima issue~\cite{wang2024learnable}. 
    % 3. 分析一下在cold场景下哪些更好:对于只用cf的方法(dreamrec, e4srec, cid),他们在cold上面效果不行。而那些利用了寓意信息的方法,在cold上表现就比较优秀。但是codebook的大多数情况仍然不如human vocab的那些方法。
    \item When recommending cold items\footnote{The higher values on cold performance are due to the limited number of cold items.}, methods that merely utilize CF information (DreamRec, E4SRec, and CID) fail to give satisfying results. 
    This is not surprising since CF information depends heavily on substantial interactions for training, thereby struggling with cold items. 
    In contrast, methods that integrate semantics into identifiers (BIGRec, IDGenRec, SemID, TIGER, and LETTER) generalize better on cold-start scenarios (superior performance under ``cold'' setting). 
    Specifically, BIGRec and IDGenRec tend to have competitive performance. 
    This is reasonable because they utilize readable human vocabulary to represent each item, which better leverages rich world knowledge encoded in LLMs. 

    % 4. 我们的方法significantly/constantly超过了其他方法。在accuracy上,我们在all,warm,和cold上都显著超越。我们利用了cf的信息,让那些拥有丰富交互的warm item能够被准确的推荐;此外我们利用了多维度的semantic信息,这让我们的模型能够泛化到cold item上面去。
    \item SETRec significantly outperform all baselines under ``all'', ``warm'', and ``cold'' settings across all four datasets. 
    The superior performance is attributed to 
    % 1) the incorporation of both CF and semantic information, which ensures the items with similar interactions have similar identifiers, thus recommending warm items accurately; 
    % 2) representation of rich semantics into multiple embeddings, which encourages the identifier to contain semantics of different dimensions, thus strengthening the cold-start generalization. 
    1) the incorporation of both CF and semantic information into a set of tokens, which ensures accurate warm item recommendation and strong generalization on cold items; 
    2) order agnosticism of identifier, which removes the possibly inaccurate dependencies across different tokens associated with an identifier. 
    
    \item From the perspective of efficiency, SETRec significantly reduces the inference time costs compared to the token-sequence identifiers. 
    SETRec achieves an average 15$\times$, 11$\times$, 18$\times$, and 8$\times$ speedup on Toys, Beauty, Sports, and Steam, respectively, compared to token-sequence identifiers. 
    The high efficiency is attributed to the simultaneous generation, which generates multiple tokens at a single LLM call, unlocking the real-world deployment of LLM-based generative recommendation. 
    % from perspective of efficiency, 我们的方法显著的超越了seq-based的这些方法,实现用一个single step就能够生成
\end{itemize}


% Overall performance on Qwen
% Please add the following required packages to your document preamble:
% \usepackage{multirow}

% \noindent$\bullet\quad$\textbf{Performance on Qwen-1.5B.} 




\subsubsection{\textbf{Performance on Qwen-1.5B}}
To evaluate SETRec on decoder-only LLMs, we instantiate SETRec and all baselines on Qwen-1.5B. We present the results on Toys and Beauty\footnote{We omit the results with similar observations on other datasets to save space.} in Table~\ref{tab:Overall_performance_on_Qwen}, from which we summarize several key different observations from performance on T5 as follows: 
% observations

\begin{itemize}[leftmargin=*]
    % 1. 在qwen上和t5不同的地方是,seq-based失去了它显著的效果,我们猜测这主要是因为qwen的参数量更大,他拥有更强的预训练知识。因此难以在数据有限的情况下很快的adpat到推荐任务上。相反的,E4SRec大部分情况能有非常competitive 的performance。我们猜测这主要是因为它把之前的vocabulary head换成了新的logits,这样利于大语言模型从原来的pre-training任务上adapt到推荐任务上,通过后面那个head高效调整。  
    \item Token-sequence identifiers show limited competitiveness compared to the counterparts on T5. 
    % We suspect that this might be caused by the magnified knowledge gap between the pre-training data and the recommendation data. 
    A possible reason is that Qwen-1.5B probably contains richer knowledge within its parameters, which amplifies the knowledge gap between the pre-training and recommendation tasks,  thereby hindering its adaptation to recommendation tasks with limited interaction data.  
    Conversely, E4SRec yields competitive performance in most cases. 
    This makes sense because E4SRec removes the original vocabulary head and replaces it with an item projection head, thus facilitating effective adaption to the recommendation tasks. 
    % 2. 和t5相比,这些用human vocab的在cold上面会有比较好的performance。-> 这个符合直觉。但是词表这种反而下降了,这个也符合直觉,因为需要更多的interaction来adapt,否则生成概率会偏低
    \item BIGRec and IDGenRec outperform their T5 counterparts on cold items on Beauty. Because they represent items with human vocabulary, which can leverage the rich world knowledge within Qwen-1.5B for better generalization. 
    On the contrary, identifiers with external tokens have inferior cold performance compared to their T5 counterparts. 
    This is also reasonable since it requires extensive interaction data to train external tokens. Otherwise, it is difficult for it to generalize to cold items accurately due to the low generation probability of these external tokens. 
    % 3. 我们的方法仍然能稳定的超过baseline,. 并且稳定的比t5要好。尤其是cold上面的performance。-> 在qwen上比较好的表现验证了我们方法在不同模型架构上的泛化能力。 
    \item SETRec constantly outperforms baselines, which is consistent with the observations on T5. 
    Notably, SETRec instantiated on Qwen-1.5B steadily surpasses SETRec on T5, especially under the ``cold'' setting. 
    This validates the strong generalization ability of SETRec on different architectures of LLMs. 
    Moreover, as the LLM size increases, the efficiency improvements over the token-sequence identifiers are more significant, resulting in an average of 20$\times$ speedup across the two datasets. 
    
\end{itemize}




\subsection{In-depth Analysis}

\subsubsection{\textbf{Ablation Study (RQ2)}} 
To study the effectiveness of each component of SETRec, we separately remove semantic tokens (``w/o Sem''), 
CF token (``w/o CF'').  
In addition, we replace learnable query vectors with random frozen vectors (``w/o Query'') and 
use the original attention mask (``w/o SA''), to evaluate the effect of query vectors and the sparse attention mask, respectively. 
The results of different ablation variants on T5 and Qwen-1.5B on Toys are presented in Figure~\ref{fig:ablation} and we omit the results on other datasets with similar observations to save space. 

From the figures, we can find similar observations on T5 and Qwen that 
% t5和qwen都有的现象:
% 1. 单独移除每一个元素在all,warm, cold 上performance都下降了。这验证了每个component的有效性。
1) removing each component causes performance drops under ``all'', ``warm'', and ``cold'' settings, which validates the effectiveness of each component of SETRec. 
% 2. 一处semantic对cold的影响非常大。这也说明了semantic对于cold start item的重要性。验证了引入semantic是必要的。
2) Discarding semantic tokens drastically degrades the recommendation accuracy under ``cold'' settings. 
This demonstrates the necessity of incorporating semantics into identifiers. 
% 3. 相比于移除cf embedding,移除semantic反而会让performance下降更多。这个interesting现象我们猜测是源于我们使用了多个embedding。这个现象也和XXX里的观测一致。我们补充了只有一个semantic的实验结果在appendix)
Interestingly, 
3) removing semantic tokens leads to worse performance compared to removing CF token. 
The possible reason for this is the utilization of multiple semantic tokens to represent each item, which highlights the significance of leveraging multi-dimensional semantic information. 
This observation is also consistent with the results in~\cite{lin2024bridging}. 
% t5和qwen不一样的现象:主要是在cold上,去掉cf有时候反而有更好的cold start performance。这个可能的原因是参数量大的模型能比参数量小的模型拥有更好的语义理解。更detialed analysis of the balance between cf and semantics are provided in XXX
Nonetheless, 
4) while removing CF tokens for T5 leads to inferior performance on cold items, using CF tokens for Qwen might negatively impact on cold items. 
A possible reason is that the larger-size Qwen is better at understanding semantics due to its stronger knowledge base encoded in the parameters, making the contribution of CF less significant. 



% % ablation figures on Toys
% \begin{figure}[t]
% % \vspace{-0.2cm}
% \setlength{\abovecaptionskip}{-0.15cm}
% \setlength{\belowcaptionskip}{-0cm}
%   \centering 
%   % \hspace{-0.7in}
%   \subfigure{
%     \includegraphics[height=1.35in]{figures/ablation-toys-t5-all.pdf}} 
%   % \hspace{-0.105in}
%   \subfigure{
%     \includegraphics[height=1.35in]{figures/ablation-toys-qwen-all.pdf}} 
%   % \hspace{-0.105in}
%   \subfigure{
%     \includegraphics[height=1.35in]{figures/ablation-toys-t5-warm.pdf}} 
%   % \hspace{-0.105in}
%   \subfigure{
%     \includegraphics[height=1.35in]{figures/ablation-toys-qwen-warm.pdf}} 
%   % \hspace{-0.105in}
%   \subfigure{
%     \includegraphics[height=1.35in]{figures/ablation-toys-t5-cold.pdf}} 
%   % \hspace{-0.105in}
%   \subfigure{
%     \includegraphics[height=1.35in]{figures/ablation-toys-qwen-cold.pdf}} 
%   % \hspace{-0.105in}
% \caption{Ablation study on Toys.}
%   \label{fig:ablation}
%   % \vspace{-0.3cm}
% \end{figure}



\begin{figure}[t]
% \vspace{-0.2cm}
\setlength{\abovecaptionskip}{0.02cm}
\setlength{\belowcaptionskip}{-0.3cm}
\centering
\includegraphics[scale=1.2]{figures/ablation.pdf}
\caption{Ablation study on Toys.}
\label{fig:ablation}
\end{figure}

\subsubsection{\textbf{Item Group Analysis (RQ3)}}
To understand how SETRec improves performance, we evaluate it over items with different popularity. 
% item group是怎么划分的 - 我们根据item popularity 排序,然后分成5组到Group1-group5, (从最popular到最不popular)
We divide the items into 5 groups according to their frequencies and test the models over each group respectively. 
The performance comparison between SETRec and two competitive baselines from token-sequence identifiers (LETTER) and single-token identifiers (E4SRec) are reported in Figure~\ref{fig:group_analysis}. 
We can observe that 
% 1. 从most popular 到least popular, item group performance 是在逐渐下降的,这符合预期。因为交互越少的item,llm能够decode出来的概率就会更低,欠拟合
1) the performance gradually drops from G1 to G5. 
This makes sense since the less popular items have fewer interactions for LLMs to learn, thus leading to worse generation probabilities. 
% 2. e4srec在第一组比letter要强很多,但是随着item的popularity降低,letter慢慢超过e4srec。这也符合我们的直觉。e4srec是纯靠cf信息的,非常依赖于大量的交互来学习cf info。而letter同时利用了语义信息,会在sparse的item上面有更好的表现
Besides, 
2) E4SRec outperforms LETTER on most popular items (G1) but usually yields inferior performance on unpopular items (G2-G5). 
This is due to that E4SRec only uses CF information, which relies on substantial interactions and therefore struggle on unpopular items. 
In contrast, LETTER additionally incorporates semantics into identifiers, thus achieving better generalization on sparse items. 
% 3. 每一组里面我们的方法都稳定的超过了competitive baselines。除此之外提升的百分比是在unpopular的item上面有更强的优势。这也部分说明了我们方法的泛化能力很强。
3) SETRec consistently excels both E4SRec and LETTER over all groups. 
Notably, the improvements over sparse items are more significant, which partially explains the superiority of SETRec regarding overall performance.  



% group analysis figure
\begin{figure}[t]
\vspace{-0.2cm}
\setlength{\abovecaptionskip}{-0.15cm}
\setlength{\belowcaptionskip}{-0cm}
  \centering 
  % \hspace{-0.7in}
  \subfigure{
    \includegraphics[height=1.65in]{figures/group_analysis_R10.pdf}} 
  % \hspace{-0.105in}
  \subfigure{
    \includegraphics[height=1.65in]{figures/group_analysis_N10.pdf}} 
\caption{Performance of SETRec, LETTER, and E4SRec (T5) on item groups with different popularity on Toys.}
  \label{fig:group_analysis}
  % \vspace{-0.3cm}
\end{figure}

\subsubsection{\textbf{Scalability on Model Parameters (RQ3)}}
To investigate whether SETRec can bring continuous performance when expanding the model parameters, we test SETRec on Qwen with different model sizes (1.5B, 3B, and 7B). 
Performance comparisons between SETRec, E4SRec, and LETTER on Toys are shown in Table~\ref{tab:scaling_performance}. 
% and the results on other datasets with similar observations are omitted to save space.
From the results, we can find that 
% 1. SETRec在cold上有比较明显的scaling,这因为模型对语义理解的能力更强。这展现了在cold start上比较promising的scaling的能力
1) SETRec clearly shows continued improvements over cold-start items when the model size scales from 1.5B to 7B, demonstrating promising scalability on cold items. 
We attribute this to the continued improvements of better semantic understanding by expanding the model parameters. 
% 2. SETRec在warm上可能已经到达瓶颈了,随着参数量的提升,对cf信息的接受没有进一步的提升。这个在e4srec的结果上也可以看得出来
Nonetheless, 
2) the performance on the warm items fails to continuously improve, indicating a relatively limited scalability over warm items. 
This shows that the larger models do not necessarily lead to better CF information understanding, which can also be indicated by the limited improvements of E4SRec under ``warm'' setting. 
% 3. 对于LETTER这种利用语意的identifier方法,也面临瓶颈。主要是因为扩展词表,其实和模型内部的语义没有很好的align,继续scale模型对于cold的提升其实作用并不明显
Besides, 
3) LETTER shows weak scalability over the three settings. 
This is mainly due to the utilization of external tokens, which do not necessarily align with the pre-trained knowledge in LLMs, thus showing limited improvements by expanding the model parameters. 


% Please add the following required packages to your document preamble:
% \usepackage{multirow}
% \begin{table*}[t]
% \setlength{\abovecaptionskip}{0.05cm}
% \setlength{\belowcaptionskip}{0.2cm}
% \caption{Performance comparison between SETRec and competitive baselines with different LLM sizes on Qwen. }
% \setlength{\tabcolsep}{2.5mm}{
% \resizebox{\textwidth}{!}{
% \begin{tabular}{clccccccccccccl}
% \toprule
% \multicolumn{15}{c}{\textbf{Toys}} \\ \hline
% \multicolumn{1}{l|}{} & \multicolumn{1}{l|}{} & \multicolumn{4}{c|}{\textbf{All}} & \multicolumn{4}{c|}{\textbf{Warm}} & \multicolumn{4}{c|}{\textbf{Cold}} & \multicolumn{1}{c}{\textbf{Inference Time (s)}} \\ \hline
% \multicolumn{1}{l|}{\textbf{Model Size}} & \multicolumn{1}{l|}{\textbf{Method}} & \textbf{R@5} & \textbf{R@10} & \textbf{N@5} & \multicolumn{1}{c|}{\textbf{N@10}} & \textbf{R@5} & \textbf{R@10} & \textbf{N@5} & \multicolumn{1}{c|}{\textbf{N@10}} & \textbf{R@5} & \textbf{R@10} & \textbf{N@5} & \multicolumn{1}{c|}{\textbf{N@10}} & \multicolumn{1}{c}{\textbf{All Users}} \\ \midrule
% \multicolumn{1}{c|}{\multirow{3}{*}{\textbf{1.5B}}} & \multicolumn{1}{l|}{\textbf{LETTER}} & 0.0057 & 0.0093 & 0.005 & \multicolumn{1}{c|}{0.0064} & 0.008 & 0.0126 & 0.0066 & \multicolumn{1}{c|}{0.0085} & 0.0217 & 0.0416 & 0.017 & \multicolumn{1}{c|}{0.0239} &  \\
% \multicolumn{1}{c|}{} & \multicolumn{1}{l|}{\textbf{E4SRec}} & 0.0065 & 0.0108 & 0.0056 & \multicolumn{1}{c|}{0.0072} & 0.0089 & 0.0144 & 0.0075 & \multicolumn{1}{c|}{0.0096} & 0.0084 & 0.0235 & 0.0055 & \multicolumn{1}{c|}{0.0111} &  \\
% \multicolumn{1}{c|}{} & \multicolumn{1}{l|}{\cellcolor{gray!16}\textbf{SETRec}} & \cellcolor{gray!16}\textbf{0.0116} & \cellcolor{gray!16}\textbf{0.0188} & \cellcolor{gray!16}\textbf{0.0095} & \multicolumn{1}{c|}{\cellcolor{gray!16}\textbf{0.012}} & \cellcolor{gray!16}\textbf{0.0144} & \cellcolor{gray!16}\textbf{0.0236} & \cellcolor{gray!16}\textbf{0.0118} & \multicolumn{1}{c|}{\cellcolor{gray!16}\textbf{0.0151}} & \cellcolor{gray!16}\textbf{0.0531} & \cellcolor{gray!16}\textbf{0.0883} & \cellcolor{gray!16}\textbf{0.0382} & \multicolumn{1}{c|}{\cellcolor{gray!16}\textbf{0.0507}} &  \\ \midrule
% \multicolumn{1}{c|}{\multirow{3}{*}{\textbf{3B}}} & \multicolumn{1}{l|}{\textbf{LETTER}} & 0.0057 & 0.0109 & 0.0053 & \multicolumn{1}{c|}{0.0072} & 0.0078 & 0.0151 & 0.0069 & \multicolumn{1}{c|}{0.0097} & 0.0254 & 0.0471 & 0.0162 & \multicolumn{1}{c|}{0.0236} &  \\
% \multicolumn{1}{c|}{} & \multicolumn{1}{l|}{\textbf{E4SRec}} & 0.0062 & 0.0096 & 0.0048 & \multicolumn{1}{c|}{0.0061} & 0.0082 & 0.0129 & 0.0062 & \multicolumn{1}{c|}{0.0081} & 0.0084 & 0.0218 & 0.0053 & \multicolumn{1}{c|}{0.0103} &  \\
% \multicolumn{1}{c|}{} & \multicolumn{1}{l|}{SETRec} & \textbf{0.0118} & \textbf{0.0195} & \textbf{0.0095} & \multicolumn{1}{c|}{\textbf{0.0123}} & \textbf{0.015} & \textbf{0.0258} & \textbf{0.0119} & \multicolumn{1}{c|}{\textbf{0.0159}} & \textbf{0.065} & \textbf{0.0964} & \textbf{0.0462} & \multicolumn{1}{c|}{\textbf{0.0571}} &  \\ \midrule
% \multicolumn{1}{c|}{\multirow{3}{*}{\textbf{7B}}} & \multicolumn{1}{l|}{\textbf{LETTER}} & 0.0057 & 0.0099 & 0.0044 & \multicolumn{1}{c|}{0.0061} & 0.0078 & 0.0137 & 0.0057 & \multicolumn{1}{c|}{0.0081} & 0.0215 & 0.0406 & 0.0144 & \multicolumn{1}{c|}{0.0216} &  \\
% \multicolumn{1}{c|}{} & \multicolumn{1}{l|}{\textbf{E4SRec}} & 0.0048 & 0.0088 & 0.0041 & \multicolumn{1}{c|}{0.0057} & 0.0062 & 0.0114 & 0.0053 & \multicolumn{1}{c|}{0.0072} & 0.0064 & 0.0133 & 0.0037 & \multicolumn{1}{c|}{0.0065} &  \\
% \multicolumn{1}{c|}{} & \multicolumn{1}{l|}{\cellcolor{gray!16}\textbf{SETRec}} & \cellcolor{gray!16}\textbf{0.0107} & \cellcolor{gray!16}\textbf{0.0194} & \cellcolor{gray!16}\textbf{0.0083} & \multicolumn{1}{c|}{\cellcolor{gray!16}\textbf{0.0115}} & \cellcolor{gray!16}\textbf{0.0127} & \cellcolor{gray!16}\textbf{0.0239} & \cellcolor{gray!16}\textbf{0.01} & \multicolumn{1}{c|}{\cellcolor{gray!16}\textbf{0.014}} & \cellcolor{gray!16}\textbf{0.0632} & \cellcolor{gray!16}\textbf{0.1016} & \cellcolor{gray!16}\textbf{0.0482} & \multicolumn{1}{c|}{\cellcolor{gray!16}\textbf{0.0613}} &  \\ \bottomrule
% \end{tabular}
% }}
% \label{tab:scaling_performance}
% \end{table*}

\begin{table}[t]
\setlength{\abovecaptionskip}{0.05cm}
\setlength{\belowcaptionskip}{0.2cm}
\caption{Performance comparison between SETRec and competitive baselines with different LLM sizes on Qwen. }
\setlength{\tabcolsep}{2.2mm}{
\resizebox{0.46\textwidth}{!}{
\begin{tabular}{clcccccc}
\toprule
% \multicolumn{8}{c}{\textbf{Toys}} \\ \midrule
\multicolumn{1}{l|}{} & \multicolumn{1}{l|}{} & \multicolumn{2}{c}{\textbf{All}} & \multicolumn{2}{c}{\textbf{Warm}} & \multicolumn{2}{c}{\textbf{Cold}} \\
\multicolumn{1}{l|}{} & \multicolumn{1}{l|}{} & \textbf{R@10} & \textbf{N@10} & \textbf{R@10} & \textbf{N@10} & \textbf{R@10} & \textbf{N@10} \\ \midrule\midrule
\multicolumn{1}{c|}{\multirow{3}{*}{\textbf{1.5B}}} & \multicolumn{1}{l|}{\textbf{LETTER}} & 0.0093 & 0.0064 & 0.0126 & 0.0085 & 0.0416 & 0.0239 \\
\multicolumn{1}{c|}{} & \multicolumn{1}{l|}{\textbf{E4SRec}} & 0.0108 & 0.0072 & 0.0144 & 0.0096 & 0.0235 & 0.0111 \\
\multicolumn{1}{c|}{} & \multicolumn{1}{l|}{\cellcolor{gray!16}\textbf{SETRec}} & \cellcolor{gray!16}\textbf{0.0188} & \cellcolor{gray!16}\textbf{0.0120} & \cellcolor{gray!16}\textbf{0.0236} & \cellcolor{gray!16}\textbf{0.0151} & \cellcolor{gray!16}\textbf{0.0883} & \cellcolor{gray!16}\textbf{0.0507} \\ \midrule
\multicolumn{1}{c|}{\multirow{3}{*}{\textbf{3B}}} & \multicolumn{1}{l|}{\textbf{LETTER}} & 0.0109 & 0.0072 & 0.0151 & 0.0097 & 0.0471 & 0.0236 \\
\multicolumn{1}{c|}{} & \multicolumn{1}{l|}{\textbf{E4SRec}} & 0.0096 & 0.0061 & 0.0129 & 0.0081 & 0.0218 & 0.0103 \\
\multicolumn{1}{c|}{} & \multicolumn{1}{l|}{\cellcolor{gray!16}\textbf{SETRec}} & \cellcolor{gray!16}\textbf{0.0195} & \cellcolor{gray!16}\textbf{0.0123} & \cellcolor{gray!16}\textbf{0.0258} & \cellcolor{gray!16}\textbf{0.0159} & \cellcolor{gray!16}\textbf{0.0964} & \cellcolor{gray!16}\textbf{0.0571} \\ \midrule
\multicolumn{1}{c|}{\multirow{3}{*}{\textbf{7B}}} & \multicolumn{1}{l|}{\textbf{LETTER}} & 0.0099 & 0.0061 & 0.0137 & 0.0081 & 0.0406 & 0.0216 \\
\multicolumn{1}{c|}{} & \multicolumn{1}{l|}{\textbf{E4SRec}} & 0.0088 & 0.0057 & 0.0114 & 0.0072 & 0.0133 & 0.0065 \\
\multicolumn{1}{c|}{} & \multicolumn{1}{l|}{\cellcolor{gray!16}\textbf{SETRec}} & \cellcolor{gray!16}\textbf{0.0194} & \cellcolor{gray!16}\textbf{0.0115} & \cellcolor{gray!16}\textbf{0.0239} & \cellcolor{gray!16}\textbf{0.0140} & \cellcolor{gray!16}\textbf{0.1016} & \cellcolor{gray!16}\textbf{0.0613} \\ \bottomrule
\end{tabular}
}}
\label{tab:scaling_performance}
\end{table}

\begin{figure*}[t]
% \vspace{-0.2cm}
\setlength{\abovecaptionskip}{-0.15cm}
\setlength{\belowcaptionskip}{-0cm}
  \centering 
  \hspace{-0.105in}
  \subfigure{
    \includegraphics[height=1.4in]{figures/hyper_alpha_R_10.pdf}} 
  % \hspace{-0.105in}
  \subfigure{
    \includegraphics[height=1.4in]{figures/hyper_alpha_N_10.pdf}} 
  \subfigure{
    \includegraphics[height=1.4in]{figures/hyper_N_R_10.pdf}}
  \subfigure{
    \includegraphics[height=1.4in]{figures/hyper_N_N_10.pdf}}
\caption{Performance of SETRec (T5) with different strength of AE loss $\alpha$ and different numbers of semantic tokens $N$.}
  \label{fig:hp}
  % \vspace{-0.3cm}
\end{figure*}

\subsubsection{\textbf{Effect of Semantic Strength $\bm{\beta}$ (RQ4)}}

\begin{figure}[t]
\vspace{-0.2cm}
\setlength{\abovecaptionskip}{-0.15cm}
\setlength{\belowcaptionskip}{-0.15cm}
  \centering 
  \hspace{-0.105in}
  \subfigure{
  \includegraphics[height=1.4in]{figures/hp_beta_warm.pdf}} 
  \hspace{-0.105in}
  \subfigure{    
  \includegraphics[height=1.4in]{figures/hp_beta_cold.pdf}} 
\caption{Performance of SETRec (T5) with different strength of semantics $\beta$ for inference.}
  \label{fig:hp_beta}
  % \vspace{-0.3cm}
\end{figure}

% 1. 如果只用cf的话,performance不行(significant inferior performance of beta=0 than beta>0)。这说明当同时利用cf和sem来encode user embedding的时候,decode也需要semantic的帮助。并且在cold上的提升要明显比warm上的提升大,这也说明了对cold start item推荐时semantic引入的必要性。
% 2. 持续增大到只用semantics时在warm和cold上仍然能取得不错的performance。说明了item rich semantic信息对于warm item的推荐也是有帮助的。这可能是因为在训练的时候semantic和cf之间achieve implicit alignment,无脑全入semantic也不会让performance掉太多。蕾丝的现象也在ablation里有观察到。
To investigate how semantic information contributes to the performance during inference, we vary $\beta$ from $0$ to $1$, where $\beta=0$ indicates that only CF score is used for ranking, and $\beta=1$ ranks items based solely on semantic scores (Eq. (\ref{eqn:single_logits})).  
From the results reported in Figure~\ref{fig:hp_beta}. 
we can find that 
1) Incorporating semantic information during inference is necessary (inferior performance of $\beta=0$ than $\beta>0$, which facilitates
global ranking over multi-dimensional information and lead to strong generalization ability. 
Notably, 
2) incorporating semantic scores brings more significant improvements on cold items, underscoring the critical role of semantic information for zero-shot scenarios.
Moreover, 
3) Gradually increase $\beta$ to rely solely on semantics ($\beta=1$), SETRec maintains competitive performance on warm items, which is probably attributed to the implicit alignment between CF and semantic tokens during training. 



\subsubsection{\textbf{Hyper-parameter Sensitivity (RQ4)}}\label{sec:exp_hyper_param}
We further study the hyper-parameter sensitivity to facilitate SETRec application.

\noindent$\bullet\quad$\textbf{Effect of $\bm{\alpha}$.} 
We vary the strength of AE loss $\alpha$ for SETRec training and present the results on Toys in Figure~\ref{fig:hp}(a-b). 
We can observe that 
% 1. alpha 从0慢慢增大,performance提升。这合理因为tokenizer肯定是需要随着llm一起优化的。
1) the performance is overall improved when $\alpha$ is increased from $0$ to $0.7$, which validates the effectiveness of reconstruction loss that encourages AE to preserve useful information in the latent space. 
% 2. 但是tokenizer的权重不能太高,和llm的loss一起做multi-task training的话,可能会导致模型偏向tokenizer更多,而这可能反过来影响llm,使他推荐能力学的差。(这个理由再想想,现在不太行)
Nonetheless, 
2) while continuously increasing $\alpha$ generally gives better performance on cold-start items, it might hurt the performance under ``warm'' setting. 
Based on the empirical results, we recommend setting $\alpha$ ranging from $0.5$ to $0.7$. 






\noindent$\bullet\quad$\textbf{Effect of $\bm{N}$.} 
We change the number of semantic tokens from $1$ to $6$ to investigate how $N$ affects the performance. 
From the results shown in Figure~\ref{fig:hp}(c-d), we can find that 
% 1. N提高,有提升。说明多侧面语义信息是有用的。
1) gradually increasing semantic tokens generally improves the performance, which validates the effectiveness of incorporating multiple tokens to mitigate the potential information conflicts~\cite{wang2024learnable} and embedding collapse issue~\cite{guoembedding}. 
% 2. 但是继续提升,反而会有所下降。这主要是因为XXX?
However, 
2) blindly increasing the number of semantic tokens might hurt the performance (decreased performance from $N=4$ to $N=6$). 
This is reasonable since it is non-trivial to recover the category-level preference aligning well with the real-world scenarios. 
Similar observations are also seen in~\cite{lin2024disentangled} and~\cite{lin2024temporally}. 





% \begin{figure}[t]
% % \vspace{-0.2cm}
% \setlength{\abovecaptionskip}{-0.15cm}
% \setlength{\belowcaptionskip}{-0cm}
%   \centering 
%   \hspace{-0.105in}
%   \subfigure{
%     \includegraphics[height=1.4in]{figures/hyper_N_R@10.pdf}} 
%   % \hspace{-0.105in}
%   \subfigure{
%     \includegraphics[height=1.4in]{figures/hyper_N_N@10.pdf}} 
% \caption{Performance of SETRec (T5) with different number of semantic embeddings on Toys.}
%   \label{fig:hp_N}
%   % \vspace{-0.3cm}
% \end{figure}


\section{Related Work}
Behavior control for Humanoid robots is a long-standing problem, initially explored with simplified humanoid agent \citep{tunyasuvunakool2020dm_control} and recently with full-size humanoid robot \citep{zhuang2024humanoid,fu2024humanplus} such as Unitree H1.
Humanoid robots are of particular interest to the reinforcement learning community because of the high-dimensional action space \citep{merel2017learning, hansen2022temporal, hansen2023td, hansen2024hierarchical}.
To overcome the challenges of exploration in high-dimensional action spaces, some algorithms learn policies by imitating human behavior \citep{fu2024humanplus} or enhance exploration through massive parallelization \citep{zhuang2024humanoid}.
In contrast, our proposed algorithm attempts to learn from scratch without the aid of massive parallelization \citep{makoviychuk2021isaac}. 
We have extensively evaluated our algorithm on the HumanoidBench \citep{sferrazza2024humanoidbench}, a benchmark built on humanoid robot with dexterous hands \citep{menagerie2022github} that contains not only  14 locomotion tasks but also  17 whole-body manipulation tasks.
In the LocoMujoco \citep{al2023locomujoco}, the H1 robot is not equipped with dexterous hands and only focus on locomotion tasks.

Confronted with tasks involving high-dimensional action spaces, model-based RL algorithms \citep{ha2018recurrent, hansen2022temporal, hafner2023mastering, hafner2019dream} often prove to be more sample-efficient compared to model-free alternatives \citep{haarnoja2018soft, fujimoto2018addressing}. 
However, when it comes to humanoid robots with dexterous hands, even the SOTA model-based algorithms struggle to solve it \citep{sferrazza2024humanoidbench}. 
Our algorithm integrates the concept of imitation learning \citep{liu2023ceil, zhang2024context} with the reinforcement learning framework, introducing a loss term of behavioral cloning \citep{pomerleau1988alvinn}.  It may bear a  resemblance to the offline RL \citep{zhuang2024reinformer, fujimoto2019off} algorithm TD3+BC \citep{fujimoto2021minimalist} but our problem setting is   completely different to theirs.
Additionally, it should be noted that the SIRL framework is fundamentally an online RL paradigm that does not rely on expert data, different from IBRL \citep{hu2023imitation} or MoDem \citep{hansen2022modem}.


\section{Conclusion}

%In this paper, w
We propose a new PEFT method called DiffoRA, which enables efficient and adaptive LLM fine-tuning based on LoRA. 
Instead of adjusting every interior rank, 
%of the decomposition matrices 
%of all modules, 
we argue that adopting LoRA module-wisely is sufficient. 
To achieve this, we construct a DAM to select the modules that are most suitable and essential to fine-tune. We theoretically analyze how the DAM impacts the convergence rate and generalization capability.
%of the pre-trained model. 
Furthermore, we adopt continuous relaxation and discretization to establish DAM.
%for each task. 
To alleviate the issue of discretization discrepancy, we utilize the weight-sharing strategy for optimization. 
%We fully implement our method and t
The experimental results demonstrate that our DiffoRA works consistently better than the baselines across all benchmarks. 

% \clearpage
% \appendix
% \input{6_SI}


{
\tiny
\bibliographystyle{ACM-Reference-Format}
\balance
\bibliography{bibfile}
}

\newpage
% \appendix
% \input{6_SI}


\end{document}
\endinput

