% This must be in the first 5 lines to tell arXiv to use pdfLaTeX, which is strongly recommended.
\pdfoutput=1
% In particular, the hyperref package requires pdfLaTeX in order to break URLs across lines.

\documentclass[11pt]{article}

% Change "review" to "final" to generate the final (sometimes called camera-ready) version.
% Change to "preprint" to generate a non-anonymous version with page numbers.
\usepackage[preprint]{acl}


% D ——> 条件
    % ——> 结论变换


% Standard package includes
\usepackage{times}
\usepackage{latexsym}

% For proper rendering and hyphenation of words containing Latin characters (including in bib files)
\usepackage[T1]{fontenc}
% For Vietnamese characters
% \usepackage[T5]{fontenc}
% See https://www.latex-project.org/help/documentation/encguide.pdf for other character sets

% This assumes your files are encoded as UTF8
\usepackage[utf8]{inputenc}

% This is not strictly necessary, and may be commented out,
% but it will improve the layout of the manuscript,
% and will typically save some space.
\usepackage{microtype}

% This is also not strictly necessary, and may be commented out.
% However, it will improve the aesthetics of text in
% the typewriter font.
\usepackage{inconsolata}

%Including images in your LaTeX document requires adding
%additional package(s)
\usepackage{graphicx}

% If the title and author information does not fit in the area allocated, uncomment the following
%
%\setlength\titlebox{<dim>}
%
% and set <dim> to something 5cm or larger.

\usepackage{color, xcolor}
\usepackage{booktabs}
\usepackage{multicol}
\usepackage{multirow, makecell, caption}
\usepackage{colortbl}
\usepackage{tikz}
\usepackage{arydshln}
\usepackage{pgfplots}
\usepackage{amsmath}
\usepackage{amssymb}
\usepackage{pgfplots}
\pgfplotsset{compat=1.18}
\usepackage{tabularx}
\usepackage{enumerate}
\definecolor{darkgreen}{RGB}{0, 100, 0}

\makeatletter
\def\adl@drawiv#1#2#3{%
        \hskip.5\tabcolsep
        \xleaders#3{#2.5\@tempdimb #1{1}#2.5\@tempdimb}%
                #2\z@ plus1fil minus1fil\relax
        \hskip.5\tabcolsep}
\newcommand{\cdashlinelr}[1]{%
  \noalign{\vskip\aboverulesep
           \global\let\@dashdrawstore\adl@draw
           \global\let\adl@draw\adl@drawiv}
  \cdashline{#1}
  \noalign{\global\let\adl@draw\@dashdrawstore
           \vskip\belowrulesep}}
\makeatother

\title{Order Doesn't Matter, But Reasoning Does:\\ Training LLMs with Order-Centric Augmentation}

% Author information can be set in various styles:
% For several authors from the same institution:
% \author{Author 1 \and ... \and Author n \\
%         Address line \\ ... \\ Address line}
% if the names do not fit well on one line use
%         Author 1 \\ {\bf Author 2} \\ ... \\ {\bf Author n} \\
% For authors from different institutions:
% \author{Author 1 \\ Address line \\  ... \\ Address line
%         \And  ... \And
%         Author n \\ Address line \\ ... \\ Address line}
% To start a separate ``row'' of authors use \AND, as in
% \author{Author 1 \\ Address line \\  ... \\ Address line
%         \AND
%         Author 2 \\ Address line \\ ... \\ Address line \And
%         Author 3 \\ Address line \\ ... \\ Address line}

% \author{First Author \\
%   Affiliation / Address line 1 \\
%   Affiliation / Address line 2 \\
%   Affiliation / Address line 3 \\
%   \texttt{email@domain} \\\And
%   Second Author \\
%   Affiliation / Address line 1 \\
%   Affiliation / Address line 2 \\
%   Affiliation / Address line 3 \\
%   \texttt{email@domain} \\}

\author{
    Qianxi He\textsuperscript{1}, Qianyu He\textsuperscript{1}, Jiaqing Liang\textsuperscript{2\textdagger}, Yanghua Xiao\textsuperscript{1\textdagger}\\\textbf{Weikang Zhou\textsuperscript{3}, Zeye Sun\textsuperscript{3}, Fei Yu\textsuperscript{3}}\\
    \\
    \textsuperscript{1}Shanghai Key Laboratory of Data Science, School of Computer Science, Fudan University \\
    \textsuperscript{2}School of Data Science, Fudan University  \textsuperscript{3}Ant Group\\
    \{qxhe23, qyhe21\}@m.fudan.edu.cn, \{liangjiaqing, shawyh\}@fudan.edu.cn\\
}

% % This must be in the first 5 lines to tell arXiv to use pdfLaTeX, which is strongly recommended.
% \pdfoutput=1
% % In particular, the hyperref package requires pdfLaTeX in order to break URLs across lines.

% \documentclass[11pt]{article}

% % Change "review" to "final" to generate the final (sometimes called camera-ready) version.
% % Change to "preprint" to generate a non-anonymous version with page numbers.
% \usepackage[review]{acl}
% % Standard package includes
% \usepackage{times}
% \usepackage{latexsym}
% \usepackage{amsmath}
% \usepackage{multirow}
% \usepackage{booktabs}
% \usepackage{arydshln}
% \usepackage{colortbl}

% % For proper rendering and hyphenation of words containing Latin characters (including in bib files)
% \usepackage[T1]{fontenc}
% % For Vietnamese characters
% % \usepackage[T5]{fontenc}
% % See https://www.latex-project.org/help/documentation/encguide.pdf for other character sets
% % This assumes your files are encoded as UTF8
% \usepackage[utf8]{inputenc}

% % This is not strictly necessary, and may be commented out,
% % but it will improve the layout of the manuscript,
% % and will typically save some space.
% \usepackage{microtype}

% % This is also not strictly necessary, and may be commented out.
% % However, it will improve the aesthetics of text in
% % the typewriter font.
% \usepackage{inconsolata}

% %Including images in your LaTeX document requires adding
% %additional package(s)
% \usepackage{graphicx}

% % If the title and author information does not fit in the area allocated, uncomment the following
% %
% %\setlength\titlebox{<dim>}
% %
% % and set <dim> to something 5cm or larger.

% \title{Instructions for *ACL Proceedings}

% % Author information can be set in various styles:
% % For several authors from the same institution:
% % \author{Author 1 \and ... \and Author n \\
% %         Address line \\ ... \\ Address line}
% % if the names do not fit well on one line use
% %         Author 1 \\ {\bf Author 2} \\ ... \\ {\bf Author n} \\
% % For authors from different institutions:
% % \author{Author 1 \\ Address line \\  ... \\ Address line
% %         \And  ... \And
% %         Author n \\ Address line \\ ... \\ Address line}
% % To start a separate ``row'' of authors use \AND, as in
% % \author{Author 1 \\ Address line \\  ... \\ Address line
% %         \AND
% %         Author 2 \\ Address line \\ ... \\ Address line \And
% %         Author 3 \\ Address line \\ ... \\ Address line}

% \author{First Author \\
%   Affiliation / Address line 1 \\
%   Affiliation / Address line 2 \\
%   Affiliation / Address line 3 \\
%   \texttt{email@domain} \\\And
%   Second Author \\
%   Affiliation / Address line 1 \\
%   Affiliation / Address line 2 \\
%   Affiliation / Address line 3 \\
%   \texttt{email@domain} \\}

%\author{
%  \textbf{First Author\textsuperscript{1}},
%  \textbf{Second Author\textsuperscript{1,2}},
%  \textbf{Third T. Author\textsuperscript{1}},
%  \textbf{Fourth Author\textsuperscript{1}},
%\\
%  \textbf{Fifth Author\textsuperscript{1,2}},
%  \textbf{Sixth Author\textsuperscript{1}},
%  \textbf{Seventh Author\textsuperscript{1}},
%  \textbf{Eighth Author \textsuperscript{1,2,3,4}},
%\\
%  \textbf{Ninth Author\textsuperscript{1}},
%  \textbf{Tenth Author\textsuperscript{1}},
%  \textbf{Eleventh E. Author\textsuperscript{1,2,3,4,5}},
%  \textbf{Twelfth Author\textsuperscript{1}},
%\\
%  \textbf{Thirteenth Author\textsuperscript{3}},
%  \textbf{Fourteenth F. Author\textsuperscript{2,4}},
%  \textbf{Fifteenth Author\textsuperscript{1}},
%  \textbf{Sixteenth Author\textsuperscript{1}},
%\\
%  \textbf{Seventeenth S. Author\textsuperscript{4,5}},
%  \textbf{Eighteenth Author\textsuperscript{3,4}},
%  \textbf{Nineteenth N. Author\textsuperscript{2,5}},
%  \textbf{Twentieth Author\textsuperscript{1}}
%\\
%\\
%  \textsuperscript{1}Affiliation 1,
%  \textsuperscript{2}Affiliation 2,
%  \textsuperscript{3}Affiliation 3,
%  \textsuperscript{4}Affiliation 4,
%  \textsuperscript{5}Affiliation 5
%\\
%  \small{
%    \textbf{Correspondence:} \href{mailto:email@domain}{email@domain}
%  }
%}

\begin{document}
\maketitle
\begin{abstract}
Hypergraphs provide a superior modeling framework for representing complex multidimensional relationships in the context of real-world interactions that often occur in groups, overcoming the limitations of traditional homogeneous graphs. However, there have been few studies on hypergraph-based contrastive learning, and existing graph-based contrastive learning methods have not been able to fully exploit the high-order correlation information in hypergraphs. Here, we propose a Hypergraph Fine-grained contrastive learning (HyFi) method designed to exploit the complex high-dimensional information inherent in hypergraphs. While avoiding traditional graph augmentation methods that corrupt the hypergraph topology, the proposed method provides a simple and efficient learning augmentation function by adding noise to node features. Furthermore, we expands beyond the traditional dichotomous relationship between positive and negative samples in contrastive learning by introducing a new relationship of weak positives. It demonstrates the importance of fine-graining positive samples in contrastive learning. Therefore, HyFi is able to produce high-quality embeddings, and outperforms both supervised and unsupervised baselines in average rank on node classification across 10 datasets. Our approach effectively exploits high-dimensional hypergraph information, shows significant improvement over existing graph-based contrastive learning methods, and is efficient in terms of training speed and GPU memory cost. The source code is available at \url{https://github.com/Noverse0/HyFi.git}.


% 하이퍼그래프는 집단에서 자주 발생하는 실제 상호작용의 맥락에서 복잡한 다차원 관계를 표현하는 데 탁월한 모델링 프레임워크를 제공하여 기존의 동질적인 그래프의 한계를 극복할 수 있습니다. 하지만 하이퍼그래프 기반 대조 학습과 관련된 연구는 많지 않으며, 기존의 그래프 기반 대조 학습 방법은 하이퍼그래프의 고차 상관관계 정보를 충분히 활용하지 못했습니다. 여기서는 하이퍼그래프에 내재된 복잡한 고차원 정보를 활용하기 위해 고안된 세분화된 하이퍼그래프 대비 학습(FG-HGCL) 방법을 소개합니다. 제안된 방법은 하이퍼그래프 토폴로지를 손상시키는 기존의 그래프 증강 방법을 피하면서 노드 특징에 노이즈를 추가하여 간단하고 효율적인 학습 증강 기능을 제공합니다. 또한 공유 하이퍼에지와 공유 노드를 동질성의 지표로 사용하는 독특한 대비 학습 방식을 사용합니다. 이 방법은 쌍을 이루는 노드 관계를 4개의 세분으로 효율적으로 분류하고 고품질 임베딩을 생성하며 10개의 데이터 세트에서 노드 분류 및 클러스터링 작업에서 감독 및 비감독 기준선보다 뛰어난 성능을 보입니다. 이 접근 방식은 고차원 하이퍼그래프 정보를 효과적으로 활용하여 기존의 그래프 기반 대비 학습 방법에 비해 상당한 개선을 보여주며, 훈련 속도와 GPU 메모리 비용 측면에서 효율적입니다. 소스 코드는 https://github.com/Noverse0/FG-HGCL.git 에서 확인할 수 있습니다.

\end{abstract}




\section{Introduction}
% 1. 逻辑推理对大模型很重要
% 2. LLM对于逻辑等价变换很敏感。
% 3. 然而,现有增强LLM逻辑的方法都并没有关注到等价性 a. evaluation b. symbolic-NL映射/translation c. 避免额外的信息干扰
% 4. 逻辑推理的核心特质是independency + commutativity。independency—


% 提出逻辑推理对大模型的重要性
Large language models (LLMs) have demonstrated exceptional performance across various real-world applications~\cite{jaech2024openai,dubey2024llama,liu2024deepseek}. Logic reasoning~\cite{Cummins1991-CUMCRA-2} is essential for LLMs. It allows models to draw valid conclusions, maintain coherence, and make reliable decisions across tasks~\cite{pan2023logic,liu2023evaluating}.

% LLMs 对逻辑序列极为敏感,难以灵活适应等价的逻辑结构。当前的 LLMs 更多依赖模式记忆和惯性进行推理,而非掌握逻辑推理的结构性特征。
% 下降了多少,用数据
However, LLMs are sensitive to reasoning order and struggle with logically equivalent transformations~\cite{chen2024premise,berglund2023reversal,tarski1956logic}. First, the models are highly sensitive to the order of premises, with perturbing the order leading to up to a 40\% performance drop~\cite{chen2024premise,liu2024conciseorganizedperceptionfacilitates}. Additionally, if the testing order is reversed compared to the training order, accuracy drops drastically. For example, in the case of data involving two entities within a single factual statement, accuracy drops from 96.7\% to 0.1\% when training is left-to-right and testing is right-to-left.~\cite{berglund2023reversal,berglund2023taken,allen2023physics}. This suggests that LLMs follow a rigid logical reasoning order driven by learned patterns rather than true logical understanding.


\begin{figure}[t] 
    \centering
        \includegraphics[width=0.5\textwidth]{order_intro_new_fig.pdf}
    % \captionsetup{font={small}} 
    \caption{A logical reasoning example. Independent premises can be freely reordered, while reasoning steps must be reordered without violating dependencies.}
    \label{fig:order_intro}
\end{figure}


% 然而,现有增强LLM逻辑的方法都并没有解决等价变化很敏感的问题 a. evaluation b. symbolic-NL映射/translation c. 避免额外的信息干扰
Existing LLM logical data augmentation methods do not effectively address the sensitivity to equivalent transformations. First, many logical datasets are specifically designed for certain domains, such as specialized fields or exam questions, primarily to broaden the scope of logical reasoning data collection and application~\cite{han2022folio,liu2020logiqa,yu2020reclor}. Second, a line of work aims to enhance the model's reasoning by mapping natural language to symbolic reasoning~\cite{olausson2023linc,xu2024faithful,pan2023logic}, but it primarily provides symbolic tools for understanding logical language rather than enhancing the logical structure itself. Lastly, another augmentation method creates a ``vacuum'' world to block interference from real-world logic~\cite{saparov2022language}, but it focuses on the impact of the model’s prior experience on reasoning, without addressing the design of logical equivalence.


In fact, \textbf{commutativity} is a crucial property of logical reasoning. As established by Gödel’s completeness theorem~\cite{godel1930completeness} and Tarski’s model theory~\cite{tarski1956logic}, commutativity means that independent logical units can be freely reordered without changing the essence of the logical structure. Therefore, in logical reasoning, first, independent premises are commutative. As shown in the upper half of Fig. \ref{fig:order_intro}, different orders of premises represent equivalent problem structures. Furthermore, as demonstrated by Gentzen’s proof theory~\cite{gentzen1935proof}, reasoning steps are also commutative, provided their dependencies are intact. As shown in the lower half of Fig. \ref{fig:order_intro}, changing the order of steps without disrupting the dependencies results in an equivalent reasoning process. However, altering the order of dependent steps disrupts inference and prevents a coherent path to the correct conclusion.


% 我们提出了一种基于逻辑乱序的数据增强框架。对于条件/回答步骤进行乱序。
% order-centric亦谓之为中心的
% teach-enable……
% 所有的说法都要统一,先给一个定义
In this work, we propose an order-centric data augmentation framework that explicitly incorporates logical commutativity into LLM training. For condition order, we randomly shuffle all independent premises. For reasoning steps, we construct a structured, step-by-step reasoning process, identify step dependencies using a directed acyclic graph (DAG), and apply topological sorting to reorder reasoning steps while preserving logical dependencies. Order-centric data augmentation allows models to learn logical equivalence through commutativity, leading to a deeper understanding of logic, rather than relying solely on fixed patterns to solve problems. Our experiments show that order-centric augmentation outperforms training on datasets with a fixed logical structure, enhancing the model's overall reasoning ability and improving its performance in complex shuffled testing scenarios.

% 贡献:提出了一种基于条件乱序和回答步骤乱序的逻辑推理数据增强方法/构建了逻辑推理过程中条件与推理步骤之间的依赖关系建模机制/系统性实验,验证了所提出方法的有效性
Our contributions are summarized as follows:
(1) We propose an order-centric logic data augmentation method based on commutativity, which permutes both condition order and reasoning step order, helping models gain a deeper understanding of logical equivalence.
(2) We introduce a method that uses DAGs to model the dependencies between reasoning steps, helping to identify valid step reorderings.
(3) We conduct extensive experiments to prove the effectiveness of our approach in enhancing logical reasoning.


\section{Related Work}
This section reviews related work on analogy in education, evaluating analogy in HCI, analogy-making with LLMs, and \chirev{LLM-assisted educational systems.} 
\subsection{Analogy in Education}
Analogies help humans understand complex concepts by linking them to familiar ones, making them a valuable tool in educational contexts. 
Many studies~\cite{thagard_analogy_1992,gick_analogical_1980,gick_schema_1983, brown_analogical_1989} have investigated analogical problem solving, where students of various ages solve unfamiliar problems using well-designed analogies. 
Through observational feedback and statistical analysis, researchers have established frameworks and several guidelines for using analogies in education. 
For example, as discussed by~\cite{gick_analogical_1980}, the source of the analogy would share similar relationships with the target, yet originate from a semantically distant field. 
However, such lab studies often involve experimenters posing problems, with students merely solving them without instructional guidance~\cite{brown_analogical_1989}, which diverges from real classroom learning.
Therefore, further research~\cite{vendetti_analogical_2015,richland_analogy_2004,treagust_science_1992,oliva_teaching_2007} have investigated how teachers and students engage with analogies in classroom settings, leading to nuanced insights on the influence of students' age and background and teachers' strategies.
% For example, the age and background of students~\cite{vendetti_analogical_2015}, along with the teacher's strategies~\cite{richland_cognitive_2007}, influence how frequently and effectively analogies are used in the classroom.

Although previous studies have explored the characteristics and use of analogies in education, they have not examined those generated by LLMs, which is crucial given the growing importance of LLM-assisted education~\cite{gao2024fine, Lyu2024evaluating}.
Our work fills this gap by leveraging LLMs to generate analogies tailored to specific education needs, incorporating established characteristics from prior literature and our interviews.
We design human-subjective studies to evaluate their effectiveness in problem-solving tests and classroom environments following prior research.
% \chirev{Finally, we developed a system to examine how LLMs support educators in constructing analogies for real-world applications.}

\subsection{Evaluating Analogy in Human-Computer Interaction}
Analogy has long been studied in HCI for its effectiveness in various context, including algorithms improvement~\cite{bureaucracy2020Pääkkönen,streetlevel2019Alkhatib}, cancer communication~\cite{capturing2024hnatyshyn}, narrative framing~\cite{reelframer2024wang}, enhancing deliberation~\cite{help2024yeo}, communicating standardized effect sizes~\cite{putting2022kim}, and sensemaking of LLM responses~\cite{supporting2024gero}.

Two key research directions about analogies in HCI are for enhancing numerical comprehension through data analogy and fostering creativity.
Data analogies link abstract data to familiar concepts to improve understanding. Researchers evaluate these analogies using controlled experiments and assess effectiveness through subjective ratings like helpfulness~\cite{toput2018riederer, improving2018hullman, generating2016kim, spatharioti_using_2024, chen_beyond_2024}, estimation errors~\cite{toput2018riederer, improving2018hullman}, and correlations between model and human ratings~\cite{spatharioti_using_2024}.
Analogies also facilitate scientific discovery and design. 
In scientific discovery, evaluations involve coding analogy types~\cite{solvent2018chan, kang_augmenting_2022}, calculating similarity metrics~\cite{solvent2018chan}, and conducting think-aloud sessions with scientists~\cite{kang_augmenting_2022}. 
For creative design, analogies are assessed by novelty~\cite{searching2014yu, bilogically2023zhu, bidtrainer2024chen}, quality~\cite{distributed2014yu, bidtrainer2024chen}, relevance and domain distance~\cite{analogymining2018Gilon}, feasibility~\cite{bilogically2023zhu}, and rationality~\cite{bidtrainer2024chen}.
Recently, Ding et al.~\cite{ding_fluid_2023} explored GPT-3's capacity to augment cross-domain analogical reasoning, finding it helpful for creative problem reformulation despite the risks of harmful content.

However, there has been limited exploration of analogy search in HCI for education~\cite{kumar2015stickipedia}. 
While researchers have adopted LLMs to help students and teachers generate novel analogies~\cite{bhavya2024analego}, systematic evaluations of their effectiveness in educational settings are lacking. 
Given the unique cognitive demands of education, existing assessments~\cite{ding_fluid_2023} may not be directly applicable. 
Our work aims to address this gap and offer insights into analogy generation for education.



\subsection{Analogy-making with Language Models}
Analogy is vital for human cognition and has attracted considerable interest from the AI research community. 
Traditionally, studies on analogy-making in AI have concentrated on creating word analogies (\eg, ``king is to man as queen is to woman'') using smaller language models (LMs), e.g., BERT~\cite{devlin_bert_2019} and GPT-2~\cite{radford_language_2019} trained on specific datasets~\cite{turney_combining_2003,mikolov_linguistic_2013,boteanu_solving_2015,gladkova_analogy-based_2016,chen_e-kar_2022, yuan_analogykb_2023}.
With the advancement of LLMs~\cite{ouyang_training_2022,team_gemini_2023,touvron_llama_2023,openai_gpt-4_2023}, there has been a shift toward generating natural language analogies, \ie, free-form analogies~\cite{bhavya_analogy_2022,webb_emergent_2022,ding_fluid_2023,wijesiriwardene_analogical_2023,jiayang_storyanalogy_2023,hu_-context_2023,sultan_parallelparc_2024} and forming structural analogies~\cite{sultan_life_2022,yuan_beneath_2023}.
Researchers typically design prompts manually for free-form analogies to guide LLMs in \chifinal{generating analogies~\cite{bhavya_analogy_2022, webb_emergent_2022}.
For example, Bhavya et al.~\cite{bhavya_analogy_2022} constructed a new dataset including standard science analogies and science analogies from academics and adopted prompt engineering to ask LLMs to generate analogies. 
The results show that LLMs are sensitive to prompt design, temperature, and injected spelling errors, particularly the distinction between questions and imperative statements.
We followed their optimal prompt format for our generation process.
}
%Recent studies on structural analogies employ LLMs to identify mappings between concepts across domains based on relational structures~\cite{yuan_beneath_2023}.
\chirev{
For evaluation of the generation quality of analogy, previous studies have relied on annotators manually evaluating analogies according to established principles of analogy cognition~\cite{sultan_life_2022}. 
}
%\sout{traditional work in the AI research community focuses on manually constructing benchmarks from student exams, formatting them as question-answering tasks to test the capabilities of LMs to recognize word analogies~\cite{mikolov_linguistic_2013,boteanu_solving_2015,gladkova_analogy-based_2016,chen_e-kar_2022}.
%For example, Chen et al.~\cite{chen_e-kar_2022} collect 1,655 analogical reasoning problems sourced from publicly available Civil Service Examinations of China to evaluate the performance of LMs.}
%When addressing free-form and structural analogies, verifying the correctness automatically becomes challenging.
%Therefore,
% For example,  et al.~\cite{_life_2022} described a method where annotators categorize generated analogies into five types: Not analogy, Self-analogy (where entities and their roles are identical), Close-analogy (topics are similar and entities are from related domains), Far-analogy (covering unrelated topics with different entities) and Sub-Analogy (only a part of one entity is analogous to a part of the other).
%give the entity and relation similarity to model the good analogy is with low entity similarity and high relation similarity.
%However, much of this research has focused primarily on assessing if LLMs can produce relevant analogies through human evaluation, without fully exploring their practical use in real-world applications.

In contrast to these approaches, our study is pioneering in investigating how analogies generated by LLMs can help students understand scientific concepts. 
We analyze the characteristics of analogies in educational settings through literature reviews and interviews and incorporate them into prompts for generation. 
Then, we use LLMs to generate educational analogies and evaluate them in real tests, classroom practice, \chirev{and a practical system}.
% \chirev{We also developed a system and conducted a user study to explore the real-world value of LLMs in assisting educators with constructing analogies.}

\subsection{\chirev{LLM-assisted Educational Systems}}
% \ysy{\todo @ysy and @szk add related work from NLP domain and HCI domain}
\chirev{
With the rapid advancement of LLMs, researchers are exploring their potential to develop efficient and practical systems that support students and teachers in educational tasks~\cite{kasneci2023chatgpt,yan2024practical}. 
For students, many studies have focused on creating intelligent tutoring systems powered by LLMs. 
Examples include enabling fully autonomous self-learning pipelines to support self-regulated learning~\cite{gao2024fine} and developing and evaluating LLM-based learning assistants in classroom settings~\cite{kazemitabaar_codeaid_2024,Lyu2024evaluating}.
For teachers, several LLM-based systems are designed to effectively monitor and analyze students' learning activities~\cite{ngoon_classinsight_2024,cflow,vizgroup}.
In addition, researchers aim to assist teachers in creating diverse teaching materials, such as lesson plans~\cite{lessonplanner2024uist}, diagrammatic problems~\cite{edgeworth}, and reading quizzes~\cite{readingquizmaker}.

Our work explores a novel aspect of LLM-driven education: evaluating the effectiveness of LLMs in generating teaching analogies. 
One preliminary research has initially explored generating educational analogies~\cite{bhavya2024analego}, while its system design lacks the support of empirical evidences and fails to address teachers' needs.
Instead, we first conducted a two-stage study to gain insights and empirical evidence and identify needs for teachers and students. 
We then developed and tested a system to support teachers in creating and refining analogies for lesson preparation and discussed future integration with diverse LLM-based educational tools for various users.
% Through two studies, we examined LLM-generated analogies' impact with and without teacher intervention, which provided insights into potential applications for students' self-learning and teachers' lesson preparation. 




% For example, Abd-Alrazaq et al.~\cite{abd2023large} use LLMs to produce clinical case studies, act as virtual test subjects or patients, accelerate research, develop course plans, and provide personalized feedback and assistance.  
% Lee et al.~\cite{lee-etal-2023-peep} design a situational dialogue-based chatbot to help foreign students learn English.
}


\section{Problem Formulation}
% Problem Formulation
% Problem Defintion:D = {P, C, l} P = {P1, P2, ... Pn}——>Premise set,C——>Conclusion。I = {l1, l2, ..., ln}——>label set
In this paper, we formulate the problem of logical reasoning in a unified representation. Let \( D = \{P, C, L\} \) represent a logical reasoning problem, where \( P = \{P_1, P_2, \dots, P_n\} \) is the set of premises, \( C \) is the conclusion, and \( L \) is the label, which takes a value from a finite set, such as \( \{ \text{true}, \text{false}, \text{uncertain} \} \), indicating whether \( C \) can be logically inferred from \( P \). In step-based data augmentation, we extend the representation to include a solution \( S = \{S_1, S_2, \dots, S_m\} \), where \( S \) consists of reasoning steps that derive the conclusion from the premises. This process can be abstracted as a directed acyclic graph (DAG). Typical logical reasoning datasets only provide labels. Therefore, we construct \( S \) ourselves. The specific construction of \( S \) will be detailed in Sec. \ref{sec:Answer order Augmentation}.


\section{Method}
\subsection{Framework}
 
\method comprises three components: large protein model encodings, graph-based classification, and post-hoc subgraph explanation, as shown in Figure \ref{fig:systemOV}. 
First, we encode AA sequence data using a pre-trained large protein model, ESM-2 \cite{ESM} (see Section \ref{pLM} ), to serve as node attributes.
% Next, we construct graphs for various bio-networks where each node corresponds to a protein and edges represent known or predicted interactions. 
To address (\ul{Task-1, Hypothesis-1}),
\classifier, a classifier combining graph neural networks (GNNs) with state-space sequence modeling (Mamba), is to capture both local node-pair interactions and global \textbf{pathway-level dependencies} (see Section \ref{sec:classifier}). 
% This model is trained on the node embeddings to predict functional or pathway-level labels.
To address (\ul{Task-2, Hypothesis-2}), \explainer, an explainer method trained with \textbf{pathway-wise masks} (see Section \ref{sec:explainer}), aims to identify the most influential subgraphs.
%by masking less relevant edges and node features.
We explicitly integrate pathway-level information into both models to satisfy \ul{Hypothesis-3}.
The large protein model ensures biologically meaningful protein representations, \classifier leverages various pathway information in knowledge bases for robust classification, and \explainer highlights minimal subgraphs that drive the final predictions, offering interpretable insights into key pathways.
We evaluate \method from both machine learning and biological perspectives, as detailed in Section \ref{sec:exp}.


% an explainer method, PathExplainer trained with \textbf{pathway-wise masks} (see Section \ref{sec:explainer}) is applied to isolate the most influential substructures by masking out less relevant edges and node features. 
% The large protein model preserves biologically meaningful sequence representations, the GNN exploits graph connectivity for robust classification, and the explainer selectively highlights the minimal subgraphs that drive the final prediction, offering interpretable insights into key pathway components.



\subsection{Data Encoding for Node Attributes}
\label{pLM}
\subsubsection{Large Protein Language Model Encoding}

The ESM-2 model \cite{ESM}, pre-trained on over 60 million AA sequences with parameter scaling up to 15 billion, is employed to encode our data. 
We evaluate different parameter variants of ESM-2, and the results are presented in Table \ref{tab:classification_result}.
% This model, pre-trained on over 60 million AA sequences, aims to address the limitations of existing methods like BLAST \cite{BLAST1}, which are unable to handle AA sequences.
% We evalute different parameter variants of ESM-2 and the results have shown in Table \ref{tab:classification_result}.
Formally, each \(S^{(m)}\) is tokenized and passed through stacked Transformers.
The output is a token vector, denoted as:
$
\mathbf{h}_i = \mathbf{H}_1^{(L)},
$
where \(\mathbf{H}_1^{(L)} \in \mathbb{R}^d\) is the embedding of the first token from the \(L\)-th (last) Transformer layer, serving as data representation. 
% The input sequence for node \(i\) is represented as \(\mathbf{X}_i = [x_1, x_2, \dots, x_L]\), where \(L\) is the sequence length and \(x_j\) denotes the \(j\)-th amino acid.
% The Transformer processes \(\mathbf{X}_i\) through stacked layers of multi-head self-attention and feed-forward networks. The attention mechanism at each layer computes:
% \[
% \text{Attention}(\mathbf{Q}, \mathbf{K}, \mathbf{V}) = \text{softmax}\left(\frac{\mathbf{Q} \mathbf{K}^\top}{\sqrt{d_k}}\right) \mathbf{V},
% \]
% where \(\mathbf{Q} = \mathbf{X}_i \mathbf{W}^Q\), \(\mathbf{K} = \mathbf{X}_i \mathbf{W}^K\), and \(\mathbf{V} = \mathbf{X}_i \mathbf{W}^V\), with \(\mathbf{W}^Q\), \(\mathbf{W}^K\), and \(\mathbf{W}^V\) as learnable weight matrices.
% Formally, each AA sequence \(S^{(m)}\) is tokenized and passed through stacked Transformers.
% The output a token vector \(\mathbf{h}_i\), where the embedding of the first token serves as data representation:
% $
% \mathbf{h}_i = \mathbf{H}_1^{(L)},
% $
% where \(\mathbf{H}_1^{(L)} \in \mathbb{R}^d\) is the embedding of the first token from the \(L\)-th (last) Transformer layer. 
% This global feature vector \(\mathbf{h}_i\) encodes the protein's sequence-level information.

\subsubsection{Positional Encoding}
To address a fundamental limitation of GNNs \cite{GIN} or hybrid models \cite{GPS} to distinguish nodes with identical local structures, 
we apply a random-walk-based positional encoding (RWPE) base on a diffusion process \cite{PE}, defined as:
% \begin{align*}
$p_i = [RW_{ii}, RW_{ii}^2, \cdots, RW_{ii}^k] \in \mathbb{R}^k $
% \end{align*}
where $RW = AD^{-1}$ is the random walk operator, constructed by the adjacency matrix $A$ and degree matrix $D$. 
For each node $i$, $RW_{ii}^k$ captures the probability of returning to node $i$ after $k$ steps of random walk.
% Unlike Laplacian eigenvector-based PE (LapPE) which suffers from sign ambiguity, RWPE provides unambiguous positional information. This makes the learning process more efficient as the network does not need to learn invariance to $2^k$ possible sign combinations. 
% Experimentally, RWPE demonstrates superior performance compared to LapPE across molecular datasets.


The final node representation combines the sequence-level features from ESM-2 and the structural information from the graph. Specifically, the sequence embedding \(\mathbf{h}_i\) and the positional encoding \(\mathbf{p}_i\) are concatenated and passed through a linear layer to obtain the final representation:
$\mathbf{x}_i = \text{Linear}([\mathbf{h}_i \| \mathbf{p}_i]),$
where \([\mathbf{h}_i \| \mathbf{p}_i] \in \mathbb{R}^{d + K}\) denotes the concatenation of \(\mathbf{h}_i \in \mathbb{R}^d\) and \(\mathbf{p}_i \in \mathbb{R}^K\). The linear transformation ensures dimensionality reduction and effective integration of both global protein sequence features and local graph structural information.
This final node feature \(\mathbf{x}_i \in \mathbb{R}^d\) is optimized for downstream tasks such as graph classification.



\subsection{\classifier: Pathway Information Learning}
\label{sec:classifier}

\classifier integrates the Graph Isomorphism Network (GIN) with a novel \textbf{pathway-wise Mamba} model.
It leverages the strengths of both global selective modeling mechanisms and message-passing GNNs. 
% We propose a novel pathway learning and classification model by integrating the Graph Isomorphism Network (GIN) with a \textbf{pathway-wise Mamba} architecture.
% % model for classifying pathway networks by combining GIN with \textbf{pathway-wise Mamba}. 
% This approach leverages the strengths of both global selective modeling mechanisms and message-passing GNNs while addressing their limitations in graph-based pathway modeling. 
% Below, we detail the core components of our method.
Specifically, inspired by GPS \cite{GPS}, our model avoids early-stage information loss that could arise from using GNNs exclusively in the initial layers. 
We hence employ pathway-wise global aggregation in combination with an efficient Mamba mechanism \cite{Mamba}. 
% While GNNs struggle with issues like over-smoothing, over-squashing, and limited expressivity against the Weisfeiler-Lehman (WL) test, our model addresses these challenges by introducing pathway-wise global aggregation in combination with an efficiency Mamba mechanism \cite{Mamba}. 
At each layer, node and edge features are updated by aggregating the outputs of a pathway-wise Mamba aggregation as: 
% illustrated in Figure \ref{fig:systemOV}. 
% This process can be expressed as follows:
\begin{eqnarray}
    X^{l+1}, &=& \texttt{PathMamba}^{l} \left( X^{l}, A \right), \\
    \textrm{computed as} \ \ \ 
    X^{l+1}_L, &=& \texttt{LocalGIN}^{l} \left( X^{l}, A \right), \\
    X^{l+1}_G, &=& \texttt{GlobalMamba}^{l} \left( X^{l}, A \right), \\
    X^{l+1}, &=& 
    \texttt{MLP}^{l}\left(X^{l+1}_L + X^{l+1}_G\right),
    \label{eqn:layer_equation}
\end{eqnarray}
where $A \in \mathbb{R}^{N \times N}$ is the adjacency matrix of a graph with $N$ nodes and $E$ edges; $X^{l} \in \mathbb{R}^{N \times d}$ represents the $d$-dimensional node features at layer $l$; $\texttt{LocalGIN}^{l}$ is a GIN; $\texttt{GlobalMamba}^{l}$ is a global pathway-wise aggregation layer; and $\texttt{MLP}^{l}$ is a two-layer multilayer perceptron (MLP) used to combine local and global features.

\subsubsection{Node-wise local aggregation}
% The local graph aggregation is performed using the GIN \cite{GIN}, which 
%  can distinguish graph structures up to the expressiveness of the Weisfeiler-Lehman graph isomorphism test.
Node features are updated by aggregating information from their local neighbors. The GIN operation can be expressed as:
\begin{equation}
    X^{l+1}_L = \text{ReLU}\left( W^{l} \cdot \big( (1 + \epsilon) X^{l} + \sum_{j \in \mathcal{N}(i)} X^{l}_j \big) \right),
\end{equation}
where $\mathcal{N}(i)$ represents the set of neighbors of node $i$, $W^{l}$ is the learnable weight matrix at layer $l$, and $\epsilon$ is a trainable parameter controlling the importance of self-loops. 
This ensures a high level of expressivity for local feature aggregation.


\subsubsection{Pathway-wise global aggregation}
\label{sub:global}

To capture long-range dependencies within pathways, we propose (1) random pathway sampling and (2) sequential pathway modeling in \classifier.

\noindent\textbf{Random Pathway Sampling. }
Formally, for each node $v_i$, we randomly sample a varied, single pathway with a maximum length of $L$. The sampling process is defined as:  
\begin{equation}
    \mathcal{Q} = \left\{ \bf{q}^i \mid \bf{q}^i \sim \text{Pathway}(v_i, L), \, |\bf{q}^i| \leq L \right\}_{i=1}^N,
\end{equation}  
where $N$ is the number of nodes in the graph, and $\bf{q}^i$ represents the sampled pathway for node $v_i$. Each pathway $\bf{q}^i$ is a sequence of nodes $\{v_i, v_{i_1}, v_{i_2}, \dots, v_{i_L}\}$, sampled according to a random walk process \cite{crawl}.  
The sampling process $\text{Pathway}(v_i, L)$ involves selecting a sequence of connected nodes starting from $v_i$.
The selection of each subsequent node is determined probabilistically, guided by the graph adjacency structure.
% where the choice of each subsequent node is determined probabilistically based on the graph's adjacency structure. 
% This ensures that each node is associated with diverse pathways.
% This ensures that each node is associated with a random path, providing a structured yet diverse representation of the graph's global architecture.

\noindent\textbf{Sequential Pathway Modeling. }
The forward propagation of the Mamba layer aggregates long-range dependencies along the sampled pathways. For each sampled pathway $\bf{q}^i \in \mathcal{Q}(X^{l})$, the Mamba layer processes the pathway sequentially as:

\begin{equation}
\begin{aligned}
\Delta_t &= \tau_\Delta(f_\Delta(\mathbf{x}_t^l)), \quad
\mathbf{B}_{t} = f_B(\mathbf{x}_t^l), \quad
\mathbf{C}_t = f_C(\mathbf{x}_t^l), \\
\mathbf{h}_t^l &= (1 - \Delta_t\cdot\mathbf{D}) \mathbf{h}_{t-1}^l + \Delta_t\cdot\mathbf{B}_t\mathbf{x}_t^l \\
X^{l+1}_G &= C \cdot h^{l+1}_{L},
\end{aligned}
\label{eqn:mamba-updates}
\end{equation}
where $\mathbf{x}_t^l $ is the $t$-th input node feature matrix in pathway $\bf{q}^i$ at layer $l$.
$f_*$ are learnable projections and $\mathbf{h}_t^e$ is hidden state. $\tau_\Delta$ is the softplus function. 
The forgetting term 
$(1 - \Delta_t^e\cdot\mathbf{D})$
implements a selective mechanism analogous to synaptic decay or inhibitory processes that diminish outdated or irrelevant information. 
Conversely, the update term 
$\Delta_t^e\cdot\mathbf{B}_t^e$
mirrors gating that selectively reinforces and integrates salient new information. 
The projection $\mathbf{C}_t^e$ translates the internal state into observable outputs.
By processing each sampled pathway individually, the Mamba layer effectively aggregates information along each pathway. 
The aggregated pathway representations are then combined to form the updated node features $X^{l+1}_G$ for the next layer.


\subsubsection{Graph Classification}
The updated node features are aggregated using a max pooling to generate a graph representation. 
This representation is passed through an MLP layer for classification:
\begin{equation}
    y = \text{MLP}\big(\text{MaxPooling}\big(\{h_{v_i}\}_{i=1}^N\big)\big),
\end{equation}
where $N$ is the number of nodes, and $y$ is the predicted class label for the pathway.
The model is trained using the cross-entropy loss:
$\mathcal{L}_\text{cross-entropy} = -\sum_{i=1}^C y_i \log \hat{y}_i,
$
where $C$ is the number of classes, $y_i$ is the ground truth label, and $\hat{y}_i$ is the prediction.

% \noindent By combining GNN and pathway-wise Mamba, our method effectively captures graph structure and pathway-specific features, leading to robust pathway network classification.

\subsection{PathExplainer: Targeted Pathway Inference}\label{sec:explainer}
\explainer directly infer subgraphs to generate targeted pathways by leveraging the interpretability of \classifier. 
Vallina GNNexplainers \cite{gnnexplainer, pgexplainer}, which focus primarily on the node or edge level, often struggle to capture the global structures at the pathway level. 
In contrast, \explainer introduces a key technical novelty by \textbf{training pathway masks}, where entire pathways (i.e., sequences of connected nodes and edges) are selectively masked during training to evaluate their contributions to \classifier.


% This enables the identification of the most influential subgraphs and node features at the pathway level.

PathExplainer formalizes the identification of important subgraphs as an optimization problem. 
For a given graph \( G = (V, E) \), the explanation is defined as \( (G_S, F_S) \), where \( G_S \subseteq G \) is the subgraph and \( F_S \) represents the selected features. 
The explanation is derived by optimizing the mutual information $\mathcal{MI}(\cdot)$ between the subgraph and the model's prediction:
\begin{equation}
\max_{G_S, F} \mathcal{MI}(Y, (G_S, F)) = H(Y) - H(Y \mid G = G_S, X = F_S),
\end{equation}
where \( H(Y) \) is the entropy of the predictions and \( H(Y \mid G = G_S, X = F_S) \) is the conditional entropy given the explanation.

The optimization is approached by learning a pathway mask \( M \) for the sampled pathway's edges and nodes. 
To enhance the interpretability and biological relevance of the pathway mask, random pathways \( \mathcal{Q} \) are sampled as described in Section~\ref{sub:global}. 
For each node \( v_i \), a single random pathway \( q_i \) of length up to \( L \) is sampled. These pathways are then used to restrict the mask learning process to edges within the sampled pathways, ensuring that the learned \( M \) focuses on them. 
Specifically, the adjacency matrix \( A \) is modified based on the pathway mask \( M \) as \( A' = A \odot \sigma(M) \), where \( \sigma \) denotes the sigmoid function. Similarly, the features are masked as \( X' = X \odot \sigma(M) \).
The loss function for PathExplainer combines two components: a cross-entropy term for prediction consistency and regularization terms for sparsity:

\begin{equation}
\resizebox{\linewidth}{!}{% Ensure the content is grouped properly
    $\begin{aligned}
    \mathcal{L}_\text{mask} := -\sum_{c=1}^{C} 1[y = c] \log P_\Phi(Y = c \mid G = A', X = X') d
    + \lambda \|M\|
    \end{aligned}$
}
\end{equation}

where \( \|M\| \) encourages sparsity in the edge selection, and $\lambda$ balances the trade-off between the classification loss and the sparsity regularization.
Hence, the identified important subgraphs and node features (referring to AA sequence data) that contribute most to specific bio-networks can considered as targeted pathways.
% We evaluate \method and resulting pathways from both machine learning (Section \ref{subsec:exp1}) and biological (Section \ref{subsec:exp2}) perspectives .


% By leveraging random pathway sampling \( \mathcal{Q} \), PathExplainer ensures that the learned pathway mask \( M \) focuses on critical subgraphs that reflect the stochastic and hierarchical nature of biological pathways. This integration enhances both the interpretability and the relevance of the extracted subgraphs, providing meaningful insights into key pathway mechanisms.












% We propose PathExplainer, a novel method designed to extract critical subgraphs within pathway networks by leveraging the interpretability of PathMamba. The primary objective of PathExplainer is to identify subgraphs and node features that are most influential in determining a model's predictions for biological pathway network analysis.

% PathExplainer formalizes the identification of important subgraphs as an optimization problem. For a given pathway \( G = (V, E) \), where \( V \) represents nodes and \( E \) represents edges, the explanation is defined as \( (G_S, F_S) \), where \( G_S \subseteq G \) is the subgraph and \( F_S \) represents the selected features. The explanation is derived by optimizing the mutual information between the subgraph and the model's prediction:
% \begin{equation}
% \max_{G_S, F} \mathrm{MI}(Y, (G_S, F)) = H(Y) - H(Y \mid G = G_S, X = F_S),
% \end{equation}
% where \( H(Y) \) is the entropy of the predictions and \( H(Y \mid G = G_S, X = F_S) \) is the conditional entropy given the explanation.

% The optimization is approached by learning two masks: a structural mask \( M \) for edges and a feature mask \( F \) for node features. The graph structure is modified by applying \( M \) to the adjacency matrix \( A \) as \( A' = A \odot \sigma(M) \), where \( \sigma \) denotes the sigmoid function. Similarly, the features are masked as \( X' = X \odot \sigma(F) \).

% The loss function for PathExplainer combines two components: a cross-entropy term for prediction consistency and regularization terms for sparsity and connectivity:
% \begin{equation}
% \begin{aligned}
% \mathcal{L}_\text{mask} = -\sum_{c=1}^{C} 1[y = c] \log P_\Phi(Y = c \mid G = A', X = X') \\
% \quad + \lambda_1 \|M\|_1 
% \quad + \lambda_2 \mathrm{Laplacian}(A'),
% \end{aligned}
% \end{equation}

% where \( \|M\|_1 \) encourages sparsity in the edge selection, and the Laplacian term promotes connectivity within the extracted subgraph.

% PathExplainer is tailored for pathway networks, enabling it to uncover biologically meaningful subgraphs that correspond to key protein interactions. Experiments on pathway network datasets demonstrate its ability to provide interpretable insights into pathway mechanisms while maintaining predictive accuracy.








\section{Experiments}
% \section{Experiments and Results}
    
    
\begin{table*}[t!]
    \centering
    \small
    
    \scalebox{0.90}{
    \setlength{\tabcolsep}{1.0pt}
    \begin{tabular}{l c c c r | c c c c c c |c  c c }
    \toprule
    \multirow{1}{*}{Method} & \multirow{1}{*}{Recipe} & \multirow{1}{*}{Complexity} & \multirow{1}{*}{\# P.} & \multirow{1}{*}{\# T.P.}& MME & MMB &POPE & \multicolumn{1}{c} {SEED} & MMMU & MM-Vet& TQA & SQA-I  & \multicolumn{1}{c}{GQA} \\
    \midrule
    \rowcolor{gray!14}
    \multicolumn{14}{l}{\textbf{\textit{Encoder-based VLMs}}} \\ 
    OpenFlamingo~\cite{openflamingo} & \underline{PT, SFT}& Quadratic & 9B& 96.6\%  & - & 4.6 & - & - & - & - & 33.6 & - & - \\
    MiniGPT-4~\cite{minigpt} & \underline{PT, SFT}& Quadratic & 13B& 94.8\%  & 581.7 & 23.0 & - & - & -& 22.1 & - & - & 32.2  \\
    Qwen-VL~\cite{qwenvl} & \underline{PT, SFT}& Quadratic & 7B& 100.0\%  & - & 38.2 & - & 56.3 & - & - & 63.8 & 67.1 & 59.3\\ 
    LLaVA-Phi~\cite{llavaphi}  & \underline{PT, SFT}& Quadratic & 3B& 90.0\%  & 1335.1 & 59.8 & 85.0 & - & - & 28.9& 48.6 & 68.4 & - \\
    MobileVLM-3B~\cite{mobilevlm} & \underline{PT, SFT}& Quadratic & 3B& 90.0\%  & 1288.9 & 59.6 & 84.9 & - & - & - & 47.5 & 61.0 & 59.0  \\
    VisualRWKV~\cite{visualrwkv} & \underline{PT, SFT}&  \textbf{Linear} & 3B& 90.0\%  & 1369.2 & 59.5 & 83.1 & - & - & - & 48.7 & 65.3 & 59.6 \\
    VL-Mamba~\cite{vlmamba} & \underline{PT, SFT}&  \textbf{Linear} & 3B& 90.0\%  & 1369.6 & 57.0 & 84.4 & - & -& 32.6 & 48.9 & 65.4 & 56.2 \\
    Cobra~\cite{cobra} & \underline{PT, SFT}&  \textbf{Linear} & 3.5B& 82.6\%  & - & - & \textbf{88.4} & - & - & - & 58.2 & - & \textbf{62.3}\\
    \midrule
    \rowcolor{gray!14}
    \multicolumn{14}{l}{\textbf{\textit{Decoder-only VLMs}}} \\
    Fuyu-8B (HD)~\cite{fuyu} & \underline{PT, SFT}& Quadratic & 8B& 100.0\%  & 728.6 & 10.7 & 74.1 & - & - & 21.4 & - & - & -\\
    SOLO~\cite{solo} & \underline{PT, SFT}& Quadratic &  7B& 100.0\%   & 1001.3 & - & - & 64.4 & - & - & - & 73.3 & -   \\    
    Chameleon-7B~\cite{chameleon}  & \underline{PT, SFT}& Quadratic &  7B& 100.0\%   & 170 & 31.1 & - & 30.6 & 25.4 & 8.3 & 4.8 & 47.2 & -\\  
    EVE-7B~\cite{eve}  & \underline{PT, SFT}& Quadratic &  7B& 100.0\%  & 1217.3 & 49.5 & 83.6 & 61.3 & \underline{32.3} & 25.6& 51.9 & 63.0 & 60.8 \\
    Emu3~\cite{emu3} & \underline{PT, SFT}& Quadratic & 8B& 100.0\%  & - & 58.5 & 85.2 & \underline{68.2} & 31.6 & \underline{37.2} & \underline{64.7} & \underline{89.2} & 60.3\\
    HoVLE~\cite{hovle} & DT, PT, SFT & Quadratic & \textbf{2.6B}& 100.0\%  & \textbf{1433.5} & \textbf{71.9} & \underline{87.6} & \textbf{70.7} & \textbf{33.7} & \textbf{44.3} & \textbf{66.0} & \textbf{94.8} & \underline{60.9} \\
    \rowcolor{green!15}
    \name{} & \textbf{DT} & \textbf{Linear} & \underline{2.7B}& \underline{14.7\%}  &1303.5 & 57.2 & 85.2 & 62.9& 30.7  & 31.1 &47.7 & 79.2 & 57.4 \\
    \rowcolor{yellow!15}
    \name{} & \textbf{DT} & \underline{Hybrid} & \underline{2.7B}& \textbf{11.2\%}  & \underline{1371.1} & \underline{63.7} & 86.7 & 66.3 & \underline{32.3} & 36.9 & 55.1 & 86.9 & 59.3  \\
    
    \bottomrule
    \end{tabular}
    }
    \vspace{-1em}
    \caption{\textbf{Comparison with existing VLMs on general VLM benchmarks.} ``Recipe'' denotes the adopted training recipe. ``PT'', ``SFT'', and ``DT'' denote the pre-training, supervised fine-tuning, and distillation training, respectively. ``Complexity'' denotes the model computation complexity with respect to the number of tokens. ``\# P.'' denotes the number of total parameters. ``\# T.P.'' denotes the percentage of trainable parameters ($\frac{\text{trainable paramters}}{\text{total parameters}}$). The best performance is highlighted in \textbf{bold} and the second-best result is \underline{underlined}.}
    \label{tab:results_general}
    \end{table*}



In this section, we perform comprehensive experiments to evaluate our proposed method to answer the following research questions:
\begin{itemize}[leftmargin=*]
\item \textbf{Q1:} Can a fully discretized EEG tokenization framework outperform baseline methods that rely on continuous embeddings? How crucial is joint frequency-temporal modeling in EEG analysis and tokenization? 

% \item \textbf{Q2:} How crucial is the joint modeling of frequency and temporal features for EEG analysis? 

\item\textbf{Q2:} How effectively do our learned tokens represent EEG features and capture class-specific characteristics? Do the generated tokens capture relevant frequency information?

% How well do our learned tokens represent EEG features, and how effective are they in capturing class-specific characteristics? Do the tokens generated by our \tokenizer capture good frequency features?

% \item\textbf{Q3:} How well do our learned tokens represent EEG features, and how effective are they in capturing class-specific characteristics?

% effectively capture both frequency and temporal features,along with class-specific characteristics, through unsupervised pretraining?}
% \askFillIn{Putting token analysis on here.}

% \item \textbf{Q4:} Do the tokens generated by our \tokenizer capture good frequency features?

% \item \askFillIn{\textbf{Q4: Do the tokens generated by our \tokenizer capture good frequency features? }}\askFillIn{Can we learn good frequency features?
% How can we learn frequencies? We can provide some in-depth frequency response analysis here.}

% \item \textbf{Q3:}  Do the tokens produced by our \tokenizer capture class-specific characteristics?
\item \textbf{Q3:}  Are the tokens generated by our \tokenizer scalable, and can they enhance the performance of existing EEG foundation models such as LaBraM?
% \askFillIn{Putting scalability analysis on here.}

\item \textbf{Q4:}  Do tokens produced by \tokenizer offer interpretability, and do they correspond to distinct, recognizable EEG patterns?
\end{itemize}

% \subsection{Datasets}
\subsection{Experiment Setup}
\subsubsection{Datasets:} We conducted our evaluation on four EEG datasets:
\begin{itemize}[leftmargin=*]
\item\textbf{TUH EEG Events (TUEV)} \cite{harati2015improved}: TUEV is a subset of the TUH EEG Corpus \cite{obeid2016temple}, which comprises clinical EEG recordings collected at Temple University Hospital between 2002 and 2017. The dataset is annotated for six EEG event types: spike and sharp wave (SPSW), generalized periodic epileptiform discharges (GPED), periodic lateralized epileptiform discharges (PLED), eye movement (EYEM), artifact (ARTF), and background (BCKG). 

\item\textbf{TUH Abnormal EEG Corpus (TUAB)} \cite{lopez2015automated}: TUAB comprises EEG recordings collected at Temple University Hospital, which are labeled for normal and abnormal EEG activity. 


% TUEV contains $11,914$ recordings, yielding a total of $112,491$ annotated samples, each with a duration of 5 seconds.

% Each sample is a 5-second recording from 16 channels, derived using the 16 bipolar montage based on the international 10–20 system.

\item\textbf{IIIC Seizure} \cite{jing2023development,ge2021deep}: The IIIC Seizure dataset is curated for the detection of six distinct ictal–interictal–injury continuum (IIIC) patterns and is sourced from \cite{jing2023development,ge2021deep}. The annotations include: (1) others (OTH), (2) seizure types (ESZ), (3) lateralized periodic discharge (LPD), (4) generalized periodic discharge (GPD), (5) lateralized rhythmic delta activity (LRDA), and (6) generalized rhythmic delta activity (GRDA). 


% The dataset is from \cite{jing2023development,ge2021deep} and comprises $2,689$ recordings, yielding a total of $135,096$ samples. Each sample is a 10-second EEG segment recorded from 16 channels.

\item\textbf{CHB-MIT} \cite{shoeb2009application}: The CHB-MIT dataset is a widely used benchmark for epilepsy seizure detection. It comprises EEG recordings from 23 pediatric subjects with intractable seizures. 

% The dataset includes $686$ recordings, yielding a total of $326,993$ EEG samples, each with a duration of 10 seconds.


% It contains $2,339$ recordings, yielding a total of $409,455$ samples, each with a duration of 10 seconds.

\end{itemize}



\subsubsection{Preprocessing:} We follow the preprocessing setup of BIOT \cite{yang2024biot}. Unlike LaBraM \cite{jiang2024large}, which utilized 23 channels in the TUEV and TUAB datasets, we adhere to the 16-channel bipolar montage from the international 10–20 system, as used in \cite{yang2024biot}. All EEG recordings are resampled to 200 Hz. For TUEV and TUAB, we apply a bandpass filter ($0.1$–$75$ Hz) and a notch filter (50 Hz), following the preprocessing pipeline of LaBraM \cite{jiang2024large}. STFT computation of the signals is performed using PyTorch, with detailed parameters provided in Appendix~\ref{app:stft_params}. For training, validation, and test splits, we follow the recommendations from \cite{yang2024biot}. Additional details on dataset statistics and splits are provided in Appendix~\ref{app:dataset_splits}.


\subsubsection{Baselines and Metrics:} We evaluated our approach against the baselines from \cite{yang2024biot} as well as the current state-of-the-art methods, including BIOT \cite{yang2024biot} and LaBraM \cite{jiang2024large}. All baselines were reproduced using their respective open-source GitHub repositories. To ensure a fair comparison, our experiments follow a single-dataset setting for all the baselines. Specifically for BIOT, we conducted their proposed unsupervised pretraining followed by fine-tuning on the same dataset. Similarly, for LaBraM, we used their base model and conducted neural tokenizer training, masked EEG modeling, and fine-tuning within the same dataset. For performance evaluation, we used balanced accuracy, Cohen's Kappa coefficient, and weighted-F1 score for multi-class classification tasks, while balanced accuracy, AUC-PR, and AUROC were used for binary classification tasks. For TUAB, we used binary cross-entropy loss for fine-tuning, while the cross-entropy loss was applied to the TUEV and IIIC datasets. Given the class imbalance in the CHB-MIT dataset, we employed focal loss for all experiments. All experiments were conducted using five different random seeds, and we report the mean and standard deviation for each metric. Appendix~\ref{app:experiment_details} provides additional details on the experiment settings.



% We reproduced the baselines from \cite{yang2024biot} borrowing their codebase. Additionally we compare with the current SOTA methods including BIOT\cite{yang2024biot} and LaBraM\cite{jiang2024large}, which we reproduced using their official GitHub repositories. In our study we conducted all the experiments in single dataset setting, as our current focus is to build and evaluate a good tokenizer. For BIOT we conducted their proposed unsupervised pretraining and fine-tuning on the same dataset. Similarly, for LaBraM, we utilized their base model and conducted neural tokenizer training, masked EEG modeling and fine-tuning on the same dataset for fair comparisons. For multi-class classification experiments (TUEV, IIIC) we utilized balanced accuracy, Cohen's Kappa coefficient and weighted-F1 score as metrics. Similarly for binary classification tasks, we utilized balanced accuracy, AUC-PR and AUROC metrics. For TUAB we used binary-cross entropy loss for fine-tuning and cross-entropy loss for TUEV and IIIC datasets. Due to the imbalance nature of CHB-MIT dataset, we used focal loss for all our experiments similar to past methods\cite{yang2024biot}. We used five different random seeds for all our experiments and report the mean and standard deviation for each metrics. 



\subsection{Q1: Performance and Importance of Joint Frequency-Temporal Modeling}
% \subsection{Q1: Performance Comparison with SOTA Continuous Embedding Based Methods}
% compare the scores. 
% talk on the parameters
% talk on the neural tokanizer param and ours and even with low params we are more efficient.
\noindent{\textbf{Q1.1 - Performance Evaluation:}} 
% In this section, we evaluate the performance of our \method framework on EEG-related tasks, comparing it against state-of-the-art (SOTA) methods based on continuous embeddings. 
Table~\ref{tab:eeg_classification} presents EEG event classification results on TUEV and abnormal detection performance on TUAB. Our \method consistently outperforms all baselines across all metrics. For event-type classification on TUEV, \method achieves a $5\%$ increase in balanced accuracy over BIOT ($0.4679\rightarrow0.4943$) and LaBraM ($0.4682\rightarrow0.4943$). In Cohen's Kappa, \method improves by $9\%$ over BIOT ($0.4890\rightarrow0.5337$) and $5\%$ over LaBraM ($0.5067\rightarrow0.5337$). In abnormal detection on TUAB, \method achieves a $5\%$ improvement across all metrics compared to LaBraM. This highlights that our fully discrete tokenization approach surpasses the performance of existing continuous embedding-based approaches. 
% The results also suggest that our tokenizer captures EEG representations more effectively than the neural tokenizer used in LaBraM for model pretraining. 
Another advantage of \method is its reduced model footprint. As shown in Table~\ref{tab:eeg_classification}, \method achieves better results with significantly fewer parameters—a 3-fold reduction compared to LaBraM (5.8M $\rightarrow$ 1.9M) and a 1.5-fold reduction compared to BIOT (3.2M $\rightarrow$ 1.9M). This reduction can be attributed to discrete tokenization approach, which compresses EEG into a token sequence, thereby reducing data complexity.


\noindent{\textbf{Q1.2 - Importance of Joint Frequency and Temporal Modeling:}} To evaluate the importance of joint frequency-temporal modeling, we conducted an ablation study comparing three tokenization variants: (1) \method-Raw Signal Only (\method-R), which uses only raw EEG patches $\{x_i\}_{i=1}^N$ to predict the spectrum $\mathbf{S}$, (2) \method-STFT Only (\method-S), and (3) \method, which jointly models both temporal and frequency features. Masked modeling was applied for token learning in the latter two, with consistent \encoder training across all variants. As shown in Table~\ref{tab:eeg_classification}, all three variants outperform existing baselines. In event classification, \method-S improves Cohen’s Kappa over \method-R ($0.5194 \rightarrow 0.5275$). However, in abnormal detection, \method-R achieves a higher AUC-PR ($0.8814 \rightarrow 0.8908$). These results indicate that different EEG tasks rely on distinct feature domains, underscoring the necessity of joint modeling. The primary \method consistently outperforms both single-domain approaches across all settings, further underscoring the importance of joint modeling.


% In the latter two variants, we employ masked modeling for token learning, while the subsequent \encoder training remains consistent across all variants. As shown in Table~\ref{tab:eeg_classification}, all three variants of \method consistently outperform existing baselines.  In event classification task \method-S achieves better performance compared to \method-R ($0.5194\rightarrow0.5275$) in Cohen's kappa. However, \method-R outperforms \method-S ($0.8814\rightarrow0.8908$) in AUC-PR score in abnormal detection tasks. This suggests that for different EEG tasks essential features comes from different domains highlighting the need of joint modeling. The main variant \method outperfoms both other single domain variants across all metrics and tasks. 

% the performance of our \method framework on EEG-related tasks and compare it against existing state-of-the-art (SOTA) methods that rely on continuous embeddings. Table~\ref{tab:eeg_classification} presents the results of EEG event type classification on TUEV and abnormal detection on TUAB datasets. Our \method outperforms all baselines across all metrics consistently. For event-type classification tasks using 5-second EEG signals, \method achieves an $5\%$ increase in balanced accuracy compared to existing state of the art methods BIOT($0.4679 \rightarrow 0.4943$) and LaBraM ($0.4682\rightarrow 0.4943$). In Cohen's Kappa \method shows an increase of $9\%$ compared to BIOT ($0.4890 \rightarrow 0.5337$) and $5\%$ increase compared to LaBraM ($0.5067 \rightarrow 0.5337$). In abnormal detection tasks given a 10-second sample, \method achieves an increase of approximately $5\%$ across all metrics compared to LaBraM ($0.7720\rightarrow 0.8152$, $0.8498 \rightarrow 0.8946$, $0.8534\rightarrow 0.8897$). This comparison highlights that our fully discrete tokenization approach achieves significant improvements compared to existing continuous embedding based models such as BIOT and hybrid method which somewhat rely on discrete tokens such as LaBraM. Another key advantage of the \method is its smaller model footprint. As shown in Table~\ref{tab:eeg_classification}, our \method archives better results with significantly  less number of parameters, with 3-fold reduction compared to LaBraM(5.8M$\rightarrow$1.9M) and around 1.5 fold reduction compared to BIOT(3.2M$\rightarrow$1.9M). This reduction in model complexity can be attributed to our discrete tokenization approach, which compresses EEG data into a sequence of tokens, thereby reducing overall data complexity.



% compares various models on the TUEV and TUAB datasets. The results demonstrate that \method consistently outperforms all baselines across all metrics. 


% Our results demonstrate that TFM-Token consistently outperforms all baselines across all metrics, highlighting the effectiveness of our tokenization method in EEG classification tasks. Another key advantage of the \method is its smaller model footprint. As shown in Table~\ref{tab:eeg_classification}, our framework performs better despite using fewer parameters than most baselines. For instance, while LaBraM-Base has 5.8M parameters and BIOT has 3.2M parameters, TFM-Token achieves superior results with only 1.9M parameters. This reduction in model complexity can be attributed to our discrete tokenization approach, which compresses EEG data into a sequence of tokens, thereby reducing overall data complexity while preserving essential information. Overall, these findings validate that discretized EEG tokenization not only matches but surpasses the performance of existing continuous embedding-based approaches while offering greater computational efficiency during inference. 

% Notably, TUAB presents a particularly challenging setting where LaBraM and BIOT exhibit lower performance than other baselines despite their pretraining and fine-tuning strategies. In contrast, TFM-Token significantly outperforms these models, which indicates that our \tokenizer learns highly informative discrete tokens. 




 





% \subsection{Q2: Importance of Joint Frequency and Temporal Modeling}
% % mention the three variants
% % mention that all our variants are better than baselines
% % highlight the best performing model
% % mention that in the next section we further deeply analyze these tokens
% To evaluate the effectiveness of joint frequency-temporal modeling, we conducted an ablation study comparing three variants of our tokenization framework: (1) TFM-Token-Raw Signal Only, which uses only raw EEG patches $\{x_i\}_{i=1}^N$ to predict the STFT transform $\mathbf{S}$, (2) TFM-Token-STFT Only, and (3) TFM-Token, which jointly models both temporal and frequency features. The latter two variants employ masked modeling for token learning, while the subsequent \encoder training remains consistent across all variants.

% As shown in Table~\ref{tab:eeg_classification}, not only does the TFM-Token variant with joint modeling consistently outperform both single-domain approaches, but notably, all three variants demonstrate superior performance compared to existing baselines. These results demonstrate the complementary nature of temporal features (capturing long-range dependencies and transient events) and frequency features (detecting characteristic oscillatory patterns) in EEG analysis. These results validate our approach of incorporating both domains to learn more informative EEG tokens, which we further analyze in detail in the following Sections~\ref{sec:token_quality_analysis} and \ref{sec:freq_learning}.




\subsection{Q2: EEG Token Quality Analysis and Frequency Learning}
\label{sec:token_quality_analysis}

\begin{table}[t]
\centering
\caption{Token Utilization and class-token uniqueness comparison }
\label{tab:token_utilization}
\resizebox{\linewidth}{!}{%
\begin{tabular}{lccccc}
\toprule
\textbf{Tokenization Method} & \textbf{\# Params}& \multicolumn{2}{c}{\textbf{Utilization}} & \multicolumn{2}{c}{\textbf{Class-Token}}\\ 
 & &\multicolumn{2}{c}{\textbf{$\%$}}&\multicolumn{2}{c}{\textbf{Uniqueness (GM) $\%$}}\\ 

&& \textbf{TUEV} & \textbf{IIIC}&\textbf{TUEV} & \textbf{IIIC} \\

\midrule
Neural Tokenizer (LaBraM)     &  8.6M  & 21.13 & 15.25& 0.034 & 0.000 \\ 
\tokenizer-R     & 1.1M     & 5.29& 7.87 & 0.000 & 0.000 \\  %40.70  & 29.71 \\
\tokenizer-S         & 1.1M       & 13.93 &11.04 & 0.004 & 0.619 \\ %1.78  &  \\
\tokenizer                   & 1.2M        & 9.78 & 8.26 & 2.14 & 1.429 \\


\bottomrule

\end{tabular}
}
%\vspace{-0.3cm}
\end{table}


% \begin{figure}[t]
%     \centering
%     % \rule{0.8\linewidth}{0.5\linewidth} 
%     % \includegraphics[width=\linewidth]{Figures/Story_Overview_Fig.pdf}
%     \includegraphics[width=\linewidth]{FIG/retrieval_test_2.pdf}
%     % \includegraphics[width=\linewidth]{Figures/retrieval_test.pdf}
%     \caption{Overview of the token quality analysis study. (a) Class-Token Uniqueness Score of the tokenizers across different classes. (b) Precision scores of tokenizers in EEG is similar to the retrieval class.}
%     \label{fig:retrieval_test}
% \end{figure}
% % talk on token utilization and impact of PE.. Mention results in appendix. 
% % talk on classwise token uniqueness.. explain the experiment and how the scores are obtained.
% % talk on the data mining test. 
% In this section, we study the quality of the EEG tokens learned by our \tokenizer by analyzing three key aspects: (1) token utilization, (2) class-specific distinctiveness, and (3) semantic coherence. We conducted our analysis using all three \tokenizer variants and the neural tokenizer from LaBraM\cite{jiang2024large}, testing them on the test splits of both the TUEV and IIIC datasets, which have multi-classes. For consistency and fair comparison, all tokenizers employed a standardized vocabulary size of $8,192$ tokens.



% talk on token utilization and impact of PE.. Mention results in appendix. 
% talk on classwise token uniqueness.. explain the experiment and how the scores are obtained.
% talk on the data mining test. 
We study the quality of the EEG tokens learned by our \tokenizer by analyzing four key aspects: (1) token utilization, (2) class-specific distinctiveness, (3) similar class retrieval, and (4) frequency learning capability. We conducted our analysis using all three \tokenizer variants and the neural tokenizer from LaBraM \cite{jiang2024large}, testing them on the test splits of both the TUEV and IIIC datasets, which have multiple classes. All tokenizers employed a fixed vocabulary size of $8,192$ tokens for consistency and fair comparison.

\noindent{\textbf{Q2.1 - Token utilization and Class uniqueness:}} Token utilization ($\%$) score was calculated as the percentage of unique tokens activated from the total available vocabulary size. To quantify whether the tokenizers capture class-distinctive representations, we introduce the Class-Token Uniqueness Score, defined as:
$$
\text{Class-Token Uniqueness \%} = \frac{\text{\# Unique Tokens in Class} }{\text{\# Tokens Utilized by Class}}\times 100 %
$$
Figure~\ref{fig:retrieval_test}a visualizes the class-token uniqueness scores for each class in both datasets. A robust tokenizer should capture class-distinctive tokens across all dataset classes through unsupervised pretraining. To assess this, we computed the geometric mean (GM) of class-token uniqueness scores, as shown in Table~\ref{tab:token_utilization}. Our \tokenizer reduces token utilization by more than two-fold compared to the neural tokenizer on TUEV ($21.13\% \rightarrow 9.78\%$) and nearly two-fold on IIIC ($15.25\% \rightarrow 8.26\%$). 
It also significantly improves learning of class-unique tokens compared to neural tokenizer ($0.034\% \rightarrow 2.14\%$on TUEV, $0.0\% \rightarrow 1.429\%$ on IIIC). These results demonstrate that the \tokenizer captures more compact and useful tokens than the neural tokenizer. Additionally, \tokenizer achieves a higher class-token uniqueness score across all classes compared to \tokenizer-R ($0.0\% \rightarrow 1.429\%$ on IIIC) and \tokenizer-S  ($0.619\% \rightarrow 1.429\%$ on IIIC), as depicted in Figure~\ref{fig:retrieval_test}a. This further validates joint frequency-temporal modeling in EEG analysis.


% \tokenizer also achieves better class-token uniqueness score across all classes compared to to \tokenizer-R ($0.0\% \rightarrow 1.429\%$ on IIIC) and \tokenizer-S ($0.619\% \rightarrow 1.429\%$ on IIIC) variants, which can be observed in Figure~\ref{fig:retrieval_test}a. This also highlights the importance of joint frequency-temporal modeling. 



% Our \tokenizer shows more than two-fold reduction in token utilization compared to neural tokenizer on TUEV ($21.13\% \rightarrow 9.78\%$) and approximately two-fold reduction on IIIC($15.25\% \rightarrow 8.26\%$) datasets. Also, our \tokenizer has a significant increase in learning class-unique tokens compared to neural tokenizer ($0.034\% \rightarrow 2.14\%$ and $0.0\% \rightarrow 1.429\%)$. Both of these results suggests that our \tokenizer captures more useful and compact tokens compared to neural tokenizer. Additionally these tokens are more effective given its performance on EEG related tasks. 



% Our \tokenizer shows lower token utilization compared to neural tokenizer ($21.13\% \rightarrow 9.78\%$, $15.25\% \rightarrow 8.78\%$,


% % In order to evaluate the tokenizers' capacity to acquire class-specific unique tokens across all classes via unsupervised pretraining, we calculated the geometric mean (GM) of the class-token uniqueness scores. Table~\ref{tab:token_utilization} presents the token utilization alongside the GM of class-token uniqueness scores across various tokenization methods. 


% Furthermore, Figure~\ref{fig:retrieval_test}a visualizes the class-token uniqueness scores for each class in both datasets. 

% Compared to the neural tokenizer, all \tokenizer variants exhibit a significantly lower token utilization percentage. However, our proposed \tokenizer, which integrates both frequency and temporal domains, achieves a higher GM of class-token uniqueness, indicating a more representative yet compact vocabulary. Figure~\ref{fig:retrieval_test}a clearly illustrates our tokenizer's effectiveness across all classes in both TUEV and IIIC datasets. Tokenizers based solely on raw signals struggle to capture class-unique tokens, underscoring the importance of joint temporal-frequency modeling. Despite having significantly fewer parameters than the neural tokenizer, our \tokenizer captures more informative and class-distinctive tokens, reflected in its superior classification performance. 

\noindent{\textbf{Q2.2 - Tokens for Similar-Class Sample Mining:}} We conducted an EEG signal mining experiment based on similar-class sample retrieval. Given a multi-channel EEG sample, we first obtain its discrete token representation. Using the Jaccard similarity score, we then retrieve the top $K$ most similar samples from the dataset and compute the precision score for correctly retrieving samples of the same class. For this study, we constructed a balanced subset from the IIIC and TUEV datasets and tested all four tokenization methods. The retrieval performance, illustrated in Figure~\ref{fig:retrieval_test}b, shows that all \tokenizer variants significantly outperform neural tokenizer. Notably, \tokenizer-S and \tokenizer achieve nearly $60\%$ precision on the TUEV for $K=1$. While the Jaccard similarity measure demonstrates initial feasibility, further research is needed to identify optimal metrics for token-based EEG retrieval. 

% A promising future direction lies in leveraging these discrete tokens for identifying positive pairs within datasets, potentially enabling novel contrastive learning frameworks for EEG analysis.

\begin{figure}[t]
    \centering
    % \rule{0.8\linewidth}{0.5\linewidth} 
    % \includegraphics[width=\linewidth]{Figures/Story_Overview_Fig.pdf}
    \includegraphics[width=\linewidth]{FIG/retrieval_test_2_new.pdf}
    % \includegraphics[width=\linewidth]{Figures/retrieval_test.pdf}
    \caption{Analysis of token quality across three \tokenizer variants and the neural tokenizer. (a) Comparison of class-token uniqueness scores across all classes. (b) Retrieval performance comparison of tokenizers in a similar-class sample mining task.}
    
    % \caption{Overview of the token quality analysis study. (a) Class-Token Uniqueness Score of the tokenizers across different classes. (b) Precision scores of tokenizers in EEG is similar to the retrieval class.}
    \label{fig:retrieval_test}
    % \vspace{-0.2cm}
\end{figure}
\noindent{\textbf{Q2.3 - Evaluating the Frequency Learning of \tokenizer Tokens:}}
\label{sec:freq_learning}
% In this experiment, we compare frequency-domain and temporal-domain Transformer encoders to evaluate their ability to capture diverse frequency features in EEG signals. 
% We first apply a discrete Fourier transform to each token to decompose it into its constituent frequency components. 
% Then, we compute spectral entropy from all amplitude values as a quantitative measure of how uniformly the energy is distributed across the frequency spectrum. 
% In our framework, a higher spectral entropy value indicates that the energy is more evenly spread among different frequency components, implying that the model has learned a broader range of frequency features.
% Figure \ref{fig:frequeny_ana} shows the results on the TUEV, TUAB, and CHBMIT datasets,
% showing that the frequency encoder produces tokens with higher spectral entropy than the temporal encoder.
% For instance, on the TUEV dataset, the frequency encoder achieved an average spectral entropy of 0.26 compared to 0.14 for the temporal encoder. 
% This result suggests that the frequency encoder is more effective at extracting diverse frequency characteristics from the raw EEG data, which may enhance performance on downstream tasks such as classification.
In this experiment, we compare the frequency and temporal-domain encoders of the \tokenizer to evaluate their ability to capture diverse frequency features in EEG signals. 
Specifically, we arrange all tokens in temporal order and perform a discrete Fourier transform on the token sequence. 
This process decomposes the tokens into frequencies, where each frequency reflects the degree of change between tokens at various scales. 
Larger changes indicate more diverse token representations.
%—that is, the model has learned clear differences between tokens.
Then, we compute spectral entropy, defined as the normalized Shannon entropy of the amplitude values, to quantify how energy is distributed across the spectrum. 
Higher spectral entropy means that the model has learned a broader range of frequency features, capturing differences from both large-scale trends and fine details. 
Figure \ref{fig:frequeny_ana} shows that on the TUEV, TUAB, and CHBMIT datasets, the frequency encoder produces tokens with significantly higher spectral entropy than the temporal encoder. 
For example, on the TUEV dataset, the frequency encoder achieved an average spectral entropy of 0.26, while the temporal encoder reached only 0.14. 
This multi-scale sensitivity benefits downstream tasks such as classification, where learning detailed differences in EEG tokens can improve performance.








% \subsection{Q4: Evaluating the Frequency Learning of \tokenizer Tokens}
% \label{sec:freq_learning}

% In this experiment, we compare the frequency and temporal Transformer encoders to assess how well they capture multiple EEG signal frequencies. 
% We apply a Fourier transform to break each token into its frequency components, then use spectral entropy to measure how diverse these components are.
% % Specifically, a higher entropy value means that more frequency components are represented in the token. 
% Specifically, a higher entropy value indicates that the energy in the frequency spectrum is distributed more uniformly, meaning that the model has learned representations from a wider range of frequency components.
% Our tests on the TUEV, TUAB, and CHBMIT datasets consistently show that the frequency encoder produces tokens with higher spectral entropy, proving that it captures a broader range of EEG frequency features.

% In this experiment, we compare frequency-domain and temporal-domain Transformer encoders to evaluate their ability to capture diverse frequency features in EEG signals. 
% We first apply a discrete Fourier transform to each token to decompose it into its constituent frequency components. 
% Then, we compute spectral entropy from all amplitude values as a quantitative measure of how uniformly the energy is distributed across the frequency spectrum. 
% In our framework, a higher spectral entropy value indicates that the energy is more evenly spread among different frequency components, implying that the model has learned a broader range of frequency features.
% Figure \ref{fig:frequeny_ana} shows the results on the TUEV, TUAB, and CHBMIT datasets,
% showing that the frequency encoder produces tokens with higher spectral entropy than the temporal encoder.
% For instance, on the TUEV dataset, the frequency encoder achieved an average spectral entropy of 0.26 compared to 0.14 for the temporal encoder. 
% This result suggests that the frequency encoder is more effective at extracting diverse frequency characteristics from the raw EEG data, which may enhance performance on downstream tasks such as classification.


\begin{figure}[t]
    \centering
    % \rule{0.8\linewidth}{0.5\linewidth} 
    % \includegraphics[width=\linewidth]{Figures/Story_Overview_Fig.pdf}
    \includegraphics[width=0.98\linewidth]{FIG/frequency_ana.pdf}
    % \includegraphics[width=\linewidth]{Figures/retrieval_test.pdf}
    \caption{
    %5 samples randomly selected from the TUEV, TUAB, and CHBMIT datasets to analyze the frequency complexity of the token sequences learned by both the temporal and frequency encoders.
    An analysis of how the proposed frequency and temporal-domain encoders capture frequency features, by using the spectral entropy of the learned token sequences from randomly selected samples. 
    Higher values indicate that the tokens contain richer frequency information.}
    \label{fig:frequeny_ana}
    % \vspace{-0.2cm}
\end{figure}



% \begin{figure}[t]
%     \centering
%     % \rule{0.8\linewidth}{0.5\linewidth} 
%     % \includegraphics[width=\linewidth]{Figures/Story_Overview_Fig.pdf}
%     \includegraphics[width=\linewidth]{FIG/Labram_tfm.pdf}
%     % \includegraphics[width=\linewidth]{Figures/retrieval_test.pdf}
%     \caption{Performance Comparison of LaBraM with their neural tokenizer vs TFM-Tokenizer}
%     \label{fig:labram_tfm}
% \end{figure}

% \subsection{Q3: Scalability of \tokenizer}
\subsection{Q3: Does \tokenizer Enhance LaBraM?}
% \begin{table}[t]
% \centering
% \caption{Performance Comparison of LaBraM with their neural tokenizer vs TFM-Tokenizer}
% \label{tab:LaBraM_with_TFM}
% \resizebox{\linewidth}{!}{%
% \begin{tabular}{lcccc}
% \toprule
% \textbf{Metric} & \multicolumn{2}{c}{\textbf{TUEV}} & \multicolumn{2}{c}{\textbf{IIIC}}\\ 

% &\textbf{Neural} & \textbf{TFM} &\textbf{Neural} & \textbf{TFM}\\
% &\textbf{Tokenizer} & \textbf{Tokenizer} &\textbf{Tokenizer} & \textbf{Tokenizer}\\

% \midrule
% Balanced Acc  & $0.4682\pm0.0856$ & \textbf{0.5147}$\pm0.0174$ & & \\
% Cohen’s Kappa & $0.5067\pm0.0413$	& \textbf{0.5220}$\pm0.0153$ & & \\
% Weighted F1   &   $0.7466\pm0.0202$ & \textbf{0.7533}$\pm0.0094$ & & \\


% \bottomrule

% \end{tabular}
% }
% \end{table}
\begin{table}[t]
\centering
\caption{Performance Comparison of LaBraM with their neural tokenizer vs TFM-Tokenizer}
\label{tab:LaBraM_with_TFM}
\resizebox{\linewidth}{!}{%
\begin{tabular}{lcccc}
\toprule
\textbf{Dataset} & \textbf{Tokenizer} & \multicolumn{3}{c}{\textbf{Performance Metrics}}\\ 
\cmidrule{3-5}
& & \textbf{Balanced Acc.} & \textbf{Cohen's Kappa} & \textbf{Weighted F1}\\ 
\midrule
\multirow{2}{*}{TUEV} & Neural Tokenizer &$0.4682 \pm 0.0856$  &  $0.5067 \pm 0.0413$  &  $0.7466 \pm 0.0202$\\
& \tokenizer &\textbf{0.5147}$\pm0.0174$ $\uparrow$	&\textbf{0.5220}$\pm0.0153$ $\uparrow$&\textbf{0.7533}$\pm0.0094$ $\uparrow$\\
\midrule
& & \textbf{Balanced Acc.} & \textbf{AUC-PR} & \textbf{AUROC}\\ 
\cmidrule{3-5}
\multirow{2}{*}{TUAB} & Neural Tokenizer & $0.7720\pm0.0046$ &$0.8498\pm0.0036$ &	$0.8534\pm0.0027$ \\
& \tokenizer & \textbf{0.7765}$\pm0.0016$ $\uparrow$ & \textbf{0.8518}$\pm0.0051$ $\uparrow$ &	\textbf{0.8584}$\pm0.0022$ $\uparrow$ \\




\bottomrule

\end{tabular}
% \vspace{-1cm}
}
\end{table}

% explain the experiment then the results
% mention the size of both tokenizers.
To assess the scalability of \tokenizer, we investigated its ability to enhance an existing EEG foundation model. We selected LaBraM \cite{jiang2024large}, which employs a neural tokenizer solely for pretraining. 
This setup makes it an ideal candidate for this study. We replaced LaBraM neural tokenizer with \tokenizer during the masked EEG modeling stage and evaluated its performance on TUEV and TUAB, presented in Table~\ref{tab:LaBraM_with_TFM}. On TUEV, LaBraM with \tokenizer achieves a $9\%$ increase in balanced accuracy ($0.4682 \rightarrow 0.5147$) and a $3\%$ increase in Cohen's Kappa ($0.5067 \rightarrow 0.5220$). 
On TUAB, \tokenizer consistently outperforms the neural tokenizer. 
These results confirm the capability of \method in enhancing the performance of EEG foundation models. The increase in balanced accuracy suggests that our tokenizer learns more class-discriminative tokens than the neural tokenizer.


% These validate that our \tokenizer is capable of existing EEG foundation models. Also improvement in balanced accuracy indicates that \tokenizer's tokens learns better class related tokens compared to neural tokenizer. 


% To evaluate the scalability of our \tokenizer, we investigated whether it can enhance the performance of an existing state-of-the-art EEG foundation model. For this study, we selected LaBraM \cite{jiang2024large}, as its training scheme includes a neural tokenizer used solely for pretraining the final model. This makes it an ideal candidate for assessing the adaptability of our tokenizer. We replaced the neural tokenizer in LaBraM with our \tokenizer during the masked EEG modeling stage and evaluated its performance on the TUEV and TUAB datasets. Table~\ref{tab:LaBraM_with_TFM} presents the performance comparison between LaBraM with its original neural tokenizer and LaBraM using our \tokenizer. The results clearly demonstrate that LaBraM with \tokenizer outperforms the original LaBraM model across all metrics, indicating that \tokenizer generates higher-quality tokens compared to the neural tokenizer. Improvement in balanced accuracy across both datasets indicates that \tokenizer learns tokens with better class discriminability compared to the neural tokenizer. These results validate the scalability of our \tokenizer, as it can be integrated into different EEG foundation models and further enhance their performance.



% \subsection{Token quality analysis}

% TODO:
% \begin{itemize}
%     \item do utilization analysis and unique token analysis
%     \item gets the unique tokens and creates some figures. Share it with Brandon.
% \end{itemize}
\begin{figure}[t]
    \centering
    % \rule{0.8\linewidth}{0.5\linewidth} 
    % \includegraphics[width=\linewidth]{Figures/Story_Overview_Fig.pdf}
    \includegraphics[width=0.98\linewidth]{FIG/token_interpret_TUEV_1_new3.pdf}
    % \includegraphics[width=\linewidth]{Figures/retrieval_test.pdf}
    \caption{Overview of motifs captured by \tokenizer on the TUEV dataset: (a) presents three samples from the PLED class, while (b) displays three samples from the GPED class.
    % {\color{red}
    % font is too small.
    % }
    }
    %\vspace{-0.5cm}
    \label{fig:interpret_TUEV_1}
\end{figure}
\subsection{Q4: Interpretability of Learned Tokens}
\label{sec:Q6}
We conducted a visual inspection study to determine whether the learned tokens by our \tokenizer capture and represent distinct time-frequency motifs in EEG signals. Our findings are presented in Figure~\ref{fig:interpret_TUEV_1}, which illustrates the unique tokens identified by \tokenizer on the TUEV dataset. 
% Each column in the Figure presents three different samples from the PLED and GPED classes. 
Each token represents a spectral window and its corresponding raw EEG patch (1s window with 0.5s overlap). For more precise visualization, we highlight each class's most frequently occurring tokens, assigning different colors to each token. This analysis confirms that our \tokenizer effectively captures class-specific distinct EEG patterns, encoding them into discrete tokens. For instance, token $4035$ in the PLED class consistently captures a characteristic drop followed by a rise in EEG signals, maintaining its structure across different samples despite variations in noise, amplitude, and minor shifts within the window. Similarly, tokens such as $5096$ and $3751$ in the GPED class demonstrate the advantages of combining frequency and temporal features, as they remain robust to minor temporal shifts and warping within a window due to emphasizing spectral patterns. However, we also identified certain limitations associated with using fixed windowing and overlapping segments for tokenization. Specifically, when large shifts cause a distinct EEG pattern to be split across two adjacent windows, the model may assign them separate tokens, potentially treating them as distinct patterns. 

% Addressing this issue could be an important direction for future research, enabling the learning of shift-invariant tokens. 


% to further enhance the learning. Overall, these findings highlight the potential of discretized tokenization for EEG, as it reduces data complexity by mitigating the effects of noise and amplitude variations, while also enabling a structured and interpretable representation of EEG patterns.





% \subsection{Discrete tokens for EEG Data Mining}
% Experiment, using the data from test set, retrieve top 10 similar samples from training and use that to assign labels for each test sample. 
% study the confusion matrix for each class for their retrieval.
% TODO:
% \begin{itemize}
%     \item Do retrieval test for the new variants of TFM-TOKEN on TUEV and IIIC
%     \item Do continuous token retrieval experiments using BIOT
% \end{itemize}


% \subsection{Does TFM-TOKEN capture time-frequency information better?}
% In this section focus on comparing TFM-TOKEN stft reconstruction with the LaBRAM frequency reconstruction and discuss. (Park)



% \subsection{Positional Encoding ablation in the tokenizer}

% \subsection{Different Setting Ablation}
% for LaBraM using the pretrained model run experiment for 8 channel only in Finetuning. Same for ours. Dataset TUEV.


% \subsection{Importance of Capturing Frequency Band Information for EEG}
% This section includes an ablation on various masking methods and compares their results on TUEV. 
%     1. Random frequency masking
%     2. Frequency band masking
% Also, an ablation of a model variant without a freq encoder will be conducted, and the results will be provided.

% \subsection{Importance of raw signal in EEG analysis}
% Provide the token utilization plots to show that including the raw signals improves the learning. Additionally provide a table to compare with and without raw signal ablation of the model. (We already have this result)




% \subsection{Is TFM-TOKEN generalizable to other biosignals?}
% add single dataset pretraining and fine-tuning results here. or recreate the BIOT setting for ECG and HAR, provide the results. 


% \subsection{TFM Tokenizer scaling experiments}

% \subsection{TFM Classifier scaling experiments}





 





\section{Analysis}
% Please add the following required packages to your document preamble:
% \usepackage{multirow}
% \usepackage[normalem]{ulem}
% \useunder{\uline}{\ul}{}
\begin{table*}[]
\renewcommand{\arraystretch}{1.2} % 增加行间距到1.5倍
% 在表格环境之前设置全局字体为罗马字体
\newcolumntype{b}{>{\columncolor{blue!4}}c}
\renewcommand{\familydefault}{\rmdefault}
\resizebox{\textwidth}{!}{
\begin{tabular}{llccccccccccb}
\toprule
\textbf{Model}                                     & \textbf{Test}       & \textbf{-1.0}    & \textbf{-0.8}    & \textbf{-0.6}         & \textbf{-0.4}    & \textbf{-0.2}    & \textbf{0.0}     & \textbf{0.2}           & \textbf{0.4}           & \textbf{0.6}     & \textbf{0.8}              & \textbf{Random}           \\
\hline
\multirow{2}{*}{LLaMA3-8B-Instruct}       & Sequential & 69.45\% & 80.00\% &  \underline{80.55\%} & 75.90\% & 64.25\% & 74.80\% & 69.65\%       & 73.15\%       & 74.40\% & 67.50\%          & \textbf{81.05\%} \\
& Shuffled   & 67.55\% & 77.90\% & \underline{77.95\%} & 74.85\% & 64.10\% & 72.90\% & 68.50\%       & 70.05\%       & 73.30\% & 66.20\%          & \textbf{78.80\%} \\
\hdashline
\multirow{2}{*}{LLaMA2-13B-Chat}          & Sequential  & 68.20\% & 65.25\% & 71.40\%       & 66.90\% & 70.60\% & 60.40\% & 71.65\%       & 70.35\%       & 72.50\% & \textbf{73.20\%} & \underline{72.20\%}    \\
& Shuffled  & 67.65\% & 63.65\% & 69.60\%       & 65.10\% & 69.40\% & 58.75\% & 69.00\%       & 68.75\%       & 69.55\% & \underline{70.60\%}    & \textbf{71.30\%} \\
\hdashline
\multirow{2}{*}{Mistral-7B-Instruct-v0.3} & Sequential  & 64.75\% & 64.80\% & 54.50\%       & 65.60\% & 69.05\% & 50.95\% & 68.95\%       & \underline{69.65\%} & 67.55\% & 54.65\%          & \textbf{70.95\%} \\
& Shuffled  & 64.80\% & 64.10\% & 55.55\%       & 66.25\% & 67.75\% & 51.15\% & \underline{67.90\%} & 66.70\%       & 67.40\% & 54.55\%          & \textbf{69.20\%}\\
\bottomrule
\end{tabular}
}
  \caption{
  The performance of the models on RuleTaker with conditionally shuffled premises at different tau values. Tau = 1 represents the original order, tau = -1 indicates a complete reversal, tau = 0 means uniform shuffling, and "Random" refers to a fully random shuffle.
  }
  \label{tab:tau}
\end{table*} 
\subsection{Condition Augmentation with Varying Shuffling Degrees}
To investigate the effects of premise order transformations, we divide the Kendall tau distance \( \tau \) between different premise orders and the original order into 10 groups, each spanning a 0.2 range within [-1,1). A \( \tau \) value of 1 indicates forward order, -1 indicates a complete reversal, and 0 represents  a more uniform shuffling. Additionally, random shuffle means that \( \tau \) values from the entire range may be included. We conduct experiments on RuleTaker using different \( \tau \) values for condition-based data augmentation.

As shown in Tab. \ref{tab:tau}, random shuffling provides the best performance across all \( \tau \) values. The level of perturbation in premise order significantly affects model accuracy,  with differences exceeding 10\%. LLaMA3-8B-Instruct excels with negative \( \tau \) values, while LLaMA2-13B-Chat and Mistral-7B-Instruct-v0.3 do better with positive \( \tau \) values. Random shuffled in training data achieves the best overall performance, emphasizing the value of diverse data augmentation for more flexible and robust models.


% Please add the following required packages to your document preamble:
% \usepackage{multirow}
\begin{table*}[]
\renewcommand{\arraystretch}{1.3} % 增加行间距到1.5倍
% 在表格环境之前设置全局字体为罗马字体
\renewcommand{\familydefault}{\rmdefault}
\resizebox{\textwidth}{!}{
\begin{tabular}{llcccccc}
\toprule
\multirow{2}{*}{\textbf{Models}}            & \multirow{2}{*}{\textbf{Training data}} & \multicolumn{2}{c}{\textbf{FOLIO}} & \multicolumn{2}{c}{\textbf{Ruletaker}} & \multicolumn{2}{c}{\textbf{LogicNLI}} \\
\cmidrule(lr){3-8}
&                         & \textbf{Sequential}   & \textbf{Shuffled}   & \textbf{Sequential}     & \textbf{Shuffled}     & \textbf{Sequential}     & \textbf{Shuffled}    \\
\hline
\multirow{2}{*}{LLaMA3-8B-Instruct}
& Answer Steps Shuffled    & 77.34\%      & 76.85\%    & 84.60\% & 82.70\%      & 43.80\% & 42.80\%     \\
& Random Steps Shuffled & 76.85\%(-0.49) & 69.21\%(-7.64) & 82.10\%(-2.50) & 81.20\%(-1.50) & 45.20\%\textcolor{darkgreen}{(+1.40)} & 42.75\%(-0.05) \\
& Condition\&Answer Shuffled 
& 74.88\% (-2.46) 
& 75.86\% (-0.99) 
& 81.15\% (-3.45) 
& 79.60\% (-3.10) 
& 42.80\% (-1.00) 
& 43.50\% (+0.70) \\
\hdashline
\multirow{2}{*}{LLaMA2-13B-Chat} 
& Answer Steps Shuffled    & 76.35\%      & 73.89\%    & 75.50\% & 72.25\%      & 46.90\% & 42.75\%     \\  
& Random Steps Shuffled & 71.92\%(-4.43) & 69.21\%(-4.68) & 74.75\%(-0.75) & 72.75\%\textcolor{darkgreen}{(+0.50)} & 43.70\%(-3.20) & 42.45\%(-0.30) \\
& Condition\&Answer Shuffled 
& 70.94\% (-5.41) 
& 67.24\% (-6.65) 
& 77.40\% (+1.90) 
& 73.80\% (+1.55) 
& 44.20\% (-2.70) 
& 41.60\% (-1.15) \\
\hdashline
\multirow{2}{*}{Mistral-7B-Instruct-v0.3} 
& Answer Steps Shuffled    & 72.91\%      & 72.17\%    & 84.10\% & 82.80\%      & 47.35\% & 47.30\%     \\
& Random Steps Shuffled & 71.43\%(-1.48) & 72.41\%\textcolor{darkgreen}{(+0.24)} & 82.95\%(-1.15) & 79.95\%(-2.85) & 44.75\%(-2.60) & 45.25\%(-2.05) \\
& Condition\&Answer Shuffled 
& 70.94\% (-1.97) 
& 70.44\% (-1.73) 
& 82.55\% (-1.55) 
& 81.70\% (-1.10) 
& 41.50\% (-5.85)
& 42.00\% (-5.30) \\
\bottomrule
\end{tabular}
}
  \caption{
  The performance of three different augmentation methods: the first row represents the original DAG-based Answer Steps Shuffled augmentation, the second row represents random step shuffling without dependencies in Sec. \ref{sec:non_DAG}, and the third row represents the combined condition and answer augmentation method in Sec. \ref{sec:both_ran}.
  }
  \label{tab:DAG_or_not}
\end{table*}
\subsection{The Importance of DAG-based Step Dependency}
\label{sec:non_DAG}
To explore the importance of using DAG for step dependencies in step augmentation, we use the Answer Step Shuffled data from Tab. \ref{tab:Data_statistics} as a baseline. We randomly shuffle the steps in the original COT process and assess its performance to evaluate the impact of random step reordering without DAG dependencies.

As shown in Tab. \ref{tab:DAG_or_not}, not utilizing DAG dependencies leads to a performance drop compared to DAG-based augmentation. The decline is particularly severe on FOLIO, where LLaMA3-8B-Instruct and LLaMA2-13B-Chat show a drop of 7.64\% and 4.68\% in the shuffled test. In contrast, Ruletaker and LogicNLI experience a smaller decline. 

To explore the underlying cause of this phenomenon, we investigate the degree of dependency between steps in the step dependency DAG. We introduce the \textbf{Topological Freedom Index (TFI)}. This metric measures how loosely or tightly connected a DAG is, and it is calculated as follows:
\begin{equation}
TFI = \frac{\text{Number of valid sequences}}{\text{Factorial of the number of steps}}
\end{equation}

Number of valid sequences represents the number of valid topological orderings that respect the dependency constraints within the DAG. Factorial of the number of steps corresponds the number of possible orderings if no dependencies were present.
The closer the TFI value is to 1, the looser the DAG structure, indicating higher reordering flexibility and greater step independence. Conversely, the closer the TFI value is to 0, the stronger the step dependencies, meaning the sequence must follow a strict order with little to no flexibility. Fig. \ref{fig:pie_TFI} presents the TFI distribution across three datasets, illustrating the degree of step dependency in different reasoning tasks.

\begin{figure}[t] 
    \centering
        \resizebox{0.48\textwidth}{!}{
            \includegraphics[width=1\textwidth]{pie_TFI_fig.pdf}
    % \captionsetup{font={small}} 
    }
    \caption{The distribution of TFI index across different intervals in the training sets of FOLIO, RuleTaker, and LogicNLI. Since none of the datasets contain data in the [0.6-0.9) interval, this portion is omitted from the presentation.}
    \label{fig:pie_TFI}
\end{figure}

In FOLIO, the majority of samples (49.9\%) fall within the 0.0–0.1 range, indicating strong dependency constraints and minimal reordering flexibility. In contrast, RuleTaker and LogicNLI exhibit a significantly higher proportion of high-TFI samples (e.g., 44.0\% and 27.0\% in the 0.9–1.0 range, respectively), suggesting that these datasets contain more loosely connected reasoning structures. These trends highlight that FOLIO imposes stricter logical dependencies. In contrast, the latter datasets have weaker step dependencies, offering greater reordering flexibility. However, this may also imply that the quality of the generated COT needs improvement.

The difference in TFI across datasets aligns with the conclusions we obtained from the Random Step Shuffled experiment. Specifically, stronger step dependencies result in greater performance loss from random shuffling. Therefore, when performing answer order augmentation, it is important to maintain the integrity of these dependencies.

\subsection{Combined Condition and Step Shuffling Leads to Performance Degradation}
\label{sec:both_ran}
To investigate the combined effect of condition and step order augmentation, we apply an additional premise shuffle to Answer Steps Shuffled data, adjusting premise references in the answers accordingly. As shown in Tab. \ref{tab:DAG_or_not}, \textbf{Condition\&Answer Shuffled} results in lower performance compared to Answer Steps Shuffled alone.

We believe the key reason is that premise and step shuffling serve different learning purposes. Premise shuffling enables the model to recognize that independent conditions with commutativity can lead to the same answer, while step shuffling allows it to understand that different reasoning paths under the same condition can yield consistent conclusions. When applied separately, each augmentation enhances the model's understanding of logical equivalence, thereby improving its overall reasoning ability. However, when condition and step shuffling are applied together, the logical structure is perturbed in two ways, requiring the model to simultaneously align different orders of both conditions and steps, increasing learning difficulty and reducing generalization. This suggests that excessive augmentation may introduce noise, making it harder for models to establish logical equivalence.

\subsection{Effect of Augmentation Frequency}
In the main experiment, we set \( |D_C'| = |D_C| \), meaning that the parameter \( k = 1 \).
To investigate the impact of augmentation quantity, we increase \( k \) and generate \( k = 5 \) augmented instances for each original training sample in RuleTaker. 
This leads to an augmented dataset \( D_C' \) containing \( 5 \times |D_C| \) instances. 
As shown in Tab. \ref{tab:condition_variants}, adding a few shuffled instances improves model accuracy, but excessive augmentation results in performance degradation. This highlights the need to control the augmentation frequency. The increase in \( k \) can lead to a certain degree of performance improvement, indicating that our order-centric data augmentation method has room for further enhancement.

\begin{table}[t]
\renewcommand{\arraystretch}{1.2} % 增加行间距到1.5倍
% 在表格环境之前设置全局字体为罗马字体
\renewcommand{\familydefault}{\rmdefault}
    \centering
    \resizebox{0.48\textwidth}{!}{
\begin{tabular}{lccccc}
\toprule
Test       & \( k \)=1 & \( k \)=2 & \( k \)=3 & \( k \)=4 & \( k \)=5 \\
\hline
Sequential & 77.30\%    & 81.35\%     & \textbf{83.45\%}    & 83.15\%     & 77.65\%     \\
Shuffled   & 76.65\%    & 80.35\%     & 81.95\%    & \textbf{83.55\%}     & 79.85\%    \\
\bottomrule
\end{tabular}
    }
  \caption{
  The performance under different augmentation frequencies, where \( k \) represents the number of condition order augmentation instances applied per training sample.
  }
  \label{tab:condition_variants}
\end{table}

\section{Conclusion}
In this paper, we systematically study how to enhance the logical reasoning ability of LLMs by addressing their limitations in reasoning order variations. We introduce an order-centric data augmentation framework based on the principles of independency and commutativity in logical reasoning. Our method involves shuffling independent premises to introduce order variations and constructing directed acyclic graphs (DAGs) to identify valid step reorderings while preserving logical dependency. Extensive experiments across multiple logical reasoning benchmarks demonstrate that our method significantly improves LLMs’ reasoning performance and their adaptability to diverse logical structures.


\section{Limitations}
Our work primarily focuses on logical commutativity within propositional reasoning tasks. However, this property extends beyond these tasks. It is also prevalent in many other reasoning scenarios, such as mathematical problems and other logic-based tasks. This remains an area for future exploration. Additionally, while we have explored the impact of condition order and answer order augmentations on model performance, how to further integrate and refine these augmentations for better logical reasoning capability is still an open question. We believe our exploration will provide valuable insights for future work on logical equivalence and commutativity in reasoning.

\bibliography{custom}
\clearpage

\appendix

\section{Appendix}
\label{sec:appendix}

\section{Additional Results}

\subsection{Performance on CHB-MIT and IIIC Seizure}
\label{app:chbmit_iiic_comparison}

Table~\ref{tab:seizure_detection_CHBMIT} and \ref{tab:eeg_classification_IIIC} presents the performance comparison of \method with baselines on seizure detection (CHB-MIT) and seizure type classification (IIIC Seizure) tasks. \method outperforms all baselines across all metrics in both datasets. On the CHB-MIT dataset with a highly imbalanced binary classification task, BIOT is the only baseline with an AUC-PR above $0.25$.  However, \method surpasses BIOT, achieving an $8\%$ improvement in AUC-PR ($0.3127 \rightarrow 0.3379$) and a $4.5\%$ increase in AUROC ($0.8456 \rightarrow 0.8839$), demonstrating better robustness to class imbalance. For the IIIC Seizure dataset, where the task is to classify 10-second, 16-channel EEG segments into six classes, \method improves Cohen’s Kappa by $9.5\%$ ($0.4549 \rightarrow 0.4985$) and Weighted F1 by $8.5\%$ ($0.5387\rightarrow 0.5847$) over ContraWR, which achieves second best results. 

The superior performance of \method across four EEG datasets shows the promise of a fully discretized framework that has the potential to enhance future EEG foundation models. These results also underscore the importance of capturing both temporal and frequency information, highlighting the critical role of frequency learning in EEG analysis.


% Table\ref{tab:seizure_detection_CHBMIT} and \ref{tab:eeg_classification_IIIC} present the performance comparison of our \method with baselines on seizure detection (CHB-MIT) and seizure type classification (IIIC Seizure) tasks. Our \method outperforms all baselines in all metrics in both datasets. CHB-MIT dataset posses a challenging binary classification task mainly due to its imbalanced settings. BIOT is the only baseline to achieves better performance in CHB-MIT datasets, where as all other baselines' AUC-PR scores are less than $0.25$. However, \method outperforms BIOT and achieves $8\%$ improvement over  BIOT in AUC-PR($0.3127 \rightarrow 0.3379$) and $4.5\%$ improvement in AUROC($0.8456 \rightarrow 0.8839$). This shows that our model is more robust to imbalance data compared to the baselines. 

% On the IIIC seizure dataset, the task is to classify each $10$ second and $16$ channel EEG segments into one of six classes. \method achieves $9.5\%$ increase in Cohen's Kappa($0.4549 \rightarrow 0.4985$) and $8.5\%$ increase in Weighted F1 ($0.5387 \rightarrow 0.5847$) compared to ContraWR, which achieves the second best performance on this dataset. Our \method's consistent superiors performance compared to other baselines on four EEG datasets suggests that fully discretization based framework would enable building better EEG foundation models in future. Additionaly these results highlight the importance of capturing both temporal and frequency information, where research on frequency learning is extremely important for EEG analysis. 

\begin{table}[thpb]
\centering
\caption{Seizure detection performance comparison on the CHB-MIT dataset}
\label{tab:seizure_detection_CHBMIT}
\resizebox{\linewidth}{!}{%
% \begin{tabular}{l p{2.2cm} p{2.7cm} p{2.7cm} p{2.7cm} p{2.7cm} p{2.7cm} p{2.7cm}} %{lccccccc}
\begin{tabular}{lcccc}
\toprule
\textbf{Models} & \textbf{Number} &  \multicolumn{3}{c}{\textbf{CHB-MIT (seizure detection)}} \\
\cmidrule(lr){3-5} 
 &\textbf{of Params} &  \textbf{Balanced Acc.} & \textbf{AUC-PR} & \textbf{AUROC} \\
\midrule

SPaRCNet\cite{jing2023development} & 0.79M & $0.5483\pm0.0164$ &	$0.1264\pm0.0112$	&$0.7953\pm0.0402$ \\

ContraWR\cite{yang2021self} & 1.6M &  $0.6479\pm0.0423$ &	$0.2312\pm0.0316$	&$0.8198\pm0.0197$\\

CNN-Transformer\cite{peh2022transformer} & 3.2M & $0.5900\pm0.0429$	& $0.2250\pm0.0512$	&$0.8333\pm0.0326$\\

FFCL\cite{li2022motor} & 2.4M &  $0.6565\pm0.0260$ &	$0.2115\pm0.0286$	&$0.8071\pm0.0117$ \\

ST-Transformer\cite{song2021transformer} & 3.5M & $0.6717\pm0.0271$	&$0.2133\pm0.0148$	&$0.8657\pm0.0139$ \\

BIOT\cite{yang2024biot} & 3.2M &  $0.6582\pm0.0896$&	$0.3127\pm0.0890$ &	$0.8456\pm0.0333$\\

LaBraM-Base\cite{jiang2024large} & 5.8M & $0.5035\pm0.0078$ &	$0.0959\pm0.0742$& $0.6624\pm0.1050$  \\

\midrule

% \textbf{TFM-Token-Raw Signal Only} & 1.8M & $0.5812 \pm 0.0029$ &	$0.5051 \pm 0.0048$ &	$0.5874 \pm 0.0029$ \\

% \textbf{TFM-Token-STFT Only} & 1.9M&  $0.7538 \pm 0.0152$ &  $0.5709 \pm 0.0059$	&$0.4904 \pm 0.0065$	& $0.5799 \pm 0.0059$\\

\textbf{\method} & 1.9M & \textbf{0.6750 $\pm$ 0.0392} &	\textbf{0.3379 $\pm$ 0.0515}	& \textbf{0.8839 $\pm$ 0.0173} \\

% & $0.5487\pm0.0038$ &	$0.4589\pm0.0045$ &	$0.5549\pm0.0044$  \\

\bottomrule
\end{tabular}
}
% \multicolumn{8}{l}{1. The number of parameters for LaBraM is only considering their classifier model. The size of their neural tokenizer used for masked EEG modeling training was 8.6M.} \\
\end{table}









\begin{table}[t]
\centering
\caption{Seizure type classification performance comparison on the IIIC Seizure dataset}
\label{tab:eeg_classification_IIIC}
\resizebox{\linewidth}{!}{%
% \begin{tabular}{l p{2.2cm} p{2.7cm} p{2.7cm} p{2.7cm} p{2.7cm} p{2.7cm} p{2.7cm}} %{lccccccc}
\begin{tabular}{lcccc}
\toprule
\textbf{Models} & \textbf{Number} &  \multicolumn{3}{c}{\textbf{IIIC Seizure (seizure type classification)}} \\
\cmidrule(lr){3-5} 
 &\textbf{of Params} &  \textbf{Balanced Acc.} & \textbf{Cohen's Kappa} & \textbf{Weighted F1} \\
\midrule

SPaRCNet\cite{jing2023development} & 0.79M &  $0.5011 \pm 0.0286$ & $0.4115 \pm 0.0297$ & $0.4996 \pm 0.0262$  \\

ContraWR\cite{yang2021self} & 1.6M &  $0.5421 \pm 0.0123$ & $0.4549 \pm 0.0166$ & $0.5387 \pm 0.0138$  \\

CNN-Transformer\cite{peh2022transformer} & 3.2M &  $0.5395 \pm 0.0144$ & $0.4500 \pm 0.0165$ & $0.5413 \pm 0.0176$ \\

FFCL\cite{li2022motor} & 2.4M &  $0.5309 \pm 0.0217$ & $0.4412 \pm 0.0253$ & $0.5315 \pm 0.0277$ \\

ST-Transformer\cite{song2021transformer} & 3.5M &  $0.5093 \pm 0.0122$ & $0.4217 \pm 0.0151$ & $0.5217 \pm 0.0110$ \\

BIOT\cite{yang2024biot} & 3.2M &  $0.4458\pm0.0183$  & $0.3418\pm0.0228$  &  $0.4511\pm0.0207$ \\

LaBraM-Base\cite{jiang2024large} & 5.8M &   $0.4736 \pm 0.0101$ & $0.3716 \pm 0.0128$ & $0.4765 \pm 0.0097$ \\

\midrule

% \textbf{TFM-Token-Raw Signal Only} & 1.8M & $0.5812 \pm 0.0029$ &	$0.5051 \pm 0.0048$ &	$0.5874 \pm 0.0029$ \\

% \textbf{TFM-Token-STFT Only} & 1.9M&  $0.7538 \pm 0.0152$ &  $0.5709 \pm 0.0059$	&$0.4904 \pm 0.0065$	& $0.5799 \pm 0.0059$\\

\textbf{\method} & 1.9M &  \textbf{0.5775 $\pm$ 0.0042} &	\textbf{0.4985 $\pm$ 0.0039} &	\textbf{0.5847 $\pm$ 0.0050}\\

% & $0.5487\pm0.0038$ &	$0.4589\pm0.0045$ &	$0.5549\pm0.0044$  \\

\bottomrule

\end{tabular}
}
% \multicolumn{8}{l}{1. The number of parameters for LaBraM is only considering their classifier model. The size of their neural tokenizer used for masked EEG modeling training was 8.6M.} \\
\end{table}






% % \setlength{\tabcolsep}{6pt}
% \begin{table}[thpb]
% \centering
% \caption{EEG abnormal detection performance comparison on TUAB}
% \label{tab:tuab_chb_results}
% \resizebox{\linewidth}{!}{%
% % \begin{tabular}{l p{2.2cm} p{2.7cm} p{2.7cm} p{2.7cm} p{2.7cm} p{2.7cm} p{2.7cm}} %{lccccccc}
% \begin{tabular}{lcccc}
% \toprule
% \textbf{Models} & \textbf{Number} & \multicolumn{3}{c}{\textbf{TUAB (abnormal detection)}} \\
% \cmidrule(lr){3-5} 
%  &\textbf{of Params} & \textbf{Balanced Acc.} & \textbf{AUC-PR} & \textbf{AUROC} \\
% \midrule

% SPaRCNet \cite{} & 0.79M & $0.7885\pm0.0205$	&$0.8804\pm0.0071$	&$0.8782\pm0.0127$ \\

% ContraWR \cite{} & 1.6M &  $0.7851\pm0.0050$&	$0.8743\pm0.0042$	&$0.8719\pm0.0076$ \\

% CNN-Transformer \cite{} & 3.2M &  $0.7898\pm0.0062$&	$0.8781\pm0.0067$&	$0.8781\pm0.0053$  \\

% FFCL \cite{} & 2.4M &  $0.7869\pm0.0054$ &$0.8761\pm0.0044$ &	$0.8694\pm0.0091$\\

% ST-Transformer \cite{} & 3.5M &  $0.8004\pm0.0037$&	$0.8798\pm0.0029$	&$0.8737\pm0.0069$  \\

% BIOT\cite{} & 3.2M &  $0.7955\pm0.0047$	& $0.8834\pm0.0041$& 	$0.8819\pm0.0046$ \\

% LaBraM-Base \cite{} & 5.8M & $0.7720\pm0.0046$ &	$0.8534\pm0.0027$& $0.8498\pm0.0036$  \\

% \midrule

% \textbf{TFM-Token-Raw Signal Only} & 1.8M & $0.8033\pm0.0021$ & $0.8849\pm0.0024$ & $0.8908\pm0.0027$ \\

% \textbf{TFM-Token-STFT Only} & 1.9M& & & \\

% \textbf{TFM-Token} & 1.9M &  \textbf{0.8152}$\pm0.0014$	& \textbf{0.8897}$\pm0.0008$	& \textbf{0.8946}$\pm0.0008$\\

% % & $0.5487\pm0.0038$ &	$0.4589\pm0.0045$ &	$0.5549\pm0.0044$  \\

% \bottomrule

% \end{tabular}
% }
% % \multicolumn{8}{l}{1. The number of parameters for LaBraM is only considering their classifier model. The size of their neural tokenizer used for masked EEG modeling training was 8.6M.} \\
% \end{table}


 

\subsection{Effect of Masked Token Prediction in EEG Tokenization}
\label{app:masked_token_prediction_ablation}
% \begin{table}[thpb]
\centering
\caption{Ablation on with and without masked token prediction pretraining on TUEV dataset.}
\label{tab:masked_token_predition_ablation}
\resizebox{\linewidth}{!}{%
% \begin{tabular}{l p{2.2cm} p{2.7cm} p{2.7cm} p{2.7cm} p{2.7cm} p{2.7cm} p{2.7cm}} %{lccccccc}
\begin{tabular}{lccc}
\toprule
\textbf{Models}  &  \multicolumn{3}{c}{\textbf{TUEV (event type classification) }} \\
\cmidrule(lr){2-4} 
 &  \textbf{Balanced Acc.} & \textbf{Cohen's Kappa} & \textbf{Weighted F1} \\
\midrule


\textbf{TFM-Token}-R & $0.4898\pm 0.0105$ & $0.5194 \pm 0.0195$ & $0.7518 \pm 0.0095$ \\

\textbf{TFM-Token}-S & $0.4708 \pm 0.0339$ & $0.5275\pm 0.0314$ & $0.7538\pm 0.0152$ \\

\textbf{TFM-Token} &   $0.4943 \pm 0.0516$  &  $0.5337 \pm 0.0306$  &  $0.7570 \pm0.0163$ \\

% & $0.5487\pm0.0038$ &	$0.4589\pm0.0045$ &	$0.5549\pm0.0044$  \\

\bottomrule

\end{tabular}
}
% \multicolumn{8}{l}{1. The number of parameters for LaBraM is only considering their classifier model. The size of their neural tokenizer used for masked EEG modeling training was 8.6M.} \\
\end{table}

We conducted an ablation study on \encoder to assess the impact of masked token prediction pretraining in a fully discretized framework. Using a pretrained \tokenizer, we compared two approaches: (1) masked token prediction pretraining followed by fine-tuning and (2) direct fine-tuning without pretraining. This experiment was performed on the TUEV dataset across all three \tokenizer variants, with results summarized in Figure~\ref{fig:MTP_ablation}. While Cohen's Kappa and Weighted F1 showed no significant differences between the two approaches, masked token prediction pretraining significantly improved balanced accuracy across all \tokenizer variants. This suggests that pretraining enhances class-wise prediction consistency by capturing token dependencies and making \encoder more robust to missing channels or time segments, a common challenge in EEG analysis.
 
\begin{figure}[h]
    \centering
    % \rule{0.8\linewidth}{0.5\linewidth} 
    % \includegraphics[width=\linewidth]{Figures/Story_Overview_Fig.pdf}
    \includegraphics[width=\linewidth]{FIG/MTP_ablation.png}
    % \includegraphics[width=\linewidth]{Figures/retrieval_test.pdf}
    \caption{Masked Token Prediction Ablation}
    \label{fig:MTP_ablation}
\end{figure}


\subsection{Removing Position Embedding in \tokenizer Improves Token Learning}
\label{app:with_wo_pe_ablation}
\begin{table}[thpb]
\centering
\caption{TFM-Tokenizer Comparison with and w/o Position Embedding (PE) on TUEV Dataset}
\label{tab:with_wo_PE}
\resizebox{\linewidth}{!}{%
\begin{tabular}{lccccc}
\toprule
\textbf{ Method} & \textbf{Utilization}& \textbf{Uniqueness} & \textbf{Balanced} &\textbf{Cohen's} & \textbf{Weighted}\\
& \textbf{$\%$} &\textbf{(GM) $\%$} & \textbf{Acc.} & \textbf{Kappa}&  \textbf{F1} \\

\midrule
\tokenizer + PE & $12.87$& $1.94$ & $0.4765\pm0.038$ & $0.5119\pm0.022$	& $0.7457\pm0.012$  \\
\tokenizer  w/o PE & $9.78$ & $2.14$ & \textbf{0.4943 $\pm 0.052$}  &  \textbf{0.5337 $\pm 0.031$}  &  \textbf{0.7570 $\pm 0.016$}   \\

\bottomrule

\end{tabular}
}
% \vspace{-0.5cm}

\end{table}




Through our empirical analysis, we found that the performance of our \method significantly improved when no position embedding was applied to the \tokenizer. EEG patterns are inherently chaotic and non-stationary, meaning similar motifs can occur at any position within the signal. An ideal tokenizer should be capable of capturing and representing such EEG motifs as distinct tokens without relying on positional information. 

We conducted an ablation study comparing the \tokenizer's performance with and without position embeddings to critically analyze this phenomenon. The results of this analysis, presented in Table~\ref{tab:with_wo_PE}, clearly show that the \tokenizer without position embedding achieves significantly better performance, with an increase of $4\%$ in Cohen's Kappa ($0.5119 \rightarrow 0.5337$).

We further studied the quality of the learned tokens in terms of token utilization and class-uniqueness scores. Token utilization decreased ($12.87\% \rightarrow 9.78\%$) when position embeddings were removed, while the class-token uniqueness score increased ($1.94\% \rightarrow 2.14\%$). This suggests that the \tokenizer, when using positional encoding, learns different tokens for the same motifs depending on their location in the signal, leading to redundancy. Removing the position embedding allows the \tokenizer to learn more compact and meaningful tokens without introducing unnecessary data complexities. This improvement is further illustrated in the motifs captured by the \tokenizer's tokens in Figure~\ref{fig:interpret_TUEV_1} in Section~\ref{sec:Q6}.




\subsection{Ablation on Token Vocabulary Size}
To evaluate the impact of token vocabulary size on performance and token learning, we conducted an ablation study by varying the vocabulary size from 256 to 8192 in powers of two. As shown in Figure~\ref{fig:codebook_ablation_metrics}, no monotonic trend was observed for Cohen's Kappa and Weighted F1 scores. However, balanced accuracy increased with larger vocabulary sizes. Further analysis of token utilization and class-token uniqueness scores is presented in Figure~\ref{fig:codebook_ablation_util_unique}. Notably, Figure~\ref{fig:codebook_ablation_util_unique}b shows that class-token uniqueness scores increase with vocabulary size, contributing to the improvement in balanced accuracy by enabling learning more unique class-specific tokens.
 


\begin{figure}[thpb]
    \centering
    % \rule{0.8\linewidth}{0.5\linewidth} 
    % \includegraphics[width=\linewidth]{Figures/Story_Overview_Fig.pdf}
    \includegraphics[width=\linewidth, trim=20 0 20 0, clip]{FIG/code_book_size_ablation.pdf}
    % \includegraphics[width=\linewidth]{Figures/retrieval_test.pdf}
    \caption{Token vocabulary size ablation with performance metrics}
    \label{fig:codebook_ablation_metrics}
    \vspace{-0.4cm}
\end{figure}

\begin{figure}[thpb]
    \centering
    % \rule{0.8\linewidth}{0.5\linewidth} 
    % \includegraphics[width=\linewidth]{Figures/Story_Overview_Fig.pdf}
    \includegraphics[width=\linewidth, trim=20 0 20 0, clip]{FIG/util_unique_code_book_size_ablation.pdf}
    % \includegraphics[width=\linewidth]{Figures/retrieval_test.pdf}
    \caption{Token vocabulary size ablation with token utilization and uniqueness}
    \label{fig:codebook_ablation_util_unique}
    \vspace{-0.4cm}
\end{figure}


% \subsection{Ablation on \tokenizer Modules}
\newpage
\subsection{Ablation on Masking}
\begin{table}[thpb]
\centering
\caption{Ablation on masking used for the pretraining of \tokenizer on TUEV Dataset}
\label{tab:masking_ablation}
\resizebox{\linewidth}{!}{%
% \begin{tabular}{l p{2.2cm} p{2.7cm} p{2.7cm} p{2.7cm} p{2.7cm} p{2.7cm} p{2.7cm}} %{lccccccc}
\begin{tabular}{lccc}
\toprule
\textbf{Masking Strategy}  &  \textbf{Balanced Acc.} & \textbf{Cohen's Kappa} & \textbf{Weighted F1} \\
\midrule


Random Masking &  $0.4351\pm0.0462$ &	$0.4772\pm0.0140$&	$0.7296\pm0.0076$\\
Frequency Bin Masking & $0.4673\pm0.0540$ &	$0.5193\pm0.0243$	&$0.7536\pm0.0125$\\
Frequency Bin  &  \multirow{2}{*}{$\mathbf{0.4946\pm0.0392}$} &	 \multirow{2}{*}{$0.5045\pm0.0221$}	& \multirow{2}{*}{$0.7462\pm0.0116$}\\
+ Temporal Masking & & & \\
\hline
Frequency Bin   & \multirow{3}{*}{$0.4943\pm 0.0516$}  &  \multirow{3}{*}{$\mathbf{0.5337\pm 0.0306}$}  &  \multirow{3}{*}{$\mathbf{0.7570\pm 0.0163}$}\\
+ Temporal Masking\\
+ Symmetric Masking\\
% & $0.5487\pm0.0038$ &	$0.4589\pm0.0045$ &	$0.5549\pm0.0044$  \\

\bottomrule

\end{tabular}
}
% \multicolumn{8}{l}{1. The number of parameters for LaBraM is only considering their classifier model. The size of their neural tokenizer used for masked EEG modeling training was 8.6M.} \\
\end{table}
We conducted an ablation study on masking strategies during \tokenizer pretraining to assess their impact on performance. Results shown in Table~\ref{tab:masking_ablation} indicate that random masking on the spectrogram $S$ performs poorly compared to other strategies, underscoring the need for effective masking to capture frequency and temporal features from EEG. Frequency bin masking significantly improves performance over random masking, with an $8\%$ increase in Cohen's Kappa ($0.4772 \rightarrow 0.5193$) and a $7\%$ increase in balanced accuracy ($0.4351 \rightarrow 0.4673$), highlighting the importance of modeling frequency band dynamics. The addition of temporal masking further boosts balanced accuracy by $5\%$ ($0.4673 \rightarrow 0.4946$), underscoring the importance of joint temporal-frequency modeling. However, temporal masking results in a decline in Cohen's Kappa and Weighted F1, which is then resolved by introducing symmetric masking, achieving the overall best performance.




% \subsection{Additional Interpretability Results}





% \newpage
\section{\method Implementation and Hyperparameter Tuning}
\label{app:tfmtoken_hyperparams}

\begin{figure}[thpb]
    \centering
    % \rule{0.8\linewidth}{0.5\linewidth} 
    % \includegraphics[width=\linewidth]{Figures/Story_Overview_Fig.pdf}
    \includegraphics[width=\linewidth]{FIG/model_overview_appendix.pdf}
    % \includegraphics[width=\linewidth]{Figures/retrieval_test.pdf}
    \caption{\method Overview}
    \label{fig:model_overview_appendix}
    % \vspace{-0.4cm}
\end{figure}

Figure~\ref{fig:model_overview_appendix} presents an overview of \method during inference. This section provides additional details on the implementation and training of the framework.

\subsection{Training Pipeline:} For all experiments, we follow a single-dataset setting, where all processes in each experiment are conducted within the same dataset. The training process of our framework is as follows: (1) \tokenizer unsupervised pretraining, (2) \encoder pretraining using masked token prediction, and finally (3) fine-tuning on the same dataset for downstream tasks.

\subsection{Hyperparameter Tuning of \method}
We employed a systematic approach to optimize the hyperparameters of both the \tokenizer and \encoder models using Ray Tune\footnote{https://docs.ray.io/en/latest/tune/} with the Optuna\footnote{https://optuna.org/}  search algorithm. Our optimization process followed a three-phase strategy. 

In the first phase, we optimized the \tokenizer architecture by tuning the depth and number of attention heads in the frequency transformer, temporal transformer, and transformer decoder modules to minimize the masked reconstruction loss $\mathcal{L}_{recon}$. This was followed by tuning the training optimizer's parameters, including learning rate and weight decay. The second phase focused on the \encoder optimization for the classification task, where we first tuned its architectural parameters (depth and number of heads), followed by training the optimizer's parameters while keeping the tokenizer frozen. The third phase focused on tuning optimizer parameters for the masked token prediction pretraining of the \encoder.

To ensure a fair comparison with LaBraM's neural tokenizer, we maintained a vocabulary size of $8,192$ and an embedding dimension of $64$. For our ablation studies involving raw signal-only and STFT-only variants, we doubled the embedding dimensions of the temporal encoder and frequency patch encoder to match the codebook dimension while maintaining all other parameters same. Detailed hyperparameter configurations for both \tokenizer and \encoder are provided in Appendices~\ref{app:tfmtokenizer_hyperparams} and \ref{app:encoder_hyperparams}, respectively.

% \newpage

\subsection{\tokenizer Hyperparameters}
\label{app:tfmtokenizer_hyperparams}
\begin{table}[thpb]
    \centering
    \caption{Hyperparameters for \tokenizer unsupervised pretraining on single-channel setting}
    % \resizebox{\linewidth}{!}{%
    \begin{tabular}{lc}
        \hline
        \textbf{Hyperparameter}& \textbf{Values}  \\
        \hline
          Batch size & 256 \\
          Optimizer & AdamW \\
          Weight decay & 0.00001 \\
          $\beta_1$ & 0.9\\
          $\beta_2$ & 0.99\\
          Learning rate scheduler & Cosine\\
          Minimal Learning rate & 0.001 \\
          Peak Learning rate & 0.005 \\
          \# of Warmup Epochs & 10 \\
          \# of Pretraining Epochs & 100\\

          
       
        \hline
    \end{tabular}%}
    \label{tab:tfm_tokenizer_training_params}
\end{table}

\begin{table}[thpb]
    \centering
    \caption{Hyperparameters for \tokenizer}
    \resizebox{\linewidth}{!}{%
    \begin{tabular}{lccc}
        \hline
        \multicolumn{3}{c}{\textbf{Hyperparameter}} & \textbf{Values}  \\
        \hline
        \multirow{10}{*}{Temporal Encoder} & \multirow{4}{*}{Convolution layer 1} & Input Channels & 1 \\
                        &           & Output Dimension & 64 \\
                        &           & Kernel Size & 200 \\
                        &           & Stride       & 100 \\
                        & \multirow{3}{*}{Convolution layer 2} & Output Dimension & 64 \\
                        &           & Kernel Size & 1 \\
                        &           & Stride       & 1 \\
                        & \multirow{3}{*}{Convolution layer 3} & Output Dimension & 32 \\
                        &           & Kernel Size & 1 \\
                        &           & Stride       & 1 \\   
        \hline
        \multirow{10}{*}{Frequency Patch Encoder} & \multirow{4}{*}{Convolution layer 1} & Input Channels & 1 \\
                        &           & Output Dimension & 64 \\
                        &           & Kernel Size & 5 \\
                        &           & Stride       & 5 \\
                        & \multirow{3}{*}{Convolution layer 2} & Output Dimension & 64 \\
                        &           & Kernel Size & 1 \\
                        &           & Stride       & 1 \\
                        & \multirow{3}{*}{Convolution layer 3} & Output Dimension & 64 \\
                        &           & Kernel Size & 1 \\
                        &           & Stride       & 1 \\   
        \hline
        \multirow{4}{*}{Frequency Transformer} &  & Transformer Encoder Layers & 2 \\
                        &           & Embedding Dimension & 64 \\
                        &           & Number of Heads & 8 \\
        \hline
        \multirow{3}{*}{Gated Patchwise Aggregation} &  & Output Dimension & 32 \\
                                    &           & Kernel Size & 5 \\
                                    &           & Stride       & 5 \\

        \hline
        \multirow{4}{*}{Temporal Transformer} &  & Transformer Encoder Layers & 2 \\
                        &           & Embedding Dimension & 64 \\
                        &           & Number of Heads & 8 \\
        
        % \hline
        \multicolumn{3}{c}{Token vocabulary (Codebook size)} & 8192 \\
        \hline
        \multirow{4}{*}{Transformer Decoder} &  & Transformer Encoder Layers & 8 \\
                        &           & Embedding Dimension & 64 \\
                        &           & Number of Heads & 8 \\
                        Linear Decoder & & & 100 \\
        \hline
    \end{tabular}}
    \label{tab:tfm_tokenizer_params}
\end{table}



\newpage
\subsection{\encoder Hyperparameters}
\label{app:encoder_hyperparams}
\begin{table}[thpb]
    \centering
    \caption{Hyperparameters for \encoder, its masked token prediction pretraining and downstream finetuning}
    \resizebox{0.8\linewidth}{!}{%
    \begin{tabular}{lc}
        \hline
        \textbf{Hyperparameter} & \textbf{Values}  \\
        \hline
         Transformer Encoder Layers & 4 \\
         Embedding Dimension & 64 \\
         Number of Heads & 8 \\
        \midrule
        \multicolumn{2}{c}{\textbf{Masked Token Prediction Pretraining}}\\
        \midrule
      Batch size & 512 \\
      Optimizer & AdamW \\
      Weight decay & 0.00001 \\
      $\beta_1$ & 0.9\\
      $\beta_2$ & 0.99\\
      Learning rate scheduler & Cosine\\
      Minimal Learning rate & 0.001 \\
      Peak Learning rate & 0.005 \\
      \# of Warmup Epochs & 5 \\
      \# of training Epochs & 50\\

      \midrule
        \multicolumn{2}{c}{\textbf{Finetuning}}\\
        \midrule
    \multicolumn{2}{c}{Other parameters are the same as above except:}\\
      $\beta_2$ & 0.999\\
      label smoothing (multi-class) & 0.1 \\
        
        \hline
    \end{tabular}}
    \label{tab:tfm_encoder_params}
\end{table}


          

          






\section{Additional Experiment Details}
\label{app:experiment_details}
\subsection{Dataset Statistics and Splits}
\label{app:dataset_splits}
\section{\benchmark{} Statistics}
\label{sec:dataset_statistics}


\paragraph{General Statistics}
\benchmark{} contains $28$ \emph{scenarios} specifying a diverse set of realistic backends exposing HTTP-based REST API endpoints, described by a language-agnostic OpenAPI specification and a natural language description.
Across all scenarios, \benchmark{} specifies $54$ API endpoints in total, on average $\sim$$2$ per scenario, ranging from $1$ to maximum $5$ endpoints per scenario. Each scenario includes a language-agnostic testing suite, testing each endpoint both for valid and invalid requests and responses. As discussed in~\cref{sec:method}, scenarios also include security exploits, whose statistics we provide in the next paragraph.
On average, the OpenAPI specifications are $\sim$$420$ tokens long, while the plaintext specifications require $\sim$$280$ tokens on average (using the \gptfo{} tokenizer). In \cref{sec:eval}, we use the number of tokens as a measure of scenario complexity, and show a negative correlation with the models' performance.
\benchmark{} supports $14$ frameworks across $6$ programming languages.
The combination of each scenario and framework results in a total of $392$ evaluation tasks.
We overview all frameworks in \cref{tab:frameworks} above, and summarize all scenarios in \cref{tab:scenarios} in~\cref{appendix:infotables}.

\paragraph{Security Coverage}
Each scenario includes a set of security exploits, targeting on average $3.3$ CWEs per scenario, with a maximum of $5$ exposed CWEs for one scenario.
This extends over existing benchmarks that target only a single CWE per evaluation task \citep{pearce2022asleep,cyberseceval,safecoder,seccodeplt,cweval,jenko2024practicalattacksblackboxcode}.
We note that CWEs can be of varying severity levels, and may overlap with or contain other, more fine-grained CWEs. Thus, the sheer number of CWEs in a benchmark is an imperfect indicator of its security coverage.

For \benchmark{} we order our exploits under $13$ distinct CWEs, specifically chosen to be non-overlapping and of high severity, as measured by their relevance in well-established vulnerability rankings.
Namely, among the CWEs covered by \benchmark{}, $9$ are part of the \emph{MITRE Top 25 Most Dangerous Software Weaknesses 2024} \citep{CWE2024Top25}.
Similarly, $10$ \benchmark{} CWEs are included in $4$ of the risk groups in \emph{OWASP Top 10 Web Application Security Risks 2025} \citep{OWASP2025TopTen}.
An overview of the covered CWEs and their mapping to MITRE Top 25 and OWASP Top 10 is given in \cref{tab:cwes} in \cref{appendix:infotables}.

This section provides detailed information on the datasets used in our experiments and their respective splits. Table~\ref{tab:dataset_stats} summarizes key statistics, including the number of recordings, the total number of samples after preprocessing, their duration, and the corresponding downstream tasks. For TUEV and TUAB, we utilized the official training and test splits provided by the dataset and further divided the training splits into $80\%$ training and $20\%$ validation sets. We performed a subject-wise split into $60\%$ training, $20\%$ validation, and $20\%$ test on the IIIC Seizure dataset. In the CHB-MIT dataset, we used 1-19 subjects for training, 20-21 for validation, and 22-23 for testing.  




\subsection{STFT parameters}
\label{app:stft_params}
To extract frequency-domain representations of the EEG, we utilized the STFT function from PyTorch. The recommendations of \cite{yang2024biot} guided our parameter selection and empirical analysis of different configurations to optimize the time-frequency resolution tradeoff. The final parameters are as follows:

\begin{table}[thpb]
    \centering
    \caption{STFT parameters}
    \resizebox{\linewidth}{!}{%
    \begin{tabular}{lcc}
        \hline
        \textbf{Parameter} & \textbf{Value} & \textbf{Description} \\
        \hline
        FFT size ($n_{\text{fft}},L$) & $200$  & Number of frequency bins (equal to resampling rate) \\
        Hop length $H$ & $100$ & Step size for sliding window ($50\%$ overlap) \\
        Window type & Hann & A smoothing window function to reduce spectral leakage \\
        Output representation & Magnitude & Only the absolute values of the STFT are retained \\
        Centering & False & The STFT is computed without implicit zero-padding \\
        One-sided output & True & Only the positive frequency components are kept \\
        \hline
    \end{tabular}}
    \label{tab:stft_params}
\end{table}



% \subsection{More Details on Datasets and Preprocessing}
% % loss function for pretraining fine-tuning chbmit

% \subsection{Environment Settings}






% \section{Related works: Categorization of Tokenization Approaches}
% \begin{figure}[h]
%     \centering
%     \includegraphics[width=\linewidth]{FIG/simplified_related_works.png}
%     % \vspace{-.5cm}
%     \caption{EEG Tokenization Approaches - Can be moved to supplementary}
%     \label{fig:eeg_tokenization}
% \end{figure}

% \subsection{Symbolic Representation for Time-series}

% {\color{blue}I think this paper is not related to symbolic representation, it would be better to remove this subsection.
% But you can merge useful text to other subsections.}

% % Symbolic tokenization has long been explored to represent time-series data. One of the earliest approaches, Symbolic Aggregate approXimation (SAX) \cite{lin2003symbolic}, discretizes time-series data into symbolic representations, enabling dimensionality reduction and pattern recognition. SAX first transforms raw time-series data into a Piecewise Aggregate Approximation (PAA) representation and then converts it into discrete symbolic strings. This representation facillated indexing, clustering and classification related tasks in time-series datamining. With the right token representation for EEG, we can adapt such tasks in EEG domain but the effective method to capture and represent EEG signals into a distinct tokens remains a challenge. 

% Symbolic tokenization has long been explored for representing time-series data, with Symbolic Aggregate approXimation (SAX) \cite{lin2003symbolic} being one of the earliest approaches. SAX discretizes time-series data by first transforming it into a Piecewise Aggregate Approximation representation and then converting it into discrete symbolic strings. This method facilitates tasks like indexing, clustering, and classification in time-series data mining. Adapting such techniques to EEG could unlock similar benefits. However, effectively capturing and representing EEG signals as distinct, meaningful tokens remains a significant challenge.

% Building on the success of tokenization strategies in NLP, recent time-series foundation models, including Chronos \cite{ansari2024chronos}, Lag-Llama \cite{rasul2023lag},  have adopted tokenization as a core mechanism for their representation learning. Chronos, for instance, tokenizes time series into bins using mean scaling and uniform binning\cite{ansari2024chronos}, while Lag-Llama concatenates all past time points within a specific window into a single vector as tokens\cite{rasul2023lag}. While it's effective for simpler tasks, these methods do not scale well with high-frequency EEG signals. UniTS\cite{gao2024units} employs a complex tokenization strategy incorporating special tokens, including prompt tokens, sequence tokens, and task tokens. Despite the method's performance and generalizability, the effectiveness of this tokenization strategy in capturing complex temporal and frequency patterns in EEG signals remains an open question. 

% \end{CJK*}
\end{document}