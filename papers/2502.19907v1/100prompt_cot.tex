\begin{table*}[t]
% \resizebox{0.95\textwidth}{!}{
\small
    \begin{tabularx}{\linewidth}{X}
    \toprule
    \color{gray}{/* \textit{Task prompt} */}\\
    Please solve the question step by step based on First-Order Logic rules such as Modus Ponens, determine whether the hypothesis is true, false, or unknown based on these premises.
    \\
    \color{gray}{/* \textit{Example} */}\\
    \textbf{Premises:} \\
1. Walter Folger Brown was an American politician and lawyer who served as the postmaster general. \\
2. Walter Folger Brown graduated from Harvard University with a Bachelor of Arts. \\
3. While they were both in Toledo, Walter Folger Brown's father practiced law with Walter Folger Brown. \\
4. Katherin Hafer married Walter Folger Brown. \\

\textbf{Premises-FOL:} \\
1. AmericanPolitician(walterBrown) \(\land\) Lawyer(walterBrown) \(\land\) ServedAs(walterBrown, postMasterGeneral) \\
2. Graduated(walterBrown, harvard) \(\land\) GraduatedWith(walterBrown, bachelorsOfArt) \\
3. \(\exists t\) (In(walterBrown, toledo, t) \(\land\) In(walterBrownFather, toledo, t) \(\land\) PracticedLawTogether(walterBrown, walterBrownFather, t)) \\
4. Married(katherinHafer, walterBrown) \\

\textbf{Hypothesis:} \\
Walter Folger Brown was not in Toledo. \\

\textbf{Hypothesis-FOL:} \\
\(\exists t\) (\(\neg\)In(walterBrownFather, toledo, t)) \\
\textbf{Label:} \\
False \\
\textbf{Solution:} \\
\textbf{Step 1: Analyze Walter Folger Brown's presence in Toledo} \\
The third premise states that there exists a time t such that: \\
In(walterBrown, toledo, t) \(\land\) In(walterBrownFather, toledo, t) \(\land\) PracticedLawTogether(walterBrown, walterBrownFather, t). \\
This means that Walter Folger Brown and his father were both in Toledo at the same time, and they practiced law together there.\\
Thus, we have clear evidence that Walter Folger Brown was indeed in Toledo at some point. \\

\textbf{Step 2: Analyze the hypothesis's claim} \\
The hypothesis states that Walter Folger Brown was not in Toledo, represented in FOL as: \\
\(\exists t\) (\(\neg\)In(walterBrownFather, toledo, t)) \\
However, this contradicts the third premise, which explicitly states that both Walter Folger Brown and his father were in Toledo at the same time. \\
Therefore, the hypothesis that Walter Folger Brown was not in Toledo is False based on the premises. \\

\textbf{Final Hypothesis:} \\
The hypothesis "Walter Folger Brown was not in Toledo" is False.
    \\
    \color{gray}{/* \textit{Input} */}\\
    ---INPUT---\\
    Premises:\\
    \{\textbf{Given\_premises and premises-FOL}\}\\
    Hypothesis:\\
    \{\textbf{Given\_hypothesis and hypothesis-FOL}\}\\
    Label:\\
    \{\textbf{Given\_label}\}\\
    ---OUTPUT---\\
    \bottomrule
    \end{tabularx}
    % }
  \caption{
    The prompt for generating a step-by-step Chain of Thought (CoT) process based on premises, hypothesis, and label. Different datasets have different sets of labels and examples. For convenience, we only show the prompt on FOLIO here. In practice, we listed two examples, but for brevity and clarity in display, we only present one.
  }
  \label{tab:100prompt_cot}
\end{table*}

