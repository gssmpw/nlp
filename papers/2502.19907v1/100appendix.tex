\subsection{Details of Generating Solutions} 
\label{sec:appendix_1}
In Sec. \ref{sec:Answer order Augmentation}, 
We discuss how to generate step-by-step solutions through \( D = \{P, C, L\} \). Specifically, we follow these steps:

(1) For datasets that do not have first-order logic (FOL) expressions, such as RuleTaker and LogicNLI, we extract their premises and conclusions, and use GPT-4o-mini with prompts as shown in Tab. \ref{tab:100prompt_FOL} to convert them into corresponding FOL representations. FOLIO, on the other hand, already includes FOL expressions, so no conversion is required.

(2) The FOL-enhanced premises and ground truth labels are input into the model, prompting it to generate step-by-step solutions. As shown in the prompt in Tab. \ref{tab:100prompt_cot}, we add two domain-specific examples from each dataset to the prompt, requiring the model to clearly define the purpose and reasoning for each step, eventually leading to the final conclusion. The Task prompt specifies the possible values for the label. Specifically, in FOLIO, the label values are \{True, False, Unknown\}, in RuleTaker they are \{entailment, not entailment\}, and in LogicNLI they are \{entailment, neutral, self\_contradiction, contradiction\}.

(3) The model then reprocesses the generated solutions, using prompts like the one shown in Tab. \ref{tab:100prompt_DAG}, to extract the premise indices and premise step indices used in each reasoning step.

\begin{table*}[]
% \resizebox{0.95\textwidth}{!}{
\small
    \begin{tabularx}{\linewidth}{X}
    \toprule
    \color{gray}{/* \textit{Task prompt} */}\\
    Please parse the context and question into First-Order Logic formulas. Please use symbols as much as possible to express, such as \( \forall \), \( \land \), \( \rightarrow \), \( \oplus \), \( \neg \), etc. 
    \\
    \color{gray}{/* \textit{Example} */}\\
\textbf{Premises:} \\
If a cartoon character is yellow, it is from the Simpsons. \\
If a cartoon character is from Simpsons, then it is loved by children. \\
Ben is ugly or yellow. \\
Ramon being real is equivalent to Rhett being not modest and Philip being lazy. \\

\textbf{Hypothesis:} \\
James does not have lunch in the company. \\

\textbf{Premises-FOL:} \\
\(\forall x (Yellow(x) \rightarrow Simpsons(x))\)

\(\forall x (Simpsons(x) \rightarrow Loved(x))\)

\((Yellow(ben) \vee Ugly(ben))\)

\(real(Ramon) \iff (modest(Rhett) \land lazy(Philip))\)

\textbf{Hypothesis-FOL:} \\
\(\neg HasLunch(james, company)\)
    \\
    \color{gray}{/* \textit{Input} */}\\
    ---INPUT---\\
    Premises:\\
    \{\textbf{Given\_premises}\}\\
    Hypothesis:\\
    \{\textbf{Given\_hypothesis}\}\\
    ---OUTPUT---\\
    \bottomrule
    \end{tabularx}
    % }
  \caption{
    The prompt for generating First-Order Logic (FOL) expressions corresponding to natural language logical propositions.
  }
  \label{tab:100prompt_FOL}
\end{table*}


\begin{table*}[t]
% \resizebox{0.95\textwidth}{!}{
\small
    \begin{tabularx}{\linewidth}{X}
    \toprule
    \color{gray}{/* \textit{Task prompt} */}\\
    Please solve the question step by step based on First-Order Logic rules such as Modus Ponens, determine whether the hypothesis is true, false, or unknown based on these premises.
    \\
    \color{gray}{/* \textit{Example} */}\\
    \textbf{Premises:} \\
1. Walter Folger Brown was an American politician and lawyer who served as the postmaster general. \\
2. Walter Folger Brown graduated from Harvard University with a Bachelor of Arts. \\
3. While they were both in Toledo, Walter Folger Brown's father practiced law with Walter Folger Brown. \\
4. Katherin Hafer married Walter Folger Brown. \\

\textbf{Premises-FOL:} \\
1. AmericanPolitician(walterBrown) \(\land\) Lawyer(walterBrown) \(\land\) ServedAs(walterBrown, postMasterGeneral) \\
2. Graduated(walterBrown, harvard) \(\land\) GraduatedWith(walterBrown, bachelorsOfArt) \\
3. \(\exists t\) (In(walterBrown, toledo, t) \(\land\) In(walterBrownFather, toledo, t) \(\land\) PracticedLawTogether(walterBrown, walterBrownFather, t)) \\
4. Married(katherinHafer, walterBrown) \\

\textbf{Hypothesis:} \\
Walter Folger Brown was not in Toledo. \\

\textbf{Hypothesis-FOL:} \\
\(\exists t\) (\(\neg\)In(walterBrownFather, toledo, t)) \\
\textbf{Label:} \\
False \\
\textbf{Solution:} \\
\textbf{Step 1: Analyze Walter Folger Brown's presence in Toledo} \\
The third premise states that there exists a time t such that: \\
In(walterBrown, toledo, t) \(\land\) In(walterBrownFather, toledo, t) \(\land\) PracticedLawTogether(walterBrown, walterBrownFather, t). \\
This means that Walter Folger Brown and his father were both in Toledo at the same time, and they practiced law together there.\\
Thus, we have clear evidence that Walter Folger Brown was indeed in Toledo at some point. \\

\textbf{Step 2: Analyze the hypothesis's claim} \\
The hypothesis states that Walter Folger Brown was not in Toledo, represented in FOL as: \\
\(\exists t\) (\(\neg\)In(walterBrownFather, toledo, t)) \\
However, this contradicts the third premise, which explicitly states that both Walter Folger Brown and his father were in Toledo at the same time. \\
Therefore, the hypothesis that Walter Folger Brown was not in Toledo is False based on the premises. \\

\textbf{Final Hypothesis:} \\
The hypothesis "Walter Folger Brown was not in Toledo" is False.
    \\
    \color{gray}{/* \textit{Input} */}\\
    ---INPUT---\\
    Premises:\\
    \{\textbf{Given\_premises and premises-FOL}\}\\
    Hypothesis:\\
    \{\textbf{Given\_hypothesis and hypothesis-FOL}\}\\
    Label:\\
    \{\textbf{Given\_label}\}\\
    ---OUTPUT---\\
    \bottomrule
    \end{tabularx}
    % }
  \caption{
    The prompt for generating a step-by-step Chain of Thought (CoT) process based on premises, hypothesis, and label. Different datasets have different sets of labels and examples. For convenience, we only show the prompt on FOLIO here. In practice, we listed two examples, but for brevity and clarity in display, we only present one.
  }
  \label{tab:100prompt_cot}
\end{table*}


\begin{table*}[t]
% \resizebox{0.95\textwidth}{!}{
\small
    \begin{tabularx}{\linewidth}{X}
    \toprule
    \color{gray}{/* \textit{Task prompt} */}\\
    I will provide you with a description of the question and its answer, and the condition of the question is specific. The answer is done in steps. I hope you can extract the conditions and prerequisite steps used in each step of the answer. Please note that I am not asking you to regenerate the answer yourself, but rather to extract the conditions and prerequisite steps used in each step from the answer I have given you. Meanwhile, the conditions used in the steps are quite explicit, but the prerequisite steps used are quite implicit. I hope you can understand and summarize the prerequisite steps used in each step. Your answer should only include Conditions and prerequisite steps used.
    \\
    \color{gray}{/* \textit{Example} */}\\
    \textbf{Question:} \\
\textbf{Premises:} \\
1. Lana Wilson directed After Tiller, The Departure, and Miss Americana. \\
2. If a film is directed by a person, the person is a filmmaker. \\
3. After Tiller is a documentary. \\
4. The documentary is a type of film. \\
5. Lana Wilson is from Kirkland. \\
6. Kirkland is a US city. \\
7. If a person is from a city in a country, the person is from the country. \\
8. After Tiller is nominated for the Independent Spirit Award for Best Documentary.

\textbf{Premises-FOL:} \\
1. DirectedBy(afterTiller, lanaWilson) \( \land \) DirectedBy(theDeparture, lanaWilson) \\
\( \land \) DirectedBy(missAmericana, lanaWilson) \\
2. \( \forall \) x \( \forall \) y (DirectedBy(x, y) \( \rightarrow \) Filmmaker(y)) \\
3. Documentary(afterTiller) \\
4. \( \forall \) x (Documentary(x) \( \rightarrow \) Film(x)) \\
5. From(lanaWilson, kirkland) \\
6. In(kirkland, unitedStates) \\
7. \( \forall \) x \( \forall \) y \( \forall \) z ((From(x, y) \( \land \) In(y, z)) \( \rightarrow \) From(x, z)) \\
8. Nomination(afterTiller, theIndependentSpiritAwardForBestDocumentary)

\textbf{Conclusion:} \\
Miss Americana is not directed by a filmmaker from Kirkland.

\textbf{Conclusion-FOL:}
\( \neg \) \(\exists t\) x (Filmmaker(x) \( \land \) From(x, kirkland) \( \land \) DirectedBy(missAmericana, x))
    \\

\textbf{Answer:} \\
Step 1: Analyze the premises regarding Lana Wilson's role as a filmmaker \\
From premise 1, we know that Lana Wilson directed Miss Americana. Therefore, we can conclude that Lana Wilson is a filmmaker based on premise 2, which states that if a film is directed by a person, that person is a filmmaker. Thus, we have:
    DirectedBy(missAmericana, lanaWilson) \( \rightarrow \) Filmmaker(lanaWilson)

Step 2: Analyze Lana Wilson's origin \\
From premise 5, we know that Lana Wilson is from Kirkland. Therefore, we can conclude:
    From(lanaWilson, kirkland)

Step 3: Combine the information \\
Since we have established that Lana Wilson is a filmmaker and she is from Kirkland, we can conclude:
    Filmmaker(lanaWilson) \( \land \) From(lanaWilson, kirkland)

Step 4: Analyze the conclusion's claim \\
The conclusion states that Miss Americana is not directed by a filmmaker from Kirkland, represented in FOL as:
    \( \neg \) \(\exists t\) x (Filmmaker(x) \( \land \) From(x, kirkland) \( \land \) DirectedBy(missAmericana, x))

However, we have already established that Lana Wilson, who directed Miss Americana, is indeed a filmmaker from Kirkland. This directly contradicts the conclusion.

\textbf{Final Conclusion:} \\
The conclusion "Miss Americana is not directed by a filmmaker from Kirkland" is False.


\textbf{Conditions and prerequisite steps used:}\\
step 1: condition 1, condition 2\\
step 2: condition 5\\
step 3: step 1, step 2\\
step 4: step 3\\

    \color{gray}{/* \textit{Input} */}\\
    ---INPUT---\\
    Question:\\
    \{\textbf{Given\_question}\}\\
    Answer:\\
    \{\textbf{Given\_answer}\}\\
    ---OUTPUT---\\
    \bottomrule
    \end{tabularx}
    % }
  \caption{
    The prompt for extracting Conditions and prerequisite steps used in each step of step-by-step solutions.
  }
  \label{tab:100prompt_DAG}
\end{table*}



\subsection{Kendall Tau Distance}
% 不同的tau的数据示例
%4.4.1里不同的tau的比例
In our study, we investigate the effects of premise order transformations by using the Kendall tau distance \( \tau \). This coefficient measures the correlation between two ordered lists, providing a quantitative way to assess how much one order differs from another. We use \( \tau \) to categorize various permutations of premise orders and assess their impact on model performance.

The Kendall tau coefficient \( \tau \) is calculated as follows:
\[
\tau = \frac{C - D}{\binom{n}{2}}
\]
where \( C \) is the number of concordant pairs (pairs of items that are in the same relative order in both lists), and \( D \) is the number of discordant pairs (pairs that are in opposite order in both lists). The total number of possible pairs is \( \binom{n}{2} \), where \( n \) is the number of items being compared.

We divide \( \tau \) values into 10 groups, each spanning a 0.2 range within the interval [-1, 1). A \( \tau \) value of 1 indicates that the order of the premises is exactly as required for the reasoning process, while -1 indicates a complete reversal of order. A \( \tau \) value of 0 indicates that the order is completely random, with no correlation to the original sequence.

For example, if the original premise order is \( [P_1, P_2, P_3, \dots, P_n] \), a permutation function \( \sigma \) might rearrange it to \( [P_3, P_1, P_2, \dots, P_n] \). This process allows us to explore different levels of order perturbation, with the goal of analyzing how such variations affect model performance. Examples of premise orders corresponding to different \( \tau \) values can be seen in Fig. \ref{fig:tau}.
\begin{figure}[t] 
    \centering
        \includegraphics[width=0.5\textwidth]{tau_fig.pdf}
    % \captionsetup{font={small}} 
    \caption{An example of showing the arrangement of premises with different tau values. The tau values do not represent exact values but rather the closest intervals for demonstration purposes.}
    \label{fig:tau}
\end{figure}



% \subsection{TFI}
% % 不同的TFI的数据示例
% % 我们介绍了Topological Freedom Index (TFI)指数,下面我们详细的介绍
% In Sec. \ref{sec:non_DAG}, 
