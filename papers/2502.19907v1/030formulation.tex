% Problem Formulation
% Problem Defintion:D = {P, C, l} P = {P1, P2, ... Pn}——>Premise set,C——>Conclusion。I = {l1, l2, ..., ln}——>label set
In this paper, we formulate the problem of logical reasoning in a unified representation. Let \( D = \{P, C, L\} \) represent a logical reasoning problem, where \( P = \{P_1, P_2, \dots, P_n\} \) is the set of premises, \( C \) is the conclusion, and \( L \) is the label, which takes a value from a finite set, such as \( \{ \text{true}, \text{false}, \text{uncertain} \} \), indicating whether \( C \) can be logically inferred from \( P \). In step-based data augmentation, we extend the representation to include a solution \( S = \{S_1, S_2, \dots, S_m\} \), where \( S \) consists of reasoning steps that derive the conclusion from the premises. This process can be abstracted as a directed acyclic graph (DAG). Typical logical reasoning datasets only provide labels. Therefore, we construct \( S \) ourselves. The specific construction of \( S \) will be detailed in Sec. \ref{sec:Answer order Augmentation}.
