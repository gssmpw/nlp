\paragraph{Mathematical notations.}
\looseness=-1
The set of probability distributions over a set $\S$ is denoted by $\sP(\S)$.
For integers $a \leq b$, let $\bbrack{a,b}\df \brace*{a,\dots, b}$, and $\bbrack{a, b} \df \emptyset$ if $a > b$.
For $\bx \in \R^N$, its $n$-th element is $\bx_n$ or $\bx(n)$.
The clipping function $\clip\brace{\bx, a, b}$ returns $\bx'$ with $\bx'_i = \min\brace{\max\brace{\bx_i, a}, b}$ for each $i$.
We define $\bzero \df \paren{0, \ldots, 0}^\top$ and $\bone \df \paren{1, \ldots, 1}^\top$, with dimensions inferred from the context.
All scalar operations and inequalities should be understood point-wise when applied to vectors and functions.
For a positive definite matrix $\bA \in \R^{d\times d}$ and $\bx \in \R^d$, we denote $\norm*{\bx}_\bA = \sqrt{\bx^\top \bA \bx}$.
For positive sequences $\brace{a_n}$ and $\brace{b_n}$ with $n=1,2, \ldots$, we write $a_n=O\left(b_n\right)$ if there exists $C>0$ such that $a_n \leq C b_n$ for all $n \geq 1$, and $a_n = \Omega(b_n)$ for the reverse inequality.
We use $\widetilde{O}(\cdot)$ and $\widetilde{\Omega}(\cdot)$ to further hide the polylogarithmic factors.
Finally, for $\bx \in \R^d$, we denote its softmax distribution as $\softmax(\bx) \in \sP(\bbrack{1, d})$ with its $i$-th component $\softmax(\bx)_i = \exp(\bx_i) / \paren*{\sum_{i} \exp(\bx_i)}$.

