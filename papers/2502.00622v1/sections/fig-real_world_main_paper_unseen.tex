

\vspace{-3mm}
\begin{figure*}[h]
    \begin{minipage}{\textwidth}
        \centering
        \begin{tabular}{c}
            \hspace{-4mm}
            \begin{minipage}{\textwidth}
                \centering
                \includegraphics[width=\columnwidth]{figures/real_world/pushTR/pushTR_baseline/eval_traj_2.png}
            \end{minipage}
            \\
            \hspace{-4mm}
            \begin{minipage}{\textwidth}
                \centering
                \includegraphics[width=\columnwidth]{figures/real_world/pushTR/pushTR_GPC_RANK/eval_traj_2.png}
            \end{minipage}
            \\
            \hspace{-4mm}
            \begin{minipage}{\textwidth}
                \centering
                \includegraphics[width=\columnwidth]{figures/real_world/pushTR/pushTR_GPC_OPT/eval_traj_2.png}
            \end{minipage}
            \\

        \end{tabular}
    \end{minipage}
    \vspace{-2mm}
    \caption{\textsc{Real-world test for out-of-domain scene.} In every test, top row shows trajectories of baseline model ($K=1, M=0$), middle row shows trajectories of \gpcrank ($K=10, M=0$), and last row shows trajectories of \gpcopt ($K=0, M=25$). This displays \nameshort would help to make policy succeed in unseen scene while baseline failed. Remaining results for push-T with collision of R are shown in Appendix~\ref{app:real_world} and supplementary videos.
    \label{fig:fig-real_world_main_paper_unseen}}
\end{figure*}