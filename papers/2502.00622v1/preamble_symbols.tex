%!TEX root = main.tex

% LC: can be inserted
% \newcommand{\qed}{{\hfill $\square$}}

% Format definition
\newrefformat{prob}{Problem\,\ref{#1}}
\newrefformat{def}{Definition\,\ref{#1}}
\newrefformat{sec}{Section\,\ref{#1}}
\newrefformat{sub}{Section\,\ref{#1}}
\newrefformat{prop}{Proposition\,\ref{#1}}
\newrefformat{app}{Appendix\,\ref{#1}}
\newrefformat{alg}{Algorithm\,\ref{#1}}
\newrefformat{cor}{Corollary\,\ref{#1}}
\newrefformat{thm}{Theorem\,\ref{#1}}
\newrefformat{lem}{Lemma\,\ref{#1}}
\newrefformat{fig}{Fig.\,\ref{#1}}
\newrefformat{tab}{Table\,\ref{#1}}

% Problem environment
\newtheorem{theorem}{Theorem}
\newtheorem{problem}{Problem}
\newtheorem{trule}[theorem]{Rule}
\newtheorem{corollary}[theorem]{Corollary}
%\newtheorem{algorithm}{Algorithm}
%\newtheorem{procedure}{\textbf{Procedure}}
\newtheorem{conjecture}[theorem]{Conjecture}
\newtheorem{lemma}[theorem]{Lemma}
\newtheorem{assumption}[theorem]{Assumption}
\newtheorem{definition}[theorem]{Definition}
\newtheorem{proposition}[theorem]{Proposition}
\newtheorem{remark}[theorem]{Remark}
\newtheorem{example}[theorem]{Example}

% Shortcuts
\newcommand{\cf}{\emph{cf.}\xspace}
\newcommand{\Figure}{Fig.}
\newcommand{\bdmath}{\begin{dmath}}
\newcommand{\edmath}{\end{dmath}}
\newcommand{\beq}{\begin{equation}}
\newcommand{\eeq}{\end{equation}}
\newcommand{\bdm}{\begin{displaymath}}
\newcommand{\edm}{\end{displaymath}}
\newcommand{\bea}{\begin{eqnarray}}
\newcommand{\eea}{\end{eqnarray}}
\newcommand{\beal}{\beq \begin{array}{ll}}
\newcommand{\eeal}{\end{array} \eeq}
\newcommand{\beas}{\begin{eqnarray*}}
\newcommand{\eeas}{\end{eqnarray*}}
\newcommand{\ba}{\begin{array}}
\newcommand{\ea}{\end{array}}
\newcommand{\bit}{\begin{itemize}}
\newcommand{\eit}{\end{itemize}}
\newcommand{\ben}{\begin{enumerate}}
\newcommand{\een}{\end{enumerate}}
\newcommand{\expl}[1]{&&\qquad\text{\color{gray}(#1)}\nonumber}

% \newcommand{\insertproof}[1]{%
%     % lem:name -> proof_lem_name.tex
%     \StrSubstitute{#1}{:}{_}[\name] % replace : with _
%     \begin{IEEEproof} %\color[rgb]{0.8,0.8,0.8}
%     \documentclass[journal]{IEEEtran}
\IEEEoverridecommandlockouts
\usepackage{cite}
\usepackage{amsmath,amssymb,amsfonts}
\usepackage{algorithmic}
\usepackage{graphicx}
\usepackage{textcomp}
\usepackage{xcolor}
\usepackage{multirow}
\usepackage{array}
\usepackage{caption}
\usepackage{booktabs}
\usepackage{subcaption}
\usepackage{balance}
\usepackage[capitalise]{cleveref}
\usepackage{tikz}
\usetikzlibrary{fit}
\makeatletter
\tikzset{
  fitting node/.style={
    inner sep=0pt,
    fill=none,
    draw=none,
    reset transform,
    fit={(\pgf@pathminx,\pgf@pathminy) (\pgf@pathmaxx,\pgf@pathmaxy)}
  },
  reset transform/.code={\pgftransformreset}
}
\makeatother

\usepackage{pgfplots}
\pgfplotsset{compat=newest} 
\pgfplotsset{plot coordinates/math parser=false} 

\usepackage[acronyms,nonumberlist,nopostdot,nomain,nogroupskip,acronymlists={hidden}]{glossaries}
\newlength\fheight
\setlength{\fheight}{0.5\columnwidth}
\newlength\fwidth
\setlength{\fwidth}{0.8\columnwidth}

\def\BibTeX{{\rm B\kern-.05em{\sc i\kern-.025em b}\kern-.08em
    T\kern-.1667em\lower.7ex\hbox{E}\kern-.125emX}}
\begin{document}

\title{O-RAN-en-el-Espacio: Habilitando Redes Inteligentes, Abiertas e Interoperables en 6G}

\author{\IEEEauthorblockN{
Eduardo Baena, 
Paolo Testolina, 
Michele Polese, 
Dimitrios Koutsonikolas, 
Josep Jornet, 
Tommaso Melodia}

\IEEEauthorblockA{Instituto para el Internet Inalámbrico de las Cosas, Universidad Northeastern, Boston, MA, EE.UU.}}

\maketitle

\begin{abstract}
Aunque las redes no terrestres (NTN) han avanzado significativamente en la expansión de la conectividad, su integración con las redes terrestres sigue siendo un desafío complejo debido a la falta de marcos de gestión cohesivos. La naturaleza descentralizada de las NTN, junto con los mecanismos de control fragmentados existentes y la falta de interfaces de gestión estandarizadas, complica la coordinación entre los dominios terrestre y orbital, resultando en ineficiencias críticas y discontinuidades operativas. Este artículo presenta la arquitectura O-RAN-en-el-Espacio, un marco novedoso diseñado para unificar los sistemas terrestres y no terrestres a través de una gestión jerárquica y distribuida. En su núcleo, el "SpaceRIC" permite la coordinación a nivel de constelación, integrando perfectamente las directivas estratégicas de los sistemas terrestres impulsados por IA y Gemelos Digitales con las operaciones en tiempo real de los satélites. La arquitectura incorpora aplicaciones distribuidas (DAPP y SAPP), lo que permite bucles de control adaptativos y garantiza la robustez operativa bajo diversas condiciones de red. Un enfoque dinámico de mapeo de interfaz-enlace mejora la flexibilidad al alinear las funciones de red con las demandas específicas de las aplicaciones, incluso bajo escenarios de conectividad intermitente y recursos limitados. Los resultados de simulaciones validan su viabilidad al demostrar el cumplimiento de los requisitos de retardo de señalización, destacando su potencial para habilitar una gestión global robusta y eficiente de la conectividad hacia las redes 6G.
\end{abstract}

\begin{IEEEkeywords}
NTN, O-RAN, Integración, 6G, Gestión
\end{IEEEkeywords}

\glsresetall

\section{Introducción}
La creciente demanda de conectividad global de alta capacidad y confiabilidad ha convertido a las redes no terrestres (NTN) en un componente clave de los sistemas de comunicación futuros. Estas redes, que incluyen satélites en órbitas bajas (LEO), medias (MEO), y geoestacionarias (GEO), así como plataformas de gran altitud (HAP), amplían la conectividad a regiones remotas y desatendidas, superando las limitaciones de la infraestructura terrestre. Sin embargo, la integración de las redes NTN con las terrestres sigue siendo compleja debido a sus diferencias operativas y restricciones de recursos.

Los satélites en órbita baja, por ejemplo, operan en entornos altamente dinámicos donde su movimiento constante requiere traspasos frecuentes y actualizaciones continuas a los sistemas de gestión de conectividad. Estas dinámicas introducen desafíos como latencia variable, conectividad intermitente y pérdidas frecuentes de línea de visión (LoS), lo que afecta la continuidad del servicio y compromete la estabilidad de la red.

\section{Arquitectura Propuesta}
\label{sec:architecture}
La arquitectura O-RAN-en-el-Espacio aborda los principales desafíos de integración de las NTN con los sistemas terrestres compatibles con O-RAN, enfocándose en un marco de control jerárquico que une la infraestructura terrestre en la nube y el cómputo en el borde a bordo de los satélites. Esta integración permite una asignación dinámica de recursos y distribución de tareas a lo largo de los segmentos terrestre y orbital, garantizando eficiencia operativa bajo condiciones dinámicas y reduciendo la dependencia de la conectividad terrestre continua.

\subsection{Componentes Clave}
\textbf{Nube Terrestre:} La nube terrestre gestiona funciones de red de alto nivel, incluyendo planificación a largo plazo, modelado de gemelos digitales y optimización de políticas. La capacidad computacional de esta capa soporta procesos no críticos en tiempo real, como entrenamiento de modelos de aprendizaje automático y análisis de datos a gran escala.

\textbf{Cómputo en el Borde a Bordo:} Los recursos a bordo, como CPU y GPU, son gestionados por un middleware avanzado que asigna recursos dinámicamente según la demanda. Este middleware, junto con una capa de virtualización, facilita el despliegue de múltiples funciones de red y garantiza la adaptabilidad y eficiencia operativa en condiciones de tráfico variables.

\section{Conclusión}
La arquitectura O-RAN-en-el-Espacio proporciona un marco robusto para la integración de redes terrestres y no terrestres, habilitando comunicaciones globales inteligentes, abiertas e interoperables hacia la era 6G.
\end{document}
%     \end{IEEEproof}
% }

% Calligraphic fonts
\newcommand{\calA}{{\cal A}}
\newcommand{\calB}{{\cal B}}
\newcommand{\calC}{{\cal C}}
\newcommand{\calD}{{\cal D}}
\newcommand{\calE}{{\cal E}}
\newcommand{\calF}{{\cal F}}
\newcommand{\calG}{{\cal G}}
\newcommand{\calH}{{\cal H}}
\newcommand{\calI}{{\cal I}}
\newcommand{\calJ}{{\cal J}}
\newcommand{\calK}{{\cal K}}
\newcommand{\calL}{{\cal L}}
\newcommand{\calM}{{\cal M}}
\newcommand{\calN}{{\cal N}}
\newcommand{\calO}{{\cal O}}
\newcommand{\calP}{{\cal P}}
\newcommand{\calQ}{{\cal Q}}
\newcommand{\calR}{{\cal R}}
\newcommand{\calS}{{\cal S}}
\newcommand{\calT}{{\cal T}}
\newcommand{\calU}{{\cal U}}
\newcommand{\calV}{{\cal V}}
\newcommand{\calW}{{\cal W}}
\newcommand{\calX}{{\cal X}}
\newcommand{\calY}{{\cal Y}}
\newcommand{\calZ}{{\cal Z}}

% SETS:
\newcommand{\setA}{\textsf{A}}
\newcommand{\setB}{\textsf{B}}
\newcommand{\setC}{\textsf{C}}
\newcommand{\setD}{\textsf{D}}
\newcommand{\setE}{\textsf{E}}
\newcommand{\setF}{\textsf{F}}
\newcommand{\setG}{\textsf{G}}
\newcommand{\setH}{\textsf{H}}
\newcommand{\setI}{\textsf{I}}
\newcommand{\setJ}{\textsf{J}}
\newcommand{\setK}{\textsf{K}}
\newcommand{\setL}{\textsf{L}}
\newcommand{\setM}{\textsf{M}}
\newcommand{\setN}{\textsf{N}}
\newcommand{\setO}{\textsf{O}}
\newcommand{\setP}{\textsf{P}}
\newcommand{\setQ}{\textsf{Q}}
\newcommand{\setR}{\textsf{R}}
\newcommand{\setS}{\textsf{S}}
\newcommand{\setT}{\textsf{T}}
\newcommand{\setU}{\textsf{U}}
\newcommand{\setV}{\textsf{V}}
\newcommand{\setX}{\textsf{X}}
\newcommand{\setY}{\textsf{Y}}
\newcommand{\setZ}{\textsf{Z}}

%General
\newcommand{\smallheading}[1]{\textit{#1}: }
\newcommand{\algostep}[1]{{\small\texttt{#1:}}\xspace}
\newcommand{\etal}{\emph{et~al.}\xspace}
\newcommand{\setal}{~\emph{et~al.}\xspace}
\newcommand{\eg}{\emph{e.g.,}\xspace}
\newcommand{\ie}{\emph{i.e.,}\xspace}
\newcommand{\myParagraph}[1]{{\bf #1.}\xspace}
% \newcommand{\email}[1]{{\smaller \textsf{#1}}}

%Typography
\newcommand{\M}[1]{{\bm #1}} % Face for matrices
\renewcommand{\boldsymbol}[1]{{\bm #1}}
\newcommand{\algoname}[1]{\textsc{#1}} % Name of algorithms

%Editing
\newcommand{\towrite}{{\color{red} To write}\xspace}
\newcommand{\xxx}{{\color{red} XXX}\xspace}
\newcommand{\AC}[1]{{\color{red} \textbf{AC}: #1}}
\newcommand{\LC}[1]{{\color{red} \textbf{LC}: #1}}
\newcommand{\FM}[1]{{\color{red} \textbf{FM}: #1}}
\newcommand{\RT}[1]{{\color{blue} \textbf{RT}: #1}}
\newcommand{\VT}[1]{{\color{blue} \textbf{VT}: #1}}
\newcommand{\hide}[1]{}
\newcommand{\wrt}{w.r.t.\xspace}
\newcommand{\highlight}[1]{{\color{red} #1}}
\newcommand{\tocheck}[1]{{\color{brown} #1}}
\newcommand{\grayout}[1]{{\color{gray} #1}}
\newcommand{\grayText}[1]{{\color{gray} \text{#1} }}
\newcommand{\grayMath}[1]{{\color{gray} #1 }}
\newcommand{\hiddenText}{{\color{gray} hidden text.}}
\newcommand{\hideWithText}[1]{\hiddenText}


\newcommand{\NA}{{\sf n/a}}
\newcommand{\versus}{\scenario{VS}\xspace}

%Basic math symbols
\newcommand{\kron}{\otimes}
\newcommand{\dist}{\mathbf{dist}}
\newcommand{\iter}{\! \rm{iter.} \;}
\newcommand{\leqt}{\!\!\! < \!\!\!}
\newcommand{\geqt}{\!\!\! > \!\!\!}
\newcommand{\mysetminus}{-} % One set minus another
\newcommand{\powerset}{\mathcal{P}}
\newcommand{\Int}[1]{ { {\mathbb Z}^{#1} } }
\newcommand{\Natural}[1]{ { {\mathbb N}^{#1} } }
\newcommand{\Complex}[1]{ { {\mathbb C}^{#1} } }
\newcommand{\one}{ {\mathbf{1}} }
\newcommand{\subject}{\text{ subject to }}
\DeclareMathOperator*{\argmax}{arg\,max}
\DeclareMathOperator*{\argmin}{arg\,min}

%% Norms
\newcommand{\normsq}[2]{\left\|#1\right\|^2_{#2}}
\newcommand{\norm}[1]{\left\| #1 \right\|}
\newcommand{\normsqs}[2]{\|#1\|^2_{#2}}
\newcommand{\infnorm}[1]{\|#1\|_{\infty}}
\newcommand{\zeronorm}[1]{\|#1\|_{0}}
\newcommand{\onenorm}[1]{\|#1\|_{1}}
\newcommand{\lzero}{\ell_{0}}
\newcommand{\lone}{\ell_{1}}
\newcommand{\linf}{\ell_{\infty}}

\newcommand{\E}{{\mathbb{E}}}
\newcommand{\EV}{\mathbb{E}}
\newcommand{\erf}{{\mathbf{erf}}}
\newcommand{\prob}[1]{{\mathbb P}\left(#1\right)}
% \newcommand{\tran}{^{\top}}
\newcommand{\tran}{^{\mathsf{T}}}
\newcommand{\traninv}{^{-\mathsf{T}}}
\newcommand{\diag}[1]{\mathrm{diag}\left(#1\right)}
\newcommand{\trace}[1]{\mathrm{tr}\left(#1\right)}
\newcommand{\conv}[1]{\mathrm{conv}\left(#1\right)}
\newcommand{\polar}[1]{\mathrm{polar}\left(#1\right)}
\newcommand{\rank}[1]{\mathrm{rank}\left(#1\right)}
\newcommand{\e}{{\mathrm e}}
\newcommand{\inv}{^{-1}}
\newcommand{\pinv}{^\dag}
\newcommand{\until}[1]{\{1,\dots, #1\}}
\newcommand{\ones}{{\mathbf 1}}
\newcommand{\zero}{{\mathbf 0}}
\newcommand{\eye}{{\mathbf I}}
\newcommand{\vect}[1]{\left[\begin{array}{c}  #1  \end{array}\right]}
\newcommand{\matTwo}[1]{\left[\begin{array}{cc}  #1  \end{array}\right]}
\newcommand{\matThree}[1]{\left[\begin{array}{ccc}  #1  \end{array}\right]}
\newcommand{\dss}{\displaystyle}
\newcommand{\Real}[1]{ { {\mathbb R}^{#1} } }
\newcommand{\reals}{\Real{}}
\newcommand{\opt}{^{\star}}
\newcommand{\only}{^{\alpha}}
\newcommand{\copt}{^{\text{c}\star}}
\newcommand{\atk}{^{(k)}}
\newcommand{\att}{^{(t)}}
\newcommand{\at}[1]{^{(#1)}}
\newcommand{\attau}{^{(\tau)}}
\newcommand{\atc}[1]{^{(#1)}}
\newcommand{\atK}{^{(K)}}
\newcommand{\atj}{^{(j)}}
\newcommand{\projector}{{\tt projector}}
\newcommand{\setdef}[2]{ \{#1 \; {:} \; #2 \} }
\newcommand{\smalleye}{\left(\begin{smallmatrix}1&0\\0&1\end{smallmatrix}\right)}

%Spaces
\newcommand{\SEtwo}{\ensuremath{\mathrm{SE}(2)}\xspace}
\newcommand{\SE}[1]{\ensuremath{\mathrm{SE}(#1)}\xspace}
\newcommand{\SEthree}{\ensuremath{\mathrm{SE}(3)}\xspace}
\newcommand{\SOtwo}{\ensuremath{\mathrm{SO}(2)}\xspace}
\newcommand{\SOthree}{\ensuremath{\mathrm{SO}(3)}\xspace}
\newcommand{\Othree}{\ensuremath{\mathrm{O}(3)}\xspace}
\newcommand{\SOn}{\ensuremath{\mathrm{SO}(n)}\xspace}
\newcommand{\SO}[1]{\ensuremath{\mathrm{SO}(#1)}\xspace}
\newcommand{\On}{\ensuremath{\mathrm{O}(n)}\xspace}
\newcommand{\sotwo}{\ensuremath{\mathrm{so}(2)}\xspace}
\newcommand{\sothree}{\ensuremath{\mathrm{so}(3)}\xspace}
\newcommand{\intexpmap}[1]{\mathrm{Exp}\left(#1\right)}
\newcommand{\intlogmap}[1]{\mathrm{Log}\left(#1\right)}
\newcommand{\logmapz}[1]{\mathrm{Log}_0(#1)}
\newcommand{\loglikelihood}{\mathrm{log}\calL}
\newcommand{\intprinlogmap}{\mathrm{Log}_0}
\newcommand{\expmap}[1]{\intexpmap{#1}}
\newcommand{\expmaps}[1]{\langle #1 \rangle_{2\pi}}
\newcommand{\biggexpmap}[1]{\left\langle #1 \right\rangle_{2\pi}}
\newcommand{\logmap}[1]{\intlogmap{#1}}
\newcommand{\round}[1]{\mathrm{round}\left( #1 \right)}

% Matrices 
\newcommand{\MA}{\M{A}}
\newcommand{\MB}{\M{B}}
\newcommand{\MC}{\M{C}}
\newcommand{\MD}{\M{D}}
\newcommand{\ME}{\M{E}}
\newcommand{\MJ}{\M{J}}
\newcommand{\MK}{\M{K}}
\newcommand{\MG}{\M{G}}
\newcommand{\MM}{\M{M}}
\newcommand{\MN}{\M{N}}
\newcommand{\MP}{\M{P}}
\newcommand{\MQ}{\M{Q}}
\newcommand{\MU}{\M{U}}
\newcommand{\MR}{\M{R}}
\newcommand{\MS}{\M{S}}
\newcommand{\MI}{\M{I}}
\newcommand{\MV}{\M{V}}
\newcommand{\MF}{\M{F}}
\newcommand{\MH}{\M{H}}
\newcommand{\ML}{\M{L}}
\newcommand{\MO}{\M{O}}
\newcommand{\MT}{\M{T}}
\newcommand{\MX}{\M{X}}
\newcommand{\MY}{\M{Y}}
\newcommand{\MW}{\M{W}}
\newcommand{\MZ}{\M{Z}}
\newcommand{\MSigma}{\M{\Sigma}} 
\newcommand{\MOmega}{\M{\Omega}}
\newcommand{\MPhi}{\M{\Phi}}
\newcommand{\MPsi}{\M{\Psi}}
\newcommand{\MDelta}{\M{\Delta}}
\newcommand{\MLambda}{\M{\Lambda}}

% vectors
\newcommand{\va}{\boldsymbol{a}} 
\newcommand{\vh}{\boldsymbol{h}} 
\newcommand{\vb}{\boldsymbol{b}}
\newcommand{\vc}{\boldsymbol{c}}
\newcommand{\vd}{\boldsymbol{d}}
\newcommand{\ve}{\boldsymbol{e}}
\newcommand{\vf}{\boldsymbol{f}}
\newcommand{\vg}{\boldsymbol{g}}
\newcommand{\vk}{\boldsymbol{k}}
\newcommand{\vl}{\boldsymbol{l}}
\newcommand{\vn}{\boldsymbol{n}}
\newcommand{\vo}{\boldsymbol{o}}
\newcommand{\vp}{\boldsymbol{p}}
\newcommand{\vq}{\boldsymbol{q}}
\newcommand{\vr}{\boldsymbol{r}}
\newcommand{\vs}{\boldsymbol{s}}
\newcommand{\vu}{\boldsymbol{u}}
\newcommand{\vv}{\boldsymbol{v}}
\newcommand{\vt}{\boldsymbol{t}}
\newcommand{\vxx}{\boldsymbol{x}} 
\newcommand{\vy}{\boldsymbol{y}}
\newcommand{\vw}{\boldsymbol{w}}
\newcommand{\vzz}{\boldsymbol{z}} 
\newcommand{\vdelta}{\boldsymbol{\delta}}
\newcommand{\vgamma}{\boldsymbol{\gamma}}  
\newcommand{\vlambda}{\boldsymbol{\lambda}}
\newcommand{\vtheta}{\boldsymbol{\theta}}
\newcommand{\valpha}{\boldsymbol{\alpha}}
\newcommand{\vbeta}{\boldsymbol{\beta}}
\newcommand{\vnu}{\boldsymbol{\nu}}
\newcommand{\vmu}{\boldsymbol{\mu}}
\newcommand{\vepsilon}{\boldsymbol{\epsilon}}
\newcommand{\vtau}{\boldsymbol{\tau}}

%Intrinsic geometry
\newcommand{\Rtheta}{\boldsymbol{R}}
\newcommand{\symf}{f} % Symmetry function

%Angles
\newcommand{\angledomain}{(-\pi,+\pi]}

% Tree, graphs, and cycle basis
\newcommand{\MCB}{\mathsf{MCB}}
\newcommand{\FCM}{\mathsf{FCM}}
\newcommand{\FCB}{\mathsf{FCB}}
\newcommand{\cyclemap}[1]{\calC^{\calG}\left(#1\right)}
\newcommand{\incidencemap}[1]{\calA^{\calG}\left(#1\right)}
\newcommand{\cyclemapk}{\calC^{\calG}_{k}}
\newcommand{\incidencemapij}{\calA^{\calG}_{ij}}
\renewcommand{\ij}{_{ij}}
\newcommand{\foralledges}{\forall(i,j) \in \calE}
\newcommand{\sumalledges}{
     \displaystyle
     \sum_{(i,j) \in \calE}}
\newcommand{\sumalledgesm}{
     \displaystyle
     \sum_{i=1}^{m}}
\newcommand{\T}{\mathsf{T}}
\newcommand{\To}{\T_{\rm o}}
\newcommand{\Tm}{\T_{\rm m}}
\newcommand{\MCBa}{\MCB_{\mathsf{a}}}
\newcommand{\FCBo}{\FCB_{\mathsf{o}}}
\newcommand{\FCBm}{\FCB_{\mathsf{m}}}

% Algorithms
\newcommand{\algoonlyname}{MOLE2D}
\newcommand{\algoml}{{\smaller\sf \algoonlyname}\xspace}
\newcommand{\algocyclebasis}{\algoname{compute-cycle-basis}}
\newcommand{\scenario}[1]{{\smaller \sf#1}\xspace}
\newcommand{\toro}{{\smaller\sf Toro}\xspace}
\newcommand{\gtwoo}{{\smaller\sf g2o}\xspace}
\newcommand{\gtwooST}{{\smaller\sf g2oST}\xspace}
\newcommand{\gtwood}{{\smaller\sf g2o{10}}\xspace}
\newcommand{\gtsam}{{\smaller\sf gtsam}\xspace}
\newcommand{\isam}{{\smaller\sf iSAM}\xspace}
\newcommand{\lago}{{\smaller\sf LAGO}\xspace}
\newcommand{\egtwoo}{{\smaller\sf \algoonlyname+g2o}\xspace}

% Datasets
%\newcommand{\grid}{\scenario{cube}}
\newcommand{\rim}{\scenario{rim}}
\newcommand{\cubicle}{\scenario{cubicle}}
\newcommand{\sphere}{\scenario{sphere}}
\newcommand{\sphereHard}{\scenario{sphere-a}}
\newcommand{\garage}{\scenario{garage}}
\newcommand{\torus}{\scenario{torus}}
\newcommand{\oneloop}{\scenario{circle}}
\newcommand{\intel}{\scenario{INTEL}}
\newcommand{\bovisa}{\scenario{Bovisa}}
\newcommand{\bov}{\scenario{B25b}}
\newcommand{\fra}{\scenario{FR079}}
\newcommand{\frb}{\scenario{FRH}}
\newcommand{\csail}{\scenario{CSAIL}}
\newcommand{\Ma}{\scenario{M3500}}
\newcommand{\Mb}{\scenario{M10000}}
\newcommand{\ATE}{\scenario{ATE}}
\newcommand{\CVX}{\scenario{CVX}}
\newcommand{\NEOS}{\scenario{NEOS}}
\newcommand{\sdptThree}{\scenario{sdpt3}}
\newcommand{\MOSEK}{\scenario{MOSEK}}
\newcommand{\NESTA}{\scenario{NESTA}}
\newcommand{\vertigo}{\scenario{Vertigo}}
\newcommand{\SDPA}{\scenario{SDPA}}

% \newcommand{\tablabel}[1]{%
%     \hspace{-1mm}%
%     \begin{sideways}{\small\scenario{#1}}\end{sideways}%
%     \hspace{-4mm}%
% }
%
% \newcommand{\tabfig}[2]{%
%     \subfloat[\label{fig:#2} #1]{%
%     \begin{minipage}{5.5cm}%
%     \centering%
%     \includegraphics[width=5.5cm,trim=4 0 4 15]{figures/#2}
%     \end{minipage}%
%     }%
% }
%
% \newcommand{\subFigure}[3]{%
%     \subfloat[\label{fig:#2} #1]{%
%     \begin{minipage}{#3cm}%
%     \centering%
%     \includegraphics[width=#3cm,trim=4 0 4 15]{figures/#2}
%     \end{minipage}%
%     }%
% }
\newcommand{\cvx}{{\sf cvx}\xspace}

% COLORS
\newcommand{\blue}[1]{{\color{blue}#1}}
\newcommand{\green}[1]{{\color{green}#1}}
\newcommand{\red}[1]{{\color{red}#1}}

% TO MANAGE REFERENCES
%============================================================================
\newcommand{\linkToPdf}[1]{\href{#1}{\blue{(pdf)}}}
\newcommand{\linkToPpt}[1]{\href{#1}{\blue{(ppt)}}}
\newcommand{\linkToCode}[1]{\href{#1}{\blue{(code)}}}
\newcommand{\linkToWeb}[1]{\href{#1}{\blue{(web)}}}
\newcommand{\linkToVideo}[1]{\href{#1}{\blue{(video)}}}
\newcommand{\linkToMedia}[1]{\href{#1}{\blue{(media)}}}
\newcommand{\award}[1]{\xspace} % {{\red{#1}}} % omit awards


% PAPER-SPECIFIC COMMANDS
%============================================================================

% Linear approximation
\newcommand{\vpose}{\boldsymbol{x}}
\newcommand{\vz}{\boldsymbol{z}}
\newcommand{\vDelta}{\boldsymbol{\Delta}}
\newcommand{\vposesub}{\hat{\vpose}}
\newcommand{\vpossub}{\hat{\vpos}}
\newcommand{\vthetasub}{\hat{\vtheta}}
\newcommand{\Pthetasub}{\MP_\vtheta}
\newcommand{\thetasub}{\hat{\theta}}
\newcommand{\vposecorr}{\tilde{\vpose}}
\newcommand{\vposcorr}{\tilde{\vpos}}
\newcommand{\vthetacorr}{\tilde{\vtheta}}
\newcommand{\vposestar}{{\vpose}^{\star}}
\newcommand{\vposstar}{{\vpos}^{\star}}
\newcommand{\vthetastar}{{\vtheta}^{\star}}
\newcommand{\vcthetastar}{{\vctheta}^{\star}}
\newcommand{\pose}{\boldsymbol{x}}
\newcommand{\pos}{\boldsymbol{p}}
\newcommand{\mease}{z} % element
\newcommand{\meas}{\boldsymbol{z}} % vector
\newcommand{\meashat}{\hat{\meas}}



