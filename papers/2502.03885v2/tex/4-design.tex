% \newpage
\vspace{-1em}
\section{\SYS{} Design}
\label{section:design}

This section first introduces the innovative design of OCS transceivers (OCSTrx) based on Silicon Photonics (SiPh) chips (\secref{sec:design:docs}), a key enabler for \sys{}, providing both cost efficiency and reconfigurability. Next, we present the DC-scale \sys{} topology design (\secref{section:design:topology}) based on OCSTrx. Finally, we outline the HBD-DCN orchestration algorithm (\secref{sec:design:orch}), designed to optimize communication efficiency for training jobs.

\vspace{-1em}
\subsection{SiPh-based OCS transceiver (OCSTrx)}
\label{sec:design:docs}

The \ocstrx \xspace is designed for reconfigurable point-to-multipoint connectivity. It incorporates a compact OCS-based switch with three Rx/Tx paths, utilizing the MZI switch ~\cite{mzi} micro-structure with thermo-optic (TO) effect~\cite{thermo-optic_2006} phase arms. This OCS-based switch is seamlessly integrated into the Photonic Integrated Chip (PIC) of the transceiver, serving as the MZI switch matrix within the Tx light path, and providing photodetector (PD) modules for each Rx paths.

\para{SiPh-Based OCS.} 
Currently, there are two predominant technological approaches for OCS.
Micro Electromechanical systems (MEMS)~\cite{urata2022missionapollo, mem-optical-switches} are attractive for commercial adoption due to supporting large port radix, up to a $320\times 320$ matrix~\cite{mems-320}.
Another option is SiPh-based OCS. 
Its structure is simpler and cheaper to manufacture, the limitation is its radix due to optical losses in multistage light path selector. 
Given that the locality of traffic and external paths number of \ocstrx{} is only two, SiPh-based OCS offers greater advantages. 

So we choose the design of MZI micro-structure~\cite{mzi} based SiPh-Based OCS.
The basic mechanism of MZI switch elements is controlling the phase difference between light paths in two phase arms, then directs the output light to specific ports through interference at the output combiner. TO effect is utilized for phase arm control, for better switching latency compared to MEMS.


\para{OCS Micro-Structure Design.} As shown in Figure~\ref{figure:design:transceiver:component}, the initial routing decision is made by two MZI switch elements, determining whether to direct the signal through external output 1\&2, or the internal loopback path.
Subsequently, an internal $N\times N$ MZI switch matrix is incorporated to facilitate the cross-lane loopback mechanism, exemplified by the blue and red paths.  Notably, this design can reduces stages count and light attenuation of output 1\&2, while ensures consistent light attenuation for them. The design is implemented on the Photonic Integrated Circuit (PIC) chip.

\para{Transceiver Design.} In \ocstrx, Tx electrical signal is amplified by linear driver and converted to optical signal by modulators as in \figref{figure:design:transceiver}. One laser is coupled into the PIC as optical source.
On the receiving end, multiple photodetectors capture the Rx optical signal from all available paths separately. The output from the activated photodetector is then amplified by a linear transimpedance amplifier (TIA).
\ocstrx{} offers significant benefits, including high compactness, low power consumption, and cost-effective mass production.








% \vspace{-10pt}
\subsection{\sys Topology}
\label{section:design:topology}


In this section, we present the \sys{} topology design (\figref{fig:overview}) integrating \docs{} that allows all GPUs within datacenter to be connected in a \textit{reconfigurable K-Hop Ring topology}, while supporting dynamic ring construction and high fault tolerance.  

\para{Intra-node Topology.} The intra-node topology is designed for \textit{dynamic ring construction} and compiles with the OCP UBB 2.0 standard~\cite{UBB2.0}. As shown in \figref{fig:degin:topo:ubb}, one node equipped with $R$ GPUs can support $R$ bundles of \ocstrx. Each \docs{} bundle is connected to a pair of GPUs, with one GPU linking to the upper-half SerDes and the other to the lower-half.
For one group of nodes connected as one line, the two GPU paris at each end can interconnect with the \ocstrx \xspace internal loopback path, forming a GPU-level ring. 
As shown in \figref{fig:overview}, nodes $N_1$ and $N_3$ are connected in a line, where $OCSTrx_1(N_1)$ and $OCSTrx_2(N_3)$ activate the cross-lane loopback path, creating a ring between the 8 GPUs of $N_1$ and $N_3$.
During ring construction, only two \docs{} bundles per node are utilized, while the remaining \docs{} operate in loopback mode. These idle \docs{} can be replaced with direct connections, such as DAC links, offering a trade-off between cost and reliability. \fig{fig:inner-topo}(a,b) shows a 4-GPU node with varying numbers of \textit{\docs{}} bundles. Note that the topology design in this section utilizes a 4-GPU node as an example, it can be easily scaled for 8-GPU nodes.

\begin{figure}[h!t]
   \centering
   \includegraphics[width=0.8\linewidth]{figs/design/ubb-standard.pdf}
   \vspace{-2ex}
   \caption{\ocstrx{} connection within nodes. Each block contains multiple \ocstrx \xspace as one bundle, .e.g, $8\times 800Gbps$ \ocstrx \xspace for a 6.4Tbps GPU. }

   \label{fig:degin:topo:ubb}
   \vspace{-2ex}
\end{figure}

\para{Inter-node Topology.}
We construct the inter-node topology by pruning the full-mesh design, based on two key observations:
i) \textit{Traffic locality}: TP Ring-AllReduce in HBD exhibits neighbor communication patterns, eliminating the need for distant connections; 
ii) \textit{Fault non-locality}: node-side failures typically occur independently at the node level, meaning consecutive multi-node failures follow an exponentially decaying probability.
Each node provides up to $2R$ external paths, allowing us to construct a DC-scale reconfigurable $K$-Hop Ring topology ($K\leq R$) by connecting them to nodes at $\pm 1,...,\pm K, K\leq R$. For AllReduce communication, only two out of the $2K$ links are activated once, with the others serving as backup links for fault isolation.
For example (\figref{fig:overview}), if $N_2$ fails, $OCSTrx_2(N_1)$ and $OCSTrx_1(N_3)$ can switch to backup links, maintaining connectivity between $N_1$ and $N_3$ while isolating $N_2$'s fault. As $K$ increases, the probability of encountering an unbypassed failure rapidly decreases, which is nearly negligible for $K=3$ (detailed analysis in Appendix~\S\ref{appendix:ft-anay}). Thus, this architecture typically achieves a node-level explosion radius.
Moreover, the K-Hop Ring can be broken into the K-Hop line topology, with the trade-off of reduced fault tolerance of $2K$ nodes at two ends.


\begin{figure}[h!t]
    \centering
    \begin{subfigure}[b]{0.22\textwidth}
        \centering
        \includegraphics[height=70pt]{figs/design/inner-topo/4gpu-2port.pdf}
        \caption{2 bundles of \ocstrx}
        \label{fig:4g3d}
    \end{subfigure}
    \hspace{10pt}
    \begin{subfigure}[b]{0.22\textwidth}
        \centering
        \includegraphics[height=70pt]{figs/design/inner-topo/4gpu-3port.pdf}
        \caption{3 bundles of \ocstrx}
        \label{fig:4g2d}
    \end{subfigure}
    \vspace{-2ex}
    \caption{4-GPU node with \docs.}
    \label{fig:inner-topo}
    \vspace{-2ex}

\end{figure}











\vspace{-1ex}
\subsection{HBD-DCN Orchestration Algorithm}  
\label{sec:design:orch}  

\sys{} is designed to work with arbitrary DCN, including Rail-Optimized~\cite{rail-optimized, sigcomm2024hpn} and Fat-Tree~\cite{sigcomm2008fattree}. This section co-optimizes communication performance for both HBD and DCN in \sys{}.  


\para{Problem Statement.} In \sys{}, GPUs communicate without routing traffic, preventing congestion at any scale. In contrast, DCNs experience inevitable congestion, leading to performance degradation. To mitigate this, we leverage traffic locality to orchestrate nodes, minimizing cross-ToR traffic. Given a job $J$ requiring $N$ nodes from an available pool of $M$ ($M \geq N$), we must select and order $N$ nodes to satisfy two requirements: (1) nodes in the same TP group should communicate via \sys{}, and (2) other parallel traffic should minimize congestion. Ideally, communication remains within the same ToR, confining congestion to switch-to-node links.  

\begin{figure}[h!t]
\centering
\begin{subfigure}[b]{0.23\textwidth}
    \centering
    \includegraphics[width=0.92\textwidth]{figs/problem-1.png}
    \caption{Orchestration scheme 1.}
    \label{figure:orchstration:problem-1}
\end{subfigure}
\hspace{2pt}
\begin{subfigure}[b]{0.23\textwidth}
    \centering
    \includegraphics[width=0.92\textwidth]{figs/problem-2.png}
    \caption{Orchestration scheme 2.}
    \label{figure:orchstration:problem-2}
\end{subfigure}
\vspace{-6ex}
\caption{Illustration for problem statement of node orchestration.}
\vspace{-1em}
\end{figure}

A naive approach is sorting nodes based on deployment order in \sys{}, fulfilling the first requirement but not the second. As shown in \figref{figure:orchstration:problem-1}, this method places ($N_1$, $N_2$) in the same TP group and ($N_1$, $N_3$) in the same DP group, forcing DP traffic across ToRs. A better scheme (\figref{figure:orchstration:problem-2}) eliminates cross-ToR traffic and congestion. However, considering failures and multiple parallel dimensions complicates orchestration, necessitating an efficient method.  



Our key insight is to arrange nodes in \sys{} based on DCN traffic locality, prioritizing appropriate network distances over minimal ones. For example, in \figref{figure:orchstration:problem-2}, $N_1$'s \sys{} neighbor is $N_3$, despite a 3-hop network distance in DCN. We propose a two-phase solution: (1) a deployment phase defining physical connections in DCN and \sys{}, and (2) a runtime phase using an algorithm to orchestrate nodes for arbitrary-scale jobs. 

\vspace{-1em}
\begin{algorithm}[h!t]
\small
\caption{Orchestration For Fat-Tree}
\label{alg:orchestration-fat-tree-overview}
\SetAlgoNlRelativeSize{-1}
\SetAlgoNlRelativeSize{1}
 \KwIn{
    Topology of DCN and HBD $G$, Faulty Node Set $F$, Job Information $J$.}
 \KwOut{Placement scheme that satisfies job scale and minimizes cross-ToR traffic.}
 Create graph $G_{deploy}=<S_{deploy},E_{deploy}>=\text{Deployment-Strategy}(G)$\;
 Initialize $high=n_{allconstraints}$, $low=0$, $placement =\{\}$\;
\While{ $low \leq high$}
{
     $mid=\lfloor \frac{low+high}{2} \rfloor$\;
     $placement=\text{Placement-Fat-Tree}(G_{deploy},mid,F,J)$\;
    \eIf {$placement$ satisfies job $J$}
    {
         $low=mid+1$\;
    }
    {
         $high=mid-1$\;
    }
}
\KwRet {$placement$}
\end{algorithm}
\vspace{-1em}

\begin{figure}[h!t]
    \centering
    \includegraphics[width=0.7\linewidth]{figs/design/deployment.pdf}
    \vspace{-1em}
    \caption{Illustration of the deployment phase, showing only two backup links for simplicity.}
    \label{fig:fat-tree-topo}
    \vspace{-8pt}
\end{figure}


\para{Deployment Phase.} \figref{fig:fat-tree-topo} shows node deployment in HBD and DCN. \sys{} connects nodes at a network distance of 3 (i.e., cross-ToR). In a DCN with $r$ nodes per ToR, node $N_n$ connects to $N_{n\pm r}$ as main links, while backup links connect to $N_{n\pm 2r}$. For $1 < n \le r$, $N_n$ connects to $N_{D + n - r - 1}$, where $D$ is the total node count (e.g., $N_3$ connects to $N_{14}$). Additionally, $N_1$ may link to the last node, forming a ring.




\para{Runtime Phase.} Without considering DCN topology, \sys{} orchestrates nodes in three steps: (1) identifying cluster faults and modeling healthy nodes as a graph, (2) using Depth-First Search to find connected components, and (3) sequentially placing TP groups within these components. Due to \sys{}’s topology, each TP group forms a ring.  


For real-world DCNs, topology constraints refine step (2) and (3). In Fat-Tree networks, congestion arises when (1) a TP group spans multiple Aggregation-Switch domains, or (2) GPUs within a ToR have mismatched TP group ranks, forcing DP, CP, PP, SP traffic across ToRs. Thus, we aim to localize TP groups within the same Aggregation-Switch domain and align ranks within each ToR. Our scheduling algorithm minimizes cross-ToR traffic while meeting job scale requirements via a binary search over constraint variables. \algref{alg:orchestration-fat-tree-overview} outlines the approach, with full details in Appendix~\S\ref{appendix:orch-algo}.  




