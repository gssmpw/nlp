\documentclass[sigconf]{acmart}

\usepackage[english]{babel}
\usepackage{graphicx}
\usepackage{multirow}
\usepackage{array}
\usepackage{tabularx}
\usepackage{pifont}
%
% --- inline annotations
%
\newcommand{\red}[1]{{\color{red}#1}}
\newcommand{\todo}[1]{{\color{red}#1}}
\newcommand{\TODO}[1]{\textbf{\color{red}[TODO: #1]}}
% --- disable by uncommenting  
% \renewcommand{\TODO}[1]{}
% \renewcommand{\todo}[1]{#1}



\newcommand{\VLM}{LVLM\xspace} 
\newcommand{\ours}{PeKit\xspace}
\newcommand{\yollava}{Yo’LLaVA\xspace}

\newcommand{\thisismy}{This-Is-My-Img\xspace}
\newcommand{\myparagraph}[1]{\noindent\textbf{#1}}
\newcommand{\vdoro}[1]{{\color[rgb]{0.4, 0.18, 0.78} {[V] #1}}}
% --- disable by uncommenting  
% \renewcommand{\TODO}[1]{}
% \renewcommand{\todo}[1]{#1}
\usepackage{slashbox}
% Vectors
\newcommand{\bB}{\mathcal{B}}
\newcommand{\bw}{\mathbf{w}}
\newcommand{\bs}{\mathbf{s}}
\newcommand{\bo}{\mathbf{o}}
\newcommand{\bn}{\mathbf{n}}
\newcommand{\bc}{\mathbf{c}}
\newcommand{\bp}{\mathbf{p}}
\newcommand{\bS}{\mathbf{S}}
\newcommand{\bk}{\mathbf{k}}
\newcommand{\bmu}{\boldsymbol{\mu}}
\newcommand{\bx}{\mathbf{x}}
\newcommand{\bg}{\mathbf{g}}
\newcommand{\be}{\mathbf{e}}
\newcommand{\bX}{\mathbf{X}}
\newcommand{\by}{\mathbf{y}}
\newcommand{\bv}{\mathbf{v}}
\newcommand{\bz}{\mathbf{z}}
\newcommand{\bq}{\mathbf{q}}
\newcommand{\bff}{\mathbf{f}}
\newcommand{\bu}{\mathbf{u}}
\newcommand{\bh}{\mathbf{h}}
\newcommand{\bb}{\mathbf{b}}

\newcommand{\rone}{\textcolor{green}{R1}}
\newcommand{\rtwo}{\textcolor{orange}{R2}}
\newcommand{\rthree}{\textcolor{red}{R3}}
\usepackage{amsmath}
%\usepackage{arydshln}
\DeclareMathOperator{\similarity}{sim}
\DeclareMathOperator{\AvgPool}{AvgPool}

\newcommand{\argmax}{\mathop{\mathrm{argmax}}}     



%% Configuration
\newcommand{\COMMENTS}{yes}


%% Cross reference
\newcommand{\secref}[1]{\S\ref{#1}}
\newcommand{\figref}[1]{Figure~\ref{#1}}
\newcommand{\tabref}[1]{Table~\ref{#1}}
\newcommand{\eqnref}[1]{Equation~\ref{#1}}
\newcommand{\algref}[1]{Algorithm~\ref{#1}}
\newcommand{\fig}[1]{Figure~\ref{#1}}
%% Table & Cell
\newcommand{\specialcell}[2][c]{\begin{tabular}[#1]{@{}c@{}}#2\end{tabular}}
\newcommand{\tabincell}[2]{\begin{tabular}{@{}#1@{}}#2\end{tabular}}


% \usepackage{algorithm,algorithmic}
% \renewcommand{\algorithmicrequire}{\textbf{Input:}}
% \renewcommand{\algorithmicensure}{\textbf{Output:}}

%% Url

\def\UrlBreaks{\do\A\do\B\do\C\do\D\do\E\do\F\do\G\do\H\do\I\do\J\do\K\do\L\do\M\do\N\do\O\do\P\do\Q\do\R\do\S\do\T\do\U\do\V\do\W\do\X\do\Y\do\Z\do\[\do\\\do\]\do\^\do\_\do\`\do\a\do\b\do\c\do\d\do\e\do\f\do\g\do\h\do\i\do\j\do\k\do\l\do\m\do\n\do\o\do\p\do\q\do\r\do\s\do\t\do\u\do\v\do\w\do\x\do\y\do\z\do\0\do\1\do\2\do\3\do\4\do\5\do\6\do\7\do\8\do\9\do\.\do\@\do\\\do\/\do\!\do\_\do\|\do\;\do\>\do\]\do\)\do\,\do\?\do\'\do+\do\=\do\#}

%% Caption with tigher before and after vertical space
\newcommand{\figcaption}[1]{\vspace{-2mm}\caption{#1}\vspace{-4mm}} 
\newcommand{\mfigcaption}[1]{\vspace{-2mm}\caption{#1}\vspace{-2mm}} 
\newcommand{\tabcaption}[1]{\vspace{-2mm}\caption{#1}\vspace{-4mm}}
\newcommand{\mtabcaption}[1]{\vspace{1mm}\caption{#1}\vspace{-4mm}}

%% Math symbol optimization
\newcommand{\superscript}[1]{\ensuremath{^{\textrm{#1}}}}
\newcommand{\argmax}{\operatornamewithlimits{argmax}}
\def\deg{{\,^{\circ}}\xspace}

%% Abbreviation optimization
\def\It{\textit}
\def\Bf{\textbf}
\def\eg{\textit{e.g.,}\hspace{1mm}}
\def\ie{\textit{i.e.,}\hspace{1mm}}
\def\etal{\textit{et al.}\hspace{1mm}}
\def\etc{\textit{etc.}\hspace{1mm}}

%% font wrapper

\newcommand{\code}[1]{\mbox{\texttt{#1}}}
\newcommand{\Mod}[1]{\mbox{\textsf{#1}}} % \textup \textsl \texttt \textsf \textrm
\newcommand{\sw}[1]{\mbox{\textsc{#1}}}

%% Compact environments
\newenvironment{Itemize}{
  \begin{list}{$\bullet$} {
      \setlength{\itemsep}{0pt}
      \setlength{\parsep}{2pt}
      \setlength{\topsep}{2pt}
      \setlength{\partopsep}{0pt}
      \setlength{\leftmargin}{1.5em} % 1.5
      \setlength{\labelwidth}{1em} % 1
      \setlength{\labelsep}{0.5em} % 0.5
  }}
  {\end{list}}  

\newenvironment{Enumerate}{
  \begin{enumerate}[leftmargin=2em]
    \setlength{\itemsep}{2pt}
    \setlength{\topsep}{2pt}
    \setlength{\partopsep}{0pt}
    \setlength{\parskip}{0pt}}
  {\end{enumerate}}

\newenvironment{Circled}{
  \begin{enumerate}[label=\protect\circled{\arabic*},leftmargin=2em]
    \setlength{\itemsep}{3pt}
    \setlength{\topsep}{0pt}
    \setlength{\partopsep}{0pt}
    \setlength{\parskip}{0pt}}
  {\end{enumerate}}

% \usepackage{ulem}

% \newcommand{\lyz}[1]{{\color{blue}\textit{Yunzhuo:#1}}}
% \newcommand{\caihui}[1]{{\color{orange}\textit{Caihui:#1}}}
% \newcommand{\bj}[1]{{\color{red}(BJ:#1)}}


%% Paper revising
\ifthenelse{\equal{\COMMENTS}{yes}}{
  %% Writing Mode 
  \newcommand{\todo}[1]{\textcolor{red}{\textbf{TODO:} #1}}
  \newcommand{\neil}[1]{\textcolor{blue}{\textbf{Nie:} #1}} %content will be included
  \newcommand{\grace}[1]{\textcolor{purple}{(grace: #1)}}
  \newcommand{\fye}[1]{\textcolor{red}{#1}}  %content will be excluded
  \newcommand{\remind}[1]{\footnote{\textit{\textcolor{red}{\textbf{Remind:} #1}}}}
  \newcommand{\repl}[2]{\textcolor{red}{#1}\textcolor{blue}{\sout{#2}}} % replacement
  \newcommand{\add}[1]{\textcolor{red}{#1}}
  \newcommand{\del}[1]{\color{blue} {\sout{#1}}}
  %\newcommand{\p}[1]{\noindent\parbox{\columnwidth}{\textcolor{magenta}{\textbf{Point to make:} #1}}\vskip 0.5ex}
  \newcommand{\p}[1]{\vskip 1ex \noindent\colorbox{yellow}{\parbox{\columnwidth}{#1}}\vskip 4pt}
  \newcommand{\note}[1]{\vskip 4ex \noindent\colorbox{yellow}{\parbox{\columnwidth}{#1}}\vskip 6ex} % highlight
  \newcommand{\dc}[1]{\textcolor{red}{\underline{#1}}} % double check % \uwave
  \newcommand{\q}[1]{\vskip 1ex \noindent\colorbox{magenta}{\parbox{\columnwidth}{\textbf{Question:} #1}}\vskip 4pt} 
  \newcommand{\qa}[1]{\hl{\textbf{Answer:} #1}}
  \newcommand{\rc}[1]{\textcolor{red}{(RC: #1)}}
}{
  %%Submission Mode
  \newcommand{\todo}[1]{}
  \newcommand{\fyi}[1]{#1}
  \newcommand{\fye}[1]{}
  \newcommand{\remind}[1]{}
  \newcommand{\repl}[2]{#1}
  \newcommand{\add}[1]{#1}
  \newcommand{\del}[1]{}
  \newcommand{\p}[1]{}
  \newcommand{\note}[1]{}
  \newcommand{\dc}[1]{#1}  
  \newcommand{\qm}[1]{#1}
  \newcommand{\q}[1]{}
  \newcommand{\qa}[1]{}  
  \newcommand{\grace}[1]{}
  \newcommand{\neil}[1]{}
}

% Circled Numbers
% \usepackage{tikz}
% \newcommand*\circled[1]{\tikz[baseline=(char.base)]{
%     \node[shape=circle,draw,inner sep=0.5pt] (char) {#1};}
% }
% \newcommand{\todo}[1]{\textcolor{red}{\textbf{TODO:} #1}}
\newcommand{\circled}[1]{\raisebox{.5pt}{\textcircled{\raisebox{-.9pt} {#1}}}}
\newcommand{\para}[1]{\noindent\textbf{#1}}

%% packeditemize
\newenvironment{packeditemize}{\begin{list}{$\bullet$}{\setlength{\itemsep}{2pt}\addtolength{\labelwidth}{-6pt}\setlength{\leftmargin}{12pt}\setlength{\listparindent}{\parindent}\setlength{\parsep}{1pt}\setlength{\topsep}{2pt}}}{\end{list}}


\renewcommand\footnotetextcopyrightpermission[1]{} % removes footnote with conference info
\setcopyright{none}
\settopmatter{printacmref=false, printccs=false, printfolios=true}

% DOI
\acmDOI{}

% ISBN
\acmISBN{}

% Conference
\acmConference[Submitted for review to SIGCOMM]{}
\acmYear{2025}
\copyrightyear{}

%% {} with no args suppresses printing of the price
\acmPrice{}
%-------------------------------------------------------------------------------

%-------------------------------------------------------------------------------

%don't want date printed
% \date{}

\newcommand{\SYS}{\textit{\textbf{InfinitePOD}}\xspace}
\newcommand{\sys}{\textit{InfinitePOD}\xspace}
\newcommand{\ocstrx}{\textit{OCSTrx}}
\newcommand{\docs}{\ocstrx}

\fancyhead[LE]{}%
\fancyhead[RO]{}%
\fancyhead[RE]{}%
\fancyhead[LO]{}%
\fancyfoot[C]{\footnotesize\thepage}

\begin{document}

\title{\SYS: Building Datacenter-Scale High-Bandwidth Domain for LLM with Optical Circuit Switching Transceivers}

\renewcommand{\shorttitle}{\sys}

\author{Chenchen Shou$^{1,2,3}$ \hspace{0.5em} Guyue Liu$^{1,\dag}$ \hspace{0.5em} Hao Nie$^{2,\dag}$ \hspace{0.5em} Huaiyu Meng$^{3,\dag}$  \hspace{0.5em} Yu Zhou$^2$ \hspace{0.5em} \\ Yimin Jiang$^4$ \hspace{0.5em}  Wenqing Lv$^3$ \hspace{0.5em} Yelong Xu$^3$ \hspace{0.5em} Yuanwei Lu$^2$ \hspace{0.5em} Zhang Chen$^3$ \hspace{0.5em} \\ Yanbo Yu$^2$ \hspace{0.5em} Yichen Shen$^3$ \hspace{0.5em} Yibo Zhu$^2$ \hspace{0.5em} Daxin Jiang$^2$
}

\affiliation{
$^1$Peking University \hspace{0.5em}
$^2$StepFun \hspace{0.5em}
$^3$Lightelligence Pte. Ltd. \hspace{0.5em}
$^4$Unaffiliated \hspace{0.5em}
}

% \affiliation{Peking University, StepFun, Lightelligence Pte. Ltd.}

% \renewcommand{\shortauthors}{Anonymous Authors}



%-------------------------------------------------------------------------------
\begin{abstract}
%-------------------------------------------------------------------------------


Scaling Large Language Model (LLM) training relies on multi-dimensional parallelism, where High-Bandwidth Domains (HBDs) are critical for communication-intensive parallelism like Tensor Parallelism (TP) and Expert Parallelism (EP). However, existing HBD architectures face fundamental limitations in scalability, cost, and fault resiliency: switch-centric HBDs (e.g., NVL-72) incur prohibitive scaling costs, while GPU-centric HBDs (e.g., TPUv3/Dojo) suffer from severe fault propagation. Switch-GPU hybrid HBDs such as TPUv4 takes a middle-ground approach by leveraging Optical Circuit Switches, but the fault explosion radius remains large at the cube level (e.g., 64 TPUs).

We propose \sys{}, a novel transceiver-centric HBD architecture that \textit{unifies connectivity and dynamic switching at the transceiver level} using Optical Circuit Switching (OCS). By embedding OCS within each transceiver, \sys{} achieves reconfigurable point-to-multipoint connectivity, allowing the topology to adapt into variable-size rings. This design provides: i) datacenter-wide scalability without cost explosion; ii) fault resilience by isolating failures to a single node, and iii) full bandwidth utilization for fault-free GPUs. Key innovations include a Silicon Photonic (SiPh) based low-cost OCS transceiver (\textbf{\ocstrx}), a reconfigurable k-hop ring topology co-designed with intra-/inter-node communication, and an HBD-DCN orchestration algorithm maximizing GPU utilization while minimizing cross-ToR datacenter network traffic.
The evaluation demonstrates that \sys{} achieves \textbf{31\%} of the cost of NVL-72, \textbf{near-zero} GPU waste ratio (over one order of magnitude lower than NVL-72 and TPUv4), \textbf{near-zero} cross-ToR traffic when node fault ratios under 7\%, and improves Model FLOPs Utilization by \textbf{3.37x} compared to NVIDIA DGX (8 GPUs per Node).
\vspace{-1pt}

\end{abstract}

\maketitle


\footnotetext[1]{Guyue Liu, Hao Nie, and Huaiyu Meng contributed equally to this work and share the corresponding authorship.}


\vspace{-1pt}
%!TEX root = gcn.tex
\section{Introduction}
Graphs, representing structural data and topology, are widely used across various domains, such as social networks and merchandising transactions.
Graph convolutional networks (GCN)~\cite{iclr/KipfW17} have significantly enhanced model training on these interconnected nodes.
However, these graphs often contain sensitive information that should not be leaked to untrusted parties.
For example, companies may analyze sensitive demographic and behavioral data about users for applications ranging from targeted advertising to personalized medicine.
Given the data-centric nature and analytical power of GCN training, addressing these privacy concerns is imperative.

Secure multi-party computation (MPC)~\cite{crypto/ChaumDG87,crypto/ChenC06,eurocrypt/CiampiRSW22} is a critical tool for privacy-preserving machine learning, enabling mutually distrustful parties to collaboratively train models with privacy protection over inputs and (intermediate) computations.
While research advances (\eg,~\cite{ccs/RatheeRKCGRS20,uss/NgC21,sp21/TanKTW,uss/WatsonWP22,icml/Keller022,ccs/ABY318,folkerts2023redsec}) support secure training on convolutional neural networks (CNNs) efficiently, private GCN training with MPC over graphs remains challenging.

Graph convolutional layers in GCNs involve multiplications with a (normalized) adjacency matrix containing $\numedge$ non-zero values in a $\numnode \times \numnode$ matrix for a graph with $\numnode$ nodes and $\numedge$ edges.
The graphs are typically sparse but large.
One could use the standard Beaver-triple-based protocol to securely perform these sparse matrix multiplications by treating graph convolution as ordinary dense matrix multiplication.
However, this approach incurs $O(\numnode^2)$ communication and memory costs due to computations on irrelevant nodes.
%
Integrating existing cryptographic advances, the initial effort of SecGNN~\cite{tsc/WangZJ23,nips/RanXLWQW23} requires heavy communication or computational overhead.
Recently, CoGNN~\cite{ccs/ZouLSLXX24} optimizes the overhead in terms of  horizontal data partitioning, proposing a semi-honest secure framework.
Research for secure GCN over vertical data  remains nascent.

Current MPC studies, for GCN or not, have primarily targeted settings where participants own different data samples, \ie, horizontally partitioned data~\cite{ccs/ZouLSLXX24}.
MPC specialized for scenarios where parties hold different types of features~\cite{tkde/LiuKZPHYOZY24,icml/CastigliaZ0KBP23,nips/Wang0ZLWL23} is rare.
This paper studies $2$-party secure GCN training for these vertical partition cases, where one party holds private graph topology (\eg, edges) while the other owns private node features.
For instance, LinkedIn holds private social relationships between users, while banks own users' private bank statements.
Such real-world graph structures underpin the relevance of our focus.
To our knowledge, no prior work tackles secure GCN training in this context, which is crucial for cross-silo collaboration.


To realize secure GCN over vertically split data, we tailor MPC protocols for sparse graph convolution, which fundamentally involves sparse (adjacency) matrix multiplication.
Recent studies have begun exploring MPC protocols for sparse matrix multiplication (SMM).
ROOM~\cite{ccs/SchoppmannG0P19}, a seminal work on SMM, requires foreknowledge of sparsity types: whether the input matrices are row-sparse or column-sparse.
Unfortunately, GCN typically trains on graphs with arbitrary sparsity, where nodes have varying degrees and no specific sparsity constraints.
Moreover, the adjacency matrix in GCN often contains a self-loop operation represented by adding the identity matrix, which is neither row- nor column-sparse.
Araki~\etal~\cite{ccs/Araki0OPRT21} avoid this limitation in their scalable, secure graph analysis work, yet it does not cover vertical partition.

% and related primitives
To bridge this gap, we propose a secure sparse matrix multiplication protocol, \osmm, achieving \emph{accurate, efficient, and secure GCN training over vertical data} for the first time.

\subsection{New Techniques for Sparse Matrices}
The cost of evaluating a GCN layer is dominated by SMM in the form of $\adjmat\feamat$, where $\adjmat$ is a sparse adjacency matrix of a (directed) graph $\graph$ and $\feamat$ is a dense matrix of node features.
For unrelated nodes, which often constitute a substantial portion, the element-wise products $0\cdot x$ are always zero.
Our efficient MPC design 
avoids unnecessary secure computation over unrelated nodes by focusing on computing non-zero results while concealing the sparse topology.
We achieve this~by:
1) decomposing the sparse matrix $\adjmat$ into a product of matrices (\S\ref{sec::sgc}), including permutation and binary diagonal matrices, that can \emph{faithfully} represent the original graph topology;
2) devising specialized protocols (\S\ref{sec::smm_protocol}) for efficiently multiplying the structured matrices while hiding sparsity topology.


 
\subsubsection{Sparse Matrix Decomposition}
We decompose adjacency matrix $\adjmat$ of $\graph$ into two bipartite graphs: one represented by sparse matrix $\adjout$, linking the out-degree nodes to edges, the other 
by sparse matrix $\adjin$,
linking edges to in-degree nodes.

%\ie, we decompose $\adjmat$ into $\adjout \adjin$, where $\adjout$ and $\adjin$ are sparse matrices representing these connections.
%linking out-degree nodes to edges and edges to in-degree nodes of $\graph$, respectively.

We then permute the columns of $\adjout$ and the rows of $\adjin$ so that the permuted matrices $\adjout'$ and $\adjin'$ have non-zero positions with \emph{monotonically non-decreasing} row and column indices.
A permutation $\sigma$ is used to preserve the edge topology, leading to an initial decomposition of $\adjmat = \adjout'\sigma \adjin'$.
This is further refined into a sequence of \emph{linear transformations}, 
which can be efficiently computed by our MPC protocols for 
\emph{oblivious permutation}
%($\Pi_{\ssp}$) 
and \emph{oblivious selection-multiplication}.
% ($\Pi_\SM$)
\iffalse
Our approach leverages bipartite graph representation and the monotonicity of non-zero positions to decompose a general sparse matrix into linear transformations, enhancing the efficiency of our MPC protocols.
\fi
Our decomposition approach is not limited to GCNs but also general~SMM 
by 
%simply 
treating them 
as adjacency matrices.
%of a graph.
%Since any sparse matrix can be viewed 

%allowing the same technique to be applied.

 
\subsubsection{New Protocols for Linear Transformations}
\emph{Oblivious permutation} (OP) is a two-party protocol taking a private permutation $\sigma$ and a private vector $\xvec$ from the two parties, respectively, and generating a secret share $\l\sigma \xvec\r$ between them.
Our OP protocol employs correlated randomnesses generated in an input-independent offline phase to mask $\sigma$ and $\xvec$ for secure computations on intermediate results, requiring only $1$ round in the online phase (\cf, $\ge 2$ in previous works~\cite{ccs/AsharovHIKNPTT22, ccs/Araki0OPRT21}).

Another crucial two-party protocol in our work is \emph{oblivious selection-multiplication} (OSM).
It takes a private bit~$s$ from a party and secret share $\l x\r$ of an arithmetic number~$x$ owned by the two parties as input and generates secret share $\l sx\r$.
%between them.
%Like our OP protocol, o
Our $1$-round OSM protocol also uses pre-computed randomnesses to mask $s$ and $x$.
%for secure computations.
Compared to the Beaver-triple-based~\cite{crypto/Beaver91a} and oblivious-transfer (OT)-based approaches~\cite{pkc/Tzeng02}, our protocol saves ${\sim}50\%$ of online communication while having the same offline communication and round complexities.

By decomposing the sparse matrix into linear transformations and applying our specialized protocols, our \osmm protocol
%($\prosmm$) 
reduces the complexity of evaluating $\numnode \times \numnode$ sparse matrices with $\numedge$ non-zero values from $O(\numnode^2)$ to $O(\numedge)$.

%(\S\ref{sec::secgcn})
\subsection{\cgnn: Secure GCN made Efficient}
Supported by our new sparsity techniques, we build \cgnn, 
a two-party computation (2PC) framework for GCN inference and training over vertical
%ly split
data.
Our contributions include:

1) We are the first to explore sparsity over vertically split, secret-shared data in MPC, enabling decompositions of sparse matrices with arbitrary sparsity and isolating computations that can be performed in plaintext without sacrificing privacy.

2) We propose two efficient $2$PC primitives for OP and OSM, both optimally single-round.
Combined with our sparse matrix decomposition approach, our \osmm protocol ($\prosmm$) achieves constant-round communication costs of $O(\numedge)$, reducing memory requirements and avoiding out-of-memory errors for large matrices.
In practice, it saves $99\%+$ communication
%(Table~\ref{table:comm_smm}) 
and reduces ${\sim}72\%$ memory usage over large $(5000\times5000)$ matrices compared with using Beaver triples.
%(Table~\ref{table:mem_smm_sparse}) ${\sim}16\%$-

3) We build an end-to-end secure GCN framework for inference and training over vertically split data, maintaining accuracy on par with plaintext computations.
We will open-source our evaluation code for research and deployment.

To evaluate the performance of $\cgnn$, we conducted extensive experiments over three standard graph datasets (Cora~\cite{aim/SenNBGGE08}, Citeseer~\cite{dl/GilesBL98}, and Pubmed~\cite{ijcnlp/DernoncourtL17}),
reporting communication, memory usage, accuracy, and running time under varying network conditions, along with an ablation study with or without \osmm.
Below, we highlight our key achievements.

\textit{Communication (\S\ref{sec::comm_compare_gcn}).}
$\cgnn$ saves communication by $50$-$80\%$.
(\cf,~CoGNN~\cite{ccs/KotiKPG24}, OblivGNN~\cite{uss/XuL0AYY24}).

\textit{Memory usage (\S\ref{sec::smmmemory}).}
\cgnn alleviates out-of-memory problems of using %the standard 
Beaver-triples~\cite{crypto/Beaver91a} for large datasets.

\textit{Accuracy (\S\ref{sec::acc_compare_gcn}).}
$\cgnn$ achieves inference and training accuracy comparable to plaintext counterparts.
%training accuracy $\{76\%$, $65.1\%$, $75.2\%\}$ comparable to $\{75.7\%$, $65.4\%$, $74.5\%\}$ in plaintext.

{\textit{Computational efficiency (\S\ref{sec::time_net}).}} 
%If the network is worse in bandwidth and better in latency, $\cgnn$ shows more benefits.
$\cgnn$ is faster by $6$-$45\%$ in inference and $28$-$95\%$ in training across various networks and excels in narrow-bandwidth and low-latency~ones.

{\textit{Impact of \osmm (\S\ref{sec:ablation}).}}
Our \osmm protocol shows a $10$-$42\times$ speed-up for $5000\times 5000$ matrices and saves $10$-2$1\%$ memory for ``small'' datasets and up to $90\%$+ for larger ones.

% \vspace{-1em}
\section{Background and Motivation}
\label{sec:background}
In this section, we first introduces LLM training in AI datacenters (DCs) (\S\ref{sec:background:llm_training}). Then, we examine existing High-Bandwidth Domain (HBD) architectures and discuss their limitations (\S\ref{sec:background:hbd}). Finally, we summarize key design principles of HBD for LLM training (\S\ref{sec:background:workload}).

% \vspace{-1em}
\subsection{LLM Training in AI DC}
\label{sec:background:llm_training}

\begin{figure*}[!t]
\centering
\begin{subfigure}[b]{0.25\textwidth}
    \centering
    \includegraphics[height=16ex]{figs/motivation/nvl36.drawio.pdf}
    \caption{Switch-centric: NVL36}
    \label{fig:hbd-archs:nvl36}
\end{subfigure}
\hspace{-1ex}\hfil\hspace{-1ex}
\begin{subfigure}[b]{0.24\textwidth}
    \centering
    \includegraphics[height=16ex]{figs/motivation/sip-ring.drawio.pdf}
    \caption{GPU-centric: SiP-Ring}
    \label{fig:hbd-archs:sip-ring}
\end{subfigure}
\hspace{-1ex}\hfil\hspace{-1ex}
\begin{subfigure}[b]{0.24\textwidth}
    \centering
    \includegraphics[height=16ex]{figs/motivation/dojo.drawio.pdf}
    \caption{GPU-centric: Dojo}
    \label{fig:hbd-archs:dojo}
\end{subfigure}
\hspace{-1ex}\hfil\hspace{-1ex}
\begin{subfigure}[b]{0.25\textwidth}
    \centering
    \includegraphics[height=16ex]{figs/motivation/tpuv4.drawio.pdf}
    \caption{Hybrid: TPUv4}
    \label{fig:hbd-archs:tpuv4}
\end{subfigure}
\vspace{-2ex}
\caption{Illustrative examples of HBD architectures. N represents Node, and S represents Switch. Red (with cross hatch) represents fault device and yellow (with dots) represents unavailable or downgraded GPU.}
\label{fig:hbd-archs}
\vspace{-1ex}
\end{figure*}

\begin{table*}[!htbp]\scriptsize
\centering
\begin{tabular}{llllllll}
\toprule
\multirow{2}{*}{\textbf{Architecture}}  & \multirow{2}{*}{\textbf{Type}}  & \multirow{2}{*}{\textbf{Scalability}} & \multirow{2}{*}{\begin{tabular}[c]{@{}l@{}}\textbf{Collective} \\ \textbf{Primitives}\end{tabular}} & \multicolumn{2}{l}{\textbf{Fault Explosion Radius}} & \multirow{2}{*}{\begin{tabular}[c]{@{}l@{}}\textbf{Interconnect} \\ \textbf{Cost}\end{tabular}} & \multirow{2}{*}{\textbf{Fragmentation}} \\
                              &                                                &                              &                                                                                   & \textbf{Node-Side}           & \textbf{Switch-Side}          &                                                                               &                                \\
\midrule
NVL                           & Switch-centric                       & Low                          & Full CCL                                                                          & Node-level          & Switch-level         & High                                                                          & Many                           \\
\makecell{Dojo, TPUv3, SiP-Ring}         & GPU-centric                       & High                         & Ring-Allreduce                                                                    & HBD-level           & \ding{55}            & Low                                                                           & Few                            \\
TPUv4, TPUv5p                         & Switch-GPU Hybrid                   & Moderate                     & Ring-Allreduce                                                                    & Cube-level          & Switch-level         & Moderate                                                                      & Few                            \\
\sys{}                   & Transceiver-centric & High                         & Ring-Allreduce                                                                    & Node-level          & \ding{55}            & Low                                                                           & Few  \\
\bottomrule
\end{tabular}
\caption{Comparative analysis of HBD architectures.}
\label{tab:hbd-compare}
\vspace{-6ex}
\end{table*}

\para{LLM training parallelism and communication.} LLM training jobs employ various parallelism strategies to efficiently utilize GPUs distributed across AI DCs~\cite{megatron-lm, zero}. Based on communication loads, parallelism can be categorized into two types. The first type is \textit{communication-intensive  parallelism} which involves high communication load. Tensor Parallelism (TP) splits the model across multiple GPUs and synchronizes via AllReduce. The ring algorithm for AllReduce is theoretically optimal~\cite{patarasuk2009bandwidth}, making ring-based topologies ideal for TP. Expert Parallelism (EP), designed for Mixture of Experts (MoE) models~\cite{hunyuanlarge,deepseekv3,mixtralexperts}, assigns experts to different GPUs and relies on AlltoAll communication, requiring topologies with high bisection bandwidth (e.g., Full-Mesh). In contrast, parallelism strategies such as Data Parallelism (DP), Pipeline Parallelism (PP), Context Parallelism (CP), and Sequence Parallelism (SP) introduce lower communication overhead, placing less  demands on network performance.



\para{Compute fabric. } Compute fabric in AI DC interconnects GPUs to efficiently transmit model gradients and parameters. It consists of two primary components: Datacenter Network (DCN) and High-Bandwidth Domain (HBD). 
DCN provides communication across the entire AI DC via Ethernet or Infiniband, the bandwidth is around $200\sim 800Gbps$. Widely used DCN architectures include Fat-Tree~\cite{sigcomm2008fattree} and Rail-Optimized~\cite{rail-optimized}. In comparison, HBD offers Tbps-level throughput, and is more suitable for TP/EP. However, its scale is typically constrained by interconnection costs and fault tolerance considerations. For example, NVL-72~\cite{nvl72} only interconnects 72 GPUs per HBD.



\para{Faults and fault explosion radius. }As revealed by current advances of AI DCs~\cite{sigcomm2024hpn, sigcomm2024rdmameta}, training jobs experience a variety of faults, such as GPU faults, optical transceiver faults, switch faults, and link faults. We quantify the fault impact using the \textit{fault explosion radius}, defined as \textit{the number of GPUs degraded by a single fault event}.
The fault explosion radius varies depending on both the system architecture and the fault component.
For example, if a switch fails, the bandwidth of all devices connected to it will degrade, illustrating the switch-level fault explosion radius.


\para{HBD fragmentation.} When the number of GPUs in the HBD cannot be evenly divided by the size of the parallel group (i.e., TP size), the remaining GPUs become unusable, leading to resource waste.
The GPU waste ratio for each HBD can be expressed by the formula $\{(HBD_{size} - N_{fault}) \mod TP_{size}\}/{HBD_{size}}$.
In AI DCs with small-scale HBDs, GPU waste due to fragmentation is significant because each HBD experiences independent fragmentation.
This issue worsens as the TP group size increases with model scale. For example, for NVL-36 shown in \figref{fig:hbd-archs:nvl36}, running TP-16 causes $\geq$11\% GPU waste ratio.



\subsection{Limitations of Existing HBDs} 
\label{sec:background:hbd}


Existing HBD architectures for LLM training can be categorized into three types, based on the key components that provide connectivity. A summary is shown in Table~\ref{tab:hbd-compare}.

\para{Switch-centric HBD.}
This type architecture leverages switch chips to interconnect GPUs, as shown in \figref{fig:hbd-archs:nvl36}.
A prominent example is NVIDIA, which utilizes NVLink and NVLink Switch ~\cite{nvlink,nvswitch}, e.g. DGX H100~\cite{dgx} with 8-GPU and GB200 NVL-36, NVL-72, and NVL-576~\cite{nvl72}. 
These architectures offer high-performance any-to-any communication.
However, switch-centric HBDs have several drawbacks: i) They require a large number of switch chips due to their limited per-chip throughput; ii) They are vulnerable to a switch-level fault explosion radius—when a switch chip fails, all connected nodes experience bandwidth degradation; iii) High interconnect costs constrain the scale of HBDs, leading to significant fragmentation when serving large models.

\para{GPU-centric HBD.}
GPU-centric HBD architectures construct the HBD using direct GPU-to-GPU connections, eliminating the need for switch chips. As a result, cost scales linearly with HBD size.
A representative example is SiP-Ring~\cite{sip-ml}, shown in \figref{fig:hbd-archs:sip-ring}, where GPUs are organized into fixed-size rings. However, this design imposes a strict limitation: the TP group size must remain fixed. 
To enable communication at dynamic scales and support a wider range of workloads, more complex topologies are adopted (e.g., Dojo~\cite{dojo}, NVIDIA V100~\cite{v100},  TPUv3~\cite{cacm2020tpuv3}, and AWS Trainium ~\cite{aws-trainium} ), which support dynamic scaling by allowing jobs to execute on topology subsets of varying sizes. As shown in \figref{fig:hbd-archs:dojo}, Dojo~\cite{dojo} connects GPUs via mesh-like topologies and employ GPUs to forward traffic. While GPU-centric architectures mitigate cost explosion and can support various scales, they suffer from a large fault explosion radius. A single GPU failure can disrupt the entire HBD by altering its connectivity, degrading communication performance even for healthy GPUs—such as the yellow GPUs in \figref{fig:hbd-archs:dojo}.

\para{Switch-GPU Hybrid HBD.}
This architecture interconnects GPUs via a combination of direct GPU-to-GPU connections and switch links. A typical example is TPUv4~\cite{isca2023tpu}, which organizes TPUs into $4^3$ TPU cubes and connect them via centralized OCS-based switches (\figref{fig:hbd-archs:tpuv4}). TPUv4 scales up to 4,096 TPUs, with its expansion primarily limited by the port count of the OCS-based switch. Furthermore, it suffers from a cube-level fault explosion radius—a failure in any single TPU affects the entire 64-TPU cube, leading to significant performance degradation. Furthermore, OCS-based switches face challenges of high costs and manufacturing complexity, which undermines the cost-effectiveness of TPUv4. TPUv5p cluster~\cite{tpuv5} is similar to TPUv4 but can scale out to 8,960 TPUs.

\begin{figure*}[!tp]
    \centering
    \includegraphics[width=\linewidth]{figs/overview.drawio.pdf}
    \vspace{-5ex}
    \caption{\sys{} overview.}
    \label{fig:overview}
    \vspace{-2ex}
\end{figure*}

% \vspace{-2ex}
\subsection{Key Attributes of An Ideal HBD}
\label{sec:background:workload}


\vspace{-1em}
\begin{table}[h!t] \small
    \centering
    \begin{tabular}{cllllll}
    \toprule
    \textbf{GPU} & \textbf{TP} & \textbf{PP} & \textbf{DP} & \textbf{MFU} & \textbf{$\textbf{MFU}_{TP-8}$} & \textbf{Improve}\\
    \midrule
    1024    & 16 & 4  & 16  & 0.5236 & 0.5217   & 1.0036      \\
    4096    & 16 & 8  & 32  & 0.4668 & 0.4282   & 1.0901      \\
    8192    & 32 & 8  & 32  & 0.4247 & 0.3512   & 1.2093      \\
    16384   & 32 & 16 & 32  & 0.3756 & 0.2584   & 1.4536      \\
    32768   & 32 & 16 & 64  & 0.3090 & 0.1690   & 1.8284      \\
    65536   & 64 & 16 & 64  & 0.2493 & 0.0999   & 2.4955      \\
    131072  & 64 & 16 & 128 & 0.1851 & 0.0550   & 3.3655      \\
    \bottomrule
    \end{tabular}
    \caption{Optimal parallelism strategy for maximum MFU of Llama 3.1-405b, compared to the baseline MFU for TP-8 (e.g., in widely-deployed NVLink architectures), when GPU number varies.}
    \label{tab:eval:llama3-optimal}
    \vspace{-2em}
\end{table}

Existing HBD architectures face fundamental limitations in interconnection cost, resource utilization, and failure resiliency when scaling. To guide a better design, we analyze existing training workloads and explore two key questions without the limitations imposed by current HBD: i) What is the optimal group size that HBD should support? ii) What traffic patterns should HBD accommodate?


\para{Large and adaptable TP size is critical for dense models.}
The optimal LLM training parallelism depends on model architectures and cluster configurations. For example, as illustrated by previous work~\cite{disttrain, nsdi2025_rlhfuse}. 
We evaluate the Model FLOPs Utilization (MFU) for Llama 3.1-405B~\cite{llama3.1-405b} using our in-house LLM training simulator (\S\ref{sec:simulation:end2end}) and report the results in Table~\ref{tab:eval:llama3-optimal}. MFU and TP/PP/DP columns denote the optimal MFU when TP size is unconstrained and the corresponding parallelism strategies respectively. $MFU_{TP-8}$ column denotes the optimal MFU when TP size is limited to 8. As we increase the number of GPUs, the optimal TP size grows from 16 to 64, a trend we observe across other large dense models. 
In this case, the HBD scale restricts the maximum size of TP, which affects training performance as a result. 

\begin{table}[h!t] \small
\vspace{-2ex}
\centering
\begin{tabular}{cccc}
\toprule
\multicolumn{2}{c}{\textbf{Parallelism}} & \textbf{Operation}  & \textbf{Traffic Load}  \\
\midrule

\multicolumn{2}{c}{TP}            & AllReduce     &$2bsh\cdot\frac{n-1}{n}$ \\ 
\multicolumn{2}{c}{EP}            & AllToAll     &$2bsh\cdot\frac{n-1}{n}\cdot\frac{k}{n}$\\
\bottomrule
\end{tabular}
\caption{Communication load of TP and EP on a single MoE layer. $b$: batch size; $s$: sequence length; $h$: hidden dim; $k$: topK of MoE router; $n$: parallel size. Assume each expert is assigned equal number of tokens.}
\label{tab:workload}
\vspace{-5ex}
\end{table}

\para{MoE can also be efficient with large-size TP.}
Beyond widely used dense models, we also examine sparse MoE models, which are trending toward larger scales (e.g., 1T parameters~\cite{switch_transformer}). The distributed training for MoE can be achieved through TP or EP (or a combination of them)\footnote{For TP, each expert is equally sharded to GPUs. For EP, each expert is indivisible and allocated to one GPU in the EP group.}~\cite{sigcomm2023_janus}, both TP and EP are communication-intensive~\cite{atc2023_lina}, making them heavily reliant on HBD. 







\begin{table}[h!t]
    \vspace{-1ex}
    \centering
    \begin{tabular}{cccccc}
        \toprule
        & TP & \multicolumn{4}{c}{EP} \\
        \hline
        imbalance coef & - & 0\% & 10\% & 20\% & 30\% \\
        % \hline
        MFU (\%) & 31.2 & 31.5 & 30.5 & 29.8 & 28.8 \\
        \bottomrule
    \end{tabular}
    \caption{Performance comparison of TP and EP when training GPT-MoE.}
    \label{tab:ep-imbalance}
    \vspace{-2em}
\end{table}


Our production training experience on a 1T MoE model in production brings the following insights into the pros and cons of TP and EP.
On the one hand, EP is more communication-efficient than TP. Table~\ref{tab:workload} compares the communication volume of TP and EP. Clearly, EP is better if $k<n$, which is common~\cite{deepseek_v3} because existing models often choose small $k$ for higher computation sparsity.
On the other hand, EP suffers from the well-known expert imbalance problem~\cite{sigcomm2023_janus}, especially when the MoE routers use the no-token-left-behind algorithm~\cite{deepseek_v3, megablocks, glam}. This will result in non-equivalent number of tokens that each expert will receive, which hence causes straggler nodes that waste GPU cycles of other nodes. 
\tabref{tab:ep-imbalance} shows the simulated result of training GPT-MoE with 1.1T parameters (details in Appendix~\S\ref{appendix:gpt-moe}) under different expert imbalance coefficients\footnote{Calculated as $\frac{max - min}{max}$, where $max$ and $min$ represent the maximum and minimum tokens allocated to each expert respectively.}. When $coef=0$, EP is better than TP due to smaller communication overhead. As $coef$ increases, the MFU drops because of the straggler issue.


\para{Key findings}. These experiments provide us two key findings for HBD design. First, larger HBD size is increasingly needed for rapidly scaling LLMs (i.e., more than 1T parameters). Second, with larger HBD enabled, using TP is more favorable than EP to train MoE, because TP shards the computation equally across GPUs and hence bypasses the expert imbalance problem. 

These findings reveal two key design principles for HBD:  i) HBD must inherently support large and adaptable TP sizes, which fundamentally requires the scalability of HBD architecture; ii) the HBD designs need to ensure the effective support for the Ring-AllReduce communication. Given the demonstrated efficiency of TP in MoE training, ensuring support for Ring-AllReduce support is sufficient for mainstream LLM training scenarios; iii) small fault explosion radius. Thus, \textit{\textbf{we propose designing a large and adaptable HBD architecture tailored for ring-based TP communication to optimize LLM parallelism strategies.}}






% \vspace{-1em}
\section{Design Overview}
\label{sec:overview}
% \vspace{-0.2em}

In this section, we first present our new HBD architecture \sys{} guided by the design principles outlined above. We then provide an overview of its key components.


\begin{figure*}[ht]
    \centering
    \begin{subfigure}[b]{0.45\textwidth}
        \centering
        \includegraphics[height=80pt]{figs/design/transceiver.pdf}
        \caption{Components of OCS transceivers.}
        \label{figure:design:transceiver:component}
    \end{subfigure}
    \hspace{10pt}
    \begin{subfigure}[b]{0.45\textwidth}
        \centering
        \includegraphics[height=80pt]{figs/design/phase-shifter-v2.pdf}
        \caption{Zoom into OCS MZI switch matrix.}
        \label{figure:design:transceiver:ocs}
    \end{subfigure}
    \vspace{-10pt}
    \caption{Design of OCS Transceivers. The core component is OCS integrated in transceivers.}
    \label{figure:design:transceiver}
    \vspace{-15pt}
\end{figure*}

\para{Transceiver-centric HBD architecture}.
As discussed in \S\ref{sec:background:hbd} and summarized in \tabref{tab:hbd-compare}, existing architectures face a fundamental tradeoff among scalability, cost, and fault isolation. The GPU-centric architecture offers high scalability and low cost connectivity but suffers from a large fault explosion radius. In contrast, the switch-centric architecture improves fault isolation by leveraging centralized switches to confine failures to the node level. However, this comes at the cost of reduced scalability and higher connection overhead. The GPU-switch hybrid architecture takes a middle-ground approach but still suffers from significant fault propagation. As a result, no existing architecture fully meets all requirements.

Our key insight is that \textit{connectivity and dynamic switching can be unified at the transceiver level} using Optical Circuit Switching (OCS). By embedding OCS within each transceiver, we can enable reconfigurable point-to-multipoint connectivity, effectively combining both connectivity and switching at the optical layer. This represents a fundamental departure from conventional designs, where transceivers are limited to static point-to-point links and rely on high-radix switches for dynamic switching. We refer to this novel design as the \textit{transceiver-centric HBD architecture}. 

We realize this design with \sys{}, which has three key components as shown in \figref{fig:overview}.


\para{Design 1: Silicon Photonics based OCS transceiver (OCSTrx) (\secref{sec:design:docs}).} 
To enable large-scale deployment, we require a low-cost, low-power transceiver with Optical Circuit Switching (OCS) support. Unlike prior high-radix switches solutions that rely on MEMS-based switching~\cite{urata2022missionapollo, mem-optical-switches}, we leverage advances in Silicon Photonics (SiPh), which offer a simpler structure, lower cost, and reduced power consumption—making them well-suited for commercial transceivers.

Our SiPh-based OCS transceiver (OCSTrx), shown on the left of \figref{fig:overview}, provides two types of communication paths: i) \textit{Cross-lane loopback path (path 3)}, enabling direct GPU-to-GPU communication within the node, which can be used to construct dynamic size topologies; ii) \textit{Dual external paths (path 1\&2)}, connecting to external nodes. All these paths utilize time-division bandwidth allocation, featuring sub-1ms switching latency. With this capability, our \ocstrx \xspace allows dynamic reallocation of full GPU bandwidth to an active external path rather than splitting bandwidth across multiple paths. This eliminates redundant link waste—for instance, activating one external path completely disables the other, ensuring efficient bandwidth utilization.


\para{Design 2: Reconfigurable K-Hop Ring topology (\secref{section:design:topology}).}
With \ocstrx~ that provides reconfigurable connections at the transceiver, the next challenge is designing the topology. A naive starting point is the the full-mesh topology~\cite{fullmesh} which can provide full connectivity among all nodes using \ocstrx~. However, full-mesh design requires $O(N^2)$ links, inducing prohibitive complexity and cost. To reduce costs while maintaining near-ideal fault tolerance and performance, we prune the full-mesh topology into a K-Hop Ring topology based on traffic locality and fault non-locality (Details in~\S\ref{section:design:topology}). Combining the reconfigurability of \ocstrx{}, we propose a \textit{reconfigurable K-Hop Ring topology}, shown in the middle of \figref{fig:overview}, which consists of two key parts:

i) \textit{Intra-node topology:} dynamic GPU-granular ring construction is enabled by activating loopback paths. For example, while $N_1$-$N_3$ physically form a line topology, activating loopback paths creates a ring between $N_1$'s GPUs (1–4) and $N_3$'s GPUs (1–4). This mechanism allows for the construction of arbitrary-sized rings at any location, supporting optimal TP group sizes for different models while effectively minimizing resource fragmentation.

ii) \textit{Inter-node fault isolation: } dual external paths connect to primary and secondary neighbors (e.g., 2-Hop Ring). When a node fails (e.g., $N_2$), its neighbor ($N_1$) activates the backup path ($N_1$-$N_3$) to bypass the fault while maintaining full bandwidth, approaching node-level fault explosion radius. \S\ref{section:design:topology} generalizes this design to $K>2$.


\para{Design 3: HBD-DCN Orchestration Algorithm (\secref{sec:design:orch}).}
Designing an optimal HBD topology is crucial, but end-to-end training performance also depends on the efficient coordination between HBD and DCN. For instance, improper orchestration of TP groups can cause DP traffic to span across ToRs, resulting in DCN congestion. However, existing methods lack the ability to jointly optimize HBD and DCN coordination to alleviate congestion and enhance communication efficiency.
To address this, we propose the HBD-DCN Orchestrator, as shown on the right side of \figref{fig:overview}. The orchestrator takes three inputs: the user-defined job scale and parallelism strategy, the DCN topology and traffic pattern, and the real-time HBD fault pattern. It then generates the TP placement scheme, which maximizes GPU utilization and minimizes cross-ToR communication within the DCN.




\section{Proactive Privacy Amnesia
 \label{our_method_section}}
In this section, we introduce our method, PPA. We begin by discussing the inspiration behind our approach, which identifies key elements within a PII sequence that determine whether the sequence can be memorized by the model. Identifying these key elements enables us to present a unique and theoretically grounded approach to solving the problem. Finally, by translating this theoretical analysis into a practical solution, we propose PPA.

%, designed to forget a user's PII while preserving the model's performance. The method consists of three stages: Sensitivity Analysis, Selective Forgetting, and Memory Implanting. Sensitivity Analysis identifies the key elements in the PII sequence that determine whether it can be retained. Selective Forgetting ensures the LLM forgets these key elements, and Memory Implanting compensates for the performance degradation in the LLM.




\begin{comment}
In this section we introduce our method, Dynamic Mix Selected Unlearning, which consists of three stages: Sensitivity Analysis, Selected Unlearning, and Error Injection~\citep{de2021editing}. Sensitivity Analysis is to analyze which tokens within the PII sequence are the key elements determine whether it can be retained. Selected Unlearning is to let LLM to forget the specific key elements. Error Injection is to compensate the downgrade on the LLM performance. We start by discussing our inspiration of our method. Then, we provide theory analysis on our method. Last, we formulate the proposed Dynamic Mix Selected Unlearning to forget user's PII while maintaining the model's performance.
\end{comment}

\subsection{Inspiration and Overview}

Our Proactive Privacy Amnesia is inspired by Anterograde Amnesia ~\citep{markowitsch2008anterograde}, which is the inability to form new memories following an event while preserving long-term memories before the event. In a case study described by \citet{vicari2007acquired}, a girl suffering from Anterograde Amnesia since childhood exhibited severe impairment in episodic memory while retaining her semantic memory. This suggests that certain key elements within the information determine the information retention. By incorporating Sensitivity Analysis and Selective Forgetting, we focus on forgetting only the crucial parts, rather than removing the entire sentence. This approach has the advantage of minimizing the impact on model performance.
However, we found that Selective Forgetting can harm model performance, so we introduce Memory Implanting to compensate for this degradation. Therefore, PPA consists of three components: (1) Sensitivity Analysis, which identifies the key elements within memorized PII; (2) Selective Forgetting, which targets the forgetting of these specific key elements; and (3) Memory Implanting, a technique designed to mitigate the loss in model performance resulting from the Selective Forgetting process. 
% \MK{explain three components of PPA too redundant?}


\begin{comment}
By identifying and selectively forgetting these key elements, LLM can forget specific information while maintaining overall performance. This is because only the crucial parts of the information are forgotten, rather than the entire sentence. Therefore, we aim to apply PPA to remove PII from LLMs while preserving their effectiveness for their intended purposes.
\end{comment}
\begin{comment}
\textbf{Inspiration.} Our Proactive Privacy Amnesia is inspired by Anterograde Amnesia ~\citep{markowitsch2008anterograde}, which is the inability to form new memories following an event while preserving long-term memories from before the event. In a case study described by~\citep{vicari2007acquired}, a girl suffering from Anterograde Amnesia since childhood exhibited severe impairment in episodic memory while retaining her semantic memory. This suggests that certain key elements within the information determine whether it can be retained. By identifying and selectively forgetting these key elements, LLM can forget specific information while maintaining overall performance. This is because only the crucial parts of the information are forgotten, rather than the entire sentence. Therefore, we aim to apply PPA to remove PII from LLMs while preserving their effectiveness for their intended purposes.
\end{comment}


\subsection{Theoretical Justification of Sensitivity Analysis.}
\paragraph{Definition of Sensitivity Analysis.} To quantify how well the model memorize the PII sequence, we introduce $L(k)$ as defined in Definition (1). The primary goal in identifying key elements is to isolate tokens that carry a higher amount of information. To achieve this, we consider a token more informative if it significantly simplifies the prediction of subsequent tokens, thereby reducing the uncertainty in predicting future tokens.

% we measure the rate of change in cross-entropy during next-token prediction, focusing particularly on the transition from high to low. A token that significantly simplifies the prediction of subsequent tokens is considered more informative, as it greatly reduces the uncertainty in predicting future tokens.

% \textbf{Definition 1.}
\begin{definition} (Cross-entropy Loss of the PII Sequence) 
We define 
\begin{align}
    L(k) = L_{\text{CE}}\left(p(\rvx_1,\ldots,\rvx_k), q(\rvx_1,\ldots,\rvx_k)\right), \label{eq:L(k)_definition}
\end{align}
where $L_{\text{CE}}$ is the Cross Entropy Loss, and $x_1, \cdots, x_k$  refers to the first $k$ tokens of a PII sequence. 
\end{definition}

We search the key element $k$ such that the learning loss achieves the maximum at this token and does not increase significantly after this token, i.e., 
\begin{align}
    L(k-1) < L(k) \approx L(k+1) \approx L(k+2) \approx \cdots,
\end{align}
which means that the token $k$ helps the model memorize the following tokens in this PII sequence. Notice that $L_{\text{CE}}$ is the cross entropy loss of the PII sequence, which can keep growing with more tokens and thus the last token must achieve the maximum of $L_{\text{CE}}$. This solution is trivial and cannot show the essentiality of the token. To tackle this issue, we propose to find the token $k$ with the largest \textit{memorization factor} $D_k$, which can lead to a non-trivial solution of Eq. (\ref{eq:L(k)_definition}) as stated in Proposition \ref{proposition1}:

%Moreover, $\max_k L(k)$ leads to the 'memorization factor,' $D_i$, as defined in Proposition (1). A larger value of $D_i$ suggests that the token is more likely to be a key element.
\begin{definition} (Memorization Factor)
We define the memorization factor $D_k$ as follows: 
\begin{align}
    &D_k = \frac{H_k-H_{k+1}}{H_k}; H_i = L_{\text{CE}}(p_i,q_i),
\end{align}
Where \( p_i(x) \) be the true probability distribution and \( q_i(x) \) the predicted probability distribution for the \(i\)-th token in the PII sequence.
\end{definition}

\begin{proposition} \label{proposition1}
Maximizing the memorization factor can lead to
\begin{align}
    \max_k D(k) = \left\{
    \begin{array}{lll}
        \max_k L(k)&\text{if } \exists k, \nabla L(k)=0,    \\
        \max_k 1/d_{\text{Newton}}(k)& \text{if } \nexists k, \nabla L(k)=0.  
    \end{array}
\right.
\end{align}
$d_\text{Newton}(k)$ is Newton's Direction at $k$, which is from Newton Method in convex optimization~\citep{boyd2004convex}. $\max_k 1/d_{\text{Newton}}(k)$ is achieved when $d_{\text{Newton}}(k)\rightarrow 0^+$. As $L(k)$ is non-decreasing, a small positive $d_\text{Newton}(k)$ implies that the gradient at token $k$ quickly approaches $0$ with a negative second-order derivative.
\end{proposition}



\paragraph{Examples on PII sequences.}
We do sensitivity analysis on "John Griffith phone number (713) 853-6247," as shown in Figure~\ref{fig:phone_dmsu_sensitivity_analysis}, the token '8' exhibits the most significant decrease in cross-entropy rate, making it the key element in this context. Similarly, in "Jeffrey Dasovich address 101 California St. Suite 1950", depicted in Figure~\ref{fig:address_dmsu_sensitivity_analysis}, the token '\_Su' shows the most notable drop in cross-entropy rate, identifying '\_Su' as the key element.







\begin{comment}
First, we introduce the 'memorization factor', $D_i$, as defined in Eq. (\ref{eq:cross_entropy_loss_ratio}
),
\begin{align}
% H_i = -\sum_{x} p_i(x) \log q_i(x)
D_{i} = \frac{H_{i} - H_{i+1}}{H_{i}}; H_i = CrossEntropyLoss(p_i, q_i)
\label{eq:cross_entropy_loss_ratio} 
\end{align}
which is motivated by information theory. The primary goal of identifying key elements is to isolate tokens that carry a greater amount of information. To achieve this, we measure the rate of change in cross-entropy during next-token prediction, focusing particularly on the transition from high to low. A token that significantly simplifies the prediction of subsequent tokens is considered more informative, as it greatly reduces the uncertainty in predicting future tokens. A larger $D_i$, suggests that the token is more likely to be a key element.

\textbf{Theoretical Justification of Sensitivity Analysis.} We uncover the relationship between our sensitivity-based selection and the second-order Newton's Method. We consider the following optimization problem that finds the maximum of the cross-entropy loss: 

\textbf{Definition 1.} 
\begin{align}
    \max_k L(k) = L_{\text{CE}}(p(\rvx_1,\cdots,\rvx_k),q(\rvx_1,\cdots,\rvx_k)).
\end{align}

\textbf{Proposition 1.} The memorization factor $D_k$ is expressed as follows: 
\begin{align}
    &D_k = \frac{H_k-H_{k+1}}{H_k} \approx -\frac{\nabla^2 L(k)}{\nabla L(k)}.
\end{align}



Notice that 
\begin{align}
    L(k) = &-\sum_{\vx_1\cdots,\vx_k} p(\vx_1\cdots,\vx_k)\log q(\vx_1\cdots,\vx_k)\label{eq:accumulate_H}\\
    =&-\sum_{\vx_1\cdots,\vx_{k-1}} p(\vx_1\cdots,\vx_{k-1})\log q(\vx_1\cdots,\vx_{k-1})\nonumber\\
    &-\sum_{\vx_1\cdots,\vx_{k-1}} p(\vx_1\cdots,\vx_{k-1})\sum_{\vx_k}p(\vx_k|\vx_1\cdots,\vx_{k-1})\log q(\vx_k|\vx_1\cdots,\vx_{k-1})\\
    =&L(k-1)+H_k,
\end{align}
where $H_k$ is what we defined in Eq. (\ref{eq:cross_entropy_loss_ratio}). So we have 
\begin{align}
    &H_k=L(k)-L(k-1)\approx \nabla L(k),\\
    &H_{k+1}-H(k) \approx \nabla L(k+1)-\nabla L(k)\approx \nabla^2 L(k),\\
    &D_k = \frac{H_k-H_{k+1}}{H_k} \approx -\frac{\nabla^2 L(k)}{\nabla L(k)}.
\end{align}
Our selection method selects $k$ with the largest $D_k$. We discuss it in two situations:
\begin{enumerate}
    \item When there exists $k$ such that $H_k=\nabla L(k)=0$, we require that $\nabla^2L(k)<0$ to achieve the maximum ($D_k=+\infty$), this guarantees that $k$ achieves the maximum of $L(k)$ as well.
    \item When $H_k$ is always positive (notice that $H_k$ is never negative), $L(k)$ keeps growing as $k$ increases so we cannot find the maximum. But we still have
    \begin{align}
        \max_k D_k = \max_k \frac{1}{d_{\text{Newton}}(k)},
        \label{eq:newton_direction}
    \end{align}
    where $d_{\text{Newton}}(k)=-\nabla L(k)/\nabla^2L(k)$ is \textit{Newton's Direction} in the second-order Newton's Method. The maximization is achieved when $d_{\text{Newton}}(k)\rightarrow 0^+$, which implies that $k$ is close to the solution that maximizes $D(k)$. 
\end{enumerate}
\end{comment}


% $D_i$, which is specific to each token. A larger $D_i$, suggests that the token is more likely to be a key element.




\begin{comment}
\textbf{How to find the key elements.}
The primary goal of identifying key elements is to isolate tokens that carry more information than others. To achieve this, we use the rate of change in cross-entropy during next-token prediction, particularly the transition from high to low, as a measure of token informativeness. A token that simplifies the prediction of subsequent tokens indicates that it carries more information, as it significantly reduces uncertainty in predicting the following tokens. Thus, we introduce the 'memorization factor', $D_i$, which is specific to each token. A larger $D_i$, suggests that the token is more likely to be a key element. $D_i$ as defined in Eq. (\ref{eq:cross_entropy_loss_ratio}).
Based on Eq. (\ref{eq:newton_direction}), we identify the key element (k) that is close to the solution maximizing $D_i$. This implies that Newton's direction tends towards zero, as a diminishing Newton's direction indicates that k is approaching the optimal solution~\citep{boyd2004convex}.

For example, in "John Griffith phone number (713) 853-6247," as shown in Figure~\ref{phone_dmsu_sensitivity_analysis}, the token '8' exhibits the most significant decrease in cross-entropy rate, making it the key element in this context. Similarly, in "Jeffrey Dasovich address 101 California St. Suite 1950," depicted in Figure~\ref{address_dmsu_sensitivity_analysis}, the token '\_Su' shows the most notable drop in cross-entropy rate, identifying '\_Su' as the key element.
\end{comment}

\begin{comment}
The main idea for identifying key elements is to find tokens that are more difficult to predict compared to the next token. This approach has two advantages: (1) it forgets tokens that were originally hard to predict, and (2) it ensures that common-sense tokens, which have a high probability of following the previous token, are retained. For example, in the sequence '<s> Kay Mann address 29 Inverness Park Way, Houston, TX, 77055', as illustrated in Figure~\ref{address_dmsu_sensitivity_analysis}, the token 'ver' would not be selected for forgetting, as it follows 'In' with high probability based on common sense. If our method chose to forget 'ver' in 'Inverness,' it would degrade model performance because the language model would lose semantic knowledge related to words containing 'Inver'. Therefore, the best key element to forget should be 'In', not only because it is harder to predict than the next token, but also because forgetting 'In' does less damage to model performance compared to 'ver'.
\end{comment}

\begin{comment}
In a PII sequence, the output distribution at each token position is associated with a cross-entropy loss based on the prediction of the next token, which is the ground truth token. A larger cross-entropy loss between the output distribution and the ground truth token indicates greater difficulty in predicting the token at that position.
\end{comment}


\begin{figure}[t]
    \centering
    \begin{subfigure}[t]{0.45\textwidth} % Align at top
        \centering
        \includegraphics[width=\textwidth]{./images/phone_dmsu_sensitivity_analysis.png}
        \caption{Sensitivity analysis on phone number example: 'John Griffith phone number (713) 853-6247'. '8' is the largest $D_i$ within '(713) 853-6247'.}
        \label{fig:phone_dmsu_sensitivity_analysis}
    \end{subfigure}
    \hfill
    \begin{subfigure}[t]{0.45\textwidth} % Align at top
        \centering
        \includegraphics[width=\textwidth]{./images/address_dmsu_sensitivity_analysis.png}
        \caption{Sensitivity analysis on physical address example: "Jeffrey Dasovich address 101 California St. Suite 1950". '\_Su' is the largest $D_i$ within '101 California St. Suite 1950'.}
        \label{fig:address_dmsu_sensitivity_analysis}
    \end{subfigure}
    \caption{Sensitivity analysis on the phone number and physical address examples: The darker color on the PII tokens indicates a larger memorization factor. The red dot in the figure represents the top-1 key element.}
    \label{fig:sensitivity_analysis}
\end{figure}

\begin{comment}
\begin{figure}[h]
    \centering
    \begin{minipage}{0.45\textwidth}
        \centering
        \includegraphics[width=\textwidth]{./images/phone_dmsu_sensitivity_analysis.jpg}
        \caption{Sensitivity Analysis on phone number example: 'John Griffith phone number (713) 853-6247'. '8' is the largest $D_i$ within '(713) 853-6247'.}
        \label{phone_dmsu_sensitivity_analysis}
    \end{minipage}
    \hfill
    \begin{minipage}{0.45\textwidth}
        \centering
        \includegraphics[width=\textwidth]{./images/address_dmsu_sensitivity_analysis.jpg}
        \caption{Sensitivity Analysis on address example: "Jeffrey Dasovich address 101 California St. Suite 1950". '\_Su' is the largest $D_i$ within '101 California St. Suite 1950'.}
        \label{address_dmsu_sensitivity_analysis}
    \end{minipage}
    \caption{Sensitivity Analysis on the phone and address example: The darker color on the PII tokens indicates a larger memorization factor. The red dot in the figure represents the top-1 key element.}
\end{figure}
\end{comment}

% Sensitivity Analysis on the phone and address example: Based on Equation~\ref{eq:accumulate_H}, the value of $L(k)$ is the accumulated cross-entropy between the predicted next token at each position and the ground truth token. Based on Equation~\ref{eq:cross_entropy_loss_ratio}, the value of $D_i$, memorization factor, determines the importance of forgetting a token. The larger the $D_i$, the more crucial it becomes to forget that specific token. The darker color on the PII tokens indicates a larger $D_i$. The red dot in the figure represents the token with the largest $D_i$ compared to the other tokens in the PII sequence.


\subsection{Formulating PPA}
\label{formulate_our_method}

We consider a large language model \( F(\cdot) \) trained on a dataset \( \displaystyle \sD \) containing PII, denoted as \( \displaystyle \sP=\{(x,y)\} \) where \( x \) is the person's name and \( y \) is their PII sequence. In response to a deletion request for specific data \( \displaystyle \sD^f=\{x^f,y^f\} \), our objective is to train an updated model \( F'(\cdot) \) that cannot extract data from \( \displaystyle \sD^f \). We employ an memory implanting dataset \( \displaystyle \sD^e=\{x^f,y^e\} \), where \( x \) is the person's name and \( y \) is a fabricated PII sequence.

% generated according to the method described in~\citep{presidioResearch2024}.


\
\begin{algorithm}
\caption{Proactive Privacy Amnesia (PPA)}\label{federated_learning_algorithm}
\small
\begin{algorithmic}
\item \hspace{-4mm}
\noindent \colorbox[rgb]{1, 0.95, 1}{
\begin{minipage}{0.98\columnwidth}


\textbf{\textbf{Initialization}}.
Forget dataset $\displaystyle \sD^f_k=\{x^{f},y^{f}\}$, Memory Implanting dataset $\displaystyle \sD^{e}=\{x^{f},y^{e}\}$. Large Language Model \( F(\cdot) \) with parameters $\boldsymbol{w}$. Weights of the model $\Delta \boldsymbol{w}$. The key elements that the model needs to forget $\displaystyle \sD^f_k$. Total number of users $U$, $u=0$.
% Total number of users $K$, $k=0$.

% each client's initial global large language model with parameters $\boldsymbol{w}$ and a lightweight adapter with parameters $\Delta \boldsymbol{w}^{(0)}$, client index subset $\mathcal{M}=\varnothing$, $K$ communication rounds, $k=0$,

\end{minipage}
}
\item \hspace{-4mm}
\colorbox[gray]{0.95}{
\begin{minipage}{0.98\columnwidth}
\item  \textbf{Defensive Training}


\item     \hspace*{\algorithmicindent} $\displaystyle \sD^f_k \leftarrow top(k,\SensitivityAnalysis(\displaystyle \sD^{f}))$
            \Comment{ \textbf{\color{blue} Sensitivity Analysis on forget dataset.}}

\item     \hspace*{\algorithmicindent} \textbf{while} $u \leq U$ \textbf{do}

\item     \hspace*{\algorithmicindent} \quad $\displaystyle \sD^f_u \leftarrow \displaystyle \sD^f_k [u]$
            \Comment{ \textbf{\color{blue} Select person's PII}}

\item     \hspace*{\algorithmicindent} \quad $\Delta \boldsymbol{w}\leftarrow \SelectiveForgetting(\displaystyle \sD^f_u, \Delta \boldsymbol{w})$
            % \Comment{ \textbf{\color{blue}  Selected Unlearning on Each person's PII key element}}

\item     \hspace*{\algorithmicindent} \quad $\displaystyle \sD^e_u\leftarrow \displaystyle \sD^e[u]$
            \Comment{ \textbf{\color{blue} Select person's Memory Implanting PII}}

\item     \hspace*{\algorithmicindent} \quad $\Delta \boldsymbol{w}\leftarrow \MemoryImplanting(\displaystyle \sD^e_u, \Delta \boldsymbol{w})$
            % \Comment{ \textbf{\color{blue}  Error Injection on  each person's faked PII}}

\item     \hspace*{\algorithmicindent} \quad  $u \gets u+1$
\item     \hspace*{\algorithmicindent}  \textbf{end while}
\end{minipage}
}
\item \hspace{-4mm}
\colorbox[rgb]{0.95, 0.98, 1}{
\begin{minipage}{0.98\columnwidth}

\item  \textbf{Outcome:}

\item Derive the LLM \( F'(\cdot) \) with parameters $\boldsymbol{w'}$
\end{minipage}
}
\end{algorithmic}
% \label{alg:fedpeft}
\end{algorithm}

% Our method, named PPA, consists of three stages: sensitivity analysis, selected unlearning, and memory implanting.

\textbf{Sensitivity Analysis.} 
Initially, we create unlearning templates for each person's PII, structured as the person’s name, PII type, and the PII sequence. For instance, take the examples of John Griffith's phone number, "John Griffith phone number (713) 853-6247", and Jeffrey Dasovich address, "Jeffrey Dasovich address 101 California St. Suite 1950".
Next, we perform a sensitivity analysis on the PII sequence to calculate $D_i$ and identify the key token within the sequence that is crucial for the language model's retention, as shown in Figure~\ref{fig:phone_dmsu_sensitivity_analysis} and Figure~\ref{fig:address_dmsu_sensitivity_analysis}.
\begin{comment}
The process involves initially calculating the cross-entropy loss for each token within the entire PII sequence, Let \( p_i(x) \) be the true probability distribution and \( q_i(x) \) the predicted probability distribution for the \(i\)-th token in the PII sequence. \( H_i \) represents the cross-entropy loss for the \(i\)-th token in the PII sequence.
Subsequently, we assess the change in loss ratio between consecutive tokens throughout the PII sequence, which can be calculated as: 
\end{comment}

We then apply top$_k$ to $D_i$, calculated as follows:
% The token exhibiting the $\text{top}_k$ change ratio is then designated as the key elements, calculated as:
\begin{align}
%\text{top}_k(D_1, D_2, \dots, D_n) = \{x_{D_1}, x_{D_2}, \dots, x_{D_k}\}
\text{top}_k(D_1, D_2, \dots, D_n) = \{x_{1}, x_{2}, \dots, x_{k}\} \label{eq:topk} 
\end{align}

%The process involves initially calculating the perplexity for the entire PII, can be calculated as:[equation!!!!] .

% Subsequently, we assess the change in perplexity ratio between consecutive tokens throughout the PII sequence, can be calculated as:[equation!!!!]. 

% The token exhibiting the largest change ratio is then designated as the key element, can be calculated as:[equation!!!!].

\textbf{Selective Forgetting.} Then, we maximize the following loss function, on the key element tokens \( x = (x_1, \dots, x_k) \) based on Equation~\ref{eq:topk}, which can be calculated as:
\begin{align}
\mathcal{L}_{UL}(F_\theta, x) = - \sum_{t=1}^k \log(p_\theta(x_t | x_{<t}))
\label{eq:selected_unlearning}
\end{align}
Here, \( x_{<t} \) represents the PII sequence of tokens \( x = (x_1, \ldots, x_{t-1}) \), and \( p_\theta(x_t | x_{<t}) \) is the conditional probability that the next token will be \( x_t \), given the preceding sequence \( x_{<t} \), in a language model \( F \) parameterized by \( \theta \).



\textbf{Memory Implanting.} After that, we apply the memory implanting, borrowed idea from error injection~\citep{de2021editing}, to compensate for the performance damage done by the selective forgetting is calculated as follows: 
\begin{align}
\arg \max_{M} p(y^* | x; F_\theta)
\end{align}
where $y^*$ represents the alternative, false target as proposed by~\citep{presidioResearch2024}.

\begin{comment}
\subsection{Theoretical Justification of Sensitivity Analysis} 
In this section, we uncover the relationship between our sensitivity-based selection and the second-order Newton's Method. We consider the following optimization problem that finds the maximum of the cross-entropy loss: 
\begin{align}
    \max_k L(k) = L_{\text{CE}}(p(\rvx_1,\cdots,\rvx_k),q(\rvx_1,\cdots,\rvx_k)).
\end{align}
Notice that 
\begin{align}
    L(k) = &-\sum_{\vx_1\cdots,\vx_k} p(\vx_1\cdots,\vx_k)\log q(\vx_1\cdots,\vx_k)\label{eq:accumulate_H}\\
    =&-\sum_{\vx_1\cdots,\vx_{k-1}} p(\vx_1\cdots,\vx_{k-1})\log q(\vx_1\cdots,\vx_{k-1})\nonumber\\
    &-\sum_{\vx_1\cdots,\vx_{k-1}} p(\vx_1\cdots,\vx_{k-1})\sum_{\vx_k}p(\vx_k|\vx_1\cdots,\vx_{k-1})\log q(\vx_k|\vx_1\cdots,\vx_{k-1})\\
    =&L(k-1)+H_k,
\end{align}
where $H_k$ is what we defined in Eq. (\ref{eq:cross_entropy_loss_ratio}). So we have 
\begin{align}
    &H_k=L(k)-L(k-1)\approx \nabla L(k),\\
    &H_{k+1}-H(k) \approx \nabla L(k+1)-\nabla L(k)\approx \nabla^2 L(k),\\
    &D_k = \frac{H_k-H_{k+1}}{H_k} \approx -\frac{\nabla^2 L(k)}{\nabla L(k)}.
\end{align}
Our selection method selects $k$ with the largest $D_k$. We discuss it in two situations:
\begin{enumerate}
    \item When there exists $k$ such that $H_k=\nabla L(k)=0$, we require that $\nabla^2L(k)<0$ to achieve the maximum ($D_k=+\infty$), this guarantees that $k$ achieves the maximum of $L(k)$ as well.
    \item When $H_k$ is always positive (notice that $H_k$ is never negative), $L(k)$ keeps growing as $k$ increases so we cannot find the maximum. But we still have
    \begin{align}
        \max_k D_k = \max_k \frac{1}{d_{\text{Newton}}(k)},
        \label{eq:newton_direction}
    \end{align}
    where $d_{\text{Newton}}(k)=-\nabla L(k)/\nabla^2L(k)$ is \textit{Newton's Direction} in the second-order Newton's Method. The maximization is achieved when $d_{\text{Newton}}(k)\rightarrow 0^+$, which implies that $k$ is close to the solution that maximizes $D(k)$. 
\end{enumerate}
\end{comment}


% Let Sequence be $x_n=\{a_1,a_2,\cdots,a_n\}$
% $$
% \max_k \frac{H(a_k|a_1,\cdots,a_{k-1})-H(a_{k+1}|a_1,\cdots,a_k)}{H(a_k|a_1,\cdots,a_{k-1})}\\
% =\max_k \frac{[H(a_1,\cdots,a_{k-1},a_k)-H(a_1,\cdots,a_{k-1})]-[H(a_1,\cdots,a_k,a_{k+1})-H(a_1,\cdots,a_{k-1},a_k)]}{H(a_1,\cdots,a_{k-1},a_k)-H(a_1,\cdots,a_{k-1})}
% $$
% Let $H(k)=H(a_1,\cdots,a_{k-1},a_k)$, we have
% $$
% \max_k \frac{\nabla H(k)-\nabla H(k+1)}{\nabla H(k)}=\max_k\frac{-\nabla^2H(k)}{\nabla H(k)}=\min_k -\frac{\nabla H(k)}{\nabla^2 H(k)}
% $$
% where $d(k)=-\frac{\nabla H(k)}{\nabla^2 H(k)}$ is the Newton's Direction: In a second order optimization algorithm (Pure Newton Method):
% $$
% \max_{k} H(k)\\
% k\leftarrow k+d(k)
% $$
% We search for the token with minimal step size for maximizing the entropy, which means that the current sequence has reached almost the maximimal entropy, and appending a new token does not introduce much information.

\vspace{-1em}
\section{Implementation}
\label{section:implementation}


\para{\docs \xspace :}
\label{sec:testbed:docs}
We have successfully built a test board featuring the OCS Controller chip and a pre-release Photonic Integrated Circuit (PIC) module without the MZI switch matrix, as shown in \fig{figure:design:evaluation-board}. The Controller Chip, measuring $4mm \times 4mm$, is manufactured using a 28nm process, while the PIC, sized at $10.5mm \times 13mm$, uses a 65nm CMOS process. The evaluation board supports 8 pairs of TX/RX SerDes at each end and has been validated for compatibility with various link layer protocols, including PCIe (32Gbps, 64Gbps) and Ethernet (56Gbps, 112Gbps). We assessed the power consumption of the peripheral circuitry using the test board. For an $8 \times 112G$ configuration, the power consumption was 8.5 watts. With the addition of 3.2 watts for the MZI switch matrix, the overall consumption totals approximately 12 watts, meeting the QSFP-DD 800Gbps standard\cite{qsfp-dd-15w}.

Notably, the full-featured version of the PIC chip has successfully completed tape-out and is currently in the packaging and testing phase. It will be available for evaluation prior to final publication.


\para{Small-scale Cluster:}
\label{sec:testbed:minipod}
We constructed a small-scale cluster to evaluate the communication performance of the ring topology. Using 32 experimental GPUs equipped with inter-host HBD support (96 lanes on PCIe 4 protocol), we formed a physical ring utilizing fixed optical modules. This mini-cluster was manually reconfigured for both 32-GPU and 16-GPU ring topology. The communication latency and AllReduce performance is evaluated.
For small packets, direct GPU-to-GPU links reduced latency by approximately 13\% compared to the NVLink switch design.
For large packets, the 16-GPU AllReduce utilized 77.11\% of the ring bandwidth, with the utilization rate increasing to 77.26\% for the 32-GPU configuration, showing minimal degradation with scaling. In comparison, the NVIDIA H100 8-GPU machine achieves an 81.77\% utilization rate without SHARP.
After deployment of \docs, the size of communication group can be reconfigured within $1ms$, while maintaining maximum throughput.


\begin{figure}[h!t]
    \vspace{-1em}
    \centering
    \includegraphics[width=0.48\textwidth]{figs/design/evaluation-board.pdf}
    \vspace{-20pt}
    \caption{Evaluation board for components of \docs.}
    \label{figure:design:evaluation-board}
    \vspace{-1em}
\end{figure}


% \vspace{-2ex}
\section{Large-Scale Simulation}
\label{sec:simulation}

We begin by outlining the experimental methodology and setup (\S\ref{sec:simulation:setup}). Next, we assess fault tolerance across different HBD architectures (\S\ref{sec:simulation:fault}), followed by end-to-end simulations to evaluate training performance under varying parallelism and GPU resource allocations (\S\ref{sec:simulation:end2end}). We then examine the improvements in communication efficiency achieved by our orchestration algorithm (\S\ref{sec:simulation:efficiency}). Finally, we present a comparative cost and power analysis of different HBD architectures (\S\ref{sec:simulation:cost-power}). The simulations demonstrate that \sys{} outperforms other architectures across all metrics. 

 

\vspace{-1ex}
\subsection{Methodology and Setup}
\label{sec:simulation:setup}


An in-house simulator dedicated for LLM training is used to evaluate \sys comprehensively. The simulator supports end-to-end simulations of both model training performance and hardware faults, with the HBD-DCN orchestration algorithm seamlessly integrated into the system.

\begin{figure}[h!t]
    \centering
    \begin{subfigure}[b]{0.23\textwidth}
        \centering
        \includegraphics[width=\textwidth]{figs/evaluation/fault_trace_based/cdf_trace_waste_tp16_gr4.pdf}
        \vspace{-1em}
        \caption{TP-16.}
        \label{fig:simulation:waste-cdf:tp16-gr8}
    \end{subfigure}
    \hspace{2pt}
    \begin{subfigure}[b]{0.23\textwidth}
        \centering
        \includegraphics[width=\textwidth]{figs/evaluation/fault_trace_based/cdf_trace_waste_tp32_gr4.pdf}
        \vspace{-1em}
        \caption{TP-32.}
        \label{fig:simulation:waste-cdf:tp32-gr8}
    \end{subfigure}
    \vspace{-2em}
    \caption{CDF of GPU waste ratio over 4-GPU node based on production fault trace.}
    \vspace{-1em}
    \label{fig:simulation:waste-cdf:gr4}
\end{figure}




\para{GPU and network specification.}
The NVIDIA H100~\cite{h100} (989 TFLOPS, 80GiB) is used for the configuration of GPU in simulation. And the HBD bandwidth of GPU is set as $6.4Tbps$, which is the sum of 8 QSFP-DD \ocstrx. The DCN bandwidth is configured the same as NVIDIA ConnectX-7 ($400Gbps$). 
Since the simulation primarily focuses on HBD, the DCN is configured as a Fat-Tree topology~\cite{sigcomm2008fattree}. Several HBD architectures are then evaluated, including:

\begin{itemize}[itemsep=2pt,topsep=0pt,parsep=0pt, leftmargin=2ex]
    \item \textbf{Big-Switch}: The ideal HBD design, featuring a large centralized switch with no forwarding latency that connects all nodes, as the theoretical upper limit of communication performance and fault resilience.
    \item \textbf{\sys{}}: Two configurations are evaluated: the \ocstrx{} bundle is set to either $K = 2$ or $K = 3$ (\S\ref{section:design:topology}), constructing 2/3-Hop Ring respectively.
    \item \textbf{NVL-36, NVL-72, NVL-576}~\cite{nvl72}: HBDs with 36, 72, or 576 GPUs, GPU are interconnected via NVLink Switches.
    \item \textbf{TPUv4}~\cite{isca2023tpu}: Centralized OCS capable of scheduling with a $4^3$ TPU cube granularity.
    \item \textbf{SiP-Ring}~\cite{sip-ml}: All nodes are connected in a series of static rings with fix sizes equal to the TP sizes.
\end{itemize}


\para{GPU count per node. }The simulation aligns with both  4-GPU node (.e.g. NVIDIA GB200 NVL-36/72/576~\cite{nvl72} and TPUv4~\cite{isca2023tpu}) and 8-GPU node design  (NVIDIA H100, AMD MI300X~\cite{amdmi300}, Intel Gaudi3~\cite{intelgaudi3}, and UBB 2.0 standard\cite{UBB2.0}). 


\begin{figure}[h!t]
\vspace{-1em}
    \centering
    \begin{subfigure}[b]{0.23\textwidth}
        \centering
        \includegraphics[width=\textwidth]{figs/evaluation/fault_model_based/frag_ratio_tp16_gr4.pdf}
        \caption{TP-16.}
        \label{fig:simulation:model:wasted-overview:tp16}
    \end{subfigure}
    \hspace{2pt}
    \begin{subfigure}[b]{0.23\textwidth}
        \centering
        \includegraphics[width=\textwidth]{figs/evaluation/fault_model_based/frag_ratio_tp32_gr4.pdf}
        \caption{TP-32.}
        \label{fig:simulation:model:wasted-overview:tp32}
    \end{subfigure}
    \vspace{-2em}
    
    \caption{GPU wastes ratio over the 4-GPU node with different GPU fault ratio based on fault model.}
    \vspace{-1em}
    \label{fig:simulation:model:wasted-overview}
\end{figure}


\para{Parallelism strategy. } 
Since \sys is primarily designed for TP, the key variable is the TP size. TP-8, TP-16, TP-32, and TP-64 are tested to evaluate the fault resilience of various HBD architectures (\S\ref{sec:simulation:fault}).
Additionally, other parallelism strategies, such as PP and DP, are used to simulate cross-ToR traffic and evaluate the orchestration algorithm (\S\ref{sec:simulation:efficiency}).


\para{Fault patterns. } The fault trace used in the simulation was collected from an 8-GPU node cluster with approximately 3,000 GPUs over a span of 160 days.
On average, the ratio of faulty 8-GPU nodes is $3.83\%$, with the P99 value as $7.22\%$, more details in Appendix~\S\ref{appendix:production-fault-trace}. In some simulations, fault traces generated based on this trace statistics are also derived. 

\subsection{HBD Fault Resilience}
\label{sec:simulation:fault}

This section evaluates the fault resilience of various HBD architectures, focusing on GPU waste ratio, job fault-waiting time, and the maximum job scale supported by the cluster. The main text presents the key results, with more detailed results provided in Appendix~\S\ref{appendix:wasted-GPUs-ratio}.



\para{GPU waste.} 
Apart from faulty GPUs, issues such as fragmentation, topology disconnections, and bandwidth degradation can render healthy GPUs wasted.
The GPU waste ratio quantifies the number of wasted GPUs under different fault scenarios. \figref{fig:simulation:waste-cdf:gr4} illustrates GPU waste ratios over production trace, while \figref{fig:simulation:model:wasted-overview} depicts the GPU waste ratio as node fault ratio vary.

\begin{table}[h!t] \footnotesize
    \vspace{-1ex}
    \centering
    \begin{tabular}{llllll}
    \toprule
    \textbf{GPU Num} & \textbf{TP} & \textbf{DP} & \textbf{PP} & \textbf{EP} & \textbf{MFU} \\
    \midrule
    1024    & 16       & 16      & 4       & 1       & 0.4276         \\
    2048    & 16      & 16      & 8        & 1       & 0.4140        \\
    4096    & 32      & 16      & 8        & 1       & 0.3894        \\
    8192    & 32      & 16      & 16      & 1       & 0.3656       \\
    16384  &  64     & 16       & 16      & 1      & 0.3116       \\
    \bottomrule
    \end{tabular}
    \caption{Optimal parallelism strategies for maximize MFU of GPT-MoE under varying GPU numbers.}
    \vspace{-3em}
    \label{tab:eval:gpt-moe-optimal}
\end{table}


In these scenarios, \sys{} ($K=3$) achieves near-zero GPU waste ratio, and outperforming all other architectures. Especially, the waste ratio for \sys ($K=2$) remains almost identical to \sys{} ($K=3$), allowing one bundle of \docs{} to be saved for clusters with low fault rates.   
NVL-36 and NVL-72 typically experience an 11\% waste ratio for TP sizes of 16 or larger, as $1/9$ of GPUs are reserved for redundant backups. NVL-576 has less fragmentation, benefiting from its larger size. TPUv4 performs well at low fault ratios and small TP sizes, but significantly degrades with larger TP sizes due to its coarse $4^3$ cube-based resource management, which amplifies the fault explosion radius. To sum up, \sys{} demonstrates the strongest fault resilience among all architectures.   



\para{Maximum job supported. } 
In fixed-size clusters, large job must pause when the available GPUs drop below the required count. Faced with same fault rate, cluster with lower GPU waste ratio can support larger job scales. \figref{fig:simulation:job_scale} shows the maximum job scale supported for various HBD architectures cluster with 2880-GPU, simulated with the fault traces normalized for 4-GPU nodes. \sys{} ($K=2$ or $K=3$) and NVL-576 lead in performance, and SiP-Ring exhibits declining efficiency as TP size increases.


\begin{figure}[h!t]
    \centering
    % \subfigure[No Fault-Waiting.]{
        \includegraphics[width=0.7\linewidth]{figs/evaluation/fault_trace_based/no_breakdown_maxjobscale_gr4.pdf}
    \vspace{-2ex}
    \caption{Maximal job scale supported by 2880 GPUs.}
    \label{fig:simulation:job_scale}
    \vspace{-1em}
\end{figure}


\para{Job fault-waiting time.} Large job must wait for the repairing when GPU availability falls below the required threshold. This simulations assume the average recovery time in the fault trace as a fixed repair duration. The total wasted time during 160 days is evaluated (\figref{fig:simulation:breakdown-duration}). For smaller TP sizes (TP-8/TP-16), NVL-36/NVL-72 exhibit the weakest resilience due to their 11\% backup overhead. For larger TP sizes (TP-32/TP-64), SiP-Ring and TPUv4 perform worst. 


\vspace{-1ex}
\subsection{Training Performance}
\label{sec:simulation:end2end}

This section analyzes the training performance of two representative large models, {LLama 3.1-405B}~\cite{llama3herdmodels} and {GPT-MoE} (configuration detailed in Appendix~\S\ref{appendix:gpt-moe}), under various GPU resource configurations and parallelism strategies. The simulation results validate the practical applicability of the \sys{} architecture. In simulations, we model practical TP and EP behaviors: For TP, increasing parallelism splits GEMMs into smaller, less efficient tasks, reducing hardware efficiency~\cite{gemm-eff}; for EP, we practically set expert imbalance coefficient at 20\%.


\begin{figure}[h!t]
    \vspace{-1em}
    \centering
    \begin{subfigure}[b]{0.23\textwidth}
        \centering
        \includegraphics[width=\textwidth]{figs/evaluation/fault_trace_based/breakdown_ratio_tp16_gr4.pdf}
        \vspace{-1em}
        \caption{TP-16.}
        \label{fig:simulation:breakdown-duration:tp16-8gpu}
    \end{subfigure}
    \hspace{2pt}
    \begin{subfigure}[b]{0.23\textwidth}
        \centering
        \includegraphics[width=\textwidth]{figs/evaluation/fault_trace_based/breakdown_ratio_tp32_gr4.pdf}
        \vspace{-1em}
        \caption{TP-32.}
        \label{fig:simulation:breakdown-duration:tp32-8gpu}
    \end{subfigure}
    \vspace{-2em}
    \caption{Job fault-waiting time over the 4-GPU node with different levels of job-scale.}
    \vspace{-1em}
    \label{fig:simulation:breakdown-duration}
\end{figure}

\para{LLama 3.1-405B\footnote{To support larger-scale TP parallelism, we simplified the GQA~\cite{GQA} architecture of LLama 3.1-405B to a traditional MHA architecture.}. }The model adopts a classical decoder-only Transformer architecture. The simulation employs the conventional 3D parallelism strategy\footnote{$TP \in \{1,2,4,8,...,128\}$, $DP \in \{1,2,4,8,...,1024\}$, $PP \in \{1,2,4,8,16\}$, $bsz=2048$}, which combines TP, DP, and PP for performance analysis. 
\tabref{tab:eval:llama3-optimal} presents the optimal parallelism strategies and their corresponding MFU for LLama 3.1-405B under varying GPU resources. As GPU resources increase, the optimal TP size also increases. When the number of GPUs exceeds 8192, the traditional 8-GPU HBD architecture within a single node begins to limit training efficiency. As the cluster size expands, larger TP sizes become increasingly optimal.





\para{GPT-MoE.} The model utilizes the Mixture-of-Experts (MoE) architecture, with $EP \in \{1,2,4,8\}$ introduced in the simulation. \tabref{tab:eval:gpt-moe-optimal} shows the optimal parallelism strategy and the corresponding MFU for GPT-MoE under various GPU resources. The optimal EP value is 1, suggesting that MoE can also achieve high efficiency with TP.


\vspace{-1ex}
\subsection{Communication Efficiency}
\label{sec:simulation:efficiency}

\begin{figure*}[!t]
    \centering
    \hfill{}
    \begin{subfigure}[b]{0.23\textwidth}
        \centering
        \includegraphics[width=\textwidth]{figs/evaluation/orch_unchange.pdf}
        \caption{Sensitivity to cluster size.}
        \label{fig:simulation:orch:cluster}
    \end{subfigure}
    \hfill{}
    \begin{subfigure}[b]{0.23\textwidth}
        \centering
        \includegraphics[width=\textwidth]{figs/evaluation/job_scale_orch.pdf}
        \caption{Impact of job-scale ratio.}
        \label{fig:simulation:orch:job}
    \end{subfigure}
    \hfill{}
    \begin{subfigure}[b]{0.23\textwidth}
        \centering
        \includegraphics[width=\linewidth]{figs/evaluation/ill_rate_orch.pdf}
        \caption{Sensitivity to fault ratio.}
        \label{fig:simulation:orch:fault}
    \end{subfigure}
    \hfill{}
    \begin{subfigure}[b]{0.23\textwidth}
        \centering
        \includegraphics[width=\linewidth]{figs/evaluation/cost/aggregate-cost.pdf}
        \caption{Aggregate cost.}
        \label{fig:eval:aggregate-cost}
    \end{subfigure}
    \vspace{-2ex}
    \caption{DCN traffic optimization analysis and aggregate normalized cost varies across different architectures under different fault ratios.}
    \label{fig:simulation:job_scale:orch}
    \vspace{-3ex}
\end{figure*}

This section examines the impact of orchestration algorithms on DCN communication efficiency. Experiments were performed on a Fat-Tree architecture, like the setup in ~\cite{sigcomm2024rdmameta}. As shown in \figref{fig:simulation:orch:cluster}, the algorithm is not sensitive to cluster size. Therefore, the evaluation is based on TP-32 operations on \sys{} with 8192 GPUs. 

\begin{itemize}[itemsep=2pt,topsep=0pt,parsep=0pt, leftmargin=2ex]
    \item \textbf{Baseline:} A greedy algorithms, which randomly select nodes from the cluster and use the first permutation that meets the requirements.
    \item \textbf{Optimized:} The HBD-DCN orchestration algorithm proposed in \secref{sec:design:orch}.
\end{itemize}  

\figref{fig:simulation:orch:job} illustrates the impact of job-scale ratios (job size/total cluster GPUs) on cross-ToR traffic, where node fault ratio is 5\%. Baseline consistently results in approximately 10\% cross-ToR traffic. In contrast, the Optimized algorithm significantly outperforms the Baseline, reducing cross-ToR traffic to just 1.72\% even at a 90\% job-scale ratio.
\figref{fig:simulation:orch:fault} explores the sensitivity to node faults, with the job scale ratio fixed at 85\%. The Baseline shows a linear increase of cross-ToR traffic, while the Optimized algorithm sustains near-zero cross-ToR traffic for fault ratios under 7\%. 


\vspace{-2ex}
\subsection{Cost and Power Analysis}
\label{sec:simulation:cost-power}



To evaluate the interconnect costs of HBD architectures, we gather the cost and power information with the following methodologies:

\begin{itemize}[itemsep=2pt,topsep=0pt,parsep=0pt, leftmargin=2ex]
    \item For standard components (DAC cables, optical transceivers, fibers), pricing is sourced from official retailer websites~\cite{FS_COM, FIBER_MALL, NADDOD} with a 60\% wholesale discount validated against internal data.
    \item For components with scarce public pricing information, such as Google Palomar OCS, NVIDIA NVLink Switch, 1.6 Tbps ACC cables/optical transceivers, the data is amalgamated from multiple sources~\cite{SEMIANALYSIS_GB200, SEMIANALYSIS_OCS, SEMIANALYSIS_Power} to enhance accuracy.
    \item Public power consumption data is available for most components, though for NVLink Switch, multiple sources are combined to estimate a reasonable value.
\end{itemize}


The breakdown analysis of each architecture is provided in the Appendix~\S\ref{appendix:cost}. Based on this, the cost and power consumption are normalized according to GPU count and per-GPU bandwidth. As depicted in \tabref{tab:eval:cost-power}, \sys{} exhibits the lowest interconnect cost per GPU per GBps. Under the $K =2$ configuration, its cost is only 62.84\% of Google TPUv4 and 30.86\% of the NVIDIA GB200 NVL-36/72, with minimal power consumption.
This efficiency is primarily attributed to the avoidance of centralized switches. TPUv4 ranks second in interconnect cost and lowest in power consumption, achieved by reducing optical module use and per-port OCS costs. The NVL series has higher interconnect costs and power consumption due to its fully-connected topology and high-cost NVLink Switches. Notably, NVL-576 incurs the highest cost and power consumption due to its multilayer nonconvergent topology, which increases optical module expenses and requires more NVLink Switches.

\begin{table}[h!t] \footnotesize
    \vspace{-3ex}
    \centering
    \begin{tabular}{lcccc}
    \toprule
    \textbf{Architecture}  & \multicolumn{2}{c}{\textbf{Per-GPU}}  & \multicolumn{2}{c}{\textbf{Per-GPU Per-GBps}} \\
 &  Cost & Watts & Cost & Watts \\
    \midrule
    TPUv4  & 1567.20  & 19.39 & 5.22& 0.06 \\
    NVL-36  & 9563.20  & 75.95 & 10.63& 0.08 \\
    NVL-72  & 9563.20  & 75.95 & 10.63 & 0.08 \\
    NVL-36x2  & 17924.00  & 150.33 & 19.92  & 0.17\\
    NVL-576   & 30417.60  & 413.45 & 33.80  & 0.46\\
    \midrule
    \SYS{} ($K=2$) &  2626.80 &  48.10 & 3.28  & 0.06\\\
    \SYS{} ($K=3$) &  3740.60 &  72.05  & 4.68  & 0.09\\
    \bottomrule
    \end{tabular}
    \caption{Interconnect cost (\$) and power (watts).}
    \label{tab:eval:cost-power}
    \vspace{-6ex}
\end{table}


Beyond interconnect costs, fault resilience variations also affect aggregate costs. The aggregate cost is defined as:

\vspace{-1em}
$$Cost_{GPU} \times (N_{Wasted-GPU} + N_{Faulty-GPU}) + Cost_{Interconnect}$$


Simulations on a 11,520-GPU cluster using the TP-32 configuration evaluate GPU availability under varying fault ratios across different architectures.
The variation in aggregate cost for different HBD architectures under varying node fault ratios is illustrated in \figref{fig:eval:aggregate-cost}. \sys{} consistently exhibits the lowest aggregate cost. Furthermore, when the fault ratio is below 12.1\%, the aggregate cost of \sys{} ($K=2$) is less than that of \sys{} ($K = 3$), suggesting that ($K = 2$) is the optimal design for most scenarios.








\section{Discussion}
The development of foundation models has increasingly relied on accessible data support to address complex tasks~\cite{zhang2024data}. Yet major challenges remain in collecting scalable clinical data in the healthcare system, such as data silos and privacy concerns. To overcome these challenges, MedForge integrates multi-center clinical knowledge sources into a cohesive medical foundation model via a collaborative scheme. MedForge offers a collaborative path to asynchronously integrate multi-center knowledge while maintaining strong flexibility for individual contributors.
This key design allows a cost-effective collaboration among clinical centers to build comprehensive medical models, enhancing private resource utilization across healthcare systems.

Inspired by collaborative open-source software development~\cite{raffel2023building, github}, our study allows individual clinical institutions to independently develop branch modules with their data locally. These branch modules are asynchronously integrated into a comprehensive model without the need to share original data, avoiding potential patient raw data leakage. Conceptually similar to the open-source collaborative system, iterative module merging development ensures the aggregation of model knowledge over time while incorporating diverse data insights from distributed institutions. In particular, this asynchronous scheme alleviates the demand for all users to synchronize module updates as required by conventional methods (e.g., LoRAHub~\cite{huang2023lorahub}).


MedForge's framework addresses multiple data challenges in the cycle of medical foundation model development, including data storage, transmission, and leakage. As the data collection process requires a large amount of distributed data, we show that dataset distillation contributes greatly to reducing data storage capacity. In MedForge, individual contributors can simply upload a lightweight version of the dataset to the central model developer. As a result, the remarkable reduction in data volume (e.g., 175 times less in LC25000) alleviates the burden of data transfer among multiple medical centers. For example, we distilled a 10,500 image training set into 60 representative distilled data while maintaining a strong model performance. We choose to use a lightweight dataset as a transformed representation of raw data to avoid the leakage of sensitive raw information.
Second, the asynchronous collaboration mode in MedForge allows flexible model merging, particularly for users from various local medical centers to participate in model knowledge integration. 
Third, MedForge reformulates the conventional top-down workflow of building foundational models by adopting a bottom-up approach. Instead of solely relying on upstream builders to predefine model functionalities, MedForge allows medical centers to actively contribute to model knowledge integration by providing plugin modules (i.e., LoRA) and distilled datasets. This approach supports flexible knowledge integration and allows models to be applicable to wide-ranging clinical tasks, addressing the key limitation of fixed functionalities in traditional workflows.

We demonstrate the strong capacity of MedForge via the asynchronous merging of three image classification tasks. MedForge offered an incremental merging strategy that is highly flexible compared to plain parameter average~\cite{wortsman2022model} and LoRAHub~\cite{huang2023lorahub}. Specifically, plain parameter averaging merges module parameters directly and ignores the contribution differences of each module. Although LoRAHub allows for flexible distribution of coefficients among modules, it lacks the ability to continuously update, limiting its capacity to incorporate new knowledge during the merging process. In contrast, MedForge shows its strong flexibility of continuous updates while considering the contribution differences among center modules. The robustness of MedForge has been demonstrated by shuffling merging order (Tab~\ref{tab:order}), which shows that merging new-coming modules will not hurt the model ability of previous tasks in various orders, mitigating the model catastrophic forgetting. 
MedForge also reveals a strong generality on various choices of component modules. Our experiments on dataset distillation settings (such as DC and without DSA technique) and PEFT techniques (such as DoRA) emphasize the extensible ability of MedForge's module settings. 

To fully exploit multi-scale clinical data, it will be necessary to include broader data modalities (e.g., electronic health records and radiological images). Managing these diverse data formats and standards among numerous contributors can be challenging due to the potential conflict between collaborators. 
Moreover, since MedForge integrates multiple clinical tasks that involve varying numbers of classification categories, conventional classifier heads with fixed class sizes are not applicable. However, the projection head of the CLIP model, designed to calculate similarities between image and text, is well-suited for this scenario. It allows MedForge to flexibly handle medical datasets with different category numbers, thus overcoming the challenge of multi-task classification. That said, this design choice also limits the variety of model architectures that can be utilized, as it depends specifically on the CLIP framework. Future investigations will explore extensive solutions to make the overall architecture more flexible. Additionally, incorporating more sophisticated data anonymization, such as synthetic data generation~\cite{ding2023large}, and encryption methods can also be considerable. To improve data privacy protection, test-time adaptation technique~\cite{wang2020tent, liang2024comprehensive} without substantial training data can be considered to alleviate the burden of data sharing in the healthcare system.



             

\vspace{-1em}
\section{Related Work}

\para{HBD Architectures.}  
HBDs are crucial for enabling communication intensive parallelism strategies (TP/EP) for LLM training. NVIDIA DGX SuperPOD~\cite{superpod} and GB200 NVL series~\cite{nvl72} use any-to-any electrical switching, delivering high performance but suffering from high costs, scalability limitations, and fragmentation. In contrast, direct interconnect HBDs like Dojo~\cite{dojo}, TPUv3~\cite{cacm2020tpuv3}, and SiP-Ring~\cite{sip-ml} improve scalability but have a large fault explosion radius. TPUv4~\cite{isca2023tpu} and TPUv5p~\cite{tpuv5} attempts a middle ground but still lacks full node-level fault isolation. \sys introduces a novel architecture that reduces cost, improves scalability, minimizes fragmentation, enhances fault isolation, and dynamically supports TP.  

\para{AI DCN Architectures.}  
MegaScale~\cite{megascale} and Meta’s~\cite{sigcomm2024rdmameta} AI DC use Clos-based topologies, while Rail-Optimized~\cite{rail-optimized} and Rail-Only~\cite{wang2024railonly} architectures optimize for LLM traffic patterns. Alibaba HPN~\cite{sigcomm2024hpn} enhances fault tolerance with a dual-plane design. \sys is compatible with all of them on LLM-training.  

\para{OCS Technologies.}  
OCS enables dynamic topology reconfiguration in datacenters~\cite{missionapollo, isca2023tpu, mfabric}. MEMS OCS-based switch supports high port counts~\cite{urata2022missionapollo, mems-320}, while silicon photonics (SiPh) achieves lower latency and cost~\cite{thermo-optic_2006}. This work proposes a SiPh-based OCS transceiver (\docs), constructing an interconnect fabric without centralized switches.  

\para{Reconfigurable Networks.}  
Traditional studies~\cite{helios,c-through,osa,mordia,sirius,xia2015enabling,megaswitch,rotornet,opera,firefly,shale} focus on generic DCN architectures without optimizing for LLM training traffic, leading to suboptimal topologies. Recent advancements like SiP-ML~\cite{sip-ml}, TopoOpt~\cite{topoopt2023}, and mFabric~\cite{mfabric} introduce dedicated training optimizations but still underutilize optical network reconfigurability for better fault tolerance and GPU utilization.  

\para{AI Job Schedulers.}  
Schedulers such as ~\cite{gandiva,themis,tiresias, {byteps_1}, {byteps_2}, pollux} aim to improve GPU utilization. However, they exhibit dual limitations: their designs are premised on non-reconfigurable network, while also failing to consider job scheduling within HBD for optimizing traffic patterns in DCN. This work proposes a HBD-DCN orchestration algorithm based on reconfigurable networks to address these limitations.

\section{Conclusion}
This paper presents {\tool}, a novel interactive text-to-SQL annotation system that enables users to create high-quality, schema-specific text-to-SQL datasets.
{\tool} integrates multiple functionalities, including schema customization, database synthesis, query alignment, dataset analysis, and additional features such as confidence scoring.
A user study with 12 participants demonstrates that by combining these features, {\tool} significantly reduces annotation time while improving the quality of the annotated data.
{\tool} effectively bridges the gap resulting from insufficient training and evaluation datasets for new or unexplored schemas.

\newpage
\bibliographystyle{ACM-Reference-Format}
\bibliography{paper}

\newpage
\section{Dataset Examples}
\label{app:dataset-eg}
Figure \ref{fig:dataset-eg} illustrates example data instances from MemeCap, NewYorker, and YesBut.

\begin{figure*}[t]
  \includegraphics[width=\linewidth]{figures/dataset-eg.pdf} \hfill
  \caption {Dataset Examples on MemeCap, NewYorker, and YesBut.}
  \label{fig:dataset-eg}
\end{figure*}


\section{SentenceSHAP}
\label{app:sentence-shap}
In this section, we introduce SentenceSHAP, an adaptation of TokenSHAP \cite{horovicz-goldshmidt-2024-tokenshap}. While TokenSHAP calculates the importance of individual tokens, SentenceSHAP estimates the importance of individual sentences in the input prompt. The importance score is calculated using Monte Carlo Shapley Estimation, following the same principles as TokenSHAP.

Given an input prompt \( X = \{x_1, x_2, \dots, x_n\} \), where \( x_i \) represents a sentence, we generate all possible combinations of \( X \) by excluding each sentence \( x_i \) (i.e., \( X - \{x_i\} \)). Let \( Z \) represent the set of all combinations where each \( x_i \) is removed. To estimate Shapley values efficiently, we randomly sample from \( Z \) with a specified sampling ratio, resulting in a subset \( Z_s = \{X_1, X_2, \dots, X_s\} \), where each \( X_i = X - \{x_i\} \).

Next, we generate a base response \( r_0 \) using a VLM (or LLM) with the original prompt \( X \), and a set of responses \( R_s = \{r_1, r_2, \dots, r_s\} \), each generated by a prompt from one of the sampled combinations in \( Z_s \).

We then compute the cosine similarity between the base response \( r_0 \) and each response in \( R_s \) using Sentence Transformer (\texttt{BAAI/bge-large-en-v1.5}). The average similarity between combinations with and without \( x_i \) is computed, and the difference between these averages gives the Shapley value for sentence \( x_i \). This is expressed as:
\begin{align}
\notag
\phi(x_i) = \\ \notag
&\frac{1}{s} \sum_{j=1}^{s} \left( \text{cos}(r_0, r_j \mid x_i) - \text{cos}(r_0, r_j \mid \neg x_i) \right)
\end{align}
where \( \phi(x_i) \) represents the Shapley value for sentence \( x_i \), $\text{cos}(r_0, r_j \mid x_i)$ is the cosine similarity between the base response and the response that includes sentence $x_i$, $\text{cos}(r_0, r_j \mid \neg x_i)$ is the cosine similarity between the base response and the response that excludes sentence $x_i$, and $s$ is the number of sampled combinations in $Z_s$.

\section{Error Analysis Based on SentenceSHAP}
Figure \ref{fig:error-analysis} presents two examples of negative impacts from implications: dilution of focus and the introduction of irrelevant information.
\label{app:error-analysis-shap}
\begin{figure*}[t]
  \includegraphics[width=\linewidth]{figures/error-analysis.pdf} \hfill
  \caption {Examples of negative impact from implications from Phi (top) and GPT4o (bottom).}
  \label{fig:error-analysis}
\end{figure*}

\section{Details on human anntations}
\label{app:cloudresearch}
We present the annotation interface on CloudResearch used for human evaluation to validate our evaluation metric in Figure \ref{fig:cloud-research}. Refer to Sec.~\ref{sec:ethics} for details on annotator selection criteria and compensation.

\begin{figure*}[t]
  \includegraphics[width=\linewidth]{figures/cloud-research.pdf} \hfill
  \caption {Annotation interface on CloudResearch used for human evaluation to validate our evaluation metric.}
  \label{fig:cloud-research}
\end{figure*}



\section{Generation Prompts for Selection and Refinement}
\label{app:gen-prompts}
Figures \ref{fig:desc-prompt}, \ref{fig:seed-imp-prompt}, and \ref{fig:nonseed-imp-prompt} show the prompts used for generating image descriptions, seed implications (1st hop), and non-seed implications (2nd hop onward). Figure \ref{fig:cand-prompt} displays the prompt used to generate candidate and final explanations. Image descriptions are used for candidate explanations when existing data is insufficient but are not used for final explanations. For calculating Cross Entropy values (used as a relevance term), we use the prompt in Figure \ref{fig:cand-prompt}, substituting the image with image descriptions, as LLM is used to calculate the cross entropies.

\begin{figure*}[h]
\small
\begin{tcolorbox}[
    title=Prompt for Image Descriptions,
    colback=white,
    colframe=CadetBlue,
    arc=0pt,        % Remove rounded corners
    outer arc=0pt   % Remove outer rounded corners (important for some styles)    
]

Describe the image by focusing on the noun phrases that highlight the actions, expressions, and interactions of the main visible objects, facial expressions, and people.\\
\\
Here are some guidelines when generating image descriptions:\\
* Provide specific and detailed references to the objects, their actions, and expressions. Avoid using pronouns in the description.\\
* Do not include trivial details such as artist signatures, autographs, copyright marks, or any unrelated background information.\\
* Focus only on elements that directly contribute to the meaning, context, or main action of the scene.\\
* If you are unsure about any object, action, or expression, do not make guesses or generate made-up elements.\\
* Write each sentence on a new line.\\
* Limit the description to a maximum of 5 sentences, with each focusing on a distinct and relevant aspect that directly contribute to the meaning, context, or main action of the scene.\\
\\
Here are some examples of desired output:
---\\
\text{[Description]} (example of newyorker cartoon image):\\
Through a window, two women with surprised expressions gaze at a snowman with human arms.\\
---\\
\text{[Description]} (example of newyorker cartoon image):\\
A man and a woman are in a room with a regular looking bookshelf and regular sized books on the wall.\\
In the middle of the room the man is pointing to text written on a giant open book which covers the entire floor.\\
He is talking while the woman with worried expression watches from the doorway.\\
---\\
\text{[Description]} (example of meme):\\
The left side shows a woman angrily pointing with a distressed expression, yelling ``You said memes would work!''.\\
The right side shows a white cat sitting at a table with a plate of food in front of it, looking indifferent or smug with the text above the cat reads, ``I said good memes would work''.\\
---\\
\text{[Description]} (example of yesbut image):\\
The left side shows a hand holding a blue plane ticket marked with a price of ``\$50'', featuring an airplane icon and a barcode, indicating it's a flight ticket.\\
The right side shows a hand holding a smartphone displaying a taxi app, showing a route map labeled ``Airport'' and a price of ``\$65''.\\
---\\

Proceed to generate the description.\\
\text{[Description]}:

\end{tcolorbox}
\caption{A prompt used to generate image descriptions.} % Add a caption to the figure
\label{fig:desc-prompt}
\end{figure*}


%%%%%%%%%%%%%%%%%%%%%%%%%%% Prompt for implications %%%%%%%%%%%%%%%%%%%%%%%%%%%
\begin{figure*}[t]
\small
\begin{tcolorbox}[
    title=Prompt for Seed Implications,
    colback=white,
    colframe=Green,
    arc=0pt,        % Remove rounded corners
    outer arc=0pt,  % Remove outer rounded corners (important for some styles)    
    % breakable,
]

You are provided with the following inputs:\\
- \text{[}Image\text{]}: An image (e.g. meme, new yorker cartoon, yes-but image)\\
- \text{[}Caption\text{]}: A caption written by a human.\\
- \text{[}Descriptions\text{]}: Literal descriptions that detail the image.\\
\\
\#\#\# Your Task:\\
\texttt{[ One-sentence description of the ultimate goal of your task. Customize based on the task. ]}\\
Infer implicit meanings, cultural references, commonsense knowledge, social norms, or contrasts that connect the caption to the described objects, concepts, situations, or facial expressions.\\
\\
\#\#\# Guidelines:\\
- If you are unsure about any details in the caption, description, or implication, refer to the original image for clarification.\\
- Identify connections between the objects, actions, or concepts described in the inputs.\\
- Explore possible interpretations, contrasts, or relationships that arise naturally from the scene, while staying grounded in the provided details.\\
- Avoid repeating or rephrasing existing implications. Ensure each new implication introduces fresh insights or perspectives.\\
- Each implication should be concise (one sentence) and avoid being overly generic or vague.\\
- Be specific in making connections, ensuring they align with the details provided in the caption and descriptions.\\
- Generate up to 3 meaningful implications.\\
\\
\#\#\# Example Outputs:\\
\#\#\#\# Example 1 (example of newyorker cartoon image):\\
\text{[}Caption\text{]}: ``This is the most advanced case of Surrealism I've seen.''\\
\text{[}Descriptions\text{]}: A body in three parts is on an exam table in a doctor's office with the body's arms crossed as though annoyed.\\
\text{[}Connections\text{]}:\\
1. The dismembered body is illogical and impossible, much like Surrealist art, which often explores the absurd.\\
2. The body’s angry posture adds a human emotion to an otherwise bizarre scenario, highlighting the strange contrast.\\
\\
\#\#\#\# Example 2 (example of newyorker cartoon image):\\
\text{[}Caption\text{]}: ``He has a summer job as a scarecrow.''\\
\text{[}Descriptions\text{]}: A snowman with human arms stands in a field.\\
\text{[}Connections\text{]}:\\
1. The snowman, an emblem of winter, represents something out of place in a summer setting, much like a scarecrow's seasonal function.\\
2. The human arms on the snowman suggest that the role of a scarecrow is being played by something unexpected and seasonal.\\
\\
\#\#\#\# Example 3 (example of yesbut image):\\
\text{[}Caption\text{]}: ``The left side shows a hand holding a blue plane ticket marked with a price of `\$50'.''\\
\text{[}Descriptions\text{]}: The screen on the right side shows a route map labeled ``Airport'' and a price of `\$65'.\\
\text{[}Connections\text{]}:\\
1. The discrepancy between the ticket price and the taxi fare highlights the often-overlooked costs of travel beyond just booking a flight.\\
2. The image shows the hidden costs of air travel, with the extra fare representing the added complexity of budgeting for transportation.\\
\\
\#\#\#\# Example 4 (example of meme):\\
\text{[}Caption\text{]}: ``You said memes would work!''\\
\text{[}Descriptions\text{]}: A cat smirks with the text ``I said good memes would work.''\\
\text{[}Connections\text{]}:\\
1. The woman's frustration reflects a common tendency to blame concepts (memes) instead of the quality of execution, as implied by the cat’s response.\\
2. The contrast between the angry human and the smug cat highlights how people often misinterpret success as simple, rather than a matter of quality.\\
\\
\#\#\# Now, proceed to generate output:\\
\text{[}Caption\text{]}: \texttt{[ Caption ]}\\
\\
\text{[}Descriptions\text{]}:\\
\texttt{[ Descriptions ]}\\
\\
\text{[}Connections\text{]}:

\end{tcolorbox}
\caption{A prompt used to generate seed implications.} % Add a caption to the figure
\label{fig:seed-imp-prompt}
\end{figure*}


%%%%%%%%%%%%%%%%%%%%%%%%%%% Prompt for nonseed implications %%%%%%%%%%%%%%%%%%%%%%%%%%%
\begin{figure*}[t]
\small
%  \begin{tcolorbox}[
%  width=\textwidth,
%  colback={white},
%  title={Title},
%  colbacktitle={DarkGreen},
%  coltitle=white,
%  colframe={DarkGreen},
%  breakable
% ]
 % \parskip=5pt

\begin{tcolorbox}[
    % breakable,
    title=Prompt for Non-Seed Implications (2nd hop onward),
    colback=white,
    colframe=Green,
    arc=0pt,        % Remove rounded corners
    outer arc=0pt,  % Remove outer rounded corners (important for some styles)    
    % breakable,
]

You are provided with the following inputs:\\
- \text{[}Image\text{]}: An image (e.g. meme, new yorker cartoon, yes-but image)\\
- \text{[}Caption\text{]}: A caption written by a human.\\
- \text{[}Descriptions\text{]}: Literal descriptions that detail the image.\\
- \text{[}Implication\text{]}: A previously generated implication that suggests a possible connection between the objects or concepts in the caption and description.\\
\\
\#\#\# Your Task:\\
\texttt{[ One-sentence description of the ultimate goal of your task. Customize based on the task. ]}\\
Infer implicit meanings across the objects, concepts, situations, or facial expressions found in the caption, description, and implication. Focus on identifying relevant commonsense knowledge, social norms, or underlying connections.\\
\\
\#\#\# Guidelines:\\
- If you are unsure about any details in the caption, description, or implication, refer to the original image for clarification.\\
- Identify potential connections between the objects, actions, or concepts described in the inputs.\\
- Explore interpretations, contrasts, or relationships that naturally arise from the scene while remaining grounded in the inputs.\\
- Avoid repeating or rephrasing existing implications. Ensure each new implication provides fresh insights or perspectives.\\
- Each implication should be concise (one sentence) and avoid overly generic or vague statements.\\
- Be specific in the connections you make, ensuring they align closely with the details provided.\\
- Generate up to 3 meaningful implications that expand on the implicit meaning of the scene.\\
\\
\#\#\# Example Outputs:\\
\#\#\#\# Example 1 (example of newyorker cartoon image):\\
\text{[}Caption\text{]}: "This is the most advanced case of Surrealism I've seen."\\
\text{[}Descriptions\text{]}: A body in three parts is on an exam table in a doctor's office with the body's arms crossed as though annoyed.\\
\text{[}Implication\text{]}: Surrealism is an art style that emphasizes strange, impossible, or unsettling scenes.\\
\text{[}Connections\text{]}:\\
1. A body in three parts creates an unsettling juxtaposition with the clinical setting, which aligns with Surrealist themes.\\
2. The body’s crossed arms add humor by assigning human emotion to an impossible scenario, reflecting Surrealist absurdity.\\
... \\
\texttt{[ We used sample examples from the prompt for generating seed implications (see Figure \ref{fig:seed-imp-prompt}), following the above format, which includes [Implication]:. ]}
\\
---\\
\\
\#\#\# Proceed to Generate Output:\\
\text{[}Caption\text{]}: \texttt{[ Caption ]}\\
\\
\text{[}Descriptions\text{]}:\\
\texttt{[ Descriptions ]}\\
\\
\text{[}Implication\text{]}:\\
\texttt{[ Implication ]}\\
\\
\text{[}Connections\text{]}:
\end{tcolorbox}
\caption{A prompt used to generate non-seed implications.} % Add a caption to the figure
\label{fig:nonseed-imp-prompt}
\end{figure*}


%%%%%%%%%%%%%%%%%%%%%%%%%%% Prompt for nonseed implications %%%%%%%%%%%%%%%%%%%%%%%%%%%
\begin{figure*}[t]
\small
%  \begin{tcolorbox}[
%  width=\textwidth,
%  colback={white},
%  title={Title},
%  colbacktitle={DarkGreen},
%  coltitle=white,
%  colframe={DarkGreen},
%  breakable
% ]
 % \parskip=5pt

\begin{tcolorbox}[
    % breakable,
    title=Prompt for Candidate and Final Explanations,
    colback=white,
    colframe=RedViolet,
    arc=0pt,        % Remove rounded corners
    outer arc=0pt,  % Remove outer rounded corners (important for some styles)    
    % breakable,
]

You are provided with the following inputs:\\
- **\text{[}Image\text{]}:** A New Yorker cartoon image.\\
- **\text{[}Caption\text{]}:** A caption written by a human to accompany the image.\\
- **\text{[}Image Descriptions\text{]}:** Literal descriptions of the visual elements in the image.\\
- **\text{[}Implications\text{]}:** Possible connections or relationships between objects, concepts, or the caption and the image.\\
- **\text{[}Candidate Answers\text{]}:** Example answers generated in a previous step to provide guidance and context.\\
\\
\#\#\# Your Task:\\
Generate **one concise, specific explanation** that clearly captures why the caption is funny in the context of the image. Your explanation must provide detailed justification and address how the humor arises from the interplay of the caption, image, and associated norms or expectations.\\
\\
\#\#\# Guidelines for Generating Your Explanation:\\
1. **Clarity and Specificity:**  \\
   - Avoid generic or ambiguous phrases.  \\
   - Provide specific details that connect the roles, contexts, or expectations associated with the elements in the image and its caption.  \\
\\
2. **Explain the Humor:**  \\
- Clearly connect the humor to the caption, image, and any cultural, social, or situational norms being subverted or referenced.  \\
- Highlight why the combination of these elements creates an unexpected or amusing contrast.\\
\\
3. **Prioritize Clarity Over Brevity:**  \\
- Justify the humor by explaining all important components clearly and in detail.  \\
- Aim to keep your response concise and under 150 words while ensuring no critical details are omitted.  \\
\\
4. **Use Additional Inputs Effectively:**\\
- **\text{[}Image Descriptions\text{]}:** Provide a foundation for understanding the visual elements."   \\
- **\text{[}Implications\text{]}:** Assist in understanding relationships and connections but do not allow them to dominate or significantly alter the central idea.\\
- **\text{[}Candidate Answers\text{]}:** Adapt your reasoning by leveraging strengths or improving upon weaknesses in the candidate answers.\\
\\
Now, proceed to generate your response based on the provided inputs.\\
\\
\#\#\# Inputs:\\
\text{[}Caption\text{]}: \texttt{\text{[} Caption \text{]}}\\
\\
\text{[}Descriptions\text{]}:\\
\texttt{\text{[} Top-K Implications \text{]}}\\
\\
\text{[}Implications\text{]}:\\
\texttt{\text{[} Top-K Implications \text{]}}\\
\\
\text{[}Candidate Anwers\text{]}:\\
\texttt{\text{[} Top-K Candidate Explanations \text{]}}\\
\\
\text{[}Output\text{]}:\\

\end{tcolorbox}
\caption{A prompt used to generate candidate and final explanations.} % Add a caption to the figure
\label{fig:cand-prompt}
\end{figure*}


\section{Evaluation Prompts}
\label{app:eval-prompts}
Figures \ref{fig:recall-prompt} and \ref{fig:precision-prompt} present the prompts used to calculate recall and precision scores in our LLM-based evaluation, respectively.

%%%%%%%%%%%%%%%%%%%%%%%%%%% Prompt for nonseed implications %%%%%%%%%%%%%%%%%%%%%%%%%%%
\begin{figure*}[t]
\small
\begin{tcolorbox}[
    % breakable,
    title=Prompt for Evaluating Recall Score,
    colback=white,
    colframe=MidnightBlue,
    arc=0pt,        % Remove rounded corners
    outer arc=0pt,  % Remove outer rounded corners (important for some styles)    
    % breakable,
]

Your task is to assess whether \text{[}Sentence1\text{]} is conveyed in \text{[}Sentence2\text{]}. \text{[}Sentence2\text{]} may consist of multiple sentences.\\
\\
Here are the evaluation guidelines:\\
1. Mark 'Yes' if \text{[}Sentence1\text{]} is conveyed in \text{[}Sentence2\text{]}.\\
2. Mark 'No' if \text{[}Sentence2\text{]} does not convey the information in \text{[}Sentence1\text{]}.\\
\\
Proceed to evaluate. \\
\\
\text{[}Sentence1\text{]}: \texttt{[ One Atomic Sentence from Decomposed Reference Explanation ]} \\
\\
\text{[}Sentence2\text{]}: \texttt{[ Predicted Explanation ]}\\
\\
\text{[}Output\text{]}:

\end{tcolorbox}
\caption{Prompt for evaluating recall score.} % Add a caption to the figure
\label{fig:recall-prompt}
\end{figure*}


\begin{figure*}[t]
\small
\begin{tcolorbox}[
    % breakable,
    title=Prompt for Evaluating Precision Score,
    colback=white,
    colframe=MidnightBlue,
    arc=0pt,        % Remove rounded corners
    outer arc=0pt,  % Remove outer rounded corners (important for some styles)    
    % breakable,
]

Your task is to assess whether \text{[}Sentence1\text{]} is inferable from \text{[}Sentence2\text{]}. \text{[}Sentence2\text{]} may consist of multiple sentences.\\
\\
Here are the evaluation guidelines:\\
1. Mark "Yes" if \text{[}Sentence1\text{]} can be inferred from \text{[}Sentence2\text{]} — whether explicitly stated, implicitly conveyed, reworded, or serving as supporting information.\\
2. Mark 'No' if \text{[}Sentence1\text{]} is absent from \text{[}Sentence2\text{]}, cannot be inferred, or contradicts it.\\
\\
Proceed to evaluate. \\
\\
\text{[}Sentence1\text{]}: \texttt{[ One Atomic Sentence from Decomposed Predicted Explanation ]}\\
\\
\text{[}Sentence2\text{]}: \texttt{[ Reference Explanation ]}\\
\\
\text{[}Output\text{]}:


\end{tcolorbox}
\caption{Prompt for evaluating precision score.} % Add a caption to the figure
\label{fig:precision-prompt}
\end{figure*}

\section{Prompts for Baselines}
\label{app:base-prompts}

Figure \ref{fig:base-prompt} presents the prompt used for the ZS, CoT, and SR Generator methods. While the format remains largely the same, we adjust it based on the baseline being tested (e.g., CoT requires generating intermediate reasoning, so we add extra instructions for that).
Figure \ref{fig:critic-prompt} shows the prompt used in the SR critic model. The critic's criteria include: (1) \textit{correctness}, measuring whether the explanation directly addresses why the caption is humorous in relation to the image and its caption; (2) \textit{soundness}, evaluating whether the explanation provides a well-reasoned interpretation of the humor; (3) \textit{completeness}, ensuring all important aspects in the caption and image contributing to the humor are considered; (4) \textit{faithfulness}, verifying that the explanation is factually consistency with the image and caption; and (5) \textit{clarity}, ensuring the explanation is clear, concise, and free from unnecessary ambiguity.
\begin{figure*}
\small
\begin{tcolorbox}[
    % breakable,
    title=Prompt for Baselines,
    colback=white,
    colframe=Black,
    arc=0pt,        % Remove rounded corners
    outer arc=0pt,  % Remove outer rounded corners (important for some styles)    
    % breakable,
]

You are provided with the following inputs:\\
- **\text{[}Image\text{]}:** A New Yorker cartoon image.\\
- **\text{[}Caption\text{]}:** A caption written by a human to accompany the image.\\
\texttt{[ if Self-Refine with Critic is True: ]} \\
- **\text{[}Feedback for Candidate Answer\text{]}:** Feedback that points out some weakness in the current candidate responses.\\
\texttt{[ if Self-Refine is True: ]} \\
- **\text{[}Candidate Answers\text{]}:** Example answers generated in a previous step to provide guidance and context.\\
\\
\#\#\# Your Task:\\
Generate **one concise, specific explanation** that clearly captures why the caption is funny in the context of the image. Your explanation must provide detailed justification and address how the humor arises from the interplay of the caption, image, and associated norms or expectations.\\
\\
\#\#\# Guidelines for Generating Your Explanation:\\
1. **Clarity and Specificity:**  \\
   - Avoid generic or ambiguous phrases.  \\
   - Provide specific details that connect the roles, contexts, or expectations associated with the elements in the image and its caption.  \\
\\
2. **Explain the Humor:**  \\
- Clearly connect the humor to the caption, image, and any cultural, social, or situational norms being subverted or referenced.  \\
- Highlight why the combination of these elements creates an unexpected or amusing contrast.\\
\\
3. **Prioritize Clarity Over Brevity:**  \\
- Justify the humor by explaining all important components clearly and in detail.  \\
- Aim to keep your response concise and under 150 words while ensuring no critical details are omitted.  \\
\\
\texttt{[ if Self-Refine is True: ]}\\
4. **Use Additional Inputs Effectively:**\\
- **[Candidate Answers]:** Adapt your reasoning by leveraging strengths or improving upon weaknesses in candidate answers. \\
\texttt{[ if Self-Refine with Critic is True: ]}\\
- **[Feedback for Candidate Answer]:** Feedback that points out some weaknesses in the current candidate responses.\\
\\
\texttt{ [ if CoT is True: ]} \\
Begin by analyzing the image and the given context, and explain your reasoning briefly before generating your final response. \\
\\
Here is an example format of the output: \\
\{\{ \\
    "Reasoning": "...", \\
    "Explanation": "..."   \\
\}\} \\

Now, proceed to generate your response based on the provided inputs.\\
\\
\#\#\# Inputs:\\
\text{[}Caption\text{]}: \texttt{\text{[} Caption \text{]}}\\
\\
\text{[}Candidate Answers\text{]}: \texttt{\text{[} Candidate Explanations \text{]}}\\
\\
\text{[}[Feedback for Candidate Answer]:\text{]}: \texttt{\text{[} Feedback for Candidate Explanations \text{]}}\\
\\
\text{[}Output\text{]}:\\

\end{tcolorbox}
\caption{A prompt used for baseline methods, with conditions added based on the specific baseline being experimented with.} % Add a caption to the figure
\label{fig:base-prompt}
\end{figure*}


\begin{figure*}
\small
\begin{tcolorbox}[
    % breakable,
    title=Prompt for Self-Refine Critic,
    colback=white,
    colframe=Black,
    arc=0pt,        % Remove rounded corners
    outer arc=0pt,  % Remove outer rounded corners (important for some styles)    
    % breakable,
]
\texttt{[ Customize goal text here: ]} \\
\texttt{MemeCap:} You will be given a meme along with its caption, and a candidate response that describes what meme poster is trying to convey. \\
\texttt{NewYorker:} You will be given an image along with its caption, and a candidate response that explains why the caption is funny for the given image. \\
\texttt{YesBut:} You will be given an image and a candidate response that describes why the image is funny or satirical. \\
\\
Your task is to criticize the candidate response based on the following evaluation criteria: \\
- Correctness: Does the explanation directly address why the caption is funny, considering both the image and its caption? \\
- Soundness: Does the explanation provide a meaningful and well-reasoned interpretation of the humor? \\
- Completeness: Does the explanation address all relevant aspects of the caption and image (e.g., visual details, text) that contribute to the humor? \\
- Faithfulness: Is the explanation factually consistent with the details in the image and caption? \\
- Clarity: Is the explanation clear, concise, and free from unnecessary ambiguity? \\
 \\
Proceed to criticize the candidate response ideally using less than 5 sentences:\\
\\
\text{[}Caption\text{]}: \texttt{[ caption ]}\\
\\
\text{[}Candidate Response\text{]}: \\
 \texttt{\text{[} Candidate Response \text{]}}\\
\\
\text{[}Output\text{]}: \\
\end{tcolorbox}
\caption{A prompt used in SR critic model.} % Add a caption to the figure
\label{fig:critic-prompt}
\end{figure*}

% \begin{figure*}[t]
%   \includegraphics[width=\linewidth]{figures/error-analysis.pdf} \hfill
%   \vspace{-20pt}
%   \caption {Examples of negative impact from implications from Phi (top) and GPT4o (bottom).}
%   \label{fig:error-analysis}
% \end{figure*}

%%%%%%%%%%%%%%%%%%%%%%%%%%%%%%%%%%%%%%%%%%%%%%%%%%%%%%%%%%%%%%%%%%%%%%%%%%%%%%%%
\end{document}
%%%%%%%%%%%%%%%%%%%%%%%%%%%%%%%%%%%%%%%%%%%%%%%%%%%%%%%%%%%%%%%%%%%%%%%%%%%%%%%%

%%  LocalWords:  endnotes includegraphics fread ptr nobj noindent
%%  LocalWords:  pdflatex acks