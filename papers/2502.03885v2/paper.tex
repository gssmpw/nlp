\documentclass[sigconf]{acmart}

\usepackage[english]{babel}
\usepackage{graphicx}
\usepackage{multirow}
\usepackage{array}
\usepackage{tabularx}
\usepackage{pifont}
\newcommand{\CG}{\mathcal{G}\xspace}
\newcommand{\CV}{\mathcal{V}\xspace}
\newcommand{\CE}{\mathcal{E}\xspace}
\newcommand{\CA}{\mathcal{A}\xspace}
\newcommand{\CF}{\mathcal{F}\xspace}
\newcommand{\CR}{\mathcal{R}\xspace}
\newcommand{\CB}{\mathcal{B}\xspace}
\newcommand{\CX}{\mathcal{X}\xspace}
\newcommand{\CK}{\mathcal{K}\xspace}
\newcommand{\CM}{\mathcal{M}\xspace}
\newcommand{\CC}{\mathcal{C}\xspace}
\newcommand{\CL}{\mathcal{L}\xspace}
\newcommand{\CI}{\mathcal{I}\xspace}
\newcommand{\CQ}{\mathcal{Q}\xspace}
\newcommand{\CO}{\mathcal{O}\xspace}
\newcommand{\CP}{\mathcal{P}\xspace}
\newcommand{\CS}{\mathcal{S}\xspace}
\newcommand{\CT}{\mathcal{T}\xspace}
\newcommand{\CJ}{\mathcal{J}\xspace}
\usepackage[para]{footmisc}
\usepackage{subfig}
% \usepackage{subcaption}
% \usepackage{array}
% \usepackage{colortbl}



%% Configuration
\newcommand{\COMMENTS}{yes}


%% Cross reference
\newcommand{\secref}[1]{\S\ref{#1}}
\newcommand{\figref}[1]{Figure~\ref{#1}}
\newcommand{\tabref}[1]{Table~\ref{#1}}
\newcommand{\eqnref}[1]{Equation~\ref{#1}}
\newcommand{\algref}[1]{Algorithm~\ref{#1}}
\newcommand{\fig}[1]{Figure~\ref{#1}}
%% Table & Cell
\newcommand{\specialcell}[2][c]{\begin{tabular}[#1]{@{}c@{}}#2\end{tabular}}
\newcommand{\tabincell}[2]{\begin{tabular}{@{}#1@{}}#2\end{tabular}}


% \usepackage{algorithm,algorithmic}
% \renewcommand{\algorithmicrequire}{\textbf{Input:}}
% \renewcommand{\algorithmicensure}{\textbf{Output:}}

%% Url

\def\UrlBreaks{\do\A\do\B\do\C\do\D\do\E\do\F\do\G\do\H\do\I\do\J\do\K\do\L\do\M\do\N\do\O\do\P\do\Q\do\R\do\S\do\T\do\U\do\V\do\W\do\X\do\Y\do\Z\do\[\do\\\do\]\do\^\do\_\do\`\do\a\do\b\do\c\do\d\do\e\do\f\do\g\do\h\do\i\do\j\do\k\do\l\do\m\do\n\do\o\do\p\do\q\do\r\do\s\do\t\do\u\do\v\do\w\do\x\do\y\do\z\do\0\do\1\do\2\do\3\do\4\do\5\do\6\do\7\do\8\do\9\do\.\do\@\do\\\do\/\do\!\do\_\do\|\do\;\do\>\do\]\do\)\do\,\do\?\do\'\do+\do\=\do\#}

%% Caption with tigher before and after vertical space
\newcommand{\figcaption}[1]{\vspace{-2mm}\caption{#1}\vspace{-4mm}} 
\newcommand{\mfigcaption}[1]{\vspace{-2mm}\caption{#1}\vspace{-2mm}} 
\newcommand{\tabcaption}[1]{\vspace{-2mm}\caption{#1}\vspace{-4mm}}
\newcommand{\mtabcaption}[1]{\vspace{1mm}\caption{#1}\vspace{-4mm}}

%% Math symbol optimization
\newcommand{\superscript}[1]{\ensuremath{^{\textrm{#1}}}}
\newcommand{\argmax}{\operatornamewithlimits{argmax}}
\def\deg{{\,^{\circ}}\xspace}

%% Abbreviation optimization
\def\It{\textit}
\def\Bf{\textbf}
\def\eg{\textit{e.g.,}\hspace{1mm}}
\def\ie{\textit{i.e.,}\hspace{1mm}}
\def\etal{\textit{et al.}\hspace{1mm}}
\def\etc{\textit{etc.}\hspace{1mm}}

%% font wrapper

\newcommand{\code}[1]{\mbox{\texttt{#1}}}
\newcommand{\Mod}[1]{\mbox{\textsf{#1}}} % \textup \textsl \texttt \textsf \textrm
\newcommand{\sw}[1]{\mbox{\textsc{#1}}}

%% Compact environments
\newenvironment{Itemize}{
  \begin{list}{$\bullet$} {
      \setlength{\itemsep}{0pt}
      \setlength{\parsep}{2pt}
      \setlength{\topsep}{2pt}
      \setlength{\partopsep}{0pt}
      \setlength{\leftmargin}{1.5em} % 1.5
      \setlength{\labelwidth}{1em} % 1
      \setlength{\labelsep}{0.5em} % 0.5
  }}
  {\end{list}}  

\newenvironment{Enumerate}{
  \begin{enumerate}[leftmargin=2em]
    \setlength{\itemsep}{2pt}
    \setlength{\topsep}{2pt}
    \setlength{\partopsep}{0pt}
    \setlength{\parskip}{0pt}}
  {\end{enumerate}}

\newenvironment{Circled}{
  \begin{enumerate}[label=\protect\circled{\arabic*},leftmargin=2em]
    \setlength{\itemsep}{3pt}
    \setlength{\topsep}{0pt}
    \setlength{\partopsep}{0pt}
    \setlength{\parskip}{0pt}}
  {\end{enumerate}}

% \usepackage{ulem}

% \newcommand{\lyz}[1]{{\color{blue}\textit{Yunzhuo:#1}}}
% \newcommand{\caihui}[1]{{\color{orange}\textit{Caihui:#1}}}
% \newcommand{\bj}[1]{{\color{red}(BJ:#1)}}


%% Paper revising
\ifthenelse{\equal{\COMMENTS}{yes}}{
  %% Writing Mode 
  \newcommand{\todo}[1]{\textcolor{red}{\textbf{TODO:} #1}}
  \newcommand{\neil}[1]{\textcolor{blue}{\textbf{Nie:} #1}} %content will be included
  \newcommand{\grace}[1]{\textcolor{purple}{(grace: #1)}}
  \newcommand{\fye}[1]{\textcolor{red}{#1}}  %content will be excluded
  \newcommand{\remind}[1]{\footnote{\textit{\textcolor{red}{\textbf{Remind:} #1}}}}
  \newcommand{\repl}[2]{\textcolor{red}{#1}\textcolor{blue}{\sout{#2}}} % replacement
  \newcommand{\add}[1]{\textcolor{red}{#1}}
  \newcommand{\del}[1]{\color{blue} {\sout{#1}}}
  %\newcommand{\p}[1]{\noindent\parbox{\columnwidth}{\textcolor{magenta}{\textbf{Point to make:} #1}}\vskip 0.5ex}
  \newcommand{\p}[1]{\vskip 1ex \noindent\colorbox{yellow}{\parbox{\columnwidth}{#1}}\vskip 4pt}
  \newcommand{\note}[1]{\vskip 4ex \noindent\colorbox{yellow}{\parbox{\columnwidth}{#1}}\vskip 6ex} % highlight
  \newcommand{\dc}[1]{\textcolor{red}{\underline{#1}}} % double check % \uwave
  \newcommand{\q}[1]{\vskip 1ex \noindent\colorbox{magenta}{\parbox{\columnwidth}{\textbf{Question:} #1}}\vskip 4pt} 
  \newcommand{\qa}[1]{\hl{\textbf{Answer:} #1}}
  \newcommand{\rc}[1]{\textcolor{red}{(RC: #1)}}
}{
  %%Submission Mode
  \newcommand{\todo}[1]{}
  \newcommand{\fyi}[1]{#1}
  \newcommand{\fye}[1]{}
  \newcommand{\remind}[1]{}
  \newcommand{\repl}[2]{#1}
  \newcommand{\add}[1]{#1}
  \newcommand{\del}[1]{}
  \newcommand{\p}[1]{}
  \newcommand{\note}[1]{}
  \newcommand{\dc}[1]{#1}  
  \newcommand{\qm}[1]{#1}
  \newcommand{\q}[1]{}
  \newcommand{\qa}[1]{}  
  \newcommand{\grace}[1]{}
  \newcommand{\neil}[1]{}
}

% Circled Numbers
% \usepackage{tikz}
% \newcommand*\circled[1]{\tikz[baseline=(char.base)]{
%     \node[shape=circle,draw,inner sep=0.5pt] (char) {#1};}
% }
% \newcommand{\todo}[1]{\textcolor{red}{\textbf{TODO:} #1}}
\newcommand{\circled}[1]{\raisebox{.5pt}{\textcircled{\raisebox{-.9pt} {#1}}}}
\newcommand{\para}[1]{\noindent\textbf{#1}}

%% packeditemize
\newenvironment{packeditemize}{\begin{list}{$\bullet$}{\setlength{\itemsep}{2pt}\addtolength{\labelwidth}{-6pt}\setlength{\leftmargin}{12pt}\setlength{\listparindent}{\parindent}\setlength{\parsep}{1pt}\setlength{\topsep}{2pt}}}{\end{list}}


\renewcommand\footnotetextcopyrightpermission[1]{} % removes footnote with conference info
\setcopyright{none}
\settopmatter{printacmref=false, printccs=false, printfolios=true}

% DOI
\acmDOI{}

% ISBN
\acmISBN{}

% Conference
\acmConference[Submitted for review to SIGCOMM]{}
\acmYear{2025}
\copyrightyear{}

%% {} with no args suppresses printing of the price
\acmPrice{}
%-------------------------------------------------------------------------------

%-------------------------------------------------------------------------------

%don't want date printed
% \date{}

\newcommand{\SYS}{\textit{\textbf{InfinitePOD}}\xspace}
\newcommand{\sys}{\textit{InfinitePOD}\xspace}
\newcommand{\ocstrx}{\textit{OCSTrx}}
\newcommand{\docs}{\ocstrx}

\fancyhead[LE]{}%
\fancyhead[RO]{}%
\fancyhead[RE]{}%
\fancyhead[LO]{}%
\fancyfoot[C]{\footnotesize\thepage}

\begin{document}

\title{\SYS: Building Datacenter-Scale High-Bandwidth Domain for LLM with Optical Circuit Switching Transceivers}

\renewcommand{\shorttitle}{\sys}

\author{Chenchen Shou$^{1,2,3}$ \hspace{0.5em} Guyue Liu$^{1,\dag}$ \hspace{0.5em} Hao Nie$^{2,\dag}$ \hspace{0.5em} Huaiyu Meng$^{3,\dag}$  \hspace{0.5em} Yu Zhou$^2$ \hspace{0.5em} \\ Yimin Jiang$^4$ \hspace{0.5em}  Wenqing Lv$^3$ \hspace{0.5em} Yelong Xu$^3$ \hspace{0.5em} Yuanwei Lu$^2$ \hspace{0.5em} Zhang Chen$^3$ \hspace{0.5em} \\ Yanbo Yu$^2$ \hspace{0.5em} Yichen Shen$^3$ \hspace{0.5em} Yibo Zhu$^2$ \hspace{0.5em} Daxin Jiang$^2$
}

\affiliation{
$^1$Peking University \hspace{0.5em}
$^2$StepFun \hspace{0.5em}
$^3$Lightelligence Pte. Ltd. \hspace{0.5em}
$^4$Unaffiliated \hspace{0.5em}
}

% \affiliation{Peking University, StepFun, Lightelligence Pte. Ltd.}

% \renewcommand{\shortauthors}{Anonymous Authors}



%-------------------------------------------------------------------------------
\begin{abstract}
%-------------------------------------------------------------------------------


Scaling Large Language Model (LLM) training relies on multi-dimensional parallelism, where High-Bandwidth Domains (HBDs) are critical for communication-intensive parallelism like Tensor Parallelism (TP) and Expert Parallelism (EP). However, existing HBD architectures face fundamental limitations in scalability, cost, and fault resiliency: switch-centric HBDs (e.g., NVL-72) incur prohibitive scaling costs, while GPU-centric HBDs (e.g., TPUv3/Dojo) suffer from severe fault propagation. Switch-GPU hybrid HBDs such as TPUv4 takes a middle-ground approach by leveraging Optical Circuit Switches, but the fault explosion radius remains large at the cube level (e.g., 64 TPUs).

We propose \sys{}, a novel transceiver-centric HBD architecture that \textit{unifies connectivity and dynamic switching at the transceiver level} using Optical Circuit Switching (OCS). By embedding OCS within each transceiver, \sys{} achieves reconfigurable point-to-multipoint connectivity, allowing the topology to adapt into variable-size rings. This design provides: i) datacenter-wide scalability without cost explosion; ii) fault resilience by isolating failures to a single node, and iii) full bandwidth utilization for fault-free GPUs. Key innovations include a Silicon Photonic (SiPh) based low-cost OCS transceiver (\textbf{\ocstrx}), a reconfigurable k-hop ring topology co-designed with intra-/inter-node communication, and an HBD-DCN orchestration algorithm maximizing GPU utilization while minimizing cross-ToR datacenter network traffic.
The evaluation demonstrates that \sys{} achieves \textbf{31\%} of the cost of NVL-72, \textbf{near-zero} GPU waste ratio (over one order of magnitude lower than NVL-72 and TPUv4), \textbf{near-zero} cross-ToR traffic when node fault ratios under 7\%, and improves Model FLOPs Utilization by \textbf{3.37x} compared to NVIDIA DGX (8 GPUs per Node).
\vspace{-1pt}

\end{abstract}

\maketitle


\footnotetext[1]{Guyue Liu, Hao Nie, and Huaiyu Meng contributed equally to this work and share the corresponding authorship.}


\vspace{-1pt}
\section{Introduction}
\label{sec:intro}

Foundational models (FMs)~\cite{zhang2024data, zhou2023comprehensive} have shown remarkable progress in the healthcare domain, enabling professional-like assessment of disease diagnosis, treatment decision-making, and monitoring~\cite{zhang2023text, wang2022medclip, lu2023mi-zero}. 
Examples include LLaVA-Med~\cite{li2023llava}, Med-PaLM Multimodal~\cite{tu2024towards}, and Med-Flamingo~\cite{moor2023med}, have demonstrated their capacity on question answering, medical image analysis, and report generation.
These studies follow a predominant top-down model development strategy that requires upstream developers to collect data and train models for downstream tasks. 
Consequently, the developed model capabilities are heavily dependent on the training data, limiting their generalization performance in diverse clinical scenarios. 
For instance, Med-Gemini~\cite{yang2024advancing} reveals promising general capabilities in report generation while it lags behind state-of-the-art (SoTA) models on classification tasks, especially for out-of-domain applications. 
This indicates that while the generalizability of the foundation model is promising, more solutions are expected to meet the various specialized clinical needs.

To address these challenges, multi-center data centralization becomes essential to enhance model capacity and robustness across varied clinical scenarios~\cite{rajpurkar2022ai}. 
Centralizing distributed data can significantly improve model training and inference performance.
However, the process of medical data storage, transfer, and aggregation among centers requires extra efforts to ensure data security and system interoperability~\cite{bradford2020international}.
Moreover, a growing concern for patient privacy makes large-scale multi-center data sharing particularly challenging. 
While efforts like federated learning~\cite{wen2023survey, li2020review} can achieve good model performance on local data, the need for synchronized system coordination presents significant challenges, as clients are unable to update asynchronously. This limitation greatly restricts the practical capability of such approaches.
As a result, without a flexible collaboration, medical community still struggles to fully utilize the isolated data and local computation resources for comprehensive medical AI model development. 
To address this dilemma, open-source platforms encourage public data sharing and knowledge integration~\cite{markiewicz2021openneuro, zenodo}.
However, these platforms focus solely on raw data sharing while seldom providing collaborative model training or cooperation between different institutions.
Recently, collaborative learning has emerged as a viable approach for enhancing multi-model robustness~\cite{boulemtafes2020review}. 
For instance, software-like model development~\cite{raffel2023building} mimics software engineering practices by introducing structured workflows, enabling merging, version control, and continuous model integration.
Under this design, model ability can be strengthened with incremental knowledge updates similar to the version updating in software development. 

Although collaborative learning provides a multi-model collaboration, two key challenges remain in the leakage of raw data during collaboration~\cite{huang2023lorahub} and the synchronization of multiple collaborators~\cite{mcmahan2017communication} in the medical AI community. It is still challenging to integrate decentralized, privacy-sensitive data across institutions, leading to under-utilized insights and fragmented knowledge sharing~\cite{kaissis2020secure, rajpurkar2022ai, abdullah2021ethics}.
 To address these challenges, inspired by the collaborative software development, we propose \textbf{Med}ical \textbf{Fo}undation Models Me\textbf{rg}ing (\textbf{MedForge}), a cooperative workflow enabling continuously community-driven foundation model (FM) development.
MedForge enables a lightweight manner for individual centers to share their knowledge among multiple centers, minimizing the burden of data transmission and integration while enhancing model robustness.
Meanwhile, MedForge facilitates asynchronous and flexible collaboration, allowing individual centers to continuously update and improve medical FMs without the need for real-time synchronization.
Similar to open-source software development, MedForge incrementally updates medical knowledge and follows a sustainable model development scheme. 
This key design emphasizes a bottom-up construction of a multi-task medical FM, allowing downstream users to collaboratively build, refine, and update the upstream model according to their local resources. Our major contributions of MedForge are as below: 
\begin{enumerate}
    \item[$\bullet$] We introduce a collaborative workflow to promote the merging scheme of open-source software development. Our proposed MedForge allows distributed clinical centers to asynchronously contribute to comprehensive medical model construction while reducing transmitting costs among centers and avoiding the leakage of raw data, thus enhancing the utilization of private resources in the healthcare system. 
    \item[$\bullet$] We propose two effective knowledge-merging strategies for the asynchronous branch contribution. The MedForge-Fusion strategy updates the plugin module parameters of the main model during the merging phase, whereas the MedForge-Mixture strategy integrates the output of the plugin module by memorizing each contributor's coefficient. These strategies make MedForge more flexible and versatile. MedForge-Fusion is friendly to implement, while the MedForge-Mixture offers better performance and robustness.
    \item[$\bullet$]  We comprehensively evaluate model merging strategies to accumulate medical knowledge among multiple branch plugin modules. MedForge yields superior performance on medical classification tasks compared to other collaborative baselines across multiple datasets. We demonstrate the robustness of MedForge by shuffling the task order and evaluating various configurations of plugin modules and dataset distillation methods.
\end{enumerate}



% \vspace{-1em}
\section{Background and Motivation}
\label{sec:background}
In this section, we first introduces LLM training in AI datacenters (DCs) (\S\ref{sec:background:llm_training}). Then, we examine existing High-Bandwidth Domain (HBD) architectures and discuss their limitations (\S\ref{sec:background:hbd}). Finally, we summarize key design principles of HBD for LLM training (\S\ref{sec:background:workload}).

% \vspace{-1em}
\subsection{LLM Training in AI DC}
\label{sec:background:llm_training}

\begin{figure*}[!t]
\centering
\begin{subfigure}[b]{0.25\textwidth}
    \centering
    \includegraphics[height=16ex]{figs/motivation/nvl36.drawio.pdf}
    \caption{Switch-centric: NVL36}
    \label{fig:hbd-archs:nvl36}
\end{subfigure}
\hspace{-1ex}\hfil\hspace{-1ex}
\begin{subfigure}[b]{0.24\textwidth}
    \centering
    \includegraphics[height=16ex]{figs/motivation/sip-ring.drawio.pdf}
    \caption{GPU-centric: SiP-Ring}
    \label{fig:hbd-archs:sip-ring}
\end{subfigure}
\hspace{-1ex}\hfil\hspace{-1ex}
\begin{subfigure}[b]{0.24\textwidth}
    \centering
    \includegraphics[height=16ex]{figs/motivation/dojo.drawio.pdf}
    \caption{GPU-centric: Dojo}
    \label{fig:hbd-archs:dojo}
\end{subfigure}
\hspace{-1ex}\hfil\hspace{-1ex}
\begin{subfigure}[b]{0.25\textwidth}
    \centering
    \includegraphics[height=16ex]{figs/motivation/tpuv4.drawio.pdf}
    \caption{Hybrid: TPUv4}
    \label{fig:hbd-archs:tpuv4}
\end{subfigure}
\vspace{-2ex}
\caption{Illustrative examples of HBD architectures. N represents Node, and S represents Switch. Red (with cross hatch) represents fault device and yellow (with dots) represents unavailable or downgraded GPU.}
\label{fig:hbd-archs}
\vspace{-1ex}
\end{figure*}

\begin{table*}[!htbp]\scriptsize
\centering
\begin{tabular}{llllllll}
\toprule
\multirow{2}{*}{\textbf{Architecture}}  & \multirow{2}{*}{\textbf{Type}}  & \multirow{2}{*}{\textbf{Scalability}} & \multirow{2}{*}{\begin{tabular}[c]{@{}l@{}}\textbf{Collective} \\ \textbf{Primitives}\end{tabular}} & \multicolumn{2}{l}{\textbf{Fault Explosion Radius}} & \multirow{2}{*}{\begin{tabular}[c]{@{}l@{}}\textbf{Interconnect} \\ \textbf{Cost}\end{tabular}} & \multirow{2}{*}{\textbf{Fragmentation}} \\
                              &                                                &                              &                                                                                   & \textbf{Node-Side}           & \textbf{Switch-Side}          &                                                                               &                                \\
\midrule
NVL                           & Switch-centric                       & Low                          & Full CCL                                                                          & Node-level          & Switch-level         & High                                                                          & Many                           \\
\makecell{Dojo, TPUv3, SiP-Ring}         & GPU-centric                       & High                         & Ring-Allreduce                                                                    & HBD-level           & \ding{55}            & Low                                                                           & Few                            \\
TPUv4, TPUv5p                         & Switch-GPU Hybrid                   & Moderate                     & Ring-Allreduce                                                                    & Cube-level          & Switch-level         & Moderate                                                                      & Few                            \\
\sys{}                   & Transceiver-centric & High                         & Ring-Allreduce                                                                    & Node-level          & \ding{55}            & Low                                                                           & Few  \\
\bottomrule
\end{tabular}
\caption{Comparative analysis of HBD architectures.}
\label{tab:hbd-compare}
\vspace{-6ex}
\end{table*}

\para{LLM training parallelism and communication.} LLM training jobs employ various parallelism strategies to efficiently utilize GPUs distributed across AI DCs~\cite{megatron-lm, zero}. Based on communication loads, parallelism can be categorized into two types. The first type is \textit{communication-intensive  parallelism} which involves high communication load. Tensor Parallelism (TP) splits the model across multiple GPUs and synchronizes via AllReduce. The ring algorithm for AllReduce is theoretically optimal~\cite{patarasuk2009bandwidth}, making ring-based topologies ideal for TP. Expert Parallelism (EP), designed for Mixture of Experts (MoE) models~\cite{hunyuanlarge,deepseekv3,mixtralexperts}, assigns experts to different GPUs and relies on AlltoAll communication, requiring topologies with high bisection bandwidth (e.g., Full-Mesh). In contrast, parallelism strategies such as Data Parallelism (DP), Pipeline Parallelism (PP), Context Parallelism (CP), and Sequence Parallelism (SP) introduce lower communication overhead, placing less  demands on network performance.



\para{Compute fabric. } Compute fabric in AI DC interconnects GPUs to efficiently transmit model gradients and parameters. It consists of two primary components: Datacenter Network (DCN) and High-Bandwidth Domain (HBD). 
DCN provides communication across the entire AI DC via Ethernet or Infiniband, the bandwidth is around $200\sim 800Gbps$. Widely used DCN architectures include Fat-Tree~\cite{sigcomm2008fattree} and Rail-Optimized~\cite{rail-optimized}. In comparison, HBD offers Tbps-level throughput, and is more suitable for TP/EP. However, its scale is typically constrained by interconnection costs and fault tolerance considerations. For example, NVL-72~\cite{nvl72} only interconnects 72 GPUs per HBD.



\para{Faults and fault explosion radius. }As revealed by current advances of AI DCs~\cite{sigcomm2024hpn, sigcomm2024rdmameta}, training jobs experience a variety of faults, such as GPU faults, optical transceiver faults, switch faults, and link faults. We quantify the fault impact using the \textit{fault explosion radius}, defined as \textit{the number of GPUs degraded by a single fault event}.
The fault explosion radius varies depending on both the system architecture and the fault component.
For example, if a switch fails, the bandwidth of all devices connected to it will degrade, illustrating the switch-level fault explosion radius.


\para{HBD fragmentation.} When the number of GPUs in the HBD cannot be evenly divided by the size of the parallel group (i.e., TP size), the remaining GPUs become unusable, leading to resource waste.
The GPU waste ratio for each HBD can be expressed by the formula $\{(HBD_{size} - N_{fault}) \mod TP_{size}\}/{HBD_{size}}$.
In AI DCs with small-scale HBDs, GPU waste due to fragmentation is significant because each HBD experiences independent fragmentation.
This issue worsens as the TP group size increases with model scale. For example, for NVL-36 shown in \figref{fig:hbd-archs:nvl36}, running TP-16 causes $\geq$11\% GPU waste ratio.



\subsection{Limitations of Existing HBDs} 
\label{sec:background:hbd}


Existing HBD architectures for LLM training can be categorized into three types, based on the key components that provide connectivity. A summary is shown in Table~\ref{tab:hbd-compare}.

\para{Switch-centric HBD.}
This type architecture leverages switch chips to interconnect GPUs, as shown in \figref{fig:hbd-archs:nvl36}.
A prominent example is NVIDIA, which utilizes NVLink and NVLink Switch ~\cite{nvlink,nvswitch}, e.g. DGX H100~\cite{dgx} with 8-GPU and GB200 NVL-36, NVL-72, and NVL-576~\cite{nvl72}. 
These architectures offer high-performance any-to-any communication.
However, switch-centric HBDs have several drawbacks: i) They require a large number of switch chips due to their limited per-chip throughput; ii) They are vulnerable to a switch-level fault explosion radius—when a switch chip fails, all connected nodes experience bandwidth degradation; iii) High interconnect costs constrain the scale of HBDs, leading to significant fragmentation when serving large models.

\para{GPU-centric HBD.}
GPU-centric HBD architectures construct the HBD using direct GPU-to-GPU connections, eliminating the need for switch chips. As a result, cost scales linearly with HBD size.
A representative example is SiP-Ring~\cite{sip-ml}, shown in \figref{fig:hbd-archs:sip-ring}, where GPUs are organized into fixed-size rings. However, this design imposes a strict limitation: the TP group size must remain fixed. 
To enable communication at dynamic scales and support a wider range of workloads, more complex topologies are adopted (e.g., Dojo~\cite{dojo}, NVIDIA V100~\cite{v100},  TPUv3~\cite{cacm2020tpuv3}, and AWS Trainium ~\cite{aws-trainium} ), which support dynamic scaling by allowing jobs to execute on topology subsets of varying sizes. As shown in \figref{fig:hbd-archs:dojo}, Dojo~\cite{dojo} connects GPUs via mesh-like topologies and employ GPUs to forward traffic. While GPU-centric architectures mitigate cost explosion and can support various scales, they suffer from a large fault explosion radius. A single GPU failure can disrupt the entire HBD by altering its connectivity, degrading communication performance even for healthy GPUs—such as the yellow GPUs in \figref{fig:hbd-archs:dojo}.

\para{Switch-GPU Hybrid HBD.}
This architecture interconnects GPUs via a combination of direct GPU-to-GPU connections and switch links. A typical example is TPUv4~\cite{isca2023tpu}, which organizes TPUs into $4^3$ TPU cubes and connect them via centralized OCS-based switches (\figref{fig:hbd-archs:tpuv4}). TPUv4 scales up to 4,096 TPUs, with its expansion primarily limited by the port count of the OCS-based switch. Furthermore, it suffers from a cube-level fault explosion radius—a failure in any single TPU affects the entire 64-TPU cube, leading to significant performance degradation. Furthermore, OCS-based switches face challenges of high costs and manufacturing complexity, which undermines the cost-effectiveness of TPUv4. TPUv5p cluster~\cite{tpuv5} is similar to TPUv4 but can scale out to 8,960 TPUs.

\begin{figure*}[!tp]
    \centering
    \includegraphics[width=\linewidth]{figs/overview.drawio.pdf}
    \vspace{-5ex}
    \caption{\sys{} overview.}
    \label{fig:overview}
    \vspace{-2ex}
\end{figure*}

% \vspace{-2ex}
\subsection{Key Attributes of An Ideal HBD}
\label{sec:background:workload}


\vspace{-1em}
\begin{table}[h!t] \small
    \centering
    \begin{tabular}{cllllll}
    \toprule
    \textbf{GPU} & \textbf{TP} & \textbf{PP} & \textbf{DP} & \textbf{MFU} & \textbf{$\textbf{MFU}_{TP-8}$} & \textbf{Improve}\\
    \midrule
    1024    & 16 & 4  & 16  & 0.5236 & 0.5217   & 1.0036      \\
    4096    & 16 & 8  & 32  & 0.4668 & 0.4282   & 1.0901      \\
    8192    & 32 & 8  & 32  & 0.4247 & 0.3512   & 1.2093      \\
    16384   & 32 & 16 & 32  & 0.3756 & 0.2584   & 1.4536      \\
    32768   & 32 & 16 & 64  & 0.3090 & 0.1690   & 1.8284      \\
    65536   & 64 & 16 & 64  & 0.2493 & 0.0999   & 2.4955      \\
    131072  & 64 & 16 & 128 & 0.1851 & 0.0550   & 3.3655      \\
    \bottomrule
    \end{tabular}
    \caption{Optimal parallelism strategy for maximum MFU of Llama 3.1-405b, compared to the baseline MFU for TP-8 (e.g., in widely-deployed NVLink architectures), when GPU number varies.}
    \label{tab:eval:llama3-optimal}
    \vspace{-2em}
\end{table}

Existing HBD architectures face fundamental limitations in interconnection cost, resource utilization, and failure resiliency when scaling. To guide a better design, we analyze existing training workloads and explore two key questions without the limitations imposed by current HBD: i) What is the optimal group size that HBD should support? ii) What traffic patterns should HBD accommodate?


\para{Large and adaptable TP size is critical for dense models.}
The optimal LLM training parallelism depends on model architectures and cluster configurations. For example, as illustrated by previous work~\cite{disttrain, nsdi2025_rlhfuse}. 
We evaluate the Model FLOPs Utilization (MFU) for Llama 3.1-405B~\cite{llama3.1-405b} using our in-house LLM training simulator (\S\ref{sec:simulation:end2end}) and report the results in Table~\ref{tab:eval:llama3-optimal}. MFU and TP/PP/DP columns denote the optimal MFU when TP size is unconstrained and the corresponding parallelism strategies respectively. $MFU_{TP-8}$ column denotes the optimal MFU when TP size is limited to 8. As we increase the number of GPUs, the optimal TP size grows from 16 to 64, a trend we observe across other large dense models. 
In this case, the HBD scale restricts the maximum size of TP, which affects training performance as a result. 

\begin{table}[h!t] \small
\vspace{-2ex}
\centering
\begin{tabular}{cccc}
\toprule
\multicolumn{2}{c}{\textbf{Parallelism}} & \textbf{Operation}  & \textbf{Traffic Load}  \\
\midrule

\multicolumn{2}{c}{TP}            & AllReduce     &$2bsh\cdot\frac{n-1}{n}$ \\ 
\multicolumn{2}{c}{EP}            & AllToAll     &$2bsh\cdot\frac{n-1}{n}\cdot\frac{k}{n}$\\
\bottomrule
\end{tabular}
\caption{Communication load of TP and EP on a single MoE layer. $b$: batch size; $s$: sequence length; $h$: hidden dim; $k$: topK of MoE router; $n$: parallel size. Assume each expert is assigned equal number of tokens.}
\label{tab:workload}
\vspace{-5ex}
\end{table}

\para{MoE can also be efficient with large-size TP.}
Beyond widely used dense models, we also examine sparse MoE models, which are trending toward larger scales (e.g., 1T parameters~\cite{switch_transformer}). The distributed training for MoE can be achieved through TP or EP (or a combination of them)\footnote{For TP, each expert is equally sharded to GPUs. For EP, each expert is indivisible and allocated to one GPU in the EP group.}~\cite{sigcomm2023_janus}, both TP and EP are communication-intensive~\cite{atc2023_lina}, making them heavily reliant on HBD. 







\begin{table}[h!t]
    \vspace{-1ex}
    \centering
    \begin{tabular}{cccccc}
        \toprule
        & TP & \multicolumn{4}{c}{EP} \\
        \hline
        imbalance coef & - & 0\% & 10\% & 20\% & 30\% \\
        % \hline
        MFU (\%) & 31.2 & 31.5 & 30.5 & 29.8 & 28.8 \\
        \bottomrule
    \end{tabular}
    \caption{Performance comparison of TP and EP when training GPT-MoE.}
    \label{tab:ep-imbalance}
    \vspace{-2em}
\end{table}


Our production training experience on a 1T MoE model in production brings the following insights into the pros and cons of TP and EP.
On the one hand, EP is more communication-efficient than TP. Table~\ref{tab:workload} compares the communication volume of TP and EP. Clearly, EP is better if $k<n$, which is common~\cite{deepseek_v3} because existing models often choose small $k$ for higher computation sparsity.
On the other hand, EP suffers from the well-known expert imbalance problem~\cite{sigcomm2023_janus}, especially when the MoE routers use the no-token-left-behind algorithm~\cite{deepseek_v3, megablocks, glam}. This will result in non-equivalent number of tokens that each expert will receive, which hence causes straggler nodes that waste GPU cycles of other nodes. 
\tabref{tab:ep-imbalance} shows the simulated result of training GPT-MoE with 1.1T parameters (details in Appendix~\S\ref{appendix:gpt-moe}) under different expert imbalance coefficients\footnote{Calculated as $\frac{max - min}{max}$, where $max$ and $min$ represent the maximum and minimum tokens allocated to each expert respectively.}. When $coef=0$, EP is better than TP due to smaller communication overhead. As $coef$ increases, the MFU drops because of the straggler issue.


\para{Key findings}. These experiments provide us two key findings for HBD design. First, larger HBD size is increasingly needed for rapidly scaling LLMs (i.e., more than 1T parameters). Second, with larger HBD enabled, using TP is more favorable than EP to train MoE, because TP shards the computation equally across GPUs and hence bypasses the expert imbalance problem. 

These findings reveal two key design principles for HBD:  i) HBD must inherently support large and adaptable TP sizes, which fundamentally requires the scalability of HBD architecture; ii) the HBD designs need to ensure the effective support for the Ring-AllReduce communication. Given the demonstrated efficiency of TP in MoE training, ensuring support for Ring-AllReduce support is sufficient for mainstream LLM training scenarios; iii) small fault explosion radius. Thus, \textit{\textbf{we propose designing a large and adaptable HBD architecture tailored for ring-based TP communication to optimize LLM parallelism strategies.}}




\goodname~enhances safety of LLM inputs and outputs while improving their quality. Specifically, it achieves two goals, 1) all user inputs are safe, contextually grounded, and effectively processed, such that the inputs to the LLMs are of high-quality and informative; and 2) the output generated by the LLMs are evaluated and enhanced, such that the outputs passed to users can be both relevant and of high quality. 
The pipeline can be partitioned into two parts, including 
1) processing before LLM inference that enhances user queries, and 2) processing after LLM inference that detects undesired content and handle them properly. We overview our pipeline in Figure~\ref{fig: system_overview}.


\noindent\underline{\textit{Pre-inference processing. }}
Before sending user queries to LLMs, \goodname~detects if there are any safety issues in the queries with \detection~and ground the queries with context knowledge with \grounding. 
\detection~monitors user inputs to identify and reject queries that might be unsafe. The monitoring includes typical safety checks, including toxicity, stereotypes, threats, obscenities, prompt injection attacks, etc. Any form of unsafe content will lead to the queries being rejected. 
Inputs that pass this initial safety check are grounded with context with \grounding, where the user query is contextualized and enhanced with relevant knowledge retrieved from the vector data storage. By equipping the query with some context knowledge, the LLM can do inference with enriched information, thus can reduce hallucinations when generating responses. The details of \detection~ and \grounding~will be introduced in \S\ref{sec:safety_detector} and \S\ref{sec:grounding}, respectively.




\noindent\underline{\textit{Post-inference processing. }}
Upon LLM finishing inference, \detection~detects safety issues in the LLM outputs, specifically, hallucinations. This is because LLM applications typically leverages well-developed LLMs or APIs, such as LLaMA~\citep{touvron2023llama} and ChatGPT API~\citep{openai-data-paper}, which are generally safe and less likely to generate toxic or other unsafe content, while hallucinations occur frequently. \detection~identifies hallucinations and provides reasons for the hallucinations, such that \goodname~can utilize the reasoning for later refinement of the LLM outputs. To achieve goal, \goodname~employs a text generation model to generate explainable results, and adjusts the loss function during training to ensure the model to produce classification results. 
After \detection~finishes detection, \fixing~fixes the problematic content or aligns the outputs with some rule-based wrappers to meet user expectations. 
If the outputs are difficult to fix, e.g., hallucinated responses, 
\fixing~will call a fixing model to fix the answers. Details about \fixing~can be found in \S\ref{sec:fixing}.

\section{ORCA Design}
\label{sec:system-design}

%%%%%%%%%%%%%%%%%%%%%%%%%%%%%%%%%%%%%%%%%%%%%
\begin{figure*}[tp]
    \centering
    \includegraphics[width=\textwidth]{figures/overview.png}
    \vspace{-0.5cm}
    \caption{\shepherd{ORCA cloud-assisted design overview.}}
    \vspace{-0.2cm}
    \label{fig:system-overview}
\end{figure*}
%%%%%%%%%%%%%%%%%%%%%%%%%%%%%%%%%%%%%%%%%%%%%

\subsection{Overview}
\label{sec:system-overview}
Based on the observations and discussions in Section~\ref{sec:background-and-related-works} and~\ref{sec:preliminary-study}, we argue that an ideal edge-cloud collaborative learning system over LPWANs should have the following design considerations. First, to tackle the unreliability of wireless channels, a cloud-assisted strategy should be adopted rather than the state-of-the-art cloud-dependent offloading. Second, to adapt to the low bit rates of LPWANs and the on-device resource constraints, we demand a more efficient information exchange strategy. Additionally, from an audio processing perspective, we look for a more effective feature selection method to reduce input size and therefore reduce on-device computation overheads while maintaining comparable accuracy performances. Informed by these demands, we introduce our novel design of a resource-aware cloud-assisted environmental sounds recognition system, primarily operating over LoRa networks. Our system features resource-aware and communication-adaptive cloud assistance, enabling efficient and flexible cloud offloading under resource constraints and unreliable communications. Furthermore, we apply a novel self-attention-based frequency band feature selection method with the wavelet transform to effectively select important features for efficient on-device inference. We illustrate the workflow of the ORCA cloud-assisted framework in Figure~\ref{fig:system-overview}:

\noindent
\shepherd{\textbf{Step~\textcircled{\small{1}}:} Initially, the edge device preprocesses audio signals using low-level WPT to generate a low-resolution spectrogram. Preprocessing details are in Section~\ref{sec:preprocess}, and optimized resolution selection based on wireless channel feedback, e.g., Adaptive Data Rate (ADR),  is discussed in Section~\ref{sec:resource-aware-cloud-assistance}.}

\noindent
\shepherd{\textbf{Step~\textcircled{\small{2}}:} The resulting low-resolution spectrogram is transmitted to the server via uplink LoRa channel, using ADR-recommended parameters.}

\noindent
\shepherd{\textbf{Step~\textcircled{\small{3}}:} Upon receiving the low-resolution spectrogram, the server processes it using a pre-trained contrastive vision transformer~\cite{dosovitskiy2020vit} to extract an attention mask through attention rollout~\cite{abnar2020quantifying}. Details of the cloud model are provided in Section~\ref{sec:attention-mask-generation}.}

\noindent
\shepherd{\textbf{Step~\textcircled{\small{4}}:} The extracted attention mask, along with ADR feedback, is sent back to the edge device via downlink. Resource efficiency adaptations using ADR feedback are further discussed in Section~\ref{sec:resource-aware-cloud-assistance}.}

\noindent
\shepherd{\textbf{Step~\textcircled{\small{5}}:} The edge device validates the received attention mask. If invalid or lost, it bypasses cloud assistance and performs standalone on-device inference. We will also discuss this in Section~\ref{sec:resource-aware-cloud-assistance}.}

\noindent
\shepherd{\textbf{Step~\textcircled{\small{6}}:} If the mask is valid, the edge device refines the resolution to construct a multi-resolution spectrogram with details in Section~\ref{sec:spectral-encoding-cnn}.}

\noindent
\shepherd{\textbf{Step~\textcircled{\small{7}}:} Finally, with the multi-resolution spectrogram, the edge device performs efficient inference using spectral encoding and spectral CNNs with details in Section~\ref{sec:spectral-encoding-cnn}.}

\shepherd{To address resource constraints and dynamic communication costs, ORCA introduces a novel resource-aware scheduler for efficient cloud assistance on batteryless devices. Our algorithm dynamically adjusts to variable communication costs, enabling optimized communication scheduling in scenarios of high communication costs for adaptive transmission and bypassing. We will detail this algorithm in Section~\ref{sec:resource-aware-cloud-assistance}. }



% \begin{figure}[h]
%     \centering
%     \includegraphics[width=0.9\linewidth]{figures/selective-wpt.png}
%     \caption{Wavelet transform based on attention masks for selective frequency band resolution refinement.}
%     \label{fig:selective-WPT}
% \end{figure}




\subsection{Preprocessing}
\label{sec:preprocess}
\shepherd{To minimize communication costs, ORCA employs a low-resolution wavelet spectrogram as a compact and informative abstraction for cloud assistance. 
We use the WPT with depth $n$ to extract coarse frequency-domain features from the input audio waveform, producing a spectrogram $S$ with a frequency dimension of $2^n$. To generalize features over time and reduce payload size, we apply average pooling along the time axis, transforming $S$ into a square matrix $S_a$ in dimension of $2^n$. We refer to $S_a$ as the cloud-assisted spectrogram and define its dimension as the cloud assistance resolution $R_a = 2^n$, with selection details in Section~\ref{sec:resource-aware-cloud-assistance}.}


\subsection{Attention Mask Generation}
\label{sec:attention-mask-generation}
\shepherd{
In this section, we discuss how the server identifies important features from the assistance spectrogram $\mathcal{S}_a$. Specifically, we define important features as the most informative frequency bands, guided by preliminary studies. The edge device then leverages this information, encoded as an attention mask, to enhance on-device inference accuracy in later steps.
}

\noindent
\subsubsection{Vision Transformer for Assistance Spectrogram.}
\shepherd{ORCA server-side design leverages the self-attention mechanism to dynamically encode the importance of input features. The server processes the assistance spectrogram $\mathcal{S}_a$ by patching it into tokens and computing a self-attention map to highlight key regions. We show the attention computation in Figure~\ref{fig:attention-block}.
First, we adopt the same architecture from the vision transformer~\cite{dosovitskiy2020vit} and divide the input spectrogram into $p^2$ patches. To preserve the spectrogram’s spectral-temporal properties, we apply positional encoding by adding trainable encoding to each patch.
Next, we pass the patches through a convolutional patch embedding layer, encoding each patch into an embedding of dimension $E$.
The resulting embedding is passed through the $i$-th attention block to compute the attention matrix $A_i$, sequentially. Formally, $A_i = \text{Softmax}(Q_i \cdot K_i^T / \sqrt{E})$, where $Q_i$ and $K_i$ are the query and key embeddings at each layer. The attention matrix $A_i$ of size $p^2 \times p^2$ captures the relative importance between patch pairs, aiding in identifying the most informative frequency bands, as discussed next.
}

\begin{figure}[tp]
    \centering
    \includegraphics[width=\linewidth]{figures/attention-block.png}
    \vspace{-0.3cm}
    \caption{Attention computation for attention rollout.}
    \label{fig:attention-block}
    \vspace{-0.3cm}
\end{figure}


\noindent
\subsubsection{Attention Mask Generation.}
Recall that the attention matrix $A_i$ represents the attention map of the $i$-th attention block, encoding the relative importance between patches in a spectrogram. 
% Sequential application of attention blocks can cause attention signals to vanish due to constant information mixing between tokens~\cite{abnar2020quantifying}. 
\shepherd{
Inspired by~\cite{abnar2020quantifying}, we compute the rollout attention map $\widetilde{A} = \Pi^{1}_{i=n} A_i = A_n A_{n-1} \cdots A_{1}$ for importance estimations.} This approach aggregates attention matrices from all blocks, enhancing interpretability and preventing attention scores from vanishing. 
The resulting rollout attention map $\widetilde{A}$ has dimensions $p^2 \times p^2$. 
Then, we aim to identify the most informative frequency bands for the edge. Intuitively, a frequency band is informative if patches within that band have high attention scores, as this indicates that the cloud model prioritizes those patches. Therefore, let $\widetilde{a}_{ij}$ represent the rollout attention between patches $i$ and $j$ in $\widetilde{A}$. We compute the column-wise summation $C$ of $\widetilde{A}$ as $C = [c_1, c_2, \cdots, c_{p^2}]$ where $c_j = \sum_{i=1}^{p^2} \widetilde{a}_{ij}$. 
The vector $C$ is reshaped into a 2D importance matrix $C'\in \mathbb{R}^{p \times p}$, where each entry represents the importance of a patch in the input WPT spectrogram. 
\shepherd{We select frequency bands by summing contiguous $k$ rows in $C'$ and identifying the highest sum, where $k$ is a predefined hyperparameter agreed upon by the server and edge device.} A binary vector of length $p$ records the selected indices, forming the spectral attention mask, which is sent to edge devices.

\noindent
\subsubsection{Contrastive Pre-Training.} \shepherd{The method above relies on a vision transformer capable of identifying informative frequency bands from the WPT spectrogram.} Given the lack of labeled data for frequency-domain feature importance information, we propose training the cloud model offline in an unsupervised manner. \shepherd{Inspired by contrastive learning, where the model learns to produce distinctive features via contrastive loss, we create attracting and contrasting pairs by masking random frequency bands and use triplet loss~\cite{schroff2015facenet} on the flattened output of vision transformer as representations.} Overall, the advantage of ORCA attention-based cloud assistance solution is twofold: first, it uses self-attention over spectrograms to guide clients in focusing on informative frequency bands, which not only improves inference accuracy on the resource-constrained edge devices but also reduces computational load by minimizing the edge model input size. Additionally, transmitting the low-resolution assistance spectrogram and attention masks is highly communication-efficient, significantly reducing communication costs and latency.


\subsection{\shepherd{Cloud-Assisted Inference}}
\label{sec:spectral-encoding-cnn}
\shepherd{
% Spectrograms provide detailed information across all frequency bands, and wavelet transform enables extraction of frequency-band details at flexible resolutions. In preliminary study, we demonstrated that higher-resolution spectrograms achieve high accuracy but increase resource usage. Inspired by the preliminary study 2, we propose a three-fold on-device inference solution: Multi-resolution Refinement, Spectral Encoding, and Multi-resolution CNN designs, selectively focusing on information-rich spectral bands from the aforementioned cloud assistance step. and still maintain high accuracy. This section details our design of a multi-resolution spectral encoding CNN (Steps \textcircled{\small{6}} and \textcircled{\small{7}} in Figure~\ref{fig:system-overview}), optimized for resource-constrained devices through the use of cloud-generated spectral attention masks discussed in the previous section.
Following the discussion on server-generated attention masks, we explore how edge devices can leverage this information for efficient on-device inference. First, we introduce the \textit{Multi-resolution Refinement} module, which extracts high-resolution frequency bands guided by attention masks. After refinement, two challenges remain: (i) embedding high-resolution spectral bands and (ii) creating a multi-resolution representation for accurate and efficient inference. For (i), we propose \textit{Spectral Encoding}, a trainable weight that encodes high-resolution frequency band-specific knowledge. For (ii), we employ \textit{Multi-resolution CNNs} to process the combination of high-resolution bands from multi-resolution refinement and their corresponding spectral encoding for efficient on-device classification.}

\noindent
\subsubsection{Multi-resolution Refinement.}
\shepherd{The server-generated spectral attention mask captures key frequency bands. It guides the edge device to selectively extract high-resolution spectrograms via wavelet transform. Let $R_l$ denote the pre-defined dimension of the low-resolution spectrogram and $R_h$ the dimension of the high-resolution spectral bands, this refinement results in $R_l$-dimensional low-resolution spectrograms and $R_h$-dimensional high-resolution spectrograms frequency bands.} To further reduce dimension, adaptive average pooling is applied along the time dimension, regularizing the size of both spectrograms.

\noindent
\subsubsection{Spectral Encoding.}
\shepherd{Since each frequency band captures unique frequency-domain properties, spectrograms from different bands should be interpreted accordingly. Using separate CNNs per band~\cite{phaye2019subspectralnet} is memory-inefficient and costly. Instead, inspired by transformer's positional encoding, we use spectral encoding, a trainable weight that encodes frequency band-specific information. It is then concatenated channel-wise to corresponding high-resolution bands, as shown in Figure~\ref{fig:spectral-encoding}. This approach helps the network to learn spectral-specific knowledge independently of the input spectrogram.}

\begin{figure}[tp]
    \centering
    \includegraphics[width=\linewidth]{figures/spectral-encoding.png}
    \vspace{-0.8cm}
    \caption{\shepherd{Spectral encoding and multi-resolution CNNs.}}
    \label{fig:spectral-encoding}
    \vspace{-0.4cm}
\end{figure}


\noindent
\subsubsection{Multi-resolution CNN.}
\shepherd{The next challenge is to create a multi-resolution representation for inference. As discussed in preliminary studies in Section~\ref{sec:preliminary-study}, discriminative information varies between spectral bands of the spectrogram. With the full low-resolution spectrogram available from preprocessing, we use two 2-layer shallow CNN as encoders, one for low resolution and one for high resolution. The encoded features are fused channel-wise into a single vector and fed into the Multi-Res classifier for final classification. This architecture reduces inference costs by leveraging spectrograms at different resolutions. If cloud assistance is unavailable, an additional Single-Res classifier is employed to process the output of the Low-Res encoder only. All components are pre-trained offline in a two-stage supervised process. First, we train the low-res encoder, high-res encoder, and multi-resolution classifier together with the attention masks generated by the pre-trained cloud vision transformer. In the second stage, we freeze all other components and train the single-resolution classifier independently.}

%%%%%%%%%%%%%%%%%%%%%%%%%%%%%%%%%%%%%%%%%%%%%%%%%%%%%%%%%%%%%%%
\begin{figure}[tp]
    \centering
    \includegraphics[width=\linewidth]{figures/intermittent.png}
    \vspace{-1.0cm}
    \caption{Execution model (up), capacitor voltage (mid), and relative power consumptions (low).  }
    \label{fig:intermittent}
\end{figure}
%%%%%%%%%%%%%%%%%%%%%%%%%%%%%%%%%%%%%%%%%%%%%%%%%%%%%%%%%%%%%%%

\subsection{Resource-Aware Scheduler}
\label{sec:resource-aware-cloud-assistance}
\shepherd{Given the high energy cost of communication and wireless uncertainty, dynamically managing data transmission size is essential for resource-efficient cloud assistance.} Experimental measurements~\cite{mileiko2023run} indicate that the uplink phase dominates energy consumption in each communication round and varies with channel conditions. Thus, a key component of our framework is optimizing uplink data transmission. 
\shepherd{We introduce a resource-aware, communication-adaptive resolution algorithm. This algorithm dynamically schedules the assistance resolution $R_a$ (as discussed in Section~\ref{sec:preprocess}) based on energy storage and communication quality for resource-efficient cloud assistance.}

\noindent
\subsubsection{Communication Model.} 
As discussed in Section~\ref{sec:system-overview}, ORCA uses two communication phases for one round of cloud assistance, uplink (Tx) and downlink (Rx). It adopts the intermittent computation model from~\cite{mileiko2023run} which concludes an uplink and a downlink in the same power cycle with a synchronized sleep period interleaved. \shepherd{The key advantage of this design is maintaining inference integrity and timeliness for cloud assistance, even during prolonged power failures in batteryless systems.} We illustrate this design in Figure~\ref{fig:intermittent}. Within one power cycle, edge device initiates by restoring the communication parameters, spreading factor (SF) and transmitting power ($P_{\text{Tx}}$) once waking up at voltage threshold $V_{\text{on}}$. Then it goes through sampling and preprocessing, Tx, sleeping, Rx, and on-device inference sequentially as discussed in Section~\ref{sec:system-overview}. Between each power cycle, our edge device checkpoints and restores SF and $P_{\text{Tx}}$ in and out of the non-volatile memory (yellow blocks in Figure~\ref{fig:intermittent}). This ensures their synchronizations to the server's recommendation for reliable communication. Here, the generic ADR algorithm~\cite{Semtech2016LoRaWAN} is employed to estimate the optimal communication parameters ensuring communication reliability. Every time the server receives an uplink packet, it calculates and compares the SNR margins to the optimal values and recommends the optimal SF and $P_{\text{Tx}}$ back to the edge device in downlink message. Edge device can then checkpoint these parameters for next round of communication. The next challenge is to complete restoring, preprocessing, Tx, sleep, Rx, inference, and checkpointing within one power cycle.

% Additionally, based on the measurements by~\cite{mileiko2023run}, uplink dominates the energy consumption and therefore, our algorithm mainly focuses on optimizing uplink data transmission.
% For uplink, the transmission time, known as the time-on-air (ToA), depends on the data rate (DR) and can be estimated by $\text{ToA} = \frac{S + S_{\text{p}}}{\text{DR}}$ for sending a payload size of $S$ with a fixed preamble $S_{\text{p}}$. 
% Conversely, a fixed time window ($T_{\text{Rx}}$) is set up for listening to the downlink signal after the sleeping period timeout. 
% Given the variability of the wireless channel, the costs of ensuring signal quality for the uplink changes constantly. To address this, we employ the adaptive data rate (ADR)~\cite{Semtech2016LoRaWAN}. By analyzing SNR history, the server recommends an optimal spreading factor (SF) and transmit power ($P_{\text{Tx}}$) for the edge device. Edge device then adjusts these parameters to ensure reliable uplink transmission. 



\begin{figure}[tp]
    \centering
    \includegraphics[width=\linewidth]{figures/workflow.png}
    \vspace{-0.5cm}
    \caption{Cloud-assisted offloading flow chart.}
    \label{fig:workflow}
    \vspace{-0.2cm}
\end{figure}

\noindent
\subsubsection{Adaptive Resolution.}
\shepherd{Given the proposed communication model and parameters, we first examine the key factors influencing energy consumption.} Since batteryless devices usually wake up at a pre-defined voltage threshold, the energy budget per power cycle is typically fixed and can be estimated by $E_{\text{cap}}=\frac{C}{2}(V_{\text{on}}^2 - V_{\text{off}}^2)$, where $C$ is capacitance, and $V_{\text{on}}$ and $V_{\text{off}}$ represent the microcontroller switching voltage thresholds (on and off, respectively), as depicted in Figure~\ref{fig:intermittent}. 
We propose ORCA resource-aware adaptive resolution algorithm for cloud assistance, designed to adapt to varying communication costs and complete each round of cloud assistance within a single power cycle. Our approach determines an optimal assistance resolution $R_a$ which in turn defines the payload size $S={R_a}^2$ for uplink. We model the adaptive resolution algorithm with the parameters followed. The uplink energy consumption $E_{\text{Tx}}$ can be estimated as: $E_{\text{Tx}} = P_{\text{Tx}} \cdot \text{ToA} = P_{\text{Tx}} ({R_a}^2+S_{\text{p}})/\text{DR}$.
where the uplink transmission time, known as the time-on-air (ToA), depends on the various data rate (DR) under different SF and can be estimated by $\text{ToA} = (S + S_{\text{p}})/\text{DR}$ for sending a payload size of $S$ with a fixed preamble $S_{\text{p}}$. 
The downlink energy cost is estimated as $E_{\text{Rx}}=P_{\text{Rx}} \cdot T_{\text{Rx}}$, the product of the downlink power and the downlink window length. Additionally, $E_{\text{Pre}}$, $E_{\text{sleep}}$, and $E_{\text{inf}}$ are for energy usage during preprocessing, sleep period, and inference, respectively, and can be considered as constants in ORCA. Moreover, to formulate the optimization problem, we define the one-hot encoded resolution selection vector $x$ for resolution ${R_a}$ and the pre-estimated accuracy vector $a$ for accuracy under different ${R_a}$ values. To complete a round of cloud assistance within a single power cycle, the model ensures $E_{\text{Pre}} + E_{\text{Tx}} + E_{\text{sleep}} + E_{\text{Rx}} + E_{\text{inf}} \leq E_{\text{cap}}$. 
We define the following optimization problem, finding the optimal resolution selection vector $x$ to maximize the accuracy under energy constraints:
% \setlength{\abovedisplayskip}{2pt}
% \setlength{\belowdisplayskip}{2pt}
\begin{equation*}
\begin{aligned}
\max_{x} \ a^{T}x \quad \textrm{s.t.} \quad & E_{\text{Tx}}(x)+E_{\text{Pre}} + E_{\text{sleep}} + E_{\text{Rx}} + E_{\text{inf}}\leq E_{\text{cap}}\\[-0.2em] 
  &\textbf{1}^Tx = 1, \ x_i = \{0, 1\} \\[-0.2em]
  \end{aligned}
\end{equation*}
The optimal resolution selection is derived by ${R_a}=\text{argmax}(x)$, and, specifically, we define ${R_a}=0$ as local bypassing without cloud assistance. In practice, since the optimization search space is small (as $R_a$ is chosen from only a few options) and the capacitor is pre-selected to ensure enough budget for at least local inference without cloud assistance, we simply iterate through all feasible solutions within the energy budget and select the one with the highest estimated accuracy.

\noindent
\subsubsection{Workflow.}
The workflow is presented in Figure~\ref{fig:workflow}. Starting with the communication parameters in the yellow block, we use the energy storage $E_{\text{cap}}$ and communication parameter recommendations from the previous round as the budget and cost inputs, respectively. These inputs are applied to the optimization problem, where the edge device determines the optimal $R_a$ for maximum assistance accuracy and then uploads the low-resolution spectrogram. The server extracts and transmits the attention masks along with the ADR in the downlink back to the edge device. The edge device verifies downlink message validity using the CRC error check or by missing packets after a downlink timeout, treating invalid messages as such. If valid, the edge device proceeds with the multi-resolution inference step as described in Section~\ref{sec:spectral-encoding-cnn}. Otherwise, due to resource constraints, the device bypasses retransmission and cloud assistance, performing single-resolution on-device inference as also detailed in Section~\ref{sec:spectral-encoding-cnn}. Overall, ORCA using fixed energy budgets and dynamic data size offers two major advantages. First, unlike reconfigurable energy storage solutions, which require additional hardware and may face durability or read-write cycle limitations~\cite{colin2018reconfigurable, bakar2022protean, mileiko2023run}, our strategy does not require extra hardware. Second, our algorithm intelligently balances communication costs and accuracy gains by adaptively selecting the amount of resources for cloud assistance. As shown in Figure~\ref{fig:resource-aware}: (i) when communication cost is low, the edge device sends a high-resolution spectrogram for better inference accuracy; (ii) when communication cost is high, it sends a low-resolution spectrogram with a smaller payload to manage energy cost, resulting in lower accuracy; (iii) if communication is unstable with packet loss, the device bypasses cloud assistance and performs local inference to avoid costly retransmissions.


\begin{figure}[tp]
    \centering
    \includegraphics[width=\linewidth]{figures/resource-aware.png}
    \vspace{-0.5cm}
    \caption{Resource-aware adaptive resolution for cloud assistance.}
    \label{fig:resource-aware}
    \vspace{-0.2cm}
\end{figure}




\vspace{-1em}
\section{Implementation}
\label{section:implementation}


\para{\docs \xspace :}
\label{sec:testbed:docs}
We have successfully built a test board featuring the OCS Controller chip and a pre-release Photonic Integrated Circuit (PIC) module without the MZI switch matrix, as shown in \fig{figure:design:evaluation-board}. The Controller Chip, measuring $4mm \times 4mm$, is manufactured using a 28nm process, while the PIC, sized at $10.5mm \times 13mm$, uses a 65nm CMOS process. The evaluation board supports 8 pairs of TX/RX SerDes at each end and has been validated for compatibility with various link layer protocols, including PCIe (32Gbps, 64Gbps) and Ethernet (56Gbps, 112Gbps). We assessed the power consumption of the peripheral circuitry using the test board. For an $8 \times 112G$ configuration, the power consumption was 8.5 watts. With the addition of 3.2 watts for the MZI switch matrix, the overall consumption totals approximately 12 watts, meeting the QSFP-DD 800Gbps standard\cite{qsfp-dd-15w}.

Notably, the full-featured version of the PIC chip has successfully completed tape-out and is currently in the packaging and testing phase. It will be available for evaluation prior to final publication.


\para{Small-scale Cluster:}
\label{sec:testbed:minipod}
We constructed a small-scale cluster to evaluate the communication performance of the ring topology. Using 32 experimental GPUs equipped with inter-host HBD support (96 lanes on PCIe 4 protocol), we formed a physical ring utilizing fixed optical modules. This mini-cluster was manually reconfigured for both 32-GPU and 16-GPU ring topology. The communication latency and AllReduce performance is evaluated.
For small packets, direct GPU-to-GPU links reduced latency by approximately 13\% compared to the NVLink switch design.
For large packets, the 16-GPU AllReduce utilized 77.11\% of the ring bandwidth, with the utilization rate increasing to 77.26\% for the 32-GPU configuration, showing minimal degradation with scaling. In comparison, the NVIDIA H100 8-GPU machine achieves an 81.77\% utilization rate without SHARP.
After deployment of \docs, the size of communication group can be reconfigured within $1ms$, while maintaining maximum throughput.


\begin{figure}[h!t]
    \vspace{-1em}
    \centering
    \includegraphics[width=0.48\textwidth]{figs/design/evaluation-board.pdf}
    \vspace{-20pt}
    \caption{Evaluation board for components of \docs.}
    \label{figure:design:evaluation-board}
    \vspace{-1em}
\end{figure}


% \vspace{-2ex}
\section{Large-Scale Simulation}
\label{sec:simulation}

We begin by outlining the experimental methodology and setup (\S\ref{sec:simulation:setup}). Next, we assess fault tolerance across different HBD architectures (\S\ref{sec:simulation:fault}), followed by end-to-end simulations to evaluate training performance under varying parallelism and GPU resource allocations (\S\ref{sec:simulation:end2end}). We then examine the improvements in communication efficiency achieved by our orchestration algorithm (\S\ref{sec:simulation:efficiency}). Finally, we present a comparative cost and power analysis of different HBD architectures (\S\ref{sec:simulation:cost-power}). The simulations demonstrate that \sys{} outperforms other architectures across all metrics. 

 

\vspace{-1ex}
\subsection{Methodology and Setup}
\label{sec:simulation:setup}


An in-house simulator dedicated for LLM training is used to evaluate \sys comprehensively. The simulator supports end-to-end simulations of both model training performance and hardware faults, with the HBD-DCN orchestration algorithm seamlessly integrated into the system.

\begin{figure}[h!t]
    \centering
    \begin{subfigure}[b]{0.23\textwidth}
        \centering
        \includegraphics[width=\textwidth]{figs/evaluation/fault_trace_based/cdf_trace_waste_tp16_gr4.pdf}
        \vspace{-1em}
        \caption{TP-16.}
        \label{fig:simulation:waste-cdf:tp16-gr8}
    \end{subfigure}
    \hspace{2pt}
    \begin{subfigure}[b]{0.23\textwidth}
        \centering
        \includegraphics[width=\textwidth]{figs/evaluation/fault_trace_based/cdf_trace_waste_tp32_gr4.pdf}
        \vspace{-1em}
        \caption{TP-32.}
        \label{fig:simulation:waste-cdf:tp32-gr8}
    \end{subfigure}
    \vspace{-2em}
    \caption{CDF of GPU waste ratio over 4-GPU node based on production fault trace.}
    \vspace{-1em}
    \label{fig:simulation:waste-cdf:gr4}
\end{figure}




\para{GPU and network specification.}
The NVIDIA H100~\cite{h100} (989 TFLOPS, 80GiB) is used for the configuration of GPU in simulation. And the HBD bandwidth of GPU is set as $6.4Tbps$, which is the sum of 8 QSFP-DD \ocstrx. The DCN bandwidth is configured the same as NVIDIA ConnectX-7 ($400Gbps$). 
Since the simulation primarily focuses on HBD, the DCN is configured as a Fat-Tree topology~\cite{sigcomm2008fattree}. Several HBD architectures are then evaluated, including:

\begin{itemize}[itemsep=2pt,topsep=0pt,parsep=0pt, leftmargin=2ex]
    \item \textbf{Big-Switch}: The ideal HBD design, featuring a large centralized switch with no forwarding latency that connects all nodes, as the theoretical upper limit of communication performance and fault resilience.
    \item \textbf{\sys{}}: Two configurations are evaluated: the \ocstrx{} bundle is set to either $K = 2$ or $K = 3$ (\S\ref{section:design:topology}), constructing 2/3-Hop Ring respectively.
    \item \textbf{NVL-36, NVL-72, NVL-576}~\cite{nvl72}: HBDs with 36, 72, or 576 GPUs, GPU are interconnected via NVLink Switches.
    \item \textbf{TPUv4}~\cite{isca2023tpu}: Centralized OCS capable of scheduling with a $4^3$ TPU cube granularity.
    \item \textbf{SiP-Ring}~\cite{sip-ml}: All nodes are connected in a series of static rings with fix sizes equal to the TP sizes.
\end{itemize}


\para{GPU count per node. }The simulation aligns with both  4-GPU node (.e.g. NVIDIA GB200 NVL-36/72/576~\cite{nvl72} and TPUv4~\cite{isca2023tpu}) and 8-GPU node design  (NVIDIA H100, AMD MI300X~\cite{amdmi300}, Intel Gaudi3~\cite{intelgaudi3}, and UBB 2.0 standard\cite{UBB2.0}). 


\begin{figure}[h!t]
\vspace{-1em}
    \centering
    \begin{subfigure}[b]{0.23\textwidth}
        \centering
        \includegraphics[width=\textwidth]{figs/evaluation/fault_model_based/frag_ratio_tp16_gr4.pdf}
        \caption{TP-16.}
        \label{fig:simulation:model:wasted-overview:tp16}
    \end{subfigure}
    \hspace{2pt}
    \begin{subfigure}[b]{0.23\textwidth}
        \centering
        \includegraphics[width=\textwidth]{figs/evaluation/fault_model_based/frag_ratio_tp32_gr4.pdf}
        \caption{TP-32.}
        \label{fig:simulation:model:wasted-overview:tp32}
    \end{subfigure}
    \vspace{-2em}
    
    \caption{GPU wastes ratio over the 4-GPU node with different GPU fault ratio based on fault model.}
    \vspace{-1em}
    \label{fig:simulation:model:wasted-overview}
\end{figure}


\para{Parallelism strategy. } 
Since \sys is primarily designed for TP, the key variable is the TP size. TP-8, TP-16, TP-32, and TP-64 are tested to evaluate the fault resilience of various HBD architectures (\S\ref{sec:simulation:fault}).
Additionally, other parallelism strategies, such as PP and DP, are used to simulate cross-ToR traffic and evaluate the orchestration algorithm (\S\ref{sec:simulation:efficiency}).


\para{Fault patterns. } The fault trace used in the simulation was collected from an 8-GPU node cluster with approximately 3,000 GPUs over a span of 160 days.
On average, the ratio of faulty 8-GPU nodes is $3.83\%$, with the P99 value as $7.22\%$, more details in Appendix~\S\ref{appendix:production-fault-trace}. In some simulations, fault traces generated based on this trace statistics are also derived. 

\subsection{HBD Fault Resilience}
\label{sec:simulation:fault}

This section evaluates the fault resilience of various HBD architectures, focusing on GPU waste ratio, job fault-waiting time, and the maximum job scale supported by the cluster. The main text presents the key results, with more detailed results provided in Appendix~\S\ref{appendix:wasted-GPUs-ratio}.



\para{GPU waste.} 
Apart from faulty GPUs, issues such as fragmentation, topology disconnections, and bandwidth degradation can render healthy GPUs wasted.
The GPU waste ratio quantifies the number of wasted GPUs under different fault scenarios. \figref{fig:simulation:waste-cdf:gr4} illustrates GPU waste ratios over production trace, while \figref{fig:simulation:model:wasted-overview} depicts the GPU waste ratio as node fault ratio vary.

\begin{table}[h!t] \footnotesize
    \vspace{-1ex}
    \centering
    \begin{tabular}{llllll}
    \toprule
    \textbf{GPU Num} & \textbf{TP} & \textbf{DP} & \textbf{PP} & \textbf{EP} & \textbf{MFU} \\
    \midrule
    1024    & 16       & 16      & 4       & 1       & 0.4276         \\
    2048    & 16      & 16      & 8        & 1       & 0.4140        \\
    4096    & 32      & 16      & 8        & 1       & 0.3894        \\
    8192    & 32      & 16      & 16      & 1       & 0.3656       \\
    16384  &  64     & 16       & 16      & 1      & 0.3116       \\
    \bottomrule
    \end{tabular}
    \caption{Optimal parallelism strategies for maximize MFU of GPT-MoE under varying GPU numbers.}
    \vspace{-3em}
    \label{tab:eval:gpt-moe-optimal}
\end{table}


In these scenarios, \sys{} ($K=3$) achieves near-zero GPU waste ratio, and outperforming all other architectures. Especially, the waste ratio for \sys ($K=2$) remains almost identical to \sys{} ($K=3$), allowing one bundle of \docs{} to be saved for clusters with low fault rates.   
NVL-36 and NVL-72 typically experience an 11\% waste ratio for TP sizes of 16 or larger, as $1/9$ of GPUs are reserved for redundant backups. NVL-576 has less fragmentation, benefiting from its larger size. TPUv4 performs well at low fault ratios and small TP sizes, but significantly degrades with larger TP sizes due to its coarse $4^3$ cube-based resource management, which amplifies the fault explosion radius. To sum up, \sys{} demonstrates the strongest fault resilience among all architectures.   



\para{Maximum job supported. } 
In fixed-size clusters, large job must pause when the available GPUs drop below the required count. Faced with same fault rate, cluster with lower GPU waste ratio can support larger job scales. \figref{fig:simulation:job_scale} shows the maximum job scale supported for various HBD architectures cluster with 2880-GPU, simulated with the fault traces normalized for 4-GPU nodes. \sys{} ($K=2$ or $K=3$) and NVL-576 lead in performance, and SiP-Ring exhibits declining efficiency as TP size increases.


\begin{figure}[h!t]
    \centering
    % \subfigure[No Fault-Waiting.]{
        \includegraphics[width=0.7\linewidth]{figs/evaluation/fault_trace_based/no_breakdown_maxjobscale_gr4.pdf}
    \vspace{-2ex}
    \caption{Maximal job scale supported by 2880 GPUs.}
    \label{fig:simulation:job_scale}
    \vspace{-1em}
\end{figure}


\para{Job fault-waiting time.} Large job must wait for the repairing when GPU availability falls below the required threshold. This simulations assume the average recovery time in the fault trace as a fixed repair duration. The total wasted time during 160 days is evaluated (\figref{fig:simulation:breakdown-duration}). For smaller TP sizes (TP-8/TP-16), NVL-36/NVL-72 exhibit the weakest resilience due to their 11\% backup overhead. For larger TP sizes (TP-32/TP-64), SiP-Ring and TPUv4 perform worst. 


\vspace{-1ex}
\subsection{Training Performance}
\label{sec:simulation:end2end}

This section analyzes the training performance of two representative large models, {LLama 3.1-405B}~\cite{llama3herdmodels} and {GPT-MoE} (configuration detailed in Appendix~\S\ref{appendix:gpt-moe}), under various GPU resource configurations and parallelism strategies. The simulation results validate the practical applicability of the \sys{} architecture. In simulations, we model practical TP and EP behaviors: For TP, increasing parallelism splits GEMMs into smaller, less efficient tasks, reducing hardware efficiency~\cite{gemm-eff}; for EP, we practically set expert imbalance coefficient at 20\%.


\begin{figure}[h!t]
    \vspace{-1em}
    \centering
    \begin{subfigure}[b]{0.23\textwidth}
        \centering
        \includegraphics[width=\textwidth]{figs/evaluation/fault_trace_based/breakdown_ratio_tp16_gr4.pdf}
        \vspace{-1em}
        \caption{TP-16.}
        \label{fig:simulation:breakdown-duration:tp16-8gpu}
    \end{subfigure}
    \hspace{2pt}
    \begin{subfigure}[b]{0.23\textwidth}
        \centering
        \includegraphics[width=\textwidth]{figs/evaluation/fault_trace_based/breakdown_ratio_tp32_gr4.pdf}
        \vspace{-1em}
        \caption{TP-32.}
        \label{fig:simulation:breakdown-duration:tp32-8gpu}
    \end{subfigure}
    \vspace{-2em}
    \caption{Job fault-waiting time over the 4-GPU node with different levels of job-scale.}
    \vspace{-1em}
    \label{fig:simulation:breakdown-duration}
\end{figure}

\para{LLama 3.1-405B\footnote{To support larger-scale TP parallelism, we simplified the GQA~\cite{GQA} architecture of LLama 3.1-405B to a traditional MHA architecture.}. }The model adopts a classical decoder-only Transformer architecture. The simulation employs the conventional 3D parallelism strategy\footnote{$TP \in \{1,2,4,8,...,128\}$, $DP \in \{1,2,4,8,...,1024\}$, $PP \in \{1,2,4,8,16\}$, $bsz=2048$}, which combines TP, DP, and PP for performance analysis. 
\tabref{tab:eval:llama3-optimal} presents the optimal parallelism strategies and their corresponding MFU for LLama 3.1-405B under varying GPU resources. As GPU resources increase, the optimal TP size also increases. When the number of GPUs exceeds 8192, the traditional 8-GPU HBD architecture within a single node begins to limit training efficiency. As the cluster size expands, larger TP sizes become increasingly optimal.





\para{GPT-MoE.} The model utilizes the Mixture-of-Experts (MoE) architecture, with $EP \in \{1,2,4,8\}$ introduced in the simulation. \tabref{tab:eval:gpt-moe-optimal} shows the optimal parallelism strategy and the corresponding MFU for GPT-MoE under various GPU resources. The optimal EP value is 1, suggesting that MoE can also achieve high efficiency with TP.


\vspace{-1ex}
\subsection{Communication Efficiency}
\label{sec:simulation:efficiency}

\begin{figure*}[!t]
    \centering
    \hfill{}
    \begin{subfigure}[b]{0.23\textwidth}
        \centering
        \includegraphics[width=\textwidth]{figs/evaluation/orch_unchange.pdf}
        \caption{Sensitivity to cluster size.}
        \label{fig:simulation:orch:cluster}
    \end{subfigure}
    \hfill{}
    \begin{subfigure}[b]{0.23\textwidth}
        \centering
        \includegraphics[width=\textwidth]{figs/evaluation/job_scale_orch.pdf}
        \caption{Impact of job-scale ratio.}
        \label{fig:simulation:orch:job}
    \end{subfigure}
    \hfill{}
    \begin{subfigure}[b]{0.23\textwidth}
        \centering
        \includegraphics[width=\linewidth]{figs/evaluation/ill_rate_orch.pdf}
        \caption{Sensitivity to fault ratio.}
        \label{fig:simulation:orch:fault}
    \end{subfigure}
    \hfill{}
    \begin{subfigure}[b]{0.23\textwidth}
        \centering
        \includegraphics[width=\linewidth]{figs/evaluation/cost/aggregate-cost.pdf}
        \caption{Aggregate cost.}
        \label{fig:eval:aggregate-cost}
    \end{subfigure}
    \vspace{-2ex}
    \caption{DCN traffic optimization analysis and aggregate normalized cost varies across different architectures under different fault ratios.}
    \label{fig:simulation:job_scale:orch}
    \vspace{-3ex}
\end{figure*}

This section examines the impact of orchestration algorithms on DCN communication efficiency. Experiments were performed on a Fat-Tree architecture, like the setup in ~\cite{sigcomm2024rdmameta}. As shown in \figref{fig:simulation:orch:cluster}, the algorithm is not sensitive to cluster size. Therefore, the evaluation is based on TP-32 operations on \sys{} with 8192 GPUs. 

\begin{itemize}[itemsep=2pt,topsep=0pt,parsep=0pt, leftmargin=2ex]
    \item \textbf{Baseline:} A greedy algorithms, which randomly select nodes from the cluster and use the first permutation that meets the requirements.
    \item \textbf{Optimized:} The HBD-DCN orchestration algorithm proposed in \secref{sec:design:orch}.
\end{itemize}  

\figref{fig:simulation:orch:job} illustrates the impact of job-scale ratios (job size/total cluster GPUs) on cross-ToR traffic, where node fault ratio is 5\%. Baseline consistently results in approximately 10\% cross-ToR traffic. In contrast, the Optimized algorithm significantly outperforms the Baseline, reducing cross-ToR traffic to just 1.72\% even at a 90\% job-scale ratio.
\figref{fig:simulation:orch:fault} explores the sensitivity to node faults, with the job scale ratio fixed at 85\%. The Baseline shows a linear increase of cross-ToR traffic, while the Optimized algorithm sustains near-zero cross-ToR traffic for fault ratios under 7\%. 


\vspace{-2ex}
\subsection{Cost and Power Analysis}
\label{sec:simulation:cost-power}



To evaluate the interconnect costs of HBD architectures, we gather the cost and power information with the following methodologies:

\begin{itemize}[itemsep=2pt,topsep=0pt,parsep=0pt, leftmargin=2ex]
    \item For standard components (DAC cables, optical transceivers, fibers), pricing is sourced from official retailer websites~\cite{FS_COM, FIBER_MALL, NADDOD} with a 60\% wholesale discount validated against internal data.
    \item For components with scarce public pricing information, such as Google Palomar OCS, NVIDIA NVLink Switch, 1.6 Tbps ACC cables/optical transceivers, the data is amalgamated from multiple sources~\cite{SEMIANALYSIS_GB200, SEMIANALYSIS_OCS, SEMIANALYSIS_Power} to enhance accuracy.
    \item Public power consumption data is available for most components, though for NVLink Switch, multiple sources are combined to estimate a reasonable value.
\end{itemize}


The breakdown analysis of each architecture is provided in the Appendix~\S\ref{appendix:cost}. Based on this, the cost and power consumption are normalized according to GPU count and per-GPU bandwidth. As depicted in \tabref{tab:eval:cost-power}, \sys{} exhibits the lowest interconnect cost per GPU per GBps. Under the $K =2$ configuration, its cost is only 62.84\% of Google TPUv4 and 30.86\% of the NVIDIA GB200 NVL-36/72, with minimal power consumption.
This efficiency is primarily attributed to the avoidance of centralized switches. TPUv4 ranks second in interconnect cost and lowest in power consumption, achieved by reducing optical module use and per-port OCS costs. The NVL series has higher interconnect costs and power consumption due to its fully-connected topology and high-cost NVLink Switches. Notably, NVL-576 incurs the highest cost and power consumption due to its multilayer nonconvergent topology, which increases optical module expenses and requires more NVLink Switches.

\begin{table}[h!t] \footnotesize
    \vspace{-3ex}
    \centering
    \begin{tabular}{lcccc}
    \toprule
    \textbf{Architecture}  & \multicolumn{2}{c}{\textbf{Per-GPU}}  & \multicolumn{2}{c}{\textbf{Per-GPU Per-GBps}} \\
 &  Cost & Watts & Cost & Watts \\
    \midrule
    TPUv4  & 1567.20  & 19.39 & 5.22& 0.06 \\
    NVL-36  & 9563.20  & 75.95 & 10.63& 0.08 \\
    NVL-72  & 9563.20  & 75.95 & 10.63 & 0.08 \\
    NVL-36x2  & 17924.00  & 150.33 & 19.92  & 0.17\\
    NVL-576   & 30417.60  & 413.45 & 33.80  & 0.46\\
    \midrule
    \SYS{} ($K=2$) &  2626.80 &  48.10 & 3.28  & 0.06\\\
    \SYS{} ($K=3$) &  3740.60 &  72.05  & 4.68  & 0.09\\
    \bottomrule
    \end{tabular}
    \caption{Interconnect cost (\$) and power (watts).}
    \label{tab:eval:cost-power}
    \vspace{-6ex}
\end{table}


Beyond interconnect costs, fault resilience variations also affect aggregate costs. The aggregate cost is defined as:

\vspace{-1em}
$$Cost_{GPU} \times (N_{Wasted-GPU} + N_{Faulty-GPU}) + Cost_{Interconnect}$$


Simulations on a 11,520-GPU cluster using the TP-32 configuration evaluate GPU availability under varying fault ratios across different architectures.
The variation in aggregate cost for different HBD architectures under varying node fault ratios is illustrated in \figref{fig:eval:aggregate-cost}. \sys{} consistently exhibits the lowest aggregate cost. Furthermore, when the fault ratio is below 12.1\%, the aggregate cost of \sys{} ($K=2$) is less than that of \sys{} ($K = 3$), suggesting that ($K = 2$) is the optimal design for most scenarios.








\section{Discussion}
\label{sec:discussion}

In this section, we first summarize the conclusion and share some key observations. Then, we reflect on the usability of our method and propose potential applications. In the end, we discuss the limitations and future work.

\subsection{Effectiveness of \name{}}
\label{sec:discuss_effectiveness}
Firstly, based on the results from Section~\ref{sec:experiment}, we can draw the following conclusions:
\begin{itemize}
    \item It is efficient to detect unknown words by combining linguistic characteristics provided by the pre-trained language model (PLM) and gaze trajectory.
    \item The prediction is mainly based on the linguistic features from the textual context captured by PLM.
    \item Gaze locates the region of interest in a timely manner, which is necessary for real-time applications. Gaze also helps improve the model performance, but its contribution is limited compared to PLM.
\end{itemize}

Additionally, it is interesting that while we typically assume that the gaze modality should contribute significantly to the task of unknown word detection, the experimental results show that the contribution of gaze to the model’s improvement is small with the existence of PLM. Based on the previous analysis of line spacing and eye tracker accuracy, a possible reason for this is that under normal reading conditions (single-line spacing, line height 3-5 mm), the eye tracker’s accuracy is insufficient to precisely detect which line the gaze belongs to, thus failing to accurately locate the gaze on the words. Furthermore, changes in user posture during long reading sessions further reduce the accuracy of the eye tracker. In our system, PLM compensates for this issue by providing linguistic information based on the text.

From another perspective, the low contribution of gaze is not necessarily a disadvantage. Our method’s reduced reliance on gaze makes it more tolerant of noise. The model’s good performance on data collected by webcams further supports this conclusion. The reduced dependency on gaze data allows our model to be applied on more affordable and accessible devices, such as webcams.

\subsection{Usability of \name{}}
\label{sec:discuss_usability}
The results from the user evaluation (Section~\ref{sec:user_evaluation}) show that our reading assistance prototype helps users read more fluently and they are more willing to use it compared to traditional click-to-translate methods. In addition to providing real-time translation and explanations during reading, our system can also benefit ESL for long-term learning. For example, based on the unknown word detected by our system, we can generate a vocabulary list for memorizing and offer memory curve tracking. Furthermore, these unknown words can also be used to generate personalized summaries and notes.

The potential issue of generalizability across users, texts and devices can be addressed through fine-tuning and reinforcement learning methods. During the initial phases of usage, the system collects both gaze and text data for fine-tuning and lets users provide feedback on the model's predictions. This allows the model to continuously learn the user's unique gaze patterns and infer their vocabulary proficiency and domain expertise from textual content, thereby improving prediction accuracy.

\subsection{Limitation and Future Works}
\label{sec:discuss_limitation}
The quality of gaze data hinders the improvement model performance. The accuracy of the eye tracker is not enough for word-level detection. Common formatting, such as single-line spacing and 10-point font, results in a line height of approximately 3-5 mm when viewed using the PDF viewer with a sidebar on a 14-inch laptop. This requires an accuracy of about $0.3-0.6^\circ$ at a reading distance of 50-60 cm. However, most eye trackers have a gaze accuracy ranging from $0.2-1.1^\circ$~\cite{gaze_survey_2024}. Combined with additional errors caused by head and upper body movements, this level of accuracy is insufficient for real-world reading scenarios. During data collection and evaluation, some participants reported that even after calibration, the error could span 1-3 lines. This makes it difficult to determine the specific word the user is focusing on based solely on gaze coordinates, explaining why gaze-based baselines performed poorly on our data.

\change{The inaccuracy of the gaze data could also lead to the inaccuracy of data labeling. To mitigate the impact of mouse clicks on gaze behavior, we asked users to label unknown words during their second pass. However, this widely adopted labeling method inherently requires "guessing" which words correspond to a given gaze trajectory. Previous works mapped each gaze coordinate directly to a specific word to establish word-gaze pairs. This method is infeasible for text with normal line spacing, so we establish gaze-word pairs by defining a bounding box based on a segment of gaze to identify the corresponding words instead. While this approach improves robustness, it may also introduce mismatches between gaze and words and thus introduce noise to the dataset. To further improve model performance, more precise labeling methods are needed.}

Additionally, reading time can be longer than several minutes in daily scenarios, so gaze drift can significantly affect data quality. In our experiments, we observed that it is difficult for participants to maintain a fixed posture after calibration, though we required them to do so. The posture shift further increases errors. Therefore, in practical applications, real-time calibration of gaze data based on user posture is crucial to ensure data quality. If the existing eye-tracking technology can combined with user posture detection~\cite{faceori}, it is possible to reduce the impact of user posture on gaze data, thereby improving the quality of gaze data.



\vspace{-1em}
\section{Related Work}

\para{HBD Architectures.}  
HBDs are crucial for enabling communication intensive parallelism strategies (TP/EP) for LLM training. NVIDIA DGX SuperPOD~\cite{superpod} and GB200 NVL series~\cite{nvl72} use any-to-any electrical switching, delivering high performance but suffering from high costs, scalability limitations, and fragmentation. In contrast, direct interconnect HBDs like Dojo~\cite{dojo}, TPUv3~\cite{cacm2020tpuv3}, and SiP-Ring~\cite{sip-ml} improve scalability but have a large fault explosion radius. TPUv4~\cite{isca2023tpu} and TPUv5p~\cite{tpuv5} attempts a middle ground but still lacks full node-level fault isolation. \sys introduces a novel architecture that reduces cost, improves scalability, minimizes fragmentation, enhances fault isolation, and dynamically supports TP.  

\para{AI DCN Architectures.}  
MegaScale~\cite{megascale} and Meta’s~\cite{sigcomm2024rdmameta} AI DC use Clos-based topologies, while Rail-Optimized~\cite{rail-optimized} and Rail-Only~\cite{wang2024railonly} architectures optimize for LLM traffic patterns. Alibaba HPN~\cite{sigcomm2024hpn} enhances fault tolerance with a dual-plane design. \sys is compatible with all of them on LLM-training.  

\para{OCS Technologies.}  
OCS enables dynamic topology reconfiguration in datacenters~\cite{missionapollo, isca2023tpu, mfabric}. MEMS OCS-based switch supports high port counts~\cite{urata2022missionapollo, mems-320}, while silicon photonics (SiPh) achieves lower latency and cost~\cite{thermo-optic_2006}. This work proposes a SiPh-based OCS transceiver (\docs), constructing an interconnect fabric without centralized switches.  

\para{Reconfigurable Networks.}  
Traditional studies~\cite{helios,c-through,osa,mordia,sirius,xia2015enabling,megaswitch,rotornet,opera,firefly,shale} focus on generic DCN architectures without optimizing for LLM training traffic, leading to suboptimal topologies. Recent advancements like SiP-ML~\cite{sip-ml}, TopoOpt~\cite{topoopt2023}, and mFabric~\cite{mfabric} introduce dedicated training optimizations but still underutilize optical network reconfigurability for better fault tolerance and GPU utilization.  

\para{AI Job Schedulers.}  
Schedulers such as ~\cite{gandiva,themis,tiresias, {byteps_1}, {byteps_2}, pollux} aim to improve GPU utilization. However, they exhibit dual limitations: their designs are premised on non-reconfigurable network, while also failing to consider job scheduling within HBD for optimizing traffic patterns in DCN. This work proposes a HBD-DCN orchestration algorithm based on reconfigurable networks to address these limitations.

\section{Conclusion}


In this paper, we studied how Bayesian mechanism design can be adapted to address the challenges posed by hallucination-prone predictions generated by modern machine learning models. By introducing a novel Bayesian framework, we modeled these imperfect signals and rigorously characterized the structure of optimal mechanisms, extending classical results like those of \citet{myerson1981optimal} to settings where posterior distributions lack continuous densities. Our findings provide new insights into how sellers can navigate uncertainty and optimize revenue in environments shaped by unreliable predictions.

Our framework has three main implications.  First, it bridges the gap between traditional auction theory and contemporary machine learning applications, offering a pathway to integrate uncertain predictive signals into practical mechanism design. Second, our comparative analysis with an alternative model, the value-with-noise model, underscores the sensitivity of optimal mechanisms to the underlying assumptions about signal generation, thereby encouraging careful model selection in real-world implementations. Finally, in contrast with the now classical formulation in the algorithm with prediction literature which assumes that advice are either correct or adversarially chosen, our Bayesian framework captures the fact that when the prediction of a machine learning model is wrong, it is in fact ``randomly'' wrong: we believe that exploring this paradigm for other problem classes could design algorithms which are not tailored towards worst-case analyses. 

Despite these contributions, several exciting questions remain. A critical open question lies in analyzing non-direct mechanisms, where signals are not directly disclosed to buyers and strategic interactions become significantly more complex. Understanding the revenue implications (if any) and computational challenges in such settings would greatly add to the value of our framework. Additionally, our results assume that the hallucination probability is known to the seller; relaxing this assumption to consider uncertainty in hallucination probabilities could further align the model with real-world applications. 

\newpage
\bibliographystyle{ACM-Reference-Format}
\bibliography{paper}

\newpage
\clearpage
\begin{appendices}

\section{Production Fault Trace}
\label{appendix:production-fault-trace}
The production fault trace was collected from an 8-GPU node pretrain cluster with 2880 GPUs over a period of 160 days. The trace includes details such as fault start time, fault end time, and the ID of the faulty node. \figref{fig:simulation:trace:timetrace} and \figref{fig:simulation:trace:cdf} provide a macro-level overview of the production fault trace. On average, the ratio of faulty 8-GPU nodes at any given time is $3.83\%$, with a p99 value of $7.22\%$.

\begin{figure}[h!t]
    \centering
    \begin{subfigure}[b]{0.23\textwidth}
        \centering
        \includegraphics[width=\textwidth]{figs/evaluation/fault_server_ratio.pdf}
        \caption{Fault Node Ratio Trace.}
        \label{fig:simulation:trace:timetrace}
    \end{subfigure}
    \hspace{2pt}
    \begin{subfigure}[b]{0.23\textwidth}
        \centering
        \includegraphics[width=\textwidth]{figs/evaluation/fault_server_cdf.pdf}
        \caption{Cumulative Distribution.}
        \label{fig:simulation:trace:cdf}
    \end{subfigure}
    \vspace{-2ex}
    \caption{Fault node trace in the production AI DC.}
    \label{fig:simulation:trace}
\end{figure}

Since most of failure events are GPU faults, we normalized the trace of 8-GPU nodes to generate 4-GPU nodes trace. Assuming that the fault rates of GPUs are i.i.d. with a fault probability of $p$ for each GPU, and considering that a node is deemed faulty if any GPU within it fails, the fault rate of an 8-GPU node is calculated as:  

\vspace{-1em}
$$
P_{fault}(8\text{-GPU}) = 1 - (1-p)^8 = 3.83\%.
$$  

From this, we derive $p = 0.49\%$. The fault rate for a 4-GPU node is then:  
$$
P_{fault}(4\text{-GPU}) = 1 - (1-p)^4 = 1.93\%.
$$  

The fault event of 4-GPU node is generate with Bayesian Equation, as:


\begin{align*}\label{eq:convert-trace}
& P_{fault}( \text{4-GPU} \mid  \text{8-GPU})\\ 
    &=\frac{P_{fault}(\text{8-GPU} \mid \text{4-GPU}) P_{fault}(\text{4-GPU})}{P_{fault}(\text{8-GPU})} \\ 
    & =  \frac{1 \times 1.93\%}{3.83\%} = 50.39\% \\
\end{align*}

Thus, whenever a fault occurs in an 8-GPU node in the original trace, each of the two corresponding 4-GPU nodes at the same location has a $50.39\%$ probability of fault. This method is used to convert the traces.

As node faults are i.i.d., the simulator linearly maps the fault trace to different network architectures.

\section{GPT-MoE Architecture}
\label{appendix:gpt-moe}
This model is a mixture-of-experts (MoE) model with the following configuration:

\para{Model Configuration:}
\begin{itemize}
    \item \textbf{Number of Layers:} 192
    \item \textbf{Inner Layer Dimension:} 49152
    \item \textbf{Embedding Dimension:} 12288
    \item \textbf{Hidden Dimension:} 12288
    \item \textbf{Vocabulary Size:} 64000
    \item \textbf{Number of Attention Heads:} 128
    \item \textbf{Maximum Sequence Length:} 2048
    \item \textbf{Number of Experts:} 8
    \item \textbf{MoE Layer Ratio:} 0.5
    \item \textbf{Top-K Experts:} 2
\end{itemize}

\para{Runtime Configuration:}
\begin{itemize}
    \item \textbf{Virtual Pipeline Parallelism:} 3
    \item \textbf{Micro Batch Size:} 1
    \item \textbf{Global Batch Size:} 1536
    \item \textbf{Max Sequence Length:} 2048
\end{itemize}




\section{Theoretical analysis of wasted GPU ratio for \sys}
\label{appendix:ft-anay}

The count of backup lines as $2K - 2$ will significantly influence the fault tolerance of \sys. We use the expectation of waste ratio caused by GPU failure and fragmentation problem to evaluate this design, the result is shown in \tabref{table:design:1.5ratio}.

For one single working server in the middle of line, the count of breakpoints $B$ on its two sides has the expectation as:

\vspace{-1em}
\begin{equation*}
E_B(\eta = 1,middle) = 2(P_s^K + P_s^{2K})
\end{equation*}

Where $P_s$ is the fail probability of GPU server, and $\eta$ is count of servers. The expectation of breakpoints count is:

Once the distance between one server and the tail of line is $\alpha < K$, it will connect to all servers between itself and the last one, so there will be no breakpoints on this side, and the expectation of breakpoints count is less than servers in the middle of line.
Then, for any server in the line topology:

\vspace{-1em}
$$
E_B(\eta = 1) \leq E_B(\eta = 1,middle) 
$$

When the distance between two servers is $\beta \geq K$, the breakpoints among them can be calculated as independent.
Once the distance $\beta < K$, as all servers in this range are connected to these two servers, there will be no breakpoints between them. So, the expectation is less than two independent servers. Then,



\vspace{-1em}
\begin{align*}
E_B(\eta =& 2) < E_B(\eta = 2, \beta \geq K) =  2E(\eta = 1)   \\ 
 E_B(\eta =& N_s) \leq N_s E_B(\eta = 1) 
\end{align*}

For a LLM job which require a ring communication size (TP .etc) as $N_t$, \sys   will cut the whole line topology into several sub lines with the length of $N_t/R$.
Once \sys is cutting a new sub line from the remaining servers in the line, 
all $N_t$ GPU will be wasted when one break point exist in the middle of this sub line required, shown in \fig{fig:subline-waste}. 
Then the expectation for waste GPU caused by one single break point is:

\vspace{-1em}
$$
E_W(B=1) = N_t R\cdot (1 - (N_t/R)^{-1} ) = R(N_t -R)
$$

\begin{figure}[h!t]
    \centering
    \includegraphics[width=0.8\linewidth]{figs/design/intra-topo/break-topo.drawio.pdf}
    \caption{Break point can cause server waste compare to ideal situation.}
    \vspace{-1em}
    \label{fig:subline-waste}
\end{figure}

As the influence between two break points only reduce the expectation of wasted GPUs, we can have this for $X$ break points:

\vspace{-1em}
\begin{equation*}
E_W(B = X) \leq XE_W(B=1) = XR(N_t-R)
\end{equation*}

So the expectation of wasted GPU for a servers cluster with $N_s$ GPU servers is:

\vspace{-1em}
\begin{align*}
E_W(\eta = N_s) &\leq \sum P(B=X ,\eta = N_s) \cdot X\cdot  E_W(B=1)\\
&= E_B(\eta = N_s)\cdot E_W(B=1)\\
&\leq  \lim_{P_s\rightarrow 0}2N_s\cdot R \cdot (N_t-R)P_s^K
\end{align*}



The final expectation of GPUs waste ratio is \eqref{eq:design:ratio}:

\begin{equation}
E_{WR}(\eta = N_s) = \frac{E_W(\eta = N_s)}{N_g} \leq 2(N_t-R)(P_s)^K
\label{eq:design:ratio}
\end{equation}

In our trace for a 160 days long pre-train job on 10K-GPU, the p99 failure rate for 8-card machines is 7\%. If a TP32 jobs is running on \sys, we can get the upper bond for waste ratio expectation for various configuration in \tabref{table:design:1.5ratio}.

\begin{table}[h!t]
\centering
\begin{tabular}{cccc}
    \toprule
        & $K=2$&$K=3$&$K=4$\\
    \midrule
     R=4& $7.35\%$ & $0.26\%$ & $9.00\times 10^{-4}$ \\
     R=8& $27.4\%$ & $1.92\%$ & $0.13\%$ \\
     \bottomrule
\end{tabular}
\caption{Upper bond for waste ratio expectation of GPU, where GPU failure rate is 0.875\% and X is 32}
\vspace{-2em}
\label{table:design:1.5ratio}
\end{table}

As shown in the table, for 4 GPU server ($R=4$) 3 bundles ($K = 3$) design, the additional waste of GPU is less than 0.26\%, while the waste ratio for $R=8,K=4$ is less than 0.13\%. This is sufficient for production clusters. 

\section{Orchestration For Fat-Tree}
\label{appendix:orch-algo}
In this section, we introduce the orchestration algorithm under Fat-Tree DCN in detail.

\para{Notations}
\label{appendix:orch-algo:notation}
To ensure rigorous mathematical reasoning, we introduce the following notations:

\begin{itemize}
    \item {
        $n$: number of nodes in the data-center.
    }
    \item {
        $K$: \docs{} bundle (see \S\ref{section:design:topology}).
    }
    \item {
        $S_{all}$: ordered set, represents all nodes numbered from 1 according to their physical connection order in DCN fabric. $|S_{all}|=n$.
    }
    \item {
        $S$: ordered subset, represents nodes, $\forall u \in S, u \in S_{all}$. Adjacent elements in $S$ are also adjacent from the perspective of the \SYS{} topology. 
    }
    \item{
        $E$: The set of edges across $S$, should be equal to $\{ (S_i, S_j) \mid 1 \leq i < j \leq n, j - i \leq K \} $, representing the connections between nodes, including both primary and backup links, and $O(|E|) = O(K|S|)$.
    }
    \item {
        $InfHBD=<S,E>$: the topology of \SYS{} as an undirected graph.
    }
    \item {
        $F$: faulty nodes.
    }
    \item {
        $HealthyHBD=<H,HE>$: healthy node subgraph where the set of healthy nodes $H = S - F$ and the edge set $HE = \{ (u, v) \mid u \in H \text{ and } v \in H \text{ and } (u, v) \in E \}$.
    }
    \item{
        $t$: TP size, number of GPUs in one TP Group.
    }
    \item{
        $r$: GPU ranks per node.
    }
    \item{
        $m=t/r$: number of nodes in a TP group.
    }
    % \item{
    %     $k$: number of rails in rail-optimized network.
    % }
    \item{
        $s$: job scale, number of GPUs required for the job.
    }
    \item{
        $d$: Aggregation-Switches Domain size. Number of nodes under coverage of one group of Aggregation-Switches.
    }
    \item{
        $n_{constrains}$: number of applied constraints in binary-search-based orchestration algorithm.
    }
    \item{
        $p$: number of nodes under each ToR.
    }
    \item{
        $l$: shortest sub-line length under fat-tree orchestration.
    }
    \item{
        $n_{maxsubline}=\lfloor \frac{nd}{p} \rfloor$: max number of sub-lines.
    }
    \item{
        $G_{deploy}=<S_{deploy},E_{deploy}>$: deployed topology. After applying the deployment strategy, the topology from the perspective of \SYS{} is described as follows: $S_{\text{deploy}}$ is an ordered set where adjacent elements correspond to adjacent nodes in \SYS{}, and $E_{\text{deploy}}$ represents the connections between nodes.
    }
    
\end{itemize}


% For the \SYS{} the orchestration algorithm in ideal conditions is relatively straightforward. The detailed steps of the algorithm are outlined in \algref{alg:orchestration-ideal}.

% Assume that the \SYS{}(with \docs{} direction $K$) is represented as an undirected graph $ \text{InfHBD} = \langle S, E \rangle $, where the ordered set of nodes $ S $ represents nodes. Adjacent elements in $S$ are also adjacent from the perspective of the \SYS{} topology. The set of edges $E$ should be equal to $\{ (S_i, S_j) \mid 1 \leq i < j \leq n, j - i \leq K \} $, representing the connections between nodes, including both primary and backup links, and $O(|E|) = O(K|S|)$. The set of faulty nodes is denoted as $ F \subseteq S $.

% The algorithm proceeds as follows:

% \begin{enumerate}
%     \item {\textbf{Extract the Healthy Node Subgraph:} First, extract the subgraph $\text{HealthyHBD} = \langle H, HE \rangle$ where the set of healthy nodes $H = S - F$ and the edge set $HE = \{ (u, v) \mid u \in H \text{ and } v \in H \text{ and } (u, v) \in E \}$. See \algref{alg:orchestration-ideal}.
%     }
%     \item {\textbf{Identify Connected Components:} Next, identify all connected components in the graph $\text{HealthyHBD}$. Faulty nodes may cause disconnections in the \SYS{} fabric, splitting the original cluster into multiple sub-HBDs. These sub-HBDs are the connected components, and TP Groups cannot span across these disconnected sub-HBDs. We use a simple Depth-First Search (DFS) algorithm here. See \algref{alg:dfs}.}
%     \item {\textbf{Generate Placement Scheme:} Given the excellent physical properties of the \SYS{}, TP Groups can be arranged sequentially within each connected component to generate placement scheme maximizing GPU utilization. See \algref{alg:orchestration-ideal}.
%     }
% \end{enumerate}

% Since each of the three steps involves traversing the entire graph's edges and nodes only once, 
The orchestration algorithm (\algref{alg:orchestration-ideal}) without considering DCN has the overall time complexity $3\cdot O(|H| + |HE|) = O(|S| + |E|) = O((K+1)|S|) = O(|S|)$.

% \begin{algorithm}[!h]
% \small
% \caption{Connected-Component-DFS}
% \label{alg:dfs}
% \SetAlgoNlRelativeSize{-1}
% \SetAlgoNlRelativeSize{1}
%  \KwIn{ $node$, $HealthyHBD$, $visited$}
%  \KwOut{ $component$}

%  Initialize $stack = [node]$ \;
%  Initialize $component = []$\;

% \While{ stack is not empty}
% {
%      $current = stack.pop()$\;
%     \If{$current$ not in $visited$}
%     {
%          Add $current$ to $visited$\;
%          Add $current$ to $component$\;
%         \For{ each neighbor in $HealthyHBD.neighbors(current)$}
%         {
%              $stack.push(neighbor)$\;
%         }
%     }
% }
        
% \KwRet{$component$}
% \end{algorithm}

\begin{algorithm}[!h]
\small
\caption{Orchestration-DCN-Free}
\label{alg:orchestration-ideal}
\SetAlgoNlRelativeSize{-1}
\SetAlgoNlRelativeSize{1}
\KwIn{$\text{InfHBD}=\langle S, E \rangle$, $F$, $m$}
\KwOut{ Placement scheme maximizing GPU utilization}

 Initialize $H = S - F$\;
 Initialize $HE = \{ (u, v) \mid u \in H \text{ and } v \in H \text{ and } (u, v) \in E \}$\;
 Create subgraph $HealthyHBD = \langle H, HE \rangle$\;
 Initialize $component\_list = []$\;
 Initialize $visited = \{\}$\;
 Initialize $placement\_scheme= \{\}$\;

\For{ each node $s$ in $H$}
{
    \uIf{ $s$ not in $visited$}
    {
         $component = Connected-Component-DFS(s, HealthyHBD, visited)$\;
         Add $component.sortedinHBD()$ to $component\_list$\;
    }
}
\For{ each $component$ in $component\_list$}
{
    \While{ $component.size()\geq m$}
    {
         Add $component.pop(m)$ to $placement\_scheme$\;
    }
}
        
 \KwRet{$placement\_scheme$}
 \end{algorithm}
 
% \subsection{Algorithms under Rail-Optimized Network}
% \label{appendix:orch-algo:rail-optimized}

% This subsection provides a detailed description of the orchestration algorithm for Rail-Optimized network.  

% The rail-optimized network topology is specifically designed for highly regular machine learning workload traffic patterns, making it a commonly used and effective architecture. As illustrated in \fig{fig:rail-topo}, Rail Switch $i$ connects to GPU $i$ in node, dividing the network into multiple rails. Let $r$ denote the GPU ranks per node, and $k$ the number of rails. In traditional rail-optimized networks, $k = r$, and a typical training strategy involves running TP $r$ within the single-node HBD, while DP operates between HBDs. Since in DP, GPUs only communicate with GPUs of the same rank in different TP groups, in other words, DP traffic is confined to the rail itself. Therefore, the Rail-Optimized topology perfectly meets this requirement.

% % \begin{figure}[!h]
% %     \centering
% %     \includegraphics[width=\linewidth]{figs/design/Orchestration/rail-optimized.drawio.pdf}
% %     \caption{Rail-Optimized Network: GPU ranks per node $r=4$, Number of rails $k=8$, Aggregation-Switches Domain size $d$, Number of Aggregation-Switches Domain $nd$, Node IDs from 1 to $nd\cdot d$. }
% %     \label{fig:rail-topo}
% % \end{figure}

% \para{Orchestration Constraints. }To minimize the cross-rail traffic which can lead to congestion and latency, the rail-optimized network introduces two key constraints for orchestration algorithms:


% \begin{itemize}
%     \item {
%         \textbf{Aggregation-Switches Domain Coverage Constraint. }
%         The coverage domian of a group of Aggregation-Switches is limited, meaning that TP groups spanning across Aggregation-Switches domains would result in cross-rail traffic, which should be avoided as much as possible.
%     }
%     \item {
%         \textbf{Node Rail State Constraint. }When$ k = r$, this constraint does not apply, as there is no cross-rail traffic.However, as HBDs extend beyond single nodes and the need for larger DP scales due to the expansion of LLM scale, scenarios with $k = p \cdot r$ may arise. This results in $p$ different node states within the data center, with each state occupying $r$ rails, and inter-state communication leads to cross-rail traffic. The specific form of this constraint depends on the deployment strategy.
%     }
% \end{itemize}

% \para{Deployment Strategy. }If the \SYS{} connections continue to follow the physical layout of nodes on the DCN Fabric, avoiding cross-rail traffic would require each TP Group to have an equal number of nodes from each state, making the algorithm to maximize GPU Utilization NP-Complete (see Appendix.\ref{appendix:np-hard-orchestration}). However, by altering the physical connection sequence of \SYS{}, this NP-Complete problem can be reduced to polynomial time. As shown in \fig{fig:parallel-line}, nodes of each state are arranged into $p$ parallel sub-lines, which are then connected end-to-end to form a single line. By restricting DP to operate within sub-lines, all DP traffic remains within the rails, effectively reducing the $k = p * r$ scenario to $k = r$. 

% % \begin{figure}[!h]
% %     \centering
% %     \includegraphics[width=\linewidth]{figs/design/Orchestration/parallel-line.drawio.pdf}
% %     \caption{The deployment strategy example with $p=4$ and Aggregation-Switches Domain size $K=8$. Node IDs from 1 to n are arranged according to their connection order in the DCN Fabric.}
% %     \label{fig:parallel-line}
% % \end{figure}

% \para{The binary search-based Orchestration algorithm.} Based on the above-mentioned constraints and the deployment strategy, we developed an orchestration algorithm that maximizes the number of constraints satisfied while meeting the job scale requirements. This is achieved using a binary search approach with the number of satisfied constraints as the variable. Both types of constraints essentially involve splitting the Line into sub-lines. Therefore, controlling the number of constraints translates to managing the number of sub-lines: fewer sub-lines mean longer sub-lines, leading to higher GPU Utilization. Since the Ideal orchestration algorithm with complexity $O(n)$ can be applied within sub-lines.

% \algref{alg:orchestration-fat-tree} is the main binary-search-based orchestration algorithm. It begins by generating the topology from the perspective of \SYS{} based on the hardware deployment strategy (\algref{alg:deployment-strategy}). Using the number of satisfied constraints as a variable, the algorithm performs a binary search to identify the placement scheme that maximizes the number of satisfied constraints while meeting the job scale requirements.  

% \algref{alg:placement-rail-optimized} calculates the placement scheme for a given number of constraints. It divides the topology into multiple ideal sub-lines and applies the ideal-case orchestration algorithm (\algref{alg:orchestration-ideal}) to each sub-line.  

% Since the time complexity of \algref{alg:orchestration-ideal} is $O(|S|)$, the complexity of \algref{alg:placement-rail-optimized} is 

% \begin{align*}
% &\sum_{i=1}^{n_{constraints}} O(|S_{subline}|) \\
% &= O(\sum_{i=1}^{n_{constraints}} |S_{subline}|) \\
% &= O(|S_{all}|) = O(n)
% \end{align*}

% Thus, the overall time complexity of \algref{alg:orchestration-rail-optimized} is $O(n \log n)$.

\begin{algorithm}[!h]
\small
\caption{Deployment-Strategy}
\label{alg:deployment-strategy}
\SetAlgoNlRelativeSize{-1}
\SetAlgoNlRelativeSize{1}
 \KwIn{Node ordered set $S$, \docs{} direction $K$, parallel factor $p$}
 \KwOut{Deployment topology $G_{deploy}=<S_{deploy},E_{deploy}>$}
 Initialize ordered set $S_{deploy}=[]$\;
 Initialize $l=\lfloor \frac{|S|}{p}\rfloor$\;
\For{$i$ in $0...p-1$}
{
    \For{$j$ in $0...l-1$}{
         Add $i+j\cdot p$ to $S_{deploy}$\;}
}
 Create $E_{deploy}=\{(S_{deploy}^i,S_{deploy}^j)|1\leq i\le j\leq |S_{deploy}|, j-i\leq K \}$\;
 \KwRet{$G_{deploy}=<S_{deploy},E_{deploy}>$}
\end{algorithm}


% \begin{algorithm}[!h]
% \small
% \caption{Placement-Rail-Optimized}
% \label{alg:placement-rail-optimized}
% \SetAlgoNlRelativeSize{-1}
% \SetAlgoNlRelativeSize{1}
%  \KwIn{Deployment topology $G_{deploy}=<S_{deploy},E_{deploy}>$, Number of applied constraints $n_{constraints}$, Faulty node $F$, Sub-line length $l$, Number of node in one TP group $m$}
%  \KwOut{Placement scheme}
%  Initialize $placement\_scheme=\{\}$\;
% \For{$i$ in $1..n_{constraints}$}
% {
%      $S_{subline}=S_{deploy}.pop(l)$\;
%      $E_{subline}=\{(u,v)\mid u\in S_{subline} \text{ and } v\in S_{subline} \text{ and } (u,v)\in E_{subline}\}$\;
%      $F_{subline}=F\cap S_{subline}$\;
%      $placement\_scheme=placement\_scheme\cup \text{Orchestration-Ideal}(<S_{subline},E_{subline}>, F_{subline}, m)$\;
% }
%  $E_{res}=\{(u,v)\mid u \in S_{deploy} \text{ and } v \in S_{deploy} \text{ and } (u,v) \in E_{deploy}\}$\;
%  $F_{res}=F\cap S_{deploy}$\;
%  $placement\_scheme=placement\_scheme\cup \text{Orchestration-Ideal}(<S_{deploy},E_{res}>, F_{res},m)$\;
%  \KwRet{$placement\_scheme$}
% \end{algorithm}


% \begin{algorithm}[!h]
% \small
% \caption{Orchestration-Rail-Optimized}
% \label{alg:orchestration-rail-optimized}
% \SetAlgoNlRelativeSize{-1}
% \SetAlgoNlRelativeSize{1}
%  \KwIn{Node ordered set $S$ (from 1 to n in DCN Fabric), GPU ranks per node $r$, Number of rails $k$, Faulty set $F$, TP size $t$, Job scale $s$ (number of GPUs required for the job), Aggregation-Switches Domain size $d$, \docs{} directions $K$.}
%  \KwOut{Placement scheme that satisfies job scale and minimizes cross-rail traffic.}
%  Initialize $p=k/r$, $m=t/r$, $n=|S|$, $l=\lfloor \frac{d}{p}\rfloor$\;
%  Create graph $G_{deploy}=<S_{deploy},E_{deploy}>=\text{Deployment-Strategy}(S,K,p)$\;
%  Initialize $high=\lfloor\frac{nd}{p}\rfloor$\;
%  Initialize $low=0$\;
%  Initialize $placement\_scheme=\{\}$\;
% \While{ $low \leq$ high}
% {
%      $mid=\lfloor \frac{low+high}{2} \rfloor$\;
%      $placement\_scheme=\text{Placement-Rail-Optimized}(G_{deploy},mid,F,l,m)$\;
%     \eIf {$|placement\_scheme|\cdot m\cdot r\ge s$}
%     {
%          $low=mid+1$\;
%     }
%     {
%          $high=mid-1$\;
%     }
% }
    
% \eIf{$|placement\_scheme|\cdot m\cdot r\ge s$}
% {
%   \KwRet {$placement\_scheme$}
% }
% {
%     \KwRet {None}
% }
% \end{algorithm}
  

Fat-Tree topology is another common data center topology. A typical training strategy for this topology aims to maximize the bandwidth utilization under ToR (Top of Rack) Switches. Using Meta's two-stage clos topology\cite{sigcomm2024meta} as a reference, it can be observed that there is an attempt to run CP under ToR.

\para{Deployment Strategy:} Assuming there are $p$ nodes under each ToR, nodes with the same index under each ToR are deployed along the same parallel sub-line, and the $p$ sub-lines are connected end-to-end, as shown in \fig{fig:fat-tree-topo}. The training strategy involves running CP $p$ across the sub-lines and running TP within them.

\para{Orchestration Constraints. }To maximize the utilization of ToR bandwidth and minimize cross-ToR traffic, the fat-tree topology introduces two constraints:

\begin{packeditemize}
    \item {
        \textbf{Aggregation-Switches Domain Constraint: }The coverage domian of a group of Aggregation Switches is limited, meaning that TP groups spanning across Aggregation Switches domains would result in cross-rail traffic, which should be avoided as much as possible.
    }
    \item {
        \textbf{TP Group Alignment Constraint: } A CP Group consists of TP Groups across parallel sub-lines. To keep CP traffic within the ToR, the TP Groups must be aligned. If a node fails under one ToR, all nodes under that ToR are considered failed, expanding the failure radius by a factor of $p$. 
    }
\end{packeditemize}

\para{Binary-Search-Based Orchestration Algorithm.} Based on the constraints and deployment strategy, we develop a binary search orchestration algorithm (see \algref{alg:orchestration-fat-tree}) that adjusts the number of satisfied constraints. The binary search first relaxes the TP Group alignment constraints within the Aggregation-Switches Domain and then relaxes the TP Group crossing constraints between Aggregation-Switch domains (see \algref{alg:placement-fat-tree}). This process is monotonic.


% \begin{figure}[!h]
%     \centering
%     \includegraphics[width=\linewidth]{figs/design/Orchestration/meta-topo.drawio.pdf}
%     \caption{Orchestration example for Fat-Tree Topology under single Aggregation-Switches Domain with $p=2$. Green indicates active node, red indicates faulty node and yellow indicates idle nodes}
%     \label{fig:meta-topo}
% \end{figure}


The time complexity of \algref{alg:orchestration-ideal} is $O(|S|)$, and the complexity of \algref{alg:placement-fat-tree} is 

$$\sum_{i=1}^{n_{subline}} O(|S_{subline}|) = O(\sum_{i=1}^{n_{subline}} |S_{subline}|) = O(|S_{all}|) = O(n)$$  

Thus, the overall time complexity of \algref{alg:orchestration-fat-tree} is $O(n \log n)$.

\begin{algorithm}[!h]
\small
\caption{Placement-Fat-Tree}
\label{alg:placement-fat-tree}
\SetAlgoNlRelativeSize{-1}
\SetAlgoNlRelativeSize{1}
 \KwIn{$G_{deploy}=<S_{deploy},E_{deploy}>$, $n_{constraints}$, $F$, $l$, $m$, $n_{maxsubline}$, $d$, $p$}
 \KwOut{Placement scheme}
 Initialize $placement\_scheme=\{\}$\;
 Initialize $n_{align}=max(0,n_{constraints}-n_{maxsubline})$, $n_{subline}=min(n_{maxsubline},n_{constraints})$\;
 
\For{$i$ in $0..n_{align}-1$}
{
    \For{$j$ in $1..d$}
    {
        $sid=i*d+j$\;
        \If{$sid \in F$}
        {
            $F\cup \{\lfloor \frac{sid-1}{p}\rfloor\cdot p+1..(\lfloor \frac{sid-1}{p}\rfloor+1)\cdot p \}$\;
        }
    }
}
\For{$i$ in $1..n_{subline}$}
{
     $S_{subline}=S_{deploy}.pop(l)$\;
     $E_{subline}=\{(u,v)\mid u\in S_{subline} \text{ and } v\in S_{subline} \text{ and } (u,v)\in E_{subline}\}$\;
     $F_{subline}=F\cap S_{subline}$\;
     $placement\_scheme=placement\_scheme\cup \text{Orchestration-Ideal}(<S_{subline},E_{subline}>, F_{subline}, m)$\;
}
 $E_{res}=\{(u,v)\mid u \in S_{deploy} \text{ and } v \in S_{deploy} \text{ and } (u,v) \in E_{deploy}\}$\;
 $F_{res}=F\cap S_{deploy}$\;
 $placement\_scheme=placement\_scheme\cup \text{Orchestration-Ideal}(<S_{deploy},E_{res}>, F_{res},m)$\;
 \KwRet{$placement\_scheme$}
\end{algorithm}

\begin{algorithm}[!h]
\small
\caption{Orchestration-Fat-Tree}
\label{alg:orchestration-fat-tree}
\SetAlgoNlRelativeSize{-1}
\SetAlgoNlRelativeSize{1}
 \KwIn{$S$, $r$, $p$, $F$, $t$, $s$, $d$, $K$.}
 \KwOut{Placement scheme that satisfies job scale and minimizes cross-rail traffic.}
 Initialize $m=t/r$, $n=|S|$, $l=\lfloor\frac{d}{p}\rfloor$\, $n_{domain}=\lfloor\frac{n}{d}\rfloor$, $n_{maxsubline}=\lfloor\frac{nd}{p}\rfloor$\;
 Create graph $G_{deploy}=<S_{deploy},E_{deploy}>=\text{Deployment-Strategy}(S,K,p)$\;
 Initialize $high=n_{domain}+n_{maxsubline}$\;
 Initialize $low=0$\;
 Initialize $placement\_scheme=\{\}$\;
\While{ $low \leq$ high}
{
     $mid=\lfloor \frac{low+high}{2} \rfloor$\;
     $placement\_scheme=\text{Placement-Fat-Tree}(G_{deploy},mid,F,l,m,n_{maxsubline},d,p)$\;
    \eIf {$|placement\_scheme|\cdot m\cdot r\ge s$}
    {
         $low=mid+1$\;
    }
    {
         $high=mid-1$\;
    }
}
    
\eIf{$|placement\_scheme|\cdot m\cdot r\ge s$}
{
    \KwRet {$placement\_scheme$}
}
{
    \KwRet {None}
}
\end{algorithm}





\section{Additional Simulation Results for Fault Resilience}
\label{appendix:wasted-GPUs-ratio}
This section presents additional simulation results related to \S\ref{sec:simulation:fault}. \figref{fig:simulation:wasted-trace} shows the variation of the GPU waste ratio over time under the production fault trace. \figref{fig:simulation:waste-cdf:gr4:supple} presents the CDF data for the GPU waste ratio. \figref{fig:simulation:model:wasted-gr4} illustrates the waste GPU ratio for different HBD architectures under various node failure rates, including the results for TP-8 to TP-64. \figref{fig:simulation:breakdown-duration-supple} shows the proportion of job-fault waiting time relative to total time for different job scales. All the aforementioned experiments include results for TP-8, TP-16, TP-32, and TP-64 configurations.








\begin{figure*}[h!t]
    \centering
    \begin{subfigure}[b]{0.23\linewidth}
        \centering
        \includegraphics[width=\linewidth]{figs/evaluation/fault_trace_based/frag_trace_tp8_gr4.pdf}
        \caption{TP-8.}
        \label{fig:simulation:wasted-trace:tp8-4gpu}
    \end{subfigure}
    \hspace{2pt}
    \begin{subfigure}[b]{0.23\linewidth}
        \centering
        \includegraphics[width=\linewidth]{figs/evaluation/fault_trace_based/frag_trace_tp16_gr4.pdf}
        \caption{TP-16.}
        \label{fig:simulation:wasted-trace:tp16-4gpu}
    \end{subfigure}
    \hspace{2pt}
    \begin{subfigure}[b]{0.23\linewidth}
        \centering
        \includegraphics[width=\linewidth]{figs/evaluation/fault_trace_based/frag_trace_tp32_gr4.pdf}
        \caption{TP-32.}
        \label{fig:simulation:wasted-trace:tp32-4gpu}
    \end{subfigure}
    \hspace{2pt}
    \begin{subfigure}[b]{0.23\linewidth}
        \centering
        \includegraphics[width=\linewidth]{figs/evaluation/fault_trace_based/frag_trace_tp64_gr4.pdf}
        \caption{TP-64.}
        \label{fig:simulation:wasted-trace:tp64-4gpu}
    \end{subfigure}

    \vspace{-1ex}
    \caption{GPU waste ratio over production fault trace, 4 GPU node.}
    \label{fig:simulation:wasted-trace}
\end{figure*}


\begin{figure*}[h!t]
    \centering
    \begin{subfigure}[b]{0.23\linewidth}
        \centering
        \includegraphics[width=\linewidth]{figs/evaluation/fault_trace_based/cdf_trace_waste_tp8_gr4.pdf}
        \caption{TP-8.}
        \label{fig:simulation:waste-cdf:tp8-gr4}
    \end{subfigure}
    \hspace{2pt}
    \begin{subfigure}[b]{0.23\linewidth}
        \centering
        \includegraphics[width=\linewidth]{figs/evaluation/fault_trace_based/cdf_trace_waste_tp16_gr4.pdf}
        \caption{TP-16.}
        \label{fig:simulation:waste-cdf:tp16-gr4}
    \end{subfigure}
    \hspace{2pt}
    \begin{subfigure}[b]{0.23\linewidth}
        \centering
        \includegraphics[width=\linewidth]{figs/evaluation/fault_trace_based/cdf_trace_waste_tp32_gr4.pdf}
        \caption{TP-32.}
        \label{fig:simulation:waste-cdf:tp32-gr4}
    \end{subfigure}
    \hspace{2pt}
    \begin{subfigure}[b]{0.23\linewidth}
        \centering
        \includegraphics[width=\linewidth]{figs/evaluation/fault_trace_based/cdf_trace_waste_tp64_gr4.pdf}
        \caption{TP-64.}
        \label{fig:simulation:waste-cdf:tp64-gr4}
    \end{subfigure}
    \vspace{-1ex}
    \caption{CDF of GPU waste ratio over production fault trace, 4 GPU node.}
    \label{fig:simulation:waste-cdf:gr4:supple}
\end{figure*}


\begin{figure*}[h!t]
    \centering
    \begin{subfigure}[b]{0.23\linewidth}
        \centering
        \includegraphics[width=\linewidth]{figs/evaluation/fault_model_based/frag_ratio_tp8_gr4.pdf}
        \caption{TP-8.}
        \label{fig:simulation:model:wasted:tp8}
    \end{subfigure}
    \hspace{2pt}
    \begin{subfigure}[b]{0.23\linewidth}
        \centering
        \includegraphics[width=\linewidth]{figs/evaluation/fault_model_based/frag_ratio_tp16_gr4.pdf}
        \caption{TP-16.}
        \label{fig:simulation:model:wasted:tp16}
    \end{subfigure}
    \hspace{2pt}
    \begin{subfigure}[b]{0.23\linewidth}
        \centering
        \includegraphics[width=\linewidth]{figs/evaluation/fault_model_based/frag_ratio_tp32_gr4.pdf}
        \caption{TP-32.}
        \label{fig:simulation:model:wasted:tp32}
    \end{subfigure}
    \hspace{2pt}
    \begin{subfigure}[b]{0.23\linewidth}
        \centering
        \includegraphics[width=\linewidth]{figs/evaluation/fault_model_based/frag_ratio_tp64_gr4.pdf}
        \caption{TP-64.}
        \label{fig:simulation:model:wasted:tp64}
    \end{subfigure}
    \vspace{-1ex}
    \caption{GPU wastes ratio with different GPU fault ratio, 4-GPU node.}
    \label{fig:simulation:model:wasted-gr4}
\end{figure*}



\begin{figure*}[h!t]
    \centering
    \begin{subfigure}[b]{0.23\linewidth}
        \centering
        \includegraphics[width=\linewidth]{figs/evaluation/fault_trace_based/breakdown_ratio_tp8_gr4.pdf}
        \caption{TP-8.}
        \label{fig:simulation:breakdown-duration:tp8-4gpu}
    \end{subfigure}
    \hspace{2pt}
    \begin{subfigure}[b]{0.23\linewidth}
        \centering
        \includegraphics[width=\linewidth]{figs/evaluation/fault_trace_based/breakdown_ratio_tp16_gr4.pdf}
        \caption{TP-16.}
        \label{fig:simulation:breakdown-duration:tp16-4gpu}
    \end{subfigure}
    \hspace{2pt}
    \begin{subfigure}[b]{0.23\linewidth}
        \centering
        \includegraphics[width=\linewidth]{figs/evaluation/fault_trace_based/breakdown_ratio_tp32_gr4.pdf}
        \caption{TP-32.}
        \label{fig:simulation:breakdown-duration:tp32-4gpu}
    \end{subfigure}
    \hspace{2pt}
    \begin{subfigure}[b]{0.23\linewidth}
        \centering
        \includegraphics[width=\linewidth]{figs/evaluation/fault_trace_based/breakdown_ratio_tp64_gr4.pdf}
        \caption{TP-64.}
        \label{fig:simulation:breakdown-duration:tp64-4gpu}
    \end{subfigure}
    \vspace{-1ex}
    \caption{Job fault-waiting duration with different levels of job-scale, 4 GPU node}
    \label{fig:simulation:breakdown-duration-supple}
\end{figure*}





\vspace{-12em}
\section{Detailed Cost and power consumption Analysis}
\label{appendix:cost}
In this section, \tabref{tab:eval:components} provides a detailed description of the quantity, cost, bandwidth, and power consumption of the interconnect components in various network architectures, including Google TPUv4~\cite{isca2023tpu}, NVIDIA GB200 NVL series~\cite{nvl72}, Alibaba HPN\cite{sigcomm2024hpn}, and \sys{}.


\begin{table*}[h!t] \small
    \centering
    \begin{tabular}{lllll}
    \toprule
    
    \textbf{Component} & \textbf{Quantity} & \textbf{Unit Cost (\$)}  & \textbf{Unit Bandwidth (GBps)} & \textbf{Unit Power (W)} \\

    \midrule
    \multicolumn{5}{c}{\textbf{Google TPUv4\cite{isca2023tpu} with 4096 GPU, bandwidth 300GBps/GPU}} \\
    
    \midrule
    OCS\cite{sigcomm2023lightwave} & 48 & 80000 & 6400 & 108 \\
    DAC Cable\cite{400G_DAC} & 5120 & 63.60 & 50 & 0.1 \\
    Optical Module\cite{400G_OPTICAL_MODULE} & 6144 & 360 & 50 & 12  \\
    Fiber\cite{FIBER}& 6144 & 6.80 & 50 & 0 \\
    
    \midrule
    \multicolumn{5}{c}{\textbf{NVIDIA GB200 NVL-36\cite{SEMIANALYSIS_GB200} with 36 GPU, bandwidth 900GBps/GPU}}\\
    \midrule
    NVLink Switch\cite{SEMIANALYSIS_Power} & 9 & 28000 & 3600 & 275 \\
    DAC Cable\cite{200G_DAC} & 2592 & 35.60 & 25 & 0.1 \\
    
    \midrule
    \multicolumn{5}{c}{\textbf{NVIDIA GB200 NVL-72\cite{nvl72}\cite{SEMIANALYSIS_GB200} with 72 GPU, bandwidth 900GBps/GPU}}\\
    \midrule
    NVLink Switch\cite{SEMIANALYSIS_Power} & 18 & 28000 & 3600 & 275 \\
    DAC Cable\cite{200G_DAC} & 5184 & 35.60 & 25 & 0.1 \\
    \midrule
    \multicolumn{5}{c}{\textbf{NVIDIA GB200 NVL-36x2\cite{SEMIANALYSIS_GB200} with 72 GPU, bandwidth 900GBps/GPU}}\\
    \midrule
    NVLink Switch\cite{SEMIANALYSIS_Power} & 36 & 28000 & 3600 &  275\\
    DAC Cable\cite{200G_DAC} & 6480 & 35.60 & 25 & 0.1 \\
    ACC Cable\cite{SEMIANALYSIS_Power} & 162 & 320 & 200 & 2.5 \\

    \midrule
    \multicolumn{5}{c}{\textbf{NVIDIA GB200 NVL-576\cite{SEMIANALYSIS_GB200} with 576 GPU, bandwidth 900GBps/GPU}}\\
    \midrule
    NVLink Switch\cite{SEMIANALYSIS_Power} & 432 & 28000 & 3600 & 275 \\
    DAC Cable\cite{200G_DAC} & 41472 & 35.60 & 25 & 0.1 \\
    Optical Module\cite{OSFPXD} & 4608 & 850 & 200 & 25 \\
    Fiber\cite{FIBER} & 4608 & 6.80 & 200 & 0 \\

    \midrule
    \multicolumn{5}{c}{\textbf{Alibaba HPN\cite{sigcomm2024hpn} with 16320 GPU, bandwidth 50GBps/GPU}}\\
    \midrule
    EPS\cite{51.2T_EPS} & 360 & 14960 & 6400 & 3145 \\
    DAC Cable\cite{200G_DAC} & 32640 & 35.60 & 25 & 0.1\\
    Optical Module\cite{400G_OPTICAL_MODULE} & 28800 & 360 & 50 & 12 \\
    Fiber\cite{FIBER} & 14400 & 6.80 & 50 & 0 \\

    \midrule
    \multicolumn{5}{c}{\textbf{\SYS{}($K=2$)  with 4 GPU, bandwidth 800GBps/GPU}}\\
    \midrule
    DAC Cable\cite{1.6T_DAC}& 4 & 199.60 & 200 & 0.1\\
    dOCS Module & 16 & 600 & 100 & 12 \\
    Fiber\cite{FIBER} & 16 & 6.80 & 100 & 0 \\

    \midrule
    \multicolumn{5}{c}{\textbf{\SYS{}($K=3$)  with 4 GPU, bandwidth 800GBps/GPU}}\\
    \midrule
    DAC Cable\cite{1.6T_DAC} & 2 & 199.60 & 200 & 0.1\\
    dOCS Module & 24 & 600 & 100 & 12 \\
    Fiber\cite{FIBER} & 24 & 6.80 & 100 & 0 \\
    \bottomrule
    \end{tabular}
    \caption{Interconnect cost and power consumption of components used in different network architectures.}
    \label{tab:eval:components}
\end{table*}


\end{appendices}





%%%%%%%%%%%%%%%%%%%%%%%%%%%%%%%%%%%%%%%%%%%%%%%%%%%%%%%%%%%%%%%%%%%%%%%%%%%%%%%%
\end{document}
%%%%%%%%%%%%%%%%%%%%%%%%%%%%%%%%%%%%%%%%%%%%%%%%%%%%%%%%%%%%%%%%%%%%%%%%%%%%%%%%

%%  LocalWords:  endnotes includegraphics fread ptr nobj noindent
%%  LocalWords:  pdflatex acks