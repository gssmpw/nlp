
\documentclass[a4paper]{article}

\usepackage[utf8]{inputenc}
\usepackage[T1]{fontenc}
\usepackage{amsmath, amssymb}
\usepackage{graphicx}
\usepackage[margin=2cm]{geometry}
\usepackage{placeins}


\pdfsuppresswarningpagegroup=1


\date{}

\begin{document}

\title{In Situ Optimization of an Optoelectronic Reservoir Computer with Digital Delayed Feedback}

\maketitle

% Author: Please give the full first and last names of authors and include * after the name of all corresponding authors

\section{Dynamics of the optoelectronic oscillator}
\textbf{Figure \ref{fig:dynamics}} presents the behavior of the optoelectronic oscillator: Figures (a-c) show cobweb diagrams in stable, periodic, and chaotic regimes, respectively, Figures d-e show simulated and experimentally obtained fixed points in the systems at different settings based on the Ikeda model~\cite{ref:ikeda1979,ref:ikeda1980,ref:ikeda1982,ref:neyer1982,ref:larger2004}.

\begin{figure}[h]
  \includegraphics[width=\linewidth]{Figures/Results.png}
    \caption{Simulated and experimental results from set-up shown in Figure 2 in the main text: calculated results based on the discrete-time model: (a) stable, (b) periodic, and (c) chaotic regimes. (d-e) Fixed points of the system as a function of (d) bias voltage $V_B$ and (e) optical power $P_\mathrm{max}$, f) Simulated dynamics using discrete-time model
    }
  \label{fig:dynamics}
\end{figure}


As discussed in~\cite{ref:neyer1982}, if the response time of the system $\tau$ is much less than the delay time $\tau_{D}$ $\tau\ll\tau_D$ the continuous-time dynamics of the delayed-feedback system can be efficiently modeled by the discrete-time difference equation
\begin{equation}
  x_{n+1}=G/2\left(1+M\sin(\pi(x_n+x_b))\right),
  \label{eq:ikeda-discrete}
\end{equation}
where $x_{n}=V(n\tau_D)/V_\pi$, $x_b=V_B/V_\pi$, and
\begin{equation}
  G=P_\mathrm{out}G^{*}/V_\pi
\end{equation}
is the net gain of the open loop. The graphical solution of the \ref{eq:ikeda-discrete} is depicted in \textbf{Figure }. The resulting path originates from $x_\mathrm{0}$ in the proximity of $x_\mathrm{1}$ and moves away from $x_\mathrm{1}$ due to instability. However, it is eventually attracted to a stable limit determined by the number of periods. For \textbf{Figure \ref{fig:dynamics}a}, the period is 1, for \textbf{Figure \ref{fig:dynamics}b} it is 2, and for \textbf{Figure \ref{fig:dynamics}c}, the period is 3.

The simulation results of the optoelectronic oscillator are represented in subplots \textbf{Figures \ref{fig:dynamics}a-c}, which show the cobweb diagrams of the different regimes.
The method involves overlapping the cobweb plot with the function $y=x$ to identify the fixed points.
\textbf{Figure \ref{fig:dynamics}a} shows the dynamics of the system in the stable regime corresponding to the parameters $P_\mathrm{max}=0.3$ mW denotes the maximum power, $V_\mathrm{b}=0$ V represents the bias voltage, $G=0.56$ signifies the feedback gain, and $M=0.983$ is defined as the modulation factor.
So, in the intersection, we obtain one unstable fixed point.
\textbf{Figure \ref{fig:dynamics}b} shows the periodic regime with $P_\mathrm{max}=0.5$ mW, $V_\mathrm{b}=0$ V and $G=0.93$. Here, we encounter two unstable fixed points.
The chaotic regime is shown in  \textbf{Figure \ref{fig:dynamics}c} with $P_\mathrm{max}=0.9$ mW and \textbf{$V_\mathrm{b}=0$ V}, $G=1.49$.
By examining the three subplots, we can observe that the stability of $x_\mathrm{n+1}$ vs. $x_\mathrm{n}$ is decreasing compared to the stable regime in \textbf{Figure \ref{fig:dynamics}a}.
%One can see the evolution of $x_\mathrm{n+1}$ vs. $x_\mathrm{n}$ losing stability compared to the stable regime \textbf{Figure \ref{fig:dynamics}a}.
Hopf bifurcations are shown in \textbf{Figures \ref{fig:dynamics}d-f}. Where  \textbf{Figures \ref{fig:dynamics}d} demonstrates the equilibrium values of the first iteration "N=1" for the "Number of the equation" considering the bias voltage. The black curve represents the simulated results, while the red curve represents the experimental measurements, considering the parameters \textbf{{$G=0.93$}} and \textbf{$M=0.983$}. The graph showcases stable regions with a single stable solution denoting system stability. In contrast, the bistable and periodic regions exhibit three solutions: two stable and one unstable. The presence of periodic solutions arises from the equation's bifurcation of stable states. Simulated results are used to validate the theoretical predictions, ensuring the accuracy and reliability of the findings.\textbf{Figures \ref{fig:dynamics}e} which exhibits a cascade of periodic solutions with fixed parameters \textbf{$G=0.93$} and \textbf{$M=0.983$}. The graph presents a depiction of the stable and unstable fixed points as a function of the maximum power, $P_\mathrm{max}$, up to iteration \textbf{$N=8$}. A significant observation is that as $P_\mathrm{max}$ increases, so does the net feedback gain, G, leading to a growing number of bifurcations. Notably, this cascade of periodic solutions appears to extend infinitely until reaching the critical value \textbf{$G_{c}$=1.49}. Upon reaching it, the system transforms into an aperiodic(chaotic) regime, resulting in the absence of periodic solutions.
\textbf{Figure \ref{fig:dynamics}f} depicts stable, periodic, and chaotic regimes in the time domain.

%\todo[inline]{(Optional) Insert the $V(P)$ graph here}

\section{Electro-optic modulator driver}

To drive the electro-optical modulator which has $50~\Omega$ characteristic impedance the low-current output signal of the Moku:Go's FIR filter needed to be amplified.
For this purpose we have built a driver circuit based on a high-speed analog operational amplifier (LM7171, Texas Instruments) as shown in Figure~\ref{fig:schematic}.
We have incorporated a voltage divider at the input of the amplifier to adjust the relative strengths of the delayed feedback and input signal.

\begin{figure}[h]
  \centering
  \includegraphics[width=0.8\linewidth]{Figures/Amplifier.png}
    \caption{(a) Schematic of the electro-optic modulator driver. (b-c) design of the amplifier board.}
  \label{fig:schematic}
\end{figure}

\FloatBarrier

\bibliographystyle{plain}
\bibliography{bibliography}

\end{document}

