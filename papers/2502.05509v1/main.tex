%%%%%%%%%%%%%%%%%%%%%%%%%%%%%%%%%%%%%
%%%%%%%%%%%%%%%%%%%%%%%%%%%%%%%%%%%%%
%% HaMi's IEEE Conference Template %%
%%%%%%%%%%%%%%%%%%%%%%%%%%%%%%%%%%%%%
%%%%%%%%%%%%%%%%%%%%%%%%%%%%%%%%%%%%%

\documentclass[conference]{IEEEtran}

\hyphenpenalty=4000

\usepackage{cite}
\usepackage{amsmath,amssymb,amsfonts}
\usepackage{algorithmic}
\usepackage{algorithm}
\usepackage{multirow}
\usepackage{graphicx}
\usepackage{svg}
\usepackage{textcomp}
\usepackage{xcolor, soul}
\usepackage{booktabs}




\def\BibTeX{{\rm B\kern-.05em{\sc i\kern-.025em b}\kern-.08em
    T\kern-.1667em\lower.7ex\hbox{E}\kern-.125emX}}
\begin{document}


\title{Do Spikes Protect Privacy? 

Investigating Black-Box Model Inversion Attacks 

in Spiking Neural Networks} % Will be changed later

% \author{\IEEEauthorblockN{Anonymous Authors}}

% \author{\IEEEauthorblockN{Full Name}
% \IEEEauthorblockA{\textit{Department of Electrical and Computer Engineering} \\
% George Mason University, Fairfax, VA, USA\\
% Email: netID@gmu.edu}
% \and
% \IEEEauthorblockN{Full Name}
% \IEEEauthorblockA{\textit{Department of Electrical and Computer Engineering} \\
% George Mason University, Fairfax, VA, USA\\
% Email: netID@gmu.edu}
% }

\author{\IEEEauthorblockN{Hamed Poursiami}
\IEEEauthorblockA{\textit{ECE Department} \\
George Mason University \\ Fairfax, VA, USA\\
Email: hpoursia@gmu.edu}

\and
\IEEEauthorblockN{Ayana Moshruba}
\IEEEauthorblockA{\textit{ECE Department} \\
George Mason University \\ Fairfax, VA, USA\\
Email: amoshrub@gmu.edu}

\and
\IEEEauthorblockN{Maryam Parsa}
\IEEEauthorblockA{\textit{ECE Department} \\
George Mason University \\ Fairfax, VA, USA\\
Email: mparsa@gmu.edu}

}






\maketitle
\begin{abstract}

As machine learning models become integral to security-sensitive applications, concerns over data leakage from adversarial attacks continue to rise. Model Inversion (MI) attacks pose a significant privacy threat by enabling adversaries to reconstruct training data from model outputs. While MI attacks on Artificial Neural Networks (ANNs) have been widely studied, Spiking Neural Networks (SNNs) remain largely unexplored in this context. Due to their event-driven and discrete computations, SNNs introduce fundamental differences in information processing that may offer inherent resistance to such attacks. A critical yet underexplored aspect of this threat lies in black-box settings, where attackers operate through queries without direct access to model parameters or gradients—representing a more realistic adversarial scenario in deployed systems. This work presents the first study of black-box MI attacks on SNNs. We adapt a generative adversarial MI framework to the spiking domain by incorporating rate-based encoding for input transformation and decoding mechanisms for output interpretation. Our results show that SNNs exhibit significantly greater resistance to MI attacks than ANNs, as demonstrated by degraded reconstructions, increased instability in attack convergence, and overall reduced attack effectiveness across multiple evaluation metrics. Further analysis suggests that the discrete and temporally distributed nature of SNN decision boundaries disrupts surrogate modeling, limiting the attacker’s ability to approximate the target model.

\end{abstract}

\begin{IEEEkeywords}
Spiking Neural Networks, Black-Box Model Inversion Attacks, Neuromorphic Privacy, Adversarial Machine Learning
\end{IEEEkeywords}



% 
% 
The widespread integration of communication networks and smart devices in modern control systems has increased the vulnerability of industrial systems to online cyber-attacks, e.g., Industroyer, Blackenergy, etc \citep{osti_1505628}.
% Modern control systems have seen a large push to include communication networks and smart devices to increase performance, made possible by improvements in communication device cost and energy consumption. This trend has been coupled with the usage of open-standard communication protocols among industrial control systems, making them vulnerable to online cyber-attacks such as Industroyer, Blackenergy, etc \citep{osti_1505628}. 
To counter this, methods have been developed to improve security by achieving attack detection, mitigation, and monitoring, among others \citep{sandberg2022secure}. This paper focuses on active attack diagnosis to mitigate stealthy attacks. 
%
%\subsection{Literature review}

Active diagnosis techniques rely on the inclusion of additional moduli to control systems
% inclusion within the control system of additional moduli 
to alter the behavior of the system compared to information known by the attacker. 
For instance, the concept of additive watermarking was introduced in \cite{mo2015physical}, where noise signals of known mean and variance are added at the plant and compensated for it at the controller. 
This compensation, however, is not exact, causing some performance degradation. Thus, trade-offs between performance and detectability  are necessary \citep{zhu2023detection}.
% A later work \citep{zhu2023detection} designs the watermark signal by trading performance for detection. Thus, although additive watermarking serves as a good detection scheme, they endure performance losses even in the nominal case. 

In encrypted control \citep{darup2021encrypted}, the sensor data is encrypted, sent to the controller, and then operated on directly. Encrypted input signals are sent back to the plant for decryption. Although encryption is widespread in IT security, in control systems it presents some concerns, such as the introduction of time delays \citep{stabile2024verifiable}, while it may present inherent weaknesses \citep{alisic2023model}.
% they are not preferred as they introduce time delays \citep{stabile2024verifiable} which can cause instability, and some encryption schemes can be very weak  \citep{alisic2023model}. 

In moving target defense \citep{griffioen2020moving}, the plant is augmented with fictitious dynamics, known to the controller. The plant output is transmitted to the controller along with the fictitious states over a network under attack. 
The additional measurements then aide in the detection of attacks. 
This comes at the cost of higher communication bandwidth needs, which increases rapidly with the dimension of the augmented systems.
% Since the dynamics of the fictitious dynamics are exactly known to the controller, the attack is detected easily. However, when the scale of the system increases, the communication bandwidth used by moving the target defense approach increases rapidly. 

Other recently proposed works include two-way coding \citep{fang2019two}, a weak encryuption technique, and dynamic masking \citep{abdalmoaty2023privacy}, which enhances privacy as well as security, have been shown to be effective against zero-dynamics attacks.
% Two-way coding \citep{fang2019two} and dynamic masking \citep{abdalmoaty2023privacy} are other recently proposed approaches. Two-way coding is another form of weak encryption technique whilst dynamic masking proposes an architecture that enhances both privacy and security. These schemes are shown to be effective against zero dynamics attacks but remain to be studied for other classes of attacks. 
% Recent extensions include \citep{mukherjee2021secure,ramos2024privacy}.
% Some other works which are related are \citep{mukherjee2021secure}, an extension of \cite{fang2019two}. The work \citep{ramos2024privacy} is an extension of moving target defense for multi-agent systems. 
Furthermore, filtering techniques for attack detection are proposed by \cite{murguia2020security,hashemi2022codesign,escudero2023safety}, while not focusing on stealthy attacks.
% The works \citep{murguia2020security,hashemi2022codesign,escudero2023safety} develop filtering techniques to guarantee safety, without being focused on stealthy covert attacks.

Multiplicative watermarking (mWM) has been proposed by the authors as a diagnosis technique \citep{ferrari2020switching}. mWM consists of a pair of filters on each communication channel between the plant and its controller; the scheme is affine to weak encryption, whereby ``encoding'' and ``decoding'' are done by changing signals' dynamic characteristics through inverse pairs of filters. This enables original signals to be recovered exactly, and thus does not lead to performance degradation.
% A multiplicative watermark is an affine to a weak encryption technique, through which the signal is ``encoded'' by a filter, changing its dynamic behavior. The use of inverse pairs means that the original signal can be recovered, through ``decoding'' via an inverse filter. As such, differently to techniques based on additive watermarking, no performance is lost due to the injection of noise, and there are no bandwidth limitations.

%\subsection{Contributions}
One of the critical features of multiplicative watermarking is that to detect stealthy attacks, the mWM filter parameters must be switched over time. In this paper, an algorithm to optimally design the mWM parameters after a switching event is presented, enhancing detection performance, without changing the switching time.
% This is done without changing the switching time, which is taken as given.

\textcolor{black}{
To formalize the filter design problem, we suppose the defender is interested in optimal performance against adversaries injecting covert attacks with matched system parameters \citep{smith2015covert}, including the mWM parameters prior to the switch. This scenario represents a worst case where malicious agents can take full control of the system while remaining undetected.
Thus, the attack strategy is explicitly included within the formulation of the closed-loop system, and the mWM filters are chosen by solving an optimization problem minimizing the attack-energy-constrained output-to-output gain (AEC-OOG) \citep{anand2023risk}, a variation of the output-to-output gain proposed in  \cite{teixeira2015strategic}.
}
The main contributions of this paper are:
% We consider an adversary injecting a covert attack with matched system parameters \citep{smith2015covert}, i.e., an attacker with full knowledge of the control system parameters, including those of the mWM filters before the switch. This scenario is taken as a worst case, as it has been shown that this class of attacks can be made stealthy. To quantitatively define a cost, the output-to-output gain (OOG) \citep{teixeira2015strategic} is leveraged,
% a metric introduced to evaluate the impact of an additive attack in a control system. %Specifically, OOG evaluates the worst-case performance loss that an attacker injecting an undetectable attack can obtain. 
% Here, the maximum performance loss caused by a stealthy adversary with limited energy is taken, the attack-energy-constrained OOG (AEC-OOG) \citep{anand2023risk}. The main contributions of this paper are:
\begin{enumerate}
%[label=\alph*.]
\item The problem of optimally designing the switching mWM filters is formulated as an optimization problem, with the AEC-OOG is taken as the objective;%where the AEC-OOG is taken as the impact metric; 
\item The worst-case scenario of a covert attack with exact knowledge of plant and mWM filter parameters is embedded within the design problem;
% The optimization problem is defined to incorporate the worst-case scenario of a covert attack with exact knowledge of plant and mWM filter parameters;
\item The feasibility of the optimization problem is shown to be dependent only on stability conditions; 
\item A solution scheme is proposed to promote randomization of the mWM filter parameters such that an eavesdropping adversary cannot remain stealthy.
\end{enumerate} 

This builds on the results of \cite{ferrari2020switching}, where the focus was on the design of the switching protocols, rather than the parameters themselves.
Compared to previous work \citep{gallo2021design}, this paper introduces an optimization problem which is always feasible (thanks to the use of AEC-OOG in the objective), while also considering a more sophisticated class of covert attacks, where the presence of watermark is known to the adversary. 
Moreover, this paper poses a different objective than \citep{zhang2023hybrid}; indeed, while \citep{zhang2023hybrid} provided a design strategy to ensure certain privacy properties, in this paper we address the problem of optimal parameter design following a switching event.


%\subsection{Organization}
The rest of the paper is organized as follows. 
After formulating the problem in Section~\ref{sec:PF}, we propose our design algorithm in Section~\ref{sec:main}, and analyze its properties. It is then evaluated through a numerical example in Section~\ref{sec:NE}, and concluding remarks are given Section~\ref{sec:Con}.
% We provide the problem background in Section~\ref{sec:PF}. We formulate the design problem in Section~\ref{sec:main}, together with an analysis of its properties. The proposed algorithm is evaluated through a numerical example in Section \ref{sec:NE}. Concluding remarks are offered in Section \ref{sec:Con}.
In this section, we first present the notation and the problem definition. Then we introduce the COD algorithm and its theoretical guarantees.

\subsection{Notations}
Let $\mI_n$ denote the identity matrix of size $n \times n$, and $\mathbf{0}_{m \times n}$ represent the $m \times n$ matrix filled with zeros. A matrix $\mX$ of size $m \times n$ can be expressed as $\mX = [\vx_1, \vx_2, \dots, \vx_n]$, where each $\vx_i \in \mathbb{R}^m$ is the $i$-th column of $\mX$. The notation $[\mX_1\quad \mX_2]$ represents the concatenation of matrices $\mX_1$ and $\mX_2$ along their column dimensions. For a vector $\vx \in \mathbb{R}^d$, we define its $\ell_2$-norm as $\|\vx\| = \sqrt{\sum_{i=1}^d x_i^2}$. For a matrix $\mX \in \mathbb{R}^{m \times n}$, its spectral norm is defined as $\|\mX\|_2 = \max_{\vu: \|\vu\| = 1} \|\mX \vu\|$, and its Frobenius norm is $\|\mX\|_F = \sqrt{\sum_{i=1}^n \|\vx_i\|^2}$, where $\vx_i$ is the $i$-th column of $\mX$. The condensed singular value decomposition (SVD) of $\mX$, written as SVD$(\mX)$, is given by $\mU \mSigma \mV^\top$, where $\mU \in \mathbb{R}^{m \times r}$ and $\mV \in \mathbb{R}^{n \times r}$ are orthonormal column matrices, and $\mSigma$ is a diagonal matrix containing the nonzero singular values $\sigma_1(\mX) \geq \sigma_2(\mX) \geq \dots \geq \sigma_r(\mX) > 0$. The QR decomposition of $\mX$, denoted as QR$(\mX)$, is given by $\mQ \mR$, where $\mQ \in \mathbb{R}^{m \times n}$ is an orthogonal matrix with orthonormal columns, and $\mR \in \mathbb{R}^{n \times n}$ is an upper triangular matrix. The LDL decomposition is a variant of the Cholesky decomposition that decomposes a positive semidefinite symmetric matrix \( \mX \in \mathbb{R}^{n\times n}\) into \( \mL \mD \mL^\top = \operatorname{LDL}(\mX) \), where \( \mL \) is a unit lower triangular matrix and \( \mD \) is a diagonal matrix. By defining \( \bar{\mL} = \sqrt{\mD} \mL^\top \), we obtain the triangular matrix decomposition \( \mX = \bar{\mL} \bar{\mL}^\top \).
%The columns of $\mQ$ form an orthonormal basis for the column space of $\mX$, and $\mR$ contains the coefficients that represent $\mX$ in this orthonormal basis.
% We let $\mI_n$ be the $n \times n$ identity matrix, and $\bf{0}_{m \times n}$ be the $m \times n$ matrix of all zeros. We can denote a $m \times n$ matrix as $\mX = [\vx_1, \vx_2, \dots, \vx_n]$, where $\vx_i \in \BR^{m}$ is the $i$-th column of $\mX$. We use $[ \mX_1, \mX_2 ] $ to denote their concatenation
% on their column dimensions. For a vector $\vx\in\BR^d$, we let $\norm{\vx}=\sqrt{\sum_{i=1}^d x_i^2}$ be its $\ell_2$-norm. For a matrix $\mX\in\BR^{m\times n}$, we let $\Norm{\mX} =  \max_{\vu:\Norm{\vu} = 1}\Norm{\mX \vu}$ be its spectral norm and  $\Norm{\mX}_F = \sqrt{\sum_{i = 1}^{n}{\Norm{\vx_i}^2}}$ be its Frobenius norm. The condensed singular value decomposition (SVD) of matrix $\mX$, written as SVD$(\mX)$, is defined as $\mU \mSigma \mV^T$ where $\mU \in \BR^{m \times r}$ and $\mV \in \BR^{n \times r}$ are column orthonormal and $\mSigma$ is a diagonal matrix with nonzero singular values $\sigma_1(\mX) \geq \sigma_2(\mX) \geq \dots \geq \sigma_r(\mX)>0$. We use $\textrm{nnz}(\mX)$ to denote number of nonzero elements of matrix $\mX$.

\subsection{Problem Setup}
We first provide the definition of correlation sketch as follows:
\begin{defn}[\cite{mroueh2017co}]
Let $\mX \in \mathbb{R}^{m_x \times n}$, $\mY \in \mathbb{R}^{m_y \times n}$, $\mA \in \mathbb{R}^{m_x \times \ell}$ and $\mB \in \mathbb{R}^{m_y \times \ell}$ where $n\ge\max(m_x,m_y)$ and $\ell \leq \min(m_x,m_y)$. We call the pair $(\mA,\mB)$ is an $\varepsilon$-correlation sketch of $(\mX,\mY)$ if the correlation error satisfies
\[ \text{corr-err}\left(\mX \mY^\top, \mA \mB^\top\right)\triangleq\frac{\norm{\mX\mY^\top-\mA\mB^\top}}{\Norm{\mX}_F \Norm{\mY}_F}\leq \varepsilon. \]
\end{defn}
This paper addresses the problem of approximate matrix multiplication (AMM) in the context of sliding windows. At each time step $t$, the algorithm receives column pairs $(\vx_t, \vy_t)$ from the original matrices $\mX$ and $\mY$. Let $N$ denote the window size. The submatrices within current window are denoted as $\mX_W$ and $\mY_W$. The goal of the algorithm is to maintain a pair of low-rank matrices $(\mA, \mB)$, which is an $\varepsilon$-correlation sketch of the matrices $(\mX_W, \mY_W)$. Similar to \cite{wei2016matrix}, we assume that the squared norms of the data columns are normalized to the range \([1, R]\) for both \( \mX \) and \( \mY \). Therefore, for any column pair \( (\vx, \vy) \), the condition \( 1 \leq \|\vx\| \|\vy\| \leq R \) holds.


\subsection{Co-occurring Directions}
Co-occurring directions (COD)~\cite{mroueh2017co} is a deterministic algorithm for correlation sketching. The core step of COD are summarized in Algorithm \ref{alg:cs}, which we call it the correlation shrinkage (CS) procedure.
\begin{algorithm}[t]
	\renewcommand{\algorithmicrequire}{\textbf{Input:}}
	\renewcommand{\algorithmicensure}{\textbf{Output:}}
	\caption{Correlation Shrinkage (CS)}
	\label{alg:cs}
	\begin{algorithmic}[1]
        \Require
            $\mathbf{A} \in \mathbb{R}^{m_x \times \ell'}, \mathbf{B} \in \mathbb{R}^{m_y \times \ell'}, \text{sketch size }\ell $.
        \State $[\mathbf{Q}_x, \mathbf{R}_x] \leftarrow \text{QR}(\mathbf{A})$,
         $[\mathbf{Q}_y, \mathbf{R}_y] \leftarrow \text{QR}(\mathbf{B})$.
        \State $[\mathbf{U}, \mathbf{\Sigma}, \mathbf{V}] \leftarrow \text{SVD}(\mathbf{R}_x \mathbf{R}_y^\top)$.
        \State $\mC \leftarrow \mQ_x\mU\sqrt{\mathbf{\Sigma}},\mD \leftarrow \mQ_y\mV\sqrt{\mathbf{\Sigma}}$ \Comment{$\mC$ and $\mD$ not computed.}
        \State $\delta \leftarrow \sigma_{\ell} (\mathbf{\Sigma})$,
        $\mathbf{\hat{\Sigma}} \leftarrow \text{max}(\mathbf{\Sigma} - \delta \mathbf{I}_{\ell'}, \mathbf{0})$.
        \State $\mathbf{A} \leftarrow \mathbf{Q}_x \mathbf{U} \sqrt{\mathbf{\hat{\Sigma}}}$,  $\mathbf{B} \leftarrow \mathbf{Q}_y \mathbf{V} \sqrt{\mathbf{\hat{\Sigma}}}$.
        \Ensure 
            $\mathbf{A} \text{ and } \mathbf{B}$.
	\end{algorithmic}  
\end{algorithm}

The COD algorithm initially set $\mA = \mathbf{0}_{m_x \times \ell}$ and $\mB = \mathbf{0}_{m_y \times \ell}$. Then, it processes the i-th column of X and Y as follows
\begin{flalign}
    &\text{Insert $\vx_i$ into a zero valued column of $\mA$} \nonumber\\
    &\text{Insert $\vy_i$ into a zero valued column of $\mB$} \nonumber\\
    &\text{\textbf{if} $\mA$ or $\mB$ has no zero valued columns \textbf{then}} \nonumber\\
    &\text{\quad\quad$[\mA,\mB] = \operatorname{CS}(\mA,\mB, \ell/2)$} \nonumber
\end{flalign}


The COD algorithm runs in $O(n(m_x + m_y)\ell)$ time and requires a space of $O((m_x+m_y)\ell)$. It returns the final sketch $\mA\mB^\top$ with correlation error bounded as:
\[
\norm{\mX\mY^\top-\mA\mB^\top} \leq \frac{2}{\ell}\|\mX\|_F\|\mY\|_F.
\]

For the convenience of expression, we present the following definition:
\begin{defn}
% We call matrix pair $(\mC,\mD)$ is an aligned pair of $(\mA,\mB)$ if it satisfies $\mC\mD^\top=\mA\mB^\top$, $\mC=\bar{\mU}\bar{\mSigma}$ and $\mD=\bar{\mV}\bar{\mSigma}$, where $\bar{\mU}$ and $\bar{\mV}$ are orthonormal matrices and $\bar{\mSigma}$ is a diagonal matrix with descending diagonal elements.
We call matrix pair $(\mC,\mD)$ is an aligned pair of $(\mA,\mB)$ if it satisfies $\mC\mD^\top=\mA\mB^\top$, $\mC=\mQ_x\mU\sqrt{\mathbf{\Sigma}}$ and $\mD=\mQ_y\mV\sqrt{\mathbf{\Sigma}}$, where $\mQ_x$, $\mQ_y$, $\mU$ and $\mV$ are orthonormal matrices and $\mathbf{\Sigma}$ is a diagonal matrix with descending diagonal elements.
\end{defn}
Notice that the line 1-3 of the CS procedure generate an aligned pair $(\mC,\mD)$ for $(\mA,\mB)$. In addition, The output of CS algorithm is a shrinked variant of the aligned pair.

%Here we define a new concept: reconstructed multipliers. With the symbols used in the CS algorithm, we have $\mA\mB^\top=\mQ_x\mU\mathbf{\Sigma}\mV^\top\mQ_y^\top = \mC\mD^\top$. We refer to the pair of matrices $\mC = \mQ_x\mU\sqrt{\mathbf{\Sigma}}$ and $\mD = \mQ_y\mV\sqrt{\mathbf{\Sigma}}$ as the reconstructed multipliers of $\mA\mB^\top$. The columns of $\mC$ and $\mD$ are orthogonal and sorted in descending order of their norms ($\|\vc_i\|=\|\vd_i\| \geq \|\vc_{i+1}\|= \|\vd_{i+1}\|$). The output of CS algorithm is a shrinked variant of the reconstructed multipliers. From a global perspective, for our target $\mX\mY^\top = \mQ_x \mU \mathbf{\Sigma} \mV^\top \mQ_y^\top$,  
% (where $[\mQ_x, \mR_x] = \operatorname{QR}(\mX)$, $[\mQ_y, \mR_y] = \operatorname{QR}(\mY)$, and $[\mU, \mathbf{\Sigma}, \mV] = \operatorname{SVD}(\mR_x \mR_y^\top)$),
%the approximate result provided by COD is $\mA\mB^\top = \mQ_x \mU \hat{\mathbf{\Sigma}} \mV^\top \mQ_y^\top$. We refer to $\mU \mathbf{\Sigma} \mV^\top$ as the product core of $\mX\mY^\top$, and correspondingly, $\mU \hat{\mathbf{\Sigma}} \mV^\top$ as the product core of $\mA\mB^\top$.
\section{Methodology}
In this section, we outline the key research questions driving this study, followed by a detailed description of the methodology used to design and conduct the survey.
\subsection{Research Questions}
\begin{enumerate}
    \item[\textbf{RQ1:}] How do developers allocate their time during a typical workweek, and how does this compare to their perception of an \textbf{ideal workweek?}
    \item[\textbf{RQ2:}] How are developer's satisfaction and productivity affected by \textbf{deviations} from their ideal workweek?
     \item[\textbf{RQ3:}] For which tasks do developers prefer using \textbf{AI tools}, and how does the frequency of AI tool usage \textbf{influence} their satisfaction and productivity?
\end{enumerate}

\subsection{Survey Design}
% Describe how the survey was conducted, survey structure, sample size, which activities were selected and how, incentives, etc. 

To gain insights into the types of activities developers engage in during a typical work week, we conducted a series of exploratory interviews with 12 randomly selected participants. These semi-structured interviews provided a qualitative foundation, allowing us to iteratively develop a comprehensive list of higher-level activities that reflect both ideal and actual workweek allocations. The findings from these interviews were instrumental in refining our survey questions and design.

% - When was it distributed
% - How many people were invited
% - how was the survey advertised
% - incentive provided to participants
% - how many responses received (with response rates)
% - Board of ethics description \& instruments
% - Describe the main questions asked in the survey

The survey was distributed in \textcolor{blue}{May 2024} to software engineers working in Microsoft teams across India and the United States. A total of 6000 developers were invited to participate via email. Framed as a study aimed at boosting developer productivity by understanding how they allocate their time in a workday, the survey received 510 complete responses (responses rate of 8.5\%). After finishing the survey, the participants could enter a sweepstake to win one out of ten \$50 Amazon.com Gift Cards.
\textcolor{blue}{description of ethics}.

The main questions in the survey were as follows:
\begin{enumerate}
    \item Their roles and years of experience in the industry/team
    \item The hours spent on various activities in their typical workweek
    \item Ideally, the percentage of time they would want to allocate to each activity in a workweek
    \item How productive and satisfied were they by their past workweek
    \item Activities they find most cognitively challenging
    \item How often do they use AI tools to assist in their daily activities
    \item Two open-ended questions about the activities they would want to automate using AI tools, and advice for new hires to boost their productivity and satisfaction levels 
\end{enumerate}



\subsection{Data Analysis \& Exploration}
% Here, we could start with discussing the survey group:
% - demographic observations
% - distribution of participants (based on the years experience in the industry/team), 

From the exploratory interviews, we identified sixteen key activities, which were subsequently used to quantify the developers' time allocation across their work week. 

\subsection{Limitations}
In this section, we empirically compare the proposed algorithm on both sequence windows and time windows with existing methods.
\paragraph{Datasets} For the sequence-based model, we used two synthetic datasets and two cross-language datasets. The statistics of the datasets are provided in Table \ref{table:statistics}:

\begin{table}[t]
    \centering
    \caption{The statistics of the datasets. The datasets satisfy $1 \leq \|\vx\|\|\vy\| \leq R $.}
    \label{table:statistics}
    \begin{tabular}{|c|c|c|c|c|c|}
    \hline
        Dataset & $n$ & $m_x$ & $m_y$ & $N$ & $R$ \\ \hline
        SYNTHETIC(1) & 100,000 & 1,000 & 2,000 & 50,000 & 65 \\ \hline
        SYNTHETIC(2) & 100,000 & 1,000 & 2,000 & 50,000 & 724 \\ \hline
        APR & 23,235 & 28,017 & 42,833 & 10,000 & 773 \\ \hline
        PAN11 & 88,977 & 5,121 & 9,959 & 10,000 & 5,548 \\ \hline
        EURO & 475,834 & 7,247 & 8,768 & 100,000 & 107,840 \\ \hline
    \end{tabular}
\end{table}

\begin{itemize}
    \item Synthetic: The elements of the two synthetic datasets are initially uniformly sampled from the range (0,1), then multiplied by a coefficient to adjust the maximum column squared norm $R$. The X matrix has 1,000 rows, and the Y matrix has 2,000 rows, each with 100,000 columns. The window size is set to 50,000.
    \item APR: The Amazon Product Reviews (APR) dataset is a publicly available collection containing product reviews and related information from the Amazon website. This dataset consists of millions of sentences in both English and French. We structured it into a review matrix where the X matrix has 28,017 rows, and the Y matrix has 42,833 rows, with both matrices sharing 23,235 columns. The window size is 10,000.
    \item PAN11: PANPC-11 (PAN11) is a dataset designed for text analysis, particularly for tasks such as plagiarism detection, author identification, and near-duplicate detection. The dataset includes texts in English and French. The X and Y matrices contain 5,121 and 9,959 rows, respectively, with both matrices having 88,977 columns. The window size is 10,000.
\end{itemize}
We evaluate the time-based model on another real-world dataset:
\begin{itemize}
    \item EURO: The Europarl (EURO) dataset is a widely used multilingual parallel corpus, comprising the proceedings of the European Parliament. We selected a subset of its English and French text portions. The X and Y matrices contain 7,247 and 8,768 rows, respectively, and both matrices share 475,834 columns. Timestamps are generated using the $Poisson$ $Arrival$ $Process$ with a rate parameter of $\lambda=2$. The window size is set to 100,000, with approximately 30,000 columns of data on average in each window.
\end{itemize}

\paragraph{Setup} For the sequence-based model, we compare the proposed hDS-COD and  aDS-COD with EH-COD~\cite{yao2024approximate} and DI-COD~\cite{yao2024approximate}. We do not consider the Sampling algorithm as a baseline, as its performance is inferior to that of EH-COD and DI-CID, as demonstrated in \cite{yao2024approximate}. %The hDS-COD is adjusted by the parameter $\ell$ and the maximum number of levels $L = \log{R}$, where $R$ is the prior estimate of the maximum squared column norm of the dataset. DI-COD similarly requires a prior estimate of $R$ to limit the maximum number of levels $L = \log{(R/\varepsilon})$. In contrast, aDS-COD and EH-COD do not require an estimate of $R$; their error-space balance is controlled by the parameter $\ell = \frac{1}{\varepsilon}$. 
For the time-based model, we compare the proposed hDS-COD and  aDS-COD with EH-COD and the Sampling algorithm since DI-COD cannot be applied to time-based sliding window model. To achieve the same error bound, the maximum number of levels for hDS-COD is set to $L = \log{(\varepsilon NR)}$, and the initial threshold for aDS-COD is set to $1$.

Our experiments aim to illustrate the trade-offs between space and approximation errors. The x-axis represents two metrics for space: final sketch size and total space cost. The final sketch size refers to the number of columns in the result sketches $\mA$ and $\mB$ generated by the algorithm, representing a compression ratio. The total space cost refers to the maximum space required during the algorithm's execution, measured by the number of columns.We evaluate the approximation performance of all algorithms based on correlation errors $\operatorname{corr-err}(\mathbf{X}_W \mathbf{Y}_W^\top, \mathbf{A} \mathbf{B}^\top)$, which is reflected on the y-axis. Every 1,000 iterations, all algorithms query the window and record the average and maximum errors across all sampled windows.

The experiments for all algorithms were conducted using MATLAB (R2023a), with all algorithms running on a Windows server equipped with 32GB of memory and a single processor of Intel i9-13900K.

\paragraph{Performance} Figure \ref{fig:error vs l} and Figure \ref{fig:error vs space} illustrate the space efficiency comparison of the algorithms on sequence-based datasets. Panels (a-d) show the average errors across all sampled windows, while panels (e-h) display the maximum errors.

Figure \ref{fig:error vs l} evaluates the compression effect of the final sketch. The hDS-COD, aDS-COD, and EH-COD show similar compression performances. But the DS series is more stable, particularly on the synthetic datasets, where they significantly outperform EH-COD and DI-COD. The performance of hDS-COD and aDS-COD is nearly the same, indicating that the adaptive threshold trick in aDS-COD does not have a noticeable negative impact on it, maintaining the same error as hDS-COD.

Figure \ref{fig:error vs space} measures the total space cost of the algorithms. hDS-COD and aDS-COD show a significant advantage over existing methods, as they can achieve the  $\varepsilon$-approximation error with much less space. For the same space cost, the correlation errors of hDS-COD and aDS-COD are much smaller than those of EH-COD and DI-COD. Also, aDS-COD has better space efficiency than hDS-COD because aDS only uses a single-level structure while hDS requires $\log R+1$ levels. We find that hDS-COD requires more space on  SYNTHETIC(2) dataset compared to SYNTHETIC(1) dataset. This phenomenon occurs because SYNTHETIC(2) dataset has a larger $R$, which confirms the dependence on $R$ as stated in Theorem~\ref{thm:hds}. 

Figure \ref{fig:time-based} compares the performance of algorithms on time-based windows. Panels (a) and (b) present the error against the final sketch size, which show that our aDS-COD and hDS-COD algorithms enjoy similar performance as EH-COD and significantly outperform the sampling algorithm. On the other hand, as shown in panels (c) and (d), our methods outperform baselines in terms of total space cost.

\section{Discussion and Future Work}\label{sec:discussion}
This paper pioneers the novel approach of selective response, showing that withholding responses can be a powerful tool for GenAI systems. By opting not to answer every query as accurately as it can---particularly when new or complex topics emerge---GenAI can encourage user participation on community-driven platforms and thereby generate more high-quality data for future training. This mechanism ultimately enhances GenAI's long-term performance and revenue. From a welfare perspective, our results indicate that such selective engagement can also benefit users, leading to better solutions and increased overall satisfaction. Since this work is the first to address selective response strategies for GenAI, numerous promising directions remain for future research; we highlight some of them below. 

First, from a technical standpoint, all of the results in this paper rely on Assumption~\ref{assumption: data lip}, involving the lipshitz condition of the accuracy function and the sensitivity parameter $\beta$. Future work could seek to relax this assumption. Furthermore, our constrained optimization approach in Subsection~\ref{sec: welfare constrained revenue maximization} could be extended to approximate the optimal (continuous) strategy instead of the optimal discrete strategy.

Second, our stylized model adopts the simplifying---though unrealistic---assumption that only a single GenAI platform exists. Admittedly, this makes it easier to focus on the idea of selective responses, and indeed, this assumption is pivotal in keeping our analysis tractable. Future research could explore scenarios with multiple GenAI platforms and human-centered forums. In such settings, one platform's selective response might redirect users not only to forums but also to competing GenAI platforms, leading to the tragedy of the commons \cite{hardin1968tragedy}: Although all GenAI platforms benefit from fresh data generation, none may choose to respond selectively if it means losing users to competitors. 

Third, we assumed Forum behaves non-strategically. In reality, human-centered platforms often monetize their data by selling it to GenAI platforms, adding a further layer of strategic interaction for GenAI. Moreover, data transfer between the platforms can form the basis for collaboration: GenAI could employ selective response to bolster Forum content creation, and Forum could, in turn, attribute that content to GenAI for subsequent use in retraining.


%Third, we make the (again) simplifying assumption that Forum is non-strategic. However, in practice, human-centered platforms can sell their data to GenAI platforms. This adds additional considerations for GenAI. Furthermore, data transmission between the platforms can also become the basis for collaboration: GenAI can use selective response to ensure enough content is generated in Forum, and Forum could provide the data attributed to this mechanism back to GenAI. 


%Second, this paper makes the simplifying yet unrealistic assumption of the existence of one GenAI platform. Indeed, this simplifies many aspects and allows us to analyze selective responses. Future work could address the data generation process with more than one GenAI platform and possibly several human-centered forums. In such a case, selective response of one GenAI platform can either drive users to forums or to other GenAI platforms; thus, we might face a tragedy of the commons situation~\ref{hardin1968tragedy}, where all GenAI platforms are interested in fresh data generation but none volunteer to selectively respond and lose users. 

%This paper examines the competition between a generative AI platform and human-based platforms, challenging the assumption that always providing answers is optimal. We analyzed the impact of withholding answers on GenAI's revenue and developed an efficient approximately optimal algorithm for this purpose. We further explored how withholding affects users, showing that it can lead to better outcomes compared to always answering. Specifically, we demonstrated that withholding can Pareto-dominate this strategy and derived the necessary and sufficient conditions for that. Finally, we proposed a second approximately optimal algorithm that maximizes GenAI's revenue while ensuring users are better off than when GenAI answers all queries.

%On a more conceptual level, our model assumes that GenAI’s data comes solely from the competing platform (Forum). Future research could explore a scenario where GenAI can purchase additional data from a third party. This extension could provide valuable insights into the interplay between withholding answers and data purchasing, and whether these two strategies can complement each other or must be traded off.
Software development is increasingly conceived as a collaboration activity between developers and AIs. Indeed, IDEs already implement features to enable interactive development, with AI suggesting implementations that are reused by developers.

Although multiple studies show this interaction can be successful, there is still limited understanding of how the models must be configured and used in the context of code generation tasks. This study addresses this gap, systematically investigating the impact of several key parameters, including the repeated submission of a prompt to accommodate for the non-deterministic nature of the models.

Our study reveals several key findings about the usage of ChatGPT. In particular, we discovered how creativity, although up to a limited extent, is useful to increase the range of methods whose code can be generated correctly. A major role is played by parameter top-p, which is commonly underrated, and instead has a major impact on the correctness of the results, with lower values producing better results. Finally, prompts should be submitted multiple times, with $5$ repetitions combined with a temperature of $1.2$ resulting in an effective configuration in our experiments.  

Future work concerns two main research directions. One is about replicating this experiment with other AI assistants, to validate our findings in multiple contexts. The second research direction concerns finding strategies to deal with the need to submit the same prompt multiple times to obtain a useful result, and thus developing approaches able to select or merge multiple responses automatically. 

% \begin{figure}[htbp]
%   \centering
%   \includegraphics[width=\linewidth]{ADDRESS}
%   \caption{CAPTION}
%   \label{fig:name}
% \end{figure}










\nocite{*}
\bibliographystyle{IEEEtran}
\bibliography{ref}
\end{document}
