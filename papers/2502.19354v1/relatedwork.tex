\section{Related Work}
\label{sec:related_work}
Localization algorithms have been extensively studied in the literature, with numerous efforts dedicated to developing low-complexity methods that enhance accuracy while maintaining computational efficiency. Specifically, the work in~\cite{ZHUO2024} introduced a low-complexity optimization technique for localization using successive greedy grid search and successive cancellation-list grid search to address a measurement-to-target association problem, while~\cite{5464262} provides a framework for a low-complexity least-squares algorithm that demonstrates improved accuracy; however, their performance remains comparable to NLS and falls short of more advanced approaches such as WNLS and optimization-based methods. Furthermore, in ~\cite{8947090_liu_weight_ippm, 8979252_liu_covar_ippm}, the authors introduced approaches for weighted PPM to enable enhanced cooperative localization, where the weights are determined using the gradient descent technique and the covariance of TOA measurements respectively. While these methods demonstrated promising results for TOA-based localization, they cannot be directly applied to TDOA-based scenarios. Further, they cannot be used if the TOA values are pseudo-range measurements, meaning they inherently include clock offsets and bias terms. In scenarios where there is no prior information on transmission time and bias offsets, such as clock offsets and non-line-of-sight (NLOS) bias, TOA values cannot be directly treated as noisy range measurements. Addressing these challenges, the authors in~\cite{10632801_harish_dyspan} proposed a reformulated version of IPPM to extend its applicability to TDOA-based localization. However, it is not straightforward to introduce suitable weighting mechanisms in PPMs to prevent overemphasizing or under-emphasizing specific TDOA measurements, particularly when certain anchor-to-UE links experience low SNR~\cite{dureppagari2025icc}. To overcome these limitations and achieve a balance between improved accuracy and computational feasibility, the current paper introduces a novel TS-WPM for robust cooperative localization. The proposed approach significantly enhances positioning accuracy, outperforming state-of-the-art approaches, including IPPM, NLS, and WNLS, while maintaining efficiency suitable for practical deployment. 
%Addressing these challenges, the authors in~\cite{dureppagari2025algo} proposed a reformulated version of IPPM to extend its applicability to TDOA-based localization. However, they also highlighted the difficulty of introducing suitable weighting mechanisms in PPMs to prevent overemphasizing or under-emphasizing specific TDOA measurements, particularly when certain anchor-to-UE links experience low SNR. To overcome these limitations and achieve a balance between improved accuracy and computational feasibility, this paper introduces a novel two-stage weighted projection method (TS-WPM) for robust cooperative localization. The proposed approach significantly enhances positioning accuracy, outperforming state-of-the-art approaches, including IPPM, NLS, and WNLS, while maintaining efficiency suitable for practical deployment.