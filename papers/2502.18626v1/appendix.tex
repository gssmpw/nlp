\appendix

\section{Moment bounds for Gaussian random vectors and matrices}

The goal of this section is to establish moment bounds for $\lVert \mtx{A} \mtx{\Omega} \rVert _2^2$ with a fixed matrix $\mtx{A}$ and a Gaussian random matrix $\mtx{\Omega}$. These bounds are needed in the proof of \reflem{lem:nystrom}, but may also be of independent interest.
The corresponding result for the Frobenius norm can be found in~\cite[Lemma 3]{kressner-2024-randomized-lowrank}. For the spectral norm, moment bounds for $\lVert \mtx{\Omega} \rVert _2^2$ (that is, $\mtx{A} = \mtx{I}$) are well established\cite{chen-2005-condition-numbers, edelman-1988-eigenvalues-condition, james-1964-distributions-matrix}. In \cite[Lemma B.1]{tropp-2023-randomized-algorithms} bounds on the first- and second-order moments for general $\mtx{A}$ are derived. In the following, we will generalize this result to moments of arbitrarily high order. To do so, we first establish two preliminary results.

\begin{lemma}{Moment bound of chi-squared random variable}{gamma}
    Let $X \sim \chi_k^2$, where $\chi_k^2$ denotes the chi-squared distribution with $k \ge 2$ degrees of freedom. Then 
    $
        \mathbb{E}^{{p}}[X] \leq k + p-1
    $ holds for every $p \ge 1$, $p\in \mathbb R$.
\end{lemma}%
\begin{proof}
It is well known that
    $
        \mathbb{E}^{p}[X] = 2 \Big( \frac{\Gamma(k/2+p)}{\Gamma(k/2)} \Big)^{\sfrac{1}{p}}.
    $ For every $\alpha \ge 0$, $\beta \ge 2$ the bound $\frac{\Gamma(\alpha+\beta)}{\Gamma(\alpha + 1)} \le (\alpha+\beta/2)^{\beta-1}$ holds~\cite[Equation 2.2]{laforgia-1984-further-inequalities}. The result of the lemma follows by using this bound with $\alpha = k/2-1$ and $\beta = p+1$.
\end{proof}

\begin{lemma}{Spectral norm moments of Gaussian random vector}{spectral-norm-moment-vector}
    Given $\mtx{A} \in \mathbb{R}^{m \times m}$ and a Gaussian random vector $\vct{\omega} \in \mathbb{R}^{m}$, the bound
    \begin{equation*}
        \mathbb{E}^{p}\big[ \lVert \mtx{A} \vct{\omega} \rVert _2^2 \big]
        \leq  (k + p - 1) \Big( \lVert \mtx{A} \rVert _2^2 + \frac{1}{k} \lVert \mtx{A} \rVert _F^2 \Big).        
    \end{equation*}
    holds for every $k \ge 2$, $k\in \mathbb{N}$, and $p \ge 1$, $p\in \mathbb R$.
\end{lemma}%
\begin{proof}
By the unitary invariance of Gaussian random vectors, we may assume w.l.o.g. that $\mtx{A} = \mtx{\Sigma} = \operatorname{diag}(\sigma_1, \dots, \sigma_m)$ with $\sigma_1 \geq \dots \geq \sigma_m \geq 0$.
Following the proof of \cite[Theorem 1]{cohen-2016-optimal-approximate}, we split the singular values into $\ell = \lceil m/k \rceil$ groups of size $k$:
    \begin{equation*}
        \overbrace{\underbrace{\sigma_1, \dots, \sigma_k}_{\leq \sigma_1}}^{\geq \sigma_{k+1}}, \overbrace{\underbrace{\sigma_{k+1}, \dots, \sigma_{2k}}_{\leq \sigma_{k+1}}}^{\geq \sigma_{2k+1}}, \dots, \overbrace{\underbrace{\sigma_{(\ell - 1)k + 1}, \dots, \sigma_{\ell k}}_{\leq \sigma_{(\ell - 1)k + 1}}}^{\geq 0}.
    \end{equation*}
    If $m$ is not a multiple of $k$, we set $\sigma_i = 0$ for $i > m$. Using that
    $\sigma_{ik + 1}^2 \leq ( \sigma_{(i-1)k + 1} + \cdots + \sigma_{ik - 1} + \sigma_{ik} ) / k$ for $i = 1,\ldots, \ell-1$, we get
    \begin{equation}
        \sum_{i=0}^{\ell-1} \sigma_{ik + 1}^2 = \lVert \mtx{\Sigma} \rVert _2^2 + \sum_{i=1}^{\ell-1} \sigma_{ik + 1}^2  \leq \lVert \mtx{\Sigma} \rVert _2^2 + \frac{1}{k} \sum_{j=1}^{(\ell - 1)k} \sigma_j^2 \leq \lVert \mtx{\Sigma} \rVert _2^2 + \frac{1}{k} \lVert \mtx{\Sigma} \rVert _F^2.
        \label{equ:singular-value-group-bound}
    \end{equation}
    This allows us to bound
    \begin{align*}
    \mathbb{E}^{p}\big[ \lVert \mtx{\Sigma} \vct{\omega} \rVert _2^2 \big] & = 
        \mathbb{E}^{p}\Big[ \sum_{i=1}^{m} \sigma_i^2 \omega_i^2 \Big]
        = \mathbb{E}^{p}\Big[ \sum_{i=0}^{\ell - 1} \sum_{j=1}^{k} \sigma_{ik + j}^2 \omega_{ik + j}^2 \Big] \\
        &\leq \mathbb{E}^{p}\Big[ \sum_{i=0}^{\ell - 1} \sigma_{ik + 1}^2 \sum_{j=1}^{k} \omega_{ik + j}^2 \Big] 
        \leq \sum_{i=0}^{\ell - 1} \sigma_{ik + 1}^2 ~ \mathbb{E}^{p}\Big[ \sum_{j=1}^{k} \omega_{ik + j}^2 \Big] \\
        &\leq (k + p - 1) \sum_{i=0}^{\ell - 1} \sigma_{ik + 1}^2 \leq (k + p - 1) \Big( \lVert \mtx{\Sigma} \rVert _2^2 + \frac{1}{k} \lVert \mtx{\Sigma} \rVert _F^2 \Big), 
    \end{align*}
    where we used the triangle inequality, \reflem{lem:gamma}, and \refequ{equ:singular-value-group-bound} for the last three inequalities.
\end{proof}

\begin{lemma}{Spectral norm moments of Gaussian random matrix}{spectral-norm-moment}
    Given $\mtx{A} \in \mathbb{R}^{m \times m}$ and a Gaussian random matrix $\mtx{\Omega} \in \mathbb{R}^{m \times k}$, the bound
    \begin{equation*}
        \mathbb{E}^{p}\left[ \lVert \mtx{A} \mtx{\Omega} \rVert _2^2 \right]
        \leq  (k + p-1) \Big( 2 \lVert \mtx{A} \rVert _2^2 + \frac{1}{k} \lVert \mtx{A} \rVert _F^2 \Big).
    \end{equation*}
    holds for every $k \ge 2$, $k\in \mathbb{N}$, and $p \ge 1$, $p\in \mathbb R$.
\end{lemma}%
\begin{proof}
    By the proof of \cite[Lemma B.1]{tropp-2023-randomized-algorithms}, we have that
    \begin{equation*}
        \mathbb{E}^{p}\big[ \lVert \mtx{A} \mtx{\Omega} \rVert _2^2 \big]
        \leq \lVert \mtx{A} \rVert _2^2 \mathbb{E}^{p}\big[ \lVert \vct{\omega}_1 \rVert _2^2 \big] + \mathbb{E}^{p}\big[ \lVert \mtx{A} \vct{\omega}_2 \rVert _2^2 \big]
    \end{equation*}
    for Gaussian random vectors $\vct{\omega}_1 \in \mathbb{R}^{k}$ and $\vct{\omega}_2 \in \mathbb{R}^{m}$. The claimed result follows from applying \reflem{lem:gamma} and \reflem{lem:spectral-norm-moment-vector}:
    \begin{equation*}
        \mathbb{E}^{p}\big[ \lVert \mtx{A} \mtx{\Omega} \rVert _2^2 \big]
        \leq (k + p - 1) \lVert \mtx{A} \rVert _2^2  + (k + p - 1) \Big( \lVert \mtx{A} \rVert _2^2 + \frac{1}{k} \lVert \mtx{A} \rVert _F^2 \Big).
    \end{equation*}
\end{proof}


In passing, we note that \reflem{lem:spectral-norm-moment}  yields the bound
    \begin{equation*}
        \mathbb{E}^{p}\big[ \lVert \mtx{A} \mtx{\Omega} \rVert _2 \big] = \sqrt{\mathbb{E}^{\sfrac{p}{2}}\big[ \lVert \mtx{A} \mtx{\Omega} \rVert _2^2 \big]} \leq \sqrt{k + p/2-1} \cdot \Big(\sqrt{2} \lVert \mtx{A} \rVert _2 + \frac{1}{\sqrt{k}}\lVert \mtx{A} \rVert_F\Big)
    \end{equation*}
    with $k\ge 2, p\ge 2$. For $p = 2$, this nearly matches the corresponding bound from~\cite[Lemma B.1]{tropp-2023-randomized-algorithms}, except for the additional factor $\sqrt{2}$.

\section{Moment bounds for sub-gamma random variables}

A real centered random variable $X$ is called sub-gamma with parameters $(\nu,c)$ if its moment generating function satisfies
\begin{equation*}
 \mathbb E[ \exp( tX ) ] \le \exp\left( \frac{ t^2 \nu}{2(1-c|t|)} \right) \ \text{for every} \ 0 < |t| < 1/c.
\end{equation*}
Any such $X$ satisfies the following tail bound~\cite[P. 29]{boucheron-2013-concentration-inequalities}:
\begin{equation}
  \mathbb P\big(|X| > \sqrt{2\nu t} + ct\big) \le 2 e^{-t} \ \text{for every} \ t>0.
  \label{equ:subgammatail}
\end{equation}

\begin{lemma}{Moment bounds for sub-gamma random variables}{sub-gamma-moments}
    Let $X$ be a centered, $(\nu,c)$-sub-gamma random variable for $\nu,c>0$. Then
    \begin{equation*}
        \mathbb{E}^p[X] \leq 2 \sqrt{2 \nu p} + 4 c p
    \end{equation*}
    holds for any $p \geq 1$, $p \in \mathbb{R}$.
\end{lemma}
\begin{proof}
The statement follows from a straightforward extension of the proof of~\cite[Theorem 2.3]{boucheron-2013-concentration-inequalities} from  even integers $p$ to general $p$. We include the detailed argument for completeness. Using integration by parts, the reparametrization $x = \sqrt{2\nu t} + ct$, and~\refequ{equ:subgammatail}, one gets
\begin{align*}
 \mathbb{E}\big[|X|^p\big] &= p \int_0^\infty  |x|^{p-1} \mathbb P\big(|X| > x\big) \,\mathrm{d}x \\
 &=  p \int_0^\infty  (\sqrt{2\nu t} + ct)^{p-1} \mathbb P\big(|X| > \sqrt{2\nu t} + ct \big) \frac{ \sqrt{2\nu t} + 2ct}{2t} \,\mathrm{d}t \\
 &\le p  \int_0^\infty  (\sqrt{2\nu t} + 2ct)^{p}  \frac{e^{-t}}{t} \,\mathrm{d}t.
\end{align*}
Using the inequality $( (a+b)/2 )^p \le (a^p + b^p)/2$ implied by the convexity of $x^p$ on $[0,\infty)$,
we have that
\begin{align*}
 \mathbb{E}\big[|X|^p\big] & \le 
 p 2^{p-1}   \int_0^\infty  \big( (2\nu t)^{\sfrac{p}{2}} + (2ct)^{p}\big)  \frac{e^{-t}}{t} \,\mathrm{d}t \\
 &= p 2^{p-1} \big( (2\nu)^{\sfrac{p}{2}} \Gamma(p/2) + (2c)^p  \Gamma(p) \big) \\
 &= 2^{p} (2\nu)^{\sfrac{p}{2}} \Gamma(p/2+1) + 2^{p-1} (2c)^p  \Gamma(p+1).
\end{align*}
Using $(a+b)^{\sfrac{1}{p}} \le a^{\sfrac{1}{p}} + b^{\sfrac{1}{p}}$, we obtain

\begin{equation*}
\mathbb{E}^p[X] \leq 2 \sqrt{2\nu} \Gamma(p/2+1)^{\sfrac{1}{p}}  + 4 c \Gamma(p+1)^{\sfrac{1}{p}}.
\end{equation*}
Combined with $\Gamma(p/2+1)^{\sfrac{1}{p}} \le \sqrt{p}$ and $\Gamma(p+1)^{\sfrac{1}{p}} \le p$, this completes the proof.
\end{proof}





