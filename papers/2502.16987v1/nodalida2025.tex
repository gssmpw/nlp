%
% File nodalida2025.tex
%
% Contact:  Sara Stymne
% Email:    sara.stymne@lingfil.uu.se
%
% Based on the instruction file for NoDaLiDa 2023 by Mark Fishel which in turn were
% Based on the instruction file for NoDaLiDa 2021 by Lilja Øvrelid which in turn were
% Based on the instruction file for NoDaLiDa 2019 by Barbara Plank and Mareike Hartmann which in turn were based on the instruction files from NoDaLiDa 2017 and 2015 by
% Beata Megyesi (beata.megyesi@lingfil.uu.se) and EACL 2014
% which in turn was based on the instruction files for previous 
% ACL and EACL conferences. The BibTeX file is based on NAACL 2019
% style files, which in turn are based on style files for ACL 2018 and NAACL 2018, which were
% Based on the style files for ACL-2015, with some improvements
%  taken from the NAACL-2016 style
% Based on the style files for ACL-2014, which were, in turn,
% based on ACL-2013, ACL-2012, ACL-2011, ACL-2010, ACL-IJCNLP-2009,
% EACL-2009, IJCNLP-2008...
% Based on the style files for EACL 2006 by 
% e.agirre@ehu.es or Sergi.Balari@uab.es
% and that of ACL 08 by Joakim Nivre and Noah Smith

\documentclass[11pt]{article}
\usepackage{nodalida2025}
\usepackage{times}
\usepackage{url}
\usepackage{latexsym}

%\aclfinalcopy % Uncomment this line for the final submission

\title{Instructions for NoDaLiDa/Baltic-HLT 2025 Proceedings}

\author{Anonymous Author \\
  Affiliation / Address line 1 \\
  Affiliation / Address line 2 \\
  Affiliation / Address line 3 \\
  {\tt email@domain} \\\And
  Anonymouser Author \\
  Affiliation / Address line 1 \\
  Affiliation / Address line 2 \\
  Affiliation / Address line 3 \\
  {\tt email@domain} \\\And
  Anonymousest Author \\
  Affiliation / Address line 1 \\
  Affiliation / Address line 2 \\
  Affiliation / Address line 3 \\
  {\tt email@domain} \\}

\date{}

\begin{document}
\maketitle
\begin{abstract}
  This document contains the instructions for preparing a camera-ready
  manuscript for the proceedings of NoDaLiDa/Batlic-HLT 2025, which is also 
  accessible directly on Overleaf. The document itself
  conforms to its own specifications and is therefore an example of
  what your manuscript should look like. These instructions should be
  used for both papers submitted for review and for final versions of
  accepted papers.  Authors are asked to conform to all the directions
  reported in this document.
\end{abstract}



\section{Credits}

This document has been adapted from the instructions for earlier
NoDaLiDa proceedings by Mark Fishel (NoDaLiDa 2025), Lilja Øvrelid (NoDaLiDa 2021), Barbara Plank and Mareike Hartmann (NoDaLiDa 2019), Be\'{a}ta Megyesi (NoDaLiDa 2017 and 2015), which are
based on (E)ACL-proceedings, including those for earlier ACL, NAACL and EACL proceedings,
including 
those for ACL 2018 by Shay Cohen, Kevin Gimpel, and Wei Lu, 
NAACL 2018 by Margaret Michell and Stephanie Lukin,
2017/2018 (NA)ACL bibtex suggestions from Jason Eisner,
ACL 2017 by Dan Gildea and Min-Yen Kan, 
NAACL 2017 by Margaret Mitchell, 
ACL 2012 by Maggie Li and Michael White, 
those from ACL 2010 by Jing-Shing Chang and Philipp Koehn, 
those for ACL 2008 by JohannaD. Moore, Simone Teufel, James Allan, and Sadaoki Furui, 
those for ACL 2005 by Hwee Tou Ng and Kemal Oflazer, 
those for ACL 2002 by Eugene Charniak and Dekang Lin, 
and earlier EACL formats, such as those for
EACL-2014 by Gosse Bouma
and Yannick Parmentier, those for ACL-2013 by Roberto Navigli
and Jing-Shing Chang, those for ACL-2012 by Maggie Li and Michael
White, those for ACL-2010 by Jing-Shing Chang and Philipp Koehn,
those for ACL-2008 by Joakim Nivre and Noah Smith, 
those for EACL-2006 by Eneko Agirre and Sergi Balari, 
those for ACL-2005 by Hwee Tou Ng and Kemal Oflazer, those for
ACL-2002 by Eugene Charniak and Dekang Lin, and earlier ACL and EACL
formats. Those versions were written by several people, including John
Chen, Henry S. Thompson and Donald Walker. Additional elements were
taken from the formatting instructions of the {\em International Joint
  Conference on Artificial Intelligence}.

\section{Introduction}
% das: removed reference to PostScript

\begin{figure*}
    \centering
    \begin{scriptsize}
         \url{https://www.overleaf.com/latex/templates/instructions-for-nodalida-baltic-hlt-2025-submissions/vntbxsmmzyqj} \\
    \caption{URL to OverLeaf template}
    \label{fig:overleaflink}
    \end{scriptsize}
\end{figure*}

The following instructions are directed to authors of papers submitted
to NoDaLiDa/Batlic-HLT 2025 or accepted for publication in its proceedings. 
All submissions must conform to the NoDaLiDa/Batlic-HLT 2025 style guides, as summarized
and exemplified in this document. All authors are required to adhere to these specifications. 
Templates are available on Overleaf, see Figure \ref{fig:overleaflink}, as well as on the conference web site\footnote{\url{https://www.nodalida-bhlt2025.eu/}} for \LaTeX\
(recommended). The NoDaLiDa Programme Committee reserves
the right not to publish papers not conforming to the standard format. 

Submission and reviewing will be on-line, managed by the OpenReview system.
The only accepted format for submitted papers is PDF.
NoDaLiDa does not allow submission of supplementary materials.
All submissions must be uploaded to OpenReview by the submission deadlines;
submissions received after that time will not be accepted.
To minimize the risk of network congestion, we encourage authors to upload
their submissions as early as possible. 
Improved versions of the submissions may be continuously uploaded until the
final deadline. 

On acceptance of a submission, precise instructions will be given on how to
send the camera ready version; following reviewers' feedback, such instructions
may include specific requests for modification.
Camera-ready versions not conforming to requests for revision made by the
Programme Committee will be considered as unsent, and hence will not be
included in the proceedings or presented at the conference.

\section{General Instructions}

Manuscripts must be in two-column format.  Exceptions to the
two-column format include the title, authors' names and complete
addresses, which must be centered at the top of the first page, and
any full-width figures or tables (see the guidelines in
Subsection~\ref{ssec:fonts}). {\bf Type single-spaced.}  Start all
pages directly under the top margin. See the guidelines later
regarding formatting the first page (Subsection~\ref{ssec:first}).
The manuscript should be printed single-sided and its length should
not exceed the maximum page limit described in
Section~\ref{sec:length}.  {\bf Do not number the pages.}
Pages are numbered for  initial submission. However, {\bf do not number the pages in the camera-ready version}.

By uncommenting {\small\verb|\aclfinalcopy|} at the top of this 
 document, it will compile to produce an example of the camera-ready formatting; by leaving it commented out, the document will be anonymized for initial submission.  

The review process is double-blind, so do not include any author information (names, addresses) when submitting a paper for review.  
However, you should maintain space for names and addresses so that they will fit in the final (accepted) version. Easiest to do so by leaving the ``Anonymous Author'' entries unchanged in the submitted paper.

The author list for submissions should include all (and only) individuals who made substantial contributions to the work presented. Each author listed on a submission to NoDaLiDa/Batlic-HLT 2025 will be notified of submissions, revisions and the final decision. No authors may be added to or removed from submissions to NoDaLiDa/Batlic-HLT 2025 after the submission deadline.

\subsection{Electronically-available resources}

NoDaLiDa/Batlic-HLT 2025 provides this description in \LaTeX2e ({\small {\tt
    nodalida2025.tex}}) and PDF format ({\small {\tt nodalida2025.pdf}}),
along with the \LaTeX2e style file used to format it ({\small {\tt
    nodalida2025.sty}}) and an ACL bibliography style ({\small {\tt
    acl\_natbib.bst}}) for the Bib\TeX\ reference management
software. These files are all available on Overleaf, see Figure \ref{fig:overleaflink}, and via the conference webpage\footnote{\url{https://www.nodalida-bhlt2025.eu/}}.
%A Microsoft Word template file ({\small  {\tt nodalida2023.docx}}) is also available at the same URL.
 We strongly
recommend the use of these style files, which have been appropriately
tailored for the NoDaLiDa/Batlic-HLT 2025 proceedings. If you have an option, we
recommend that you use the \LaTeX2e version. \textbf{If you will be
  using Microsoft Word, we require you to anonymize
  your source file so that the pdf produced does not retain your
  identity.}  This can be done by removing any personal information
from your source document properties.



\subsection{Format of Electronic Manuscript}
\label{sect:pdf}

For the production of the electronic manuscript you must use Adobe's
Portable Document Format (PDF). The easiest way is to use Overleaf\footnote{\url{https://www.overleaf.com}}.
%% Add instructions about using pdflatex
This format can also be generated directly from \LaTeX2e files or from
postscript ones. On Linux/Unix-like systems, you can use {\tt
  pdflatex} to generate a PDF file from \LaTeX2e files, or {\tt
  ps2pdf} to convert from postscript to PDF. 
%%
In Microsoft Windows, you can use Adobe's Distiller, or if you have
\texttt{cygwin} installed, you can use \texttt{dvipdf} or
\texttt{ps2pdf}. Note that some word processing programs generate PDF
which may not include all the necessary fonts (esp. tree diagrams,
symbols). When you print or create the PDF file, there is usually an
option in your printer setup to include none, all or just non-standard
fonts.  Please make sure that you select the option of including ALL
the fonts. {\em Before sending it, test your PDF by printing it from a
  computer different from the one where it was created.} Moreover,
some word processors may generate very large postscript/PDF files,
where each page is rendered as an image. Such images may reproduce
poorly. In this case, try alternative ways to obtain the postscript
and/or PDF. One way on some systems is to install a driver for a
postscript printer, send your document to the printer specifying
``Output to a file'', then convert the file to PDF.
Please keep in mind that it is of utmost importance to use \textbf{A4 format}.

%%Removed by YP (use of the \special command above instead):
%% It is of utmost importance to specify the \textbf{A4 format} (21 cm x
%% 29.7 cm) when formatting the paper. 
%% When working with {\tt dvips}, for instance, one should specify {\tt -t a4}.

Print-outs of the PDF file on A4 paper should be identical to the
hardcopy version. If you cannot meet the above requirements about the
production of your electronic submission, please contact the
program chair above as soon as possible.


\subsection{Layout}
\label{ssec:layout}

Format manuscripts two columns to a page, in the manner these
instructions are formatted. The exact dimensions for a page on A4
paper are:

\begin{itemize}
\item Left and right margins: 2.5 cm
\item Top margin: 2.5 cm
\item Bottom margin: 2.5 cm
\item Column width: 7.7 cm
\item Column height: 24.7 cm
\item Gap between columns: 0.6 cm
\end{itemize}

\noindent Papers should not be submitted on any other paper size.
 If you cannot meet the above requirements about the production of your electronic submission, 
please contact the program chair above as soon as possible.

\subsection{Fonts}
\label{ssec:fonts}
For reasons of uniformity, Adobe's {\bf Times Roman} font should be
used. In \LaTeX2e{} this is accomplished by putting

\begin{quote}
\begin{verbatim}
\usepackage{times}
\usepackage{latexsym}
\end{verbatim}
\end{quote}
in the preamble. If Times Roman is unavailable, use {\bf Computer
  Modern Roman} (\LaTeX2e{}'s default).  Note that the latter is about
  10\% less dense than Adobe's Times Roman font.


\begin{table}
\begin{center}
\begin{tabular}{|l|rl|}
\hline \bf Type of Text & \bf Font Size & \bf Style \\ \hline
paper title & 15 pt & bold \\
author names & 12 pt & bold \\
author affiliation & 12 pt & \\
the word ``Abstract'' & 12 pt & bold \\
section titles & 12 pt & bold \\
document text & 11 pt  &\\
captions & 11 pt & \\
abstract text & 10 pt & \\
bibliography & 10 pt & \\
footnotes & 9 pt & \\
\hline
\end{tabular}
\end{center}
\caption{\label{font-table} Font guide. }
\end{table}

\subsection{The First Page}
\label{ssec:first}

Center the title, author's name(s) and affiliation(s) across both
columns. Do not use footnotes for affiliations. Do not include the
paper ID number assigned during the submission process. Use the
two-column format only when you begin the abstract.

{\bf Title}: Place the title centered at the top of the first page, in
a 15-point bold font. (For a complete guide to font sizes and styles, 
see Table~\ref{font-table}.) Long titles should be typed on two lines without
a blank line intervening. Approximately, put the title at 2.5 cm from
the top of the page, followed by a blank line, then the author's
names(s), and the affiliation on the following line. Do not use only
initials for given names (middle initials are allowed). Do not format surnames
in all capitals (e.g., use ``Schlangen'' not ``SCHLANGEN'').
Do not format title and section headings in all capitals as well
except for proper names (such as ``BLEU'') that are conventionally
in all capitals.
The affiliation should contain the author's complete address, and if
possible, an electronic mail address. Leave about 2 cm between the
affiliation and the body of the first page.
The title, author names and addresses should be completely
identical to those entered to the electronical paper submission
website in order to maintain the consistency of author information
among all publications of the conference.

{\bf Abstract}: Type the abstract at the beginning of the first
column. The width of the abstract text should be smaller than the
width of the columns for the text in the body of the paper by about
0.6 cm on each side. Center the word {\bf Abstract} in a 12 point bold
font above the body of the abstract. The abstract should be a concise
summary of the general thesis and conclusions of the paper. It should
be no longer than 200 words. The abstract text should be in 10 point font.

{\bf Text}: Begin typing the main body of the text immediately after
the abstract, observing the two-column format as shown in 
the present document. Do not include page numbers.

{\bf Indent} when starting a new paragraph. Use 11 points for text and 
subsection headings, 12 points for section headings and 15 points for
the title. 

\subsection{Sections}

{\bf Headings}: Type and label section and subsection headings in the
style shown on the present document.  Use numbered sections (Arabic
numerals) in order to facilitate cross references. Number subsections
with the section number and the subsection number separated by a dot,
in Arabic numerals. Do not number subsubsections.

{\bf Citations}: Citations within the text appear
in parentheses as~\cite{vaswani2017attention} or, if the author's name appears in
the text itself, as \newcite{vaswani2017attention}. 
Append lowercase letters to the year in cases of ambiguity.  
Treat double authors as in~\cite{Aho:72}, but write as
 in~\cite{Chandra:81} when more than two authors are involved. Collapse multiple citations as
in~\cite{Gusfield:97,Aho:72}. Also refrain from using full citations as sentence constituents. We
suggest that instead of
\begin{quote}
  ``\cite{vaswani2017attention} showed that ...''
\end{quote}
you use
\begin{quote}
``\newcite{vaswani2017attention} showed that ...''
\end{quote}

If you are using the provided \LaTeX{} and Bib\TeX{} style files, you
can use the command \verb|\newcite| to get ``author (year)'' citations.

As reviewing will be double-blind, the submitted version of the papers should not include the
authors' names and affiliations. Furthermore, self-references that
reveal the author's identity, e.g.,
\begin{quote}
``We previously showed \cite{vaswani2017attention} ...''  
\end{quote}
should be avoided. Instead, use citations such as 
\begin{quote}
``\newcite{vaswani2017attention}
previously showed ... ''
\end{quote}

\textbf{Please do not  use anonymous citations} and  do not include acknowledgements when submitting your paper. Papers that do not conform
to these requirements may be rejected without review. 

\textbf{References}: Gather the full set of references together under
the heading {\bf References}; place the section before any Appendices,
unless they contain references. Arrange the references alphabetically
by first author, rather than by order of occurrence in the text.
Provide as complete a citation as possible, using a consistent format,
 the one for {\em Computational Linguistics\/}.  Use of full names for
authors rather than initials is preferred. 

{\bf Appendices}: Appendices, if any, directly follow the text and the
references (but see above).  Letter them in sequence and provide an
informative title: {\bf Appendix A. Title of Appendix}.

\textbf{Acknowledgement} section should appear in accepted manuscripts
only. It should go as a last section immediately
before the references.  Do not number the acknowledgement section.


\subsection{Footnotes}

{\bf Footnotes}: Put footnotes at the bottom of the page and use 9
points text. They may be numbered or referred to by asterisks or other
symbols.\footnote{This is how a footnote should appear.} Footnotes
should be separated from the text by a line.\footnote{Note the line
separating the footnotes from the text.}

\subsection{Graphics}

{\bf Illustrations}: Place figures, tables, and photographs in the
paper near where they are first discussed, rather than at the end, if
possible.  Wide illustrations may run across both columns.  Color
illustrations are allowed, provided you have verified that  
they will be understandable when printed in black ink.

{\bf Captions}: Provide a caption for every illustration; number each one
sequentially in the form:  ``Figure 1. Caption of the Figure.'' ``Table 1.
Caption of the Table.''  Type the captions of the figures and 
tables below the body, using 11 point text.  

\section{Translation of non-English Terms}

It is also advised to supplement non-English characters and terms
with appropriate transliterations and/or translations
since not all readers understand all such characters and terms.
Inline transliteration or translation can be represented in
the order of: original-form transliteration ``translation''.

\section{Length of Submission}
\label{sec:length}

Long papers may consist of up to 8 pages of content (excluding
references) and short papers and demo papers may consist of up to 
four (4) pages plus an unlimited number of pages for references in the proceedings.  Papers that do not conform to the
specified length and formatting requirements are subject to be
rejected without review. Note that we do not allow any additonal pages for appendices in submitted papers. 

Final versions of accepted papers will be given one additional page of content (up to 9 pages for long papers, up to 5 pages for short papers and demo papers) to address reviewers’ comments.

\section*{Acknowledgments}

Do not number the acknowledgment section. Do not include this section
when submitting your paper for review.

~

\noindent {\bf Preparing References:}

\noindent Include your own bib file like this:

\noindent \verb|\bibliographystyle{acl_natbib}|
\verb|\bibliography{nodalida2025}| 

\noindent where \verb|nodalida2025| corresponds to a nodalida2025.bib file.



\bibliographystyle{acl_natbib}
\bibliography{nodalida2025}

\end{document}
