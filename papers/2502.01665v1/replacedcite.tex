\section{Related works}
\label{sec_related_works}

One rock heterogeneity characterization approach involves identifying related features. ____ proposed the notion that an inertial coefficient substantially greater than the characteristic length indicates a sample that is nonhomogeneous with respect to permeability. ____ employed the fractal dimension to quantify heterogeneity, as it represents a measure of the complexity and irregularities exhibited by rock pore structures. A positive correlation with heterogeneity was qualitatively determined by comparing pore size distributions. However, both of these approaches rely on laboratory measurements (since the fractal dimension is determined from capillary pressure curves), which can be costly.

____ introduced a manually labelled homogeneous coefficient for quantifying rock heterogeneity on the basis of CT scans. Building on this research, ____ used image recognition to predict rock properties, incorporating the previously proposed coefficient. Their methodology involves a convolutional neural network (CNN) that includes a 3D spatial grid based on the observed mineral distribution. Detailed information concerning the spatial arrangement of rock components is provided, allowing the CNN to predict coefficients for new samples. By incorporating the rock homogeneity coefficient, the loss values induced when predicting rock strength and elastic modulus values were reduced by 25.7\% and 3.9\%, respectively. However, the training data employed for the homogeneity coefficient consisted of prelabelled samples with coefficients discretized into only six values, adding subjectivity.

____ extended an existing method to automate the heterogeneity estimation process at the pore scale for two-phase media; this approach is independent of the input image resolution. The algorithm calculates porosity within a moving window at various radii based on the maximum solid-to-pore distance observed in the given 3D image. Porosity variances are then plotted against the radius index, providing a heterogeneity measure that is influenced by these radius-based bounds, with larger radii generally leading to smaller variations. While the method can incorporate other features, the presented results focused on the spatial variability of porosity to obtain variability measures. ____ categorized rock heterogeneity based on pore size and pore arrangement variability. By setting a distinct threshold for each parameter, they divided the observed heterogeneity into four distinct areas. Both of the above approaches, despite being designed to quantify heterogeneity at the pore scale, require segmentation to implemented beforehand, which introduces a new source of error due to the limited accuracy of this process ____.

____ identified two approaches for estimating spatial entropy. The first method accounts for unequally partitioned subareas and is computed only for binary variables, indicating the presence or absence of an attribute at each location. The second approach introduces spatial considerations by transforming the examined variable to account for the distances between realizations (cooccurrences). One area that extensively applies spatial entropy involves the quantification and characterization of landscape patterns, where entropy-related metrics have rapidly developed into efficient tools for this purpose ____.

Given the advancements achieved in terms of high-resolution imaging technologies, texture analysis has emerged as a promising option for assessing heterogeneity via automated approaches. It describes the spatial arrangement and relationships of the components of a rock, thereby aiding in identifying variability and heterogeneity. Recently, ____ proposed a methodology that uses a grey-level cooccurrence matrix (GLCM) to extract textural attributes and captures additional compositional attributes from well logs. Principal component analysis ____ is applied to these attributes to represent the variability at each depth with a unique value. Finally, an index is analytically calculated to quantify the heterogeneity at each depth. While this method is applied on a macro scale, it underscores the potential of texture analysis for assessing rock heterogeneity, since applying this approach at a finer scale with more details could yield even better results.