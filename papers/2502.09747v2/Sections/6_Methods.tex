\section*{Supplementary Information}



\subsection*{Overview of the Consumer Complaint Data}
\label{main:subsec:Consumer Complaint-data}

The Consumer Complaint Database, maintained by the Consumer Financial Protection Bureau (CFPB), is a publicly accessible resource that collects complaints about consumer financial products and services. These complaints are forwarded to companies for their response, while the CFPB—a U.S. government agency—is dedicated to ensuring that banks, lenders, and other financial institutions treat consumers fairly. We focus on 687,241 consumer complaint narrative, starting from January 2022 and ending in August 2024. The dataset offers the mailing ZIP code provided by the consumer, which allow us to check heterogeneity via the educational level and the degree of urbanization by region. Specifically, we employ Rural Urban Commuting Area (RUCA) codes to assess urbanization levels and measure the educational level by the percentage of individuals aged 25 and older who have earned a bachelor’s degree. Corresponding data is available at 
\href{https://www.ers.usda.gov/data-products/rural-urban-commuting-area-codes}{\texttt{here}}
and \href{https://data.census.gov/table/ACSST1Y2023.S1501}{\texttt{here}} respectively.


\subsection*{Overview of the LinkedIn Job Posting Data}
\label{main:subsec:job-posting-data}


We use data from the Revelio Labs universe, which collects, cleans and aggregates individual-level job postings sourced from publicly available online sources, such as LinkedIn. The raw dataset includes all LinkedIn postings (active, inactive, removed), the company identifier, the company founding year, the full text of job listings, and associated information (title, salary, etc.).  The raw data are broken out by Revelio Labs into eight job categories: Administration, Engineering, Finance, Marketing, Operations, Sales, Scientist, and Unclassified. We focus on 304,270,122 job postings, starting from January 2021 and ending in October 2023. We focus on the full text of the job postings. To analyze the heterogeneity of LLM usage by company characteristics, we combine the job listings information with the Revelio Labs associated LinkedIn employee data. Similarly to the job postings data, the baseline workforce data was scraped, cleaned and aggregated at the firm level. The workforce data is available going back up to 2008. We define firm characteristics based on pre-ChatGPT introduction characteristics. We define two different definitions for small firms: in our sample, small firms are companies with either 10 or fewer registered employees in 2021 or companies posting less than or equal to about 2 postings per year. We also check heterogeneity via founding year, splitting in terms of years 2015-onwards, 2000-2015, 1980-2000 and before 1980. These time periods are determined based on quantiles of the founding year distribution. Note that although the median number of postings per company per year is 3, the total number of postings drops from 304,270,122 to 1,440,912 when we focus on small companies. This indicates that small companies contribute a relatively minor share to the total posting volume compared to larger companies.












\subsection*{Overview of the Corporate Press Release Data}
\label{main:subsec:press-release-data}

We collect corporate press release data using the NewsAPI service, which aggregates online news content from various sources. We collected data from: PRNewswire, PRWeb, and Newswire, three of the main companies distributing corporate press releases online. These were chosen due to data avilability and cost. PR Newswire, founded in 1954, is one of the oldest and most widely recognized press release distribution services, offering an extensive network that reaches major news outlets, journalists, and online platforms worldwide. It serves a broad range of clients, from large corporations to small businesses. PRWeb, launched in 1997, focuses primarily on online distribution and SEO optimization, making it a more budget-friendly option for businesses looking to enhance their digital presence. Newswire distributes press releases to both traditional media and online platforms, catering to businesses of various sizes. While all three services offer some level of editorial support, their primary business focus remains distribution.

With a focus on English-language text, we gathered up to 537,413 press releases from January 2022 to September 2024.  Our analysis primarily focused on the full body text. Due to the limited number of articles post-ChatGPT introduction available from Newswire, we conducted detailed robustness checks only on PR Newswire and PRWeb data, which provided sufficient volume for heterogeneity analysis.  We  classified the press releases by four overarching categories: Business \& Money, Science \& Tech, People \& Culture, and Other. 








\subsection*{Overview of the UN Press Release Data}
\label{main:subsec:press-release-data}

We collect United Nations release data using customized scripts. The United Nations (UN), founded in 1945, is an international organization dedicated to fostering global peace, security, and cooperation among its member states~\cite{shin2024adoption}. 
Country teams of United Nations regularly update on the latest developments in that country.
To ensure consistency and maintain a focus on English-language content, articles were selected from the English-language websites of 97 country teams. From January 2019 to September 2024, up to 15,919 press releases were collected, with the analysis primarily concentrating on the full body text. Our investigation revealed that among the remaining 96 country teams, 57 do not have their own websites, 33 lack English-language websites, and 6 do not operate press release websites.




















\subsection*{Data Split, Model Fitting, and Evaluation}
\label{main:subsec:training-validation}



For model fitting, we count word frequencies for the corpora written before the release of ChatGPT and the LLM-modified corpora. We fit the model with data from 2021 (2019 for UN press release), and use data from January 2022 onwards for validation and inference. We developed individual models for each major category in LinkedIn job postings and for each distribution platform in corporate press releases. For UN press releases and consumer complaints, we fit one model for each domain. During inference, we randomly sample up to 2,000 records per month (per quarter for UN press release) to analyze the increasing temporal trends of LLM usage across various writing domains.

To evaluate model accuracy and calibration under temporal distribution shift, we collected a sample of 2000 records from January 1, 2022, to November 29, 2022, a time period prior to the release of ChatGPT, as the validation data. We construct validation sets with LLM-modified content proportions ($\alpha$) ranging from 0\% to 25\%, in 2.5\% increments, and compared the model's estimated $\alpha$ with the ground truth $\alpha$ (Table \ref{t1}, \ref{t2}, \ref{t3}, \ref{t4}, \ref{t5}). Our models all performed well in our application, with a prediction error consistently less than 3.3\% at the population level across various ground truth $\alpha$ values. 





