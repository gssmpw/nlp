

\clearpage
\newpage

\begin{suppfigure}[htb!]
\begin{lstlisting}
The aim here is to reverse-engineer the author's writing process by taking a piece of text from a consumer complaint and compressing it into a more concise form. This process simulates how an author might distill their thoughts and key points into a structured, yet not overly condensed form. 

Now as a first step, given a complete piece of text from a consumer complaint, reverse-engineer it into a list of bullet points.
\end{lstlisting}
\caption{
Example prompt for summarizing a consumer complaint into a skeleton: This process simulates how an author might first only write the main ideas and core information into a concise outline. The goal is to capture the essence of the complaint in a structured and succinct manner, serving as a foundation for the next prompt.
}
\label{fig:skeleton-prompt-1}
\end{suppfigure}



\begin{suppfigure}[htb!]
\begin{lstlisting}
Following the initial step of reverse-engineering the author's writing process by compressing a text segment from a consumer complaint, you now enter the second phase. Here, your objective is to expand upon the concise version previously crafted. This stage simulates how an author elaborates on the distilled thoughts and key points, enriching them into a detailed, structured narrative. 

Given the concise output from the previous step, your task is to develop it into a fully fleshed-out text.
\end{lstlisting}
\caption{
Example prompt for expanding the skeleton into a full text: The aim here is to simulate the process of using the structured outline as a basis to generate comprehensive and coherent text. This step mirrors the way an author might flesh out the outline into detailed paragraphs, effectively transforming the condensed ideas into a fully articulated consumer complaint. The format and depth of the expansion can vary, reflecting the diverse styles and requirements of different consumer complaints.
}
\label{fig:skeleton-prompt-2}
\end{suppfigure}



