
\begin{figure}[htb]
\centering
\includegraphics[width=\textwidth]{new_Figures/main.pdf}
\caption{
\textbf{Temporal dynamics of large language model (LLM) adoption across diverse writing domains.}
Analysis of LLM-generated or substantially modified content across four domains: (a) Consumer complaints filed with the Consumer Financial Protection Bureau showed algorithm false positive rate of 1.5\% pre-ChatGPT release (November 2022), followed by genuine LLM adoption rising to 15.3\% by August 2023, before plateauing at 17.7\% through August 2024. (b) Corporate press releases demonstrated consistent adoption patterns across platforms: Newswire platform showed rapid uptake reaching 24.3\% by December 2023, stabilizing at 23.8\% through September 2024; PRNewswire demonstrated similar trends with peak adoption at 16.4\% (December 2023) maintaining at 16.5\% through September 2024; PRWeb showed comparable patterns (data available through January 2024). (c) LinkedIn job postings from small organizations (below median job postings) displayed consistent trends across professional categories, with adoption increasing post-ChatGPT release (5-month lag), peaking in July 2023 before plateauing or slightly declining through October 2023. (d) United Nations government press releases showed two phases: rapid initial adoption (Q1 2023: 3.1\% to Q3 2023: 10.1\%), followed by a more gradual increase to 13.7\% by Q3 2024. This figure displays the fraction ($\alpha$) of sentences estimated to have been substantially modified by LLM using our previous method \cite{liang2024monitoring}. Error bars indicate 95\% confidence intervals by bootstrap.
}
\label{fig:main:1}
\end{figure}

\clearpage
\newpage


\begin{figure}[htb]
\centering
\includegraphics[width=\textwidth]{new_Figures/complaint.pdf}
\caption{
\textbf{Geographic and demographic patterns of LLM adoption in Consumer Financial Protection Bureau complaints.}
(a) State-level analysis (January-August 2024) revealed substantial geographic variation, with highest adoption in Arkansas (29.2\%), Missouri (26.9\%), and North Dakota (24.8\%), contrasting with lowest rates in West Virginia (2.6\%), Idaho (3.8\%), and Vermont (4.8\%). Notable population centers showed moderate adoption (California: 17.4\%, New York: 16.6\%). (b) Analysis by Rural Urban Commuting Area (RUCA) codes showed similar adoption trajectories between highly urbanized and non-highly urbanized areas during initial uptake (2023Q1-2023Q3), before diverging to equilibrium levels of 18.2\% and 10.9\%, respectively. (c) Comparison of areas above and below state median levels of bachelor's degree attainment (population aged 25+) revealed comparable initial adoption patterns (2023Q1-2023Q2), followed by higher stabilized rates in areas with lower educational attainment (19.9\% vs 17.4\% by 2024Q3). (d) Within highly urbanized areas, this educational attainment pattern persisted, with lower-education areas showing higher adoption rates (21.4\% vs 17.8\% by 2024Q3). 
}
\label{fig:main:2}
\end{figure}

\clearpage
\newpage






\begin{figure}[htb]
\centering
\includegraphics[width=\textwidth]{MoreFigures/PressRelease/robustness_check_press_release.pdf}
\caption{
\textbf{Sectoral patterns of LLM adoption in corporate press releases across major distribution platforms.}
Analysis of press releases by sector revealed consistent patterns across platforms, with Science \& Technology showing marginally higher adoption rates. (a) PRNewswire demonstrated similar sectoral patterns by 2023Q4: Science \& Technology (16.8\%), People \& Culture (14.3\%), Business \& Money (14.0\%), and Other sectors (11.4\%). (b) PRWeb exhibited comparable sectoral distribution: Science \& Technology (16.8\%), Business \& Money (15.6\%), People \& Culture (13.6\%), and Other sectors (11.7\%). All sectors showed similar temporal adoption patterns following ChatGPT's release, with initial lag followed by sustained growth through 2023. 
}
\label{fig:main:3}
\end{figure}
\clearpage
\newpage



\begin{figure}[htb]
\centering
\includegraphics[width=\textwidth]{new_Figures/linkedin.pdf}
\caption{
\textbf{Organization age and LLM adoption patterns in LinkedIn job postings from small organizations across professional categories.}
(a) Among small organizations (less than median job vacancies), analysis stratified by number of employees revealed higher LLM adoption rates in firms with below median employees (11.1\% vs 6.2\% by October 2023). (b) Among small organizations (less than median job vacancies), analysis stratified by founding year revealed higher LLM adoption rates in more recently established firms (founded after 2015: 14.1\%; 2010-2015: 10.2\%; 1980-2000: 7.2\%; pre-1980: 4.0\%). (c-i) This age-dependent pattern persisted across professional categories: Admin (c), Engineer (d), Finance (e), Marketing (f), Operations (g), Sales (h), and Scientist (i), with newer organizations consistently showing higher adoption rates. We defined small organizations based on having 2 or less job vacancy postings in a year (median is 3).
}
\label{fig:main:4}
\end{figure}
\clearpage
\newpage




\clearpage



\begin{suppfigure}[htb!]
\centering
\includegraphics[width=1.00\textwidth]{MoreFigures/UNPressRelease/robust_state.pdf}
\caption{
\textbf{Regional variation in LLM adoption across United Nations Member States' press releases.} 
Temporal analysis of estimated fraction ($\alpha$) of LLM-modified content stratified by regional groups shows differential adoption patterns. After ChatGPT's launch (November 30, 2022), Latin American and Caribbean States demonstrated the highest adoption rate, reaching approximately 21\% by 2024, while African States, Asia-Pacific States, and Eastern European States showed more moderate increases to 11-14\%. Error bars indicate 95\% confidence intervals obtained through bootstrap analysis. Regional variations may reflect differences in technological infrastructure, language diversity, and institutional policies across Member States.
}
\label{fig: supp-robust-US-country-groups}
\end{suppfigure}


\clearpage



\begin{suppfigure}[ht!]
    \centering
    \includegraphics[width=0.5\textwidth]{MoreFigures/JobPosting/slight_increase_when_majority_company_included.pdf}
\caption{
\textbf{Analysis of LLM adoption in LinkedIn job postings across the full sample.} Temporal analysis of the estimated fraction ($\alpha$) of LLM-modified content in job postings across all company sizes shows a modest but statistically significant increase from pre-ChatGPT baseline to approximately 3\% adoption following ChatGPT's introduction (November 30, 2022). This aggregate analysis includes all companies regardless of size, with larger firms (who post more frequent vacancies and typically have dedicated HR resources) representing a greater proportion of the sample. Error bars represent 95\% confidence intervals obtained through bootstrap analysis.
}
\label{fig: full-sample-LinkedIn}
\end{suppfigure}

\begin{suppfigure}[ht!]
    \centering
    \includegraphics[width=0.5\textwidth]{new_Figures/linkedin_year_employee.pdf}
\caption{
\textbf{LLM adoption patterns in LinkedIn job postings from small organizations ($\leq$10 employees).} 
Temporal analysis of estimated fraction ($\alpha$) of LLM-modified content across professional categories (Finance, Marketing, Admin, Operation, Engineer, Scientist, Sales) shows patterns consistent with main findings based on vacancy frequency. Following ChatGPT's launch (November 30, 2022), organizations with $\leq$10 employees demonstrate similar adoption trajectories to those posting $\leq$2 vacancies annually, with estimated $\alpha$ increasing from 0-2\% pre-launch to 7-15\% by October 2023. Scientist positions show highest adoption ($\approx$15\%), followed by Marketing and Finance ($>$10\%), while Admin, Engineer, Sales and Operations show more moderate adoption (7-9\%). Error bars indicate 95\% confidence intervals obtained through bootstrap analysis. This consistency across different definitions of small organizations (by employee count or vacancy frequency) strengthens the robustness of observed adoption patterns.
}
\label{fig: supp-robust-small-company-definition}
\end{suppfigure}





\clearpage






\begin{suppfigure}[htb]
\centering
\includegraphics[width=0.75\textwidth]{MoreFigures/PressRelease/model_ckpt.pdf}
\caption{
\textbf{Robustness analysis of LLM adoption estimates across different press release platforms using multiple GPT models for training data generation.}
(a) PRNewswire, (b) PRWeb, and (c) Newswire press releases show consistent temporal patterns regardless of the GPT model used for training data generation. Estimated fraction ($\alpha$) of LLM-modified content was calculated using three different models: GPT-3.5-turbo (used in main analysis, released January 25, 2024), GPT-4-0125-preview (released January 25, 2024), and GPT-4-2024-08-06 (released August 6, 2024). While all models reveal similar adoption trajectories following ChatGPT's launch (November 30, 2022), the most recent model GPT-4-2024-08-06 generates marginally higher estimates across platforms, suggesting our main results may be conservative. Error bars indicate 95\% confidence intervals obtained through bootstrap analysis.
}
\end{suppfigure}













    































\clearpage
\newpage


\begin{supptable}[htb!]
\small
\begin{center}

\caption{
\textbf{Performance validation of our model} across Consumer Complaint (all predating ChatGPT's launch), using a blend of official human and LLM-generated complaints. 
}
\label{t1}
\begin{tabular}{lrcllc}
\cmidrule[\heavyrulewidth]{1-6}
\multirow{2}{*}{\bf No.} 
& \multirow{2}{*}{\bf \begin{tabular}[c]{@{}c@{}} Validation \\ Data Source 
\end{tabular} } 
& \multirow{2}{*}{\bf \begin{tabular}[c]{@{}c@{}} Ground \\ Truth $\alpha$
\end{tabular}}  
&\multicolumn{2}{l}{\bf Estimated} 
& \multirow{2}{*}{\bf \begin{tabular}[c]{@{}c@{}} Prediction \\ Error 
\end{tabular} } 
\\
\cmidrule{4-5}
 & & & $\alpha$ & $CI$ ($\pm$) & \\
\cmidrule{1-6}
(1) & \emph{Consumer Complaint} & 0.0\% & 1.8\% & 0.2\% & 1.8\% \\
(2) & \emph{Consumer Complaint} & 2.5\% & 4.6\% & 0.2\% & 2.1\% \\
(3) & \emph{Consumer Complaint} & 5.0\% & 7.3\% & 0.2\% & 2.3\% \\
(4) & \emph{Consumer Complaint} & 7.5\% & 9.8\% & 0.2\% & 2.3\% \\
(5) & \emph{Consumer Complaint} & 10.0\% & 12.2\% & 0.3\% & 2.2\% \\
(6) & \emph{Consumer Complaint} & 12.5\% & 14.6\% & 0.2\% & 2.1\% \\
(7) & \emph{Consumer Complaint} & 15.0\% & 17.1\% & 0.3\% & 2.1\% \\
(8) & \emph{Consumer Complaint} & 17.5\% & 19.4\% & 0.3\% & 1.9\% \\
(9) & \emph{Consumer Complaint} & 20.0\% & 21.8\% & 0.3\% & 1.8\% \\
(10) & \emph{Consumer Complaint} & 22.5\% & 24.2\% & 0.3\% & 1.7\% \\
(11) & \emph{Consumer Complaint} & 25.0\% & 26.5\% & 0.3\% & 1.5\% \\
\cmidrule[\heavyrulewidth]{1-6}
\end{tabular}
\end{center}
\vspace{-5mm}
\end{supptable}

\begin{supptable}[htb!]
\small
\begin{center}

\caption{
\textbf{Performance validation of our model} across UN Press Release (all predating ChatGPT's launch), using a blend of official human and LLM-generated press releases. 
}
\label{t2}
\begin{tabular}{lrcllc}
\cmidrule[\heavyrulewidth]{1-6}
\multirow{2}{*}{\bf No.} 
& \multirow{2}{*}{\bf \begin{tabular}[c]{@{}c@{}} Validation \\ Data Source 
\end{tabular} } 
& \multirow{2}{*}{\bf \begin{tabular}[c]{@{}c@{}} Ground \\ Truth $\alpha$
\end{tabular}}  
&\multicolumn{2}{l}{\bf Estimated} 
& \multirow{2}{*}{\bf \begin{tabular}[c]{@{}c@{}} Prediction \\ Error 
\end{tabular} } 
\\
\cmidrule{4-5}
 & & & $\alpha$ & $CI$ ($\pm$) & \\
\cmidrule{1-6}
(1) & \emph{UN Press Release} & 0.0\% & 2.5\% & 0.2\% & 2.5\% \\
(2) & \emph{UN Press Release} & 2.5\% & 5.4\% & 0.2\% & 2.9\% \\
(3) & \emph{UN Press Release} & 5.0\% & 8.1\% & 0.3\% & 3.1\% \\
(4) & \emph{UN Press Release} & 7.5\% & 10.7\% & 0.3\% & 3.2\% \\
(5) & \emph{UN Press Release} & 10.0\% & 13.1\% & 0.3\% & 3.1\% \\
(6) & \emph{UN Press Release} & 12.5\% & 15.6\% & 0.3\% & 3.1\% \\
(7) & \emph{UN Press Release} & 15.0\% & 18.0\% & 0.3\% & 3.0\% \\
(8) & \emph{UN Press Release} & 17.5\% & 20.4\% & 0.3\% & 2.9\% \\
(9) & \emph{UN Press Release} & 20.0\% & 22.8\% & 0.3\% & 2.8\% \\
(10) & \emph{UN Press Release} & 22.5\% & 25.1\% & 0.3\% & 2.6\% \\
(11) & \emph{UN Press Release} & 25.0\% & 27.5\% & 0.3\% & 2.5\% \\
\cmidrule[\heavyrulewidth]{1-6}
\end{tabular}
\end{center}
\vspace{-5mm}
\end{supptable}


\begin{supptable}[htb!]
\small
\begin{center}

\caption{
\textbf{Performance validation of our model} across PRNewswire, PRWeb, Newswire (all predating ChatGPT's launch), using a blend of official human and LLM-generated press releases. 
Our algorithm demonstrates high accuracy with less than 3.3\% prediction error in identifying the proportion of LLM press release within the validation set.
}
\label{t3}
\begin{tabular}{lrcllc}
\cmidrule[\heavyrulewidth]{1-6}
\multirow{2}{*}{\bf No.} 
& \multirow{2}{*}{\bf \begin{tabular}[c]{@{}c@{}} Validation \\ Data Source 
\end{tabular} } 
& \multirow{2}{*}{\bf \begin{tabular}[c]{@{}c@{}} Ground \\ Truth $\alpha$
\end{tabular}}  
&\multicolumn{2}{l}{\bf Estimated} 
& \multirow{2}{*}{\bf \begin{tabular}[c]{@{}c@{}} Prediction \\ Error 
\end{tabular} } 
\\
\cmidrule{4-5}
 & & & $\alpha$ & $CI$ ($\pm$) & \\
\cmidrule{1-6}
(1) & \emph{PRNewswire} & 0.0\% & 2.9\% & 0.3\% & 2.9\% \\
(2) & \emph{PRNewswire} & 2.5\% & 5.7\% & 0.3\% & 3.2\% \\
(3) & \emph{PRNewswire} & 5.0\% & 8.3\% & 0.3\% & 3.3\% \\
(4) & \emph{PRNewswire} & 7.5\% & 10.8\% & 0.3\% & 3.3\% \\
(5) & \emph{PRNewswire} & 10.0\% & 13.2\% & 0.3\% & 3.2\% \\
(6) & \emph{PRNewswire} & 12.5\% & 15.6\% & 0.3\% & 3.1\% \\
(7) & \emph{PRNewswire} & 15.0\% & 18.0\% & 0.3\% & 3.0\% \\
(8) & \emph{PRNewswire} & 17.5\% & 20.3\% & 0.3\% & 2.8\% \\
(9) & \emph{PRNewswire} & 20.0\% & 22.7\% & 0.3\% & 2.7\% \\
(10) & \emph{PRNewswire}& 22.5\% & 25.0\% & 0.3\% & 2.5\% \\
(11) & \emph{PRNewswire}& 25.0\% & 27.3\% & 0.3\% & 2.3\% \\
\cmidrule{1-6}
(12) & \emph{PRWeb} & 0.0\% & 2.1\% & 0.2\% & 2.1\% \\
(13) & \emph{PRWeb} & 2.5\% & 5.2\% & 0.2\% & 2.7\% \\
(14) & \emph{PRWeb} & 5.0\% & 7.8\% & 0.2\% & 2.8\% \\
(15) & \emph{PRWeb} & 7.5\% & 10.4\% & 0.2\% & 2.9\% \\
(16) & \emph{PRWeb} & 10.0\% & 12.9\% & 0.3\% & 2.9\% \\
(17) & \emph{PRWeb} & 12.5\% & 15.4\% & 0.3\% & 2.9\% \\
(18) & \emph{PRWeb} & 15.0\% & 17.8\% & 0.3\% & 2.8\% \\
(19) & \emph{PRWeb} & 17.5\% & 20.2\% & 0.3\% & 2.7\% \\
(20) & \emph{PRWeb} & 20.0\% & 22.6\% & 0.3\% & 2.6\% \\
(21) & \emph{PRWeb} & 22.5\% & 25.0\% & 0.3\% & 2.5\% \\
(22) & \emph{PRWeb} & 25.0\% & 27.3\% & 0.3\% & 2.3\% \\
\cmidrule{1-6}
(23) & \emph{Newswire} & 0.0\% & 2.3\% & 0.2\% & 2.3\% \\
(24) & \emph{Newswire} & 2.5\% & 5.3\% & 0.2\% & 2.8\% \\
(25) & \emph{Newswire} & 5.0\% & 7.9\% & 0.3\% & 2.9\% \\
(26) & \emph{Newswire} & 7.5\% & 10.5\% & 0.3\% & 3.0\% \\
(27) & \emph{Newswire} & 10.0\% &13.0\% & 0.3\% & 3.0\% \\
(28) & \emph{Newswire} & 12.5\% & 15.4\% & 0.3\% & 2.9\% \\
(29) & \emph{Newswire} & 15.0\% & 17.9\% & 0.3\% & 2.9\% \\
(30) & \emph{Newswire} & 17.5\% & 20.3\% & 0.3\% & 2.8\% \\
(31) & \emph{Newswire} & 20.0\% & 22.6\% & 0.3\% & 2.6\% \\
(32) & \emph{Newswire} & 22.5\% & 25.0\% & 0.3\% & 2.5\% \\
(33) & \emph{Newswire} & 25.0\% & 27.4\% & 0.3\% & 2.4\% \\
\cmidrule[\heavyrulewidth]{1-6}
\end{tabular}
\end{center}
\vspace{-5mm}
\end{supptable}


\begin{supptable}[htb!]
\small
\begin{center}

\caption{
\textbf{Performance validation of our model} across Admin, Engineer, Finance, Marketing (all predating ChatGPT's launch), using a blend of official human and LLM-generated job postings. 
}
\label{t4}
\begin{tabular}{lrcllc}
\cmidrule[\heavyrulewidth]{1-6}
\multirow{2}{*}{\bf No.} 
& \multirow{2}{*}{\bf \begin{tabular}[c]{@{}c@{}} Validation \\ Data Category 
\end{tabular} } 
& \multirow{2}{*}{\bf \begin{tabular}[c]{@{}c@{}} Ground \\ Truth $\alpha$
\end{tabular}}  
&\multicolumn{2}{l}{\bf Estimated} 
& \multirow{2}{*}{\bf \begin{tabular}[c]{@{}c@{}} Prediction \\ Error 
\end{tabular} } 
\\
\cmidrule{4-5}
 & & & $\alpha$ & $CI$ ($\pm$) & \\
\cmidrule{1-6}
(1) & \emph{Admin} & 0.0\% & 1.2\% & 0.5\% & 1.2\% \\
(2) & \emph{Admin} & 2.5\% & 4.0\% & 0.6\% & 1.5\% \\
(3) & \emph{Admin} & 5.0\% & 6.6\% & 0.7\% & 1.6\% \\
(4) & \emph{Admin} & 7.5\% & 9.1\% & 0.7\% & 1.6\% \\
(5) & \emph{Admin} & 10.0\% & 11.6\% & 0.8\% & 1.6\% \\
(6) & \emph{Admin} & 12.5\% & 14.1\% & 0.8\% & 1.6\% \\
(7) & \emph{Admin} & 15.0\% & 16.7\% & 0.8\% & 1.7\% \\
(8) & \emph{Admin} & 17.5\% & 19.1\% & 0.8\% & 1.6\% \\
(9) & \emph{Admin} & 20.0\% & 21.6\% & 0.9\% & 1.6\% \\
(10) & \emph{Admin}& 22.5\% & 24.0\% & 0.9\% & 1.5\% \\
(11) & \emph{Admin}& 25.0\% & 26.4\% & 0.9\% & 1.4\% \\
\cmidrule{1-6}
(12) & \emph{Engineer} & 0.0\% & 0.9\% & 0.5\% & 0.9\% \\
(13) & \emph{Engineer} & 2.5\% & 3.6\% & 0.6\% & 1.1\% \\
(14) & \emph{Engineer} & 5.0\% & 6.2\% & 0.7\% & 1.2\% \\
(15) & \emph{Engineer} & 7.5\% & 8.8\% & 0.8\% & 1.3\% \\
(16) & \emph{Engineer} & 10.0\% & 11.3\% & 0.8\% & 1.3\% \\
(17) & \emph{Engineer} & 12.5\% & 13.8\% & 0.8\% & 1.3\% \\
(18) & \emph{Engineer} & 15.0\% & 16.4\% & 0.9\% & 1.4\% \\
(19) & \emph{Engineer} & 17.5\% & 18.9\% & 0.8\% & 1.4\% \\
(20) & \emph{Engineer} & 20.0\% & 21.4\% & 0.9\% & 1.4\% \\
(21) & \emph{Engineer} & 22.5\% & 23.9\% & 0.9\% & 1.4\% \\
(22) & \emph{Engineer} & 25.0\% & 26.4\% & 0.9\% & 1.4\% \\
\cmidrule{1-6}
(23) & \emph{Finance} & 0.0\% & 0.7\% & 0.4\% & 0.7\% \\
(24) & \emph{Finance} & 2.5\% & 3.5\% & 0.6\% & 1.0\% \\
(25) & \emph{Finance} & 5.0\% & 6.0\% & 0.7\% & 1.0\% \\
(26) & \emph{Finance} & 7.5\% & 8.5\% & 0.7\% & 1.0\% \\
(27) & \emph{Finance} & 10.0\% &10.9\% & 0.7\% & 0.9\% \\
(28) & \emph{Finance} & 12.5\% & 13.4\% & 0.7\% & 0.9\% \\
(29) & \emph{Finance} & 15.0\% & 15.9\% & 0.8\% & 0.9\% \\
(30) & \emph{Finance} & 17.5\% & 18.3\% & 0.8\% & 0.8\% \\
(31) & \emph{Finance} & 20.0\% & 20.7\% & 0.9\% & 0.7\% \\
(32) & \emph{Finance} & 22.5\% & 23.1\% & 0.8\% & 0.6\% \\
(33) & \emph{Finance} & 25.0\% & 25.5\% & 0.9\% & 0.5\% \\
\cmidrule{1-6}
(23) & \emph{Marketing} & 0.0\% & 0.6\% & 0.5\% & 0.6\% \\
(24) & \emph{Marketing} & 2.5\% & 3.4\% & 0.6\% & 0.9\% \\
(25) & \emph{Marketing} & 5.0\% & 5.9\% & 0.6\% & 0.9\% \\
(26) & \emph{Marketing} & 7.5\% & 8.4\% & 0.7\% & 0.9\% \\
(27) & \emph{Marketing} & 10.0\% &10.9\% & 0.8\% & 0.9\% \\
(28) & \emph{Marketing} & 12.5\% & 13.4\% & 0.8\% & 0.9\% \\
(29) & \emph{Marketing} & 15.0\% & 15.8\% & 0.8\% & 0.8\% \\
(30) & \emph{Marketing} & 17.5\% & 18.3\% & 0.9\% & 0.8\% \\
(31) & \emph{Marketing} & 20.0\% & 20.8\% & 0.8\% & 0.8\% \\
(32) & \emph{Marketing} & 22.5\% & 23.3\% & 0.9\% & 0.8\% \\
(33) & \emph{Marketing} & 25.0\% & 25.7\% & 0.9\% & 0.7\% \\
\cmidrule[\heavyrulewidth]{1-6}
\end{tabular}
\end{center}
\vspace{-5mm}
\end{supptable}
\clearpage
\newpage
\begin{supptable}[htb!]
\small
\begin{center}

\caption{
\textbf{Performance validation of our model} across Operation, Sales, Scientist (all predating ChatGPT's launch), using a blend of official human and LLM-generated job postings. 
}
\label{t5}
\begin{tabular}{lrcllc}
\cmidrule[\heavyrulewidth]{1-6}
\multirow{2}{*}{\bf No.} 
& \multirow{2}{*}{\bf \begin{tabular}[c]{@{}c@{}} Validation \\ Data Category 
\end{tabular} } 
& \multirow{2}{*}{\bf \begin{tabular}[c]{@{}c@{}} Ground \\ Truth $\alpha$
\end{tabular}}  
&\multicolumn{2}{l}{\bf Estimated} 
& \multirow{2}{*}{\bf \begin{tabular}[c]{@{}c@{}} Prediction \\ Error 
\end{tabular} } 
\\
\cmidrule{4-5}
 & & & $\alpha$ & $CI$ ($\pm$) & \\
\cmidrule{1-6}
(1) & \emph{Operation} & 0.0\% & 0.8\% & 0.5\% & 0.8\% \\
(2) & \emph{Operation} & 2.5\% & 3.3\% & 0.6\% & 0.8\% \\
(3) & \emph{Operation} & 5.0\% & 5.9\% & 0.7\% & 0.9\% \\
(4) & \emph{Operation} & 7.5\% & 8.4\% & 0.7\% & 0.9\% \\
(5) & \emph{Operation} & 10.0\% & 10.9\% & 0.8\% & 0.9\% \\
(6) & \emph{Operation} & 12.5\% & 13.3\% & 0.8\% & 0.8\% \\
(7) & \emph{Operation} & 15.0\% & 15.8\% & 0.8\% & 0.8\% \\
(8) & \emph{Operation} & 17.5\% & 18.2\% & 0.9\% & 0.7\% \\
(9) & \emph{Operation} & 20.0\% & 20.7\% & 0.9\% & 0.7\% \\
(10) & \emph{Operation}& 22.5\% & 23.2\% & 0.9\% & 0.7\% \\
(11) & \emph{Operation}& 25.0\% & 25.6\% & 0.9\% & 0.6\% \\
\cmidrule{1-6}
(12) & \emph{Sales} & 0.0\% & 1.2\% & 0.5\% & 1.2\% \\
(13) & \emph{Sales} & 2.5\% & 3.7\% & 0.6\% & 1.2\% \\
(14) & \emph{Sales} & 5.0\% & 6.2\% & 0.7\% & 1.2\% \\
(15) & \emph{Sales} & 7.5\% & 8.6\% & 0.8\% & 1.1\% \\
(16) & \emph{Sales} & 10.0\% & 11.0\% & 0.8\% & 1.0\% \\
(17) & \emph{Sales} & 12.5\% & 13.4\% & 0.8\% & 0.9\% \\
(18) & \emph{Sales} & 15.0\% & 15.8\% & 0.8\% & 0.8\% \\
(19) & \emph{Sales} & 17.5\% & 18.2\% & 0.8\% & 0.7\% \\
(20) & \emph{Sales} & 20.0\% & 20.7\% & 0.9\% & 0.7\% \\
(21) & \emph{Sales} & 22.5\% & 23.1\% & 0.9\% & 0.6\% \\
(22) & \emph{Sales} & 25.0\% & 25.5\% & 0.9\% & 0.5\% \\
\cmidrule{1-6}
(23) & \emph{Scientist} & 0.0\% &  2.0\% & 0.6\% & 2.0\% \\
(24) & \emph{Scientist} & 2.5\% &  4.8\% & 0.7\% & 2.3\% \\
(25) & \emph{Scientist} & 5.0\% &  7.3\% & 0.7\% & 2.3\% \\
(26) & \emph{Scientist} & 7.5\% &  9.8\% & 0.8\% & 2.3\% \\
(27) & \emph{Scientist} & 10.0\% & 12.3\% & 0.8\% & 2.3\% \\
(28) & \emph{Scientist} & 12.5\% & 14.7\% & 0.9\% & 2.2\% \\
(29) & \emph{Scientist} & 15.0\% & 17.2\% & 0.9\% & 2.2\% \\
(30) & \emph{Scientist} & 17.5\% & 19.7\% & 1.0\% & 2.2\% \\
(31) & \emph{Scientist} & 20.0\% & 22.1\% & 0.9\% & 2.1\% \\
(32) & \emph{Scientist} & 22.5\% & 24.5\% & 1.0\% & 2.0\% \\
(33) & \emph{Scientist} & 25.0\% & 27.0\% & 1.0\% & 2.0\% \\
\cmidrule[\heavyrulewidth]{1-6}
\end{tabular}
\end{center}
\vspace{-5mm}
\end{supptable}

