The recent advances in large language models (LLMs) attracted significant public and policymaker interest in its adoption patterns.  In this paper, we systematically analyze LLM-assisted writing across four domains—consumer complaints, corporate communications, job postings, and international organization press releases—from January 2022 to September 2024. Our dataset includes 687,241 consumer complaints, 537,413 corporate press releases, 304.3 million job postings, and 15,919 United Nations (UN) press releases.
%
Using a robust population-level statistical framework, we find that LLM usage surged following the release of ChatGPT in November 2022. By late 2024, roughly 18\% of financial consumer complaint text appears to be LLM-assisted, with adoption patterns spread broadly across regions and slightly higher in urban areas. For corporate press releases, up to 24\% of the text is attributable to LLMs. In job postings, LLM-assisted writing accounts for just below 10\% in small firms, and is even more common among younger firms. UN press releases also reflect this trend, with nearly 14\% of content being generated or modified by LLMs.
% 
Although adoption climbed rapidly post-ChatGPT, growth appears to have stabilized by 2024, reflecting either saturation in LLM adoption or increasing subtlety of more advanced models. Our study shows the emergence of a new reality in which firms, consumers and even international organizations substantially rely on generative AI for communications. 


