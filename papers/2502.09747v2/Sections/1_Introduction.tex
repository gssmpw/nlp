


\section*{Introduction}
\label{sec:introduction}


The  emergence of large language models (LLMs) marked a significant moment in artificial intelligence, offering unprecedented capabilities in natural language processing and generation.  
This rapid proliferation of LLMs generated both excitement and concern. On one hand, LLMs have the potential to greatly enhance productivity; in the writing space specifically, it can democratize content creation (especially for non-native speakers). On the other hand, policymakers fear an erosion of trust, risks of biases and discrimination, and job displacement \cite{whitehouse2022ai,bommasani2021opportunities,StanfordAIIndex2024}; businesses worry about reliability and data privacy; academics debate the implications for research integrity and teaching \cite{Dwivedi2023,Kasneci2023}; and the public is concerned about misinformation, deepfakes, and authenticity \cite{bender2021,weidinger2021ethical}. Further complicating the discourse is the question of how LLMs may widen or potentially bridge socioeconomic gaps, given differential access to these advanced technologies.

Although some early adoption stories or isolated examples have drawn significant media attention, and survey studies have explored LLM adoption from an individual user perspective \cite{humlum2024chatgpt, bick2024rapid}, there remains a lack of systematic evidence about the patterns and extent of LLM adoption across various diverse writing domains. While some previous work used commercial software to detect such patterns \cite{brooks2024riseaigeneratedcontentwikipedia, shin2024adoption},  these studies often been constrained to single domains, relied on black-box commercial AI detectors, or analyzed relatively small datasets.
To address this gap, we conduct the first large-scale, systematic analysis of LLM adoption patterns across consumer, firm and institution communications. 
Our analysis leverages a statistical framework validated in our previous work \cite{liang2024monitoring} 
to quantify the prevalence of LLM-modified content. This framework has demonstrated superior robustness, transparency (and lower cost) compared to commercial AI content detectors \cite{liang2024monitoring, liang2024mapping, Liang2023GPTDA}, allowing us to track adoption trajectories and uncover key demographic and organizational factors driving LLM integration. 

We focus on four domains where LLMs are likely to influence communication and decision-making: consumer complaints, corporate press releases, job postings, and United Nations press releases. Consumer complaints offer insight into user–business interactions and show how these technologies may extend beyond AI-powered customer service \cite{Brynjolfsson2023}. Corporate press releases reflect strategic organizational usage, as firms incorporate LLMs into their investor relations, public relations, and broader business communications. Job postings reveal how recruiters and human resource departments harness LLMs, shedding light on broader labor market trends. Finally, UN press releases showcase the growing institutional adoption of AI for regulatory, policy, and public outreach efforts.\footnote{We also conducted a similar analysis of patent applications. However, due to the standard 18-month embargo between application and publication, our study period did not yield sufficient data to draw robust conclusions. Still, in the limited sample of late-2024 published patents, we observed a (very) moderate uptick in LLM-generated text.}

This comprehensive approach reveals several patterns. First, we observe a consistent trajectory across all the analyzed domains: rapid initial adoption following ChatGPT's release, followed by a distinctive stabilizing trend highlighting widespread adoption. One of the remarkable results from our analysis is how similar adoption is between these diverse domains. By the end of the period we analyzed, in the financial dataset we estimate about 18\% of the data was generated by LLM, around 24\% in company press releases, up to 15\% for young and small companies job postings, and 14\% for international organizations. Second, we uncover some heterogeneity in adoption rates across geographic regions, demographic groups, and organizational characteristics. Third, we find that organizational age and size emerge as the most important predictor of differential adoption, with smaller and younger firms showing markedly higher utilization rates. 

Our findings provide crucial insights into the first wave of LLM integration across society, revealing how various socioeconomic and organizational factors shape technology adoption patterns. This understanding is essential for policymakers, business leaders, and researchers as they navigate the implications of AI integration across different sectors of society and work to ensure equitable access to and responsible deployment of these powerful new tools in the future.
