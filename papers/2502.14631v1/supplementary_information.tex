% ****** Start of file aipsamp.tex ******
%
%   This file is part of the AIP files in the AIP distribution for REVTeX 4.
%   Version 4.1 of REVTeX, October 2009
%
%   Copyright (c) 2009 American Institute of Physics.
%
%   See the AIP README file for restrictions and more information.
%
% TeX'ing this file requires that you have AMS-LaTeX 2.0 installed
% as well as the rest of the prerequisites for REVTeX 4.1
% 
% It also requires running BibTeX. The commands are as follows:
%
%  1)  latex  aipsamp
%  2)  bibtex aipsamp
%  3)  latex  aipsamp
%  4)  latex  aipsamp
%
% Use this file as a source of example code for your aip document.
% Use the file aiptemplate.tex as a template for your document.
\documentclass[%
aip,
onecolumn,
% jmp,
% bmf,
% sd,
% rsi,
 amsmath,amssymb,
%preprint,%
% reprint,%
%author-year,%
%author-numerical,%
% Conference Proceedings
]{revtex4-1}

\usepackage{graphicx}% Include figure files
\usepackage{dcolumn}% Align table columns on decimal point
\usepackage{bm}% bold math
%\usepackage[mathlines]{lineno}% Enable numbering of text and display math
%\linenumbers\relax % Commence numbering lines

\usepackage{lipsum}

\usepackage[utf8]{inputenc}
\usepackage[T1]{fontenc}
%\usepackage{mathptmx}
\usepackage{multirow}
\usepackage{booktabs}
\usepackage{xcolor}
\usepackage{hyperref} 
\usepackage{etoolbox}
\newtheorem{definition}{Definition}

\newcommand{\dtoprule}{\specialrule{1pt}{0pt}{0.4pt}%
            \specialrule{0.3pt}{0pt}{\belowrulesep}%
            }
\newcommand{\dbottomrule}{\specialrule{0.3pt}{0pt}{0.4pt}%
            \specialrule{1pt}{0pt}{\belowrulesep}%
            }

\renewcommand{\figurename}{Supplementary Figure}

\renewcommand{\tablename}{Supplementary Table}
\renewcommand{\thetable}{\arabic{table}}

%% Apr 2021: AIP requests that the corresponding 
%% email to be moved after the affiliations
\makeatletter
\def\@email#1#2{%
 \endgroup
 \patchcmd{\titleblock@produce}
  {\frontmatter@RRAPformat}
  {\frontmatter@RRAPformat{\produce@RRAP{*#1\href{mailto:#2}{#2}}}\frontmatter@RRAPformat}
  {}{}
}%
\makeatother
\begin{document}

\preprint{AIP/123-QED}

\title[Supplementary Information]{Supplementary Information}
% Force line breaks with \\

\author{Minh-Quyet Ha}
\affiliation{Japan Advanced Institute of Science and Technology, 1-1 Asahidai, Nomi, Ishikawa 923-1292, Japan}
 
 \author{Dinh-Khiet Le}
\affiliation{Japan Advanced Institute of Science and Technology, 1-1 Asahidai, Nomi, Ishikawa 923-1292, Japan}

 \author{Duc-Anh Dao}
\affiliation{Japan Advanced Institute of Science and Technology, 1-1 Asahidai, Nomi, Ishikawa 923-1292, Japan}

 \author{Tien-Sinh Vu}
\affiliation{Japan Advanced Institute of Science and Technology, 1-1 Asahidai, Nomi, Ishikawa 923-1292, Japan}
 
\author{Duong-Nguyen Nguyen}%
\affiliation{Japan Advanced Institute of Science and Technology, 1-1 Asahidai, Nomi, Ishikawa 923-1292, Japan}

\author{Viet-Cuong Nguyen}
\affiliation{%
HPC SYSTEMS Inc., Minato, Tokyo 108-0022, Japan%\\This line break forced% with \\
}

%\author{Takahiro Nagata}
% \affiliation{RCFM, National Institute for Materials Science, 1-2-1 Sengen, Tsukuba, Ibaraki 305-0044, Japan}
% 
%\author{Toyohiro Chikyow}
% \affiliation{MaDIS, National Institute for Materials Science, 1-2-1 Sengen, Tsukuba, Ibaraki 305-0044, Japan} 
 
\author{Hiori Kino}
 \affiliation{MaDIS, National Institute for Materials Science, 1-2-1 Sengen, Tsukuba, Ibaraki 305-0044, Japan}

%\author{Takashi Miyake}
% \affiliation{CD-FMat, AIST, 1-1-1 Umezono, Tsukuba, Ibaraki 305-8568, Japan}

\author{Thierry Denœux}
 \affiliation{Université de technologie de Compiègne, CNRS, UMR 7253 Heudiasyc, Compiègne, France}

\author{Van-Nam Huynh}
 \affiliation{Japan Advanced Institute of Science and Technology, 1-1 Asahidai, Nomi, Ishikawa 923-1292, Japan}

\author{Hieu-Chi Dam}
\email{dam@jaist.ac.jp}
\affiliation{Japan Advanced Institute of Science and Technology, 1-1 Asahidai, Nomi, Ishikawa 923-1292, Japan}

\date{\today}% It is always \today, today,
             %  but any date may be explicitly specified



\maketitle

\section{\label{sec:example}{Illustrative examples}}

The following examples provide explanations of how the evidence theory work to learn the similarity and infer the HEA formation for new element combinations, identifying equiatomic alloys.

\textbf{Example 1:} Suppose we have collected four pairs of alloys from experiments. Three of those pairs are alloys that both form HEA phase: $pair_1=(\{A^1,B^1,C^1,D\}, \{A^1,B^1,C^1,E\})$; $pair_2=(\{A^2,B^2,C^2,D\},\{A^2,B^2,C^2,E\})$; and $pair_3=(\{A^3,B^3,C^3,D\}, \{A^3,B^3,C^3,E\})$.  The fourth pair $pair_4=(\{A^4,B^4,C^4,D\},\{A^4,B^4,C^4,E\})$ is different from the other three, in which $\{A^4,B^4,C^4,D\}$ forms HEA phase while $\{A^4,B^4,C^4,E\}$ does not form HEA phase. We consider each pair as a source of evidence support that $\{D\}$ is similar to $\{E\}$ in term of substitutability to form the HEA phase. Each evidence is modeled using mass function as follows:

\begin{equation*}
\begin{aligned}
&m^{\{C\},\{D\}}_{pair_1}(\{similar\})=0.1,\\
&m^{\{C\},\{D\}}_{pair_1}(\{dissimilar\})=0,\\
&m^{\{C\},\{D\}}_{pair_1}(\{similar, dissimilar\})=0.9
\end{aligned}
\end{equation*}

\begin{equation*}
\begin{aligned}
&m^{\{C\},\{D\}}_{pair_2}(\{similar\})=0.1,\\
&m^{\{C\},\{D\}}_{pair_2}(\{dissimilar\})=0,\\
&m^{\{C\},\{D\}}_{pair_2}(\{similar, dissimilar\})=0.9
\end{aligned}
\end{equation*}

\begin{equation*}
\begin{aligned}
&m^{\{C\},\{D\}}_{pair_3}(\{similar\})=0.1,\\
&m^{\{C\},\{D\}}_{pair_3}(\{dissimilar\})=0,\\
&m^{\{C\},\{D\}}_{pair_3}(\{similar, dissimilar\})=0.9
\end{aligned}
\end{equation*}

\begin{equation*}
\begin{aligned}
&m^{\{C\},\{D\}}_{pair_4}(\{similar\})=0,\\
&m^{\{C\},\{D\}}_{pair_4}(\{dissimilar\})=0.1,\\
&m^{\{C\},\{D\}}_{pair_4}(\{similar, dissimilar\})=0.9
\end{aligned}
\end{equation*}


The three pieces of evidence are combined using the Dempster' rule of combination to accumulate the believe that $\{D\}$ is similar to $\{E\}$:

\begin{equation*}
\begin{aligned}
&m^{\{C\},\{D\}}(\{similar\})=0.25,\\
&m^{\{C\},\{D\}}(\{dissimilar\})=0.075,\\
&m^{\{C\},\{D\}}(\{similar, dissimilar\})=0.675
\end{aligned}
\end{equation*}

Next, if we observed (included in the data) that the HEA phase exists for alloy $\{G,H,I,D\}$, the ERS (which focuses on finding some chance for discovering new combination of elements that the HEA phase exist and ignores the belief regarding $\neg HEA$) will consider that there is some believe that the HEA phase also exists for $\{G,H,I,E\}$ (by substituting $\{D\}$ with $\{E\}$). The evidence is modeled using mass function as follows:

\begin{equation*}
\begin{aligned}
&m^{\{G,H,I,E\}}_{\{G,H,I,D\}, \{D\}\leftarrow \{E\}}(\{\neg HEA\})=0,\\
&m^{\{G,H,I,E\}}_{\{G,H,I,D\}, \{D\}\leftarrow \{E\}}(\{HEA\})=m^{C,D}(\{similar\})=0.25,\\
&m^{\{G,H,I,E\}}_{\{G,H,I,D\}, \{D\}\leftarrow \{E\}}(\{HEA, \neg HEA\})=1-m^{C,D}(\{similar\})=0.75
\end{aligned}
\end{equation*}

\textbf{Example 2:} In a same manner but for an extrapolative recommendation: if the HEA phases exist for all the alloys in the three following pairs: $pair_1=(\{A^1,B^1,C\}, \{A^1,B^1,D,E\})$, $pair_2=(\{A^2,B^2,C\}, \{A^2,B^2,D,E\})$, $pair_3=(\{A^3,B^3,C\}, \{A^3,B^3,D,E\})$.  In the fourth pair $pair_4=(\{A^4,B^4,C\},\{A^4,B^4,D,E\})$,  $\{A^4,B^4,C\}$ forms HEA phase while $\{A^4,B^4,D,E\}$ does not form HEA phase. The algorithm will accumulate the believe that $\{C\}$ is similar to $\{D,E\}$ as follows: 

\begin{equation*}
\begin{aligned}
&m^{\{C\},\{D,E\}}(\{similar\})=0.25,\\
&m^{\{C\},\{D,E\}}(\{dissimilar\})=0.075,\\
&m^{\{C\},\{D,E\}}(\{similar, dissimilar\})=0.675
\end{aligned}
\end{equation*}

Consequently, if we observed (included in the data) that the HEA phase exists for $\{G,H,I,C\}$, the algorithm (which focuses on finding some chance for discovering new combination of elements that the HEA phase exist and ignores the belief regarding $\neg HEA$) will consider that there is some believe that the HEA phase also exists for $\{G,H,I,D,E\}$ (by substituting $\{C\}$ with $\{D,E\}$).

\begin{equation*}
\begin{aligned}
&m^{\{G,H,I,D,E\}}_{\{G,H,I,C\}, \{C\}\leftarrow \{D,E\}}(\{\neg HEA\})=0,\\
&m^{\{G,H,I,D,E\}}_{\{G,H,I,C\}, \{C\}\leftarrow \{D,E\}}(\{HEA\})=m^{C,D}(\{similar\})=0.25,\\
&m^{\{G,H,I,D,E\}}_{\{G,H,I,C\}, \{C\}\leftarrow \{D,E\}}(\{HEA, \neg HEA\})=1-m^{C,D}(\{similar\})=0.75
\end{aligned}
\end{equation*}

 \section{\label{subsec:dempster_rule}{Combining pieces of evidence using Dempster's rule of combination}}
 
We assume that we can collect $q$ pieces of evidence from $\mathcal{D}$ to compare a specific pair of element combinations, $C_t$ and $C_v$. If no evidence is found, the mass function $m^{C_t,C_v}_{none}$ is initialized, which assigns a probability mass of 1 to subset $\{similar, dissimilar\}$. $m^{C_t,C_v}_{none}$ models the condition under which no information about the similarity (or dissimilarity) between $C_t$ and $C_v$ is available. Any two pieces of evidence $a$ and $b$ modeled by the corresponding mass functions $m_a^{C_t,C_v}$ and $m_b^{C_t,C_v}$ can be combined using the Dempster rule \cite{dempster1968} to assign the joint mass $m_{a,b}^{C_t,C_v}$ to each subset $\omega$ of $\Omega_{sim}$ (i.e. $\{similar\}$, $\{dissimilar\}$, or $\{similar, dissimilar\}$) as follows:
\begin{equation}
\begin{aligned}
m_{a,b}^{C_t,C_v}(\omega)={}& \left(m_a^{C_t,C_v} \oplus m_b^{C_t,C_v} \right)(\omega) \\
={}& { \frac{\sum\limits_{\forall\omega_k \cap \omega_h = \omega} {m_a^{C_t,C_v}(\omega_k) \times m_b^{C_t,C_v}(\omega_h) }}{1 - \sum\limits_{\forall\omega_k \cap \omega_h = \emptyset} {m_a^{C_t,C_v}(\omega_k) \times m_b^{C_t,C_v}(\omega_h)}}},
\end{aligned}
\end{equation}
where $\omega$, $\omega_k$ and $\omega_h$ are subsets of $\Omega_{sim}$. Note that the Dempster rule is commutative and yields the same result by changing the order of $m_a^{C_t,C_v}$ and $m_b^{C_t,C_v}$. All the $q$ obtained mass functions corresponding to the $q$ collected pieces of evidence from available sources are then combined using the Dempster rule to assign the final mass $m^{C_t,C_v}$ as follows: 
\begin{equation}
m^{C_t,C_v}_{\mathcal{D}}(\omega) = \left(m_1^{C_t,C_v} \oplus m_2^{C_t,C_v} \oplus \dots \oplus m_q^{C_t,C_v} \right)(\omega).
\end{equation} 

%Multiple pieces of evidence about the label of each new alloy are combined using the similar manner. We assume that for a specific hypothetical alloy, $A_{new}$, we can collect pieces of evidence about its properties from $\mathcal{D}$ (pair of $A_{host}$ and the corresponding substitution to obtain $A_{new}$ from $A_{host}$). If no evidence is found, $m^{A_{new}}_{none}$ is initialized and a probability mass of 1 is applied to set $\{HEA, \neg{HEA}\}$. $m^{A_{new}}_{none}$ models the condition that no information about the label of $A_{new}$ can be obtained from $\mathcal{D}$. The obtained mass functions for ${A_{new}}$ are then combined using the Dempster rule\cite{dempster1968} to obtain a final mass function $m^{A_{new}}_{\mathcal{D}}$ on $\Omega_{HEA}$.
 
 \section{\label{subsec:descriptors}{Materials descriptors}}

Descriptors, which are the representation of alloys, play a crucial role in building a recommender system to explore potential new HEAs. In this research, the raw data of alloys is represented in the form of elements combination. Several descriptors have been studied in materials informatics to represent the compounds \cite{Seko2018}. To employ the data-driven approaches for this work, we applied compositional descriptor\cite{Seko2017}. % and binary elemental descriptor\cite{Seko2018}.

Compositional descriptor represents an alloy by a set of 135 features composed of means, standard deviations, and covariance of established atomic representations that form the alloy. The descriptor can be applied not only to crystalline systems but also to molecular system. We adopted 15 atomic representations: (1) atomic number, (2) atomic mass,(3) period and (4) group in the periodic table, (5) first ionization energy, (6) second ionization energy, (7) Pauling electronegativity, (8) Allen electronegativity, (9) van der Waals radius, (10) covalentradius, (11) atomic radius, (12) melting point, (13) boiling point, (14) density, and (15) specific heat. However, the compositional descriptor hardly distinguishes compounds which have different numbers of the atom because it is to regard the atomic representations of a compound as distributions of data. Therefore, the compositional descriptor cannot be applied in the case of having extrapolation in the number of components.

%The rating matrix representation, which is a descriptor-free approach, shows a robust performance of recommendations for a wide variety of data sets in the Machine Learning community. Seko et al. adopted the representation to build a recommender system for exploring currently unknown chemically relevant compositions\cite{Seko2018MatrixAT}. In that work, a composition data set needs to be transformed into just two feature sets, which corresponds to users and items in a user-item rating matrix. Ratings of missing elements are approximately predicted based on the similarity of features given by the representation. To build a recommender system for HEA, we first define the candidate alloys as $AB$, where $A$ and $B$ correspond to elementary components of the alloys. We introduce two kinds of matrix representations for the eight alloys data sets. An alloy is decomposed into two elementary components with the following number of elements.
%
%\begin{itemize}
%\item \textit{Type 1}: $|A|\in\{1, 2\}$ and $|B|\in\{1, 2, 3\}$. The numbers of possible components $A$ and $B$ are respectively 378 and 3303. The size of the rating  matrix is $(378\times 3303)$.
%\item \textit{Type 2}: $|A|=1$ and $|B|\in\{1, 2, 3, 4\}$.  The numbers of possible components $A$ and $B$ are 27 and 20853, respectively. The size of the rating  matrix is $(27\times 20853)$.
%\end{itemize}

%Binary elemental descriptors is simply a binary digit representing the presence of chemical elements. The number of binary elemental descriptors corresponds to the number of element types included in the training data. For instance, the quaternary alloys in the dataset $\mathcal{D}_{0.9T_m}$ are composed of twenty-six elements. Thus, an alloy is described by a 26-dimensional binary vector with elements of one or zero.

\section{\label{subsec:alpha_estimation}Tuning hyper-parameter of the predictive models}

Because the datasets used in this study are derived from calculation-based prediction methods, we introduce a degree of uncertainty $\alpha$ into the mass function that models substitutability evidence from material datasets. For each dataset, a grid search is performed to determine the optimal $\alpha$, which best reproduces the alloy labels by achieving the highest cross-validation score. The cross-validation scheme employs a 10-fold cross-validation repeated three times. The search space for $\alpha$ ranges from 0.01 to 0.5, with increments of 0.01.

In contrast, the degree of uncertainty $\beta$ in the mass function representing substitutability evidence from domain knowledge is assigned a fixed value of $1/N_{domains}$, where $N_{domains}$ denotes the number of knowledge domains applied in the study. This ensures balanced weighting of evidence across multiple sources.

Additionally, the grid search parameter tuning is also applied to other traditional models. The parameter grid for logistic regression is as follows:
\begin{itemize}
\item \textbf{Penalty}: {\texttt{l1}, \texttt{l2}}
\item \textbf{Solver}: {
\begin{itemize}
\item \texttt{l1} penalty: [\texttt{liblinear}, \texttt{saga}]
\item \texttt{l2} penalty:
\begin{itemize}
\item \texttt{dual=True}: [\texttt{liblinear}]
\item \texttt{dual=False}: [\texttt{newton-cg}, \texttt{lbfgs}, \texttt{sag}, \texttt{saga}, \texttt{liblinear}]
\end{itemize}
\end{itemize}
}
\item \textbf{Regularization Parameter} $C$: [0.01, 0.1, 1, 10, 100]
\end{itemize}

\section{Hybrid distance measurement for alloys}\label{sec.ers_jaccard_distance_matrix}

To identify the subgroup to which unobserved alloys belong, we propose an approach that constructs an embedding space for both observed and unobserved alloys based on a combined distance matrix $\overline{M}$. This matrix is computed as the element-wise product of the distance matrix obtained from our proposed substitutability-based similarity measurement and the distance matrix derived using the Jaccard index for the binary descriptors of the alloys.

For any pair of alloys $A_i$ and $A_j$, the combined distance is defined as follows:
\begin{equation}
\overline{M}[i,j] = \left(m^{C_t,C_v}(\{\mathrm{dissimilar}\}) + \frac{m^{C_t,C_v}(\{\mathrm{similar}, \mathrm{dissimilar}\})}{2}\right) \times (1 - J(A_i, A_j)).
\end{equation}

Here, $m^{C_t,C_v}$ represents the substitutability between the two element combinations, $C_t = A_i \setminus (A_i \cap A_j) \quad \text{and} \quad C_v = A_j \setminus (A_i \cap A_j)$, obtained through our method, while $J(A_i, A_j)$ denotes the Jaccard similarity coefficient between the element sets of the two alloys.

If one of the alloys in the pair is unobserved in the dataset, our method assigns no similarity information between the alloys, resulting in $m^{C_t,C_v}(\{\mathrm{similar}, \mathrm{dissimilar}\})= 1$ and $m^{C_t,C_v}(\{\mathrm{dissimilar}\}) = 0$. In this case, the combined distance simplifies to:
\begin{equation}
\overline{M}[i,j] = \frac{1 - J(A_i, A_j)}{2}
\end{equation}

This approach allows for the effective grouping of unobserved alloys by leveraging both compositional similarity and substitutability information.


\clearpage


\begin{table*}[t]
\centering
\caption{\label{tab:classification-results-Os}Classification report of our proposed framework for extrapolation experiment of Os-based alloys on the dataset $\mathcal{D}_{0.9T_m}$.}
\begin{tabular*}{\textwidth}{@{\extracolsep{\fill}}lcccc}
%\hline
 &   \textbf{precision}  & \textbf{recall} & \textbf{f1-score} & \textbf{support} \\
\hline
Single-phase & $0.90$ & $0.78$ & $0.84$ & $882$ alloys \\
Multi-phase & $0.88$ & $0.95$ & $0.91$ & $1,418$ alloys \\
 &  &  & &  \\
Accuracy &  &  & $0.88$ & $2,300$ alloys \\
Macro average& $0.89$ & $0.87$ & $0.87$ & $2,300$ alloys \\
Micro average & $0.89$ & $0.88$ & $0.88$ & $2,300$ alloys \\
%\hline
\end{tabular*}
\end{table*}

\clearpage
\begin{figure}[t]
\centering
  \includegraphics[width=\linewidth]{figures/figS1.pdf}
\caption{\label{fig.s1}Proportions of constituent elements in the four alloys datasets $\mathcal{D}_{\text{Mag}}$ (a), $\mathcal{D}_{T_C}$(b), $\mathcal{D}_{\text{0.9}T_{m}}$ (c),  and $\mathcal{D}_{\text{1350K}}$ (d).}
\end{figure}

\clearpage
\begin{figure}[t]
\centering
  \includegraphics[width=\linewidth]{figures/figS2.pdf}
\caption{\label{fig.s2}The area under the ROC curve (AUC) for extrapolation experiments on constituent elements across the four alloy datasets:  $\mathcal{D}_{\text{Mag}}$, $\mathcal{D}_{T_C}$, $\mathcal{D}_{\text{0.9}T_{m}}$,  and $\mathcal{D}_{\text{1350K}}$.}
\end{figure}

\clearpage
\begin{figure}[t]
\centering
  \includegraphics[width=\linewidth]{figures/figS3.pdf}
\caption{\label{fig.s3}(Top) Heatmap illustrating similarity matrices derived from domain knowledge distilled by GPT-4o. The heatmap visualizes the pairwise similarity between elements, based on substitutability evidence extracted from domain-specific knowledge using GPT-4o. The color intensity indicates the degree of similarity, with higher values representing stronger substitutability relationships. (Bottom) Circular hierarchical clustering (HAC) of elements based on substitutability between elements.The circular dendrogram displays the hierarchical clustering of all constituent elements, constructed using hierarchical agglomerative clustering (HAC) with the "complete" linkage criterion. The substitutability information is derived from LLM-based knowledge. Blue labels represent early transition metals, orange labels indicate late transition metals, and red labels denote coinage metals, including copper (Cu), silver (Ag), and gold (Au).}
\end{figure}

\clearpage
\begin{figure}[t]
\centering
  \includegraphics[width=\linewidth]{figures/figS4.pdf}
\caption{\label{fig.s4}Updated map after integrating Os-based alloys into the dataset. The visualization highlights structural changes and reorganization in alloy clusters following the inclusion of Os-based alloys.}
\end{figure}

\clearpage
\begin{figure}[t]
\centering
  \includegraphics[width=\linewidth]{figures/figS5.pdf}
\caption{\label{fig.s5}Analysis of synthesized effects between transition metals in $\mathcal{E}$ and other elements. (a) Summary of the number of single-phase and multi-phase alloys formed by combining transition metals in $\mathcal{E}$ with other elements. (b) Heatmap showing the stability rate (proportion of single-phase alloys) for alloys formed exclusively from modified sets of $\mathcal{E}$, where one element is replaced by silicon (Si), palladium (Pd), tantalum (Ta), niobium (Nb), or titanium (Ti).}
\end{figure}


\clearpage
\bibliography{supplementary_information.bib} %You need to replace "rsc" on this line with the name of your .bib file
\bibliographystyle{rsc} %the RSC's .bst file
\end{document}
%
% ****** End of file aipsamp.tex ******