% !TEX root = paper.tex
\section{Moment estimation for $0<t<1$}\label{sec:small-t}

\subsection{Moment estimation using Proportional Sampling for 1/2<t<1} We use Algorithm \ref{alg:estmoments} to estimate moment $S_t$ for $1/2<t<1$. We adapt the analysis in Section \ref{sec:moments} to prove the following result.

\begin{thm} There exists an algorithm $ALG$ that given proportional sampling access to the weights of the elements of a set $A$ and parameters $1/2<t<1$, $\eps\in (0,1)$, provides an $(\eps,1/3)$-estimate of $S_t$ using $O(\frac{\sqrt{n}}{\eps}+\frac{n^{\frac{1}{t}-1}}{\eps^2})$ samples. \end{thm}

The sample complexity of the algorithm for $1/2<t<1$ is given as follows. Note that the sample complexity of our algorithm is given as $\frac{W (||w(A)||_{2t-1})^{2t-1}}{\eps^2 (||w(A)||_t)^{2t}}$. We can rewrite it as $\frac{||w(A)||_1}{\eps^2 ||w(A)||_t} \cdot (\frac{||w(A)||_{2t-1}}{||w(A)||_t})^{2t-1}$. Using Fact \ref{fact:norms}, this is upper bounded as $n^{\frac{1}{t}-1}/\eps^2$. Overall sample complexity of the $(\eps,1/3)$-estimator of the moment is $O(\sqrt{n}/\eps+ n^{\frac{1}{t}-1}/\eps^2)$. Using the median trick we transform the $(\eps,1/3)$-estimator to an $(\eps,\delta)$-estimator with sample complexity $O((\sqrt{n}/\eps+ n^{\frac{1}{t}-1}/\eps^2) \ln 1/\delta)$.


\subsection{Lower bound for $t\leq 1/2$} We show that there are no sublinear algorithms for the problem when $t\leq 1/2$.

\begin{thm} For any $\eps>0$ and $t\leq 1/2$, any randomized algorithm that computes an $(\eps,1/3)$-estimate of $S_t$ requires $\Omega(n)$ proportional samples. \end{thm}
\begin{proof} We construct two families of instances which we show are hard to be distinguished using proportional sampling. In one instance we have one element with weight $(n-1)$, and the rest $(n-1)$ elements with weight $0$. In another instance, we have one element of weight $(n-1)$, and $(n-1)$ elements have weight $\frac{\eps^{1/t}}{n-1}$. The moment values of the two instances are given as $(n-1)^t$ and $(n-1)^t + \eps (n-1)^{1-t}\geq (1+\eps) (n-1)^t$ as $t\leq 1/2$. Hence, there exists a $(1+\eps)$-multiplicative gap between the moment values of these two instances.

These two instances are distinguished using proportional sampling once we sample an element of weight $\frac{\eps^{1/t}}{n-1}$. The probability of sampling such an element is at most $\frac{\eps^{1/t}}{\eps^{1/t} + (n-1)}$. Hence, one requires $\Omega(n)$ samples to distinguish the two instances with constant probability. \end{proof}

