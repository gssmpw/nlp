% !TEX root = paper.tex
\section{Characterization of Sample Complexity}\label{sec:characterize}

We define a {\it moment-density} parameter of the input that governs the sample complexity of the moment estimation problem using proportional sampling. For $S_t=\sum_{a\in A} w(a)^t$ and $W=\sum_{a\in A} w(a)$, we define the moment-density parameter as $$\rho=\max_{L\subseteq A} \frac{\frac{\sum_{a\in L} w(a)^t}{S_t}}{\frac{\sum_{a\in L}w(a)}{W}}=\max_{L\subseteq A} \frac{\sum_{a\in L} w(a)^t}{\sum_{a\in L} w(a)} \cdot \frac{W}{S_t}$$ 

We give an upper bound for an $(\eps,\delta)$-estimator of $S_t$ using $O(({\sqrt{n}}/{\eps}+\rho/\eps^2)\ln 1/\delta)$ proportional samples. 

\begin{thm}\label{thm:char-upper} There exists an algorithm $ALG$ that given proportional sampling access to the weights of elements of $A$ having moment-density parameter $\rho$ and parameters $t>1,\eps,\delta\in (0,1)$, provides an $(\eps,\delta)$-estimate of $S_t$ using $O(({\sqrt{n}}/{\eps}+\rho/\eps^2)\ln 1/\delta)$ proportional samples. \end{thm}

\begin{proof} Algorithm \ref{alg:estmoments} gives the above sample complexity bound. We adapt the calculations from Section \ref{sec:moments}. Following Equation \ref{eqn:upper-sample}, we can write the sample complexity of our $(\eps,1/3)$ estimator to be at most $O(\sqrt{n}/\eps + \frac{1}{\eps^2} \frac{W\cdot \sum_{a\in A} w(a)^{2t-1}}{S_t^2})$. We will show that $\frac{W\cdot \sum_{a\in A} w(a)^{2t-1}}{S_t^2} \leq \rho$.  
\begin{align*}
\frac{W\cdot \sum_{a\in A} w(a)^{2t-1}}{S_t^2}
& = \frac{W}{S_t} \cdot \frac{\sum_{a\in A} w(a)^{2t-1}}{\sum_{a\in A} w(a)^t}\\
%& = \frac{W}{S_t} \cdot \frac{\sum_{a\in A} w(a)^{t-1} \cdot w(a)^t}{\sum_{a\in A} w(a)^{t-1} \cdot w(a)}\\
%& = \frac{W}{S_t} \cdot \frac{\sum_{a\in A} \frac{w(a)^{t-1}}{\sum_{b\in A} w(b)^{t-1}} \cdot w(a)^t}{\sum_{a\in A} \frac{w(a)^{t-1}}{\sum_{b\in A} w(b)^{t-1}} \cdot w(a)}\\
& \leq \frac{W}{S_t} \cdot \max_{a\in A} \frac{w(a)^t}{w(a)}\\
&= \rho
\end{align*}

Therefore, the sample complexity of our $(\eps,1/3)$-estimate of $S_t$ is $O({\sqrt{n}}/{\eps}+\rho/\eps^2)$. The $(\eps,1/3)$-estimate of the moment can be improved to an $(\eps,\delta)$-estimate with a multiplicative $\ln 1/\delta$-factor sample complexity overhead using the standard median trick.
\end{proof}
Next, we show a $\Omega(\frac{\rho \ln 1/\delta}{\eps})$ lower bound on the sample complexity of any algorithm for the $(\eps,\delta)$-moment estimation problem on any instance with the moment-density parameter $\rho$. We construct two families of input distributions that are hard to be distinguished using a small number of proportional samples. The lower bound instances for the moment estimation using proportional sampling described in Section \ref{sec:lower-proportional} have the moment-density parameter values as $\rho=1$ and $\rho=O(n^{1-1/t}/\eps)$. Next, we show that similar lower bounds hold even when we restrict the input instances to have their moment-density parameter values to be within a constant factor.

\begin{thm}\label{thm:char-lower} For any $\rho$, $\eps,\delta>0$ and $t>1$, any randomized algorithm for an $(\eps,\delta)$-estimate of $S_t$ on an instance with the moment-density parameter $\rho$ requires $\Omega(\frac{\rho \ln 1/\delta}{\eps})$ proportional samples. \end{thm}
\begin{proof} 
  Given any $\rho,\eps,\delta$ as input, we construct two families of instances for which their moment-density parameter values $\rho_1,\rho_2=O(\rho)$ and their moment values $S_t$ differ by a $(1\pm \eps)$ factor. There are $n$ elements in both the instances. In the first instance $n_1$ elements will have weight $d_1$ and $n_2$ elements will have value $d_2$. In the second instance there are $n_1$ elements of weight $d_1$, $\frac{n_2}{3}$ elements of weight $d_2$, and $\frac{2n_2}{3}$ elements of weight $0$. We set the values of the parameters $n_1,n_2,d_1,d_2$ as follows:

\begin{gather}
\begin{align*}
\qquad \qquad \quad n_1 & =  \frac{n^2}{n +(3 \eps)^{\frac{2t-1}{t-1}}} & d_1 & =n^{1-1/t}( 3\eps)^{1/(t-1)}\\
\qquad \qquad \quad n_2 & =  \frac{n (3\eps)^{\frac{2t-1}{t-1}}}{n +(3 \eps)^{\frac{2t-1}{t-1}}} & d_2&=n
\end{align*}
\end{gather}

\noindent The moment value for the first instance is:
\begin{align*}
S_t = n_1d_1^t+n_2d_2^t&=\frac{n^2}{n +(3 \eps)^{\frac{2t-1}{t-1}}}\cdot n^{t-1}(3\eps)^{\frac{t}{t-1}}+\frac{n^{t+1} (3\eps)^{\frac{2t-1}{t-1}}}{n +(3 \eps)^{\frac{2t-1}{t-1}}} \\
&= \frac{n^{t+1}(3\eps)^{\frac{t}{t-1}}+n^{t+1}(3\eps)^{\frac{2t-1}{t-1}}}{n +(3 \eps)^{\frac{2t-1}{t-1}}}\\
&=\frac{n^{t+1}(3\eps)^\frac{t}{t-1}}{n+(3\eps)^{\frac{2t-1}{t-1}}}(1+3\eps)
\end{align*}

\noindent The moment value for the second instance is:
\begin{align*}
S_t = n_1d_1^t+\frac{n_2}{3}d_2^t+\frac{2}{3}n_2\cdot0 &=\frac{n^2}{n +(3 \eps)^{\frac{2t-1}{t-1}}}\cdot n^{t-1}(3\eps)^{\frac{t}{t-1}}+\frac{1}{3}\frac{n^{t+1} (3\eps)^{\frac{2t-1}{t-1}}}{n +(3 \eps)^{\frac{2t-1}{t-1}}}\\
&=\frac{n^{t+1}(3\eps)^\frac{t}{t-1}}{n+(3\eps)^{\frac{2t-1}{t-1}}}(1+\frac{3\eps}{3})\\
&=\frac{n^{t+1}(3\eps)^\frac{t}{t-1}}{n+(3\eps)^{\frac{2t-1}{t-1}}}(1+\eps)
\end{align*}

The moment values of the above two instances differ by a factor of $\frac{1+3\eps}{1+\eps}>1+\eps$. Let $\rho_1$ and $\rho_2$ be the moment-density parameters for these two instances, respectively. We compute the $\rho$ values of the instances as follows. 
\begin{align*}
\rho_1 & = \frac{n^t}{n} \cdot \frac{n_1 d_1 + n_2 d_2}{n_1 d_1^t+n_2 d_2^t}\\
& = n^{t-1} \cdot \frac{\frac{n^2}{n +(3 \eps)^{\frac{2t-1}{t-1}}} \cdot n^{1-1/t}( 3\eps)^{1/(t-1)} + \frac{n (3\eps)^{\frac{2t-1}{t-1}}}{n +(3 \eps)^{\frac{2t-1}{t-1}}}\cdot n }{\frac{n^2}{n +(3 \eps)^{\frac{2t-1}{t-1}}} \cdot n^{t-1}( 3\eps)^{t/(t-1)} + \frac{n (3\eps)^{\frac{2t-1}{t-1}}}{n +(3 \eps)^{\frac{2t-1}{t-1}}}\cdot n^t}\\
&=\frac{n^{1-1/t}+(3\eps)^2}{3\eps+(3\eps)^2}=\frac{n^{1-1/t}+9\eps^2}{3\eps+9\eps^2}
\end{align*}

Similarly, we compute $\rho_2$ as
\begin{align*}
\rho_2 & =\frac{n^t}{n} \cdot \frac{n_1 \cdot d_1 + n_2/3 \cdot d_2 + 2n_2/3 \cdot 0}{n_1 \cdot d_1^t + n_2/3 \cdot d_2^t + 2n_2/3 \cdot 0}\\ 
& = \frac{n^{1-1/t}+\frac{1}{3}(3\eps)^2}{3\eps(1+\eps)}= \frac{n^{1-1/t}+3\eps^2}{3\eps+3\eps^2} 
\end{align*}

We have two instances as given above where their $\rho$ values are within a constant factor of each other and the moment values differ by at least a factor of $(1+\eps)$. These instances using proportional sampling remain indistinguishable until an element with weight $d_2$ is sampled. The probability of sampling an element of weight $d_2$ using proportional sampling is at most:
\begin{align*}
\frac{n_2d_2}{n_1d_1+n_2d_2}=\frac{1}{1+\frac{n_1d_1}{n_2d_2}}=\frac{1}{1+\frac{n^2n^{1-1/t}(3\eps)^{1/(t-1)}}{n^2(3\eps)^{\frac{2t-1}{t-1}}}}=\frac{1}{1+\frac{n^{1-1/t}}{9\eps^2}}\leq \frac{1}{1+O(\frac{\rho}{\eps})}
\end{align*} 

Therefore, we require at least $\Omega(\rho/\eps)$ samples to distinguish the two instances with constant probability of success. Any algorithm giving an $(\eps,\delta)$-estimate would need to make $\Omega (\frac{\rho \ln 1/\delta}{\eps})$ proportional samples.
\end{proof}
