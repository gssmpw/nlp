\documentclass{article}

\usepackage[T1]{fontenc}
\usepackage{graphicx}


%
\setlength\unitlength{1mm}
\newcommand{\twodots}{\mathinner {\ldotp \ldotp}}
% bb font symbols
\newcommand{\Rho}{\mathrm{P}}
\newcommand{\Tau}{\mathrm{T}}

\newfont{\bbb}{msbm10 scaled 700}
\newcommand{\CCC}{\mbox{\bbb C}}

\newfont{\bb}{msbm10 scaled 1100}
\newcommand{\CC}{\mbox{\bb C}}
\newcommand{\PP}{\mbox{\bb P}}
\newcommand{\RR}{\mbox{\bb R}}
\newcommand{\QQ}{\mbox{\bb Q}}
\newcommand{\ZZ}{\mbox{\bb Z}}
\newcommand{\FF}{\mbox{\bb F}}
\newcommand{\GG}{\mbox{\bb G}}
\newcommand{\EE}{\mbox{\bb E}}
\newcommand{\NN}{\mbox{\bb N}}
\newcommand{\KK}{\mbox{\bb K}}
\newcommand{\HH}{\mbox{\bb H}}
\newcommand{\SSS}{\mbox{\bb S}}
\newcommand{\UU}{\mbox{\bb U}}
\newcommand{\VV}{\mbox{\bb V}}


\newcommand{\yy}{\mathbbm{y}}
\newcommand{\xx}{\mathbbm{x}}
\newcommand{\zz}{\mathbbm{z}}
\newcommand{\sss}{\mathbbm{s}}
\newcommand{\rr}{\mathbbm{r}}
\newcommand{\pp}{\mathbbm{p}}
\newcommand{\qq}{\mathbbm{q}}
\newcommand{\ww}{\mathbbm{w}}
\newcommand{\hh}{\mathbbm{h}}
\newcommand{\vvv}{\mathbbm{v}}

% Vectors

\newcommand{\av}{{\bf a}}
\newcommand{\bv}{{\bf b}}
\newcommand{\cv}{{\bf c}}
\newcommand{\dv}{{\bf d}}
\newcommand{\ev}{{\bf e}}
\newcommand{\fv}{{\bf f}}
\newcommand{\gv}{{\bf g}}
\newcommand{\hv}{{\bf h}}
\newcommand{\iv}{{\bf i}}
\newcommand{\jv}{{\bf j}}
\newcommand{\kv}{{\bf k}}
\newcommand{\lv}{{\bf l}}
\newcommand{\mv}{{\bf m}}
\newcommand{\nv}{{\bf n}}
\newcommand{\ov}{{\bf o}}
\newcommand{\pv}{{\bf p}}
\newcommand{\qv}{{\bf q}}
\newcommand{\rv}{{\bf r}}
\newcommand{\sv}{{\bf s}}
\newcommand{\tv}{{\bf t}}
\newcommand{\uv}{{\bf u}}
\newcommand{\wv}{{\bf w}}
\newcommand{\vv}{{\bf v}}
\newcommand{\xv}{{\bf x}}
\newcommand{\yv}{{\bf y}}
\newcommand{\zv}{{\bf z}}
\newcommand{\zerov}{{\bf 0}}
\newcommand{\onev}{{\bf 1}}

% Matrices

\newcommand{\Am}{{\bf A}}
\newcommand{\Bm}{{\bf B}}
\newcommand{\Cm}{{\bf C}}
\newcommand{\Dm}{{\bf D}}
\newcommand{\Em}{{\bf E}}
\newcommand{\Fm}{{\bf F}}
\newcommand{\Gm}{{\bf G}}
\newcommand{\Hm}{{\bf H}}
\newcommand{\Id}{{\bf I}}
\newcommand{\Jm}{{\bf J}}
\newcommand{\Km}{{\bf K}}
\newcommand{\Lm}{{\bf L}}
\newcommand{\Mm}{{\bf M}}
\newcommand{\Nm}{{\bf N}}
\newcommand{\Om}{{\bf O}}
\newcommand{\Pm}{{\bf P}}
\newcommand{\Qm}{{\bf Q}}
\newcommand{\Rm}{{\bf R}}
\newcommand{\Sm}{{\bf S}}
\newcommand{\Tm}{{\bf T}}
\newcommand{\Um}{{\bf U}}
\newcommand{\Wm}{{\bf W}}
\newcommand{\Vm}{{\bf V}}
\newcommand{\Xm}{{\bf X}}
\newcommand{\Ym}{{\bf Y}}
\newcommand{\Zm}{{\bf Z}}

% Calligraphic

\newcommand{\Ac}{{\cal A}}
\newcommand{\Bc}{{\cal B}}
\newcommand{\Cc}{{\cal C}}
\newcommand{\Dc}{{\cal D}}
\newcommand{\Ec}{{\cal E}}
\newcommand{\Fc}{{\cal F}}
\newcommand{\Gc}{{\cal G}}
\newcommand{\Hc}{{\cal H}}
\newcommand{\Ic}{{\cal I}}
\newcommand{\Jc}{{\cal J}}
\newcommand{\Kc}{{\cal K}}
\newcommand{\Lc}{{\cal L}}
\newcommand{\Mc}{{\cal M}}
\newcommand{\Nc}{{\cal N}}
\newcommand{\nc}{{\cal n}}
\newcommand{\Oc}{{\cal O}}
\newcommand{\Pc}{{\cal P}}
\newcommand{\Qc}{{\cal Q}}
\newcommand{\Rc}{{\cal R}}
\newcommand{\Sc}{{\cal S}}
\newcommand{\Tc}{{\cal T}}
\newcommand{\Uc}{{\cal U}}
\newcommand{\Wc}{{\cal W}}
\newcommand{\Vc}{{\cal V}}
\newcommand{\Xc}{{\cal X}}
\newcommand{\Yc}{{\cal Y}}
\newcommand{\Zc}{{\cal Z}}

% Bold greek letters

\newcommand{\alphav}{\hbox{\boldmath$\alpha$}}
\newcommand{\betav}{\hbox{\boldmath$\beta$}}
\newcommand{\gammav}{\hbox{\boldmath$\gamma$}}
\newcommand{\deltav}{\hbox{\boldmath$\delta$}}
\newcommand{\etav}{\hbox{\boldmath$\eta$}}
\newcommand{\lambdav}{\hbox{\boldmath$\lambda$}}
\newcommand{\epsilonv}{\hbox{\boldmath$\epsilon$}}
\newcommand{\nuv}{\hbox{\boldmath$\nu$}}
\newcommand{\muv}{\hbox{\boldmath$\mu$}}
\newcommand{\zetav}{\hbox{\boldmath$\zeta$}}
\newcommand{\phiv}{\hbox{\boldmath$\phi$}}
\newcommand{\psiv}{\hbox{\boldmath$\psi$}}
\newcommand{\thetav}{\hbox{\boldmath$\theta$}}
\newcommand{\tauv}{\hbox{\boldmath$\tau$}}
\newcommand{\omegav}{\hbox{\boldmath$\omega$}}
\newcommand{\xiv}{\hbox{\boldmath$\xi$}}
\newcommand{\sigmav}{\hbox{\boldmath$\sigma$}}
\newcommand{\piv}{\hbox{\boldmath$\pi$}}
\newcommand{\rhov}{\hbox{\boldmath$\rho$}}
\newcommand{\upsilonv}{\hbox{\boldmath$\upsilon$}}

\newcommand{\Gammam}{\hbox{\boldmath$\Gamma$}}
\newcommand{\Lambdam}{\hbox{\boldmath$\Lambda$}}
\newcommand{\Deltam}{\hbox{\boldmath$\Delta$}}
\newcommand{\Sigmam}{\hbox{\boldmath$\Sigma$}}
\newcommand{\Phim}{\hbox{\boldmath$\Phi$}}
\newcommand{\Pim}{\hbox{\boldmath$\Pi$}}
\newcommand{\Psim}{\hbox{\boldmath$\Psi$}}
\newcommand{\Thetam}{\hbox{\boldmath$\Theta$}}
\newcommand{\Omegam}{\hbox{\boldmath$\Omega$}}
\newcommand{\Xim}{\hbox{\boldmath$\Xi$}}


% Sans Serif small case

\newcommand{\Gsf}{{\sf G}}

\newcommand{\asf}{{\sf a}}
\newcommand{\bsf}{{\sf b}}
\newcommand{\csf}{{\sf c}}
\newcommand{\dsf}{{\sf d}}
\newcommand{\esf}{{\sf e}}
\newcommand{\fsf}{{\sf f}}
\newcommand{\gsf}{{\sf g}}
\newcommand{\hsf}{{\sf h}}
\newcommand{\isf}{{\sf i}}
\newcommand{\jsf}{{\sf j}}
\newcommand{\ksf}{{\sf k}}
\newcommand{\lsf}{{\sf l}}
\newcommand{\msf}{{\sf m}}
\newcommand{\nsf}{{\sf n}}
\newcommand{\osf}{{\sf o}}
\newcommand{\psf}{{\sf p}}
\newcommand{\qsf}{{\sf q}}
\newcommand{\rsf}{{\sf r}}
\newcommand{\ssf}{{\sf s}}
\newcommand{\tsf}{{\sf t}}
\newcommand{\usf}{{\sf u}}
\newcommand{\wsf}{{\sf w}}
\newcommand{\vsf}{{\sf v}}
\newcommand{\xsf}{{\sf x}}
\newcommand{\ysf}{{\sf y}}
\newcommand{\zsf}{{\sf z}}


% mixed symbols

\newcommand{\sinc}{{\hbox{sinc}}}
\newcommand{\diag}{{\hbox{diag}}}
\renewcommand{\det}{{\hbox{det}}}
\newcommand{\trace}{{\hbox{tr}}}
\newcommand{\sign}{{\hbox{sign}}}
\renewcommand{\arg}{{\hbox{arg}}}
\newcommand{\var}{{\hbox{var}}}
\newcommand{\cov}{{\hbox{cov}}}
\newcommand{\Ei}{{\rm E}_{\rm i}}
\renewcommand{\Re}{{\rm Re}}
\renewcommand{\Im}{{\rm Im}}
\newcommand{\eqdef}{\stackrel{\Delta}{=}}
\newcommand{\defines}{{\,\,\stackrel{\scriptscriptstyle \bigtriangleup}{=}\,\,}}
\newcommand{\<}{\left\langle}
\renewcommand{\>}{\right\rangle}
\newcommand{\herm}{{\sf H}}
\newcommand{\trasp}{{\sf T}}
\newcommand{\transp}{{\sf T}}
\renewcommand{\vec}{{\rm vec}}
\newcommand{\Psf}{{\sf P}}
\newcommand{\SINR}{{\sf SINR}}
\newcommand{\SNR}{{\sf SNR}}
\newcommand{\MMSE}{{\sf MMSE}}
\newcommand{\REF}{{\RED [REF]}}

% Markov chain
\usepackage{stmaryrd} % for \mkv 
\newcommand{\mkv}{-\!\!\!\!\minuso\!\!\!\!-}

% Colors

\newcommand{\RED}{\color[rgb]{1.00,0.10,0.10}}
\newcommand{\BLUE}{\color[rgb]{0,0,0.90}}
\newcommand{\GREEN}{\color[rgb]{0,0.80,0.20}}

%%%%%%%%%%%%%%%%%%%%%%%%%%%%%%%%%%%%%%%%%%
\usepackage{hyperref}
\hypersetup{
    bookmarks=true,         % show bookmarks bar?
    unicode=false,          % non-Latin characters in AcrobatÕs bookmarks
    pdftoolbar=true,        % show AcrobatÕs toolbar?
    pdfmenubar=true,        % show AcrobatÕs menu?
    pdffitwindow=false,     % window fit to page when opened
    pdfstartview={FitH},    % fits the width of the page to the window
%    pdftitle={My title},    % title
%    pdfauthor={Author},     % author
%    pdfsubject={Subject},   % subject of the document
%    pdfcreator={Creator},   % creator of the document
%    pdfproducer={Producer}, % producer of the document
%    pdfkeywords={keyword1} {key2} {key3}, % list of keywords
    pdfnewwindow=true,      % links in new window
    colorlinks=true,       % false: boxed links; true: colored links
    linkcolor=red,          % color of internal links (change box color with linkbordercolor)
    citecolor=green,        % color of links to bibliography
    filecolor=blue,      % color of file links
    urlcolor=blue           % color of external links
}
%%%%%%%%%%%%%%%%%%%%%%%%%%%%%%%%%%%%%%%%%%%


\allowdisplaybreaks

\usepackage{color}
\renewcommand\UrlFont{\color{blue}\rmfamily}
\urlstyle{rm}

\begin{document}

\title{Improved Sublinear-time Moment Estimation using Weighted Sampling} %TODO Please add

%\titlerunning{Dummy short title} %TODO optional, please use if title is longer than one line

\author{
Anup Bhattacharya\thanks{Authors are ordered alphabetically}\\ NISER, Bhubaneswar, India \\ anup@niser.ac.in \and
Pinki Pradhan \\ NISER, Bhubaneswar, India \\ pinki.pradhan@niser.ac.in}

%\institute{National Institute of Science Education and Research, An OCC of Homi Bhabha National Institute, Bhubaneswar, India\\\email{\{anup, pinki.pradhan\}@niser.ac.in}}

%\author{Jane {Open Access}}{Dummy University Computing Laboratory, [optional: Address], Country \and My second affiliation, Country \and \url{http://www.myhomepage.edu} }{johnqpublic@dummyuni.org}{https://orcid.org/0000-0002-1825-0097}{(Optional) author-specific funding acknowledgements}%TODO mandatory, please use full name; only 1 author per \author macro; first two parameters are mandatory, other parameters can be empty. Please provide at least the name of the affiliation and the country. The full address is optional. Use additional curly braces to indicate the correct name splitting when the last name consists of multiple name parts.

%\author{Joan R. Public\footnote{Optional footnote, e.g. to mark corresponding author}}{Department of Informatics, Dummy College, [optional: Address], Country}{joanrpublic@dummycollege.org}{[orcid]}{[funding]}

%\authorrunning{J. Open Access and J.\,R. Public} %TODO mandatory. First: Use abbreviated first/middle names. Second (only in severe cases): Use first author plus 'et al.'


%\relatedversiondetails[linktext={opt. text shown instead of the URL}, cite=DBLP:books/mk/GrayR93]{Classification (e.g. Full Version, Extended Version, Previous Version}{URL to related version} %linktext and cite are optional

%\supplement{}%optional, e.g. related research data, source code, ... hosted on a repository like zenodo, figshare, GitHub, ...
%\supplementdetails[linktext={opt. text shown instead of the URL}, cite=DBLP:books/mk/GrayR93, subcategory={Description, Subcategory}, swhid={Software Heritage Identifier}]{General Classification (e.g. Software, Dataset, Model, ...)}{URL to related version} %linktext, cite, and subcategory are optional

%\funding{(Optional) general funding statement \dots}%optional, to capture a funding statement, which applies to all authors. Please enter author specific funding statements as fifth argument of the \author macro.

%\acknowledgements{I want to thank \dots}%optional

%\nolinenumbers %uncomment to disable line numbering



\maketitle

\begin{abstract}  
Test time scaling is currently one of the most active research areas that shows promise after training time scaling has reached its limits.
Deep-thinking (DT) models are a class of recurrent models that can perform easy-to-hard generalization by assigning more compute to harder test samples.
However, due to their inability to determine the complexity of a test sample, DT models have to use a large amount of computation for both easy and hard test samples.
Excessive test time computation is wasteful and can cause the ``overthinking'' problem where more test time computation leads to worse results.
In this paper, we introduce a test time training method for determining the optimal amount of computation needed for each sample during test time.
We also propose Conv-LiGRU, a novel recurrent architecture for efficient and robust visual reasoning. 
Extensive experiments demonstrate that Conv-LiGRU is more stable than DT, effectively mitigates the ``overthinking'' phenomenon, and achieves superior accuracy.
\end{abstract}  
\section{Introduction}


\begin{figure}[t]
\centering
\includegraphics[width=0.6\columnwidth]{figures/evaluation_desiderata_V5.pdf}
\vspace{-0.5cm}
\caption{\systemName is a platform for conducting realistic evaluations of code LLMs, collecting human preferences of coding models with real users, real tasks, and in realistic environments, aimed at addressing the limitations of existing evaluations.
}
\label{fig:motivation}
\end{figure}

\begin{figure*}[t]
\centering
\includegraphics[width=\textwidth]{figures/system_design_v2.png}
\caption{We introduce \systemName, a VSCode extension to collect human preferences of code directly in a developer's IDE. \systemName enables developers to use code completions from various models. The system comprises a) the interface in the user's IDE which presents paired completions to users (left), b) a sampling strategy that picks model pairs to reduce latency (right, top), and c) a prompting scheme that allows diverse LLMs to perform code completions with high fidelity.
Users can select between the top completion (green box) using \texttt{tab} or the bottom completion (blue box) using \texttt{shift+tab}.}
\label{fig:overview}
\end{figure*}

As model capabilities improve, large language models (LLMs) are increasingly integrated into user environments and workflows.
For example, software developers code with AI in integrated developer environments (IDEs)~\citep{peng2023impact}, doctors rely on notes generated through ambient listening~\citep{oberst2024science}, and lawyers consider case evidence identified by electronic discovery systems~\citep{yang2024beyond}.
Increasing deployment of models in productivity tools demands evaluation that more closely reflects real-world circumstances~\citep{hutchinson2022evaluation, saxon2024benchmarks, kapoor2024ai}.
While newer benchmarks and live platforms incorporate human feedback to capture real-world usage, they almost exclusively focus on evaluating LLMs in chat conversations~\citep{zheng2023judging,dubois2023alpacafarm,chiang2024chatbot, kirk2024the}.
Model evaluation must move beyond chat-based interactions and into specialized user environments.



 

In this work, we focus on evaluating LLM-based coding assistants. 
Despite the popularity of these tools---millions of developers use Github Copilot~\citep{Copilot}---existing
evaluations of the coding capabilities of new models exhibit multiple limitations (Figure~\ref{fig:motivation}, bottom).
Traditional ML benchmarks evaluate LLM capabilities by measuring how well a model can complete static, interview-style coding tasks~\citep{chen2021evaluating,austin2021program,jain2024livecodebench, white2024livebench} and lack \emph{real users}. 
User studies recruit real users to evaluate the effectiveness of LLMs as coding assistants, but are often limited to simple programming tasks as opposed to \emph{real tasks}~\citep{vaithilingam2022expectation,ross2023programmer, mozannar2024realhumaneval}.
Recent efforts to collect human feedback such as Chatbot Arena~\citep{chiang2024chatbot} are still removed from a \emph{realistic environment}, resulting in users and data that deviate from typical software development processes.
We introduce \systemName to address these limitations (Figure~\ref{fig:motivation}, top), and we describe our three main contributions below.


\textbf{We deploy \systemName in-the-wild to collect human preferences on code.} 
\systemName is a Visual Studio Code extension, collecting preferences directly in a developer's IDE within their actual workflow (Figure~\ref{fig:overview}).
\systemName provides developers with code completions, akin to the type of support provided by Github Copilot~\citep{Copilot}. 
Over the past 3 months, \systemName has served over~\completions suggestions from 10 state-of-the-art LLMs, 
gathering \sampleCount~votes from \userCount~users.
To collect user preferences,
\systemName presents a novel interface that shows users paired code completions from two different LLMs, which are determined based on a sampling strategy that aims to 
mitigate latency while preserving coverage across model comparisons.
Additionally, we devise a prompting scheme that allows a diverse set of models to perform code completions with high fidelity.
See Section~\ref{sec:system} and Section~\ref{sec:deployment} for details about system design and deployment respectively.



\textbf{We construct a leaderboard of user preferences and find notable differences from existing static benchmarks and human preference leaderboards.}
In general, we observe that smaller models seem to overperform in static benchmarks compared to our leaderboard, while performance among larger models is mixed (Section~\ref{sec:leaderboard_calculation}).
We attribute these differences to the fact that \systemName is exposed to users and tasks that differ drastically from code evaluations in the past. 
Our data spans 103 programming languages and 24 natural languages as well as a variety of real-world applications and code structures, while static benchmarks tend to focus on a specific programming and natural language and task (e.g. coding competition problems).
Additionally, while all of \systemName interactions contain code contexts and the majority involve infilling tasks, a much smaller fraction of Chatbot Arena's coding tasks contain code context, with infilling tasks appearing even more rarely. 
We analyze our data in depth in Section~\ref{subsec:comparison}.



\textbf{We derive new insights into user preferences of code by analyzing \systemName's diverse and distinct data distribution.}
We compare user preferences across different stratifications of input data (e.g., common versus rare languages) and observe which affect observed preferences most (Section~\ref{sec:analysis}).
For example, while user preferences stay relatively consistent across various programming languages, they differ drastically between different task categories (e.g. frontend/backend versus algorithm design).
We also observe variations in user preference due to different features related to code structure 
(e.g., context length and completion patterns).
We open-source \systemName and release a curated subset of code contexts.
Altogether, our results highlight the necessity of model evaluation in realistic and domain-specific settings.





% !TEX root = paper.tex
\section{Estimation of Moments using Proportional Sampling}\label{sec:moments}

We describe our algorithm for the moment estimation problem using proportional sampling. We mentioned earlier that for $t>1$, our upper bounds match those of Aliakbarpour~\etal~\cite{ABGPRY2018}. However, since our algorithm works in strictly more general settings and the algorithm for $1/2< t<1$ uses the same ideas, we describe it in detail. Let $A$ be a set of $n$ weighted elements. We assume access to a proportional sampling oracle on the weights of the elements in $A$. For a proportional sample, the oracle returns an element $a_j\in A$ with probability $w(a_j)/W$, where $W=\sum_{a_j\in A} w(a_j)$. Given parameters $t>1,\eps,\delta\in (0,1)$, we design an $(\eps,\delta)$-estimate of $S_t=\sum_{a_j\in A} w(a_j)^t$.

\begin{thm} There exists an algorithm $ALG$ that given proportional sampling access to the weights of the elements in a set $A$ and parameters $t>1,\eps,\delta\in (0,1)$, provides an $(\eps,\delta)$-estimate of $S_t$ using $O(\frac{\sqrt{n}\log 1/\delta}{\eps} + \frac{n^{1-1/t} \log 1/\delta}{\eps^2})$ samples. \end{thm}

\begin{algorithm}[H]
    \caption{Moment Estimation using Proportional Sampling}
    \label{alg:estmoments}
    \begin{algorithmic}[1] % The number tells where the line numbering should start
        \Procedure{MomentEstimator}{$A,t,\eps,\delta$} %\Comment{The g.c.d. of a and b}
                \State Let $\wt{W}$ denote an $(\eps_1=\eps/2,\delta/2)$-estimate of $W$ using the sum estimation algorithm of \cite{BT2022}. This step requires $480\cdot \frac{\sqrt{n}\log (2/\delta)}{\eps}$ proportional samples.
                \For $~r=1$ to $v = 48 \cdot \log 2/\delta$  %median of means improvement
                        \For $~j=1$ to $l=48 \cdot n^{1-1/t}/\eps^2$
                                \State Let $a_j$ denote a proportional sample of weight $w(a_j)$. 
				\State Compute $\ti{p}_j=\frac{w(a_j)}{\wt{W}}$.
                                \State Set $X_j=\frac{w(a_j)^t}{\ti{p}_j}$
                        \EndFor
                        \State $Y_r = \frac{\sum_{j=1}^l X_j}{l}$
                \EndFor %median of means improvement
                \State \textbf{return} median$(Y_1,\ldots,Y_v)$
        \EndProcedure
    \end{algorithmic}
\end{algorithm}

Algorithm \ref{alg:estmoments} first computes an $(\eps_1,\delta/2)$-estimate $\wt{W}$ of the sum $W$ of the weights of elements in $A$ using the sum estimation algorithm in \cite{BT2022}\footnote{\cite{BT2022} states the sample complexity for probability of success at least $2/3$. Here, we are stating the bounds for an $(\eps_1,\delta/2)$-estimate. This is obtained using an application of the medians of means technique.}. Let the probability of sampling element $a_j$ using proportional sampling be given as $p_j=\frac{w(a_j)}{W}$. Note that we don't know $p_j$, however we can obtain a good approximation of it as follows. For a proportional sample $a_j$, we know its weight $w(a_j)$, and $\wt{W}$ gives us an approximation of $W$. Using this we get a approximation for $p_j$ as $\ti{p}_j=\frac{w(a_j)}{\wt{W}}$. Let $\E$ denote the event that $\wt{W}\in [(1-\eps_1)W,(1+\eps_1)W]$. In what follows we condition on event $\E$.

\begin{claim}\label{claim:prob} Conditioned on $\E$, for any $j\in [n]$, we have $\frac{p_j}{1+\eps_1}\leq \ti{p}_j\leq \frac{p_j}{1-\eps_1}$. \end{claim}
\begin{proof} Conditioned on $\E$, $(1-\eps_1)W\leq \wt{W}\leq (1+\eps_1)W$. The above inequality follows. \end{proof}

Conditioned on $\E$, for any $j$, we have $\frac{p_j}{1+\eps_1}\leq \ti{p}_j\leq \frac{p_j}{1-\eps_1}$. Given a proportional sample $a_j$, we define a random variable $X_j$ with value $\frac{w(a_j)^t}{\ti{p}_j}$. Then, we have $(1-\eps_1)S_t\leq \EE[X_j]\leq (1+\eps_1)S_t$. Here, $X_j$ is not an unbiased estimator of $S_t$. Next, we bound the variance of this estimator given as $\var[X_j]=\EE[X_j^2]-\EE^2[X_j]\leq \EE[X_j^2]$. 

\begin{align}
\EE[X_j^2]
& = \sum_{a_j\in A} \frac{w(a_j)^{2t}}{\ti{p}_j^2} p_j \nonumber\\
& \leq (1+\eps_1)^2 \sum_{a_j\in A} \frac{w(a_j)^{2t}}{p_j} \nonumber\\
& = (1+\eps_1)^2 \cdot W \cdot \sum_{a_j\in A} \frac{w(a_j)^{2t}}{w(a_j)} && \text{(Substituting $p_j$ with $\frac{w(a_j)}{W}$)} \nonumber\\
& = (1+\eps_1)^2 \cdot W \cdot \sum_{a_j\in A} w(a_j)^{2t-1} \label{eqn:upper-sample}
\end{align}

Let us obtain $l$ independent samples using proportional sampling and let these random variables be $X_1,\ldots,X_l$. Let $X=\frac{1}{l} \sum_{j=1}^l X_j$. We have $(1-\eps_1)S_t\leq \EE[X]=\EE[X_j]\leq (1+\eps_1)S_t$, and $\var[X]=\frac{\var[X_j]}{l}$. Using Chebyshev's inequality, we have $\Pr[|X-S_t|>\eps S_t]\leq \Pr[|X-\EE[X]|>(\eps-\eps_1) S_t]]\leq \frac{\var[X]}{(\eps-\eps_1)^2 S_t^2}$. For appropriate choice of parameter $l$, we show this probability to be at most a small constant using the following claim. 
%bound this probability to be at most $\frac{(1+\eps_1)^2}{l(\eps-\eps_1)^2} \cdot n^{1-1/t}$ in the following claim whose proof is given in Appendix \ref{sec:omit}.

%\begin{claim} $\frac{\var[X]}{(\eps-\eps_1)^2 S_t^2}\leq \frac{(1+\eps_1)^2}{l(\eps-\eps_1)^2} \cdot n^{1-1/t}$. \end{claim}

\begin{claim} $\frac{\var[X]}{(\eps-\eps_1)^2 S_t^2}\leq \frac{(1+\eps_1)^2}{l(\eps-\eps_1)^2} \cdot n^{1-1/t}$. \end{claim}

\begin{proof}
\begin{align*}
& \frac{\var[X]}{(\eps-\eps_1)^2 S_t^2} \\
\leq &\frac{\EE[X_j^2]}{l(\eps-\eps_1)^2 \cdot S_t^2} \\
 \leq &\frac{(1+\eps_1)^2}{l(\eps-\eps_1)^2} \cdot \frac{W \cdot \sum_{a_j\in A} w(a_j)^{2t-1}}{S_t^2} && \text{(Using Equation (\ref{eqn:upper-sample}))}\\
 = & \frac{(1+\eps_1)^2}{l(\eps-\eps_1)^2} \cdot \frac{W \cdot ||w(A)||_{2t-1}^{2t-1}}{S_t^2} && \text{(vector $w(A)$ has length $n$)} \\
 \leq &\frac{(1+\eps_1)^2}{l(\eps-\eps_1)^2} \cdot \frac{W \cdot {(||w(A)||_t^t)}^{2-1/t}}{S_t^2} && \text{(Using Fact~\ref{fact:norms}, $||w(A)||_{2t-1}\leq ||w(A)||_t$)} \\
 = &\frac{(1+\eps_1)^2}{l(\eps-\eps_1)^2} \cdot \frac{||w(A)||_1 \cdot (||w(A)||_t^t)^{2-1/t}}{(||w(A)||_t^t)^2}\\
 = &\frac{(1+\eps_1)^2}{l(\eps-\eps_1)^2} \cdot \frac{||w(A)||_1}{||w(A)||_t}\\
 \leq &\frac{(1+\eps_1)^2}{l(\eps-\eps_1)^2} \cdot n^{1-1/t} && \text{(Using Fact~\ref{fact:norms}, $||w(A)||_1\leq n^{1-1/t} ||w(A)||_t$)}
\end{align*}
\end{proof}


Let $\eps_1=\eps/2$. For $l=48n^{1-1/t}/\eps^2$, this failure probability is at most $1/3$. Using the standard median trick, we show that for $v=48\log 2/\delta$, this failure probability can be reduced to be at most $\delta/2$. Let us define independent Bernoulli random variables $Z_1,\ldots,Z_v$ such that $\Pr[Z_i=1]=2/3$ for all $i$. Let $Z=\sum_{r=1}^v Z_r$. Now, conditioned on $\E$, the probability that the output of Algorithm \ref{alg:estmoments} does not lie in the interval $[(1-\eps)S_t,(1+\eps)S_t]$ is the same as the probability that the median of $Y_1,\ldots,Y_v$ lies outside the interval $[(1-\eps)S_t,(1+\eps)S_t]$. This probability is at most $\Pr[Z<v/2]$. Using a standard application of a Chernoff bound, given in Lemma \ref{lem:chernoff}, we have $\Pr[Z<v/2]\leq \delta/2$.

\paragraph{Correctness and Sample complexity bounds}

We need to show that Algorithm \ref{alg:estmoments} returns an estimate $ALG(A,t,\eps,\delta)$ for which with probability at least $1-\delta$, we have $(1-\eps)S_t\leq ALG(A,t,\eps,\delta)\leq (1+\eps)S_t$. Step $2$ of Algorithm \ref{alg:estmoments} uses $\Theta{(\frac{\sqrt{n}\log (2/\delta)}{\eps_1})}$ proportional samples to obtain an $(\eps_1,\delta/2)$ estimate $\wt{W}$ of $W$. Algorithm \ref{alg:estmoments} fails if either $\E$ does not hold or the estimate in Step $10$ is incorrect. Since both of these failure probabilities are at most $\delta/2$, the algorithm succeeds with probability at least $1-\delta$. 

Algorithm \ref{alg:estmoments} uses $\Theta{(\frac{\sqrt{n}\log (2/\delta)}{\eps_1})}$ proportional samples in Step 2 and uses $O(\frac{n^{1-1/t}\log 1/\delta}{\eps^2})$ proportional samples in Steps 3 and 4. Therefore, the required number of proportional samples is $O(\frac{\sqrt{n}\log 1/\delta}{\eps_1} + \frac{n^{1-1/t}\log 1/\delta}{\eps^2})$. For $\eps_1=\eps/2$, this gives $O(\frac{\sqrt{n}\log 1/\delta}{\eps} + \frac{n^{1-1/t}\log 1/\delta}{\eps^2})$.


\section{Lower bound for Moment Estimation using Proportional Sampling}\label{sec:lower-proportional}

We use Yao's minimax lemma to prove the sample complexity lower bound for obtaining an $(\eps,\delta)$ estimate of $S_t$ using any randomized algorithm. We construct two families of instances on which $S_t$ differs by at least a $(1\pm \eps)$-factor and show that it is hard to distinguish these two instances using a small number of proportional samples.

Our lower bound constructions are as follows. There are $n_1$ elements of weight $d_1$ and $n_2$ elements of weight $d_2$ in both the instances, where the values of the parameters $n_1, n_2$ and $d_1$ are the same in both instances, and the instances differ in the value of parameter $d_2$. The exact values of these parameters will be set below. In one instance we set $d_2=n$, where as for the other instance $d_2=0$. These choices for parameter values creates a gap of a multiplicative $(1\pm \eps)$ factor between the $S_t$ values of the two instances. One can differentiate these two instances only when an element of weight $d_2=n$ is sampled using proportional sampling, and we show that this requires a lot of samples. 

\begin{thm} For any $\eps,\delta \in (0,1)$ and $t>1$, any randomized algorithm that computes an $(\eps,\delta)$-estimate of $S_t$ requires $\Omega(\frac{n^{1-1/t}\ln 1/\delta }{\eps^2})$ proportional samples. \end{thm}

\begin{proof} We construct two families of instances which we show are hard to be distinguished using a few proportional samples by Yao's lemma. In the first instance we have $n_1$ elements with weight $d_1$ and $n_2$ elements of weight $0$, and in the second instance, there are $n_1$ elements of weight $d_1$ and $n_2$ elements of weight $d_2$, where the following values are used. The parameter values used in the lower bound constructions of the two instances are given as follows. 

%\begin{tabular}{p{8cm}|p{8cm}}
%Instance $1$ & Instance $2$\\ 
%$n_1  =  \frac{n^2}{n + \eps^{\frac{2t-1}{t-1}}}$ & $n_1  =  \frac{n^2}{n+ \eps^{\frac{2t-1}{t-1}}}$\\
%$d_1  =  n^{1-1/t} \eps^{1/(t-1)}$ & $d_1 =  n^{1-1/t} \eps^{1/(t-1)}$\\
%$n_2  =  \frac{n \eps^{\frac{2t-1}{t-1}}}{n + \eps^{\frac{2t-1}{t-1}}}$ & $n_2  =  \frac{n \eps^{\frac{2t-1}{t-1}}}{n + \eps^{\frac{2t-1}{t-1}}}$\\
%$d_2  =  0$ & $d_2 = n$
%\end{tabular}

\begin{gather}
\begin{align*}
\qquad \qquad \quad n_1 & =  \frac{n^2}{n + \eps^{\frac{2t-1}{t-1}}} & n_1 & =  \frac{n^2}{n+ \eps^{\frac{2t-1}{t-1}}}\\
\qquad \qquad \quad d_1 & =  n^{1-1/t} \eps^{1/(t-1)} & d_1 & =  n^{1-1/t} \eps^{1/(t-1)}\\
\qquad \qquad \quad n_2 & =  \frac{n \eps^{\frac{2t-1}{t-1}}}{n + \eps^{\frac{2t-1}{t-1}}} & n_2 & =  \frac{n \eps^{\frac{2t-1}{t-1}}}{n + \eps^{\frac{2t-1}{t-1}}}\\
\qquad \qquad \quad d_2 & =  0 & d_2 & = n
\end{align*}
\end{gather}


\noindent The $S_t$ value for the first instance is given as follows.
\begin{align*}
n_1 \cdot d_1^t + n_2 \cdot 0
& = \frac{n^2}{n + \eps^{\frac{2t-1}{t-1}}} \cdot n^{t-1} \eps^{t/(t-1)}\\
& = \frac{n^{t+1} \eps^{t/(t-1)}}{n + \eps^{\frac{2t-1}{t-1}}} 
\end{align*}


\noindent The $S_t$ value for the second instance is 
\begin{align*}
n_1\cdot d_1^t + n_2 \cdot d_2^t 
& = \frac{n^{t+1} \eps^{t/(t-1)}}{n + \eps^{\frac{2t-1}{t-1}}} +  \frac{n \eps^{\frac{2t-1}{t-1}}}{n + \eps^{\frac{2t-1}{t-1}}} \cdot n^t\\
& = \frac{n^{t+1} \eps^{t/(t-1)}}{n + \eps^{\frac{2t-1}{t-1}}} + \eps \frac{n^{t+1} \eps^{t/(t-1)}}{n + \eps^{\frac{2t-1}{t-1}}}\\
& = (1+\eps) \frac{n^{t+1} \eps^{t/(t-1)}}{n +\eps^{\frac{2t-1}{t-1}}}
\end{align*}

The above two instances differ in their $S_t$ values by a multiplicative factor of $(1+\eps)$. In order to distinguish these two instances, one is required to sample an element of weight $d_2=n$. Using proportional sampling, the probability of sampling an element of weight $n$ in the above instance is given to be at least


\begin{align*}
\frac{n_2 d_2}{n_2 d_2 + n_1 d_1}
& = \frac{\frac{n \eps^{\frac{2t-1}{t-1}}}{n + \eps^{\frac{2t-1}{t-1}}} \cdot n} {\frac{n \eps^{\frac{2t-1}{t-1}}}{n + \eps^{\frac{2t-1}{t-1}}} \cdot n + \frac{n^2 }{n + \eps^{\frac{2t-1}{t-1}}} \cdot n^{1-1/t} \eps^{1/(t-1)}}\\
& =  \frac{n^2 \eps^{\frac{2t-1}{t-1}}}{n^2 \eps^{\frac{2t-1}{t-1}} + n^2 \cdot n^{1-1/t} \cdot \eps^{\frac{1}{t-1}}}\\
& =  \frac{1}{1+\frac{n^{1-1/t}}{\eps^2}}
\end{align*}

Let $p=\frac{1}{1+\frac{n^{1-1/t}}{\eps^2}}$. The lower bound on the sample complexity for this instance distinguishing problem is given as the number of samples required to observe a \textit{success} with probability at least $(1-\delta)$ while drawing independent samples from $Geom(p)$. Here, $Geom(p)$ denotes a geometric distribution with success probability $p$. The number of samples required to observe one \textit{success} from $Geom(p)$ with probability at least $(1-\delta)$ is at least $\Omega(\frac{\ln 1/\delta}{p})$. Therefore, $\Omega(\frac{n^{1-1/t}\ln1/\delta}{\eps^2})$ samples are required to distinguish these two instances with probability at least $1-\delta$.

\end{proof}

\section{Estimation of Moments using Hybrid Sampling}\label{sec:lower-hybrid}

In this section we prove a lower bound showing that for the moment estimation problem, the hybrid sampling framework does not provide any significant advantage over access to just the proportional sampling oracle. In contrast, note that for the sum estimation problem, hybrid sampling-based algorithms in fact give much better sample complexity bounds over proportional sampling \cite{MPX2007,BT2022}. We prove the following result.

\begin{thm} For any $\eps,\delta>0$ and $t>1$, any algorithm having access to a hybrid sampling oracle requires to make at least $\Omega(\frac{n^{1-1/t}\ln 1/\delta}{\eps^2})$ queries to compute an $(\eps,\delta)$-estimate for $S_t$. \end{thm}

We show that the lower bound instance described in Section \ref{sec:lower-proportional} yields a lower bound for the hybrid sampling as well. In order to distinguish these instances, one is required to sample an element of weight $n$. We have seen that using just proportional sampling $\Omega(\frac{n^{1-1/t} \log 1/\delta}{\eps^2})$ samples are required. The probability of sampling an element of weight $d_2$ using uniform sampling is given as $\frac{n_2}{n_1+n_2}$. This probability using the values of the parameters from Section \ref{sec:lower-proportional} equals $\frac{n_2}{n_1+n_2} = \frac{1}{1+\frac{n}{\eps^{\frac{2t-1}{t-1}}}}$. The instances are distinguished if an element of weight $n$ is sampled using either proportional sampling or uniform sampling. These two probabilities are given as $\frac{1}{1+\frac{n^{1-1/t}}{\eps^2}}$ and $\frac{1}{1+\frac{n}{\eps^{\frac{2t-1}{t-1}}}}$, respectively. Overall, we get a lower bound of $\Omega(\min\{\frac{n^{1-1/t}}{\eps^2}, \frac{n}{\eps^{\frac{2t-1}{t-1}}}\}\ln 1/\delta)=\Omega(\frac{n^{1-1/t}\ln 1/\delta}{\eps^2})$ for the $(\eps,\delta)$ moment estimation problem using hybrid sampling.

% !TEX root = paper.tex
\section{Characterization of Sample Complexity}\label{sec:characterize}

We define a {\it moment-density} parameter of the input that governs the sample complexity of the moment estimation problem using proportional sampling. For $S_t=\sum_{a\in A} w(a)^t$ and $W=\sum_{a\in A} w(a)$, we define the moment-density parameter as $$\rho=\max_{L\subseteq A} \frac{\frac{\sum_{a\in L} w(a)^t}{S_t}}{\frac{\sum_{a\in L}w(a)}{W}}=\max_{L\subseteq A} \frac{\sum_{a\in L} w(a)^t}{\sum_{a\in L} w(a)} \cdot \frac{W}{S_t}$$ 

We give an upper bound for an $(\eps,\delta)$-estimator of $S_t$ using $O(({\sqrt{n}}/{\eps}+\rho/\eps^2)\ln 1/\delta)$ proportional samples. 

\begin{thm}\label{thm:char-upper} There exists an algorithm $ALG$ that given proportional sampling access to the weights of elements of $A$ having moment-density parameter $\rho$ and parameters $t>1,\eps,\delta\in (0,1)$, provides an $(\eps,\delta)$-estimate of $S_t$ using $O(({\sqrt{n}}/{\eps}+\rho/\eps^2)\ln 1/\delta)$ proportional samples. \end{thm}

\begin{proof} Algorithm \ref{alg:estmoments} gives the above sample complexity bound. We adapt the calculations from Section \ref{sec:moments}. Following Equation \ref{eqn:upper-sample}, we can write the sample complexity of our $(\eps,1/3)$ estimator to be at most $O(\sqrt{n}/\eps + \frac{1}{\eps^2} \frac{W\cdot \sum_{a\in A} w(a)^{2t-1}}{S_t^2})$. We will show that $\frac{W\cdot \sum_{a\in A} w(a)^{2t-1}}{S_t^2} \leq \rho$.  
\begin{align*}
\frac{W\cdot \sum_{a\in A} w(a)^{2t-1}}{S_t^2}
& = \frac{W}{S_t} \cdot \frac{\sum_{a\in A} w(a)^{2t-1}}{\sum_{a\in A} w(a)^t}\\
%& = \frac{W}{S_t} \cdot \frac{\sum_{a\in A} w(a)^{t-1} \cdot w(a)^t}{\sum_{a\in A} w(a)^{t-1} \cdot w(a)}\\
%& = \frac{W}{S_t} \cdot \frac{\sum_{a\in A} \frac{w(a)^{t-1}}{\sum_{b\in A} w(b)^{t-1}} \cdot w(a)^t}{\sum_{a\in A} \frac{w(a)^{t-1}}{\sum_{b\in A} w(b)^{t-1}} \cdot w(a)}\\
& \leq \frac{W}{S_t} \cdot \max_{a\in A} \frac{w(a)^t}{w(a)}\\
&= \rho
\end{align*}

Therefore, the sample complexity of our $(\eps,1/3)$-estimate of $S_t$ is $O({\sqrt{n}}/{\eps}+\rho/\eps^2)$. The $(\eps,1/3)$-estimate of the moment can be improved to an $(\eps,\delta)$-estimate with a multiplicative $\ln 1/\delta$-factor sample complexity overhead using the standard median trick.
\end{proof}
Next, we show a $\Omega(\frac{\rho \ln 1/\delta}{\eps})$ lower bound on the sample complexity of any algorithm for the $(\eps,\delta)$-moment estimation problem on any instance with the moment-density parameter $\rho$. We construct two families of input distributions that are hard to be distinguished using a small number of proportional samples. The lower bound instances for the moment estimation using proportional sampling described in Section \ref{sec:lower-proportional} have the moment-density parameter values as $\rho=1$ and $\rho=O(n^{1-1/t}/\eps)$. Next, we show that similar lower bounds hold even when we restrict the input instances to have their moment-density parameter values to be within a constant factor.

\begin{thm}\label{thm:char-lower} For any $\rho$, $\eps,\delta>0$ and $t>1$, any randomized algorithm for an $(\eps,\delta)$-estimate of $S_t$ on an instance with the moment-density parameter $\rho$ requires $\Omega(\frac{\rho \ln 1/\delta}{\eps})$ proportional samples. \end{thm}
\begin{proof} 
  Given any $\rho,\eps,\delta$ as input, we construct two families of instances for which their moment-density parameter values $\rho_1,\rho_2=O(\rho)$ and their moment values $S_t$ differ by a $(1\pm \eps)$ factor. There are $n$ elements in both the instances. In the first instance $n_1$ elements will have weight $d_1$ and $n_2$ elements will have value $d_2$. In the second instance there are $n_1$ elements of weight $d_1$, $\frac{n_2}{3}$ elements of weight $d_2$, and $\frac{2n_2}{3}$ elements of weight $0$. We set the values of the parameters $n_1,n_2,d_1,d_2$ as follows:

\begin{gather}
\begin{align*}
\qquad \qquad \quad n_1 & =  \frac{n^2}{n +(3 \eps)^{\frac{2t-1}{t-1}}} & d_1 & =n^{1-1/t}( 3\eps)^{1/(t-1)}\\
\qquad \qquad \quad n_2 & =  \frac{n (3\eps)^{\frac{2t-1}{t-1}}}{n +(3 \eps)^{\frac{2t-1}{t-1}}} & d_2&=n
\end{align*}
\end{gather}

\noindent The moment value for the first instance is:
\begin{align*}
S_t = n_1d_1^t+n_2d_2^t&=\frac{n^2}{n +(3 \eps)^{\frac{2t-1}{t-1}}}\cdot n^{t-1}(3\eps)^{\frac{t}{t-1}}+\frac{n^{t+1} (3\eps)^{\frac{2t-1}{t-1}}}{n +(3 \eps)^{\frac{2t-1}{t-1}}} \\
&= \frac{n^{t+1}(3\eps)^{\frac{t}{t-1}}+n^{t+1}(3\eps)^{\frac{2t-1}{t-1}}}{n +(3 \eps)^{\frac{2t-1}{t-1}}}\\
&=\frac{n^{t+1}(3\eps)^\frac{t}{t-1}}{n+(3\eps)^{\frac{2t-1}{t-1}}}(1+3\eps)
\end{align*}

\noindent The moment value for the second instance is:
\begin{align*}
S_t = n_1d_1^t+\frac{n_2}{3}d_2^t+\frac{2}{3}n_2\cdot0 &=\frac{n^2}{n +(3 \eps)^{\frac{2t-1}{t-1}}}\cdot n^{t-1}(3\eps)^{\frac{t}{t-1}}+\frac{1}{3}\frac{n^{t+1} (3\eps)^{\frac{2t-1}{t-1}}}{n +(3 \eps)^{\frac{2t-1}{t-1}}}\\
&=\frac{n^{t+1}(3\eps)^\frac{t}{t-1}}{n+(3\eps)^{\frac{2t-1}{t-1}}}(1+\frac{3\eps}{3})\\
&=\frac{n^{t+1}(3\eps)^\frac{t}{t-1}}{n+(3\eps)^{\frac{2t-1}{t-1}}}(1+\eps)
\end{align*}

The moment values of the above two instances differ by a factor of $\frac{1+3\eps}{1+\eps}>1+\eps$. Let $\rho_1$ and $\rho_2$ be the moment-density parameters for these two instances, respectively. We compute the $\rho$ values of the instances as follows. 
\begin{align*}
\rho_1 & = \frac{n^t}{n} \cdot \frac{n_1 d_1 + n_2 d_2}{n_1 d_1^t+n_2 d_2^t}\\
& = n^{t-1} \cdot \frac{\frac{n^2}{n +(3 \eps)^{\frac{2t-1}{t-1}}} \cdot n^{1-1/t}( 3\eps)^{1/(t-1)} + \frac{n (3\eps)^{\frac{2t-1}{t-1}}}{n +(3 \eps)^{\frac{2t-1}{t-1}}}\cdot n }{\frac{n^2}{n +(3 \eps)^{\frac{2t-1}{t-1}}} \cdot n^{t-1}( 3\eps)^{t/(t-1)} + \frac{n (3\eps)^{\frac{2t-1}{t-1}}}{n +(3 \eps)^{\frac{2t-1}{t-1}}}\cdot n^t}\\
&=\frac{n^{1-1/t}+(3\eps)^2}{3\eps+(3\eps)^2}=\frac{n^{1-1/t}+9\eps^2}{3\eps+9\eps^2}
\end{align*}

Similarly, we compute $\rho_2$ as
\begin{align*}
\rho_2 & =\frac{n^t}{n} \cdot \frac{n_1 \cdot d_1 + n_2/3 \cdot d_2 + 2n_2/3 \cdot 0}{n_1 \cdot d_1^t + n_2/3 \cdot d_2^t + 2n_2/3 \cdot 0}\\ 
& = \frac{n^{1-1/t}+\frac{1}{3}(3\eps)^2}{3\eps(1+\eps)}= \frac{n^{1-1/t}+3\eps^2}{3\eps+3\eps^2} 
\end{align*}

We have two instances as given above where their $\rho$ values are within a constant factor of each other and the moment values differ by at least a factor of $(1+\eps)$. These instances using proportional sampling remain indistinguishable until an element with weight $d_2$ is sampled. The probability of sampling an element of weight $d_2$ using proportional sampling is at most:
\begin{align*}
\frac{n_2d_2}{n_1d_1+n_2d_2}=\frac{1}{1+\frac{n_1d_1}{n_2d_2}}=\frac{1}{1+\frac{n^2n^{1-1/t}(3\eps)^{1/(t-1)}}{n^2(3\eps)^{\frac{2t-1}{t-1}}}}=\frac{1}{1+\frac{n^{1-1/t}}{9\eps^2}}\leq \frac{1}{1+O(\frac{\rho}{\eps})}
\end{align*} 

Therefore, we require at least $\Omega(\rho/\eps)$ samples to distinguish the two instances with constant probability of success. Any algorithm giving an $(\eps,\delta)$-estimate would need to make $\Omega (\frac{\rho \ln 1/\delta}{\eps})$ proportional samples.
\end{proof}

% !TEX root = paper.tex
\section{Moment estimation for $0<t<1$}\label{sec:small-t}

\subsection{Moment estimation using Proportional Sampling for 1/2<t<1} We use Algorithm \ref{alg:estmoments} to estimate moment $S_t$ for $1/2<t<1$. We adapt the analysis in Section \ref{sec:moments} to prove the following result.

\begin{thm} There exists an algorithm $ALG$ that given proportional sampling access to the weights of the elements of a set $A$ and parameters $1/2<t<1$, $\eps\in (0,1)$, provides an $(\eps,1/3)$-estimate of $S_t$ using $O(\frac{\sqrt{n}}{\eps}+\frac{n^{\frac{1}{t}-1}}{\eps^2})$ samples. \end{thm}

The sample complexity of the algorithm for $1/2<t<1$ is given as follows. Note that the sample complexity of our algorithm is given as $\frac{W (||w(A)||_{2t-1})^{2t-1}}{\eps^2 (||w(A)||_t)^{2t}}$. We can rewrite it as $\frac{||w(A)||_1}{\eps^2 ||w(A)||_t} \cdot (\frac{||w(A)||_{2t-1}}{||w(A)||_t})^{2t-1}$. Using Fact \ref{fact:norms}, this is upper bounded as $n^{\frac{1}{t}-1}/\eps^2$. Overall sample complexity of the $(\eps,1/3)$-estimator of the moment is $O(\sqrt{n}/\eps+ n^{\frac{1}{t}-1}/\eps^2)$. Using the median trick we transform the $(\eps,1/3)$-estimator to an $(\eps,\delta)$-estimator with sample complexity $O((\sqrt{n}/\eps+ n^{\frac{1}{t}-1}/\eps^2) \ln 1/\delta)$.


\subsection{Lower bound for $t\leq 1/2$} We show that there are no sublinear algorithms for the problem when $t\leq 1/2$.

\begin{thm} For any $\eps>0$ and $t\leq 1/2$, any randomized algorithm that computes an $(\eps,1/3)$-estimate of $S_t$ requires $\Omega(n)$ proportional samples. \end{thm}
\begin{proof} We construct two families of instances which we show are hard to be distinguished using proportional sampling. In one instance we have one element with weight $(n-1)$, and the rest $(n-1)$ elements with weight $0$. In another instance, we have one element of weight $(n-1)$, and $(n-1)$ elements have weight $\frac{\eps^{1/t}}{n-1}$. The moment values of the two instances are given as $(n-1)^t$ and $(n-1)^t + \eps (n-1)^{1-t}\geq (1+\eps) (n-1)^t$ as $t\leq 1/2$. Hence, there exists a $(1+\eps)$-multiplicative gap between the moment values of these two instances.

These two instances are distinguished using proportional sampling once we sample an element of weight $\frac{\eps^{1/t}}{n-1}$. The probability of sampling such an element is at most $\frac{\eps^{1/t}}{\eps^{1/t} + (n-1)}$. Hence, one requires $\Omega(n)$ samples to distinguish the two instances with constant probability. \end{proof}


\section{Conclusion}
In this work, we propose a simple yet effective approach, called SMILE, for graph few-shot learning with fewer tasks. Specifically, we introduce a novel dual-level mixup strategy, including within-task and across-task mixup, for enriching the diversity of nodes within each task and the diversity of tasks. Also, we incorporate the degree-based prior information to learn expressive node embeddings. Theoretically, we prove that SMILE effectively enhances the model's generalization performance. Empirically, we conduct extensive experiments on multiple benchmarks and the results suggest that SMILE significantly outperforms other baselines, including both in-domain and cross-domain few-shot settings.

\subsubsection{Acknowledgements} Anup Bhattacharya is supported by the Science and Engineering Research Board (SERB) via the project (CRG/2023/002119). The authors thank the reviewers for suggestions to improve the manuscript. 

\bibliographystyle{splncs04}
\bibliography{ref}


\end{document}
