\documentclass{article}

\usepackage[T1]{fontenc}
\usepackage{graphicx}

\newcommand{\thought}[1]{{\color[rgb]{0.2,0.39,0.66}(#1)}}
\newcommand{\todo}[1]{{\color[rgb]{1.0,0.0,0.0}(#1)}}
\newcommand{\hsh}[1]{{\color{green!50!black} Henrik: #1}}
\newcommand{\st}[1]{{\color{red!50!black} Sebastian: #1}}

\newcommand{\ulm}[1]{_{\scaleto{\mathrm{#1}}{3pt}}}
\newcommand\at[2]{\left.#1\right|_{#2}}











\newtheorem{assumption}{Assumption}

\DeclareMathOperator*{\argmax}{arg\,max}
\DeclareMathOperator*{\argmin}{arg\,min}

\newcommand{\swname}[1]{\texttt{#1}}
\newcommand{\ie}{i\/.\/e\/.,\/~}
\newcommand{\eg}{e\/.\/g\/.,\/~}
\newcommand{\cf}{cf\/.\/~}

\newcommand{\fig}{Fig\/.\/~}
\newcommand{\defn}{Def\/.\/~}
\newcommand{\sect}{Sec\/.\/~}
\newcommand{\tabl}{Tab\/.\/~}
\newcommand{\algo}{Algorithm~}
\newcommand{\theo}{Theorem~}

\newcommand{\bnnl}{3 hidden layers}
\newcommand{\bnnn}{50 neurons}
\newcommand{\bnna}{tanh activations}

\newcommand{\capt}[1]{\mdseries{\emph{#1}}}

\newcommand{\videolink}{at \url{https://youtu.be/_d7AqTRjz6g}}
\newcommand{\codelink}{\url{https://github.com/wheelbot/mini-wheelbot}}

\newcommand{\fakepar}[1]{\vspace{0mm}\noindent\textbf{#1.}}

\newcommand{\needref}{\textcolor{red}{[REF]}}

\newcommand{\plotfontsize}{9pt}

\allowdisplaybreaks

\usepackage{color}
\renewcommand\UrlFont{\color{blue}\rmfamily}
\urlstyle{rm}

\begin{document}

\title{Improved Sublinear-time Moment Estimation using Weighted Sampling} %TODO Please add

%\titlerunning{Dummy short title} %TODO optional, please use if title is longer than one line

\author{
Anup Bhattacharya\thanks{Authors are ordered alphabetically}\\ NISER, Bhubaneswar, India \\ anup@niser.ac.in \and
Pinki Pradhan \\ NISER, Bhubaneswar, India \\ pinki.pradhan@niser.ac.in}

%\institute{National Institute of Science Education and Research, An OCC of Homi Bhabha National Institute, Bhubaneswar, India\\\email{\{anup, pinki.pradhan\}@niser.ac.in}}

%\author{Jane {Open Access}}{Dummy University Computing Laboratory, [optional: Address], Country \and My second affiliation, Country \and \url{http://www.myhomepage.edu} }{johnqpublic@dummyuni.org}{https://orcid.org/0000-0002-1825-0097}{(Optional) author-specific funding acknowledgements}%TODO mandatory, please use full name; only 1 author per \author macro; first two parameters are mandatory, other parameters can be empty. Please provide at least the name of the affiliation and the country. The full address is optional. Use additional curly braces to indicate the correct name splitting when the last name consists of multiple name parts.

%\author{Joan R. Public\footnote{Optional footnote, e.g. to mark corresponding author}}{Department of Informatics, Dummy College, [optional: Address], Country}{joanrpublic@dummycollege.org}{[orcid]}{[funding]}

%\authorrunning{J. Open Access and J.\,R. Public} %TODO mandatory. First: Use abbreviated first/middle names. Second (only in severe cases): Use first author plus 'et al.'


%\relatedversiondetails[linktext={opt. text shown instead of the URL}, cite=DBLP:books/mk/GrayR93]{Classification (e.g. Full Version, Extended Version, Previous Version}{URL to related version} %linktext and cite are optional

%\supplement{}%optional, e.g. related research data, source code, ... hosted on a repository like zenodo, figshare, GitHub, ...
%\supplementdetails[linktext={opt. text shown instead of the URL}, cite=DBLP:books/mk/GrayR93, subcategory={Description, Subcategory}, swhid={Software Heritage Identifier}]{General Classification (e.g. Software, Dataset, Model, ...)}{URL to related version} %linktext, cite, and subcategory are optional

%\funding{(Optional) general funding statement \dots}%optional, to capture a funding statement, which applies to all authors. Please enter author specific funding statements as fifth argument of the \author macro.

%\acknowledgements{I want to thank \dots}%optional

%\nolinenumbers %uncomment to disable line numbering



\maketitle

\begin{abstract}
Retrieval-Augmented Generation (RAG) is often used with Large Language Models (LLMs) to infuse domain knowledge or user-specific information. In RAG, given a user query, a retriever extracts chunks of relevant text from a knowledge base. These chunks are sent to an LLM as part of the input prompt. Typically, any given chunk is repeatedly retrieved across user questions. However, currently, for every question, attention-layers in LLMs fully compute the key values (KVs) repeatedly for the input chunks, as state-of-the-art methods cannot reuse KV-caches when chunks appear at arbitrary locations with arbitrary contexts. Naive reuse leads to output quality degradation.  This leads to potentially redundant computations on expensive GPUs and increases latency. In this work, we propose \sys, a system for managing and reusing precomputed KVs corresponding to the text chunks (we call \textit{chunk-caches}) in RAG-based systems. We present how to identify \hl{\textit{chunk-caches} that are reusable}, how to efficiently perform a small fraction of recomputation to \textit{fix} the cache to maintain output quality, and how to efficiently store and evict \textit{chunk-caches} in the hardware for maximizing reuse while masking any overheads. With real production workloads as well as synthetic datasets, we show that \sys reduces redundant computation by \textbf{51\%} over SOTA prefix-caching and \textbf{75\%} over full recomputation.
\hl{Additionally, with continuous batching on a real production workload, we get a \textbf{1.6$\times$} speedup in throughput and a \textbf{2$\times$} reduction in end-to-end response latency over prefix-caching while maintaining quality, for both the \llama-3-8B and \llama-3-70B models. 
}
\end{abstract}





\section{Introduction}
\label{sec:intro}

\begin{figure*}[tb]
    \centering
    \includegraphics[width=0.848\linewidth]{figs/circuitnn.pdf} 
    \caption{Illustration of differentiable CircuitNN. CircuitNN is designed based on differentiable NAND gates. After DAS is guided by PI and PO pairs of the truth table, CircuitNN can get the precise circuit architecture logic equivalent to the truth table.}
    \label{fig:circuitnn}
\end{figure*}

% 1. Describe the importance of logic synthesis
% 2. Existing Problems
% (a) Neural Architecture Search: Unstable, Predefined Setting, etc.
% (b) Circuit Generation: Probabilistic Model, Logic Equivalence

With the rapid advancement of technology, the scale of integrated circuits (ICs) has expanded exponentially. 
This expansion has introduced significant challenges in chip manufacturing, particularly concerning power and area metrics.
A primary objective in IC design is achieving the same circuit function with fewer transistors, thereby reducing power usage and area occupancy.

Logic synthesis~\cite{hachtel2005logicsynth}, a critical step in electronic design automation (EDA), transforms behavioral-level circuit designs into optimized gate-level circuits, ultimately yielding the final IC layout. 
The primary goal of logic synthesis is to identify the physical implementation with the fewest gates for a given circuit function. 
This task constitutes a challenging NP-hard combinatorial optimization problem. 
Current logic synthesis tools~\cite{brayton2010abc, wolf2013yosys} rely on human-designed heuristics, often leading to sub-optimal outcomes.

Differentiable architecture search (DAS) techniques~\cite{liu2018darts, chu2020darts} offer novel perspectives on addressing challenges in this problem.
Circuit functions can be represented through truth tables, which map binary inputs to their corresponding outputs. 
Truth tables provide a precise representation of input-output relationships, ensuring the design of functionally equivalent circuits.
Inspired by this, researchers~\cite{deepmind2024ai4sys, wang2024tnet} have begun exploring the application of DAS to synthesize circuits directly from truth tables.
Specifically, \citet{deepmind2024ai4sys} proposed CircuitNN, a framework that learns differentiable connection structures with logic gates, enabling the automatic generation of logic circuits from truth tables.
This approach significantly reduces the complexity of traditional circuit generation. 
Building on this, \citet{wang2024tnet} introduced T-Net, a triangle-shaped variant of CircuitNN, incorporating regularization techniques to enhance the efficiency of DAS.

Despite these advancements, several challenges remain. 
The computational complexity of DAS grows quadratically with the number of gates, posing scalability issues.
Although triangle-shaped architecture~\cite{wang2024tnet} partially mitigates this problem, redundancy persists. 
%Additionally, DAS is susceptible to converging to local optima, limiting the ability to search architectures that satisfy the given truth tables~\cite{liu2018darts}. 
%Furthermore, hyperparameters (network depth and layer width) require extensive searches, introducing complexity and prolonging the synthesis process. 
Additionally, DAS is susceptible to converging to local optima~\cite{liu2018darts} and hyperparameters (network depth and layer width) require extensive searches. 
The challenges arise from the vast search space in DAS. 
% Even with predefined settings for CircuitNN, finding a configuration that meets the truth table requires extensive trial and error during the DAS process. 
Intuitively, limiting the search space through predefined parameters (network depth, gates per layer, and connection probabilities) can significantly reduce the complexity.

Recent advances~\cite{openai2023gpt4, abramson2024alphafold3, esser2024sd3, li2024mar} in conditional generative models have demonstrated remarkable performance across language, vision, and graph generation tasks. 
Motivated by these developments, we propose a novel approach to circuit generation that generates preliminary circuit structures to guide DAS in generating refined circuits matching specified truth tables. 
Firstly, we introduce CircuitVQ, a tokenizer with a discrete codebook for circuit tokenization. 
Built upon our Circuit AutoEncoder framework~\cite{hou2022graphmae,li2023maskgae,wu2025mgvga}, CircuitVQ is trained through a circuit reconstruction task. 
Specifically, the CircuitVQ encoder encodes input circuits into discrete tokens using a learnable codebook, while the decoder reconstructs the circuit adjacency matrix based on these tokens.
Subsequently, the CircuitVQ encoder serves as a circuit tokenizer for CircuitAR pretraining, which employs a masked autoregressive modeling paradigm~\cite{chang2022maskgit, li2023mage}. 
In this process, the discrete codes function as supervision signals. 
After training, CircuitAR can generate discrete tokens progressively, which can be decoded into initial circuit structures by the decoder of the CircuitVQ. 
These prior insights can guide DAS in producing refined circuits that match the target truth tables precisely.

Our key contributions can be summarized as follows:
\begin{itemize}
\item We introduce CircuitVQ, a circuit tokenizer that facilitates graph autoregressive modeling for circuit generation, based on our Circuit AutoEncoder framework;
\item Develop CircuitAR, a model trained using masked autoregressive modeling, which generates initial circuit structures conditioned on given truth tables;
\item Propose a refinement framework that integrates differentiable architecture search to produce functionally equivalent circuits guided by target truth tables;
\item Comprehensive experiments demonstrating the scalability and capability emergence of our CircuitAR and the superior performance of the proposed circuit generation approach.
\end{itemize}

% Motivation
% (a) Diffusion (Vision, Graph), Autoregressive (Language, Vision)
% (b) Circuit Generation for Predefined Setting
% (c) Neural Architecture Search for Strict Logic Equivalence

% Contribution
% (a) Circuit Tokenizer (new transformer arch, training strategy)
% (b) CircuitAR (train and gen strategies, post-ar strategy)
% (c) Extensive Evaluation including BitD (Bit Distance) for Scalability

% !TEX root = paper.tex
\section{Estimation of Moments using Proportional Sampling}\label{sec:moments}

We describe our algorithm for the moment estimation problem using proportional sampling. We mentioned earlier that for $t>1$, our upper bounds match those of Aliakbarpour~\etal~\cite{ABGPRY2018}. However, since our algorithm works in strictly more general settings and the algorithm for $1/2< t<1$ uses the same ideas, we describe it in detail. Let $A$ be a set of $n$ weighted elements. We assume access to a proportional sampling oracle on the weights of the elements in $A$. For a proportional sample, the oracle returns an element $a_j\in A$ with probability $w(a_j)/W$, where $W=\sum_{a_j\in A} w(a_j)$. Given parameters $t>1,\eps,\delta\in (0,1)$, we design an $(\eps,\delta)$-estimate of $S_t=\sum_{a_j\in A} w(a_j)^t$.

\begin{thm} There exists an algorithm $ALG$ that given proportional sampling access to the weights of the elements in a set $A$ and parameters $t>1,\eps,\delta\in (0,1)$, provides an $(\eps,\delta)$-estimate of $S_t$ using $O(\frac{\sqrt{n}\log 1/\delta}{\eps} + \frac{n^{1-1/t} \log 1/\delta}{\eps^2})$ samples. \end{thm}

\begin{algorithm}[H]
    \caption{Moment Estimation using Proportional Sampling}
    \label{alg:estmoments}
    \begin{algorithmic}[1] % The number tells where the line numbering should start
        \Procedure{MomentEstimator}{$A,t,\eps,\delta$} %\Comment{The g.c.d. of a and b}
                \State Let $\wt{W}$ denote an $(\eps_1=\eps/2,\delta/2)$-estimate of $W$ using the sum estimation algorithm of \cite{BT2022}. This step requires $480\cdot \frac{\sqrt{n}\log (2/\delta)}{\eps}$ proportional samples.
                \For $~r=1$ to $v = 48 \cdot \log 2/\delta$  %median of means improvement
                        \For $~j=1$ to $l=48 \cdot n^{1-1/t}/\eps^2$
                                \State Let $a_j$ denote a proportional sample of weight $w(a_j)$. 
				\State Compute $\ti{p}_j=\frac{w(a_j)}{\wt{W}}$.
                                \State Set $X_j=\frac{w(a_j)^t}{\ti{p}_j}$
                        \EndFor
                        \State $Y_r = \frac{\sum_{j=1}^l X_j}{l}$
                \EndFor %median of means improvement
                \State \textbf{return} median$(Y_1,\ldots,Y_v)$
        \EndProcedure
    \end{algorithmic}
\end{algorithm}

Algorithm \ref{alg:estmoments} first computes an $(\eps_1,\delta/2)$-estimate $\wt{W}$ of the sum $W$ of the weights of elements in $A$ using the sum estimation algorithm in \cite{BT2022}\footnote{\cite{BT2022} states the sample complexity for probability of success at least $2/3$. Here, we are stating the bounds for an $(\eps_1,\delta/2)$-estimate. This is obtained using an application of the medians of means technique.}. Let the probability of sampling element $a_j$ using proportional sampling be given as $p_j=\frac{w(a_j)}{W}$. Note that we don't know $p_j$, however we can obtain a good approximation of it as follows. For a proportional sample $a_j$, we know its weight $w(a_j)$, and $\wt{W}$ gives us an approximation of $W$. Using this we get a approximation for $p_j$ as $\ti{p}_j=\frac{w(a_j)}{\wt{W}}$. Let $\E$ denote the event that $\wt{W}\in [(1-\eps_1)W,(1+\eps_1)W]$. In what follows we condition on event $\E$.

\begin{claim}\label{claim:prob} Conditioned on $\E$, for any $j\in [n]$, we have $\frac{p_j}{1+\eps_1}\leq \ti{p}_j\leq \frac{p_j}{1-\eps_1}$. \end{claim}
\begin{proof} Conditioned on $\E$, $(1-\eps_1)W\leq \wt{W}\leq (1+\eps_1)W$. The above inequality follows. \end{proof}

Conditioned on $\E$, for any $j$, we have $\frac{p_j}{1+\eps_1}\leq \ti{p}_j\leq \frac{p_j}{1-\eps_1}$. Given a proportional sample $a_j$, we define a random variable $X_j$ with value $\frac{w(a_j)^t}{\ti{p}_j}$. Then, we have $(1-\eps_1)S_t\leq \EE[X_j]\leq (1+\eps_1)S_t$. Here, $X_j$ is not an unbiased estimator of $S_t$. Next, we bound the variance of this estimator given as $\var[X_j]=\EE[X_j^2]-\EE^2[X_j]\leq \EE[X_j^2]$. 

\begin{align}
\EE[X_j^2]
& = \sum_{a_j\in A} \frac{w(a_j)^{2t}}{\ti{p}_j^2} p_j \nonumber\\
& \leq (1+\eps_1)^2 \sum_{a_j\in A} \frac{w(a_j)^{2t}}{p_j} \nonumber\\
& = (1+\eps_1)^2 \cdot W \cdot \sum_{a_j\in A} \frac{w(a_j)^{2t}}{w(a_j)} && \text{(Substituting $p_j$ with $\frac{w(a_j)}{W}$)} \nonumber\\
& = (1+\eps_1)^2 \cdot W \cdot \sum_{a_j\in A} w(a_j)^{2t-1} \label{eqn:upper-sample}
\end{align}

Let us obtain $l$ independent samples using proportional sampling and let these random variables be $X_1,\ldots,X_l$. Let $X=\frac{1}{l} \sum_{j=1}^l X_j$. We have $(1-\eps_1)S_t\leq \EE[X]=\EE[X_j]\leq (1+\eps_1)S_t$, and $\var[X]=\frac{\var[X_j]}{l}$. Using Chebyshev's inequality, we have $\Pr[|X-S_t|>\eps S_t]\leq \Pr[|X-\EE[X]|>(\eps-\eps_1) S_t]]\leq \frac{\var[X]}{(\eps-\eps_1)^2 S_t^2}$. For appropriate choice of parameter $l$, we show this probability to be at most a small constant using the following claim. 
%bound this probability to be at most $\frac{(1+\eps_1)^2}{l(\eps-\eps_1)^2} \cdot n^{1-1/t}$ in the following claim whose proof is given in Appendix \ref{sec:omit}.

%\begin{claim} $\frac{\var[X]}{(\eps-\eps_1)^2 S_t^2}\leq \frac{(1+\eps_1)^2}{l(\eps-\eps_1)^2} \cdot n^{1-1/t}$. \end{claim}

\begin{claim} $\frac{\var[X]}{(\eps-\eps_1)^2 S_t^2}\leq \frac{(1+\eps_1)^2}{l(\eps-\eps_1)^2} \cdot n^{1-1/t}$. \end{claim}

\begin{proof}
\begin{align*}
& \frac{\var[X]}{(\eps-\eps_1)^2 S_t^2} \\
\leq &\frac{\EE[X_j^2]}{l(\eps-\eps_1)^2 \cdot S_t^2} \\
 \leq &\frac{(1+\eps_1)^2}{l(\eps-\eps_1)^2} \cdot \frac{W \cdot \sum_{a_j\in A} w(a_j)^{2t-1}}{S_t^2} && \text{(Using Equation (\ref{eqn:upper-sample}))}\\
 = & \frac{(1+\eps_1)^2}{l(\eps-\eps_1)^2} \cdot \frac{W \cdot ||w(A)||_{2t-1}^{2t-1}}{S_t^2} && \text{(vector $w(A)$ has length $n$)} \\
 \leq &\frac{(1+\eps_1)^2}{l(\eps-\eps_1)^2} \cdot \frac{W \cdot {(||w(A)||_t^t)}^{2-1/t}}{S_t^2} && \text{(Using Fact~\ref{fact:norms}, $||w(A)||_{2t-1}\leq ||w(A)||_t$)} \\
 = &\frac{(1+\eps_1)^2}{l(\eps-\eps_1)^2} \cdot \frac{||w(A)||_1 \cdot (||w(A)||_t^t)^{2-1/t}}{(||w(A)||_t^t)^2}\\
 = &\frac{(1+\eps_1)^2}{l(\eps-\eps_1)^2} \cdot \frac{||w(A)||_1}{||w(A)||_t}\\
 \leq &\frac{(1+\eps_1)^2}{l(\eps-\eps_1)^2} \cdot n^{1-1/t} && \text{(Using Fact~\ref{fact:norms}, $||w(A)||_1\leq n^{1-1/t} ||w(A)||_t$)}
\end{align*}
\end{proof}


Let $\eps_1=\eps/2$. For $l=48n^{1-1/t}/\eps^2$, this failure probability is at most $1/3$. Using the standard median trick, we show that for $v=48\log 2/\delta$, this failure probability can be reduced to be at most $\delta/2$. Let us define independent Bernoulli random variables $Z_1,\ldots,Z_v$ such that $\Pr[Z_i=1]=2/3$ for all $i$. Let $Z=\sum_{r=1}^v Z_r$. Now, conditioned on $\E$, the probability that the output of Algorithm \ref{alg:estmoments} does not lie in the interval $[(1-\eps)S_t,(1+\eps)S_t]$ is the same as the probability that the median of $Y_1,\ldots,Y_v$ lies outside the interval $[(1-\eps)S_t,(1+\eps)S_t]$. This probability is at most $\Pr[Z<v/2]$. Using a standard application of a Chernoff bound, given in Lemma \ref{lem:chernoff}, we have $\Pr[Z<v/2]\leq \delta/2$.

\paragraph{Correctness and Sample complexity bounds}

We need to show that Algorithm \ref{alg:estmoments} returns an estimate $ALG(A,t,\eps,\delta)$ for which with probability at least $1-\delta$, we have $(1-\eps)S_t\leq ALG(A,t,\eps,\delta)\leq (1+\eps)S_t$. Step $2$ of Algorithm \ref{alg:estmoments} uses $\Theta{(\frac{\sqrt{n}\log (2/\delta)}{\eps_1})}$ proportional samples to obtain an $(\eps_1,\delta/2)$ estimate $\wt{W}$ of $W$. Algorithm \ref{alg:estmoments} fails if either $\E$ does not hold or the estimate in Step $10$ is incorrect. Since both of these failure probabilities are at most $\delta/2$, the algorithm succeeds with probability at least $1-\delta$. 

Algorithm \ref{alg:estmoments} uses $\Theta{(\frac{\sqrt{n}\log (2/\delta)}{\eps_1})}$ proportional samples in Step 2 and uses $O(\frac{n^{1-1/t}\log 1/\delta}{\eps^2})$ proportional samples in Steps 3 and 4. Therefore, the required number of proportional samples is $O(\frac{\sqrt{n}\log 1/\delta}{\eps_1} + \frac{n^{1-1/t}\log 1/\delta}{\eps^2})$. For $\eps_1=\eps/2$, this gives $O(\frac{\sqrt{n}\log 1/\delta}{\eps} + \frac{n^{1-1/t}\log 1/\delta}{\eps^2})$.


\section{Lower bound for Moment Estimation using Proportional Sampling}\label{sec:lower-proportional}

We use Yao's minimax lemma to prove the sample complexity lower bound for obtaining an $(\eps,\delta)$ estimate of $S_t$ using any randomized algorithm. We construct two families of instances on which $S_t$ differs by at least a $(1\pm \eps)$-factor and show that it is hard to distinguish these two instances using a small number of proportional samples.

Our lower bound constructions are as follows. There are $n_1$ elements of weight $d_1$ and $n_2$ elements of weight $d_2$ in both the instances, where the values of the parameters $n_1, n_2$ and $d_1$ are the same in both instances, and the instances differ in the value of parameter $d_2$. The exact values of these parameters will be set below. In one instance we set $d_2=n$, where as for the other instance $d_2=0$. These choices for parameter values creates a gap of a multiplicative $(1\pm \eps)$ factor between the $S_t$ values of the two instances. One can differentiate these two instances only when an element of weight $d_2=n$ is sampled using proportional sampling, and we show that this requires a lot of samples. 

\begin{thm} For any $\eps,\delta \in (0,1)$ and $t>1$, any randomized algorithm that computes an $(\eps,\delta)$-estimate of $S_t$ requires $\Omega(\frac{n^{1-1/t}\ln 1/\delta }{\eps^2})$ proportional samples. \end{thm}

\begin{proof} We construct two families of instances which we show are hard to be distinguished using a few proportional samples by Yao's lemma. In the first instance we have $n_1$ elements with weight $d_1$ and $n_2$ elements of weight $0$, and in the second instance, there are $n_1$ elements of weight $d_1$ and $n_2$ elements of weight $d_2$, where the following values are used. The parameter values used in the lower bound constructions of the two instances are given as follows. 

%\begin{tabular}{p{8cm}|p{8cm}}
%Instance $1$ & Instance $2$\\ 
%$n_1  =  \frac{n^2}{n + \eps^{\frac{2t-1}{t-1}}}$ & $n_1  =  \frac{n^2}{n+ \eps^{\frac{2t-1}{t-1}}}$\\
%$d_1  =  n^{1-1/t} \eps^{1/(t-1)}$ & $d_1 =  n^{1-1/t} \eps^{1/(t-1)}$\\
%$n_2  =  \frac{n \eps^{\frac{2t-1}{t-1}}}{n + \eps^{\frac{2t-1}{t-1}}}$ & $n_2  =  \frac{n \eps^{\frac{2t-1}{t-1}}}{n + \eps^{\frac{2t-1}{t-1}}}$\\
%$d_2  =  0$ & $d_2 = n$
%\end{tabular}

\begin{gather}
\begin{align*}
\qquad \qquad \quad n_1 & =  \frac{n^2}{n + \eps^{\frac{2t-1}{t-1}}} & n_1 & =  \frac{n^2}{n+ \eps^{\frac{2t-1}{t-1}}}\\
\qquad \qquad \quad d_1 & =  n^{1-1/t} \eps^{1/(t-1)} & d_1 & =  n^{1-1/t} \eps^{1/(t-1)}\\
\qquad \qquad \quad n_2 & =  \frac{n \eps^{\frac{2t-1}{t-1}}}{n + \eps^{\frac{2t-1}{t-1}}} & n_2 & =  \frac{n \eps^{\frac{2t-1}{t-1}}}{n + \eps^{\frac{2t-1}{t-1}}}\\
\qquad \qquad \quad d_2 & =  0 & d_2 & = n
\end{align*}
\end{gather}


\noindent The $S_t$ value for the first instance is given as follows.
\begin{align*}
n_1 \cdot d_1^t + n_2 \cdot 0
& = \frac{n^2}{n + \eps^{\frac{2t-1}{t-1}}} \cdot n^{t-1} \eps^{t/(t-1)}\\
& = \frac{n^{t+1} \eps^{t/(t-1)}}{n + \eps^{\frac{2t-1}{t-1}}} 
\end{align*}


\noindent The $S_t$ value for the second instance is 
\begin{align*}
n_1\cdot d_1^t + n_2 \cdot d_2^t 
& = \frac{n^{t+1} \eps^{t/(t-1)}}{n + \eps^{\frac{2t-1}{t-1}}} +  \frac{n \eps^{\frac{2t-1}{t-1}}}{n + \eps^{\frac{2t-1}{t-1}}} \cdot n^t\\
& = \frac{n^{t+1} \eps^{t/(t-1)}}{n + \eps^{\frac{2t-1}{t-1}}} + \eps \frac{n^{t+1} \eps^{t/(t-1)}}{n + \eps^{\frac{2t-1}{t-1}}}\\
& = (1+\eps) \frac{n^{t+1} \eps^{t/(t-1)}}{n +\eps^{\frac{2t-1}{t-1}}}
\end{align*}

The above two instances differ in their $S_t$ values by a multiplicative factor of $(1+\eps)$. In order to distinguish these two instances, one is required to sample an element of weight $d_2=n$. Using proportional sampling, the probability of sampling an element of weight $n$ in the above instance is given to be at least


\begin{align*}
\frac{n_2 d_2}{n_2 d_2 + n_1 d_1}
& = \frac{\frac{n \eps^{\frac{2t-1}{t-1}}}{n + \eps^{\frac{2t-1}{t-1}}} \cdot n} {\frac{n \eps^{\frac{2t-1}{t-1}}}{n + \eps^{\frac{2t-1}{t-1}}} \cdot n + \frac{n^2 }{n + \eps^{\frac{2t-1}{t-1}}} \cdot n^{1-1/t} \eps^{1/(t-1)}}\\
& =  \frac{n^2 \eps^{\frac{2t-1}{t-1}}}{n^2 \eps^{\frac{2t-1}{t-1}} + n^2 \cdot n^{1-1/t} \cdot \eps^{\frac{1}{t-1}}}\\
& =  \frac{1}{1+\frac{n^{1-1/t}}{\eps^2}}
\end{align*}

Let $p=\frac{1}{1+\frac{n^{1-1/t}}{\eps^2}}$. The lower bound on the sample complexity for this instance distinguishing problem is given as the number of samples required to observe a \textit{success} with probability at least $(1-\delta)$ while drawing independent samples from $Geom(p)$. Here, $Geom(p)$ denotes a geometric distribution with success probability $p$. The number of samples required to observe one \textit{success} from $Geom(p)$ with probability at least $(1-\delta)$ is at least $\Omega(\frac{\ln 1/\delta}{p})$. Therefore, $\Omega(\frac{n^{1-1/t}\ln1/\delta}{\eps^2})$ samples are required to distinguish these two instances with probability at least $1-\delta$.

\end{proof}

\section{Estimation of Moments using Hybrid Sampling}\label{sec:lower-hybrid}

In this section we prove a lower bound showing that for the moment estimation problem, the hybrid sampling framework does not provide any significant advantage over access to just the proportional sampling oracle. In contrast, note that for the sum estimation problem, hybrid sampling-based algorithms in fact give much better sample complexity bounds over proportional sampling \cite{MPX2007,BT2022}. We prove the following result.

\begin{thm} For any $\eps,\delta>0$ and $t>1$, any algorithm having access to a hybrid sampling oracle requires to make at least $\Omega(\frac{n^{1-1/t}\ln 1/\delta}{\eps^2})$ queries to compute an $(\eps,\delta)$-estimate for $S_t$. \end{thm}

We show that the lower bound instance described in Section \ref{sec:lower-proportional} yields a lower bound for the hybrid sampling as well. In order to distinguish these instances, one is required to sample an element of weight $n$. We have seen that using just proportional sampling $\Omega(\frac{n^{1-1/t} \log 1/\delta}{\eps^2})$ samples are required. The probability of sampling an element of weight $d_2$ using uniform sampling is given as $\frac{n_2}{n_1+n_2}$. This probability using the values of the parameters from Section \ref{sec:lower-proportional} equals $\frac{n_2}{n_1+n_2} = \frac{1}{1+\frac{n}{\eps^{\frac{2t-1}{t-1}}}}$. The instances are distinguished if an element of weight $n$ is sampled using either proportional sampling or uniform sampling. These two probabilities are given as $\frac{1}{1+\frac{n^{1-1/t}}{\eps^2}}$ and $\frac{1}{1+\frac{n}{\eps^{\frac{2t-1}{t-1}}}}$, respectively. Overall, we get a lower bound of $\Omega(\min\{\frac{n^{1-1/t}}{\eps^2}, \frac{n}{\eps^{\frac{2t-1}{t-1}}}\}\ln 1/\delta)=\Omega(\frac{n^{1-1/t}\ln 1/\delta}{\eps^2})$ for the $(\eps,\delta)$ moment estimation problem using hybrid sampling.

% !TEX root = paper.tex
\section{Characterization of Sample Complexity}\label{sec:characterize}

We define a {\it moment-density} parameter of the input that governs the sample complexity of the moment estimation problem using proportional sampling. For $S_t=\sum_{a\in A} w(a)^t$ and $W=\sum_{a\in A} w(a)$, we define the moment-density parameter as $$\rho=\max_{L\subseteq A} \frac{\frac{\sum_{a\in L} w(a)^t}{S_t}}{\frac{\sum_{a\in L}w(a)}{W}}=\max_{L\subseteq A} \frac{\sum_{a\in L} w(a)^t}{\sum_{a\in L} w(a)} \cdot \frac{W}{S_t}$$ 

We give an upper bound for an $(\eps,\delta)$-estimator of $S_t$ using $O(({\sqrt{n}}/{\eps}+\rho/\eps^2)\ln 1/\delta)$ proportional samples. 

\begin{thm}\label{thm:char-upper} There exists an algorithm $ALG$ that given proportional sampling access to the weights of elements of $A$ having moment-density parameter $\rho$ and parameters $t>1,\eps,\delta\in (0,1)$, provides an $(\eps,\delta)$-estimate of $S_t$ using $O(({\sqrt{n}}/{\eps}+\rho/\eps^2)\ln 1/\delta)$ proportional samples. \end{thm}

\begin{proof} Algorithm \ref{alg:estmoments} gives the above sample complexity bound. We adapt the calculations from Section \ref{sec:moments}. Following Equation \ref{eqn:upper-sample}, we can write the sample complexity of our $(\eps,1/3)$ estimator to be at most $O(\sqrt{n}/\eps + \frac{1}{\eps^2} \frac{W\cdot \sum_{a\in A} w(a)^{2t-1}}{S_t^2})$. We will show that $\frac{W\cdot \sum_{a\in A} w(a)^{2t-1}}{S_t^2} \leq \rho$.  
\begin{align*}
\frac{W\cdot \sum_{a\in A} w(a)^{2t-1}}{S_t^2}
& = \frac{W}{S_t} \cdot \frac{\sum_{a\in A} w(a)^{2t-1}}{\sum_{a\in A} w(a)^t}\\
%& = \frac{W}{S_t} \cdot \frac{\sum_{a\in A} w(a)^{t-1} \cdot w(a)^t}{\sum_{a\in A} w(a)^{t-1} \cdot w(a)}\\
%& = \frac{W}{S_t} \cdot \frac{\sum_{a\in A} \frac{w(a)^{t-1}}{\sum_{b\in A} w(b)^{t-1}} \cdot w(a)^t}{\sum_{a\in A} \frac{w(a)^{t-1}}{\sum_{b\in A} w(b)^{t-1}} \cdot w(a)}\\
& \leq \frac{W}{S_t} \cdot \max_{a\in A} \frac{w(a)^t}{w(a)}\\
&= \rho
\end{align*}

Therefore, the sample complexity of our $(\eps,1/3)$-estimate of $S_t$ is $O({\sqrt{n}}/{\eps}+\rho/\eps^2)$. The $(\eps,1/3)$-estimate of the moment can be improved to an $(\eps,\delta)$-estimate with a multiplicative $\ln 1/\delta$-factor sample complexity overhead using the standard median trick.
\end{proof}
Next, we show a $\Omega(\frac{\rho \ln 1/\delta}{\eps})$ lower bound on the sample complexity of any algorithm for the $(\eps,\delta)$-moment estimation problem on any instance with the moment-density parameter $\rho$. We construct two families of input distributions that are hard to be distinguished using a small number of proportional samples. The lower bound instances for the moment estimation using proportional sampling described in Section \ref{sec:lower-proportional} have the moment-density parameter values as $\rho=1$ and $\rho=O(n^{1-1/t}/\eps)$. Next, we show that similar lower bounds hold even when we restrict the input instances to have their moment-density parameter values to be within a constant factor.

\begin{thm}\label{thm:char-lower} For any $\rho$, $\eps,\delta>0$ and $t>1$, any randomized algorithm for an $(\eps,\delta)$-estimate of $S_t$ on an instance with the moment-density parameter $\rho$ requires $\Omega(\frac{\rho \ln 1/\delta}{\eps})$ proportional samples. \end{thm}
\begin{proof} 
  Given any $\rho,\eps,\delta$ as input, we construct two families of instances for which their moment-density parameter values $\rho_1,\rho_2=O(\rho)$ and their moment values $S_t$ differ by a $(1\pm \eps)$ factor. There are $n$ elements in both the instances. In the first instance $n_1$ elements will have weight $d_1$ and $n_2$ elements will have value $d_2$. In the second instance there are $n_1$ elements of weight $d_1$, $\frac{n_2}{3}$ elements of weight $d_2$, and $\frac{2n_2}{3}$ elements of weight $0$. We set the values of the parameters $n_1,n_2,d_1,d_2$ as follows:

\begin{gather}
\begin{align*}
\qquad \qquad \quad n_1 & =  \frac{n^2}{n +(3 \eps)^{\frac{2t-1}{t-1}}} & d_1 & =n^{1-1/t}( 3\eps)^{1/(t-1)}\\
\qquad \qquad \quad n_2 & =  \frac{n (3\eps)^{\frac{2t-1}{t-1}}}{n +(3 \eps)^{\frac{2t-1}{t-1}}} & d_2&=n
\end{align*}
\end{gather}

\noindent The moment value for the first instance is:
\begin{align*}
S_t = n_1d_1^t+n_2d_2^t&=\frac{n^2}{n +(3 \eps)^{\frac{2t-1}{t-1}}}\cdot n^{t-1}(3\eps)^{\frac{t}{t-1}}+\frac{n^{t+1} (3\eps)^{\frac{2t-1}{t-1}}}{n +(3 \eps)^{\frac{2t-1}{t-1}}} \\
&= \frac{n^{t+1}(3\eps)^{\frac{t}{t-1}}+n^{t+1}(3\eps)^{\frac{2t-1}{t-1}}}{n +(3 \eps)^{\frac{2t-1}{t-1}}}\\
&=\frac{n^{t+1}(3\eps)^\frac{t}{t-1}}{n+(3\eps)^{\frac{2t-1}{t-1}}}(1+3\eps)
\end{align*}

\noindent The moment value for the second instance is:
\begin{align*}
S_t = n_1d_1^t+\frac{n_2}{3}d_2^t+\frac{2}{3}n_2\cdot0 &=\frac{n^2}{n +(3 \eps)^{\frac{2t-1}{t-1}}}\cdot n^{t-1}(3\eps)^{\frac{t}{t-1}}+\frac{1}{3}\frac{n^{t+1} (3\eps)^{\frac{2t-1}{t-1}}}{n +(3 \eps)^{\frac{2t-1}{t-1}}}\\
&=\frac{n^{t+1}(3\eps)^\frac{t}{t-1}}{n+(3\eps)^{\frac{2t-1}{t-1}}}(1+\frac{3\eps}{3})\\
&=\frac{n^{t+1}(3\eps)^\frac{t}{t-1}}{n+(3\eps)^{\frac{2t-1}{t-1}}}(1+\eps)
\end{align*}

The moment values of the above two instances differ by a factor of $\frac{1+3\eps}{1+\eps}>1+\eps$. Let $\rho_1$ and $\rho_2$ be the moment-density parameters for these two instances, respectively. We compute the $\rho$ values of the instances as follows. 
\begin{align*}
\rho_1 & = \frac{n^t}{n} \cdot \frac{n_1 d_1 + n_2 d_2}{n_1 d_1^t+n_2 d_2^t}\\
& = n^{t-1} \cdot \frac{\frac{n^2}{n +(3 \eps)^{\frac{2t-1}{t-1}}} \cdot n^{1-1/t}( 3\eps)^{1/(t-1)} + \frac{n (3\eps)^{\frac{2t-1}{t-1}}}{n +(3 \eps)^{\frac{2t-1}{t-1}}}\cdot n }{\frac{n^2}{n +(3 \eps)^{\frac{2t-1}{t-1}}} \cdot n^{t-1}( 3\eps)^{t/(t-1)} + \frac{n (3\eps)^{\frac{2t-1}{t-1}}}{n +(3 \eps)^{\frac{2t-1}{t-1}}}\cdot n^t}\\
&=\frac{n^{1-1/t}+(3\eps)^2}{3\eps+(3\eps)^2}=\frac{n^{1-1/t}+9\eps^2}{3\eps+9\eps^2}
\end{align*}

Similarly, we compute $\rho_2$ as
\begin{align*}
\rho_2 & =\frac{n^t}{n} \cdot \frac{n_1 \cdot d_1 + n_2/3 \cdot d_2 + 2n_2/3 \cdot 0}{n_1 \cdot d_1^t + n_2/3 \cdot d_2^t + 2n_2/3 \cdot 0}\\ 
& = \frac{n^{1-1/t}+\frac{1}{3}(3\eps)^2}{3\eps(1+\eps)}= \frac{n^{1-1/t}+3\eps^2}{3\eps+3\eps^2} 
\end{align*}

We have two instances as given above where their $\rho$ values are within a constant factor of each other and the moment values differ by at least a factor of $(1+\eps)$. These instances using proportional sampling remain indistinguishable until an element with weight $d_2$ is sampled. The probability of sampling an element of weight $d_2$ using proportional sampling is at most:
\begin{align*}
\frac{n_2d_2}{n_1d_1+n_2d_2}=\frac{1}{1+\frac{n_1d_1}{n_2d_2}}=\frac{1}{1+\frac{n^2n^{1-1/t}(3\eps)^{1/(t-1)}}{n^2(3\eps)^{\frac{2t-1}{t-1}}}}=\frac{1}{1+\frac{n^{1-1/t}}{9\eps^2}}\leq \frac{1}{1+O(\frac{\rho}{\eps})}
\end{align*} 

Therefore, we require at least $\Omega(\rho/\eps)$ samples to distinguish the two instances with constant probability of success. Any algorithm giving an $(\eps,\delta)$-estimate would need to make $\Omega (\frac{\rho \ln 1/\delta}{\eps})$ proportional samples.
\end{proof}

% !TEX root = paper.tex
\section{Moment estimation for $0<t<1$}\label{sec:small-t}

\subsection{Moment estimation using Proportional Sampling for 1/2<t<1} We use Algorithm \ref{alg:estmoments} to estimate moment $S_t$ for $1/2<t<1$. We adapt the analysis in Section \ref{sec:moments} to prove the following result.

\begin{thm} There exists an algorithm $ALG$ that given proportional sampling access to the weights of the elements of a set $A$ and parameters $1/2<t<1$, $\eps\in (0,1)$, provides an $(\eps,1/3)$-estimate of $S_t$ using $O(\frac{\sqrt{n}}{\eps}+\frac{n^{\frac{1}{t}-1}}{\eps^2})$ samples. \end{thm}

The sample complexity of the algorithm for $1/2<t<1$ is given as follows. Note that the sample complexity of our algorithm is given as $\frac{W (||w(A)||_{2t-1})^{2t-1}}{\eps^2 (||w(A)||_t)^{2t}}$. We can rewrite it as $\frac{||w(A)||_1}{\eps^2 ||w(A)||_t} \cdot (\frac{||w(A)||_{2t-1}}{||w(A)||_t})^{2t-1}$. Using Fact \ref{fact:norms}, this is upper bounded as $n^{\frac{1}{t}-1}/\eps^2$. Overall sample complexity of the $(\eps,1/3)$-estimator of the moment is $O(\sqrt{n}/\eps+ n^{\frac{1}{t}-1}/\eps^2)$. Using the median trick we transform the $(\eps,1/3)$-estimator to an $(\eps,\delta)$-estimator with sample complexity $O((\sqrt{n}/\eps+ n^{\frac{1}{t}-1}/\eps^2) \ln 1/\delta)$.


\subsection{Lower bound for $t\leq 1/2$} We show that there are no sublinear algorithms for the problem when $t\leq 1/2$.

\begin{thm} For any $\eps>0$ and $t\leq 1/2$, any randomized algorithm that computes an $(\eps,1/3)$-estimate of $S_t$ requires $\Omega(n)$ proportional samples. \end{thm}
\begin{proof} We construct two families of instances which we show are hard to be distinguished using proportional sampling. In one instance we have one element with weight $(n-1)$, and the rest $(n-1)$ elements with weight $0$. In another instance, we have one element of weight $(n-1)$, and $(n-1)$ elements have weight $\frac{\eps^{1/t}}{n-1}$. The moment values of the two instances are given as $(n-1)^t$ and $(n-1)^t + \eps (n-1)^{1-t}\geq (1+\eps) (n-1)^t$ as $t\leq 1/2$. Hence, there exists a $(1+\eps)$-multiplicative gap between the moment values of these two instances.

These two instances are distinguished using proportional sampling once we sample an element of weight $\frac{\eps^{1/t}}{n-1}$. The probability of sampling such an element is at most $\frac{\eps^{1/t}}{\eps^{1/t} + (n-1)}$. Hence, one requires $\Omega(n)$ samples to distinguish the two instances with constant probability. \end{proof}


\section*{Conclusion}
This paper aims to enhance our understanding of the computational complexity of computing various Shapley value variants. We found that for various ML models --- including decision trees, regression tree ensembles, weighted automata, and linear regression --- both local and global interventional and baseline SHAP can be computed in polynomial time under HMM modeled distributions. This extends popular algorithms, such as TreeSHAP, beyond their empirical distributional scope. We also establish strict complexity gaps between the various SHAP variants (baseline, interventional, and conditional) and prove the intractability of computing SHAP for tree ensembles and neural networks in simplified scenarios. Overall, we present SHAP as a versatile framework whose complexity depends on four key factors: \begin{inparaenum}[(i)] \item model type, \item SHAP variant, \item distribution modeling approach, \item and local vs. global explanations\end{inparaenum}. We believe this perspective provides deeper insight into the computational complexity of SHAP, paving the way for future work.




%We believe that our framework provides a more intricate understanding of SHAP computation complexity across different models, distributions, and variants, paving the way for further research.

Our work opens promising directions for future research. First, expanding our computational analysis to other SHAP-related metrics, such as asymmetric SHAP~\citep{frye20} and SAGE~\citep{covert2020understanding}, would be valuable. Additionally, we aim to explore more expressive distribution classes and relaxed assumptions beyond those in Section \ref{sec:tractable} while maintaining tractable SHAP computation. Finally, when exact computation is intractable (Section \ref{sec:intractable}), investigating the approximability of SHAP metrics through approximation and parameterized complexity theory~\citep{downey2012parameterized} is an important direction.

%Our work opens several promising avenues for future research on the computational properties of explainable AI methods, with a particular focus on SHAP. First, it would be interesting to broaden the computational analysis conducted in this work to include other popular SHAP-related metrics in the literature, such as asymmetric SHAP \cite{frye20} and SAGE \cite{covert2020understanding}. Also, in the future, we aim to explore more expressive distribution classes and relaxed distributional assumptions—extending beyond those examined in Section \ref{sec:tractable} —that still yield tractable SHAP computation. Finally, when exact computation proves intractable (Section \ref{sec:intractable}), it is worthwhile to theoretically investigate the question of the approximability of computing the SHAP metrics across various configurations, through the lens of approximation and parametrized complexity theory \cite{arora2009computational}.

%This paper aims to deepen our understanding of the computational complexity involved in obtaining different Shapley value variants. We found that for a variety of ML models, including decision trees, tree ensembles for regression, weighted automata, and linear regression models — computing both local and global interventional and baseline SHAP can be done in polynomial time when distributions are modeled by HMMs. This extends the distributional scope of popular algorithms like TreeSHAP, which is limited to empirical distributions. Additionally, we demonstrate a strict complexity gap between SHAP variants, showing that interventional and baseline SHAP can be strictly easier to compute than conditional SHAP. Despite these positive results, we uncovered intractability for various SHAP variants in neural networks and tree ensembles. Finally, we provided generalized complexity relations across SHAP variants. We believe that our framework offers a deeper understanding of the complexity involved in computing SHAP across various variants, models, distributions, as well as in both local and global computations, laying the groundwork for future research.

\subsubsection{Acknowledgements} Anup Bhattacharya is supported by the Science and Engineering Research Board (SERB) via the project (CRG/2023/002119). The authors thank the reviewers for suggestions to improve the manuscript. 

\bibliographystyle{splncs04}
\bibliography{ref}


\end{document}
