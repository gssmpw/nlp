\section{Related Work}
In recent years, several studies have focused on the automated segmentation of bone structures.
Besler~\textit{et al}.~\cite{besler_2021_Bone} proposed an enhance-and-segment pipeline, which combined Hessian-based filtering and graph cut segmentation for proximal femur segmentation in CT.
Gao~\textit{et al}.~\cite{gao_2022_Deep} introduced a multi-angle projection network for rib fracture segmentation in CT, utilizing rib extraction, fracture segmentation, and multi-angle projection fusion modules to capture rib and fracture features.
Liu~\textit{et al}.~\cite{liu_2023_Pelvic} developed a two-stage approach for pelvic fracture segmentation, using a pelvic bone segmentation network followed by a fracture segmentation network.
Cai~\textit{et al}.~\cite{cai_2024_Automatic} proposed a 3D U-Net-based method for knee CT segmentation, generating 3D fracture maps and aiding Schatzker classification to improve clinical diagnostic efficiency.
Zhou~\textit{et al}.~\cite{zhou_2024_CrossScale} designed a framework for hip fracture segmentation, incorporating a cross-scale attention mechanism and a surface supervision strategy to enhance fracture representation and boundary accuracy.

Some studies have focused on semi-supervised segmentation of bone structures.
Zhao~\textit{et al}.~\cite{zhao_2019_SemiSupervised} proposed a self-training method for finger bone segmentation in hand X-rays, utilizing a U-Net model with a conditional random field module to generate pseudo-labels for unlabeled images.
Burton~\textit{et al}.~\cite{burton_2020_Semisupervised} introduced a teacher-student model for knee MRI segmentation, employing a Monte Carlo patch sampling strategy to improve accuracy.
Liu~\textit{et al}.~\cite{liu_2020_Semisupervised} explored generative adversarial network (GAN)-based methods for lumbosacral structure segmentation on thin-layer CT, using semi-cGAN and few-shot-GAN models.
Li~\textit{et al}.~\cite{li_2023_Patchshufflebased} proposed a patch-shuffle-based augmentation technique, leveraging bone structure uniqueness and consistency loss to align features between original and shuffled CT slices.
Li~\textit{et al}.~\cite{li_2025_Semisupervised} developed a semi-supervised tooth segmentation method combining an entropy-guided mean-teacher approach and a weakly mutual consistency network.