%TC:ignore
\section{Scenario Variants}
\label{appendix:scenarios}

This section provides the complete background text provided to participants for each scenario variant. Scenarios are labeled as \texttt{Context-Partner}.

\subsection{Research-AI}
Imagine that you’re an employee of a large, international software company. As part of your work, you're writing a research paper that will be peer reviewed and published at an international conference. The paper describes findings and implications of a research study that you conducted.
% The following three sections present different ways of working (or not working) with AI to help you write the paper.

\subsection{Research-Human}
Imagine that you’re an employee of a large, international software company. As part of your work, you're writing a research paper that will be peer reviewed and published at an international conference. The paper describes findings and implications of a research study that you conducted.
% The following three sections present different ways of working (or not working) with a colleague to help you write the paper.

\subsection{Professional-AI}
Imagine that you’re an employee of a large, international software company. As part of your work, you’re writing an article that gives health advice. This article will be published on the company's public-facing website. It includes tips and recommendations for improving emotional well-being.
% The following three sections present different ways of working (or not working) with AI to help you write the article.

\subsection{Professional-Human}
Imagine that you’re an employee of a large, international software company. As part of your work, you’re writing an article that gives health advice. This article will be published on the company's public-facing website. It includes tips and recommendations for improving emotional well-being.
% The following three sections present different ways of working (or not working) with a colleague to help you write the article.

\subsection{Technical-AI}
Imagine that you’re an employee of a large, international software company. As part of your work, you're writing documentation on a new technology. This documentation will be published on the company's public-facing website. It includes details of how the technology works and how to use it.
% The following three sections present different ways of working (or not working) with AI to help you write the documentation.

\subsection{Technical-Human}
Imagine that you’re an employee of a large, international software company. As part of your work, you're writing documentation on a new technology. This documentation will be published on the company's public-facing website. It includes details of how the technology works and how to use it.
% The following three sections present different ways of working (or not working) with a colleague to help you write the documentation.

\section{Survey Instrument}
\label{appendix:survey-instrument}

The survey completed by study participants is presented in this section. Portions of the survey text were variable, based on the writing context and writing partner conditions described in Section~\ref{section:study-design}. Variable text based on the participant's condition appears within [square brackets]. References to ``[AI / your colleague],'' ``[an AI / your colleague],'' or ``[the AI / your colleague]'' appeared based on writing partner condition (AI or Human). References to the ``[artifact]'' were replaced by ``research paper,'' ``article,'' or ``documentation'' based on the writing context (research, professional, or technical writing).

In the survey presented to participants, the three sections on type of contribution, amount of contribution, and initiative were shown in a random order to minimize order effects.

% \subsection{Instructions}
% \textit{Note: These instructions were the same for all participants.}

% This survey is part of a study conducted in IBM Research to understand people’s perceptions of how to attribute authorship for collaborative work. Survey responses may be used to inform design suggestions for IBM products and included in a research paper. All responses will be anonymized. You may exit the survey at any time by closing this browser tab if you do not wish to continue.

% If you have any questions, please contact Jessica He via Slack or email. Thank you!

\subsection{Experience screener}
\label{survey-screener}

\textit{Note: Respondents who selected ``No'' were disqualified from participating in our study.}

\begin{itemize}[leftmargin=0pt, itemindent=2em]
    \item[1.] Have you used generative AI applications (such as such as watsonx.ai, ChatGPT, DALL-E, Gemini, etc.) in any capacity?
    \item Yes
    \item No
\end{itemize}

\subsection{Scenarios}
\label{survey-scenarios}

[Insert scenario variant text from Appendix~\ref{appendix:scenarios}]

The following three sections present different ways of working (or not working) with [AI / your colleague] to help you write the [artifact]. For each scenario, please indicate what you think is the most accurate way to attribute authorship. Remember that there are no right or wrong answers.

\textit{Participants used the following scale to rate the questions in each contribution dimension.}

\begin{itemize}
    \item You are the sole author
    \item You are the primary author; [AI / your colleague] is acknowledged but not as an author
    \item You are the primary author; [AI / your colleague] is the secondary author
    \item You and [AI / your colleague] have equal authorship
    \item {[AI / your colleague]} is the primary author; you are the secondary author
    \item {[AI / your colleague]} is the primary author; you are acknowledged but not as an author
    \item {[AI / your colleague]} is the sole author
    \item Unsure
\end{itemize}

\subsubsection{Type of contribution}
The following scenarios will focus on your perceptions of authorship for different types of contributions made by [the AI / your colleague]. For all scenarios, assume that contributions from [the AI / your colleague] are included in the final [artifact].
    
\begin{itemize}[leftmargin=0pt, itemindent=2em]
    \item[2.] You write the [artifact] and ask [the AI / your colleague] to elaborate on your ideas.
    \item[3.] You write the [artifact] and ask [the AI / your colleague] to make wording changes to improve readability and clarity.
    \item[4.] You write the [artifact] and ask [the AI / your colleague] to identify out-of-date or incorrect information to correct.
    \item[5.] You write the [artifact] and ask [the AI / your colleague] to modify the organization or structure of the [artifact].
    \item[6.] You write notes and ideas and ask [the AI / your colleague] to synthesize the information into a cohesive [artifact].
    \item[7.] You write the [artifact] and ask [the AI / your colleague] to modify the tone and style to ensure it is suitable for the target audience.
    \item[8.] You write the [artifact] and ask [the AI / your colleague] to make spelling and grammar corrections.
    \item[9.] You write the [artifact] and ask [the AI / your colleague] to contribute new ideas.
    \item[10.] You write the [artifact] and ask [the AI / your colleague] to narrow the scope by removing less important ideas.
\end{itemize}

\subsubsection{Amount of contribution}
The following scenarios will focus on your perceptions of authorship for different amounts of contribution made by [the AI / your colleague]. For all scenarios, assume that contributions from [the AI / your colleague] are included in the final [artifact].

\begin{itemize}[leftmargin=0pt, itemindent=2em]
    \item[11.] You ask [an AI / your colleague] to write the full [artifact].
    \item[12.] You write a few sentences of the [artifact] and ask [an AI / your colleague] to write the rest.
    \item[13.] You write most of the [artifact] and ask [an AI / your colleague] to write a few sentences.
    \item[14.] You write the [artifact] on your own, without [AI / your colleague's] assistance.
    \item[15.] You write a few paragraphs of the [artifact] and ask [an AI / your colleague] to write a few other paragraphs.
\end{itemize}

\subsubsection{Initiative}
The following scenarios will focus on your perceptions of authorship when different parties initiate different types of contributions. For all scenarios, assume that contributions from [the AI / your colleague] are included in the final [artifact].

\begin{itemize}[leftmargin=0pt, itemindent=2em]
    \item[16.] {[The AI / Your colleague]} proactively suggests a complete [artifact] without being asked. The [artifact] is published without any modifications.
    \item[17.] You ask [the AI / your colleague] for ideas or feedback. You incorporate some of its recommendations into the [artifact].
    \item[18.] You ask [the AI / your colleague] to write the [artifact]. The [artifact] is published without any modifications.
    \item[19.] {[The AI / your colleague]} proactively provides ideas or feedback without being asked. You incorporate some of its recommendations into the [artifact].
\end{itemize}

\subsection{Attribution determination}
% \label{survey-Importance}

\begin{itemize}[leftmargin=0pt, itemindent=2em]
    \item[20.] \textit{Open response:} Please explain how you determined appropriate attribution for the scenarios in the previous three sections.
\end{itemize}

\begin{itemize}[leftmargin=0pt, itemindent=2em]
    \item[21.] Please indicate the importance of the following factors when you are determining who to attribute authorship to.
    \item Type of contribution (e.g. who wrote or modified wording vs. ideas)
    \item Amount of contribution (e.g. who wrote the most words)
    \item Initiative (e.g. who wrote the words or ideas first)
\end{itemize}

\textit{Participants rated the above items on the following scale.}

\begin{itemize}
    \item Not at all important
    \item Not very important
    \item Neither important nor unimportant
    \item Important
    \item Very important
\end{itemize}

\begin{itemize}[leftmargin=0pt, itemindent=2em]
    \item[22.] \textit{Open response:} Would any other factors affect your decision on who to attribute authorship to? If so, how would you describe the importance of these factors? 
\end{itemize}


% Let’s say both you and <an AI/your colleague> have contributed to a piece of writing in some capacity. For each scenario, consider if or why attribution information should be shown.

% For the purposes of this question, attribution means any form of credit or acknowledgment for <the AI's/your colleague's> involvement that you consider appropriate. Assume any attribution information shown will be visible to all stakeholders and consumers.

% \begin{itemize}[leftmargin=0pt, itemindent=2em]
%     {\setlength\itemindent{0pt} \item[] In the following scenarios, attribution for the AI is...
%     (Select all that apply.)}
    
%     \textit{Check-box options for each scenario:
%     \item Not important at all
%     \item Important for giving acknowledgment or authorship credit
%     \item Important for transparency or ethical reasons
%     \item Important for liability and legal reasons
%     \item Important for recording how content was created
%     }
     
%     \textit{Scenarios:}
%     \item You work with <an AI/your colleague> to write something that will not be published and will only be visible to you.
%     \item You work with <an AI/your colleague> to write something that will be shared with your teammates.
%     \item You work with <an AI/your colleague> to write something that will be published on the Internet.
% \end{itemize}

\subsection{Closing}
% \label{survey-Closing}

\begin{itemize}[leftmargin=0pt, itemindent=2em]
    \item[23.] \textit{Open response:} Is there anything else you’d like to share on your perception of authorship in collaborative work?
\end{itemize}

\subsection{Demographic information}
% \label{survey-Demographics}

Our research will investigate authorship in collaborative work between people and generative AI systems. As such, we’d like to understand more about your background with generative AI, in addition to a few standard demographic questions.

\begin{itemize}[leftmargin=0pt, itemindent=2em]
    \item[24.] In what ways do you use or work with generative AI applications (such as such as watsonx.ai, ChatGPT, DALL-E, Gemini, etc.)? Select all that apply.
    \item I use generative AI for school/academic purposes.
    \item I use generative AI for work purposes.
    \item I use generative AI for personal purposes (outside of my job or school).
    \item I develop, design, or study generative AI as part of my job.
    \item I am involved in training or tuning generative AI models.
\end{itemize}

\begin{itemize}[leftmargin=0pt, itemindent=2em]
    \item[25.] On average, how frequently do you use generative AI applications (such as watsonx.ai, ChatGPT, DALL-E, Gemini, etc.)?
    \item A couple times in my life
    \item Yearly
    \item Monthly
    \item Weekly
    \item Daily
\end{itemize}

\begin{itemize}[leftmargin=0pt, itemindent=2em]
    \item[26.] \textit{Optional:} What is your primary job category?
    \item Architect
    \item Communications
    \item Consultant
    \item Data Science
    \item Design
    \item Enterprise Operations
    \item Finance
    \item General Management
    \item Hardware Development \& Support
    \item Human Resources
    \item Information Technology \& Services
    \item Legal
    \item Manufacturing
    \item Marketing
    \item Marketing \& Communications
    \item Product Management
    \item Product Services
    \item Project Executive
    \item Project Management
    \item Research
    \item Sales
    \item Services Solutions Management
    \item Site Reliability Engineer
    \item Software Development \& Support
    \item Supply Chain
    \item Technical Services
    \item Technical Specialist
    \item Other: (write in)
\end{itemize}

\begin{itemize}[leftmargin=0pt, itemindent=2em]
    \item[27.] \textit{Optional:} What geography are you located in?
    \item Americas
    \item APAC
    \item EMEA
    \item Japan
\end{itemize}
%TC:endignore
