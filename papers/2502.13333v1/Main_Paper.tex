%File: formatting-instructions-latex-2025.tex
%release 2025.0
\documentclass[letterpaper]{article} % DO NOT CHANGE THIS
\usepackage{aaai25}  % DO NOT CHANGE THIS
\usepackage{times}  % DO NOT CHANGE THIS
\usepackage{helvet}  % DO NOT CHANGE THIS
\usepackage{courier}  % DO NOT CHANGE THIS
\usepackage[hyphens]{url}  % DO NOT CHANGE THIS
\usepackage{graphicx} % DO NOT CHANGE THIS
\urlstyle{rm} % DO NOT CHANGE THIS
\def\UrlFont{\rm}  % DO NOT CHANGE THIS
\usepackage{natbib}  % DO NOT CHANGE THIS AND DO NOT ADD ANY OPTIONS TO IT
\usepackage{caption} % DO NOT CHANGE THIS AND DO NOT ADD ANY OPTIONS TO IT
\frenchspacing  % DO NOT CHANGE THIS
\setlength{\pdfpagewidth}{8.5in}  % DO NOT CHANGE THIS
\setlength{\pdfpageheight}{11in}  % DO NOT CHANGE THIS
%
% These are recommended to typeset algorithms but not required. See the subsubsection on algorithms. Remove them if you don't have algorithms in your paper.
\usepackage{algorithm}
\usepackage{algorithmic}
\usepackage{amssymb}
\usepackage{amsmath}
\usepackage{booktabs}
\usepackage{comment}

\DeclareMathOperator*{\argmin}{arg\,min}

%
% These are are recommended to typeset listings but not required. See the subsubsection on listing. Remove this block if you don't have listings in your paper.
\usepackage{newfloat}
\usepackage{listings}
\DeclareCaptionStyle{ruled}{labelfont=normalfont,labelsep=colon,strut=off} % DO NOT CHANGE THIS
\lstset{%
	basicstyle={\footnotesize\ttfamily},% footnotesize acceptable for monospace
	numbers=left,numberstyle=\footnotesize,xleftmargin=2em,% show line numbers, remove this entire line if you don't want the numbers.
	aboveskip=0pt,belowskip=0pt,%
	showstringspaces=false,tabsize=2,breaklines=true}
\floatstyle{ruled}
\newfloat{listing}{tb}{lst}{}
\floatname{listing}{Listing}
%
% Keep the \pdfinfo as shown here. There's no need
% for you to add the /Title and /Author tags.
\pdfinfo{
/TemplateVersion (2025.1)
}

% DISALLOWED PACKAGES
% \usepackage{authblk} -- This package is specifically forbidden
% \usepackage{balance} -- This package is specifically forbidden
% \usepackage{color (if used in text)
% \usepackage{CJK} -- This package is specifically forbidden
% \usepackage{float} -- This package is specifically forbidden
% \usepackage{flushend} -- This package is specifically forbidden
% \usepackage{fontenc} -- This package is specifically forbidden
% \usepackage{fullpage} -- This package is specifically forbidden
% \usepackage{geometry} -- This package is specifically forbidden
% \usepackage{grffile} -- This package is specifically forbidden
% \usepackage{hyperref} -- This package is specifically forbidden
% \usepackage{navigator} -- This package is specifically forbidden
% (or any other package that embeds links such as navigator or hyperref)
% \indentfirst} -- This package is specifically forbidden
% \layout} -- This package is specifically forbidden
% \multicol} -- This package is specifically forbidden
% \nameref} -- This package is specifically forbidden
% \usepackage{savetrees} -- This package is specifically forbidden
% \usepackage{setspace} -- This package is specifically forbidden
% \usepackage{stfloats} -- This package is specifically forbidden
% \usepackage{tabu} -- This package is specifically forbidden
% \usepackage{titlesec} -- This package is specifically forbidden
% \usepackage{tocbibind} -- This package is specifically forbidden
% \usepackage{ulem} -- This package is specifically forbidden
% \usepackage{wrapfig} -- This package is specifically forbidden
% DISALLOWED COMMANDS
% \nocopyright -- Your paper will not be published if you use this command
% \addtolength -- This command may not be used
% \balance -- This command may not be used
% \baselinestretch -- Your paper will not be published if you use this command
% \clearpage -- No page breaks of any kind may be used for the final version of your paper
% \columnsep -- This command may not be used
% \newpage -- No page breaks of any kind may be used for the final version of your paper
% \pagebreak -- No page breaks of any kind may be used for the final version of your paperr
% \pagestyle -- This command may not be used
% \tiny -- This is not an acceptable font size.
% \vspace{- -- No negative value may be used in proximity of a caption, figure, table, section, subsection, subsubsection, or reference
% \vskip{- -- No negative value may be used to alter spacing above or below a caption, figure, table, section, subsection, subsubsection, or reference

\setcounter{secnumdepth}{0} %May be changed to 1 or 2 if section numbers are desired.

% The file aaai25.sty is the style file for AAAI Press
% proceedings, working notes, and technical reports.
%

% Title

% Your title must be in mixed case, not sentence case.
% That means all verbs (including short verbs like be, is, using,and go),
% nouns, adverbs, adjectives should be capitalized, including both words in hyphenated terms, while
% articles, conjunctions, and prepositions are lower case unless they
% directly follow a colon or long dash
\title{An Uncertainty-Aware Data-Driven Predictive Controller for Hybrid Power Plants}
\author{
    %Authors
    % All authors must be in the same font size and format.
   Manavendra Desai, Himanshu Sharma, Sayak Mukherjee, Sonja Glavaski
}
\affiliations{
    %Afiliations
    Pacific Northwest National Laboratory
    % If you have multiple authors and multiple affiliations
    % use superscripts in text and roman font to identify them.
    % For example,

    % Sunil Issar\textsuperscript{\rm 2}, 
    % J. Scott Penberthy\textsuperscript{\rm 3}, 
    % George Ferguson\textsuperscript{\rm 4},
    % Hans Guesgen\textsuperscript{\rm 5}
    % Note that the comma should be placed after the superscript

    Richland, WA, 99354,USA\\
    % Washington, DC 20004 USA\\
    % email address must be in roman text type, not monospace or sans serif
    \{manavendrabalwant.desai, himanshu.sharma,sayak.mukherjee, sonja.glavaski\}@pnnl.gov
%
% See more examples next
}

%Example, Single Author, ->> remove \iffalse,\fi and place them surrounding AAAI title to use it
\iffalse
\title{My Publication Title --- Single Author}
\author {
    Author Name
}
\affiliations{
    Affiliation\\
    Affiliation Line 2\\
    name@example.com
}
\fi

\iffalse
%Example, Multiple Authors, ->> remove \iffalse,\fi and place them surrounding AAAI title to use it
\title{My Publication Title --- Multiple Authors}
\author {
    % Authors
    First Author Name\textsuperscript{\rm 1,\rm 2},
    Second Author Name\textsuperscript{\rm 2},
    Third Author Name\textsuperscript{\rm 1}
}
\affiliations {
    % Affiliations
    \textsuperscript{\rm 1}Affiliation 1\\
    \textsuperscript{\rm 2}Affiliation 2\\
    firstAuthor@affiliation1.com, secondAuthor@affilation2.com, thirdAuthor@affiliation1.com
}
\fi


% REMOVE THIS: bibentry
% This is only needed to show inline citations in the guidelines document. You should not need it and can safely delete it.
\usepackage{bibentry}
% END REMOVE bibentry

\begin{document}

\maketitle

% Uncomment the following to link to your code, datasets, an extended version or similar.
%
% \begin{links}
%     \link{Code}{https://aaai.org/example/code}
%     \link{Datasets}{https://aaai.org/example/datasets}
%     \link{Extended version}{https://aaai.org/example/extended-version}
% \end{links}

%\section{Key ingredients of a position paper}

%Take a stance on a particular research topic or research direction, backed by statistics/facts/relevant literature. Is not based on opinions but is a personal position that best reflects a personal understanding of a topic that is supported with credible facts. Challenge your thesis. Collect supporting evidence. A strong position paper takes a clear position on a topic that people can disagree about. When there’s no disagreement about a topic, it’s difficult to write a compelling position paper.

% \section{Skeleton}

% \begin{enumerate}
%   \item Abstract
%   \item \textcolor{red}{Introduction (convince readers that data-driven optimal control is relevant to the realm of AI for Science and Engineering)}.
%   \item \textcolor{red}{Motivate importance of hybrid energy power plants}
%   \item Motivate necessity to plan for uncertainties in the face of increased penetration of renewable energy in electricity power grid 
%   \item 
  
%   \begin{enumerate}
%       \item challenge for grid operators to forecast load due to increased incorporation of behind-the-meter renewable energy-based electricity generators

%       \item grid operators required to purchase energy (from energy storage devices or capacity markets) as contingency to reduced output from renewable energy-based generators (wind farm/solar farm) - can be expensive \$\$\$

%       \item challenge for power plant owners to make market bids, plan set-points, and control hybrid-energy power plants, in the face of uncertain weather forecasts and hard-to-model power plant dynamics/performance

%       \item \textcolor{red}{herein lies an opportunity for adding adaptive resiliency, robustness, and safety, from concepts of control theory, into data-driven schemes - leading to an \textit{intelligent} system/assistant with provable (and quantifiable?) stability/robustness/resiliency with expected control actions/plans that can be \textit{explained} through theoretical analysis}
      
%   \end{enumerate}
  
%   \item State objective and main contribution. This paper investigates the potential of a data-driven predictive controller as an intelligent planner/decision-maker/supervisor for a hybrid energy power plant. Critical analyses leading to the need and avenues available to incorporate uncertainty-awareness in such an intelligent planner are made. 
  
%   \item Intelligent supervisor for a hybrid energy power plant \textcolor{red}{(Please feel free to add/edit content on the supervisor)}.
%   \begin{enumerate}
%       \item Schematic of supervisory controller wrapped around inner-loop of wind farm, solar farm and battery with their respective controllers 
%       \item Explain concept. The supervisor, through its collection of behavior data, is expected to provide a sufficiently good representation of the closed loop performance of the inner-loop, and a sufficiently good prediction of the power output and battery utilization over a future horizon. 
%   \end{enumerate}

%   \item Simulation studies
%   \begin{enumerate}
%       \item Demonstrate ability of SPC to predict state of charge and discharge of battery and provide insight into expected power plant output over future horizon
%       \item Demonstrate compromised FO performance in the face of uncertainty. Make a case for uncertainty-aware predictive control.
% \end{enumerate} 

% \item Questions open to investigation \textcolor{red}{(Please feel free to add/edit research questions you may think are relevant)}

% \item Concluding remarks and next steps. \textcolor{red}{Highlight/remark that herein lies an opportunity to extend proofs and guarantees of stability/convergence/robustness from control-theory into the intelligent data-driven planner, thereby making the decisions (i.e., SPC optimization outputs) and performance bounds \textit{explainable}}. 
  
% \end{enumerate}

% \section{Key sentences to include Workshop theme}
% \begin{enumerate}
%     \item Predictive modeling for forecasting, and decision-making and control, of hybrid power plants

%     \item Sufficiently rich data-driven predictive models can be used as surrogates for high-order complex dynamical equations (prone to bias, assumptions and modeling errors) to accelerate simulations

%     \item Learning effective representation of structured data  

%     \item Challenges in applying and deploying in the real-world
    
%     \item Explainable predictions and decision making
% \end{enumerate}
% \pagebreak

\begin{abstract}
 Given the advancements in data-driven modeling for complex engineering and scientific applications, this work utilizes a data-driven predictive control method, namely \textit{subspace predictive control}, to coordinate hybrid power plant components and meet a desired power demand despite the presence of weather uncertainties. An uncertainty-aware data-driven predictive controller is proposed and its potential is analyzed using real-world electricity demand profiles. For the analysis, a hybrid power plant with a wind, solar, and co-located energy storage capacity of 4 MW each, is considered. The analysis shows that the predictive controller can track a real-world inspired electricity demand profile despite the presence of weather-induced uncertainties and be an intelligent forecaster for HPP performance.
\end{abstract}

\section{Introduction}
% Relevant References for Hybrid pkatn :
% https://emp.lbl.gov/hybrid
\noindent \textbf{Background:} The world is rapidly transitioning towards energy diversification to meet energy demands and reduce energy cost. Advances in renewable energy generation technologies, such as solar panels and wind turbines, along with energy storage solutions, have heightened the focus on their reliable and sustainable integration with conventional energy systems \cite{bolinger2023hybrid}. Hybrid power plants (HPP) have emerged as a promising solution, combining two or more technologies, which may include wind turbines, photovoltaic (PV) or concentrated solar power systems, energy storage, geothermal power, hydropower, biomass, natural gas, oil, coal, or nuclear power. In this study, we focus on HPPs designed for electricity production, specifically those leveraging renewable generation, such as wind and solar, combined with energy storage \cite{bolinger2023hybrid}. In the U.S., as of the end of 2023, solar PV and energy storage plants have led the market, with a growing presence of wind, solar, and energy storage combinations. Additionally, there is strong interest in hybridization among developers of renewable energy systems, evidenced by an increase in interconnection requests for HPPs.

Developers are also exploring advancements in power electronics for integrating renewable assets like solar, wind, and energy storage, to mimic capabilities of traditional power plants in terms of their capacity value, dispatchability, ancillary services, and reliability. The heightened penetration of renewable energy-dominant systems also motivates researchers to reconsider revenue streams for wind-based HPPs, comparable to conventional power plants \cite{dykes2020opportunities}. Additionally, this has inspired research in simultaneously optimizing for system design and operation (i.e control co-design) \cite{garcia2019control,herber2017advances, bird2024set, SharmaGraph} to ensure long term efficiency and reliable operation of the system. Furthermore, new regulations for grid operators and agencies are being assessed to mandate roles of renewable-energy generators in fortifying the power grid during contingencies \cite{li2023review}. Consequently, the design objectives for HPPs are shifting from merely reducing the levelized cost of energy to identifying income streams that maximize revenue through participation in time-varying energy pricing markets, ancillary services, and capacity markets. Further discussions on HPPs are available in \cite{dykes2020opportunities}.

\noindent \textbf{Motivation:} The ability of HPPs to function comparably to traditional power-generation plants is often compromised by the uncertainty and variability inherent in wind, solar, and energy storage technologies. This presents both a challenge and an opportunity for developers of HPPs to innovate control systems that deliver economic benefits akin to traditional power plants. Addressing this uncertainty can enhance the reliability of and decision-making in HPPs. 
% This study proposes to examine the role of accounting for uncertainty in the design of data-driven predictive controllers for hybrid power plants. 
With the advancement and proliferation of sensing technology in complex physical systems, data-driven predictive controller designs are anticipated to offer the necessary flexibility for HPP deployment, surpassing that of model-based controller designs. For hybrid systems encompassing multiple assets, such as wind, solar, and storage, capturing the dynamic operations of the plant is crucial for effective controller design.

\noindent \textbf{Related work:} In the age of "big data", data-driven modeling and decision-making for various engineering and scientific applications intersect significantly with the field of control theory \cite{annaswamy2024control}. Both fields mutually enhance their applicability to complex engineering challenges \cite{brunton2022data}. Recent advances in model-free decision-making have spurred significant developments in reinforcement learning and adaptive dynamic programming, both used in Markov decision processes (MDPs) \cite{sutton2018reinforcement, powell2007approximate}, as well as in control-theoretic state-space approaches with rigorous guarantees \cite{vrabie2009adaptive, jiang2012computational, mukherjee2021reduced}. Among these data-driven control methodologies, behavioral system theory has shown considerable promise, as highlighted in works such as \cite{de2019formulas, van2020data}. This progress has led the development of predictive extensions, including methods like data-enabled predictive control \cite{coulson2019data} and advancements in subspace predictive control \cite{fiedler2021relationship}. A comprehensive overview of data-driven control from a control theoretical perspective is provided by \cite{prag2022toward}, and for process control applications, by \cite{tang2022data}. A systematic exploration of how data-driven models and approaches can leverage cutting-edge machine learning methods to identify system dynamics crucial for control design is provided in \cite{brunton2022data}. Lastly, the integration of learned components into traditional control techniques, or deriving control actions directly from data, is increasingly popular due to advances in computing power and data acquisition. Reinforcement learning-based approaches have been identified as promising control solutions across various engineering applications, with detailed reviews presented by \cite{wang2022deep,hewing2020learning}.
% The behavioral approach to systems theory is being increasingly revisited for data-driven control \cite{willems2007behavioral}. 
% It represents a system as a collection of trajectories of input-output data capable of representing the entire space of possible system trajectories......The work specifically focuses on u

% next we briefly discuss the intersections of the two fields to motivate readers for identifying existing gaps in the literature. 
% The increase of renewable energy systems for meeting power demands support decarbonization goal, while at the same time creates additional challenges for grid operators due to the variability associated with the renewable generation resources. Therefore efficient control and decision making strategy is required for plants owner to to efficiently support the grid operations. Increasingly the 
% - Increased penetration good for the environment but creates challenges for grid operators for load forecasting (due to behind-the-meter RE devices) and for HPP owners for bidding in electricity markets and controlling HPPs - uncertain weather, hard-to-model HPPs, necessity to fulfill electricity demand, minimize purchase of backup energy storage and capacity power plants (coal fired/non-renewable energy PPs; back-up energy is expensive)
% \subsection{Intersection of control theory and data-driven control}
% Modeling of complex  physical  and engineering systemsare crucial for development and innovation for c. The moFor most of the engineering and physics based system desing and  the first principle f approach has been a 
% desing and
% \noindent \textbf{Motivation:} 
% ..... Direct data-driven control circumvents unmodelled dynamics and biases by virtue of directly using relevant system behavior data to make control decisions.   
% \noindent \textbf{Related work:} Past work on data-driven control. Challenges of uncertainty quantification/incorporation in data-driven modeling - Past work on managing uncertainty in constraints and models in data-driven decision-making and control - Shortcomings in dealing with uncertainty in data-driven decision making.
\noindent \textbf{Objective and main contributions:} This paper proposes, and investigates the potential of, an uncertainty-aware data-driven predictive controller as an intelligent decision-maker and supervisor for a hybrid power plant.
In this work, intelligence of the controller is considered and demonstrated in the context of its potential ability to a) encode data-driven HPP dynamics, b) provide forecasts of the power output of each component of the HPP, and c) anticipate and control the state of charge/discharge of the energy storage device in the event of peak demands from the electricity power grid.  

Next, a simulation study that utilizes real-world electricity demand profiles is presented and the performance of the controller is analyzed. Lastly, open research inquiries to make the decisions of the controller trustworthy and explainable, are discussed. 

The rest of this paper is organized as follows. Section \ref{sec:method} introduces the HPP considered in this work and the design and application of the uncertainty-aware data-driven predictive controller. Results of the simulation study and related discussions are provided in Sections \ref{sec:results} and \ref{sec:disc}, respectively. Finally, concluding remarks are made in Section \ref{sec:conc}.
\begin{figure}
    \centering
\includegraphics[width=\linewidth]{Figures/hsc_control_diag.png}
    \caption{The control scheme of the hybrid power plant (HPP) considered in this work. The HPP supervisory controller ($HSC$) attempts to track a load reference $P_r$ by selecting optimal setpoints (superscript $u$) for the underlying plant components. Measured outputs (superscript $y$) are summed to obtain the overall HPP output $P_l$. $\mathbf{C}$ and $\mathbf{G}$ respectively represent controller and system dynamics, for each plant component.}
    \label{fig:hsc_control_diag}
\end{figure}

\section{Methodology}
\label{sec:method}
% We present the details about the simulation environment, dataset and the predictive controller design for the hybrid energy power plant.
\noindent \textbf{Hybrid power plant:} This work considers an HPP that comprises a wind farm, solar farm, and battery as an energy storage device. Figure \ref{fig:hsc_control_diag} shows the \textit{supervisory} control scheme for HPP control used in this work. Setpoints are denoted by $u$, and measured outputs are denoted by $y$. The HPP supervisory controller ($HSC$) selects optimal set-points for tracking an electricity load power reference $P_r$. The subscripts $w$, $s$, and $b$, denote the relation to the wind farm, solar farm, and battery, respectively. The wind and solar farms generate power outputs, $P^{y}_w$ and $P^{y}_s$, respectively. Excess power produced (i.e., when $P^{y}_w + P^{y}_s > P_r$), is used to charge the battery ($P^{u}_b<0$). In the event of a deficit (i.e., when $P^{y}_w + P^{y}_s < P_r$), the battery is discharged (i.e., $P^{u}_b>0$) to support load-tracking. Each plant energy generation component (solar PV, turbines) comprises a dedicated and well-tuned proportional-integral-derivative (PID) controller, $\mathbf{C}$, for tracking the respective setpoints - $P^{u}_w$, $P^{u}_s$ and $P^{u}_b$. The wind farm, solar farm, and battery, for tracking the reference load, are approximately modeled as first-order systems, $\mathbf{G}$. Note that $P^y_w$ and $P^y_s$ are limited by the wind and solar energy available, thus making weather forecasts and the associated uncertainties in available wind and solar energy critical to HPP performance.

\noindent \textbf{Subspace predictive control:} 
The data-driven predictive controller introduced in this work is based on the method of \textit{subspace predictive control} \cite{favoreel1999spc}. Subspace predictive control (SPC) is a parameter-free technique that identifies a multi-step prediction model of a system using only previously measured input-output data and uses this model to formulate an optimal control problem.

Consider the measured input and output data, $U$ and $Y$, respectively, to comprise $T$ trajectories of a multi-input and multi-output dynamic system. Each trajectory is $L=T_{ini}+N$ samples long. $N$ is the length of the prediction horizon of the controller. $T_{ini}$ is the number of the most recent measurements of $U$ and $Y$ of relevance. Let $u$ $\epsilon$ $\mathbb{R}^{m}$ and $y$ $\epsilon$ $\mathbb{R}^{p}$. Therefore, $U$ and $Y$ can be expressed as $U=[u^1_L~u^2_L~...~u^{T}_L]$ and $Y=[y^1_L~y^2_L~...~y^{T}_L]$, where $u^j_L=[(u^j_1)^{\intercal}~(u^j_2)^{\intercal}~...~(u^j_L)^{\intercal}]^{\intercal}$  $\epsilon$ $\mathbb{R}^{Lm}$ and $y^j_L=[(y^j_1)^{\intercal}~(y^j_2)^{\intercal}~...~(y^j_L)^{\intercal}]^{\intercal}$ $\epsilon$ $\mathbb{R}^{Lp}$. Fulfilling Assumptions 1, 2, and 3, discussed in \cite{fiedler2021relationship}, ensures that the values of $T_{ini}$, $N$ and $T$, satisfy observability requirements for data-driven predictive control, and make $U$ persistently exciting. Splitting each of $U$ and $Y$ as $U=[~U_{T_{ini}}^{\intercal}~~~U_N^{\intercal}~]^{\intercal}$ and $U=[~Y_{T_{ini}}^{\intercal}~~~Y_N^{\intercal}~]^{\intercal}$, a matrix $S^{*}$ can be calculated as

\begin{equation}
    S^{*} = \argmin_{S} ||S \underbrace{\begin{bmatrix}
       Y_{T_{ini}} \\ U_{T_{ini}} \\ U_N
    \end{bmatrix}}_M - Y_N||^2_F, 
    \label{eq:Sstar}
\end{equation}
to give $S^{*}=Y_NM^{\dagger}$, where $\dagger$ denotes a Moore-Penrose pseudo-inverse. Here, $U_{T_{ini}}$ $\epsilon$ $\mathbb{R}^{T_{ini}m \times T}$ and $Y_{T_{ini}}$ $\epsilon$ $\mathbb{R}^{T_{ini}p \times T}$. Additionally, $U_N$ $\epsilon$ $\mathbb{R}^{Nm \times T}$ and $Y_{N}$ $\epsilon$ $\mathbb{R}^{Np \times T}$.

If $U$ and $Y$ are sufficiently rich to cover diverse trajectories or \textit{behaviors} of the HPP system, the matrix $S^{*}$ can be used to represent any single input-output trajectory $y_{T_{ini}}$, $y_N$, $u_{T_{ini}}$, $u_N$, as
\begin{equation}
    y_N=S^{*} \begin{bmatrix}
       y_{T_{ini}} \\ u_{T_{ini}} \\ u_N
    \end{bmatrix},
    \label{eq:y_N}
\end{equation}
where $y_{T_{ini}}$ $\epsilon$ $\mathbb{R}^{T_{ini}p}$, $u_{T_{ini}}$ $\epsilon$ $\mathbb{R}^{T_{ini}m}$, $u_{N}$ $\epsilon$ $\mathbb{R}^{mN}$, and $y_{N}$ $\epsilon$ $\mathbb{R}^{pN}$. Therefore, given the past $T_{ini}$ sequence of input-output measurement data, \eqref{eq:Sstar} provides a relation between inputs and output over a future horizon of $N$ samples. Since $S^{*}$ is calculated using diverse trajectories of the HPP system, the trajectory of ($u_N$, $y_N$) is expected to be admissible by the HPP dynamics.
\begin{figure}[tb]
    \centering
\includegraphics[width=\linewidth]{Figures/Test_load.pdf}
    \caption{A sample load-tracking result. A constrained feedback-optimization based controller generates input-output data for multiple load references $P_r$ \cite{ortmann2022online}.}
    \label{fig:Test_load}
\end{figure}

\noindent \textbf{Data curation:} In the simulation study presented in this work, the HPP is considered to serve a region with electricity demands equivalent to the combined day-long power profiles of the Hornsea 1 wind farm, and the solar farms of PES Region 10 in East England \cite{heftdata}. 

The input-output data required to determine $S^{*}$ in \eqref{eq:Sstar} is obtained by tracking day-long demand profiles of multiple days, spread over a year. For data collection, a constrained feedback-optimization (FO) based \textit{reactive} controller is utilized as the $HSC$ of the HPP in Figure \ref{fig:hsc_control_diag}. The FO uses a single output measurement (i.e., $y$) of the most recently sampled interval to optimally select setpoints (i.e., $u$) for the next control action \cite{ortmann2022online}. Perfect setpoint-tracking by the inner control-loops of the HPP, and completely certain weather forecasts, are assumed. A noise-to-signal ratio of 0.02 is considered while recording $y$ and $u$.  Figure \ref{fig:Test_load} shows a sample result for tracking an electricity load-reference $P_r$ using the FO controller. Here, the power outputs of the wind farm (i.e., $P^y_w$) and solar farm (i.e., $P^y_s$) are upper bounded by the completely certain and maximum available wind (i.e., $P^{\max}_w$) and solar energy (i.e., $P^{\max}_s$).

\noindent \textbf{Uncertainty-aware predictive controller:} This work introduces an SPC-based data-driven predictive controller for supervisory control of an HPP (see figure \ref{fig:hsc_control_diag}), and analyzes its decision-making in the presence of uncertain weather conditions, particularly wind speed and the wind energy available to the wind farm. The controller is framed as follows.
\begin{eqnarray}
\nonumber && \argmin_{u_N, y_N, \sigma_u, \sigma_y} \sum_{k=1}^{N} \left\{\frac{1}{2}Q_r{J_T^k}^2 + \lambda {\Delta P^{y,k}_w}^2 + \lambda {\Delta P^{u,k}_w}^2 \right\} \\ && + 
 \nonumber \lambda_u||\bf{\sigma_u}||^2 + \lambda_y||\bf{\sigma_y}||^2 \\
&&\text{s.t.} ~ J^k_T = P^{k}_r-P^{y,k}_w-P^{y,k}_s \label{eq:J}\\ &&
~~~~~~y_N = S^*\begin{bmatrix}
    y_{T_{ini}} \\ u_{T_{ini}} \\ u_N \\
\end{bmatrix} + \begin{bmatrix}
    \bf{\sigma_y} \\ \bf{\sigma_u} \\ \bf{0}
\end{bmatrix} \label{eq:FL} \\&&
~~~P^{u,k}_b=P^{k}_r-P^{y,k}_w-P^{y,k}_s \label{eq:Pbd}\\&&
~~~0 \le P^{u,k}_w, P^{y,k}_w \le q_{P^{\max}_{w}} \label{ineq:Pwmxmn}\\&&
~~~0 \le P^{u,k}_s, P^{y,k}_s \le P^{\max}_{s} \label{ineq:Psmxmn}\\&& \label{ineq:Pbmxmn}
~~~P^{\min}_{b} \le P^{u,k}_b, P^{y,k}_b \le P^{\max}_{b}.
\end{eqnarray}
Here, $u_{N}$ is a solution trajectory of setpoints sent by the $HSC$ to the HPP components. $y_{N}$ is the prediction of HPP outputs over the horizon of $N$ samples. $Q_r$ is a penalty on the sample-wise tracking cost $J^{k}_T$ in \eqref{eq:J}. $\lambda$ penalizes large changes in setpoint and power output of the wind farm. \eqref{eq:FL} constrains $u_{N}$ and $y_{N}$ to within the admissible system trajectories, as shown in \eqref{eq:y_N}. $\bf{\sigma_u}$ $\epsilon$ $\mathbb{R}^{T_{ini}m}$ and $\bf{\sigma_y}$ $\epsilon$ $\mathbb{R}^{T_{ini}p}$ are heavily penalized relaxations added in \eqref{eq:FL} to add robustness against measurement noise \cite{coulson2019data}. In \eqref{eq:Pbd}, the setpoint for the battery is constrained to autonomously charge the excess ($P^{u,k}_b<0$), or discharge to support a deficit ($P^{u,k}_b>0$), in the combined output of the wind and solar farms. Constraints \eqref{ineq:Pwmxmn}-\eqref{ineq:Pbmxmn} set bounds on the admissible setpoints to, and power outputs of, the HPP components. $q_{P^{\max}_w}$ is an upper bound for the available wind power in the presence of uncertain forecasts for wind speed. The uncertainty is assumed to have a normal distribution and $q_{{P^{\max}_w}}$ is set using a suitable quantile value. Therefore, \eqref{ineq:Pwmxmn} is an individual probabilistic constraint and adds uncertainty-awareness to the predictive controller. In this work, $N=20$, $T_{ini}=20$, $L=40$, and $T=1000$. Additionally, $p=3$ and $m=3$.

\noindent \textbf{Simulation setup:} The HPP in Figure \ref{fig:hsc_control_diag} is simulated using the uncertainty-aware data-driven predictive controller as the $HSC$. The wind and solar farms, and the battery, have a rated capacity of 4 MW each. The battery is assumed to have a sufficient state-of-charge at all times and its cycling limits are not considered. The electricity demand profiles in \cite{heftdata} are linearly scaled to the order of five megawatts to be consistent with the rated output of the HPP. The available wind power (i.e., $P^{\max}_w$) and solar power (i.e., $P^{\max}_s$) are functions of the wind speed and solar irradiance, respectively (see Figure \ref{fig:power_curves}). For this study, their profiles are assumed sinusoidal and are generated synthetically. To test the uncertainty-awareness of the predictive controller, a normally distributed uncertainty in wind speed, of mean 0 m/s and standard deviation 0.1 m/s, is applied. Therefore, $q_{P^{\max}_w}=\mu_{P^{\max}_w}+q_w\sigma_{P^{\max}_w}$ in \eqref{ineq:Pwmxmn}. Here, the quantile $q_w=-0.4$, and $\mu_{P^{\max}_w}$ and $\sigma_{P^{\max}_w}$ are mean and standard deviation of the uncertain wind power profile. Figure \ref{fig:Test_load} shows the load-profile $P_r$ used as the reference to the predictive controller. 

The optimization problem in \eqref{eq:J}-\eqref{ineq:Pbmxmn} is solved using the NLP solver from CasADi on a laptop with an 11th Gen Intel(R) Core(TM) i7-1185G7 3.00GHz processor and 16GB RAM. The mean solve time was 0.3 s, much lower than the sampling interval of 20 s of the predictive controller. Lastly, given this sampling interval, the predictive controller solves for the next 20$N$ seconds using measurement data from the past 20 $T_{ini}$ seconds.
\begin{figure}[tb]
    \centering
\includegraphics[width=\linewidth]{Figures/weather.pdf}
    \caption{Day-long forecasts for available wind and solar power to be used in the simulation study in Section \ref{sec:results}. Uncertainty added to the wind-speed translates to uncertainty in available wind power ($P^{\max}_{w,un}$).}
    \label{fig:power_curves}
\end{figure}
\begin{figure}[tb]
    \centering
\includegraphics[width=\linewidth]{Figures/SPC_OL_un_wind.pdf}
    \caption{Open-loop uncertainty-aware setpoint selection and tracking for the wind farm. $P^{y,ol}_w$ is the \textit{open-loop} response of the wind farm to setpoints $P^{u,SPC}_w$ of the predictive controller, in the presence of uncertainty in available wind power $P^{max}_{w,un}$. The error of the predicted output $P^{y,SPC}_w$  with respect to $P^{y,ol}_w$, normalized by the mean of $P^{y,SPC}_w$, is 6.5\%.}
    \label{fig:SPC_OL_un_wind}
\end{figure}
\begin{figure}[tb]
    \centering
\includegraphics[width=\linewidth]{Figures/SPC_OL_un_Solar_Batt.pdf}
    \caption{Open-loop setpoint selection and tracking for the solar farm and battery. $P^{y,ol}_s$ and $P^{y,ol}_b$ are respective \textit{open-loop} responses of the solar farm  and battery to setpoints $P^{u,SPC}_s$ and $P^{u,SPC}_b$ of the predictive controller. The error of the predicted solar farm output $P^{y,SPC}_s$  with respect to $P^{y,ol}_s$, normalized by the mean of $P^{y,SPC}_s$, is 8.5\%. For the battery, this value is 10\%.}
    \label{fig:SPC_OL_un_Solar_Batt}
\end{figure}
\begin{figure}[tb]
    \centering
\includegraphics[width=\linewidth]{Figures/SPC_OL_un_load.pdf}
    \caption{Open-loop uncertainty-aware load tracking for the hybrid power plant in Figure \ref{fig:hsc_control_diag}. $P^{ol}_l$ is the \textit{open-loop} output of the hybrid power plant in response to setpoints of the predictive controller, in the presence of uncertainty in available wind power. The error of the predicted HPP output, $P^{SPC}_l=P^{y,SPC}_w+P^{y,SPC}_s+P^{y,SPC}_b$, with respect to $P^{ol}_l$, and normalized by the mean of $P_r$ over the prediction horizon, is 6.1\%. The combined setpoints to the wind and solar farm, $P^{u,ol}_{w+s}$, obey the probabilistic constraint in \eqref{ineq:Pwmxmn} to remain below the available wind and solar energy, $P^{\max}_{w,un+s}$. }
    \label{fig:SPC_OL_un_load}
\end{figure}
% \begin{figure}[tb]
%     \centering
% \includegraphics[width=\linewidth]{Figures/SPC_CL_un_load.pdf}
%     \caption{Uncertainty-aware closed-loop data-driven predictive control of the hybrid power plant in Figure \ref{fig:hsc_control_diag}. The plant output $P_l$ tracks the load reference $P_r$ well. Once the wind and solar power output saturate ($P^{y,SPC,cl}_w \approx P^{max}_{w,un}$ and $P^{y,SPC,cl}_s \approx P^{max}_s$), the battery begins to autonomously discharge ($P^{y,SPC,cl}_b>0$) to support load tacking.}
%     \label{fig:SPC_CL_un_load}
% \end{figure}
% \begin{figure}[tb]
%     \centering
% \includegraphics[width=\linewidth]{Figures/SPC_CL_wind_fore.pdf}
%     \caption{The solution of the uncertainty-aware data-driven predictive controller can be treated as a \textit{forecast} or expected trajectory of the power plant output over a future horizon. The uncertain and available wind power $P^{max}_{w,un}$ is dotted at specific sample times. At these sample times, solid lines represent the prediction of future setpoints, in (a), and wind farm output, in (b). The underlying dashed lines in black represent the setpoints and output obtained in closed-loop running of the wind farm.}
%     \label{fig:SPC_CL_fore}
% \end{figure}
% \begin{figure}[tb]
%     \centering
% \includegraphics[width=\linewidth]{Figures/SPC_CL_bat_load_fore.pdf}
%     \caption{(a) Each solid line is a forecast of the battery output over the prediction horizon of the predictive controller. The dashed cyan line represents the excess ($\Delta P_l>0$) or deficit ($\Delta P_l<0$) in power production of the wind and solar farms, relative to the load reference. In the event of a deficit, the battery must discharge ($P^{y}_b>0$) to support the load. As seen at $t \approx 6.2$ hours, the battery setpoints in solid orange \textit{forecasts} the need to discharge and support the load, and turns positive toward its tail. This demonstrates the capability of the predictive controller to autonomously drive the state of the battery despite the presence of uncertainty in available wind power. (b) Each solid line represents a forecast of the plant output given by the predicted controller. This is obtained by summing the predicted outputs, $P^{y,SPC}_{w}$, $P^{y,SPC}_{s}$, and $P^{y,SPC}_{b}$. Given that the forecasts match the true output of the power plant (in dashed black), the predictive controller is shown to be uncertainty-aware and capable in providing useful forecasts of plant performance.}
%     \label{fig:SPC_CL_bat_plant_fore}
% \end{figure}
% \begin{table}[tb]
%     \caption{Impact of uncertainty, weights, and quantiles, on the relaxations $\bf{\sigma_y}$ and $\bf{\sigma_u}$ of the behavior constraint in \eqref{eq:FL}. $||\cdot||_F$ is the Frobenius norm.$|\Delta P^{y}_w|$ reflects the change in solution of the predictive controller, for wind power output $P^{y,SPC}_w$, from a baseline scenario of no uncertainty and $\lambda_u$=$\lambda_y$=10. The prediction horizon considered is $N$=20 samples long and begins at 11.88 hours for the load profile $P_r$ shown in Figure \ref{fig:Test_load}.}
%     \centering
%     \begin{tabular}{l l l l}
%         \toprule
%         Test & $||\bf{\sigma_u}||_F$ & $||\bf{\sigma_y}||_F$ & $|\Delta P^{y}_w|$\\ 
%         \hline
%         No uncertainty & 8 kW & 5 kW & 0 kW\\
%         Uncertainty, ${\lambda_u}$=${\lambda_y}$= 10 & 38 kW & 17 kW & 115 kW\\
%         Uncertainty, ${\lambda_u}$=${\lambda_y}$= 10$^5$ & 1 kW & 0.8 kW & 118 kW \\
%         Uncertainty, $q_w$ = -0.4 & 38 kW & 17 kW & 115 kW \\
%         Uncertainty, $q_w$ = -1.6 & 152 kW & 66 kW & 660 kW\\
%          \bottomrule
%     \end{tabular}
%     \label{tab:relaxation_norm}
% \end{table}

%Simulations
\section{Results}
\label{sec:results}

\noindent \textbf{Open-loop control:} The potential of the uncertainty-aware data-driven predictive controller is tested by first applying its chosen setpoints, $P^{u,SPC}_w$, $P^{u,SPC}_s$, and $P^{u,SPC}_b$, to the HPP of Figure \ref{fig:hsc_control_diag}, in an open-loop fashion, over the prediction horizon of $N$ samples. Therefore, in open-loop control, the entire sequence of setpoints $u_N$, solved by the predictive controller in \eqref{eq:J}-\eqref{ineq:Pbmxmn}, is applied to the HPP without any feedback of the measured HPP output to the controller for a revised calculation of $u_N$. The HPP output obtained is the open-loop response. The simulation is conducted over $N=20$ samples and with the initial sample at 11.88 hours. At this time of the day, Figure \ref{fig:Test_load} shows maximum utilization of wind and solar energy, albeit in the absence of uncertainty. Therefore, such an initial condition can be used to test the uncertainty-awareness of the predictive controller since $P^{u,k}_w$ and $P^{y,k}_w$ are expected to approach an uncertain $P^{\max}_w$. 

Figure \ref{fig:SPC_OL_un_wind} plots the open-loop response of the wind farm. The setpoints $P^{u,SPC}_w$ remain below the uncertain wind power $P^{\max}_{w,un}$, demonstrating the uncertainty-awareness of the predictive controller. The error of the predicted wind farm output $P^{y,SPC}_w$  with respect to the open-loop power output $P^{y,ol}_w$, normalized by the mean of $P^{y,SPC}_w$, is 6.5\%. $P^{y,ol}_w$ follows $P^{u,SPC}_w$ reasonably well given that the inner control-loop for the wind farm is well-tuned. The deviation of $P^{y,SPC}_w$ from $P^{u,SPC}_w$ can be attributed to noise in the data collected to calculate $S^*$ in \eqref{eq:Sstar}, the relaxation terms in \eqref{eq:FL}, and the input-output dynamics of the HPP captured in $S^{*}$.  

Figure \ref{fig:SPC_OL_un_Solar_Batt} shows the open-loop response of the solar farm, $P^{y,ol}_s$, and battery, $P^{y,ol}_b$, to the setpoints $P^{u,SPC}_s$ and $P^{u,SPC}_b$, respectively. The error of the predicted solar farm output $P^{y,SPC}_s$  with respect to $P^{y,ol}_s$, normalized by the mean of $P^{y,SPC}_s$, is 8.5\%. For the battery, this value is 10\%.

Figure \ref{fig:SPC_OL_un_load} plots the predicted total power output of the HPP, $P^{SPC}_l=P^{y,SPC}_w+P^{y,SPC}_s+P^{y,SPC}_b$, over the prediction horizon of $N$ samples. The error of $P^{SPC}_l$ with respect to $P^{ol}_l$, normalized by the mean of $P_r$ over the prediction horizon, is 6.1\%. The uncertainty in the maximum available wind power, $P^{\max}_{w,un}$, leads to non-smooth variation in $P^{SPC}_l$ and the total available wind and solar energy $P^{SPC}_{w,un + s}$. The total power setpoint to the wind and solar farms in open-loop is $P^{u,ol}_{w+s}=P^{u,SPC}_w+P^{u,SPC}_s$. Its increasing deviation from $P^{\max}_{w,un + s}$ is undesired since it is beneficial to utilize the available and renewable wind and solar energy before discharging the battery to support the tracking of $P_r$. Given the reasonably low normalized errors in the predictions of the data-driven controller for the open-loop outputs of the power plant components, the predictive controller demonstrated potential in providing its solutions as \textit{forecasts} of HPP power generation.
\begin{figure}[tb]
    \centering
\includegraphics[width=\linewidth]{Figures/SPC_CL_un_load.pdf}
    \caption{Uncertainty-aware closed-loop data-driven predictive control of the hybrid power plant in Figure \ref{fig:hsc_control_diag}. The plant output $P_l$ tracks the load reference $P_r$ well. The predictive controller anticipates $P_r$  to exceed the maximum available wind and solar energy $P^{\max}_{w,un +s}$ and discharges the battery ($P^{y,SPC,cl}_b>0$) to support load tracking. Maximum available solar energy is utilized ($P^{y,SPC,cl}_s \approx P^{\max}_{s}$) and the wind farm output $P^{y,SPC,cl}_{w}$ is limited by the uncertain available wind power $P^{\max}_{w,un}$.}
    \label{fig:SPC_CL_un_load}
\end{figure}
\begin{figure}[tb]
    \centering
\includegraphics[width=\linewidth]{Figures/SPC_CL_wind_fore.pdf}
    \caption{The solution of the uncertainty-aware data-driven predictive controller can be treated as a \textit{forecast} or expected trajectory of the power plant output over a future horizon. The uncertain and available wind power $P^{max}_{w,un}$ is dotted at specific sample times. At these sample times, solid lines represent the prediction of future setpoints, in (a), and wind farm output, in (b). The underlying dashed lines in black represent the setpoints and true output obtained in simulated closed-loop control of the hybrid power plant.}
    \label{fig:SPC_CL_fore}
\end{figure}
\begin{figure}[tb]
    \centering
\includegraphics[width=\linewidth]{Figures/SPC_CL_bat_load_fore.pdf}
    \caption{(a) Each solid line is a forecast of the setpoint to the battery, over the prediction horizon of the predictive controller. $\Delta P_l>0$ is the excess ($\Delta P_l>0$) or deficit ($\Delta P_l<0$) in the combined power production of the wind and solar farms, relative to the load reference. In the event of a deficit, the battery must discharge ($P^{y}_b>0$) to support the load. At $\approx$ 6.2 hours, the battery setpoints in solid orange \textit{forecast} the need to discharge and support the load, and turn positive toward the tail of the forecast. This demonstrates the capability of the predictive controller to anticipate and control the state of the battery, despite the presence of uncertainty in the available wind power. (b) Each solid line represents a forecast of the total output of the hybrid power plant given by the predictive controller. This output is obtained by summing the predicted outputs, $P^{y,SPC}_{w}$, $P^{y,SPC}_{s}$, and $P^{y,SPC}_{b}$. Since the true output of the power plant (in dashed black) closely follows the forecasts, the predictive controller is shown to be uncertainty-aware and capable of providing useful forecasts of power generation.}
    \label{fig:SPC_CL_bat_plant_fore}
\end{figure}

\noindent \textbf{Closed-loop control:} Next, the predictive controller is tested in a closed-loop fashion. In closed-loop control, the setpoints selected by the predictive controller are applied to the HPP in a receding horizon fashion. The first setpoint of the solution $u_N$, $u_N^1=\{P^{u,1}_w, P^{u,1}_s, P^{u,1}_b\}$ is applied and the measurement of the resulting output $y=\{P^{y}_w, P^{y}_s, P^{y}_b\}$ is fedback to the controller to solve the optimization again. Noise to signal ratio of zero is considered to isolate the response of the controller to uncertainty in the available wind power.

Figure \ref{fig:SPC_CL_un_load} shows the effective closed-loop tracking of the load reference $P_r$ by the uncertainty-aware predictive controller between 4am and 8am for the test data in Figure \ref{fig:Test_load}. The load reference $P_r$ is tracked well by the HPP load output $P^{cl}_l$. The predictive controller anticipates $P_r$  to exceed the maximum available wind and solar energy $P^{\max}_{w,un +s}$ and discharges the battery ($P^{y,SPC,cl}_b>0$) to support load tracking. Maximum available solar energy is utilized ($P^{y,SPC,cl}_s \approx P^{\max}_{s}$) and the wind farm output $P^{y,SPC,cl}_{w}$ is limited by the uncertain available wind power $P^{\max}_{w,un}$. 

\noindent \textbf{Forecasting power plant output:} Figure \ref{fig:SPC_CL_fore} provides the $N=20$ samples long solution of the predictive controller at evenly spaced sampling intervals between 6am and 7am for the test data in Figure \ref{fig:Test_load}. At each sampling interval, the corresponding (uncertain) upper bound on available wind power, $P^{\max}_{w,un}$, is plotted as a cyan dot. Given that the predictive controller showed potential to predict the open-loop performance of the HPP, its solutions can be explored as \textit{forecasts} for the HPP outputs over a future horizon. 

Figure \ref{fig:SPC_CL_fore} (a) provides forecasts of the setpoints to the wind farm, $P^{u,SPC,cl}_w$, each plotted as a solid line of different color. Figure \ref{fig:SPC_CL_fore} (b) provides forecasts of the power output of the wind farm, $P^{y,SPC,cl}_w$, each plotted as a solid line of different color. All forecasts obey the probabilistic constraint in \eqref{ineq:Pwmxmn} and remain below $P^{\max}_{w,un}$. The underlying black dashed line in the respective plots represents the setpoint and true output recorded in the simulated closed-loop control of the HPP.

Figure \ref{fig:SPC_CL_bat_plant_fore} (a) provides forecasts of the setpoints to the battery, $P^{u,SPC,cl}_b$, each plotted as a solid line of different color. The underlying dashed line in black represents the setpoint utilized in closed-loop control of the HPP. The dashed cyan line represents the excess ($\Delta P_l>0$) or deficit ($\Delta P_l<0$) in power production of the wind and solar farms, relative to the load reference $P_r$. In the event of a deficit, the battery must discharge ($P^{y}_b>0$) to support the load. As seen at $\approx$ 6.2 hours, the battery setpoints in solid orange \textit{forecasts} the need to discharge and support the load, and turns positive toward its tail. This demonstrates the capability of the predictive controller to anticipate and control the state of the battery, despite the presence of uncertainty in the available wind power, in the event of a peak demand from the electricity power grid. Such a capability can provide data, on the future participation of the battery, to the manager of an HPP.

Figure \ref{fig:SPC_CL_bat_plant_fore} (b) provides a forecast of the total output of the HPP with respect to the load reference $P_r$. Each forecast is a solid line of a different color and is obtained by summing the predicted outputs, $P^{y,SPC}_{w}$, $P^{y,SPC}_{s}$, and $P^{y,SPC}_{b}$, over the prediction horizon. The underlying dashed line in black is the true total output obtained in the simulated closed-loop control of the HPP. Given that the true output of the power plant (in dashed black) closely follows the forecasts, the predictive controller is shown to be uncertainty-aware and capable of providing useful forecasts of power generation. This capability can give a manager of an HPP data on the anticipated behavior and output of the plant and assist in HPP management.  
%Point to Table \ref{tab:relaxation_norm}.
\begin{table}[tb]
    \caption{Impact of uncertainty, weights, and quantiles, on the relaxations $\bf{\sigma_y}$ and $\bf{\sigma_u}$ of the behavior constraint in \eqref{eq:FL}. $||\cdot||_F$ is the Frobenius norm.$|\Delta P^{y}_w|$ reflects the change in solution of the predictive controller, for wind power output $P^{y,SPC}_w$, from a baseline scenario of no uncertainty and $\lambda_u$=$\lambda_y$=10. The prediction horizon considered is $N$=20 samples long and begins at 11.88 hours for the load profile $P_r$ shown in Figure \ref{fig:Test_load}.}
    \centering
    \begin{tabular}{l l l l}
        \toprule
        Test & $||\bf{\sigma_u}||_F$ & $||\bf{\sigma_y}||_F$ & $|\Delta P^{y}_w|$\\ 
        \hline
        No uncertainty & 8 kW & 5 kW & 0 kW\\
        Uncertainty, ${\lambda_u}$=${\lambda_y}$= 10 & 38 kW & 17 kW & 115 kW\\
        Uncertainty, ${\lambda_u}$=${\lambda_y}$= 10$^5$ & 1 kW & 0.8 kW & 118 kW \\
        Uncertainty, $q_w$ = -0.4 & 38 kW & 17 kW & 115 kW \\
        Uncertainty, $q_w$ = -1.6 & 152 kW & 66 kW & 660 kW\\
         \bottomrule
    \end{tabular}
    \label{tab:relaxation_norm}
\end{table}
\section{Discussions}
\label{sec:disc}
The previous section showed that the data-driven predictive controller has potential for uncertainty-aware closed loop control of a hybrid power plant. Additionally, in the form of a supervisor, i.e., the $HSC$ in Figure \ref{fig:hsc_control_diag}, the controller is shown to be capable of intelligent decision-making for a) optimal setpoint selection for each HPP component, and b) anticipation and control of the state of battery charge/discharge, despite the presence of uncertainty. Next, we discuss a few research inquiries to retain this predictive controller intelligence, and making the decisions and forecasts trustworthy and explainable.
\begin{enumerate}
    
    \item \textit{How good is my data?}

    The uncertainty-aware predictive controller presented in this work is completely data-driven. Its ability to constrain permissible future trajectories of the system in \eqref{eq:FL} relies on the quality of the HPP input-output measurement data used to define the matrix $S^{*}$ in \eqref{eq:Sstar}. Therefore, it is necessary to develop a quantitative and/or qualitative metric of how well the data $U$ and $Y$ represent the dynamics of the HPP. If this is not the case, the most recent input-output measurements, $u_{T_{ini}}$ and $y_{T_{ini}}$, used in \eqref{eq:FL}, will represent system dynamics different from that encoded by $U$, $Y$, and $S^{*}$. This can lead to

    \begin{enumerate}
        \item     large values of relaxations $\bf{\sigma_u}$ and $\bf{\sigma_y}$ in \eqref{eq:FL} to maintain a feasible optimization, and

        \item incorrect setpoint selection (i.e, decision making) and forecasts of HPP power output.
    \end{enumerate}   

A data-quality metric for noise-corrupted data and data prone to privacy related obfuscation is introduced in \cite{mieth2024data}, while a min-max data-driven optimization technique is introduced in \cite{huang2023robust} for distributional robustness against uncertain data. However, the predictive controller introduced in this work is expected to run in-house, making all required measurement data available and free of obfuscation. The question asked above seeks to determine a metric that provides analytical insight on the appropriateness of the data for predictive control and inform the need to update the data collected, and $S^{*}$.

    \item \textit{Is my data-driven optimization biased?}

    The potential of the data-driven predictive controller to serve as a supervisor for HPP control can lead to an evolution of the objective function and constraints utilized in \eqref{eq:J}-\eqref{ineq:Pbmxmn}. The design of the objective function, constraints, and the tuning of penalties such as $Q_r$, $\lambda_u$, and $\lambda_y$, affect the admissible system outputs, $y_N$, and setpoints, $u_N$, solved by the predictive controller over a prediction horizon. If the gains are tuned too strongly to optimize the objective function or the objective demands an uncharacteristic system trajectory, the constraint in \eqref{eq:FL} may be relaxed excessively to accommodate the solution ($u_N$, $y_N$). This trajectory could be biased toward the objective and may not be representable by $U$, $Y$, and $S^{*}$, in \eqref{eq:Sstar}. An example of how optimization may be optimistically biased is provided in \cite{hobbs1989optimization}.

    \item \textit{Does constraint relaxation of the admissible system trajectories offer any clues?}

    Excessive relaxation of the behavior constraint in \eqref{eq:FL} can lead to increased values of $\bf{\sigma_u}$ and $\bf{\sigma_u}$ in the solution of the predictive controller. An increasing relaxation is a sign of the solution trajectory ($u_N$, $y_N$) being increasingly different than the system dynamics encoded in $U$, $Y$, and $S^{*}$, in \eqref{eq:Sstar}. This may happen due to unaccounted uncertainty in the behavior data (old input-output measurement data of the HPP), incorrectly tuned probabilistic constraints (weather forecasts of high-variance), and improper tuning of the penalties in the objective function.    

    For instance, Table \ref{tab:relaxation_norm} provides the Frobenius norm of the relaxations, $\bf{\sigma_u}$ and $\bf{\sigma_y}$, under different penalties, quantiles $q_w$, and uncertainty. A baseline scenario of no uncertainty (wind speed and available wind power are completely certain) is considered, with $\bf{\sigma_u}$ and $\bf{\sigma_y}$ penalized with $\lambda_y$=$\lambda_u$=10. For a given set of ($\lambda_y$, $\lambda_u$, $q_w$), $|\Delta P^{y}_w|$ reflects the change in solution of the predictive controller, for wind power output $P^{y,SPC}_w$, with respect to the baseline scenario. The prediction horizon considered is $N$=20 samples long and begins at 11.88 hours for the load profile $P_r$ shown in Figure \ref{fig:Test_load}. 
    
    Overall, the presence of uncertainty in the weather forecasts (i.e., wind speed in this study) leads to an increased relaxation of the order of a 100 kW. This can be attributed to the following. The wind farm output is bounded by the probabilistic constraint in \eqref{ineq:Pwmxmn} to remain below the quantile $q_w$ of the uncertain and maximum available wind power, as shown in Figure \ref{fig:SPC_CL_un_load}. The resulting conservative but uncertainty-aware production of wind power differs in behavior from the uncertainty-free input-output measurement data, $U$ and $Y$, collected using feedback-optimization in Figure \ref{fig:Test_load}. Thus, the presence of a probabilistic constraint in \eqref{ineq:Pwmxmn}, to provide uncertainty-awareness, necessitates an increased relaxation of the constraint in \eqref{eq:FL}.
    Applying high penalties and setting $\lambda_y$=$\lambda_u$=10$^5$, lowers relaxation of the behavior constraint. However, this comes at a cost of increased change in solution for the outputs of the solar farm and battery (not reported here for sake of brevity). Lastly, using a more conservative quantile value of $q_w=-1.6$ to set a lower bound on the available wind power in \eqref{ineq:Pwmxmn}, leads to a large change in the trajectory of the wind power output, leading to excessive relaxation of the constraint in \eqref{eq:FL}.

    Therefore, developing a threshold for admissible relaxations can provide insight into a change in underlying dynamics of the HPP. Additionally, such a threshold can point toward a need for improved tuning of penalties in the objective function, and the quantiles for the probabilistic constraints.

    A starting point to determine such a threshold could be determining the relaxation required in the behavior constraint in \eqref{eq:FL} to represent the most recent HPP trajectory, ($u_{T_{ini}}$, $y_{T_{ini}}$). The sequence of measurements ($u_{T_{ini}}$, $y_{T_{ini}}$) will encode the up-to-date input-output dynamics of the HPP. Therefore, larger than expected relaxations, to describe ($u_{T_{ini}}$, $y_{T_{ini}}$), may point toward the need to examine the the optimization framework \eqref{eq:J}-\eqref{ineq:Pbmxmn} for lack of awareness of uncertainty in the data and probabilistic constraints. 

\end{enumerate}
\section{Conclusion}
\label{sec:conc}
This work proposed, and investigated the potential of, an uncertainty-aware data-driven predictive controller as an intelligent decision-maker and supervisor for a hybrid power plant (HPP). Intelligence of the controller is considered and demonstrated in the context of its ability to a) encode dynamics of an HPP through the mere use of available input-output measurement data, b) provide forecasts of the power output of each component of the HPP, and c) anticipate and control the state of charge/discharge of the energy storage device in the event of peak demands from the electricity power grid. The potential of the predictive controller was demonstrated using a simulation study that utilized real-world electricity demand profiles. The controller provided good closed-loop control, and forecasts of setpoints to and outputs of the HPP. Lastly, a few research inquiries to retain intelligence, and make the decisions and forecasts of the predictive controller trustworthy and explainable, are discussed. The next steps involve testing and analyzing the predictive controller's performance using real-world weather forecasts, higher-fidelity plant models, and joint probabilistic constraints to enhance uncertainty-awareness and evaluate its impact on solutions. 

The broader objective of this work is to explore and investigate the utility of the predictive controller as an assistant to the management of an HPP, not only in producing electricity, but also in its participating in electricity markets.

% \textcolor{red}{The immediate next steps in this work are to further test and analyze the performance of the predictive controller by a) using real-world weather forecasts, b) using higher-fidelity plant models for the wind farm, solar farm, and battery, and c) using joint probabilistic constraints for uncertainty-awareness and studying its impact on the solutions of the data-driven predictive controller.}



% This work introduced an uncertainty-aware data-driven predictive controller to serve as an intelligent supervisor of hybrid power plants. The hybrid power plant considered in this work comprises a wind farm, solar farm, and a battery as the energy storage device. This supervisor only uses input-output measurement data of these components, and imposes chance constraints on uncertain weather forecasts, to optimally select setpoints for each component, to track a electricity demand profile generated from real-world data.

% The solutions of the predictive controller were shown to be sufficiently good to be treated as forecasts for the power plant output and state of battery (charging/discharging mode). These forecasts can be used by power plant owners to anticipate power plant performance, make better bids in the electricity market, and better prepare energy storage devices during peak demands in electricity.

% Finally, questions open to investigating the convergence and robustness of this predictive control were stated. Given that control theory provides methodologies to conduct such investigations, there lies an opportunity to search for adding adaptive resiliency, robustness, and safety, into data-driven schemes. This can lead to an \textit{intelligent} data-driven assistant with control actions and plans may be \textit{explainable} through theoretical analysis.




\begin{comment}
\section{Preparing an Anonymous Submission}

This document details the formatting requirements for anonymous submissions. The requirements are the same as for camera ready papers but with a few notable differences:

\begin{itemize}
    \item Anonymous submissions must not include the author names and affiliations. Write ``Anonymous Submission'' as the ``sole author'' and leave the affiliations empty.
    \item The PDF document's metadata should be cleared with a metadata-cleaning tool before submitting it. This is to prevent leaked information from revealing your identity.
    \item References must be anonymized whenever the reader can infer that they are to the authors' previous work.
    \item AAAI's copyright notice should not be included as a footer in the first page.
    \item Only the PDF version is required at this stage. No source versions will be requested, nor any copyright transfer form.
\end{itemize}

You can remove the copyright notice and ensure that your names aren't shown by including \texttt{submission} option when loading the \texttt{aaai25} package:

\begin{quote}\begin{scriptsize}\begin{verbatim}
\documentclass[letterpaper]{article}
\usepackage[submission]{aaai25}
\end{verbatim}\end{scriptsize}\end{quote}

The remainder of this document are the original camera-
ready instructions. Any contradiction of the above points
ought to be ignored while preparing anonymous submis-
sions.

\section{Camera-Ready Guidelines}

Congratulations on having a paper selected for inclusion in an AAAI Press proceedings or technical report! This document details the requirements necessary to get your accepted paper published using PDF\LaTeX{}. If you are using Microsoft Word, instructions are provided in a different document. AAAI Press does not support any other formatting software.

The instructions herein are provided as a general guide for experienced \LaTeX{} users. If you do not know how to use \LaTeX{}, please obtain assistance locally. AAAI cannot provide you with support and the accompanying style files are \textbf{not} guaranteed to work. If the results you obtain are not in accordance with the specifications you received, you must correct your source file to achieve the correct result.

These instructions are generic. Consequently, they do not include specific dates, page charges, and so forth. Please consult your specific written conference instructions for details regarding your submission. Please review the entire document for specific instructions that might apply to your particular situation. All authors must comply with the following:

\begin{itemize}
\item You must use the 2025 AAAI Press \LaTeX{} style file and the aaai25.bst bibliography style files, which are located in the 2025 AAAI Author Kit (aaai25.sty, aaai25.bst).
\item You must complete, sign, and return by the deadline the AAAI copyright form (unless directed by AAAI Press to use the AAAI Distribution License instead).
\item You must read and format your paper source and PDF according to the formatting instructions for authors.
\item You must submit your electronic files and abstract using our electronic submission form \textbf{on time.}
\item You must pay any required page or formatting charges to AAAI Press so that they are received by the deadline.
\item You must check your paper before submitting it, ensuring that it compiles without error, and complies with the guidelines found in the AAAI Author Kit.
\end{itemize}

\section{Copyright}
All papers submitted for publication by AAAI Press must be accompanied by a valid signed copyright form. They must also contain the AAAI copyright notice at the bottom of the first page of the paper. There are no exceptions to these requirements. If you fail to provide us with a signed copyright form or disable the copyright notice, we will be unable to publish your paper. There are \textbf{no exceptions} to this policy. You will find a PDF version of the AAAI copyright form in the AAAI AuthorKit. Please see the specific instructions for your conference for submission details.

\section{Formatting Requirements in Brief}
We need source and PDF files that can be used in a variety of ways and can be output on a variety of devices. The design and appearance of the paper is strictly governed by the aaai style file (aaai25.sty).
\textbf{You must not make any changes to the aaai style file, nor use any commands, packages, style files, or macros within your own paper that alter that design, including, but not limited to spacing, floats, margins, fonts, font size, and appearance.} AAAI imposes requirements on your source and PDF files that must be followed. Most of these requirements are based on our efforts to standardize conference manuscript properties and layout. All papers submitted to AAAI for publication will be recompiled for standardization purposes. Consequently, every paper submission must comply with the following requirements:

\begin{itemize}
\item Your .tex file must compile in PDF\LaTeX{} --- (you may not include .ps or .eps figure files.)
\item All fonts must be embedded in the PDF file --- including your figures.
\item Modifications to the style file, whether directly or via commands in your document may not ever be made, most especially when made in an effort to avoid extra page charges or make your paper fit in a specific number of pages.
\item No type 3 fonts may be used (even in illustrations).
\item You may not alter the spacing above and below captions, figures, headings, and subheadings.
\item You may not alter the font sizes of text elements, footnotes, heading elements, captions, or title information (for references and mathematics, please see the limited exceptions provided herein).
\item You may not alter the line spacing of text.
\item Your title must follow Title Case capitalization rules (not sentence case).
\item \LaTeX{} documents must use the Times or Nimbus font package (you may not use Computer Modern for the text of your paper).
\item No \LaTeX{} 209 documents may be used or submitted.
\item Your source must not require use of fonts for non-Roman alphabets within the text itself. If your paper includes symbols in other languages (such as, but not limited to, Arabic, Chinese, Hebrew, Japanese, Thai, Russian and other Cyrillic languages), you must restrict their use to bit-mapped figures. Fonts that require non-English language support (CID and Identity-H) must be converted to outlines or 300 dpi bitmap or removed from the document (even if they are in a graphics file embedded in the document).
\item Two-column format in AAAI style is required for all papers.
\item The paper size for final submission must be US letter without exception.
\item The source file must exactly match the PDF.
\item The document margins may not be exceeded (no overfull boxes).
\item The number of pages and the file size must be as specified for your event.
\item No document may be password protected.
\item Neither the PDFs nor the source may contain any embedded links or bookmarks (no hyperref or navigator packages).
\item Your source and PDF must not have any page numbers, footers, or headers (no pagestyle commands).
\item Your PDF must be compatible with Acrobat 5 or higher.
\item Your \LaTeX{} source file (excluding references) must consist of a \textbf{single} file (use of the ``input" command is not allowed.
\item Your graphics must be sized appropriately outside of \LaTeX{} (do not use the ``clip" or ``trim'' command) .
\end{itemize}

If you do not follow these requirements, your paper will be returned to you to correct the deficiencies.

\section{What Files to Submit}
You must submit the following items to ensure that your paper is published:
\begin{itemize}
\item A fully-compliant PDF file.
\item Your \LaTeX{} source file submitted as a \textbf{single} .tex file (do not use the ``input" command to include sections of your paper --- every section must be in the single source file). (The only allowable exception is .bib file, which should be included separately).
\item The bibliography (.bib) file(s).
\item Your source must compile on our system, which includes only standard \LaTeX{} 2020 TeXLive support files.
\item Only the graphics files used in compiling paper.
\item The \LaTeX{}-generated files (e.g. .aux,  .bbl file, PDF, etc.).
\end{itemize}

Your \LaTeX{} source will be reviewed and recompiled on our system (if it does not compile, your paper will be returned to you. \textbf{Do not submit your source in multiple text files.} Your single \LaTeX{} source file must include all your text, your bibliography (formatted using aaai25.bst), and any custom macros.

Your files should work without any supporting files (other than the program itself) on any computer with a standard \LaTeX{} distribution.

\textbf{Do not send files that are not actually used in the paper.} Avoid including any files not needed for compiling your paper, including, for example, this instructions file, unused graphics files, style files, additional material sent for the purpose of the paper review, intermediate build files and so forth.

\textbf{Obsolete style files.} The commands for some common packages (such as some used for algorithms), may have changed. Please be certain that you are not compiling your paper using old or obsolete style files.

\textbf{Final Archive.} Place your source files in a single archive which should be compressed using .zip. The final file size may not exceed 10 MB.
Name your source file with the last (family) name of the first author, even if that is not you.


\section{Using \LaTeX{} to Format Your Paper}

The latest version of the AAAI style file is available on AAAI's website. Download this file and place it in the \TeX\ search path. Placing it in the same directory as the paper should also work. You must download the latest version of the complete AAAI Author Kit so that you will have the latest instruction set and style file.

\subsection{Document Preamble}

In the \LaTeX{} source for your paper, you \textbf{must} place the following lines as shown in the example in this subsection. This command set-up is for three authors. Add or subtract author and address lines as necessary, and uncomment the portions that apply to you. In most instances, this is all you need to do to format your paper in the Times font. The helvet package will cause Helvetica to be used for sans serif. These files are part of the PSNFSS2e package, which is freely available from many Internet sites (and is often part of a standard installation).

Leave the setcounter for section number depth commented out and set at 0 unless you want to add section numbers to your paper. If you do add section numbers, you must uncomment this line and change the number to 1 (for section numbers), or 2 (for section and subsection numbers). The style file will not work properly with numbering of subsubsections, so do not use a number higher than 2.

\subsubsection{The Following Must Appear in Your Preamble}
\begin{quote}
\begin{scriptsize}\begin{verbatim}
\documentclass[letterpaper]{article}
% DO NOT CHANGE THIS
\usepackage[submission]{aaai25} % DO NOT CHANGE THIS
\usepackage{times} % DO NOT CHANGE THIS
\usepackage{helvet} % DO NOT CHANGE THIS
\usepackage{courier} % DO NOT CHANGE THIS
\usepackage[hyphens]{url} % DO NOT CHANGE THIS
\usepackage{graphicx} % DO NOT CHANGE THIS
\urlstyle{rm} % DO NOT CHANGE THIS
\def\UrlFont{\rm} % DO NOT CHANGE THIS
\usepackage{graphicx}  % DO NOT CHANGE THIS
\usepackage{natbib}  % DO NOT CHANGE THIS
\usepackage{caption}  % DO NOT CHANGE THIS
\frenchspacing % DO NOT CHANGE THIS
\setlength{\pdfpagewidth}{8.5in} % DO NOT CHANGE THIS
\setlength{\pdfpageheight}{11in} % DO NOT CHANGE THIS
%
% Keep the \pdfinfo as shown here. There's no need
% for you to add the /Title and /Author tags.
\pdfinfo{
/TemplateVersion (2025.1)
}
\end{verbatim}\end{scriptsize}
\end{quote}

\subsection{Preparing Your Paper}

After the preamble above, you should prepare your paper as follows:
\begin{quote}
\begin{scriptsize}\begin{verbatim}
\begin{document}
\maketitle
\begin{abstract}
%...
\end{abstract}\end{verbatim}\end{scriptsize}
\end{quote}

\noindent If you want to add links to the paper's code, dataset(s), and extended version or similar this is the place to add them, within a \emph{links} environment:
\begin{quote}%
\begin{scriptsize}\begin{verbatim}
\begin{links}
  \link{Code}{https://aaai.org/example/guidelines}
  \link{Datasets}{https://aaai.org/example/datasets}
  \link{Extended version}{https://aaai.org/example}
\end{links}\end{verbatim}\end{scriptsize}
\end{quote}
\noindent Make sure that you do not de-anonymize yourself with these links.

\noindent You should then continue with the body of your paper. Your paper must conclude with the references, which should be inserted as follows:
\begin{quote}
\begin{scriptsize}\begin{verbatim}
% References and End of Paper
% These lines must be placed at the end of your paper
\bibliography{Bibliography-File}
\end{document}
\end{verbatim}\end{scriptsize}
\end{quote}

\begin{quote}
\begin{scriptsize}\begin{verbatim}
\begin{document}\\
\maketitle\\
...\\
\bibliography{Bibliography-File}\\
\end{document}\\
\end{verbatim}\end{scriptsize}
\end{quote}

\subsection{Commands and Packages That May Not Be Used}
\begin{table*}[t]
\centering

\begin{tabular}{l|l|l|l}
\textbackslash abovecaption &
\textbackslash abovedisplay &
\textbackslash addevensidemargin &
\textbackslash addsidemargin \\
\textbackslash addtolength &
\textbackslash baselinestretch &
\textbackslash belowcaption &
\textbackslash belowdisplay \\
\textbackslash break &
\textbackslash clearpage &
\textbackslash clip &
\textbackslash columnsep \\
\textbackslash float &
\textbackslash input &
\textbackslash input &
\textbackslash linespread \\
\textbackslash newpage &
\textbackslash pagebreak &
\textbackslash renewcommand &
\textbackslash setlength \\
\textbackslash text height &
\textbackslash tiny &
\textbackslash top margin &
\textbackslash trim \\
\textbackslash vskip\{- &
\textbackslash vspace\{- \\
\end{tabular}
%}
\caption{Commands that must not be used}
\label{table1}
\end{table*}

\begin{table}[t]
\centering
%\resizebox{.95\columnwidth}{!}{
\begin{tabular}{l|l|l|l}
    authblk & babel & cjk & dvips \\
    epsf & epsfig & euler & float \\
    fullpage & geometry & graphics & hyperref \\
    layout & linespread & lmodern & maltepaper \\
    navigator & pdfcomment & pgfplots & psfig \\
    pstricks & t1enc & titlesec & tocbind \\
    ulem
\end{tabular}
\caption{LaTeX style packages that must not be used.}
\label{table2}
\end{table}

There are a number of packages, commands, scripts, and macros that are incompatable with aaai25.sty. The common ones are listed in tables \ref{table1} and \ref{table2}. Generally, if a command, package, script, or macro alters floats, margins, fonts, sizing, linespacing, or the presentation of the references and citations, it is unacceptable. Note that negative vskip and vspace may not be used except in certain rare occurances, and may never be used around tables, figures, captions, sections, subsections, subsubsections, or references.


\subsection{Page Breaks}
For your final camera ready copy, you must not use any page break commands. References must flow directly after the text without breaks. Note that some conferences require references to be on a separate page during the review process. AAAI Press, however, does not require this condition for the final paper.


\subsection{Paper Size, Margins, and Column Width}
Papers must be formatted to print in two-column format on 8.5 x 11 inch US letter-sized paper. The margins must be exactly as follows:
\begin{itemize}
\item Top margin: 1.25 inches (first page), .75 inches (others)
\item Left margin: .75 inches
\item Right margin: .75 inches
\item Bottom margin: 1.25 inches
\end{itemize}


The default paper size in most installations of \LaTeX{} is A4. However, because we require that your electronic paper be formatted in US letter size, the preamble we have provided includes commands that alter the default to US letter size. Please note that using any other package to alter page size (such as, but not limited to the Geometry package) will result in your final paper being returned to you for correction.


\subsubsection{Column Width and Margins.}
To ensure maximum readability, your paper must include two columns. Each column should be 3.3 inches wide (slightly more than 3.25 inches), with a .375 inch (.952 cm) gutter of white space between the two columns. The aaai25.sty file will automatically create these columns for you.

\subsection{Overlength Papers}
If your paper is too long and you resort to formatting tricks to make it fit, it is quite likely that it will be returned to you. The best way to retain readability if the paper is overlength is to cut text, figures, or tables. There are a few acceptable ways to reduce paper size that don't affect readability. First, turn on \textbackslash frenchspacing, which will reduce the space after periods. Next, move all your figures and tables to the top of the page. Consider removing less important portions of a figure. If you use \textbackslash centering instead of \textbackslash begin\{center\} in your figure environment, you can also buy some space. For mathematical environments, you may reduce fontsize {\bf but not below 6.5 point}.


Commands that alter page layout are forbidden. These include \textbackslash columnsep,  \textbackslash float, \textbackslash topmargin, \textbackslash topskip, \textbackslash textheight, \textbackslash textwidth, \textbackslash oddsidemargin, and \textbackslash evensizemargin (this list is not exhaustive). If you alter page layout, you will be required to pay the page fee. Other commands that are questionable and may cause your paper to be rejected include \textbackslash parindent, and \textbackslash parskip. Commands that alter the space between sections are forbidden. The title sec package is not allowed. Regardless of the above, if your paper is obviously ``squeezed" it is not going to to be accepted. Options for reducing the length of a paper include reducing the size of your graphics, cutting text, or paying the extra page charge (if it is offered).


\subsection{Type Font and Size}
Your paper must be formatted in Times Roman or Nimbus. We will not accept papers formatted using Computer Modern or Palatino or some other font as the text or heading typeface. Sans serif, when used, should be Courier. Use Symbol or Lucida or Computer Modern for \textit{mathematics only. }

Do not use type 3 fonts for any portion of your paper, including graphics. Type 3 bitmapped fonts are designed for fixed resolution printers. Most print at 300 dpi even if the printer resolution is 1200 dpi or higher. They also often cause high resolution imagesetter devices to crash. Consequently, AAAI will not accept electronic files containing obsolete type 3 fonts. Files containing those fonts (even in graphics) will be rejected. (Authors using blackboard symbols must avoid packages that use type 3 fonts.)

Fortunately, there are effective workarounds that will prevent your file from embedding type 3 bitmapped fonts. The easiest workaround is to use the required times, helvet, and courier packages with \LaTeX{}2e. (Note that papers formatted in this way will still use Computer Modern for the mathematics. To make the math look good, you'll either have to use Symbol or Lucida, or you will need to install type 1 Computer Modern fonts --- for more on these fonts, see the section ``Obtaining Type 1 Computer Modern.")

If you are unsure if your paper contains type 3 fonts, view the PDF in Acrobat Reader. The Properties/Fonts window will display the font name, font type, and encoding properties of all the fonts in the document. If you are unsure if your graphics contain type 3 fonts (and they are PostScript or encapsulated PostScript documents), create PDF versions of them, and consult the properties window in Acrobat Reader.

The default size for your type must be ten-point with twelve-point leading (line spacing). Start all pages (except the first) directly under the top margin. (See the next section for instructions on formatting the title page.) Indent ten points when beginning a new paragraph, unless the paragraph begins directly below a heading or subheading.


\subsubsection{Obtaining Type 1 Computer Modern for \LaTeX{}.}

If you use Computer Modern for the mathematics in your paper (you cannot use it for the text) you may need to download type 1 Computer fonts. They are available without charge from the American Mathematical Society:
http://www.ams.org/tex/type1-fonts.html.

\subsubsection{Nonroman Fonts.}
If your paper includes symbols in other languages (such as, but not limited to, Arabic, Chinese, Hebrew, Japanese, Thai, Russian and other Cyrillic languages), you must restrict their use to bit-mapped figures.

\subsection{Title and Authors}
Your title must appear centered over both text columns in sixteen-point bold type (twenty-four point leading). The title must be written in Title Case according to the Chicago Manual of Style rules. The rules are a bit involved, but in general verbs (including short verbs like be, is, using, and go), nouns, adverbs, adjectives, and pronouns should be capitalized, (including both words in hyphenated terms), while articles, conjunctions, and prepositions are lower case unless they directly follow a colon or long dash. You can use the online tool \url{https://titlecaseconverter.com/} to double-check the proper capitalization (select the "Chicago" style and mark the "Show explanations" checkbox).

Author's names should appear below the title of the paper, centered in twelve-point type (with fifteen point leading), along with affiliation(s) and complete address(es) (including electronic mail address if available) in nine-point roman type (the twelve point leading). You should begin the two-column format when you come to the abstract.

\subsubsection{Formatting Author Information.}
Author information has to be set according to the following specification depending if you have one or more than one affiliation. You may not use a table nor may you employ the \textbackslash authorblk.sty package. For one or several authors from the same institution, please separate them with commas and write all affiliation directly below (one affiliation per line) using the macros \textbackslash author and \textbackslash affiliations:

\begin{quote}\begin{scriptsize}\begin{verbatim}
\author{
    Author 1, ..., Author n\\
}
\affiliations {
    Address line\\
    ... \\
    Address line\\
}
\end{verbatim}\end{scriptsize}\end{quote}


\noindent For authors from different institutions, use \textbackslash textsuperscript \{\textbackslash rm x \} to match authors and affiliations. Notice that there should not be any spaces between the author name (or comma following it) and the superscript.

\begin{quote}\begin{scriptsize}\begin{verbatim}
\author{
    AuthorOne\equalcontrib\textsuperscript{\rm 1,\rm 2},
    AuthorTwo\equalcontrib\textsuperscript{\rm 2},
    AuthorThree\textsuperscript{\rm 3},\\
    AuthorFour\textsuperscript{\rm 4},
    AuthorFive \textsuperscript{\rm 5}}
}
\affiliations {
    \textsuperscript{\rm 1}AffiliationOne,\\
    \textsuperscript{\rm 2}AffiliationTwo,\\
    \textsuperscript{\rm 3}AffiliationThree,\\
    \textsuperscript{\rm 4}AffiliationFour,\\
    \textsuperscript{\rm 5}AffiliationFive\\
    \{email, email\}@affiliation.com,
    email@affiliation.com,
    email@affiliation.com,
    email@affiliation.com
}
\end{verbatim}\end{scriptsize}\end{quote}

You can indicate that some authors contributed equally using the \textbackslash equalcontrib command. This will add a marker after the author names and a footnote on the first page.

Note that you may want to  break the author list for better visualization. You can achieve this using a simple line break (\textbackslash  \textbackslash).

\subsection{\LaTeX{} Copyright Notice}
The copyright notice automatically appears if you use aaai25.sty. It has been hardcoded and may not be disabled.

\subsection{Credits}
Any credits to a sponsoring agency should appear in the acknowledgments section, unless the agency requires different placement. If it is necessary to include this information on the front page, use
\textbackslash thanks in either the \textbackslash author or \textbackslash title commands.
For example:
\begin{quote}
\begin{small}
\textbackslash title\{Very Important Results in AI\textbackslash thanks\{This work is
 supported by everybody.\}\}
\end{small}
\end{quote}
Multiple \textbackslash thanks commands can be given. Each will result in a separate footnote indication in the author or title with the corresponding text at the botton of the first column of the document. Note that the \textbackslash thanks command is fragile. You will need to use \textbackslash protect.

Please do not include \textbackslash pubnote commands in your document.

\subsection{Abstract}
Follow the example commands in this document for creation of your abstract. The command \textbackslash begin\{abstract\} will automatically indent the text block. Please do not indent it further. {Do not include references in your abstract!}

\subsection{Page Numbers}

Do not print any page numbers on your paper. The use of \textbackslash pagestyle is forbidden.

\subsection{Text}
The main body of the paper must be formatted in black, ten-point Times Roman with twelve-point leading (line spacing). You may not reduce font size or the linespacing. Commands that alter font size or line spacing (including, but not limited to baselinestretch, baselineshift, linespread, and others) are expressly forbidden. In addition, you may not use color in the text.

\subsection{Citations}
Citations within the text should include the author's last name and year, for example (Newell 1980). Append lower-case letters to the year in cases of ambiguity. Multiple authors should be treated as follows: (Feigenbaum and Engelmore 1988) or (Ford, Hayes, and Glymour 1992). In the case of four or more authors, list only the first author, followed by et al. (Ford et al. 1997).

\subsection{Extracts}
Long quotations and extracts should be indented ten points from the left and right margins.

\begin{quote}
This is an example of an extract or quotation. Note the indent on both sides. Quotation marks are not necessary if you offset the text in a block like this, and properly identify and cite the quotation in the text.

\end{quote}

\subsection{Footnotes}
Use footnotes judiciously, taking into account that they interrupt the reading of the text. When required, they should be consecutively numbered throughout with superscript Arabic numbers. Footnotes should appear at the bottom of the page, separated from the text by a blank line space and a thin, half-point rule.

\subsection{Headings and Sections}
When necessary, headings should be used to separate major sections of your paper. Remember, you are writing a short paper, not a lengthy book! An overabundance of headings will tend to make your paper look more like an outline than a paper. The aaai25.sty package will create headings for you. Do not alter their size nor their spacing above or below.

\subsubsection{Section Numbers.}
The use of section numbers in AAAI Press papers is optional. To use section numbers in \LaTeX{}, uncomment the setcounter line in your document preamble and change the 0 to a 1. Section numbers should not be used in short poster papers and/or extended abstracts.

\subsubsection{Section Headings.}
Sections should be arranged and headed as follows:
\begin{enumerate}
\item Main content sections
\item Appendices (optional)
\item Ethical Statement (optional, unnumbered)
\item Acknowledgements (optional, unnumbered)
\item References (unnumbered)
\end{enumerate}

\subsubsection{Appendices.}
Any appendices must appear after the main content. If your main sections are numbered, appendix sections must use letters instead of arabic numerals. In \LaTeX{} you can use the \texttt{\textbackslash appendix} command to achieve this effect and then use \texttt{\textbackslash section\{Heading\}} normally for your appendix sections.

\subsubsection{Ethical Statement.}
You can write a statement about the potential ethical impact of your work, including its broad societal implications, both positive and negative. If included, such statement must be written in an unnumbered section titled \emph{Ethical Statement}.

\subsubsection{Acknowledgments.}
The acknowledgments section, if included, appears right before the references and is headed ``Acknowledgments". It must not be numbered even if other sections are (use \texttt{\textbackslash section*\{Acknowledgements\}} in \LaTeX{}). This section includes acknowledgments of help from associates and colleagues, credits to sponsoring agencies, financial support, and permission to publish. Please acknowledge other contributors, grant support, and so forth, in this section. Do not put acknowledgments in a footnote on the first page. If your grant agency requires acknowledgment of the grant on page 1, limit the footnote to the required statement, and put the remaining acknowledgments at the back. Please try to limit acknowledgments to no more than three sentences.

\subsubsection{References.}
The references section should be labeled ``References" and must appear at the very end of the paper (don't end the paper with references, and then put a figure by itself on the last page). A sample list of references is given later on in these instructions. Please use a consistent format for references. Poorly prepared or sloppy references reflect badly on the quality of your paper and your research. Please prepare complete and accurate citations.

\subsection{Illustrations and  Figures}

\begin{figure}[t]
\centering
\includegraphics[width=0.9\columnwidth]{figure1} % Reduce the figure size so that it is slightly narrower than the column. Don't use precise values for figure width.This setup will avoid overfull boxes.
\caption{Using the trim and clip commands produces fragile layers that can result in disasters (like this one from an actual paper) when the color space is corrected or the PDF combined with others for the final proceedings. Crop your figures properly in a graphics program -- not in LaTeX.}
\label{fig1}
\end{figure}

\begin{figure*}[t]
\centering
\includegraphics[width=0.8\textwidth]{figure2} % Reduce the figure size so that it is slightly narrower than the column.
\caption{Adjusting the bounding box instead of actually removing the unwanted data resulted multiple layers in this paper. It also needlessly increased the PDF size. In this case, the size of the unwanted layer doubled the paper's size, and produced the following surprising results in final production. Crop your figures properly in a graphics program. Don't just alter the bounding box.}
\label{fig2}
\end{figure*}

% Using the \centering command instead of \begin{center} ... \end{center} will save space
% Positioning your figure at the top of the page will save space and make the paper more readable
% Using 0.95\columnwidth in conjunction with the


Your paper must compile in PDF\LaTeX{}. Consequently, all your figures must be .jpg, .png, or .pdf. You may not use the .gif (the resolution is too low), .ps, or .eps file format for your figures.

Figures, drawings, tables, and photographs should be placed throughout the paper on the page (or the subsequent page) where they are first discussed. Do not group them together at the end of the paper. If placed at the top of the paper, illustrations may run across both columns. Figures must not invade the top, bottom, or side margin areas. Figures must be inserted using the \textbackslash usepackage\{graphicx\}. Number figures sequentially, for example, figure 1, and so on. Do not use minipage to group figures.

If you normally create your figures using pgfplots, please create the figures first, and then import them as pdfs with proper bounding boxes, as the bounding and trim boxes created by pfgplots are fragile and not valid.

When you include your figures, you must crop them \textbf{outside} of \LaTeX{}. The command \textbackslash includegraphics*[clip=true, viewport 0 0 10 10]{...} might result in a PDF that looks great, but the image is \textbf{not really cropped.} The full image can reappear (and obscure whatever it is overlapping) when page numbers are applied or color space is standardized. Figures \ref{fig1}, and \ref{fig2} display some unwanted results that often occur.

If your paper includes illustrations that are not compatible with PDF\TeX{} (such as .eps or .ps documents), you will need to convert them. The epstopdf package will usually work for eps files. You will need to convert your ps files to PDF in either case.

\subsubsection {Figure Captions.}The illustration number and caption must appear \textit{under} the illustration. Labels and other text with the actual illustration must be at least nine-point type. However, the font and size of figure captions must be 10 point roman. Do not make them smaller, bold, or italic. (Individual words may be italicized if the context requires differentiation.)

\subsection{Tables}

\subsection{Tables}

Tables should be presented in 10 point roman type. If necessary, they may be altered to 9 point type. You must not use \texttt{\textbackslash resizebox} or other commands that resize the entire table to make it smaller, because you can't control the final font size this way.
If your table is too large you can use \texttt{\textbackslash setlength\{\textbackslash tabcolsep\}\{1mm\}} to compress the columns a bit or you can adapt the content (e.g.: reduce the decimal precision when presenting numbers, use shortened column titles, make some column duble-line to get it narrower).

Tables that do not fit in a single column must be placed across double columns. If your table won't fit within the margins even when spanning both columns and using the above techniques, you must split it in two separate tables.

\subsubsection {Table Captions.} The number and caption for your table must appear \textit{under} (not above) the table.  Additionally, the font and size of table captions must be 10 point roman and must be placed beneath the figure. Do not make them smaller, bold, or italic. (Individual words may be italicized if the context requires differentiation.)



\subsubsection{Low-Resolution Bitmaps.}
You may not use low-resolution (such as 72 dpi) screen-dumps and GIF files---these files contain so few pixels that they are always blurry, and illegible when printed. If they are color, they will become an indecipherable mess when converted to black and white. This is always the case with gif files, which should never be used. The resolution of screen dumps can be increased by reducing the print size of the original file while retaining the same number of pixels. You can also enlarge files by manipulating them in software such as PhotoShop. Your figures should be 300 dpi when incorporated into your document.

\subsubsection{\LaTeX{} Overflow.}
\LaTeX{} users please beware: \LaTeX{} will sometimes put portions of the figure or table or an equation in the margin. If this happens, you need to make the figure or table span both columns. If absolutely necessary, you may reduce the figure, or reformat the equation, or reconfigure the table.{ \bf Check your log file!} You must fix any overflow into the margin (that means no overfull boxes in \LaTeX{}). \textbf{Nothing is permitted to intrude into the margin or gutter.}


\subsubsection{Using Color.}
Use of color is restricted to figures only. It must be WACG 2.0 compliant. (That is, the contrast ratio must be greater than 4.5:1 no matter the font size.) It must be CMYK, NOT RGB. It may never be used for any portion of the text of your paper. The archival version of your paper will be printed in black and white and grayscale. The web version must be readable by persons with disabilities. Consequently, because conversion to grayscale can cause undesirable effects (red changes to black, yellow can disappear, and so forth), we strongly suggest you avoid placing color figures in your document. If you do include color figures, you must (1) use the CMYK (not RGB) colorspace and (2) be mindful of readers who may happen to have trouble distinguishing colors. Your paper must be decipherable without using color for distinction.

\subsubsection{Drawings.}
We suggest you use computer drawing software (such as Adobe Illustrator or, (if unavoidable), the drawing tools in Microsoft Word) to create your illustrations. Do not use Microsoft Publisher. These illustrations will look best if all line widths are uniform (half- to two-point in size), and you do not create labels over shaded areas. Shading should be 133 lines per inch if possible. Use Times Roman or Helvetica for all figure call-outs. \textbf{Do not use hairline width lines} --- be sure that the stroke width of all lines is at least .5 pt. Zero point lines will print on a laser printer, but will completely disappear on the high-resolution devices used by our printers.

\subsubsection{Photographs and Images.}
Photographs and other images should be in grayscale (color photographs will not reproduce well; for example, red tones will reproduce as black, yellow may turn to white, and so forth) and set to a minimum of 300 dpi. Do not prescreen images.

\subsubsection{Resizing Graphics.}
Resize your graphics \textbf{before} you include them with LaTeX. You may \textbf{not} use trim or clip options as part of your \textbackslash includegraphics command. Resize the media box of your PDF using a graphics program instead.

\subsubsection{Fonts in Your Illustrations.}
You must embed all fonts in your graphics before including them in your LaTeX document.

\subsubsection{Algorithms.}
Algorithms and/or programs are a special kind of figures. Like all illustrations, they should appear floated to the top (preferably) or bottom of the page. However, their caption should appear in the header, left-justified and enclosed between horizontal lines, as shown in Algorithm~\ref{alg:algorithm}. The algorithm body should be terminated with another horizontal line. It is up to the authors to decide whether to show line numbers or not, how to format comments, etc.

In \LaTeX{} algorithms may be typeset using the {\tt algorithm} and {\tt algorithmic} packages, but you can also use one of the many other packages for the task.

\begin{algorithm}[tb]
\caption{Example algorithm}
\label{alg:algorithm}
\textbf{Input}: Your algorithm's input\\
\textbf{Parameter}: Optional list of parameters\\
\textbf{Output}: Your algorithm's output
\begin{algorithmic}[1] %[1] enables line numbers
\STATE Let $t=0$.
\WHILE{condition}
\STATE Do some action.
\IF {conditional}
\STATE Perform task A.
\ELSE
\STATE Perform task B.
\ENDIF
\ENDWHILE
\STATE \textbf{return} solution
\end{algorithmic}
\end{algorithm}

\subsubsection{Listings.}
Listings are much like algorithms and programs. They should also appear floated to the top (preferably) or bottom of the page. Listing captions should appear in the header, left-justified and enclosed between horizontal lines as shown in Listing~\ref{lst:listing}. Terminate the body with another horizontal line and avoid any background color. Line numbers, if included, must appear within the text column.

\begin{listing}[tb]%
\caption{Example listing {\tt quicksort.hs}}%
\label{lst:listing}%
\begin{lstlisting}[language=Haskell]
quicksort :: Ord a => [a] -> [a]
quicksort []     = []
quicksort (p:xs) = (quicksort lesser) ++ [p] ++ (quicksort greater)
	where
		lesser  = filter (< p) xs
		greater = filter (>= p) xs
\end{lstlisting}
\end{listing}

\subsection{References}
The AAAI style includes a set of definitions for use in formatting references with BibTeX. These definitions make the bibliography style fairly close to the ones  specified in the Reference Examples appendix below. To use these definitions, you also need the BibTeX style file ``aaai25.bst," available in the AAAI Author Kit on the AAAI web site. Then, at the end of your paper but before \textbackslash end{document}, you need to put the following lines:

\begin{quote}
\begin{small}
\textbackslash bibliography\{bibfile1,bibfile2,...\}
\end{small}
\end{quote}

Please note that the aaai25.sty class already sets the bibliographystyle for you, so you do not have to place any \textbackslash bibliographystyle command in the document yourselves. The aaai25.sty file is incompatible with the hyperref and navigator packages. If you use either, your references will be garbled and your paper will be returned to you.

References may be the same size as surrounding text.
However, in this section (only), you may reduce the size to {\em \textbackslash small} (9pt) if your paper exceeds the allowable number of pages. Making it any smaller than 9 point with 10 point linespacing, however, is not allowed.

The list of files in the \textbackslash bibliography command should be the names of your BibTeX source files (that is, the .bib files referenced in your paper).

The following commands are available for your use in citing references:
\begin{quote}
{\em \textbackslash cite:} Cites the given reference(s) with a full citation. This appears as ``(Author Year)'' for one reference, or ``(Author Year; Author Year)'' for multiple references.\smallskip\\
{\em \textbackslash shortcite:} Cites the given reference(s) with just the year. This appears as ``(Year)'' for one reference, or ``(Year; Year)'' for multiple references.\smallskip\\
{\em \textbackslash citeauthor:} Cites the given reference(s) with just the author name(s) and no parentheses.\smallskip\\
{\em \textbackslash citeyear:} Cites the given reference(s) with just the date(s) and no parentheses.
\end{quote}
You may also use any of the \emph{natbib} citation commands.


\section{Proofreading Your PDF}
Please check all the pages of your PDF file. The most commonly forgotten element is the acknowledgements --- especially the correct grant number. Authors also commonly forget to add the metadata to the source, use the wrong reference style file, or don't follow the capitalization rules or comma placement for their author-title information properly. A final common problem is text (expecially equations) that runs into the margin. You will need to fix these common errors before submitting your file.

\section{Improperly Formatted Files }
In the past, AAAI has corrected improperly formatted files submitted by the authors. Unfortunately, this has become an increasingly burdensome expense that we can no longer absorb). Consequently, if your file is improperly formatted, it will be returned to you for correction.

\section{Naming Your Electronic File}
We require that you name your \LaTeX{} source file with the last name (family name) of the first author so that it can easily be differentiated from other submissions. Complete file-naming instructions will be provided to you in the submission instructions.

\section{Submitting Your Electronic Files to AAAI}
Instructions on paper submittal will be provided to you in your acceptance letter.

\section{Inquiries}
If you have any questions about the preparation or submission of your paper as instructed in this document, please contact AAAI Press at the address given below. If you have technical questions about implementation of the aaai style file, please contact an expert at your site. We do not provide technical support for \LaTeX{} or any other software package. To avoid problems, please keep your paper simple, and do not incorporate complicated macros and style files.

\begin{quote}
\noindent AAAI Press\\
1101 Pennsylvania Ave, NW Suite 300\\
Washington, DC 20004 USA\\
\textit{Telephone:} 1-202-360-4062\\
\textit{E-mail:} See the submission instructions for your particular conference or event.
\end{quote}

\section{Additional Resources}
\LaTeX{} is a difficult program to master. If you've used that software, and this document didn't help or some items were not explained clearly, we recommend you read Michael Shell's excellent document (testflow doc.txt V1.0a 2002/08/13) about obtaining correct PS/PDF output on \LaTeX{} systems. (It was written for another purpose, but it has general application as well). It is available at www.ctan.org in the tex-archive.

\appendix
\section{Reference Examples}
\label{sec:reference_examples}

\nobibliography*
Formatted bibliographies should look like the following examples. You should use BibTeX to generate the references. Missing fields are unacceptable when compiling references, and usually indicate that you are using the wrong type of entry (BibTeX class).

\paragraph{Book with multiple authors~\nocite{em:86}} Use the \texttt{@book} class.\\[.2em]
\bibentry{em:86}.

\paragraph{Journal and magazine articles~\nocite{r:80, hcr:83}} Use the \texttt{@article} class.\\[.2em]
\bibentry{r:80}.\\[.2em]
\bibentry{hcr:83}.

\paragraph{Proceedings paper published by a society, press or publisher~\nocite{c:83, c:84}} Use the \texttt{@inproceedings} class. You may abbreviate the \emph{booktitle} field, but make sure that the conference edition is clear.\\[.2em]
\bibentry{c:84}.\\[.2em]
\bibentry{c:83}.

\paragraph{University technical report~\nocite{r:86}} Use the \texttt{@techreport} class.\\[.2em]
\bibentry{r:86}.

\paragraph{Dissertation or thesis~\nocite{c:79}} Use the \texttt{@phdthesis} class.\\[.2em]
\bibentry{c:79}.

\paragraph{Forthcoming publication~\nocite{c:21}} Use the \texttt{@misc} class with a \texttt{note="Forthcoming"} annotation.
\begin{quote}
\begin{footnotesize}
\begin{verbatim}
@misc(key,
  [...]
  note="Forthcoming",
)
\end{verbatim}
\end{footnotesize}
\end{quote}
\bibentry{c:21}.

\paragraph{ArXiv paper~\nocite{c:22}} Fetch the BibTeX entry from the "Export Bibtex Citation" link in the arXiv website. Notice it uses the \texttt{@misc} class instead of the \texttt{@article} one, and that it includes the \texttt{eprint} and \texttt{archivePrefix} keys.
\begin{quote}
\begin{footnotesize}
\begin{verbatim}
@misc(key,
  [...]
  eprint="xxxx.yyyy",
  archivePrefix="arXiv",
)
\end{verbatim}
\end{footnotesize}
\end{quote}
\bibentry{c:22}.

\paragraph{Website or online resource~\nocite{c:23}} Use the \texttt{@misc} class. Add the url in the \texttt{howpublished} field and the date of access in the \texttt{note} field:
\begin{quote}
\begin{footnotesize}
\begin{verbatim}
@misc(key,
  [...]
  howpublished="\url{http://...}",
  note="Accessed: YYYY-mm-dd",
)
\end{verbatim}
\end{footnotesize}
\end{quote}
\bibentry{c:23}.

\end{comment}


\section*{Acknowledgments}
The work was authored by the Pacific Northwest National Laboratory (PNNL) is operated for the DOE by Battelle Memorial Institute under contract DE-AC06-76RL01830. Funding is provided by the U.S. Department of Energy Wind Energy Technologies Office for project: Path to Nationwide Deployment of Fully Coupled Wind-Based Hybrid Energy Systems. The views expressed in the article do not necessarily represent the views of the DOE or the U.S. Government. The U.S. Government retains and the publisher, by accepting the article for publication, acknowledges that the U.S. Government retains a nonexclusive, paid-up, irrevocable, worldwide license to publish or reproduce the published form of this work, or allow others to do so, for U.S. Government purposes.

\bibliography{aaai25}

% \pagebreak

\end{document}
