%Version 3 December 2023
% See section 11 of the User Manual for version history
%
%%%%%%%%%%%%%%%%%%%%%%%%%%%%%%%%%%%%%%%%%%%%%%%%%%%%%%%%%%%%%%%%%%%%%%
%%                                                                 %%
%% Please do not use \input{...} to include other tex files.       %%
%% Submit your LaTeX manuscript as one .tex document.              %%
%%                                                                 %%
%% All additional figures and files should be attached             %%
%% separately and not embedded in the \TeX\ document itself.       %%
%%                                                                 %%
%%%%%%%%%%%%%%%%%%%%%%%%%%%%%%%%%%%%%%%%%%%%%%%%%%%%%%%%%%%%%%%%%%%%%

%%\documentclass[referee,sn-basic]{sn-jnl}% referee option is meant for double line spacing

%%=======================================================%%
%% to print line numbers in the margin use lineno option %%
%%=======================================================%%

%%\documentclass[lineno,sn-basic]{sn-jnl}% Basic Springer Nature Reference Style/Chemistry Reference Style

%%======================================================%%
%% to compile with pdflatex/xelatex use pdflatex option %%
%%======================================================%%

%%\documentclass[pdflatex,sn-basic]{sn-jnl}% Basic Springer Nature Reference Style/Chemistry Reference Style


%%Note: the following reference styles support Namedate and Numbered referencing. By default the style follows the most common style. To switch between the options you can add or remove “Numbered” in the optional parenthesis. 
%%The option is available for: sn-basic.bst, sn-vancouver.bst, sn-chicago.bst%  
 
%%\documentclass[pdflatex,sn-nature]{sn-jnl}% Style for submissions to Nature Portfolio journals
%%\documentclass[pdflatex,sn-basic]{sn-jnl}% Basic Springer Nature Reference Style/Chemistry Reference Style
\documentclass[pdflatex,sn-mathphys-num]{sn-jnl}% Math and Physical Sciences Numbered Reference Style 
%%\documentclass[pdflatex,sn-mathphys-ay]{sn-jnl}% Math and Physical Sciences Author Year Reference Style
%%\documentclass[pdflatex,sn-aps]{sn-jnl}% American Physical Society (APS) Reference Style
%%\documentclass[pdflatex,sn-vancouver,Numbered]{sn-jnl}% Vancouver Reference Style
%%\documentclass[pdflatex,sn-apa]{sn-jnl}% APA Reference Style 
%%\documentclass[pdflatex,sn-chicago]{sn-jnl}% Chicago-based Humanities Reference Style

%%%% Standard Packages
%%<additional latex packages if required can be included here>

\usepackage{graphicx}%
\usepackage{multirow}%
\usepackage{amsmath,amssymb,amsfonts}%
\usepackage{amsthm}%
\usepackage{mathrsfs}%
\usepackage[title]{appendix}%
\usepackage{xcolor}%
\usepackage{textcomp}%
\usepackage{manyfoot}%
\usepackage{booktabs}%
\usepackage{algorithm}%
\usepackage{algorithmicx}%
\usepackage{algpseudocode}%
\usepackage{listings}%
%%%%

%%%%%=============================================================================%%%%
%%%%  Remarks: This template is provided to aid authors with the preparation
%%%%  of original research articles intended for submission to journals published 
%%%%  by Springer Nature. The guidance has been prepared in partnership with 
%%%%  production teams to conform to Springer Nature technical requirements. 
%%%%  Editorial and presentation requirements differ among journal portfolios and 
%%%%  research disciplines. You may find sections in this template are irrelevant 
%%%%  to your work and are empowered to omit any such section if allowed by the 
%%%%  journal you intend to submit to. The submission guidelines and policies 
%%%%  of the journal take precedence. A detailed User Manual is available in the 
%%%%  template package for technical guidance.
%%%%%=============================================================================%%%%

%% as per the requirement new theorem styles can be included as shown below
\theoremstyle{thmstyleone}%
\newtheorem{theorem}{Theorem}%  meant for continuous numbers
%%\newtheorem{theorem}{Theorem}[section]% meant for sectionwise numbers
%% optional argument [theorem] produces theorem numbering sequence instead of independent numbers for Proposition
\newtheorem{proposition}[theorem]{Proposition}% 
%%\newtheorem{proposition}{Proposition}% to get separate numbers for theorem and proposition etc.

\theoremstyle{thmstyletwo}%
\newtheorem{example}{Example}%
\newtheorem{remark}{Remark}%

\theoremstyle{thmstylethree}%
\newtheorem{definition}{Definition}%

\raggedbottom
%%\unnumbered% uncomment this for unnumbered level heads

\begin{document}

\begin{flushleft}
Sarah Ball \\
LMU Munich \& Munich Center for Machine \\ Learning (MCML)
\\\\
Simeon Allmendinger \\
University of Bayreuth \& Fraunhofer Institute \\ for Applied Information Technology (FIT)
\\\\
Frauke Kreuter \\
LMU Munich \& MCML
\\\\
Niklas Kühl \\
University of Bayreuth \& Fraunhofer FIT
\\\\
To: Nature Computational Science
\end{flushleft}

\vspace{2cm}

\begin{flushright}
    {January 20, 2025}
\end{flushright}

\subsubsection*{Presubmission Inquiry}
\vspace{1cm}
Dear Editors,
\\\\
We are exploring whether our recent study on the technical reliability and applicability of large language models (LLMs) in public opinion research aligns with your interests as a \textit{Perspective} or \textit{Comment} piece in \textit{Nature Computational Science}.
\\\\
\textbf{Title:} Human Preferences in Language Model Latent Space: A Technical Analysis on the Reliability of Synthetic Data
%The Technical Frontiers of Using Language Models in Survey Research
\\\\
\textbf{Motivation and Summary.}
With the release of ChatGPT in November 2022, the world has seen a spike in interest in large language models (LLMs). Many academic disciplines, as well as the business world, wonder if and how they can integrate LLMs into their endeavors. One emerging---and highly debated---topic is the usage of LLMs for (public) opinion research. The idea is that one can leverage LLMs to substitute for surveying humans. Yet, the question remains as to how valid and reliable it is to substitute humans with LLMs from a technical perspective. Previous research mainly focuses on comparing LLM predictions based on personas to a gold standard survey prediction for these personas. The results of such analyses are mixed \citep{argyle2023out,durmus2023towards,kim2023ai}, revealing problems like, e.g., prediction instability that occurs with slight formulation changes in the prompt \cite{bisbee2023synthetic} and performance differences between countries \cite{von2024united}. While such approaches might give first insights into how well LLMs can predict general past questions of interest, we lack a deeper understanding of how ``opinion formation'' works on a \textit{technical} level in LLMs. 

Our research builds on hypotheses already registered on the Open Science Framework (OSF), where we address two central questions:
\begin{enumerate}
    \item How can language models mimic the distribution of voting outcomes for different demographic subgroups in their latent space?
    \item How does higher entropy in voting outcomes correlate with greater prompt instability?
\end{enumerate}

To answer these questions, our research analyzes the latent mechanisms of LLMs, focusing on understanding persona-to-party mappings in multi- and dual-party contexts. To this end, we identify model-specific value vectors associated with political affiliations and employ trained probes (Multi-Layer-Perceptrons) to uncover which persona attributes trigger these vectors. Building on this analysis, we study how minor textual changes impact the persona-to-party mapping and downstream model predictions. Our investigation provides a technical foundation to evaluate the usability of LLMs in survey-like tasks and reveals limitations that practitioners must address.
\\\\
\textbf{Significance.} Generative AI is increasingly used in survey contexts to simulate public opinions and predict voting outcomes~\cite{vonderheyde2024uniteddiversitycontextualbiases}. However, significant gaps remain in understanding the technical validity and reliability of these approaches. Studies highlight cultural, demographic, and thematic biases, with performance favoring Western, English-speaking contexts~\cite{qu2024performance}. While frontier LLMs like ChatGPT show promise in replicating nuanced response patterns~\cite{argyle2023out}, they still often exhibit systematic distortions in behavioral simulations~\cite{aher2023using}. Hence, to leverage the advantages of efficient large-scale experimentation via LLMs a deeper technical understanding of the possibilities and limitations is necessary~\cite{horton2023large}.

Our research addresses the need for a deeper technical understanding by analyzing how demographic attributes and prompt variations influence latent opinion mappings in LLMs. By uncovering biases and instabilities at a technical level, we provide actionable insights into improving representativeness and reliability. This understanding is essential to critically evaluate whether LLMs can effectively serve as a reliable substitute or supplement for traditional human samples, particularly in contexts requiring equitable and accurate representation.
\\\\
\textbf{Impact.} As the adoption of LLMs continues to grow, it is crucial to ground these advancements in rigorous technical evaluations. By linking LLM mechanisms with practical considerations, our study bridges methodological insights and real-world implementation challenges, advancing the discourse on AI's role in public opinion research.
\\\\
We appreciate your feedback on the suitability of this topic for a \textit{Perspective} or \textit{Comment} piece in your journal.
\\\\
Sarah Ball, \\Simeon Allmendinger, \\Frauke Kreuter, \\Niklas Kühl

\newpage
\bibliography{99_References/references}


\end{document}
