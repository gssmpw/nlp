\begin{figure*}[!h]
  \centering
  \includegraphics[width=\linewidth]{figs/overview_v1.pdf}
  \caption{
\textbf{Overview of our method.}
Given a user-provided prompt $\inputprompt$ and a partial sketch $\inputsketch$, our method first (a) stylizes the input prompt by augmenting it using style descriptions generated by the VLM (\textbf{bold text}).
Using the stylized prompt $\finalprompt$, the method then performs (b) stroke optimization to generate strokes that fill the missing regions, thus ensuring that the style-agnostic completed sketch $\intermediatesketch$ can fully represents the content of the user-provided prompt.
To align the styles of $\intermediatesketch$ and $\inputsketch$, we (c) instruct the VLM to generate an executable style adjustment code that modifies the strokes of $\intermediatesketch$.
Finally, we obtain a final completed sketch $\completesketch$ wherein the styles of the strokes are aligned to those of the $\inputsketch$.
}
\label{fig:overview}
\end{figure*}





\section{Overview}
In~\cref{fig:overview}, we illustrate the overview of our method.
Our method takes a text prompt $\inputprompt$ and a partial sketch $\inputsketch$ as inputs.
The prompt describes the content to be illustrated in the completed sketch, but the user-provided partial sketch represents only some of the content described in the prompt.
The output is a completed sketch $\completesketch=\inputsketch \cup \optsketch$ that fully represents the content of $\inputprompt$.
Our method has two stages: \textit{style-agnostic sketch completion} and \textit{sketch style adjustment}.

In the first stage, the goal is to optimize a set of parametric strokes that, when combined with the user-provided partial sketch, ensure that the complete sketch represents the content of $\inputprompt$ without consideration of sketch styles.
First, we stylize $\inputprompt$ by leveraging a large vision-language model (VLM) to produce style descriptions $\augprompt$ of the given partial sketch $\inputsketch$, \ie~$\finalprompt=\{\inputprompt \cup \augprompt\}$ (\cref{fig:overview}(a)).
Then, we optimize the parameters of $\optsketch$ using a diffusion prior conditioned on the stylized text prompt $\finalprompt$ (\cref{fig:overview}(b)) and obtains $\intermediatesketch$.

In the second stage, the goal is to adjust the styles of $\intermediatesketch$ to ensure a coherent style across the final sketch. 
We task the VLM using a carefully crafted prompt that contains the completed sketch of the first stage in SVG format and the text prompt $\finalprompt$.
The VLM then generates executable code that adjusts the styles of the new strokes in $\intermediatesketch$ to the style of the original partial sketch. 
