\section{Stage 1: Style-agnostic Sketch Completion}
Inspired by previous works~\cite{xing2023diffsketcher,vinker2023clipascene}, we take a ``synthesis through optimization'' approach. 
We optimize the parameters of a group of strokes by leveraging the prior of a pretrained text-to-image (T2I) model.
Unlike previous works, our method employs a user-provided partial sketch $\mathcal{S}_{\text{input}}$ as an additional input.
Therefore, we employ a conditional T2I model (\eg~ControlNet Scribble\footnote{\url{https://huggingface.co/lllyasviel/sd-controlnet-scribble}}) to optimize the stroke parameters.

\subsection{Prompt Stylization}
\label{sec:method_step1}
Although the conditional T2I model generates images that match the input text prompt $\inputprompt$, the styles of the generated images are often not those of the given partial sketch $\mathcal{S}_{\text{input}}$.
It is likely that the style of the optimized sketch will deviate from that of $\mathcal{S}_{\text{input}}$.
To address this issue, we first stylize the input prompt $\mathcal{P}_{\text{input}}$, \ie~augment style descriptions to $\mathcal{P}_{\text{input}}$ using the VLM.
Specifically, we render the partial sketch $\inputsketch$ into a raster image and then request the VLM to generate textual descriptions capturing both the semantic and stylistic cues of the rendered image.
Then, we augment the style descriptions $\mathcal{P}_{\text{aug}}$ to the input prompt.
The final prompt becomes: \ie~$\finalprompt=\{\mathcal{P}_{\text{input}} \cup \mathcal{P}_{\text{aug}}\}$.


\begin{figure}[!t]
  \centering
  \includegraphics[width=\linewidth]{figs/stroke_opt_v1.pdf}
  \caption{
\textbf{Overview of stroke optimization.} 
Given (a) the user-provided partial sketch and (b) the guidance image generated by the conditional T2I model, our method (c) iteratively updates the position, opacity, and width of each stroke.
This ensures that the resulting style-agnostic completed sketch is in visual alignment with the guidance image but does not overlap with the user-provided partial sketch. 
}
  \label{fig:stroke_opt}
\end{figure}

\subsection{Stroke Optimization for Completion}
\label{sec:method_step2}
Using the stylized prompt $\finalprompt$, we generate strokes that fill the empty regions of the user-provided partial sketch.
We define the strokes to be optimized as $\optsketch=\{s_1,\dots,s_n\}$, and the stroke parameterization as:
\begin{align}
    s_i = \left\{ \{p^j_i\}_{j=1}^4,o_i, w_i \right\},
\end{align}
where $\{p^j_i\}_{j=1}^4$ are the control points of a cubic Bézier curve, $o_i$ denotes an opacity attribute, and $w_i$ denotes the stroke width.
Specifically, we first generate an guidance image $\guideimg$ using a conditional T2I model that is based on the stylized prompt $\finalprompt$.
Then, we optimize the control points to obtain a sketch that is consistent with both the stylized prompt $\finalprompt$ and the guidance image $\guideimg$ (\cref{fig:stroke_opt}).
Specifically, at iteration $t$, we rasterize the strokes using a differentiable rasterizer $R$ to generate the raster sketch: $\rastersketch = R(\completesketch)$, and we optimize the following objective function when updating the strokes:
\begin{align}
    L_{\mathrm{all}}
    &= \alpha \;\tcboxmath[colback=white,colframe=purpleD,title=CLIP visual alignment]{(1 - \text{sim}\bigl(\phi_{\mathrm{vis}}(\rastersketch), \phi_{\mathrm{vis}}(\guideimg)\bigr))} \label{eq:clip_alignment}\\ 
    &+ \beta \;\tcboxmath[colback=white,colframe=greenD,title=perceptual loss]{(LPIPS(\rastersketch, \guideimg))} \label{eq:percep}\\ 
    &+ \gamma \;\tcboxmath[colback=white,colframe=lightblue,title=Overlap penalty]{\sum_{x_k \in \mathbf{x}} \mathds{1} \left[ \mathbf{M}(x_k) = 1 \right]},
    \label{eq:overlap_penalty}
\end{align}
where $\alpha,\beta,\gamma$ control the relative importance of the three terms.
The first term measures the visual alignment between the guidance image $\guideimg$ and the raster sketch $\rastersketch$ using the CLIP visual encoder $\phi_{\mathrm{img}}(\cdot)$, where $\text{sim}(\mathbf{x},\mathbf{y})=\frac{\mathbf{x}\cdot\mathbf{y}}{\|\mathbf{x}\|\cdot \|\mathbf{y}\|}$ is the cosine similarity.
Additionally, we further minimize the LPIPS loss to enhance the visual similarity of $\rastersketch$ and $\guideimg$.

To ensure that the strokes do not overlap with those of the user-provided sketch $\inputsketch$, we introduce an overlap penalty term. 
Specifically, we first define a binary mask $\mathbf{M}$ that encodes the regions in $\inputsketch$ where strokes already exist and should thus not be altered:
\begin{equation}
    \mathbf{M}(x) = 
    \begin{cases}
    1, & \text{if pixel } x \text{ belongs to strokes in } \inputsketch, \\
    0, & \text{otherwise}.
    \end{cases}
\end{equation}

\begin{wrapfigure}{r}{0.12\textwidth}
\includegraphics[width=0.10\textwidth]{figs/mask_penalty.pdf}
\end{wrapfigure} 
Then, we sample $10$ points on each stroke $s_i \in \optsketch$.
For each sample point $x_k$, if that point falls in $M$ (the filled black circles in the inset), we introduce a penalty, where $\mathds{1}[\cdot]$ in~\cref{eq:overlap_penalty} is the indicator function.

After optimizing $L_{\mathrm{all}}$, we obtain the style-agnostic completed sketch $\mathcal{\bar{S}}_\mathrm{complete}$ by combining the optimized strokes $\mathcal{\bar{S}}_{\mathrm{opt}}$ with those of $\inputsketch$.
The strokes in $\mathcal{\bar{S}}_\mathrm{complete}$ contain the overall content in $\inputprompt$, but the styles are not coherent.













