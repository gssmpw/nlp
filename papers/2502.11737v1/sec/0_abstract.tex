\begin{abstract}
\pspace{}The digital age requires strong security measures to protect online activities. \Ac{2fa} has emerged as a critical solution. However, its implementation presents significant challenges, particularly in terms of accessibility for people with disabilities. This paper examines the intricacies of deploying \ac{2fa} in a way that is secure and accessible to all users by outlining the concrete challenges for people who are affected by various types of impairments. This research investigates the implications of \ac{2fa} on digital inclusivity and proposes solutions to enhance accessibility. An analysis was conducted to examine the implementation and availability of various \ac{2fa} methods across popular online platforms. The results reveal a diverse landscape of authentication strategies. While \ac{2fa} significantly improves account security, its current adoption is hampered by inconsistencies across platforms and a lack of standardised, accessible options for users with disabilities. Future advancements in \ac{2fa} technologies, including but not limited to autofill capabilities and the adoption of \ac{fido} protocols, offer possible directions for more inclusive authentication mechanisms. However, ongoing research is necessary to address the evolving needs of users with disabilities and to mitigate new security challenges. This paper proposes a collaborative approach among stakeholders to ensure that security improvements do not compromise accessibility. It promotes a digital environment where security and inclusivity mutually reinforce each other.
\end{abstract}