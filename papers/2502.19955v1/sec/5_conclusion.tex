\section{Limitations}
\label{sec:limitations}
Our co-visibility map generation pipeline lacks robustness to instance changes; for example, when one car is replaced by another, the depth check and normal check may still be satisfied, leading to the region being incorrectly considered co-visible. Some examples are in the Supplementary Material.

\section{Conclusion}
\label{sec:conclusion}
We introduced \BN, a novel benchmark that provides a systematic way to evaluate camera pose estimation methods across well-defined geometric challenges. By organizing 16.5K image pairs into 33 difficulty levels based on overlap, scale ratio, and viewpoint angle, our benchmark revealed several important insights. First, recent detector-free methods (DUSt3R, MASt3R, RoMa) significantly outperform traditional approaches, achieving success rates above 47\%, but at the cost of higher computational requirements (150-600ms vs. 40-70ms for detector-based methods). Second, even the best performing methods (DUSt3R and MASt3R) fail to correctly estimate poses more than 45\% of image pairs, highlighting significant room for improvement, particularly in challenging scenarios combining low overlap, large scale differences, and extreme viewpoint changes.
By providing a fine-grained understanding of method limitations, \BN opens new perspectives for developing more robust pose estimation approaches, particularly for challenging geometric configurations that current methods struggle with.

\section*{Acknowledgment}
This project has received funding from the Bosch Research Foundation
(Bosch Forschungsstiftung), and was granted access to the HPC resources of IDRIS under the allocation 2024-AD011014905R1 made by GENCI.
