% \subsection*{A}

\subsection{Discussion:\name works if all responses are from the same LM}
% Please add the following required packages to your document preamble:
% \usepackage[table,xcdraw]{xcolor}
% Beamer presentation requires \usepackage{colortbl} instead of \usepackage[table,xcdraw]{xcolor}
\begin{table}[h]
\small
\centering
\begin{tabular}{c
>{\columncolor[HTML]{F3F4FF}}c ccc}
\hline
 &
  \textsc{\name} &
  \textsc{\begin{tabular}[c]{@{}c@{}}Qwen2.5\\ -72B\end{tabular}} &
  \textsc{\begin{tabular}[c]{@{}c@{}}Llama3.1\\ -405B\end{tabular}} &
  \textsc{\begin{tabular}[c]{@{}c@{}}Gemma\\ -it-9B\end{tabular}} \\ \hline
\textbf{LC} &
  \textbf{28.1} &
  25.8 &
  16.3 &
  26.9 \\ \hline
\textbf{WR} &
  \textbf{33.1} &
  24.3 &
  17.0 &
  20.2 \\ \hline
\end{tabular}
\caption{Alpaca-Eval2 Results On Magpie-Zoo~\cite{xu2024strongermodelsstrongerteachers}.}
\label{tab:magpie_zoo}
\end{table}
In Section~\ref{sec:experiments_ultra}, we demonstrated how \name enables practitioners to refine model-generated responses for improved training outcomes. Beyond that, \name can optimize responses from a single generator. Using the Magpie-Zoo instruction set~\cite{xu2024strongermodelsstrongerteachers}, we sample 10 responses per instruction from Qwen2.5-72B-Instruct, select in-distribution responses with \name, to train a Mistral-v0.3-7B model. We also compare with training on top-3 reported datasets of Magpie-Zoo. As shown in Table~\ref{tab:magpie_zoo}, \name-selected responses further boost the performance. Practically, batch sampling of responses cause little latency~\cite{zhong2024distservedisaggregatingprefilldecoding,zhou2024survey}. One therefore can sample batches of responses from a single strong model and apply \name on top of it to find the response closest to the base model's distribution efficiently. 

\subsection{Discussion: Is In-Distribution A Silver Bullet?}
\label{sec:self_distillation}
The success of \name verifies our key hypothesis that matching SFT distribution to the base model benefits the performance. Yet, we argue that this cannot be pushed to the limit of SFT using data from the same distribution. 
By experimenting with training on solutions sampled from a trained version of the same base model, we notice a drastic performance drop from the original model as shown in Table~\ref{tab:self_distill_ultra}. 
\begin{table}[h]
\small
\centering
\begin{tabular}{ccc}
\hline
                                       &                                        &                              \\
\multirow{-2}{*}{\textbf{Model}} & \multirow{-2}{*}{\textbf{Data}} & \multirow{-2}{*}{\textbf{Avg.}} \\ \hline
                                       & \cellcolor[HTML]{F8E8E7}Self-Distilled & \cellcolor[HTML]{F8E8E7}28.4\textcolor{red}{\textbf{(-)} } \\
\multirow{-2}{*}{\textsc{Mistral-7B}}  & Original-UI                            & 32.7                         \\ \hline
                                       & \cellcolor[HTML]{F8E8E7}Self-Distilled & \cellcolor[HTML]{F8E8E7}29.4\textcolor{red}{\textbf{(-)} } \\
\multirow{-2}{*}{\textsc{Llama3.1-8B}} & Original-UI                            & 36.8                         \\ \hline
                                       & \cellcolor[HTML]{F8E8E7}Self-Distilled & \cellcolor[HTML]{F8E8E7}15.1\textcolor{red}{\textbf{(-)} } \\
\multirow{-2}{*}{\textsc{Llama3.2-3B}} & Original-UI                            & 20.3                         \\ \hline
\end{tabular}

\caption{Performance Degradation From Self-Distillation On UI.}\
\label{tab:self_distill_ultra}
\end{table}
We explored this using the \textbf{MATH} \cite{hendrycks2021measuringmathematicalproblemsolving} dataset. A base model was fine-tuned using solutions from stronger models like \textsc{Llama3.1-70B-Instruct} or GPT-3.5-Turbo-augmented \textbf{MetaMathQA} \cite{yu2024metamath}. This fine-tuned model then generated new solutions, which were used to further fine-tune another base model instance, simulating iterative on-policy tuning.
\begin{table}[h]
\centering
\small
\begin{tabular}{ccccc}
\hline
\textbf{Dataset} & \textbf{Model} & \textbf{Response Generator} & \textbf{N} & \textbf{Acc.} \\ \hline
MATH & \textsc{Mistral} & Llama3.1-70B      & 10 & 18.2     \\
\rowcolor[HTML]{F8E8E7} 
MATH & \textsc{Mistral} & FT-Mistral & 10 & 15.9 \textcolor{red}{\textbf{(-)} }\\ \hline
MATH & \textsc{Llemma}  & Llama3.1-70B      & 10 & 26.2     \\
\rowcolor[HTML]{F8E8E7} 
MATH & \textsc{Llemma}  & FT-Llemma  & 10 & 23.6 \textcolor{red}{\textbf{(-)} } \\ \hline
MATH & \textsc{Mistral} & MM-AnsAug   & -  & 22.3     \\
\rowcolor[HTML]{F8E8E7} 
MATH & \textsc{Mistral} & FT-Mistral & 10 & 20.6 \textcolor{red}{\textbf{(-)} } \\ \hline
MATH & \textsc{Llemma}  & MM-AnsAug   & -  & 28.1     \\
\rowcolor[HTML]{F8E8E7} 
MATH & \textsc{Llemma}  & FT-Llemma  & 10 & 21.4 \textcolor{red}{\textbf{(-)} }\\ \hline
\end{tabular}
\caption{Effect of self-generation on MATH dataset. \textsc{FT-Mistral} refers to the model right above that row finetuned from either MM-AnsAug or Llama3.1-70B-Instruct produced solutions. N stands for the number of responses sampled.}
\label{tab:self_gen_math}
\end{table}
As shown in Table~\ref{tab:self_gen_math}, performance declined when solutions were sampled from the fine-tuned model itself. This decline stems from reduced solution diversity, leading to distributional drift and poorer generalization. 

These results validate our claim in \S\ref{sec:on_policy} that robust fine-tuning requires more than alignment with the base model—it needs complementary strategies to maintain response diversity and quality. 