\begin{table}[h]
\centering
\small
\begin{tabular}{ccccc}
\hline
\textbf{Dataset} & \textbf{Model} & \textbf{Response Generator} & \textbf{N} & \textbf{Acc.} \\ \hline
MATH & \textsc{Mistral} & Llama3.1-70B      & 10 & 18.2     \\
\rowcolor[HTML]{F8E8E7} 
MATH & \textsc{Mistral} & FT-Mistral & 10 & 15.9 \textcolor{red}{\textbf{(-)} }\\ \hline
MATH & \textsc{Llemma}  & Llama3.1-70B      & 10 & 26.2     \\
\rowcolor[HTML]{F8E8E7} 
MATH & \textsc{Llemma}  & FT-Llemma  & 10 & 23.6 \textcolor{red}{\textbf{(-)} } \\ \hline
MATH & \textsc{Mistral} & MM-AnsAug   & -  & 22.3     \\
\rowcolor[HTML]{F8E8E7} 
MATH & \textsc{Mistral} & FT-Mistral & 10 & 20.6 \textcolor{red}{\textbf{(-)} } \\ \hline
MATH & \textsc{Llemma}  & MM-AnsAug   & -  & 28.1     \\
\rowcolor[HTML]{F8E8E7} 
MATH & \textsc{Llemma}  & FT-Llemma  & 10 & 21.4 \textcolor{red}{\textbf{(-)} }\\ \hline
\end{tabular}
\caption{Effect of self-generation on MATH dataset. \textsc{FT-Mistral} refers to the model right above that row finetuned from either MM-AnsAug or Llama3.1-70B-Instruct produced solutions. N stands for the number of responses sampled.}
\label{tab:self_gen_math}
\end{table}