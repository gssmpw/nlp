% This must be in the first 5 lines to tell arXiv to use pdfLaTeX, which is strongly recommended.
\pdfoutput=1
% In particular, the hyperref package requires pdfLaTeX in order to break URLs across lines.

\documentclass[11pt]{article}

% Change "review" to "final" to generate the final (sometimes called camera-ready) version.
% Change to "preprint" to generate a non-anonymous version with page numbers.
\usepackage[final]{acl}

% Standard package includes
\usepackage{times}
\usepackage{latexsym}
\usepackage{url}

% For proper rendering and hyphenation of words containing Latin characters (including in bib files)
\usepackage[T1]{fontenc}
% For Vietnamese characters
% \usepackage[T5]{fontenc}
% See https://www.latex-project.org/help/documentation/encguide.pdf for other character sets

% This assumes your files are encoded as UTF8
\usepackage[utf8]{inputenc}

% This is not strictly necessary, and may be commented out,
% but it will improve the layout of the manuscript,
% and will typically save some space.
\usepackage{microtype}

% This is also not strictly necessary, and may be commented out.
% However, it will improve the aesthetics of text in
% the typewriter font.
\usepackage{inconsolata}

%Including images in your LaTeX document requires adding
%additional package(s)
\usepackage{graphicx}

% Standard package includes
\usepackage{times}
\usepackage{booktabs}
\usepackage{latexsym}
\usepackage{algorithm}
\usepackage{multirow, multicol}

%\usepackage{algorithmic}
\usepackage{bbm}
\usepackage{algpseudocode}
\usepackage{amsmath}
\usepackage{soul}
\usepackage{pifont}
\usepackage{paralist}
\usepackage[most]{tcolorbox}
\definecolor{green1}{RGB}{215, 255, 215}  % Medium light green
\definecolor{green2}{RGB}{138, 209, 138}  % Medium green

\newcommand{\sa}[1]{\textcolor{red}{\{#1 -- SA\}}}
\newcommand{\mo}[1]{\textcolor{red}{#1 -- MC}}

\newcommand{\skp}[1]{\textcolor{blue}{#1 -- SKP}}
\newcommand{\sv}[1]{\textcolor{blue}{#1 -- SV}}
\newcommand{\dd}[1]{\textcolor{blue}{#1 -- DD}}


\algnewcommand\algorithmicinput{\textbf{Input:}}
\algnewcommand\INPUT{\item[\algorithmicinput]}

\algnewcommand\algorithmicrepresentation{\textbf{Representation:}}
\algnewcommand\REPRESENTATION{\item[\algorithmicrepresentation]}

\algnewcommand\algorithmicoutput{\textbf{Output:}}
\algnewcommand\OUTPUT{\item[\algorithmicoutput]}


\usepackage{tikz}
\newcommand*\circled[1]{\tikz[baseline=(char.base)]{
        \node[shape=circle,draw,inner sep=1pt] (char) {#1};}}


\usepackage{graphicx}
\usepackage{amssymb}
\usepackage{inconsolata}

% If the title and author information does not fit in the area allocated, uncomment the following
%
%\setlength\titlebox{<dim>}
%
% and set <dim> to something 5cm or larger.

\title{SMAB: MAB based word Sensitivity Estimation Framework and its Applications in Adversarial Text Generation}

% Author information can be set in various styles:
% For several authors from the same institution:
% \author{Author 1 \and ... \and Author n \\
%         Address line \\ ... \\ Address line}
% if the names do not fit well on one line use
%         Author 1 \\ {\bf Author 2} \\ ... \\ {\bf Author n} \\
% For authors from different institutions:
% \author{Author 1 \\ Address line \\  ... \\ Address line
%         \And  ... \And
%         Author n \\ Address line \\ ... \\ Address line}
% To start a separate ``row'' of authors use \AND, as in
% \author{Author 1 \\ Address line \\  ... \\ Address line
%         \AND
%         Author 2 \\ Address line \\ ... \\ Address line \And
%         Author 3 \\ Address line \\ ... \\ Address line}

% \author{First Author \\
%   Affiliation / Address line 1 \\
%   Affiliation / Address line 2 \\
%   Affiliation / Address line 3 \\
%   \texttt{email@domain} \\\And
%   Second Author \\
%   Affiliation / Address line 1 \\
%   Affiliation / Address line 2 \\
%   Affiliation / Address line 3 \\
%   \texttt{email@domain} \\}

\author{
\textbf{Saurabh Kumar Pandey\textsuperscript{1*}},
  \textbf{Sachin Vashistha\textsuperscript{2*}},
  \textbf{Debrup Das\textsuperscript{3}},
\\
  \textbf{Somak Aditya\textsuperscript{2}},
  \textbf{Monojit Choudhury\textsuperscript{1}}
  \\ 
  \texttt{saurabh2000.iitkgp@gmail.com, sachinvashistha6916@gmail.com,}
\\
  \texttt{saditya@cse.iitkgp.ac.in, monojit.choudhury@mbzuai.ac.ae}
   \\
  \textsuperscript{1}MBZUAI, 
  \\
  \textsuperscript{2}Indian Institute of Technology, Kharagpur
  \\
  \textsuperscript{3}University of Massachusetts Amherst 
%  \textsuperscript{5}Affiliation 5
%\\
}

%\author{
%  \textbf{First Author\textsuperscript{1}},
%  \textbf{Second Author\textsuperscript{1,2}},
%  \textbf{Third T. Author\textsuperscript{1}},
%  \textbf{Fourth Author\textsuperscript{1}},
%\\
%  \textbf{Fifth Author\textsuperscript{1,2}},
%  \textbf{Sixth Author\textsuperscript{1}},
%  \textbf{Seventh Author\textsuperscript{1}},
%  \textbf{Eighth Author \textsuperscript{1,2,3,4}},
%\\
%  \textbf{Ninth Author\textsuperscript{1}},
%  \textbf{Tenth Author\textsuperscript{1}},
%  \textbf{Eleventh E. Author\textsuperscript{1,2,3,4,5}},
%  \textbf{Twelfth Author\textsuperscript{1}},
%\\
%  \textbf{Thirteenth Author\textsuperscript{3}},
%  \textbf{Fourteenth F. Author\textsuperscript{2,4}},
%  \textbf{Fifteenth Author\textsuperscript{1}},
%  \textbf{Sixteenth Author\textsuperscript{1}},
%\\
%  \textbf{Seventeenth S. Author\textsuperscript{4,5}},
%  \textbf{Eighteenth Author\textsuperscript{3,4}},
%  \textbf{Nineteenth N. Author\textsuperscript{2,5}},
%  \textbf{Twentieth Author\textsuperscript{1}}
%\\
%\\
%  \textsuperscript{1}Affiliation 1,
%  \textsuperscript{2}Affiliation 2,
%  \textsuperscript{3}Affiliation 3,
%  \textsuperscript{4}Affiliation 4,
%  \textsuperscript{5}Affiliation 5
%\\
%  \small{
%    \textbf{Correspondence:} \href{mailto:email@domain}{email@domain}
%  }
%}

\begin{document}
\maketitle

\setlength{\abovedisplayskip}{1pt}
\setlength{\belowdisplayskip}{1pt}

\begin{abstract}
\begin{abstract}
Retrieval-Augmented Generation (RAG) is often used with Large Language Models (LLMs) to infuse domain knowledge or user-specific information. In RAG, given a user query, a retriever extracts chunks of relevant text from a knowledge base. These chunks are sent to an LLM as part of the input prompt. Typically, any given chunk is repeatedly retrieved across user questions. However, currently, for every question, attention-layers in LLMs fully compute the key values (KVs) repeatedly for the input chunks, as state-of-the-art methods cannot reuse KV-caches when chunks appear at arbitrary locations with arbitrary contexts. Naive reuse leads to output quality degradation.  This leads to potentially redundant computations on expensive GPUs and increases latency. In this work, we propose \sys, a system for managing and reusing precomputed KVs corresponding to the text chunks (we call \textit{chunk-caches}) in RAG-based systems. We present how to identify \hl{\textit{chunk-caches} that are reusable}, how to efficiently perform a small fraction of recomputation to \textit{fix} the cache to maintain output quality, and how to efficiently store and evict \textit{chunk-caches} in the hardware for maximizing reuse while masking any overheads. With real production workloads as well as synthetic datasets, we show that \sys reduces redundant computation by \textbf{51\%} over SOTA prefix-caching and \textbf{75\%} over full recomputation.
\hl{Additionally, with continuous batching on a real production workload, we get a \textbf{1.6$\times$} speedup in throughput and a \textbf{2$\times$} reduction in end-to-end response latency over prefix-caching while maintaining quality, for both the \llama-3-8B and \llama-3-70B models. 
}
\end{abstract}





\let\thefootnote\relax\footnotetext[1]{* indicates equal contribution, Order chosen at random}
\end{abstract}

% \iftaclpubformat

\section{Introduction}
\documentclass[../main.tex]{subfiles}
\graphicspath{{../images/}}
\makeatletter
\def\input@path{{../images/}}
\makeatother
\begin{document}
\section{Introduction}
\begin{figure}
\centering
\begin{tikzpicture}
\node[inner sep=0pt] (ws) at (0, 0) {
\includegraphics[height=.4\textwidth, trim={10cm 0 10cm 0},clip]{world_space.png}};
\node[inner sep=0pt] (cs) at (6,0) {\includegraphics[height=.4\textwidth, trim={10cm 1cm 10cm 4cm},clip]{conf_space.png}};
\end{tikzpicture}
\vspace{-5pt}
\label{fig:pbrm_intro}
\caption{\textbf{Left}: Shows world space obstacles as grey spheres. Robots start and goal configuration is colored red and green, respectively. Configurations along the computed path are colored transparent blue. \textbf{Right:} Mapped world space scenario to configuration space. Obstacle region is the grey mesh. Red spheres are collision-free regions computed by the neural SCDF. The optimized shortest path in the convex corridor is the blue curve.}
\vspace{-25pt}
\end{figure}
Motion planning is the problem of finding a collision-free trajectory that connects a given start and goal configuration. The planning takes place in the configuration space of the robot. For single body robots, like mobile robots or drones, the configuration space and the world space are usually the same. This simplifies the planning, since explicit obstacle representations are available which enables geometrical tools like separating hyperplanes, smallest distance to obstacles etc., to be used when designing motion planning algorithms. For multi-body robots like manipulators, the situation is completely different. The world space obstacles are usually mapped to non-convex regions, and to make the problem even harder, the mapping is usually not known. Forming explicit representations of the obstacle region in the configuration space is usually too expensive or intractable. Despite all of this, sampling based planners are used with great success, which mainly is due to their use of implicit representations of the obstacle region. The basic idea is to construct a graph in the configuration space that covers and connects the collision-free region. From this graph, a path can be extracted that connects a given start and goal configuration. The approach is computationally expensive, since the graph is constructed with the smallest geometrical building block available, points, which represents a collision-check. Furthermore, the extracted paths from the graph are non-smooth and jagged due to the stochastic nature of the approach. This adds an additional post-processing step to the process, where the paths are shortcutted and smoothened, before the path can be used for tracking. Clearly a lot of time is invested to form this graph and produce smooth paths. Thus, if the obstacles start to move, then all of this work is done in no use, since all points that make up this graph need to be re-verified, which is simply too time consuming to be done in real time.
\\\\
In this work, we want to address the existing drawbacks of the sampling based planners. Our main contribution is an improved motion planner where each vertex in the graph covers a collision-free region in the form of a sphere instead of a point and where the edges are formed with neighboring intersecting spheres. This representation has the advantage of instead of returning piecewise linear paths, returning a sequence of overlapping spheres, i.e. a convex corridor, that connects a given start and goal configuration, illustrated in Figure \ref{fig:pbrm_intro}. This convex corridor allows us to use convex optimization to produce smooth trajectories, instead of computationally expensive post-processing methods. The representation further allows us to estimate the coverage of the collision-free space, which gives us awareness and feedback in the offline roadmap construction phase. Finally, our representation is simple to adapt to moving obstacles, simply requery for the new radii and recheck for intersections. 
\\\\
The spherical collision-free regions are formed using a signed distance function (SDF), which is a function that returns the smallest distance from an arbitrary point to the boundary of an obstacle. As the name implies, the distance is signed, thus if the point is inside the obstacle it is negative otherwise positive. If the distance is positive, a sphere with radius equal to the distance is guaranteed to cover a collision-free region. Using an SDF in motion planning is not new, but what is novel about our approach is that we express the distance in the configuration space instead of the world space and by doing so allows us to form these convex collision-free regions. We refer to the resulting SDF as a signed configuration distance function (SCDF). Computing an SCDF analytically is non-trivial, our approach is therefore to parameterize the SCDF with a deep neural network and learn the mapping by supervised learning. Our resulting neural SCDF can compute distances for different parameter values of obstacle shapes and we also show how multiple distances can be combined, thus making our approach flexible.
\section{Related work}
Motion planning algorithms can roughly be divided into three families, grid-based, sampling based and optimization based methods. Grid-based methods (GBM) discretize the planning space from which a graph is then compiled. A standard search method is A$^\star$ \citep{a_star}, which is classified as an \textit{informed} search method, since it employs a heuristic function to speed up the search. A$^\star$ guarantees to return an optimal path at the level of discretization used. GBMs usually discretize the planning space by a regular lattice and this limits the GBMs to problems with low dimensionality due to the curse of dimensionality. Thus, GBMs are usually limited to single-body robots where the degrees of freedom (DOF) are low. To overcome the inherent scaling problem with the GBMs, stochastic methods are usually used for multi-body robots. These methods are termed as sampling-based methods (SBM) and core members within this family are the rapidly-exploring random trees (RRT) \citep{rrt} and the probabilistic roadmap (PRM) \citep{prm}. RRT grows a tree from the start configuration and explores the collision-free region in a rapid way until it is able to connect to the goal region. RRT is usually improved by bi-directional planning \citep{rrt_connect}, i.e. an additional tree is grown from the goal configuration and the trees are tested for connection after any tree has been expanded. RRT is a single-query method, thus it searches for a path from scratch each time it is queried. Contrary to this, PRM is a multi-query method, which solves for multiple queries without starting from scratch. PRM does this by creating a roadmap (graph) that covers the collision-free space as an offline step. The graph is then used to solve for multiple queries. PRMs are used in cases where the environment does not change since the extra offline step is too computationally costly and needs to be re-done if the environment is changed. In our work, we address this inherent issue by using a different roadmap representation. Our vertices in the graph cover a collision-free region in the form of spheres and we form the edges by checking for intersecting spheres. If something in the environment changes, we recompute the spheres radii and recheck the intersections, without relying on collision detection. We use a trained neural network to compute the sphere radius, therefore querying for the radius can be done fast, hence our representation enables the PRM for dynamic environments.
\\\\
In the recent decades, optimization based methods (OBM) \citep{chomp, schulman, itomp, stomp} have been introduced as an alternative to SBM for multi-body robots. Like the SBM, the OBMs scale well to higher dimensional problems and produce smoother motion. It is common to use a SDF in the optimization since it is a smooth function, thus enabling gradient-based methods. However, the standard way of expressing the SDF is in world space. The distance therefore needs to be mapped to the configuration space by the forward kinematics. This mapping makes the optimization problem a non-linear program (NLP), which is computationally expensive to solve. Recently, a different approach has been proposed. In \cite{mp_gcs} motion planning is formulated as a convex optimization problem by using the graph of convex sets framework \citep{gcs}. The underlying idea is to decompose the collision-free space into intersecting convex sets from which a convex optimization problem is formulated. In cases where an explicit representation of the obstacles in the configuration space exists, like for single-body robots, creating collision-free convex regions can be done fast \citep{iris}. For multi-body robots, this is non-trivial. Existing work does this successfully \citep{iris_nlp, iris_c} by an optimization based approach, but the methods are still too time consuming to be used in the presence of moving obstacles. Our approach is instead to use deep learning to learn an SDF expressed in the configuration space. With this, we can query for shortest distances to the collision boundary, which allows us to expand spherical regions which are collision-free. Our approach is fast and therefore enables our suggested roadmap planner to be used in dynamic environments.
\\\\
Recent research has focused on learning collision detection \citep{fk_kernel_distance, diffco, graphdistnet} by predicting the signed distance between the robot links and the surrounding obstacles in the world space. The learned SDF is used in trajectory optimization but since the distance is expressed in the world space, the problem becomes an NLP and therefore takes a long time to solve. We take a novel approach and suggest to instead express the signed distance in the configuration space. This allows us to improve the PRM at the same time as it enables convex optimization for trajectory optimization, which runs faster and is more reliable than NLP solvers. In \cite{cspf} a learned signed distance function in the configuration space is proposed similar to our approach. However, their approach is restricted to point cloud representations, while we propose to represent the obstacles as parameterized geometric shapes, e.g. spheres. Furthermore, we also show how to use our learned SCDF to improve an existing roadmap planner.
\section{Problem formulation}
A robot is located in the world space, $\W \subset \R^3 $. The unique location of the robot is given by its configuration $\q \in \C$, where $\C$ is the configuration space. The set of points covered by the robots bodies at a certain configuration is expressed as $\B(\q) \subset \W$. The robot is surrounded by $\NrObst$ obstacles $\O = \bigcup_{i=1}^{\NrObst} \O_i$, where  $\O_i \subset \W$. The representation of the obstacle in the configuration space is the set $\C\O_i = \{\q \in \C \: |\: \B(\q) \cap \O_i \neq \emptyset \}$. The obstacle space is formed as $\Co = \bigcup_{i=1}^{\NrObst} \C \O_i$. The complement is referred to as the free space, $\Cf = \C \setminus \Co$. The path planning problem is a tuple, ($\Cf$, $\qStart$, $\qGoal$), where we want to connect a query pair, consisting of a start, $\qStart$, and goal configuration, $\qGoal$, with a geometric path, $\q(s): [0, 1] \mapsto \Cf$, such that $\q(0)=\qStart$ and $\q(1)=\qGoal$, or report correctly when such a path does not exist.
\end{document}


\section{Methodology}

\section{\label{sec:method}Methodology}

Each SIEM system uses its own RDL to define threat detection rules, and each RDL has its own schema.
For example, the Splunk SIEM uses the SPL to define its threat detection rules.
The task of understanding threat detection rules and recommending relevant MITRE ATT\&CK techniques (or sub-techniques) requires complex reasoning skills.
In the case of LLMs, this can be achieved with a technique called prompt chaining in which each task is divided into multiple sub-tasks in order to understand the complex reasoning behind the task.
Therefore, we employ a multi-phase architecture based on prompt chaining that leverages the power of LLMs to take a SIEM rule defined in any RDL and map it to relevant MITRE ATT\&CK techniques using the power of LLMs.
Our approach is based on the following intuitions:
\begin{itemize}[nosep,leftmargin=*]
    \item \textit{LLMs' implicit knowledge}: LLMs possess deep understanding of diverse RDLs. This enables them to interpret any rule, regardless of the RDL it is defined in, and convert it into comprehensible natural language text.
    \item \textit{LLMs' similarity comparison capability}: LLMs are adept at analyzing and comparing textual descriptions. 
    They can intelligently assess the similarity between two textual inputs to establish a meaningful connection.
\end{itemize}

\methodName has two main phases: (1) the rule to text translation phase, and (2) the MITRE ATT\&CK techniques recommendation phase.
These two phases in the pipeline include six key steps to determine relevant TTPs, as illustrated in Figure~\ref{fig:r2t}.

Although LLMs excel at translating SIEM rules into natural language, they often lack critical domain-specific contextual information related to IoCs in the rules.
To overcome this limitation, the \textit{rule to text translation} phase includes three steps: IoC extraction, contextual information retrieval, and natural language translation.

The workflow begins with the extraction of IoCs from the rules (for example, processes, log source, event codes, and file names) that the rule searches for in the logs (step (1)).In the next sstep a web search agent performs the task of obtaining additional contextual information about the IoCs discovered ((step 2)).
By incorporating this additional domain-specific information, the pipeline enhances the language translation, resulting in a more accurate and meaningful interpretation of SIEM rules.
The rule itself and the IoCs' contextual additional information from the previous stage are then used to translate the rule from RDL to natural language (step (3)).

The \textit{MITRE ATT\&CK techniques} recommendation phase of the pipeline includes the following three steps.
The rule in processed in data source identification step in which the probable origin of the data is identified (step (4)).
The description of the rule is then used to determine probable MITRE ATT\&CK techniques based on the implicit knowledge of the LLM (step (5)).
Finally, using chain-of-thought~\cite{wei2022chain} prompting, the most relevant techniques are extracted from the list of probable techniques (step (6)).
Each step of our method is further described in detail below.


% [bb=0 0 1440 900,width=1.43\linewidth,height=0.9\textwidth]
\begin{figure*}[htbp]
   \includegraphics[width=\textwidth]{Images/stages.jpg}
    
   \caption{An illustration of the different steps in \methodName.}
   \label{fig:stages}
\end{figure*} 

\subsection{IoC Extraction}
The context associated with a SIEM detection rule is crucial for its accurate interpretation and effective application. 
Obtaining this contextual understanding requires comprehensive analysis of the embedded IoCs in the SIEM rule.
In the first step, \methodName systematically identifies and extracts all IoCs, identifying the types of IoCs and their corresponding values that form the foundational elements of the detection rules. 
Leveraging the LLM's inherent understanding of rule structures and IoCs, we employ a zero-shot prompting approach for this task. 
Zero-shot prompting enables the direct extraction of IoCs from the rules without requiring extensive pre-training on specific datasets.

\noindent The result of this stage is a dictionary structure, where:
\begin{itemize}[nosep,leftmargin=*]
    \item Keys represent types of IoC, such as processes, files, IP addresses, and log sources.
    \item Values are lists containing specific IoC details, such as process names, file names, IP addresses, and log source identifiers.
\end{itemize}

In the example depicted in Figure~\ref{fig:stages}(a), the pipeline processes a rule for which relevant MITRE ATT\&CK techniques need to be recommended. 
The IoC extractor LLM produces a dictionary structure as output, organizing the IoCs in a structured format to support subsequent stages in the analysis pipeline. 



\subsection{Contextual Information Retrieval}
In this step, an LLM agent is employed to retrieve relevant information pertaining to the IoCs extracted from the rule.
A REACT agent~\cite{react} was used in this case to generate both reasoning traces and task-specific actions in an interleaved manner.
REACT agents interact with external tools to retrieve additional information that leads to more factual and reliable responses.
The LLM agent conducts a systematic search across web resources to gather additional contextual information for each IoC value present in the rule. 
This step addresses LLMS' lack of up-to-date knowledge or specialized domain expertise (which is critical to understanding the role and significance of the IoCs in the rule), without the need for retraining or fine-tuning.
Figure~\ref{fig:stages}(b) presents an example in which the rule includes the process name \texttt{soaphound.exe} as an IoC.
As can be seen, the web search results indicate that \texttt{soaphound.exe} is being used for active directory (AD) enumeration, which is important for the understanding of the attack. 

\subsection{Natural Language Translation}

The translation of detection rules into natural language textual descriptions fulfills three key objectives:
\begin{enumerate}[nosep,leftmargin=*]
    \item \textbf{Ensures that \methodName is format-agnostic}: It converts rules defined in various RDL formats into a generic, unstructured text format, ensuring compatibility with different SIEM systems, regardless of the specific rule format.
    \item \textbf{Provides contextual explanation}: It includes all relevant contextual information to produce a concise and comprehensible explanation of the rule.
    \item \textbf{Enhances the comprehension for LLMs}: It enables LLMs to more effectively compare the translated rule with descriptions in the MITRE ATT\&CK framework by providing a unified textual representation.
\end{enumerate}
To achieve these objectives, a zero-shot prompting technique is employed. 
The input to the LLM comprises two components:
\begin{itemize}
    \item \textbf{Syntactical information}: The rule itself, providing the structural and operational details.
    \item \textbf{Contextual information}: Details of the IoCs extracted from the rule, providing semantic insights into the rule's intent and function.
\end{itemize}
The LLM utilizes these inputs to generate a natural language textual description of the rule. 
This transformation not only ensures a more interpretable representation but also facilitates further steps of analysis and comparison, particularly in aligning the rule with MITRE ATT\&CK techniques and sub-techniques.



\subsection{Data Source or Mitigation Identification}
Identifying the most relevant data component or mitigation associated with the rule description in this step is critical for filtering out irrelevant MITRE ATT\&CK techniques (or sub-techniques) in subsequent steps of the pipeline.
In the MITRE ATT\&CK framework, data sources represent various categories of information that can be gathered from sensors or logs. 
These data sources include \textit{data components}, which are specific attributes or properties within a data source that are directly relevant to detecting a particular technique or sub-technique~. 
For example, in the context of the rule described in Figure~\ref{fig:stages}(a), the term \texttt{Endpoint.Processes} indicates that the activity is happening on an endpoint. 
Presence of the terms such as, \texttt{soaphound.exe}, \texttt{--buildcache}, \texttt{--certdump} and etc. indicate that the rule searches for command line execution of an executable named \texttt{soaphound.exe} with specific parameters. 
Therefore, the appropriate data source in this example is \textit{Command}, with the corresponding data component being \textit{Command Execution}.
Additionally, \textit{mitigations} are defined as categories of technologies or strategies that can prevent or reduce the impact of specific techniques or sub-techniques. 
The MITRE ATT\&CK framework explicitly establishes relationships between data components, mitigations, and techniques (or sub-techniques), enabling a systematic approach for identifying relevant elements.

To identify the most relevant data component or mitigation associated with a given rule description, we utilize agentic retrieval augmented generation (RAG), which incorporates an AI Agent-based implementation of the RAG framework.
Data from the MITRE ATT\&CK framework, specifically related to data components and mitigations, is stored in a vector database (e.g., ChromaDB). 
The process begins with the rule description from the previous stage, which serves as the input to the AI Agent. 
The LLM-powered agent automatically generates a search query tailored to retrieve relevant information from the RAG database.

For each query, the system retrieves the five most similar documents from the database, each containing contextual information about data components or mitigations. 
These documents are then utilized by the LLM agent to contextualize the rule description. 
By comparing the content of these retrieved documents with the rule description, the LLM agent determines and outputs the most relevant data component or mitigation along with a chain-of-thought as to why the data component or mitigation is related to the rule.


\subsection{Probable Technique Recommendation}

In this step, an LM agent is utilized to propose probable MITRE ATT\&CK techniques (and sub-techniques) that may be relevant to the description of the provided rule.
We used a REACT agent in this step as well to utilize both implicit and explicit knowledge during reasoning.
For explicit knowledge, the agent searches the MITRE ATT\&CK framework to obtain the list of probable techniques (and sub-techniques).
The natural language description of the rule from the previous step serves as input to the LLM agent.
The output of this stage consists of a list of JSON objects, each containing the MITRE technique ID, technique name, and technique description as seen in Figure~\ref{fig:stages}(c).

Throughout our experiments, we observed that as the number of recommendations ($k$) increases, both the framework's average recall and precision initially improve, however beyond a certain threshold of $k$, the %average 
precision begins to decline.
Based on these observations(please refer Table~\ref{tab:results3}), we selected a $k$-value of 11 to ensure a high recall.



\subsection{Relevant Technique Extraction}
In this step, \methodName refines the set of probable MITRE ATT\&CK techniques identified in the previous stage by eliminating irrelevant entries.
This step in the pipeline serves two primary purposes: (1) to enhance precision while maintaining recall achieved in previous step, and (2) to provide a clear rationale for the selection of the labels, ensuring transparency and interpretability of the mapping process.
This refinement process is grounded in the assumption that LLMs are effective for text similarity matching tasks.

The process comprises two key steps:
\begin{itemize}
    \item \textit{Rule-technique comparison}: The description of each technique in the set of probable techniques is compared with the rule description. 
    A chain-of-thought technique is then applied to elucidate the reasoning behind the association of each technique with the rule.
    \item \textit{Confidence calculation}: The generated chain-of-thought rationale for each technique (or sub-technique) is compared with the rule description to compute a relevance (or confidence) score, as done in prior work~\cite{freitas2024ai}.
    % \item \textbf{Reasoning}: \new{Add here the reasoning that it provides - explaining in NLP why it was selected...}
\end{itemize}

Techniques with higher confidence scores are deemed more relevant to the rule. 
Conversely, techniques with scores falling below a predefined threshold are excluded.
The techniques retained after this filtering step represent the most relevant techniques corresponding to the given rule's description. 


The chain-of-thought (CoT) rationale generated during the comparison of each rule to its probable technique is also provided as an output in this step.
This rationale offers a detailed natural language explanation, articulating why a particular technique is relevant to the given rule. 
Such explanations are highly valuable for security analysts, as they provide clear and transparent reasoning behind the mapping, enabling analysts to better understand and validate the association between the rule and the technique.
Other classification models proposed in previous works within this domain also suffer from the limitation of being black-box models, which lack the ability to provide clear reasoning or explanations. 
Unlike \methodName, these models fail to generate transparent, CoT rationales that explain why a particular rule is mapped to a specific technique, making them less interpretable and less useful for security analysts.

\section{SMAB: A Case Study on \textsc{CheckList}}
\begin{figure}
    \centering
    \includegraphics[width=\linewidth]{figures/MCQA_checklist.pdf}
    \vspace{-4.75ex}
    \setlength{\fboxsep}{0pt}
    \caption{\small Example unanswerable MCQ from MMLU \cite{gema2024we}, along with rubric criteria from \citet{haladyna1989taxonomy} flagged by OpenAI's o1 \cite{jaech2024openai}.}
    \label{fig:checklist}
    \vspace{-1.7ex}
\end{figure}

\section{Sensitivity as a Proxy for Accuracy}
\begin{tikzpicture}
    \begin{axis}[
        width=\linewidth,
        ylabel style={font=\scriptsize,yshift=-0.6em},
        y tick label style={font=\scriptsize},
        x tick label style={font=\scriptsize},
        ybar,
        %axis lines=left,  
        ymajorgrids,
        symbolic x coords={XGBoost, gMLP, PedCA-FT},
        %xtick={XGBoost, LightGBM, {ours}},
        ylabel={Sensitivity},
        ymin=0,
        ymax=55,
        bar shift=0pt,
        %bar width=0.5cm,
        nodes near coords, 
        nodes near coords style={font=\scriptsize}, 
        %enlargelimits=0.10,
    ]
        \addplot[
            fill=Set2-A,
            ybar,
            error bars/.cd,
            y dir=both,
            y explicit,
        ] coordinates {
            (XGBoost, 39.04) += (0, 8.1) -= (0, 7.53)
        };
        \addplot[
            fill=Set2-B,
            ybar,
            error bars/.cd,
            y dir=both,
            y explicit,
        ] coordinates {
            (gMLP, 8.22) += (0, 5.6) -= (0, 3.46)
        };
        \addplot[
            fill=Set2-C,
            ybar,
            error bars/.cd,
            y dir=both,
            y explicit,
        ] coordinates {
            (PedCA-FT, 42.47) += (0, 8.11) -= (0, 4.73)
        };
    \end{axis}
\end{tikzpicture}

\section{Adversarial example generation}
\section{Toward Multi-dimensional Concept of Safety Fine-tuning Vulnerabilities}
\label{sec:application}

Previous analysis presents a multi-dimensional framework for understanding learned safety behaviors, where distinct features and dynamics emerge along different directions in residual space. In this section, we demonstrate how this framework provides practical insights into safety fine-tuning vulnerabilities by showing manipulating non-dominant directions can bypass learned safety capabilities. We explore two methods to circumvent the learned safety capabilities while preserving the model's refusal ability: (1) suppressing non-dominant components and (2) removing or rephrasing trigger tokens from jailbreak prompts. Here, we define "trigger tokens" as specific token sequences that induce changes in feature directions, as demonstrated in \autoref{tab:plrp_logitlens}.

\paragraph{Suppressing Non-Dominant Directions}
As shown in \autoref{iterpret_tokens}, removing \texttt{L14-C6} explains the model's learned ability to refuse PAIR-like jailbreaks. Building on this insight, we investigate the effect of suppressing most non-dominant components while leaving dominant components untouched. Formally:

\[
    \mathbf{x} := \mathbf{x} - \sum_{v_i \in V^{t:}} \alpha_i \mathbf{v}_i
    \label{eq:intervene_all}
\]

This approach allows us to examine whether safety alignment can be reversed by blocking only indirect features. To preserve the model's ability to refuse plainly harmful prompts, we exclude component directions with harmfulness correlations above 0.7. 


\paragraph{Trigger Removal Attack}
We next introduce a procedure to remove trigger tokens from jailbreaks. First, we apply token-wise PLRP to dominant directions of the final layers to identify a list of top trigger tokens that explain the refusal output. Then, we employ another LLM to iteratively rephrase the harmful prompt while avoiding these trigger tokens, similar to TAP~\cite{mehrotra2023tree}. These modified jailbreak prompts are incorporated into the safety fine-tuning dataset, and we evaluate the detection accuracy on a validation split. The detailed algorithm is provided in the Appendix~\ref{appd:trigger_removal}.

\subsection{Results}
\paragraph{Disrupting Non-dominant Directions Reduces Refusal}
In \autoref{fig:component_projections}, we analyze how different attacks affect the projection values compared to default prompts (\texttt{Harmful} and \texttt{Benign}). Both non-dominant suppression and trigger removal attacks cause the dominant component projection to deviate from harmful samples. This deviation leads to a lower refusal rate as projection values on the dominant component increase. Our analysis reveals that indirect features from non-dominant directions greatly influence the dominant directions. Interestingly, while trigger removal attacks shift projections closer to benign samples, non-dominant suppression pushes them in the opposite direction.

\paragraph{Trigger Removal is Resilient to Safety Fine-tuning}

\autoref{tab:exposure_acc} shows that removing triggers effectively prevents safety fine-tuning from generalizing to these attacks. The initial attack success rate is comparable to other methods for a pre-fine-tuned model. However, after fine-tuning on 80 samples per jailbreak, while the success rate of other jailbreaks drops to near zero, the Trigger Removal Attack maintains approximately 40\% effectiveness.


Overall, these findings confirm that non-dominant directions causally impact both the dominant component and safety behavior. Since these non-dominant directions capture features beyond query harmfulness like specific jail-break patterns, this suggests that safety training may model \emph{spurious correlations}~\cite{geirhos2020shortcut} in certain jailbreak patterns, allowing out-of-domain jailbreaks like the Trigger Removal Attack to weaken or bypass the learned alignment.

\begin{table}[t]
    \caption{Attack Pass Rate of jailbreak prompts on safety fine-tuned models under different exposure settings. \textsc{n-shot} indicates the number of samples of each jailbreak presented in the fine-tuning dataset.}
    \label{tab:exposure_acc}
    \vskip 0.15in
    \begin{center}
    \begin{small}
    \begin{sc}
    \setlength\tabcolsep{4pt}
    \begin{tabular}{lcccccc}
    \toprule
    Method & 0-shot & 10 & 20 & 40 & 80 & 160 \\
            & Success   & shot & shot & shot & shot & shot \\
    \midrule
    GPTFuzz  & 0.02 & 0.02 & 0.02 & 0.03 & 0.03 & 0.03 \\
    Flip     & 0.78 & 0.12 & 0.22 & 0.03 & 0.03 & 0.03 \\
    Pair     & 0.82 & 0.75 & 0.45 & 0.17 & 0.12 & 0.05 \\
    ReNellm  & 0.61 & 0.00 & 0.00 & 0.00 & 0.00 & 0.00 \\
    \midrule
    \begin{tabular}[c]{@{}l@{}} Trigger \\ Removal \end{tabular}     & 0.77 & 0.78 & 0.62 & 0.52 & 0.42 & 0.30 \\
    \bottomrule
    \end{tabular}
    \end{sc}
    \end{small}
    \end{center}
    \vskip -0.2in
\end{table}


\section{Related Work}
\section{Related Work}
% \subsection{Vision Language Model}
% 시각장애인에서 상황을 설명할 DB가 없으니 만들었다. 그리고 이를 VLM에 튜닝했다.
\subsection{Technical approaches for assisting the visually-impaired}


\subsection{Datasets for visual instruction tuning}


\section{Conclusion}
We introduce the notion of \textit{local} (sentence-level) and \textit{global} (word-level) sensitivities to capture the intricacies of a text classifier for a given dataset. We introduce a novel, cost-effective sensitivity estimation framework, SMAB. Through experiments on \textsc{CheckList}-generated dataset, we show that our SMAB framework captures high-sensitive and low-sensitive words effectively. We observe that the comparative accuracy between two models (for the same language or for across language on the same task) has strong negative correlations with KL divergence between (global) sensitivity distributions of the models -- showing sensitivity can be used as an unsupervised proxy for accuracy (drops). Further, we define three word-level perturbation instructions utilizing the global sensitivity values obtained from the SMAB framework to attack LLMs such as GPT-3.5 with a high success rate. We also show that sensitivity can be used as an additional reward in paraphrase-based attacks to improve the attack success rate of adversarial models. Hence, word-level sensitivities provide a closer look at how opaque language models work.

\section*{Limitations}
The work explores the proposed framework for sequence classification tasks. Further exploration is needed to extend to other tasks, such as generation and translation. The hypothesis of sensitivity acting as an unsupervised proxy is valid under the specific conditions we tested. A more detailed study of various families of models and tasks might provide deeper insights into the correlation, which will be highly useful for evaluating and benchmarking low-resource languages. It is also important to note that we performed experiments concerning adversarial example generation in English, and a full-fledged multilingual study needs to be performed.

\section*{Ethics Statement}
Although our framework helps identify words with different sensitivity levels, there can be a few repercussions. It is important to note that the method does not guarantee that the examples generated will always be adversarial. The framework, and hence the sensitivity values, may be misused by people to develop better jailbreak techniques.

\section*{Acknowledgements}
This research is partially supported by SERB SRG/2022/000648. We acknowledge the OpenAI and Azure credits from the Microsoft Accelerate Foundation Models Research (AFMR) Grant. Sachin Vashistha is partially supported by the Prime Minister's Research Fellowship (PMRF) grant.
  
% This must be in the first 5 lines to tell arXiv to use pdfLaTeX, which is strongly recommended.
\pdfoutput=1
% In particular, the hyperref package requires pdfLaTeX in order to break URLs across lines.

\documentclass[11pt]{article}

% Change "review" to "final" to generate the final (sometimes called camera-ready) version.
% Change to "preprint" to generate a non-anonymous version with page numbers.
\usepackage[final]{acl}

% Standard package includes
\usepackage{times}
\usepackage{latexsym}

% For proper rendering and hyphenation of words containing Latin characters (including in bib files)
\usepackage[T1]{fontenc}
% For Vietnamese characters
% \usepackage[T5]{fontenc}
% See https://www.latex-project.org/help/documentation/encguide.pdf for other character sets

% This assumes your files are encoded as UTF8
\usepackage[utf8]{inputenc}

% This is not strictly necessary, and may be commented out,
% but it will improve the layout of the manuscript,
% and will typically save some space.
\usepackage{microtype}

% This is also not strictly necessary, and may be commented out.
% However, it will improve the aesthetics of text in
% the typewriter font.
\usepackage{inconsolata}

%Including images in your LaTeX document requires adding
%additional package(s)
\usepackage{graphicx}
\usepackage{xcolor}

% If the title and author information does not fit in the area allocated, uncomment the following
%
%\setlength\titlebox{<dim>}
%
% and set <dim> to something 5cm or larger.

\usepackage{booktabs}
\usepackage{hyperref}
\usepackage{multirow}
\usepackage{multicol}
\usepackage[most]{tcolorbox}
\usepackage{adjustbox}
\usepackage{graphicx}
\usepackage{fullpage}
\usepackage{times}
\usepackage{fancyhdr,graphicx,amsmath,amssymb}
%\usepackage[ruled,vlined]{algorithm2e}
\usepackage{algorithm}
\usepackage{algpseudocode}
\usepackage{booktabs}
\usepackage{adjustbox}
\usepackage{url}
\usepackage{hyperref}
\usepackage{amssymb}
\usepackage{marvosym}
\usepackage{multirow}
\usepackage{subcaption}
\DeclareMathOperator*{\argmax}{arg\,max}
\DeclareMathOperator*{\argmin}{arg\,min}


\newtcolorbox{promptbox}[2][]{
  colback=gray!10,
  colframe=gray!50,
  arc=3mm,
  boxrule=1pt,
  left=10pt,
  right=10pt,
  top=8pt,
  bottom=8pt,
  before skip=12pt,
  after skip=12pt,
  fonttitle=\bfseries,
  title=#2,
  #1
}

\title{Quality-Aware Decoding: Unifying Quality Estimation and Decoding}

% Author information can be set in various styles:
% For several authors from the same institution:
% \author{Author 1 \and ... \and Author n \\
%         Address line \\ ... \\ Address line}
% if the names do not fit well on one line use
%         Author 1 \\ {\bf Author 2} \\ ... \\ {\bf Author n} \\
% For authors from different institutions:
% \author{Author 1 \\ Address line \\  ... \\ Address line
%         \And  ... \And
%         Author n \\ Address line \\ ... \\ Address line}
% To start a separate ``row'' of authors use \AND, as in
% \author{Author 1 \\ Address line \\  ... \\ Address line
%         \AND
%         Author 2 \\ Address line \\ ... \\ Address line \And
%         Author 3 \\ Address line \\ ... \\ Address line}

\author{Sai Koneru$^{1}$,
  Matthias Huck$^{2}$,
  Miriam Exel$^{2}$, \textnormal{and}
  Jan Niehues$^{1}$ \\
  $^{1}$ Karlsruhe Institute of Technology \\
  $^{2}$ SAP SE, Dietmar-Hopp-Allee 16, 69190 Walldorf, Germany \\
  \texttt{\{sai.koneru, jan.niehues\}@kit.edu} \\
  \texttt{\{matthias.huck, miriam.exel\}@sap.com}}

%\author{
%  \textbf{First Author\textsuperscript{1}},
%  \textbf{Second Author\textsuperscript{1,2}},
%  \textbf{Third T. Author\textsuperscript{1}},
%  \textbf{Fourth Author\textsuperscript{1}},
%\\
%  \textbf{Fifth Author\textsuperscript{1,2}},
%  \textbf{Sixth Author\textsuperscript{1}},
%  \textbf{Seventh Author\textsuperscript{1}},
%  \textbf{Eighth Author \textsuperscript{1,2,3,4}},
%\\
%  \textbf{Ninth Author\textsuperscript{1}},
%  \textbf{Tenth Author\textsuperscript{1}},
%  \textbf{Eleventh E. Author\textsuperscript{1,2,3,4,5}},
%  \textbf{Twelfth Author\textsuperscript{1}},
%\\
%  \textbf{Thirteenth Author\textsuperscript{3}},
%  \textbf{Fourteenth F. Author\textsuperscript{2,4}},
%  \textbf{Fifteenth Author\textsuperscript{1}},
%  \textbf{Sixteenth Author\textsuperscript{1}},
%\\
%  \textbf{Seventeenth S. Author\textsuperscript{4,5}},
%  \textbf{Eighteenth Author\textsuperscript{3,4}},
%  \textbf{Nineteenth N. Author\textsuperscript{2,5}},
%  \textbf{Twentieth Author\textsuperscript{1}}
%\\
%\\
%  \textsuperscript{1}Affiliation 1,
%  \textsuperscript{2}Affiliation 2,
%  \textsuperscript{3}Affiliation 3,
%  \textsuperscript{4}Affiliation 4,
%  \textsuperscript{5}Affiliation 5
%\\
%  \small{
%    \textbf{Correspondence:} \href{mailto:email@domain}{email@domain}
%  }
%}

\begin{document}
\maketitle
\begin{abstract}
% Neural Machine Translation (NMT) has achieved high-quality translations in many scenarios, pushing the boundaries of tasks such as instruction-following and multimodal translation. 
Quality Estimation (QE) models for Neural Machine Translation (NMT) predict the quality of the hypothesis without having access to the reference.
An emerging research direction in NMT involves the use of QE models, which have demonstrated high correlations with human judgment and can enhance translations through Quality-Aware Decoding. Although several approaches have been proposed based on sampling multiple candidate translations and picking the best candidate, none have integrated these models directly into the decoding process. In this paper, we address this by proposing a novel token-level QE model capable of reliably scoring partial translations. We build a uni-directional QE model for this, as decoder models are inherently trained and efficient on partial sequences. We then present a decoding strategy that integrates the QE model for Quality-Aware decoding and demonstrate that the translation quality improves when compared to the N-best list re-ranking with state-of-the-art QE models (up to $1.39$ XCOMET-XXL $\uparrow$). Finally, we show that our approach provides significant benefits in document translation tasks, where the quality of N-best lists is typically suboptimal\footnote{Code can be found at \url{https://github.com/SAP-samples/quality-aware-decoding-translation}}
\end{abstract}
\section{Introduction}

Large language models (LLMs) have significantly impacted various Natural Language Processing (NLP) tasks \citep{brown2020language, jiang2023mistral, dubey2024llama}, including Neural Machine Translation (NMT). The field of NMT is transitioning from using dedicated encoder-decoder transformers \citep{vaswani2017attention, nllb2024scaling} to leveraging decoder-only LLM-based translation models \citep{kocmi2024findings}. This shift is driven by LLMs' ability to retain knowledge, handle large contexts, and follow instructions, learned during extensive pre-training \citep{xu2024contrastive, alves2024tower}. As a result, LLM-based MT models have achieved state-of-the-art translation quality \citep{kocmi2024findings}.

In parallel, Quality Estimation (QE) has become a well-researched subfield within NMT. QE models are trained to predict the quality of a translation without requiring access to the reference \citep{rei2021references,rei2022cometkiwi}. Interestingly, QE models can achieve performance in assessing translation quality that is comparable to MT evaluation models, which do have access to the reference \citep{zerva2024findings}.

This led to the question: "\textit{Can we integrate QE into the current translation process to improve quality?}" Incorporating QE into NMT offers several benefits. First, having a expert QE model guiding the decoding can further improve the quality. Second, by adapting the QE model with feedback from human annotators, we can generate future translations guided with the newly obtained feedback.

\begin{figure*}[!ht]
\includegraphics[width=\textwidth]{Figures/nbestlist_problem.png}
 \caption{Example from WMT'23 English → German \#ID: 10: The paragraph begins with 'Department of Homeland Security,' which should be translated as 'Ministerium für \textbf{I}nnere Sicherheit.' However, the top 25 beams do not contain the correct translation and begin with an error, making N-best list re-ranking insufficient. Although the top-5 tokens at the decoding contain the correct forms 'Inn' or 'Inner,' the probabilities split among them giving highest mass to the incorrect token 'inn.' Quality-Aware decoding can prevent errors with earlier integration.}
\label{fig:nbestlist}
\end{figure*}


Several approaches have been explored to integrate QE into the translation process. These include re-ranking the N-best list \citep{fernandes2022quality}, applying minimum Bayes risk (MBR) decoding on a quality-filtered N-best list \citep{tomani2024quality}, and training additional models for post-editing based on QE-predicted errors \citep{treviso2024xtower}. However, all these methods operate on fully generated sequences before the QE model can exert influence. Integrating QE earlier in the decoding process, referred in this paper as \textit{Quality-Aware Decoding}, could enhance translation quality and reduce reliance on the N-best list. This is especially relevant when dealing with long inputs as good translations during decoding are likely to be pruned and may need sampling larger number of finished hypothesis. We illustrate this in Figure \ref{fig:nbestlist}.

To achieve this, a QE model capable of predicting the quality of partial translations is required. However, current leading QE models face challenges in this area, as they are typically not trained to predict scores for incomplete hypotheses. \textit{Therefore, developing QE models that can handle partial translations is essential for implementing Quality-Aware Decoding during the translation process}.

In this work, we propose adapting LLM-based MT models to perform QE on partial translations and incorporating this model into the decoding. We create a token-level synthetic QE dataset using WMT Multidimensional Quality Metrics (MQM) data \citep{burchardt2013multidimensional, freitag2024llms}. We then adapt a uni-directional LLM-based MT model to predict whether a token is \textit{Good} or \textit{Bad}. Training QE models on these token-level tasks alleviates the data challenge and allows us to exploit the MQM data while simultaneously making the task easier for the model compared to predicting a score directly.

\begin{figure*}[!ht]
\includegraphics[width=\textwidth]{Figures/annotation_scheme.png}
 \caption{Token-level label annotation scheme using the MQM error tags. \textit{MASK} indicates that this token label will not be used in training to prevent incorrect learning signal.}
\label{fig:annotation}
\end{figure*}

Furthermore, integrating the QE model into NMT during decoding is not trivial, as we need to combine the QE estimates during decoding. Therefore, we use the decoding strategy from \citet{koneru2024plug}, and modify it to incorporate token-level predictions efficiently with the adapted QE model to provide real-time feedback during the decoding process. We summarize our main findings and contributions below.

\begin{itemize}
    \item We present a novel uni-directional QE model which estimates quality on incomplete hypotheses by averaging the probabilities of each token being classified as \textit{Good}. 
    
    %We demonstrate that it achieves improved correlation with human annotations on WMT 23 English $\rightarrow$ German, compared to the log probabilities of the same LLM-based NMT model.

    \item We propose a decoding strategy that combines the token-level QE model on partial hypothesis and the NMT model to perform Quality-Aware Decoding. 
    
    \item We show through experiments that early integration is essential and the translation quality is improved even when compared to re-ranking the N-best list with state-of-the-art QE models.

    \item We highlight the significance of our approach in document translation scenarios, where post-generation QE techniques fall short due to their reliance on the quality of the N-best list, a challenge that becomes more difficult as the input length increases.
\end{itemize}



\section{Quality-Aware Decoding}

The primary objective of this paper is to achieve Quality-Aware Decoding in MT. To accomplish this, it is essential to predict the quality of partial translations and integrate this information during the decoding process. Our approach proposes using one NMT model for generating translations and another adapted NMT model to predict the quality of the candidate translations produced by the first model.

First, we explain why relying solely on the NMT model to predict the quality of a hypothesis is insufficient and why an additional model is necessary. Next, we outline the adaptation of the NMT model for QE on partial translations, detailing the creation of a token-level QE dataset, the modifications made to the NMT model for this task, and the process of estimating the sentence-level quality score. Finally, we describe the algorithm used to incorporate the QE score into the decoding process.

\subsection{Decomposing Decoding: Translation + QE}
NMT models generate a token-by-token sequence and provide the probability of each token at the decoding step. The average of the log-probabilities is often used as a proxy to score the candidate during search. 

While NMT models are capable of generating high-quality translations, using the average log-probabilities of hypotheses as a scoring metric tends to yield poor correlation with actual translation quality \citep{eikema2020map, freitag2020bleu}. In many cases, a translation can continue in several different ways, all of which may be acceptable. If the starting tokens for these continuations differ, the probability mass may be spread across multiple options which is used during the search. However, from a quality perspective, all these continuations could still achieve a high score, as the QE scores are independent and need not sum to $1$.

Therefore, we propose a expert model that focuses on quality to estimate the scores better during decoding and  improve the search space leading to a better hypothesis.


% Therefore, relying solely on the average log-probabilities during decoding is not ideal, as it computes the score independently for each token and does not fully correlate with the overall quality of the current hypothesis.

\subsection{Quality Estimation on Partial Sequences}

% NMT models decode sequences token-by-token. 
To provide a quality score during decoding, the QE model must be capable of handling incomplete sequences. It should not penalize a sequence if there is a potential extension that could lead to a perfect translation.

Estimating the score in this way is not feasible with current QE models, such as COMET \citep{rei2021references}, as they were not trained for this specific task and cannot provide reliable scores in the context of partial translations. Hence, we need to develop a partial QE system.

When building a partial QE system, several factors need to be considered. First, should the model use a uni-directional or bi-directional architecture? A \textbf{uni-directional} model is more efficient, as it allows for caching the hidden states, which can then be used for subsequent steps without re-encoding, unlike a bi-directional model.

Next, we need to decide whether to predict the QE score at the sequence level or at the token level. For \textbf{token-level QE}, we can directly use data from MQM annotations, as we already know which tokens are \textit{Good} or \textit{Bad}. However, for segment-level scoring, we need to consider how to synthetically create the training data. 

% Additionally, COMET models are encoder-only architectures pre-trained on full sentences, rather than partial sentences as required in this case. Moreover, predicting the score of partial translations naturally favors decoder-only models due to their efficiency. New tokens only need to process the preceding sequence, avoiding the need to re-encode the entire sequence. Additionally, this approach simplifies training, as we do not require synthetically shorter samples. 

%  Furthermore, there is no readily available dataset containing partial translations along with their quality scores. Hence, we need to design the adaptation process with a QE model that is uni-directional and exploit already available human annotated data.

% \subsubsection{Token-level Quality Estimation}

Therefore, we decide adapt the uni-directional model into a token-level QE system that predicts whether each token is \textit{Good} or \textit{Bad} (a binary decision) by adding an additional classifier head. This adaptation enables us to estimate the score for a sequence by calculating the average probability that each token is classified as \textit{Good}. We hypothesize that adapting the model in this way, rather than directly predicting the score, provides greater stability, as the last hidden states inherently contain token-level information and do not require mapping the entire sequence to a single score.

For training this model, we leverage the WMT MQM data containing error annotations in MT outputs. We can treat tokens before an error as \textit{Good} and those containing inside an error as \textit{Bad}. Then, we can train in uni-directional manner where each token's label is predicted using only the preceding context in the hypothesis. This is crucial as we only have the preceding context to estimate the quality for partial hypothesis.

\subsubsection{Learning the Right Signal}

\begin{algorithm*}[!t]
\caption{Computing merged score of partial hypothesis with translation and token-level QE models.}
\begin{algorithmic}[1]
\setlength{\baselineskip}{1.2em}
\Procedure{MergeScore}{}
    \State \textbf{Input:}   Hypothesis tokens $h_1, h_2, h_3, \dots, h_{n}$, Translation Model $\mathcal{M}_{NMT}$, QE model $\mathcal{M}_{QE}$, Source sentence $\mathcal{S}$, Re-ranking weight $\alpha$,
    \State \textbf{Output:} $merged\_score$
    \State $Score_{NMT} \gets \frac{1}{n}\sum \log \mathcal{P}(h_1, h_2,\dots, h_{n}|\mathcal{S};\mathcal{M}_{NMT})$ 
    \State $Score_{QE} \gets \frac{1}{n}\sum \log \mathcal{P}(0_{1}, 0_{2},\dots,0_{n} | h_1, h_2,\dots, h_{n},\mathcal{S};\mathcal{M}_{QE})$ 
    \State $merged\_score \gets (\alpha) \times Score_{NMT} + (1 - \alpha) \times Score_{QE}$
\EndProcedure
\end{algorithmic}
\label{alg:joint}
\end{algorithm*}

The straightforward approach to creating labels is to assign $1$ to all tokens within the error span and $0$ otherwise. However, MQM annotations can mark errors from words to phrases, and the starting tokens of an error span may not always be wrong. This is illustrated in Figure \ref{fig:annotation}.

For example, consider the German sentence \textit{"Ich spiele Tennis"} translated by three different NMT systems, each annotated with MQM error labels. In this work, we focus on learning a binary decision: whether an error is present, ignoring error severity.

\textbf{System 1: No error}: The translation \textit{"I play Tennis"} is perfect, and all tokens are labeled as "\textit{Good}."

\textbf{System 2: Partial error}: The translation \textit{"I played Tennis"} has an error in the verb form ("played" instead of "play"). The error is in the token span \textit{"played"}, but not all tokens in this span are incorrect (e.g., "pla" is correct). Assigning a "\textit{Bad}" label to the entire span would lead to incorrect learning. A more refined approach is needed to mark errors accurately at the token level.

\textbf{System 3: Full error}: The translation \textit{"I enjoy Tennis"} contains an error in \textit{"enjoy"}, so all tokens in this span should be labeled as "\textit{Bad}."

It is not trivial to decide when the prefix of an error span is correct/incorrect. To achieve accurate labeling, we propose the following scheme:

\begin{itemize} \item Apply a \texttt{<MASK>} operation to all tokens within the error span. \item Only the last token in the span is assigned the label "\textit{Bad}", as the error is considered complete at the end of the span. \end{itemize}

If the error token is in the middle, we still train the model to predict "\textit{Bad}" in the end and let the model determine which tokens should be part of the error span during inference. This approach ensures that errors are identified without explicitly defining the error span. 

\subsubsection{Sequence-Level Quality Estimation}


After fine-tuning a token-level classification model to predict the quality of the tokens, we still need to map these predictions into a sequence-level score that can be integrated during the decoding process. There are several potential ways to achieve this.

One approach is to simply count how many tokens are classified as \textit{Bad} in the current hypothesis. However, this method has limitations. The number of errors should be normalized based on the length of the hypothesis to account for varying sizes. Additionally, converting the probabilities into a fixed number of error tokens would need to account for different error types according to the MQM format, as each error counts differently.

To avoid such strict scoring schemes, we take a simpler approach. We average the log probabilities of all tokens that are classified as \textit{Good}. This method inherently accounts for the length of the hypothesis, and it provides a score on the scale of log probabilities, which aligns with the decoding process. Therefore, we use this averaged log probability as a proxy metric for the QE score, where a higher score indicates better quality
(\textbf{Line 5} in Algorithm \ref{alg:joint}).

\subsubsection{Fusing Translation and Quality}

We can use a token-level QE system to evaluate the quality of a source and partial hypothesis during decoding. However, integrating these probabilities into all candidates is computationally expensive, as each beam considers extensions equal to the vocabulary size.

To address this, we adopt a simplified decoding strategy from \citet{koneru2024plug}, which ensembles models with different vocabularies. By adapting the same MT model for token-level QE, we simplify the merging process, as the vocabularies match. This restriction is reasonable, as it is also beneficial to leverage the knowledge learned by the specialized MT for token-level QE.

The core idea is to re-rank the top candidates at each decoding step using the QE model. After re-ranking, the translation and QE scores are merged, and the process repeats until the end-of-sentence token is generated, for each beam. This strategy allows us to efficiently incorporate the QE model’s estimate, improving translation quality.

During decoding, at each step, we have scores for $n$ beams and $V$ possible extensions from the vocabulary. In typical beam search, we select the top $n$ extensions and expand the hypothesis. To make the decoding process Quality-aware, we estimate the quality of these extensions. Since estimating all extensions is computationally expensive, we limit the candidates by selecting a specified number of top candidates.

To achieve this, we use a hyper-parameter $topk$, which selects the best $topk$ extensions for each beam. For each of these top $topk$ extensions, we compute a combined score, detailed in Algorithm \ref{alg:joint}. This combined score incorporates both the translation model score and the quality estimation score, ensuring the quality is considered during decoding.

For a top extension at decoding step $n$, let the current tokens be $h_1, h_2, h_3, \dots, h_n$. The NMT model score is computed as the average log probabilities of each token (Line 4). For the token-level QE model, we compute the average probability of each token being classified as '\textit{Good}' (Line 5). The merged score is equal to weighted linear combination of these probabilities, with weight $\alpha$ (Line 6).

Thus, to make the decoding process Quality-Aware, we first train a token-level QE system by adapting the same NMT model to ensure vocabulary matching. We then combine the scores from both models to improve the sequence estimates explored during search.


\begin{table*}[!ht]
\resizebox{2\columnwidth}{!}{
\begin{tabular}{@{}ccccc@{}}
\toprule
\multicolumn{1}{c|}{Model}            & \multicolumn{1}{c|}{Beams}                & \multicolumn{1}{c|}{Re-ranking}              & MetricX ($\downarrow$)     & XCOMET-XXL ($\uparrow$)    \\ \midrule
\multicolumn{5}{c}{\textit{English $\rightarrow$ German}}                                                                                                          \\ \midrule
\multicolumn{1}{c|}{Tower}            & \multicolumn{1}{c|}{5}                    & \multicolumn{1}{c|}{\_}                      & 2.52          & 86.93          \\
\multicolumn{1}{c|}{Tower}            & \multicolumn{1}{c|}{25}                   & \multicolumn{1}{c|}{XCOMET-XL QE}            & 2.37          & 87.79          \\
\multicolumn{1}{c|}{Tower}            & \multicolumn{1}{c|}{25}                   & \multicolumn{1}{c|}{Tower QE} & 2.38          & 87.40          \\
\multicolumn{1}{c|}{Tower + Tower QE} & \multicolumn{1}{c|}{5 (25* for Tower QE)} & \multicolumn{1}{c|}{\_}                      & 2.12          & 88.95          \\
\multicolumn{1}{c|}{Tower + Tower QE} & \multicolumn{1}{c|}{5 (25* for Tower QE)} & \multicolumn{1}{c|}{XCOMET-XL QE}            & \textbf{2.09} & \textbf{89.08} \\ \midrule
\multicolumn{5}{c}{\textit{Chinese $\rightarrow$ English}}                                                                                                         \\ \midrule
\multicolumn{1}{c|}{Tower}            & \multicolumn{1}{c|}{5}                    & \multicolumn{1}{c|}{\_}                      & 2.42          & 88.91          \\
\multicolumn{1}{c|}{Tower}            & \multicolumn{1}{c|}{25}                   & \multicolumn{1}{c|}{XCOMET-XL QE}            & 2.30          & 89.49          \\
\multicolumn{1}{c|}{Tower}            & \multicolumn{1}{c|}{25}                   & \multicolumn{1}{c|}{Tower QE} & 2.32          & 89.51          \\
\multicolumn{1}{c|}{Tower + Tower QE} & \multicolumn{1}{c|}{5 (25* for Tower QE)} & \multicolumn{1}{c|}{\_}                      & 2.26          & 89.82          \\
\multicolumn{1}{c|}{Tower + Tower QE} & \multicolumn{1}{c|}{5 (25* for Tower QE)} & \multicolumn{1}{c|}{XCOMET-XL QE}            & \textbf{2.24} & \textbf{90.00} \\ \bottomrule
\end{tabular}
}
\caption{Translation Quality on WMT23 English $\rightarrow$ German Test set. Both XCOMET and MetricX columns use reference for reporting translation quality where as XCOMET-XL QE does not use for re-ranking.}
\label{tab:qadecoding}
\end{table*}

\begin{table}[!ht]
\resizebox{\columnwidth}{!}{
\centering
\begin{tabular}{@{}c|ccc@{}}
\toprule
                                                                                      & Pearson        & Spearmann      & Kendall        \\ \midrule
COMETQE                                                                               & \textbf{44.41} & 41.29          & 31.19          \\ \midrule
COMETQE-XL                                                                            & 41.23          & \textbf{42.17} & \textbf{31.84} \\ \midrule
Tower Avg. Log Prob                                                                        & 32.32          & 16.74          & 12.77          \\ \midrule
\begin{tabular}[c]{@{}c@{}}Tower QE\end{tabular} & 40.56          & 33.96          & 25.87          \\ \bottomrule
\end{tabular}
}
\caption{Correlation on WMT 23 for English $\rightarrow$ German Test set. The scores are calculated after removing the few sentences labeled for hallucination detection. Best scores according to each coefficient are highlighted in \textbf{bold}.}
\label{tab:correlation}
\end{table}

\section{Experimental Setup}
\paragraph{Datasets:} We focus on two language directions given their availability of MQM data: English $\rightarrow$ German and Chinese $\rightarrow$ English. To train our token-level QE systems, we use the MQM datasets\footnote{https://github.com/google/wmt-mqm-human-evaluation} from WMT \citep{freitag2021experts}. Specifically, we use the datasets until 2022 for training, 2024 for validation, and 2023 for testing \citep{kocmi2024findings}. This setup is consistent with all the other QE metrics, and we do not use any additional data beyond these datasets.
\vspace{-0.1cm}
\paragraph{Models:} 
Our proposed approach achieves Quality-Aware decoding by combining an NMT model with a token-level QE model, where we adapt the same NMT for QE by adding a classification head. We use the state-of-the-art NMT model, Tower 7B\footnote{Unbabel/TowerInstruct-7B-v0.2} \citep{alves2024tower}, which provides high-quality translations and has already been exposed to MQM data during instruction-tuning. This ensures that the gains observed in our approach stem from integrating Quality-Aware decoding into the NMT process, rather than introducing new data. Additional details on training and hyper-parameters are provided in Appendix \ref{sec:training_detail}.
\vspace{-0.1cm}
\paragraph{Metrics:}
For reporting the translation quality, we consistently use XCOMET-XXL\footnote{Unbabel/XCOMET-XXL} \citep{guerreiro2024xcomet} and MetricX\footnote{google/metricx-24-hybrid-xl-v2p6} \citep{juraska2024metricx} \textbf{with the reference}. To compare with N-best list re-ranking, we use the XCOMET-XL QE\footnote{Unbabel/XCOMET-XL} \textbf{without the reference}. This approach allows us to avoid biasing toward a single metric during the re-ranking process and enables us to measure the gains achieved by differently trained metrics. 

\section{Results}



\begin{table*}[!ht]
\centering
\resizebox{2\columnwidth}{!}{
\begin{tabular}{@{}ccccc@{}}
\toprule
\multicolumn{1}{c|}{Model}            & \multicolumn{1}{c|}{Beams}                        & \multicolumn{1}{c|}{Re-ranking}               & MetricX ($\downarrow$) & XCOMET-XXL ($\uparrow$) \\ \midrule
\multicolumn{5}{c}{\textit{English $\rightarrow$ German}}                                                                                                          \\ \midrule
\multicolumn{1}{c|}{Tower}            & \multicolumn{1}{c|}{25}                           & \multicolumn{1}{c|}{XCOMET-XL QE}             & 2.37     & 87.79      \\
\multicolumn{1}{c|}{Tower}            & \multicolumn{1}{c|}{25}                           & \multicolumn{1}{c|}{Tower QE}         & 2.38     & 87.40      \\
\multicolumn{1}{c|}{Tower}            & \multicolumn{1}{c|}{25}                           & \multicolumn{1}{c|}{Tower Distill QE} & 2.38     & 87.39      \\
\multicolumn{1}{c|}{Tower + Tower QE} & \multicolumn{1}{c|}{5 (25* for Tower QE)}         & \multicolumn{1}{c|}{\_}                       & 2.12     & \textbf{88.95}      \\
\multicolumn{1}{c|}{Tower + Tower QE} & \multicolumn{1}{c|}{5 (25* for Tower Distill QE)} & \multicolumn{1}{c|}{\_}                       & \textbf{2.11}     & 88.76      \\ \bottomrule
\end{tabular}
}
\caption{Performance of Unidirectional QE trained with/without distillation on WMT23 English $\rightarrow$ German Test set. Best scores according to each metric are highlighted in \textbf{bold}.}
\label{tab:towerdistill}
\end{table*}


\begin{table*}[!ht]
\centering
\resizebox{2\columnwidth}{!}{
\begin{tabular}{@{}cccccc@{}}
\toprule
\multicolumn{1}{c|}{Model}            & \multicolumn{1}{c|}{Beams}                & \multicolumn{1}{c|}{Re-ranking}       & XCOMET-XL ($\uparrow$)     & \multicolumn{1}{c|}{XCOMET-XXL ($\uparrow$)}     & Impact                                                                                       \\ \midrule
\multicolumn{6}{c}{\textit{Paragraph-Level}}                                                                                                                                                                                                                                    \\ \midrule
\multicolumn{1}{c|}{Tower}            & \multicolumn{1}{c|}{25}                   & \multicolumn{1}{c|}{XCOMET-XL QE}     & \textbf{86.56} & \multicolumn{1}{c|}{87.79}          & \multirow{3}{*}{\begin{tabular}[c]{@{}c@{}}$\delta$ = + 1.16\\ (88.95 - 87.79)\end{tabular}} \\
\multicolumn{1}{c|}{Tower}            & \multicolumn{1}{c|}{25}                   & \multicolumn{1}{c|}{Tower QE} & 85.40          & \multicolumn{1}{c|}{87.40}          &                                                                                              \\
\multicolumn{1}{c|}{Tower + Tower QE} & \multicolumn{1}{c|}{5 (25* for Tower QE)} & \multicolumn{1}{c|}{\_}               & 86.36          & \multicolumn{1}{c|}{\textbf{88.95}} &                                                                                              \\ \midrule
\multicolumn{6}{c}{\textit{Sentence-Level}}                                                                                                                                                                                                                                     \\ \midrule
\multicolumn{1}{c|}{Tower}            & \multicolumn{1}{c|}{25}                   & \multicolumn{1}{c|}{XCOMET-XL QE}     & \textbf{86.42}          & \multicolumn{1}{c|}{87.68}          & \multirow{3}{*}{\begin{tabular}[c]{@{}c@{}}$\delta$ = + 0.38\\ (88.06 - 87.68)\end{tabular}} \\
\multicolumn{1}{c|}{Tower}            & \multicolumn{1}{c|}{25}                   & \multicolumn{1}{c|}{Tower QE} & 85.23          & \multicolumn{1}{c|}{87.41}          &                                                                                              \\
\multicolumn{1}{c|}{Tower + Tower QE} & \multicolumn{1}{c|}{5 (25* for Tower QE)} & \multicolumn{1}{c|}{\_}               & 85.96          & \multicolumn{1}{c|}{\textbf{88.06}}          &                                                                                              \\ \bottomrule
\end{tabular}
}
\caption{Impact of integrating Unidirectional QE during decoding with paragraphs vs sentences on WMT23 English $\rightarrow$ German Test set. $\delta$ denotes the improvement in translation quality from re-ranking N-best list with XCOMET-XL QE to integrating unidirectional Tower QE during the decoding. Best scores according to each metric are highlighted in \textbf{bold}.}
\label{tab:sentvspara}
\end{table*}



We conduct a series of experiments to validate the effectiveness of Quality-Aware decoding and identify the scenarios where it provides the most benefit. First, we evaluate whether our token-level QE model can better estimate sequence quality compared to the log probabilities of the NMT model. Next, we assess the impact of Quality-Aware decoding by comparing it with other approaches to determine if it improves translation quality. We also perform an ablation study to examine whether training the QE model on errors from the same NMT model enhances its performance. Finally, we explore the impact of source sentence length to highlight the limitations of N-best list re-ranking.

\subsection{Quality Estimation Performance}

First, we evaluate the agreement between the Tower-based token-level QE model (\textbf{Tower QE}) and human scores for a given hypothesis. It is only beneficial if we achieve higher correlation than the average of the NMT model log probabilities to show the need to integrate it during decoding. Therefore, we report the correlation with human scores of different models on WMT 23 English $\rightarrow$ German in Table \ref{tab:correlation}. 

We observe that the best-performing systems are the Comet QE models, which predict a single score using the full hypothesis. This is expected, as these models assess quality after the hypothesis is fully generated. In contrast, both log probabilities and Tower QE scores are based on the predicted token of each decoding step, using only the preceding context. Log probabilities perform poorly in this setup, while our proposed model, Tower QE, achieves twice the correlation with human judgments compared to log probabilities, despite scoring token by token with preceding context. This result highlights the potential of integrating our approach into the decoding process.

\subsection{Unified Decoding for NMT}


To validate the effectiveness of our unified decoding approach, we compare it with several baselines in Table \ref{tab:qadecoding}. First, we evaluate whether our approach outperforms generating translations with the NMT model alone. Next, we check if the quality of translations improves compared to N-best list re-ranking. To make the setups comparable, we set $topk$ and $num\_beams$ to $5$ and compare with re-ranking the top $25$ beams using XCOMET-XL. Finally, to demonstrate that re-ranking the N-best list remains a viable and complementary approach, we re-rank the top $5$ beams obtained from Quality-Aware decoding using the same QE model. 

We find that re-ranking with XCOMET-XL and Tower QE yields similar results, indicating that our partial QE model does not over-fit to any specific metric. Furthermore, we observe that the unified decoding approach outperforms N-best list re-ranking across both metrics in both language pairs. For example, the MetricX score improves from $2.37$ to $2.12$ for English $\rightarrow$ German. Note that Tower has already seen this data during instruction-tuning and the improvement is not from new data but from Quality-Aware decoding. Moreover, re-ranking the top $5$ beams obtained from unified decoding with XCOMET-XL leads to a slight further improvement in quality. This highlights the robustness and generalizability of our approach across different evaluation metrics.
%\footnotetext{\href{https://github.com/WMT-QE-Task/wmt-qe-2023-data}{WMT 23 English $\rightarrow$ German QE Data}}

\subsection{Adapting for Tower Errors}

We use the MQM annotations from WMT to train our Tower QE model, which contains error annotations from other systems. However, a viable alternative would be to adapt Tower QE specifically to the errors it typically makes. To maintain a similar data setup, we first generate translations using Tower on these source sentences. Then, we annotate the generated hypotheses with XCOMET-XL using the reference and fine-tune Tower QE on this synthetic dataset, which we refer to as \textbf{Tower Distill QE}. We evaluate the performance of the new distill QE model and report the results in Table \ref{tab:towerdistill}.

We observe that the distilled QE model performs very similarly to the QE model trained on errors from other systems. This indicates that there was no significant benefit in adapting the QE model to the specific errors typically made by Tower. However, further analysis on larger datasets and different domains is needed to fully validate the effectiveness of the distillation approach as the current synthetic data generated is small.

\subsection{Sentence vs Document-level Translation}

From Table \ref{tab:qadecoding}, we observe that the gains for English $\rightarrow$ German (paragraph-level) are much higher than for Chinese $\rightarrow$ English (sentence-level). We hypothesize that this discrepancy arises from the length of the sentences, as the N-best list re-ranking is likely sufficient for shorter sentences. To confirm this, we take the English paragraphs and split them into sentences using a tokenizer while tracking the paragraph IDs. We then perform the entire decoding process similarly, and later join the sentences back using the paragraph IDs before evaluation. We report the results in Table \ref{tab:sentvspara}.

We define the impact as the improvement in translation quality from re-ranking the N-best list with XCOMET-XL QE to integrating Tower QE. Comparing the results at the paragraph level to those at the sentence level, we observe that the impact decreases, which confirms our hypothesis. Additionally, we obtain better scores at the document level, further highlighting the potential benefits of Quality-Aware Decoding.

\section{Related Work}

\textbf{Integrating QE in NMT:} Several advancements have been made in improving QE for NMT over the years \citep{rei2021references, rei2022cometkiwi, blain2023findings, zerva2024findings, guerreiro2024xcomet}. These developments have led to the integration of QE in various ways.
One common approach involves applying QE after generating multiple sequences through techniques such as QE re-ranking \citep{fernandes2022quality, faria2024quest} or Minimum Bayes Risk (MBR) decoding \citep{tomani2024quality}. Another direction focuses on removing noisy data using QE models, followed by fine-tuning on high-quality data \citep{xu2024contrastive, finkelstein2024introducing}. \citet{vernikos2024don} proposes to generate diverse translations as a first step and then combine them. We perform this explicitly by integrating the QE directly into decoding.
Recently, \citet{zhang2024learning} exploited the MQM data by training models to penalize tokens within an error span, improving translation quality. In contrast, our approach adopts a modular framework, where we propose an expert QE model that is trained independently for targeted training. This modular approach aims to improve performance by decomposing the task into separate translation and QE components.

\textbf{Reward Modeling in NLG:}  Quality-Aware decoding shares several similarities with controllable text generation methods, particularly in the use of an additional "Quality/Reward" model that guides the decoding. A well-explored approach for controlling text is altering the decoding with a reward model (Weighted Decoding) \citep{yang2021fudge}. This method modifies the decoding by adjusting token probabilities based on the reward model, allowing for more controlled generation.
Similarly, \citet{deng-raffel-2023-reward} also used a uni-directional reward model, with the aim of maintaining efficiency during generation. This approach minimizes computational complexity while still benefiting from the guiding influence of the reward model. Moreover, recent work by \citet{li-etal-2024-reinforcement} introduced a token-level reinforcement learning-based reward model, providing more fine-grained feedback that enhances control over text generation at a granular level. While similar, the key contribution in our work lies in the development of the first uni-directional QE model for translation. 


\section{Conclusion}
We have shown the importance of Quality-Aware decoding to improve translation quality, rather than relying on post-generation techniques. In this work, we demonstrated how MQM data can be used to build a uni-directional token-level QE model, which is then integrated into the decoding process. Through a series of experiments, we showed that our Quality-Aware decoding approach results in measurable improvements in translation quality. Notably, we did not introduce new training data to the NMT model, and show that the gains stem from Quality-Aware decoding.


\section{Limitations}
While our Quality-Aware decoding improves translation quality, it adds considerable computational complexity to the inference process. Theoretically, this approach would double the time needed to generate a translation and require additional memory to utilize the token-level QE model. One potential solution to mitigate this issue could be to use token-level QE as a reward model for training via Reinforcement Learning.

Additionally, we trained our model on a limited set of human-annotated MQM data. However, current QE models, such as XCOMET, are capable of predicting error tags using the reference with reasonable quality. This suggests that further improvements could be achieved if these models were trained on larger-scale datasets, providing more nuanced feedback and refining translation quality even further.

Lastly, our proposed token-level QE model does not account for error severity. Ideally, it should be able to predict the category of errors, allowing for more nuanced feedback and enabling the model to generate translations with only minor errors when necessary.


% Bibliography entries for the entire Anthology, followed by custom entries
%\bibliography{anthology,custom}
% Custom bibliography entries only
\bibliography{custom}

\appendix

\section{Appendix}
\label{sec:appendix}

% \begin{table*}[!ht]
% \centering
% \begin{tabular}{@{}ccccc@{}}
% \toprule
% \multicolumn{1}{c|}{Model}            & \multicolumn{1}{c|}{Beams}                & \multicolumn{1}{c|}{Re-ranking}              & XCOMET-XL      & XCOMET-XXL     \\ \midrule
% \multicolumn{5}{c}{\textit{English $\rightarrow$ German}}                                                                                                          \\ \midrule
% \multicolumn{1}{c|}{Tower}            & \multicolumn{1}{c|}{5}           & \multicolumn{1}{c|}{\_}                      & 84.93          & 86.93          \\
% \multicolumn{1}{c|}{Tower}            & \multicolumn{1}{c|}{25}                   & \multicolumn{1}{c|}{\textbf{\_}}             & 84.87 & 86.45          \\
% \multicolumn{1}{c|}{Tower MBR}        & \multicolumn{1}{c|}{25}                   & \multicolumn{1}{c|}{\_}                      & 85.23          & 87.09          \\
% \multicolumn{1}{c|}{Tower}            & \multicolumn{1}{c|}{25}                   & \multicolumn{1}{c|}{XCOMET-XL QE}            & 86.56          & 87.79          \\
% \multicolumn{1}{c|}{Tower}            & \multicolumn{1}{c|}{5}                    & \multicolumn{1}{c|}{Tower QE} & 85.34          & 87.33          \\
% \multicolumn{1}{c|}{Tower}            & \multicolumn{1}{c|}{25}                   & \multicolumn{1}{c|}{Tower QE} & 85.40          & 87.40          \\
% \multicolumn{1}{c|}{Tower + Tower QE} & \multicolumn{1}{c|}{5 (25* for Tower QE)} & \multicolumn{1}{c|}{\_}                      & 86.36          & 88.95          \\
% \multicolumn{1}{c|}{Tower + Tower QE} & \multicolumn{1}{c|}{5 (25* for Tower QE)} & \multicolumn{1}{c|}{XCOMET-XL QE}            & \textbf{86.88} & \textbf{89.08} \\ \midrule
% \multicolumn{5}{c}{\textit{Chinese $\rightarrow$ English}}                                                                                                         \\ \midrule
% \multicolumn{1}{c|}{Tower}            & \multicolumn{1}{c|}{5}                    & \multicolumn{1}{c|}{\_}                      & 85.38          & 88.91          \\
% \multicolumn{1}{c|}{Tower}            & \multicolumn{1}{c|}{25}                   & \multicolumn{1}{c|}{\_}                      & 85.29          & 88.71          \\
% \multicolumn{1}{c|}{Tower MBR}        & \multicolumn{1}{c|}{25}                   & \multicolumn{1}{c|}{\_}                      & 86.00          & 89.23          \\
% \multicolumn{1}{c|}{Tower}            & \multicolumn{1}{c|}{25}                   & \multicolumn{1}{c|}{XCOMET-XL QE}            & 87.04          & 89.49          \\
% \multicolumn{1}{c|}{Tower}            & \multicolumn{1}{c|}{5}                    & \multicolumn{1}{c|}{Tower QE} & 85.64          & 89.10          \\
% \multicolumn{1}{c|}{Tower}            & \multicolumn{1}{c|}{25}                   & \multicolumn{1}{c|}{Tower QE} & 85.93          & 89.51          \\
% \multicolumn{1}{c|}{Tower + Tower QE} & \multicolumn{1}{c|}{5 (25* for Tower QE)} & \multicolumn{1}{c|}{\_}                      & 86.01          & 89.82          \\
% \multicolumn{1}{c|}{Tower + Tower QE} & \multicolumn{1}{c|}{5 (25* for Tower QE)} & \multicolumn{1}{c|}{XCOMET-XL QE}            & \textbf{86.67} & \textbf{90.00} \\ \bottomrule
% \end{tabular}
% \caption{COMET scores on WMT23 English $\rightarrow$ German Test set. Both XCOMET metric columns use reference for reporting translation quality and do not when used for re-ranking }
% \end{table*}


\subsection{Training details}
\label{sec:training_detail}

We use the transformers library \citep{wolf-etal-2020-transformers} for training and inference with Tower-Instruct V2.  For adapting Tower to token-level QE, we use LoRA \citep{hulora} based fine-tuning with an additional classifier head. Therefore, we only train the adapters and the weights for classification head.

We add the adapters to the modules \textit{q\_proj,k\_proj,v\_proj,gate\_proj,up\_proj} and \textit{down\_proj}. We set a batch size for each device to 12 initially and enable \textit{auto\_find\_batch\_size} to \textit{True} on 4 NVIDIA RTX A6000 GPU's. For having a  larger batch size during training, we set \textit{gradient\_accumulation\_steps} to 6. We use a \textit{learning\_rate} of $1e^{-5}$. We set the \textit{eval\_steps} to $50$ and \textit{num\_train\_epochs} to $10$. The other parameters are set to default.

Using the cross-entropy loss for token-level QE directly is insufficient due to the fact that the majority of tokens are classified as '\textit{Good}'. Hence, we find that the weighted cross-entropy loss is essential when fine-tuning the model. For the training on human MQM data, we set the weights to $0.05,0.95$ to '\textit{Good}' and '\textit{Bad}' labels respectively. In the case of distilling from XCOMET, we observed more errors. Therefore, we find that setting them $0.2,0.8$ to '\textit{Good}' and '\textit{Bad}' labels respectively provided stable training.

We train on data until WMT'22 for training and use WMT'24 for validation. We calculate the macro '\textit{F1}' on token-level predictions as the validation metric and stop training if it does not improve for 10 consecutive \textit{eval\_steps}.

\subsection{Partial vs Full Sequence Quality Estimation}

We also compare the difference in performance between our proposed token-level QE for partial sequences with Tower trained for full sequence QE. We achieve this by adding a regression head to predict the score at the end-of-sentence token. Hence, the model uses the source and hypothesis to predict the score using regression head at the end.

We fine-tune the model using only direct assesment data \citep{zerva2024findings} (\textbf{Tower Full DA}). Furthermore, we use this as initialisation and continue fine-tuning on the MQM data (\textbf{Tower Full DA + MQM}). We also use LoRA similarly to the previous model with a regression head to adapt the model. We report the scores in Table \ref{tab:correlation_ablation}.

We see that the both Tower QE models based on full sentences outperforms the partial model. However, this is expected as it has seen the entire context and was also trained on larger amounts of data. Nonetheless, the partial model still achieves much higher correlaiton that the log probabilities showcasing its potential for Quality-Aware decoding.

\subsection{Robustness to re-ranking weight}

In our method, we introduce a hyperparameter, $\alpha$, to merge the probabilities from the token-level QE model and the translation model. This section analyzes the impact of $\alpha$ on the final translation quality.

To efficiently evaluate its effect, we re-rank the N-best list using different values of $\alpha$. This approach allows us to estimate the ideal value of $\alpha$ without the need for joint decoding multiple times. If the re-ranking model (in this case, Tower QE) is beneficial, we expect that any $\alpha$ less than 1 will improve translation quality, as it demonstrates that incorporating the probabilities from the QE model is helpful.

We visualize this impact in Figure \ref{fig:mainfigure}. The results show that using an $\alpha$ less than 1 leads to improved translation quality in both scenarios. This indicates that relying entirely on the NMT model does not yield the best results and highlights the importance of the Tower QE model.

Thus, we emphasize that re-ranking the N-best list provides an effective way to tune the value of $\alpha$, and it remains robust to different values.

\begin{figure*}[!htpb]
\begin{promptbox}[title={Tower Translation Prompt}]
    \small
    <|im\_start|>user\\
    Translate the sentence from English into German.\\
    English: \{src\_sent\}\\
    German:\\
    <|im\_end|>\\
    <|im\_start|>assistant
\end{promptbox}

\begin{promptbox}[title={Tower Token-Level QE Prompt}]
    \small
    English:\{src\_sent\}\\
    German: \{tgt\_sent\}
\end{promptbox}
\caption{Prompts used in our experiments for translation and QE model. \{src\_sent\} and \{tgt\_sent\} represent the source and target sentence. We replace the language with Chinese and English when experimenting with that language pair.}
\end{figure*}

\begin{figure*}[!htpb]
    \centering
    % First subfigure
    \begin{subfigure}[b]{0.5\textwidth}
        \centering
        \includegraphics[width=\textwidth]{Figures/alphas_ende_25.png} % Replace with your image path
        \caption{English $\rightarrow$ German}
        \label{fig:subfigure1}
    \end{subfigure}
    
    \vspace{0.5cm} % Adjust space between the two subfigures

    % Second subfigure
    \begin{subfigure}[b]{0.5\textwidth}
        \centering
        \includegraphics[width=\textwidth]{Figures/alphas_zhen_25.png} % Replace with your image path
        \caption{Chinese $\rightarrow$ English}
        \label{fig:subfigure2}
    \end{subfigure}
    
    \caption{Impact of $\alpha$ when re-ranking with token-level Tower QE on WMT'23 Test sets.}
    \label{fig:mainfigure}
\end{figure*}


\begin{table*}[!ht]
\centering
\begin{tabular}{@{}c|ccc@{}}
\toprule
                                                                                      & Pearson        & Spearmann      & Kendall        \\ \midrule
COMETQE                                                                               & \textbf{44.41} & 41.29          & 31.19          \\ \midrule
COMETQE-XL                                                                            & 41.23          & \textbf{42.17} & \textbf{31.84} \\ \midrule
\begin{tabular}[c]{@{}c@{}}COMETQE Scratch\\      Fine-tuned (ours)\end{tabular}      & 36.32          & 33.66          & 25.24          \\ \midrule
Tower Log Prob                                                                        & 32.32          & 16.74          & 12.77          \\ \midrule
\begin{tabular}[c]{@{}c@{}}Tower Partial QE\end{tabular} & 40.56          & 33.96          & 25.87          \\ \midrule
Tower Full DA                                                                        & 33.67          & 36.46          & 27.38          \\ \midrule
Tower Full DA + MQM                                                                 & 32.03          & 40.85          & 30.38          \\ \bottomrule
\end{tabular}
\caption{Full Correlation results on WMT 23 for English $\rightarrow$ German Test set. Partial indicates that the QE model predict scores via token-level where as full indicates predicting the score at the end-of-sentence token. The scores are calculated after removing the few sentences labelled for hallucination detection. Best scores according to each coefficient are highlighted in \textbf{bold}.}
\label{tab:correlation_ablation}
\end{table*}




\end{document}


\appendix
%You may include other additional sections here.
% \clearpage

\section{MAB Framework: Additional Details}
\label{sec:appendix_SMAB}
We summarize the details of the SMAB framework in an Algorithm format in Algorithm~\ref{algo:SMAB}. 

\paragraph{Outer Arm Selection.} We employ the UCB sampling strategy to choose a specific word $w$ at each step $t$ using:
\begin{equation*}
    w^*_{t+1} = \operatorname*{argmax}_{w \in W} \left( Beta(\alpha, \beta) \right)
\end{equation*}

% \begin{equation}
%      w^*_{t+1} = \operatorname*{arg\,max}_{w \in W} \left( \ G^w_{t} + \sqrt{\frac{2*log(1+t)}{1+N^w}} \right),
% \end{equation}
where $W$ is the set of all the words or outer arms. $N^w$ is the no. of times a word $w$ has been picked so far. 

\paragraph{Estimation of Total Regret.}~~In MAB, Total Regret $R_t$ is defined as the total loss we get by not selecting the optimal action up to the step or iteration $t$. Let the outer arm or word $w$ be picked up at the step or iteration $t$. Now, in turn, we will pick up sentences $s_1$ and $s_2$ $\in$ $S_w$. Let $L_w$ be the local sensitivity. Hence, the Total Regret $R_t$ up to the iteration $t$ is defined as:
\begin{equation}
    R_t = R_{t-1} \: + \: ([L_{w^*}\: - \: L_w] \: * \: G^w_t)
\end{equation}
where $G^w_t$ is the Global sensitivity value of (the outer arm) the word $w$ that was picked and $L_{W^*}$ is the highest value of local sensitivity that can be obtained out of the set $S^{w}$.

\paragraph{Comparison of Time Complexity.} Let $|P|$ be the size of the subset, $|D|$ be the size of the dataset i.e. the number of input sentences in the dataset, $|\sum|$ be the total number of words in the dataset, $|V|$ be the vocabulary size of the Language Model, and $cost(f)$ be the cost to use a Language Model for various purposes like classification, Masked Language Modeling.\\
\textbf{Time Complexity for Subset-sensitivity:} For every input sentence in the dataset, we use an MLM to generate $|V|^{|P|}$ perturbed strings and then classify them using an LLM to see if the label flipped. Hence, the Time Complexity is:
\begin{equation}
    O(|D|\cdot |V|^{|P|} \cdot cost(f) )
\end{equation}
For calculating the block sensitivity, the subset sensitivity is calculated K times, corresponding to the K partitions, which can go exponential. \\
\textbf{Time Complexity of SMAB:} Local sensitivity in our algorithm is closely related to the subset sensitivity defined in \citet{hahn-etal-2021-sensitivity}. Our local sensitivity is obtained by taking $|P| = 1$ (singleton sensitivity) and global sensitivity is calculated using equation 3 in our paper, which is O(1). The time complexity of our SMAB algorithm given $T$ as the total number of iterations is:
\begin{equation}
    O(T\cdot (|\sum| \, + \, (|D| \cdot |V|\cdot cost(f))))
\end{equation}
This shows that our proposed algorithm is computationally more efficient than the existing algorithms.
In practice, the top few replacements (perturbations) from the MLM probability distribution are used instead of iterating over all vocabulary symbols.


\begin{algorithm}[htbp]
    \caption{Multi-Armed Bandit Algorithm}
    \label{algo:SMAB}
    \textbf{Input: } A set of words/outer-arms \textbf{W}, Dictionary \textbf{D} containing the set $\textbf{S}^{w}$ of sentences as a \textit{value} for every \textit{key} i.e. word $w \in \textbf{W}$ and total number of iterations $\textbf{T} \gets 200000$.\\
    \textbf{Output: } The set \textbf{G} containing final global-sensitivity values for every word $w \in \textbf{W}$
    % , and the Total Regret \textbf{R}.
    \vspace{0.1cm}
    \hrule % Line after Output
    \vspace{0.1cm}

    \begin{algorithmic}[1]
        \State Initialize the set \textbf{G} as the initial values of the global sensitivities of the words. Here, $|\textbf{G}| = |\textbf{W}|$

        \State Initialize the set \textbf{N} ($|\textbf{N}| = |\textbf{W}|$) to zero. \textbf{N} represent the count of every word $w \in \textbf{W}$.

        \State $t \gets 0$
        \State \textbf{Repeat steps 5 to 9 until $t \neq \textbf{T}$ :}
        \State Select a word $w \in \textbf{W}$ such that 
        \begin{equation*}
            w^*_{t+1} = \operatorname*{argmax}_{w \in W} \left( Beta(\alpha, \beta) \right)
        \end{equation*}
    
        \State $\textbf{S}^{w} \gets \textbf{D}[w^{*}]$, $N^w \gets N^w + 1$
        \State Select two sentences \( s_1, s_2 \in \textbf{S}^{w} \) and calculate Local sensitivity as:
\begin{equation*}
    L_w = \epsilon \cdot r_1 + (1-\epsilon) \cdot r_2
\end{equation*}

        \State Update Global Sensitivity $G^w_t$ as
        \begin{equation*}
            G^{w}_{t} = \frac{(N^w \:* \:G^{w}_{t - 1} \: + \: L_{w})}{ 1 \:+ \:N^w}
        \end{equation*}

        \State Final Global Sensitivity Values: $\textbf{G}[w] \gets G^{w}_{t}$ 

        \State Total Regret $R_t = R_{t-1} \: + \: ([L_{w*}\: - \: L_{w}] \: * \: G^w_t)$
    \end{algorithmic}
\end{algorithm}

\section{\textsc{CheckList}: Additional Results}

In Figure ~\ref{fig:scatter_plot}, we show a scatter plot of the word sensitivities estimated using SMAB (with Thomspon Sampling). As mentioned in the main paper, we observe that words from DIR templates have a larger distribution, present in the $0.2-1$ range, whereas, the words from the invariant test cases (INV template) have lower sensitivities (mostly between $0$ and $0.2$).

\begin{figure}[!t]
\centering
    \includegraphics[height=0.32\textheight]{figures/scatter_checklist_1000_xlabel.png}
    \caption{Scatter plot of estimated global sensitivities of arms of INV and DIR templates using TS. Words from DIR templates have higher estimated global sensitivity and are spread in the whole space as opposed to words from INV templates. %\sa{change the colors to more distinct ones. dark blue, and light red with black border.}
    } 
    \label{fig:scatter_plot}
\end{figure}


\subsection{Dataset details}
\label{sec:Appendix:subsection:dataset details}

\subsubsection{Hate Speech Dataset}
\label{sec:appendix:subsection:hate_speech_details}
In this section, we describe the sources of our compiled dataset for the hate classification task for different languages. The complete statistics for all the languages can be found in Table~\ref{tab:hate_dataset_used}.

\paragraph{English:} We used the \textbf{Stage 1: High Level Categorization} of the \textit{Implicit Hate} dataset provided by \cite{elsherief-etal-2021-latent}. It contains three labels namely implicit\_hate, explicit\_hate, and not\_hate. We converted this multi-classification task into a binary classification task where the labels implicit\_hate, and explicit\_hate are treated as the hate label.
\paragraph{Bengali:} We used a subset of the Bengali Hate speech dataset provided by \cite{romim2020hate} which includes the categories \textit{crime}, \textit{religion}, and \textit{politics} for hate label and all the categories for non-hate label.
\paragraph{Hindi:} For the Hindi language, we combined datasets collected from two different sources: 1) \cite{10.1145/3368567.3368584} provides a binary (hate/no-hate) version of the Hindi dataset. 2) We used a subset of the Hindi dataset (excluding the \textit{fake} category) provided by \cite{bhardwaj2020hostility} which includes the category \textit{non-hostile} for non-hate label and all remaining categories for hate label.

\paragraph{Spanish:}We used two datasets for the Spanish language: 1) HatEval dataset is provided by \cite{basile-etal-2019-semeval}, and 2) \cite{PereiraKohatsu2019DetectingAM} has provided the hate speech dataset in the Spanish language.

% \paragraph{Brazilian Portuguese:} We used the dataset provided by \cite{leite2020toxic} in our experiments.

\begin{table}[tbhp]
    \centering
    \begin{tabular}{c|c|c|c}
        \toprule
        \textbf{Language}   & \textbf{Hate}  & \textbf{No Hate} & \textbf{Total} \\
        \midrule
        English    &  1026 & 1658 & 2684 \\
        Bengali    & 806  & 1194 & 2000 \\
        French     & 1593 & 421 & 2014\\
        German     & 904 & 1607 & 2511\\
        Greek      & 501 & 1100 & 1601\\
        Hindi      & 783  & 549 & 1332 \\
        Spanish    & 1010  & 1290 & 2300 \\
        Turkish    & 400 & 1290 & 1690 \\
       \bottomrule
    \end{tabular}
    \caption{Dataset Statistics for val split of mHate dataset}
    \label{tab:hate_dataset_used}
\end{table}


\section{KLD v/s Accuracy Experiments}
\label{sec:appendix:subsection:kld_vs_acc}


\paragraph{Correlation Within Langauge.}
We experiment with various models on mHate and XNLI dataset. We use 5 different target classifiers for a language to estimate global sensitivities. We use XLM-R \cite{conneau2020unsupervisedcrosslingualrepresentationlearning}, mBERT, mDeBERTa \cite{he2021debertadecodingenhancedbertdisentangled}, FlanT5-L \cite{chung2022scalinginstructionfinetunedlanguagemodels} and GPT-3.5 for predictions on mHate and XNLI dataset. For a given language, we then compare the difference in sensitivity distributions of different models with respect to a base model (XLM-R), measured as KL Divergence with the performance of different models on that langauge (accuracy). We plot KLD v/s Accuracy plots for different models under study. 

\begin{figure}[htbp]
    \includegraphics[width=\columnwidth]{figures/kld_vs_accuracy/kld_ts_wl_mhate.png}
    \caption{KLD v/s accuracy within language on mHate-English dataset.}
    \label{fig:mhate_within_languges}
\end{figure}

\section{Improving Paraphrase Attack with Local Sensitivity}
\label{appendix:paraphraseattack_details}
% In this section, we provide additional details about improving paraphrase attacks using local sensitivity. In Table~\ref{tab:paraphrase_attack_HS}, we show the flip success rates for different variants on the hate speech dataset.


\subsection{Senisitivity Reward Calculation}
\label{appendix:sensitvity_reward_calc}
We adjust the original reward function $R(x, x')$ by incorporating an additional \textbf{Sensitivity reward} $S(x, x')$ weighted by the scaling constant $\alpha \in (0, 1)$ as shown:
\begin{equation}
    R(x, x') = R(x, x') + \alpha \, S(x, x').
\end{equation}
Here, $S(x, x')$ is the difference between the sensitivity of the input text $x$ and the sensitivity of the generated text 
$x'$: $S(x, x') = s(x) - s(x')$.
The sensitivity of a text is calculated as 
\begin{equation}
    s(x) = \frac{\sum_{i=1}^{m} \sum_{j=1}^{n_i} L_{s}^{ij}}{\sum_{i=1}^{m} n_i}
\end{equation}
where ${m}$ represents the total number of keyphrases, ${n_i}$ represents the number of words in the i-th keyphrase and $L_{s}^{ij}$ represents the local sensitivity of $j^{th}$ word in the $i^{th}$ keyphrase. For our experiments, we used Scaling constant $\alpha = 0.25$. We extract the keyphrases from a text using a TopicRank keyphrase extraction model using an open source toolkit \texttt{pke} \cite{boudin:2016:COLINGDEMO}.


\begin{table*}[t]
\centering
\resizebox{0.7\textwidth}{!}{%
\begin{tabular}{l|l|c}
            \toprule
            \textbf{Purpose} & \textbf{Model}  &\textbf{Threshold}\\ 
            \midrule
            Paraphraser & \texttt{prithivida/parrot\_paraphraser\_on\_T5}  & - \\ 
            \midrule
            \begin{tabular}[c]{@{}l@{}}Target (RT) \end{tabular} & \texttt{textattack/distilbert-base-uncased-rotten-tomatoes}  & - \\
            \midrule
             % \begin{tabular}[c]{@{}l@{}}Victim (HS) \end{tabular}& \texttt{bert-base-multilingual-cased}  & - \\
             % \midrule
             \begin{tabular}[c]{@{}l@{}}Sensitivity (RT) \end{tabular} & \texttt{textattack/bert-base-uncased-rotten-tomatoes}  
            & - \\ 
            \midrule
            % \begin{tabular}[c]{@{}l@{}}Sensitivity (HS) \end{tabular} & Finetuned \texttt{bert-base-multilingual-cased}  & - \\ 
            % \midrule
            \begin{tabular}[c]{@{}l@{}}Linguistic \\ Acceptability\end{tabular} & \texttt{textattack/albert-base-v2-CoLA}   &0.5\\ 
            \midrule
            \begin{tabular}[c]{@{}l@{}}Semantic \\ Consistency\end{tabular} & \texttt{sentence-transformers/paraphrase-MiniLM-L12-v2}   &0.8\\
            \midrule
            \begin{tabular}[c]{@{}l@{}}Label \\ Invariance\end{tabular} & \texttt{howey/electra-small-mnli}  &0.2\\
            \midrule
\end{tabular}%
}
\caption{Various models as target model, in sensitivity reward functions and constraints used in Paraphrase Attack using Local Sensitivity based adversarial example generation experiment. \textbf{RT}: Rotten Tomatoes dataset. }
% \textbf{HS}: Hate speech dataset.}
\label{tab:various_models_used_paraphrase}
\end{table*}

\begin{table}[htbp]
\resizebox{\columnwidth}{!}{%
    \centering
    \begin{tabular}{l c c c}
    \toprule
    \multirow{1}{*}{\textbf{\begin{tabular}[c]{@{}l@{}}Strategy\end{tabular}}} &
    \multirow{1}{*}{\textbf{\begin{tabular}[c]{@{}l@{}}Temp \end{tabular}}} & 
      \textbf{\begin{tabular}[c]{@{}l@{}}Total generated \\ paraphrases\end{tabular}} & 
      \textbf{\begin{tabular}[c]{@{}l@{}}Total used \\paraphrases \end{tabular}} \\ 
    \midrule
    
    \multicolumn{4}{c}{\rule{0pt}{2ex} SMAB Model $\rightarrow$ \ding{55}}\\
    \hline

    DBS + (\ding{55}) & 0.85  & 62 & 62\\
    BS + (\ding{55}) & 0.85 & 53 & 53\\
    BS + (\ding{55})  & 1.00 & 110 & 100\\
    BS + (\ding{55})  & 1.15 & 68 & 68\\                       
    \hline

    \multicolumn{4}{c}{\rule{0pt}{2ex} SMAB Model $\rightarrow$ \textbf{BERT}}\\
    \hline
    DBS + (\ding{51})  & 0.85 & 181 & 100\\
    BS + (\ding{51}) & 0.85 & 71 & 71\\
    BS + (\ding{51})  & 1.00 & 79 & 79\\
    BS + (\ding{51}) & 1.15 & 183 & 100\\
    \hline

    \multicolumn{4}{c}{\rule{0pt}{2ex} SMAB Model $\rightarrow$ \textbf{DistilBERT}}\\
    \hline
    DBS + (\ding{51})   & 0.85  & 319 & 100\\
    BS + (\ding{51})  & 0.85  & 140 & 100\\ 
    BS + (\ding{51})   & 1.00  & 334 & 100\\
    BS + (\ding{51})  & 1.15  & 75 & 75\\ 
    \bottomrule
\end{tabular}%
}
\caption{Total unique paraphrases generated by each model variation from a total of 359 test instances that resulted in a label flip and the total paraphrases selected for human evaluation for the \textbf{Rotten Tomatoes} dataset.}
\label{tab:UniqueParaphrase_RT_appendix}
\end{table}


% Here is the 7.2 table

\begin{table}[htbp]
\resizebox{\columnwidth}{!}{%
    \centering
    \begin{tabular}{l c c c}
    \toprule
    \multirow{1}{*}{\textbf{\begin{tabular}[c]{@{}l@{}}Strategy\end{tabular}}} &
    \multirow{1}{*}{\textbf{\begin{tabular}[c]{@{}l@{}}Temp \end{tabular}}} &
    \multirow{1}{*}{\textbf{\begin{tabular}[c]{@{}l@{}}FSR $\uparrow$ \end{tabular}}} & \textbf{\begin{tabular}[c]{@{}l@{}}After Attack \\ Accuracy $\downarrow$ \end{tabular}} \\
    \hline
    
    \multicolumn{4}{c}{\rule{0pt}{1ex} SMAB Model $\rightarrow$ \ding{55}}\\
    \hline

    DBS + (\ding{55}) & 0.85  & 24.79 & 69.35 \\
    BS + (\ding{55}) & 0.85 & 15.87  & 82.72 \\ 
    BS + (\ding{55})  &1.00 &  17.54  & 81.05 \\
    BS + (\ding{55})  & 1.15 & 13.64 & 85.23 \\  
    \hline

    \multicolumn{4}{c}{\rule{0pt}{1ex} SMAB Model $\rightarrow$ \textbf{BERT}}\\
    \hline
    DBS + (\ding{51})  & 0.85 & \cellcolor{green1}44.87  & \cellcolor{green1}49.58 \\
    BS + (\ding{51}) & 0.85 & 18.10  & 80.22 \\
    BS + (\ding{51})  & 1.00  & 20.33  & 77.99 \\
    BS + (\ding{51}) & 1.15 & \cellcolor{green1}47.62  & \cellcolor{green1}49.02 \\
    \hline

    \multicolumn{4}{c}{\rule{0pt}{1ex} SMAB Model $\rightarrow$ \textbf{DistilBERT}}\\
    \hline
    DBS + (\ding{51})   & 0.85  & \cellcolor{green1}86.35  & \cellcolor{green1}11.14 \\
    BS + (\ding{51})  & 0.85  & \cellcolor{green1}35.93  & \cellcolor{green1}61.00 \\ 
    BS + (\ding{51})   & 1.00  & \cellcolor{green2}92.20  & \cellcolor{green2}6.96 \\
    BS + (\ding{51})  & 1.15  & 20.33 & 79.10 \\
    \bottomrule
\end{tabular}%
}
\caption{ParaphraseAttack Results on Rotten Tomatoes (\textbf{RT}) dataset. The strategy represents the combination of decoding mechanism (\textbf{DBS}: Diverse Beam Search, \textbf{BS}: Beam Search) and usage of sensitivity reward (\ding{55}: No sensitivity, \ding{51}: Sensitivity using bert/distilbert). Temp: decoding temperature. FSR/After Attack Accuracy in ($\%$). \textbf{Before Attack Accuracy} is 100\%. All variants outperforming the highest baseline score are highlighted in \colorbox{green1}{green} and the best performing variant is highlighted in \colorbox{green2}{green}.}
\label{tab:paraphrase_attack_RT_appendix}
\end{table}



% SMAB table %
\begin{table*}[htbp]
\resizebox{\textwidth}{!}{
    \begin{tabular}{|c|c|l|c|c|c|c|}
    \hline
    \textbf{Task}  & \textbf{Dataset} & \multicolumn{1}{c|}{\textbf{Target Classifier}} & \textbf{Languages} & \textbf{\#datapoints} & \textbf{\#arms} & \textbf{\#edges} \\ 
    \hline
    
    \multirow{5}{*}{\begin{tabular}[c]{@{}c@{}}Sentiment \\ Analysis\end{tabular}}          & CheckList & cardiffnlp/twitter-roberta-base-sentiment-latest & english & 34486                 & 8498 & 52359 \\ 
    \cline{2-7} & RT & textattack/distilbert-base-uncased-rotten-tomatoes & english & 1066 & 5104 & 18835 \\ \cline{2-7}  & \multirow{3}{*}{\begin{tabular}[c]{@{}c@{}}SST-2 \end{tabular}} & \multirow{3}{*}{\begin{tabular}[c]{@{}l@{}}distilbert/distilbert-base-uncased-finetuned-sst-2-english \\ gpt-3.5-turbo\\ meta-llama/Llama-2-7b \end{tabular}} & english & 872  & 4088  & 8041 \\
    \cline{4-7} & & & english & 872 & 4088 & 8041 \\
    \cline{4-7} & & & english & 872 & 4088 & 8041 \\
    \hline
    
    \multirow{9}{*}{\begin{tabular}[c]{@{}c@{}}Hate \\ Classification\end{tabular}}         & \multirow{9}{*}{\begin{tabular}[c]{@{}c@{}}Multilingual\\ hate\end{tabular}} & \multirow{9}{*}{\begin{tabular}[c]{@{}l@{}}google-bert/bert-base-multilingual-uncased\\ google/flan-t5-large\\ MoritzLaurer/deberta-v3-base-zeroshot-v1\\ gpt-3.5-turbo\end{tabular}} & bengali & 2000 & 7880 & 18929 \\ 
    \cline{4-7}  &   &  & english & 2684 & 7061 & 23149 \\ 
    \cline{4-7} &  &  & french  & 2014 & 5680 & 14184 \\ 
    \cline{4-7}  &   &   & german & 2511 & 11767 & 24226\\ 
    \cline{4-7}  &  &  & greek & 1601 & 9799 & 22031 \\ 
    \cline{4-7}  &  &   & hindi  & 1332 & 7181 & 20140 \\ 
    \cline{4-7} & & & italian & 1846 & 7551 & 17876 \\ \cline{4-7}  &  &  & spanish & 2150 & 8780            & 21982 \\ 
    \cline{4-7}  & &  & turkish & 1690 & 12308 & 20631  \\ \hline
    
    \multirow{6}{*}{\begin{tabular}[c]{@{}c@{}}Natural\\ Language\\ Inference\end{tabular}} & \multirow{6}{*}{XNLI} & \multirow{6}{*}{MoritzLaurer/mDeBERTa-v3-base-mnli-xnli} & english & 5010  & 9543  & 68156 \\ \cline{4-7}  & & & french & 5010 & 12431 & 75322  \\ \cline{4-7}  & & & german & 5010 & 13099 & 61780 \\ \cline{4-7}  & & & greek  & 5010  & 15114  & 90149 \\ \cline{4-7} & & & hindi & 5010 & 9484 & 70938 \\ \cline{4-7}  &  &  & spanish & 5010 & 11994 & 69166 \\ \hline
    \end{tabular}
    }
    \caption{Training details of SMAB for each task and target classifier and the langauges. \#datapoints represent the number of sentences in the training set, \#arms represents the unique words present in the dataset after preprocessing.}
    \label{tab: SMAB_training_details}
\end{table*}

\subsection{Key Phrase sensitivity Calculation}
\label{Keyphrase_sensitivity}
We present the key-phrase sensitivity estimation process in a pseudo-code form in Algorithm \ref{algo:alg_kp_sensitivity}. This algorithm takes as input the original text $x$ and calculates its sensitivity. The algorithm has three main steps: In the \textbf{first step}, it extracts the key phrases from the input text $x$ using the Topic Rank Key phrase extraction Model $M$ and stores them in a list $K$. This first step is shown in line $6$ of the pseudo-algorithm. In the \textbf{second step}, it calculates the sensitivities of each key phrase. The algorithm iterates through all the key phrases stored in the list $K$ and does the following: \circled{1} It extracts all the words that are present in the current key phrase and stores it in the list $words$. Similarly, it maintains a global variable $Total\_words$ that will store the total number of words extracted from each Keyphrase. \circled{2} It creates a masked sentence $x_{masked}$ from the input text $x$ by replacing all the words that are present in the current key phrase with [MASK]. \circled{3} It then uses \texttt{bert-large-uncased} to generate $10$ perturbed sentences by predicting new words at the locations where [MASK] is present in $x_{masked}$. These $10$ perturbations are stored in the list $Masked\_output$. \circled{4} The algorithm then predicts the label of the input text $x$ and $10$ perturbed sentence present in the list $Masked\_output$ and stores the predictions in the variable $P rediction\_original$ and in the list $Prediction\_List$ respectively. \circled{5} Then, it iterates through all the labels in the $Prediction\_List$ and increments the variable $flips$ if the $P rediction\_original$
doesn't match the current label. 
 \circled{6} Finally, it calculates the local sensitivity of the current keyphrase as the proportion of the total flips that we got out of the total labels i.e. $flips / \texttt{len}(Prediction\_List)$ and stores it in a list $Keyphrase\_sensitivity$ at the index corresponding to the current key phrase. This second step is shown in lines $7$ to $25$ in the pseudo-algorithm. In the \textbf{third step}, the algorithm adds up all the values in the list $Keyphrase\_sensitivity$ and divides it by $Total\_words$ to get the sensitivity of the input text $x$. This third step is shown in lines $27$ to $33$ in the pseudo-algorithm.

\subsection{Qualitative analysis of the type of Attacks}
\label{sec:appendix_qual_examples_RT}
Table \ref{tab:qual_examples_PP} shows the three different types of attacks carried out by different variants when using \textbf{distilbert} as the SMAB Model. \circled{1} \textbf{Type 1} attack involves generating adversarial examples by replacing one or more words in a sentence with new words and rephrasing the original sentence. Model variant with decoding strategy \textbf{BS + (\ding{51})} and a temperature of $0.85$  has learned to carry out \textbf{Type 1} attacks. \circled{2} \textbf{Type 2} attack involves generating adversarial examples by removing one or more words from the original sentence while preserving its semantic meaning. Model variant with decoding strategy \textbf{BS + (\ding{51})} and a temperature of $1.15$  has learned to carry out \textbf{Type 2} attacks. \circled{3} \textbf{Type 3} attack involves appending various suffixes to sentences, such as adding \textit{but it's true}, or words like \textit{but why} followed by a \textit{?}. Model variant with decoding strategy \textbf{BS + (\ding{51})} and a temperature of $1$  has learned to carry out \textbf{Type 3} attacks.

% Qualitative Examples - ParaphraseAttack %
\begin{table}[htbp!]
\scriptsize 
\begin{tabular}{l l r}
\toprule
\textbf{Attack} & \textbf{Examples} & \textbf{Flip Result} \\
\midrule
\begin{tabular}[c]{@{}l@{}}Type 1\end{tabular} & \begin{tabular}[c]{@{}l@{}} \underline{Original}: smarter than its commercials make \\it seem. \\ 

\underline{Perturbed}: 
it's smarter than the commercials \\make it appear. \\

\hline
\\
\underline{Original}: it has become apparent that the \\franchise's best years are long past. \\ 
\underline{Perturbed}: it's clear that the best years of \\the franchise are long gone.\end{tabular} & \begin{tabular}[c]{@{}r@{}}\texttt{\textcolor{blue}{pos}} $\rightarrow$ \texttt{\textcolor{red}{neg}}\\ \\ \\ \\ \texttt{\textcolor{red}{neg}} $\rightarrow$ \texttt{\textcolor{blue}{pos}}\end{tabular} \\ 
\midrule
\begin{tabular}[c]{@{}l@{}}Type 2\end{tabular} & \begin{tabular}[c]{@{}l@{}} \underline{Original}: crush is so warm and fuzzy you \\might be able to forgive its mean-spirited \\second half. \\ 

\underline{Perturbed}: crush is so warm and fuzzy you \\might forgive its mean-spirited second half \\ 
\hline
\\
\underline{Original}: you can practically hear george \\orwell turning over.
\\ 

\underline{Perturbed}: You can hear george orwell \\turning over. \\ 

\end{tabular} & \begin{tabular}[c]{@{}r@{}}\texttt{\textcolor{blue}{pos}} $\rightarrow$ \texttt{\textcolor{red}{neg}}\\ \\ \\ \\ \\ \texttt{\textcolor{red}{neg}} $\rightarrow$ \texttt{\textcolor{blue}{pos}}\end{tabular} \\ 
\midrule
\begin{tabular}[c]{@{}l@{}}Type 3\end{tabular} & \begin{tabular}[c]{@{}l@{}}\underline{Original}: provides a porthole into that noble, \\trembling incoherence that defines us all. \\ 

\underline{Perturbed}: It provides a porthole into that \\noble trembling incoherence that defines us \\all. But why? \\ 
\hline
\\
\underline{Original}: this feature is about as necessary \\as a hole in the head.
\\ 

\underline{Perturbed}: This feature is about as necessary \\as a hole in the head. But it's true. \\ 

\end{tabular} & \begin{tabular}[c]{@{}r@{}}\texttt{\textcolor{blue}{pos}} $\rightarrow$ \texttt{\textcolor{red}{neg}}\\ \\ \\ \\ \\ \\ \texttt{\textcolor{red}{neg}} $\rightarrow$ \texttt{\textcolor{blue}{pos}}\end{tabular} \\
\bottomrule
\end{tabular}
\caption{Examples of successful adversarial attacks across the three different attack types when using DistilBERT as the SMAB Model. The \textbf{Flip Result} column shows label changes from \texttt{Original Label} $\rightarrow$ \texttt{New Label}, where \texttt{pos} and \texttt{neg} denote \textit{positive} and \textit{negative} classes respectively in the \textbf{Rotten Tomatoes} dataset.}
\label{tab:qual_examples_PP}
\end{table}


% Classification Prompts Table %
\begin{table*}[htbp!]
\resizebox{\textwidth}{!}{%
\centering
    \begin{tabular}{l | l }
    \toprule
    \textbf{\begin{tabular}[c]{@{}l@{}}LLM\end{tabular}} &
    \textbf{\begin{tabular}[c]{@{}l@{}}Prompt \end{tabular}} \\ 
    \midrule
    \textbf{GPT-3.5} & \begin{tabular}[c]{@{}l@{}}Given a sentence that is a movie review, your task is to assign a label based on its sentiment. Label 1 if the sentence is a \\ positive review and Label 0 if the sentence is a negative review. Remember to only provide the label.\\
    Sentence: [input\_sentence]\\
    Label: \end{tabular} \\ 
    \midrule
    \textbf{Llama-2-7B} & \begin{tabular}[c]{@{}l@{}}Please label the sentiment of the given movie review text. The sentiment label should be "positive" or "negative". \\
    Answer only a single word for the sentiment label. Do not leave answer as empty. Do not generate any extra text. \\ 
    Text: [input\_sentence] \\
    Answer: \end{tabular} \\ 
    \midrule
    \textbf{Qwen-2.5-7B} & \begin{tabular}[c]{@{}l@{}}Please label the sentiment of the given movie review text. The sentiment label should be "positive" or "negative". \\
    Answer only a single word for the sentiment label. Do not leave answer as empty. Do not generate any extra text. \\ 
    Text: [input\_sentence] \\
    Answer: \end{tabular} \\
    \bottomrule
\end{tabular}%
}
\caption{Prompts used for sentiment classification task on SST-2 dataset}
\label{tab:sentiment_classification_prompt}
\end{table*}


% Qualitative examples (PromptAttack) %
\begin{table*}[htbp]
\resizebox{\textwidth}{!}{%
\centering
    \begin{tabular}{l l l}
    \toprule
    \textbf{\begin{tabular}[c]{@{}l@{}}Perturb \\Type\end{tabular}} &
    \textbf{\begin{tabular}[c]{@{}l@{}}Qualitative \\Example \end{tabular}} &
    \textbf{\begin{tabular}[c]{@{}l@{}}Flip \\ Result \end{tabular}}\\ 
    \midrule
    $W4$ & \text{\begin{tabular}[c]{@{}l@{}} \textbf{Original Sentence:} the title not only describes its main characters, but the lazy people behind the camera as well.\\
    \textbf{New Sentence:} The title not only portrays its main protagonists, but also the laid-back crew behind the camera as well.
    \end{tabular}} & \textcolor{red}{neg} $\rightarrow$ \textcolor{blue}{pos}              \\ 
    \midrule
    $W5$ & \text{\begin{tabular}[c]{@{}l@{}} \textbf{Original Sentence:} an important movie, a reminder of the power of film to move us and to make us examine our values.\\
    \textbf{New Sentence:} an important movie, a reminder of the influence of cinema to move us and to force us to question our beliefs.
    \end{tabular}} & \textcolor{blue}{pos} $\rightarrow$ \textcolor{red}{neg}\\ 
    \midrule
    $W6$ & \text{\begin{tabular}[c]{@{}l@{}} \textbf{Original Sentence:} It's a charming and often affecting journey.\\
    \textbf{New Sentence:} It's a charming but sometimes disconcerting journey.
    \end{tabular}} & \textcolor{blue}{pos} $\rightarrow$ \textcolor{red}{neg}\\
    \bottomrule
\end{tabular}%
}
\caption{Qualitative examples
for perturbation types $W4$, $W5$, and $W6$ when using Llama-2-13B both as the as the \textsc{Smab} Model and the target Model. The \textbf{Flip Result} column shows label changes from \texttt{Original Label} $\rightarrow$ \texttt{New Label}, where \texttt{pos} and \texttt{neg} denote \textit{positive} and \textit{negative} classes respectively in the \textbf{SST-2} dataset. In $W4$, the placeholders "\texttt{[Word1]}", "\texttt{[Word2]}", \texttt{GS1}, and \texttt{GS2} are substituted with \textit{characters}, \textit{people}, 0.9865988772394045, and 0.968535617081986 respectively. In $W5$, the placeholders "\texttt{[Word1]}" and "\texttt{[Word2]}" are substituted with \textit{make} and \textit{film} respectively. In $W6$, the placeholders "\texttt{[Word1]}" and "\texttt{[Word2]}" are substituted with \textit{affecting} and \textit{often} respectively.}
\label{tab:Qual_Examples_promptAttack}
\end{table*}

% Keyphrase sensitivity algo %
\clearpage

\begin{algorithm*}[htbp]
\begin{minipage}{\textwidth}
\begin{algorithmic}[1]
\INPUT $x \gets$ Input Text
\Require $M \gets$ TopicRank keyphrase extraction model, $C \gets$ Classifier Model, $m \gets$ \texttt{bert-large-uncased} Model 
\REPRESENTATION $G(x) \gets$ Gold label of the Input Text $x$, $C(x) \gets$ Predicted label of the Input Text $x$ using the Classifier Model $C$, $m(x) \gets$ generates $K = 10$ perturbations of the input text x using the model $m$
\OUTPUT $s \gets$ Sensitivity of the Input Text

\State \textcolor{blue}{// Initialization}
\State $s \gets 0$ \Comment{\textcolor{blue}{This variable will store the sensitivity of the input text $x$.}}
\State $Total\_words \gets 0$ \Comment{\textcolor{blue}{This variable represents the total number of words across all the keyphrases.}}
\State $Keyphrase\_sensitivity \gets \{\}$ \Comment{\textcolor{blue}{A dictionary to store the sensitivity values of each keyphrase.}}

\State \textcolor{blue}{// Extract keyphrases using the TopicRank Model $M$ and store them in the list $K$.}
\State $K \gets M(x)$

\State \textcolor{blue}{// Main algorithm}
\For{$k$ \textbf{in} $K$} 
    \State $words \gets k.\texttt{split()}$ \Comment{\textcolor{blue}{Get a list of all the words in the current keyphrase.}}
    \State $Total\_words \mathrel{+}= \texttt{len}(words)$

    \For{$w$ \textbf{in} $words$}
        \State $x_{masked} \gets$ mask the word $w$ in the text $x$ using "[MASK]"
    \EndFor

    \State $Masked\_output \gets m(x_{masked})$ \Comment{\textcolor{blue}{Store all the perturbed samples in this list.}}
    \State $Prediction\_List \gets C(Masked\_output)$ \Comment{\textcolor{blue}{Store the prediction labels of all the perturbed samples in this list.}}
    \State $Prediction\_original \gets C(x)$ \Comment{\textcolor{blue}{Store the prediction label of the input text $x$.}}

    \State $flips \gets 0$ \Comment{\textcolor{blue}{A temporary variable to count the flips.}}
    \For{$labels$ \textbf{in} $Prediction\_List$}
        \If{$labels \neq Prediction\_original$}
            \State $flips \mathrel{+}= 1$
        \EndIf
    \EndFor
    \State $local\_sensitivity \gets flips / \texttt{len}(Prediction\_List)$ \Comment{\textcolor{blue}{Total flips normalized by the total number of labels.}}
    \State $Keyphrase\_sensitivity[k] \gets local\_sensitivity$
\EndFor

\State \textcolor{blue}{// Calculate the sensitivity of the input text $x$.}
\If{$Total\_words \neq 0$}
    \State $t \gets 0$ \Comment{\textcolor{blue}{A temporary variable.}}
    \For{$value$ \textbf{in} $Keyphrase\_sensitivity.\texttt{values}()$}
        \State $t \mathrel{+}= value$
    \EndFor
    \State $s \gets t / Total\_words$
\EndIf

\Return $s$
\end{algorithmic}
\end{minipage}
\caption{Algorithm to calculate the sensitivity of Input Text}
\label{algo:alg_kp_sensitivity}

\end{algorithm*}



\end{document}