\documentclass[a4paper,11pt]{article}

% --- Packages ---
\usepackage{amsmath, amssymb, amsthm}  % Math symbols and formatting
\usepackage{graphicx, subfigure, wrapfig}  % Images
\usepackage{hyperref}  % Clickable links
\usepackage{geometry}  % Page layout
\usepackage{titlesec}  % Custom section formatting
\usepackage{natbib}  % Bibliography support
\usepackage{lmodern}  % Better font rendering
\usepackage{microtype}  % Better typography
\usepackage{fancyhdr}  % Headers/footers
\usepackage{booktabs, multirow}  % Professional tables
\usepackage{algorithm, algpseudocode}  % Algorithms
\usepackage{xcolor}  % Colors
\usepackage{enumitem}  % Custom list formatting
\usepackage{cleveref}  % Smart cross-referencing
\usepackage{framed}  % Framing text
\usepackage{caption}  % Custom captions
\usepackage{mathabx}  % Provides \bigtimes
\usepackage{mathtools}  % Enhanced math notation
\usepackage{forloop}  % Looping in LaTeX
\usepackage{eso-pic}  % Background elements
\usepackage[normalem]{ulem}  % Better underlining (without modifying \emph)
\usepackage{bm}  % Bold math symbols

% Custom macros (Ensure these files exist)
\usepackage{utils/amir_macros}
\def\vnu{\nu^{(v)}}
\def\mnu{\nu^{(m)}}
\def\pnu{\nu^{(p)}}
\def\pomega{\omega^{(p)}}
\def\pphi{\phi^{(p)}}
\def\bfpphi{\bm \phi^{(p)}}


\def\barphi{\bar{\phi}}
\def\barnu{\bar{\nu}}
\def\barbfv{\bar{\bfv}}

\def\bfD{{\bf D}}
\def\bfZ{{\bf Z}}
\def\bfW{{\bf W}}
\def\bfB{{\bf B}}
\def\bfE{{\bf E}}
\def\bfV{{\bf V}}
\def\bfA{{\bf A}}
\def\bfR{{\bf R}}
\def\bfb{{\bf b}}
\def\bff{{\bf f}}

\def\tilbfX{{\bf \tilde{X}}}
\def\bfphi{{\bm \phi}}

%%%% Raccourcis pour les caract�res gras math�matiques (ensembles R, N, Z, C etc)

\def\WW{{\mathbb W}}
\def\NN{{\mathbb N}}    %naturels
\def\ZZ{{\mathbb Z}}     %entiers relatifs
\def\RR{{\mathbb R}}    %r�els
\def\QQ{{\mathbb Q}}    %r�els
\def\CC{{\mathbb C}}    %complexes
\def\HH{{\mathbb H}}    %quaternions / espace hyperbolique
\def\PP{{\mathbb P}}     %espace projectif / probabilité
\def\KK{{\mathbb K}}     %corps quelconques
\def\EE{{\mathbb E}}    % espérance
\def\VV{{\mathbb V}}
\def\11{{\mathbf 1}}    % indicatrice
\def\AA{{\mathbb A}}

%%%%%%raccourcis lettres calligraphi�es
\def\cA{{\mathcal A}}  \def\cG{{\mathcal G}} \def\cM{{\mathcal M}} \def\cS{{\mathcal S}} \def\cB{{\mathcal B}}  \def\cH{{\mathcal H}} \def\cN{{\mathcal N}} \def\cT{{\mathcal T}} \def\cC{{\mathcal C}}  \def\cI{{\mathcal I}} \def\cO{{\mathcal O}} \def\cU{{\mathcal U}} \def\cD{{\mathcal D}}  \def\cJ{{\mathcal J}} \def\cP{{\mathcal P}} \def\cV{{\mathcal V}} \def\cE{{\mathcal E}}  \def\cK{{\mathcal K}} \def\cQ{{\mathcal Q}} \def\cW{{\mathcal W}} \def\cF{{\mathcal F}}  \def\cL{{\mathcal L}} \def\cR{{\mathcal R}} \def\cX{{\mathcal X}} \def\cY{{\mathcal Y}}  \def\cZ{{\mathcal Z}}

\def\cPone{{\mathcal P^{(1)}}}
\def\cPtwo{{\mathcal P^{(2)}}}
\def\ptwo{p^{(2)}}
\def\pone{p^{(1)}}
\def\Kone{K^{(1)}}
\def\Ktwo{K^{(2)}}

%%%%%%raccourcis lettres gothiques

\def\mfA{{\mathfrak A}} \def\mfA{{\mathfrak P}} \def\mfS{{\mathfrak S}}\def\mfZ{{\mathfrak Z}} \def\mfM{{\mathfrak M}} \def\mfQ{{\mathfrak Q}} \def\mfE{{\mathfrak E}} \def\mfL{{\mathfrak L}} \def\mfW{{\mathfrak W}} \def\mfR{{\mathfrak R}} \def\mfK{{\mathfrak K}} \def\mfX{{\mathfrak X}} \def\mfT{{\mathfrak T}} \def\mfJ{{\mathfrak J}} \def\mfC{{\mathfrak C}} \def\mfY{{\mathfrak Y}} \def\mfH{{\mathfrak H}} \def\mfV{{\mathfrak V}}\def\mfU{{\mathfrak U}}\def\mfG{{\mathfrak G}} \def\mfB{{\mathfrak B}} \def\mfI{{\mathfrak I}} \def\mfF{{\mathfrak F}} \def\mfN{{\mathfrak N}} \def\mfO{{\mathfrak O}} \def\mfD{{\mathfrak D}} 

\def\mfa{{\mathfrak a}} \def\mfp{{\mathfrak p}} \def\mfs{{\mathfrak s}}  \def\mfz{{\mathfrak z}} \def\mfm{{\mathfrak m}} \def\mfq{{\mathfrak q}}  \def\mfe{{\mathfrak e}} \def\mfl{{\mathfrak l}} \def\mfw{{\mathfrak w}} \def\mfr{{\mathfrak r}} \def\mfk{{\mathfrak k}} \def\mfx{{\mathfrak x}} \def\mft{{\mathfrak t}} \def\mfj{{\mathfrak j}} \def\mfc{{\mathfrak c}} \def\mfy{{\mathfrak y}} \def\mfh{{\mathfrak h}} \def\mfv{{\mathfrak v}} \def\mfu{{\mathfrak u}} \def\mfg{{\mathfrak g}} \def\mfb{{\mathfrak b}} \def\mfi{{\mathfrak i}} \def\mff{{\mathfrak f}} \def\mfn{{\mathfrak n}} \def\mfo{{\mathfrak o}} \def\mfd{{\mathfrak d}}


\def\muhat{{\hat{\mu}}}

%%%%%%raccourcis lettres gras
\def\boldx{{\boldsymbol x}} \def\boldt{{\boldsymbol t}}
\def\bfx{{\bf x}} \def\bfy{{\bf y}} \def\bfz{{\bf z}} \def\bfw{{\bf w}}
\def\bfk{{\bf k}} \def\bfK{{\bf K}} \def\bfell{{\bf \ell}}
\def\bfL{{\bf L}} \def\bfQ{{\bf Q}} \def\bfA{{\bf A}}
\def\bfPhi{{\bf \Phi}} \def\bfPsi{{\bf \Psi}}
\def\boldupsilon{\boldsymbol{\Upsilon}}
\def\bfeta{\boldsymbol{\eta}} \def\bfSigma{\boldsymbol{\Sigma}}
\def\bfb{{\bf b}}
\def\bfv{{\bf v}}
\def\bfX{{\bf X}} \def\bfY{{\bf Y}} 
\def\bfKXX{{\bf K}_{\bf XX}}
\def\bfM{{\bf M}}


\def\btheta{\boldsymbol{\theta}}

%%%%%%raccourcis lettres romaines
\def\rmC{{\mathrm C}}
\def\rmD{{\mathrm D}}
\def\rmc{{\mathrm c}}
\def\rmd{{\mathrm d}}

%%%% Tilde notation
\def\tilx{{\tilde{x}}}
\def\tily{{\tilde{y}}}
\def\tilm{\tilde{m}}
\def\tilk{\tilde{k}}
\def\tilbfx{{\bf \tilde{x}}}

\def\balpha{\boldsymbol{\alpha}}

%%%%%%raccourcis lettres gras
%\def\ba{{\mathbf a}} \def\bb{{\mathbf b}} \def\bc{{\mathbf c}} \def\bd{{\mathbf d}} \def\be{{\mathbf e}} \def\bf{{\mathbf f}} \def\bg{{\mathbf g}} \def\bh{{\mathbf h}} \def\bi{{\mathbf i}} \def\bj{{\mathbf j}} \def\bk{{\mathbf k}} \def\bl{{\mathbf l}} \def\bm{{\mathbf m}} \def\bn{{\mathbf n}} \def\bo{{\mathbf o}} \def\bp{{\mathbf p}} \def\bq{{\mathbf q}} \def\br{{\mathbf r}} \def\bs{{\mathbf s}} \def\bt{{\mathbf t}} \def\bu{{\mathbf u}} \def\bv{{\mathbf v}} \def\bw{{\mathbf w}} \def\bx{{\mathbf x}} \def\by{{\mathbf y}} \def\bz{{\mathbf z}} 

%\def\bA{{\mathbf A}} \def\bB{{\mathbf B}} \def\bC{{\mathbf C}} \def\bD{{\mathbf D}} \def\bE{{\mathbf E}} \def\bF{{\mathbf F}} \def\bG{{\mathbf G}} \def\bH{{\mathbf H}} \def\bI{{\mathbf I}} \def\bJ{{\mathbf J}} \def\bK{{\mathbf K}} \def\bL{{\mathbf L}} \def\bM{{\mathbf M}} \def\bN{{\mathbf N}} \def\bO{{\mathbf O}} \def\bP{{\mathbf P}} \def\bQ{{\mathbf Q}} \def\bR{{\mathbf R}} \def\bS{{\mathbf S}} \def\bT{{\mathbf T}} \def\bU{{\mathbf U}} \def\bV{{\mathbf V}} \def\bW{{\mathbf W}} \def\bX{{\mathbf X}} \def\bY{{\mathbf Y}} \def\bZ{{\mathbf Z}} 



%%%%%%random variables
\newcommand{\indep}{\perp \!\!\!\!\!\; \perp}

%%%%%%definition d applications
    
\newcommand{\deffunction}[5]{
{#1}:
\left|
  \begin{array}{rcl}
    {#2} & \longrightarrow & {#3} \\
    {#4} & \longmapsto & {#5} \\
  \end{array}
\right.
}


%%%%%%application restriction
\newcommand{\restrict}[2]{{#1}_{\mkern 2mu \vrule height 2.5ex\mkern2mu {#2}}}



%%%%%%differentiation
\def\d{\,{\mathrm d}}


%%%%% operators
\def\tr{\operatorname{tr}}
\newcommand{\Range}[1]{\operatorname{Range}({#1})}
\newcommand{\Span}[1]{\operatorname{Span}\left\{{#1}\right\}}
\newcommand{\cSpan}[1]{\overline{\operatorname{Span}}\left\{{#1}\right\}}



\def\Sc{{S^c}}

\def\barm{{\bar{m}}}
\def\bark{{\bar{k}}}


\def\overcap{\overline{\sqcap}}
\def\undercap{\underline{\sqcap}}

% Custom Commands
\renewcommand{\d}{D}
\newcommand{\TT}{T}  % Avoid potential conflicts with \T

% --- Theorems ---
\theoremstyle{plain}
\newtheorem{theorem}{Theorem}[section]
\newtheorem{proposition}[theorem]{Proposition}
\newtheorem{lemma}[theorem]{Lemma}
\newtheorem{corollary}[theorem]{Corollary}

\theoremstyle{definition}
\newtheorem{definition}[theorem]{Definition}
\newtheorem{assumption}[theorem]{Assumption}

\theoremstyle{remark}
\newtheorem{remark}[theorem]{Remark}

% Colored comments
\newcommand\misha[1]{\textcolor{blue}{(MB: #1)}}
\newcommand\hz[1]{\textcolor{orange}{(HZ: #1)}}

% --- Customizing Section Titles ---
\titleformat{\section}{\large\bfseries}{\thesection.}{0.2em}{} 
\titleformat{\subsection}{\normalsize\bfseries}{\thesubsection.}{0.2em}{} 




% --- Title Formatting ---

% \makeatletter
% \renewcommand{\maketitle}{
%   \begin{flushleft} % Left-align title and author block
%     {\fontsize{16}{22} \bfseries \@title \par} % Title font size (18pt, line spacing 22pt)
%     \vskip 1em
%     {\fontsize{10}{14} \bfseries \@author \par} % Author font size (12pt, line spacing 14pt)
%     \vskip 0.5em
%     {\small \@date \par} % Date (if needed)
%   \end{flushleft}
%   \renewcommand{\thefootnote}{\fnsymbol{footnote}} % Symbol footnotes for equal contribution
%   \footnotetext[1]{Equal contribution.}
%   \renewcommand{\thefootnote}{\arabic{footnote}} % Reset footnote numbering
% }
% \makeatother

\makeatletter
\renewcommand{\maketitle}{
  \noindent % Ensure the content starts at the very left
  \hspace*{-3em} % Force a left shift
  \begin{minipage}{1.2\textwidth} % Extend width to allow content to fit
    \parbox{1.1\textwidth}{ % Adjust width (e.g., 1.1\textwidth)
      {\fontsize{15}{28} \bfseries \@title} % Title font size (14pt, line spacing 16pt)
    }
    \vskip 2em
    {  \fontsize{10}{16} \bfseries \@author \par} % Author font size (10pt, line spacing 14pt)
    \vskip 0.5em
    {\small \@date \par} % Date (if needed)
  \end{minipage}
  \renewcommand{\thefootnote}{\fnsymbol{footnote}} % Symbol footnotes for equal contribution
  \footnotetext[1]{Equal contribution.}
  \renewcommand{\thefootnote}{\arabic{footnote}} % Reset footnote numbering
}
\makeatother






\title{\textbf{Task Generalization With AutoRegressive Compositional Structure: \\ \strut 
\hspace{16mm}
Can Learning From $\d$ Tasks Generalize to $\d^{T}$ Tasks?}}

% \author{
%     \textbf{Amirhesam Abedsoltan}$^{1}$\footnotemark[1],  
%     \textbf{Huaqing Zhang}$^{3}$\footnotemark[1],  
%     \textbf{Kaiyue Wen}$^{4}$,  
%     \textbf{Hongzhou Lin}$^{5}$, 
%     \textbf{Jingzhao Zhang}$^{3}$,  
%     \textbf{Mikhail Belkin}$^{1,2}$
% }

% \date{}  % No date


\renewcommand{\thefootnote}{\fnsymbol{footnote}}


\author{
    \begin{minipage}{\textwidth}
        \centering
    Amirhesam Abedsoltan$^{1}$\footnotemark[1], 
    Huaqing Zhang$^{3}$\footnotemark[1], 
    Kaiyue Wen$^{4}$, 
    Hongzhou Lin$^{5}$,\\ 
    Jingzhao Zhang$^{3}$, 
    Mikhail Belkin$^{1,2}$
    \end{minipage}
}

\date{}  % Removes date

% --- Running Headers & Footers ---
\pagestyle{fancy}  
\fancyhf{}  
\fancyhead[C]{\small Task Generalization With AutoRegressive Compositional Structure}  
\fancyfoot[C]{\thepage}  
\renewcommand{\headrulewidth}{0.4pt}  
\renewcommand{\footrulewidth}{0pt}  

% Suppress header on first page
\fancypagestyle{plain}{
  \fancyhf{}  
  \fancyfoot[C]{\thepage}  
  \renewcommand{\headrulewidth}{0pt}  
}

\begin{document}

\newcommand{\amir}[1]{\textcolor{teal}{Amir:#1}}



\maketitle


% \footnotetext[1]{\textit{Equal contribution.}}
\footnotetext[1]{\textit{Department of Computer Science and Engineering, UC San Diego.}}
\footnotetext[2]{\textit{Halicioglu Data Science Institute, UC San Diego}}
\footnotetext[3]{\textit{Institute for Interdisciplinary Information Sciences, Tsinghua University.}}
\footnotetext[4]{\textit{Stanford University.}}
\footnotetext[5]{\textit{Amazon. This work is independent of and outside of the work at Amazon.}}



% \footnotetext[1]{Department of Computer Science and Engineering, UC San Diego}
% \footnotetext[2]{Halicioglu Data Science Institute, UC San Diego}
% \footnotetext[3]{Institute for Interdisciplinary Information Sciences, Tsinghua University}
% \footnotetext[4]{Stanford University}
% \footnotetext[5]{Amazon}


\thispagestyle{plain}  % Ensure the first page uses the plain style

% --- Abstract ---
\begin{abstract}
Large language models (LLMs) exhibit remarkable task generalization, solving tasks they were never explicitly trained on with only a few demonstrations. This raises a fundamental question: When can learning from a small set of tasks generalize to a large task family? In this paper, we investigate task generalization through the lens of AutoRegressive Compositional (ARC) structure, where each task is a composition of $T$ operations, and each operation is among a finite family of $\d$ subtasks. This yields a total class of size~\( \d^\TT \). We first show that generalization to all \( \d^\TT \) tasks is theoretically achievable by training on only \( \tilde{O}(\d) \) tasks. Empirically, we demonstrate that Transformers achieve such exponential task generalization on sparse parity functions via in-context learning (ICL) and Chain-of-Thought (CoT) reasoning. We further demonstrate this generalization in arithmetic and language translation, extending beyond parity functions.
\end{abstract}

% --- Main Sections ---
\section{Introduction}




Large language models (LLMs) demonstrate a remarkable ability to solve tasks they were never explicitly trained on. Unlike classical supervised learning, which typically assumes that the test data distribution follows the training data distribution, LLMs can generalize to new task distributions with just a few demonstrations—a phenomenon known as in-context learning (ICL) \citep{brown2020language, wei2022emergent, garg2022can}. Recent studies suggest that trained Transformers implement algorithmic learners capable of solving various statistical tasks—such as linear regression—at inference time in context \citep{li2023transformers, bai2023transformers}. Despite their success in tasks such as learning conjunctions or linear regression, Transformers relying solely on in-context learning (ICL) struggle with more complex problems, particularly those requiring hierarchical reasoning.






A notable case where Transformers struggle with in-context learning (ICL) is the learning of parity functions, as examined in \cite{bhattamishra2024understanding}. In this setting, a Transformer is provided with a sequence of demonstrations $
(\bm{x}_1, f(\bm{x}_1)), \dots, (\bm{x}_n, f(\bm{x}_n))
$
and is required to predict \( f(\bm{x}_{\mathrm{query}}) \) for a new input \( \bm{x}_{\mathrm{query}} \). Specifically, they focused on parity functions from the class \( \text{Parity}(10,2) \), where each function is defined by a secret key of length \( k=2 \) within a length space of \( d=10 \). Each function \( f \) corresponds to a distinct learning task, resulting in 45 possible tasks. To assess generalization, a subset of tasks was held out during training. Their results demonstrate that Transformers trained via ICL fail to generalize to unseen tasks, even when the new tasks require only a simple XOR operation. These findings, along with other empirical studies \cite{an2023context, xu2024do}, suggest that standard ICL struggles with tasks requiring hierarchical or compositional reasoning.




\begin{figure}
    \centering
    \includegraphics[width=0.5\textwidth]{Figures/in_out_distribution_accuracy.png}
    \caption{We train a Transformer to learn parity functions through In-Context Learning (ICL): given a demonstration sequence \((\bm x_1, f(\bm x_1)), \dots, (\bm x_n, f(\bm x_n))\), infer the target \( f(\bm x_{\mathrm{query}}) \) from a new input $\bm x_{\mathrm{query}}$. Intuitively, each function \( f \) defines a distinct learning task. In this prototype experiment, tasks are sampled from the parity function family \(Parity (10,2)\) with secret length $k=2$ and bit length $d=10$, which consists of 45 tasks in total. To evaluate task generalization, we withhold a subset of tasks and train only on different subset of the remaining ones. Consistent with prior work \cite{bhattamishra2024understanding}, we observe that standard ICL fails to generalize across tasks. In contrast, incorporating Chain-of-Thought (CoT) reasoning significantly improves performance on unseen tasks.}
    \label{fig:in_out_dist_prelim}
\end{figure}    
% \end{wrapfigure} 

In contrast, we found that incorporating Chain-of-Thought (CoT) reasoning—introducing intermediate reasoning steps to the model—allows Transformers to easily generalize to unseen tasks, as illustrated in Figure~\ref{fig:in_out_dist_prelim}. Consistent with \cite{bhattamishra2024understanding}, we observe that Transformers without CoT perform only slightly better than chance level, no matter how many training tasks are presented to the model. However, as the number of training tasks increases, Transformers with CoT achieve near-perfect generalization on the held-out set of unseen tasks. We see that the extra information provided by CoT enables the model to exploit the compositional structure of the parity problem.
















Motivated by this example, we aim to systematically analyze how models can leverage autoregressive compositional structures to extend their capabilities beyond the training tasks. Conventionally, learning involves approximating a target function \(f^*\) drawn from a function class \(\mathcal{F}\) using examples from a training distribution over the input space \(\mathcal{X}\); generalization is then measured by testing $f^*$ on new  examples. In contrast, our focus is on \textbf{task generalization}, where training is restricted to a subset of functions or ``tasks'' \( \mathcal{F}_{\mathrm{train}} \subset \mathcal{F} \), leaving the remaining functions, unseen during training. Our goal is to investigate whether a model trained on tasks from \( \mathcal{F}_{\mathrm{train}} \) (with inputs from \( \mathcal{X} \)) can generalize to \textit{all tasks}, including \textit{unseen} tasks. 


This notion of task generalization goes beyond the standard out-of-distribution (OOD) settings (see, e.g., \cite{zhou2022domain} for review) by shifting the focus from adapting to new input distributions to learning entirely new tasks. Specifically, we ask:

\medskip
\textit{How can we quantify the number of tasks a model must be trained on to generalize to the entire class $\mathcal{F}$?}
\medskip



To analyze task generalization, we consider a finite set of functions \(\mathcal{F}\), where each function maps an input \(\bm x \in \mathcal{X}\) to a tuple of random variables 
$\bm y = (y_1, \dots, y_T)$. We assume each function can be characterized by a parameter tuple
\[
\theta = (\theta_1, \theta_2, \dots, \theta_T).
\]
The outputs are generated autoregressively: first, \(y_1\) is produced from \(\bm x\); then \(y_2\) is generated from \(\bm x\) and \(y_1\); and then \(y_3\) is generated from \(\bm x\), \(y_1\) and \(y_2\); and this process continues until \(y_T\) is produced. 
 Specifically, the sequence is generated sequentially as
\[
y_t \sim P_{\theta_t}(y_t \mid \bm x, \bm y_{<t}), \quad \text{for } t = 1, \dots, T,
\]
where \(\bm y_{<t} = (y_1, \dots, y_{t-1})\) denotes the previously generated outputs, and $P_{\theta_t}$ is some conditional probability distribution that is parametrized by $\theta_{t}$ and is conditioned on $\bm y_{<t}$ and $\bm x$.This structure can also be interpreted as a sequence of compositions,






\vspace{-2pt}


\begin{align*}
    \bm x &\xrightarrow{P_{\theta_1}} y_1 \\
    \bm x, y_1 &\xrightarrow{P_{\theta_2}} y_2 \\
    &\dots \\
    \bm x, y_1, \dots, y_{T-1} &\xrightarrow{P_{\theta_{T-1}}} y_T\;.
\end{align*}



We will call this function class \textit{AutoRegressive Compositional structure}. 
Assuming that the cardinality of the set of possible values for each parameter $\theta_t$ is finite and is equal to \( \d \), we will use the notation $\mathcal{F} =ARC(T, \d)$. The cardinality of this class is $\d^T$.


For the sparse parity problem with \(k\) secret keys in this framework, the output sequence has length \(T = k\). Given an input \(\bm x \in \mathcal{X} = \{0,1\}^n\), let the secret keys correspond to indices \(i_1, i_2, \dots, i_k\) (in a predetermined order). The output sequence \(\bm y = (y_1, y_2, \dots, y_k)\) is defined as follows, 
$$y_1 = x_{i_1}, \; y_2 = x_{i_1} \oplus x_{i_2}, \; \dots, \; y_k = x_{i_1} \oplus x_{i_2} \oplus \dots \oplus x_{i_k}.$$


That is, each \(y_t\) recovers the XOR of the first \(t\) secret coordinates. In this example, the output distribution at each step is deterministic, assigning probability 1 to the correct XOR value and 0 to all other values.


.

\noindent We can now address the following fundamental question: 

\medskip
\textit{How many tasks in \(\mathcal{F}_{\mathrm{train}}\) must a model be trained on to generalize to all tasks in \(\mathcal{F}\), including those it has not seen? In particular, can a model trained on \( \tilde{O}(\d) \) tasks generalize across the entire set of \( \d^T \) tasks?} 
\medskip

\noindent Our main contributions are:
% \vspace{-3mm}
\begin{itemize}[leftmargin=0.4 cm]
    \item We define AutoRegressive Compositional structure and introduce a framework to quantitatively analyze task generalization when the function class follows an AutoRegressive Compositional structure. (Sections \ref{sec: ARC} and \ref{sec: task generalization})
    
    \item We establish that under this structure, task generalization to all \( \d^T \) tasks is theoretically achievable by training on  \( \tilde{O}(\d) \) tasks up to logarithmic terms (\cref{sec: Exp Task Generalization}).
    
    \item We demonstrate how the parity problem aligns with our framework and empirically show that Transformers trained on i.i.d. sampled tasks exhibit exponential task generalization via chain-of-thought (CoT) reasoning, consistent with theoretical scaling (\cref{sec: Experiments}).
    
    \item Finally, we show that the selection of training tasks significantly impacts generalization to unseen tasks. If tasks are chosen adversarially, training on even nearly all $\d^T$ of the tasks with CoT may fail to generalize to the remaining tasks (\cref{sec: Experiment beyond iid sampling}).\vspace{-3mm}
\end{itemize}







\section{Related works}
Implicit Neural Representations are designed to learn continuous representations of target functions by taking advantages of the approximation power of neural networks.
%
Their inherent continuous property can beneficial in many cases like video compression~\citep{chen2021nerv,strumpler2022implicit}, 3D modeling~\citep{park2019deepsdf,atzmon2020sal,9010266,gropp2020implicit,sitzmann2019scene} and volume rendering~\citep{pumarola2021d, barron2021mip,martin2021nerf,barron2023zip}.
%
However, simply employing MLPs may result in spectral bias, where oversmoothed outputs are generated due to the inherent tendency of MLPs to prioritize learning low-frequency components first. Consequently, many studies have focused on these drawbacks and explored various methods to address this issue.
%
The most straightforward way to address this issue is by projecting the coordinates into the higher dimension~\citep{tancik2020fourier, wang2021spline}.
%
However, these methods can lead to noisy outputs if there is a mismatch in the embeddings variance.
%
To address this, \citet{landgraf2022pins} propose dividing the Random Fourier Features into multiple levels of detail, allowing the MLPs to disregard unnecessary high-frequency components. Another type of approach to mitigating the spectral bias introduced by the ReLU activation function, as proposed by \citet{sitzmann2020implicit}, \citet{ramasinghe2022beyond}, \citet{saragadam2023wire}, and \citet{shenouda2024relus}, is to modify the activation function itself by using alternatives such as the Sine function, Wavelets, or a combination of ReLU with other functions. There are also efforts to modify network structures to mitigate spectral bias~\citep{mujkanovic2024neural}. 
%
\citet{lindell2022bacon} introduce a network design that treats MLPs as filters applied to the input of the next layer, known as Multiplicative Filter Networks (MFNs). 
%
Additionally, based on the discrete nature of signals like images and videos, grid-based approaches (e.g., Grid Tangent Kernel~\citep{zhao2024grounding}, DINER~\citep{xie2023diner}, and Fourier Filter Bank~\citep{wu2023neural}) have been proposed to address spectral bias, as the grid property allows for sharp changes in features, which facilitates learning fine details.
Even though, there are some prior works trying to solve the inherent problems of Fourier features embeddings ~\citep{landgraf2022pins, yuce2022structured, hertz2021sape, saratchandran2024sampling}, limited research has addressed both the underlying causes of high-frequency noise and provides a non-heuristic solution even if these embeddings are widely employed into many downstream tasks.
\section{Experiments}
\label{sec:Experiments} 

We conduct several experiments across different problem settings to assess the efficiency of our proposed method. Detailed descriptions of the experimental settings are provided in \cref{sec:apendix_experiments}.
%We conduct experiments on optimizing PINNs for convection, wave PDEs, and a reaction ODE. 
%These equations have been studied in previous works investigating difficulties in training PINNs; we use the formulations in \citet{krishnapriyan2021characterizing, wang2022when} for our experiments. 
%The coefficient settings we use for these equations are considered challenging in the literature \cite{krishnapriyan2021characterizing, wang2022when}.
%\cref{sec:problem_setup_additional} contains additional details.

%We compare the performance of Adam, \lbfgs{}, and \al{} on training PINNs for all three classes of PDEs. 
%For Adam, we tune the learning rate by a grid search on $\{10^{-5}, 10^{-4}, 10^{-3}, 10^{-2}, 10^{-1}\}$.
%For \lbfgs, we use the default learning rate $1.0$, memory size $100$, and strong Wolfe line search.
%For \al, we tune the learning rate for Adam as before, and also vary the switch from Adam to \lbfgs{} (after 1000, 11000, 31000 iterations).
%These correspond to \al{} (1k), \al{} (11k), and \al{} (31k) in our figures.
%All three methods are run for a total of 41000 iterations.

%We use multilayer perceptrons (MLPs) with tanh activations and three hidden layers. These MLPs have widths 50, 100, 200, or 400.
%We initialize these networks with the Xavier normal initialization \cite{glorot2010understanding} and all biases equal to zero.
%Each combination of PDE, optimizer, and MLP architecture is run with 5 random seeds.

%We use 10000 residual points randomly sampled from a $255 \times 100$ grid on the interior of the problem domain. 
%We use 257 equally spaced points for the initial conditions and 101 equally spaced points for each boundary condition.

%We assess the discrepancy between the PINN solution and the ground truth using $\ell_2$ relative error (L2RE), a standard metric in the PINN literature. Let $y = (y_i)_{i = 1}^n$ be the PINN prediction and $y' = (y'_i)_{i = 1}^n$ the ground truth. Define
%\begin{align*}
%    \mathrm{L2RE} = \sqrt{\frac{\sum_{i = 1}^n (y_i - y'_i)^2}{\sum_{i = 1}^n y'^2_i}} = \sqrt{\frac{\|y - y'\|_2^2}{\|y'\|_2^2}}.
%\end{align*}
%We compute the L2RE using all points in the $255 \times 100$ grid on the interior of the problem domain, along with the 257 and 101 points used for the initial and boundary conditions.

%We develop our experiments in PyTorch 2.0.0 \cite{paszke2019pytorch} with Python 3.10.12.
%Each experiment is run on a single NVIDIA Titan V GPU using CUDA 11.8.
%The code for our experiments is available at \href{https://github.com/pratikrathore8/opt_for_pinns}{https://github.com/pratikrathore8/opt\_for\_pinns}.


\subsection{2D Allen Cahn Equation}
\begin{figure*}[t]
    \centering
    \includegraphics[scale=0.38]{figs/Burgers_operator.pdf}
    \caption{1D Burgers' Equation (Operator Learning): Steady-state solutions for different initializations $u_0$ under varying viscosity $\varepsilon$: (a) $\varepsilon = 0.5$, (b) $\varepsilon = 0.1$, (c) $\varepsilon = 0.05$. The results demonstrate that all final test solutions converge to the correct steady-state solution. (d) Illustration of the evolution of a test initialization $u_0$ following homotopy dynamics. The number of residual points is $\nres = 128$.}
    \label{fig:Burgers_result}
\end{figure*}
First, we consider the following time-dependent problem:
\begin{align}
& u_t = \varepsilon^2 \Delta u - u(u^2 - 1), \quad (x, y) \in [-1, 1] \times [-1, 1] \nonumber \\
& u(x, y, 0) = - \sin(\pi x) \sin(\pi y) \label{eq.hom_2D_AC}\\
& u(-1, y, t) = u(1, y, t) = u(x, -1, t) = u(x, 1, t) = 0. \nonumber
\end{align}
We aim to find the steady-state solution for this equation with $\varepsilon = 0.05$ and define the homotopy as:
\begin{equation}
    H(u, s, \varepsilon) = (1 - s)\left(\varepsilon(s)^2 \Delta u - u(u^2 - 1)\right) + s(u - u_0),\nonumber
\end{equation}
where $s \in [0, 1]$. Specifically, when $s = 1$, the initial condition $u_0$ is automatically satisfied, and when $s = 0$, it recovers the steady-state problem. The function $\varepsilon(s)$ is given by
\begin{equation}
\varepsilon(s) = 
\left\{\begin{array}{l}
s, \quad s \in [0.05, 1], \\
0.05, \quad s \in [0, 0.05].
\end{array}\right.\label{eq:epsilon_t}
\end{equation}

Here, $\varepsilon(s)$ varies with $s$ during the first half of the evolution. Once $\varepsilon(s)$ reaches $0.05$, it remains fixed, and only $s$ continues to evolve toward $0$. As shown in \cref{fig:2D_Allen_Cahn_Equation}, the relative $L_2$ error by homotopy dynamics is $8.78 \times 10^{-3}$, compared with the result obtained by PINN, which has a $L_2$ error of $9.56 \times 10^{-1}$. This clearly demonstrates that the homotopy dynamics-based approach significantly improves accuracy.

\subsection{High Frequency Function Approximation }
We aim to approximate the following function:
$u=    \sin(50\pi x), \quad x \in [0,1].$
The homotopy is defined as $H(u,\varepsilon) = u - \sin(\frac{1}{\varepsilon}\pi x), $
where $\varepsilon \in [\frac{1}{50},\frac{1}{15}]$.

\begin{table}[htbp!]
    \caption{Comparison of the lowest loss achieved by the classical training and homotopy dynamics for different values of $\varepsilon$ in approximating $\sin\left(\frac{1}{\varepsilon} \pi x\right)$
    }
    \vskip 0.15in
    \centering
    \tiny
    \begin{tabular}{|c|c|c|c|c|} 
    \hline 
    $ $ & $\varepsilon = 1/15$ & $\varepsilon = 1/35$ & $\varepsilon = 1/50$ \\ \hline 
    Classical Loss                & 4.91e-6     & 7.21e-2     & 3.29e-1       \\ \hline 
    Homotopy Loss $L_H$                      & 1.73e-6     & 1.91e-6     & \textbf{2.82e-5}       \\ \hline
    \end{tabular}
    % On convection, \al{} provides 14.2$\times$ and 1.97$\times$ improvement over Adam or \lbfgs{} on L2RE. 
    % On reaction, \al{} provides 1.10$\times$ and 1.99$\times$ improvement over Adam or \lbfgs{} on L2RE.
    % On wave, \al{} provides 6.32$\times$ and 6.07$\times$ improvement over Adam or \lbfgs{} on L2RE.}
    \label{tab:loss_approximate}
\end{table}

As shown in \cref{fig:high_frequency_result}, due to the F-principle \cite{xu2024overview}, training is particularly challenging when approximating high-frequency functions like $\sin(50\pi x)$. The loss decreases slowly, resulting in poor approximation performance. However, training based on homotopy dynamics significantly reduces the loss, leading to a better approximation of high-frequency functions. This demonstrates that homotopy dynamics-based training can effectively facilitate convergence when approximating high-frequency data. Additionally, we compare the loss for approximating functions with different frequencies $1/\varepsilon$ using both methods. The results, presented in \cref{tab:loss_approximate}, show that the homotopy dynamics training method consistently performs well for high-frequency functions.





\subsection{Burgers Equation}
In this example, we adopt the operator learning framework to solve for the steady-state solution of the Burgers equation, given by:
\begin{align}
& u_t+\left(\frac{u^2}{2}\right)_x - \varepsilon u_{xx}=\pi \sin (\pi x) \cos (\pi x), \quad x \in[0, 1]\nonumber\\
& u(x, 0)=u_0(x),\label{eq:1D_Burgers} \\
& u(0, t)=u(1, t)=0, \nonumber 
\end{align}
with Dirichlet boundary conditions, where $u_0 \in L_{0}^2((0, 1); \mathbb{R})$ is the initial condition and $\varepsilon \in \mathbb{R}$ is the viscosity coefficient. We aim to learn the operator mapping the initial condition to the steady-state solution, $G^{\dagger}: L_{0}^2((0, 1); \mathbb{R}) \rightarrow H_{0}^r((0, 1); \mathbb{R})$, defined by $u_0 \mapsto u_{\infty}$ for any $r > 0$. As shown in Theorem 2.2 of \cite{KREISS1986161} and Theorems 2.5 and 2.7 of \cite{hao2019convergence}, for any $\varepsilon > 0$, the steady-state solution is independent of the initial condition, with a single shock occurring at $x_s = 0.5$. Here, we use DeepONet~\cite{lu2021deeponet} as the network architecture. 
The homotopy definition, similar to ~\cref{eq.hom_2D_AC}, can be found in \cref{Ap:operator}. The results can be found in \cref{fig:Burgers_result} and \cref{tab:loss_burgers}. Experimental results show that the homotopy dynamics strategy performs well in the operator learning setting as well.


\begin{table}[htbp!]
    \caption{Comparison of loss between classical training and homotopy dynamics for different values of $\varepsilon$ in the Burgers equation, along with the MSE distance to the ground truth shock location, $x_s$.}
    \vskip 0.15in
    \centering
    \tiny
    \begin{tabular}{|c|c|c|c|c|} 
    \hline  
    $ $ & $\varepsilon = 0.5$ & $\varepsilon = 0.1$ & $\varepsilon = 0.05$ \\ \hline 
    Homotopy Loss $L_H$                &  7.55e-7     & 3.40e-7     & 7.77e-7       \\ \hline 
    L2RE                      & 1.50e-3     & 7.00e-4     & 2.52e-2       \\ \hline
        MSE Distance $x_s$                      & 1.75e-8     & 9.14e-8      & 1.2e-3      \\ \hline
    \end{tabular}
    % On convection, \al{} provides 14.2$\times$ and 1.97$\times$ improvement over Adam or \lbfgs{} on L2RE. 
    % On reaction, \al{} provides 1.10$\times$ and 1.99$\times$ improvement over Adam or \lbfgs{} on L2RE.
    % On wave, \al{} provides 6.32$\times$ and 6.07$\times$ improvement over Adam or \lbfgs{} on L2RE.}
    \label{tab:loss_burgers}
\end{table}



% \begin{itemize}
%     \item Relate the curvature in the problem to the differential operator. Use this to demonstrate why the problem is ill-conditioned
%     \item Give an argument for why using Adam + L-BFGS is better than just using L-BFGS outright. The idea is that Adam lowers the errors to the point where the rest of the optimization becomes convex \ldots
%     \item Show why we need second-order methods. We would like to prove that once we are close to the optimum, second-order methods will give condition-number free linear convergence. Specialize this to the Gauss-Newton setting, with the randomized low-rank approximation.
%     % \item Show that it is not possible to get superlinear convergence under the interpolation assumption with an overparameterized neural network. This should be true b/c the Hessian at the optimum will have rank $\min(n, d)$, and when $d > n$, this guarantees that we cannot have strong convexity.
% \end{itemize}
In this section, we empirically compare the proposed algorithm on both sequence windows and time windows with existing methods.
\paragraph{Datasets} For the sequence-based model, we used two synthetic datasets and two cross-language datasets. The statistics of the datasets are provided in Table \ref{table:statistics}:

\begin{table}[t]
    \centering
    \caption{The statistics of the datasets. The datasets satisfy $1 \leq \|\vx\|\|\vy\| \leq R $.}
    \label{table:statistics}
    \begin{tabular}{|c|c|c|c|c|c|}
    \hline
        Dataset & $n$ & $m_x$ & $m_y$ & $N$ & $R$ \\ \hline
        SYNTHETIC(1) & 100,000 & 1,000 & 2,000 & 50,000 & 65 \\ \hline
        SYNTHETIC(2) & 100,000 & 1,000 & 2,000 & 50,000 & 724 \\ \hline
        APR & 23,235 & 28,017 & 42,833 & 10,000 & 773 \\ \hline
        PAN11 & 88,977 & 5,121 & 9,959 & 10,000 & 5,548 \\ \hline
        EURO & 475,834 & 7,247 & 8,768 & 100,000 & 107,840 \\ \hline
    \end{tabular}
\end{table}

\begin{itemize}
    \item Synthetic: The elements of the two synthetic datasets are initially uniformly sampled from the range (0,1), then multiplied by a coefficient to adjust the maximum column squared norm $R$. The X matrix has 1,000 rows, and the Y matrix has 2,000 rows, each with 100,000 columns. The window size is set to 50,000.
    \item APR: The Amazon Product Reviews (APR) dataset is a publicly available collection containing product reviews and related information from the Amazon website. This dataset consists of millions of sentences in both English and French. We structured it into a review matrix where the X matrix has 28,017 rows, and the Y matrix has 42,833 rows, with both matrices sharing 23,235 columns. The window size is 10,000.
    \item PAN11: PANPC-11 (PAN11) is a dataset designed for text analysis, particularly for tasks such as plagiarism detection, author identification, and near-duplicate detection. The dataset includes texts in English and French. The X and Y matrices contain 5,121 and 9,959 rows, respectively, with both matrices having 88,977 columns. The window size is 10,000.
\end{itemize}
We evaluate the time-based model on another real-world dataset:
\begin{itemize}
    \item EURO: The Europarl (EURO) dataset is a widely used multilingual parallel corpus, comprising the proceedings of the European Parliament. We selected a subset of its English and French text portions. The X and Y matrices contain 7,247 and 8,768 rows, respectively, and both matrices share 475,834 columns. Timestamps are generated using the $Poisson$ $Arrival$ $Process$ with a rate parameter of $\lambda=2$. The window size is set to 100,000, with approximately 30,000 columns of data on average in each window.
\end{itemize}

\paragraph{Setup} For the sequence-based model, we compare the proposed hDS-COD and  aDS-COD with EH-COD~\cite{yao2024approximate} and DI-COD~\cite{yao2024approximate}. We do not consider the Sampling algorithm as a baseline, as its performance is inferior to that of EH-COD and DI-CID, as demonstrated in \cite{yao2024approximate}. %The hDS-COD is adjusted by the parameter $\ell$ and the maximum number of levels $L = \log{R}$, where $R$ is the prior estimate of the maximum squared column norm of the dataset. DI-COD similarly requires a prior estimate of $R$ to limit the maximum number of levels $L = \log{(R/\varepsilon})$. In contrast, aDS-COD and EH-COD do not require an estimate of $R$; their error-space balance is controlled by the parameter $\ell = \frac{1}{\varepsilon}$. 
For the time-based model, we compare the proposed hDS-COD and  aDS-COD with EH-COD and the Sampling algorithm since DI-COD cannot be applied to time-based sliding window model. To achieve the same error bound, the maximum number of levels for hDS-COD is set to $L = \log{(\varepsilon NR)}$, and the initial threshold for aDS-COD is set to $1$.

Our experiments aim to illustrate the trade-offs between space and approximation errors. The x-axis represents two metrics for space: final sketch size and total space cost. The final sketch size refers to the number of columns in the result sketches $\mA$ and $\mB$ generated by the algorithm, representing a compression ratio. The total space cost refers to the maximum space required during the algorithm's execution, measured by the number of columns.We evaluate the approximation performance of all algorithms based on correlation errors $\operatorname{corr-err}(\mathbf{X}_W \mathbf{Y}_W^\top, \mathbf{A} \mathbf{B}^\top)$, which is reflected on the y-axis. Every 1,000 iterations, all algorithms query the window and record the average and maximum errors across all sampled windows.

The experiments for all algorithms were conducted using MATLAB (R2023a), with all algorithms running on a Windows server equipped with 32GB of memory and a single processor of Intel i9-13900K.

\paragraph{Performance} Figure \ref{fig:error vs l} and Figure \ref{fig:error vs space} illustrate the space efficiency comparison of the algorithms on sequence-based datasets. Panels (a-d) show the average errors across all sampled windows, while panels (e-h) display the maximum errors.

Figure \ref{fig:error vs l} evaluates the compression effect of the final sketch. The hDS-COD, aDS-COD, and EH-COD show similar compression performances. But the DS series is more stable, particularly on the synthetic datasets, where they significantly outperform EH-COD and DI-COD. The performance of hDS-COD and aDS-COD is nearly the same, indicating that the adaptive threshold trick in aDS-COD does not have a noticeable negative impact on it, maintaining the same error as hDS-COD.

Figure \ref{fig:error vs space} measures the total space cost of the algorithms. hDS-COD and aDS-COD show a significant advantage over existing methods, as they can achieve the  $\varepsilon$-approximation error with much less space. For the same space cost, the correlation errors of hDS-COD and aDS-COD are much smaller than those of EH-COD and DI-COD. Also, aDS-COD has better space efficiency than hDS-COD because aDS only uses a single-level structure while hDS requires $\log R+1$ levels. We find that hDS-COD requires more space on  SYNTHETIC(2) dataset compared to SYNTHETIC(1) dataset. This phenomenon occurs because SYNTHETIC(2) dataset has a larger $R$, which confirms the dependence on $R$ as stated in Theorem~\ref{thm:hds}. 

Figure \ref{fig:time-based} compares the performance of algorithms on time-based windows. Panels (a) and (b) present the error against the final sketch size, which show that our aDS-COD and hDS-COD algorithms enjoy similar performance as EH-COD and significantly outperform the sampling algorithm. On the other hand, as shown in panels (c) and (d), our methods outperform baselines in terms of total space cost.

\section{Discussion and Future Work}\label{sec:discussion}
This paper pioneers the novel approach of selective response, showing that withholding responses can be a powerful tool for GenAI systems. By opting not to answer every query as accurately as it can---particularly when new or complex topics emerge---GenAI can encourage user participation on community-driven platforms and thereby generate more high-quality data for future training. This mechanism ultimately enhances GenAI's long-term performance and revenue. From a welfare perspective, our results indicate that such selective engagement can also benefit users, leading to better solutions and increased overall satisfaction. Since this work is the first to address selective response strategies for GenAI, numerous promising directions remain for future research; we highlight some of them below. 

First, from a technical standpoint, all of the results in this paper rely on Assumption~\ref{assumption: data lip}, involving the lipshitz condition of the accuracy function and the sensitivity parameter $\beta$. Future work could seek to relax this assumption. Furthermore, our constrained optimization approach in Subsection~\ref{sec: welfare constrained revenue maximization} could be extended to approximate the optimal (continuous) strategy instead of the optimal discrete strategy.

Second, our stylized model adopts the simplifying---though unrealistic---assumption that only a single GenAI platform exists. Admittedly, this makes it easier to focus on the idea of selective responses, and indeed, this assumption is pivotal in keeping our analysis tractable. Future research could explore scenarios with multiple GenAI platforms and human-centered forums. In such settings, one platform's selective response might redirect users not only to forums but also to competing GenAI platforms, leading to the tragedy of the commons \cite{hardin1968tragedy}: Although all GenAI platforms benefit from fresh data generation, none may choose to respond selectively if it means losing users to competitors. 

Third, we assumed Forum behaves non-strategically. In reality, human-centered platforms often monetize their data by selling it to GenAI platforms, adding a further layer of strategic interaction for GenAI. Moreover, data transfer between the platforms can form the basis for collaboration: GenAI could employ selective response to bolster Forum content creation, and Forum could, in turn, attribute that content to GenAI for subsequent use in retraining.


%Third, we make the (again) simplifying assumption that Forum is non-strategic. However, in practice, human-centered platforms can sell their data to GenAI platforms. This adds additional considerations for GenAI. Furthermore, data transmission between the platforms can also become the basis for collaboration: GenAI can use selective response to ensure enough content is generated in Forum, and Forum could provide the data attributed to this mechanism back to GenAI. 


%Second, this paper makes the simplifying yet unrealistic assumption of the existence of one GenAI platform. Indeed, this simplifies many aspects and allows us to analyze selective responses. Future work could address the data generation process with more than one GenAI platform and possibly several human-centered forums. In such a case, selective response of one GenAI platform can either drive users to forums or to other GenAI platforms; thus, we might face a tragedy of the commons situation~\ref{hardin1968tragedy}, where all GenAI platforms are interested in fresh data generation but none volunteer to selectively respond and lose users. 

%This paper examines the competition between a generative AI platform and human-based platforms, challenging the assumption that always providing answers is optimal. We analyzed the impact of withholding answers on GenAI's revenue and developed an efficient approximately optimal algorithm for this purpose. We further explored how withholding affects users, showing that it can lead to better outcomes compared to always answering. Specifically, we demonstrated that withholding can Pareto-dominate this strategy and derived the necessary and sufficient conditions for that. Finally, we proposed a second approximately optimal algorithm that maximizes GenAI's revenue while ensuring users are better off than when GenAI answers all queries.

%On a more conceptual level, our model assumes that GenAI’s data comes solely from the competing platform (Forum). Future research could explore a scenario where GenAI can purchase additional data from a third party. This extension could provide valuable insights into the interplay between withholding answers and data purchasing, and whether these two strategies can complement each other or must be traded off.
\newpage
% --- Bibliography ---
\bibliography{ref}
\bibliographystyle{icml2025}

% --- Appendix ---
\newpage
\appendix
\onecolumn

\subsection{Proof for Satisfaction of Marginal Constraints.}
% In this section, we will first show that our procedure satisfying the marginal conditions for our coupling $q(\rvx_0, \rvx_1)$:
% \begin{equation}
%     \int q(\rvx_0, \rvx_1) d\rvx_1 = q_0(\rvx_0), \int q(\rvx_0, \rvx_1) d\rvx_0 = q_1(\rvx_1).
% \end{equation}

% \begin{itemize}
%     \item For independent couple $q(x_0) = \int q(\mathcal{S}) \int q(x_1 | \mathcal{S}) q(x_0) dx_0 d_\mathcal{S}$ and $q(x_1) = \int q(\mathcal{S}) \int q(x_1 | \mathcal{S}) q(x_0) dx_1 d_\mathcal{S}$, we just need to show $\int q(x_0, x_1 | \mathcal{S}) dx_0 = q(x_1 | \mathcal{S})$ and $\int q(x_0, x_1 | \mathcal{S}) dx_1 = q(x_0)$.
%     \item $q(x_0, x_1 | \mathcal{S})$ is independent, so we can decompose it into $\prod q(x_0^i, x_1^i | \mathcal{S})$.
%     \item we can show $\int q(x_0^i, x_1^i | \mathcal{S}) dx_0 = q(x_1^i | \mathcal{S})$ and $\int q(x_0^i, x_1^i | \mathcal{S}) dx_0 = q(x_1^i)$
%     \item $q(x_0| \mathcal{S})$ and $q(x_1| \mathcal{S})$ are independent, so we can decompose it into $\prod q(x_1^i | \mathcal{S})$ and $\prod q(x_0^i)$.
%     \item The first part is done.
%     \item The second part is to show adding noise will not affect $q(x_0^i)$

% \end{itemize}

% In particular, the proof will be divided into four parts.
% %
% First, we will introduce the main theorem to apply to obtain the results, and show the random subsampling of a Dense Gaussian noise will converge to Gaussian distribution if the sample superset is large enough.
% %
% Second, by a proper construction, we can show that subsampling of a dense point superset can converge to direct subsampling of the surfaces when the size of the superset is also large enough.
% %
% Third, by considering our random subsampling procedure, we can show that our sampling is still random subsampling for Gaussian noise superset and point superset.
% %
% Lastly, we show that even introduction of the coupling interpolation, the results mariginal remain the same due to careful considerations.

\newtheorem{proposition}{Proposition}
\newtheorem{lemma}{Lemma}
\subsubsection{Law of Large Numbers}


\begin{proposition}\label{prop:large_samples}
Given $(X_1, \cdots, X_n)$, which are independently and identically distributed (IID) real $d$-diemsnion random variables, following a probability distribution $p(X)$,~\ie, $X_i \sim p(X), X \in \mathbb{R}^d$.
%
We have an additional random variable $Y$ that is random uniform sample of these variables,~\ie, $P(Y = X_i) = \frac{1}{n}$.
%
The cumulative distribution function (CDF) $\bar{F}(t)$ of random variable $Y$ will converge to the $F(X)$,~\ie, CDF of $X$.
\end{proposition}



% Assume $(X_1, \cdots, X_n)$ are independently and identically distributed (IID) real $d$-diemsnion random variables following a probability distribution $p(X)$, \ie, $X_i  \sim p(X), X \in \mathbb{R}^d$.
% %
% We also denote the cumulative distribution function of $p(X)$ to be $F(x)$.
%
Proof:
We first define an empirical cumulative distribution function $\hat{F}_n(X)$ over the random variables $(X_1, \cdots, X_n)$:
\begin{equation}
    \hat{F}_n (t) = \frac{1}{n} \sum_{i=1}^{n} \mathbf{1}_{X_i \leq t},
\end{equation}
where $\mathbf{1}_{X_i \leq t}$ is an indicator for $X_i^d \leq t^d$ for all dimensions $\{1, \cdots, d\}$.

The Glivenko–Cantelli theorem states that this empirical distribution function $\hat{F}_n(X)$ will converge to the cumulative distribution $F(X)$ if $n$ is sufficiently large:
\begin{equation}
    \textbf{sup}_{t \in \mathbb{R}^d} | \hat{F}_n(t) - F(t) | \rightarrow 0.
\end{equation}

If we have an additional random variable $Y$ that its value is a random subsample of the variables $(X_1, \cdots, X_n)$:
\begin{equation}
    P(Y = X_i) = \frac{1}{n}, \forall i = 1, 2, \cdots, n.
\end{equation}

The CDF of this variable $\bar{F}(t)$ is:
\begin{equation}
    \bar{F}(t) = P(Y \leq t) = \sum_{i=1}^{n} P(Y = X_i) \cdot \mathbf{1}_{X_i \leq t} = \frac{1}{n} \sum_{i=1}^{n} \mathbf{1}_{X_i \leq t} = \hat{F}_n(t).
\end{equation}
Therefore, the CDF of $Y$ also converges to the original underlying CDF $F(t)$ if $n$ is sufficiently large.

\begin{proposition}\label{prop:ot}
Assume we have $n$ random samples $(X_1, \cdots, X_n) \sim p_1$, and another $n$ random samples $(Y_1, \cdots, Y_n) \sim p_2$, and we are also given an arbitrary bijective map between random variables, \ie, $\Pi: \{1, \cdots, n\} \leftrightarrow \{1, \cdots, n\}$.
%
If we construct a new random variable $Z : (X, Y)$ follows the following couplings:

\[
    P(X = X_i, Y = Y_j) =
    \begin{cases}
    \frac{1}{n}, & \text{if } j = \Pi(i) \\ 
        0, & \text{else } j \neq \Pi(i);
    \end{cases}
\]

The CDF of the marginal $P(X)$ will converge the CDF of $p_1$, while the CDF of the marginal $P(Y)$ will converge to the CDF of $p_2$.
\end{proposition}

Proof:
Since $\Pi$ is bijective, we can compute the marginal $P(X = X_i)$ directly:
\begin{equation}
    \begin{split}
            P(X = X_i) = \sum_{j=1}^{n} P(X = X_i, Y = Y_j) \\
            = P(X = X_i, Y = Y_{\Pi(i)}) + \sum_{j \neq \Pi(i)} P(X = X_i, Y = Y_j) \\
            = \frac{1}{n} + 0 = \frac{1}{n}
    \end{split}
\end{equation}

Similarly, we can show the marginal of P(Y) is also $\frac{1}{n}$.
%
By leveraging Proposition~\ref{prop:large_samples}, we show that $P(X)$ will converge the CDF of $p_1$, and the CDF of $P(Y)$ will converge to the CDF of $p_2$.

% \begin{lemma}\label{lemma:independent}
% The Gaussian noises $x_0$ are independently and identically distributed (IID), \ie, $q_0(x_0) = \prod_{i}^N q_0(x_0^i)$, where $x^i_0$ is the $i$-th noises and $x^i_0 \sim q_0$ .
% %
% Also, the point cloud $x_1$ given a 3D shape $S$ is also independently and identically distributed (IID), \ie, $q_1(x_1|S) = \prod_{i}^N q_1(x_1^i | S)$, where $x^i_1$ is the $i$-th point and $x^i_0 \sim q_{1|S}$.
% %
% Lastly, the training pair $(x_0, x_1)$ from our coupling  given a shape $S$ is also independently and identically distributed (IID), \ie, $q(x_0, x_1 | S) = \prod_{i}^N q(x_0^i, x_1^i | S)$, where $(x_0^i, x_1^i$) is the $i$-th pair in the training pair.
% \end{lemma}

% \begin{lemma}\label{lemma:joint}
%     The sample distribution of a point $x_1^i$ involves modeling of underlying shape $S$ and the modeling of the point distribution given $S$, \ie, $q_1(x_1^i) = \int q_1(x_1^i | S) q(S) dS$.
%     %
%     However, the distribution of noises $q_0(x^i_0)$ is unrelated to a given shape $S$, \ie, $q_0(x^i_0 | S) = q_0(x^i_0)$.
% \end{lemma}

% By considering the $p(X)$ be a Gaussian distribution $N(0, I)$ or a sampling distribution of 3D points given a Shape $\mathcal{S}$, \ie, $q(x|\mathcal{S})$, we can show the random sample $Y$ still follows the original distribution.

% If we consider $M$ random variables, where each of them is an 3D Gaussian noise, denoted as $\epsilon_i \sim N(0, I), \epsilon_i \in \mathbb{R}^3$.
% %
% We also define another variable $\epsilon$ is a random sample of these random variables, \ie, $P(\epsilon = \epsilon_i) = \frac{1}{M}$.
% %
% Since each dimension in $\epsilon$ is independent, CDF of $\epsilon^j$ will follows the by leverage the above results, where $j$ is the j-th dimension of the noise.
% We consider a dense 3D Gaussian noises with $M \times 3$ random variables, $\{x_1, y_1, z_1, \cdots, x_M, y_M, z_M\}$, where we denote $x_i$, $y_i$, and $z_i$ to be the coordinates of in x, y, and z dimensions, respectively and $x_i, y_i, z_i \sim N(0, I)$.
% %
% If we can consider a random variable $\hat{x}$, which is random sample of this dense gausian in x dimension, \ie,  P$(\hat{x} = x_i) = \frac{1}{M}$.
% %
% By the above results, the CDF follows the original distribution, which is the Gaussian distribution $N(0, I)$.
% %
% By considering also y and z dimension, we can show that a random sampling of Gaussian point converge to Gaussian distribution.
\newtheorem{theorem}{Theorem}
\subsubsection{Proof of Our OT Approximation}
\label{subsec:our_ot_proof}

We first give a definition of coupling $q(x_0, x_1)$ in our case before showing its marginal fullfils the marginal requirements.
%
In particular, we denote $x_0 \in R^{N \times 3}$ and $x_1 \in R^{N \times 3}$ as two random variables following the distributions, $q_0(x_0)$ and $q_1(x_1)$, respectively.
%
It is noted that $q_0 := N(0, I)$, which is the standard Gaussian for each dimension in $x_0$, and $q_1(x_1)$ is the distribution all possible point clouds, which involves the joint modeling of point cloud distribution given a shape $S$ ($q_{1}(x_1|S)$) and the distribution of shape ($q(S)$), \ie, $q_1(x_1) = \int q_{1}(x_1|S) q(S) dS$.
%

We can notice that each row in $x_0$ is independently and identically distributed (IID), \ie, $q_0(x_0) = \prod_{i}^N \hat{q_0}(x_0^i)$, where we denote the $i$-th row of $x_0$ as $x_0^i$ and distribution of $x_0^i$ as $\hat{q_0}(x_0^i)$, which is 3-dimensional unit Gaussian.
%
We can also assume each point in $x_1$ is IID given a shape, \ie, $q_{1}(x_1 | S) = \prod_{i}^N \hat{q_{1}}(x_1^i|S)$,  where we denote the $i$-th row of $x_1$ as $x^i_1$ and the distribution of $x^i_1$ as $\hat{q_{1}}(x_1^i|S)$. 

In our superset OT precomputation for a given shape $S$, we pre-sample a set of random variables $(x^1_0 \cdots, x^j_0, \cdots, x^M_0) \sim \hat{q}_0$, and a set of random variables  $(x^1_1, \cdots, x^k_1,\cdots, x^M_1) \sim \hat{q}_1$, and have a precomputed bijective mapping $\Pi : \{1, \cdots, M\} \leftrightarrow \{1, \cdots, M\}$.
%
With these defined, our coupling $\hat{q}(x^i_0, x^i_1 |S)$ for one row in the training pair $(x^i_0, x^i_1)$ given $S$ can be formulated as:
\[
    \hat{q}(x^i_0 = x^j_0, x^i_1 = x^k_1 | S) =
    \begin{cases}
    \frac{1}{n}, & \text{if } j = \Pi(k) \\ 
        0, & \text{else } j \neq \Pi(k);
    \end{cases}
\]
%
Since the each row in the training pairs are independently subsampled, the coupling of the training pair $(x_0, x_1)$ given a shape is defined as $q(x_0, x_1 |S) = \prod_{i}^N \hat{q}(x_0^i, x_1^i | S)$.
%
In the end, the coupling over all training pairs can be obtained by marginalize over all possible shapes, \ie, $\int q(x_0, x_1 | S) q(S) dS$.

\begin{theorem}

% Our coupling $q(x_0, x_1)$ for a given Gaussian noise $x_0 \in R^{N \times 3}$ and a given point cloud $x_1 \in R^{N \times 3}$

Our coupling without blending converge the following marginal if the superset size $M$ is sufficiently large:
\begin{equation}\label{eq:mariginals}
    \int q(\rvx_0, \rvx_1) d\rvx_1 = q_0(\rvx_0), \int q(\rvx_0, \rvx_1) d\rvx_0 = q_1(\rvx_1).
\end{equation}
\end{theorem}

Proof:
We first show the left constraint:
% \begin{equation}
\begin{align}
LHS & = \int q(x_0, x_1) dx_1 = \int \int q(x_0, x_1 | S) q(S) dS dx_1  \\
& = \int q(S) \int q(x_0, x_1 | S) dx_1 dS && \text{change the order of integration} \\
& = \int q(S) \int \prod_i^N \hat{q}(x_0^i, x_1^i|S) d(x_1^1, \cdots, x_1^N) dS  && \text{independent assumption of each row in training pair}\\
& = \int q(S) \prod_i^N \int \hat{q}(x_0^i, x_1^i|S) dx_1^j dS && \text{integrals of independent products}\\
& = \int q(S) \prod_i \sum_k^M \hat{q}(x_0^i, x_1^k|S) dS && \text{restricting to discrete values in supersets}\\
& = \int q(S) \prod_i \hat{q}_0(x^i_0) dS && \text{Proposition~\ref{prop:ot}}\\
& = \int q(S) q_0(x_0) dS = q_0(x_0) && \text{independent assumption of each row in Gaussian noises} \\
\end{align}
% \end{equation}

Similarly, we perform the same computation on the right constraint:
% \begin{equation}
\begin{align}
LHS & = \int q(x_0, x_1) dx_0 = \int \int q(x_0, x_1 | S) q(S) dS dx_0   \\
 & = \int q(S) \int q(x_0, x_1 | S) dx_0 dS && \text{change the order of integration} \\
& = \int q(S) \int \prod_i^N \hat{q}(x_0^i, x_1^i|S) d(x_0^1, \cdots, x_0^N) dS && \text{independent assumption of each row in training pair} \\
& = \int q(S) \prod_i^N \int \hat{q}(x_0^i, x_1^i|S) dx_0^i dS  && \text{integrals of independent products} \\
& = \int q(S) \prod_i \sum_j^M \hat{q}(x_0^j, x_1^i|S) dS 
 && \text{restricting to discrete values in supersets} \\
& = \int q(S) \prod_i \hat{q}_1(x^i_1 | S) dS  && \text{Proposition~\ref{prop:ot}} \\
& = \int q(S) q_1(x_1 | S) dS = q_1(x_1) && \text{independent assumption of each row in point cloud} \\
\end{align}
% \end{equation}


% We first consider the RHS of Left Constraints (Equation~\ref{eq:mariginals}), we can reformulate it as follows:
% \begin{equation}
%     \begin{split}
%             RHS = q_0(x_0) = \int q(S) q_0(x_0 | S) dS = \int q(S) q_0(x_0) dS \\
%             % = \int q_0(x_0) (\int q_1(x_1 |S) q(S) dS) dx_1 \text{, by Lemma~\ref{lemma:joint}} \\
%             % = \int q(S) \int q_0(x_0) q_1(x_1|S) dx_1 d_S \text{, by rearranging the integrals} \\
%     \end{split}
% \end{equation}
% Considering LHS:
% \begin{equation}
%     \begin{split}
%         LHS = \int q(x_0, x_1) dx_1 = \int \int q(x_0, x_1 | S) q(S) dS dx_1 \\
%         = \int q(S) \int q(x_0, x_1 | S) dx_1 dS
%     \end{split}
% \end{equation}

% By comparing LHS and RHS, it is sufficient to show that $\int q(x_0, x_1 |S) dx_1 = q_0(x_0)$ for the first constraint.
% Similarly, for the second constraint RHS:
% \begin{equation}
%     \begin{split}
%             RHS = q_1(x_1) = \int q(S) q_1(x_1|S) dS \\
%             % = \int q_0(x_0) (\int q_1(x_1 |S) q(S) dS) dx_1 \text{, by Lemma~\ref{lemma:joint}} \\
%             % = \int q(S) \int q_0(x_0) q_1(x_1|S) dx_1 d_S \text{, by rearranging the integrals} \\
%     \end{split}
% \end{equation}
% Considering LHS:
% \begin{equation}
%     \begin{split}
%         LHS = \int q(x_0, x_1) dx_0 = \int \int q(x_0, x_1 | S) q(S) dS dx_0 \\
%         = \int q(S) \int q(x_0, x_1 | S) dx_0 dS
%     \end{split}
% \end{equation}
% Then it is sufficient to show $\int q(x_0, x_1 | S) dx_0 = q_1(x_1|S) $.

% To show first equation, we can apply Lemma~\ref{lemma:independent}:
% \begin{equation}
% \label{eq:left_LHS}
%     \begin{split}
%         LHS = \int q(x_0, x_1 | S) dx_1 = \int \prod_i q(x_0^i, x_1^i | S) d(x_1^i, \cdots, x_1^N) \\
%         = \prod_i \int q(x_0^i, x_1^i|S) dx_1^i 
%     \end{split}
% \end{equation}

% \begin{equation}
% \label{eq:left_RHS}
%     RHS = q_0(x_0) = \prod_i q_0(x^i_0)
% \end{equation}
% By this computation, we are also sufficient to show $\int q(x_0^i, x_1^i | S) dx_1^i = q_0(x_0^i)$ and by similar computation:
% \begin{equation}
% \label{eq:right_LHS}
%     \begin{split}
%         LHS = \int q(x_0, x_1 | S) dx_0 = \int \prod_i q(x_0^i, x_1^i | S) d(x_0^i, \cdots, x_0^N) \\
%         = \prod_i \int q(x_0^i, x_1^i|S) dx_0^i 
%     \end{split}
% \end{equation}

% \begin{equation}
% \label{eq:right_RHS}
%     RHS = q_1(x_0|S) = \prod_i q_1(x^i_1|S)
% \end{equation}
% Therefore, we are sufficient to show $\int q(x^i_0, x^i_1) dx_0^i = q_1(x_1^i |S)$.

% By considering the fact that, we pre-sample a set of random variables $(x^1_0 \cdots, x^j_0, \cdots, x^M_0) \sim q_0$, and a set of random variables  $(x^1_1, \cdots, x^k_1,\cdots, x^M_1) \sim q_{1|S}$, and have a precomputed bijective mapping $\Pi : \{1, \cdots, M\} \leftrightarrow \{1, \cdots, M\}$.
% %
% With these defined, our coupling $q(x^i_0, x^i_1 |S)$ given $S$ can formulated as:
% \[
%     P(x^i_0 = x^j_0, x^i_1 = x^k_1) =
%     \begin{cases}
%     \frac{1}{n}, & \text{if } k = \Pi(j) \\ 
%         0, & \text{else } k \neq \Pi(j);
%     \end{cases}
% \]
% By Proposition~\ref{prop:ot}, if the superset size $M$ is large enough, we can show that the CDF of Equation~\ref{eq:left_LHS} converge to Equation~\ref{eq:left_RHS}, also the CDF of Equation~\ref{eq:left_LHS} converges to Equation~\ref{eq:left_RHS}.

% To show our coupling maintain the correct marginal, we assume we have $M$ random variables $(X_1, \cdots, X_M) \sim p_1$, and another $M$ random random variables $(Y_1, \cdots, Y_M) \sim p_2$.
% %
% We can additionally take an arbitrary bijective map $\Pi$ between random variables, \ie, $\Pi : \{1, \cdots, M\} \leftrightarrow \{1, \cdots, M\}$.
% %
% If we only sample the a pair of variables based on the bijective map, we can then construct a new random Variable $Z: \{X, Y\}$:
% \[
%     P(X = X_i, Y = Y_j) =
%     \begin{cases}
%     \frac{1}{M}, & \text{if } j = \Pi(i) \\ 
%         0, & \text{else } j \neq \Pi(i);
%     \end{cases}
% \]

% Since $\Pi$ is a bijective mapping, the mariginal distribution of $P(X = X_i)$ and $P(Y = Y_j)$ is also $\frac{1}{M}$.
% %
% Following the result in the previous section, we can show the random variable $X$ ($Y$) still follows $p_1$ ($p_2$).
% %
% In our case, we consider $p_1$ to be a 3D Gaussian distribution, and $p_2$ to be point sample distribution given a Shape $\mathcal{S}$.

% The last part we need to show is that $q_0(x_0)$ and $q_1(x_1|\mathcal{S})$ is independently sampled for each of the point, \ie, $q_0(x_0) = \prod_{i} q_0(x_0^i)$ and \ie, $q_1(x_1) = \prod_{i} q_1(x_1^i | \mathcal{S})$, where $x_0^i$ and $x_1^i$ is the $i$-th point in $x_0$ and $x_1$, respectively.
% %
% For Gaussian distribution $q_0(x_0)$, this is true because it is an unit Gaussian distribution.
% %
% For surface point distribution $q_1(x_1|S)$, it is also correct since the points are indepdently sampled.



% Additionally, for a Gaussian noise sets arranged in the matrix format, $x_0 \in \mathbb{R}^{N \times 3}, x_0 \sim$
\subsubsection{Proof of Hybrid Coupling}

In the last, we would like to show even with our hybrid coupling, the marginal still fulfills the requirements.
%
In particular, we define a new noises $x_0'$ after perturbation:
\begin{equation}
    x_0' = \sqrt{1 - \beta} x_0 + \sqrt{\beta} \epsilon, \epsilon \sim N(\epsilon; 0, I),
\end{equation}
where $\beta \in [0, 1]$ is the blending coefficient. We denoted this as a conditional distribution $q(x_0'| x_0)$, which has a form of $N(x_0'; \sqrt{1 - \beta}x_0, \beta)$.
%
It is noted that since $\epsilon \sim N(\epsilon, 0, I)$, each row of $x'_0$ is also IID given $x_0$, \ie, $q_0(x_0' | x_0) = \prod_i^N \hat{q}_0(x_0^{'i} | x_0^i)$.
%
Due to the independent properties, it is sufficient to show that:
\begin{equation}
    \int q(x_0^{i'}, x_1^i | S) dx_0^{i'} = q_1(x^i_1|S), 
    \int q(x_0^{i'}, x_1^i | S) dx_1^{i} = q_0(x_0^i).
\end{equation}

For the sake of simplicity, we remove all the index $i$ and shape $S$ in the folloings.
We first show the left constraint:
\iffalse
\begin{align}
    q(x_1) & = \int q(x_0', x_1) dx_0' = \int \int q_0(x_0) q(x_0'| x_0) q(x_1|x_0, x_0') dx_0 dx_0' \\
    & = \int \int q_0(x_0) q(x_0'| x_0) q(x_1|x_0) dx_0 dx_0' \\
    & =  \int \int  q_0(x_0) q(x_0'| x_0) q(x_1|x_0)  dx_0' dx_0 \\
    & = \int q_0(x_0) q(x_1|x_0) \int  q(x_0'| x_0)  dx_0' dx_0 \\
    & = \int q_0(x_0) q(x_1|x_0) (1) dx_0 \\
    & = \int q(x_0, x_1) dx_0  = \frac{1}{M} \\
\end{align}
\fi
\begin{align}
    \int q(x_0', x_1) dx_0' & = \int \int q_0(x_0', x_0, x_1) dx_0 dx_0' \\
    & = \int \int q_0(x_0'|x_0) q(x_0, x_1) dx_0 dx_0' \\
    & =  \int q(x_0, x_1) \int  q_0(x_0'|x_0)  dx_0' dx_0 \\
    & = \int q(x_0, x_1) (1) dx_0 \\
    & = q(x_1)
\end{align}
By Proposition~\ref{prop:large_samples}, we can show $q(x_1)$ still converge to the right CDF if $M$ is sufficient large.

On the other hand, we show that:
\iffalse
\begin{align}
    \int q(x_0', x_1) dx_1 &= \int \int q_1(x_0' | x_0, x_1) q_0(x_0|x_1) q(x_1) dx_0 dx_1 \\
    &= \int \int q(x_0' | x_0, x_1) q(x_0|x_1) q(x_1) dx_1 dx_0 \\
    &= \int \int q(x_0' | x_0) q(x_0|x_1) q(x_1) dx_1 dx_0 \\
    & = \int q(x_0'|x_0) \int q(x_0|x_1) q(x_1) dx_1 dx_0 \\
    & = \int q(x_0'|x_0) \sum_{x_1} q(x_0, x_1) dx_0 \\
    & = \int q(x_0'|x_0) \frac{1}{M} dx_0 \\
    & = \frac{1}{M} \sum_{x_0} q(x_0' | x_0) \\
    & = \frac{1}{M} \sum_{x_0} N(x_0'; \sqrt{1 - \beta} x_0, \beta)
\end{align}
\fi
\begin{align}
    \int q(x_0', x_1) dx_1 &= \int \int q_0(x_0', x_0, x_1) dx_0 dx_1 \\
    &= \int \int q_0(x_0', x_0) dx_0\\
    &= \int \int  q_0(x_0'|x_0) q(x_0) dx_0 \\
    & = N(0, I)
\end{align}
where the last equality is obtained by inserting $q(x_0) = N(0, I)$ and $q_0(x_0'|x_0) = N(x_0'; \sqrt{1 - \beta}x_0, \beta I)$.

\iffalse
When $M \rightarrow \infty$, it becomes a convolution of two Gaussian $N(0, (1 - \beta) I)$ and $N(0, \beta I)$.
%
By convolution of Gaussian, we can observe that:
\begin{align}
    \int q(x_0', x_1) dx_1 & = N(0, (1 - \beta)I + \beta I) \\
    & = N(0, I)\\
\end{align}
\fi
\section{Non-Asymptotic Analysis of Training-Time Demonstration Sample Complexity}
\label{appendix:nonasymp-result}

In Theorem~\ref{thm:exponential task generalization}, we established only an \emph{asymptotic} result, showing that as {the number of demonstration samples per task at training time }$n_x \to \infty$, the probability of correctly identifying subtask families $\fP_{\Theta_t}$  tends to one. However, by imposing an additional assumption on the total variation gap between the true distributions and any other hypotheses, it is possible to derive a \emph{non-asymptotic} guarantee on how large $n_x$ must be for accurate subtask identification.

Although maximum-likelihood estimation (MLE) does not directly yield such a non-asymptotic bound in this setting, we can use the same distribution discrimination approach introduced in the inference stage (Lemma~\ref{lem:dist_discrim}). 


\begin{assumption}[Compositional Identifiability with fixed tv marigin]\label{assm: compositional structure with gap}
The autoregressive task class $\mathcal{F}$ satisfies:
\begin{enumerate}
    \item \textbf{Finite Subtask Families}: For each $t \in [T]$, the hypothesis class $\mathcal{P}_{\Xi_t}$ is of size at most $H$ and the subtask conditional distribution family $\mathcal{P}_{\Theta_t} \subseteq \mathcal{P}_{\Xi_t}$ has size $|\mathcal{P}_{\Theta_t}| = \d$.

    \item \textbf{Task Identifiability}: For any $t \in [T]$,  $\theta_{1:t-1} \in \bigtimes_{s=1}^{t-1} \Theta_s$, and  $\theta_t \in \Theta_t$, $\zeta_t \in \Xi_t $, $P_{\zeta_t}\neq P_{\theta_t}$, the induced distributions stasify:
    \[
    \tv\left(P_{\theta_{1:t-1}, \theta_t}, P_{\theta_{1:t-1}, \zeta_t}\right) \geq r > 0.
    \]
    
    Furthermore, for any timestep $t \in [T]$,  $\theta_{1:t-1} \in \bigtimes_{s=1}^{t-1} \Theta_s$, and $\theta_t \neq \theta_t' \in \Theta_t$, the induced distributions satisfy:
    \[
    \tv\left(P_{\theta_{1:t-1},\theta_t}, P_{\theta_{1:t-1},\theta_t'}\right) \geq c > 0.
    \]

\end{enumerate}
\end{assumption}

\begin{theorem}[Exponential Task Generalization]\label{appendix thm:exponential task generalization}
Let $\mathcal{F}$ be an autoregressive compositional task class satisfying Assumption~\ref{assm: compositional structure}. Then there exists a learner $\mathcal{A}$ with the following property: if during training, one samples $n_{\theta}$ tasks uniformly and independently from $\mathcal{F}$, each provided with $n_x$ i.i.d.\ demonstration samples as the training dataset, and if at inference one observes $\ell$
i.i.d.\ demonstration samples from a previously unseen task $P_{\tilde{\theta}}\in\mathcal{F}$, then
\[
\Pr\Bigl[
  \mathcal{A}\bigl(\mathcal{D}_{\demo};\,\mathcal{D}_{\mathrm{train}}\bigr)
  \;\neq\;
  \bigl(P_{\tilde{\theta}_1}, \dots, P_{\tilde{\theta}_T}\bigr)
\Bigr]
\;\le\; \d Te^{-n_\theta/\d} + n_\theta T e^{-c^2\ell/2} + n_\theta THe^{-r^2 n_x/2}.
\]
where $\fD_\train$ and $\fD_\demo$ denote the training dataset and inference-time demonstration samples respectively, and the probability is taken over the random selection of training tasks 
$\mathcal{F}_{\mathrm{train}} \subseteq \mathcal{F}$, 
the training data $\mathcal{D}_{\mathrm{train}}$, 
and the inference time demonstration samples $\mathcal{D}_{\demo}$. 
\end{theorem}

\begin{proof}


Denote the hypothesis class $\fP_{\Xi_t} = \{P_{\xi_{t,1}},\cdots, P_{\xi_{t,|\Xi_t|}}\}$, we present the training stage of the learner.

\begin{algorithm}[H]
\caption{Training Stage with Distribution Dislimination}
\begin{algorithmic}[1]
\Require Training set $\mathcal{D}_{\mathrm{train}}=\{\mathcal{D}_i\}_{i=1}^{n_\theta}$
\For{$i = 1$ to $n_\theta$}
    \For{$t = 1$ to $T$}
      \State Initialize \(P_{\hat\theta_t^i} \gets P_{\xi_{t,1}}\). 
      \For{$k = 2$ to $|\Xi_t|$}
        \State Compute
          \[
          \phi 
          \;\leftarrow\; 
          \frac{1}{n_x}\,\sum_{(\bm x^{i,j},\bm y^{i,j})\in \fD_i}
          \;(-1)^{\,\mathbf{1}\bigl[
            P_{\hat\theta_{1:t-1},\,\hat{\theta}_t^i}({\bm x}^{i,j},{\bm y}^{i,j}_{1:t})
            \;<\;
            P_{\hat\theta_{1:t-1},\,\xi_{t,k}}({\bm x}^{i,j},{\bm y}^{i,j}_{1:t})
          \bigr]}\,.
          \]
        \If{\begin{align*}
          &\left|\,
            \phi 
            \;-\;
            \sum_{(\bm x,\bm y_{1:t})\in \fX\times \fY^t}
            P_{\hat\theta_{1:t-1},\,\xi_{t,k}}(\bm x,\bm y_{1:t})\;
            (-1)^{\,\mathbf{1}\bigl[
            P_{\hat\theta_{1:t-1},\,\hat{\theta}_t^i}({\bm x},{\bm y}_{1:t})
            \;<\;
            P_{\hat\theta_{1:t-1},\,\xi_{t,k}}({\bm x},{\bm y}_{1:t})
          \bigr]}
          \right|\\<&
          \left|\,
            \phi 
            \;-\;
            \sum_{(\bm x,\bm y_{1:t})\in \fX\times \fY^t} 
            P_{\hat\theta_{1:t-1},\,\hat{\theta}_t^i}(\bm x,\bm y_{1:t})\; 
            (-1)^{\,\mathbf{1}\bigl[
            P_{\hat\theta_{1:t-1},\,\hat{\theta}_t^i}({\bm x},{\bm y}_{1:t})
            \;<\;
            P_{\hat\theta_{1:t-1},\,\xi_{t,k}}({\bm x},{\bm y}_{1:t})
          \bigr]}
          \right|.
          \end{align*}}
          \State Update \( P_{\hat{\theta}_t^i} \gets P_{\xi_{t,k}}\).
        \EndIf
      \EndFor
    \EndFor
\EndFor
\State \textbf{return} $\fP_{\hat\Theta_t}=\{{P}_{\hat{\theta}_t^i}\}_{i=1}^{n_\theta}$ for each $t\in[T]$.
\end{algorithmic}
\end{algorithm}

Using the same approach as in Step 2 of the proof of ~\cref{thm:exponential task generalization}, 
\[
\Pr[(P_{\hat \theta^i_1},\cdots,P_{\hat \theta_T^i})\neq (P_{\theta^i_1},\cdots,P_{\theta_T^i})]\leq THe^{-{r^2 n_x}/2}.
\]
By union bound, 
\begin{align*}
\Pr[\exists ~t \in [T]:~ \fP_{\hat \Theta_t} \neq \fP_{\Theta_t }]\leq \Pr[~\exists (t,i):~ ~P_{\hat \theta^i_t}\neq P_{\theta^i_t}]\leq n_\theta T He^{-r^2 n_x/2}.
\end{align*}
The remainder of the proof then proceeds exactly as in ~\cref{thm:exponential task generalization}.

\end{proof}


\section{Extra experiments}\label{appendix:extra_exps}


\paragraph{Effect of Context Length.} The theory assumes access to an infinite number of examples for each training task but does not require infinite demonstrations during inference. However, in practice, we cannot train on an infinite number of examples. Figure \ref{fig:cl_effect} shows that providing sufficient context length during both training and inference is crucial for strong performance. Empirically, we observed that a context length of 40 works reasonably well across all experiments with dimensions up to $d = 20$.

\begin{figure}[h!] 
    \centering
    \includegraphics[width=0.3\textwidth]{Figures/effect_cl.png}
    \caption{The effect of context length on performance.}
    \label{fig:cl_effect}
\end{figure}

\paragraph{ ICL with no CoT fails in even in-distribution generalziation.} We observe in Figure \ref{fig:in_distribution_effect} that transformers with ICL and no CoT struggle to generalize even in simpler in-distribution settings as the number of tasks increases. In the parity task, we refer to in-distribution generalization as a setting where the model is trained on $\mathcal{F}_{train}$ tasks and $\mathcal{S}_{train}$ sequences, and then evaluated on the same set of tasks $\mathcal{F}_{train}$ but with entirely new sequences $\mathcal{S}_{test}$ that were not seen during training.

Here, the setting is the same as in \cite{bhattamishra2024understanding} for \( \text{Parity}(10,2) \), but we used the same tasks during both training and testing. We trained on half of the total sequences, $2^9$ and tested on unseen sequences while keeping the tasks unchanged.



\begin{figure}[h!] 
    \centering
    \includegraphics[width=0.5\textwidth]{Figures/in_distribution_accuracy.png}
    \caption{ICL without CoT even fails to generalize in distribution.}
    \label{fig:in_distribution_effect}
\end{figure}




\section{Experiment Details}\label{appendix: experiment details}

\paragraph{Model and optimization.} We used the transformers library from Hugging Face \cite{wolf2020transformers} to instantiate and train our GPT-2 model from scratch. In all experiments, we used a 3-layer, 1-head configuration. We used the Wadam optimizer \cite{kingma2014adam} with a learning rate of $8 \times 10^{-5}$ and a batch size of 64.


\paragraph{Parity and arithmetic.} In all experiments shown in Figures \ref{fig:ood_generalization} and \ref{fig:arithmetic} for both parity and arithmetic tasks, we used a context length of 40. 

For the arithmetic problem, across all dimensions, we used a total of 25,000 training examples, equally distributed across the training tasks. 

For the parity problem, we used 20,000 training samples, equally distributed across the training tasks for dimensions up to 15. For dimension 20, we increased the total number of training samples to 50,000.


At testing time, we always randomly select the minimum between 200 subsets and all remaining tasks, each containing 500 different sequences with the same context length of 40.

\paragraph{Language experiments.} For the translation experiments, we train a 2-layer Transformer with 3 heads and embedding dimension 768. We use an Adam optimizer with betas being $0.9, 0.95$ and learning rate 3e-4. We will keep the number of total training samples to be $1e6$ and train for 1 pass for 6250 steps.  We choose the languages randomly from the following set $\{ English, French, Spanish, Chinese,
        German, Italian, Japanese, Russian,\\
        Portuguese, Arabic \}$ and meanings (in English) from $\{ cat, dog, house, apple, sky, car ,road\\, tree, bed, water, sun, moon\}$.  We use a GPT-2 tokenizer and in our demonstrations, we will prepend the language of the corresponding word before each word in the following format like ``English: cat''.


\paragraph{Linear Probing}


We append a linear classifier to the checkpoints of models of ``Increasing $D$ for a fixed $T$'' tasks, trained on the hidden states of the final attention layer when generating the $i$-th token in the Chain-of-Thought, with the goal of predicting the $i$-th "secret index." The models are trained on a total number of of $20,000$, $20,000$, and $50,000$ training samples for $d = 10$, $15$, and $20$, respectively. The tasks used for training and validation are disjoint. Only the linear classifier is trained, while the parameters of the transformer are frozen. We use the Adam optimizer with a learning rate of $4 \times 10^{-5}$, and the batch size is set to be $32$.

\end{document}