\section{Introduction}

In his seminal 1981 paper on OS support for database management, Michael Stonebraker described how existing OS buffer cache mechanisms were ill-suited for the needs of databases at the time~\cite{stonebraker_support}. He observed that the buffer cache's one-size-fits-all eviction policy, approximate least-recently used (LRU), cannot possibly address the heterogeneity of database workloads. Nevertheless, in the intervening decades, despite wide-ranging efforts to rethink the UNIX/Linux OS page cache~\cite{parallel-page-cache,mglru,new-readahead}, design customizable file systems~\cite{kai-li-caching,kai-li-caching2,exokernel, bento}, and build clean-slate extensible kernels~\cite{VINO,VINO-2,SPIN}, applications by and large still contend with Linux's opaque and inflexible OS page cache policy. 

At the same time, the diversity of applications and workloads running on Linux has only increased, from enterprise file systems and large-scale distributed datacenter ML training, to multimedia rich applications running on an Android phone. All of these applications must use Linux's decades-old approximate LRU policy, despite the fact it is widely known to be inadequate for many workloads and scenarios (e.g., large scans~\cite{LIRS,frequency-replacement,LRU-K}, multi-core applications~\cite{parallel-page-cache}).
For example, an application that searches through files in a codebase (a scan-based workload) would benefit from using a most-recently used (MRU) policy, while a key-value store running a fixed, skewed-distribution workload would improve under least-frequently used (LFU). However, both of these workloads currently run with the default Linux eviction policy.

The reasons applications are ``stuck'' with the same old eviction policy are twofold. First, modifying the Linux page cache is a hard task, requiring extensive kernel knowledge and attention to detail. Second, upstreaming changes to the page cache is difficult, %
because the changes have to work well for the wide range of applications that run on Linux, forcing a lowest common denominator. For instance, it took Google years to upstream its proposed Multi-Generational LRU (MGLRU) algorithm into the Linux kernel, and even after several years, it is still disabled by default in upstream~\cite{mglru,mglru2}.

In this paper, we attempt to finally answer Stonebraker's plea for better OS support for buffer management, within Linux. To this end, we design a novel framework, \name, which provides visibility and control of the OS page cache, without requiring the application to make kernel changes. \name takes advantage of eBPF~\cite{ebpf}, a Linux (and Windows) supported runtime that allows safely running application code inside the kernel. We take a cue from \texttt{sched\_ext}, an eBPF-based framework that allows applications to customize the OS scheduler~\cite{syrup,ghost} and has been adopted by Linux~\cite{sched_ext}.


\name's design is motivated by four main insights. First, modern storage devices are very fast and support millions of IOPS, so custom page cache policies must run with low overhead. Therefore, we design \name so that its eBPF-based policies run in the kernel, avoiding expensive and frequent synchronization between the kernel and userspace.
Second, caching algorithms are very diverse and may use complex data structures. %
To address this challenge, \name exposes a simple yet flexible interface that allows applications to define one or more variable-sized lists of pages, and a set of policy functions (\eg admission, eviction) that operate on these lists, which can be used to express a wide range of eviction policies.
Third, in order for \name to be useful in multi-tenant scenarios, it should allow each application to use its own policy without interfering with others. We identify cgroups as a natural isolation boundary. Thus, \name allows each cgroup to implement its own eviction policy without interfering with other cgroups.
Finally, custom policies determine which pages to evict and return page references to the kernel. However, these references may be invalid, which could lead to kernel crashes or security breaches. To solve this, \name maintains a registry of valid page references, which is used to validate the page references returned by the user-defined policies.

We demonstrate \name's utility and flexibility by implementing four custom eviction policies, which include both ``classic'' and recently-designed policies: most-recently used (MRU), least-frequently used (LFU), S3-FIFO~\cite{s3-fifo}, and least hit density (LHD)~\cite{lhd}.
We also show how \name enables \emph{application-informed} policies with only minor policy changes, allowing applications to design policies that take into account application-level insights.
For example, a database can implement a custom policy that prioritizes point queries over scans, yielding higher throughput for point queries.
We compare these \name policies with the kernel's default eviction policy and its different options (\eg \texttt{fadvise()}), and with the recently-upstreamed MGLRU algorithm. We show that with \name, developers can significantly improve their applications' performance far beyond the existing algorithms provided by the Linux page cache.
In general, we find that there is no one-size-fits-all policy that improves all workloads -- customization is necessary in order to maximize performance. %
In particular, applications can use \name to improve throughput by up to 38\% using ``generic'' policies, and achieve up to 70\% higher throughput and 58\% lower P99 latency with application-informed policies.





We will open source \name and all our implemented policies upon publication. A key benefit of \name is that any publicly available eviction policy can be used by other developers, lowering the barrier to using the system and experimenting with eviction policies on different workloads. 

Our primary contributions are:
\begin{denseitemize}
\item \name, a flexible, scalable, and safe eBPF-based framework for running custom eviction policies in the Linux kernel page cache.
\item A suite of custom eviction policies and userspace libraries allowing developers to easily create new policies.
\item An evaluation of \name across various applications, demonstrating how they benefit from customized policies.
\end{denseitemize}

