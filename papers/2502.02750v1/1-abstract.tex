\begin{abstract}

The page cache is a central part of an OS. It reduces repeated accesses to storage by deciding which pages to retain in memory. As a result, the page cache has a significant impact on the performance of many applications. However, its one-size-fits-all eviction policy performs poorly in many workloads. While the systems community has experimented with a plethora of new and adaptive eviction policies in non-OS settings (\eg key-value stores, CDNs), it is very difficult to implement such policies in the page cache, due to the complexity of modifying kernel code. To address these shortcomings, we design a novel eBPF-based framework for the Linux page cache, called \name, that allows developers to customize the page cache without modifying the kernel.
\name enables applications to customize the page cache policy for their specific needs, while also ensuring that different applications' policies do not interfere with each other and preserving the page cache's ability to share memory across different processes.
We demonstrate the flexibility of \name's interface by using it to implement several eviction policies. Our evaluation shows that it is indeed beneficial for applications to customize the page cache to match their workloads' unique properties, and that they can achieve up to 70\% higher throughput and 58\% lower tail latency.

\end{abstract}
