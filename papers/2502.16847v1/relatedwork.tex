\section{Related Work}
Pedestrian trajectory prediction is an active field of research with different application areas in autonomous driving and social robotics \cite{rudenko2020human}. Various approaches have been used for this trajectory prediction from physics-based models such as social force model \cite{rudenko2018joint,rinke2017multi,predhumeau2021agent} to data-driven models especially deep learning ones \cite{salzmann2020trajectron++,chandra2019traphic,cheng2020mcenet}. Corresponding to this line of research, many real-world datasets have been collected from different environments to be used as a benchmark for these trajectory prediction algorithms. While the nature of the environment and vehicles' presence can have a considerable influence on the trajectory profile of pedestrians, no clear classification has been made between the different datasets based on quantitative measures such as the pedestrians’ and vehicles' trajectory features. 

Different datasets have been compared in terms of the complexity of the human trajectories for the prediction task by introducing some indicators in \cite{amirian2020opentraj}. However, the focus of this comparison was only on pedestrians’ behaviour and therefore mostly included datasets in which pedestrians are the only agent present without being influenced by vehicles. 

Robicquet et al. have detected different classes of navigation styles for pedestrians in a dataset based on the different ways pedestrians handle their interaction in a collision avoidance situation \cite{robicquet2016learning}. But it is still unclear how these navigation styles can also be affected by the type of the environment. 

Furthermore, traffic participants’ spatial and movement behaviour in different shared space layouts is studied in \cite{batista2022investigating}, considering both vehicles and pedestrians. But the number of behavioural features studied was limited to mean speed, number of interactions, and yield ratio.

Also, in another study focusing on pedestrians, pedestrians’ behaviour in a shopping mall was classified into three categories of “walking straight”,  “finding a way”, and “walking around” in \cite{okamoto2011classification}, showing that roaming around is a common trajectory behaviour that can be seen in a shopping mall environment.
However, the analysis was restricted to one dataset with only pedestrians moving in a mall. Nevertheless, we get inspiration from some of the behavioural measures used in these papers (e.g., path efficiency \cite{amirian2020opentraj}, stop ratio \cite{okamoto2011classification}) for building the quantitative features that can describe unstructured vs. structured environments.