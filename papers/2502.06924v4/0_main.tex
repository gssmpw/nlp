
\documentclass{article} % For LaTeX2e
\usepackage{iclr2025_conference,times}
% \usepackage[iclrfinal]{iclr2025_conference}
% \renewcommand{\baselinestretch}{0.955}
\renewcommand{\baselinestretch}{1}

% Optional math commands from https://github.com/goodfeli/dlbook_notation.
%%%%% NEW MATH DEFINITIONS %%%%%

\usepackage{amsmath,amsfonts,bm}
\usepackage{derivative}
% Mark sections of captions for referring to divisions of figures
\newcommand{\figleft}{{\em (Left)}}
\newcommand{\figcenter}{{\em (Center)}}
\newcommand{\figright}{{\em (Right)}}
\newcommand{\figtop}{{\em (Top)}}
\newcommand{\figbottom}{{\em (Bottom)}}
\newcommand{\captiona}{{\em (a)}}
\newcommand{\captionb}{{\em (b)}}
\newcommand{\captionc}{{\em (c)}}
\newcommand{\captiond}{{\em (d)}}

% Highlight a newly defined term
\newcommand{\newterm}[1]{{\bf #1}}

% Derivative d 
\newcommand{\deriv}{{\mathrm{d}}}

% Figure reference, lower-case.
\def\figref#1{figure~\ref{#1}}
% Figure reference, capital. For start of sentence
\def\Figref#1{Figure~\ref{#1}}
\def\twofigref#1#2{figures \ref{#1} and \ref{#2}}
\def\quadfigref#1#2#3#4{figures \ref{#1}, \ref{#2}, \ref{#3} and \ref{#4}}
% Section reference, lower-case.
\def\secref#1{section~\ref{#1}}
% Section reference, capital.
\def\Secref#1{Section~\ref{#1}}
% Reference to two sections.
\def\twosecrefs#1#2{sections \ref{#1} and \ref{#2}}
% Reference to three sections.
\def\secrefs#1#2#3{sections \ref{#1}, \ref{#2} and \ref{#3}}
% Reference to an equation, lower-case.
\def\eqref#1{equation~\ref{#1}}
% Reference to an equation, upper case
\def\Eqref#1{Equation~\ref{#1}}
% A raw reference to an equation---avoid using if possible
\def\plaineqref#1{\ref{#1}}
% Reference to a chapter, lower-case.
\def\chapref#1{chapter~\ref{#1}}
% Reference to an equation, upper case.
\def\Chapref#1{Chapter~\ref{#1}}
% Reference to a range of chapters
\def\rangechapref#1#2{chapters\ref{#1}--\ref{#2}}
% Reference to an algorithm, lower-case.
\def\algref#1{algorithm~\ref{#1}}
% Reference to an algorithm, upper case.
\def\Algref#1{Algorithm~\ref{#1}}
\def\twoalgref#1#2{algorithms \ref{#1} and \ref{#2}}
\def\Twoalgref#1#2{Algorithms \ref{#1} and \ref{#2}}
% Reference to a part, lower case
\def\partref#1{part~\ref{#1}}
% Reference to a part, upper case
\def\Partref#1{Part~\ref{#1}}
\def\twopartref#1#2{parts \ref{#1} and \ref{#2}}

\def\ceil#1{\lceil #1 \rceil}
\def\floor#1{\lfloor #1 \rfloor}
\def\1{\bm{1}}
\newcommand{\train}{\mathcal{D}}
\newcommand{\valid}{\mathcal{D_{\mathrm{valid}}}}
\newcommand{\test}{\mathcal{D_{\mathrm{test}}}}

\def\eps{{\epsilon}}


% Random variables
\def\reta{{\textnormal{$\eta$}}}
\def\ra{{\textnormal{a}}}
\def\rb{{\textnormal{b}}}
\def\rc{{\textnormal{c}}}
\def\rd{{\textnormal{d}}}
\def\re{{\textnormal{e}}}
\def\rf{{\textnormal{f}}}
\def\rg{{\textnormal{g}}}
\def\rh{{\textnormal{h}}}
\def\ri{{\textnormal{i}}}
\def\rj{{\textnormal{j}}}
\def\rk{{\textnormal{k}}}
\def\rl{{\textnormal{l}}}
% rm is already a command, just don't name any random variables m
\def\rn{{\textnormal{n}}}
\def\ro{{\textnormal{o}}}
\def\rp{{\textnormal{p}}}
\def\rq{{\textnormal{q}}}
\def\rr{{\textnormal{r}}}
\def\rs{{\textnormal{s}}}
\def\rt{{\textnormal{t}}}
\def\ru{{\textnormal{u}}}
\def\rv{{\textnormal{v}}}
\def\rw{{\textnormal{w}}}
\def\rx{{\textnormal{x}}}
\def\ry{{\textnormal{y}}}
\def\rz{{\textnormal{z}}}

% Random vectors
\def\rvepsilon{{\mathbf{\epsilon}}}
\def\rvphi{{\mathbf{\phi}}}
\def\rvtheta{{\mathbf{\theta}}}
\def\rva{{\mathbf{a}}}
\def\rvb{{\mathbf{b}}}
\def\rvc{{\mathbf{c}}}
\def\rvd{{\mathbf{d}}}
\def\rve{{\mathbf{e}}}
\def\rvf{{\mathbf{f}}}
\def\rvg{{\mathbf{g}}}
\def\rvh{{\mathbf{h}}}
\def\rvu{{\mathbf{i}}}
\def\rvj{{\mathbf{j}}}
\def\rvk{{\mathbf{k}}}
\def\rvl{{\mathbf{l}}}
\def\rvm{{\mathbf{m}}}
\def\rvn{{\mathbf{n}}}
\def\rvo{{\mathbf{o}}}
\def\rvp{{\mathbf{p}}}
\def\rvq{{\mathbf{q}}}
\def\rvr{{\mathbf{r}}}
\def\rvs{{\mathbf{s}}}
\def\rvt{{\mathbf{t}}}
\def\rvu{{\mathbf{u}}}
\def\rvv{{\mathbf{v}}}
\def\rvw{{\mathbf{w}}}
\def\rvx{{\mathbf{x}}}
\def\rvy{{\mathbf{y}}}
\def\rvz{{\mathbf{z}}}

% Elements of random vectors
\def\erva{{\textnormal{a}}}
\def\ervb{{\textnormal{b}}}
\def\ervc{{\textnormal{c}}}
\def\ervd{{\textnormal{d}}}
\def\erve{{\textnormal{e}}}
\def\ervf{{\textnormal{f}}}
\def\ervg{{\textnormal{g}}}
\def\ervh{{\textnormal{h}}}
\def\ervi{{\textnormal{i}}}
\def\ervj{{\textnormal{j}}}
\def\ervk{{\textnormal{k}}}
\def\ervl{{\textnormal{l}}}
\def\ervm{{\textnormal{m}}}
\def\ervn{{\textnormal{n}}}
\def\ervo{{\textnormal{o}}}
\def\ervp{{\textnormal{p}}}
\def\ervq{{\textnormal{q}}}
\def\ervr{{\textnormal{r}}}
\def\ervs{{\textnormal{s}}}
\def\ervt{{\textnormal{t}}}
\def\ervu{{\textnormal{u}}}
\def\ervv{{\textnormal{v}}}
\def\ervw{{\textnormal{w}}}
\def\ervx{{\textnormal{x}}}
\def\ervy{{\textnormal{y}}}
\def\ervz{{\textnormal{z}}}

% Random matrices
\def\rmA{{\mathbf{A}}}
\def\rmB{{\mathbf{B}}}
\def\rmC{{\mathbf{C}}}
\def\rmD{{\mathbf{D}}}
\def\rmE{{\mathbf{E}}}
\def\rmF{{\mathbf{F}}}
\def\rmG{{\mathbf{G}}}
\def\rmH{{\mathbf{H}}}
\def\rmI{{\mathbf{I}}}
\def\rmJ{{\mathbf{J}}}
\def\rmK{{\mathbf{K}}}
\def\rmL{{\mathbf{L}}}
\def\rmM{{\mathbf{M}}}
\def\rmN{{\mathbf{N}}}
\def\rmO{{\mathbf{O}}}
\def\rmP{{\mathbf{P}}}
\def\rmQ{{\mathbf{Q}}}
\def\rmR{{\mathbf{R}}}
\def\rmS{{\mathbf{S}}}
\def\rmT{{\mathbf{T}}}
\def\rmU{{\mathbf{U}}}
\def\rmV{{\mathbf{V}}}
\def\rmW{{\mathbf{W}}}
\def\rmX{{\mathbf{X}}}
\def\rmY{{\mathbf{Y}}}
\def\rmZ{{\mathbf{Z}}}

% Elements of random matrices
\def\ermA{{\textnormal{A}}}
\def\ermB{{\textnormal{B}}}
\def\ermC{{\textnormal{C}}}
\def\ermD{{\textnormal{D}}}
\def\ermE{{\textnormal{E}}}
\def\ermF{{\textnormal{F}}}
\def\ermG{{\textnormal{G}}}
\def\ermH{{\textnormal{H}}}
\def\ermI{{\textnormal{I}}}
\def\ermJ{{\textnormal{J}}}
\def\ermK{{\textnormal{K}}}
\def\ermL{{\textnormal{L}}}
\def\ermM{{\textnormal{M}}}
\def\ermN{{\textnormal{N}}}
\def\ermO{{\textnormal{O}}}
\def\ermP{{\textnormal{P}}}
\def\ermQ{{\textnormal{Q}}}
\def\ermR{{\textnormal{R}}}
\def\ermS{{\textnormal{S}}}
\def\ermT{{\textnormal{T}}}
\def\ermU{{\textnormal{U}}}
\def\ermV{{\textnormal{V}}}
\def\ermW{{\textnormal{W}}}
\def\ermX{{\textnormal{X}}}
\def\ermY{{\textnormal{Y}}}
\def\ermZ{{\textnormal{Z}}}

% Vectors
\def\vzero{{\bm{0}}}
\def\vone{{\bm{1}}}
\def\vmu{{\bm{\mu}}}
\def\vtheta{{\bm{\theta}}}
\def\vphi{{\bm{\phi}}}
\def\va{{\bm{a}}}
\def\vb{{\bm{b}}}
\def\vc{{\bm{c}}}
\def\vd{{\bm{d}}}
\def\ve{{\bm{e}}}
\def\vf{{\bm{f}}}
\def\vg{{\bm{g}}}
\def\vh{{\bm{h}}}
\def\vi{{\bm{i}}}
\def\vj{{\bm{j}}}
\def\vk{{\bm{k}}}
\def\vl{{\bm{l}}}
\def\vm{{\bm{m}}}
\def\vn{{\bm{n}}}
\def\vo{{\bm{o}}}
\def\vp{{\bm{p}}}
\def\vq{{\bm{q}}}
\def\vr{{\bm{r}}}
\def\vs{{\bm{s}}}
\def\vt{{\bm{t}}}
\def\vu{{\bm{u}}}
\def\vv{{\bm{v}}}
\def\vw{{\bm{w}}}
\def\vx{{\bm{x}}}
\def\vy{{\bm{y}}}
\def\vz{{\bm{z}}}

% Elements of vectors
\def\evalpha{{\alpha}}
\def\evbeta{{\beta}}
\def\evepsilon{{\epsilon}}
\def\evlambda{{\lambda}}
\def\evomega{{\omega}}
\def\evmu{{\mu}}
\def\evpsi{{\psi}}
\def\evsigma{{\sigma}}
\def\evtheta{{\theta}}
\def\eva{{a}}
\def\evb{{b}}
\def\evc{{c}}
\def\evd{{d}}
\def\eve{{e}}
\def\evf{{f}}
\def\evg{{g}}
\def\evh{{h}}
\def\evi{{i}}
\def\evj{{j}}
\def\evk{{k}}
\def\evl{{l}}
\def\evm{{m}}
\def\evn{{n}}
\def\evo{{o}}
\def\evp{{p}}
\def\evq{{q}}
\def\evr{{r}}
\def\evs{{s}}
\def\evt{{t}}
\def\evu{{u}}
\def\evv{{v}}
\def\evw{{w}}
\def\evx{{x}}
\def\evy{{y}}
\def\evz{{z}}

% Matrix
\def\mA{{\bm{A}}}
\def\mB{{\bm{B}}}
\def\mC{{\bm{C}}}
\def\mD{{\bm{D}}}
\def\mE{{\bm{E}}}
\def\mF{{\bm{F}}}
\def\mG{{\bm{G}}}
\def\mH{{\bm{H}}}
\def\mI{{\bm{I}}}
\def\mJ{{\bm{J}}}
\def\mK{{\bm{K}}}
\def\mL{{\bm{L}}}
\def\mM{{\bm{M}}}
\def\mN{{\bm{N}}}
\def\mO{{\bm{O}}}
\def\mP{{\bm{P}}}
\def\mQ{{\bm{Q}}}
\def\mR{{\bm{R}}}
\def\mS{{\bm{S}}}
\def\mT{{\bm{T}}}
\def\mU{{\bm{U}}}
\def\mV{{\bm{V}}}
\def\mW{{\bm{W}}}
\def\mX{{\bm{X}}}
\def\mY{{\bm{Y}}}
\def\mZ{{\bm{Z}}}
\def\mBeta{{\bm{\beta}}}
\def\mPhi{{\bm{\Phi}}}
\def\mLambda{{\bm{\Lambda}}}
\def\mSigma{{\bm{\Sigma}}}

% Tensor
\DeclareMathAlphabet{\mathsfit}{\encodingdefault}{\sfdefault}{m}{sl}
\SetMathAlphabet{\mathsfit}{bold}{\encodingdefault}{\sfdefault}{bx}{n}
\newcommand{\tens}[1]{\bm{\mathsfit{#1}}}
\def\tA{{\tens{A}}}
\def\tB{{\tens{B}}}
\def\tC{{\tens{C}}}
\def\tD{{\tens{D}}}
\def\tE{{\tens{E}}}
\def\tF{{\tens{F}}}
\def\tG{{\tens{G}}}
\def\tH{{\tens{H}}}
\def\tI{{\tens{I}}}
\def\tJ{{\tens{J}}}
\def\tK{{\tens{K}}}
\def\tL{{\tens{L}}}
\def\tM{{\tens{M}}}
\def\tN{{\tens{N}}}
\def\tO{{\tens{O}}}
\def\tP{{\tens{P}}}
\def\tQ{{\tens{Q}}}
\def\tR{{\tens{R}}}
\def\tS{{\tens{S}}}
\def\tT{{\tens{T}}}
\def\tU{{\tens{U}}}
\def\tV{{\tens{V}}}
\def\tW{{\tens{W}}}
\def\tX{{\tens{X}}}
\def\tY{{\tens{Y}}}
\def\tZ{{\tens{Z}}}


% Graph
\def\gA{{\mathcal{A}}}
\def\gB{{\mathcal{B}}}
\def\gC{{\mathcal{C}}}
\def\gD{{\mathcal{D}}}
\def\gE{{\mathcal{E}}}
\def\gF{{\mathcal{F}}}
\def\gG{{\mathcal{G}}}
\def\gH{{\mathcal{H}}}
\def\gI{{\mathcal{I}}}
\def\gJ{{\mathcal{J}}}
\def\gK{{\mathcal{K}}}
\def\gL{{\mathcal{L}}}
\def\gM{{\mathcal{M}}}
\def\gN{{\mathcal{N}}}
\def\gO{{\mathcal{O}}}
\def\gP{{\mathcal{P}}}
\def\gQ{{\mathcal{Q}}}
\def\gR{{\mathcal{R}}}
\def\gS{{\mathcal{S}}}
\def\gT{{\mathcal{T}}}
\def\gU{{\mathcal{U}}}
\def\gV{{\mathcal{V}}}
\def\gW{{\mathcal{W}}}
\def\gX{{\mathcal{X}}}
\def\gY{{\mathcal{Y}}}
\def\gZ{{\mathcal{Z}}}

% Sets
\def\sA{{\mathbb{A}}}
\def\sB{{\mathbb{B}}}
\def\sC{{\mathbb{C}}}
\def\sD{{\mathbb{D}}}
% Don't use a set called E, because this would be the same as our symbol
% for expectation.
\def\sF{{\mathbb{F}}}
\def\sG{{\mathbb{G}}}
\def\sH{{\mathbb{H}}}
\def\sI{{\mathbb{I}}}
\def\sJ{{\mathbb{J}}}
\def\sK{{\mathbb{K}}}
\def\sL{{\mathbb{L}}}
\def\sM{{\mathbb{M}}}
\def\sN{{\mathbb{N}}}
\def\sO{{\mathbb{O}}}
\def\sP{{\mathbb{P}}}
\def\sQ{{\mathbb{Q}}}
\def\sR{{\mathbb{R}}}
\def\sS{{\mathbb{S}}}
\def\sT{{\mathbb{T}}}
\def\sU{{\mathbb{U}}}
\def\sV{{\mathbb{V}}}
\def\sW{{\mathbb{W}}}
\def\sX{{\mathbb{X}}}
\def\sY{{\mathbb{Y}}}
\def\sZ{{\mathbb{Z}}}

% Entries of a matrix
\def\emLambda{{\Lambda}}
\def\emA{{A}}
\def\emB{{B}}
\def\emC{{C}}
\def\emD{{D}}
\def\emE{{E}}
\def\emF{{F}}
\def\emG{{G}}
\def\emH{{H}}
\def\emI{{I}}
\def\emJ{{J}}
\def\emK{{K}}
\def\emL{{L}}
\def\emM{{M}}
\def\emN{{N}}
\def\emO{{O}}
\def\emP{{P}}
\def\emQ{{Q}}
\def\emR{{R}}
\def\emS{{S}}
\def\emT{{T}}
\def\emU{{U}}
\def\emV{{V}}
\def\emW{{W}}
\def\emX{{X}}
\def\emY{{Y}}
\def\emZ{{Z}}
\def\emSigma{{\Sigma}}

% entries of a tensor
% Same font as tensor, without \bm wrapper
\newcommand{\etens}[1]{\mathsfit{#1}}
\def\etLambda{{\etens{\Lambda}}}
\def\etA{{\etens{A}}}
\def\etB{{\etens{B}}}
\def\etC{{\etens{C}}}
\def\etD{{\etens{D}}}
\def\etE{{\etens{E}}}
\def\etF{{\etens{F}}}
\def\etG{{\etens{G}}}
\def\etH{{\etens{H}}}
\def\etI{{\etens{I}}}
\def\etJ{{\etens{J}}}
\def\etK{{\etens{K}}}
\def\etL{{\etens{L}}}
\def\etM{{\etens{M}}}
\def\etN{{\etens{N}}}
\def\etO{{\etens{O}}}
\def\etP{{\etens{P}}}
\def\etQ{{\etens{Q}}}
\def\etR{{\etens{R}}}
\def\etS{{\etens{S}}}
\def\etT{{\etens{T}}}
\def\etU{{\etens{U}}}
\def\etV{{\etens{V}}}
\def\etW{{\etens{W}}}
\def\etX{{\etens{X}}}
\def\etY{{\etens{Y}}}
\def\etZ{{\etens{Z}}}

% The true underlying data generating distribution
\newcommand{\pdata}{p_{\rm{data}}}
\newcommand{\ptarget}{p_{\rm{target}}}
\newcommand{\pprior}{p_{\rm{prior}}}
\newcommand{\pbase}{p_{\rm{base}}}
\newcommand{\pref}{p_{\rm{ref}}}

% The empirical distribution defined by the training set
\newcommand{\ptrain}{\hat{p}_{\rm{data}}}
\newcommand{\Ptrain}{\hat{P}_{\rm{data}}}
% The model distribution
\newcommand{\pmodel}{p_{\rm{model}}}
\newcommand{\Pmodel}{P_{\rm{model}}}
\newcommand{\ptildemodel}{\tilde{p}_{\rm{model}}}
% Stochastic autoencoder distributions
\newcommand{\pencode}{p_{\rm{encoder}}}
\newcommand{\pdecode}{p_{\rm{decoder}}}
\newcommand{\precons}{p_{\rm{reconstruct}}}

\newcommand{\laplace}{\mathrm{Laplace}} % Laplace distribution

\newcommand{\E}{\mathbb{E}}
\newcommand{\Ls}{\mathcal{L}}
\newcommand{\R}{\mathbb{R}}
\newcommand{\emp}{\tilde{p}}
\newcommand{\lr}{\alpha}
\newcommand{\reg}{\lambda}
\newcommand{\rect}{\mathrm{rectifier}}
\newcommand{\softmax}{\mathrm{softmax}}
\newcommand{\sigmoid}{\sigma}
\newcommand{\softplus}{\zeta}
\newcommand{\KL}{D_{\mathrm{KL}}}
\newcommand{\Var}{\mathrm{Var}}
\newcommand{\standarderror}{\mathrm{SE}}
\newcommand{\Cov}{\mathrm{Cov}}
% Wolfram Mathworld says $L^2$ is for function spaces and $\ell^2$ is for vectors
% But then they seem to use $L^2$ for vectors throughout the site, and so does
% wikipedia.
\newcommand{\normlzero}{L^0}
\newcommand{\normlone}{L^1}
\newcommand{\normltwo}{L^2}
\newcommand{\normlp}{L^p}
\newcommand{\normmax}{L^\infty}

\newcommand{\parents}{Pa} % See usage in notation.tex. Chosen to match Daphne's book.

\DeclareMathOperator*{\argmax}{arg\,max}
\DeclareMathOperator*{\argmin}{arg\,min}

\DeclareMathOperator{\sign}{sign}
\DeclareMathOperator{\Tr}{Tr}
\let\ab\allowbreak


\usepackage{hyperref}
\usepackage{url}
\usepackage{graphicx}

% Added by Arghadip for Table
% \usepackage[format=iclr]{neurips_2023} % ICLR style
\usepackage{booktabs} % For better table formatting
\usepackage{array}    % For extended column formatting
\usepackage{graphicx} % For resizing tables if needed
\usepackage{siunitx}  % For better number alignment
\usepackage{caption}  % For caption formatting


% \title{XAMBA: Enabling Efficient State Space {Models} on Resource-Constrained Neural Processing Units}
\title{XAMBA: Enabling Efficient State Space \\ Models on Resource-Constrained Neural Processing Units}


% Authors must not appear in the submitted version. They should be hidden
% as long as the \iclrfinalcopy macro remains commented out below.
% Non-anonymous submissions will be rejected without review.

% \author{Antiquus S.~Hippocampus, Natalia Cerebro \& Amelie P. Amygdale \thanks{ Use footnote for providing further information
% about author (webpage, alternative address)---\emph{not} for acknowledging
% funding agencies.  Funding acknowledgements go at the end of the paper.} \\
% Department of Computer Science\\
% Cranberry-Lemon University\\
% Pittsburgh, PA 15213, USA \\
% \texttt{\{hippo,brain,jen\}@cs.cranberry-lemon.edu} \\
% \And
% Ji Q. Ren \& Yevgeny LeNet \\
% Department of Computational Neuroscience \\
% University of the Witwatersrand \\
% Joburg, South Africa \\
% \texttt{\{robot,net\}@wits.ac.za} \\
% \AND
% Coauthor \\
% Affiliation \\
% Address \\
% \texttt{email}
% }

% \author{Arghadip Das\thanks{Corresponding author. \texttt{das169@purdue.edu}} \\
% Elmore Family School of Electrical and Computer Engineering\\
% Purdue University, West Lafayette, IN, USA \\
% \texttt{vr@purdue.edu} \\
% \And
% Shamik Kundu, Arnab Raha,\\ \textbf{Soumendu Ghosh, Deepak Mathaikutty} \\
% Advanced Architecture Research Team,\\ NPU IP, CGAI (CCG) \\
% Intel Corporation, Santa Clara, CA, USA \\
% \texttt{\{shamik.kundu, arnab.raha, soumendu.ghosh, deepak.a.mathaikutty\}@intel.com} 
% }

\author{Arghadip Das$^{1}$\thanks{Corresponding author.  \texttt{das169@purdue.edu}}~, Arnab Raha$^{2}$, Shamik Kundu$^{2}$, Soumendu Kumar Ghosh$^{2}$,\\ \textbf{Deepak Mathaikutty}$^{2}$,
    \textbf{and Vijay Raghunathan}$^{1}$ \\
    $^{1}$Electrical and Computer Engineering, Purdue University \\
    $^{2}$Advanced Architecture Research Team, NPU IP, CGAI (CCG), Intel Corporation\\
    \texttt{\{das169, vr\}@purdue.edu} \\
    \texttt{\{arnab.raha, shamik.kundu\}@intel.com} \\
    \texttt{\{soumendu.ghosh, deepak.a.mathaikutty\}@intel.com} 
}



% The \author macro works with any number of authors. There are two commands
% used to separate the names and addresses of multiple authors: \And and \AND.
%
% Using \And between authors leaves it to \LaTeX{} to determine where to break
% the lines. Using \AND forces a linebreak at that point. So, if \LaTeX{}
% puts 3 of 4 authors names on the first line, and the last on the second
% line, try using \AND instead of \And before the third author name.

\newcommand{\fix}{\marginpar{FIX}}
\newcommand{\new}{\marginpar{NEW}}

\setlength{\abovecaptionskip}{0pt}  % Adjust space above the caption
\setlength{\belowcaptionskip}{0pt}  % Adjust space below the caption

\setlength{\textfloatsep}{10pt plus 2pt minus 2pt}  % Space between figure and text (top or bottom)
\setlength{\floatsep}{10pt plus 2pt minus 2pt}      % Space between figures
\setlength{\intextsep}{10pt plus 2pt minus 2pt}     % Space between in-text figures and text

% \usepackage{lineno}  

\iclrfinalcopy


%\iclrfinalcopy % Uncomment for camera-ready version, but NOT for submission.
\begin{document}
% \nolinenumbers

\maketitle

\begin{abstract}
State-Space Models (SSMs) have emerged as efficient alternatives to transformers for sequential data tasks, offering linear or near-linear scalability with sequence length, unlike transformers with quadratic-complexity attention. This makes SSMs ideal for long-sequence tasks in natural language processing (NLP), vision, and edge AI applications such as real-time transcription, translation, and contextual search. These applications demand lightweight, high-performance models for deployment on resource-constrained devices like laptops and PCs. 
\textcolor{black}{While specialized accelerators have been proposed for emerging neural networks, designing new hardware is time-intensive, costly, and impractical for every model. Instead, optimizing models for existing neural processing units (NPUs) in AI PCs offers a scalable and efficient solution.
Towards this end, we propose \textbf{XAMBA}, \textit{the first framework to enable and optimize SSMs on commercial off-the-shelf (COTS) state-of-the-art (SOTA) NPUs}.}
Our approach follows a systematic three-step methodology: (1) enabling SSMs on NPUs, (2) optimizing performance to meet target Key Performance Indicator (KPI) requirements, and (3) trading accuracy for additional performance gains. After enabling SSMs on NPUs, XAMBA addresses key performance bottlenecks with two techniques: \textbf{CumBA} and \textbf{ReduBA}. These replace sequential CumSum and ReduceSum operations with matrix-based computations, significantly improving execution speed and memory efficiency. In addition, \textbf{ActiBA} further enhances performance by mapping computationally expensive activation functions (\emph{e.g.}, Swish, Softplus) to NPU hardware using piecewise linear approximations, reducing latency with minimal accuracy loss. Experimental evaluations on an Intel\textregistered\ Core\texttrademark\ Ultra Series 2 AI PC demonstrate that XAMBA achieves significant performance improvements, reducing execution latency by up to 4.8$\times$ compared to baseline implementation. Our code implementation is available at \href{https://github.com/arghadippurdue/XAMBA}{this link}.

% Our code implementation is available at \href{https://anonymous.4open.science/r/XAMBA/}{https://anonymous.4open.science/r/XAMBA/}.
% XAMBA bridges the gap between SSMs and existing hardware, demonstrating how NPUs can be repurposed to accelerate SSMs without requiring new hardware designs. 




% v-2
% State-Space Models (SSMs) have emerged as efficient alternatives to transformers for sequential data tasks, offering linear or near-linear scalability with sequence length, unlike transformers with quadratic-complexity attention. This makes SSMs ideal for long-sequence tasks in natural language processing (NLP), vision, and edge AI applications such as real-time transcription, translation, and contextual search. These applications demand lightweight, high-performance models for deployment on resource-constrained devices like laptops and PCs. Neural Processing Units (NPUs), specialized for data-parallel tasks, are key to unlocking SSMs' potential at the edge. However, deploying SSMs on NPUs is challenging due to misaligned sequential computations (e.g., Cumulative Sum a.k.a. CumSum, Swish, Softplus) and inefficient memory usage, leading to resource underutilization and increased latency.
% To address these challenges, we propose \textbf{XAMBA}, the first framework to enable and optimize SSMs on commercial off-the-shelf (COTS) state-of-the-art (SOTA) NPUs. Our approach follows a systematic three-step methodology: (1) enabling SSMs on NPUs, (2) optimizing performance, and (3) trading accuracy for further gains. After enabling SSMs on NPUs, XAMBA addresses key performance bottlenecks with two techniques: \textbf{CumBA} and \textbf{ReduBA}. These replace sequential CumSum and ReduceSum operations with matrix-based computations, significantly improving execution speed and memory efficiency. Building on this, another technique \textbf{ActiBA} further enhances performance by mapping computationally expensive activation functions (e.g., Swish, Softplus) to NPU hardware using piecewise linear approximations, reducing latency with minimal accuracy loss. Experimental evaluations on an Intel\textregistered\ Core\texttrademark\ Ultra Series 2 AI PC demonstrate that, XAMBA achieves significant performance improvements, reducing execution latency by up to 2.6$\times$ compared to baseline implementation. 


% State-Space Models (SSMs) have emerged as efficient alternatives to transformers for sequential data tasks, offering linear or near-linear scalability with sequence length.
% DeepSeek
% Unlike transformers, which rely on quadratic-complexity attention mechanisms, SSMs achieve computational efficiency by leveraging principles from convolutional and recurrent neural networks. This makes SSMs highly suitable for long-sequence tasks like natural language processing, computer vision, and edge applications such as personal assistants, real-time transcription, language translation, and contextual search. These applications demand lightweight, high-performance models for deployment on resource-constrained client devices like laptops and PCs. Enabling SSMs on Neural Processing Units (NPUs), specialized accelerators for data-parallel tasks, is critical to unlocking their full potential for edge AI. However, deploying SSMs on NPUs presents unique challenges due to misaligned sequential computations (e.g., CumSum, Swish, Softplus) and inefficient memory usage, leading to underutilized resources and increased latency.
% ChatGPT
% Unlike transformers with quadratic-complexity attention, SSMs achieve efficiency using convolutional and recurrent principles, making them ideal for long-sequence tasks in natural language processing (NLP), vision, and edge AI (e.g., real-time transcription, translation, and contextual search). 





% By bridging the gap between SSMs and NPUs, XAMBA makes SSMs viable for real-world applications on resource-constrained devices.
% \noindent State-Space Models (SSMs) have emerged as efficient alternatives to transformers for sequential data tasks, offering linear or near-linear scalability with sequence length. Unlike transformers, which rely on quadratic-complexity attention mechanisms, SSMs achieve computational efficiency by leveraging principles from convolutional and recurrent neural networks. This makes SSMs highly suitable for long-sequence tasks like natural language processing, computer vision, and medicine, as well as edge applications such as personal assistants and real-time transcription, where they enable transformative AI with reduced resource consumption and improved energy efficiency. However, deploying SSMs on Neural Processing Units (NPUs) presents unique challenges due to misaligned sequential computations (e.g., CumSum, Swish, Softplus) and inefficient memory usage. To address these challenges, we propose \textbf{XAMBA}, the first framework to enable and optimize SSMs on commercial off-the-shelf (COTS) NPUs. XAMBA introduces three key techniques: \textbf{CumBA}, which replaces sequential CumSum with matrix multiplication; \textbf{ReduBA}, which optimizes ReduceSum via matrix-vector multiplication; and \textbf{ActiBA}, which maps activation functions to NPU hardware using piecewise linear approximations with negligible accuracy loss. These innovations reduce execution latency by up to 2.6$\times$ while enhancing memory efficiency, making SSMs viable for real-world applications on resource-constrained devices.
% \noindent State-Space Models (SSMs) such as Mamba and Mamba-2 are increasingly employed in applications requiring sequential data processing, including signal processing, time-series forecasting, and natural language modeling. Efficient execution of these models on Neural Processing Units (NPUs) is crucial to achieving low-latency, energy-efficient inference in resource-constrained environments. However, SSMs pose significant challenges for NPUs due to their memory-bound nature, sequential operations, and irregular data access patterns, which lead to suboptimal performance and resource utilization.
% In this paper, we introduce \textbf{XAMBA}, a framework that addresses these challenges through three key optimizations: \textbf{CumBA}, \textbf{ReduBA}, and \textbf{ActiBA}. CumBA leverages precomputed masks and matrix multiplication to replace the inherently sequential CumSum operation, improving data reuse and eliminating redundant memory accesses. ReduBA employs matrix-vector multiplication for ReduceSum, reducing memory traffic and increasing effective bandwidth by reusing precomputed masks across operations. ActiBA integrates computationally expensive activation functions, such as Swish and Softplus, into the data-drain phase using Configurable Lookup Tables (C-LUTs), thereby avoiding memory overhead and eliminating sequential execution bottlenecks. Additionally, XAMBA exploits the inherent sparsity of CumSum masks using Zero Value Compression (ZVC) and compute skipping via sparsity bitmaps, further reducing memory storage, bandwidth, and compute requirements.
% Our framework is implemented with minimal compiler-level changes, ensuring compatibility with existing NPU architectures without requiring hardware modifications. Experimental results on Intel\textregistered\ Core\texttrademark\ Ultra Series 2 NPUs demonstrate significant performance gains, with CumBA achieving $1.8\times$, ReduBA $1.1\times$, and ActiBA up to $2.6\times$ reductions in execution latency, resulting in an overall speedup of up to $2.6\times$ for Mamba-based models compared to initial out-of-the-box mappings. XAMBA represents a lightweight yet effective solution for optimizing SSM execution on NPUs, improving both performance and resource efficiency.

\end{abstract}


\section{Introduction}

Video generation has garnered significant attention owing to its transformative potential across a wide range of applications, such media content creation~\citep{polyak2024movie}, advertising~\citep{zhang2024virbo,bacher2021advert}, video games~\citep{yang2024playable,valevski2024diffusion, oasis2024}, and world model simulators~\citep{ha2018world, videoworldsimulators2024, agarwal2025cosmos}. Benefiting from advanced generative algorithms~\citep{goodfellow2014generative, ho2020denoising, liu2023flow, lipman2023flow}, scalable model architectures~\citep{vaswani2017attention, peebles2023scalable}, vast amounts of internet-sourced data~\citep{chen2024panda, nan2024openvid, ju2024miradata}, and ongoing expansion of computing capabilities~\citep{nvidia2022h100, nvidia2023dgxgh200, nvidia2024h200nvl}, remarkable advancements have been achieved in the field of video generation~\citep{ho2022video, ho2022imagen, singer2023makeavideo, blattmann2023align, videoworldsimulators2024, kuaishou2024klingai, yang2024cogvideox, jin2024pyramidal, polyak2024movie, kong2024hunyuanvideo, ji2024prompt}.


In this work, we present \textbf{\ours}, a family of rectified flow~\citep{lipman2023flow, liu2023flow} transformer models designed for joint image and video generation, establishing a pathway toward industry-grade performance. This report centers on four key components: data curation, model architecture design, flow formulation, and training infrastructure optimization—each rigorously refined to meet the demands of high-quality, large-scale video generation.


\begin{figure}[ht]
    \centering
    \begin{subfigure}[b]{0.82\linewidth}
        \centering
        \includegraphics[width=\linewidth]{figures/t2i_1024.pdf}
        \caption{Text-to-Image Samples}\label{fig:main-demo-t2i}
    \end{subfigure}
    \vfill
    \begin{subfigure}[b]{0.82\linewidth}
        \centering
        \includegraphics[width=\linewidth]{figures/t2v_samples.pdf}
        \caption{Text-to-Video Samples}\label{fig:main-demo-t2v}
    \end{subfigure}
\caption{\textbf{Generated samples from \ours.} Key components are highlighted in \textcolor{red}{\textbf{RED}}.}\label{fig:main-demo}
\end{figure}


First, we present a comprehensive data processing pipeline designed to construct large-scale, high-quality image and video-text datasets. The pipeline integrates multiple advanced techniques, including video and image filtering based on aesthetic scores, OCR-driven content analysis, and subjective evaluations, to ensure exceptional visual and contextual quality. Furthermore, we employ multimodal large language models~(MLLMs)~\citep{yuan2025tarsier2} to generate dense and contextually aligned captions, which are subsequently refined using an additional large language model~(LLM)~\citep{yang2024qwen2} to enhance their accuracy, fluency, and descriptive richness. As a result, we have curated a robust training dataset comprising approximately 36M video-text pairs and 160M image-text pairs, which are proven sufficient for training industry-level generative models.

Secondly, we take a pioneering step by applying rectified flow formulation~\citep{lipman2023flow} for joint image and video generation, implemented through the \ours model family, which comprises Transformer architectures with 2B and 8B parameters. At its core, the \ours framework employs a 3D joint image-video variational autoencoder (VAE) to compress image and video inputs into a shared latent space, facilitating unified representation. This shared latent space is coupled with a full-attention~\citep{vaswani2017attention} mechanism, enabling seamless joint training of image and video. This architecture delivers high-quality, coherent outputs across both images and videos, establishing a unified framework for visual generation tasks.


Furthermore, to support the training of \ours at scale, we have developed a robust infrastructure tailored for large-scale model training. Our approach incorporates advanced parallelism strategies~\citep{jacobs2023deepspeed, pytorch_fsdp} to manage memory efficiently during long-context training. Additionally, we employ ByteCheckpoint~\citep{wan2024bytecheckpoint} for high-performance checkpointing and integrate fault-tolerant mechanisms from MegaScale~\citep{jiang2024megascale} to ensure stability and scalability across large GPU clusters. These optimizations enable \ours to handle the computational and data challenges of generative modeling with exceptional efficiency and reliability.


We evaluate \ours on both text-to-image and text-to-video benchmarks to highlight its competitive advantages. For text-to-image generation, \ours-T2I demonstrates strong performance across multiple benchmarks, including T2I-CompBench~\citep{huang2023t2i-compbench}, GenEval~\citep{ghosh2024geneval}, and DPG-Bench~\citep{hu2024ella_dbgbench}, excelling in both visual quality and text-image alignment. In text-to-video benchmarks, \ours-T2V achieves state-of-the-art performance on the UCF-101~\citep{ucf101} zero-shot generation task. Additionally, \ours-T2V attains an impressive score of \textbf{84.85} on VBench~\citep{huang2024vbench}, securing the top position on the leaderboard (as of 2025-01-25) and surpassing several leading commercial text-to-video models. Qualitative results, illustrated in \Cref{fig:main-demo}, further demonstrate the superior quality of the generated media samples. These findings underscore \ours's effectiveness in multi-modal generation and its potential as a high-performing solution for both research and commercial applications.

% \section{Motivation} \label{sec_motivation}


While SSMs provide notable efficiency advantages, deploying them on NPUs presents unique challenges due to their computational patterns and hardware requirements. Unlike traditional deep learning models, SSMs exhibit characteristics that deviate from standard kernel operations, necessitating specialized optimizations. Existing NPUs are designed primarily for data-parallel operations like matrix multiplications, which dominate workloads in transformers and CNNs. SSMs, however, involve sequential computations and specialized operators, such as activation functions (e.g., Swish and Softplus) and cumulative summations (CumSum). These operations do not align with the highly parallelized architecture of NPUs, leading to inefficient execution when mapped directly. Fig.~\ref{fig:motivation_exec_lat_brkdwn} highlights execution bottlenecks for Mamba and Mamba-2 models on the Intel\textregistered\ Core\texttrademark\ Ultra Series 2~\cite{lnl} NPU. For Mamba, the majority of execution time is consumed by activation functions, such as Swish (SiLU) and Softplus, which are executed sequentially on DSPs. These DSPs are less optimized for such operations, resulting in prolonged execution times and underutilization of the data-parallel units. In Mamba-2, CumSum and ReduceSum emerge as primary bottlenecks, as these operations also rely on DSPs for sequential processing. This sequential nature hinders efficient reuse of local SRAM, increasing memory traffic and access latency. Both models further face challenges with elementwise multiplication (Multiply), which similarly runs on DSPs and contributes to inefficiencies. Handling long sequences in SSMs requires careful memory optimization. Limited on-chip memory must be utilized effectively to avoid frequent off-chip memory accesses, which incur significant latency and energy costs. The lack of optimized dataflow alignment for SSM computations exacerbates this issue, leading to poor performance. Blind, out-of-the-box mapping of SSMs on NPUs results in suboptimal performance, leaving much of their potential benefits untapped. Addressing these challenges is essential to fully leverage the advantages of SSMs in resource-constrained environments.






% \section{Prior Arts}\label{sec_prior_art}
Several efforts have been made to enable and optimize State Space Models (SSMs) for execution on hardware accelerators like GPUs. These works have made progress in improving the computational efficiency of state-space models for sequence modeling. However, they face significant limitations, which are addressed by the proposed XAMBA approach. Yang et al. (2019)~\cite{ssm_gpu} presented a GPU-optimized framework for accelerating state-space models, focusing on efficiently handling sequential data by optimizing memory transfers and reducing overheads. While this work demonstrated improved throughput, it still relied on GPU-centric architectures, which are not ideal for the memory and throughput challenges of resource-constrained edge devices. Moreover, the solution did not fully explore the specific needs of SSMs in terms of memory management and sequential computation, which are crucial for NPUs. This gap is significant because NPUs are optimized for high-throughput, parallelized computation, which requires more specialized mapping techniques for sequential models like SSMs. Wang et al. (2018)~\cite{rnn_npu} explored deploying LSTM models on NPUs, proposing custom kernels and memory optimization strategies that reduce off-chip memory access and improve data locality. Their work demonstrated improvements in the execution of recurrent models on NPUs, focusing on reducing latency and optimizing memory usage. However, their focus was primarily on traditional recurrent networks (LSTMs) rather than state-space models. The sequential nature of SSMs and the unique dataflow patterns they require for efficient execution on NPUs were not fully addressed. This is a limitation because SSMs, like Mamba, require novel kernel designs and optimizations specific to their compact representations and sequential computations. Without such optimizations, NPUs cannot fully exploit the advantages of SSMs, leading to suboptimal performance. Kiamarzi et al. (2021)~\cite{ssm_fpga} proposed a systematic approach for implementing state-space models in register transfer level (RTL), with a focus on neural network implementation. Their method can be applied to linear/nonlinear and time-varying/time-invariant systems, offering a framework for hardware synthesis of state-space equations. While this approach provides a foundation for hardware implementation, it does not specifically address the unique challenges of deploying SSMs on NPUs, such as dataflow alignment and memory optimization. Additionally, the approach may not fully leverage the parallel processing capabilities of NPUs, which are crucial for achieving high performance in resource-constrained environments. Behrouz et al. (2024)~\cite{mambamixer} introduced MambaMixer, an efficient selective state-space model with dual token and channel selection. This architecture aims to enhance the performance of SSMs by selectively mixing information across tokens and channels. While MambaMixer offers improvements in model efficiency and performance, it does not specifically address the hardware optimization aspects required for efficient execution on NPUs. The integration of such models into hardware accelerators necessitates specialized kernel designs and memory management strategies to fully exploit the capabilities of NPUs. Kiamarzi et al. (2024)~\cite{marca} proposed MARCA, a reconfigurable accelerator architecture designed to address the incompatibilities between Tensor Core-based architectures and Mamba's compute patterns. MARCA introduces a reduction-alternative PE array architecture, reusable nonlinear function units, and advanced buffer management strategies to address the bottlenecks of Mamba computations. While MARCA offers significant performance and energy efficiency improvements for SSMs on high-performance hardware, it does not directly address the unique kernel optimizations and memory alignment strategies critical for NPUs. Additionally, its focus on reconfigurable architecture may introduce overheads unsuitable for resource-constrained edge devices. These limitations highlight the need for more specialized approaches for deploying SSMs on NPUs. These prior works have made valuable contributions to optimizing sequential models for hardware accelerators. However, none of them directly target the unique characteristics of SSMs, especially on NPUs. The proposed XAMBA approach addresses these limitations by introducing novel kernel mappings, optimizing dataflow for sequential computations, and applying efficient memory management techniques specifically for SSMs. By leveraging the full potential of NPUs, XAMBA achieves superior performance and energy efficiency for state-space models, addressing challenges that past work has left unresolved.

% \section{Background}
% \vspace{-11px}
\subsection{Radar Detectors}
% \vspace{-11px}
% A typical radar pipeline is composed of three modules: the hardware setup, the signal processor, and the detector (point cloud extractor). Each is explained below:

% \textbf{Hardware Setup: }The radar hardware setup features a 1D or 2D antenna array for transmitting and receiving electromagnetic signals. At its core is a transceiver, which generates high-power RF signals, transmits them, and switches to receive mode to capture and digitize reflected echoes for post-signal processing \cite{richards2010principles}.

% \textbf{Signal Processor: }The radar signal processor extracts key information—range, velocity, and angle—from digitized signals. Range is calculated from echo time delays, velocity from Doppler frequency shifts, and angle through beamforming techniques like Bartlett's method or FFT. Together, these processes convert raw signals into actionable radar data.
Radar detectors process signals to differentiate targets from noise by applying a decision threshold, comparing the signal strength to a predefined value, and outputting results as point clouds. Conventional radar detectors, like CA-CFAR and OS-CFAR, aim to maintain a consistent false alarm rate by dynamically adjusting decision thresholds based on surrounding noise and clutter. CA-CFAR works well in homogeneous environments but struggles with the heterogeneity of vehicular surroundings. OS-CFAR handles heterogeneous environments but requires precise prior knowledge of target numbers, which might be challenging to estimate. However, a recent deep learning-based radar detector by \cite{roldan2024see} trained on lidar point clouds addresses these limitations, but struggles with sparse representations of its surroundings, compared to lidar. In this work, we produce denser depth maps using these sparse representations.

%-----------------------%
\subsection{Bartlett's Algorithm for Spatial Power Spectrum Estimation}
Bartlett's algorithm, also known as periodogram averaging \cite{Bartlett1948}, is used in signal processing and time series analysis to estimate the power spectral density of a random sequence. It divides the sequence into \( M \) overlapping segments, computes their periodograms, and averages them, with the number of segments being proportional to the spectral resolution. For received signals at spatially distant receptors, like signals received at different camera physical pixels or signals received in multi-antenna wireless systems, the $M$ segments correspond to the signals received at the $M$ receptor. The time difference between segments introduces a phase shift proportional to the spatial frequency \(\omega\), referenced against the complex sinusoidal signal at the first receptor\cite{AoA}, described as:

\begin{equation}
x_1(n) = e^{-j\omega n} , n=0,1,...,N
\end{equation}
where $N$ is the number of samples in the signal. Assuming that we have $M$ segments, each segment has a time delay that is translated into a phase shift in $x_1(n)$, expressed as:
\begin{equation}
x_m(n) = x_1(n)e^{-jm\phi} =e^{-j(\omega n + m\phi)} , m=0,1,...,M-1
\end{equation}
Hence, the $M$ segments matrix $\textbf{S}\in \mathbb{C}^{N\times M}$ is found as:
\begin{equation}
    \textbf{S} = \begin{bmatrix} \textbf{x}_{1} & \textbf{x}_{2} & \cdots & \textbf{x}_{M}  \end{bmatrix}
\end{equation}
Since we are computing the spatial power spectrum, our goal is to calculate the spectrum of the signal that is described by relative phases between segments, which can be found in the covariance matrix $\textbf{C}\in \mathbb{C}^{M\times M}$ of $\textbf{S}$ as:
\begin{equation}
    \textbf{C}= \frac{1}{N}\textbf{S}^H\textbf{S}
\end{equation}
where $H$ denotes the Hermitian or complex conjugate. Therefore, the power spectral density at $\phi$, $P(\phi)$, is found as:
\begin{equation}
    P(\phi) = \textbf{a}(\phi)^H\textbf{C}\textbf{a}(\phi)
\end{equation}
where $\textbf{a}(\phi) = \begin{bmatrix} 1 & e^{-j\phi} & e^{-j2\phi} & \cdots & e^{-j(M-1)\phi}  \end{bmatrix}^T$ is our basis vector. Note that the choice of basis vector depends on the application, pattern of interest, and desired resolution \cite{priestley1981spectral}. A different basis vector is used for our proposed approach, defined in the next section.
% Therefore, the Discrete Fourier Transform (DFT) \cite{mitra2006digital} can be used to calculate the spatial power spectrum as:
% \begin{equation}
%     \textbf{P}(\omega)= \left|\sum_{k=0}^{K-1} \textbf{FCF}^H\right|
% \end{equation}
% where $\textbf{F}$ is the DFT matrix and $K$ is its number of spatial frequency bins ranging from 0 to $M$ with a step of $\frac{m}{M}$.




\section{XAMBA Design Methodology}\label{sec_xamba_design}

\begin{figure}[t!]
\begin{center}
\includegraphics[width=\columnwidth]{Figures/CumBA_ReduBA_ActiBA.pdf}
\end{center}
\caption{XAMBA:
(a) NPU architecture
(b) Sequential CumSum and ReduceSum computation on a DSP.
(c) CumBA and ReduBA masks for optimized computations.
(d) Sequential execution of activation functions (Swish/SiLU and SoftPlus) on DSP.
(e) ActiBA: Efficient execution of SoftPlus and Swish activations using C-LUT in PLU.}\label{fig:xamba}
\end{figure}


Before detailing the design methodology, it is essential first to understand the underlying system architecture (refer Figure~\ref{fig:xamba}). We consider an output-stationary MPU architecture (as shown in Figure~\ref{fig:xamba}(a)) inspired by ~\cite{flexnn}. 
% The core component is the Data Processing Unit (DPU), an $M\times M$ grid of Versatile Processing Elements (VPEs). Each VPE comprises an $N\times N$ array of MAC Processing Elements (MPEs) designed for efficient Multiply-and-Accumulate (MAC) operations. This DPU is well-suited for operations like matrix multiplication, which are fundamental to many neural network computations.
% The architecture includes local SRAM for storing activations and weights, a tensor distribution network for data flow to and from the DPU, and control logic for managing computation, accumulation, and output extraction. MAC operations, integral to DNNs, calculate dot products of weights and activations to produce output feature maps. Each MPE leverages a local data path with register files, multipliers, and accumulators to perform these tasks. Additionally, a Digital Signal Processor (DSP) handles non-linear activation functions, complementing the data-parallel DPU. 
Although our case study considers an output-stationary MPU architecture, the proposed techniques are generic and can be applied to other NPUs without loss of generality. \textbf{Step-1:} The \textit{first step} of XAMBA is to enable SSMs on NPUs. Since NPUs generally support static input shapes, we use a prefill model with a fixed number of input tokens to generate the hidden states, applying padding for smaller inputs. For subsequent token generation, we employ a separate model that uses the cached hidden states to ensure efficiency. 
% This approach ensures compatibility with NPU constraints while maintaining performance.


\subsection{Step-2: Optimizing SSM Performance to meet target KPI requirements}

\textbf{CumBA:}  
As highlighted in Figure~\ref{fig:motivation_exec_lat_brkdwn}, one of the major bottlenecks in executing Mamba-2 on NPUs is the CumSum operation.
\textcolor{black}{A deeper analysis reveals that Mamba-2 contains three CumSum operations per block (Figure ~\ref{fig:mamba_vs_mamba2}), but the primary bottleneck, denoted as $CumSum_b$, accounts for more than 99.9\% of the total CumSum execution time. This bottleneck arises in step-1 of the SSD framework of Mamba-2 (Listing 1 of ~\cite{mamba2}).  
Within SSD, the input sequence is split into blocks, and step 1 computes the intra-chunk output, assuming an initial state of zero. $CumSum_b$, appearing at the start of this step, computes semi-separable (SS) matrices essential for modeling long-range dependencies across input segments.
% Within SSD, the input sequence is partitioned into multiple blocks, and step 1 involves determining the intra-chunk output. This step computes the contribution of each chunk independently, assuming an initial state of zero. $CumSum_b$ corresponds to the efficient computation of semi-separable (SS) matrices at the beginning of step 1, which are fundamental for modeling long-range dependencies across different input segments.  
The bottleneck stems from the large matrix dimensions associated with $CumSum_b$: In Mamba-2 130M, $CumSum_b$ operates on a $256 \times 256$ matrix, whereas the other CumSum operations in the model involve significantly smaller dimensions ($256$ and $2 \times 2$). }
% The sequential nature of CumSum, combined with these large dimensions, exacerbates execution latency, making it a critical performance limiter.}
% \textcolor{red}{A deeper analysis reveals that Mamba-2 contains three CumSum operations per block (Fig.~\ref{fig:mamba_vs_mamba2}), but the bottleneck ($CumSum_b$)—accounting for more than 99.9\% of the total CumSum execution time—is the one in the step-1 of the Structured State Space Duality (SSD) framework of Mamba-2~\cite{mamba2}. Within SSD, the input sequence is split into multiple blocks, and the first step is to determine the intra-chunk output, which computes the contribution of each chunk independently, assuming an initial state of zero. $CumSum_b$ corresponds to the efficient computation of semi-separable (SS) matrices, essential for modeling long-range dependencies across different input segments. The bottleneck arises due to the large matrix dimensions involved with $CumSum_b$: For Mamba-2 130M $CumSum_b$ operates on a $256\times 256$ matrix, whereas the other CumSums in the model have significantly smaller dimensions ($256$ and $2\times 2$).}
% Given that CumSum is inherently sequential, executing it on the DSP leads to excessive latency, as it requires n-width vector adder.}
Figure~\ref{fig:xamba}(b) depicts that executing CumSum on NPUs leads to high latency due to its \textit{sequential nature} on the DSP. Given an input tensor $\mathbf{X} \in \mathbb{R}^{m \times n}$, the standard CumSum operation along the row dimension is expressed as $\mathbf{C}_{i,j} = \sum_{k=1}^{i} \mathbf{X}_{k,j}$ for all $i \in [1, m]$, $j \in [1, n]$. This requires $m$ sequential cycles, assuming the DSP has an $n$-width vector adder. CumSum is processed in smaller chunks for higher-dimensional tensors exceeding register file capacity, requiring frequent on-chip SRAM memory transfers. This increases latency, memory traffic, and bandwidth consumption, leading to inefficient data reuse and performance degradation.
% Additionally, for large tensors, intermediate results must be stored in off-chip memory, increasing memory traffic.

To address these inefficiencies, XAMBA introduces \textbf{CumBA}, which transforms CumSum into a MatMul, leveraging the parallel processing capabilities of the NPU’s MPU. Specifically, CumBA precomputes (at compile-time) a lower triangular mask $\mathbf{M}_{\text{CumBA}}$ (Figure~\ref{fig:xamba}(c)), where $\mathbf{M}_{\text{CumBA}}(i, j) = 1$ if $j \leq i$ and $0$ otherwise. This enables CumSum to be computed as $\mathbf{C} = \mathbf{M}_{\text{CumBA}} \cdot \mathbf{X}$. By remapping CumSum to matrix multiplication, CumBA enables parallel execution, leveraging the MPU’s high-frequency MAC array to perform multiple operations simultaneously. It improves data reuse by utilizing the MPU’s larger local register files, reducing redundant memory reads/writes to SRAM compared to DSP-based execution. The MPU processes MatMul in a tiled manner, further improving data reuse and minimizing costly on-and-off-chip memory transfers. 
% By addressing both computational and memory inefficiencies of sequential CumSum, CumBA significantly enhances performance and resource utilization on NPUs.
% By utilizing the high-frequency MAC array in the DPU, CumBA enables parallel computation, reduces memory traffic, and minimizes SRAM accesses.

\textit{CumBA memory savings using Zero Value Compression (ZVC):}
XAMBA applies ZVC~\cite{zvc} to compress the CumBA mask, a lower triangular binary matrix with $\sim$50\% zeros, significantly reducing storage and memory traffic (Figure~\ref{fig:cumba_zvc}). This compression leads to substantial memory savings, as only non-zero elements are stored. Additionally, modern NPUs utilize sparsity bitmaps to skip zero-value computations, further improving execution speed and energy efficiency. While CumBA benefits from ZVC-driven optimizations, the weights in Mamba and Mamba-2 exhibit minimal inherent sparsity, limiting acceleration gains from sparsity-aware execution. As future work, we plan to explore structured sparsity techniques in SSMs to further enhance NPU efficiency.
% The $\sim$50\% zeros in the CumSum mask of CumBA, represented as a lower triangular binary matrix, enable efficient memory and compute optimization. As shown in Figure~\ref{fig:cumba_zvc}, ZVC~\cite{zvc} reduces storage and memory traffic by storing only non-zero elements. \textcolor{black}{Modern NPUs leverage sparsity bitmaps to skip zero-value computations, enhancing speed and energy efficiency. While CumBA benefits from this, Mamba weights show minimal inherent sparsity. Future work will explore injecting sparsity in SSMs for further acceleration.}
% \textcolor{blue}{Modern NPUs already support unstructured sparsity acceleration using sparsity bitmaps, allowing zero-value computations to be skipped, further improving execution speed and energy efficiency. While CumBA naturally benefits from this sparsity-aware execution, we observe that the weights of Mamba models exhibit negligible inherent sparsity. As part of future work, we plan to explore techniques for injecting sparsity in SSMs to further accelerate computing on NPUs.}
% The NPU’s sparsity-aware compute logic, using sparsity bitmaps, skips zero-value computations, further accelerating execution and improving energy efficiency. 
% As shown in Figure~\ref{fig:cumba_zvc}, storing only non-zero elements significantly reduces memory footprint, while the NPU's sparsity-aware compute logic skips redundant operations, further improving efficiency.
% Both Lunar Lake~\cite{lnl} and Meteor Lake~\cite{mtl} architectures support these capabilities, making them ideal for handling such sparse workloads.

% The $\sim$50\% zeros in the CumSum mask of CumBA, represented as a lower triangular binary matrix, provide a significant opportunity for memory and compute optimization. As shown in Figure~\ref{fig:cumba_zvc}, ZVC compresses the mask by storing only non-zero elements, which reduces both storage requirements and memory traffic. Additionally, the NPU’s sparsity-aware compute logic, using sparsity bitmaps, skips computations for zero values, further accelerating execution and improving energy efficiency. Both Lunar Lake~\cite{lnl} and Meteor Lake~\cite{mtl} architectures support ZVC and sparsity-aware compute, making them well-suited for such workloads. Figure~\ref{fig:cumba_zvc} illustrates this process, where the ZVC-compressed mask is stored in memory and the NPU bypasses unnecessary operations, leveraging both weight and activation sparsity to enhance computational throughput and reduce energy consumption.
% The $\sim$50\% sparsity in $\mathbf{M}_{\text{CumBA}}$ allows for \textbf{Zero Value Compression (ZVC)}~\cite{zvc}, reducing storage and memory traffic. As shown in Figure~\ref{fig:cumba_zvc}, storing only non-zero elements significantly reduces memory footprint, while the NPU's sparsity-aware compute logic skips redundant operations, further improving efficiency.

\textbf{ReduBA:} 
As illustrated in Figure~\ref{fig:motivation_exec_lat_brkdwn}, another significant bottleneck in the execution of Mamba-2 on NPUs is the ReduceSum operation.
\textcolor{black}{A detailed analysis shows that these bottlenecks originate from the reduction sum operations present in every step of the SSD framework in Mamba-2 (Listing 1 of ~\cite{mamba2}).}
Similar to CumSum, the ReduceSum operation suffers from high latency due to sequential DSP execution, as illustrated in Figure~\ref{fig:xamba}(b). Given an input matrix $\mathbf{X} \in \mathbb{R}^{m \times n}$, the ReduceSum along the row dimension is defined as $\mathbf{R}_{j} = \sum_{i=1}^{m} \mathbf{X}_{i,j} = \mathbf{C}_{m,j}$ for all $j \in [1, n]$.  

To mitigate this, XAMBA introduces \textbf{ReduBA}, which reformulates ReduceSum as a matrix-vector multiplication (MVM) using a precomputed vector mask $\mathbf{M}_{\text{ReduBA}}$ (Figure~\ref{fig:xamba}(c)), where $\mathbf{M}_{\text{ReduBA}}(i) = 1$ for all $i$. The ReduceSum operation is then computed as $\mathbf{R} = \mathbf{M}_{\text{ReduBA}} \cdot \mathbf{X}$. 
ReduBA improves upon CumBA by reusing the ReduBA vector mask $\mathbf{M}_{\text{ReduBA}}$ across all operations, reducing memory traffic. Unlike CumBA's matrix-matrix operations, where each computation fetches a new part of the mask, ReduBA’s matrix-vector multiplication applies the same mask repeatedly, minimizing memory accesses and optimizing bandwidth. ReduBA also leverages multiple MAC units in the MPU and a tiled computation strategy, further enhancing data reuse and reducing on-chip memory accesses, resulting in lower latency and improved memory efficiency for ReduceSum operations in Mamba-2 on NPUs.
% ReduBA improves upon CumBA by reusing the ReduBA vector mask $\mathbf{M}_{\text{ReduBA}}$ across all operations, which leads to significant reductions in memory traffic. Unlike the matrix-matrix operations in CumBA, where each computation requires fetching a new mask section, ReduBA's matrix-vector multiplication allows the mask to be applied repeatedly without reloading it. This reuse minimizes memory accesses and maximizes the use of available memory bandwidth. Furthermore, ReduBA takes advantage of multiple MAC units within the DPU and employs a tiled computation strategy, which further enhances data reuse and reduces the number of on-chip memory accesses. These optimizations collectively reduce latency and improve memory efficiency, significantly accelerating ReduceSum operations in Mamba-2 on NPUs.
% Unlike CumBA, ReduBA achieves superior memory savings by reusing $\mathbf{M}_{\text{ReduBA}}$ across all computations, significantly reducing memory traffic and optimizing bandwidth utilization.

\begin{figure}[t!]
\begin{center}
\includegraphics[width=\columnwidth]{Figures/CumBA_ZVC_1_row.pdf}
\end{center}
\caption{CumBA: Enhancing memory, bandwidth, and compute efficiency by exploiting CumBA mask sparsity and two-sided sparsity acceleration in the NPU datapath.}\label{fig:cumba_zvc}
\end{figure}

\subsection{Step-3: Trading Accuracy for Additional Performance Gains}

\textbf{ActiBA:}  
As illustrated in Figure~\ref{fig:motivation_exec_lat_brkdwn}, two of the most significant bottlenecks in Mamba’s execution on NPUs are the Swish (a.k.a. SiLU) and Softplus activation functions. They introduce significant execution overhead when processed sequentially on the DSP, as depicted in Figure~\ref{fig:xamba}(d). 
The SiLU function is defined as $\text{SiLU}(x) = x \cdot \sigma(x)$ with $\sigma(x) = \frac{1}{1 + e^{-x}}$, while Softplus is given by $\text{Softplus}(x) = \frac{1}{\beta} \log (1 + e^{\beta x})$. Both functions exhibit nonlinearity near the origin but transition to linear behavior elsewhere, making them suitable for approximation using piecewise linear functions.
% These functions are defined as $\text{SiLU}(x) = x \cdot \sigma(x)$ with $\sigma(x) = \frac{1}{1 + e^{-x}}$ and $\text{Softplus}(x) = \frac{1}{\beta} \log (1 + e^{\beta x})$. These functions are highly nonlinear near the origin but exhibit linear behavior elsewhere, enabling approximation using piecewise linear functions.

XAMBA introduces \textbf{ActiBA}, which leverages the Piecewise Linear Unit (PLU) found in modern NPUs to efficiently map the Swish and Softplus activation functions. The PLU, located within the Arithmetic Unit (AU) in the MPU's drain path, integrates a C-LUT, as shown in Figure~\ref{fig:xamba}(e).
% XAMBA introduces \textbf{ActiBA}, which efficiently maps the Swish and Softplus activation functions to the NPU’s PLU located within the Arithmetic Unit (AU) in the MPU's drain path. The PLU integrates C-LUT, as shown in Figure~\ref{fig:xamba}(e). 
The C-LUT stores precomputed slopes and intercepts for linear segments, enabling the approximation $f(x) \approx m_k x + c_k$ over intervals $[x_k, x_{k+1}]$. During runtime, the activation function is evaluated directly using the C-LUT, avoiding sequential DSP execution and significantly reducing latency. ActiBA also utilizes \textit{vertical fusion} by performing activation computations during the drain phase, which eliminates the need for storing and reloading intermediate outputs. This reduces memory access overhead and optimizes memory bandwidth usage. The simple linear computations, integrated into the data drain process, further minimize execution latency. By addressing both computational and memory inefficiencies, ActiBA drastically lowers end-to-end latency for Mamba-based models with \textit{negligible loss in quality}~\cite{dse_act}. Increasing the number of linear segments in the non-linear section of the activation functions can further reduce this loss without significantly impacting performance~\cite{flex_sfu}.

% XAMBA introduces \textbf{ActiBA}, which maps Swish and Softplus to the NPU’s Software Programmable Lookup Tables (Spr-LUTs), as illustrated in Figure~\ref{fig:xamba}(e). The LUT stores precomputed slopes and intercepts for linear segments, approximating $f(x) \approx m_k x + c_k$ for $x \in [x_k, x_{k+1}]$. At runtime, the function is evaluated using the LUT, eliminating sequential DSP execution and significantly reducing latency. Additionally, ActiBA leverages vertical fusion, performing activation calculations during the drain phase of preceding layers, minimizing memory overhead and improving dataflow efficiency.







% Version-1: with commented equations
% \subsection{Optimizing SSM Performance on NPUs}

% \textbf{CumBA: Optimizing CumSum Execution}
% As shown in Figure~\ref{fig:xamba}(b), executing cumulative sum (CumSum) on NPUs leads to high latency due to its sequential nature on the DSP. Given an input tensor $\mathbf{X} \in \mathbb{R}^{m \times n}$, the standard CumSum operation along the row dimension is expressed as $\mathbf{C}_{i,j} = \sum_{k=1}^{i} \mathbf{X}_{k,j}$ for all $i \in [1, m]$, $j \in [1, n]$.
% % $\mathbf{Y}{i,j} = \sum{k=1}^{i} \mathbf{X}_{k,j}$ for all $i \in [1, m]$, $j \in [1, n]$:
% % \begin{equation}
% %     \mathbf{Y}_{i,j} = \sum_{k=1}^{i} \mathbf{X}_{k,j}, \quad \forall i \in [1, m], \; j \in [1, n].
% % \end{equation}
% This requires $m$ sequential cycles for each column assuming the DSP has a n-width vector adder. Additionally, for large tensors, intermediate results must be stored in off-chip memory, increasing memory traffic.
% To address these inefficiencies, XAMBA introduces \textbf{CumBA}, which transforms CumSum into a matrix multiplication (MatMul), leveraging the parallel processing capabilities of the NPU’s DPU. Specifically, CumBA precomputes (at compile-time) a lower triangular $\mathbf{M}_{\text{CumBA}}$ (Figure~\ref{fig:xamba}(c)), where $\mathbf{M}_{\text{CumBA}}(i, j) = 1$ if $j \leq i$ and $0$ otherwise.
% % \begin{equation}
% %     \mathbf{M}_{\text{CumBA}} = 
% %     \begin{bmatrix}
% %         1 & 0 & 0 & \cdots & 0 \\
% %         1 & 1 & 0 & \cdots & 0 \\
% %         1 & 1 & 1 & \cdots & 0 \\
% %         \vdots & \vdots & \vdots & \ddots & \vdots \\
% %         1 & 1 & 1 & \cdots & 1
% %     \end{bmatrix} \in \mathbb{R}^{m \times m},
% % \end{equation}
% This enables CumSum to be computed as $\mathbf{C} = \mathbf{M}_{\text{CumBA}} \cdot \mathbf{X}$.
% % \begin{equation}
% %     \mathbf{Y} = \mathbf{M}_{\text{CumBA}} \cdot \mathbf{X}.
% % \end{equation}
% By utilizing the high-frequency MAC array in the DPU, CumBA enables parallel computation, reduces memory traffic, and minimizes SRAM accesses.

% \textbf{Memory Savings with Zero Value Compression (ZVC):}
% The $\sim$50\% sparsity in $\mathbf{M}_{\text{CumBA}}$ allows for \textbf{Zero Value Compression (ZVC)}~\cite{zvc}, reducing storage and memory traffic. As shown in Figure~\ref{fig:cumba_zvc}, storing only non-zero elements significantly reduces memory footprint, while the NPU's sparsity-aware compute logic skips redundant operations, further improving efficiency.

% \textbf{ReduBA: Addressing Sequential Execution Bottlenecks in ReduceSum}
% Similar to CumSum, the ReduceSum operation suffers from high latency due to sequential DSP execution, as illustrated in Figure~\ref{fig:xamba}(b). Given an input matrix $\mathbf{X} \in \mathbb{R}^{m \times n}$, the ReduceSum along the row dimension is defined as $\mathbf{R}_{j} = \sum_{i=1}^{m} \mathbf{X}_{i,j} = \mathbf{C}_{m,j}$ for all $j \in [1, n]$.
% % \begin{equation}
% %     \mathbf{Y}_{j} = \sum_{i=1}^{m} \mathbf{X}_{i,j}, \quad \forall j \in [1, n].
% % \end{equation}
% To mitigate this, XAMBA introduces \textbf{ReduBA}, which reformulates ReduceSum as a matrix-vector multiplication (MVM) using a precomputed vector mask $\mathbf{M}_{\text{ReduBA}}$ (Figure~\ref{fig:xamba}(c)), where $\mathbf{M}_{\text{ReduBA}}(i) = 1$ for all $i$.
% % \begin{equation}
% %     \mathbf{M}_{\text{ReduBA}} = 
% %     \begin{bmatrix}
% %         1 & 1 & 1 & \cdots & 1
% %     \end{bmatrix}^T \in \mathbb{R}^{m \times 1}.
% % \end{equation}
% The ReduceSum operation is then computed as $\mathbf{R} = \mathbf{M}_{\text{ReduBA}}^T \cdot \mathbf{X}$.
% % \begin{equation}
% %     \mathbf{Y} = \mathbf{M}_{\text{ReduBA}}^T \cdot \mathbf{X}.
% % \end{equation}
% Unlike CumBA, ReduBA achieves superior memory savings by reusing $\mathbf{M}_{\text{ReduBA}}$ across all computations, significantly reducing memory traffic and optimizing bandwidth utilization.

% \subsection{Trading Accuracy for Performance Gains}
% % \subsubsection{ActiBA: Efficient Activation Function Mapping}
% Activation functions such as Swish (SiLU) and Softplus introduce significant execution overhead when processed sequentially on the DSP, as depicted in Figure~\ref{fig:xamba}(d). 
% These functions are defined as $\text{SiLU}(x) = x \cdot \sigma(x)$ with $\sigma(x) = \frac{1}{1 + e^{-x}}$ and $\text{Softplus}(x) = \frac{1}{\beta} \log (1 + e^{\beta x})$.
% % Their standard formulations are:
% % \begin{equation}
% %     \text{SiLU}(x) = x \cdot \sigma(x), \quad \sigma(x) = \frac{1}{1 + e^{-x}},
% % \end{equation}
% % \begin{equation}
% %     \text{Softplus}(x) = \frac{1}{\beta} \log (1 + e^{\beta x}).
% % \end{equation}
% These functions are highly nonlinear near the origin but exhibit linear behavior elsewhere, enabling approximation using piecewise linear functions. XAMBA introduces \textbf{ActiBA}, which maps Swish and Softplus to the NPU’s Software Programmable Lookup Tables (Spr-LUTs), as illustrated in Figure~\ref{fig:xamba}(e). The LUT stores precomputed slopes and intercepts for linear segments, approximating $f(x) \approx m_k x + c_k$ for $x \in [x_k, x_{k+1}]$.
% % The LUT stores precomputed slopes and intercepts for piecewise segments:
% % \begin{equation}
% %     f(x) \approx a_k x + b_k, \quad \text{for } x \in [x_k, x_{k+1}].
% % \end{equation}
% At runtime, the function is evaluated using the LUT, eliminating sequential DSP execution and significantly reducing latency. Additionally, ActiBA leverages vertical fusion, performing activation calculations during the drain phase of preceding layers, minimizing memory overhead and improving dataflow efficiency.

% \subsection{Summary}
% XAMBA’s optimizations—CumBA for CumSum, ReduBA for ReduceSum, and ActiBA for activations—effectively mitigate sequential execution inefficiencies in SSMs on NPUs. By leveraging data-parallel execution, memory optimizations, and hardware-aware activation approximations, XAMBA significantly improves performance and efficiency for SSM workloads on edge AI devices.




% % XAMBA introduces an end-to-end methodology (as shown in Figure~\ref{fig:xamba_e2e}) for deploying pre-trained SSMs on a Neural Processing Unit (NPU), such as Intel’s NPU, without requiring retraining or hardware modifications. The approach ensures efficient execution while maintaining model performance by leveraging optimized software and compiler techniques. Key contributions include mapping CumSum and ReduceSum operations as matrix multiplications on the DPU during model compilation, as outlined by CumBA and ReduBA, and programming the Spr-LUT within the PPE to support SiLU and SoftPlus activation functions through ActiBA. This methodology maximizes compatibility and performance, enabling the efficient execution of SSMs on existing NPU hardware. Table~\ref{tab:xamba_techniques} summarizes the novel XAMBA techniques and the areas they target to improve, including compute acceleration, memory bandwidth enhancement, and data transfer reduction.

% % \begin{figure}[t!]
% % \begin{center}
% % \includegraphics[width=0.6\columnwidth]{Figures/XAMBA_e2e.pdf}
% % \end{center}
% % \caption{XAMBA end-to-end methodology highlighting the key steps}\label{fig:xamba_e2e}
% % \end{figure}


% % \begin{table}[h!]
% % \centering
% % \caption{Overview of XAMBA techniques and their corresponding optimization targets, including compute acceleration, memory bandwidth improvement, and data transfer reduction.}
% % \label{tab:xamba_techniques}
% % \begin{tabular}{|l|p{10cm}|}
% % \hline
% % \textbf{XAMBA Technique} & \textbf{Targeted Area} \\ \hline
% % CumBA & Accelerates compute with matrix multiplication for CumSum and boosts memory bandwidth by improving data reuse and reducing redundant memory accesses. \\ \hline
% % ReduBA & Enhances compute with matrix-vector multiplication for ReduceSum and reduces memory traffic by reusing the reduce sum mask. \\ \hline
% % ActiBA & Speeds up compute by offloading activation functions to specialized hardware and reduces memory overhead by avoiding intermediate output storage. \\ \hline
% % \end{tabular}
% % \end{table}



% \subsection{CumBA: Optimizing CumSum Execution on NPUs Using Matrix Multiplication}

% As highlighted in Figure~\ref{fig:motivation_exec_lat_brkdwn}, one of the major bottlenecks in executing Mamba-2 on NPUs is the cumulative sum (CumSum) operation, which suffers from high latency due to its sequential execution on the DSP. Figure~\ref{fig:cumba} provides a detailed view of this process, where CumSum is performed on a matrix of shape $m \times n$, with the summation occurring along the $m$-axis. Given that the DSP is equipped with an $n$-width vector adder, the output for each column is computed sequentially over $m$ cycles. For higher-dimensional tensors, the CumSum operation must be broken down into smaller chunks and processed sequentially, further exacerbating the latency. This approach also causes a significant increase in memory traffic and inefficient data reuse, particularly for tensors whose dimensions exceed the local SRAM capacity, as intermediate results must be written back and forth to off-chip memory.

% To overcome these inefficiencies, XAMBA introduces CumBA, a novel technique that remaps the CumSum operation to a matrix multiplication (MatMul) utilizing the data-parallel processing capabilities of the NPU’s Data Processing Unit (DPU). The DPU, an array of high-frequency Multiply-and-Accumulate (MAC) processing elements, is designed to handle matrix operations with significantly greater parallelism and efficiency compared to the sequential DSP. As shown in Figure~\ref{fig:cumba}, CumBA precomputes a mask during compile time with a shifting and saturating pattern of 1’s tailored to the CumSum operation. This mask enables the transformation of CumSum into a MatMul by reshaping the input tensor as needed.

% This remapping ensures the computation is performed in parallel, leveraging the DPU’s ability to execute multiple operations simultaneously. Additionally, CumBA improves data reuse through DPU-based stencils and eliminates redundant memory reads and writes to SRAM, addressing the inefficiencies of DSP-based execution. The DPU processes MatMul in a tiled manner, further enhancing data reuse within local register files and minimizing costly on-chip SRAM accesses. The result is an accurate output with significantly reduced execution latency. By tackling both computational and memory inefficiencies of sequential CumSum, CumBA achieves substantial improvements in performance and resource utilization on NPUs.


% % \begin{figure}[t!]
% % \begin{center}
% % \includegraphics[width=\columnwidth]{Figures/CumSum_as_MatMul.pdf}
% % \end{center}
% % \caption{CumBA: remapping the CumSum operation to matrix multiplication (MatMul) for efficient execution on NPUs}\label{fig:cumba}
% % \end{figure}


% The $\sim$50\% zeros in the CumSum mask in CumBA, represented as a lower triangular binary matrix, present an opportunity for significant memory and compute optimizations. As shown in Figure~\ref{fig:cumba_zvc}, all elements above the triangular are zero, making the mask highly sparse. By employing Zero Value Compression (ZVC), the storage requirements for the mask are greatly reduced, as only the non-zero elements are stored. This compression also minimizes memory traffic by reducing the volume of data transferred between memory and processing units. Furthermore, the NPU’s support for compute skipping using sparsity bitmaps enables additional acceleration by bypassing computations for zero values. Both Lunar Lake~\cite{lnl} and Meteor Lake~\cite{mtl} architectures support ZVC and sparse compute capabilities, making them well-suited for efficiently handling such workloads. Figure~\ref{fig:cumba_zvc} illustrates this process, showcasing the ZVC-compressed mask stored in memory and the NPU datapath with 2-sided sparsity acceleration logic. The sparsity bitmap allows the NPU to efficiently skip unnecessary operations, leveraging both weight and activation sparsity. This dual approach not only reduces memory usage but also decreases compute operations, accelerating the execution of CumSum and further enhancing energy efficiency for SSMs.




% \subsection{ReduBA: Addressing Sequential Execution Bottlenecks for ReduceSum on NPUs}
% As illustrated in Figure~\ref{fig:motivation_exec_lat_brkdwn}, another significant bottleneck in the execution of Mamba-2 on NPUs is the ReduceSum operation. This inefficiency stems from its sequential execution on the DSP. Figure~\ref{fig:reduba} shows the execution flow of ReduceSum for a tensor of shape $m \times n$ processed along the $m$-axis. The DSP, equipped with an $n$-width vector adder, produces the output over $m$ cycles, as depicted in the figure. For tensors with higher dimensions, the ReduceSum operation is divided into smaller workloads, further increasing execution time. Additionally, for tensors with shapes exceeding the vector width of the DSP, multiple intermediate results must be written to memory, leading to high memory traffic and inefficient local SRAM utilization.

% To mitigate this latency and memory inefficiency, XAMBA introduces ReduBA. As shown in Figure~\ref{fig:reduba}, ReduBA leverages the data-parallel architecture of the Data Processing Unit (DPU) to compute ReduceSum as a matrix-vector multiplication (MVM). DPUs, which consist of arrays of Multiply-and-Accumulate (MAC) processing elements, operate at a higher frequency and support parallel computation more effectively than DSPs. In ReduBA, a vector mask is precomputed at compile time for the ReduceSum operation. This mask captures the reduction pattern, allowing the ReduceSum operation to be reformulated as an MVM. The input tensor is reshaped as required, and the operation is then mapped to the DPU.

% By leveraging matrix-vector multiplication (MVM), ReduBA achieves superior data reuse compared to the matrix-matrix operations in CumBA. Specifically, the reduce sum mask is reused across all operations, significantly reducing memory traffic and effectively increasing the available memory bandwidth. Additionally, ReduBA utilizes multiple MAC units within the DPU and employs a tiled computation strategy, further enhancing data reuse and minimizing on-chip memory accesses. These optimizations collectively result in reduced latency and optimized memory usage for ReduceSum operations, significantly improving the execution efficiency of Mamba-2 on NPUs.


% % \begin{figure}[t!]
% % \begin{center}
% % \includegraphics[width=\columnwidth]{Figures/ReduceSum_as_MatMul.pdf}
% % \end{center}
% % \caption{ReduBA: transforming the ReduceSum operation into matrix-vector multiplication (MVM) to leverage the DPU's parallel processing capabilities}\label{fig:reduba}
% % \end{figure}


% \subsection{ActiBA: Efficient Mapping of Activation Functions Using Software Programmable Lookup Tables}
% As illustrated in Figure~\ref{fig:motivation_exec_lat_brkdwn}, two of the most significant bottlenecks in Mamba’s execution on NPUs are the Swish (SiLU) and Softplus activation functions. Figure~\ref{fig:actiba} demonstrates how these activation functions are processed in a sequential loop on the DSP, where each assembly instruction exhibits varying execution times based on complexity. This sequential processing results in high latency, contributing significantly to overall inefficiencies. 

% \[
% \text{SiLU}(x) = x \cdot \left( \frac{1}{1 + e^{-x}} \right), \quad \text{Softplus}(x) = \frac{1}{\beta} \log \left(1 + e^{\beta x}\right)
% \]

% However, the Swish and Softplus activation functions exhibit linear behavior over most of their domain, except for regions near the origin. This property enables their approximation using piecewise linear functions with minimal computational overhead. To address these bottlenecks, XAMBA introduces ActiBA, which leverages piecewise linear approximation to compute these functions efficiently. ActiBA uses more linear segments near the origin, where the functions are highly nonlinear, and fewer segments farther from the origin, where they become nearly linear.

% Most modern NPUs, including the Intel NPUs~\cite{lnl, mtl} considered in this work, incorporate Software Programmable Lookup Tables (Spr-LUTs) as part of the Data Processing Unit (DPU). Spr-LUTs are specialized hardware designed to approximate nonlinear activation functions directly on the main compute pipeline of the NPU. Unlike traditional approaches that compute these activations on a separate DSP, Spr-LUTs offer a significant performance advantage by avoiding additional communication overhead and exploiting the higher clock frequencies of DPUs.

% In ActiBA, the slopes and intercepts of the piecewise linear segments for Swish and Softplus are precomputed and programmed into the look-up table within the Spr-LUT during compile time (as shown in Figure~\ref{fig:actiba}). At runtime, as the output of the preceding layer is drained from the DPU, it directly passes through the Arithmetic Unit (AU), also known as the Post Processing Element (PPE). During this drain phase, the activation computations are performed in a fused manner on the DPU, leveraging the precomputed slopes and intercepts stored in the Spr-LUT. This approach eliminates the need to offload activation functions to the DSP, thereby avoiding the latency and inefficiencies associated with sequential DSP execution.

% Furthermore, since ActiBA performs activation computations during the drain phase, it eliminates the need to store the intermediate outputs of the preceding layer in memory and subsequently reload them for activation processing (also known as vertical fusion). This significantly reduces memory access overhead, improves memory bandwidth utilization, and enhances overall dataflow efficiency. As these operations are simple linear computations integrated into the data drain process, the execution latency is further minimized. By addressing both computational and memory inefficiencies, ActiBA achieves a substantial reduction in end-to-end latency for Mamba-based models without compromising accuracy.

% % \begin{figure}[t!]
% % \begin{center}
% % \includegraphics[width=\columnwidth]{Figures/SiLU_SoftPlus_SptLUT.pdf}
% % \end{center}
% % \caption{ActiBA: optimize the Swish and Softplus activation functions, using piecewise linear approximations and leveraging Spr-LUTs for efficient computation on NPUs}\label{fig:actiba}
% % \end{figure}




\section{Experimental Methodology}\label{sec_expt_meth}
\textbf{Networks and Datasets:}  
Experiments use pretrained state-space models from HuggingFace, specifically \texttt{mamba-130m-hf} and \texttt{mamba2-130m-hf}, with fixed input tokens of 4.
\textbf{Preprocessing and Conversion:}  
The models are converted from PyTorch to ONNX. They are then converted to OpenVINO~\cite{openvino} IR files (compressing weights to FP16 precision) and compiled into a binary using the OpenVINO NPU compiler. CumBA and ReduBA optimizations are applied during conversion, and ActiBA is emulated by replacing activation functions with ReLU.
% \textbf{Baseline:}  
% Comparisons are made between original models (no optimizations) and optimized models using CumBA, ReduBA, and ActiBA.
\textbf{Platform:}  
Experiments run on an Intel\textregistered\ Core\texttrademark\ Ultra Series 2~\cite{lnl} platform (ASUS Zenbook S 14, 16GB RAM, 256V NPU).
\textbf{Performance Evaluation:}  
Models are evaluated using OpenVINO’s \texttt{benchmark\_app} tool, focusing on inference latency, with optimizations (CumBA, ReduBA, ActiBA) affecting NPU execution efficiency.
\textit{All results were collected using public frameworks (OpenVINO, PyTorch) and are replicable via the provided code (see abstract).}
% \textbf{Results and Replication:}  
% Results, collected via OpenVINO, HWINFO, and PyTorch, are available for replication with optimized models (link in abstract).



% \textbf{Networks and Datasets:}
% The experiments are conducted using pretrained state-space models (SSMs) available from HuggingFace, specifically the models \texttt{mamba-130m-hf} and \texttt{mamba2-130m-hf}. These models are designed to work with a specific input shape (4) and precision (FP16), which are defined before deployment on the NPU.
% \textbf{Preprocessing and Conversion:}
% The models are first exported from PyTorch to ONNX format, and during this conversion, weight compression is applied to reduce the model size. Following this, the models are converted into OpenVINO IR (Intermediate Representation) files, which consist of \texttt{Model.xml} and \texttt{Model.bin}. These IR files are then compiled into a hardware-optimized model blob using the NPU compiler and NPU UMD driver. During the compilation step, optimizations such as CumBA and ReduBA are applied to the models.
% To simulate the effect of ActiBA, the activation functions in the models are replaced with ReLU, and the performance is measured accordingly.
% \textbf{Baseline:}
% The baseline for comparison includes the original models without any optimizations, as well as models using the standard activation functions (before applying ActiBA). The performance is compared between these baseline models and the optimized models that leverage CumBA, ReduBA, and ActiBA.
% \textbf{Platform:}
% Experiments were conducted on an Intel\textregistered\ Core\texttrademark\ Ultra Series 2~\cite{lnl} platform, specifically an ASUS Zenbook S 14 with 16GB RAM and a 256V NPU. The NPU is used for deploying the models and evaluating their performance.
% \textbf{Performance Evaluation:}
% The performance of the models is evaluated using OpenVINO’s \texttt{benchmark\_app} tool, which configures performance hints and input/output precision. The evaluation focuses on latency and throughput, with specific attention given to how the optimizations (CumBA, ReduBA, and ActiBA) impact the execution efficiency on the NPU.
% \textbf{Results and Replication:}
% All results presented are collected using public frameworks, including OpenVINO, HWINFO, and PyTorch. These results can be replicated by using the optimized models provided at the link (referenced in the abstract).

% Fig.~\ref{fig:xamba_e2e} illustrates the experimental workflow for deploying pretrained state-space models (SSMs), such as those from HuggingFace/Transformers, on an NPU (Neural Processing Unit) using OpenVINO. The process begins with the pretrained PyTorch models, where the input shape and precision (e.g., FP16) are defined. These models are converted into OpenVINO IR (Intermediate Representation) by exporting from PyTorch to ONNX format and then compressing the weights during the conversion process. 
% The OpenVINO IR files, represented as \texttt{Model.xml} and \texttt{Model.bin}, are compiled into a hardware-optimized model blob using the NPU compiler and NPU UMD driver. During this step, CumBA and ReduBA are applied for efficient execution. Specialized hardware features, such as Spr-LUT, are leveraged for piecewise linear execution of computationally expensive activation functions like Softplus and SiLU, ensuring faster inference. 
% The compiled model blob is loaded into the OpenVINO Inference Engine, which interfaces directly with the NPU for deployment.



\section{Numerical Results} \label{sec:results}


In this section, numerical results are presented to illustrate the advantages of PASS and validate the effectiveness of the proposed algorithms. Unless stated otherwise, the following setup is used throughout the simulations. We consider $N = 5$ waveguides, each equipped with $M = 6$ pinching antennas and fed by a dedicated RF chain, serving $K = 4$ users. The geometric configuration of the setup is depicted in Fig. \ref{setup}, where the key parameters are set as $d_0 = 15$ m, $d_x = 30$ m, $d_y = 3$ m, and $d_z = 10$ m. The waveguides are uniformly deployed in parallel along the $x$-axis with a spacing of $6$ m, while the users are randomly positioned within the designated service area. The maximum range and minimum spacing of pinching antenna positions are set to $x_{\max} = 50$ m and $\Delta x = 0.1$ m, respectively. For discrete activation, the number of discrete positions is set to $10$ per meter. 


\begin{figure}[t!]
  \centering
  \includegraphics[width=0.45\textwidth]{./Simulation_setup.pdf}
  \caption{The simulation setup.}
  \label{setup}
\end{figure} 

Furthermore, the carrier frequency, channel gain, noise power, and effective index of the waveguide are set to $15$ GHz, $|\eta|^2 = \left(\frac{\lambda}{4 \pi}\right)^2 = -56$ dB, $\sigma_k^2 = -80$ dBm, and $n_g = 1.4$, respectively. The total power radiated from all pinching antennas along each waveguide is constrained by $\sum_{m=1}^{M} \alpha_m^2 = 0.9$, which applies to both equal and proportional power models. The minimum SINR of all users is set to $20$ dB. For the proposed algorithms, the penalty factor is initialized as $\rho = 10$, the reduction factor is set to $\epsilon = 0.1$, and the convergence threshold is set to $10^{-3}$. The number of search points in the one dimensional search for continuous activation is set to $10^6$. Furthermore, due to its sensitivity to initialization, the penalty-based method utilizes the antenna positions obtained from the ZF-based algorithm as its starting point. Unless otherwise specified, all subsequent results are obtained by averaging over $100$ random samples of user positions.

\begin{figure*}[t!]
  \centering
  \begin{subfigure}[t]{0.32\textwidth}
    \includegraphics[width=1\textwidth]{./convergence_2.eps}
    \caption{Transmit power in \textbf{Algorithm \ref{alg:PDD}}.}
    \label{convergence_1}
  \end{subfigure}
  \begin{subfigure}[t]{0.32\textwidth}
    \includegraphics[width=1\textwidth]{./convergence_1.eps}
    \caption{Constraint violation in \textbf{Algorithm \ref{alg:PDD}}.}
    \label{convergence_2}
  \end{subfigure}
  \begin{subfigure}[t]{0.32\textwidth}
    \includegraphics[width=1\textwidth]{./convergence_3.eps}
    \caption{Transmit power in \textbf{Algorithm \ref{alg:ZF}}.}
    \label{convergence_3}
  \end{subfigure}
  \caption{Convergence behavior of the proposed algorithms.}
  \label{convergence}
  \vspace{-0.5cm}
\end{figure*} 

For performance comparison, we consider the following benchmark schemes:
\begin{itemize}
  \item \textbf{Conventional MIMO:} In this benchmark, a conventional MIMO BS is positioned at $(0,0,3)$ and equipped with a uniform linear array along $x$-axis  comprising $N = 5$ antennas, each connected to a dedicated RF chain. The antenna spacing is set to half the wavelength. Under this setup, the signal received by user $k$ is given by 
  \begin{equation}
    \overline{y}_k = \overline{\mathbf{h}}_k^H \sum_{i=1}^K \mathbf{w}_i c_i + n_k,
  \end{equation}
  where $\overline{\mathbf{h}}_k \in \mathbb{C}^{N \times 1}$ is the channel vector. The $k$-th entry of $\overline{\mathbf{h}}_k$ is $\frac{\eta}{\overline{r}_{n,k}}e^{j \beta_0 \overline{r}_{n,k}}$, with $\overline{r}_{n,k}$ denoting the distance between $n$-th antenna and user $k$. The corresponding transmit power minimization problem can be solved using the method described in \cite{bjornson2014optimal}.
  
  \item \textbf{Massive MIMO:} In this benchmark, we consider a massive MIMO BS positioned at $(0,0,3)$, equipped with a uniform linear array along $x$-axis. The number of antennas and RF chains is set to $N = 30$ and $N_{\mathrm{RF}} = 5$, respectively, matching the configuration of the considered PASS system. Furthermore, since each RF chain in the PASS system is connected to only a subset of overall pinching antennas, we assume a similar sub-connected hybrid beamforming architecture at the massive MIMO BS, where each RF chain is connected to a subset of antennas via phase shifters \cite{yu2016alternating}. Under this setup, the signal received by user $k$ is given by 
  \begin{equation}
    \overline{\overline{y}}_k = \overline{\overline{\mathbf{h}}}_k^H \mathbf{W}_{\mathrm{RF}} \sum_{i=1}^K \mathbf{w}_i c_i + n_k,
  \end{equation}
  where $\overline{\overline{\mathbf{h}}}_k \in \mathbb{C}^{N \times 1}$ is the channel vector exhibiting the same form as $\overline{\mathbf{h}}_k$, and $\mathbf{W}_{\mathrm{RF}} \in \mathbb{C}^{N \times N_{\mathrm{RF}}}$ is the analog beamforming matrix realized by the phase shifters. The corresponding transmit power minimization problem is solved by integrating the algorithms in \cite{yu2016alternating} and \cite{shi2018spectral}.  
\end{itemize}

\subsection{Performance of the Proposed Algorithms}



\begin{figure}[t!]
  \centering
  \includegraphics[width=0.45\textwidth]{./algorithm_compare.eps}
  \caption{Comparison between the proposed algorithms under different simulation setup.}
  \label{power_algorithm_compare}
\end{figure} 


Fig. \ref{convergence} illustrates the convergence behavior the proposed algorithms.  Specifically, the convergence behavior of the transmit power $P$ in \textbf{Algorithm \ref{alg:PDD}} is shown in Fig. \ref{convergence_1}. It is interesting to observe that the transmit power $P$ is not reduced monotonically as the iteration progresses, but exhibits an oscillating upward trend before converging to a stable value. This phenomenon is due to the utilization of the penalty method. Specifically, in the initial iterations, when the penalty factor $\rho$ is large, the transmit power $P$ is minimized with an almost unconstrained auxiliary channel matrix $\mathbf{U}$, allowing it to converge to a very low stable value in the inner loop. However, as the penalty factor decreases in the outer loop, constraint violations gradually diminish as shown in Fig. \ref{convergence_2}, forcing the auxiliary channel matrix $\mathbf{U}$ to conform to the channel structure and constraints in PASS. Consequently, the transmit power increases as the outer loop progresses. Fig. \ref{convergence_3} demonstrates the fast convergence of the proposed \textbf{Algorithm \ref{alg:ZF}}. Despite its low complexity, \textbf{Algorithm \ref{alg:ZF}} achieves performance comparable to \textbf{Algorithm \ref{alg:PDD}} across various system setups, as shown in \textbf{Fig. \ref{power_algorithm_compare}}. Therefore, in the following simulations, only the results obtained by \textbf{Algorithm \ref{alg:PDD}} are presented to focus on the performance gains achieved by PASS.

\begin{figure}[t!]
  \centering
  \includegraphics[width=0.45\textwidth]{./power_vs_SINR.eps}
  \caption{Transmit power versus the minimum SINR.}
  \label{power_vs_SINR}
\end{figure} 

Furthermore, Fig. \ref{convergence} reveals that different values of the reduction factor $\epsilon$ affect the convergence speed. Specifically, the algorithm converges more quickly with a smaller $\epsilon$, while a larger $\epsilon$ leads to improved performance. However, it is worth noting that the performance gain from increasing $\epsilon$ from $0.1$ to $0.5$ is negligible. Therefore, we set $\epsilon = 0.1$ for the remaining simulations.

\subsection{Transmit Power Versus the Minimum SINR}

Fig. \ref{power_vs_SINR} shows the impact of the minimum SINR on transmit power. As expected, the transmit power increases with higher minimum SINR requirements, while PASS consistently achieves the lowest transmit power within the considered range. For instance, considering the PASS system with continuous activation and the equal power model, when the minimum SINR of users is 20 dB, PASS significantly reduces the transmit power by $99.3\%$, i.e., from $26.6$ dBm to $4.9$ dBm, compared to conventional MIMO, and by $96.6\%$, i.e., from $19.6$ dBm to $4.9$ dBm, compared to massive MIMO. This remarkable power reduction is primarily due to the ability of PASS to reduce free-space path loss significantly. It is also noteworthy that such an enhancement is achieved by incorporating only low-cost pinching antennas, rather than relying on massive expensive phase shifters as in massive MIMO.

It can also be observed from Fig. \ref{power_vs_SINR} that the proportional power model exhibits almost an identical performance as the equal power model, unveiling the negligible impact of unbalanced antenna efficiency in PASS. Furthermore, while discretely deploying pinching antennas results in non-negligible performance loss, the system still achieves a significant reduction in transmit power compared to both conventional MIMO and massive MIMO. For example, when the minimum SINR is $20$ dB, the PASS with discrete activation reduce the transmit power by $95\%$ and $99\%$ compared to conventional MIMO and massive MIMO, respectively.


\subsection{Transmit Power Versus the Distance}


\begin{figure}[t!]
  \centering
  \includegraphics[width=0.45\textwidth]{./power_vs_distance.eps}
  \caption{Transmit power versus the distance $d_0$.}
  \label{power_vs_distance}
\end{figure} 

Fig. \ref{power_vs_distance} examines the impact of distance $d_0$ on transmit power, where a larger $d_0$ indicates an increased separation between the BS and the serving area. It can be observed that the transmit power achieved by PASS remains almost unchanged as $d_0$ increases. This phenomenon can be attributed to two key factors. First, the pathloss associated with in-waveguide propagation is negligible. Second, the pinching antennas can always be deployed in proximity to the serving area, ensuring an almost constant free-space pathloss. 

Furthermore, it is somewhat counterintuitive that the transmit power in conventional MIMO initially decreases before increasing. This phenomenon arises because conventional MIMO performance is influenced not only by free-space path loss but also by the directivity of the antenna array. More specifically, the uniform linear array used in conventional MIMO can generate a highly directional beam when transmitting in its broadside direction (i.e., with a small angle of departure) \cite{kallnichev2001analysis}. However, as the transmission direction shifts closer to the array's end-fire (i.e., with a large angle of departure), its beamforming capability gradually weakens \cite{kallnichev2001analysis}, resulting in increased inter-user interference. In the considered simulation setup, as illustrated in Fig. \ref{setup}, when the distance $d_0$ is smaller, users are more likely to be positioned in the end-fire directions of the antenna array. This leads to significant inter-user interference, which, in turn, increases the required transmit power. This explains why, for conventional MIMO, its transmit power increases despite the reduction in free-space path loss as the serving area moves closer to the BS. However, for massive MIMO, a smaller distance always results in lower transmit power. This is because a large antenna array can significantly enhance beamforming capability even in the end-fire direction, mitigating the directivity limitations faced by conventional MIMO. Additionally, thanks to the flexibility of pinching antenna deployment, PASS can effectively address this directivity issue as well.

\subsection{Transmit Power Versus the Number of Antennas}

\begin{figure}[t!]
  \centering
  \includegraphics[width=0.45\textwidth]{./power_vs_antenna.eps}
  \caption{Transmit power versus the number of antennas.}
  \label{power_vs_antenna}
  % \vspace{-0.3cm}
\end{figure}

\begin{figure}[t!]
  \centering
  \includegraphics[width=0.45\textwidth]{./power_vs_discrete.eps}
  \caption{Transmit power versus the number of discrete positions.}
  \label{power_vs_discrete}
\end{figure}

Fig. \ref{power_vs_antenna} studies the impact of the number of antennas on transmit power. It can be observed that increasing the number of antennas leads to a reduction in transmit power for all the considered schemes. For instance, when the number of antennas increases from 10 to 50, the transmit power is reduced by $78\%$ for PASS with continuous activation and $68.8\%$ for PASS with discrete activation. This improvement is primarily due to the enhanced beamforming capability and the higher array gain provided by a larger antenna array, which not only mitigates inter-user interference but also strengthens the desired signal, resulting in more efficient power usage. 


\subsection{Transmit Power Versus the Number of Discrete Positions}


Fig. \ref{power_vs_discrete} provides more insights into the discrete activation of PASS, specifically focusing on the impact of the number of available discrete positions. As expected, the performance of discrete activation gradually approaches that of continuous activation as the number of discrete positions increases. However, achieving performance comparable to continuous activation requires a significantly large number of discrete positions, i.e., larger than $300$ per meter. This is because effective beamforming gain becomes challenging to achieve with a limited number of discrete positions. To understand this, consider the phase shift $e^{-j \beta_{\mathrm{g}} x_{\mathrm{p}}}$ induced by a pinching antenna located at position $x_{\mathrm{p}}$, where $\beta_{\mathrm{g}} = \frac{2 \pi n_{\mathrm{g}}}{\lambda} \approx 440$ under the given simulation setup. For optimal beamforming capability, $\beta_{\mathrm{g}} x_{\mathrm{p}}$ must be adjustable across the full range of $[0, 2\pi]$. However, due to the large value of $\beta_{\mathrm{g}}$, achieving this flexibility requires sampling $x_{\mathrm{p}}$ at very fine intervals, necessitating a high density of discrete positions. This result underscores the importance of developing high-resolution pinching antenna activation structures for the practical PASS implementation.    

\section{Conclusion \& Future Work}\label{conclusion}
This work presents XAMBA, the first framework optimizing SSMs on COTS NPUs, removing the need for specialized accelerators. XAMBA mitigates key bottlenecks in SSMs like CumSum, ReduceSum, and activations using ActiBA, CumBA, and ReduBA, transforming sequential operations into parallel computations. These optimizations improve latency, throughput (Tokens/s), and memory efficiency. Future work will extend XAMBA to other models, explore compression, and develop dynamic optimizations for broader hardware platforms.



% This work introduces XAMBA, the first framework to optimize SSMs on COTS NPUs, eliminating the need for specialized hardware accelerators. XAMBA addresses key bottlenecks in SSM execution, including CumSum, ReduceSum, and activation functions, through techniques like ActiBA, CumBA, and ReduBA, which restructure sequential operations into parallel matrix computations. These optimizations reduce latency, enhance throughput, and improve memory efficiency. 
% Experimental results show up to 2.6$\times$ performance improvement on Intel\textregistered\ Core\texttrademark\ Ultra Series 2 AI PC. 
% Future work will extend XAMBA to other models, incorporate compression techniques, and explore dynamic optimization strategies for broader hardware platforms.


% This work presents XAMBA, an optimization framework that enhances the performance of SSMs on NPUs. Unlike transformers, SSMs rely on structured state transitions and implicit recurrence, which introduce sequential dependencies that challenge efficient hardware execution. XAMBA addresses these inefficiencies by introducing CumBA, ReduBA, and ActiBA, which optimize cumulative summation, ReduceSum, and activation functions, respectively, significantly reducing latency and improving throughput. By restructuring sequential computations into parallelizable matrix operations and leveraging specialized hardware acceleration, XAMBA enables efficient execution of SSMs on NPUs. Future work will extend XAMBA to other state-space models, integrate advanced compression techniques like pruning and quantization, and explore dynamic optimization strategies to further enhance performance across various hardware platforms and frameworks.
% This work presents XAMBA, an optimization framework that enhances the performance of SSMs on NPUs. Key techniques, including CumBA, ReduBA, and ActiBA, achieve significant latency reductions by optimizing operations like cumulative summation, ReduceSum, and activation functions. Future work will focus on extending XAMBA to other state-space models, integrating advanced compression techniques, and exploring dynamic optimization strategies to further improve performance across various hardware platforms and frameworks.

% This work introduces XAMBA, an optimization framework for improving the performance of Mamba-2 and Mamba models on NPUs. XAMBA includes three key techniques: CumBA, ReduBA, and ActiBA. CumBA reduces latency by transforming cumulative summation operations into matrix multiplication using precomputed masks. ReduBA optimizes the ReduceSum operation through matrix-vector multiplication, reducing execution time. ActiBA accelerates activation functions like Swish and Softplus by mapping them to specialized hardware during the DPU’s drain phase, avoiding sequential execution bottlenecks. Additionally, XAMBA enhances memory efficiency by reducing SRAM access, increasing data reuse, and utilizing Zero Value Compression (ZVC) for masks. The framework provides significant latency reductions, with CumBA, ReduBA, and ActiBA achieving up to 1.8X, 1.1X, and 2.6X reductions, respectively, compared to the baseline.
% Future work includes extending XAMBA to other state-space models (SSMs) and exploring further hardware optimizations for emerging NPUs. Additionally, integrating advanced compression techniques like pruning and quantization, and developing adaptive strategies for dynamic optimization, could enhance performance. Expanding XAMBA's compatibility with other frameworks and deployment environments will ensure broader adoption across various hardware platforms.

\subsubsection*{Acknowledgments}
This work was supported in part by the Center for the Co-Design of Cognitive Systems (CoCoSYS) and the Center on Cognitive Multispectral Sensors (CogniSense), two research centers under the Joint University Microelectronics Program (JUMP) 2.0, a Semiconductor Research Corporation (SRC) initiative sponsored by DARPA. The authors would like to thank Souvik Kundu (Intel Labs) and Zhifan Ye (Georgia Tech) for insightful discussions that contributed to this work.

% 

\newpage
\appendix
\section{Applicability of SparseTransX for dense graphs} 
\label{A:density}
Even for fully dense graphs, our KGE computations remain highly sparse. This is because our SpMM leverages the incidence matrix for triplets, rather than the graph's adjacency matrix. In the paper, the sparse matrix $A \in \{-1,0,1\}^{M \times (N+R)}$ represents the triplets, where $N$ is the number of entities, $R$ is the number of relations, and $M$ is the number of triplets. This representation remains extremely sparse, as each row contains exactly three non-zero values (or two in the case of the "ht" representation). Hence, the sparsity of this formulation is independent of the graph's structure, ensuring computational efficiency even for dense graphs.

\section{Computational Complexity}
\label{A:complexity}
 For a sparse matrix $A$ with $m \times k$ having $nnz(A)=$ number of non zeros and dense matrix $X$ with $k \times n$ dimension, the computational complexity of the SpMM is $O(nnz(A) \cdot n)$ since there are a total of $nnz(A)$ number of dot products each involving $n$ components. Since our sparse matrix contains exactly three non-zeros in each row, $nnz(A) = 3m$. Therefore, the complexity of SpMM is $O(3m \cdot n)$ or $O(m \cdot n)$, meaning the complexity increases when triplet counts or embedding dimension is increased. Memory access pattern will change when the number of entities is increased and it will affect the runtime, but the algorithmic complexity will not be affected by the number of entities/relations.

\section{Applicability to Non-translational Models}
\label{A:non_trans}
Our paper focused on translational models using sparse operations, but the concept extends broadly to various other knowledge graph embedding (KGE) methods. Neural network-based models, which are inherently matrix-multiplication-based, can be seamlessly integrated into this framework. Additionally, models such as DistMult, ComplEx, and RotatE can be implemented with simple modifications to the SpMM operations. Implementing these KGE models requires modifying the addition and multiplication operators in SpMM, effectively changing the semiring that governs the multiplication.   

In the paper, the sparse matrix $A \in \{-1,0,1\}^{M \times (N+R)}$ represents the triplets, and the dense matrix $E \in \mathbb{R}^{(N+R) \times d}$ represents the embedding matrix, where $N$ is the number of entities, $R$ is the number of relations, and $M$ is the number of triplets. TransE’s score function, defined as $h + r - t$, is computed by multiplying $A$ and $E$ using an SpMM followed by the L2 norm. This operation can be generalized using a semiring-based SpMM model: $Z_{ij} = \bigoplus_{k=1}^{n} (A_{ik} \otimes E_{kj})$

Here, $\oplus$ represents the semiring addition operator, and $\otimes$ represents the semiring multiplication operator. For TransE, these operators correspond to standard arithmetic addition and multiplication, respectively.

\subsection*{DistMult} 
DistMult’s score function has the expression $h \odot r \odot t$. To adapt SpMM for this model, two key adjustments are required: The sparse matrix $A$ stores $+1$ at the positions corresponding to $h_{\text{idx}}$, $t_{\text{idx}}$, and $r_{\text{idx}}$. Both the semiring addition and multiplication operators are set to arithmetic multiplication. These changes enable the use of SpMM for the DistMult score function.

\subsection*{ComplEx} 
ComplEx’s score function has $h \odot r \odot \bar{t}$, where embeddings are stored as complex numbers (e.g., using PyTorch). In this case, the semiring operations are similar to DistMult, but with complex number multiplication replacing real number multiplication.

\subsection*{RotatE} 
RotatE’s score function has $h \odot r - t$. For this model, the semiring requires both arithmetic multiplication and subtraction for $\oplus$. With minor modifications to our SpMM implementation, the semiring addition operator can be adapted to compute $h \odot r - t$.

\subsection*{Support from other libraries}
Many existing libraries, such as GraphBLAS (Kimmerer, Raye, et al., 2024), Ginkgo (Anzt, Hartwig, et al., 2022), and Gunrock (Wang, Yangzihao, et al., 2017), already support custom semirings in SpMM. We can leverage C++ templates to extend support for KGE models with minimal effort.


\begin{figure*}[t]
\centering     %%% not \center
\includegraphics[width=\textwidth]{figures/all-eval.pdf}
\caption{Loss curve for sparse and non-sparse approach. Sparse approach eventually reaches the same loss value with similar Hits@10 test accuracy.}
\label{fig:loss_curve}
\end{figure*}

\section{Model Performance Evaluation and Convergence}
\label{A:eval}
SpTransX follows a slightly different loss curve (see Figure \ref{fig:loss_curve}) and eventually converges with the same loss as other non-sparse implementations such as TorchKGE. We test SpTransX with the WN18 dataset having embedding size 512 (128 for TransR and TransH due to memory limitation) and run 200-1000 epochs. We compute average Hits@10 of 9 runs with different initial seeds and a learning rate scheduler. The results are shown below. We find that Hits@10 is generally comparable to or better than the Hits@10 achieved by TorchKGE.

\begin{table}[h]
\centering
\caption{Average of 9 Hits@10 Accuracy for WN18 dataset}
\begin{tabular}{|c|c|c|}
\hline
\textbf{Model} & \textbf{TorchKGE} & \textbf{SpTransX} \\ \hline
TransE         & 0.79 ± 0.001700   & 0.79 ± 0.002667   \\ \hline
TransR         & 0.29 ± 0.005735   & 0.33 ± 0.006154   \\ \hline
TransH         & 0.76 ± 0.012285   & 0.79 ± 0.001832   \\ \hline
TorusE         & 0.73 ± 0.003258   & 0.73 ± 0.002780   \\ \hline
\end{tabular}
\label{table:perf_eval}
\end{table}

% We also plot the loss curve for different models in Figure \ref{fig:loss_curve}. We observe that the sparse approach follows a similar loss curve and eventually converges to the same final loss.

\section{Distributed SpTransX and Its Applicability to Large KGs}
\label{A:dist}
SpTransX framework includes several features to support distributed KGE training across multi-CPU, multi-GPU, and multi-node setups. Additionally, it incorporates modules for model and dataset streaming to handle massive datasets efficiently. 

Distributed SpTransX relies on PyTorch Distributed Data Parallel (DDP) and Fully Sharded Data Parallel (FSDP) support to distribute sparse computations across multiple GPUs. 

\begin{table}[h]
\centering
\caption{Average Time of 15 Epochs (seconds). Training time of TransE model with Freebase dataset (250M triplets, 77M entities. 74K relations, batch size 393K)  on 32 NVIDIA A100 GPUs. FSDP enables model training with larger embedding when DDP fails.}
\begin{tabular}{|p{2cm}|p{2.5cm}|p{2.5cm}|}
\hline
\textbf{Embedding Size} & \textbf{DDP (Distributed Data Parallel)} & \textbf{FSDP (Fully Sharded Data Parallel)} \\ \hline
16                      & 65.07 ± 1.641                            & 63.35 ± 1.258                               \\ \hline
20                      & Out of Memory                            & 96.44 ± 1.490                               \\ \hline
\end{tabular}
\end{table}

We run an experiment with a large-scale KG to showcase the performance of distributed SpTransX. Freebase (250M triplets, 77M entities. 74K relations, batch size 393K) dataset is trained using the TransE model on 32 NVIDIA A100 GPUs of NERSC using various distributed settings. SpTransX’s Streaming dataset module allows fetching only the necessary batch from the dataset and enables memory-efficient training. FSDP enables model training with larger embedding when DDP fails.

\section{Scaling and Communication Bottlenecks for Large KG Training}
\label{A:scaling}
Communication can be a significant bottleneck in distributed KGE training when using SpMM. However, by leveraging Distributed Data-Parallel (DDP) in PyTorch, we successfully scale distributed SpTransX to 64 NVIDIA A100 GPUs with reasonable efficiency. The training time for the COVID-19 dataset with 60,820 entities, 62 relations, and 1,032,939 triplets is in Table \ref{table:scaling}. 
% \vspace{-.3cm}
\begin{table}[h]
\centering
\caption{Scaling TransE model on COVID-19 dataset}
\begin{tabular}{|c|c|}
\hline
\textbf{Number of GPUs} & \textbf{500 epoch time (seconds)} \\ \hline
4                       & 706.38                            \\ \hline
8                       & 586.03                            \\ \hline
16                      & 340.00                               \\ \hline
32                      & 246.02                            \\ \hline
64                      & 179.95                            \\ \hline
\end{tabular}
\label{table:scaling}
\end{table}
% \vspace{-.2cm}
It indicates that communication is not a bottleneck up to 64 GPUs. If communication becomes a performance bottleneck at larger scales, we plan to explore alternative communication-reducing algorithms, including 2D and 3D matrix distribution techniques, which are known to minimize communication overhead at extreme scales. Additionally, we will incorporate model parallelism alongside data parallelism for large-scale knowledge graphs.

\section{Backpropagation of SpMM}
\label{A:backprop}
 Our main computational kernel is the sparse-dense matrix multiplication (SpMM). The computation of backpropagation of an SpMM w.r.t. the dense matrix is also another SpMM. To see how, let's consider the sparse-dense matrix multiplication $AX = C$ which is part of the training process. As long as the computational graph reduces to a single scaler loss $\mathfrak{L}$, it can be shown that $\frac{\partial C}{\partial X} = A^T$. Here, $X$ is the learnable parameter (embeddings), and $A$ is the sparse matrix. Since $A^T$ is also a sparse matrix and $\frac{\partial \mathfrak{L}}{\partial C}$ is a dense matrix, the computation $\frac{\partial \mathfrak{L}}{\partial X} = \frac{\partial C}{\partial X} \times \frac{\partial \mathfrak{L}}{\partial C} = A^T \times \frac{\partial \mathfrak{L}}{\partial C} $ is an SpMM. This means that both forward and backward propagation of our approach benefit from the efficiency of a high-performance SpMM.

\subsection*{Proof that $\frac{\partial C}{\partial X} = A^T$}
 To see why $\frac{\partial C}{\partial X} = A^T$ is used in the gradient calculation, we can consider the following small matrix multiplication without loss of generality.
\begin{align*}
A &= \begin{bmatrix}
a_1 & a_2 \\
a_3 & a_4
\end{bmatrix} \\ 
 X &= \begin{bmatrix}
x_1 & x_2 \\
x_3 & x_4
\end{bmatrix} \\
 C &=  \begin{bmatrix}
c_1 & c_2 \\
c_3 & c_4
\end{bmatrix}
\end{align*}
Where $C=AX$, thus-
\begin{align*}
c_1&=f(x_1, x_3) \\
c_2&=f(x_2, x_4) \\
c_3&=f(x_1, x_3) \\
c_4&=f(x_2, x_4) \\
\end{align*}
Therefore-
\begin{align*}
\frac{\partial \mathfrak{L}}{\partial x_1} &= \frac{\partial \mathfrak{L}}{\partial c_1} \times \frac{\partial c_1}{\partial x_1} + \frac{\partial \mathfrak{L}}{\partial c_2} \times \frac{\partial c_2}{\partial x_1} + \frac{\partial \mathfrak{L}}{\partial c_3} \times \frac{\partial c_3}{\partial x_1} + \frac{\partial \mathfrak{L}}{\partial c_4} \times \frac{\partial c_4}{\partial x_1}\\
&= \frac{\partial \mathfrak{L}}{\partial c_1} \times \frac{\partial \mathfrak{c_1}}{\partial x_1} + 0 + \frac{\partial \mathfrak{L}}{\partial c_3} \times \frac{\partial \mathfrak{c_3}}{\partial x_1} + 0\\
&= a_1 \times \frac{\partial \mathfrak{L}}{\partial c_1} + a_3 \times \frac{\partial \mathfrak{L}}{\partial c_3}\\
\end{align*}

Similarly-
\begin{align*}
\frac{\partial \mathfrak{L}}{\partial x_2}
&= a_1 \times \frac{\partial \mathfrak{L}}{\partial c_2} + a_3 \times \frac{\partial \mathfrak{L}}{\partial c_4}\\
\frac{\partial \mathfrak{L}}{\partial x_3}
&= a_2 \times \frac{\partial \mathfrak{L}}{\partial c_1} + a_4 \times \frac{\partial \mathfrak{L}}{\partial c_3}\\
\frac{\partial \mathfrak{L}}{\partial x_4}
&= a_2 \times \frac{\partial \mathfrak{L}}{\partial c_2} + a_4 \times \frac{\partial \mathfrak{L}}{\partial c_4}\\
\end{align*}
This can be expressed as a matrix equation in the following manner-
\begin{align*}
\frac{\partial \mathfrak{L}}{\partial X} &= \frac{\partial C}{\partial X} \times \frac{\partial \mathfrak{L}}{\partial C}\\
\implies \begin{bmatrix}
\frac{\partial \mathfrak{L}}{\partial x_1} & \frac{\partial \mathfrak{L}}{\partial x_2} \\
\frac{\partial \mathfrak{L}}{\partial x_3} & \frac{\partial \mathfrak{L}}{\partial x_4}
\end{bmatrix} &= \frac{\partial C}{\partial X} \times \begin{bmatrix}
\frac{\partial \mathfrak{L}}{\partial c_1} & \frac{\partial \mathfrak{L}}{\partial c_2} \\
\frac{\partial \mathfrak{L}}{\partial c_3} & \frac{\partial \mathfrak{L}}{\partial c_4}
\end{bmatrix}
\end{align*}
By comparing the individual partial derivatives computed earlier, we can say-

\begin{align*}
\begin{bmatrix}
\frac{\partial \mathfrak{L}}{\partial x_1} & \frac{\partial \mathfrak{L}}{\partial x_2} \\
\frac{\partial \mathfrak{L}}{\partial x_3} & \frac{\partial \mathfrak{L}}{\partial x_4}
\end{bmatrix} &= \begin{bmatrix}
a_1 & a_3 \\
a_2 & a_4
\end{bmatrix} \times \begin{bmatrix}
\frac{\partial \mathfrak{L}}{\partial c_1} & \frac{\partial \mathfrak{L}}{\partial c_2} \\
\frac{\partial \mathfrak{L}}{\partial c_3} & \frac{\partial \mathfrak{L}}{\partial c_4}
\end{bmatrix}\\
\implies \begin{bmatrix}
\frac{\partial \mathfrak{L}}{\partial x_1} & \frac{\partial \mathfrak{L}}{\partial x_2} \\
\frac{\partial \mathfrak{L}}{\partial x_3} & \frac{\partial \mathfrak{L}}{\partial x_4}
\end{bmatrix} &= A^T \times \begin{bmatrix}
\frac{\partial \mathfrak{L}}{\partial c_1} & \frac{\partial \mathfrak{L}}{\partial c_2} \\
\frac{\partial \mathfrak{L}}{\partial c_3} & \frac{\partial \mathfrak{L}}{\partial c_4}
\end{bmatrix}\\
\implies \frac{\partial \mathfrak{L}}{\partial X} &= A^T \times \frac{\partial \mathfrak{L}}{\partial C}\\
\therefore \frac{\partial C}{\partial X} &= A^T \qed
\end{align*}


\bibliography{xamba}
\bibliographystyle{iclr2025_conference}

\appendix


\newpage
\appendix
\section{Applicability of SparseTransX for dense graphs} 
\label{A:density}
Even for fully dense graphs, our KGE computations remain highly sparse. This is because our SpMM leverages the incidence matrix for triplets, rather than the graph's adjacency matrix. In the paper, the sparse matrix $A \in \{-1,0,1\}^{M \times (N+R)}$ represents the triplets, where $N$ is the number of entities, $R$ is the number of relations, and $M$ is the number of triplets. This representation remains extremely sparse, as each row contains exactly three non-zero values (or two in the case of the "ht" representation). Hence, the sparsity of this formulation is independent of the graph's structure, ensuring computational efficiency even for dense graphs.

\section{Computational Complexity}
\label{A:complexity}
 For a sparse matrix $A$ with $m \times k$ having $nnz(A)=$ number of non zeros and dense matrix $X$ with $k \times n$ dimension, the computational complexity of the SpMM is $O(nnz(A) \cdot n)$ since there are a total of $nnz(A)$ number of dot products each involving $n$ components. Since our sparse matrix contains exactly three non-zeros in each row, $nnz(A) = 3m$. Therefore, the complexity of SpMM is $O(3m \cdot n)$ or $O(m \cdot n)$, meaning the complexity increases when triplet counts or embedding dimension is increased. Memory access pattern will change when the number of entities is increased and it will affect the runtime, but the algorithmic complexity will not be affected by the number of entities/relations.

\section{Applicability to Non-translational Models}
\label{A:non_trans}
Our paper focused on translational models using sparse operations, but the concept extends broadly to various other knowledge graph embedding (KGE) methods. Neural network-based models, which are inherently matrix-multiplication-based, can be seamlessly integrated into this framework. Additionally, models such as DistMult, ComplEx, and RotatE can be implemented with simple modifications to the SpMM operations. Implementing these KGE models requires modifying the addition and multiplication operators in SpMM, effectively changing the semiring that governs the multiplication.   

In the paper, the sparse matrix $A \in \{-1,0,1\}^{M \times (N+R)}$ represents the triplets, and the dense matrix $E \in \mathbb{R}^{(N+R) \times d}$ represents the embedding matrix, where $N$ is the number of entities, $R$ is the number of relations, and $M$ is the number of triplets. TransE’s score function, defined as $h + r - t$, is computed by multiplying $A$ and $E$ using an SpMM followed by the L2 norm. This operation can be generalized using a semiring-based SpMM model: $Z_{ij} = \bigoplus_{k=1}^{n} (A_{ik} \otimes E_{kj})$

Here, $\oplus$ represents the semiring addition operator, and $\otimes$ represents the semiring multiplication operator. For TransE, these operators correspond to standard arithmetic addition and multiplication, respectively.

\subsection*{DistMult} 
DistMult’s score function has the expression $h \odot r \odot t$. To adapt SpMM for this model, two key adjustments are required: The sparse matrix $A$ stores $+1$ at the positions corresponding to $h_{\text{idx}}$, $t_{\text{idx}}$, and $r_{\text{idx}}$. Both the semiring addition and multiplication operators are set to arithmetic multiplication. These changes enable the use of SpMM for the DistMult score function.

\subsection*{ComplEx} 
ComplEx’s score function has $h \odot r \odot \bar{t}$, where embeddings are stored as complex numbers (e.g., using PyTorch). In this case, the semiring operations are similar to DistMult, but with complex number multiplication replacing real number multiplication.

\subsection*{RotatE} 
RotatE’s score function has $h \odot r - t$. For this model, the semiring requires both arithmetic multiplication and subtraction for $\oplus$. With minor modifications to our SpMM implementation, the semiring addition operator can be adapted to compute $h \odot r - t$.

\subsection*{Support from other libraries}
Many existing libraries, such as GraphBLAS (Kimmerer, Raye, et al., 2024), Ginkgo (Anzt, Hartwig, et al., 2022), and Gunrock (Wang, Yangzihao, et al., 2017), already support custom semirings in SpMM. We can leverage C++ templates to extend support for KGE models with minimal effort.


\begin{figure*}[t]
\centering     %%% not \center
\includegraphics[width=\textwidth]{figures/all-eval.pdf}
\caption{Loss curve for sparse and non-sparse approach. Sparse approach eventually reaches the same loss value with similar Hits@10 test accuracy.}
\label{fig:loss_curve}
\end{figure*}

\section{Model Performance Evaluation and Convergence}
\label{A:eval}
SpTransX follows a slightly different loss curve (see Figure \ref{fig:loss_curve}) and eventually converges with the same loss as other non-sparse implementations such as TorchKGE. We test SpTransX with the WN18 dataset having embedding size 512 (128 for TransR and TransH due to memory limitation) and run 200-1000 epochs. We compute average Hits@10 of 9 runs with different initial seeds and a learning rate scheduler. The results are shown below. We find that Hits@10 is generally comparable to or better than the Hits@10 achieved by TorchKGE.

\begin{table}[h]
\centering
\caption{Average of 9 Hits@10 Accuracy for WN18 dataset}
\begin{tabular}{|c|c|c|}
\hline
\textbf{Model} & \textbf{TorchKGE} & \textbf{SpTransX} \\ \hline
TransE         & 0.79 ± 0.001700   & 0.79 ± 0.002667   \\ \hline
TransR         & 0.29 ± 0.005735   & 0.33 ± 0.006154   \\ \hline
TransH         & 0.76 ± 0.012285   & 0.79 ± 0.001832   \\ \hline
TorusE         & 0.73 ± 0.003258   & 0.73 ± 0.002780   \\ \hline
\end{tabular}
\label{table:perf_eval}
\end{table}

% We also plot the loss curve for different models in Figure \ref{fig:loss_curve}. We observe that the sparse approach follows a similar loss curve and eventually converges to the same final loss.

\section{Distributed SpTransX and Its Applicability to Large KGs}
\label{A:dist}
SpTransX framework includes several features to support distributed KGE training across multi-CPU, multi-GPU, and multi-node setups. Additionally, it incorporates modules for model and dataset streaming to handle massive datasets efficiently. 

Distributed SpTransX relies on PyTorch Distributed Data Parallel (DDP) and Fully Sharded Data Parallel (FSDP) support to distribute sparse computations across multiple GPUs. 

\begin{table}[h]
\centering
\caption{Average Time of 15 Epochs (seconds). Training time of TransE model with Freebase dataset (250M triplets, 77M entities. 74K relations, batch size 393K)  on 32 NVIDIA A100 GPUs. FSDP enables model training with larger embedding when DDP fails.}
\begin{tabular}{|p{2cm}|p{2.5cm}|p{2.5cm}|}
\hline
\textbf{Embedding Size} & \textbf{DDP (Distributed Data Parallel)} & \textbf{FSDP (Fully Sharded Data Parallel)} \\ \hline
16                      & 65.07 ± 1.641                            & 63.35 ± 1.258                               \\ \hline
20                      & Out of Memory                            & 96.44 ± 1.490                               \\ \hline
\end{tabular}
\end{table}

We run an experiment with a large-scale KG to showcase the performance of distributed SpTransX. Freebase (250M triplets, 77M entities. 74K relations, batch size 393K) dataset is trained using the TransE model on 32 NVIDIA A100 GPUs of NERSC using various distributed settings. SpTransX’s Streaming dataset module allows fetching only the necessary batch from the dataset and enables memory-efficient training. FSDP enables model training with larger embedding when DDP fails.

\section{Scaling and Communication Bottlenecks for Large KG Training}
\label{A:scaling}
Communication can be a significant bottleneck in distributed KGE training when using SpMM. However, by leveraging Distributed Data-Parallel (DDP) in PyTorch, we successfully scale distributed SpTransX to 64 NVIDIA A100 GPUs with reasonable efficiency. The training time for the COVID-19 dataset with 60,820 entities, 62 relations, and 1,032,939 triplets is in Table \ref{table:scaling}. 
% \vspace{-.3cm}
\begin{table}[h]
\centering
\caption{Scaling TransE model on COVID-19 dataset}
\begin{tabular}{|c|c|}
\hline
\textbf{Number of GPUs} & \textbf{500 epoch time (seconds)} \\ \hline
4                       & 706.38                            \\ \hline
8                       & 586.03                            \\ \hline
16                      & 340.00                               \\ \hline
32                      & 246.02                            \\ \hline
64                      & 179.95                            \\ \hline
\end{tabular}
\label{table:scaling}
\end{table}
% \vspace{-.2cm}
It indicates that communication is not a bottleneck up to 64 GPUs. If communication becomes a performance bottleneck at larger scales, we plan to explore alternative communication-reducing algorithms, including 2D and 3D matrix distribution techniques, which are known to minimize communication overhead at extreme scales. Additionally, we will incorporate model parallelism alongside data parallelism for large-scale knowledge graphs.

\section{Backpropagation of SpMM}
\label{A:backprop}
 Our main computational kernel is the sparse-dense matrix multiplication (SpMM). The computation of backpropagation of an SpMM w.r.t. the dense matrix is also another SpMM. To see how, let's consider the sparse-dense matrix multiplication $AX = C$ which is part of the training process. As long as the computational graph reduces to a single scaler loss $\mathfrak{L}$, it can be shown that $\frac{\partial C}{\partial X} = A^T$. Here, $X$ is the learnable parameter (embeddings), and $A$ is the sparse matrix. Since $A^T$ is also a sparse matrix and $\frac{\partial \mathfrak{L}}{\partial C}$ is a dense matrix, the computation $\frac{\partial \mathfrak{L}}{\partial X} = \frac{\partial C}{\partial X} \times \frac{\partial \mathfrak{L}}{\partial C} = A^T \times \frac{\partial \mathfrak{L}}{\partial C} $ is an SpMM. This means that both forward and backward propagation of our approach benefit from the efficiency of a high-performance SpMM.

\subsection*{Proof that $\frac{\partial C}{\partial X} = A^T$}
 To see why $\frac{\partial C}{\partial X} = A^T$ is used in the gradient calculation, we can consider the following small matrix multiplication without loss of generality.
\begin{align*}
A &= \begin{bmatrix}
a_1 & a_2 \\
a_3 & a_4
\end{bmatrix} \\ 
 X &= \begin{bmatrix}
x_1 & x_2 \\
x_3 & x_4
\end{bmatrix} \\
 C &=  \begin{bmatrix}
c_1 & c_2 \\
c_3 & c_4
\end{bmatrix}
\end{align*}
Where $C=AX$, thus-
\begin{align*}
c_1&=f(x_1, x_3) \\
c_2&=f(x_2, x_4) \\
c_3&=f(x_1, x_3) \\
c_4&=f(x_2, x_4) \\
\end{align*}
Therefore-
\begin{align*}
\frac{\partial \mathfrak{L}}{\partial x_1} &= \frac{\partial \mathfrak{L}}{\partial c_1} \times \frac{\partial c_1}{\partial x_1} + \frac{\partial \mathfrak{L}}{\partial c_2} \times \frac{\partial c_2}{\partial x_1} + \frac{\partial \mathfrak{L}}{\partial c_3} \times \frac{\partial c_3}{\partial x_1} + \frac{\partial \mathfrak{L}}{\partial c_4} \times \frac{\partial c_4}{\partial x_1}\\
&= \frac{\partial \mathfrak{L}}{\partial c_1} \times \frac{\partial \mathfrak{c_1}}{\partial x_1} + 0 + \frac{\partial \mathfrak{L}}{\partial c_3} \times \frac{\partial \mathfrak{c_3}}{\partial x_1} + 0\\
&= a_1 \times \frac{\partial \mathfrak{L}}{\partial c_1} + a_3 \times \frac{\partial \mathfrak{L}}{\partial c_3}\\
\end{align*}

Similarly-
\begin{align*}
\frac{\partial \mathfrak{L}}{\partial x_2}
&= a_1 \times \frac{\partial \mathfrak{L}}{\partial c_2} + a_3 \times \frac{\partial \mathfrak{L}}{\partial c_4}\\
\frac{\partial \mathfrak{L}}{\partial x_3}
&= a_2 \times \frac{\partial \mathfrak{L}}{\partial c_1} + a_4 \times \frac{\partial \mathfrak{L}}{\partial c_3}\\
\frac{\partial \mathfrak{L}}{\partial x_4}
&= a_2 \times \frac{\partial \mathfrak{L}}{\partial c_2} + a_4 \times \frac{\partial \mathfrak{L}}{\partial c_4}\\
\end{align*}
This can be expressed as a matrix equation in the following manner-
\begin{align*}
\frac{\partial \mathfrak{L}}{\partial X} &= \frac{\partial C}{\partial X} \times \frac{\partial \mathfrak{L}}{\partial C}\\
\implies \begin{bmatrix}
\frac{\partial \mathfrak{L}}{\partial x_1} & \frac{\partial \mathfrak{L}}{\partial x_2} \\
\frac{\partial \mathfrak{L}}{\partial x_3} & \frac{\partial \mathfrak{L}}{\partial x_4}
\end{bmatrix} &= \frac{\partial C}{\partial X} \times \begin{bmatrix}
\frac{\partial \mathfrak{L}}{\partial c_1} & \frac{\partial \mathfrak{L}}{\partial c_2} \\
\frac{\partial \mathfrak{L}}{\partial c_3} & \frac{\partial \mathfrak{L}}{\partial c_4}
\end{bmatrix}
\end{align*}
By comparing the individual partial derivatives computed earlier, we can say-

\begin{align*}
\begin{bmatrix}
\frac{\partial \mathfrak{L}}{\partial x_1} & \frac{\partial \mathfrak{L}}{\partial x_2} \\
\frac{\partial \mathfrak{L}}{\partial x_3} & \frac{\partial \mathfrak{L}}{\partial x_4}
\end{bmatrix} &= \begin{bmatrix}
a_1 & a_3 \\
a_2 & a_4
\end{bmatrix} \times \begin{bmatrix}
\frac{\partial \mathfrak{L}}{\partial c_1} & \frac{\partial \mathfrak{L}}{\partial c_2} \\
\frac{\partial \mathfrak{L}}{\partial c_3} & \frac{\partial \mathfrak{L}}{\partial c_4}
\end{bmatrix}\\
\implies \begin{bmatrix}
\frac{\partial \mathfrak{L}}{\partial x_1} & \frac{\partial \mathfrak{L}}{\partial x_2} \\
\frac{\partial \mathfrak{L}}{\partial x_3} & \frac{\partial \mathfrak{L}}{\partial x_4}
\end{bmatrix} &= A^T \times \begin{bmatrix}
\frac{\partial \mathfrak{L}}{\partial c_1} & \frac{\partial \mathfrak{L}}{\partial c_2} \\
\frac{\partial \mathfrak{L}}{\partial c_3} & \frac{\partial \mathfrak{L}}{\partial c_4}
\end{bmatrix}\\
\implies \frac{\partial \mathfrak{L}}{\partial X} &= A^T \times \frac{\partial \mathfrak{L}}{\partial C}\\
\therefore \frac{\partial C}{\partial X} &= A^T \qed
\end{align*}

% You may include other additional sections here.

% \section{Submission of conference papers to ICLR 2025}

% ICLR requires electronic submissions, processed by
% \url{https://openreview.net/}. See ICLR's website for more instructions.

% If your paper is ultimately accepted, the statement {\tt
%   {\textbackslash}iclrfinalcopy} should be inserted to adjust the
% format to the camera ready requirements.

% The format for the submissions is a variant of the NeurIPS format.
% Please read carefully the instructions below, and follow them
% faithfully.

% \subsection{Style}

% Papers to be submitted to ICLR 2025 must be prepared according to the
% instructions presented here.

% %% Please note that we have introduced automatic line number generation
% %% into the style file for \LaTeXe. This is to help reviewers
% %% refer to specific lines of the paper when they make their comments. Please do
% %% NOT refer to these line numbers in your paper as they will be removed from the
% %% style file for the final version of accepted papers.

% Authors are required to use the ICLR \LaTeX{} style files obtainable at the
% ICLR website. Please make sure you use the current files and
% not previous versions. Tweaking the style files may be grounds for rejection.

% \subsection{Retrieval of style files}

% The style files for ICLR and other conference information are available online at:
% \begin{center}
%    \url{http://www.iclr.cc/}
% \end{center}
% The file \verb+iclr2025_conference.pdf+ contains these
% instructions and illustrates the
% various formatting requirements your ICLR paper must satisfy.
% Submissions must be made using \LaTeX{} and the style files
% \verb+iclr2025_conference.sty+ and \verb+iclr2025_conference.bst+ (to be used with \LaTeX{}2e). The file
% \verb+iclr2025_conference.tex+ may be used as a ``shell'' for writing your paper. All you
% have to do is replace the author, title, abstract, and text of the paper with
% your own.

% The formatting instructions contained in these style files are summarized in
% sections \ref{gen_inst}, \ref{headings}, and \ref{others} below.

% \section{General formatting instructions}
% \label{gen_inst}

% The text must be confined within a rectangle 5.5~inches (33~picas) wide and
% 9~inches (54~picas) long. The left margin is 1.5~inch (9~picas).
% Use 10~point type with a vertical spacing of 11~points. Times New Roman is the
% preferred typeface throughout. Paragraphs are separated by 1/2~line space,
% with no indentation.

% Paper title is 17~point, in small caps and left-aligned.
% All pages should start at 1~inch (6~picas) from the top of the page.

% Authors' names are
% set in boldface, and each name is placed above its corresponding
% address. The lead author's name is to be listed first, and
% the co-authors' names are set to follow. Authors sharing the
% same address can be on the same line.

% Please pay special attention to the instructions in section \ref{others}
% regarding figures, tables, acknowledgments, and references.


% There will be a strict upper limit of 10 pages for the main text of the initial submission, with unlimited additional pages for citations. 

% \section{Headings: first level}
% \label{headings}

% First level headings are in small caps,
% flush left and in point size 12. One line space before the first level
% heading and 1/2~line space after the first level heading.

% \subsection{Headings: second level}

% Second level headings are in small caps,
% flush left and in point size 10. One line space before the second level
% heading and 1/2~line space after the second level heading.

% \subsubsection{Headings: third level}

% Third level headings are in small caps,
% flush left and in point size 10. One line space before the third level
% heading and 1/2~line space after the third level heading.

% \section{Citations, figures, tables, references}
% \label{others}

% These instructions apply to everyone, regardless of the formatter being used.

% \subsection{Citations within the text}

% Citations within the text should be based on the \texttt{natbib} package
% and include the authors' last names and year (with the ``et~al.'' construct
% for more than two authors). When the authors or the publication are
% included in the sentence, the citation should not be in parenthesis using \verb|\citet{}| (as
% in ``See \citet{Hinton06} for more information.''). Otherwise, the citation
% should be in parenthesis using \verb|\citep{}| (as in ``Deep learning shows promise to make progress
% towards AI~\citep{Bengio+chapter2007}.'').

% The corresponding references are to be listed in alphabetical order of
% authors, in the \textsc{References} section. As to the format of the
% references themselves, any style is acceptable as long as it is used
% consistently.

% \subsection{Footnotes}

% Indicate footnotes with a number\footnote{Sample of the first footnote} in the
% text. Place the footnotes at the bottom of the page on which they appear.
% Precede the footnote with a horizontal rule of 2~inches
% (12~picas).\footnote{Sample of the second footnote}

% \subsection{Figures}

% All artwork must be neat, clean, and legible. Lines should be dark
% enough for purposes of reproduction; art work should not be
% hand-drawn. The figure number and caption always appear after the
% figure. Place one line space before the figure caption, and one line
% space after the figure. The figure caption is lower case (except for
% first word and proper nouns); figures are numbered consecutively.

% Make sure the figure caption does not get separated from the figure.
% Leave sufficient space to avoid splitting the figure and figure caption.

% You may use color figures.
% However, it is best for the
% figure captions and the paper body to make sense if the paper is printed
% either in black/white or in color.
% \begin{figure}[h]
% \begin{center}
% %\framebox[4.0in]{$\;$}
% \fbox{\rule[-.5cm]{0cm}{4cm} \rule[-.5cm]{4cm}{0cm}}
% \end{center}
% \caption{Sample figure caption.}
% \end{figure}

% \subsection{Tables}

% All tables must be centered, neat, clean and legible. Do not use hand-drawn
% tables. The table number and title always appear before the table. See
% Table~\ref{sample-table}.

% Place one line space before the table title, one line space after the table
% title, and one line space after the table. The table title must be lower case
% (except for first word and proper nouns); tables are numbered consecutively.

% \begin{table}[t]
% \caption{Sample table title}
% \label{sample-table}
% \begin{center}
% \begin{tabular}{ll}
% \multicolumn{1}{c}{\bf PART}  &\multicolumn{1}{c}{\bf DESCRIPTION}
% \\ \hline \\
% Dendrite         &Input terminal \\
% Axon             &Output terminal \\
% Soma             &Cell body (contains cell nucleus) \\
% \end{tabular}
% \end{center}
% \end{table}

% \section{Default Notation}

% In an attempt to encourage standardized notation, we have included the
% notation file from the textbook, \textit{Deep Learning}
% \cite{goodfellow2016deep} available at
% \url{https://github.com/goodfeli/dlbook_notation/}.  Use of this style
% is not required and can be disabled by commenting out
% \texttt{math\_commands.tex}.


% \centerline{\bf Numbers and Arrays}
% \bgroup
% \def\arraystretch{1.5}
% \begin{tabular}{p{1in}p{3.25in}}
% $\displaystyle a$ & A scalar (integer or real)\\
% $\displaystyle \va$ & A vector\\
% $\displaystyle \mA$ & A matrix\\
% $\displaystyle \tA$ & A tensor\\
% $\displaystyle \mI_n$ & Identity matrix with $n$ rows and $n$ columns\\
% $\displaystyle \mI$ & Identity matrix with dimensionality implied by context\\
% $\displaystyle \ve^{(i)}$ & Standard basis vector $[0,\dots,0,1,0,\dots,0]$ with a 1 at position $i$\\
% $\displaystyle \text{diag}(\va)$ & A square, diagonal matrix with diagonal entries given by $\va$\\
% $\displaystyle \ra$ & A scalar random variable\\
% $\displaystyle \rva$ & A vector-valued random variable\\
% $\displaystyle \rmA$ & A matrix-valued random variable\\
% \end{tabular}
% \egroup
% \vspace{0.25cm}

% \centerline{\bf Sets and Graphs}
% \bgroup
% \def\arraystretch{1.5}

% \begin{tabular}{p{1.25in}p{3.25in}}
% $\displaystyle \sA$ & A set\\
% $\displaystyle \R$ & The set of real numbers \\
% $\displaystyle \{0, 1\}$ & The set containing 0 and 1 \\
% $\displaystyle \{0, 1, \dots, n \}$ & The set of all integers between $0$ and $n$\\
% $\displaystyle [a, b]$ & The real interval including $a$ and $b$\\
% $\displaystyle (a, b]$ & The real interval excluding $a$ but including $b$\\
% $\displaystyle \sA \backslash \sB$ & Set subtraction, i.e., the set containing the elements of $\sA$ that are not in $\sB$\\
% $\displaystyle \gG$ & A graph\\
% $\displaystyle \parents_\gG(\ervx_i)$ & The parents of $\ervx_i$ in $\gG$
% \end{tabular}
% \vspace{0.25cm}


% \centerline{\bf Indexing}
% \bgroup
% \def\arraystretch{1.5}

% \begin{tabular}{p{1.25in}p{3.25in}}
% $\displaystyle \eva_i$ & Element $i$ of vector $\va$, with indexing starting at 1 \\
% $\displaystyle \eva_{-i}$ & All elements of vector $\va$ except for element $i$ \\
% $\displaystyle \emA_{i,j}$ & Element $i, j$ of matrix $\mA$ \\
% $\displaystyle \mA_{i, :}$ & Row $i$ of matrix $\mA$ \\
% $\displaystyle \mA_{:, i}$ & Column $i$ of matrix $\mA$ \\
% $\displaystyle \etA_{i, j, k}$ & Element $(i, j, k)$ of a 3-D tensor $\tA$\\
% $\displaystyle \tA_{:, :, i}$ & 2-D slice of a 3-D tensor\\
% $\displaystyle \erva_i$ & Element $i$ of the random vector $\rva$ \\
% \end{tabular}
% \egroup
% \vspace{0.25cm}


% \centerline{\bf Calculus}
% \bgroup
% \def\arraystretch{1.5}
% \begin{tabular}{p{1.25in}p{3.25in}}
% % NOTE: the [2ex] on the next line adds extra height to that row of the table.
% % Without that command, the fraction on the first line is too tall and collides
% % with the fraction on the second line.
% $\displaystyle\frac{d y} {d x}$ & Derivative of $y$ with respect to $x$\\ [2ex]
% $\displaystyle \frac{\partial y} {\partial x} $ & Partial derivative of $y$ with respect to $x$ \\
% $\displaystyle \nabla_\vx y $ & Gradient of $y$ with respect to $\vx$ \\
% $\displaystyle \nabla_\mX y $ & Matrix derivatives of $y$ with respect to $\mX$ \\
% $\displaystyle \nabla_\tX y $ & Tensor containing derivatives of $y$ with respect to $\tX$ \\
% $\displaystyle \frac{\partial f}{\partial \vx} $ & Jacobian matrix $\mJ \in \R^{m\times n}$ of $f: \R^n \rightarrow \R^m$\\
% $\displaystyle \nabla_\vx^2 f(\vx)\text{ or }\mH( f)(\vx)$ & The Hessian matrix of $f$ at input point $\vx$\\
% $\displaystyle \int f(\vx) d\vx $ & Definite integral over the entire domain of $\vx$ \\
% $\displaystyle \int_\sS f(\vx) d\vx$ & Definite integral with respect to $\vx$ over the set $\sS$ \\
% \end{tabular}
% \egroup
% \vspace{0.25cm}

% \centerline{\bf Probability and Information Theory}
% \bgroup
% \def\arraystretch{1.5}
% \begin{tabular}{p{1.25in}p{3.25in}}
% $\displaystyle P(\ra)$ & A probability distribution over a discrete variable\\
% $\displaystyle p(\ra)$ & A probability distribution over a continuous variable, or over
% a variable whose type has not been specified\\
% $\displaystyle \ra \sim P$ & Random variable $\ra$ has distribution $P$\\% so thing on left of \sim should always be a random variable, with name beginning with \r
% $\displaystyle  \E_{\rx\sim P} [ f(x) ]\text{ or } \E f(x)$ & Expectation of $f(x)$ with respect to $P(\rx)$ \\
% $\displaystyle \Var(f(x)) $ &  Variance of $f(x)$ under $P(\rx)$ \\
% $\displaystyle \Cov(f(x),g(x)) $ & Covariance of $f(x)$ and $g(x)$ under $P(\rx)$\\
% $\displaystyle H(\rx) $ & Shannon entropy of the random variable $\rx$\\
% $\displaystyle \KL ( P \Vert Q ) $ & Kullback-Leibler divergence of P and Q \\
% $\displaystyle \mathcal{N} ( \vx ; \vmu , \mSigma)$ & Gaussian distribution %
% over $\vx$ with mean $\vmu$ and covariance $\mSigma$ \\
% \end{tabular}
% \egroup
% \vspace{0.25cm}

% \centerline{\bf Functions}
% \bgroup
% \def\arraystretch{1.5}
% \begin{tabular}{p{1.25in}p{3.25in}}
% $\displaystyle f: \sA \rightarrow \sB$ & The function $f$ with domain $\sA$ and range $\sB$\\
% $\displaystyle f \circ g $ & Composition of the functions $f$ and $g$ \\
%   $\displaystyle f(\vx ; \vtheta) $ & A function of $\vx$ parametrized by $\vtheta$.
%   (Sometimes we write $f(\vx)$ and omit the argument $\vtheta$ to lighten notation) \\
% $\displaystyle \log x$ & Natural logarithm of $x$ \\
% $\displaystyle \sigma(x)$ & Logistic sigmoid, $\displaystyle \frac{1} {1 + \exp(-x)}$ \\
% $\displaystyle \zeta(x)$ & Softplus, $\log(1 + \exp(x))$ \\
% $\displaystyle || \vx ||_p $ & $\normlp$ norm of $\vx$ \\
% $\displaystyle || \vx || $ & $\normltwo$ norm of $\vx$ \\
% $\displaystyle x^+$ & Positive part of $x$, i.e., $\max(0,x)$\\
% $\displaystyle \1_\mathrm{condition}$ & is 1 if the condition is true, 0 otherwise\\
% \end{tabular}
% \egroup
% \vspace{0.25cm}



% \section{Final instructions}
% Do not change any aspects of the formatting parameters in the style files.
% In particular, do not modify the width or length of the rectangle the text
% should fit into, and do not change font sizes (except perhaps in the
% \textsc{References} section; see below). Please note that pages should be
% numbered.

% \section{Preparing PostScript or PDF files}

% Please prepare PostScript or PDF files with paper size ``US Letter'', and
% not, for example, ``A4''. The -t
% letter option on dvips will produce US Letter files.

% Consider directly generating PDF files using \verb+pdflatex+
% (especially if you are a MiKTeX user).
% PDF figures must be substituted for EPS figures, however.

% Otherwise, please generate your PostScript and PDF files with the following commands:
% \begin{verbatim}
% dvips mypaper.dvi -t letter -Ppdf -G0 -o mypaper.ps
% ps2pdf mypaper.ps mypaper.pdf
% \end{verbatim}

% \subsection{Margins in LaTeX}

% Most of the margin problems come from figures positioned by hand using
% \verb+\special+ or other commands. We suggest using the command
% \verb+\includegraphics+
% from the graphicx package. Always specify the figure width as a multiple of
% the line width as in the example below using .eps graphics
% \begin{verbatim}
%    \usepackage[dvips]{graphicx} ...
%    \includegraphics[width=0.8\linewidth]{myfile.eps}
% \end{verbatim}
% or % Apr 2009 addition
% \begin{verbatim}
%    \usepackage[pdftex]{graphicx} ...
%    \includegraphics[width=0.8\linewidth]{myfile.pdf}
% \end{verbatim}
% for .pdf graphics.
% See section~4.4 in the graphics bundle documentation (\url{http://www.ctan.org/tex-archive/macros/latex/required/graphics/grfguide.ps})

% A number of width problems arise when LaTeX cannot properly hyphenate a
% line. Please give LaTeX hyphenation hints using the \verb+\-+ command.

% \subsubsection*{Author Contributions}
% If you'd like to, you may include  a section for author contributions as is done
% in many journals. This is optional and at the discretion of the authors.

% \subsubsection*{Acknowledgments}
% Use unnumbered third level headings for the acknowledgments. All
% acknowledgments, including those to funding agencies, go at the end of the paper.





\end{document}
