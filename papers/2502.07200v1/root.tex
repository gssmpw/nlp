%%%%%%%%%%%%%%%%%%%%%%%%%%%%%%%%%%%%%%%%%%%%%%%%%%%%%%%%%%%%%%%%%%%%%%%%%%%%%%%%
%2345678901234567890123456789012345678901234567890123456789012345678901234567890
%        1         2         3         4         5         6         7         8

\documentclass[letterpaper, 10 pt, conference]{ieeeconf}  % Comment this line out if you need a4paper

%\documentclass[a4paper, 10pt, conference]{ieeeconf}      % Use this line for a4 paper

\IEEEoverridecommandlockouts                              % This command is only needed if 
                                                          % you want to use the \thanks command

\overrideIEEEmargins                                      % Needed to meet printer requirements.

%In case you encounter the following error:
%Error 1010 The PDF file may be corrupt (unable to open PDF file) OR
%Error 1000 An error occurred while parsing a contents stream. Unable to analyze the PDF file.
%This is a known problem with pdfLaTeX conversion filter. The file cannot be opened with acrobat reader
%Please use one of the alternatives below to circumvent this error by uncommenting one or the other
%\pdfobjcompresslevel=0
%\pdfminorversion=4

% See the \addtolength command later in the file to balance the column lengths
% on the last page of the document

% The following packages can be found on http:\\www.ctan.org
%\usepackage{graphics} % for pdf, bitmapped graphics files
%\usepackage{epsfig} % for postscript graphics files
%\usepackage{mathptmx} % assumes new font selection scheme installed
%\usepackage{times} % assumes new font selection scheme installed
%\usepackage{amsmath} % assumes amsmath package installed
%\usepackage{amssymb}  % assumes amsmath package installed
\usepackage{graphicx}
\usepackage{dirtytalk}
\usepackage{etoolbox}
\usepackage{booktabs}
\usepackage{multirow}

\usepackage[mathscr]{euscript}

\usepackage[table,xcdraw]{xcolor}

\usepackage{hyperref}

\usepackage{amsmath}
\usepackage{amssymb}

\usepackage{algorithm}
\usepackage{algpseudocode}
\usepackage{enumerate}

\usepackage{threeparttable}
\usepackage{scalerel}
\usepackage{tikz}
\usepackage{cite}

\hyphenation{op-tical net-works semi-conduc-tor}

\title{\LARGE \bf
Color-Quality Invariance for Robust Medical Image Segmentation
}


\author{Ravi Shah, Atsushi Fukuda, and Quan Huu Cap
% <-this % stops a space
\thanks{All authors are with the AI Development Department, Aillis, Inc., Tokyo, Japan. Corresponding author: Quan Huu Cap ({\tt\small quan.cap@aillis.jp})}
}


\begin{document}



\maketitle
\thispagestyle{empty}
\pagestyle{empty}


%%%%%%%%%%%%%%%%%%%%%%%%%%%%%%%%%%%%%%%%%%%%%%%%%%%%%%%%%%%%%%%%%%%%
\begin{abstract}
    \begin{abstract}\label{00_Abstract}
Research in the field of automated vehicles, or more generally cognitive cyber-physical systems that operate in the real world, is leading to increasingly complex systems. Among other things, artificial intelligence enables an ever-increasing degree of autonomy. In this context, the V-model, which has served for decades as a process reference model of the system development lifecycle is reaching its limits. To the contrary, innovative processes and frameworks have been developed that take into account the characteristics of emerging autonomous systems. To bridge the gap and merge the different methodologies, we present an extension of the V-model for iterative data-based development processes that harmonizes and formalizes the existing methods towards a generic framework. The iterative approach allows for seamless integration of continuous system refinement. While the data-based approach constitutes the consideration of data-based development processes and formalizes the use of synthetic and real world data. In this way, formalizing the process of development, verification, validation, and continuous integration contributes to ensuring the safety of emerging complex systems that incorporate AI. 
\end{abstract}


\begin{IEEEkeywords}
	Process Reference Model, V-Model, Continuous Integration, AI Systems, Autonomy Technology, Safety Assurance
\end{IEEEkeywords}

% \newline

% \indent \textit{Clinical relevance}— This is a brief additional statement on why a this might be of interest to practicing clinicians. Example: This establishes the anesthetic efficacy of 10\% intraosseous injections with epinephrine to positively influence cardiovascular function.
\end{abstract}

% KEYWORDS
\begin{keywords}
Pharyngeal images, endoscopy images, medical image segmentation, semantic segmentation, deep learning.
\end{keywords}

%%%%%%%%%%%%%%%%%%%%%%%%%%%%%%%%%%%%%%%%%%%%%%%%%%%%%%%%%%%%%%%%%%%%%%%%%%%%%%%%
\section{INTRODUCTION}
    \section{Introduction}

% \textcolor{red}{Still on working}

% \textcolor{red}{add label for each section}


Robot learning relies on diverse and high-quality data to learn complex behaviors \cite{aldaco2024aloha, wang2024dexcap}.
Recent studies highlight that models trained on datasets with greater complexity and variation in the domain tend to generalize more effectively across broader scenarios \cite{mann2020language, radford2021learning, gao2024efficient}.
% However, creating such diverse datasets in the real world presents significant challenges.
% Modifying physical environments and adjusting robot hardware settings require considerable time, effort, and financial resources.
% In contrast, simulation environments offer a flexible and efficient alternative.
% Simulations allow for the creation and modification of digital environments with a wide range of object shapes, weights, materials, lighting, textures, friction coefficients, and so on to incorporate domain randomization,
% which helps improve the robustness of models when deployed in real-world conditions.
% These environments can be easily adjusted and reset, enabling faster iterations and data collection.
% Additionally, simulations provide the ability to consistently reproduce scenarios, which is essential for benchmarking and model evaluation.
% Another advantage of simulations is their flexibility in sensor integration. Sensors such as cameras, LiDARs, and tactile sensors can be added or repositioned without the physical limitations present in real-world setups. Simulations also eliminate the risk of damaging expensive hardware during edge-case experiments, making them an ideal platform for testing rare or dangerous scenarios that are impractical to explore in real life.
By leveraging immersive perspectives and interactions, Extended Reality\footnote{Extended Reality is an umbrella term to refer to Augmented Reality, Mixed Reality, and Virtual Reality \cite{wikipediaExtendedReality}}
(XR)
is a promising candidate for efficient and intuitive large scale data collection \cite{jiang2024comprehensive, arcade}
% With the demand for collecting data, XR provides a promising approach for humans to teach robots by offering users an immersive experience.
in simulation \cite{jiang2024comprehensive, arcade, dexhub-park} and real-world scenarios \cite{openteach, opentelevision}.
However, reusing and reproducing current XR approaches for robot data collection for new settings and scenarios is complicated and requires significant effort.
% are difficult to reuse and reproduce system makes it hard to reuse and reproduce in another data collection pipeline.
This bottleneck arises from three main limitations of current XR data collection and interaction frameworks: \textit{asset limitation}, \textit{simulator limitation}, and \textit{device limitation}.
% \textcolor{red}{ASSIGN THESE CITATION PROPERLY:}
% \textcolor{red}{list them by time order???}
% of collecting data by using XR have three main limitations.
Current approaches suffering from \textit{asset limitation} \cite{arclfd, jiang2024comprehensive, arcade, george2025openvr, vicarios}
% Firstly, recent works \cite{jiang2024comprehensive, arcade, dexhub-park}
can only use predefined robot models and task scenes. Configuring new tasks requires significant effort, since each new object or model must be specifically integrated into the XR application.
% and it takes too much effort to configure new tasks in their systems since they cannot spawn arbitrary models in the XR application.
The vast majority of application are developed for specific simulators or real-world scenarios. This \textit{simulator limitation} \cite{mosbach2022accelerating, lipton2017baxter, dexhub-park, arcade}
% Secondly, existing systems are limited to a single simulation platform or real-world scenarios.
significantly reduces reusability and makes adaptation to new simulation platforms challenging.
Additionally, most current XR frameworks are designed for a specific version of a single XR headset, leading to a \textit{device limitation} 
\cite{lipton2017baxter, armada, openteach, meng2023virtual}.
% and there is no work working on the extendability of transferring to a new headsets as far as we know.
To the best of our knowledge, no existing work has explored the extensibility or transferability of their framework to different headsets.
These limitations hamper reproducibility and broader contributions of XR based data collection and interaction to the research community.
% as each research group typically has its own data collection pipeline.
% In addition to these main limitations, existing XR systems are not well suited for managing multiple robot systems,
% as they are often designed for single-operator use.

In addition to these main limitations, existing XR systems are often designed for single-operator use, prohibiting collaborative data collection.
At the same time, controlling multiple robots at once can be very difficult for a single operator,
making data collection in multi-robot scenarios particularly challenging \cite{orun2019effect}.
Although there are some works using collaborative data collection in the context of tele-operation \cite{tung2021learning, Qin2023AnyTeleopAG},
there is no XR-based data collection system supporting collaborative data collection.
This limitation highlights the need for more advanced XR solutions that can better support multi-robot and multi-user scenarios.
% \textcolor{red}{more papers about collaborative data collection}

To address all of these issues, we propose \textbf{IRIS},
an \textbf{I}mmersive \textbf{R}obot \textbf{I}nteraction \textbf{S}ystem.
This general system supports various simulators, benchmarks and real-world scenarios.
It is easily extensible to new simulators and XR headsets.
IRIS achieves generalization across six dimensions:
% \begin{itemize}
%     \item \textit{Cross-scene} : diverse object models;
%     \item \textit{Cross-embodiment}: diverse robot models;
%     \item \textit{Cross-simulator}: 
%     \item \textit{Cross-reality}: fd
%     \item \textit{Cross-platform}: fd
%     \item \textit{Cross-users}: fd
% \end{itemize}
\textbf{Cross-Scene}, \textbf{Cross-Embodiment}, \textbf{Cross-Simulator}, \textbf{Cross-Reality}, \textbf{Cross-Platform}, and \textbf{Cross-User}.

\textbf{Cross-Scene} and \textbf{Cross-Embodiment} allow the system to handle arbitrary objects and robots in the simulation,
eliminating restrictions about predefined models in XR applications.
IRIS achieves these generalizations by introducing a unified scene specification, representing all objects,
including robots, as data structures with meshes, materials, and textures.
The unified scene specification is transmitted to the XR application to create and visualize an identical scene.
By treating robots as standard objects, the system simplifies XR integration,
allowing researchers to work with various robots without special robot-specific configurations.
\textbf{Cross-Simulator} ensures compatibility with various simulation engines.
IRIS simplifies adaptation by parsing simulated scenes into the unified scene specification, eliminating the need for XR application modifications when switching simulators.
New simulators can be integrated by creating a parser to convert their scenes into the unified format.
This flexibility is demonstrated by IRIS’ support for Mujoco \cite{todorov2012mujoco}, IsaacSim \cite{mittal2023orbit}, CoppeliaSim \cite{coppeliaSim}, and even the recent Genesis \cite{Genesis} simulator.
\textbf{Cross-Reality} enables the system to function seamlessly in both virtual simulations and real-world applications.
IRIS enables real-world data collection through camera-based point cloud visualization.
\textbf{Cross-Platform} allows for compatibility across various XR devices.
Since XR device APIs differ significantly, making a single codebase impractical, IRIS XR application decouples its modules to maximize code reuse.
This application, developed by Unity \cite{unity3dUnityManual}, separates scene visualization and interaction, allowing developers to integrate new headsets by reusing the visualization code and only implementing input handling for hand, head, and motion controller tracking.
IRIS provides an implementation of the XR application in the Unity framework, allowing for a straightforward deployment to any device that supports Unity. 
So far, IRIS was successfully deployed to the Meta Quest 3 and HoloLens 2.
Finally, the \textbf{Cross-User} ability allows multiple users to interact within a shared scene.
IRIS achieves this ability by introducing a protocol to establish the communication between multiple XR headsets and the simulation or real-world scenarios.
Additionally, IRIS leverages spatial anchors to support the alignment of virtual scenes from all deployed XR headsets.
% To make an seamless user experience for robot learning data collection,
% IRIS also tested in three different robot control interface
% Furthermore, to demonstrate the extensibility of our approach, we have implemented a robot-world pipeline for real robot data collection, ensuring that the system can be used in both simulated and real-world environments.
The Immersive Robot Interaction System makes the following contributions\\
\textbf{(1) A unified scene specification} that is compatible with multiple robot simulators. It enables various XR headsets to visualize and interact with simulated objects and robots, providing an immersive experience while ensuring straightforward reusability and reproducibility.\\
\textbf{(2) A collaborative data collection framework} designed for XR environments. The framework facilitates enhanced robot data acquisition.\\
\textbf{(3) A user study} demonstrating that IRIS significantly improves data collection efficiency and intuitiveness compared to the LIBERO baseline.

% \begin{table*}[t]
%     \centering
%     \begin{tabular}{lccccccc}
%         \toprule
%         & \makecell{Physical\\Interaction}
%         & \makecell{XR\\Enabled}
%         & \makecell{Free\\View}
%         & \makecell{Multiple\\Robots}
%         & \makecell{Robot\\Control}
%         % Force Feedback???
%         & \makecell{Soft Object\\Supported}
%         & \makecell{Collaborative\\Data} \\
%         \midrule
%         ARC-LfD \cite{arclfd}                              & Real        & \cmark & \xmark & \xmark & Joint              & \xmark & \xmark \\
%         DART \cite{dexhub-park}                            & Sim         & \cmark & \cmark & \cmark & Cartesian          & \xmark & \xmark \\
%         \citet{jiang2024comprehensive}                     & Sim         & \cmark & \xmark & \xmark & Joint \& Cartesian & \xmark & \xmark \\
%         \citet{mosbach2022accelerating}                    & Sim         & \cmark & \cmark & \xmark & Cartesian          & \xmark & \xmark \\
%         ARCADE \cite{arcade}                               & Real        & \cmark & \cmark & \xmark & Cartesian          & \xmark & \xmark \\
%         Holo-Dex \cite{holodex}                            & Real        & \cmark & \xmark & \cmark & Cartesian          & \cmark & \xmark \\
%         ARMADA \cite{armada}                               & Real        & \cmark & \xmark & \cmark & Cartesian          & \cmark & \xmark \\
%         Open-TeleVision \cite{opentelevision}              & Real        & \cmark & \cmark & \cmark & Cartesian          & \cmark & \xmark \\
%         OPEN TEACH \cite{openteach}                        & Real        & \cmark & \xmark & \cmark & Cartesian          & \cmark & \cmark \\
%         GELLO \cite{wu2023gello}                           & Real        & \xmark & \cmark & \cmark & Joint              & \cmark & \xmark \\
%         DexCap \cite{wang2024dexcap}                       & Real        & \xmark & \cmark & \xmark & Cartesian          & \cmark & \xmark \\
%         AnyTeleop \cite{Qin2023AnyTeleopAG}                & Real        & \xmark & \xmark & \cmark & Cartesian          & \cmark & \cmark \\
%         Vicarios \cite{vicarios}                           & Real        & \cmark & \xmark & \xmark & Cartesian          & \cmark & \xmark \\     
%         Augmented Visual Cues \cite{augmentedvisualcues}   & Real        & \cmark & \cmark & \xmark & Cartesian          & \xmark & \xmark \\ 
%         \citet{wang2024robotic}                            & Real        & \cmark & \cmark & \xmark & Cartesian          & \cmark & \xmark \\
%         Bunny-VisionPro \cite{bunnyvisionpro}              & Real        & \cmark & \cmark & \cmark & Cartesian          & \cmark & \xmark \\
%         IMMERTWIN \cite{immertwin}                         & Real        & \cmark & \cmark & \cmark & Cartesian          & \xmark & \xmark \\
%         \citet{meng2023virtual}                            & Sim \& Real & \cmark & \cmark & \xmark & Cartesian          & \xmark & \xmark \\
%         Shared Control Framework \cite{sharedctlframework} & Real        & \cmark & \cmark & \cmark & Cartesian          & \xmark & \xmark \\
%         OpenVR \cite{openvr}                               & Real        & \cmark & \cmark & \xmark & Cartesian          & \xmark & \xmark \\
%         \citet{digitaltwinmr}                              & Real        & \cmark & \cmark & \xmark & Cartesian          & \cmark & \xmark \\
        
%         \midrule
%         \textbf{Ours} & Sim \& Real & \cmark & \cmark & \cmark & Joint \& Cartesian  & \cmark & \cmark \\
%         \bottomrule
%     \end{tabular}
%     \caption{This is a cross-column table with automatic line breaking.}
%     \label{tab:cross-column}
% \end{table*}

% \begin{table*}[t]
%     \centering
%     \begin{tabular}{lccccccc}
%         \toprule
%         & \makecell{Cross-Embodiment}
%         & \makecell{Cross-Scene}
%         & \makecell{Cross-Simulator}
%         & \makecell{Cross-Reality}
%         & \makecell{Cross-Platform}
%         & \makecell{Cross-User} \\
%         \midrule
%         ARC-LfD \cite{arclfd}                              & \xmark & \xmark & \xmark & \xmark & \xmark & \xmark \\
%         DART \cite{dexhub-park}                            & \cmark & \cmark & \xmark & \xmark & \xmark & \xmark \\
%         \citet{jiang2024comprehensive}                     & \xmark & \cmark & \xmark & \xmark & \xmark & \xmark \\
%         \citet{mosbach2022accelerating}                    & \xmark & \cmark & \xmark & \xmark & \xmark & \xmark \\
%         ARCADE \cite{arcade}                               & \xmark & \xmark & \xmark & \xmark & \xmark & \xmark \\
%         Holo-Dex \cite{holodex}                            & \cmark & \xmark & \xmark & \xmark & \xmark & \xmark \\
%         ARMADA \cite{armada}                               & \cmark & \xmark & \xmark & \xmark & \xmark & \xmark \\
%         Open-TeleVision \cite{opentelevision}              & \cmark & \xmark & \xmark & \xmark & \cmark & \xmark \\
%         OPEN TEACH \cite{openteach}                        & \cmark & \xmark & \xmark & \xmark & \xmark & \cmark \\
%         GELLO \cite{wu2023gello}                           & \cmark & \xmark & \xmark & \xmark & \xmark & \xmark \\
%         DexCap \cite{wang2024dexcap}                       & \xmark & \xmark & \xmark & \xmark & \xmark & \xmark \\
%         AnyTeleop \cite{Qin2023AnyTeleopAG}                & \cmark & \cmark & \cmark & \cmark & \xmark & \cmark \\
%         Vicarios \cite{vicarios}                           & \xmark & \xmark & \xmark & \xmark & \xmark & \xmark \\     
%         Augmented Visual Cues \cite{augmentedvisualcues}   & \xmark & \xmark & \xmark & \xmark & \xmark & \xmark \\ 
%         \citet{wang2024robotic}                            & \xmark & \xmark & \xmark & \xmark & \xmark & \xmark \\
%         Bunny-VisionPro \cite{bunnyvisionpro}              & \cmark & \xmark & \xmark & \xmark & \xmark & \xmark \\
%         IMMERTWIN \cite{immertwin}                         & \cmark & \xmark & \xmark & \xmark & \xmark & \xmark \\
%         \citet{meng2023virtual}                            & \xmark & \cmark & \xmark & \cmark & \xmark & \xmark \\
%         \citet{sharedctlframework}                         & \cmark & \xmark & \xmark & \xmark & \xmark & \xmark \\
%         OpenVR \cite{george2025openvr}                               & \xmark & \xmark & \xmark & \xmark & \xmark & \xmark \\
%         \citet{digitaltwinmr}                              & \xmark & \xmark & \xmark & \xmark & \xmark & \xmark \\
        
%         \midrule
%         \textbf{Ours} & \cmark & \cmark & \cmark & \cmark & \cmark & \cmark \\
%         \bottomrule
%     \end{tabular}
%     \caption{This is a cross-column table with automatic line breaking.}
% \end{table*}

% \begin{table*}[t]
%     \centering
%     \begin{tabular}{lccccccc}
%         \toprule
%         & \makecell{Cross-Scene}
%         & \makecell{Cross-Embodiment}
%         & \makecell{Cross-Simulator}
%         & \makecell{Cross-Reality}
%         & \makecell{Cross-Platform}
%         & \makecell{Cross-User}
%         & \makecell{Control Space} \\
%         \midrule
%         % Vicarios \cite{vicarios}                           & \xmark & \xmark & \xmark & \xmark & \xmark & \xmark \\     
%         % Augmented Visual Cues \cite{augmentedvisualcues}   & \xmark & \xmark & \xmark & \xmark & \xmark & \xmark \\ 
%         % OpenVR \cite{george2025openvr}                     & \xmark & \xmark & \xmark & \xmark & \xmark & \xmark \\
%         \citet{digitaltwinmr}                              & \xmark & \xmark & \xmark & \xmark & \xmark & \xmark &  \\
%         ARC-LfD \cite{arclfd}                              & \xmark & \xmark & \xmark & \xmark & \xmark & \xmark &  \\
%         \citet{sharedctlframework}                         & \cmark & \xmark & \xmark & \xmark & \xmark & \xmark &  \\
%         \citet{jiang2024comprehensive}                     & \cmark & \xmark & \xmark & \xmark & \xmark & \xmark &  \\
%         \citet{mosbach2022accelerating}                    & \cmark & \xmark & \xmark & \xmark & \xmark & \xmark & \\
%         Holo-Dex \cite{holodex}                            & \cmark & \xmark & \xmark & \xmark & \xmark & \xmark & \\
%         ARCADE \cite{arcade}                               & \cmark & \cmark & \xmark & \xmark & \xmark & \xmark & \\
%         DART \cite{dexhub-park}                            & Limited & Limited & Mujoco & Sim & Vision Pro & \xmark &  Cartesian\\
%         ARMADA \cite{armada}                               & \cmark & \cmark & \xmark & \xmark & \xmark & \xmark & \\
%         \citet{meng2023virtual}                            & \cmark & \cmark & \xmark & \cmark & \xmark & \xmark & \\
%         % GELLO \cite{wu2023gello}                           & \cmark & \xmark & \xmark & \xmark & \xmark & \xmark \\
%         % DexCap \cite{wang2024dexcap}                       & \xmark & \xmark & \xmark & \xmark & \xmark & \xmark \\
%         % AnyTeleop \cite{Qin2023AnyTeleopAG}                & \cmark & \cmark & \cmark & \cmark & \xmark & \cmark \\
%         % \citet{wang2024robotic}                            & \xmark & \xmark & \xmark & \xmark & \xmark & \xmark \\
%         Bunny-VisionPro \cite{bunnyvisionpro}              & \cmark & \cmark & \xmark & \xmark & \xmark & \xmark & \\
%         IMMERTWIN \cite{immertwin}                         & \cmark & \cmark & \xmark & \xmark & \xmark & \xmark & \\
%         Open-TeleVision \cite{opentelevision}              & \cmark & \cmark & \xmark & \xmark & \cmark & \xmark & \\
%         \citet{szczurek2023multimodal}                     & \xmark & \xmark & \xmark & Real & \xmark & \cmark & \\
%         OPEN TEACH \cite{openteach}                        & \cmark & \cmark & \xmark & \xmark & \xmark & \cmark & \\
%         \midrule
%         \textbf{Ours} & \cmark & \cmark & \cmark & \cmark & \cmark & \cmark \\
%         \bottomrule
%     \end{tabular}
%     \caption{TODO, Bruce: this table can be further optimized.}
% \end{table*}

\definecolor{goodgreen}{HTML}{228833}
\definecolor{goodred}{HTML}{EE6677}
\definecolor{goodgray}{HTML}{BBBBBB}

\begin{table*}[t]
    \centering
    \begin{adjustbox}{max width=\textwidth}
    \renewcommand{\arraystretch}{1.2}    
    \begin{tabular}{lccccccc}
        \toprule
        & \makecell{Cross-Scene}
        & \makecell{Cross-Embodiment}
        & \makecell{Cross-Simulator}
        & \makecell{Cross-Reality}
        & \makecell{Cross-Platform}
        & \makecell{Cross-User}
        & \makecell{Control Space} \\
        \midrule
        % Vicarios \cite{vicarios}                           & \xmark & \xmark & \xmark & \xmark & \xmark & \xmark \\     
        % Augmented Visual Cues \cite{augmentedvisualcues}   & \xmark & \xmark & \xmark & \xmark & \xmark & \xmark \\ 
        % OpenVR \cite{george2025openvr}                     & \xmark & \xmark & \xmark & \xmark & \xmark & \xmark \\
        \citet{digitaltwinmr}                              & \textcolor{goodred}{Limited}     & \textcolor{goodred}{Single Robot} & \textcolor{goodred}{Unity}    & \textcolor{goodred}{Real}          & \textcolor{goodred}{Meta Quest 2} & \textcolor{goodgray}{N/A} & \textcolor{goodred}{Cartesian} \\
        ARC-LfD \cite{arclfd}                              & \textcolor{goodgray}{N/A}        & \textcolor{goodred}{Single Robot} & \textcolor{goodgray}{N/A}     & \textcolor{goodred}{Real}          & \textcolor{goodred}{HoloLens}     & \textcolor{goodgray}{N/A} & \textcolor{goodred}{Cartesian} \\
        \citet{sharedctlframework}                         & \textcolor{goodred}{Limited}     & \textcolor{goodred}{Single Robot} & \textcolor{goodgray}{N/A}     & \textcolor{goodred}{Real}          & \textcolor{goodred}{HTC Vive Pro} & \textcolor{goodgray}{N/A} & \textcolor{goodred}{Cartesian} \\
        \citet{jiang2024comprehensive}                     & \textcolor{goodred}{Limited}     & \textcolor{goodred}{Single Robot} & \textcolor{goodgray}{N/A}     & \textcolor{goodred}{Real}          & \textcolor{goodred}{HoloLens 2}   & \textcolor{goodgray}{N/A} & \textcolor{goodgreen}{Joint \& Cartesian} \\
        \citet{mosbach2022accelerating}                    & \textcolor{goodgreen}{Available} & \textcolor{goodred}{Single Robot} & \textcolor{goodred}{IsaacGym} & \textcolor{goodred}{Sim}           & \textcolor{goodred}{Vive}         & \textcolor{goodgray}{N/A} & \textcolor{goodgreen}{Joint \& Cartesian} \\
        Holo-Dex \cite{holodex}                            & \textcolor{goodgray}{N/A}        & \textcolor{goodred}{Single Robot} & \textcolor{goodgray}{N/A}     & \textcolor{goodred}{Real}          & \textcolor{goodred}{Meta Quest 2} & \textcolor{goodgray}{N/A} & \textcolor{goodred}{Joint} \\
        ARCADE \cite{arcade}                               & \textcolor{goodgray}{N/A}        & \textcolor{goodred}{Single Robot} & \textcolor{goodgray}{N/A}     & \textcolor{goodred}{Real}          & \textcolor{goodred}{HoloLens 2}   & \textcolor{goodgray}{N/A} & \textcolor{goodred}{Cartesian} \\
        DART \cite{dexhub-park}                            & \textcolor{goodred}{Limited}     & \textcolor{goodred}{Limited}      & \textcolor{goodred}{Mujoco}   & \textcolor{goodred}{Sim}           & \textcolor{goodred}{Vision Pro}   & \textcolor{goodgray}{N/A} & \textcolor{goodred}{Cartesian} \\
        ARMADA \cite{armada}                               & \textcolor{goodgray}{N/A}        & \textcolor{goodred}{Limited}      & \textcolor{goodgray}{N/A}     & \textcolor{goodred}{Real}          & \textcolor{goodred}{Vision Pro}   & \textcolor{goodgray}{N/A} & \textcolor{goodred}{Cartesian} \\
        \citet{meng2023virtual}                            & \textcolor{goodred}{Limited}     & \textcolor{goodred}{Single Robot} & \textcolor{goodred}{PhysX}   & \textcolor{goodgreen}{Sim \& Real} & \textcolor{goodred}{HoloLens 2}   & \textcolor{goodgray}{N/A} & \textcolor{goodred}{Cartesian} \\
        % GELLO \cite{wu2023gello}                           & \cmark & \xmark & \xmark & \xmark & \xmark & \xmark \\
        % DexCap \cite{wang2024dexcap}                       & \xmark & \xmark & \xmark & \xmark & \xmark & \xmark \\
        % AnyTeleop \cite{Qin2023AnyTeleopAG}                & \cmark & \cmark & \cmark & \cmark & \xmark & \cmark \\
        % \citet{wang2024robotic}                            & \xmark & \xmark & \xmark & \xmark & \xmark & \xmark \\
        Bunny-VisionPro \cite{bunnyvisionpro}              & \textcolor{goodgray}{N/A}        & \textcolor{goodred}{Single Robot} & \textcolor{goodgray}{N/A}     & \textcolor{goodred}{Real}          & \textcolor{goodred}{Vision Pro}   & \textcolor{goodgray}{N/A} & \textcolor{goodred}{Cartesian} \\
        IMMERTWIN \cite{immertwin}                         & \textcolor{goodgray}{N/A}        & \textcolor{goodred}{Limited}      & \textcolor{goodgray}{N/A}     & \textcolor{goodred}{Real}          & \textcolor{goodred}{HTC Vive}     & \textcolor{goodgray}{N/A} & \textcolor{goodred}{Cartesian} \\
        Open-TeleVision \cite{opentelevision}              & \textcolor{goodgray}{N/A}        & \textcolor{goodred}{Limited}      & \textcolor{goodgray}{N/A}     & \textcolor{goodred}{Real}          & \textcolor{goodgreen}{Meta Quest, Vision Pro} & \textcolor{goodgray}{N/A} & \textcolor{goodred}{Cartesian} \\
        \citet{szczurek2023multimodal}                     & \textcolor{goodgray}{N/A}        & \textcolor{goodred}{Limited}      & \textcolor{goodgray}{N/A}     & \textcolor{goodred}{Real}          & \textcolor{goodred}{HoloLens 2}   & \textcolor{goodgreen}{Available} & \textcolor{goodred}{Joint \& Cartesian} \\
        OPEN TEACH \cite{openteach}                        & \textcolor{goodgray}{N/A}        & \textcolor{goodgreen}{Available}  & \textcolor{goodgray}{N/A}     & \textcolor{goodred}{Real}          & \textcolor{goodred}{Meta Quest 3} & \textcolor{goodred}{N/A} & \textcolor{goodgreen}{Joint \& Cartesian} \\
        \midrule
        \textbf{Ours}                                      & \textcolor{goodgreen}{Available} & \textcolor{goodgreen}{Available}  & \textcolor{goodgreen}{Mujoco, CoppeliaSim, IsaacSim} & \textcolor{goodgreen}{Sim \& Real} & \textcolor{goodgreen}{Meta Quest 3, HoloLens 2} & \textcolor{goodgreen}{Available} & \textcolor{goodgreen}{Joint \& Cartesian} \\
        \bottomrule
        \end{tabular}
    \end{adjustbox}
    \caption{Comparison of XR-based system for robots. IRIS is compared with related works in different dimensions.}
\end{table*}



\section{METHODS}
    \subsection{Dynamic Color Image Normalization}
Fig. \ref{fig:fig_2} (blue dashed box) shows the data flow of our dynamic color image normalization (DCIN) method. 
For a given input test image from a non-source domain, the reference image selection module strategically identifies suitable reference images from the training source domain. 
The color transfer in the perception-based color space $l\alpha \beta$ \cite{reinhard2001color} is then applied to align the color distribution of the reference images with that of the test image. 
The reference image selection module incorporates two strategies: \say{global} reference selection and \say{local} reference selection. 
The global strategy assigns a single reference image to all test images, whereas the local strategy selects a unique reference image for each individual test image. 
Detailed descriptions of these strategies are described below. 

\subsubsection{Global Reference Image Selection}
We propose the global reference image selection (GRIS) strategy that utilizes color histograms such that each image in the source domain is converted into a $b$-bin normalized color histogram vector. 
The global reference image $x_g$ is selected as the image whose color histogram minimizes the average pairwise distance between histograms of all other images in the source domain. 
The average pairwise distance $\mathcal{D}_{\mathrm{pairwise}}(x_i) = \frac1N\overset N{\underset{j=1}{\sum d_{i,j}}}$, where $d_{i,j}=\sqrt{\sum_{k=1}^b\left(H_k\left(x_i\right)-H_k\left(x_j\right)\right)^2}$ is the Euclidean distance between two normalized histogram vectors. 
In this case, $H_k(x)$ is the $k$-th bin value of the image $x$, and $N$ is the number of images in the source domain. 
The selected image $x_g$ will be used as the color normalization reference of \textit{all} test images before making predictions. 

\subsubsection{Local Reference Image Selection}
As stated previously, segmentation results vary on different reference images. 
Thus, we believe selecting a semantically similar image to the test image from the training data can benefit the segmentation performance. 
We propose a local reference image selection (LRIS) strategy that utilizes a pre-trained CNN model to select a more tailored reference image for each test image. 
First, the pre-trained CNN was used to extract feature vectors from all source images, which were then normalized into unit vectors. 
For a given image $x_{test}$ in a test domain, a local reference image $x_l$ is selected as the one whose feature vector has the highest cosine similarity with $x_{test}$. 
The selected image $x_l$ will be used as the color normalization reference of the test image $x_{test}$ before making predictions. 

\subsubsection{Ensembling Both Selected Reference Images}
To further improve generalization capabilities, the results of the above two strategies can be ensembles. 
From the selected global and local reference images, two color-normalized input images were produced, which were then used to generate two corresponding output masks. 
Finally, the final prediction is formed by taking a pixel-wise mean of the two predicted masks. 
Here, we refer to the \say{full DCIN} as the complete module with this ensemble method (i.e., GRIS + LRIS). 

\subsection{The Color-Quality Generalization Loss}
The color-quality generalization loss (CQG) is a training objective function inspired by contrastive loss. 
The idea is that given the same input presented in varying colors and qualities, the model should produce identical segmentation masks. 
This approach encourages the model to adapt effectively to images with diverse variations. 
Fig. \ref{fig:fig_2} (purple dashed box) illustrates the data flow of the CQG loss. 
For each training image $x$, we apply transformations randomly to generate two inputs for the segmentation model. 
The first input $x_1$ is obtained by applying geometric transformations, and the second input $x_2$ is from both geometric and photometric transformations. 
Geometric transformations are applied to change the input $x$ geometrically while photometric transformations change its color and quality. 
Specifically, geometric transformations include random horizontal flip, shear, shift, scale, rotation, and elastic transform. 
Photometric transformations include random blur, sharpening, Gaussian noise, brightness contrast, and RGB shifts. 
Both $x_1$ and $x_2$ have the same ground-truth mask $y$.

Given a segmentation model $S$ and a ground-truth mask $y$, we have $y_1 = S(x_1)$, and $y_2 = S(x_2)$. 
Our CQG loss is defined as:
\begin{equation}
    \mathcal{L} = \lambda_{1}\mathrm{DC}(y,y_1) + \lambda_{2}\mathrm{DC}(y,y_2) + \lambda_{3}\mathrm{MSE}(y_1,y_2),
\end{equation}
where $\mathrm{DC}(y,y\prime)$ is the sum of the Dice loss and cross-entropy loss between the ground truth $y$ and predicted mask $y\prime$. 
$\mathrm{MSE}(y_1,y_2)$ is the mean squared error loss between the predicted masks $y_1$ and $y_2$. 
Here, $\lambda_{1}, \lambda_{2}, \lambda_{3}$ are the hyper-parameters controlling the weight of each loss term. 

The CQG loss can be viewed as a form of image augmentation. By leveraging this loss, the model is encouraged to produce identical predictions for both original and augmented images, regardless of color and quality shifts. 

\section{EXPERIMENTAL RESULTS}
    \subsection{Data Collection}
In this work, we collected 16,000 high-quality (HQ) clean throat images of around 900 patients from several hospitals in Japan and refer to it as the $\mathrm{HQ}_{all}$ dataset. 
They were obtained by a special camera type designed for taking pharyngeal images. 
Among the $\mathrm{HQ}_{all}$ dataset, there are 2,000 images that were labeled by experts containing the semantic pixel-level annotations of four areas inside the throat: uvula, tongue, tonsil, and pharyngeal wall. 
We randomly split 1,600 images for training (refer as $\mathrm{HQ}_{seg/train}$ set) and 400 images for validation (refer as $\mathrm{HQ}_{seg/val}$ set). 
In addition, the $\mathrm{HQ}_{all}$ dataset is used for color reference image selection as in the DCIN module. 

To evaluate the performance of the medical image segmentation models, we additionally collected data from two non-source domains. 
First, we collected 255 LQ images (e.g., blurry, hazy, compression artifacts, poor lighting, etc.) from over 150 patients. 
These images were taken by different camera devices, and we refer this as the $\mathrm{LQ}_{seg}$ dataset. 
Second, we collected 125 smartphone (SP) images. 
These images were taken primarily with iPhone SE (1st gen) and Sony Xperia XZ1 (SO-01K) cameras, and we refer to this as the $\mathrm{SP}_{seg}$ dataset. 
Images from all datasets are resized to the size of $768 \times 512$ pixels. 
Please refer to Fig. \ref{fig:fig_1} for some samples of these datasets. 

\subsection{Reference Image Selection}
For each test image, we utilize two images that correspond to the GRIS and LRIS strategies in our DCIN module. 
The global reference image $x_g$ is selected by applying the color histogram reference image selection pipeline on the $\mathrm{HQ}_{seg/train}$ dataset. 
We created the histograms in RGB space and used eight bins per channel based on our preliminary experiments. 
Note again that the image is the same across all test images and is selected before the test time. 
The local reference image $x_l$ is selected at inference time for each test image. 
Specifically, for all images in the $\mathrm{HQ}_{all}$ dataset, we pre-computed embedding vectors using a Swin-V2-Large model \cite{liu2021swin} which was pre-trained on the ImageNet dataset \cite{deng09imagenet}. 

To further validate the effectiveness of our GRIS and LRIS strategies, we incorporated a reference image selected by physicians for color transfer before testing. 
The image was chosen based on specific criteria, including cleanliness and color balance, ensuring its suitability for diagnostic purposes. 
We refer to this image as the expert-selected reference image (ExRI). 
% Fig. 3
%Figure 3
\begin{figure*}[!t]
\centering
\includegraphics[width=0.95\textwidth]{figures/Figure_results_comparison_final.pdf}
\caption{
    Visual comparison of the results on the $\mathrm{LQ}_{seg}$ (top) and $\mathrm{SP}_{seg}$ (bottom) datasets from different segmentation models. 
    Results with DCIN are from the full DCIN setting (i.e., GRIS + LRIS).
}
\label{fig:fig_3}
\end{figure*}

\subsection{Training Throat Segmentation Models}
To evaluate the effectiveness of our proposals for supporting SDG medical image segmentation tasks, in this experiment, we trained three different throat image segmentation models on the $\mathrm{HQ}_{seg/train}$ dataset. 
All models are built based on the U-Net \cite{ronneberger15unet} model with the pre-trained EfficientNet-B2 \cite{tan2019efficientnet} on the ImageNet dataset as the backbone. 
The three segmentation models are:
\begin{itemize}
    \item Baseline ($S_{\mathrm{base}}$): The baseline model is trained with minimal preprocessing, consisting only of resizing and normalization. 
    A simple loss function comprising the sum of Dice and cross-entropy losses is applied without any data augmentations. 
    
    \item Baseline + augmentations ($S_{\mathrm{aug}}$): The baseline model is additionally trained with the augmentations as in the CQG loss. 
    The data augmentation only creates one output from an input image and the CQG loss is not applied. 
    
    \item Baseline + CQG ($S_{\mathrm{CQG}}$): The baseline model is trained with the color-quality generalization (CQG) loss function. 
    We chose to use $\lambda_{1}=0.3, \lambda_{2}=0.7, \lambda_{3}=1.0$ based on our preliminary experiments. 
\end{itemize}

All three throat segmentation models were optimized using the Adam optimizer \cite{kingma2015adam} with a learning rate of $1 \times 10^{-3}$. 
The batch size was set to 2 and training was completed after 15 epochs. 
For the evaluation metric, we employed the commonly used Dice score to measure the overlap between the prediction and the segmentation ground-truth masks. 

\subsection{Results of Throat Segmentation Models}
After training, the three segmentation models $S_{\mathrm{base}}$, $S_{\mathrm{aug}}$, and $S_{\mathrm{CQG}}$ achieved Dice scores of 88.9, 87.8, and 88.6 on the $\mathrm{HQ}_{seg/val}$ dataset, respectively. 
All models demonstrated high accuracy on the source domain, indicating the capability of accurately segmenting HQ images. 

Table \ref{tab:table_1} and \ref{tab:table_2} summarize the Dice scores for the three models on the $\mathrm{LQ}_{seg}$ and $\mathrm{SP}_{seg}$ datasets (the best performance of each model across all DCIN configurations is in \textbf{bold} text). 
The visual comparison of segmentation results is provided in Fig. \ref{fig:fig_3}. 
Note that the results with DCIN depicted in Fig. \ref{fig:fig_3} are from the full DCIN (i.e., GRIS + LIRS). 
All models experienced a severe performance drop when moving from the HQ dataset to other non-source datasets. 
For instance, without DCIN module, the $S_{\mathrm{base}}$ model's Dice score largely dropped from 88.9 on $\mathrm{HQ}_{seg/val}$ to 36.9 on the $\mathrm{LQ}_{seg}$ dataset, and down to 55.3 on the $\mathrm{SP}_{seg}$ dataset. 

Both the DCIN module and CQG loss consistently boosted the segmentation performances in all models. 
The most effective configuration was achieved by combining CQG training with the full DCIN (i.e., GRIS + LRIS), resulting in increments in Dice score over the baseline models, from 36.9 to 69.2 on the $\mathrm{LQ}_{seg}$, and 55.3 to 68.6 on the $\mathrm{SP}_{seg}$ datasets. 
% Table I
\begin{table}[t]
\centering
\caption{Performance comparison in Dice score on the $\mathrm{LQ}_{seg}$ dataset}
\label{tab:table_1}
\resizebox{0.97\linewidth}{!}{
\begin{tabular}{@{}llllll@{}}
\toprule
\multicolumn{1}{c}{\multirow{2}{*}{Model}} & \multirow{2}{*}{W/o DCIN} & \multicolumn{4}{c}{DCIN}                    \\ \cmidrule(l){3-6} 
\multicolumn{1}{c}{}                        &                           & ExRI & GRIS & LRIS          & GRIS + LRIS   \\ \midrule
$S_{\mathrm{base}}$                                    & 36.9                      & 53.7 & 52.4 & \textbf{58.9} & 56.9          \\ \midrule
$S_{\mathrm{aug}}$                                & 54.0                      & 55.9 & 53.7 & \textbf{58.1} & 57.2          \\ \midrule
$S_{\mathrm{CQG}}$                                & 64.5                      & 67.2 & 67.5 & 68.5          & \textbf{69.2} \\ \bottomrule
\end{tabular}
}
\end{table}
% Table II
\begin{table}[t]
\centering
\caption{Performance comparison in Dice score on the $\mathrm{SP}_{seg}$ dataset}
\label{tab:table_2}
\resizebox{0.97\linewidth}{!}{
\begin{tabular}{@{}llllll@{}}
\toprule
\multicolumn{1}{c}{\multirow{2}{*}{Model}} & \multirow{2}{*}{W/o DCIN} & \multicolumn{4}{c}{DCIN}                    \\ \cmidrule(l){3-6} 
\multicolumn{1}{c}{}                       &                           & ExRI & GRIS & LRIS          & GRIS + LRIS   \\ \midrule
$S_{\mathrm{base}}$                                   & 55.3                      & 61.0 & 63.8 & 60.8 & \textbf{65.5} \\ \midrule
$S_{\mathrm{aug}}$                               & 60.6                      & 62.4 & 66.1 & 64.4 & \textbf{66.7} \\ \midrule
$S_{\mathrm{CQG}}$                               & 63.5                      & 65.0 & 68.1 & 67.1          & \textbf{68.6} \\ \bottomrule
\end{tabular}
}
\end{table}

\section{DISCUSSION}
    In this study, we demonstrated the effectiveness of our framework in identifying regions where the prescribed material law did not hold using the precision parameter. In regions where the material law was valid, the predicted material parameters were correctly inferred, while in invalid regions, the applied prior flattened the field, preventing inaccurate predictions. Additionally, the inferred stress fields adhered to the conservation law, further validating our approach.

Our method enabled fast inference by leveraging the weighted residual scheme proposed by Scholz et al., demonstrating its computational efficiency. However, several avenues remain for future work.

To further validate and extend our approach, we plan to:
\begin{itemize}
    \item Conduct experiments on higher-dimensional grids (e.g., 64 × 64 resolution).
    \item Test the framework using real or phantom data to assess its practical applicability.
    \item Investigate scenarios with multiple inclusions, including one with a valid and one with an invalid material law. Initial tests showed that both inclusions were flagged as invalid, highlighting a need for refinement.
    \item Implement latent-based fields to downscale large neural networks and improve computational efficiency.
\end{itemize}

These future investigations will enhance the framework's robustness and broaden its applicability in inverse modeling problems.


\section{CONCLUSION}
    \section{Conclusion}
We introduce the Difficulty and Uncertainty-Aware Lightweight (DUAL) score, a novel scoring metric designed for cost-effective pruning. The DUAL score is the first metric to integrate both difficulty and uncertainty into a single measure, and its effectiveness in identifying the most informative samples early in training is further supported by theoretical analysis. Additionally, we propose pruning-ratio-adaptive sampling to account for sample diversity, particularly when the pruning ratio is extremely high. Our proposed pruning methods, DUAL score and DUAL score combined with Beta sampling demonstrate remarkable performance, particularly in realistic scenarios involving label noise and image corruption, by effectively distinguishing noisy samples.

Data pruning research has been evolving in a direction that contradicts its primary objective of reducing computational and storage costs while improving training efficiency. This is mainly because the computational cost of pruning often exceeds that of full training. By introducing our DUAL method, we take a crucial step toward overcoming this challenge by significantly reducing the computation cost associated with data pruning, making it feasible for practical scenarios. Ultimately, we believe this will help minimize resource waste and enhance training efficiency.

\section*{ACKNOWLEDGMENT}
We would like to thank all researchers at Aillis Inc., especially the AI Development team, Dr. Memori Fukuda (MD), Dr. Kei Katsuno (MD), and Dr. Takaya Hanawa (MD) for their valuable comments and feedback. 

\nocite{*}
\footnotesize{
\bibliographystyle{IEEEtran}
\bibliography{reference}
}

\end{document}
