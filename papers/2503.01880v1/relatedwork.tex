\section{Related work}
\label{sec:LR}

Thematic analysis is a key qualitative method used to identify patterns and themes within textual data. As social media has become a rich source of public discourse, the ability to analyze this unstructured and large-scale data has become crucial. Traditional methods of thematic analysis, though valuable for small datasets, often struggle with the volume, diversity, and complexity of social media text. This literature review outlines the progression of thematic analysis methods, highlighting the limitations of traditional techniques and how recent advancements, particularly our proposed approach, fill these gaps.
\newline
\subsection{Traditional Thematic Analysis in Social Media Research}

Thematic analysis has long been used to explore qualitative data, with foundational guidelines established by \cite{braun2006using}. Their method involves a systematic, manual coding process that identifies recurring themes through a close reading of the text. While this method remains highly respected in qualitative research, its application to social media data is problematic due to the large volume and complexity of text involved \cite{braun2019reflecting}.
Manual coding becomes time-consuming and unfeasible when applied to datasets that may include millions of posts, such as Twitter data \cite{boyatzis1998transforming}. In such cases, traditional thematic analysis often requires a reduced sample size or relies on researchers' subjective judgments to identify salient themes. This introduces biases and limits the generalizability of the findings \cite{markham2017research}. Additionally, the informal, fast-paced nature of social media language—frequently composed of abbreviations, slang, and emotive content—makes manual coding less reliable, as it struggles to capture nuances or variations in context \cite{golder2014digital}.
Due to these limitations, researchers began exploring computational methods for thematic analysis. Early approaches like word frequency analysis and topic modeling with Latent Dirichlet Allocation (LDA) became popular tools to analyze large datasets \cite{blei2003latent}. LDA has been applied extensively in social media research, particularly in studies focused on public opinion, social movements, and political discourse \cite{nguyen2015sentiment}. However, LDA's assumption that topics are merely distributions of words across documents oversimplifies the language and cannot capture the nuances of meaning, sentiment, and context present in complex social media texts \cite{chuang2012termite}.
Moreover, topic modeling techniques such as LDA struggle with the diverse, informal nature of social media language, often resulting in themes that are overly broad or too fragmented to be meaningful \cite{hong2010empirical}. Our proposed methodology directly addresses these limitations by incorporating advanced techniques like tweet embeddings from pre-trained language models, enabling a more nuanced understanding of social media discourse by preserving contextual relationships between words \cite{devlin2018bert}.
\newline
\subsection{Advances in Machine Learning and NLP for Thematic Analysis}

Recent advancements in natural language processing (NLP) have enhanced thematic analysis techniques, allowing researchers to overcome some of the limitations posed by traditional methods. The introduction of transformer-based language models, such as BERT \cite{devlin2018bert}, GPT-2/3 \cite{brown2020language}, and others, has significantly improved the ability to understand semantic relationships in text. These models leverage vast amounts of training data to generate contextualized word embeddings, offering deeper insights into text than earlier models like LDA or TF-IDF \cite{mikolov2013distributed}.

Two main studies rely on topic modeling and sentiment analysis to understand themes related to autism on Twitter. The first study by \cite{corti2022social} uses Non-Negative Matrix Factorization (NMF) for topic modeling and sentiment analysis with AFiNN, while the second study \cite{gabarron2023autistic} applies the BERTopic model for clustering tweets and extracting themes using dimensionality reduction via UMAP. Despite their contributions, these approaches exhibit limitations. NMF, although useful for topic coherence, lacks the ability to capture deep linguistic nuances, particularly in social media language. The BERTopic approach, while more advanced, relies on static embeddings and bag-of-words methods that may overlook semantic richness and context within tweets, especially in complex discourse like the autism community’s.

The use of pre-trained language models in thematic analysis has been explored in recent studies. \cite{wu2022study} applied BERT embeddings to cluster social media posts about public health issues, demonstrating improved coherence in the resulting themes compared to LDA. Likewise, \cite{yin2020detecting} showed that using contextualized embeddings from language models improved the ability to detect latent topics in crisis communication data. These studies highlight the superior capacity of pre-trained models to handle social media’s evolving terminology and informal language structures.

However, while embeddings capture rich semantic information, high dimensional representations pose challenges for downstream analysis. Dimensionality reduction techniques like autoencoder neural network and matrix factorization have emerged as effective ways to reduce complexity while preserving thematic content. \cite{chauhan2024tracking} demonstrated the utility of autoencoder neural network in compressing embeddings for more efficient clustering in sentiment analysis, which improves both the scalability and interpretability of results.

Our methodology builds on these approaches by integrating autoencoder neural network with matrix factorization to uncover latent themes, retaining the rich semantic information encoded in the tweet embeddings while effectively reducing dimensionality. This combination ensures that nuanced themes can be extracted from large datasets without losing the contextual meaning captured by the language models.
\newline
\subsection{Generative AI and Iterative Theme Refinement}

Generative AI models, such as GPT-3, have shown promise in thematic analysis by automating tasks such as text summarization, classification, and thematic extraction \cite{brown2020language}. These models generate coherent and contextually appropriate text, which can be useful for extracting or refining themes from large datasets. For example, studies by \cite{dong2024understanding} explored generative models for summarizing social media content, underscoring their potential for condensing vast amounts of text into coherent themes.

Despite this potential, generative AI has rarely been used in a more iterative, refining role in thematic analysis. Previous studies have typically applied generative models for summarization or categorization, without refining or validating the themes through multiple stages of analysis. This paper's \cite{wanna2024topictag} use CoT prompting is limited in that it primarily focuses on improving topic labels through token-based features and manual filtering, without fully leveraging the iterative reasoning potential of CoT for deeper semantic understanding. Additionally, their approach to CoT is more narrowly applied to optimize prompt tuning rather than refining the thematic structure of the topics themselves.
This gap is addressed in our methodology by using an iterative CoT prompting mechanism, which allows for iterative theme extraction and refinement. This process ensures that the themes are coherent, contextually accurate, and relevant to the dataset.

Although recent advancements in NLP and machine learning have improved thematic analysis, several critical gaps remain. First, many existing models still struggle to capture the complexity and nuance of informal social media language, particularly in communities with specialized or evolving vocabularies, such as the autistic community. LDA and word frequency-based methods, while useful for structured data, oversimplify language and miss deeper contextual meanings \cite{chuang2012termite}.
Second, many machine learning methods lack iterative processes for refining and validating themes. Pre-trained language models and clustering methods have advanced theme extraction, but they do not inherently include mechanisms for ensuring thematic consistency or addressing ambiguous themes \cite{wu2022study,yin2020detecting}. Moreover, generative AI models applied to thematic analysis have mostly focused on single-pass theme identification without multiple stages of validation or refinement, leading to incomplete or inconsistent results and the LLMs have fixed context windows \cite{dong2024understanding} and are not able to ingest text beyond that which is why clustering needs to be done before feeding the results to LLMs.
% should this be part of methodology or good to have in literature review
The proposed methodology addresses these gaps by:
\begin{enumerate}
    \item Generating tweet embeddings from pre-trained language models to capture semantic relationships and nuances in social media language. This is crucial for analyzing discourse in specialized communities, like the autistic community, which is the focus of this case study.
    \item Applying dimensionality reduction by integrating autoencoder neural network and matrix factorization to reduce the complexity of embeddings without losing key thematic information while clustering the relevant social media posts together. This approach overcomes the limitations of high-dimensional data.
    \item To balance the high cost of token generation using LLMs with the need to retain meaningful cluster information, a sample size was chosen that is representative enough to capture the overall themes of the entire cluster while minimizing the data passed to the LLMs.
    \item Using generative AI for iterative refinement by employing agentic CoT prompting mechanism to iteratively extract and refine themes of the extracted clusters, ensuring that the identified themes are accurate and contextually relevant. The inclusion of a secondary LLM for quality control further enhances the reliability of the analysis, addressing common issues with theme consistency in automated methods.
\end{enumerate}
Table \ref{tab:lit_rev} summarizes previous research conducted to address the problem of social media theme extraction and highlights how the proposed methodology aims to fill the gaps identified in these studies.
\begin{table}[H]
\centering
\caption{Comparison of Thematic Analysis Methodologies for Social Media Research}
\label{tab:lit_rev}
\begin{adjustbox}{max width=\textwidth}
\begin{tabular}{>{\raggedright\arraybackslash}p{3.5cm} >{\raggedright\arraybackslash}p{3.5cm} >{\raggedright\arraybackslash}p{3.5cm} >{\raggedright\arraybackslash}p{3.5cm}}
\hline
\textbf{Methodology} & \textbf{Focus} & \textbf{Weaknesses} & \textbf{Advantage of Proposed Methodology} \\ \hline
Traditional Thematic Analysis \cite{braun2006using} & Manual coding of text  & Unfeasible for large datasets, prone to bias & Scalable embedding-based approach retains theme detail while handling large volumes \\ \hline
Latent Dirichlet Allocation (LDA) \cite{blei2003latent} & Probabilistic topic modeling for large datasets  & Overly broad themes, lacks nuanced context & Embedding-based clustering captures semantic nuance; maintains coherence \\ \hline
Non-Negative Matrix Factorization (NMF) \cite{corti2022social} & Topic coherence via dimensionality reduction  & Limited in capturing informal language nuance & Embeddings and autoencoder neural network refine themes with more linguistic context \\ \hline
BERTopic \cite{gabarron2023autistic} & Clustering and topic modeling using static embeddings & Loses contextual depth in dynamic social media language & BERT embeddings with CoT refine themes; more adaptable to language dynamics \\ \hline
Pre-trained Language Models (e.g., BERT) \cite{devlin2018bert} & Contextualized embeddings for semantic insights & High-dimensional output challenging for analysis & Dimensionality reduction (autoencoder) makes clustering efficient, preserves semantic richness \\ \hline
Generative AI \cite{dong2024understanding} & Theme extraction via summarization & Single-pass extraction; lacks iterative refinement & CoT prompting iteratively refines themes, ensures context relevance \\ \hline
\end{tabular}
\end{adjustbox}
\end{table}