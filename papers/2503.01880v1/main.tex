%% 
%% Copyright 2007-2024 Elsevier Ltd
%% 
%% This file is part of the 'Elsarticle Bundle'.
%% ---------------------------------------------
%% 
%% It may be distributed under the conditions of the LaTeX Project Public
%% License, either version 1.3 of this license or (at your option) any
%% later version.  The latest version of this license is in
%%    http://www.latex-project.org/lppl.txt
%% and version 1.3 or later is part of all distributions of LaTeX
%% version 1999/12/01 or later.
%% 
%% The list of all files belonging to the 'Elsarticle Bundle' is
%% given in the file `manifest.txt'.
%% 
%% Template article for Elsevier's document class `elsarticle'
%% with numbered style bibliographic references
%% SP 2008/03/01
%% $Id: elsarticle-template-num.tex 249 2024-04-06 10:51:24Z rishi $
%%
\documentclass[12pt]{elsarticle}

%% Use the option review to obtain double line spacing
%% \documentclass[authoryear,preprint,review,12pt]{elsarticle}

%% Use the options 1p,twocolumn; 3p; 3p,twocolumn; 5p; or 5p,twocolumn
%% for a journal layout:
%% \documentclass[final,1p,times]{elsarticle}
%% \documentclass[final,1p,times,twocolumn]{elsarticle}
%% \documentclass[final,3p,times]{elsarticle}
%% \documentclass[final,3p,times,twocolumn]{elsarticle}
%% \documentclass[final,5p,times]{elsarticle}
%% \documentclass[final,5p,times,twocolumn]{elsarticle}

%% For including figures, graphicx.sty has been loaded in
%% elsarticle.cls. If you prefer to use the old commands
%% please give \usepackage{epsfig}

%% The amssymb package provides various useful mathematical symbols
\usepackage{amssymb}
%% The amsmath package provides various useful equation environments.
\usepackage{subcaption}
\usepackage{multirow}
\usepackage{makecell}
\usepackage{amsmath}
\usepackage{afterpage}
\usepackage{rotating}
\usepackage{subcaption}
\usepackage{algorithm}
\usepackage{lineno}
\usepackage{algpseudocode}
\usepackage{amsmath}
\usepackage{graphicx}
\usepackage{adjustbox}

\usepackage[utf8]{inputenc} % allow utf-8 input
\usepackage[T1]{fontenc}    % use 8-bit T1 fonts
\usepackage{hyperref}       % hyperlinks
\usepackage{url}            % simple URL typesetting
\usepackage{booktabs}       % professional-quality tables
\usepackage{amsfonts}       % blackboard math symbols
\usepackage{nicefrac}       % compact symbols for 1/2, etc.
\usepackage{microtype}      % microtypography
\usepackage{lipsum}
\usepackage{graphicx}
\graphicspath{{media/}}     % organize your images and other figures under media/ folder
\usepackage{float}
%% The amsthm package provides extended theorem environments
%% \usepackage{amsthm}

%% The lineno packages adds line numbers. Start line numbering with
%% \begin{linenumbers}, end it with \end{linenumbers}. Or switch it on
%% for the whole article with \linenumbers.
%% \usepackage{lineno}

\journal{arxiv}

\begin{document}

\begin{frontmatter}

%% Title, authors and addresses

%% use the tnoteref command within \title for footnotes;
%% use the tnotetext command for theassociated footnote;
%% use the fnref command within \author or \affiliation for footnotes;
%% use the fntext command for theassociated footnote;
%% use the corref command within \author for corresponding author footnotes;
%% use the cortext command for theassociated footnote;
%% use the ead command for the email address,
%% and the form \ead[url] for the home page:
%% \title{Title\tnoteref{label1}}
%% \tnotetext[label1]{}
%% \author{Name\corref{cor1}\fnref{label2}}
%% \ead{email address}
%% \ead[url]{home page}
%% \fntext[label2]{}
%% \cortext[cor1]{}
%affiliation{organization={},
%%             addressline={},
%%             city={},
%%             postcode={},
%%             state={},
%%             country={}}
%% \fntext[label3]{}

\title{BEYONDWORDS is All You Need: Agentic Generative AI based Social Media Themes Extractor}

%% use optional labels to link authors explicitly to addresses:
%% \author[label1,label2]{}
%% \affiliation[label1]{organization={},
%%             addressline={},
%%             city={},
%%             postcode={},
%%             state={},
%%             country={}}
%%
%% \affiliation[label2]{organization={},
%%             addressline={},
%%             city={},
%%             postcode={},
%%             state={},
%%             country={}}

\author[label1]{Mohammed-Khalil Ghali} %% Author name
\author[label1]{Abdelrahman Farrag} %% Author name
\author[label1]{Sarah Lam}
\author[label1]{Daehan Won}

%% Author name

%% Author affiliation
\affiliation[label1]{organization={School of Systems Science and Industrial Engineering, State University of New York at Binghamton},%Department and Organization
            addressline={4400 Vestal Pkwy}, 
            city={Binghamton},
            postcode={13902}, 
            state={NY},
            country={USA}}
            

%% Abstract
\begin{abstract}
%% Text of abstract
Thematic analysis of social media posts provides a major understanding of public discourse, yet traditional methods often struggle to capture the complexity and nuance of unstructured, large-scale text data. This study introduces a novel methodology for thematic analysis that integrates tweet embeddings from pre-trained language models, dimensionality reduction using  and matrix factorization, and generative AI to identify and refine latent themes. Our approach clusters compressed tweet representations and employs generative AI to extract and articulate themes through an agentic Chain of Thought (CoT) prompting, with a secondary LLM for quality assurance.
This methodology is applied to tweets from the autistic community, a group that increasingly uses social media to discuss their experiences and challenges. By automating the thematic extraction process, the aim is to uncover key insights while maintaining the richness of the original discourse. This autism case study demonstrates the utility of the proposed approach in improving thematic analysis of social media data, offering a scalable and adaptable framework that can be applied to diverse contexts. The results highlight the potential of combining machine learning and Generative AI to enhance the depth and accuracy of theme identification in online communities.
\end{abstract}

%\begin{graphicalabstract}
%    \centering
%    \begin{figure}
%        \centering
%        \begin{subfigure}{\textwidth}
%            \centering
%            \includegraphics[width=\linewidth]{images/Methodology (2).png}
%            \caption{GTR model}
%            \label{fig:imageA}
%        \end{subfigure}
%        \vspace{0.5cm} % Space between the images
%        \begin{subfigure}{\textwidth}
%            \centering
%            \includegraphics[width=\linewidth]{images/texttosql.png}
%            \caption{GTR-T model}
%        \end{subfigure}
%        \caption{GTR and GTR-T models graphical abstracts}
%        \label{fig:general}
%    \end{figure}
%\end{graphicalabstract}



%%Research highlights
\begin{highlights}

\item The paper presents BEYONDWORDS, a novel generative AI-based framework for theme extraction, aiming to automatically extract and refine thematic structures from large-scale social media data
\item It integrates embeddings, autoencoder neural network for dimensionality compression, matrix factorization, and Chain of Thought prompting with Large Language Models(LLMS) to recursively uncover and enhance the themes
\item The results show that the three main topics that are widely discussed in autistic community tweets are \textbf{social media content quality and engagement; advocacy for autistic rights and acceptance; mental health and well-being}
\end{highlights}

%% Keywords
\begin{keyword}
Named Entity Recognition \sep Large Language Models \sep Generative Artificial Intelligence \sep Medical Data Extraction \sep Prompts Engineering

%% keywords here, in the form: keyword \sep keyword

%% PACS codes here, in the form: \PACS code \sep code

%% MSC codes here, in the form: \MSC code \sep code
%% or \MSC[2008] code \sep code (2000 is the default)

\end{keyword}

\end{frontmatter}

%% Add \usepackage{lineno} before \begin{document} and uncomment 
%% following line to enable line numbers
%% \linenumbers

%% main text
%%

%% Use \section commands to start a section
\section{Introduction}
The rapid growth of social media has created vast repositories of user-generated content, offering unique insights into diverse communities and social phenomena. For researchers, social media platforms such as X(Twitter previously) provide rich datasets to study public discourse, opinions, and trends. However, the unstructured and high dimensional nature of social media data presents challenges for thematic analysis, as traditional methods, like manual coding or frequency-based approaches (e.g., TF-IDF), often fail to capture the semantic depth and context embedded in posts. These limitations call for more sophisticated techniques capable of handling large-scale textual data and uncovering latent themes that reflect the nuanced perspectives expressed online.

In recent years, advancements in natural language processing (NLP) have introduced embedding techniques powered by pre-trained language models, which enable the transformation of raw text into dense, high dimensional vector representations that encapsulate both semantic meaning and contextual information. These embeddings, when coupled with dimensionality compression techniques such as autoencoder neural network and matrix factorization, can be leveraged to identify underlying themes in vast corpora of social media posts. Furthermore, the integration of generative AI models offers a novel approach to extracting and articulating these themes, moving beyond surface-level keyword extraction to generate coherent and meaningful and accurate themes.

This paper presents a general methodology for thematic analysis of social media data that combines embedding-based representations, dimensionality reduction, and generative AI. The proposed framework consists of three key components: (1) extracting social media embeddings using pre-trained language models, (2) applying dimensionality compression through autoencoder neural network, matrix factorization to uncover latent themes, and (3) utilizing generative AI through LLMs with a Chain of Thought (CoT) prompting mechanism to extract and refine these themes. 

As a case study, this methodology is applied to the analysis of tweets from the autistic community. Social media platforms have become an important space for autistic individuals to share their experiences, challenges, and perspectives on autism, often bypassing traditional gatekeepers of discourse. However, manual thematic analysis of such posts is labor-intensive and subject to biases introduced by human coders. By automating the process through the integration of machine learning and AI techniques, this methodology aims to extract themes that offer a more comprehensive and nuanced understanding of the discourse within the autistic community.

The contributions of this paper are twofold. First, it proposes a scalable and adaptable framework for thematic analysis of large-scale social media datasets, addressing the limitations of existing methods. Second, it provides insights into the experiences and challenges of the autistic community, potentially supporting more specialized care for this community, which requires the utmost support. This case study demonstrates the practical applicability of the methodology. The approach can be adapted for thematic analysis in other contexts, making it a versatile framework for researchers studying online discourse across various social media platforms.

The paper is structured as follows: Section 2 reviews related work on social media analysis, Section 3 discusses the components of the proposed methodology in detail, Section 4 discusses the case study and the dataset used for testing the proposed framework and Section 5 presents the results of the case study. Section 6 concludes with an evaluation of the strengths and limitations of the approach and a discussion of future directions for research.



\section{Related work}\label{sec:LR}

Thematic analysis is a key qualitative method used to identify patterns and themes within textual data. As social media has become a rich source of public discourse, the ability to analyze this unstructured and large-scale data has become crucial. Traditional methods of thematic analysis, though valuable for small datasets, often struggle with the volume, diversity, and complexity of social media text. This literature review outlines the progression of thematic analysis methods, highlighting the limitations of traditional techniques and how recent advancements, particularly our proposed approach, fill these gaps.
\newline
\subsection{Traditional Thematic Analysis in Social Media Research}

Thematic analysis has long been used to explore qualitative data, with foundational guidelines established by \cite{braun2006using}. Their method involves a systematic, manual coding process that identifies recurring themes through a close reading of the text. While this method remains highly respected in qualitative research, its application to social media data is problematic due to the large volume and complexity of text involved \cite{braun2019reflecting}.
Manual coding becomes time-consuming and unfeasible when applied to datasets that may include millions of posts, such as Twitter data \cite{boyatzis1998transforming}. In such cases, traditional thematic analysis often requires a reduced sample size or relies on researchers' subjective judgments to identify salient themes. This introduces biases and limits the generalizability of the findings \cite{markham2017research}. Additionally, the informal, fast-paced nature of social media language—frequently composed of abbreviations, slang, and emotive content—makes manual coding less reliable, as it struggles to capture nuances or variations in context \cite{golder2014digital}.
Due to these limitations, researchers began exploring computational methods for thematic analysis. Early approaches like word frequency analysis and topic modeling with Latent Dirichlet Allocation (LDA) became popular tools to analyze large datasets \cite{blei2003latent}. LDA has been applied extensively in social media research, particularly in studies focused on public opinion, social movements, and political discourse \cite{nguyen2015sentiment}. However, LDA's assumption that topics are merely distributions of words across documents oversimplifies the language and cannot capture the nuances of meaning, sentiment, and context present in complex social media texts \cite{chuang2012termite}.
Moreover, topic modeling techniques such as LDA struggle with the diverse, informal nature of social media language, often resulting in themes that are overly broad or too fragmented to be meaningful \cite{hong2010empirical}. Our proposed methodology directly addresses these limitations by incorporating advanced techniques like tweet embeddings from pre-trained language models, enabling a more nuanced understanding of social media discourse by preserving contextual relationships between words \cite{devlin2018bert}.
\newline
\subsection{Advances in Machine Learning and NLP for Thematic Analysis}

Recent advancements in natural language processing (NLP) have enhanced thematic analysis techniques, allowing researchers to overcome some of the limitations posed by traditional methods. The introduction of transformer-based language models, such as BERT \cite{devlin2018bert}, GPT-2/3 \cite{brown2020language}, and others, has significantly improved the ability to understand semantic relationships in text. These models leverage vast amounts of training data to generate contextualized word embeddings, offering deeper insights into text than earlier models like LDA or TF-IDF \cite{mikolov2013distributed}.

Two main studies rely on topic modeling and sentiment analysis to understand themes related to autism on Twitter. The first study by \cite{corti2022social} uses Non-Negative Matrix Factorization (NMF) for topic modeling and sentiment analysis with AFiNN, while the second study \cite{gabarron2023autistic} applies the BERTopic model for clustering tweets and extracting themes using dimensionality reduction via UMAP. Despite their contributions, these approaches exhibit limitations. NMF, although useful for topic coherence, lacks the ability to capture deep linguistic nuances, particularly in social media language. The BERTopic approach, while more advanced, relies on static embeddings and bag-of-words methods that may overlook semantic richness and context within tweets, especially in complex discourse like the autism community’s.

The use of pre-trained language models in thematic analysis has been explored in recent studies. \cite{wu2022study} applied BERT embeddings to cluster social media posts about public health issues, demonstrating improved coherence in the resulting themes compared to LDA. Likewise, \cite{yin2020detecting} showed that using contextualized embeddings from language models improved the ability to detect latent topics in crisis communication data. These studies highlight the superior capacity of pre-trained models to handle social media’s evolving terminology and informal language structures.

However, while embeddings capture rich semantic information, high dimensional representations pose challenges for downstream analysis. Dimensionality reduction techniques like autoencoder neural network and matrix factorization have emerged as effective ways to reduce complexity while preserving thematic content. \cite{chauhan2024tracking} demonstrated the utility of autoencoder neural network in compressing embeddings for more efficient clustering in sentiment analysis, which improves both the scalability and interpretability of results.

Our methodology builds on these approaches by integrating autoencoder neural network with matrix factorization to uncover latent themes, retaining the rich semantic information encoded in the tweet embeddings while effectively reducing dimensionality. This combination ensures that nuanced themes can be extracted from large datasets without losing the contextual meaning captured by the language models.
\newline
\subsection{Generative AI and Iterative Theme Refinement}

Generative AI models, such as GPT-3, have shown promise in thematic analysis by automating tasks such as text summarization, classification, and thematic extraction \cite{brown2020language}. These models generate coherent and contextually appropriate text, which can be useful for extracting or refining themes from large datasets. For example, studies by \cite{dong2024understanding} explored generative models for summarizing social media content, underscoring their potential for condensing vast amounts of text into coherent themes.

Despite this potential, generative AI has rarely been used in a more iterative, refining role in thematic analysis. Previous studies have typically applied generative models for summarization or categorization, without refining or validating the themes through multiple stages of analysis. This paper's \cite{wanna2024topictag} use CoT prompting is limited in that it primarily focuses on improving topic labels through token-based features and manual filtering, without fully leveraging the iterative reasoning potential of CoT for deeper semantic understanding. Additionally, their approach to CoT is more narrowly applied to optimize prompt tuning rather than refining the thematic structure of the topics themselves.
This gap is addressed in our methodology by using an iterative CoT prompting mechanism, which allows for iterative theme extraction and refinement. This process ensures that the themes are coherent, contextually accurate, and relevant to the dataset.

Although recent advancements in NLP and machine learning have improved thematic analysis, several critical gaps remain. First, many existing models still struggle to capture the complexity and nuance of informal social media language, particularly in communities with specialized or evolving vocabularies, such as the autistic community. LDA and word frequency-based methods, while useful for structured data, oversimplify language and miss deeper contextual meanings \cite{chuang2012termite}.
Second, many machine learning methods lack iterative processes for refining and validating themes. Pre-trained language models and clustering methods have advanced theme extraction, but they do not inherently include mechanisms for ensuring thematic consistency or addressing ambiguous themes \cite{wu2022study,yin2020detecting}. Moreover, generative AI models applied to thematic analysis have mostly focused on single-pass theme identification without multiple stages of validation or refinement, leading to incomplete or inconsistent results and the LLMs have fixed context windows \cite{dong2024understanding} and are not able to ingest text beyond that which is why clustering needs to be done before feeding the results to LLMs.
% should this be part of methodology or good to have in literature review
The proposed methodology addresses these gaps by:
\begin{enumerate}
    \item Generating tweet embeddings from pre-trained language models to capture semantic relationships and nuances in social media language. This is crucial for analyzing discourse in specialized communities, like the autistic community, which is the focus of this case study.
    \item Applying dimensionality reduction by integrating autoencoder neural network and matrix factorization to reduce the complexity of embeddings without losing key thematic information while clustering the relevant social media posts together. This approach overcomes the limitations of high-dimensional data.
    \item To balance the high cost of token generation using LLMs with the need to retain meaningful cluster information, a sample size was chosen that is representative enough to capture the overall themes of the entire cluster while minimizing the data passed to the LLMs.
    \item Using generative AI for iterative refinement by employing agentic CoT prompting mechanism to iteratively extract and refine themes of the extracted clusters, ensuring that the identified themes are accurate and contextually relevant. The inclusion of a secondary LLM for quality control further enhances the reliability of the analysis, addressing common issues with theme consistency in automated methods.
\end{enumerate}
Table \ref{tab:lit_rev} summarizes previous research conducted to address the problem of social media theme extraction and highlights how the proposed methodology aims to fill the gaps identified in these studies.
\begin{table}[H]
\centering
\caption{Comparison of Thematic Analysis Methodologies for Social Media Research}
\label{tab:lit_rev}
\begin{adjustbox}{max width=\textwidth}
\begin{tabular}{>{\raggedright\arraybackslash}p{3.5cm} >{\raggedright\arraybackslash}p{3.5cm} >{\raggedright\arraybackslash}p{3.5cm} >{\raggedright\arraybackslash}p{3.5cm}}
\hline
\textbf{Methodology} & \textbf{Focus} & \textbf{Weaknesses} & \textbf{Advantage of Proposed Methodology} \\ \hline
Traditional Thematic Analysis \cite{braun2006using} & Manual coding of text  & Unfeasible for large datasets, prone to bias & Scalable embedding-based approach retains theme detail while handling large volumes \\ \hline
Latent Dirichlet Allocation (LDA) \cite{blei2003latent} & Probabilistic topic modeling for large datasets  & Overly broad themes, lacks nuanced context & Embedding-based clustering captures semantic nuance; maintains coherence \\ \hline
Non-Negative Matrix Factorization (NMF) \cite{corti2022social} & Topic coherence via dimensionality reduction  & Limited in capturing informal language nuance & Embeddings and autoencoder neural network refine themes with more linguistic context \\ \hline
BERTopic \cite{gabarron2023autistic} & Clustering and topic modeling using static embeddings & Loses contextual depth in dynamic social media language & BERT embeddings with CoT refine themes; more adaptable to language dynamics \\ \hline
Pre-trained Language Models (e.g., BERT) \cite{devlin2018bert} & Contextualized embeddings for semantic insights & High-dimensional output challenging for analysis & Dimensionality reduction (autoencoder) makes clustering efficient, preserves semantic richness \\ \hline
Generative AI \cite{dong2024understanding} & Theme extraction via summarization & Single-pass extraction; lacks iterative refinement & CoT prompting iteratively refines themes, ensures context relevance \\ \hline
\end{tabular}
\end{adjustbox}
\end{table}

\section{Methodology}\label{sec:method}
This research introduces BEYONDWORDS, an agentic generative AI-driven framework for extracting themes from social media. An overview of the methodology is shown in Figure \ref{abstract}
\begin{figure}[H]
	\centering
	\includegraphics[width=\textwidth]{BEYONDWORDSgraphicalabstract.jpg}
	\caption{Framework overview of BEYONDWORDS showing posts embedding, compression, matrix factorization, clustering and generative AI based extraction }\label{abstract}
\end{figure}

\subsection{Text Tokenization and Embeddings Extraction}

The process of tokenizing and extracting embeddings from text involves converting the raw text into numerical representations that capture both syntactic structure and semantic meaning. Tokenization serves as the initial step, where each tweet is split into a sequence of tokens. Given a tweet \( T \) consisting of \( n \) words \( w_1, w_2, \ldots, w_n \), tokenization can be represented as:

\begin{equation}
    T = \{w_1, w_2, \ldots, w_n\} \rightarrow \{t_1, t_2, \ldots, t_m\}
\end{equation}


where \( t_i \) represents each token after tokenization, and \( m \) is the total number of tokens generated, which depends on the tokenization method and vocabulary size of the embedding model. Most models use a subword-level tokenization, which splits words into smaller units (subwords), capturing even rare or compound words effectively.

After tokenization, each tweet is converted into a fixed-dimensional embedding vector by passing it through a pre-trained language model. These models use large-scale datasets to learn embeddings that encode semantic relationships between words and phrases. For instance, similar words or concepts are represented by vectors that are close to each other in the embedding space, thus capturing semantic similarity. Let \( E \) denote the embedding vector of a tweet, which can be formulated as:

\begin{equation}
    E = Embed\left(t_1, t_2, \ldots, t_m \right)
\end{equation}


where \( Embed \) is the function that processes the token representations and maps them to a high-dimensional space that captures the semantic features of the text.

Three models of different sizes were used to generate embeddings, each capturing semantic features with varying degrees of granularity and complexity. These embeddings have the following characteristics:

\begin{itemize}
    \item \textbf{all-MiniLM-L6-v2} (small) generates embeddings vectors of size 312. Its smaller architecture captures basic semantic relationships, fewer dimensions are less resource-intensive but may miss some subtleties in meaning.
    \item \textbf{bge-base-en-v1.5} (medium) provides embeddings of size 728. This intermediate model balances between efficiency and semantic detail, capturing moderate complexities in sentence structure and meaning.
    \item \textbf{bge-m3} (large) outputs embeddings of size 1024. With a larger number of parameters, it captures nuanced semantic relationships and fine-grained meanings, making it well-suited for tasks requiring high semantic accuracy but comes at a high cost compared to small or medium size representations.
\end{itemize}
Table \ref{tab:embedding_dimensions} summarizes the propoerties of embeddings creation models.

\begin{table}[H]
    \caption{Properties of each model used in text tokenization and embeddings extraction.}
    \centering
    \begin{tabular}{c c c c}
        \hline
        \textbf{Model} & \textbf{Size} & \textbf{Number of Parameters} & \textbf{Dimensions} \\
        \hline
        all-MiniLM-L6-v2 & Small & 23M & 312 \\
        bge-base-en-v1.5 & Medium & 109M & 728 \\
        bge-m3 & Large & 567M & 1024 \\
        \hline
    \end{tabular}
    \label{tab:embedding_dimensions}
\end{table}


The embedding vectors produced by these models capture semantic information by representing words and phrases in a multi-dimensional space, where similar meanings are close together. This first step allows for downstream tasks, such as clustering and classification as explained in subsequent sections.

\subsection{Embeddings Dimensionality Reduction with autoencoder}

To enhance interpretability and reduce computational complexity, embeddings generated from text are further processed through dimensionality reduction using autoencoder. An autoencoder neural network\cite{bank2023autoencoders} consists of two primary components: an encoder \( f_{\theta} \) and a decoder \( g_{\phi} \), parameterized by \( \theta \) and \( \phi \) respectively. Given an input embedding \( E \in \mathbb{R}^d \), the encoder transforms \( E \) through a series of non-linear mappings to a compressed latent representation \( Z \in \mathbb{R}^k \), where \( k < d \). Formally, the encoding process can be described as:

\begin{equation}
Z = f_{\theta}(E) = \sigma \left( W^{(l)} \sigma \left( W^{(l-1)} \cdots \sigma \left( W^{(1)} E + b^{(1)} \right) \cdots + b^{(l-1)} \right) + b^{(l)} \right)
\end{equation}

where \( W^{(i)} \) and \( b^{(i)} \) are the weight matrices and bias vectors for the \( i \)-th layer of the encoder, and \( \sigma \) denotes an activation function (e.g., ReLU or sigmoid). This encoding function learns a transformation that captures the essential structure of \( E \) in the lower-dimensional space \( \mathbb{R}^k \).

The decoder then reconstructs the input embedding \( E \) from the latent representation \( Z \) by applying the inverse transformation, approximating the original embedding through the following mapping:

\begin{equation}
\hat{E} = g_{\phi}(Z) = \sigma \left( W'^{(1)} \sigma \left( W'^{(2)} \cdots \sigma \left( W'^{(m)} Z + b'^{(m)} \right) \cdots + b'^{(2)} \right) + b'^{(1)} \right)
\end{equation}

where \( W'^{(j)} \) and \( b'^{(j)} \) represent the weight matrices and bias vectors of the \( j \)-th layer in the decoder. Here, \( m \) denotes the number of layers in the decoder, and \( \hat{E} \in \mathbb{R}^d \) is the reconstructed embedding that approximates the original \( E \).

The autoencoder neural network is trained by minimizing the reconstruction loss \( L(E, \hat{E}) \), typically formulated as mean squared error (MSE):

\begin{equation}
L(E, \hat{E}) = \frac{1}{d} \sum_{i=1}^{d} \left( E_i - \hat{E}_i \right)^2,
\end{equation}

where \( E_i \) and \( \hat{E}_i \) denote the \( i \)-th components of the original and reconstructed embeddings, respectively. This loss function encourages the model to learn a compact representation that captures the core features of the input embeddings while discarding noise.

Three different compression ratios were explored for the latent dimension \( k \): 1/2, 1/3, and 1/4. Each ratio was assessed to balance the trade-off between dimensionality reduction and reconstruction fidelity, with detailed results discussed in the Results section.

After training, the encoder \( f_{\theta} \) alone is utilized to project high-dimensional embeddings into their corresponding lower-dimensional representations \( Z \), enabling more efficient downstream processing, including Singular Value Decomposition (SVD) for latent theme extraction and clustering with k-means.



\subsection{Matrix Factorization and Clustering}

Matrix factorization using SVD \cite{stewart1993early} was applied to identify underlying themes within the tweet embeddings. Instead of using the original high-dimensional embeddings, the compressed embeddings from the autoencoder neural network were utilized as input. This approach significantly reduces computational cost while preserving essential semantic features, as the dimensionality reduction effectively discards non-informative components.

Given a compressed embedding matrix \( C \in \mathbb{R}^{n \times k} \), where \( n \) is the number of tweet embeddings and \( k \) is the compressed dimension, SVD decomposes \( C \) into three matrices:

\begin{equation}
C = U \Sigma V^T
\end{equation}

where \( U \in \mathbb{R}^{n \times r} \) and \( V \in \mathbb{R}^{k \times r} \) are orthogonal matrices, \( \Sigma \in \mathbb{R}^{r \times r} \) is a diagonal matrix of singular values, and \( r \) represents the rank of \( C \). The columns of \( U \) correspond to the principal components capturing the main themes within the tweet embeddings.

The importance of matrix factorization in this study lies in its ability to distill complex, high-dimensional data into a more manageable form while retaining the most significant features that represent underlying patterns and themes. By decomposing the compressed embeddings, SVD allows for the identification of latent structures that can reveal insights into the thematic content of the social media posts.

To group similar posts based on these themes, \( k \)-means clustering was applied to the reduced representation of the embeddings. This clustering helps in identifying coherent groups of posts, facilitating the extraction of distinct themes. The combination of SVD and \( k \)-means clustering \cite{kodinariya2013review} provides a computationally efficient method to uncover and organize thematic patterns in the data, providing essential component of the proposed approach in this research.
The equation that models this proposed methodology of clustering is expressed as follows:
\begin{equation}
C_{\text{posts}} = \operatorname{k\text{-means}}\left( U, k \right) \quad 
\end{equation}
\begin{equation}
\text{where}  \quad 
U = \operatorname{SVD}_U \left( f_{\theta^*} \left( \operatorname{Embed} \left( \bigcup_{i=1}^N \{t_{i1}, t_{i2}, \dots, t_{im_i}\} \right) \right) \right)
\end{equation}

\begin{equation}
\text{and}  \quad 
\theta^* = \operatorname{argmin}_{\theta} \sum_{i=1}^N L \left( E_i, g_{\phi} \left( f_{\theta}(E_i) \right) \right)
\end{equation}
where the term \(\theta^*\) represents the selection of the optimal compression ratio \(\theta\) that minimize the reconstruction loss \(L(E_i, \hat{E}_i)\) across all embeddings \(E_i\). Here, \(f_{\theta}\) is the encoder function, and \(g_{\phi}\) is the decoder function that reconstructs \(E_i\) from its compressed representation, aiming to minimize the difference between \(E_i\) and its reconstruction \(\hat{E}_i\).


Algorithm \ref{alg:part1} explains the steps performed for clustering the tweets using the proposed approach to prepare them for thematic analysis using LLMs. 

\begin{algorithm}[H]
    \caption{Embedding Extraction and Compression, Matrix Factorization and Clustering}
    \label{alg:part1}
    \nolinenumbers
    \begin{algorithmic}
        \State \textbf{Input:} Tweets dataset $T$, Compression ratios $CR$, Number of clusters $k$
        \State \textbf{Output:} Clustered posts $C_{posts}$

        \State \text{1. Text Tokenization:}
        \State \quad For each tweet \( T_i \in T \):
        \State \quad \quad $T_i = \{w_1, w_2, \ldots, w_n\} \rightarrow \{t_1, t_2, \ldots, t_m\}$

        \State \text{2. Embedding Extraction:}
        \For{each model in \{X, Y, Z\}}
            \State \quad $E_i \gets \operatorname{Embed}(t_1, t_2, \ldots, t_m)$
        \EndFor

        \State \text{3. Dimensionality Reduction with Autoencoders:}
        \For{each compression ratio $\theta \in CR$}
            \State \quad $Z_i^{(\theta)} = f_{\theta}(E_i)$ \quad \text{(Encode)}
            \State \quad $\hat{E}_i^{(\theta)} = g_{\phi_\theta}(Z_i^{(r)})$ \quad \text{(Decode)}
            \State \quad Calculate $L(E_i, \hat{E}_i^{(r)})$
        \EndFor
        \State \quad Select $\theta^* = \operatorname{argmin}_{\theta \in CR} \sum_{i} L(E_i, \hat{E}_i^{(r)})$
        \State \quad Set $Z_i = Z_i^{(\theta^*)}$ for each $i$

        \State \text{4. Matrix Factorization (SVD):}
        \State \quad $C \gets \text{Compressed Embedding Matrix for selected } \theta^*$
        \State \quad Decompose $C$ using SVD: $C = U \Sigma V^T$

        \State \text{5. Clustering:}
        \State \quad $C_{posts} \gets k\text{-means}(U, k)$ \quad \text{(Cluster posts)}
        
    \end{algorithmic}
\end{algorithm}


\subsection{Generative AI for Themes Extraction}
To effectively manage the analysis of social media posts, a two-step sampling strategy was employed. It is based on the Cochran formula to determine an appropriate sample size from each cluster. This approach allowed conducting a focused thematic analysis while adhering to the limitations of generative AI regarding context windows and computational costs.
The Cochran formula for sample size determination is given by:

\begin{equation}
n = \frac{Z^2 \cdot p \cdot (1 - p)}{e^2}
\end{equation}
Where: \(n\) is the sample size, \(Z\) is the Z-score corresponding to the desired confidence level (e.g., 1.64 for a 90\% confidence level), \(p\) is the estimated proportion of the population (we use \(0.5\) for maximum sample size), \(e\) is the margin of error (the desired level of precision).

Using this formula, the sample size \(n\) for each cluster was calculated. For a 90\% confidence the final sample size is determined to be 68. This statistically representative sample enables the model to apply the thematic extraction methodology efficiently without processing the entire batch of posts.
Following the determination of the sample size, the selection was further improved by employing the silhouette score to identify the top \(n\) texts with the highest cohesion and separation from other clusters. The silhouette score is defined as:

\begin{equation}
s(i) = \frac{b(i) - a(i)}{\max\{a(i), b(i)\}}
\label{silhouette}
\end{equation}
Where: \(s(i)\) is the silhouette score for the \(i\)-th data point, \(a(i)\) is the average distance between the \(i\)-th data point and all other points in the same cluster, \(b(i)\) is the average distance between the \(i\)-th data point and all points in the nearest cluster.

By calculating the silhouette score for all posts within each cluster, the texts selected are the highest scorers, indicating that they are well-clustered and representative of their themes. 

\subsubsection{CoT Strategy}

The CoT strategy serves as a pivotal component of the proposed methodology, guiding the generative process through structured and sequential tasks. The process can be detailed as follows:

\begin{enumerate}
    \item The LLM is prompted to identify significant keywords and phrases within the tweets, focusing on content that reflects important topics and sentiments.
    \item The extracted keywords are organized into coherent groups based on common themes, topics, or ideas, enhancing clarity and coherence.
    \item For each category, the LLM work as a specialized agent in thematic extraction, synthesizes high-level themes, providing concise descriptions that encapsulate the essence of the discussions in the tweets.
\end{enumerate}

\subsubsection{Recursive Theme Refinement via LLM Feedback Mechanism}

To further refine the theme extraction process, a recursive feedback mechanism involving a second agent LLM is implemented. This agent acts as a grading system, following these steps:

\begin{enumerate}
    \item The secondary LLM evaluates the themes generated by the primary LLM against predefined quality thresholds.
    \item If the themes do not meet the established criteria, feedback is relayed back to the primary LLM.
    \item The primary LLM utilizes the feedback (score + comment) to reevaluate the extraction process, revisiting the previous steps as necessary.
    \item This cycle of evaluation and adjustment continues, improving the quality of theme extraction with each iteration until either the threshold score or the maximum number of iterations are reached.
\end{enumerate}
    
This iterative refinement mimics the Generative Adversarial Networks (GANs) paradigm \cite{goodfellow2020generative}, where a generator produces outputs while a discriminator assesses their quality. In this context, the generator is the primary LLM that generates thematic outputs, while the discriminator is the secondary LLM that critiques these outputs. Since the discriminator LLM provides the score and feedback as part of a sentence, a mechanism is needed to extract only the score and accurate feedback. This is essential to isolate the components required to decide whether the extracted theme is good enough by comparing the score with a threshold, or to pass only these two elements to the first step to redo the extraction process, taking into consideration the new score and feedback from the discriminator. To serve this purpose, we use an LLM-based entity extractor model \cite{ghali2024gamedx}. This discourse between LLMs ensures continuous improvement in theme extraction, allowing the methodology to adapt and enhance its performance with each iteration. By combining structured prompting with feedback-driven refinement. 

The following equation describes the recursive refinement strategy proposed by modeling the theme extraction process for cluster \( k \), where \( T_{\text{themes}, k} \) is the final set of themes. It uses the indicator function \( \mathbb{I} \) to choose between refining the initial themes, \( T_{\text{initial}} \), or using them directly, based on the evaluation score. If the score is below the threshold \( Q \), the themes are refined using feedback through \( \mathcal{R}(T, \text{feedback}) \), otherwise, the initial themes are selected. The scoring function \( \mathcal{S}(\text{score}(T)) \) evaluates the quality of the themes, while \( \mathcal{M}_1 \) and \( \mathcal{M}_2 \) represent the language models used for theme extraction and evaluation, respectively.
\begin{equation}
T_{\text{themes}, k} = 
\arg \max_{T} \left[ \mathbb{I}_{\left( \mathcal{S}(\text{score}(T)) < Q \right)} \cdot \mathcal{R}(T, \text{feedback}) + \mathbb{I}_{\left( \mathcal{S}(\text{score}(T)) \geq Q \right)} \cdot T_{\text{initial}} \right]
\end{equation}

\(
\text{where} \quad 
T_{\text{initial}} = \mathcal{M}_1 \left( C_{\text{posts}, k} \right)
\quad \text{and} \quad
\left( \text{score}, \text{feedback} \right) = \mathcal{M}_2 \left( T_{\text{initial}} \right)
\)


Algorithm \ref{alg:genai} summarizes in details the steps to extract the themes:
\begin{algorithm}[H]
    \caption{Agentic CoT Model for Theme Extraction}
    \label{alg:genai}
    \nolinenumbers
    \begin{algorithmic}
        \State \textbf{Input:} Clustered posts $C_{posts,k}$
        \State \textbf{Output:} Extracted themes for each cluster $k$: $T_{themes,k}$
        
        \For{each cluster $k$ in $C_{posts,k}$}
            \State \text{1. Sample Selection:}
            \State \quad Calculate sample size $n$ using Cochran's formula
            \State \quad Sample $n$ posts from cluster $k$

            \State \text{2. Initial Theme Extraction:}
            \State \quad $T_{initial} \gets \text{LLM}_1.extract\_themes(C_{posts,k})$

            \State \text{3. Quality Evaluation:}
            \State \quad $score, feedback \gets \text{LLM}_2.evaluate\_and\_extract(T_{initial})$
            
            \If{$score < Q$}
                \State \text{4. Feedback and Refinement:}
                \State \quad $T_{refined} \gets \text{LLM}_1.refine\_themes(T_{initial}, feedback)$
                \State \quad $T_{themes,k} \gets T_{refined}$
            \Else
                \State \quad $T_{themes,k} \gets T_{initial}$
            \EndIf
        \EndFor
        
        \State \text{5. Output Extracted Themes:} $T_{themes,k}$ for each cluster $k$
    \end{algorithmic}
\end{algorithm}



\section{Autism Case Study}

\subsection{Dataset Overview}

The dataset used for this study comprises a selection of tweets from individuals who identify as autistic, using the hashtag \#actuallyautistic to share their experiences, insights, and perspectives. This hashtag represents a movement within the autism community that seeks to amplify the voices of autistic individuals offering a more personal and direct narrative on living with autism. 
The original dataset was gathered using \texttt{snscrape}, a Python-based scraping tool, to extract tweets from X(Twitter previously) between January 2014 and December 2022. The dataset includes tweets from individuals who self-identified as autistic through keywords such as “autism,” “autistic,” or “neurodiverse” in their profiles, usernames or tweets. This initial dataset consists of approximately 3.1 million tweets from 17,323 unique users, with additional metadata on usernames, account creation dates, and other relevant features.
Figure \ref{datapreprocessing} presents an overview of the preprocessing process.

\begin{figure}[H]
	\centering
	\includegraphics[width=\textwidth]{datapreprocessing.jpg}
	\caption{Overview of the dataset preprocessing used in this research }\label{datapreprocessing}
\end{figure}
\subsection{Dataset Preprocessing}
From the original dataset \cite{jaiswal2023actuallyautistic}, only tweets that contained \#actuallyautistic hashtag with its different textual variations were retained to ensure relevance to autistic individuals. Additionally, tweets were filtered to include only those written in English, resulting in a subset of approximately 200,000 tweets for further analysis. To prepare the data for processing, a text-cleaning script was applied to each tweet, removing URLs, mentions, hashtags, and special characters. This preprocessing step created a refined dataset that maintains linguistic consistency and readability for subsequent analysis.

\section{Experiments Results and Discussion}
\subsection{Dimensionality Reduction Analysis}
This section presents the findings from experiments conducted on autoencoder neural network models designed for dimensionality reduction of text embeddings. The primary goal is to evaluate the performance of three distinct model architectures—small, medium, and large—across varying levels of compression. Specifically, the analysis focuses on how each model performs with three different compression ratios: 1/2, 1/3, and 1/4 of the original embedding dimensionality. The ultimate goal is to identify the model that minimizes information loss, ensuring the highest possible accuracy for subsequent SVD and clustering using the minimum computations possible.

\begin{figure}[H]
	\centering
	\includegraphics[width=\textwidth]{autoencoder_loss_combined.jpg}
	\caption{Autoencoder loss plot for a) small, b) medium, and c) large embedding models}\label{loss}
\end{figure}
Figure \ref{loss} presents the validation loss across training epochs for the autoencoder model. For the small embedding size (384 dimensions), as shown in Figure \ref{loss}a, the 1/2 compression ratio achieves the lowest validation loss, indicating that halving the dimensionality retains the essential features for accurate reconstruction, balancing information preservation and reduced computational complexity. For the medium embedding size (768 dimensions), highlighted in Figure \ref{loss}b, the 1/4 compression ratio is optimal, suggesting that medium-sized embeddings tolerate higher compression while still enabling precise reconstruction. This indicates that the medium model is more resilient to dimensionality reduction, allowing for more efficient computation in subsequent SVD and clustering steps. For the large embedding size (1024 dimensions), as shown in Figure \ref{loss}c, the best performance is again at the 1/2 compression ratio, reflecting that halving the dimensionality remains optimal for capturing necessary data patterns despite the larger embedding size. Overall, the analysis highlights the autoencoder’s ability to retain critical information across all embedding sizes and compression ratios, as evidenced by the convergence of validation loss to low values. The preservation of embedding structure post-compression is crucial for accurate clustering with k-means, enabling efficient theme extraction with minimal information loss.




\subsection{Matrix Factorization and Clustering}
%%%%%%%%%%%%%%%SVD%%%%%%%%%%%%%%%
\begin{figure}[H]
	\centering
	\includegraphics[width=\textwidth]{svdcombined.jpg}
	\caption{SVD plot for a) small, b) medium, and c) large embedding models}\label{svdcombined}
 
\end{figure}

The results of the SVD analysis, illustrated in Figure \ref{svdcombined}, demonstrate the effectiveness of the proposed dimensionality reduction methodology for embeddings of varying sizes (small, medium, and large). Following the initial compression achieved by autoencoders, SVD was applied to further reduce the embeddings' dimensionality, facilitating efficient clustering and theme extraction. This step is particularly crucial in the methodology, as matrix factorization not only enhances interpretability but also optimizes computational resources, making it well-suited for large-scale social media posts analysis.
In Figure \ref{svdcombined}a, which corresponds to the small-size embeddings, the cumulative explained variance approaches 90\% with approximately 100 components. This finding suggests that SVD effectively captures the majority of the data’s variance, while significantly reducing the dimensional space. 
For medium-size embeddings, shown in Figure \ref{svdcombined}b, a similar pattern emerges. The cumulative explained variance also reaches close to 90\% with fewer than 100 components. This consistency across both small and medium embeddings indicates that SVD is highly effective in distilling essential features, even as the embedding size increases.
Figure \ref{svdcombined}c presents the results for large-size embeddings, where approximately 150 components are required to capture a comparable level of variance. This increase in required components is expected, as larger embeddings inherently contain more complex information. Despite this, the curve stabilizes, confirming that SVD provides a robust means of reducing even high-dimensional embeddings to a manageable form, preserving critical information for clustering.

%%%%%%%%%%%%%%%%%%%%%%%%%%%%%%%%%%%

%%%%%%%%%%%%%%%Elbow%%%%%%%%%%%%%%%
\subsection{Clustering (combine all 6 pictures in one big picture)}

The elbow method (Figure \ref{combinedclusters}) was used to determine the optimal number of clusters for latent themes derived via SVD from compressed embeddings. For all embedding sizes (Figures \ref{combinedclusters}a, \ref{combinedclusters}b, \ref{combinedclusters}c), the elbow point consistently appears at 3 clusters suggesting that additional clusters would contribute minimal new information, confirming that 3 clusters capture the primary themes in the data.
%%%%%%%%%%%%%%%%%%%%%%%%%%%%%%%%%%%
\begin{figure}[H]
	\centering
	\includegraphics[width=\textwidth]{combined_clusters.jpg}
	\caption{Elbow and k-means themes clusters plot for a) small, b) medium, and c) large embedding models}\label{combinedclusters}
\end{figure}
%%%%%%%%%%%%%%%clusters%%%%%%%%%%%%%%%
The clustering results demonstrate the robustness and scalability of the proposed methodology. Across all embedding sizes, three distinct clusters are consistently formed, each reflecting well-separated themes within the social media data. 
In Figure \ref{combinedclusters}a, the small-sized embeddings display clear separation among the clusters, with each color representing a unique latent theme. The compact arrangement of points within each cluster suggests that the themes identified are coherent and well-differentiated even with a limited embedding dimensionality. Figure \ref{combinedclusters}b, representing medium-sized embeddings, shows a similar pattern, with clusters that retain distinct boundaries and minimal overlap. This demonstrates that the clustering method is adaptable, producing stable clusters as the embedding dimensionality increases.
Figure \ref{combinedclusters}c, showing results for the large-sized embeddings, presents a particularly compelling case for the methodology's effectiveness. The clusters remain well-defined, with minimal inter-cluster overlap, despite the increase in data complexity and embedding dimensionality. The large embedding size allows for capturing more nuanced details in the data, yet the clusters continue to exhibit clear separation, indicating that the approach is capable of handling high-dimensional representations while preserving theme clarity.
The consistency of cluster shapes and separation across embedding sizes reinforces the suitability of this methodology for identifying latent themes in social media data. The SVD step plays a pivotal role in this outcome by reducing dimensionality while retaining essential information, enabling k-means clustering to form coherent clusters that are robust to changes in embedding size.

%%%%%%%%%%%%%%%%%%%%%%%%%%%%%%%%%%%%%%%%%%%%%%%%%%%%%%

%%%%%%%%%%%%%%%%%%%%quality of clusters%%%%%%%%%%%%%%%%

To accurately measure the quality of the clustering results after applying the proposed methodology, and compare the effectiveness of using autoencoder neural network for dimensionality reduction in text embedding compression, it was essential to conduct an analysis with and without autoencoder-based compression across different embedding sizes. This analysis allows for a deeper understanding of how compression impacts clustering quality, which is crucial when applying subsequent dimensionality reduction techniques like SVD for latent theme extraction and clustering. To evaluate the clustering performance, three metrics were considered:
\textbf{Calinski-Harabasz Index (CH Index):} Measures the ratio of the sum of between-cluster dispersion to within-cluster dispersion. Higher values indicate better-defined clusters.
   \begin{equation}
   \text{CH Index} = \frac{\text{Tr}(B_k)}{\text{Tr}(W_k)} \cdot \frac{N - k}{k - 1}
   \end{equation}
   where \( \text{Tr}(B_k) \) is the trace of the between-cluster dispersion matrix, \( \text{Tr}(W_k) \) is the trace of the within-cluster dispersion matrix, \( N \) is the total number of points, and \( k \) is the number of clusters.
\newline
\newline
\textbf{Davies-Bouldin Index (DB Index):} Measures the average "similarity" ratio of within-cluster distances to between-cluster distances. Lower values indicate better clustering quality.
   \begin{equation}
   \text{DB Index} = \frac{1}{k} \sum_{i=1}^{k} \max_{j \neq i} \left( \frac{\sigma_i + \sigma_j}{d(c_i, c_j)} \right)
   \end{equation}
   where \( \sigma_i \) and \( \sigma_j \) are the within-cluster distances for clusters \( i \) and \( j \), respectively, and \( d(c_i, c_j) \) is the distance between cluster centers \( c_i \) and \( c_j \).
   \newline
   \newline
\textbf{Silhouette Score:} Reflects the compactness and separation of clusters. It ranges from -1 to 1, where higher values indicate well-separated clusters as explained in details in Equation \ref{silhouette}.
\newline
The tables below summarizes the clustering metrics results for different embedding sizes with and without autoencoder-based compression.

\begin{table}[H]
\centering
\small
\caption{Clustering metrics for small size embeddings}
\begin{tabular}{lcc}
\hline
\textbf{Metrics} & \textbf{With Autoencoder} & \textbf{Without Autoencoder} \\
                 & \textbf{Compression}      & \textbf{Compression} \\
\hline
CH Index & 24991 & 5161 \\
DB Index & 2.93 & 8.22 \\
Silhouette Score & 0.04 & 0.04 \\
\hline
\end{tabular}
\label{table:small}
\end{table}

\begin{table}[H]
\centering
\small
\caption{Clustering metrics for medium size embeddings}
\begin{tabular}{lcc}
\hline
\textbf{Metrics} & \textbf{With Autoencoder} & \textbf{Without Autoencoder} \\
                 & \textbf{Compression}      & \textbf{Compression} \\
\hline
CH Index & 11806 & 11879 \\
DB Index & 4.01 & 3.80 \\
Silhouette Score & 0.06 & 0.06 \\
\hline
\end{tabular}
\label{table:medium}
\end{table}

\begin{table}[H]
\centering
\small
\caption{Clustering metrics for large size embeddings}
\begin{tabular}{lcc}
\hline
\textbf{Metrics} & \textbf{With Autoencoder} & \textbf{Without Autoencoder} \\
                 & \textbf{Compression}      & \textbf{Compression} \\
\hline
CH Index & 366243 & 6235 \\
DB Index & 0.62 & 5.22 \\
Silhouette Score & 0.48 & 0.04 \\
\hline
\end{tabular}
\label{table:large}
\end{table}


The clustering metrics presented in Tables \ref{table:small}, \ref{table:medium}, and \ref{table:large} serve as the final quality assessment of the methodology for extracting themes from text embeddings. This analysis follows the dimensionality reduction via autoencoder, SVD for latent theme extraction, and clustering with k-means. The results validate the effectiveness of this multi-step approach in capturing meaningful structure within the data.
The high CH Index values and low DB Index values achieved, particularly in the configurations with autoencoder neural network compression, demonstrate the ability of this methodology to create well-separated and compact clusters, indicative of high clustering quality. For instance, in the large embedding size, the CH Index reaches 366243 with a DB Index of 0.62 when autoencoder neural network are used, signaling exceptionally distinct clusters. This trend is consistent across embedding sizes, with the small embedding size achieving a CH Index of 24991 and a DB Index of 2.93, and the medium size showing similarly favorable metrics. 
Moreover, the silhouette scores across the tests further confirm the consistency and stability of the clustering structure. These metrics collectively indicate that the clusters generated are not only compact and well-separated but also robust in representing latent themes within the data as evidenced by this final quality assurance test. 

%%%%%%%%%%%%%%%%%%%%%%%%%%%%%%%%%%%%%%%%%%%%%%%%%%%%%%%
\subsection{Generative AI Theme Extraction Analysis}
In the thematic extraction phase, large embedding model's themes clusters were used as it achieved the lowest loss value and highest clustering quality across two out of three evaluation metrics. GPT -4o mini model was used for both the primary extractor (LLM1) and evaluator (LLM2) roles as it is known for its lightweight design, per token cost effectiveness and strong performance across major evaluation benchmarks \cite{openai2024gpt4ocard}. This analysis section presents both word cloud visualizations of keywords and Sankey diagrams illustrating the connections between keywords and their coherent groups, providing a comprehensive view of the relationships identified to further examine and validate the final themes extracted.

Through the structured CoT and recursive theme refinement strategy explained in details in methodology section, the LLM1 first identified significant keywords and phrases in the tweets text, focusing on elements that reflect key topics and sentiments. These keywords were then organized into multiple coherent groups for each cluster yielding at the end the final high-level themes:

\begin{itemize}
    \item \textbf{Social media content quality and engagement:} This theme contains keywords related to the quality and engagement potential of social media content, such as "Excellent content," "Highly recommended," and "Promising article." The associated Sankey diagram (Figure \ref{theme1}b) illustrates the connections between these engagement-oriented terms, while the word cloud (Figure \ref{theme1}a) highlights the prominence of keywords within this theme.
    
    \item \textbf{Advocacy for autistic rights and acceptance:} This theme emphasizes the importance of representation, empowerment, and critiques of dominant autism narratives. Keywords like "Autistic civil rights," "Neurodiversity," and "Acceptance and love" underscore the advocacy focus within this group. Figure~\ref{theme2}b provides the Sankey diagram showing how terms in this theme are interrelated, and Figure~\ref{theme2}a displays the word cloud, showcasing the relevance of advocacy-related terms.

    \item \textbf{Mental health and well-being:} This theme captures mental health challenges and coping strategies, with keywords like "Burnout," "Executive dysfunction," "Coping strategies," and "Isolation." The Sankey diagram (Figure~\ref{theme3}b) shows connections between terms related to mental health, while Figure~\ref{theme3}a displays a word cloud of keywords within this theme, underscoring the emphasis on managing mental and emotional challenges.
\end{itemize}

The themes revealed by LLM1 were iteratively evaluated by the LLM2 until they reached the predefined quality acceptance criteria. If the themes did not meet these criteria, feedback was provided to LLM1 for adjustments. This recursive feedback loop significantly enhanced the clarity and coherence of the identified themes, allowing for an even better and accurate thematic extraction as evidenced by the final themes above.

\begin{figure}[H]
	\centering
	\includegraphics[width=\textwidth]{theme1.jpg}
	\caption{LLMs based thematic analysis for theme 1: a) word cloud of keywords extracted by LLM1, b) Sankey diagram of keywords extracted and their coherent groups clustering}\label{theme1}
\end{figure}

\begin{figure}[H]
	\centering
	\includegraphics[width=\textwidth]{theme2.jpg}
	\caption{LLMs based thematic analysis for theme 2: a) word cloud of keywords extracted by LLM1, b) Sankey diagram of keywords extracted and their coherent groups clustering}\label{theme2}
\end{figure}

\begin{figure}[H]
	\centering
	\includegraphics[width=\textwidth]{theme3.jpg}
	\caption{LLMs based thematic analysis for theme 3: a) word cloud of keywords extracted by LLM1, b) Sankey diagram of keywords extracted and their coherent groups clustering}\label{theme3}
\end{figure}

\subsection{Discussion}

\subsubsection{Thematic Study Insights}
The thematic analysis provided valuable insights across the three primary areas, highlighting interesting discussions and sentiments within the social media content of autistic people.

The first extracted theme \textbf{social media content quality and engagement} reveals high engagement indicators such as "Excellent content," "Promising article," and "Highly recommended," indicating a strong focus on content quality. This underscores the value placed on credible, high-quality resources in social media conversations and what it means to autistic people.
    
The second theme \textbf{advocacy for autistic rights and acceptance} with keywords such as "Autistic civil rights," "Neurodiversity," and "Acceptance and love" reflect the community's emphasis on representation and self-determined identity. This theme highlights the role of social media in autistic advocacy, providing a platform to advocate for acceptance, critique organizations (e.g., “Autism Speaks criticism”), and promote neurodiversity.
    
The last extracted theme \textbf{mental health and well-being} groups keywords around mental health challenges and coping strategies, such as "Burnout," "Executive dysfunction," and "Coping strategies." The keywords highlight the community's struggles with mental and emotional well-being, as well as the importance of resilience and coping mechanisms, particularly in the context of neurodiverse experiences.

\subsubsection{Implications for Autism Advocacy and Informed Decision-Making}
The identified themes have implications for both advocacy and informed decision-making. For example, the second theme reveals community-driven discussions about identity-first language and critiques of existing narratives, guiding policymakers and advocacy groups to better align their communication and policies with community values.

In addition, the third theme underscores the need for targeted mental health resources for autistic individuals, pointing to the importance of tailored interventions for this population. These insights support the goal of using data-driven themes to foster inclusivity and improve access to relevant resources without neglecting the fact that they value quality content on social media as depicted by the first theme.



\section{Conclusion}
This paper presents BEYONDWORDS, a novel methodology for extracting latent themes from social media posts, with a specific focus on content related to autistic individuals. The approach integrates embeddings, dimensionality reduction, SVD for theme extraction, K-means clustering, and an agentic generative AI model with iterative feedback. This pipeline successfully identified three primary themes: \textbf{social media content quality and engagement}, \textbf{advocacy for autistic rights and acceptance}, and \textbf{mental health and well-being}. The thematic analysis provides valuable insights, revealing high engagement with quality content, a strong emphasis on autistic rights and acceptance, and the importance of mental health and coping strategies within the autistic community. While the methodology offers a comprehensive and scalable approach to theme extraction, it is not without limitations. The complexity of the multistep process and the dependency on high-quality embedding models are notable challenges. Additionally, potential biases in the AI models and the sensitivity of SVD and K-means to parameter tuning require careful consideration.
Future work will focus on addressing these limitations by developing techniques to mitigate biases, exploring real-time processing capabilities, and incorporating user feedback mechanisms to further refine the accuracy and applicability of the approach. These enhancements will make the methodology a robust tool for social media analysis, particularly in understanding and supporting the autistic community.

\newpage

% This must be in the first 5 lines to tell arXiv to use pdfLaTeX, which is strongly recommended.
\pdfoutput=1
% In particular, the hyperref package requires pdfLaTeX in order to break URLs across lines.

\documentclass[11pt]{article}

% Change "review" to "final" to generate the final (sometimes called camera-ready) version.
% Change to "preprint" to generate a non-anonymous version with page numbers.
\usepackage{acl}

% Standard package includes
\usepackage{times}
\usepackage{latexsym}

% Draw tables
\usepackage{booktabs}
\usepackage{multirow}
\usepackage{xcolor}
\usepackage{colortbl}
\usepackage{array} 
\usepackage{amsmath}

\newcolumntype{C}{>{\centering\arraybackslash}p{0.07\textwidth}}
% For proper rendering and hyphenation of words containing Latin characters (including in bib files)
\usepackage[T1]{fontenc}
% For Vietnamese characters
% \usepackage[T5]{fontenc}
% See https://www.latex-project.org/help/documentation/encguide.pdf for other character sets
% This assumes your files are encoded as UTF8
\usepackage[utf8]{inputenc}

% This is not strictly necessary, and may be commented out,
% but it will improve the layout of the manuscript,
% and will typically save some space.
\usepackage{microtype}
\DeclareMathOperator*{\argmax}{arg\,max}
% This is also not strictly necessary, and may be commented out.
% However, it will improve the aesthetics of text in
% the typewriter font.
\usepackage{inconsolata}

%Including images in your LaTeX document requires adding
%additional package(s)
\usepackage{graphicx}
% If the title and author information does not fit in the area allocated, uncomment the following
%
%\setlength\titlebox{<dim>}
%
% and set <dim> to something 5cm or larger.

\title{Wi-Chat: Large Language Model Powered Wi-Fi Sensing}

% Author information can be set in various styles:
% For several authors from the same institution:
% \author{Author 1 \and ... \and Author n \\
%         Address line \\ ... \\ Address line}
% if the names do not fit well on one line use
%         Author 1 \\ {\bf Author 2} \\ ... \\ {\bf Author n} \\
% For authors from different institutions:
% \author{Author 1 \\ Address line \\  ... \\ Address line
%         \And  ... \And
%         Author n \\ Address line \\ ... \\ Address line}
% To start a separate ``row'' of authors use \AND, as in
% \author{Author 1 \\ Address line \\  ... \\ Address line
%         \AND
%         Author 2 \\ Address line \\ ... \\ Address line \And
%         Author 3 \\ Address line \\ ... \\ Address line}

% \author{First Author \\
%   Affiliation / Address line 1 \\
%   Affiliation / Address line 2 \\
%   Affiliation / Address line 3 \\
%   \texttt{email@domain} \\\And
%   Second Author \\
%   Affiliation / Address line 1 \\
%   Affiliation / Address line 2 \\
%   Affiliation / Address line 3 \\
%   \texttt{email@domain} \\}
% \author{Haohan Yuan \qquad Haopeng Zhang\thanks{corresponding author} \\ 
%   ALOHA Lab, University of Hawaii at Manoa \\
%   % Affiliation / Address line 2 \\
%   % Affiliation / Address line 3 \\
%   \texttt{\{haohany,haopengz\}@hawaii.edu}}
  
\author{
{Haopeng Zhang$\dag$\thanks{These authors contributed equally to this work.}, Yili Ren$\ddagger$\footnotemark[1], Haohan Yuan$\dag$, Jingzhe Zhang$\ddagger$, Yitong Shen$\ddagger$} \\
ALOHA Lab, University of Hawaii at Manoa$\dag$, University of South Florida$\ddagger$ \\
\{haopengz, haohany\}@hawaii.edu\\
\{yiliren, jingzhe, shen202\}@usf.edu\\}



  
%\author{
%  \textbf{First Author\textsuperscript{1}},
%  \textbf{Second Author\textsuperscript{1,2}},
%  \textbf{Third T. Author\textsuperscript{1}},
%  \textbf{Fourth Author\textsuperscript{1}},
%\\
%  \textbf{Fifth Author\textsuperscript{1,2}},
%  \textbf{Sixth Author\textsuperscript{1}},
%  \textbf{Seventh Author\textsuperscript{1}},
%  \textbf{Eighth Author \textsuperscript{1,2,3,4}},
%\\
%  \textbf{Ninth Author\textsuperscript{1}},
%  \textbf{Tenth Author\textsuperscript{1}},
%  \textbf{Eleventh E. Author\textsuperscript{1,2,3,4,5}},
%  \textbf{Twelfth Author\textsuperscript{1}},
%\\
%  \textbf{Thirteenth Author\textsuperscript{3}},
%  \textbf{Fourteenth F. Author\textsuperscript{2,4}},
%  \textbf{Fifteenth Author\textsuperscript{1}},
%  \textbf{Sixteenth Author\textsuperscript{1}},
%\\
%  \textbf{Seventeenth S. Author\textsuperscript{4,5}},
%  \textbf{Eighteenth Author\textsuperscript{3,4}},
%  \textbf{Nineteenth N. Author\textsuperscript{2,5}},
%  \textbf{Twentieth Author\textsuperscript{1}}
%\\
%\\
%  \textsuperscript{1}Affiliation 1,
%  \textsuperscript{2}Affiliation 2,
%  \textsuperscript{3}Affiliation 3,
%  \textsuperscript{4}Affiliation 4,
%  \textsuperscript{5}Affiliation 5
%\\
%  \small{
%    \textbf{Correspondence:} \href{mailto:email@domain}{email@domain}
%  }
%}

\begin{document}
\maketitle
\begin{abstract}
Recent advancements in Large Language Models (LLMs) have demonstrated remarkable capabilities across diverse tasks. However, their potential to integrate physical model knowledge for real-world signal interpretation remains largely unexplored. In this work, we introduce Wi-Chat, the first LLM-powered Wi-Fi-based human activity recognition system. We demonstrate that LLMs can process raw Wi-Fi signals and infer human activities by incorporating Wi-Fi sensing principles into prompts. Our approach leverages physical model insights to guide LLMs in interpreting Channel State Information (CSI) data without traditional signal processing techniques. Through experiments on real-world Wi-Fi datasets, we show that LLMs exhibit strong reasoning capabilities, achieving zero-shot activity recognition. These findings highlight a new paradigm for Wi-Fi sensing, expanding LLM applications beyond conventional language tasks and enhancing the accessibility of wireless sensing for real-world deployments.
\end{abstract}

\section{Introduction}

In today’s rapidly evolving digital landscape, the transformative power of web technologies has redefined not only how services are delivered but also how complex tasks are approached. Web-based systems have become increasingly prevalent in risk control across various domains. This widespread adoption is due their accessibility, scalability, and ability to remotely connect various types of users. For example, these systems are used for process safety management in industry~\cite{kannan2016web}, safety risk early warning in urban construction~\cite{ding2013development}, and safe monitoring of infrastructural systems~\cite{repetto2018web}. Within these web-based risk management systems, the source search problem presents a huge challenge. Source search refers to the task of identifying the origin of a risky event, such as a gas leak and the emission point of toxic substances. This source search capability is crucial for effective risk management and decision-making.

Traditional approaches to implementing source search capabilities into the web systems often rely on solely algorithmic solutions~\cite{ristic2016study}. These methods, while relatively straightforward to implement, often struggle to achieve acceptable performances due to algorithmic local optima and complex unknown environments~\cite{zhao2020searching}. More recently, web crowdsourcing has emerged as a promising alternative for tackling the source search problem by incorporating human efforts in these web systems on-the-fly~\cite{zhao2024user}. This approach outsources the task of addressing issues encountered during the source search process to human workers, leveraging their capabilities to enhance system performance.

These solutions often employ a human-AI collaborative way~\cite{zhao2023leveraging} where algorithms handle exploration-exploitation and report the encountered problems while human workers resolve complex decision-making bottlenecks to help the algorithms getting rid of local deadlocks~\cite{zhao2022crowd}. Although effective, this paradigm suffers from two inherent limitations: increased operational costs from continuous human intervention, and slow response times of human workers due to sequential decision-making. These challenges motivate our investigation into developing autonomous systems that preserve human-like reasoning capabilities while reducing dependency on massive crowdsourced labor.

Furthermore, recent advancements in large language models (LLMs)~\cite{chang2024survey} and multi-modal LLMs (MLLMs)~\cite{huang2023chatgpt} have unveiled promising avenues for addressing these challenges. One clear opportunity involves the seamless integration of visual understanding and linguistic reasoning for robust decision-making in search tasks. However, whether large models-assisted source search is really effective and efficient for improving the current source search algorithms~\cite{ji2022source} remains unknown. \textit{To address the research gap, we are particularly interested in answering the following two research questions in this work:}

\textbf{\textit{RQ1: }}How can source search capabilities be integrated into web-based systems to support decision-making in time-sensitive risk management scenarios? 
% \sq{I mention ``time-sensitive'' here because I feel like we shall say something about the response time -- LLM has to be faster than humans}

\textbf{\textit{RQ2: }}How can MLLMs and LLMs enhance the effectiveness and efficiency of existing source search algorithms? 

% \textit{\textbf{RQ2:}} To what extent does the performance of large models-assisted search align with or approach the effectiveness of human-AI collaborative search? 

To answer the research questions, we propose a novel framework called Auto-\
S$^2$earch (\textbf{Auto}nomous \textbf{S}ource \textbf{Search}) and implement a prototype system that leverages advanced web technologies to simulate real-world conditions for zero-shot source search. Unlike traditional methods that rely on pre-defined heuristics or extensive human intervention, AutoS$^2$earch employs a carefully designed prompt that encapsulates human rationales, thereby guiding the MLLM to generate coherent and accurate scene descriptions from visual inputs about four directional choices. Based on these language-based descriptions, the LLM is enabled to determine the optimal directional choice through chain-of-thought (CoT) reasoning. Comprehensive empirical validation demonstrates that AutoS$^2$-\ 
earch achieves a success rate of 95–98\%, closely approaching the performance of human-AI collaborative search across 20 benchmark scenarios~\cite{zhao2023leveraging}. 

Our work indicates that the role of humans in future web crowdsourcing tasks may evolve from executors to validators or supervisors. Furthermore, incorporating explanations of LLM decisions into web-based system interfaces has the potential to help humans enhance task performance in risk control.






\section{Related Work}
\label{sec:relatedworks}

% \begin{table*}[t]
% \centering 
% \renewcommand\arraystretch{0.98}
% \fontsize{8}{10}\selectfont \setlength{\tabcolsep}{0.4em}
% \begin{tabular}{@{}lc|cc|cc|cc@{}}
% \toprule
% \textbf{Methods}           & \begin{tabular}[c]{@{}c@{}}\textbf{Training}\\ \textbf{Paradigm}\end{tabular} & \begin{tabular}[c]{@{}c@{}}\textbf{$\#$ PT Data}\\ \textbf{(Tokens)}\end{tabular} & \begin{tabular}[c]{@{}c@{}}\textbf{$\#$ IFT Data}\\ \textbf{(Samples)}\end{tabular} & \textbf{Code}  & \begin{tabular}[c]{@{}c@{}}\textbf{Natural}\\ \textbf{Language}\end{tabular} & \begin{tabular}[c]{@{}c@{}}\textbf{Action}\\ \textbf{Trajectories}\end{tabular} & \begin{tabular}[c]{@{}c@{}}\textbf{API}\\ \textbf{Documentation}\end{tabular}\\ \midrule 
% NexusRaven~\citep{srinivasan2023nexusraven} & IFT & - & - & \textcolor{green}{\CheckmarkBold} & \textcolor{green}{\CheckmarkBold} &\textcolor{red}{\XSolidBrush}&\textcolor{red}{\XSolidBrush}\\
% AgentInstruct~\citep{zeng2023agenttuning} & IFT & - & 2k & \textcolor{green}{\CheckmarkBold} & \textcolor{green}{\CheckmarkBold} &\textcolor{red}{\XSolidBrush}&\textcolor{red}{\XSolidBrush} \\
% AgentEvol~\citep{xi2024agentgym} & IFT & - & 14.5k & \textcolor{green}{\CheckmarkBold} & \textcolor{green}{\CheckmarkBold} &\textcolor{green}{\CheckmarkBold}&\textcolor{red}{\XSolidBrush} \\
% Gorilla~\citep{patil2023gorilla}& IFT & - & 16k & \textcolor{green}{\CheckmarkBold} & \textcolor{green}{\CheckmarkBold} &\textcolor{red}{\XSolidBrush}&\textcolor{green}{\CheckmarkBold}\\
% OpenFunctions-v2~\citep{patil2023gorilla} & IFT & - & 65k & \textcolor{green}{\CheckmarkBold} & \textcolor{green}{\CheckmarkBold} &\textcolor{red}{\XSolidBrush}&\textcolor{green}{\CheckmarkBold}\\
% LAM~\citep{zhang2024agentohana} & IFT & - & 42.6k & \textcolor{green}{\CheckmarkBold} & \textcolor{green}{\CheckmarkBold} &\textcolor{green}{\CheckmarkBold}&\textcolor{red}{\XSolidBrush} \\
% xLAM~\citep{liu2024apigen} & IFT & - & 60k & \textcolor{green}{\CheckmarkBold} & \textcolor{green}{\CheckmarkBold} &\textcolor{green}{\CheckmarkBold}&\textcolor{red}{\XSolidBrush} \\\midrule
% LEMUR~\citep{xu2024lemur} & PT & 90B & 300k & \textcolor{green}{\CheckmarkBold} & \textcolor{green}{\CheckmarkBold} &\textcolor{green}{\CheckmarkBold}&\textcolor{red}{\XSolidBrush}\\
% \rowcolor{teal!12} \method & PT & 103B & 95k & \textcolor{green}{\CheckmarkBold} & \textcolor{green}{\CheckmarkBold} & \textcolor{green}{\CheckmarkBold} & \textcolor{green}{\CheckmarkBold} \\
% \bottomrule
% \end{tabular}
% \caption{Summary of existing tuning- and pretraining-based LLM agents with their training sample sizes. "PT" and "IFT" denote "Pre-Training" and "Instruction Fine-Tuning", respectively. }
% \label{tab:related}
% \end{table*}

\begin{table*}[ht]
\begin{threeparttable}
\centering 
\renewcommand\arraystretch{0.98}
\fontsize{7}{9}\selectfont \setlength{\tabcolsep}{0.2em}
\begin{tabular}{@{}l|c|c|ccc|cc|cc|cccc@{}}
\toprule
\textbf{Methods} & \textbf{Datasets}           & \begin{tabular}[c]{@{}c@{}}\textbf{Training}\\ \textbf{Paradigm}\end{tabular} & \begin{tabular}[c]{@{}c@{}}\textbf{\# PT Data}\\ \textbf{(Tokens)}\end{tabular} & \begin{tabular}[c]{@{}c@{}}\textbf{\# IFT Data}\\ \textbf{(Samples)}\end{tabular} & \textbf{\# APIs} & \textbf{Code}  & \begin{tabular}[c]{@{}c@{}}\textbf{Nat.}\\ \textbf{Lang.}\end{tabular} & \begin{tabular}[c]{@{}c@{}}\textbf{Action}\\ \textbf{Traj.}\end{tabular} & \begin{tabular}[c]{@{}c@{}}\textbf{API}\\ \textbf{Doc.}\end{tabular} & \begin{tabular}[c]{@{}c@{}}\textbf{Func.}\\ \textbf{Call}\end{tabular} & \begin{tabular}[c]{@{}c@{}}\textbf{Multi.}\\ \textbf{Step}\end{tabular}  & \begin{tabular}[c]{@{}c@{}}\textbf{Plan}\\ \textbf{Refine}\end{tabular}  & \begin{tabular}[c]{@{}c@{}}\textbf{Multi.}\\ \textbf{Turn}\end{tabular}\\ \midrule 
\multicolumn{13}{l}{\emph{Instruction Finetuning-based LLM Agents for Intrinsic Reasoning}}  \\ \midrule
FireAct~\cite{chen2023fireact} & FireAct & IFT & - & 2.1K & 10 & \textcolor{red}{\XSolidBrush} &\textcolor{green}{\CheckmarkBold} &\textcolor{green}{\CheckmarkBold}  & \textcolor{red}{\XSolidBrush} &\textcolor{green}{\CheckmarkBold} & \textcolor{red}{\XSolidBrush} &\textcolor{green}{\CheckmarkBold} & \textcolor{red}{\XSolidBrush} \\
ToolAlpaca~\cite{tang2023toolalpaca} & ToolAlpaca & IFT & - & 4.0K & 400 & \textcolor{red}{\XSolidBrush} &\textcolor{green}{\CheckmarkBold} &\textcolor{green}{\CheckmarkBold} & \textcolor{red}{\XSolidBrush} &\textcolor{green}{\CheckmarkBold} & \textcolor{red}{\XSolidBrush}  &\textcolor{green}{\CheckmarkBold} & \textcolor{red}{\XSolidBrush}  \\
ToolLLaMA~\cite{qin2023toolllm} & ToolBench & IFT & - & 12.7K & 16,464 & \textcolor{red}{\XSolidBrush} &\textcolor{green}{\CheckmarkBold} &\textcolor{green}{\CheckmarkBold} &\textcolor{red}{\XSolidBrush} &\textcolor{green}{\CheckmarkBold}&\textcolor{green}{\CheckmarkBold}&\textcolor{green}{\CheckmarkBold} &\textcolor{green}{\CheckmarkBold}\\
AgentEvol~\citep{xi2024agentgym} & AgentTraj-L & IFT & - & 14.5K & 24 &\textcolor{red}{\XSolidBrush} & \textcolor{green}{\CheckmarkBold} &\textcolor{green}{\CheckmarkBold}&\textcolor{red}{\XSolidBrush} &\textcolor{green}{\CheckmarkBold}&\textcolor{red}{\XSolidBrush} &\textcolor{red}{\XSolidBrush} &\textcolor{green}{\CheckmarkBold}\\
Lumos~\cite{yin2024agent} & Lumos & IFT  & - & 20.0K & 16 &\textcolor{red}{\XSolidBrush} & \textcolor{green}{\CheckmarkBold} & \textcolor{green}{\CheckmarkBold} &\textcolor{red}{\XSolidBrush} & \textcolor{green}{\CheckmarkBold} & \textcolor{green}{\CheckmarkBold} &\textcolor{red}{\XSolidBrush} & \textcolor{green}{\CheckmarkBold}\\
Agent-FLAN~\cite{chen2024agent} & Agent-FLAN & IFT & - & 24.7K & 20 &\textcolor{red}{\XSolidBrush} & \textcolor{green}{\CheckmarkBold} & \textcolor{green}{\CheckmarkBold} &\textcolor{red}{\XSolidBrush} & \textcolor{green}{\CheckmarkBold}& \textcolor{green}{\CheckmarkBold}&\textcolor{red}{\XSolidBrush} & \textcolor{green}{\CheckmarkBold}\\
AgentTuning~\citep{zeng2023agenttuning} & AgentInstruct & IFT & - & 35.0K & - &\textcolor{red}{\XSolidBrush} & \textcolor{green}{\CheckmarkBold} & \textcolor{green}{\CheckmarkBold} &\textcolor{red}{\XSolidBrush} & \textcolor{green}{\CheckmarkBold} &\textcolor{red}{\XSolidBrush} &\textcolor{red}{\XSolidBrush} & \textcolor{green}{\CheckmarkBold}\\\midrule
\multicolumn{13}{l}{\emph{Instruction Finetuning-based LLM Agents for Function Calling}} \\\midrule
NexusRaven~\citep{srinivasan2023nexusraven} & NexusRaven & IFT & - & - & 116 & \textcolor{green}{\CheckmarkBold} & \textcolor{green}{\CheckmarkBold}  & \textcolor{green}{\CheckmarkBold} &\textcolor{red}{\XSolidBrush} & \textcolor{green}{\CheckmarkBold} &\textcolor{red}{\XSolidBrush} &\textcolor{red}{\XSolidBrush}&\textcolor{red}{\XSolidBrush}\\
Gorilla~\citep{patil2023gorilla} & Gorilla & IFT & - & 16.0K & 1,645 & \textcolor{green}{\CheckmarkBold} &\textcolor{red}{\XSolidBrush} &\textcolor{red}{\XSolidBrush}&\textcolor{green}{\CheckmarkBold} &\textcolor{green}{\CheckmarkBold} &\textcolor{red}{\XSolidBrush} &\textcolor{red}{\XSolidBrush} &\textcolor{red}{\XSolidBrush}\\
OpenFunctions-v2~\citep{patil2023gorilla} & OpenFunctions-v2 & IFT & - & 65.0K & - & \textcolor{green}{\CheckmarkBold} & \textcolor{green}{\CheckmarkBold} &\textcolor{red}{\XSolidBrush} &\textcolor{green}{\CheckmarkBold} &\textcolor{green}{\CheckmarkBold} &\textcolor{red}{\XSolidBrush} &\textcolor{red}{\XSolidBrush} &\textcolor{red}{\XSolidBrush}\\
API Pack~\cite{guo2024api} & API Pack & IFT & - & 1.1M & 11,213 &\textcolor{green}{\CheckmarkBold} &\textcolor{red}{\XSolidBrush} &\textcolor{green}{\CheckmarkBold} &\textcolor{red}{\XSolidBrush} &\textcolor{green}{\CheckmarkBold} &\textcolor{red}{\XSolidBrush}&\textcolor{red}{\XSolidBrush}&\textcolor{red}{\XSolidBrush}\\ 
LAM~\citep{zhang2024agentohana} & AgentOhana & IFT & - & 42.6K & - & \textcolor{green}{\CheckmarkBold} & \textcolor{green}{\CheckmarkBold} &\textcolor{green}{\CheckmarkBold}&\textcolor{red}{\XSolidBrush} &\textcolor{green}{\CheckmarkBold}&\textcolor{red}{\XSolidBrush}&\textcolor{green}{\CheckmarkBold}&\textcolor{green}{\CheckmarkBold}\\
xLAM~\citep{liu2024apigen} & APIGen & IFT & - & 60.0K & 3,673 & \textcolor{green}{\CheckmarkBold} & \textcolor{green}{\CheckmarkBold} &\textcolor{green}{\CheckmarkBold}&\textcolor{red}{\XSolidBrush} &\textcolor{green}{\CheckmarkBold}&\textcolor{red}{\XSolidBrush}&\textcolor{green}{\CheckmarkBold}&\textcolor{green}{\CheckmarkBold}\\\midrule
\multicolumn{13}{l}{\emph{Pretraining-based LLM Agents}}  \\\midrule
% LEMUR~\citep{xu2024lemur} & PT & 90B & 300.0K & - & \textcolor{green}{\CheckmarkBold} & \textcolor{green}{\CheckmarkBold} &\textcolor{green}{\CheckmarkBold}&\textcolor{red}{\XSolidBrush} & \textcolor{red}{\XSolidBrush} &\textcolor{green}{\CheckmarkBold} &\textcolor{red}{\XSolidBrush}&\textcolor{red}{\XSolidBrush}\\
\rowcolor{teal!12} \method & \dataset & PT & 103B & 95.0K  & 76,537  & \textcolor{green}{\CheckmarkBold} & \textcolor{green}{\CheckmarkBold} & \textcolor{green}{\CheckmarkBold} & \textcolor{green}{\CheckmarkBold} & \textcolor{green}{\CheckmarkBold} & \textcolor{green}{\CheckmarkBold} & \textcolor{green}{\CheckmarkBold} & \textcolor{green}{\CheckmarkBold}\\
\bottomrule
\end{tabular}
% \begin{tablenotes}
%     \item $^*$ In addition, the StarCoder-API can offer 4.77M more APIs.
% \end{tablenotes}
\caption{Summary of existing instruction finetuning-based LLM agents for intrinsic reasoning and function calling, along with their training resources and sample sizes. "PT" and "IFT" denote "Pre-Training" and "Instruction Fine-Tuning", respectively.}
\vspace{-2ex}
\label{tab:related}
\end{threeparttable}
\end{table*}

\noindent \textbf{Prompting-based LLM Agents.} Due to the lack of agent-specific pre-training corpus, existing LLM agents rely on either prompt engineering~\cite{hsieh2023tool,lu2024chameleon,yao2022react,wang2023voyager} or instruction fine-tuning~\cite{chen2023fireact,zeng2023agenttuning} to understand human instructions, decompose high-level tasks, generate grounded plans, and execute multi-step actions. 
However, prompting-based methods mainly depend on the capabilities of backbone LLMs (usually commercial LLMs), failing to introduce new knowledge and struggling to generalize to unseen tasks~\cite{sun2024adaplanner,zhuang2023toolchain}. 

\noindent \textbf{Instruction Finetuning-based LLM Agents.} Considering the extensive diversity of APIs and the complexity of multi-tool instructions, tool learning inherently presents greater challenges than natural language tasks, such as text generation~\cite{qin2023toolllm}.
Post-training techniques focus more on instruction following and aligning output with specific formats~\cite{patil2023gorilla,hao2024toolkengpt,qin2023toolllm,schick2024toolformer}, rather than fundamentally improving model knowledge or capabilities. 
Moreover, heavy fine-tuning can hinder generalization or even degrade performance in non-agent use cases, potentially suppressing the original base model capabilities~\cite{ghosh2024a}.

\noindent \textbf{Pretraining-based LLM Agents.} While pre-training serves as an essential alternative, prior works~\cite{nijkamp2023codegen,roziere2023code,xu2024lemur,patil2023gorilla} have primarily focused on improving task-specific capabilities (\eg, code generation) instead of general-domain LLM agents, due to single-source, uni-type, small-scale, and poor-quality pre-training data. 
Existing tool documentation data for agent training either lacks diverse real-world APIs~\cite{patil2023gorilla, tang2023toolalpaca} or is constrained to single-tool or single-round tool execution. 
Furthermore, trajectory data mostly imitate expert behavior or follow function-calling rules with inferior planning and reasoning, failing to fully elicit LLMs' capabilities and handle complex instructions~\cite{qin2023toolllm}. 
Given a wide range of candidate API functions, each comprising various function names and parameters available at every planning step, identifying globally optimal solutions and generalizing across tasks remains highly challenging.



\section{Preliminaries}
\label{Preliminaries}
\begin{figure*}[t]
    \centering
    \includegraphics[width=0.95\linewidth]{fig/HealthGPT_Framework.png}
    \caption{The \ourmethod{} architecture integrates hierarchical visual perception and H-LoRA, employing a task-specific hard router to select visual features and H-LoRA plugins, ultimately generating outputs with an autoregressive manner.}
    \label{fig:architecture}
\end{figure*}
\noindent\textbf{Large Vision-Language Models.} 
The input to a LVLM typically consists of an image $x^{\text{img}}$ and a discrete text sequence $x^{\text{txt}}$. The visual encoder $\mathcal{E}^{\text{img}}$ converts the input image $x^{\text{img}}$ into a sequence of visual tokens $\mathcal{V} = [v_i]_{i=1}^{N_v}$, while the text sequence $x^{\text{txt}}$ is mapped into a sequence of text tokens $\mathcal{T} = [t_i]_{i=1}^{N_t}$ using an embedding function $\mathcal{E}^{\text{txt}}$. The LLM $\mathcal{M_\text{LLM}}(\cdot|\theta)$ models the joint probability of the token sequence $\mathcal{U} = \{\mathcal{V},\mathcal{T}\}$, which is expressed as:
\begin{equation}
    P_\theta(R | \mathcal{U}) = \prod_{i=1}^{N_r} P_\theta(r_i | \{\mathcal{U}, r_{<i}\}),
\end{equation}
where $R = [r_i]_{i=1}^{N_r}$ is the text response sequence. The LVLM iteratively generates the next token $r_i$ based on $r_{<i}$. The optimization objective is to minimize the cross-entropy loss of the response $\mathcal{R}$.
% \begin{equation}
%     \mathcal{L}_{\text{VLM}} = \mathbb{E}_{R|\mathcal{U}}\left[-\log P_\theta(R | \mathcal{U})\right]
% \end{equation}
It is worth noting that most LVLMs adopt a design paradigm based on ViT, alignment adapters, and pre-trained LLMs\cite{liu2023llava,liu2024improved}, enabling quick adaptation to downstream tasks.


\noindent\textbf{VQGAN.}
VQGAN~\cite{esser2021taming} employs latent space compression and indexing mechanisms to effectively learn a complete discrete representation of images. VQGAN first maps the input image $x^{\text{img}}$ to a latent representation $z = \mathcal{E}(x)$ through a encoder $\mathcal{E}$. Then, the latent representation is quantized using a codebook $\mathcal{Z} = \{z_k\}_{k=1}^K$, generating a discrete index sequence $\mathcal{I} = [i_m]_{m=1}^N$, where $i_m \in \mathcal{Z}$ represents the quantized code index:
\begin{equation}
    \mathcal{I} = \text{Quantize}(z|\mathcal{Z}) = \arg\min_{z_k \in \mathcal{Z}} \| z - z_k \|_2.
\end{equation}
In our approach, the discrete index sequence $\mathcal{I}$ serves as a supervisory signal for the generation task, enabling the model to predict the index sequence $\hat{\mathcal{I}}$ from input conditions such as text or other modality signals.  
Finally, the predicted index sequence $\hat{\mathcal{I}}$ is upsampled by the VQGAN decoder $G$, generating the high-quality image $\hat{x}^\text{img} = G(\hat{\mathcal{I}})$.



\noindent\textbf{Low Rank Adaptation.} 
LoRA\cite{hu2021lora} effectively captures the characteristics of downstream tasks by introducing low-rank adapters. The core idea is to decompose the bypass weight matrix $\Delta W\in\mathbb{R}^{d^{\text{in}} \times d^{\text{out}}}$ into two low-rank matrices $ \{A \in \mathbb{R}^{d^{\text{in}} \times r}, B \in \mathbb{R}^{r \times d^{\text{out}}} \}$, where $ r \ll \min\{d^{\text{in}}, d^{\text{out}}\} $, significantly reducing learnable parameters. The output with the LoRA adapter for the input $x$ is then given by:
\begin{equation}
    h = x W_0 + \alpha x \Delta W/r = x W_0 + \alpha xAB/r,
\end{equation}
where matrix $ A $ is initialized with a Gaussian distribution, while the matrix $ B $ is initialized as a zero matrix. The scaling factor $ \alpha/r $ controls the impact of $ \Delta W $ on the model.

\section{HealthGPT}
\label{Method}


\subsection{Unified Autoregressive Generation.}  
% As shown in Figure~\ref{fig:architecture}, 
\ourmethod{} (Figure~\ref{fig:architecture}) utilizes a discrete token representation that covers both text and visual outputs, unifying visual comprehension and generation as an autoregressive task. 
For comprehension, $\mathcal{M}_\text{llm}$ receives the input joint sequence $\mathcal{U}$ and outputs a series of text token $\mathcal{R} = [r_1, r_2, \dots, r_{N_r}]$, where $r_i \in \mathcal{V}_{\text{txt}}$, and $\mathcal{V}_{\text{txt}}$ represents the LLM's vocabulary:
\begin{equation}
    P_\theta(\mathcal{R} \mid \mathcal{U}) = \prod_{i=1}^{N_r} P_\theta(r_i \mid \mathcal{U}, r_{<i}).
\end{equation}
For generation, $\mathcal{M}_\text{llm}$ first receives a special start token $\langle \text{START\_IMG} \rangle$, then generates a series of tokens corresponding to the VQGAN indices $\mathcal{I} = [i_1, i_2, \dots, i_{N_i}]$, where $i_j \in \mathcal{V}_{\text{vq}}$, and $\mathcal{V}_{\text{vq}}$ represents the index range of VQGAN. Upon completion of generation, the LLM outputs an end token $\langle \text{END\_IMG} \rangle$:
\begin{equation}
    P_\theta(\mathcal{I} \mid \mathcal{U}) = \prod_{j=1}^{N_i} P_\theta(i_j \mid \mathcal{U}, i_{<j}).
\end{equation}
Finally, the generated index sequence $\mathcal{I}$ is fed into the decoder $G$, which reconstructs the target image $\hat{x}^{\text{img}} = G(\mathcal{I})$.

\subsection{Hierarchical Visual Perception}  
Given the differences in visual perception between comprehension and generation tasks—where the former focuses on abstract semantics and the latter emphasizes complete semantics—we employ ViT to compress the image into discrete visual tokens at multiple hierarchical levels.
Specifically, the image is converted into a series of features $\{f_1, f_2, \dots, f_L\}$ as it passes through $L$ ViT blocks.

To address the needs of various tasks, the hidden states are divided into two types: (i) \textit{Concrete-grained features} $\mathcal{F}^{\text{Con}} = \{f_1, f_2, \dots, f_k\}, k < L$, derived from the shallower layers of ViT, containing sufficient global features, suitable for generation tasks; 
(ii) \textit{Abstract-grained features} $\mathcal{F}^{\text{Abs}} = \{f_{k+1}, f_{k+2}, \dots, f_L\}$, derived from the deeper layers of ViT, which contain abstract semantic information closer to the text space, suitable for comprehension tasks.

The task type $T$ (comprehension or generation) determines which set of features is selected as the input for the downstream large language model:
\begin{equation}
    \mathcal{F}^{\text{img}}_T =
    \begin{cases}
        \mathcal{F}^{\text{Con}}, & \text{if } T = \text{generation task} \\
        \mathcal{F}^{\text{Abs}}, & \text{if } T = \text{comprehension task}
    \end{cases}
\end{equation}
We integrate the image features $\mathcal{F}^{\text{img}}_T$ and text features $\mathcal{T}$ into a joint sequence through simple concatenation, which is then fed into the LLM $\mathcal{M}_{\text{llm}}$ for autoregressive generation.
% :
% \begin{equation}
%     \mathcal{R} = \mathcal{M}_{\text{llm}}(\mathcal{U}|\theta), \quad \mathcal{U} = [\mathcal{F}^{\text{img}}_T; \mathcal{T}]
% \end{equation}
\subsection{Heterogeneous Knowledge Adaptation}
We devise H-LoRA, which stores heterogeneous knowledge from comprehension and generation tasks in separate modules and dynamically routes to extract task-relevant knowledge from these modules. 
At the task level, for each task type $ T $, we dynamically assign a dedicated H-LoRA submodule $ \theta^T $, which is expressed as:
\begin{equation}
    \mathcal{R} = \mathcal{M}_\text{LLM}(\mathcal{U}|\theta, \theta^T), \quad \theta^T = \{A^T, B^T, \mathcal{R}^T_\text{outer}\}.
\end{equation}
At the feature level for a single task, H-LoRA integrates the idea of Mixture of Experts (MoE)~\cite{masoudnia2014mixture} and designs an efficient matrix merging and routing weight allocation mechanism, thus avoiding the significant computational delay introduced by matrix splitting in existing MoELoRA~\cite{luo2024moelora}. Specifically, we first merge the low-rank matrices (rank = r) of $ k $ LoRA experts into a unified matrix:
\begin{equation}
    \mathbf{A}^{\text{merged}}, \mathbf{B}^{\text{merged}} = \text{Concat}(\{A_i\}_1^k), \text{Concat}(\{B_i\}_1^k),
\end{equation}
where $ \mathbf{A}^{\text{merged}} \in \mathbb{R}^{d^\text{in} \times rk} $ and $ \mathbf{B}^{\text{merged}} \in \mathbb{R}^{rk \times d^\text{out}} $. The $k$-dimension routing layer generates expert weights $ \mathcal{W} \in \mathbb{R}^{\text{token\_num} \times k} $ based on the input hidden state $ x $, and these are expanded to $ \mathbb{R}^{\text{token\_num} \times rk} $ as follows:
\begin{equation}
    \mathcal{W}^\text{expanded} = \alpha k \mathcal{W} / r \otimes \mathbf{1}_r,
\end{equation}
where $ \otimes $ denotes the replication operation.
The overall output of H-LoRA is computed as:
\begin{equation}
    \mathcal{O}^\text{H-LoRA} = (x \mathbf{A}^{\text{merged}} \odot \mathcal{W}^\text{expanded}) \mathbf{B}^{\text{merged}},
\end{equation}
where $ \odot $ represents element-wise multiplication. Finally, the output of H-LoRA is added to the frozen pre-trained weights to produce the final output:
\begin{equation}
    \mathcal{O} = x W_0 + \mathcal{O}^\text{H-LoRA}.
\end{equation}
% In summary, H-LoRA is a task-based dynamic PEFT method that achieves high efficiency in single-task fine-tuning.

\subsection{Training Pipeline}

\begin{figure}[t]
    \centering
    \hspace{-4mm}
    \includegraphics[width=0.94\linewidth]{fig/data.pdf}
    \caption{Data statistics of \texttt{VL-Health}. }
    \label{fig:data}
\end{figure}
\noindent \textbf{1st Stage: Multi-modal Alignment.} 
In the first stage, we design separate visual adapters and H-LoRA submodules for medical unified tasks. For the medical comprehension task, we train abstract-grained visual adapters using high-quality image-text pairs to align visual embeddings with textual embeddings, thereby enabling the model to accurately describe medical visual content. During this process, the pre-trained LLM and its corresponding H-LoRA submodules remain frozen. In contrast, the medical generation task requires training concrete-grained adapters and H-LoRA submodules while keeping the LLM frozen. Meanwhile, we extend the textual vocabulary to include multimodal tokens, enabling the support of additional VQGAN vector quantization indices. The model trains on image-VQ pairs, endowing the pre-trained LLM with the capability for image reconstruction. This design ensures pixel-level consistency of pre- and post-LVLM. The processes establish the initial alignment between the LLM’s outputs and the visual inputs.

\noindent \textbf{2nd Stage: Heterogeneous H-LoRA Plugin Adaptation.}  
The submodules of H-LoRA share the word embedding layer and output head but may encounter issues such as bias and scale inconsistencies during training across different tasks. To ensure that the multiple H-LoRA plugins seamlessly interface with the LLMs and form a unified base, we fine-tune the word embedding layer and output head using a small amount of mixed data to maintain consistency in the model weights. Specifically, during this stage, all H-LoRA submodules for different tasks are kept frozen, with only the word embedding layer and output head being optimized. Through this stage, the model accumulates foundational knowledge for unified tasks by adapting H-LoRA plugins.

\begin{table*}[!t]
\centering
\caption{Comparison of \ourmethod{} with other LVLMs and unified multi-modal models on medical visual comprehension tasks. \textbf{Bold} and \underline{underlined} text indicates the best performance and second-best performance, respectively.}
\resizebox{\textwidth}{!}{
\begin{tabular}{c|lcc|cccccccc|c}
\toprule
\rowcolor[HTML]{E9F3FE} &  &  &  & \multicolumn{2}{c}{\textbf{VQA-RAD \textuparrow}} & \multicolumn{2}{c}{\textbf{SLAKE \textuparrow}} & \multicolumn{2}{c}{\textbf{PathVQA \textuparrow}} &  &  &  \\ 
\cline{5-10}
\rowcolor[HTML]{E9F3FE}\multirow{-2}{*}{\textbf{Type}} & \multirow{-2}{*}{\textbf{Model}} & \multirow{-2}{*}{\textbf{\# Params}} & \multirow{-2}{*}{\makecell{\textbf{Medical} \\ \textbf{LVLM}}} & \textbf{close} & \textbf{all} & \textbf{close} & \textbf{all} & \textbf{close} & \textbf{all} & \multirow{-2}{*}{\makecell{\textbf{MMMU} \\ \textbf{-Med}}\textuparrow} & \multirow{-2}{*}{\textbf{OMVQA}\textuparrow} & \multirow{-2}{*}{\textbf{Avg. \textuparrow}} \\ 
\midrule \midrule
\multirow{9}{*}{\textbf{Comp. Only}} 
& Med-Flamingo & 8.3B & \Large \ding{51} & 58.6 & 43.0 & 47.0 & 25.5 & 61.9 & 31.3 & 28.7 & 34.9 & 41.4 \\
& LLaVA-Med & 7B & \Large \ding{51} & 60.2 & 48.1 & 58.4 & 44.8 & 62.3 & 35.7 & 30.0 & 41.3 & 47.6 \\
& HuatuoGPT-Vision & 7B & \Large \ding{51} & 66.9 & 53.0 & 59.8 & 49.1 & 52.9 & 32.0 & 42.0 & 50.0 & 50.7 \\
& BLIP-2 & 6.7B & \Large \ding{55} & 43.4 & 36.8 & 41.6 & 35.3 & 48.5 & 28.8 & 27.3 & 26.9 & 36.1 \\
& LLaVA-v1.5 & 7B & \Large \ding{55} & 51.8 & 42.8 & 37.1 & 37.7 & 53.5 & 31.4 & 32.7 & 44.7 & 41.5 \\
& InstructBLIP & 7B & \Large \ding{55} & 61.0 & 44.8 & 66.8 & 43.3 & 56.0 & 32.3 & 25.3 & 29.0 & 44.8 \\
& Yi-VL & 6B & \Large \ding{55} & 52.6 & 42.1 & 52.4 & 38.4 & 54.9 & 30.9 & 38.0 & 50.2 & 44.9 \\
& InternVL2 & 8B & \Large \ding{55} & 64.9 & 49.0 & 66.6 & 50.1 & 60.0 & 31.9 & \underline{43.3} & 54.5 & 52.5\\
& Llama-3.2 & 11B & \Large \ding{55} & 68.9 & 45.5 & 72.4 & 52.1 & 62.8 & 33.6 & 39.3 & 63.2 & 54.7 \\
\midrule
\multirow{5}{*}{\textbf{Comp. \& Gen.}} 
& Show-o & 1.3B & \Large \ding{55} & 50.6 & 33.9 & 31.5 & 17.9 & 52.9 & 28.2 & 22.7 & 45.7 & 42.6 \\
& Unified-IO 2 & 7B & \Large \ding{55} & 46.2 & 32.6 & 35.9 & 21.9 & 52.5 & 27.0 & 25.3 & 33.0 & 33.8 \\
& Janus & 1.3B & \Large \ding{55} & 70.9 & 52.8 & 34.7 & 26.9 & 51.9 & 27.9 & 30.0 & 26.8 & 33.5 \\
& \cellcolor[HTML]{DAE0FB}HealthGPT-M3 & \cellcolor[HTML]{DAE0FB}3.8B & \cellcolor[HTML]{DAE0FB}\Large \ding{51} & \cellcolor[HTML]{DAE0FB}\underline{73.7} & \cellcolor[HTML]{DAE0FB}\underline{55.9} & \cellcolor[HTML]{DAE0FB}\underline{74.6} & \cellcolor[HTML]{DAE0FB}\underline{56.4} & \cellcolor[HTML]{DAE0FB}\underline{78.7} & \cellcolor[HTML]{DAE0FB}\underline{39.7} & \cellcolor[HTML]{DAE0FB}\underline{43.3} & \cellcolor[HTML]{DAE0FB}\underline{68.5} & \cellcolor[HTML]{DAE0FB}\underline{61.3} \\
& \cellcolor[HTML]{DAE0FB}HealthGPT-L14 & \cellcolor[HTML]{DAE0FB}14B & \cellcolor[HTML]{DAE0FB}\Large \ding{51} & \cellcolor[HTML]{DAE0FB}\textbf{77.7} & \cellcolor[HTML]{DAE0FB}\textbf{58.3} & \cellcolor[HTML]{DAE0FB}\textbf{76.4} & \cellcolor[HTML]{DAE0FB}\textbf{64.5} & \cellcolor[HTML]{DAE0FB}\textbf{85.9} & \cellcolor[HTML]{DAE0FB}\textbf{44.4} & \cellcolor[HTML]{DAE0FB}\textbf{49.2} & \cellcolor[HTML]{DAE0FB}\textbf{74.4} & \cellcolor[HTML]{DAE0FB}\textbf{66.4} \\
\bottomrule
\end{tabular}
}
\label{tab:results}
\end{table*}
\begin{table*}[ht]
    \centering
    \caption{The experimental results for the four modality conversion tasks.}
    \resizebox{\textwidth}{!}{
    \begin{tabular}{l|ccc|ccc|ccc|ccc}
        \toprule
        \rowcolor[HTML]{E9F3FE} & \multicolumn{3}{c}{\textbf{CT to MRI (Brain)}} & \multicolumn{3}{c}{\textbf{CT to MRI (Pelvis)}} & \multicolumn{3}{c}{\textbf{MRI to CT (Brain)}} & \multicolumn{3}{c}{\textbf{MRI to CT (Pelvis)}} \\
        \cline{2-13}
        \rowcolor[HTML]{E9F3FE}\multirow{-2}{*}{\textbf{Model}}& \textbf{SSIM $\uparrow$} & \textbf{PSNR $\uparrow$} & \textbf{MSE $\downarrow$} & \textbf{SSIM $\uparrow$} & \textbf{PSNR $\uparrow$} & \textbf{MSE $\downarrow$} & \textbf{SSIM $\uparrow$} & \textbf{PSNR $\uparrow$} & \textbf{MSE $\downarrow$} & \textbf{SSIM $\uparrow$} & \textbf{PSNR $\uparrow$} & \textbf{MSE $\downarrow$} \\
        \midrule \midrule
        pix2pix & 71.09 & 32.65 & 36.85 & 59.17 & 31.02 & 51.91 & 78.79 & 33.85 & 28.33 & 72.31 & 32.98 & 36.19 \\
        CycleGAN & 54.76 & 32.23 & 40.56 & 54.54 & 30.77 & 55.00 & 63.75 & 31.02 & 52.78 & 50.54 & 29.89 & 67.78 \\
        BBDM & {71.69} & {32.91} & {34.44} & 57.37 & 31.37 & 48.06 & \textbf{86.40} & 34.12 & 26.61 & {79.26} & 33.15 & 33.60 \\
        Vmanba & 69.54 & 32.67 & 36.42 & {63.01} & {31.47} & {46.99} & 79.63 & 34.12 & 26.49 & 77.45 & 33.53 & 31.85 \\
        DiffMa & 71.47 & 32.74 & 35.77 & 62.56 & 31.43 & 47.38 & 79.00 & {34.13} & {26.45} & 78.53 & {33.68} & {30.51} \\
        \rowcolor[HTML]{DAE0FB}HealthGPT-M3 & \underline{79.38} & \underline{33.03} & \underline{33.48} & \underline{71.81} & \underline{31.83} & \underline{43.45} & {85.06} & \textbf{34.40} & \textbf{25.49} & \underline{84.23} & \textbf{34.29} & \textbf{27.99} \\
        \rowcolor[HTML]{DAE0FB}HealthGPT-L14 & \textbf{79.73} & \textbf{33.10} & \textbf{32.96} & \textbf{71.92} & \textbf{31.87} & \textbf{43.09} & \underline{85.31} & \underline{34.29} & \underline{26.20} & \textbf{84.96} & \underline{34.14} & \underline{28.13} \\
        \bottomrule
    \end{tabular}
    }
    \label{tab:conversion}
\end{table*}

\noindent \textbf{3rd Stage: Visual Instruction Fine-Tuning.}  
In the third stage, we introduce additional task-specific data to further optimize the model and enhance its adaptability to downstream tasks such as medical visual comprehension (e.g., medical QA, medical dialogues, and report generation) or generation tasks (e.g., super-resolution, denoising, and modality conversion). Notably, by this stage, the word embedding layer and output head have been fine-tuned, only the H-LoRA modules and adapter modules need to be trained. This strategy significantly improves the model's adaptability and flexibility across different tasks.


\section{Experiment}
\label{s:experiment}

\subsection{Data Description}
We evaluate our method on FI~\cite{you2016building}, Twitter\_LDL~\cite{yang2017learning} and Artphoto~\cite{machajdik2010affective}.
FI is a public dataset built from Flickr and Instagram, with 23,308 images and eight emotion categories, namely \textit{amusement}, \textit{anger}, \textit{awe},  \textit{contentment}, \textit{disgust}, \textit{excitement},  \textit{fear}, and \textit{sadness}. 
% Since images in FI are all copyrighted by law, some images are corrupted now, so we remove these samples and retain 21,828 images.
% T4SA contains images from Twitter, which are classified into three categories: \textit{positive}, \textit{neutral}, and \textit{negative}. In this paper, we adopt the base version of B-T4SA, which contains 470,586 images and provides text descriptions of the corresponding tweets.
Twitter\_LDL contains 10,045 images from Twitter, with the same eight categories as the FI dataset.
% 。
For these two datasets, they are randomly split into 80\%
training and 20\% testing set.
Artphoto contains 806 artistic photos from the DeviantArt website, which we use to further evaluate the zero-shot capability of our model.
% on the small-scale dataset.
% We construct and publicly release the first image sentiment analysis dataset containing metadata.
% 。

% Based on these datasets, we are the first to construct and publicly release metadata-enhanced image sentiment analysis datasets. These datasets include scenes, tags, descriptions, and corresponding confidence scores, and are available at this link for future research purposes.


% 
\begin{table}[t]
\centering
% \begin{center}
\caption{Overall performance of different models on FI and Twitter\_LDL datasets.}
\label{tab:cap1}
% \resizebox{\linewidth}{!}
{
\begin{tabular}{l|c|c|c|c}
\hline
\multirow{2}{*}{\textbf{Model}} & \multicolumn{2}{c|}{\textbf{FI}}  & \multicolumn{2}{c}{\textbf{Twitter\_LDL}} \\ \cline{2-5} 
  & \textbf{Accuracy} & \textbf{F1} & \textbf{Accuracy} & \textbf{F1}  \\ \hline
% (\rownumber)~AlexNet~\cite{krizhevsky2017imagenet}  & 58.13\% & 56.35\%  & 56.24\%& 55.02\%  \\ 
% (\rownumber)~VGG16~\cite{simonyan2014very}  & 63.75\%& 63.08\%  & 59.34\%& 59.02\%  \\ 
(\rownumber)~ResNet101~\cite{he2016deep} & 66.16\%& 65.56\%  & 62.02\% & 61.34\%  \\ 
(\rownumber)~CDA~\cite{han2023boosting} & 66.71\%& 65.37\%  & 64.14\% & 62.85\%  \\ 
(\rownumber)~CECCN~\cite{ruan2024color} & 67.96\%& 66.74\%  & 64.59\%& 64.72\% \\ 
(\rownumber)~EmoVIT~\cite{xie2024emovit} & 68.09\%& 67.45\%  & 63.12\% & 61.97\%  \\ 
(\rownumber)~ComLDL~\cite{zhang2022compound} & 68.83\%& 67.28\%  & 65.29\% & 63.12\%  \\ 
(\rownumber)~WSDEN~\cite{li2023weakly} & 69.78\%& 69.61\%  & 67.04\% & 65.49\% \\ 
(\rownumber)~ECWA~\cite{deng2021emotion} & 70.87\%& 69.08\%  & 67.81\% & 66.87\%  \\ 
(\rownumber)~EECon~\cite{yang2023exploiting} & 71.13\%& 68.34\%  & 64.27\%& 63.16\%  \\ 
(\rownumber)~MAM~\cite{zhang2024affective} & 71.44\%  & 70.83\% & 67.18\%  & 65.01\%\\ 
(\rownumber)~TGCA-PVT~\cite{chen2024tgca}   & 73.05\%  & 71.46\% & 69.87\%  & 68.32\% \\ 
(\rownumber)~OEAN~\cite{zhang2024object}   & 73.40\%  & 72.63\% & 70.52\%  & 69.47\% \\ \hline
(\rownumber)~\shortname  & \textbf{79.48\%} & \textbf{79.22\%} & \textbf{74.12\%} & \textbf{73.09\%} \\ \hline
\end{tabular}
}
\vspace{-6mm}
% \end{center}
\end{table}
% 

\subsection{Experiment Setting}
% \subsubsection{Model Setting.}
% 
\textbf{Model Setting:}
For feature representation, we set $k=10$ to select object tags, and adopt clip-vit-base-patch32 as the pre-trained model for unified feature representation.
Moreover, we empirically set $(d_e, d_h, d_k, d_s) = (512, 128, 16, 64)$, and set the classification class $L$ to 8.

% 

\textbf{Training Setting:}
To initialize the model, we set all weights such as $\boldsymbol{W}$ following the truncated normal distribution, and use AdamW optimizer with the learning rate of $1 \times 10^{-4}$.
% warmup scheduler of cosine, warmup steps of 2000.
Furthermore, we set the batch size to 32 and the epoch of the training process to 200.
During the implementation, we utilize \textit{PyTorch} to build our entire model.
% , and our project codes are publicly available at https://github.com/zzmyrep/MESN.
% Our project codes as well as data are all publicly available on GitHub\footnote{https://github.com/zzmyrep/KBCEN}.
% Code is available at \href{https://github.com/zzmyrep/KBCEN}{https://github.com/zzmyrep/KBCEN}.

\textbf{Evaluation Metrics:}
Following~\cite{zhang2024affective, chen2024tgca, zhang2024object}, we adopt \textit{accuracy} and \textit{F1} as our evaluation metrics to measure the performance of different methods for image sentiment analysis. 



\subsection{Experiment Result}
% We compare our model against the following baselines: AlexNet~\cite{krizhevsky2017imagenet}, VGG16~\cite{simonyan2014very}, ResNet101~\cite{he2016deep}, CECCN~\cite{ruan2024color}, EmoVIT~\cite{xie2024emovit}, WSCNet~\cite{yang2018weakly}, ECWA~\cite{deng2021emotion}, EECon~\cite{yang2023exploiting}, MAM~\cite{zhang2024affective} and TGCA-PVT~\cite{chen2024tgca}, and the overall results are summarized in Table~\ref{tab:cap1}.
We compare our model against several baselines, and the overall results are summarized in Table~\ref{tab:cap1}.
We observe that our model achieves the best performance in both accuracy and F1 metrics, significantly outperforming the previous models. 
This superior performance is mainly attributed to our effective utilization of metadata to enhance image sentiment analysis, as well as the exceptional capability of the unified sentiment transformer framework we developed. These results strongly demonstrate that our proposed method can bring encouraging performance for image sentiment analysis.

\setcounter{magicrownumbers}{0} 
\begin{table}[t]
\begin{center}
\caption{Ablation study of~\shortname~on FI dataset.} 
% \vspace{1mm}
\label{tab:cap2}
\resizebox{.9\linewidth}{!}
{
\begin{tabular}{lcc}
  \hline
  \textbf{Model} & \textbf{Accuracy} & \textbf{F1} \\
  \hline
  (\rownumber)~Ours (w/o vision) & 65.72\% & 64.54\% \\
  (\rownumber)~Ours (w/o text description) & 74.05\% & 72.58\% \\
  (\rownumber)~Ours (w/o object tag) & 77.45\% & 76.84\% \\
  (\rownumber)~Ours (w/o scene tag) & 78.47\% & 78.21\% \\
  \hline
  (\rownumber)~Ours (w/o unified embedding) & 76.41\% & 76.23\% \\
  (\rownumber)~Ours (w/o adaptive learning) & 76.83\% & 76.56\% \\
  (\rownumber)~Ours (w/o cross-modal fusion) & 76.85\% & 76.49\% \\
  \hline
  (\rownumber)~Ours  & \textbf{79.48\%} & \textbf{79.22\%} \\
  \hline
\end{tabular}
}
\end{center}
\vspace{-5mm}
\end{table}


\begin{figure}[t]
\centering
% \vspace{-2mm}
\includegraphics[width=0.42\textwidth]{fig/2dvisual-linux4-paper2.pdf}
\caption{Visualization of feature distribution on eight categories before (left) and after (right) model processing.}
% 
\label{fig:visualization}
\vspace{-5mm}
\end{figure}

\subsection{Ablation Performance}
In this subsection, we conduct an ablation study to examine which component is really important for performance improvement. The results are reported in Table~\ref{tab:cap2}.

For information utilization, we observe a significant decline in model performance when visual features are removed. Additionally, the performance of \shortname~decreases when different metadata are removed separately, which means that text description, object tag, and scene tag are all critical for image sentiment analysis.
Recalling the model architecture, we separately remove transformer layers of the unified representation module, the adaptive learning module, and the cross-modal fusion module, replacing them with MLPs of the same parameter scale.
In this way, we can observe varying degrees of decline in model performance, indicating that these modules are indispensable for our model to achieve better performance.

\subsection{Visualization}
% 


% % 开始使用minipage进行左右排列
% \begin{minipage}[t]{0.45\textwidth}  % 子图1宽度为45%
%     \centering
%     \includegraphics[width=\textwidth]{2dvisual.pdf}  % 插入图片
%     \captionof{figure}{Visualization of feature distribution.}  % 使用captionof添加图片标题
%     \label{fig:visualization}
% \end{minipage}


% \begin{figure}[t]
% \centering
% \vspace{-2mm}
% \includegraphics[width=0.45\textwidth]{fig/2dvisual.pdf}
% \caption{Visualization of feature distribution.}
% \label{fig:visualization}
% % \vspace{-4mm}
% \end{figure}

% \begin{figure}[t]
% \centering
% \vspace{-2mm}
% \includegraphics[width=0.45\textwidth]{fig/2dvisual-linux3-paper.pdf}
% \caption{Visualization of feature distribution.}
% \label{fig:visualization}
% % \vspace{-4mm}
% \end{figure}



\begin{figure}[tbp]   
\vspace{-4mm}
  \centering            
  \subfloat[Depth of adaptive learning layers]   
  {
    \label{fig:subfig1}\includegraphics[width=0.22\textwidth]{fig/fig_sensitivity-a5}
  }
  \subfloat[Depth of fusion layers]
  {
    % \label{fig:subfig2}\includegraphics[width=0.22\textwidth]{fig/fig_sensitivity-b2}
    \label{fig:subfig2}\includegraphics[width=0.22\textwidth]{fig/fig_sensitivity-b2-num.pdf}
  }
  \caption{Sensitivity study of \shortname~on different depth. }   
  \label{fig:fig_sensitivity}  
\vspace{-2mm}
\end{figure}

% \begin{figure}[htbp]
% \centerline{\includegraphics{2dvisual.pdf}}
% \caption{Visualization of feature distribution.}
% \label{fig:visualization}
% \end{figure}

% In Fig.~\ref{fig:visualization}, we use t-SNE~\cite{van2008visualizing} to reduce the dimension of data features for visualization, Figure in left represents the metadata features before model processing, the features are obtained by embedding through the CLIP model, and figure in right shows the features of the data after model processing, it can be observed that after the model processing, the data with different label categories fall in different regions in the space, therefore, we can conclude that the Therefore, we can conclude that the model can effectively utilize the information contained in the metadata and use it to guide the model for classification.

In Fig.~\ref{fig:visualization}, we use t-SNE~\cite{van2008visualizing} to reduce the dimension of data features for visualization.
The left figure shows metadata features before being processed by our model (\textit{i.e.}, embedded by CLIP), while the right shows the distribution of features after being processed by our model.
We can observe that after the model processing, data with the same label are closer to each other, while others are farther away.
Therefore, it shows that the model can effectively utilize the information contained in the metadata and use it to guide the classification process.

\subsection{Sensitivity Analysis}
% 
In this subsection, we conduct a sensitivity analysis to figure out the effect of different depth settings of adaptive learning layers and fusion layers. 
% In this subsection, we conduct a sensitivity analysis to figure out the effect of different depth settings on the model. 
% Fig.~\ref{fig:fig_sensitivity} presents the effect of different depth settings of adaptive learning layers and fusion layers. 
Taking Fig.~\ref{fig:fig_sensitivity} (a) as an example, the model performance improves with increasing depth, reaching the best performance at a depth of 4.
% Taking Fig.~\ref{fig:fig_sensitivity} (a) as an example, the performance of \shortname~improves with the increase of depth at first, reaching the best performance at a depth of 4.
When the depth continues to increase, the accuracy decreases to varying degrees.
Similar results can be observed in Fig.~\ref{fig:fig_sensitivity} (b).
Therefore, we set their depths to 4 and 6 respectively to achieve the best results.

% Through our experiments, we can observe that the effect of modifying these hyperparameters on the results of the experiments is very weak, and the surface model is not sensitive to the hyperparameters.


\subsection{Zero-shot Capability}
% 

% (1)~GCH~\cite{2010Analyzing} & 21.78\% & (5)~RA-DLNet~\cite{2020A} & 34.01\% \\ \hline
% (2)~WSCNet~\cite{2019WSCNet}  & 30.25\% & (6)~CECCN~\cite{ruan2024color} & 43.83\% \\ \hline
% (3)~PCNN~\cite{2015Robust} & 31.68\%  & (7)~EmoVIT~\cite{xie2024emovit} & 44.90\% \\ \hline
% (4)~AR~\cite{2018Visual} & 32.67\% & (8)~Ours (Zero-shot) & 47.83\% \\ \hline


\begin{table}[t]
\centering
\caption{Zero-shot capability of \shortname.}
\label{tab:cap3}
\resizebox{1\linewidth}{!}
{
\begin{tabular}{lc|lc}
\hline
\textbf{Model} & \textbf{Accuracy} & \textbf{Model} & \textbf{Accuracy} \\ \hline
(1)~WSCNet~\cite{2019WSCNet}  & 30.25\% & (5)~MAM~\cite{zhang2024affective} & 39.56\%  \\ \hline
(2)~AR~\cite{2018Visual} & 32.67\% & (6)~CECCN~\cite{ruan2024color} & 43.83\% \\ \hline
(3)~RA-DLNet~\cite{2020A} & 34.01\%  & (7)~EmoVIT~\cite{xie2024emovit} & 44.90\% \\ \hline
(4)~CDA~\cite{han2023boosting} & 38.64\% & (8)~Ours (Zero-shot) & 47.83\% \\ \hline
\end{tabular}
}
\vspace{-5mm}
\end{table}

% We use the model trained on the FI dataset to test on the artphoto dataset to verify the model's generalization ability as well as robustness to other distributed datasets.
% We can observe that the MESN model shows strong competitiveness in terms of accuracy when compared to other trained models, which suggests that the model has a good generalization ability in the OOD task.

To validate the model's generalization ability and robustness to other distributed datasets, we directly test the model trained on the FI dataset, without training on Artphoto. 
% As observed in Table 3, compared to other models trained on Artphoto, we achieve highly competitive zero-shot performance, indicating that the model has good generalization ability in out-of-distribution tasks.
From Table~\ref{tab:cap3}, we can observe that compared with other models trained on Artphoto, we achieve competitive zero-shot performance, which shows that the model has good generalization ability in out-of-distribution tasks.


%%%%%%%%%%%%
%  E2E     %
%%%%%%%%%%%%


\section{Conclusion}
In this paper, we introduced Wi-Chat, the first LLM-powered Wi-Fi-based human activity recognition system that integrates the reasoning capabilities of large language models with the sensing potential of wireless signals. Our experimental results on a self-collected Wi-Fi CSI dataset demonstrate the promising potential of LLMs in enabling zero-shot Wi-Fi sensing. These findings suggest a new paradigm for human activity recognition that does not rely on extensive labeled data. We hope future research will build upon this direction, further exploring the applications of LLMs in signal processing domains such as IoT, mobile sensing, and radar-based systems.

\section*{Limitations}
While our work represents the first attempt to leverage LLMs for processing Wi-Fi signals, it is a preliminary study focused on a relatively simple task: Wi-Fi-based human activity recognition. This choice allows us to explore the feasibility of LLMs in wireless sensing but also comes with certain limitations.

Our approach primarily evaluates zero-shot performance, which, while promising, may still lag behind traditional supervised learning methods in highly complex or fine-grained recognition tasks. Besides, our study is limited to a controlled environment with a self-collected dataset, and the generalizability of LLMs to diverse real-world scenarios with varying Wi-Fi conditions, environmental interference, and device heterogeneity remains an open question.

Additionally, we have yet to explore the full potential of LLMs in more advanced Wi-Fi sensing applications, such as fine-grained gesture recognition, occupancy detection, and passive health monitoring. Future work should investigate the scalability of LLM-based approaches, their robustness to domain shifts, and their integration with multimodal sensing techniques in broader IoT applications.


% Bibliography entries for the entire Anthology, followed by custom entries
%\bibliography{anthology,custom}
% Custom bibliography entries only
\bibliography{main}
\newpage
\appendix

\section{Experiment prompts}
\label{sec:prompt}
The prompts used in the LLM experiments are shown in the following Table~\ref{tab:prompts}.

\definecolor{titlecolor}{rgb}{0.9, 0.5, 0.1}
\definecolor{anscolor}{rgb}{0.2, 0.5, 0.8}
\definecolor{labelcolor}{HTML}{48a07e}
\begin{table*}[h]
	\centering
	
 % \vspace{-0.2cm}
	
	\begin{center}
		\begin{tikzpicture}[
				chatbox_inner/.style={rectangle, rounded corners, opacity=0, text opacity=1, font=\sffamily\scriptsize, text width=5in, text height=9pt, inner xsep=6pt, inner ysep=6pt},
				chatbox_prompt_inner/.style={chatbox_inner, align=flush left, xshift=0pt, text height=11pt},
				chatbox_user_inner/.style={chatbox_inner, align=flush left, xshift=0pt},
				chatbox_gpt_inner/.style={chatbox_inner, align=flush left, xshift=0pt},
				chatbox/.style={chatbox_inner, draw=black!25, fill=gray!7, opacity=1, text opacity=0},
				chatbox_prompt/.style={chatbox, align=flush left, fill=gray!1.5, draw=black!30, text height=10pt},
				chatbox_user/.style={chatbox, align=flush left},
				chatbox_gpt/.style={chatbox, align=flush left},
				chatbox2/.style={chatbox_gpt, fill=green!25},
				chatbox3/.style={chatbox_gpt, fill=red!20, draw=black!20},
				chatbox4/.style={chatbox_gpt, fill=yellow!30},
				labelbox/.style={rectangle, rounded corners, draw=black!50, font=\sffamily\scriptsize\bfseries, fill=gray!5, inner sep=3pt},
			]
											
			\node[chatbox_user] (q1) {
				\textbf{System prompt}
				\newline
				\newline
				You are a helpful and precise assistant for segmenting and labeling sentences. We would like to request your help on curating a dataset for entity-level hallucination detection.
				\newline \newline
                We will give you a machine generated biography and a list of checked facts about the biography. Each fact consists of a sentence and a label (True/False). Please do the following process. First, breaking down the biography into words. Second, by referring to the provided list of facts, merging some broken down words in the previous step to form meaningful entities. For example, ``strategic thinking'' should be one entity instead of two. Third, according to the labels in the list of facts, labeling each entity as True or False. Specifically, for facts that share a similar sentence structure (\eg, \textit{``He was born on Mach 9, 1941.''} (\texttt{True}) and \textit{``He was born in Ramos Mejia.''} (\texttt{False})), please first assign labels to entities that differ across atomic facts. For example, first labeling ``Mach 9, 1941'' (\texttt{True}) and ``Ramos Mejia'' (\texttt{False}) in the above case. For those entities that are the same across atomic facts (\eg, ``was born'') or are neutral (\eg, ``he,'' ``in,'' and ``on''), please label them as \texttt{True}. For the cases that there is no atomic fact that shares the same sentence structure, please identify the most informative entities in the sentence and label them with the same label as the atomic fact while treating the rest of the entities as \texttt{True}. In the end, output the entities and labels in the following format:
                \begin{itemize}[nosep]
                    \item Entity 1 (Label 1)
                    \item Entity 2 (Label 2)
                    \item ...
                    \item Entity N (Label N)
                \end{itemize}
                % \newline \newline
                Here are two examples:
                \newline\newline
                \textbf{[Example 1]}
                \newline
                [The start of the biography]
                \newline
                \textcolor{titlecolor}{Marianne McAndrew is an American actress and singer, born on November 21, 1942, in Cleveland, Ohio. She began her acting career in the late 1960s, appearing in various television shows and films.}
                \newline
                [The end of the biography]
                \newline \newline
                [The start of the list of checked facts]
                \newline
                \textcolor{anscolor}{[Marianne McAndrew is an American. (False); Marianne McAndrew is an actress. (True); Marianne McAndrew is a singer. (False); Marianne McAndrew was born on November 21, 1942. (False); Marianne McAndrew was born in Cleveland, Ohio. (False); She began her acting career in the late 1960s. (True); She has appeared in various television shows. (True); She has appeared in various films. (True)]}
                \newline
                [The end of the list of checked facts]
                \newline \newline
                [The start of the ideal output]
                \newline
                \textcolor{labelcolor}{[Marianne McAndrew (True); is (True); an (True); American (False); actress (True); and (True); singer (False); , (True); born (True); on (True); November 21, 1942 (False); , (True); in (True); Cleveland, Ohio (False); . (True); She (True); began (True); her (True); acting career (True); in (True); the late 1960s (True); , (True); appearing (True); in (True); various (True); television shows (True); and (True); films (True); . (True)]}
                \newline
                [The end of the ideal output]
				\newline \newline
                \textbf{[Example 2]}
                \newline
                [The start of the biography]
                \newline
                \textcolor{titlecolor}{Doug Sheehan is an American actor who was born on April 27, 1949, in Santa Monica, California. He is best known for his roles in soap operas, including his portrayal of Joe Kelly on ``General Hospital'' and Ben Gibson on ``Knots Landing.''}
                \newline
                [The end of the biography]
                \newline \newline
                [The start of the list of checked facts]
                \newline
                \textcolor{anscolor}{[Doug Sheehan is an American. (True); Doug Sheehan is an actor. (True); Doug Sheehan was born on April 27, 1949. (True); Doug Sheehan was born in Santa Monica, California. (False); He is best known for his roles in soap operas. (True); He portrayed Joe Kelly. (True); Joe Kelly was in General Hospital. (True); General Hospital is a soap opera. (True); He portrayed Ben Gibson. (True); Ben Gibson was in Knots Landing. (True); Knots Landing is a soap opera. (True)]}
                \newline
                [The end of the list of checked facts]
                \newline \newline
                [The start of the ideal output]
                \newline
                \textcolor{labelcolor}{[Doug Sheehan (True); is (True); an (True); American (True); actor (True); who (True); was born (True); on (True); April 27, 1949 (True); in (True); Santa Monica, California (False); . (True); He (True); is (True); best known (True); for (True); his roles in soap operas (True); , (True); including (True); in (True); his portrayal (True); of (True); Joe Kelly (True); on (True); ``General Hospital'' (True); and (True); Ben Gibson (True); on (True); ``Knots Landing.'' (True)]}
                \newline
                [The end of the ideal output]
				\newline \newline
				\textbf{User prompt}
				\newline
				\newline
				[The start of the biography]
				\newline
				\textcolor{magenta}{\texttt{\{BIOGRAPHY\}}}
				\newline
				[The ebd of the biography]
				\newline \newline
				[The start of the list of checked facts]
				\newline
				\textcolor{magenta}{\texttt{\{LIST OF CHECKED FACTS\}}}
				\newline
				[The end of the list of checked facts]
			};
			\node[chatbox_user_inner] (q1_text) at (q1) {
				\textbf{System prompt}
				\newline
				\newline
				You are a helpful and precise assistant for segmenting and labeling sentences. We would like to request your help on curating a dataset for entity-level hallucination detection.
				\newline \newline
                We will give you a machine generated biography and a list of checked facts about the biography. Each fact consists of a sentence and a label (True/False). Please do the following process. First, breaking down the biography into words. Second, by referring to the provided list of facts, merging some broken down words in the previous step to form meaningful entities. For example, ``strategic thinking'' should be one entity instead of two. Third, according to the labels in the list of facts, labeling each entity as True or False. Specifically, for facts that share a similar sentence structure (\eg, \textit{``He was born on Mach 9, 1941.''} (\texttt{True}) and \textit{``He was born in Ramos Mejia.''} (\texttt{False})), please first assign labels to entities that differ across atomic facts. For example, first labeling ``Mach 9, 1941'' (\texttt{True}) and ``Ramos Mejia'' (\texttt{False}) in the above case. For those entities that are the same across atomic facts (\eg, ``was born'') or are neutral (\eg, ``he,'' ``in,'' and ``on''), please label them as \texttt{True}. For the cases that there is no atomic fact that shares the same sentence structure, please identify the most informative entities in the sentence and label them with the same label as the atomic fact while treating the rest of the entities as \texttt{True}. In the end, output the entities and labels in the following format:
                \begin{itemize}[nosep]
                    \item Entity 1 (Label 1)
                    \item Entity 2 (Label 2)
                    \item ...
                    \item Entity N (Label N)
                \end{itemize}
                % \newline \newline
                Here are two examples:
                \newline\newline
                \textbf{[Example 1]}
                \newline
                [The start of the biography]
                \newline
                \textcolor{titlecolor}{Marianne McAndrew is an American actress and singer, born on November 21, 1942, in Cleveland, Ohio. She began her acting career in the late 1960s, appearing in various television shows and films.}
                \newline
                [The end of the biography]
                \newline \newline
                [The start of the list of checked facts]
                \newline
                \textcolor{anscolor}{[Marianne McAndrew is an American. (False); Marianne McAndrew is an actress. (True); Marianne McAndrew is a singer. (False); Marianne McAndrew was born on November 21, 1942. (False); Marianne McAndrew was born in Cleveland, Ohio. (False); She began her acting career in the late 1960s. (True); She has appeared in various television shows. (True); She has appeared in various films. (True)]}
                \newline
                [The end of the list of checked facts]
                \newline \newline
                [The start of the ideal output]
                \newline
                \textcolor{labelcolor}{[Marianne McAndrew (True); is (True); an (True); American (False); actress (True); and (True); singer (False); , (True); born (True); on (True); November 21, 1942 (False); , (True); in (True); Cleveland, Ohio (False); . (True); She (True); began (True); her (True); acting career (True); in (True); the late 1960s (True); , (True); appearing (True); in (True); various (True); television shows (True); and (True); films (True); . (True)]}
                \newline
                [The end of the ideal output]
				\newline \newline
                \textbf{[Example 2]}
                \newline
                [The start of the biography]
                \newline
                \textcolor{titlecolor}{Doug Sheehan is an American actor who was born on April 27, 1949, in Santa Monica, California. He is best known for his roles in soap operas, including his portrayal of Joe Kelly on ``General Hospital'' and Ben Gibson on ``Knots Landing.''}
                \newline
                [The end of the biography]
                \newline \newline
                [The start of the list of checked facts]
                \newline
                \textcolor{anscolor}{[Doug Sheehan is an American. (True); Doug Sheehan is an actor. (True); Doug Sheehan was born on April 27, 1949. (True); Doug Sheehan was born in Santa Monica, California. (False); He is best known for his roles in soap operas. (True); He portrayed Joe Kelly. (True); Joe Kelly was in General Hospital. (True); General Hospital is a soap opera. (True); He portrayed Ben Gibson. (True); Ben Gibson was in Knots Landing. (True); Knots Landing is a soap opera. (True)]}
                \newline
                [The end of the list of checked facts]
                \newline \newline
                [The start of the ideal output]
                \newline
                \textcolor{labelcolor}{[Doug Sheehan (True); is (True); an (True); American (True); actor (True); who (True); was born (True); on (True); April 27, 1949 (True); in (True); Santa Monica, California (False); . (True); He (True); is (True); best known (True); for (True); his roles in soap operas (True); , (True); including (True); in (True); his portrayal (True); of (True); Joe Kelly (True); on (True); ``General Hospital'' (True); and (True); Ben Gibson (True); on (True); ``Knots Landing.'' (True)]}
                \newline
                [The end of the ideal output]
				\newline \newline
				\textbf{User prompt}
				\newline
				\newline
				[The start of the biography]
				\newline
				\textcolor{magenta}{\texttt{\{BIOGRAPHY\}}}
				\newline
				[The ebd of the biography]
				\newline \newline
				[The start of the list of checked facts]
				\newline
				\textcolor{magenta}{\texttt{\{LIST OF CHECKED FACTS\}}}
				\newline
				[The end of the list of checked facts]
			};
		\end{tikzpicture}
        \caption{GPT-4o prompt for labeling hallucinated entities.}\label{tb:gpt-4-prompt}
	\end{center}
\vspace{-0cm}
\end{table*}
% \section{Full Experiment Results}
% \begin{table*}[th]
    \centering
    \small
    \caption{Classification Results}
    \begin{tabular}{lcccc}
        \toprule
        \textbf{Method} & \textbf{Accuracy} & \textbf{Precision} & \textbf{Recall} & \textbf{F1-score} \\
        \midrule
        \multicolumn{5}{c}{\textbf{Zero Shot}} \\
                Zero-shot E-eyes & 0.26 & 0.26 & 0.27 & 0.26 \\
        Zero-shot CARM & 0.24 & 0.24 & 0.24 & 0.24 \\
                Zero-shot SVM & 0.27 & 0.28 & 0.28 & 0.27 \\
        Zero-shot CNN & 0.23 & 0.24 & 0.23 & 0.23 \\
        Zero-shot RNN & 0.26 & 0.26 & 0.26 & 0.26 \\
DeepSeek-0shot & 0.54 & 0.61 & 0.54 & 0.52 \\
DeepSeek-0shot-COT & 0.33 & 0.24 & 0.33 & 0.23 \\
DeepSeek-0shot-Knowledge & 0.45 & 0.46 & 0.45 & 0.44 \\
Gemma2-0shot & 0.35 & 0.22 & 0.38 & 0.27 \\
Gemma2-0shot-COT & 0.36 & 0.22 & 0.36 & 0.27 \\
Gemma2-0shot-Knowledge & 0.32 & 0.18 & 0.34 & 0.20 \\
GPT-4o-mini-0shot & 0.48 & 0.53 & 0.48 & 0.41 \\
GPT-4o-mini-0shot-COT & 0.33 & 0.50 & 0.33 & 0.38 \\
GPT-4o-mini-0shot-Knowledge & 0.49 & 0.31 & 0.49 & 0.36 \\
GPT-4o-0shot & 0.62 & 0.62 & 0.47 & 0.42 \\
GPT-4o-0shot-COT & 0.29 & 0.45 & 0.29 & 0.21 \\
GPT-4o-0shot-Knowledge & 0.44 & 0.52 & 0.44 & 0.39 \\
LLaMA-0shot & 0.32 & 0.25 & 0.32 & 0.24 \\
LLaMA-0shot-COT & 0.12 & 0.25 & 0.12 & 0.09 \\
LLaMA-0shot-Knowledge & 0.32 & 0.25 & 0.32 & 0.28 \\
Mistral-0shot & 0.19 & 0.23 & 0.19 & 0.10 \\
Mistral-0shot-Knowledge & 0.21 & 0.40 & 0.21 & 0.11 \\
        \midrule
        \multicolumn{5}{c}{\textbf{4 Shot}} \\
GPT-4o-mini-4shot & 0.58 & 0.59 & 0.58 & 0.53 \\
GPT-4o-mini-4shot-COT & 0.57 & 0.53 & 0.57 & 0.50 \\
GPT-4o-mini-4shot-Knowledge & 0.56 & 0.51 & 0.56 & 0.47 \\
GPT-4o-4shot & 0.77 & 0.84 & 0.77 & 0.73 \\
GPT-4o-4shot-COT & 0.63 & 0.76 & 0.63 & 0.53 \\
GPT-4o-4shot-Knowledge & 0.72 & 0.82 & 0.71 & 0.66 \\
LLaMA-4shot & 0.29 & 0.24 & 0.29 & 0.21 \\
LLaMA-4shot-COT & 0.20 & 0.30 & 0.20 & 0.13 \\
LLaMA-4shot-Knowledge & 0.15 & 0.23 & 0.13 & 0.13 \\
Mistral-4shot & 0.02 & 0.02 & 0.02 & 0.02 \\
Mistral-4shot-Knowledge & 0.21 & 0.27 & 0.21 & 0.20 \\
        \midrule
        
        \multicolumn{5}{c}{\textbf{Suprevised}} \\
        SVM & 0.94 & 0.92 & 0.91 & 0.91 \\
        CNN & 0.98 & 0.98 & 0.97 & 0.97 \\
        RNN & 0.99 & 0.99 & 0.99 & 0.99 \\
        % \midrule
        % \multicolumn{5}{c}{\textbf{Conventional Wi-Fi-based Human Activity Recognition Systems}} \\
        E-eyes & 1.00 & 1.00 & 1.00 & 1.00 \\
        CARM & 0.98 & 0.98 & 0.98 & 0.98 \\
\midrule
 \multicolumn{5}{c}{\textbf{Vision Models}} \\
           Zero-shot SVM & 0.26 & 0.25 & 0.25 & 0.25 \\
        Zero-shot CNN & 0.26 & 0.25 & 0.26 & 0.26 \\
        Zero-shot RNN & 0.28 & 0.28 & 0.29 & 0.28 \\
        SVM & 0.99 & 0.99 & 0.99 & 0.99 \\
        CNN & 0.98 & 0.99 & 0.98 & 0.98 \\
        RNN & 0.98 & 0.99 & 0.98 & 0.98 \\
GPT-4o-mini-Vision & 0.84 & 0.85 & 0.84 & 0.84 \\
GPT-4o-mini-Vision-COT & 0.90 & 0.91 & 0.90 & 0.90 \\
GPT-4o-Vision & 0.74 & 0.82 & 0.74 & 0.73 \\
GPT-4o-Vision-COT & 0.70 & 0.83 & 0.70 & 0.68 \\
LLaMA-Vision & 0.20 & 0.23 & 0.20 & 0.09 \\
LLaMA-Vision-Knowledge & 0.22 & 0.05 & 0.22 & 0.08 \\

        \bottomrule
    \end{tabular}
    \label{full}
\end{table*}




\end{document}
 

%% For numbered reference style
%% \bibitem{label}
%% Text of bibliographic item

%%\bibliographystyle{elsarticle-num} 
%%\bibliography{references}






%% The Appendices part is started with the command \appendix;
%% appendix sections are then done as normal sections

\end{document}

%% If you have bib database file and want bibtex to generate the
%% bibitems, please use
%%
%%  \bibliographystyle{elsarticle-num} 
%%  \bibliography{<your bibdatabase>}

%% else use the following coding to input the bibitems directly in the
%% TeX file.

%% Refer following link for more details about bibliography and citations.
%% https://en.wikibooks.org/wiki/LaTeX/Bibliography_Management



\endinput
%%
%% End of file `elsarticle-template-num.tex'.
