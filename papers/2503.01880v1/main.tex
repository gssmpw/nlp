%% 
%% Copyright 2007-2024 Elsevier Ltd
%% 
%% This file is part of the 'Elsarticle Bundle'.
%% ---------------------------------------------
%% 
%% It may be distributed under the conditions of the LaTeX Project Public
%% License, either version 1.3 of this license or (at your option) any
%% later version.  The latest version of this license is in
%%    http://www.latex-project.org/lppl.txt
%% and version 1.3 or later is part of all distributions of LaTeX
%% version 1999/12/01 or later.
%% 
%% The list of all files belonging to the 'Elsarticle Bundle' is
%% given in the file `manifest.txt'.
%% 
%% Template article for Elsevier's document class `elsarticle'
%% with numbered style bibliographic references
%% SP 2008/03/01
%% $Id: elsarticle-template-num.tex 249 2024-04-06 10:51:24Z rishi $
%%
\documentclass[12pt]{elsarticle}

%% Use the option review to obtain double line spacing
%% \documentclass[authoryear,preprint,review,12pt]{elsarticle}

%% Use the options 1p,twocolumn; 3p; 3p,twocolumn; 5p; or 5p,twocolumn
%% for a journal layout:
%% \documentclass[final,1p,times]{elsarticle}
%% \documentclass[final,1p,times,twocolumn]{elsarticle}
%% \documentclass[final,3p,times]{elsarticle}
%% \documentclass[final,3p,times,twocolumn]{elsarticle}
%% \documentclass[final,5p,times]{elsarticle}
%% \documentclass[final,5p,times,twocolumn]{elsarticle}

%% For including figures, graphicx.sty has been loaded in
%% elsarticle.cls. If you prefer to use the old commands
%% please give \usepackage{epsfig}

%% The amssymb package provides various useful mathematical symbols
\usepackage{amssymb}
%% The amsmath package provides various useful equation environments.
\usepackage{subcaption}
\usepackage{multirow}
\usepackage{makecell}
\usepackage{amsmath}
\usepackage{afterpage}
\usepackage{rotating}
\usepackage{subcaption}
\usepackage{algorithm}
\usepackage{lineno}
\usepackage{algpseudocode}
\usepackage{amsmath}
\usepackage{graphicx}
\usepackage{adjustbox}

\usepackage[utf8]{inputenc} % allow utf-8 input
\usepackage[T1]{fontenc}    % use 8-bit T1 fonts
\usepackage{hyperref}       % hyperlinks
\usepackage{url}            % simple URL typesetting
\usepackage{booktabs}       % professional-quality tables
\usepackage{amsfonts}       % blackboard math symbols
\usepackage{nicefrac}       % compact symbols for 1/2, etc.
\usepackage{microtype}      % microtypography
\usepackage{lipsum}
\usepackage{graphicx}
\graphicspath{{media/}}     % organize your images and other figures under media/ folder
\usepackage{float}
%% The amsthm package provides extended theorem environments
%% \usepackage{amsthm}

%% The lineno packages adds line numbers. Start line numbering with
%% \begin{linenumbers}, end it with \end{linenumbers}. Or switch it on
%% for the whole article with \linenumbers.
%% \usepackage{lineno}

\journal{arxiv}

\begin{document}

\begin{frontmatter}

%% Title, authors and addresses

%% use the tnoteref command within \title for footnotes;
%% use the tnotetext command for theassociated footnote;
%% use the fnref command within \author or \affiliation for footnotes;
%% use the fntext command for theassociated footnote;
%% use the corref command within \author for corresponding author footnotes;
%% use the cortext command for theassociated footnote;
%% use the ead command for the email address,
%% and the form \ead[url] for the home page:
%% \title{Title\tnoteref{label1}}
%% \tnotetext[label1]{}
%% \author{Name\corref{cor1}\fnref{label2}}
%% \ead{email address}
%% \ead[url]{home page}
%% \fntext[label2]{}
%% \cortext[cor1]{}
%affiliation{organization={},
%%             addressline={},
%%             city={},
%%             postcode={},
%%             state={},
%%             country={}}
%% \fntext[label3]{}

\title{BEYONDWORDS is All You Need: Agentic Generative AI based Social Media Themes Extractor}

%% use optional labels to link authors explicitly to addresses:
%% \author[label1,label2]{}
%% \affiliation[label1]{organization={},
%%             addressline={},
%%             city={},
%%             postcode={},
%%             state={},
%%             country={}}
%%
%% \affiliation[label2]{organization={},
%%             addressline={},
%%             city={},
%%             postcode={},
%%             state={},
%%             country={}}

\author[label1]{Mohammed-Khalil Ghali} %% Author name
\author[label1]{Abdelrahman Farrag} %% Author name
\author[label1]{Sarah Lam}
\author[label1]{Daehan Won}

%% Author name

%% Author affiliation
\affiliation[label1]{organization={School of Systems Science and Industrial Engineering, State University of New York at Binghamton},%Department and Organization
            addressline={4400 Vestal Pkwy}, 
            city={Binghamton},
            postcode={13902}, 
            state={NY},
            country={USA}}
            

%% Abstract
\begin{abstract}
%% Text of abstract
Thematic analysis of social media posts provides a major understanding of public discourse, yet traditional methods often struggle to capture the complexity and nuance of unstructured, large-scale text data. This study introduces a novel methodology for thematic analysis that integrates tweet embeddings from pre-trained language models, dimensionality reduction using  and matrix factorization, and generative AI to identify and refine latent themes. Our approach clusters compressed tweet representations and employs generative AI to extract and articulate themes through an agentic Chain of Thought (CoT) prompting, with a secondary LLM for quality assurance.
This methodology is applied to tweets from the autistic community, a group that increasingly uses social media to discuss their experiences and challenges. By automating the thematic extraction process, the aim is to uncover key insights while maintaining the richness of the original discourse. This autism case study demonstrates the utility of the proposed approach in improving thematic analysis of social media data, offering a scalable and adaptable framework that can be applied to diverse contexts. The results highlight the potential of combining machine learning and Generative AI to enhance the depth and accuracy of theme identification in online communities.
\end{abstract}

%\begin{graphicalabstract}
%    \centering
%    \begin{figure}
%        \centering
%        \begin{subfigure}{\textwidth}
%            \centering
%            \includegraphics[width=\linewidth]{images/Methodology (2).png}
%            \caption{GTR model}
%            \label{fig:imageA}
%        \end{subfigure}
%        \vspace{0.5cm} % Space between the images
%        \begin{subfigure}{\textwidth}
%            \centering
%            \includegraphics[width=\linewidth]{images/texttosql.png}
%            \caption{GTR-T model}
%        \end{subfigure}
%        \caption{GTR and GTR-T models graphical abstracts}
%        \label{fig:general}
%    \end{figure}
%\end{graphicalabstract}



%%Research highlights
\begin{highlights}

\item The paper presents BEYONDWORDS, a novel generative AI-based framework for theme extraction, aiming to automatically extract and refine thematic structures from large-scale social media data
\item It integrates embeddings, autoencoder neural network for dimensionality compression, matrix factorization, and Chain of Thought prompting with Large Language Models(LLMS) to recursively uncover and enhance the themes
\item The results show that the three main topics that are widely discussed in autistic community tweets are \textbf{social media content quality and engagement; advocacy for autistic rights and acceptance; mental health and well-being}
\end{highlights}

%% Keywords
\begin{keyword}
Named Entity Recognition \sep Large Language Models \sep Generative Artificial Intelligence \sep Medical Data Extraction \sep Prompts Engineering

%% keywords here, in the form: keyword \sep keyword

%% PACS codes here, in the form: \PACS code \sep code

%% MSC codes here, in the form: \MSC code \sep code
%% or \MSC[2008] code \sep code (2000 is the default)

\end{keyword}

\end{frontmatter}

%% Add \usepackage{lineno} before \begin{document} and uncomment 
%% following line to enable line numbers
%% \linenumbers

%% main text
%%

%% Use \section commands to start a section
\section{Introduction}
The rapid growth of social media has created vast repositories of user-generated content, offering unique insights into diverse communities and social phenomena. For researchers, social media platforms such as X(Twitter previously) provide rich datasets to study public discourse, opinions, and trends. However, the unstructured and high dimensional nature of social media data presents challenges for thematic analysis, as traditional methods, like manual coding or frequency-based approaches (e.g., TF-IDF), often fail to capture the semantic depth and context embedded in posts. These limitations call for more sophisticated techniques capable of handling large-scale textual data and uncovering latent themes that reflect the nuanced perspectives expressed online.

In recent years, advancements in natural language processing (NLP) have introduced embedding techniques powered by pre-trained language models, which enable the transformation of raw text into dense, high dimensional vector representations that encapsulate both semantic meaning and contextual information. These embeddings, when coupled with dimensionality compression techniques such as autoencoder neural network and matrix factorization, can be leveraged to identify underlying themes in vast corpora of social media posts. Furthermore, the integration of generative AI models offers a novel approach to extracting and articulating these themes, moving beyond surface-level keyword extraction to generate coherent and meaningful and accurate themes.

This paper presents a general methodology for thematic analysis of social media data that combines embedding-based representations, dimensionality reduction, and generative AI. The proposed framework consists of three key components: (1) extracting social media embeddings using pre-trained language models, (2) applying dimensionality compression through autoencoder neural network, matrix factorization to uncover latent themes, and (3) utilizing generative AI through LLMs with a Chain of Thought (CoT) prompting mechanism to extract and refine these themes. 

As a case study, this methodology is applied to the analysis of tweets from the autistic community. Social media platforms have become an important space for autistic individuals to share their experiences, challenges, and perspectives on autism, often bypassing traditional gatekeepers of discourse. However, manual thematic analysis of such posts is labor-intensive and subject to biases introduced by human coders. By automating the process through the integration of machine learning and AI techniques, this methodology aims to extract themes that offer a more comprehensive and nuanced understanding of the discourse within the autistic community.

The contributions of this paper are twofold. First, it proposes a scalable and adaptable framework for thematic analysis of large-scale social media datasets, addressing the limitations of existing methods. Second, it provides insights into the experiences and challenges of the autistic community, potentially supporting more specialized care for this community, which requires the utmost support. This case study demonstrates the practical applicability of the methodology. The approach can be adapted for thematic analysis in other contexts, making it a versatile framework for researchers studying online discourse across various social media platforms.

The paper is structured as follows: Section 2 reviews related work on social media analysis, Section 3 discusses the components of the proposed methodology in detail, Section 4 discusses the case study and the dataset used for testing the proposed framework and Section 5 presents the results of the case study. Section 6 concludes with an evaluation of the strengths and limitations of the approach and a discussion of future directions for research.



\section{Related work}\label{sec:LR}

Thematic analysis is a key qualitative method used to identify patterns and themes within textual data. As social media has become a rich source of public discourse, the ability to analyze this unstructured and large-scale data has become crucial. Traditional methods of thematic analysis, though valuable for small datasets, often struggle with the volume, diversity, and complexity of social media text. This literature review outlines the progression of thematic analysis methods, highlighting the limitations of traditional techniques and how recent advancements, particularly our proposed approach, fill these gaps.
\newline
\subsection{Traditional Thematic Analysis in Social Media Research}

Thematic analysis has long been used to explore qualitative data, with foundational guidelines established by \cite{braun2006using}. Their method involves a systematic, manual coding process that identifies recurring themes through a close reading of the text. While this method remains highly respected in qualitative research, its application to social media data is problematic due to the large volume and complexity of text involved \cite{braun2019reflecting}.
Manual coding becomes time-consuming and unfeasible when applied to datasets that may include millions of posts, such as Twitter data \cite{boyatzis1998transforming}. In such cases, traditional thematic analysis often requires a reduced sample size or relies on researchers' subjective judgments to identify salient themes. This introduces biases and limits the generalizability of the findings \cite{markham2017research}. Additionally, the informal, fast-paced nature of social media language—frequently composed of abbreviations, slang, and emotive content—makes manual coding less reliable, as it struggles to capture nuances or variations in context \cite{golder2014digital}.
Due to these limitations, researchers began exploring computational methods for thematic analysis. Early approaches like word frequency analysis and topic modeling with Latent Dirichlet Allocation (LDA) became popular tools to analyze large datasets \cite{blei2003latent}. LDA has been applied extensively in social media research, particularly in studies focused on public opinion, social movements, and political discourse \cite{nguyen2015sentiment}. However, LDA's assumption that topics are merely distributions of words across documents oversimplifies the language and cannot capture the nuances of meaning, sentiment, and context present in complex social media texts \cite{chuang2012termite}.
Moreover, topic modeling techniques such as LDA struggle with the diverse, informal nature of social media language, often resulting in themes that are overly broad or too fragmented to be meaningful \cite{hong2010empirical}. Our proposed methodology directly addresses these limitations by incorporating advanced techniques like tweet embeddings from pre-trained language models, enabling a more nuanced understanding of social media discourse by preserving contextual relationships between words \cite{devlin2018bert}.
\newline
\subsection{Advances in Machine Learning and NLP for Thematic Analysis}

Recent advancements in natural language processing (NLP) have enhanced thematic analysis techniques, allowing researchers to overcome some of the limitations posed by traditional methods. The introduction of transformer-based language models, such as BERT \cite{devlin2018bert}, GPT-2/3 \cite{brown2020language}, and others, has significantly improved the ability to understand semantic relationships in text. These models leverage vast amounts of training data to generate contextualized word embeddings, offering deeper insights into text than earlier models like LDA or TF-IDF \cite{mikolov2013distributed}.

Two main studies rely on topic modeling and sentiment analysis to understand themes related to autism on Twitter. The first study by \cite{corti2022social} uses Non-Negative Matrix Factorization (NMF) for topic modeling and sentiment analysis with AFiNN, while the second study \cite{gabarron2023autistic} applies the BERTopic model for clustering tweets and extracting themes using dimensionality reduction via UMAP. Despite their contributions, these approaches exhibit limitations. NMF, although useful for topic coherence, lacks the ability to capture deep linguistic nuances, particularly in social media language. The BERTopic approach, while more advanced, relies on static embeddings and bag-of-words methods that may overlook semantic richness and context within tweets, especially in complex discourse like the autism community’s.

The use of pre-trained language models in thematic analysis has been explored in recent studies. \cite{wu2022study} applied BERT embeddings to cluster social media posts about public health issues, demonstrating improved coherence in the resulting themes compared to LDA. Likewise, \cite{yin2020detecting} showed that using contextualized embeddings from language models improved the ability to detect latent topics in crisis communication data. These studies highlight the superior capacity of pre-trained models to handle social media’s evolving terminology and informal language structures.

However, while embeddings capture rich semantic information, high dimensional representations pose challenges for downstream analysis. Dimensionality reduction techniques like autoencoder neural network and matrix factorization have emerged as effective ways to reduce complexity while preserving thematic content. \cite{chauhan2024tracking} demonstrated the utility of autoencoder neural network in compressing embeddings for more efficient clustering in sentiment analysis, which improves both the scalability and interpretability of results.

Our methodology builds on these approaches by integrating autoencoder neural network with matrix factorization to uncover latent themes, retaining the rich semantic information encoded in the tweet embeddings while effectively reducing dimensionality. This combination ensures that nuanced themes can be extracted from large datasets without losing the contextual meaning captured by the language models.
\newline
\subsection{Generative AI and Iterative Theme Refinement}

Generative AI models, such as GPT-3, have shown promise in thematic analysis by automating tasks such as text summarization, classification, and thematic extraction \cite{brown2020language}. These models generate coherent and contextually appropriate text, which can be useful for extracting or refining themes from large datasets. For example, studies by \cite{dong2024understanding} explored generative models for summarizing social media content, underscoring their potential for condensing vast amounts of text into coherent themes.

Despite this potential, generative AI has rarely been used in a more iterative, refining role in thematic analysis. Previous studies have typically applied generative models for summarization or categorization, without refining or validating the themes through multiple stages of analysis. This paper's \cite{wanna2024topictag} use CoT prompting is limited in that it primarily focuses on improving topic labels through token-based features and manual filtering, without fully leveraging the iterative reasoning potential of CoT for deeper semantic understanding. Additionally, their approach to CoT is more narrowly applied to optimize prompt tuning rather than refining the thematic structure of the topics themselves.
This gap is addressed in our methodology by using an iterative CoT prompting mechanism, which allows for iterative theme extraction and refinement. This process ensures that the themes are coherent, contextually accurate, and relevant to the dataset.

Although recent advancements in NLP and machine learning have improved thematic analysis, several critical gaps remain. First, many existing models still struggle to capture the complexity and nuance of informal social media language, particularly in communities with specialized or evolving vocabularies, such as the autistic community. LDA and word frequency-based methods, while useful for structured data, oversimplify language and miss deeper contextual meanings \cite{chuang2012termite}.
Second, many machine learning methods lack iterative processes for refining and validating themes. Pre-trained language models and clustering methods have advanced theme extraction, but they do not inherently include mechanisms for ensuring thematic consistency or addressing ambiguous themes \cite{wu2022study,yin2020detecting}. Moreover, generative AI models applied to thematic analysis have mostly focused on single-pass theme identification without multiple stages of validation or refinement, leading to incomplete or inconsistent results and the LLMs have fixed context windows \cite{dong2024understanding} and are not able to ingest text beyond that which is why clustering needs to be done before feeding the results to LLMs.
% should this be part of methodology or good to have in literature review
The proposed methodology addresses these gaps by:
\begin{enumerate}
    \item Generating tweet embeddings from pre-trained language models to capture semantic relationships and nuances in social media language. This is crucial for analyzing discourse in specialized communities, like the autistic community, which is the focus of this case study.
    \item Applying dimensionality reduction by integrating autoencoder neural network and matrix factorization to reduce the complexity of embeddings without losing key thematic information while clustering the relevant social media posts together. This approach overcomes the limitations of high-dimensional data.
    \item To balance the high cost of token generation using LLMs with the need to retain meaningful cluster information, a sample size was chosen that is representative enough to capture the overall themes of the entire cluster while minimizing the data passed to the LLMs.
    \item Using generative AI for iterative refinement by employing agentic CoT prompting mechanism to iteratively extract and refine themes of the extracted clusters, ensuring that the identified themes are accurate and contextually relevant. The inclusion of a secondary LLM for quality control further enhances the reliability of the analysis, addressing common issues with theme consistency in automated methods.
\end{enumerate}
Table \ref{tab:lit_rev} summarizes previous research conducted to address the problem of social media theme extraction and highlights how the proposed methodology aims to fill the gaps identified in these studies.
\begin{table}[H]
\centering
\caption{Comparison of Thematic Analysis Methodologies for Social Media Research}
\label{tab:lit_rev}
\begin{adjustbox}{max width=\textwidth}
\begin{tabular}{>{\raggedright\arraybackslash}p{3.5cm} >{\raggedright\arraybackslash}p{3.5cm} >{\raggedright\arraybackslash}p{3.5cm} >{\raggedright\arraybackslash}p{3.5cm}}
\hline
\textbf{Methodology} & \textbf{Focus} & \textbf{Weaknesses} & \textbf{Advantage of Proposed Methodology} \\ \hline
Traditional Thematic Analysis \cite{braun2006using} & Manual coding of text  & Unfeasible for large datasets, prone to bias & Scalable embedding-based approach retains theme detail while handling large volumes \\ \hline
Latent Dirichlet Allocation (LDA) \cite{blei2003latent} & Probabilistic topic modeling for large datasets  & Overly broad themes, lacks nuanced context & Embedding-based clustering captures semantic nuance; maintains coherence \\ \hline
Non-Negative Matrix Factorization (NMF) \cite{corti2022social} & Topic coherence via dimensionality reduction  & Limited in capturing informal language nuance & Embeddings and autoencoder neural network refine themes with more linguistic context \\ \hline
BERTopic \cite{gabarron2023autistic} & Clustering and topic modeling using static embeddings & Loses contextual depth in dynamic social media language & BERT embeddings with CoT refine themes; more adaptable to language dynamics \\ \hline
Pre-trained Language Models (e.g., BERT) \cite{devlin2018bert} & Contextualized embeddings for semantic insights & High-dimensional output challenging for analysis & Dimensionality reduction (autoencoder) makes clustering efficient, preserves semantic richness \\ \hline
Generative AI \cite{dong2024understanding} & Theme extraction via summarization & Single-pass extraction; lacks iterative refinement & CoT prompting iteratively refines themes, ensures context relevance \\ \hline
\end{tabular}
\end{adjustbox}
\end{table}

\section{Methodology}\label{sec:method}
This research introduces BEYONDWORDS, an agentic generative AI-driven framework for extracting themes from social media. An overview of the methodology is shown in Figure \ref{abstract}
\begin{figure}[H]
	\centering
	\includegraphics[width=\textwidth]{BEYONDWORDSgraphicalabstract.jpg}
	\caption{Framework overview of BEYONDWORDS showing posts embedding, compression, matrix factorization, clustering and generative AI based extraction }\label{abstract}
\end{figure}

\subsection{Text Tokenization and Embeddings Extraction}

The process of tokenizing and extracting embeddings from text involves converting the raw text into numerical representations that capture both syntactic structure and semantic meaning. Tokenization serves as the initial step, where each tweet is split into a sequence of tokens. Given a tweet \( T \) consisting of \( n \) words \( w_1, w_2, \ldots, w_n \), tokenization can be represented as:

\begin{equation}
    T = \{w_1, w_2, \ldots, w_n\} \rightarrow \{t_1, t_2, \ldots, t_m\}
\end{equation}


where \( t_i \) represents each token after tokenization, and \( m \) is the total number of tokens generated, which depends on the tokenization method and vocabulary size of the embedding model. Most models use a subword-level tokenization, which splits words into smaller units (subwords), capturing even rare or compound words effectively.

After tokenization, each tweet is converted into a fixed-dimensional embedding vector by passing it through a pre-trained language model. These models use large-scale datasets to learn embeddings that encode semantic relationships between words and phrases. For instance, similar words or concepts are represented by vectors that are close to each other in the embedding space, thus capturing semantic similarity. Let \( E \) denote the embedding vector of a tweet, which can be formulated as:

\begin{equation}
    E = Embed\left(t_1, t_2, \ldots, t_m \right)
\end{equation}


where \( Embed \) is the function that processes the token representations and maps them to a high-dimensional space that captures the semantic features of the text.

Three models of different sizes were used to generate embeddings, each capturing semantic features with varying degrees of granularity and complexity. These embeddings have the following characteristics:

\begin{itemize}
    \item \textbf{all-MiniLM-L6-v2} (small) generates embeddings vectors of size 312. Its smaller architecture captures basic semantic relationships, fewer dimensions are less resource-intensive but may miss some subtleties in meaning.
    \item \textbf{bge-base-en-v1.5} (medium) provides embeddings of size 728. This intermediate model balances between efficiency and semantic detail, capturing moderate complexities in sentence structure and meaning.
    \item \textbf{bge-m3} (large) outputs embeddings of size 1024. With a larger number of parameters, it captures nuanced semantic relationships and fine-grained meanings, making it well-suited for tasks requiring high semantic accuracy but comes at a high cost compared to small or medium size representations.
\end{itemize}
Table \ref{tab:embedding_dimensions} summarizes the propoerties of embeddings creation models.

\begin{table}[H]
    \caption{Properties of each model used in text tokenization and embeddings extraction.}
    \centering
    \begin{tabular}{c c c c}
        \hline
        \textbf{Model} & \textbf{Size} & \textbf{Number of Parameters} & \textbf{Dimensions} \\
        \hline
        all-MiniLM-L6-v2 & Small & 23M & 312 \\
        bge-base-en-v1.5 & Medium & 109M & 728 \\
        bge-m3 & Large & 567M & 1024 \\
        \hline
    \end{tabular}
    \label{tab:embedding_dimensions}
\end{table}


The embedding vectors produced by these models capture semantic information by representing words and phrases in a multi-dimensional space, where similar meanings are close together. This first step allows for downstream tasks, such as clustering and classification as explained in subsequent sections.

\subsection{Embeddings Dimensionality Reduction with autoencoder}

To enhance interpretability and reduce computational complexity, embeddings generated from text are further processed through dimensionality reduction using autoencoder. An autoencoder neural network\cite{bank2023autoencoders} consists of two primary components: an encoder \( f_{\theta} \) and a decoder \( g_{\phi} \), parameterized by \( \theta \) and \( \phi \) respectively. Given an input embedding \( E \in \mathbb{R}^d \), the encoder transforms \( E \) through a series of non-linear mappings to a compressed latent representation \( Z \in \mathbb{R}^k \), where \( k < d \). Formally, the encoding process can be described as:

\begin{equation}
Z = f_{\theta}(E) = \sigma \left( W^{(l)} \sigma \left( W^{(l-1)} \cdots \sigma \left( W^{(1)} E + b^{(1)} \right) \cdots + b^{(l-1)} \right) + b^{(l)} \right)
\end{equation}

where \( W^{(i)} \) and \( b^{(i)} \) are the weight matrices and bias vectors for the \( i \)-th layer of the encoder, and \( \sigma \) denotes an activation function (e.g., ReLU or sigmoid). This encoding function learns a transformation that captures the essential structure of \( E \) in the lower-dimensional space \( \mathbb{R}^k \).

The decoder then reconstructs the input embedding \( E \) from the latent representation \( Z \) by applying the inverse transformation, approximating the original embedding through the following mapping:

\begin{equation}
\hat{E} = g_{\phi}(Z) = \sigma \left( W'^{(1)} \sigma \left( W'^{(2)} \cdots \sigma \left( W'^{(m)} Z + b'^{(m)} \right) \cdots + b'^{(2)} \right) + b'^{(1)} \right)
\end{equation}

where \( W'^{(j)} \) and \( b'^{(j)} \) represent the weight matrices and bias vectors of the \( j \)-th layer in the decoder. Here, \( m \) denotes the number of layers in the decoder, and \( \hat{E} \in \mathbb{R}^d \) is the reconstructed embedding that approximates the original \( E \).

The autoencoder neural network is trained by minimizing the reconstruction loss \( L(E, \hat{E}) \), typically formulated as mean squared error (MSE):

\begin{equation}
L(E, \hat{E}) = \frac{1}{d} \sum_{i=1}^{d} \left( E_i - \hat{E}_i \right)^2,
\end{equation}

where \( E_i \) and \( \hat{E}_i \) denote the \( i \)-th components of the original and reconstructed embeddings, respectively. This loss function encourages the model to learn a compact representation that captures the core features of the input embeddings while discarding noise.

Three different compression ratios were explored for the latent dimension \( k \): 1/2, 1/3, and 1/4. Each ratio was assessed to balance the trade-off between dimensionality reduction and reconstruction fidelity, with detailed results discussed in the Results section.

After training, the encoder \( f_{\theta} \) alone is utilized to project high-dimensional embeddings into their corresponding lower-dimensional representations \( Z \), enabling more efficient downstream processing, including Singular Value Decomposition (SVD) for latent theme extraction and clustering with k-means.



\subsection{Matrix Factorization and Clustering}

Matrix factorization using SVD \cite{stewart1993early} was applied to identify underlying themes within the tweet embeddings. Instead of using the original high-dimensional embeddings, the compressed embeddings from the autoencoder neural network were utilized as input. This approach significantly reduces computational cost while preserving essential semantic features, as the dimensionality reduction effectively discards non-informative components.

Given a compressed embedding matrix \( C \in \mathbb{R}^{n \times k} \), where \( n \) is the number of tweet embeddings and \( k \) is the compressed dimension, SVD decomposes \( C \) into three matrices:

\begin{equation}
C = U \Sigma V^T
\end{equation}

where \( U \in \mathbb{R}^{n \times r} \) and \( V \in \mathbb{R}^{k \times r} \) are orthogonal matrices, \( \Sigma \in \mathbb{R}^{r \times r} \) is a diagonal matrix of singular values, and \( r \) represents the rank of \( C \). The columns of \( U \) correspond to the principal components capturing the main themes within the tweet embeddings.

The importance of matrix factorization in this study lies in its ability to distill complex, high-dimensional data into a more manageable form while retaining the most significant features that represent underlying patterns and themes. By decomposing the compressed embeddings, SVD allows for the identification of latent structures that can reveal insights into the thematic content of the social media posts.

To group similar posts based on these themes, \( k \)-means clustering was applied to the reduced representation of the embeddings. This clustering helps in identifying coherent groups of posts, facilitating the extraction of distinct themes. The combination of SVD and \( k \)-means clustering \cite{kodinariya2013review} provides a computationally efficient method to uncover and organize thematic patterns in the data, providing essential component of the proposed approach in this research.
The equation that models this proposed methodology of clustering is expressed as follows:
\begin{equation}
C_{\text{posts}} = \operatorname{k\text{-means}}\left( U, k \right) \quad 
\end{equation}
\begin{equation}
\text{where}  \quad 
U = \operatorname{SVD}_U \left( f_{\theta^*} \left( \operatorname{Embed} \left( \bigcup_{i=1}^N \{t_{i1}, t_{i2}, \dots, t_{im_i}\} \right) \right) \right)
\end{equation}

\begin{equation}
\text{and}  \quad 
\theta^* = \operatorname{argmin}_{\theta} \sum_{i=1}^N L \left( E_i, g_{\phi} \left( f_{\theta}(E_i) \right) \right)
\end{equation}
where the term \(\theta^*\) represents the selection of the optimal compression ratio \(\theta\) that minimize the reconstruction loss \(L(E_i, \hat{E}_i)\) across all embeddings \(E_i\). Here, \(f_{\theta}\) is the encoder function, and \(g_{\phi}\) is the decoder function that reconstructs \(E_i\) from its compressed representation, aiming to minimize the difference between \(E_i\) and its reconstruction \(\hat{E}_i\).


Algorithm \ref{alg:part1} explains the steps performed for clustering the tweets using the proposed approach to prepare them for thematic analysis using LLMs. 

\begin{algorithm}[H]
    \caption{Embedding Extraction and Compression, Matrix Factorization and Clustering}
    \label{alg:part1}
    \nolinenumbers
    \begin{algorithmic}
        \State \textbf{Input:} Tweets dataset $T$, Compression ratios $CR$, Number of clusters $k$
        \State \textbf{Output:} Clustered posts $C_{posts}$

        \State \text{1. Text Tokenization:}
        \State \quad For each tweet \( T_i \in T \):
        \State \quad \quad $T_i = \{w_1, w_2, \ldots, w_n\} \rightarrow \{t_1, t_2, \ldots, t_m\}$

        \State \text{2. Embedding Extraction:}
        \For{each model in \{X, Y, Z\}}
            \State \quad $E_i \gets \operatorname{Embed}(t_1, t_2, \ldots, t_m)$
        \EndFor

        \State \text{3. Dimensionality Reduction with Autoencoders:}
        \For{each compression ratio $\theta \in CR$}
            \State \quad $Z_i^{(\theta)} = f_{\theta}(E_i)$ \quad \text{(Encode)}
            \State \quad $\hat{E}_i^{(\theta)} = g_{\phi_\theta}(Z_i^{(r)})$ \quad \text{(Decode)}
            \State \quad Calculate $L(E_i, \hat{E}_i^{(r)})$
        \EndFor
        \State \quad Select $\theta^* = \operatorname{argmin}_{\theta \in CR} \sum_{i} L(E_i, \hat{E}_i^{(r)})$
        \State \quad Set $Z_i = Z_i^{(\theta^*)}$ for each $i$

        \State \text{4. Matrix Factorization (SVD):}
        \State \quad $C \gets \text{Compressed Embedding Matrix for selected } \theta^*$
        \State \quad Decompose $C$ using SVD: $C = U \Sigma V^T$

        \State \text{5. Clustering:}
        \State \quad $C_{posts} \gets k\text{-means}(U, k)$ \quad \text{(Cluster posts)}
        
    \end{algorithmic}
\end{algorithm}


\subsection{Generative AI for Themes Extraction}
To effectively manage the analysis of social media posts, a two-step sampling strategy was employed. It is based on the Cochran formula to determine an appropriate sample size from each cluster. This approach allowed conducting a focused thematic analysis while adhering to the limitations of generative AI regarding context windows and computational costs.
The Cochran formula for sample size determination is given by:

\begin{equation}
n = \frac{Z^2 \cdot p \cdot (1 - p)}{e^2}
\end{equation}
Where: \(n\) is the sample size, \(Z\) is the Z-score corresponding to the desired confidence level (e.g., 1.64 for a 90\% confidence level), \(p\) is the estimated proportion of the population (we use \(0.5\) for maximum sample size), \(e\) is the margin of error (the desired level of precision).

Using this formula, the sample size \(n\) for each cluster was calculated. For a 90\% confidence the final sample size is determined to be 68. This statistically representative sample enables the model to apply the thematic extraction methodology efficiently without processing the entire batch of posts.
Following the determination of the sample size, the selection was further improved by employing the silhouette score to identify the top \(n\) texts with the highest cohesion and separation from other clusters. The silhouette score is defined as:

\begin{equation}
s(i) = \frac{b(i) - a(i)}{\max\{a(i), b(i)\}}
\label{silhouette}
\end{equation}
Where: \(s(i)\) is the silhouette score for the \(i\)-th data point, \(a(i)\) is the average distance between the \(i\)-th data point and all other points in the same cluster, \(b(i)\) is the average distance between the \(i\)-th data point and all points in the nearest cluster.

By calculating the silhouette score for all posts within each cluster, the texts selected are the highest scorers, indicating that they are well-clustered and representative of their themes. 

\subsubsection{CoT Strategy}

The CoT strategy serves as a pivotal component of the proposed methodology, guiding the generative process through structured and sequential tasks. The process can be detailed as follows:

\begin{enumerate}
    \item The LLM is prompted to identify significant keywords and phrases within the tweets, focusing on content that reflects important topics and sentiments.
    \item The extracted keywords are organized into coherent groups based on common themes, topics, or ideas, enhancing clarity and coherence.
    \item For each category, the LLM work as a specialized agent in thematic extraction, synthesizes high-level themes, providing concise descriptions that encapsulate the essence of the discussions in the tweets.
\end{enumerate}

\subsubsection{Recursive Theme Refinement via LLM Feedback Mechanism}

To further refine the theme extraction process, a recursive feedback mechanism involving a second agent LLM is implemented. This agent acts as a grading system, following these steps:

\begin{enumerate}
    \item The secondary LLM evaluates the themes generated by the primary LLM against predefined quality thresholds.
    \item If the themes do not meet the established criteria, feedback is relayed back to the primary LLM.
    \item The primary LLM utilizes the feedback (score + comment) to reevaluate the extraction process, revisiting the previous steps as necessary.
    \item This cycle of evaluation and adjustment continues, improving the quality of theme extraction with each iteration until either the threshold score or the maximum number of iterations are reached.
\end{enumerate}
    
This iterative refinement mimics the Generative Adversarial Networks (GANs) paradigm \cite{goodfellow2020generative}, where a generator produces outputs while a discriminator assesses their quality. In this context, the generator is the primary LLM that generates thematic outputs, while the discriminator is the secondary LLM that critiques these outputs. Since the discriminator LLM provides the score and feedback as part of a sentence, a mechanism is needed to extract only the score and accurate feedback. This is essential to isolate the components required to decide whether the extracted theme is good enough by comparing the score with a threshold, or to pass only these two elements to the first step to redo the extraction process, taking into consideration the new score and feedback from the discriminator. To serve this purpose, we use an LLM-based entity extractor model \cite{ghali2024gamedx}. This discourse between LLMs ensures continuous improvement in theme extraction, allowing the methodology to adapt and enhance its performance with each iteration. By combining structured prompting with feedback-driven refinement. 

The following equation describes the recursive refinement strategy proposed by modeling the theme extraction process for cluster \( k \), where \( T_{\text{themes}, k} \) is the final set of themes. It uses the indicator function \( \mathbb{I} \) to choose between refining the initial themes, \( T_{\text{initial}} \), or using them directly, based on the evaluation score. If the score is below the threshold \( Q \), the themes are refined using feedback through \( \mathcal{R}(T, \text{feedback}) \), otherwise, the initial themes are selected. The scoring function \( \mathcal{S}(\text{score}(T)) \) evaluates the quality of the themes, while \( \mathcal{M}_1 \) and \( \mathcal{M}_2 \) represent the language models used for theme extraction and evaluation, respectively.
\begin{equation}
T_{\text{themes}, k} = 
\arg \max_{T} \left[ \mathbb{I}_{\left( \mathcal{S}(\text{score}(T)) < Q \right)} \cdot \mathcal{R}(T, \text{feedback}) + \mathbb{I}_{\left( \mathcal{S}(\text{score}(T)) \geq Q \right)} \cdot T_{\text{initial}} \right]
\end{equation}

\(
\text{where} \quad 
T_{\text{initial}} = \mathcal{M}_1 \left( C_{\text{posts}, k} \right)
\quad \text{and} \quad
\left( \text{score}, \text{feedback} \right) = \mathcal{M}_2 \left( T_{\text{initial}} \right)
\)


Algorithm \ref{alg:genai} summarizes in details the steps to extract the themes:
\begin{algorithm}[H]
    \caption{Agentic CoT Model for Theme Extraction}
    \label{alg:genai}
    \nolinenumbers
    \begin{algorithmic}
        \State \textbf{Input:} Clustered posts $C_{posts,k}$
        \State \textbf{Output:} Extracted themes for each cluster $k$: $T_{themes,k}$
        
        \For{each cluster $k$ in $C_{posts,k}$}
            \State \text{1. Sample Selection:}
            \State \quad Calculate sample size $n$ using Cochran's formula
            \State \quad Sample $n$ posts from cluster $k$

            \State \text{2. Initial Theme Extraction:}
            \State \quad $T_{initial} \gets \text{LLM}_1.extract\_themes(C_{posts,k})$

            \State \text{3. Quality Evaluation:}
            \State \quad $score, feedback \gets \text{LLM}_2.evaluate\_and\_extract(T_{initial})$
            
            \If{$score < Q$}
                \State \text{4. Feedback and Refinement:}
                \State \quad $T_{refined} \gets \text{LLM}_1.refine\_themes(T_{initial}, feedback)$
                \State \quad $T_{themes,k} \gets T_{refined}$
            \Else
                \State \quad $T_{themes,k} \gets T_{initial}$
            \EndIf
        \EndFor
        
        \State \text{5. Output Extracted Themes:} $T_{themes,k}$ for each cluster $k$
    \end{algorithmic}
\end{algorithm}



\section{Autism Case Study}

\subsection{Dataset Overview}

The dataset used for this study comprises a selection of tweets from individuals who identify as autistic, using the hashtag \#actuallyautistic to share their experiences, insights, and perspectives. This hashtag represents a movement within the autism community that seeks to amplify the voices of autistic individuals offering a more personal and direct narrative on living with autism. 
The original dataset was gathered using \texttt{snscrape}, a Python-based scraping tool, to extract tweets from X(Twitter previously) between January 2014 and December 2022. The dataset includes tweets from individuals who self-identified as autistic through keywords such as “autism,” “autistic,” or “neurodiverse” in their profiles, usernames or tweets. This initial dataset consists of approximately 3.1 million tweets from 17,323 unique users, with additional metadata on usernames, account creation dates, and other relevant features.
Figure \ref{datapreprocessing} presents an overview of the preprocessing process.

\begin{figure}[H]
	\centering
	\includegraphics[width=\textwidth]{datapreprocessing.jpg}
	\caption{Overview of the dataset preprocessing used in this research }\label{datapreprocessing}
\end{figure}
\subsection{Dataset Preprocessing}
From the original dataset \cite{jaiswal2023actuallyautistic}, only tweets that contained \#actuallyautistic hashtag with its different textual variations were retained to ensure relevance to autistic individuals. Additionally, tweets were filtered to include only those written in English, resulting in a subset of approximately 200,000 tweets for further analysis. To prepare the data for processing, a text-cleaning script was applied to each tweet, removing URLs, mentions, hashtags, and special characters. This preprocessing step created a refined dataset that maintains linguistic consistency and readability for subsequent analysis.

\section{Experiments Results and Discussion}
\subsection{Dimensionality Reduction Analysis}
This section presents the findings from experiments conducted on autoencoder neural network models designed for dimensionality reduction of text embeddings. The primary goal is to evaluate the performance of three distinct model architectures—small, medium, and large—across varying levels of compression. Specifically, the analysis focuses on how each model performs with three different compression ratios: 1/2, 1/3, and 1/4 of the original embedding dimensionality. The ultimate goal is to identify the model that minimizes information loss, ensuring the highest possible accuracy for subsequent SVD and clustering using the minimum computations possible.

\begin{figure}[H]
	\centering
	\includegraphics[width=\textwidth]{autoencoder_loss_combined.jpg}
	\caption{Autoencoder loss plot for a) small, b) medium, and c) large embedding models}\label{loss}
\end{figure}
Figure \ref{loss} presents the validation loss across training epochs for the autoencoder model. For the small embedding size (384 dimensions), as shown in Figure \ref{loss}a, the 1/2 compression ratio achieves the lowest validation loss, indicating that halving the dimensionality retains the essential features for accurate reconstruction, balancing information preservation and reduced computational complexity. For the medium embedding size (768 dimensions), highlighted in Figure \ref{loss}b, the 1/4 compression ratio is optimal, suggesting that medium-sized embeddings tolerate higher compression while still enabling precise reconstruction. This indicates that the medium model is more resilient to dimensionality reduction, allowing for more efficient computation in subsequent SVD and clustering steps. For the large embedding size (1024 dimensions), as shown in Figure \ref{loss}c, the best performance is again at the 1/2 compression ratio, reflecting that halving the dimensionality remains optimal for capturing necessary data patterns despite the larger embedding size. Overall, the analysis highlights the autoencoder’s ability to retain critical information across all embedding sizes and compression ratios, as evidenced by the convergence of validation loss to low values. The preservation of embedding structure post-compression is crucial for accurate clustering with k-means, enabling efficient theme extraction with minimal information loss.




\subsection{Matrix Factorization and Clustering}
%%%%%%%%%%%%%%%SVD%%%%%%%%%%%%%%%
\begin{figure}[H]
	\centering
	\includegraphics[width=\textwidth]{svdcombined.jpg}
	\caption{SVD plot for a) small, b) medium, and c) large embedding models}\label{svdcombined}
 
\end{figure}

The results of the SVD analysis, illustrated in Figure \ref{svdcombined}, demonstrate the effectiveness of the proposed dimensionality reduction methodology for embeddings of varying sizes (small, medium, and large). Following the initial compression achieved by autoencoders, SVD was applied to further reduce the embeddings' dimensionality, facilitating efficient clustering and theme extraction. This step is particularly crucial in the methodology, as matrix factorization not only enhances interpretability but also optimizes computational resources, making it well-suited for large-scale social media posts analysis.
In Figure \ref{svdcombined}a, which corresponds to the small-size embeddings, the cumulative explained variance approaches 90\% with approximately 100 components. This finding suggests that SVD effectively captures the majority of the data’s variance, while significantly reducing the dimensional space. 
For medium-size embeddings, shown in Figure \ref{svdcombined}b, a similar pattern emerges. The cumulative explained variance also reaches close to 90\% with fewer than 100 components. This consistency across both small and medium embeddings indicates that SVD is highly effective in distilling essential features, even as the embedding size increases.
Figure \ref{svdcombined}c presents the results for large-size embeddings, where approximately 150 components are required to capture a comparable level of variance. This increase in required components is expected, as larger embeddings inherently contain more complex information. Despite this, the curve stabilizes, confirming that SVD provides a robust means of reducing even high-dimensional embeddings to a manageable form, preserving critical information for clustering.

%%%%%%%%%%%%%%%%%%%%%%%%%%%%%%%%%%%

%%%%%%%%%%%%%%%Elbow%%%%%%%%%%%%%%%
\subsection{Clustering (combine all 6 pictures in one big picture)}

The elbow method (Figure \ref{combinedclusters}) was used to determine the optimal number of clusters for latent themes derived via SVD from compressed embeddings. For all embedding sizes (Figures \ref{combinedclusters}a, \ref{combinedclusters}b, \ref{combinedclusters}c), the elbow point consistently appears at 3 clusters suggesting that additional clusters would contribute minimal new information, confirming that 3 clusters capture the primary themes in the data.
%%%%%%%%%%%%%%%%%%%%%%%%%%%%%%%%%%%
\begin{figure}[H]
	\centering
	\includegraphics[width=\textwidth]{combined_clusters.jpg}
	\caption{Elbow and k-means themes clusters plot for a) small, b) medium, and c) large embedding models}\label{combinedclusters}
\end{figure}
%%%%%%%%%%%%%%%clusters%%%%%%%%%%%%%%%
The clustering results demonstrate the robustness and scalability of the proposed methodology. Across all embedding sizes, three distinct clusters are consistently formed, each reflecting well-separated themes within the social media data. 
In Figure \ref{combinedclusters}a, the small-sized embeddings display clear separation among the clusters, with each color representing a unique latent theme. The compact arrangement of points within each cluster suggests that the themes identified are coherent and well-differentiated even with a limited embedding dimensionality. Figure \ref{combinedclusters}b, representing medium-sized embeddings, shows a similar pattern, with clusters that retain distinct boundaries and minimal overlap. This demonstrates that the clustering method is adaptable, producing stable clusters as the embedding dimensionality increases.
Figure \ref{combinedclusters}c, showing results for the large-sized embeddings, presents a particularly compelling case for the methodology's effectiveness. The clusters remain well-defined, with minimal inter-cluster overlap, despite the increase in data complexity and embedding dimensionality. The large embedding size allows for capturing more nuanced details in the data, yet the clusters continue to exhibit clear separation, indicating that the approach is capable of handling high-dimensional representations while preserving theme clarity.
The consistency of cluster shapes and separation across embedding sizes reinforces the suitability of this methodology for identifying latent themes in social media data. The SVD step plays a pivotal role in this outcome by reducing dimensionality while retaining essential information, enabling k-means clustering to form coherent clusters that are robust to changes in embedding size.

%%%%%%%%%%%%%%%%%%%%%%%%%%%%%%%%%%%%%%%%%%%%%%%%%%%%%%

%%%%%%%%%%%%%%%%%%%%quality of clusters%%%%%%%%%%%%%%%%

To accurately measure the quality of the clustering results after applying the proposed methodology, and compare the effectiveness of using autoencoder neural network for dimensionality reduction in text embedding compression, it was essential to conduct an analysis with and without autoencoder-based compression across different embedding sizes. This analysis allows for a deeper understanding of how compression impacts clustering quality, which is crucial when applying subsequent dimensionality reduction techniques like SVD for latent theme extraction and clustering. To evaluate the clustering performance, three metrics were considered:
\textbf{Calinski-Harabasz Index (CH Index):} Measures the ratio of the sum of between-cluster dispersion to within-cluster dispersion. Higher values indicate better-defined clusters.
   \begin{equation}
   \text{CH Index} = \frac{\text{Tr}(B_k)}{\text{Tr}(W_k)} \cdot \frac{N - k}{k - 1}
   \end{equation}
   where \( \text{Tr}(B_k) \) is the trace of the between-cluster dispersion matrix, \( \text{Tr}(W_k) \) is the trace of the within-cluster dispersion matrix, \( N \) is the total number of points, and \( k \) is the number of clusters.
\newline
\newline
\textbf{Davies-Bouldin Index (DB Index):} Measures the average "similarity" ratio of within-cluster distances to between-cluster distances. Lower values indicate better clustering quality.
   \begin{equation}
   \text{DB Index} = \frac{1}{k} \sum_{i=1}^{k} \max_{j \neq i} \left( \frac{\sigma_i + \sigma_j}{d(c_i, c_j)} \right)
   \end{equation}
   where \( \sigma_i \) and \( \sigma_j \) are the within-cluster distances for clusters \( i \) and \( j \), respectively, and \( d(c_i, c_j) \) is the distance between cluster centers \( c_i \) and \( c_j \).
   \newline
   \newline
\textbf{Silhouette Score:} Reflects the compactness and separation of clusters. It ranges from -1 to 1, where higher values indicate well-separated clusters as explained in details in Equation \ref{silhouette}.
\newline
The tables below summarizes the clustering metrics results for different embedding sizes with and without autoencoder-based compression.

\begin{table}[H]
\centering
\small
\caption{Clustering metrics for small size embeddings}
\begin{tabular}{lcc}
\hline
\textbf{Metrics} & \textbf{With Autoencoder} & \textbf{Without Autoencoder} \\
                 & \textbf{Compression}      & \textbf{Compression} \\
\hline
CH Index & 24991 & 5161 \\
DB Index & 2.93 & 8.22 \\
Silhouette Score & 0.04 & 0.04 \\
\hline
\end{tabular}
\label{table:small}
\end{table}

\begin{table}[H]
\centering
\small
\caption{Clustering metrics for medium size embeddings}
\begin{tabular}{lcc}
\hline
\textbf{Metrics} & \textbf{With Autoencoder} & \textbf{Without Autoencoder} \\
                 & \textbf{Compression}      & \textbf{Compression} \\
\hline
CH Index & 11806 & 11879 \\
DB Index & 4.01 & 3.80 \\
Silhouette Score & 0.06 & 0.06 \\
\hline
\end{tabular}
\label{table:medium}
\end{table}

\begin{table}[H]
\centering
\small
\caption{Clustering metrics for large size embeddings}
\begin{tabular}{lcc}
\hline
\textbf{Metrics} & \textbf{With Autoencoder} & \textbf{Without Autoencoder} \\
                 & \textbf{Compression}      & \textbf{Compression} \\
\hline
CH Index & 366243 & 6235 \\
DB Index & 0.62 & 5.22 \\
Silhouette Score & 0.48 & 0.04 \\
\hline
\end{tabular}
\label{table:large}
\end{table}


The clustering metrics presented in Tables \ref{table:small}, \ref{table:medium}, and \ref{table:large} serve as the final quality assessment of the methodology for extracting themes from text embeddings. This analysis follows the dimensionality reduction via autoencoder, SVD for latent theme extraction, and clustering with k-means. The results validate the effectiveness of this multi-step approach in capturing meaningful structure within the data.
The high CH Index values and low DB Index values achieved, particularly in the configurations with autoencoder neural network compression, demonstrate the ability of this methodology to create well-separated and compact clusters, indicative of high clustering quality. For instance, in the large embedding size, the CH Index reaches 366243 with a DB Index of 0.62 when autoencoder neural network are used, signaling exceptionally distinct clusters. This trend is consistent across embedding sizes, with the small embedding size achieving a CH Index of 24991 and a DB Index of 2.93, and the medium size showing similarly favorable metrics. 
Moreover, the silhouette scores across the tests further confirm the consistency and stability of the clustering structure. These metrics collectively indicate that the clusters generated are not only compact and well-separated but also robust in representing latent themes within the data as evidenced by this final quality assurance test. 

%%%%%%%%%%%%%%%%%%%%%%%%%%%%%%%%%%%%%%%%%%%%%%%%%%%%%%%
\subsection{Generative AI Theme Extraction Analysis}
In the thematic extraction phase, large embedding model's themes clusters were used as it achieved the lowest loss value and highest clustering quality across two out of three evaluation metrics. GPT -4o mini model was used for both the primary extractor (LLM1) and evaluator (LLM2) roles as it is known for its lightweight design, per token cost effectiveness and strong performance across major evaluation benchmarks \cite{openai2024gpt4ocard}. This analysis section presents both word cloud visualizations of keywords and Sankey diagrams illustrating the connections between keywords and their coherent groups, providing a comprehensive view of the relationships identified to further examine and validate the final themes extracted.

Through the structured CoT and recursive theme refinement strategy explained in details in methodology section, the LLM1 first identified significant keywords and phrases in the tweets text, focusing on elements that reflect key topics and sentiments. These keywords were then organized into multiple coherent groups for each cluster yielding at the end the final high-level themes:

\begin{itemize}
    \item \textbf{Social media content quality and engagement:} This theme contains keywords related to the quality and engagement potential of social media content, such as "Excellent content," "Highly recommended," and "Promising article." The associated Sankey diagram (Figure \ref{theme1}b) illustrates the connections between these engagement-oriented terms, while the word cloud (Figure \ref{theme1}a) highlights the prominence of keywords within this theme.
    
    \item \textbf{Advocacy for autistic rights and acceptance:} This theme emphasizes the importance of representation, empowerment, and critiques of dominant autism narratives. Keywords like "Autistic civil rights," "Neurodiversity," and "Acceptance and love" underscore the advocacy focus within this group. Figure~\ref{theme2}b provides the Sankey diagram showing how terms in this theme are interrelated, and Figure~\ref{theme2}a displays the word cloud, showcasing the relevance of advocacy-related terms.

    \item \textbf{Mental health and well-being:} This theme captures mental health challenges and coping strategies, with keywords like "Burnout," "Executive dysfunction," "Coping strategies," and "Isolation." The Sankey diagram (Figure~\ref{theme3}b) shows connections between terms related to mental health, while Figure~\ref{theme3}a displays a word cloud of keywords within this theme, underscoring the emphasis on managing mental and emotional challenges.
\end{itemize}

The themes revealed by LLM1 were iteratively evaluated by the LLM2 until they reached the predefined quality acceptance criteria. If the themes did not meet these criteria, feedback was provided to LLM1 for adjustments. This recursive feedback loop significantly enhanced the clarity and coherence of the identified themes, allowing for an even better and accurate thematic extraction as evidenced by the final themes above.

\begin{figure}[H]
	\centering
	\includegraphics[width=\textwidth]{theme1.jpg}
	\caption{LLMs based thematic analysis for theme 1: a) word cloud of keywords extracted by LLM1, b) Sankey diagram of keywords extracted and their coherent groups clustering}\label{theme1}
\end{figure}

\begin{figure}[H]
	\centering
	\includegraphics[width=\textwidth]{theme2.jpg}
	\caption{LLMs based thematic analysis for theme 2: a) word cloud of keywords extracted by LLM1, b) Sankey diagram of keywords extracted and their coherent groups clustering}\label{theme2}
\end{figure}

\begin{figure}[H]
	\centering
	\includegraphics[width=\textwidth]{theme3.jpg}
	\caption{LLMs based thematic analysis for theme 3: a) word cloud of keywords extracted by LLM1, b) Sankey diagram of keywords extracted and their coherent groups clustering}\label{theme3}
\end{figure}

\subsection{Discussion}

\subsubsection{Thematic Study Insights}
The thematic analysis provided valuable insights across the three primary areas, highlighting interesting discussions and sentiments within the social media content of autistic people.

The first extracted theme \textbf{social media content quality and engagement} reveals high engagement indicators such as "Excellent content," "Promising article," and "Highly recommended," indicating a strong focus on content quality. This underscores the value placed on credible, high-quality resources in social media conversations and what it means to autistic people.
    
The second theme \textbf{advocacy for autistic rights and acceptance} with keywords such as "Autistic civil rights," "Neurodiversity," and "Acceptance and love" reflect the community's emphasis on representation and self-determined identity. This theme highlights the role of social media in autistic advocacy, providing a platform to advocate for acceptance, critique organizations (e.g., “Autism Speaks criticism”), and promote neurodiversity.
    
The last extracted theme \textbf{mental health and well-being} groups keywords around mental health challenges and coping strategies, such as "Burnout," "Executive dysfunction," and "Coping strategies." The keywords highlight the community's struggles with mental and emotional well-being, as well as the importance of resilience and coping mechanisms, particularly in the context of neurodiverse experiences.

\subsubsection{Implications for Autism Advocacy and Informed Decision-Making}
The identified themes have implications for both advocacy and informed decision-making. For example, the second theme reveals community-driven discussions about identity-first language and critiques of existing narratives, guiding policymakers and advocacy groups to better align their communication and policies with community values.

In addition, the third theme underscores the need for targeted mental health resources for autistic individuals, pointing to the importance of tailored interventions for this population. These insights support the goal of using data-driven themes to foster inclusivity and improve access to relevant resources without neglecting the fact that they value quality content on social media as depicted by the first theme.



\section{Conclusion}
This paper presents BEYONDWORDS, a novel methodology for extracting latent themes from social media posts, with a specific focus on content related to autistic individuals. The approach integrates embeddings, dimensionality reduction, SVD for theme extraction, K-means clustering, and an agentic generative AI model with iterative feedback. This pipeline successfully identified three primary themes: \textbf{social media content quality and engagement}, \textbf{advocacy for autistic rights and acceptance}, and \textbf{mental health and well-being}. The thematic analysis provides valuable insights, revealing high engagement with quality content, a strong emphasis on autistic rights and acceptance, and the importance of mental health and coping strategies within the autistic community. While the methodology offers a comprehensive and scalable approach to theme extraction, it is not without limitations. The complexity of the multistep process and the dependency on high-quality embedding models are notable challenges. Additionally, potential biases in the AI models and the sensitivity of SVD and K-means to parameter tuning require careful consideration.
Future work will focus on addressing these limitations by developing techniques to mitigate biases, exploring real-time processing capabilities, and incorporating user feedback mechanisms to further refine the accuracy and applicability of the approach. These enhancements will make the methodology a robust tool for social media analysis, particularly in understanding and supporting the autistic community.

\newpage

\documentclass{MITstyle}

%\usepackage[table]{xcolor}
\usepackage{chngcntr}
\usepackage{hyperref}
\usepackage{microtype}

\title{A Lightweight and Extensible Cell Segmentation and Classification Model for Whole Slide Images}

\author{Nikita Shvetsov~$^{1, }$\footnote{Correspondence e-mail: nikita.shvetsov@uit.no}, Thomas K. Kilvaer~$^{2, 3}$, Masoud Tafavvoghi~$^{4}$, Anders Sildnes~$^{1}$, \\ Kajsa Møllersen~$^{4}$, Lill-Tove Rasmussen Busund~$^{5, 6}$, Lars Ailo Bongo~$^{1}$ \\
%
\vspace{1em} % Space between authors and afilliations
%
\normalfont{\small $^{1}$Department of Computer Science, UiT The Arctic University of Norway}\\
\normalfont{\small $^{2}$Department of Oncology, University Hospital of North Norway}\\
\normalfont{\small $^{3}$Department of Clinical Medicine, UiT The Arctic University of Norway}\\
\normalfont{\small $^{4}$Department of Community Medicine, UiT The Arctic University of Norway}\\
\normalfont{\small $^{5}$Department of Medical Biology, UiT The Arctic University of Norway} \\
\normalfont{\small $^{6}$Department of Clinical Pathology, University Hospital of North Norway} %\vspace{2em}
}

\begin{document}
\maketitle

\section*{Abstract}

% \begin{abstract}
% Developing clinically useful cell-level analysis tools in digital pathology remains challenging due to limitations in dataset granularity, inconsistent annotations, computational demands of advanced models, and difficulties in integrating new technologies into clinical workflows. To address these challenges, we propose a multi-faceted solution that enhances data quality, model performance, and usability to create a lightweight and extensible cell segmentation and classification model.

% First, we update data labels by employing a cross-relabeling process that refines the labels of two existing datasets, PanNuke and MoNuSAC, to create a new unified dataset with enhanced granularity, encompassing seven distinct cell types. Second, we leverage the H-Optimus foundation model as a fixed encoder to improve feature representation for simultaneous cell segmentation and classification tasks. Third, to address the computational demands of foundation models, we employ knowledge distillation to reduce model size and complexity while maintaining comparable performance. Finally, to facilitate integration into clinical workflows, we integrate the distilled model into the QuPath software, a widely used open-source platform in digital pathology.

% Our results demonstrate improvements in cell segmentation and classification performance using the H‑Optimus-based model compared to a CNN-based model. Specifically, the average $R^2$ improved from 0.575 to 0.871, and the average $PQ$ score improved from 0.450 to 0.492, indicating better alignment with actual cell counts and enhanced segmentation and classification quality. Furthermore, the distilled student model maintains performance comparable to the larger foundation model while reducing the parameter count by a factor of 48.
% Overall, by reducing computational complexity and integrating it into existing workflows, the proposed approach may significantly impact diagnostic processes, reduce the workload of pathologists, and contribute to improved patient outcomes. Though our approach shows potential enhancements in efficiency and usability of cell segmentation and classification models in digital pathology, extensive validation is needed to deploy these models in clinical practice.
% \end{abstract}

%%% shortened abstract
\begin{abstract}
Developing clinically useful cell-level analysis tools in digital pathology remains challenging due to limitations in dataset granularity, inconsistent annotations, high computational demands, and difficulties integrating new technologies into workflows. To address these issues, we propose a solution that enhances data quality, model performance, and usability by creating a lightweight, extensible cell segmentation and classification model. 

First, we update data labels through cross-relabeling to refine annotations of PanNuke and MoNuSAC, producing a unified dataset with seven distinct cell types. Second, we leverage the H-Optimus foundation model as a fixed encoder to improve feature representation for simultaneous segmentation and classification tasks. Third, to address foundation models' computational demands, we distill knowledge to reduce model size and complexity while maintaining comparable performance. Finally, we integrate the distilled model into QuPath, a widely used open-source digital pathology platform. 

Results demonstrate improved segmentation and classification performance using the H-Optimus-based model compared to a CNN-based model. Specifically, average $R^2$ improved from 0.575 to 0.871, and average $PQ$ score improved from 0.450 to 0.492, indicating better alignment with actual cell counts and enhanced segmentation quality. The distilled model maintains comparable performance while reducing parameter count by a factor of 48. By reducing computational complexity and integrating into workflows, this approach may significantly impact diagnostics, reduce pathologist workload, and improve outcomes. Although the method shows promise, extensive validation is necessary prior to clinical deployment.
\end{abstract}
\clearpage

\section{Introduction}
In digital pathology, accurate segmentation and classification of cells are crucial for many diagnostic, prognostic, and predictive analyses \cite{Jaber_Beziaeva_etal._2019,Lin_Pan_etal._2022,Park_Ock_etal._2022,Shen_Choi_etal._2024}. Nowadays, developments in computational pathology offer multiple solutions \cite{H._Qu_P._Wu_etal._2020,Javed_Mahmood_etal._2020} to utilize cell-level datasets to train machine learning models that solve these problems. The quality and specificity of training datasets are critical for robust and accurate models. Adhering to the principle of "garbage in, garbage out", it is essential to ensure that these datasets are extensively and accurately labeled with distinct classes that reflect the diverse biological characteristics of different cell types. Unfortunately, the number of open-source datasets comprising such high-quality annotations is limited. Existing cell segmentation datasets \cite{Gamper_Koohbanani_etal._2019,Graham_Vu_etal._2019,Verma_Kumar_etal._2021} may offer extensive annotations for certain cell types while providing more general labels for others. For example, in PanNuke, which is one of the largest open-source datasets comprising labeled cells, various types of morphologically and functionally different inflammatory cells like macrophages and lymphocytes are clustered in a broad "inflammatory" class. Consequently, these classes are frequently omitted from analyses or aggregated into broader meta-classes \cite{Gamper_Koohbanani_etal._2020} and likely interfere with other cell classes included in the dataset. This and similar inconsistencies in annotation granularity limit the ability of machine learning models to learn the comprehensive and nuanced features necessary for accurate cell segmentation and classification. To address these challenges, methods for refining and standardizing dataset annotations are essential to enhance the quality of training data.

A complementary approach to mitigate the absence of high-quality training data is the use of foundation models. Foundation models as encoders are defined as large-scale, versatile networks pre-trained on vast, diverse datasets using self-supervised learning, contrasting with convolutional neural network (CNN) pre-trained encoders that rely on supervised learning with labeled data. In practice, foundation models leverage enormous amounts of weakly or unlabeled data from millions of whole slide images (WSIs) and employ self-attention mechanisms to capture long-range dependencies and global context \cite{Chen_Ding_etal._2024,Saillard_Jenatton_etal._2024,Vorontsov_Bozkurt_etal._2024,Xu_Usuyama_etal._2024}. As a consequence, foundation models are able to produce transferable feature representations across different cell types and tissue environments. The feature representations can be leveraged by decoder networks to produce segmentation masks and pixel-level classifications. Because foundation models have comprehensive feature representations, they can be effectively fine-tuned using much smaller amounts of cell-level data compared to the large datasets needed to train models from scratch. Furthermore, foundation models incorporate adversarial training elements or contrastive learning \cite{Chen_Ding_etal._2024,Xu_Usuyama_etal._2024}, enhancing their resilience and adaptability by exposing them to challenging and varied scenarios during training. This may result in more generalizable models, often making them well-suited for diverse and complex tasks in digital pathology.

Despite the inherent advantages of foundation models, their deployment for practical use faces its own obstacles. In particular, they require substantial computational power, financial investments and rigorous testing to ensure reliability and efficacy for a given task \cite{Akkus_Dangott_etal._2022,Dragomir_Cocuz_etal._2022,Go_2022,Jafri_Farooqui_etal._2024}. Moreover, while foundation models enhance feature representation and performance, they depend on the quality of available annotations for decoder fine-tuning and, like any other model, cannot resolve existing inconsistencies or ambiguities in data labels. Therefore, there remains a critical need for solutions that address both data quality and practical deployment considerations.
Further, integrating new technologies into existing clinical workflows often encounters resistance, as it necessitates adjustments to established diagnostic processes. So, there is a need to develop solutions that could be integrated into current practices, minimizing the burden on medical professionals to adopt new tools \cite{King_Williams_etal._2023}.

Existing solutions \cite{Goldsborough_Philps_etal._2024,Hörst_Rempe_etal._2024}, while addressing some aspects of these challenges, fall short in providing a comprehensive approach. To address the data quality and clinical deployment issues, we propose a multi-faceted solution that encompasses data refinement, model optimization, and integration with existing pathology tools (\hyperref[fig:fig1]{Figure 1}). The outcome is a lightweight cell segmentation and classification model that can be integrated into digital pathology workflows for practical clinical use.

\begin{figure}[h!]
    \centering
    \includegraphics[width=\textwidth, height=0.82\textheight, keepaspectratio]{images/Figure_1.pdf}
    \caption{Overview of the proposed solution, including 1) Data refinement using cross-relabeling, 2) Teacher model development and fine tuning, 3) Student model optimization with knowledge distillation and 4) Student model and QuPath integration}
    \label{fig:fig1}
\end{figure}
\clearpage

Our approach begins with preparing the data for the fine-tuning and training of the machine learning models. We create a refined dataset, acquired via cross-relabeling two cell-level datasets, enhancing annotation specificity and consistency of the labeled data. Subsequently, we create a cell segmentation and classification model based on the foundation model. We leverage the foundation model as a fixed encoder and fine-tune a decoder using the refined dataset to improve generalization across diverse tissue- and cell types.
To ensure that the model remains lightweight and deployable in a possibly resource-constrained environment, we employ knowledge distillation to approximate the functionality of the foundation model. Finally, to facilitate the practical application of our model in digital pathology workflows, we integrate it with the QuPath \cite{Bankhead_Loughrey_etal._2017} application. Each methodological component contributes to the overarching goal of enhancing model performance, generalizability, and usability in clinical settings.

The primary contributions of this paper are:
\begin{enumerate}
    \item \textit{Data labels refinement through cross-relabeling:}
    
    We propose a new method for refining labels of cell-level datasets through cross-relabeling. This method employs classification models to re-label broad and ambiguous instances, resulting in a more diverse dataset. Our evaluation demonstrates that these classification models achieve high accuracy on test subsets, indicating the reliability of the method for label refinement.

    \item \textit{Enhanced model performance via foundation models:}
    
    We employ a foundation model as a feature extractor for the cell segmentation and classification task. In comparison with training a CNN model from scratch, the foundation model backbone only needs fine-tuning, which significantly reduces training time, computational resources and data requirements. We show that using a foundation model encoder leads to better performance in cell segmentation and classification networks than using a CNN-based encoder. This improvement may enable the model to generalize more effectively across various tissue types and imaging methods.
    
    \item \textit{Model optimization through knowledge distillation:}
    
    We show that a smaller student model trained using knowledge distillation on the refined dataset obtained via our cross-relabeling approach from a foundation model achieves comparable performance in cell segmentation and quantification tasks. As a result, this model is more suitable for deployment in environments without high-performance computing resources.
    
    \item \textit{Integration with QuPath:}
    
    We integrate the distilled cell segmentation and classification model into QuPath, a widely used open-source digital pathology platform, to accelerate clinical adaptation by enabling pathologists to more easily incorporate advanced computational tools into their existing workflows.
\end{enumerate}

Through these methodological steps, we aim to bridge the gap between advanced machine learning techniques and practical clinical applications, making accurate and efficient digital pathology accessible in a broader range of healthcare settings.

\section{Refining Existing Datasets Using Cross-Relabeling}
To address the limitations of sparse and ambiguous labeling of cell-level datasets, we propose a generalizable cross-relabeling strategy that can be applied to any dataset containing broadly categorized or imprecisely labeled cell types. This approach involves training and subsequently leveraging classification models to refine broad categories into more specific or biologically relevant classes.
When applied to cell-level data, the methodology includes extracting individual cell images from the dataset patches, preprocessing these images to standardize the size and accommodate partial cells, and then training deep learning classifiers capable of distinguishing between the finer cell subtypes within the coarser categories. 
To illustrate our approach, we focus on the PanNuke \cite{Gamper_Koohbanani_etal._2020, Gamper_Koohbanani_etal._2019} and MoNuSAC \cite{Verma_Kumar_etal._2021} datasets that we have used to train models for cell quantification in our previous works \cite{Shvetsov_Grønnesby_etal._2022,Shvetsov_Sildnes_etal._2024}. We find that for better cell differentiation we have to introduce more granular labels. PanNuke includes a broad classification of "inflammatory" cells, encompassing lymphocytes, macrophages, and neutrophils. Each cell type differs significantly in structure, function, and clinical relevance. Conversely, MoNuSAC uses the label "epithelial" for a class that comprises both benign epithelial cells and malignant neoplastic cells. This practice makes it challenging to differentiate between benign and malignant epithelial cells in the dataset, which is a critical distinction when identifying tumor areas within tissue samples. To address these issues, we implement a cross-relabeling strategy as shown in \hyperref[fig:fig2]{Figure 2}. The key components are two classification models: one is trained on singular cell images from PanNuke data to classify the epithelial meta-class into epithelial and neoplastic classes. The other is trained on MoNuSAC to refine the inflammatory class into lymphocytes, neutrophils, and macrophages.

\begin{figure}[h!]
    \centering
    \includegraphics[width=\textwidth]{images/Figure_2.pdf}
    \caption{Refined dataset generation via cross relabeling}
    \label{fig:fig2}
\end{figure}

The refining approach consists of three consecutive steps. The first is the preprocessing step, in which we extract individual cells from both datasets (\hyperref[fig:fig3]{Figure 3}). The specifics of PanNuke and MoNuSAC patch preparation before cell preprocessing are provided in \hyperref[chap:S1]{Appendix S1}.

\begin{figure}[h!]
    \centering
    \includegraphics[width=\textwidth]{images/Figure_3.pdf}
    \caption{Cell instances preprocessing including (1) cell map extraction, (2) bounding box delineation, (3) adjusting cell boxes and (4) cropping and resizing of cell images}
    \label{fig:fig3}
\end{figure}

During preprocessing, we extract cell type maps from the ground truth label mask and calculate bounding boxes around each cell instance. To accommodate partial cells at patch borders, a common issue in cropped patch images, we employ mirror padding and extend the field of view of the cell label by 15 pixels to capture adjacent cells. We then crop and resize the identified regions to $64 \times 64$ pixels using bicubic interpolation.

The preprocessed PanNuke dataset comprises 68,031 neoplastic and 23,207 epithelial cell images, while MoNuSAC comprises  33,104 lymphocytes, 1,252 neutrophils, and 1,695 macrophages, which we subsequently use in training cell classification models and classifying the cell image data \hyperref[fig:S2]{Appendix Figure S2 (1)}. 

The next step is to train two distinct ResNet50-based classifiers tailored to address the specific labeling challenges inherent in each dataset. We use ResNet50 for classification models due to its proven effectiveness for image classification tasks in histopathology \cite{pan2022reviewmachinelearningapproaches}, and its compatibility with small images. For the PanNuke dataset, we design the classifier, trained on MoNuSAC data, to disaggregate the heterogeneous "inflammatory" cell category into distinct subtypes: lymphocytes, macrophages, and neutrophils. Similarly, for the MoNuSAC dataset, the classifier is trained on PanNuke data and distinguishes between benign and malignant epithelial cells within the overarching "epithelial" label. By applying these targeted classifiers to their respective datasets, we assign more specific labels to individual cell instances, thus enabling us to create a unified dataset.
To ensure a balanced representation of classes, we train both models on datasets that had been equalized to match the size of the least represented class. Thus, we obtain datasets comprising 23,207 samples per class for PanNuke and 1,252 samples per class for MoNuSAC data. Next, we partition both of them into training (70\%), validation (20\%), and testing (10\%) subsets. To mitigate the risk of overfitting, we use a single dropout layer with a rate of p=0.5 in both models and data augmentation using randomized color perturbations, rotation, and horizontal and vertical flipping. We employ AdamW optimizer and the cross-entropy loss function for the training criterion.

To evaluate the two trained models, we measure the classification accuracy on the respective test subsets. The accuracies on the test subset for both classifiers are presented in \hyperref[tab:1]{Table 1}. The PanNuke model achieves an average accuracy of 93.57\%, with higher accuracy for neoplastic cells (96.06\%) compared to epithelial cells (86.26\%). The confusion matrix in Figure A3.1 shows that the model predominantly distinguishes accurately between epithelial and neoplastic tissues, with a substantial number of correct classifications and relatively few misclassifications. The MoNuSAC model demonstrates an average accuracy of 98.92\%, excelling in classifying lymphocytes (99.67\%) and macrophages (94.12\%), with lower performance for neutrophils (85.71\%). The confusion matrix in Figure A3.2 shows that the model identifies lymphocytes and performs reasonably well with macrophages and neutrophils.

\begin{table}[h!]
\renewcommand{\arraystretch}{1.5}
  \centering
  \caption{Cell classification results for PanNuke and MoNuSAC trained models (CI 95\%).}
  \label{tab:1}
  \begin{tabular}{|l|c|c|}
   \hline
   %\rowcolor{gray!30}
    Accuracy               & PanNuke model              & MoNuSAC model              \\
    \hline
    Average      & 0.936 (0.931--0.941)         & 0.989 (0.986--0.993)        \\
    \hline
    Neoplastic   & 0.961 (0.956--0.965)         & -                          \\
    \hline
    Epithelial   & 0.863 (0.849--0.877)         & -                          \\
    \hline
    Lymphocytes  & -                          & 0.997 (0.995--0.999)        \\
    \hline
    Neutrophils  & -                          & 0.857 (0.796--0.918)        \\
    \hline
    Macrophages  & -                          & 0.941 (0.906--0.976)        \\
    \hline
  \end{tabular}
\end{table}

Finally, during the last step, we use the model trained on PanNuke data for epithelial cells in MoNuSAC and the model trained on MoNuSAC for the inflammatory cells class in PanNuke. Specifically, we use classifier models to relabel epithelial cells in MoNuSAC and inflammatory cells in PanNuke data. Then we combine cells with refined labels and the rest of the cells in both datasets to create a refined dataset (\hyperref[fig:S2]{Appendix Figure S2 (2)}). The process of relabeling cells and visualizing them on a patch is shown in \hyperref[fig:fig4]{Figure 4}. The cell counts in the refined dataset are provided in \hyperref[tab:S4]{Appendix Table S4}.

\begin{figure}[h!]
    \centering
    \includegraphics[width=\textwidth, height=0.42\textheight, keepaspectratio]{images/Figure_4.pdf}
    \caption{Cell relabeling procedure for epithelial and inflammatory cell classes}
    \label{fig:fig4}
\end{figure}

%\hfill

Relabeling and combining datasets have been explored in a prior study \cite{Parulekar_Kanwat_etal._2023}, where consecutive fine-tuning on multiple datasets was employed to account for hierarchical class label structures. While the method presented in \cite{Parulekar_Kanwat_etal._2023} is intuitive, it often lacks consistency and requires multiple fine-tuning runs, which can be cumbersome and time-consuming. 
In contrast, cross-relabeling simplifies this process by using specialized classification models tailored to each dataset's specific labeling challenges. This approach provides better transparency and produces a unified dataset encompassing seven distinct cell types across multiple tissue samples, enhancing data diversity for further model training or fine-tuning.

Despite these improvements, cross-relabeling does not entirely resolve issues related to poor labeling quality or the amount of labeled data. Specifically, our results show lower accuracies persist for underrepresented classes, such as macrophages, which may stem from a limited sample availability and intrinsic challenges in distinguishing these cells based solely on H\&E staining. Furthermore, while our method enhances label specificity, it relies on the initial quality of the broad labels; thus, any fundamental inaccuracies in the original annotations can propagate through the relabeling process. Addressing the overall problem of limited data labels may require integrating additional data sources or utilizing complementary immunohistochemical staining methods.
Although the reported performance metrics are obtained from evaluations on the native test sets of each dataset, it is important to note that the primary application of these classifiers is to perform cross-relabeling, where a model trained on one dataset (e.g., PanNuke) is applied to another (e.g., MoNuSAC) and vice versa. We acknowledge that a more systematic evaluation of cross-dataset generalization is needed and could be performed in future work.

Overall, the refined dataset produced by our approach can enhance the supervised training or fine-tuning of cell segmentation and classification models, especially those that utilize pre-trained foundation models to improve feature extraction robustness. In addition, these models can detect nuanced classes that enable researchers to conduct more detailed analyses of biological processes in computational pathology.

\section{Foundation models for robust cell segmentation and classification}

Accurate cell segmentation and classification in digital pathology are hindered by limited labeled data and the fact that conventional CNNs are unable to capture global contextual information due to their local receptive field constraints \cite{Gheflati_Rivaz_2022,Yang_Marcus_etal.}. Traditional approaches in cell quantification have predominantly relied on CNN encoders, such as ResNet50, given their proven effectiveness in semantic segmentation tasks \cite{Deshmane_2023,Graham_Vu_etal._2019,Mukasheva_Koishiyeva_etal._2024,Stringer_Wang_etal._2021}. However, approaches that include fine-tuning of pretrained CNNs, data augmentation, and stain normalization to partially increase data variability and address staining differences often fail to achieve the necessary generalization and robustness across diverse tissue types and staining conditions \cite{G._Wang_W._Li_etal._2018,Gao_Bagci_etal._2018,Karim_El_Khoury_Martin_Fockedey_etal._2021}.

To overcome these challenges, we leverage an encoder-decoder network that uses a foundation model as the encoder and a CNN upsampling decoder (\hyperref[fig:fig5]{Figure 5}) for simultaneous cell segmentation and classification in 2D patches extracted from WSIs. Foundation models with transformer-based architectures are viable alternatives to CNN-based encoders \cite{Shamshad_Khan_etal._2023,Sourget_2023}. They enable the creation of more advanced architectures that can decode or transform learned features more effectively \cite{Chen_Duan_etal._2023,Cheng_Misra_etal._2022,Xie_Wang_etal._2021}.

\begin{figure}[h!]
    \centering
    \includegraphics[width=\textwidth]{images/Figure_5.pdf}
    \caption{UNETR-like model with foundational model as backbone}
    \label{fig:fig5}
\end{figure}

By utilizing a transformer-based encoder, we incorporate global contextual information into the feature extraction process, which is a key advantage of such architectures \cite{Chen_Lu_etal._2021}. This foundation model integration facilitates accurate pixel-wise segmentation and classification without the need for extensive encoder training, thereby potentially improving generalization across varied cellular structures and tissue types.
In our implementation, we employ a modified UNETR \cite{Hatamizadeh_Tang_etal._2021} architecture that combines a vision transformer (ViT) \cite{Dosovitskiy_Beyer_etal._2021} encoder with a CNN-based decoder. The encoder utilizes the pretrained H-Optimus foundation model, which contains 1.1 billion parameters and is trained on over 500,000 H\&E stained WSIs \cite{Saillard_Jenatton_etal._2024}. We extract outputs from four evenly spaced transformer blocks $Z_i$, where $i \in [1, 14, 26, 38]$, to serve as residual connections for the CNN decoder. We select these blocks based on our observation that features from non-adjacent levels of the encoder lead to better overall performance on the test subset.

The CNN decoder upsamples the feature representations, acquired from the transformer blocks, to generate an intermediate vector that is handled by two task-specific layers that generate cell segmentation and classification masks. The first task-specific layer is the ‘Cellpose head’,  which is used to delineate cell instances. The layer generates horizontal and vertical gradient maps to form vector fields that are refined through gradient tracking in a post-processing step using the Cellpose algorithm \cite{Stringer_Wang_etal._2021}, known for its efficacy in cell segmentation tasks and generalizability across multiple domains \cite{Pachitariu_Stringer_2022,Stringer_Pachitariu_2024}. The second task-specific layer is the "Cell type head", which assigns labels to individual pixels. In the post-processing step, we determine the output classification label of each segmented cell instance by majority voting over the labeled pixels that comprise the cell in the segmentation map.

To evaluate model performance and measure the impact of adding a foundation model as backbone, we compare it to a ResNet50-based model. ResNet50 is a widely used solution for encoders in segmentation architectures in the medical domain \cite{Deshmane_2023,Graham_Vu_etal._2019,Mukasheva_Koishiyeva_etal._2024,Stringer_Wang_etal._2021}. For the H-Optimus-based model, we utilize frozen weights for the encoder and only fine-tune the decoder to take advantage of the extensive pre-training of the foundation model. For the ResNet50-based model we start with ImageNet \cite{Deng_Dong_etal.} weights and train both encoder and decoder parts. Hyperparameters for the training step are set to be identical, where possible, for comparable evaluation. 
For this evaluation, we deliberately use the PanNuke dataset to provide a standardized and controlled comparison between the H‑Optimus and ResNet50-based models (\hyperref[fig:S2]{Appendix Figure S2 (3)}). Specifically, we use two of the default PanNuke dataset splits (66\%) for training and validation, and reserve the third split (33\%) for testing.

To address the challenge of cell class imbalance in the PanNuke dataset, which is a common characteristic in most cell-level H\&E patch datasets, both models’ training processes employ a weighted loss function comprising cross-entropy and focal loss \cite{Lin_Goyal_etal._2018}. The focal loss component is adjusted with coefficients derived from each cell class' instance frequency, emphasizing learning from underrepresented classes and enhancing the model's sensitivity to rare but significant cellular patterns. The cross-entropy loss is augmented with spectral decoupling regularization \cite{Pezeshki_Kaba_etal._2021,Pohjonen_Stürenberg_etal._2022} and spatially varying label smoothing \cite{Islam_Glocker_2021}, which potentially stabilizes training and improves generalization in case of complex tissue morphologies. For optimization, we employ the \textit{AdamW} \cite{Loshchilov_Hutter_2019} to counter unbalanced class scenarios, with cosine annealing learning rate scheduler.

We utilize the scikit-learn library \cite{Van_der_Walt_Schönberger_etal._2014} and HoVer-Net \cite{Graham_Vu_etal._2019} implementations of $R^2$ (the coefficient of determination) and $PQ$ (panoptic quality) to evaluate our experiments. Complete mathematical formulations and detailed explanations of these metrics are provided in \hyperref[chap:S5]{Appendix S5}. To compute confidence intervals, we use nonparametric bootstrapping, where after calculating the metric on the full sample, we generated 1000 bootstrap replicates by resampling with replacement and then determined the 95\% confidence intervals as the 2.5th and 97.5th percentiles of the resulting empirical distribution.

%\hfill

The model comparisons are summarized in \hyperref[tab:2]{Table 2}. The H‑Optimus-based model achieves higher $R^2$ across all cell classes compared to the ResNet50-based model, which means that its predictions are more closely aligned with the PanNuke cell counts, indicating a stronger correlation with the observed data. Notably, the improvement of $R^2_{dead}$ may be an indicator of better global contextual representations provided by the foundation model backbone. In terms of segmentation and classification quality combined, measured by the PQ score, the H‑Optimus-based model demonstrates notable improvements across most cell classes. Overall, the average $R^2$ improved from 0.575 to 0.871, while the average $PQ$ score improved from 0.450 to 0.492, demonstrating better performance of the H-Optimus-based model.

\begin{table}[h!]
\renewcommand{\arraystretch}{1.5}
  \centering
  \caption{Cell quantification metrics for baseline and proposed models (CI 95\%).}
  \label{tab:2}
  \begin{tabular}{|l|c|c|}
    \hline
    %\rowcolor{gray!30}
    Metric             & Resnet50-based            & H-optimus-based              \\
    \hline
    $R^2_{neoplastic}$    & 0.681 (0.576--0.769)       & \textbf{0.941 (0.917--0.960)} \\
    \hline
    $R^2_{inflammatory}$  & 0.863 (0.778--0.903)       & \textbf{0.949 (0.918--0.966)} \\
    \hline
    $R^2_{connective}$    & 0.600 (0.488--0.698)       & 0.609 (0.436--0.772)          \\
    \hline
    $R^2_{dead}$          & 0.097 (-11.389--0.669)     & 0.925 (0.404--0.982)          \\
    \hline
    $R^2_{epithelial}$    & 0.635 (0.490--0.747)       & \textbf{0.930 (0.886--0.964)} \\
    \hline
    $PQ_{neoplastic}$       & 0.517 (0.499--0.535)       & \textbf{0.589 (0.575--0.604)} \\
    \hline
    $PQ_{inflammatory}$     & 0.455 (0.429--0.482)       & \textbf{0.528 (0.507--0.549)} \\
    \hline
    $PQ_{connective}$       & 0.416 (0.400--0.431)       & \textbf{0.451 (0.436--0.465)} \\
    \hline
    $PQ_{dead}$             & 0.374 (0.342--0.408)       & 0.292 (0.209--0.365)          \\
    \hline
    $PQ_{epithelial}$       & 0.488 (0.460--0.519)       & \textbf{0.599 (0.579--0.618)} \\
    \hline
  \end{tabular}
\end{table}

Our results  show that integrating the H‑Optimus foundation model within the UNETR architecture enhances the model's ability to segment and classify cells across diverse tissues from PanNuke data. The pretrained transformer encoder provides robust feature representations, resulting in higher average $R^2$ and $PQ$ scores compared to the CNN-based model. This leads to more reliable cell quantification and more accurate downstream analysis. Additionally, the streamlined fine-tuning process reduces computational overhead and training time, making the model more adaptable for new data.

Despite these advancements, the foundation model-based approach does not fully resolve all challenges related to cell segmentation and classification. We observe lower metric scores for underrepresented classes in the training data. Furthermore, foundation models typically encompass billions of parameters, resulting in substantial computational and memory requirements. It therefore poses challenges for deployment in resource-constrained environments, limiting their practical applicability in certain clinical settings.

\section{Model optimization via Knowledge Distillation}

To address the limitations posed by the extensive size of foundation models, we implement knowledge distillation — a model compression technique that leverages the teacher-student paradigm \cite{Hinton_Vinyals_etal._2015}. By training a smaller, more efficient student model to replicate the output of a larger, pre-trained teacher model, we retain performance while significantly reducing the model's complexity and resource requirements (\hyperref[fig:fig6]{Figure 6}).

\begin{figure}[h!]
    \centering
    \includegraphics[width=\textwidth, height=0.45\textheight, keepaspectratio]{images/Figure_6.pdf}
    \caption{Knowledge distillation framework for training a student model using a pre-trained teacher}
    \label{fig:fig6}
\end{figure}

We employ knowledge distillation to compress the H‑Optimus-based teacher model into a more efficient student model. The teacher model is the modified UNETR architecture with the H‑Optimus foundation model described in the previous chapter. The student model is based on a UNet architecture augmented with residual connections and incorporates a smaller ViT encoder with 9 million parameters \cite{Steiner_Kolesnikov_etal._2022,Wightman_2019}. 

First, we fine-tune the teacher model using the refined dataset from the cross-relabeling procedure (Section 2). Initially we train the decoder of the teacher model while keeping the encoder weights frozen. We split the refined dataset into train (70\%), validation (20\%) and test (10\%) subsets (\hyperref[fig:S2]{Appendix Figure S2 (4)}). During fine-tuning, we use the train and validation subsets, while leaving the test subset for model evaluation. We set the training procedure and model hyperparameters to be identical to those that were used to demonstrate the utility of foundation models for the simultaneous cell segmentation and classification task.

Next, we perform knowledge distillation from teacher to student using the refined dataset used to fine-tune the teacher model. The student model is trained to replicate the teacher model's outputs. We utilize a specialized loss function that aligns the student's predicted probability distribution with the teacher's, incorporating the teacher's class probability distribution derived from the output. Following the methodology of Hinton et al. \cite{Hinton_Vinyals_etal._2015}, we experiment with various hyperparameter settings for the temperature ($T$) and the balancing coefficients ($\alpha$ and $\beta$) in the loss function. We vary $T$ from 1 to 20 and adjust $\alpha$ and $\beta$ to balance the distillation and student losses. Through iterative tuning and evaluation, we identify that setting $T=14$, $\alpha=0.3$, and $\beta=0.7$ yields a configuration that converges and closely approximates the teacher model's performance during training.

Finally, we assess the performance of both models using the $R^2$ and $PQ$ (defined in \hyperref[chap:S5]{Appendix S5}) on the test set of the refined dataset (\hyperref[tab:3]{Table 3}). We observe that the 95\% confidence intervals overlap for most cell types, so we cannot claim statistically significant performance differences between the teacher and student models. One exception appears in the neoplastic class. The teacher model produces an $R^2$ of 0.919, while the student model shows an $R^2$ of 0.852. In addition, the student model achieves higher $PQ$ values for the neoplastic and connective classes, though the confidence intervals show overlap.

\begin{table}[h!]
\renewcommand{\arraystretch}{1.5}
  \centering
  \caption{Cell quantification metrics for teacher and distilled student models (CI 95\%).}
  \label{tab:3}
  \begin{tabular}{|l|c|c|}
    \hline
    %\rowcolor{gray!30}
    Metric & Teacher & Student \\
    \hline
    $R^2_{neoplastic}$    & \textbf{0.919} (0.898--0.939) & 0.852 (0.800--0.891) \\
    \hline
    $R^2_{lymphocyte}$    & 0.969 (0.956--0.977)         & 0.969 (0.956--0.978) \\
    \hline
    $R^2_{connective}$    & 0.694 (0.548--0.809)         & 0.618 (0.469--0.741) \\
    \hline
    $R^2_{dead}$          & 0.755 (0.400--0.908)         & 0.424 (0.100--0.731) \\
    \hline
    $R^2_{epithelial}$    & 0.922 (0.870--0.958)         & 0.843 (0.738--0.917) \\
    \hline
    $R^2_{macrophage}$    & 0.384 (-0.369--0.724)        & 0.704 (0.352--0.859) \\
    \hline
    $R^2_{neutrofil}$     & 0.854 (0.578--0.929)         & 0.833 (0.502--0.925) \\
    \hline
    $PQ_{neoplastic}$       & 0.581 (0.569--0.593)         & 0.601 (0.588--0.613) \\
    \hline
    $PQ_{lymphocyte}$       & 0.536 (0.520--0.553)         & 0.563 (0.544--0.579) \\
    \hline
    $PQ_{connective}$       & 0.436 (0.421--0.451)         & 0.457 (0.441--0.474) \\
    \hline
    $PQ_{dead}$             & 0.272 (0.235--0.315)         & 0.279 (0.201--0.369) \\
    \hline
    $PQ_{epithelial}$       & 0.522 (0.500--0.545)         & 0.530 (0.506--0.555) \\
    \hline
    $PQ_{macrophage}$       & 0.524 (0.459--0.588)         & 0.474 (0.405--0.543) \\
    \hline
    $PQ_{neutrofil}$        & 0.541 (0.490--0.592)         & 0.565 (0.522--0.607) \\
    \hline
  \end{tabular}
\end{table}


We further decompose the $PQ$ metric into its $SQ$ and $DQ$ components (\hyperref[tab:S6]{Appendix Table S6}). Both models produce nearly identical $SQ$ values, which indicates that they predict instance boundaries with similar precision. Although the student model shows some improvement in $DQ$ scores for certain classes, the confidence intervals overlap and do not confirm a statistically significant difference.

We observe that the student and teacher models yield comparable detection performance despite the student model using a much smaller and simpler architecture. A model with fewer parameters reduces the risk of overfitting when training data are scarce relative to the model’s complexity \cite{Farias_Ludermir_etal._2022}. The knowledge distillation process also encourages the student model to focus on the most generalizable detection features learned from the teacher. These factors enable the student model to achieve similar detection performance across different cell types.

Additionally, considering the model sizes reported in \hyperref[tab:4]{Table 4}, the distilled model achieves a significant reduction compared to the teacher model, with a 48-fold decrease in parameter count and a 5.5-fold reduction in on-disk size. In inference mode, the teacher model requires 16 GB of VRAM for a batch size of 32, while the distilled model only needs 3 GB of VRAM for the same batch size. These reductions make the distilled model significantly more practical for fine-tuning and deployment in resource-constrained environments.

\begin{table}[h!]
\renewcommand{\arraystretch}{1.5}
  \centering
  \caption{Parameter counts and size of teacher and distilled model}
  \label{tab:4}
  \adjustbox{max width=\textwidth}{%
  \begin{tabular}{|l|c|c|c|}
    \hline
    %\rowcolor{gray!30}
    Metric & H-optimus-based (Teacher) & mobileViT-based (Student) & Magnitude of difference \\
    \hline
    Parameters count       & 1,158,917,906   & \textbf{24,093,393}   & \textbf{48x}  \\
    \hline
    Estimated Total Size (MB) & 87,912       & \textbf{15,935}    & \textbf{5.5x} \\
    \hline
  \end{tabular}%
}
\end{table}

%\hfill

With recent advancements in complex network architectures and the use of pretrained encoders to achieve state-of-the-art performance \cite{Baumann_Dislich_etal._2024,Hörst_Rempe_etal._2024} in cell segmentation and classification tasks, model size, computational complexity, and processing times have increased. This limits the scalability and accessibility of these models. As we demonstrate, this may be mitigated using knowledge distillation. Studies in the field of natural language processing have demonstrated the efficacy of knowledge distillation in retaining the capabilities of the teacher model while achieving significant reductions in size and complexity \cite{Huangpu_Gao_2024,Sun_Yu_etal.}. 

We demonstrate the feasibility of knowledge distillation in digital pathology, specifically for cell segmentation and classification tasks. Moreover, we achieve this performance while also significantly reducing the parameter count. In addressing the challenge of knowledge transfer, we found that distillation from a transformer-based model to a smaller transformer is more straightforward than attempting to map transformer features to CNN blocks. In our experiments, using a CNN-based network as a student results in worse cell quantification performance due to the structural constraints of CNN feature space dimensions. 

Although our primary approach relies on a transformer-based student model that performs well, it can be further optimized to incorporate advantages from CNN architectures. For example, employing alternative techniques such as using ViT adapters \cite{Chen_Duan_etal._2023} or $1 \times 1$ convolutions to adjust feature map sizes may be beneficial for harnessing CNN advantages like enhanced local feature extraction. Moreover, if additional performance improvements are desired, the process can be further enhanced by applying supplementary knowledge distillation techniques, such as self-distillation \cite{Zhang_Song_etal._2019} or online distillation \cite{Houyon_Cioppa_etal._2023}.

Despite these promising results, further validation on independent datasets is necessary to fully understand the model's limitations. Underrepresented classes may pose challenges when addressing complex cases. Pathologists need to validate these models to adopt them in clinical settings. While the distilled models are smaller and more deployable, a technological gap persists because pathologists traditionally rely on established methods for inspecting WSIs and diagnosing diseases. Addressing the complexities involved in deploying models for inference and supporting pathologists in adopting new tools is essential for integrating these models into clinical workflows.

\section{Model integration with QuPath}
Digital pathology tools with graphical user interfaces are essential for visualizing and analyzing WSIs. To make our student model useful in clinical pathology workflows, it needs to be integrated into a tool that enables inspecting regions, creating annotations, and providing quantitative analyses of biomarkers. Therefore, we integrate the trained student model from the previous chapter into the QuPath open‑source platform \cite{Bankhead_Loughrey_etal._2017}. QuPath provides the required annotation, visualization, and analysis tools to interpret complex histological data, including workflows for cell segmentation, classification, and quantification (\hyperref[fig:fig7]{Figure 7}). 

\begin{figure}[h!]
    \centering
    \includegraphics[width=\textwidth]{images/Figure_7.pdf}
    \caption{Visualization of model-generated cell quantification annotations (left) and the corresponding unannotated slide (right) in QuPath}
    \label{fig:fig7}
\end{figure}

To identify the regions in a WSI critical for prognosticating tumor development, such as specific tumor areas or border regions without overlapping healthy tissue, the pathologist uses QuPath to outline these regions. Then, the pathologist initiates a cell segmentation and classification script through the QuPath interface for the selected regions. The resulting annotations and quantified cell information are then directly overlaid onto the WSI in the QuPath interface. Additional design and implementation details are in \hyperref[chap:S7]{Appendix S7}. 

Two common approaches for integrating deep learning models into QuPath are Java‑based native QuPath extensions \cite{Goldsborough_Philps_etal._2024} and the execution of RESTful API requests to a model server coupled with handling the response via an extension, as demonstrated in the application of cell segmentation models applied to immunofluorescence images \cite{Sugawara_2023}. While the community is actively working on these integration strategies, there is currently no universal solution that fully addresses all integration and performance requirements.

Extensions may offer better integration with QuPath, allowing slightly improved performance and more widespread usage of the built-in QuPath models, but they lack the flexibility to customize models and modify their behavior. For example, the newest version of QuPath includes models such as StarDist \cite{Weigert_Schmidt} and InstanSeg \cite{Goldsborough_Philps_etal._2024} that can perform cell segmentation. Both models pose limitations when applied to simultaneous cell segmentation and classification. StarDist performs well only on convex, round shapes by design, whereas some neoplastic, inflammatory, and connective cells exhibit complex and non-convex shapes. InstanSeg provides only semantic segmentation without assigning classes to the segmented cells.

%\hfill

In contrast, our approach offers an alternative integration strategy. It utilizes the paquo library to directly interact with QuPath’s internal application programming interface from within Python. This enables data exchange and processing without the need for intermediate conversion steps and provides greater control over model customization, retraining, and the incorporation of custom processing steps.

The integration of our custom model with QuPath underscores its potential to significantly enhance the diagnostic process by reducing the time burden on pathologists and enabling them to focus on more complex interpretative tasks using familiar software. Leveraging a tool that is already well-established among pathologists increases the likelihood of its adoption into daily clinical workflows. The quantitative data generated through the automated workflow is critical for both clinical decision-making and research, facilitating more accurate biomarker analysis, enabling robust statistical evaluations, and supporting hypothesis generation and testing. Additionally, by streamlining cell segmentation and classification, the tool enhances the scalability and reproducibility of pathological assessments, ultimately contributing to improved diagnostic accuracy and patient outcomes.

\section{Conclusion and future work}

In this study, we address critical challenges in digital pathology and tackle the usability and deployment issues of the developed models in standard computing environments without the need for high-performance computing systems. Our multi-faceted approach encompasses data refinement through cross-relabeling, leveraging foundation models for robust cell segmentation and classification, optimizing model performance via knowledge distillation, and integrating the optimized model into the QuPath software for practical application. This approach is used to construct a capable, versatile, and adjustable model for cell segmentation and classification, with enhanced performance and usability.

\begin{sloppypar}
While our approach shows potential in the field of computational pathology, certain limitations persist. 
For example, our implementation currently exhibits lower performance in detecting macrophages. 
This serves as an instance of the broader challenge of accurately identifying complex cell types. In order to address this issue, extending our approach to incorporate additional data sources, exploring alternative modeling approaches, and integrating other imaging modalities such as immunohistochemical staining may help improve detection accuracy. Moreover, although the distilled model reduces computational demands, integrating advanced deep learning models into clinical practice requires addressing technological gaps and potential resistance to adopting new tools within established diagnostic processes.
\end{sloppypar}

Future work could focus on several key areas to refine the proposed approach and facilitate its adoption in clinical environments. Enhancing the cell-relabeling process with additional datasets \cite{Graham_Jahanifar_etal._2021} could improve the representation of underrepresented cell types and enhance overall model performance. Also, incorporating additional data sources, such as multi-modal imaging or complementary staining methods, may address limitations related to cell type differentiation and class imbalance. Exploring other foundation models \cite{Vorontsov_Bozkurt_etal._2024,Zimmermann_Vorontsov_etal._2024} or introducing additional modalities \cite{Ding_Wagner_etal._2024,Vaidya_Zhang_etal._2025} may provide alternative architectures better suited to specific tasks or offer improved efficiency. Implementing more complex knowledge distillation techniques \cite{Houyon_Cioppa_etal._2023,Zhang_Song_etal._2019} could further optimize the model's performance and adaptability. Additionally, deeper integration with QuPath or other digital pathology software could provide pathologists more control over cell quantification analysis directly within the QuPath interface, thereby increasing accessibility and usability. Such enhancements would not only refine model performance but also ensure greater adaptability and scalability within various clinical environments. Finally, extensive validation of the model by pathologists and benchmarking against independent datasets are essential steps toward establishing the model's reliability and fostering confidence in its clinical utility.

\section*{Acknowledgments} 
This work was funded in part by the Research Council of Norway grant no. 309439 SFI Visual Intelligence, and the North Norwegian Health Authority grant no. HNF1521-20.

\bibliographystyle{IEEEtran}
\begin{sloppypar}
\begin{thebibliography}{99}

\bibitem{chaplot2020neural} Chaplot, Devendra Singh, et al. "Neural topological slam for visual navigation." Proceedings of the IEEE/CVF conference on computer vision and pattern recognition. 2020.

\bibitem{maksymets2021thda} Maksymets, Oleksandr, et al. "Thda: Treasure hunt data augmentation for semantic navigation." Proceedings of the IEEE/CVF International Conference on Computer Vision. 2021.

\bibitem{mezghan2022memory} Mezghan, Lina, et al. "Memory-augmented reinforcement learning for image-goal navigation." 2022 IEEE/RSJ International Conference on Intelligent Robots and Systems (IROS). IEEE, 2022.

\bibitem{al2022zero} Al-Halah, Ziad, Santhosh Kumar Ramakrishnan, and Kristen Grauman. "Zero experience required: Plug \& play modular transfer learning for semantic visual navigation." Proceedings of the IEEE/CVF Conference on Computer Vision and Pattern Recognition. 2022.

\bibitem{ye2021auxiliary} Ye, Joel, et al. "Auxiliary tasks and exploration enable objectgoal navigation." Proceedings of the IEEE/CVF international conference on computer vision. 2021.

\bibitem{chaplot2020object} Chaplot, Devendra Singh, et al. "Object goal navigation using goal-oriented semantic exploration." Advances in Neural Information Processing Systems 33 (2020)

\bibitem{ramakrishnan2022poni} Ramakrishnan, Santhosh Kumar, et al. "Poni: Potential functions for objectgoal navigation with interaction-free learning." Proceedings of the IEEE/CVF Conference on Computer Vision and Pattern Recognition. 2022.

\bibitem{ramrakhya2022habitat} Ramrakhya, Ram, et al. "Habitat-web: Learning embodied object-search strategies from human demonstrations at scale." Proceedings of the IEEE/CVF Conference on Computer Vision and Pattern Recognition. 2022.

\bibitem{mousavian2019visual} Mousavian, Arsalan, et al. "Visual representations for semantic target driven navigation." 2019 International Conference on Robotics and Automation (ICRA). IEEE, 2019.

\bibitem{dhariwal2021diffusion} Dhariwal, Prafulla, and Alexander Nichol. "Diffusion models beat gans on image synthesis." Advances in neural information processing systems 34 (2021)

\bibitem{ho2022classifier} Ho, Jonathan, and Tim Salimans. "Classifier-free diffusion guidance." arXiv preprint arXiv:2207.12598 (2022).

\bibitem{nichol2021glide} Nichol, Alex, et al. "Glide: Towards photorealistic image generation and editing with text-guided diffusion models." arXiv preprint arXiv:2112.10741 (2021)

\bibitem{brooks2023instructpix2pix} Brooks, Tim, Aleksander Holynski, and Alexei A. Efros. "Instructpix2pix: Learning to follow image editing instructions." Proceedings of the IEEE/CVF Conference on Computer Vision and Pattern Recognition. 2023.

\bibitem{fu2023guiding} Fu, Tsu-Jui, et al. "Guiding instruction-based image editing via multimodal large language models." arXiv preprint arXiv:2309.17102 (2023).

\bibitem{geng2024instructdiffusion} Geng, Zigang, et al. "Instructdiffusion: A generalist modeling interface for vision tasks." Proceedings of the IEEE/CVF Conference on Computer Vision and Pattern Recognition. 2024.

\bibitem{zhou2024minedreamer} Zhou, Enshen, et al. "Minedreamer: Learning to follow instructions via chain-of-imagination for simulated-world control." arXiv preprint arXiv:2403.12037 (2024).

\bibitem{zhou2023esc} Zhou, Kaiwen, et al. "Esc: Exploration with soft commonsense constraints for zero-shot object navigation." International Conference on Machine Learning. PMLR, 2023.

\bibitem{yu2023l3mvn} Yu, Bangguo, Hamidreza Kasaei, and Ming Cao. "L3mvn: Leveraging large language models for visual target navigation." 2023 IEEE/RSJ International Conference on Intelligent Robots and Systems (IROS). IEEE, 2023.

\bibitem{gadre2023cows} Gadre, Samir Yitzhak, et al. "Cows on pasture: Baselines and benchmarks for language-driven zero-shot object navigation." Proceedings of the IEEE/CVF Conference on Computer Vision and Pattern Recognition. 2023.

\bibitem{shah2023navigation} Shah, Dhruv, et al. "Navigation with large language models: Semantic guesswork as a heuristic for planning." Conference on Robot Learning. PMLR, 2023.

\bibitem{cai2024bridging} Cai, Wenzhe, et al. "Bridging zero-shot object navigation and foundation models through pixel-guided navigation skill." 2024 IEEE International Conference on Robotics and Automation (ICRA). IEEE, 2024.

\bibitem{yu2023co} Yu, Bangguo, Hamidreza Kasaei, and Ming Cao. "Co-NavGPT: Multi-robot cooperative visual semantic navigation using large language models." arXiv preprint arXiv:2310.07937 (2023).

\bibitem{wu2024voronav} Wu, Pengying, et al. "Voronav: Voronoi-based zero-shot object navigation with large language model." arXiv preprint arXiv:2401.02695 (2024).

\bibitem{qin2023mp5} Qin, Yiran, et al. "Mp5: A multi-modal open-ended embodied system in minecraft via active perception." arXiv preprint arXiv:2312.07472 (2023).

\bibitem{du2024learning} Du, Yilun, et al. "Learning universal policies via text-guided video generation." Advances in Neural Information Processing Systems 36 (2024).

\bibitem{ajay2024compositional} Ajay, Anurag, et al. "Compositional foundation models for hierarchical planning." Advances in Neural Information Processing Systems 36 (2024).

\bibitem{liang2024skilldiffuser} Liang, Zhixuan, et al. "Skilldiffuser: Interpretable hierarchical planning via skill abstractions in diffusion-based task execution." Proceedings of the IEEE/CVF Conference on Computer Vision and Pattern Recognition. 2024.

\bibitem{heusel2017gans} Heusel, Martin, et al. "Gans trained by a two time-scale update rule converge to a local nash equilibrium." Advances in neural information processing systems 30 (2017).

\bibitem{zhang2018unreasonable} Zhang, Richard, et al. "The unreasonable effectiveness of deep features as a perceptual metric." Proceedings of the IEEE conference on computer vision and pattern recognition. 2018.

\bibitem{brown2020language} Brown, Tom B. "Language models are few-shot learners." arXiv preprint arXiv:2005.14165 (2020).

\bibitem{podell2023sdxl} Podell, Dustin, et al. "Sdxl: Improving latent diffusion models for high-resolution image synthesis." arXiv preprint arXiv:2307.01952 (2023).

\bibitem{brohan2022rt} Brohan, Anthony, et al. "Rt-1: Robotics transformer for real-world control at scale." arXiv preprint arXiv:2212.06817 (2022).

\bibitem{brohan2023rt} Brohan, Anthony, et al. "Rt-2: Vision-language-action models transfer web knowledge to robotic control." arXiv preprint arXiv:2307.15818 (2023).

\bibitem{li2024manipllm} Li, Xiaoqi, et al. "Manipllm: Embodied multimodal large language model for object-centric robotic manipulation." Proceedings of the IEEE/CVF Conference on Computer Vision and Pattern Recognition. 2024.

\bibitem{shah2023vint} Shah, Dhruv, et al. "ViNT: A foundation model for visual navigation." arXiv preprint arXiv:2306.14846 (2023).

\bibitem{liu2024visual} Liu, Haotian, et al. "Visual instruction tuning." Advances in neural information processing systems 36 (2024).

\bibitem{hu2021lora} Hu, Edward J., et al. "Lora: Low-rank adaptation of large language models." arXiv preprint arXiv:2106.09685 (2021).

\bibitem{qin2023supfusion} Qin, Yiran, et al. "SupFusion: Supervised LiDAR-camera fusion for 3D object detection." Proceedings of the IEEE/CVF International Conference on Computer Vision. 2023.

\bibitem{qin2024worldsimbench} Qin, Yiran, et al. "Worldsimbench: Towards video generation models as world simulators." arXiv preprint arXiv:2410.18072 (2024).

\bibitem{yu2025gamefactory} Yu, Jiwen, et al. "GameFactory: Creating New Games with Generative Interactive Videos." arXiv preprint arXiv:2501.08325 (2025).

\bibitem{zhou2024code} Zhou, Enshen, et al. "Code-as-Monitor: Constraint-aware Visual Programming for Reactive and Proactive Robotic Failure Detection." arXiv preprint arXiv:2412.04455 (2024).

\bibitem{zhang2024ad} Zhang, Zaibin, et al. "AD-H: Autonomous Driving with Hierarchical Agents." arXiv preprint arXiv:2406.03474 (2024).

\bibitem{wang2024toward} Wang, Chaoqun, et al. "Toward Accurate Camera-based 3D Object Detection via Cascade Depth Estimation and Calibration." arXiv preprint arXiv:2402.04883 (2024).

\bibitem{huang2024story3d} Huang, Yuzhou, et al. "Story3d-agent: Exploring 3d storytelling visualization with large language models." arXiv preprint arXiv:2408.11801 (2024).

\bibitem{savinov2018semi} Savinov, Nikolay, Alexey Dosovitskiy, and Vladlen Koltun. "Semi-parametric topological memory for navigation." arXiv preprint arXiv:1803.00653 (2018).

\bibitem{majumdar2022zson} Majumdar, Arjun, et al. "Zson: Zero-shot object-goal navigation using multimodal goal embeddings." Advances in Neural Information Processing Systems 35 (2022): 32340-32352.

\bibitem{yadav2023offline} Yadav, Karmesh, et al. "Offline visual representation learning for embodied navigation." Workshop on Reincarnating Reinforcement Learning at ICLR 2023. 2023.

\bibitem{yadav2023ovrl} Yadav, Karmesh, et al. "Ovrl-v2: A simple state-of-art baseline for imagenav and objectnav." arXiv preprint arXiv:2303.07798 (2023).

\bibitem{sun2024fgprompt} Sun, Xinyu, et al. "FGPrompt: fine-grained goal prompting for image-goal navigation." Advances in Neural Information Processing Systems 36 (2024).

\bibitem{zhu2017target} Zhu, Yuke, et al. "Target-driven visual navigation in indoor scenes using deep reinforcement learning." 2017 IEEE international conference on robotics and automation (ICRA). IEEE, 2017.

\bibitem{koh2024generating} Koh, Jing Yu, Daniel Fried, and Russ R. Salakhutdinov. "Generating images with multimodal language models." Advances in Neural Information Processing Systems 36 (2024).

\bibitem{krantz2022instance} Krantz, Jacob, et al. "Instance-specific image goal navigation: Training embodied agents to find object instances." arXiv preprint arXiv:2211.15876 (2022).

\bibitem{schulman2017proximal} Schulman, John, et al. "Proximal policy optimization algorithms." arXiv preprint arXiv:1707.06347 (2017).

\bibitem{anderson2018evaluation} Anderson, Peter, et al. "On evaluation of embodied navigation agents." arXiv preprint arXiv:1807.06757 (2018).

\bibitem{lin2024navcot} Lin, Bingqian, et al. "NavCoT: Boosting LLM-Based Vision-and-Language Navigation via Learning Disentangled Reasoning." arXiv preprint arXiv:2403.07376 (2024).

\bibitem{NavGPT} Zhou, Gengze, Yicong Hong, and Qi Wu. "Navgpt: Explicit reasoning in vision-and-language navigation with large language models." Proceedings of the AAAI Conference on Artificial Intelligence.

\bibitem{hahn2021no} Hahn, Meera, et al. "No rl, no simulation: Learning to navigate without navigating." Advances in Neural Information Processing Systems 34 (2021): 26661-26673.

\bibitem{li2025t2isafety} Li, Lijun, et al. "T2ISafety: Benchmark for Assessing Fairness, Toxicity, and Privacy in Image Generation." arXiv preprint arXiv:2501.12612 (2025).

\bibitem{an2024agfsync} An, Jingkun, et al. "AGFSync: Leveraging AI-Generated Feedback for Preference Optimization in Text-to-Image Generation." arXiv preprint arXiv:2403.13352 (2024).


\end{thebibliography}
\end{sloppypar}

\clearpage
\beginsupplement
\section*{Appendix}
\renewcommand{\thesubsection}{S\arabic{subsection}}

\subsection{\label{chap:S1}PanNuke and MoNuSAC preprocessing}
The PanNuke dataset comprises a set of 7,901 RGB patches, each with dimensions of $256 \times 256$ pixels, which we set as the standard patch size for our analysis. In contrast, the MoNuSAC dataset encompasses 294 images of heterogeneous dimensions. To standardize the MoNuSAC images with our experiments, we implement a standardization protocol. Specifically, for images exceeding the dimensions of $256 \times 256$ pixels, we segment them into equal-sized patches and apply mirror padding to the remaining portions to avoid information loss at the peripherals. Patches with dimensions less than $128 \times 128$ pixels are excluded from the dataset due to the insufficient resolution to capture relevant cellular details. For patches where either dimension falls between 128 and 256 pixels, we employ upsampling to achieve the standard patch size. As a result, we obtain a total of 2,823 RGB patches derived from the MoNuSAC dataset for subsequent analysis. For additional details on the MoNuSAC data preparation process, refer to the source code \cite{Shvetsov_2025a}.
\clearpage

\subsection{\label{chap:S2}Data usage for the methodology}

\counterwithin{figure}{subsection}
\renewcommand{\thefigure}{S\arabic{subsection}}

\begin{figure}[h!]
    \centering
    \includegraphics[width=\textwidth, height=0.85\textheight, keepaspectratio]{images/A2.pdf}
    \caption{Overview of the methodology for cross-labeling, dataset refinement, and model comparison. (1) Cross-relabeling - training and testing cell classification models, (2) Cross-relabeling - using cell classification models to create refined dataset, (3) Fine-tuning and training models for comparison, (4) Student knowledge distillation with refined dataset}
    \label{fig:S2}
\end{figure}
\clearpage

\subsection{\label{chap:S3}Confusion matrices for classification models}
\counterwithin{figure}{subsection}
\renewcommand{\thefigure}{S\arabic{subsection}.\arabic{figure}}

\begin{figure}[h!]
    \centering
    \includegraphics[width=\textwidth, height=0.4\textheight, keepaspectratio]{images/A3_1.pdf}
    \caption{Confusion matrix for PanNuke trained model}
    \label{fig:S3.1}
\end{figure}

\begin{figure}[h!]
    \centering
    \includegraphics[width=\textwidth, height=0.4\textheight, keepaspectratio]{images/A3_2.pdf}
    \caption{Confusion matrix for MoNuSAC trained model}
    \label{fig:S3.2}
\end{figure}

\clearpage

\subsection{\label{chap:S4}Datasets cell counts}

\counterwithin{table}{subsection}
\renewcommand{\thetable}{S\arabic{subsection}}

\begin{table}[h!]
\renewcommand{\arraystretch}{2.0}
\centering
\caption{\label{tab:S4}Cell counts for PanNuke, MoNuSAC and refined datasets. Numbers in parentheses indicate preprocessed cell counts for cell classifier models training and testing.}
%\adjustbox{max width=\textwidth}{%
\begin{tabular}{|l|c|c|c|}
\hline
%\rowcolor{gray!30}
Cell type & PanNuke & MoNuSAC & Refined \\
\hline
Neoplastic & 77,403 (68,031) & - & 105,451 \\
\hline
Epithelial & 26,572 (23,207) & - & 29,926 \\
\hline
Epithelial (benign and malignant) & - & 31,402 & - \\
\hline
Inflammatory & 32,276 & - & - \\
\hline
Lymphocytes & - & 37,045 (33,104) & 65,275 \\
\hline
Neutrophils & - & 1,355 (1,252) & 3,833 \\
\hline
Macrophage & - & 1,842 (1,695) & 3,410 \\
\hline
Dead & 2,908 & - & 2,908 \\
\hline
Connective & 50,585 & - & 50,585 \\
\hline
\end{tabular}
%
%}
\end{table}



\clearpage

\subsection{\label{chap:S5}Definition of validation metrics}
\counterwithin{equation}{subsection}
\renewcommand{\theequation}{\arabic{equation}}

\subsubsection{\label{chap:S5.1}R\textsuperscript{2}}
The coefficient of determination, denoted as $R^2$, is a statistical measure that represents the proportion of variance in the dependent variable that is predictable from the independent variables. In the context of cell quantification in pathology, $R^2$ is used to assess how well the predicted quantities of different cell types in a patch align with the actual quantities observed in the ground truth data, with higher values representing more accurate quantification. $R^2$ is defined as
\begin{equation*}
R^2 = 1 - \frac{\sum_{i=1}^n (y_i - \hat{y}_i)^2}{\sum_{i=1}^n (y_i - \bar{y})^2},
\end{equation*}
where $y_i$ represents the actual number of cells of a specific type in the $i$-th image, $\hat{y}_i$ represents the predicted number of cells of that type in the $i$-th image, $\bar{y}$ is the mean of the actual numbers across all images, and $n$ is the total number of images in the dataset.

The $R^2$ metric has a range of $(-\infty, 1]$. An $R^2$ of 1 indicates perfect prediction, where all predicted values exactly match the actual values. An $R^2$ of 0 suggests that the model explains none of the variability of the response data around its mean. If $R^2$ is negative, it indicates that the model performs worse than a model that simply predicts the mean of the actual values for all observations.

\subsubsection{\label{chap:S5.2}PQ}
Panoptic Quality ($PQ$) is a comprehensive metric used to evaluate the performance of segmentation models in tasks that require both instance segmentation and classification. $PQ$ provides a single score that encapsulates both the detection accuracy (i.e., how many objects were correctly identified) and the segmentation quality (i.e., how accurately the objects' boundaries were delineated). This metric is particularly useful in multiclass scenarios where each pixel is classified into distinct categories, such as different cell types in pathology images.

$PQ$ is calculated as the product of two terms: Detection Quality ($DQ$) and Segmentation Quality ($SQ$). It can be expressed as
\begin{equation*}
PQ = DQ \cdot SQ,
\end{equation*}
where
\begin{equation*}
DQ = \frac{TP}{TP + 0.5\, FP + 0.5\, FN},
\end{equation*}
\begin{equation*}
SQ = \frac{\sum_{(p, g) \in \mathcal{M}} IoU(p, g)}{TP}.
\end{equation*}
In these formulas, $TP$ denotes the number of correctly matched instances between ground truth and prediction, $FP$ denotes the predicted instances that have no corresponding ground truth, $FN$ denotes the ground truth instances that were not detected, $IoU(p, g)$ is the Intersection over Union for a pair of matched instances $p$ (prediction) and $g$ (ground truth), and $\mathcal{M}$ is the set of matched pairs.

The $PQ$ metric is calculated for each class and is averaged across classes to provide a global performance measure.

The $PQ$ score has a range of $[0, 1.0]$, where a higher score indicates better performance in both detecting and segmenting the instances correctly. A $PQ$ of 1 signifies perfect identification and segmentation of all instances, whereas a $PQ$ of 0 indicates that no instances were correctly identified and segmented.

\clearpage

\subsection{\label{chap:S6}Segmentation and Detection quality metrics for teacher and student models}

\begin{table}[h!]
\renewcommand{\arraystretch}{2.0}
\centering
\caption{Segmentation and detection quality for student and teacher models (CI 95\%)}
\label{tab:S6}
%\adjustbox{max width=\textwidth}{%
\begin{tabular}{|l|c|c|}
\hline
%\rowcolor{gray!30}
Metric & Teacher & Student \\
\hline
$SQ_{neoplastic}$ & 0.819 (0.815--0.823) & 0.824 (0.819--0.828) \\
\hline
$SQ_{lymphocyte}$ & 0.795 (0.788--0.802) & 0.790 (0.783--0.796) \\
\hline
$SQ_{connective}$ & 0.770 (0.762--0.776) & 0.780 (0.772--0.786) \\
\hline
$SQ_{dead}$ & 0.659 (0.623--0.688) & 0.657 (0.624--0.695) \\
\hline
$SQ_{epithelial}$ & 0.780 (0.770--0.790) & 0.788 (0.779--0.797) \\
\hline
$SQ_{macrophage}$ & 0.788 (0.760--0.810) & 0.757 (0.730--0.783) \\
\hline
$SQ_{neutrofil}$ & 0.782 (0.761--0.801) & 0.775 (0.759--0.792) \\
\hline
$DQ_{neoplastic}$ & 0.706 (0.692--0.719) & 0.727 (0.712--0.741) \\
\hline
$DQ_{lymphocyte}$ & 0.675 (0.656--0.698) & 0.713 (0.691--0.734) \\
\hline
$DQ_{connective}$ & 0.566 (0.546--0.584) & 0.583 (0.565--0.602) \\
\hline
$DQ_{dead}$ & 0.410 (0.361--0.465) & 0.435 (0.306--0.561) \\
\hline
$DQ_{epithelial}$ & 0.668 (0.639--0.694) & 0.673 (0.644--0.702) \\
\hline
$DQ_{macrophage}$ & 0.657 (0.583--0.727) & 0.615 (0.531--0.703) \\
\hline
$DQ_{neutrofil}$ & 0.691 (0.625--0.753) & 0.729 (0.679--0.778) \\
\hline
\end{tabular}
%
%}
\end{table}

\clearpage

\subsection{\label{chap:S7}QuPath integration method}
We adopt an integration strategy leveraging the paquo \cite{Bayer_AG} library, a Python package that enables direct interaction with QuPath’s internal API, thereby facilitating seamless data exchange without intermediate conversion steps. The data processing pipeline (\hyperref[fig:S7]{Appendix Figure S7}) begins with the acquisition of WSIs and their associated annotations from QuPath, which are represented as Shapely \cite{Gillies_Wel_etal._2024} polygons. Utilizing paquo, we directly read, create, and modify these annotations and detections within a QuPath project in the Python environment. Images are then cropped using these polygons and processed by cell segmentation and classification models employing standard vision processing toolkits such as OpenCV, pyvips, and PyTorch. Additionally, QuPath employs Groovy scripts to initiate a Python process that starts the entire pipeline from QuPath graphical interface: fetching polygons, extracting images from them, and running deep learning model inference on the cropped images. 
The results are returned to QuPath, leveraging paquo's Python bindings to manipulate QuPath data while minimizing the computational overhead typically associated with cross-environment communication.

\counterwithin{figure}{subsection}
\renewcommand{\thefigure}{S\arabic{subsection}}

\begin{figure}[h!]
    \centering
    \includegraphics[width=\textwidth]{images/A7.pdf}
    \caption{QuPath integration workflow using Python environment}
    \label{fig:S7}
\end{figure}

Compared to traditional workflows that involve exporting annotations as GeoJSON, classifying them in Python, and reimporting them into QuPath, our approach offers several advantages. We eliminate the need to switch between programming languages, providing a cohesive and streamlined development process entirely within QuPath software and removing the necessity to use other tools. Meanwhile, we avoid storing annotations as intermediate JSON files unless required for external use or archiving. By conducting the entire inference and post-processing workflow within the Python environment, we leverage the power and flexibility of Python libraries for image processing and machine learning. This approach also enables adjustments to any set of labels and models, thereby improving its applicability.

%\hfill

The distilled model and QuPath integration code are packaged into a Docker container, enabling streamlined execution with the Docker engine. Detailed integration code and deployment instructions can be found in the GitHub repository \cite{Shvetsov_2025b}.

Despite these benefits, we acknowledge that the paquo library is a proof‑of‑concept project in its early development stage and has not been tested across all versions of QuPath.

\clearpage

\subsection{\label{chap:S8}Data and code availability statement}
All datasets, models, and code used in this study are publicly available and can be obtained from the repositories listed below. 
The PanNuke \cite{Gamper_Koohbanani_etal._2019} and MoNuSAC \cite{Verma_Kumar_etal._2021} datasets are publicly accessible, and download information along with detailed descriptions can be found in their respective articles. Preprocessing scripts for PanNuke and MoNuSAC data, as well as individual cell extraction scripts, are available on GitHub \cite{Shvetsov_2025a}. The H-Optimus foundation model used in our experiments can be downloaded from the HuggingFace repository \cite{hoptimus2024}, and model information is available on GitHub \cite{Saillard_Jenatton_etal._2024}. In addition, the integration code for QuPath and the distilled model packaged in a Docker container are provided in the repository \cite{Shvetsov_2025b}, and paquo Python library is available from the authors GitHub repository \cite{Bayer_AG}.
\clearpage

\end{document}
 

%% For numbered reference style
%% \bibitem{label}
%% Text of bibliographic item

%%\bibliographystyle{elsarticle-num} 
%%\bibliography{references}






%% The Appendices part is started with the command \appendix;
%% appendix sections are then done as normal sections

\end{document}

%% If you have bib database file and want bibtex to generate the
%% bibitems, please use
%%
%%  \bibliographystyle{elsarticle-num} 
%%  \bibliography{<your bibdatabase>}

%% else use the following coding to input the bibitems directly in the
%% TeX file.

%% Refer following link for more details about bibliography and citations.
%% https://en.wikibooks.org/wiki/LaTeX/Bibliography_Management



\endinput
%%
%% End of file `elsarticle-template-num.tex'.
