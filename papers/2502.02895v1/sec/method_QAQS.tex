% \subsection{Problem formulation (書かなくて良さそう)}
% 物体検知モデルの性能は通常、評価指標$E: \mathrm{GT}\times\mathrm{Pred}\to\R$の大きさに基づいて評価される。したがって、評価指標$E$を最大化するような後処理アルゴリズムを適用することができれば、検知モデルの性能を向上することができる。一方で評価指標$E$の計算には正解(GT)が必要なため、正解データにアクセスできないテスト時に評価指標を計算することはできない。そこで、予測(Pred)のみを入力とする、評価指標を代替するような目的関数を設計する。本研究ではQUBOとして代替関数をモデル化し、QUBOを最大化することでより適切な後処理を実現する。

\subsection{Construction of coefficient matrix}
\noindent\textbf{Quantum Appearance QUBO Suppression (QAQS)} \\
% \paragraph{Quantum Appearance QUBO Suppression (QAQS)}
We introduce an appearance feature to the non-diagonal elements of the QSQS coefficient matrix to distinguish redundant prediction from occluded objects. This modification is based on the hypothesis that redundant predictions share a similar image feature. The coefficient matrix $Q$ is defined as follows.
\begin{equation}
    Q = w_1L - (w_2 P_1 + w_3 P_2)\odot A,\label{eq:qaqs}
\end{equation}
where $L=\mathrm{diag}(\bm{v})$ is a diagonal matrix whose elements are confidence scores, $P_1$ and $P_2$ are the same as in QSQS, symmetric matrices whose elements are IoU and spatial feature, $A$ is a symmetric matrix of appearance feature, and $w_1\geq0, w_2\geq0, w_3\geq0$ are hyperparameters which satisfy $w_1+w_2+w_3=1$. $\odot$ denotes the element-wise product.

We choose SSIM as the appearance feature for the following reasons. First, the SSIM value is less than or equal to one for arbitrary image pairs. This feature is desirable to maintain the scale of $Q$ within a reasonable range. Second, SSIM is symmetric and equal to one only if the two images are identical. IoU and the spatial feature have the same characteristics. Finally, SSIM is efficiently computed on GPUs, as detailed in \cref{sec:ssim_impl}.
% The implementation details of SSIM computation are described \cref{sec:ssim_impl}.

\vskip.5\baselineskip\noindent\textbf{QAQS with Confidence-based weighting (QAQS-C)} \\
% \paragraph{QAQS with Confidence-based weighting (QAQS-C)}
In addition to introducing an appearance feature, we further incorporate the confidence score into the non-diagonal part of the coefficient matrix based on the hypothesis that (partially-) occluded objects have low confidence scores.
DPP~\cite{lee2016individual}, which integrates the confidence score to the similarity term, also motivates this modification. The final formulation is provided as follows.
\begin{equation}
    Q = w_1L - \bm{v}\left((w_2 P_1 + w_3 P_2)\odot A\right)\bm{v}^\top.\label{eq:qaqs_c}
\end{equation}
% where $L=\mathrm{diag}(\bm{s})$ is a diagonal matrix whose elements are confidence scores, $P_1$ and $P_2$ are symmetric matrices whose elements are IoU and spatial feature, $A$ is a symmetric matrix of appearance feature, and $w_1\geq0, w_2\geq0, w_3\geq0$ are hyperparameters which satisfy $w_1+w_2+w_3=1$.
Notations are the same as QAQS. Here, $\bm{v}\in\R^{n \times 1}$.
% \subsection{QAQS-3}
% QAQS-2の改善版. \Eqref{eq:decompose_qaqs_2}の$\sum_{j}\mathcal{S}_{i, j}L_{j}x_{j}$部分のスケールが予測ボックス数$d$に対して一定になるようにしたい、というモチベーション。
% \begin{align}
%     Q_{i, j} &= \left\{
%     \begin{array}{ll}
%         \mathcal{L}_{i, j} & i=j \\
%         -\dfrac{1}{d}\times\mathcal{S}_{i, j}\times L_{i, i}\times  L_{j,j} & i\neq j
%     \end{array} \right.,
% \end{align}
% where $L\in\R^{d\times d}$: confidence, $P_1, P_2$: spatial feature, $w_1+w_2+w_3=1$, and $A$: appearance feature.
% \subsection{C-NMS-A}
% Confluence-NMS \cite{shepley2023confluence} + Appearance feature.
% \paragraph{Confluence-NMS}
% Confluence-NMS (C-NMS) \cite{shepley2023confluence} is a variation of NMS.
% C-NMS use the normalized proximity $P((b_m,b_i)_\mathrm{norm})$ instead of IoU.

% \paragraph{Confluence-NMS with Appearance feature (C-NMS-A)}
% C-NMS-A use $P((b_m,b_i)_\mathrm{norm})\times (1 - A_{m, i})$ to suppress bounding boxes.