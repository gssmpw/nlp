\subsection{One reason for higher performance of the proposed formulations}
% 実験結果を見ると、mAPとmARの両方でQAQSの方がQSQSよりも高い値を取っている。
% これは、QSQSの方が物体の過検知だけでなく見逃しも多いということを意味している。
% 見逃しが多いということは、後処理によって過剰にboxを削除しているということである。
% したがって、QAQSの方がQSQSよりも、bboxを過剰に削除しづらい手法だと考えられる。
% 実際、次の計算によってQAQSの方がQSQSよりも非対角成分の絶対値が小さくなることがわかる。
% QSQS, QAQS, QAQS-2の非対角成分は、それぞれ$(w_2 P_1+w_3 P_2)_{ij},((w_2 P_1+w_3 P_2)\odot A)_{ij},((w_2 P_1+w_3 P_2)\odot A)_{ij}\bm{s}_i\bm{s}_j$で与えられる。SSIMの性質から、$A_ij\leq1$, $\bm{s}_i$はconfidence scoreのため$0\leq \bm{s}_i\leq1$である。したがって、Q行列の非対角成分の絶対値は、QSQS, QAQS, QAQS-2の順に小さくなる。以上の分析から、QAQSはQSQSよりも予測の重なりに対するペナルティが小さいため見逃しを起こしづらい後処理アルゴリズムだと考えられる。
The higher mAP and mAR of proposed methods than QSQS indicate that they are less likely to suppress true positives. We can mathematically explain the reason for this behavior.
The non-diagonal elements of the coefficient matrix represent the penalty among predictions. If the penalty among predictions is low, the predictions are less likely to be suppressed.
The absolute value of non-diagonal elements of the QAQS coefficient matrix is smaller than that of QSQS because SSIM is less than or equal to 1.
Similarly, the absolute value of non-diagonal elements of the QAQS-C coefficient matrix is smaller than that of QAQS because the confidence score is between 0 and 1.
This analysis indicates that QSQS imposes a more severe penalty than the penalties applied by QAQS and QAQS-C.
Therefore, the proposed formulations, QAQS and QAQS-C, are less likely to suppress true positives than QSQS.

% , i.e., $|((w_2 P_1+w_3 P_2)\odot A)_{ij}|\leq |(w_2 P_1+w_3 P_2)_{ij}|$,, i.e., $|(\bm{v}(w_2 P_1+w_3 P_2)\odot A\bm{v}^\top)_{ij}|\leq |((w_2 P_1+w_3 P_2)\odot A)_{ij}|$,
% まず、QUBOの目的関数を以下のように書き下す。
% \begin{equation}
%     \bm{x}^\top Q \bm{x} = \sum_{i}\left(\sum_{j}Q_{i, j}x_{j}\right)x_{i}
% \end{equation}
% この式変形から、今回解いている問題は、box i の重要度が$\sum_j Q_{i,j}x_j$で定義されており、重要度の和が最大となるように削除するボックスを決定する問題である、と解釈することができる。
% QSQS, QAQS, QAQS-2におけるbox iの重要度はそれぞれ次のように書き下すことができる。
% \begin{align}
%     \sum_{i}\left(1 - \frac{1}{L_{i}}\sum_{j}(w_2P_1 + w_3P_2)_{i,j}x_{j}\right)\mathcal{L}_{i}x_{i} & ~\mathrm{(QSQS)}\\
%     \sum_{i}\left(1 - \frac{1}{L_{i}}\sum_{j}\mathcal{S}_{i, j}x_{j}\right)\mathcal{L}_{i}x_{i} & ~\mathrm{(QAQS)}\\
%     \sum_{i}\left(1 - \sum_{j}\mathcal{S}_{i, j}L_{j}x_{j}\right)\mathcal{L}_{i}x_{i} & ~\mathrm{(QAQS-2)}
% \end{align}
% ここで、$0\leq L_j\leq 1, A_{i,j}\leq 1$であるから、$L, P, w$が同じだと仮定すると、$\dfrac{1}{L_{i}}(w_2P_1 + w_3P_2)_{i,j}\geq \dfrac{1}{L_{i}}\mathcal{S}_{i,j}\geq L_j\mathcal{S}_{i,j}$が成立する。これは、QSQSの方がQAQS, QAQS-2よりも類似したボックスに大きなペナルティを与えることを意味している。
% したがって、box iの重要度はQSQS, QAQS, QAQS-2の順に大きくなる。

\subsection{Guidelines for further acceleration}
% 係数行列のスパース性が高まると、SSIM計算と最適化計算に必要な計算コストが低下する可能性がある。
% SSIM計算に関しては、単純に計算すべきペア数が少なくなるため、演算回数を減らすことができる。
% 最適化計算に関しては、Q行列のスパース性を活用した高速なQUBOソルバー~\cite{Rehfeldt2023}が提案されている。
% 係数行列が疎であれば、この特徴を活用してより高速に最適解が求まる可能性が高い。
By improving the sparsity of the coefficient matrix, the runtime for SSIM and optimization calculation might be reduced. The high sparsity indicates that the number of pairs for SSIM computation is small. Thus, we can simply reduce the number of SSIM computations. Additionally, the QUBO may be efficiently solved by exploiting the sparsity of the coefficient matrix~\cite{Rehfeldt2023}.

% Q行列の疎性を向上させるため、次の変更を行った。まず、IoUをGeneralized IoU (GIoU)~\cite{RezatofighiTGS019GIoU}に, Spatial FeatureをProximity~\cite{shepley2023confluence}に変更する。
% 次に、IoUを用いる場合と値の範囲を揃えるため、GIoUが0未満になるようなペアに対してはGIoUを0に設定する。従来の定式化では、IoUが0になるようなペアに対してはspatial featureが0になる。従来法との整合性を保つため、GIoUが0になるようなペアに対してはProximityを0に設定する。
% 上記手順に従って更新されたQ行列を用いた場合、QAQSとQAQS-2は精度を保ったまま、QSQSは精度の低下とともに処理時間が半分以下になった。最適解に影響を与えずに係数行列の疎性を向上できるかどうかは試行錯誤によるところが大きいと予想されるが、この方法は一つの有望なヒューリスティックであると考えられる。
% 結果としてGurobiの求解速度+SSIMの計算速度が向上したものと思われる。予測スコアの閾値を0.25に設定しているので、他の予測とのIoUは低いが明らかに不要な予測がすでに削られていると考えられる。GIoUをclipしても問題なかったのはこれが原因じゃないかな?
To improve the sparsity of the coefficient matrix, we replace IoU with Generalized IoU (GIoU)~\cite{RezatofighiTGS019GIoU}. GIoU takes a smaller value than IoU by definition. Considering the consistency with the case of IoU, the pairwise score ($P_1$ and $P_2$) of the pairs with negative GIoU values is set to 0. With this modification, the runtime of QAQS and QAQS-C is almost halved, maintaining the performance. The runtime of QSQS is also halved with a slight degradation. It will depend largely on trial and error whether the sparsity of the coefficient matrix is improved without affecting the optimal solution, but this method would be a promising heuristic to achieve it.
See Supplementary C for more details.