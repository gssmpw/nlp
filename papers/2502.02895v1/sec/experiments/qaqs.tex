\subsection{Ablation study of coefficient matrix}
% 提案手法である、外観特徴量としてのSSIMの追加及び予測スコアを用いた非対角成分の重みづけの有効性を検証するため、ablation studyを行う。
We investigate the individual contribution of two features newly introduced to the $Q$ matrix through an ablation study.
% \Cref{tab:ablation_coco,tab:ablation_crowd}にそれぞれCOCOデータセット、CrowdHumanデータセットに対する実験結果を示す。
\Cref{tab:ablation_qubo} show the results for the COCO and CrowdHuman datasets.
% \Cref{tab:ablation_coco,tab:ablation_crowd}中のQSQS-Cは、QSQSの係数行列における非対角成分、すなわち$-(w_2P_1+w_3P_2)$を$-\bm{s}^\top(w_2 P_1+w_3 P_2)\bm{s}$に置き換えた定式化を表す。
QSQS-C denotes the formulation where the pairwise score of QSQS $(w_2P_1+w_3P_2)$ is replaced with $\bm{v}(w_2 P_1+w_3 P_2)\bm{v}^\top$.
% 実験結果から、外観特徴量の考慮と予測スコアを用いた非対角成分の重みづけのいずれか一方を適用した場合であっても、ベースラインのQSQSと比較して評価指標が改善することがわかる。
% 特に外観特徴量の考慮による精度向上が顕著である。
The results show that both features improve performance compared to baseline QSQS, and the appearance feature is mainly affected.
\setlength\tabcolsep{0.75mm} 
\begin{table}[tbh]
    \centering
    \small
    \begin{tabular}{l ccc ccc}
        \toprule
        & \multicolumn{3}{c}{COCO 2017}& \multicolumn{3}{c}{CrowdHuman}\\
         \cmidrule(lr){1-4}
         \cmidrule(lr){5-7}
         Method & Time$^*$ & mAP & mAR@100 & Time$^*$ & mAP & mAR@100 \\
         \cmidrule(lr){1-4}
         \cmidrule(lr){5-7}
         % QSQS   & 108 & 34.03 & 42.90 & 140 & 31.23 & 36.19 \\ % QSQS_Algorithm4\\ % amp1, QSQS_Algorithm_4の結果
         % QSQS-C & 108 & 34.50 & 44.33 & 133 & 33.10 & 39.19 \\ % QSQS_confidence2\\ % amp1, QSQS_confidence_1の結果
         % QAQS   & 122 & 35.25 & 45.50 & 163 & 35.62 & 42.57 \\ % QAQS_ssim_batch8\\ % amp1, 3の結果
         % QAQS-C & 121 & 35.35 & 45.90 & 155 & 35.77 & 42.91 \\ % 36\\ % amp1, 139の結果
         QSQS   & 21 & 34.03 & 42.90 & 32 & 31.23 & 36.19 \\ % QSQS_Algorithm4\\ % amp1, QSQS_Algorithm_4の結果
         QSQS-C & 21 & 34.50 & 44.33 & 30 & 33.10 & 39.19 \\ % QSQS_confidence2\\ % amp1, QSQS_confidence_1の結果
         QAQS   & 24 & 35.25 & 45.50 & 37 & 35.62 & 42.57 \\ % QAQS_ssim_batch8\\ % amp1, 3の結果
         QAQS-C & 24 & 35.35 & 45.90 & 35 & 35.77 & 42.91 \\ % 36\\ % amp1, 139の結果
        \bottomrule
        \multicolumn{7}{l}{\footnotesize{$^*$ Runtime for suppression. Milliseconds per image [ms/image]}}
    \end{tabular}
    \caption{Ablation study of QUBO-based suppression. }
    \label{tab:ablation_qubo}
\end{table}

% Solverの計算時間を含む後処理全体の実行時間は、外観特徴量の計算を含む分、提案手法の方がQSQSよりもやや長い。しかし、その差は画像一枚あたり2-6msであり、実行全体から見るとわずかなオーバーヘッドである。
% COCOデータセットを対象とした場合、本実験で使用したFaster R-CNNの後処理を含まない推論時間は約54msである。COCOデータセットではQUBOを用いた後処理が約24msである。したがって、QUBOを用いた後処理を含む全体のスループットは約12.8 fps相当である。
The total runtime of QAQS and QAQS-C is slightly longer than that of QSQS and QSQS-C due to the appearance feature calculation. However, the difference is 2-6~ms per image, which is a negligible overhead compared to the total runtime.
The frames-per-seconds (fps) is calculated as the reciprocal of the total detection runtime per image that equals the sum of forward and suppression time per image.
For the COCO dataset, the forward time for Faster R-CNN without suppression is around 54~ms per image, and the runtime of our QAQS-C is approximately 24~ms per image. 
Thus, the overall detection throughput is approximately $1000\times\frac{1}{54+24} \fallingdotseq 12.8$ fps.
% 計算時間の内訳を\cref{fig:breakdown_coco}に示す。本実験におけるボトルネックはQUBO solverである。
% QUBOを用いた後処理の計算時間のうち、このうち14msがソルバーの実行時間に相当する。もしソルバーの実行時間を1msに削減できれば、約18.2fpsまでスループットが改善する。このように、将来的にQUBOソルバーを高速な量子コンピュータ(古典コンピュータはアーキテクチャの制約から中小規模の問題を超高速 (micro secondのオーダー)に解くのがほぼ不可能なので、全然違うアーキテクチャを持つ量子コンピューターに期待している, NTTのLASOLVの論文をrefした方が良いかも)に置き換えることで大幅な高速化が期待される。
\Cref{fig:breakdown_coco} shows the runtime breakdown for QUBO-based suppressions. The current major bottleneck is the runtime of the QUBO solver, which spends approximately half of the runtime. If the solver runtime can be reduced from 14~ms to 1~ms by replacing the QUBO solver with a faster quantum computer in the future, the overall throughput will improve to approximately 18.2~fps.
10-20~fps is sufficiently fast for some industrial applications, such as surveillance inside the factory, because objects do not move fast in these situations.

\begin{figure}
    \centering
    \includegraphics[width=0.8\linewidth]{fig/breakdown_coco_v3.pdf}
    \caption{Breakdown of QUBO-based suppression runtime per image of COCO dataset shown in \cref{tab:ablation_qubo}. Results on the CrowdHuman dataset show a similar tendency.}
    \label{fig:breakdown_coco}
\end{figure}
% \begin{itemize}
%     \item Confidenceを追加することで、soft-scoringの時間が短縮された様子
%     \item 精度への影響は外観特徴量の追加によるところが大きそう
% \end{itemize}
% \textcolor{gray}{
% \paragraph{メモ}
% \begin{itemize}
%     \item YOLOとFaster R-CNNでは適切なパラメータが違う
%     \item どちらの場合もNMSと比較してmAPを1\~2\%改善できるので、あとは実際の応用で要求されるフレームレートに応じて使い分け? (最近のリアルタイムモデルは100 fps近いものもあるので多少後処理に時間がかかっても許容できる)
%     \item DETR系の新しめモデルと比較して小さな物体の検出が上手であればさらにアピれそう。
% \end{itemize}
% }
% \begin{table}[tbh]
%     \centering
%     \begin{tabular}{l ccc}
%         \toprule
%         \multicolumn{4}{c}{COCO 2017}\\
%          \midrule
%          Method & Time [s] & mAP & mAR@100 \\
%          \midrule
%          QSQS   & 108 & 34.03 & 42.90 \\ % amp1, QSQS_Algorithm_4の結果
%          QSQS-C & 108 & 34.50 & 44.33 \\ % amp1, QSQS_confidence_1の結果
%          QAQS   & 122 & 35.25 & 45.50 \\ % amp1, 3の結果
%          QAQS-C & 121 & 35.35 & 45.90 \\ % amp1, 139の結果
%         \bottomrule
%         \multicolumn{4}{c}{CrowdHuman}\\
%          \midrule
%          QSQS   &  140 & 31.23 & 36.19 \\ % QSQS_Algorithm4
%          QSQS-C &  133 & 33.10 & 39.19 \\ % QSQS_confidence2
%          QAQS   &  163 & 35.62 & 42.57 \\ % QAQS_ssim_batch8
%          QAQS-C &  155 & 35.77 & 42.91 \\ % 36
%         \bottomrule
%     \end{tabular}
%     \caption{Ablation study of QUBO formulation.}
%     \label{tab:ablation_qubo}
% \end{table}
% \begin{table}[tbh]
%     \centering
%     \caption{Ablation study on COCO 2017 dataset.}
%     \label{tab:ablation_coco}
%     \begin{tabular}{l ccc}
%         \toprule
%          Method & Time [s] & mAP & mAR@100 \\
%         \midrule
%          QSQS & 108 & 34.03 & 42.90 \\ % amp1, QSQS_Algorithm_4の結果
%          QSQS-C & 108 & 34.50 & 44.33 \\ % amp1, QSQS_confidence_1の結果
%          QAQS & 122 & 35.25 & 45.50 \\ % amp1, 3の結果
%          QAQS-C & 121 & 35.35 & 45.90 \\ % amp1, 139の結果
%         \bottomrule
%     \end{tabular}
% \end{table}
% \begin{table}[tbh]
%     \centering
%     \caption{COCO2017, Faster R-CNN. Optunaを用いたハイパラ探索後$(w_1, w_2, w_3) = (0.55, 0.3, 0.15)$の結果.}
%     \label{tab:tmp2}
%     \begin{tabular}{l cccc}
%         \toprule
%          Method & Time (s) & OC & mAP & mAR@100 \\
%         \midrule
%          QSQS & 127 & 32.59 & 34.91 & 44.55 \\
%          + Confidence & 106 & 35.40 & 35.19 & 45.59 \\
%          + Appearance (QAQS) & 120 & 35.40 & 35.34 & 45.82 \\
%          + Confidence + Appearance (QAQS-2) & 123 & 35.60 & 35.36 & 45.92 \\
%         \bottomrule
%     \end{tabular}
% \end{table}
% \setlength\tabcolsep{0.75mm} 
% \begin{table}[tbh]
%     \centering
%     \caption{Ablation study on CrowdHuman dataset.}
%     \label{tab:ablation_crowd}
%     \begin{tabular}{l cccc}
%         \toprule
%          Method & Time [s] & mAP & mAR@100 \\
%         \midrule
%          QSQS & 140 & 31.23 & 36.19 \\ % QSQS_Algorithm4
%          QSQS-C & 133 & 33.10 & 39.19 \\ % QSQS_confidence2
%          QAQS & 163 & 35.62 & 42.57 \\ % QAQS_ssim_batch8
%          QAQS-C & 155 & 35.77 & 42.91 \\ % 36
%         \bottomrule
%     \end{tabular}
% \end{table}
% \begin{table}[tbh]
%     \centering
%     \caption{CrowdHuman, Faster R-CNN. Optunaを用いたハイパラ探索後$(w_1, w_2, w_3) = (0.55, 0.3, 0.15)$の結果.}
%     \label{tab:tmp2}
%     \begin{tabular}{l cccc}
%         \toprule
%          Method & Time (s) & OC & mAP & mAR@100 \\
%         \midrule
%          QSQS & 164 & 22.21 & 34.59 & 40.57 \\
%          + Confidence & 144 & 26.65 & 35.55 & 42.46 \\
%          + Appearance (QAQS) & 174 & 27.38 & 35.71 & 42.86 \\
%          + Confidence + Appearance (QAQS-2) & 151 & 27.60 & 35.77 & 42.91 \\
%         \bottomrule
%     \end{tabular}
% \end{table}

% \paragraph{まとめ}
% \begin{itemize}
%     \item Confidence (対角成分)よりもbbox類似度 (非対角成分)に大きく重み付けした方が検知精度向上に繋がる
%     \item 空間特徴量よりもIoU特徴量を大きく重み付けした方が高い検知精度を得られる可能性が高い
%     \item 非対角成分にConfidenceを掛けることで対角成分の重要度が下がる
% \end{itemize}
% \begin{figure}
%   \begin{minipage}[b]{0.48\columnwidth}
%     \centering
%     \includegraphics[width=\columnwidth]{fig/qsqs_qaqs/_QSQS_Algorithm_mAP_contour_plot.pdf}
%     \caption{COCO, YOLOv8l, QSQS}
%   \end{minipage}
%   \begin{minipage}[b]{0.48\columnwidth}
%     \centering
%     \includegraphics[width=\columnwidth]{fig/qsqs_qaqs/QAQS_ssim_batch_mAP_contour_plot.pdf}
%     \caption{COCO, YOLOv8l, QAQS}
%   \end{minipage}\\
%   \begin{minipage}[b]{0.48\columnwidth}
%     \centering
%     \includegraphics[width=\columnwidth]{fig/qsqs_qaqs/QSQS_Algorithm_mAP_contour_plot.pdf}
%     \caption{COCO, YOLOv8l, QSQS-2}
%   \end{minipage}
%   \begin{minipage}[b]{0.48\columnwidth}
%     \centering
%     \includegraphics[width=\columnwidth]{fig/qsqs_qaqs/QAQS_ssim_batch_confluence_mAP_contour_plot.pdf}
%     \caption{COCO, YOLOv8l, QAQS-2}
%   \end{minipage}
% \end{figure}