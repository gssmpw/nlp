\subsection{Quadratic Unconstrained Binary Optimization}
% Quadratic Unconstrained Binary Optimization (QUBO) とは以下の形式で定式化される最適化問題である。
Quadratic Unconstrained Binary Optimization (QUBO) is a combinatorial optimization problem formulated as follows.
\begin{equation}
    \max_{\bm{x}\in\{0, 1\}^n}\bm{x}^\top Q\bm{x} = \sum_{i=1}^{n}\sum_{j=1}^{n}Q_{ij}\bm{x}_{i}\bm{x}_{j}+\sum_{i=1}^{n} Q_{ii}\bm{x}_i,
\end{equation}
where $Q\in\R^{n\times n}$.
% 右辺の式変形は、$\bm{x}_i\in\{0, 1\}$であるから、$\bm{x}_{i}^2=\bm{x}_{i}$となることを利用した。
This equation arise from the fact that $\bm{x}_{i}^2=\bm{x}_{i}$ holds when $\bm{x}_i\in\{0, 1\}$.
% この形式の最適化問題はイジングモデルと等価であり、量子アニーリングやQAOAなどの量子アルゴリズムを用いて効率的に解けることが知られている。
QUBO is equivalent to the Ising model, which is efficiently optimized via quantum algorithms such as quantum annealing~\cite{PhysRevE.58.5355Nishimori,morita2008mathematical} and Quantum Approximate Optimization Algorithm (QAOA)~\cite{farhi2014quantum}.
% QUBOは物体検知モデルの後処理に応用されており、crowded and non-crowdedなデータセットでNMSに対する優位性が確かめられている~\cite{rujikietgumjorn2013qubo,li2020qsqs}。
QUBO was originally used for the suppression in pedestrian detection~\cite{rujikietgumjorn2013qubo} and extended to generic object detection later~\cite{li2020qsqs}. These studies demonstrated the superiority of QUBO-based suppression over NMS for crowded and non-crowded datasets.
% 本研究は、外観特徴量の考慮及び予測スコアに基づく重みづけを行うことでocclusionに起因する予測boxのoverlapと予測の冗長性に起因するoverlapを区別し、crowded and non-crowdedなデータセットにおけるQUBOベースの後処理手法を用いたaccuracyをさらに改善する。
We aim to propose enhanced QUBO formulations that distinguish whether the overlapped prediction is due to occluded objects or redundant prediction.
Our formulations significantly improve QUBO-based suppression accuracy for crowded and non-crowded datasets.

\vskip.5\baselineskip\noindent\textbf{QUBO framework.} 
% \paragraph{QUBO framework.}
% QUBO framework (QF)~\cite{rujikietgumjorn2013qubo}は物体検知、特に歩行者検出における後処理をQUBOとして定式化した初めての手法である。QFは歩行者検出における、false positiveの削減を目的として提案された。occlusionが発生しやすい、歩行者検出においてNMSよりも高い精度を示したが、non-crowdedなデータセットに対しては有効でないと結論づけている。QFにおける係数行列は以下のように定義される。
QUBO Framework (QF)~\cite{rujikietgumjorn2013qubo} first formulate the suppression of pedestrian detection as QUBO. QF aims to reduce false positives in pedestrian detection. Although QF performed better than NMS for pedestrian detection, where occlusion is likely to occur, the authors concluded that QF is not better than NMS for non-crowded scenes. The coefficient matrix of QF is defined as follows.
\begin{equation}
    Q = w_1 L - w_2 P_1,
\end{equation}
where $L=\mathrm{diag}(\bm{v})$ is a diagonal matrix whose elements are confidence scores, $P_1$ is a symmetric matrix of pairwise scores such as Intersection over Union (IoU), and $w_1\geq0, w_2\geq0$ are hyperparameters which satisfy $w_1+w_2=1$.

\vskip.5\baselineskip\noindent\textbf{Quantum-Soft QUBO Suppression.} 
% \paragraph{Quantum-Soft QUBO Suppression}
% Quantum-Soft QUBO Suppression (QSQS)~\cite{li2020qsqs}は空間特徴量を考慮した係数行列とSoft-NMSと類似したスコアリングを含むquantum-classical hybridアルゴリズムによって構成される。
% QSQSは歩行者検出における性能改善に加え、PASCAL VOC 2007やCOCOなどのより複雑なデータセットに対しても有望な結果を示した。
% QSQSはまずQUBOを解いたあと、QUBOによって棄却されたboxに対してSoftNMSと等価なsoft-scoringを行い、閾値以上のスコアを持つ検知ボックス数を追加する。QSQSにおける係数行列は以下のように定義される。
Quantum-Soft QUBO Suppression (QSQS)~\cite{li2020qsqs} is a quantum-classical hybrid algorithm that consists of QUBO with a modified coefficient matrix considering IoU and spatial feature~\cite{lee2016individual} and soft-scoring similar to Soft-NMS~\cite{bodla2017snms}.
QSQS performed better than conventional NMS for generic object detection such as PASCAL VOC 2007~\cite{Everingham15pascal} and COCO in addition to pedestrian detection.
QSQS solves QUBO first, then executes soft-scoring for predictions suppressed through QUBO to retain predictions whose score is higher than the pre-defined threshold.
The coefficient matrix for QSQS is defined as follows.
\begin{equation}
    Q = w_1L - (w_2 P_1 + w_3 P_2),
\end{equation}
where $L=\mathrm{diag}(\bm{v})$ is a diagonal matrix whose elements are confidence scores, $P_1$ and $P_2$ are symmetric matrices whose elements are IoU and spatial feature~\cite{lee2016individual,li2020qsqs}, and $w_1\geq0, w_2\geq0, w_3\geq0$ are hyperparameters which satisfy $w_1+w_2+w_3=1$.

\subsection{Structural similarity}
% Structural similarity (SSIM)~\cite{wang2004ssim,hore2010psnr_ssim} とは、画像パッチの輝度、コントラスト、構造の類似度に基づいて計算される指標であり、画像品質評価などに用いられる指標である。
% 画像の局所領域$\bm{\mathrm{x}}, \bm{\mathrm{y}}$に対するSSIMは以下の式にしたがって計算される。
Structural SIMilarity (SSIM)~\cite{wang2004ssim,hore2010psnr_ssim} is based on the similarity of the brightness, contrast, and structure of image patches and is used for image quality evaluation.
SSIM between corresponding image patches ($\bm{\mathrm{x}}$ and $\bm{\mathrm{y}}$) of two images ($X$ and $Y$) is defined as follows.
\begin{align}
    & SSIM(\bm{\mathrm{x}}, \bm{\mathrm{y}}) \\
    & = l(\bm{\mathrm{x}}, \bm{\mathrm{y}})\cdot c(\bm{\mathrm{x}}, \bm{\mathrm{y}}) \cdot s(\bm{\mathrm{x}}, \bm{\mathrm{y}}) \\
    & = \left(\dfrac{2\mu_{\bm{\mathrm{x}}}\mu_{\bm{\mathrm{y}}}+C_1}{\mu_{\bm{\mathrm{x}}}^2+\mu_{\bm{\mathrm{y}}}^2+C_1}\right)\cdot \left(\dfrac{2\sigma_{\bm{\mathrm{x}}}\sigma_{\bm{\mathrm{y}}} + C_2}{\sigma_{\bm{\mathrm{x}}}^2+\sigma_{\bm{\mathrm{y}}}^2+C_2}\right) \cdot \left(\dfrac{2\sigma_{\bm{\mathrm{x}}\bm{\mathrm{y}}}+C_3}{\sigma_{\bm{\mathrm{x}}}\sigma_{\bm{\mathrm{y}}}+C_3}\right),
\end{align}
where $C_1=0.01^2, C_2=0.03^2, C_3=2C_2$ are parameters to mitigate tiny denominator, and $\mu_{\bm{\mathrm{x}}}$ and $\sigma_{\bm{\mathrm{x}}}$ represents the mean and standard deviation over image patch $\bm{\mathrm{x}}$, respectively.
% ここで、$\mu, \sigma$は画素値の平均と分散を意味する。
% 画像全体に対するSSIMは、画像パッチ集合$W$に対するSSIMの平均$\frac{1}{|W|}\sum_{w\in W}SSIM(\bm{\mathrm{x}}_w, \bm{\mathrm{y}}_w)$として計算される。本論文ではこの値をSSIMと呼ぶ。
% SSIMは以下の特徴を持つ。
SSIM between two images $X$ and $Y$ is defined as the average of SSIM for image patches over a set of image patches $W$, i.e., $\frac{1}{|W|}\sum_{w\in W}SSIM(\bm{\mathrm{x}}_w, \bm{\mathrm{y}}_w)$. We call this value as SSIM.
SSIM has the following characteristics.
\begin{enumerate}
    \item Symmetric: $SSIM(\bm{\mathrm{x}}, \bm{\mathrm{y}})=SSIM(\bm{\mathrm{y}}, \bm{\mathrm{x}})$.
    \item Upper bounded: $SSIM(\bm{\mathrm{x}}, \bm{\mathrm{y}})\leq 1$.
    \item $SSIM(\bm{\mathrm{x}}, \bm{\mathrm{y}})=1$ if and only if $\bm{\mathrm{x}} = \bm{\mathrm{y}}$.
\end{enumerate}
% これらの特性は、QUBOの係数行列を構成するその他の項も備えている性質であり、係数行列の値のスケールを一定の範囲に収めやすいという点でも望ましい。
% 加えて、セクション3.3で述べるように効率的に計算することができる。これらの理由から、本研究では外観特徴量としてSSIMを用いる。