\begin{abstract}
    % abstractの概要
    % occlusionがあるとtrue positiveを見逃しやすいというNMSの弱点に対処する、QUBO-based suppressionが知られている。
    % 量子コンピュータはQUBOに対する高品質な解を高速に発見できるので、物体検知のsuppressionは量子コンピュータの有望な応用先である。
    % QUBOベースの後処理はNMSよりもoccluded objectを見逃しづらいが、スコアと予測の空間的重複しか考慮していないという点で改善の余地がある
    % この研究は、予測の空間的重複が、occlusionに起因するのか予測の冗長性に起因するのかを見分けることを目的とした、新しいQUBO定式化を提案する。
    % 冗長な予測は類似した画像特徴を持つという仮説に基づき、画像類似度指標を非対角成分に反映させる
    % 重なった物体の信頼度スコアは低いという仮説に基づき、非対角成分を信頼度スコアで重みづける
    % 提案手法は最先端のQUBO-based suppressionに対して、目立った計算時間の増加なく、大幅な改善を示す。
    Quadratic Unconstrained Binary Optimization (QUBO)-based suppression in object detection is known to have superiority to conventional Non-Maximum Suppression (NMS), especially for crowded scenes where NMS possibly suppresses the (partially-) occluded true positives with low confidence scores.
    % Quantum computers are expected to find high-quality solutions for QUBO in short runtime. Thus, suppression in object detection is a promising application of quantum computers.
    Whereas existing QUBO formulations are less likely to miss occluded objects than NMS, there is room for improvement because existing QUBO formulations naively consider confidence scores and pairwise scores based on spatial overlap between predictions.
    This study proposes new QUBO formulations that aim to distinguish whether the overlap between predictions is due to the occlusion of objects or due to redundancy in prediction, i.e., multiple predictions for a single object.
    The proposed QUBO formulation integrates two features into the pairwise score of the existing QUBO formulation: i) the appearance feature calculated by the image similarity metric and ii) the product of confidence scores.
    These features are derived from the hypothesis that redundant predictions share a similar appearance feature and (partially-) occluded objects have low confidence scores, respectively.
    % The proposed QUBO formulation reflects the image similarity metric to pairwise scores based on the hypothesis that redundant predictions share a similar image feature.
    % Additionally, the proposed QUBO formulation adopts pairwise scores weighted by confidence scores based on the hypothesis that (partially-) occluded objects have low confidence scores.
    The proposed methods demonstrate significant advancement over state-of-the-art QUBO-based suppression without a notable increase in runtime, achieving up to 4.54 points improvement in mAP and 9.89 points gain in mAR.
\end{abstract}