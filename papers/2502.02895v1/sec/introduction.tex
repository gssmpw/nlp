% 古典コンピューターに対する超越性~\cite{Arute2019}が期待されている量子コンピューターは実用化に向けた取り組みが活発化しており、物理量子ビット数が着々と増加傾向にある。
% 現在一般にアクセス可能な量子コンピューターは論理ビット数が限られているが、実用的な量子コンピューターの登場に先駆けて、量子コンピュータの産業応用可能性を探索することは非常に重要である。
% 量子最適化アルゴリズムの応用先の一つに、深層学習モデルを用いた物体検出のsuppressionがある。
% 画像に含まれる物体の種類と位置を予測する物体検出は、監視カメラや自動運転、ロボットのための歩行者検出~\cite{Dominguez-Sanchez18,GeigerLU12,MeiHPT15}や顔認識~\cite{YangLLT16}を含む、様々な産業応用をもつ~\cite{Kaur2022}。
Quantum computers, which are expected to have quantum supremacy~\cite{Arute2019} over classical computers, have attracted significant attention, and the number of physical qubits has been steadily increasing.
Although the number of logical qubits in publicly available quantum computers is currently limited, exploring its potential for industrial applications in advance is extremely important.
One application of quantum computers is the suppression of object detection using deep learning models.
Object detection, which predicts the category and location of objects in images, has various industrial applications, including facial recognition~\cite{YangLLT16} and pedestrian detection for surveillance~\cite{Dominguez-Sanchez18}, automatic driving~\cite{GeigerLU12}, and robots~\cite{MeiHPT15}.

% 深層学習を用いた物体検出では、Convolutional Neural Network ~\cite{SimonyanZ14aVGG,HeZRS16resnet}やVision Transformer~\cite{DosovitskiyB0WZ21ViT,LiuL00W0LG21Swin}に類するアーキテクチャを用いて抽出した画像特徴を用いて物体の種類と位置を予測する。
% 多くの物体検出モデルは、事前定義された領域 (anchor-base)、もしくは特徴マップの各点 (anchor-free)につき一つの出力を出力されるように学習される。
% したがって、その出力には冗長な検出が含まれやすい。冗長な検出を抑制する方法として、古くからNon-Maximum Suppression (NMS)が用いられてきた。
% 一方で、NMSに基づくgreedyな抑制は部分的に隠れた物体の見逃しに繋がりやすいという課題が指摘されている。
% この課題に対処するために、NMSの変種~\cite{bodla2017snms,He2019SofterNMS,Liu2019AdaNMS,Nils2020vgnms,Huang2020VFGNMS,shepley2023confluence}やクラスタリングに基づく手法~\cite{ShenJXLK22CPCluster}、Quadratic Unconstrained Binary Optimization (QUBO)に基づく手法~\cite{rujikietgumjorn2013qubo,li2020qsqs}などが提案されている。
% QUBOは組合せ最適化問題の一種で、量子最適化アルゴリズムを適用することで高速に良質の近似解を求めることが期待される。
% 本研究はQUBOを用いた後処理手法をさらに洗練および改善し、量子コンピューターを比較的容易に統合できるquantum-readyな手法の開発を目的としている。
Deep learning-based object detectors predict the category and location of objects from image features extracted by neural networks such as convolutional neural networks~\cite{SimonyanZ14aVGG,HeZRS16resnet} and vision transformers~\cite{DosovitskiyB0WZ21ViT,LiuL00W0LG21Swin}.
Most detectors are trained to detect objects from each predefined region (anchor-based) or each point in the feature map (anchor-free).
Therefore, the detectors likely output redundant detections, i.e., multiple predictions for a single object. Non-Maximum Suppression (NMS) has been used to suppress such redundant detections for a long time.

However, NMS-based greedy suppression is known to miss the partially occluded objects.
To address this issue, NMS variants~\cite{bodla2017snms,He2019SofterNMS,Liu2019AdaNMS,Nils2020vgnms,Huang2020VFGNMS,shepley2023confluence}, clustering-based suppression~\cite{ShenJXLK22CPCluster}, and Quadratic Unconstrained Binary Optimization (QUBO)-based suppression~\cite{rujikietgumjorn2013qubo,li2020qsqs} have been studied.
QUBO is a type of combinatorial optimization problem where quantum optimization algorithms are expected to find high-quality approximate solutions quickly.
We focus on refining the QUBO-based suppression to develop a quantum-ready method that can be integrated with a quantum computer.

% QUBOベースの後処理は、元来混雑したデータに対する効果的な後処理手法として提案された~\cite{rujikietgumjorn2013qubo}。
% \citet{li2020qsqs}らはQUBOを用いた後処理とSoftNMSと類似したスコアリングを組み合わせた古典量子のハイブリッドアルゴリズムである、Quantum-soft QUBO Suppression (QSQS)を提案した。QSQSは混雑したデータセットだけでなく、一般的なデータセットに対してもNMSに対する優位性を示した。
% 私たちは、混雑したデータセットで重要なのは、部分的に隠れた物体に対する予測と単に冗長な予測を区別することであるという仮説に基づき、QUBOの係数行列の構成方法を工夫することでより適切なQUBO定式化を提案する。
QUBO-based suppression was originally proposed as an effective suppression for pedestrian detection where the objects are likely to be occluded~\cite{rujikietgumjorn2013qubo}.
\citet{li2020qsqs} proposed Quantum-soft QUBO Suppression (QSQS), a hybrid classical-quantum algorithm that combines QUBO-based suppression with soft-scoring similar to Soft-NMS~\cite{bodla2017snms}. % QSQS showed superiority over NMS not only for crowded scenes but also for general non-crowded scenes. QSQS is the state-of-the-art (SOTA) QUBO-based suppression.
QSQS is the state-of-the-art (SOTA) QUBO-based suppression that has demonstrated superiority over NMS not only in crowded scenes but also in general non-crowded scenes.
Despite this great advancement, there is room for improvement because the QUBO formulation of QSQS considers only the confidence score and pairwise score based on the spatial overlap between predictions.

% 本論文の提案は、部分的に隠れた物体に対する予測と単に冗長な予測の違いが予測box内の画像情報と予測boxの信頼度スコアに反映されるという仮説に基づく。私たちは、予想boxの画像情報をStructural Similarity (SSIM)~\cite{wang2004ssim}, 信頼度スコアを非対角成分への重みづけとしてQUBOの係数行列に反映させた、新しいQUBOベースの後処理定式化を提案する。SSIMは主に画像品質評価のために用いられる指標であり、対称性、値が上から1で抑えられる、値が1となるなら二つの入力は真に等しいという特性がある。これらの特性により、係数行列を対称にできる、他の項とのスケール調整が容易である、などの利点がもたらされる。一方、通常のSSIMの計算方法では計算時間もしくはメモリ消費量の観点で難がある。そこで私たちは、分割統治法~\cite{doi:10.1137/S0895479892241287,Dwyer1987}とGPUを用いた並列計算を活用することでSSIMの計算方法を工夫する。本工夫により、約1GiBのGPUメモリの使用で、もともと2.0sec/image, 10.6sec/imageであったSSIM計算を、6ms/image, 14ms/imageで処理できるようになった。
In this study, we propose enhanced QUBO formulations that distinguish predictions for partially occluded objects from purely redundant predictions.
The proposed QUBO formulations integrate two features into the QUBO formulation of QSQS: i) the appearance feature that reflects image similarity and ii) the product of confidence scores between prediction pairs.
This integration is based on the hypothesis that the visual appearance and confidence score reflect the difference between predictions for partially occluded objects and redundant predictions.
We use Structural SIMilarity (SSIM)~\cite{wang2004ssim} as the appearance feature because SSIM has beneficial characteristics to construct the QUBO coefficient matrix.
To reduce the computation cost of SSIM, we utilize the divide-and-conquer algorithm~\cite{doi:10.1137/S0895479892241287,Dwyer1987} and GPU parallelization. The proposed SSIM implementation reduces the runtime from 2 and 11~s/image (seconds per image) to 6 and 14~ms/image (milliseconds per image) for non-crowded and crowded scenes, respectively, under approximately 1~GB GPU memory usage.

% オクルージョンが発生しやすい混雑したシーンを含むデータセットであるCrowdHumanデータセット~\cite{shao2018crowdhuman}と一般的な物体検出データセットであるCOCOデータセット~\cite{lin2015microsoft}を用いた実験では、提案手法はQUBOベースの最先端後処理手法であるQSQSを一貫して上回る結果を得た。
% 提案手法は特に混雑したシーンで有効性が高い。
% 本実験ではQUBO定式化の優位性を公平に比較するため、QUBOを用いた後処理の結果を直接比較する。
% 現時点で公開されている量子ゲートコンピューターは、有効な論理量子ビット数が数ビット程度であり、主要な実験に使用するのは難しい状況である。
% それゆえ、本実験は分枝限定法に基づく数理計画ソルバーであるGurobi~\cite{gurobi}を用いて古典コンピューター上で行った。
% 一方で私たちのソフトウェアは量子実機に簡単に対応できるように設計されている。
% その意味で提案手法はquantum-readyである。
% 本論文の貢献を改めてまとめる。
    % \item 予測ボックスの位置情報に加えて画像情報を考慮し、予測スコアを反映した係数行列を構成するという工夫に基づき、QUBO-based suppressionに対する新しい定式化を与えた
    % \item 画像情報を定量化するための指標であるSSIMの計算方法を工夫し、元々-10 sec / image程度だった計算時間を、-14 ms / image まで高速化した
    % \item solver部分を量子回路に簡単に置き換えることができる、ソフトウェアを作成し、公開する
    % \item QUBOベースのstate-of-the-artな後処理アルゴリズムであるQSQSと比較してmAPを最大4.54 point, mARを最大9.89 point改善した
The proposed methods consistently outperform QSQS, the SOTA QUBO-based suppression, on CrowdHuman~\cite{shao2018crowdhuman} dataset for crowded and on COCO~\cite{lin2015microsoft} dataset for non-crowded scenes.
The proposed methods especially show superior performance on the crowded dataset.
To fairly compare the advantages among QUBO formulations, we directly compare the results only with the QUBO solution, not including the results of soft-scoring. 
Our experiments use Gurobi Optimizer (Gurobi)~\cite{gurobi}, a branch-and-bound-based classical solver, instead of quantum solvers because the currently available quantum gate computers have only a few valid logical qubits.
However, our software is designed to integrate quantum computers, i.e., quantum-ready. See Supplementary A for usage of our software.
We summarize our contributions below.
\begin{enumerate}
    \item We propose new formulations for QUBO-based suppression that integrate the appearance feature and product of confidence scores between predictions.
    \item We propose a faster implementation of SSIM, which is used as the appearance feature, and reduce the runtime from up to 11~s/image to up to 14~ms/image.
    \item We publish a quantum-ready software that can be integrated with quantum computers after paper acceptance.%\footnote{Our software will be public when this paper is accepted.}.
    \item The proposed formulations outperform QSQS, the SOTA QUBO-based suppression, achieving up to 4.54 points improvement in mAP and 9.89 points gain in mAR without a notable increase in runtime.
\end{enumerate}



% \paragraph{強調したいこと}
% \begin{itemize}
%     \item 産業応用が盛んなタスクにおいて、量子コンピューターが適用可能な手法を提案したこと
%     \item 既存手法と比べて、大幅な計算ボトルネックなしに、大幅な性能改善を達成したこと
%     \item アルゴリズム全体のボトルネックは数理最適化ソルバーであり、これは量子計算機で代替できること
% \end{itemize}
% \paragraph{言われたくないこと}
% \begin{itemize}
%     \item NMS等古典的な手法と比較しないのか
%     \item 最先端じゃないのに何の意味があるのか
% \end{itemize}


% \paragraph{メモ書き}
% \begin{itemize}
%     \item 物体検知と量子最適化は重要なトピック
%     \begin{itemize}
%         \item 物体検知は産業アプリケーションをたくさん持っている
%         \item 量子コンピュータの開発が進んでおり、実用的な応用先の模索が重要
%         \item 本研究の目的は、量子コンピューターreadyなアルゴリズムを洗練・改善すること
%     \end{itemize}
%     \item 物体検知モデルには後処理が必要である
%     \begin{itemize}
%         \item Faster R-CNNやFCOSなどのDense-to-Sparseなモデルは予測数が極めて多い (anchor-base, anchor-free)
%         \item Sparse R-CNNなどのアンカーボックス自体を学習する手法はSparse-to-Sparseであるが、学習するボックスの候補数が多いほど性能が高い
%         \item 総じて、新しい手法であっても冗長な検出を含む可能性が高く、従って後処理が必要である
%     \end{itemize}
%     \item 既存の後処理手法は検出結果を過剰に抑制してしまうため、物体同士が重なりやすいケースに対して弱い
%     \begin{itemize}
%         \item NMS: IoU閾値で抑制する
%         \item Soft-NMS: IoUで予測スコアを重みづけ. IoUの影響は指数的に反映されることが多い (Gaussian)
%     \end{itemize}
%     \item QUBOベースの後処理は混雑したデータにおいて、NMSベース手法の代替となりうる有力な選択肢
%     \begin{itemize}
%         \item QSQS: QUBOベースの後処理手法。NMSよりも過剰に抑制しづらいとされている
%         \item 残す予測全体を考慮して後処理できる
%         \item QUBOは量子コンピューターで解くことができる
%         \item QUBOベースの後処理は無制約問題なので、実行可能性を気にしなくて良い→量子コンピューターに適している(?)
%     \end{itemize}
%     \item 提案アルゴリズムの基本思想
%     \begin{itemize}
%         \item 予測の冗長性に基づく予測ボックス同士の重なりと、オクルージョンに起因する予測ボックス同士の重なりを区別する
%         \begin{itemize}
%             \item 外観特徴量の導入: 予測ボックス内の画像情報の違いに基づいて、冗長性に起因する予測の重なりとオクルージョンに起因する予測の重なりを区別できるのではないか
%             \item 予測スコアを考慮: オクルージョンを含む予測の方が予測スコアが低いだろう。したがって予測に対するペナルティは予測スコアを反映する形で重みづけすべきではないか
%         \end{itemize}
%         \item なるべく短い計算時間で物体検出が終わる方が望ましいので、外観特徴量を高速に計算するための工夫を行った
%     \end{itemize}
%     \item (実用性を意識した工夫)
%     \begin{itemize}
%         \item SSIM計算の高速化、省メモリ化
%         \item モジュール化されたソフトウェアの公開
%     \end{itemize}
%     \item 実験結果に基づく提案手法の有用性
%     \begin{itemize}
%         \item 既存のQUBOベース手法よりも極めて優れた性能
%     \end{itemize}
%     \item 課題と制限
%     \begin{itemize}
%         \item 計算時間 (主に求解時間)→量子コンピュータが解決してくれる (してほしい)
%         \item Q行列作成コストは予測数の2乗のオーダー (確かNMSも同じ)→並列化できるだろう
%         \item 適切なハイパラはモデルやデータセットの特性によって異なる (モデルの特性を考慮したハイパラ or Q行列の設定, future work?)
        
%     \end{itemize}
% \end{itemize}