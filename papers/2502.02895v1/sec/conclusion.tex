% 本論文は、occlusionに起因する予測のoverlapと予測の冗長性に起因するoverlapを区別することが物体検知性能の向上に寄与すると仮説のもと、QUBOベースの後処理アルゴリズムの新しい定式化を提案する。
% 予測box同士の画像類似度指標SSIMと予測boxのスコアを係数行列の非対角成分に反映することで、重複した予測に対するペナルティを適切に重みづける。
% 混雑した状況を含むと想定される、CrowdHumanデータセット、一般的な物体検知データセットであるCOCOを用いた実験を通して、提案手法はstate-of-the-artなQUBOベースの後処理手法であるQSQSと比較してmAPを最大4.54 point, mARを最大9.89 point改善した。
% 特にocclusionが発生しやすいと考えられる、CrowdHumanデータセットで顕著な性能改善を確認した。この実験結果は、提案した工夫がoccludedな場合により有効であることを示唆しており、私たちの仮説が正しいことをsupportする状況証拠である。
% 本論文の実験は、古典ソルバーを用いて行われているが、私たちが開発したソフトウェアは量子回路シミュレーター及びQPUに対応している。
% その意味で提案手法はquantum-readyであり、実用に耐える規模の論理量子ビットを持つ量子ゲートコンピュータが公開されれば、それらを容易に適用することができる。
We propose new formulations of QUBO-based suppression based on the hypothesis that distinguishing predictions for occluded objects from redundant predictions enhances object detection performance.
Our formulations integrate the SSIM and the product of confidence scores between predictions into the coefficient matrix to differentiate occlusion from redundancy.
The proposed methods outperform the SOTA QUBO-based suppression, achieving up to 4.54 points improvement in mAP and up to 9.89 points gain in mAR for crowded and non-crowded datasets.
The proposed methods show a significant performance improvement for the crowded dataset. These results empirically validate our hypothesis.
The experiments are conducted with a classical solver, but our software is compatible with quantum solvers. Our quantum-ready methods will be more effective and beneficial with the advent of practical quantum computers.
