\section{Related work}
\label{sec:related_work}
In this section, the literature on event annotation, data formatting, and predictive modeling for soccer is reviewed, with attention to recent advancements and existing challenges. Section \ref{ssec:event_annotation_in_soccer} examines methods for soccer event annotation, including semi-automatic, automatic, and deep-learning approaches. Section \ref{ssec:event_data_format_in_soccer} addresses the diversity of soccer event data formats across providers. Finally, Sections \ref{ssec:predictive_modeling_for_soccer_event} and \ref{ssec:related_rl_soccer} review predictive modeling and RL approaches in soccer.



\subsection{Event annotation in soccer}
\label{ssec:event_annotation_in_soccer}

In soccer, analyzing player interactions and game events is essential for strategic insights. However, the sport’s complexity and fast pace make accurate event annotation challenging. Reliable and detailed event data, including event types and spatiotemporal information, is crucial for coaches, analysts, and players. Over the years, various methods have been developed to automate the detection and annotation of soccer events from broadcast videos.

Initial approaches relied on semi-automatic methods, which include clustering techniques \citep{ballan2010semantic} and rule-based algorithms \citep{stein2016director}, often enhanced with external text sources \citep{alan2008ontological}. These approaches required extensive domain knowledge, as rule accuracy and cluster reliability are sensitive to factors like camera angles and filming conditions. Subsequently, automatic methods were introduced to improve efficiency. These leveraged audiovisual features and fuzzy rule-based reasoning systems \citep{hosseini2013fuzzy}, visual cues paired with finite state machines \citep{assfalg2003semantic}, and synchronized match reports and video analysis \citep{wang2016soccer}.

The rapid advancement of deep learning has further expanded the capabilities of automatic event detection in soccer. Recent approaches include CNN-based methods \citep{theagarajan2020automated,cioppa2020context}, combinations of inertial sensor data with Random Forest classifiers \citep{barra2023footapp}, self-attention mechanisms \citep{ma2020event}, and hybrid CNN-RNN architectures \citep{sorano2021automatic}. Since 2021, the SoccerNet Challenge \citep{cioppa2024soccernet2024challengesresults} has provided a competitive platform for testing and improving these techniques. Additionally, simulated data has been explored to enhance event detection \citep{morra2020soccer}.

Despite these advancements, both semi-automatic and automatic methods face limitations. Semi-automatic methods require deep domain expertise in both computer science and soccer to ensure the accuracy of rules and clusters, and their performance can be highly affected by variations in video capturing. On the other hand, automatic methods struggle with precision, with the current state-of-the-art method in SoccerNet Challenge 2024 \citep{cioppa2024soccernet2024challengesresults} achieving only a mean average precision of 73. A significant gap in most approaches is the lack of attention to spatial elements in event data, even though modern datasets often include both event types and spatiotemporal information.

Currently, no publicly available annotation tool allows users to comprehensively label event types along with their spatial and temporal dimensions. To address this, we propose the STE Label Tool\footref{fn:event_label_tool}, designed to support the detailed and accurate annotation of event types and integrate spatiotemporal data for more holistic analysis. This tool aims to bridge the current gap in soccer event data annotation, providing an essential resource for researchers and analysts seeking accurate and spatially aware event data for all types of soccer matches.



\subsection{Event data format in soccer}
\label{ssec:event_data_format_in_soccer}

In professional soccer, numerous sports data vendors provide human-annotated event data, which is indispensable for modern soccer analytics, spanning applications in performance analysis, fan engagement, and machine learning-based research. However, using event data from multiple providers presents several challenges \citep{decroos2020vaep}:

\begin{enumerate}
\item \textbf{Diverse Objectives}: Vendors structure data differently to meet diverse objectives, such as audience engagement, in-depth performance analytics, and research, leading to variability in data formats.

\item \textbf{Inconsistent Terminology}: Each vendor uses distinct terminologies and definitions to describe in-game events, making it challenging to standardize the data for comparative or integrative studies.

\item \textbf{Fragmented Event Information}: Many vendors offer additional, optional information for specific event types, creating variability in the data and complicating the application of automated analysis tools consistently across events.

\end{enumerate}

Existing Python packages SoccerAction (SPDAL) \citep{decroos2020vaep} and Kloppy\footnote{Kloppy: \url{https://github.com/PySport/kloppy/tree/master}} have aimed to address these issues by grouping event data into broader categories, facilitating standardized event handling. However, these packages still face limitations: common actions, such as different pass types, are often aggregated under a single category, while rare actions (e.g., penalty shots) are classified separately, limiting flexibility in data analysis \citep{yeung2024unveiling}. Moreover, deep learning and statistical analyses benefit from features that capture transitions between events, which are underdeveloped in these tools.

To address these limitations, we introduce the Unified and Integrated Event Data (UIED) format, implemented in the OpenSTARLab Pre-processing package\footref{fn:preprocessing}. UIED unifies data from various vendors—such as StatsBomb, Wyscout, and DataStadium—into a consistent format that also aligns with the Google Research Football (GRF) RL environment \citep{kurach2020google}. The UIED format integrates essential spatial and temporal features, such as changes in distance, time, and the distance and angle to goal, which have proven valuable in previous studies \citep{yeung2024unveiling,yeung2023events,yeung2023transformer,simpson2022seq2event}. These features support advanced analyses without dependence on optional, vendor-specific fields, making UIED a robust tool for consistent event data analysis. Further details on the UIED format are provided in Section \ref{ssec:UIED_Format}.



\subsection{Predictive modeling for soccer event}
\label{ssec:predictive_modeling_for_soccer_event}

\begin{table}[h]
\centering
\begin{tabular}{lllll}
\toprule
\parbox{2cm}{ \textbf{Model} \\ \textbf{(Year)}}    & \textbf{Data}       & \textbf{Input Variables}                 & \parbox{3cm}{ \textbf{Model Architecture} \\ \textbf{(Cost Function)}}           & \parbox{3cm}{ \textbf{Target Variables} \\ \textbf{(Event at $t+1$)}}            \\ 
\midrule
Seq2Event \citep{simpson2022seq2event}        & Wyscout \cite{pappalardo2019public}            & - xy coordinates                & Transformer encoder,            & - xy coordinates             \\ 
(2022)                 &                     & - action                        & MLP decoder                     & - action          \\ 
                 &                     & - time                          & (CE + RMSE)                              &                              \\ 
                 &                     & - score advantage               &                                 &                              \\ 
                 &                     & - derived features\footnotemark[2]              &                                 &                              \\ 
\hline
NMSTPP \citep{yeung2023transformer}           & Wyscout \cite{pappalardo2019public}            & - zone                          & Transformer encoder,            & - zone                      \\ 
(2023)                 &                     & - action                        & NPP decoder                     & - action                    \\ 
                 &                     & - inter-event time              & (CE + RMSE)                               & - inter-event time \\ 
                 &                     & - derived features\footnotemark[2]              &                                 &                              \\ 
\hline
NMSTPP+360      & StatsBomb           & - zone                          & Transformer encoder,            & - zone                      \\ 
\citep{yeung2024unveiling,yeung2023events}                  &   360 Data                  & - action                        & NPP decoder                     & - action                    \\ (2023, 2024)
                 &                     & - inter-event time              & (CE + RMSE)                                & - inter-event time \\ 
                 &                     & - player coordinates            &                                 &                              \\ 
                 &                     & - derived features\footnotemark[2]              &                                 &                              \\ 
\hline
LEM \citep{mendes2024towards,mendes2024forecasting}            & Wyscout \cite{pappalardo2019public}            & - xy coordinates                & 3 MLPs (non end-to-end)                          & - xy coordinates            \\ 
(2024, 2024)                 &                     & - action                        & (CE)                              & - action                    \\ 
                 &                     & - time                          &                                 & - time                      \\ 
                 &                     & - home team indicator                         &                                 & - home team indicator                   \\ 
                 &                     & - accuracy indicator                      &                                 & - accuracy indicator    \\ 
                 &                     & - goal indicator                &                                 & - goal indicator              \\ 
                 &                     & - match scores                        &                                 &                              \\ 
\hline
FMS \citep{baron2024foundation}              & StatsBomb            & - xy coordinates                & Transformer decoder             & - action        \\ 
 (2024)                & Open Data\footnotemark[1]                    & - action                        & (CE)                              &                              \\ 
                 &                     & - team                          &                                 &                              \\ 
\botrule
\end{tabular}
\caption{Comparison of soccer event prediction models in recent studies. The models are ranked by publication year, and studies that use similar models but differ only in factors other than architecture, inputs, target, or cost function are grouped into the same row. In the model architecture and cost function, MLP denotes multi-layer perceptron, NPP denotes neural point process, CE denotes cross-entropy, and RMSE denotes root mean square error.}
\footnotetext[1]{StatsBomb open data available at \url{https://github.com/statsbomb/open-data}}
\footnotetext[2]{Derived features include the change in time and distance, angle to goal, and distance to goal.}
\label{Tab:compare_event_model}
\end{table}

Predictive modeling in soccer provides quantitative insights into tactical dynamics, team performance, and potential outcomes by forecasting on-field events. A prominent example is the expected goals (xG) model \citep{eggels2016expected}, which predicts the probability of a goal. This concept has been extended to models such as expected assists (xA)\footnote{Expected Assist Model \url{https://www.statsperform.com/opta-analytics/}} and expected threat (xT)\footnote{Expected Threat Model \url{https://karun.in/blog/expected-threat.html}}, which evaluate the likelihood of creating scoring opportunities or increasing the potential threat of a play.

The Valuing Actions by Estimating Probabilities (VAEP) framework \citep{decroos2019actions} further enhances predictive capabilities by modeling the likelihood of scoring or conceding in future time frames. Variants such as VDEP \citep{toda2022evaluation} and GVDEP \citep{umemoto2022location} extend these principles to focus specifically on defensive evaluations. However, these traditional approaches often lack comprehensive representations of future events, limiting their predictive scope. Recent advances in deep learning, particularly with transformer models \citep{vaswani2017attention}, have revolutionized soccer analytics by enabling the capture of complex sequences in match data. These methods provide more robust and comprehensive predictions of future events, offering new possibilities for analyzing and optimizing team performance.

The Seq2Event model \citep{simpson2022seq2event} marked a foundational step in this field. Utilizing a transformer encoder for feature embedding and a multi-layer perceptron (MLP) for decoding, Seq2Event predicts the xy coordinates and action probabilities at the next time step t+1. This model introduced a novel metric, the possession utility (poss-util) score, which assesses team possession.

Building on Seq2Event, the Neural Marked Spatio-Temporal Point Process (NMSTPP) model \citep{yeung2023transformer} integrates a neural marked spatio-temporal point process \citep{Shchur2021} framework. Replacing the MLP decoder, NMSTPP predicts event timing and spatial attributes sequentially, extending the possession utility score into a Holistic Possession Utilization Score (HPUS) by adding temporal considerations. The NMSTPP+360 \citep{yeung2023events} further enhances predictive accuracy by incorporating player coordinate data, and its potential in RL applications is explored in \citep{yeung2024unveiling}.

Further advancements include the  Large Events Model (LEM) model \citep{mendes2024towards}, which forecasts soccer events by independently training three MLPs to predict event attributes sequentially. LEM also introduces the Expected Points Added (xP+) metric to quantify a player’s contribution toward match outcomes. The model has demonstrated potential for both short- and long-term match simulations \citep{mendes2024forecasting}. Lastly, the Foundation Model for Soccer (FMS) \citep{baron2024foundation}, a transformer decoder-based model, predicts upcoming actions with potential applications in action simulations. Table \ref{Tab:compare_event_model} summarizes these soccer event prediction models.

Despite these advancements, existing studies use varying frameworks and evaluation metrics, complicating direct comparisons. Additionally, most models lack comprehensive simulation performance evaluations. To address these gaps, we introduce the UIED format alongside the OpenStarLab Event package\footref{fn:event}. Together, they establish a standardized platform for event data, facilitating consistent training, evaluation, and optimization methods, thereby enabling reliable model comparison and performance benchmarking, as detailed in Section \ref{sec:experiment}.

\subsection{Reinforcement learning in soccer}
\label{ssec:related_rl_soccer}
The application of agent-based modeling in sports analytics has grown significantly, particularly for analyzing player behaviors and interactions. 
Data-driven analysis involves estimating various variables and functions (such as state variables, reward and policy functions, and action evaluation) in Markov decision processes (MDPs) from observed data representing a sub-problem of RL known as the inverse approach. In contrast, constructing an RL model to generate data in a virtual environment (i.e., the forward approach such as \cite{kurach2020google, fujii2024adaptive}) provides another way to understand the behavior of the modeled agents; however, the forward approach is beyond the scope of this study because it does not usually use the measured data. In this paper, we focus on introducing deep RL (DRL) based methods, these inverse approaches leverage large-scale data to evaluate actions based on their potential to maximize future rewards (for other approaches including the forward approach, see \cite{fujii2021data, fujii2025machine}).

In the context of on-ball actions for team sports, numerous studies have focused on estimating state-action value functions (Q-function) or other policy-related functions using soccer event data \cite{Liu2020,van2021leaving,van2023markov} and tracking data \cite{rahimian2021towards,rahimian2022beyond,rahimian2024towards}, as well as in sports like ice hockey \cite{Liu2018,Schulte2017}, badminton \cite{ding2022deep}, and basketball \cite{yanai2022q, chen2022reliable}. 
Despite these advancements, most models primarily represent teams using on-ball events, often neglecting evaluations of off-ball player dynamics at all time steps. To address this limitation, Nakahara et al. \cite{nakahara2023action} proposed a holistic RL framework capable of estimating Q-functions for multiple players, enabling simultaneous evaluation of both on- and off-ball players, even in the absence of explicit events. In this paper, we introduce the SAR format and RLearn package\footref{fn:rlearn} for the first time, which includes preprocessing, modeling, and visualization tools to improve accessibility and facilitate easier integration and analysis.
For other topics, research in inverse RL (IRL) has explored estimating reward functions directly from data \cite{Luo2020,rahimian2022inferring}. To model policies, trajectory predictions have been conducted using imitation learning \cite{Le17,le2017data,Teranishi2020,fujii2024decentralized} and behavioral modeling \cite{Zhan19,Yeh19,Li2021,fujii2024estimating}, which focus on mimicking observed behaviors through neural networks rather than optimizing them.