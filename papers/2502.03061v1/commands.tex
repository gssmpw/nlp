\usepackage{microtype}
\usepackage{graphicx}
% \usepackage{subfigure}
% \usepackage[margin=1in]{geometry}
\usepackage{booktabs}
\usepackage{enumitem}

\usepackage{mathtools}
\DeclarePairedDelimiter\norm{\lVert}{\rVert}


\usepackage{tikz}
\usetikzlibrary{positioning}
\usetikzlibrary{shapes, arrows, arrows.meta, positioning}
\usepackage{caption}
\usepackage{subcaption}
%\usepackage{subfig}

% \tikzstyle{block} = [draw, fill=white, circle, text centered,inner sep=0.25cm]
% \tikzstyle{block-s} = [draw, fill=white, circle, double, double distance=1pt, text centered,inner sep=0.25cm]

% \tikzstyle{roundnode} = [circle, draw=black!60,, fill=white, thick, inner sep = 1pt]
% \tikzstyle{dashednode} = [circle, draw=black!60, dashed, fill=white, thick, inner sep=1pt]
% \tikzset{edge/.style = {->,> = latex',-{Latex[width=1.5mm]}}}
% \tikzset{edge2/.style={Latex-Latex,dashed}}


% Attempt to make hyperref and algorithmic work together better:
% \newcommand{\theHalgorithm}{\arabic{algorithm}}



% For theorems and such
\usepackage{amsmath}
\usepackage{amssymb}
\usepackage{mathtools}
\usepackage{amsthm}
\usepackage{thmtools,thm-restate}
\usepackage{multirow}

% for graphs
%For figures
\usepackage{mwe}
\usepackage{tikz} % DAG
\usetikzlibrary{arrows.meta,shapes} % DAG
\usepackage{pgf}
\usepackage{bbm}
\usepackage{extarrows}

% if you use cleveref..
%\usepackage[capitalize,noabbrev]{cleveref}



\newcommand{\M}[0]{\mathcal{M}}
\newcommand{\A}[0]{\mathcal{A}}
\newcommand{\V}[0]{\mathcal{V}}
\newcommand{\scal}[0]{\mathcal{S}}
\newcommand{\ocal}[0]{\mathcal{O}}
\newcommand{\E}[0]{\mathcal{E}}
\newcommand{\I}[0]{\mathcal{I}}
\newcommand{\Is}[0]{\mathcal{I}^s}
\newcommand{\pr}[0]{P}
\newcommand{\C}[0]{\mathcal{C}}
\newcommand{\ch}[0]{\text{conv}}
\newcommand{\N}[0]{\mathcal{N}}
\newcommand{\hcal}[0]{\mathcal{H}}
\newcommand{\lcal}[0]{\mathcal{L}}


\newcommand{\ub}{\mathbf{u}}
\newcommand{\wb}{\mathbf{w}}
\newcommand{\pb}{\mathbf{p}}
\newcommand{\Nb}{\mathbf{N}}




\newcommand{\pol}[0]{\pb}



\newcommand{\bmu}[0]{\boldsymbol{\mu}}
\newcommand{\muh}[0]{\hat{\mu}}
\newcommand{\bmuh}[0]{\hat{\bmu}}
\newcommand{\muht}[0]{\hat{\bmu}(t)}
\newcommand{\lamht}[0]{\hat{\Lambda}(t)}
\newcommand{\delh}[0]{\hat{\Delta}}
\newcommand{\delhit}[1]{\hat{\Delta}_{#1}(t)}
\newcommand{\Aht}[0]{\hat{\A}(t)}
\newcommand{\murf}[1]{\hat{\mu}_{#1}(r_f)}
\newcommand{\rhat}[1]{\hat{R}(r, #1)}
\newcommand{\rfhat}[1]{\hat{R}(r_f, #1)}
\newcommand{\logdel}[0]{\ln(1 / \delta)}  
\newcommand{\logDel}[0]{\ln(\frac{1} {\delta})}  
\newcommand{\sumij}[0]{\sum_{\substack{i \in [n] \\ j \in [k]}}}
\newcommand{\cdelt}[0]{\hat{c}_t(\delta)}
\newcommand{\fiwmu}[0]{f_i(\mathbf{w}, \bmu, \A)}
\newcommand{\istarmu}[0]{i^*(\bmu)}
\newcommand{\istarmut}[0]{i^*(\muht)}
\newcommand{\wstar}[1]{\mathbf{w}^*(#1, \A)}
\newcommand{\wzstar}[1]{\mathbf{w}_z^*(#1, \A)}
\newcommand{\wepsstar}[1]{\mathbf{w}^*_{#1}(\bmu, \A)}
\newcommand{\infnorm}[1]{\left\Vert #1 \right\Vert _{\infty}}
\newcommand{\lsnorm}[1]{\left\Vert #1 \right\Vert _{2}}
\newcommand{\dinf}[2]{dist_{\infty}\left(#1, #2\right)}
\newcommand{\dls}[2]{dist_{L^2}\left(#1, #2\right)}
\newcommand{\eteps}[0]{\E_T(\epsilon)}
\newcommand{\etepspr}[0]{\E'_T(\epsilon)}
\newcommand{\cstareps}[0]{C^*_{\epsilon}(\bmu, \A)}
\newcommand{\taudel}[0]{\tau_{\delta}}
\newcommand{\amin}[0]{a_{\min}}
\newcommand{\expec}[1]{\mathbb{E}[#1]}

\newcommand{\numaction}[0]{N^{X}_{i}(t)}
\newcommand{\numcontext}[0]{N^{Z}_{j}(t)}
\newcommand{\numac}[0]{N_{j, i}(t)}




% \DeclareMathOperator*{\argmax}{arg\,max}
% \DeclareMathOperator*{\argmin}{arg\,min}
\newcommand{\argmin}{\operatornamewithlimits{argmin}}
\newcommand{\argmax}{\operatornamewithlimits{argmax}}



%%%%%%%%%%%%%%%%%%%%%%%%%%%%%%%%
% THEOREMS
%%%%%%%%%%%%%%%%%%%%%%%%%%%%%%%%
\theoremstyle{plain}
\newtheorem{theorem}{Theorem}[section]
\newtheorem{proposition}[theorem]{Proposition}
\newtheorem{lemma}[theorem]{Lemma}
\newtheorem{corollary}[theorem]{Corollary}
\theoremstyle{definition}
\newtheorem{definition}[theorem]{Definition}
\newtheorem{assumption}[theorem]{Assumption}
\theoremstyle{remark}
\newtheorem{remark}[theorem]{Remark}

% Todonotes is useful during development; simply uncomment the next line
%    and comment out the line below the next line to turn off comments
%\usepackage[disable,textsize=tiny]{todonotes}
\usepackage[textsize=tiny]{todonotes}