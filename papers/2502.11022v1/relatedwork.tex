\section{Related Work}
This study is closely related to the fields of Text-to-SQL and NoSQL Databases, as briefly surveyed below.
\subsection{Text-to-SQL}
Early research on Text-to-SQL primarily focused on meticulously designed rule-based methods, such as those in \citep{baik2020duoquest,li2014constructing,li2014nalir,quamar2022natural,sen2020athena++}, these methods used predefined rules or semantic parsers to translate NLQs into SQL but were inflexible and inadequate for handling increasingly complex database structures.
%
With the rise of deep learning, the focus of Text-to-SQL research has gradually shifted towards methods that utilize deep neural networks, such as attention mechanisms~\citep{liu2023multi}  and graph-based encoding strategies~\citep{hui-etal-2022-s2sql,li2023graphix,qi-etal-2022-rasat,wang-etal-2020-rat,xu-etal-2018-sql,zheng-etal-2022-hie,yu2021grappa,xiang2023g3r}. Alternatively, some approaches treat Text-to-SQL as a sequence-to-sequence problem by using encoder-decoder structured Pre-trained Language Models (PLMs) to generate SQL queries~\citep{10.5555/3304222.3304323,popescu2022addressing,qi-etal-2022-rasat}.

In recent years, large language models (LLMs), which have demonstrated remarkable success across various domains, have also garnered increasing attention in the Text-to-SQL field~\citep{dong2023c3,gan-etal-2021-natural-sql,10.14778/3641204.3641221,li2023resdsql,lin-etal-2020-bridging,pourreza2024din,qi-etal-2022-rasat,rubin-berant-2021-smbop,scholak-etal-2021-picard}. Current literature primarily focuses on two approaches with LLMs: prompt engineering and pretraining/fine-tuning. Prompt engineering methods typically involve using specific reasoning workflows which can be categorized into several reasoning modes, including Chain-of-Thought (CoT) \citep{wei2022chain} and its variants \citep{pourreza2024din,liu2023divide,zhang-etal-2024-coe,zhang-etal-2023-act}, Least-to-Most \citep{zhou2023leasttomost,gan-etal-2021-natural-sql,arora-etal-2023-adapt}, and Decomposition \citep{khot2023decomposed,tai-etal-2023-exploring,pourreza2024din,wang-etal-2025-mac,xie-etal-2024-decomposition}. To evaluate Text-to-SQL model performance in practical applications, several large-scale benchmark datasets have been developed and released, including WikiSQL \citep{zhong2018seqsql}, Spider \citep{yu-etal-2018-spider}, KaggleDBQA \citep{lee-etal-2021-kaggledbqa}, BIRD \citep{10.5555/3666122.3667957}, and Bull \citep{10.1145/3626246.3653375} etc.

\subsection{NoSQL Database}
 %SQL, as the standard query language for relational databases, is essential for data management and retrieval. However, 


Traditional SQL databases face limitations with large-scale, unstructured, or semi-structured data in the internet and big data era, prompting the rise of NoSQL databases, which provide flexibility, scalability, and high performance in web applications and real-time data analysis \cite{moniruzzaman2013nosql}.
%
In the field of databases and NLP, current research primarily focuses on several key areas of NoSQL databases,including achieving scalability in data storage systems within large-scale user environments \citep{cattell2011scalable}, ensuring consistency in NoSQL databases \citep{diogo2019consistency}, addressing multi-tenant NoSQL data storage issues in cloud computing environments, particularly in scenarios involving resource and data sharing \citep{zeng2015resource}, and realizing scalability, elasticity, and autonomy in database management systems (DBMS) within cloud computing environments \citep{agrawal2011database}.

Despite the extensive research on NoSQL across various domains, its accessibility remains a challenge, especially for non-expert users. Although Text-to-NoSQL tasks have been proposed to address this issue, existing NoSQL generation primarily supports English and overlooks the needs of non-English users.
To tackle this issue, we introduce the Multilingual Text-to-NoSQL task, which is based on existing Text-to-NoSQL research and not only aims to reduce the barrier for non-expert users to utilize NoSQL databases by automatically converting NLQs into NoSQL queries but also addresses the gap in existing Text-to-NoSQL tasks that mainly support English while neglecting non-English users' needs. In this task, we also introduce MultiTEND, the largest multilingual benchmark for natural language to NoSQL query generation.

\newpage