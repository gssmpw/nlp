
\documentclass{article} % For LaTeX2e
\usepackage[preprint]{colm2025_conference}

\usepackage{microtype}
\usepackage{hyperref}
\usepackage{url}
\usepackage{booktabs}

\usepackage{lineno}
\usepackage{graphicx}

\usepackage{multirow}
\usepackage{xspace}
\usepackage{adjustbox}
\usepackage{pifont}
\usepackage{caption}
\usepackage{makecell}
\usepackage{subcaption}
\usepackage{hyperref}
\usepackage{tcolorbox}
\usepackage{lipsum} % for placeholder text
\usepackage{amsmath}
\usepackage{tablefootnote}

\definecolor{darkblue}{rgb}{0, 0, 0.5}
\hypersetup{colorlinks=true, citecolor=darkblue, linkcolor=darkblue, urlcolor=darkblue}

\newcommand{\zhiyu}[1]{\textcolor{red}{[zhiyu: #1]}}
\newcommand{\mian}[1]{\textcolor{blue}{[Mian: #1]}}

\title{Preference Learning Unlocks LLMs' Psycho-Counseling Skills}

% Aligning LLMs to Respond in Psychotherapy
% Authors must not appear in the submitted version. They should be hidden
% as long as the \colmfinalcopy macro remains commented out below.
% Non-anonymous submissions will be rejected without review.

\author{Mian Zhang$^{\alpha}$, Shaun M. Eack$^{\beta}$, Zhiyu Zoey Chen$^{\alpha}$ \\
$^{\alpha}$Department of Computer Science, University of Texas at Dallas\\ $^{\beta}$School of Social Work, University of Pittsburgh \\
\texttt{\{mian.zhang, zhiyu.chen2\}@utdallas.edu}
% Antiquus S.~Hippocampus, Natalia Cerebro \& Amelie P. Amygdale \thanks{ Use footnote for providing further information
% about author (webpage, alternative address)---\emph{not} for acknowledging
% funding agencies.  Funding acknowledgements go at the end of the paper.} \\
% Department of Computer Science\\
% Cranberry-Lemon University\\
% Pittsburgh, PA 15213, USA \\
% \texttt{\{hippo,brain,jen\}@cs.cranberry-lemon.edu} \\
% \And
% Ji Q. Ren \& Yevgeny LeNet \\
% Department of Computational Neuroscience \\
% University of the Witwatersrand \\
% Joburg, South Africa \\
% \texttt{\{robot,net\}@wits.ac.za} \\
% \And 
}

% The \author macro works with any number of authors. There are two commands
% used to separate the names and addresses of multiple authors: \And and \AND.
%
% Using \And between authors leaves it to \LaTeX{} to determine where to break
% the lines. Using \AND forces a linebreak at that point. So, if \LaTeX{}
% puts 3 of 4 authors names on the first line, and the last on the second
% line, try using \AND instead of \And before the third author name.

\newcommand{\fix}{\marginpar{FIX}}
\newcommand{\new}{\marginpar{NEW}}
\newcommand{\huggingface}{\raisebox{-1.5pt}{\includegraphics[height=1.05em]{figures/hf-logo.pdf}}\xspace}
\newcommand{\dataset}{\texttt{PsychoCounsel-Preference}}
\newcommand{\rewards}{\texttt{PsychoCounsel-Llama3-3B-Reward}}
\newcommand{\rewardl}{\texttt{PsychoCounsel-Llama3-8B-Reward}}
\newcommand{\policys}{\texttt{PsychoCounsel-Llama3-3B}}
\newcommand{\policyl}{\texttt{PsychoCounsel-Llama3-8B}}

\definecolor{darkred}{RGB}{139,0,0}
\newcommand{\red}[1]{\textcolor{red}{#1}}
\newcommand{\blue}[1]{\textcolor{blue}{#1}}
\newcommand{\darkred}[1]{\textcolor{darkred}{#1}}


\begin{document}
\ifcolmsubmission
\linenumbers
\fi
\maketitle

\begin{abstract}
Applying large language models (LLMs) to assist in psycho-counseling is an emerging and meaningful approach, driven by the significant gap between patient needs and the availability of mental health support. However, current LLMs struggle to consistently provide effective responses to client speeches, largely due to the lack of supervision from high-quality real psycho-counseling data, whose content is typically inaccessible due to client privacy concerns. Furthermore, the quality of therapists’ responses in available sessions can vary significantly based on their professional training and experience. Assessing the quality of therapists’ responses remains an open challenge.
In this work, we address these challenges by first proposing a set of professional and comprehensive principles to evaluate therapists’ responses to client speeches. Using these principles, we create a preference dataset, \dataset{}, which contains 36k high-quality preference comparison pairs. This dataset aligns with the preferences of professional psychotherapists, providing a robust foundation for evaluating and improving LLMs in psycho-counseling.
Experiments on reward modeling and preference learning demonstrate that \dataset{} is an excellent resource for LLMs to acquire essential skills for responding to clients in a counseling session. Our best-aligned model, \policyl{}, achieves an impressive win rate of 87\% against GPT-4o. We release \dataset{}, \policyl{} and the reward model \rewardl{} to facilitate the research of psycho-counseling with LLMs at:



\end{abstract}

% \vspace{6pt}
\begin{center}
\begin{tabular}{cll}
\huggingface & \textbf{Dataset \& Models} & \url{https://hf.co/Psychotherapy-LLM}

\end{tabular}
\end{center}


\documentclass[../main.tex]{subfiles}
\graphicspath{{../images/}}
\makeatletter
\def\input@path{{../images/}}
\makeatother
\begin{document}
\section{Introduction}
\begin{figure}
\centering
\begin{tikzpicture}
\node[inner sep=0pt] (ws) at (0, 0) {
\includegraphics[height=.4\textwidth, trim={10cm 0 10cm 0},clip]{world_space.png}};
\node[inner sep=0pt] (cs) at (6,0) {\includegraphics[height=.4\textwidth, trim={10cm 1cm 10cm 4cm},clip]{conf_space.png}};
\end{tikzpicture}
\vspace{-5pt}
\label{fig:pbrm_intro}
\caption{\textbf{Left}: Shows world space obstacles as grey spheres. Robots start and goal configuration is colored red and green, respectively. Configurations along the computed path are colored transparent blue. \textbf{Right:} Mapped world space scenario to configuration space. Obstacle region is the grey mesh. Red spheres are collision-free regions computed by the neural SCDF. The optimized shortest path in the convex corridor is the blue curve.}
\vspace{-25pt}
\end{figure}
Motion planning is the problem of finding a collision-free trajectory that connects a given start and goal configuration. The planning takes place in the configuration space of the robot. For single body robots, like mobile robots or drones, the configuration space and the world space are usually the same. This simplifies the planning, since explicit obstacle representations are available which enables geometrical tools like separating hyperplanes, smallest distance to obstacles etc., to be used when designing motion planning algorithms. For multi-body robots like manipulators, the situation is completely different. The world space obstacles are usually mapped to non-convex regions, and to make the problem even harder, the mapping is usually not known. Forming explicit representations of the obstacle region in the configuration space is usually too expensive or intractable. Despite all of this, sampling based planners are used with great success, which mainly is due to their use of implicit representations of the obstacle region. The basic idea is to construct a graph in the configuration space that covers and connects the collision-free region. From this graph, a path can be extracted that connects a given start and goal configuration. The approach is computationally expensive, since the graph is constructed with the smallest geometrical building block available, points, which represents a collision-check. Furthermore, the extracted paths from the graph are non-smooth and jagged due to the stochastic nature of the approach. This adds an additional post-processing step to the process, where the paths are shortcutted and smoothened, before the path can be used for tracking. Clearly a lot of time is invested to form this graph and produce smooth paths. Thus, if the obstacles start to move, then all of this work is done in no use, since all points that make up this graph need to be re-verified, which is simply too time consuming to be done in real time.
\\\\
In this work, we want to address the existing drawbacks of the sampling based planners. Our main contribution is an improved motion planner where each vertex in the graph covers a collision-free region in the form of a sphere instead of a point and where the edges are formed with neighboring intersecting spheres. This representation has the advantage of instead of returning piecewise linear paths, returning a sequence of overlapping spheres, i.e. a convex corridor, that connects a given start and goal configuration, illustrated in Figure \ref{fig:pbrm_intro}. This convex corridor allows us to use convex optimization to produce smooth trajectories, instead of computationally expensive post-processing methods. The representation further allows us to estimate the coverage of the collision-free space, which gives us awareness and feedback in the offline roadmap construction phase. Finally, our representation is simple to adapt to moving obstacles, simply requery for the new radii and recheck for intersections. 
\\\\
The spherical collision-free regions are formed using a signed distance function (SDF), which is a function that returns the smallest distance from an arbitrary point to the boundary of an obstacle. As the name implies, the distance is signed, thus if the point is inside the obstacle it is negative otherwise positive. If the distance is positive, a sphere with radius equal to the distance is guaranteed to cover a collision-free region. Using an SDF in motion planning is not new, but what is novel about our approach is that we express the distance in the configuration space instead of the world space and by doing so allows us to form these convex collision-free regions. We refer to the resulting SDF as a signed configuration distance function (SCDF). Computing an SCDF analytically is non-trivial, our approach is therefore to parameterize the SCDF with a deep neural network and learn the mapping by supervised learning. Our resulting neural SCDF can compute distances for different parameter values of obstacle shapes and we also show how multiple distances can be combined, thus making our approach flexible.
\section{Related work}
Motion planning algorithms can roughly be divided into three families, grid-based, sampling based and optimization based methods. Grid-based methods (GBM) discretize the planning space from which a graph is then compiled. A standard search method is A$^\star$ \citep{a_star}, which is classified as an \textit{informed} search method, since it employs a heuristic function to speed up the search. A$^\star$ guarantees to return an optimal path at the level of discretization used. GBMs usually discretize the planning space by a regular lattice and this limits the GBMs to problems with low dimensionality due to the curse of dimensionality. Thus, GBMs are usually limited to single-body robots where the degrees of freedom (DOF) are low. To overcome the inherent scaling problem with the GBMs, stochastic methods are usually used for multi-body robots. These methods are termed as sampling-based methods (SBM) and core members within this family are the rapidly-exploring random trees (RRT) \citep{rrt} and the probabilistic roadmap (PRM) \citep{prm}. RRT grows a tree from the start configuration and explores the collision-free region in a rapid way until it is able to connect to the goal region. RRT is usually improved by bi-directional planning \citep{rrt_connect}, i.e. an additional tree is grown from the goal configuration and the trees are tested for connection after any tree has been expanded. RRT is a single-query method, thus it searches for a path from scratch each time it is queried. Contrary to this, PRM is a multi-query method, which solves for multiple queries without starting from scratch. PRM does this by creating a roadmap (graph) that covers the collision-free space as an offline step. The graph is then used to solve for multiple queries. PRMs are used in cases where the environment does not change since the extra offline step is too computationally costly and needs to be re-done if the environment is changed. In our work, we address this inherent issue by using a different roadmap representation. Our vertices in the graph cover a collision-free region in the form of spheres and we form the edges by checking for intersecting spheres. If something in the environment changes, we recompute the spheres radii and recheck the intersections, without relying on collision detection. We use a trained neural network to compute the sphere radius, therefore querying for the radius can be done fast, hence our representation enables the PRM for dynamic environments.
\\\\
In the recent decades, optimization based methods (OBM) \citep{chomp, schulman, itomp, stomp} have been introduced as an alternative to SBM for multi-body robots. Like the SBM, the OBMs scale well to higher dimensional problems and produce smoother motion. It is common to use a SDF in the optimization since it is a smooth function, thus enabling gradient-based methods. However, the standard way of expressing the SDF is in world space. The distance therefore needs to be mapped to the configuration space by the forward kinematics. This mapping makes the optimization problem a non-linear program (NLP), which is computationally expensive to solve. Recently, a different approach has been proposed. In \cite{mp_gcs} motion planning is formulated as a convex optimization problem by using the graph of convex sets framework \citep{gcs}. The underlying idea is to decompose the collision-free space into intersecting convex sets from which a convex optimization problem is formulated. In cases where an explicit representation of the obstacles in the configuration space exists, like for single-body robots, creating collision-free convex regions can be done fast \citep{iris}. For multi-body robots, this is non-trivial. Existing work does this successfully \citep{iris_nlp, iris_c} by an optimization based approach, but the methods are still too time consuming to be used in the presence of moving obstacles. Our approach is instead to use deep learning to learn an SDF expressed in the configuration space. With this, we can query for shortest distances to the collision boundary, which allows us to expand spherical regions which are collision-free. Our approach is fast and therefore enables our suggested roadmap planner to be used in dynamic environments.
\\\\
Recent research has focused on learning collision detection \citep{fk_kernel_distance, diffco, graphdistnet} by predicting the signed distance between the robot links and the surrounding obstacles in the world space. The learned SDF is used in trajectory optimization but since the distance is expressed in the world space, the problem becomes an NLP and therefore takes a long time to solve. We take a novel approach and suggest to instead express the signed distance in the configuration space. This allows us to improve the PRM at the same time as it enables convex optimization for trajectory optimization, which runs faster and is more reliable than NLP solvers. In \cite{cspf} a learned signed distance function in the configuration space is proposed similar to our approach. However, their approach is restricted to point cloud representations, while we propose to represent the obstacles as parameterized geometric shapes, e.g. spheres. Furthermore, we also show how to use our learned SCDF to improve an existing roadmap planner.
\section{Problem formulation}
A robot is located in the world space, $\W \subset \R^3 $. The unique location of the robot is given by its configuration $\q \in \C$, where $\C$ is the configuration space. The set of points covered by the robots bodies at a certain configuration is expressed as $\B(\q) \subset \W$. The robot is surrounded by $\NrObst$ obstacles $\O = \bigcup_{i=1}^{\NrObst} \O_i$, where  $\O_i \subset \W$. The representation of the obstacle in the configuration space is the set $\C\O_i = \{\q \in \C \: |\: \B(\q) \cap \O_i \neq \emptyset \}$. The obstacle space is formed as $\Co = \bigcup_{i=1}^{\NrObst} \C \O_i$. The complement is referred to as the free space, $\Cf = \C \setminus \Co$. The path planning problem is a tuple, ($\Cf$, $\qStart$, $\qGoal$), where we want to connect a query pair, consisting of a start, $\qStart$, and goal configuration, $\qGoal$, with a geometric path, $\q(s): [0, 1] \mapsto \Cf$, such that $\q(0)=\qStart$ and $\q(1)=\qGoal$, or report correctly when such a path does not exist.
\end{document}

\section{Related Work}
\label{sec:related-work}
%We now contextualize our work with related literature so that our contributions are highlighted. We cover FMTS, perturbations in time-series, 
% robustness testing of FMs, 
%and rating of AI systems. 

\noindent \textbf{Foundation Models Supporting Time Series} 
The use of FMs for time series forecasting has advanced significantly. 
% \cite{lu2022frozen} first demonstrated that transformers pre-trained on text data (LLMs) can effectively solve sequence modeling tasks in other modalities, paving the way for leveraging language pre-trained transformers for time series analysis. Recent studies have focused on reprogramming LLMs for time series tasks through parameter-efficient fine-tuning and suitable tokenization strategies \cite{zhou2023one, gruver2024large, jin2023time, cao2023tempo, ekambaram2024tiny}. These methods have successfully adapted transformers to the unique challenges of time series forecasting. \cite{zhou2023one} and \cite{jin2023time} further illustrate the versatility and robustness of fine-tuned language pre-trained transformers for diverse time series tasks.
\cite{lu2022frozen} showed that transformers pre-trained on text data can solve sequence modeling tasks in other modalities, enabling their application to time series analysis. Recent studies have reprogrammed LLMs for time series tasks through parameter-efficient fine-tuning and tokenization strategies \cite{zhou2023one, gruver2024large, jin2023time, cao2023tempo, ekambaram2024tiny}. 
% These methods have successfully adapted transformers to the unique challenges of time series forecasting. 
\cite{zhou2023one} and \cite{jin2023time} further illustrate the versatility and robustness of fine-tuned language pre-trained transformers for diverse time series tasks.
% Several models have contributed to the advancement of time series forecasting. \cite{ansari2024chronos} and \cite{woo2024unified} have improved forecasting accuracy and model generalization.  
% % \cite{ansari2024chronos} and \cite{woo2024unified} have pushed the boundaries of forecasting accuracy and model generalization. 
% \cite{rasul2023lag} and \cite{das2023decoder} have explored new tokenization strategies and fine-tuning methods to improve model performance. Additionally, \cite{garza2023timegpt} and \cite{ekambaram2024tiny} have focused on creating lightweight and efficient models for real-time applications. \cite{talukder2024totem} stands out with its unique approach to integrating multiple temporal patterns, enhancing forecasting precision.
% FMs trained from scratch have achieved SOTA on time series tasks. Zero-shot forecasting, exemplified by \cite{gruver2024large}, showcases the ability of these models to make accurate predictions without domain-specific training. \cite{cao2023tempo} and \cite{goswami2024moment} have introduced approaches to enhance the performance and efficiency of time series models, leveraging transformer architectures to capture temporal dependencies more effectively. In our experiments, we select Gemini-V and Phi-3 as the GP models and Chronos and MOMENT as TS models due to their SOTA performance in their respective categories.
Several models have advanced time series forecasting. \cite{ansari2024chronos} and \cite{woo2024unified} have improved forecasting accuracy and model generalization, while
% \cite{ansari2024chronos} and \cite{woo2024unified} have pushed the boundaries of forecasting accuracy and model generalization. 
\cite{rasul2023lag} and \cite{das2023decoder} have explored new tokenization strategies and fine-tuning methods. \cite{garza2023timegpt} and \cite{ekambaram2024tiny} developed lightweight models for real-time applications, and \cite{talukder2024totem} integrated multiple temporal patterns to improve precision. FMs trained from scratch, like \cite{gruver2024large}, achieved SOTA in zero-shot forecasting, with \cite{cao2023tempo} and \cite{goswami2024moment} further improving model performance. 
%In our experiments, we select Gemini-V and Phi-3 as the GP models and Chronos and MOMENT as TS models due to their SOTA performance in their respective categories.
Please see Section~\ref{sec:exp_app} for the FMTS we selected due to their SOTA performance in their respective categories.

%The use of FMs for time series forecasting has seen significant advancements in recent years. \cite{lu2022frozen} first demonstrated that transformers pre-trained on text data (LLMs) can effectively solve sequence modeling tasks in other modalities. This work opened the door to leveraging language pre-trained transformers for time series analysis. Recent studies have built on this foundation, focusing on reprogramming LLMs for time series tasks through parameter-efficient fine-tuning and suitable tokenization strategies \cite{zhou2023one, gruver2024large, jin2023time, cao2023tempo, ekambaram2024tiny}. These methods have proven successful in adapting the powerful capabilities of transformers to the unique challenges of time series forecasting. OneFitsAll \cite{zhou2023one} and Time-LLM \cite{jin2023time} further illustrate how language pre-trained transformers can be fine-tuned for diverse time series tasks, demonstrating their versatility and robustness. 
% \zhen{reason why we didn't include these models in our study, weights not available? or other justification, to prevent that naturally raised question from readers.}\kl{Good point. We need to discuss. I added 2 sentences at the bottom but they are probably not very convincing.}
%Several other models have contributed to the advancement of time series forecasting. Chronos \cite{ansari2024chronos} and Moirai \cite{woo2024unified} have pushed the boundaries of forecasting accuracy and model generalization. Lag-llama \cite{rasul2023lag} and TimesFM \cite{das2023decoder} have explored new tokenization strategies and fine-tuning methods to improve model performance. Additionally, Time-GPT1 \cite{garza2023timegpt} and Tiny-Time Mixers \cite{ekambaram2024tiny} have focused on creating lightweight and efficient models suitable for real-time applications. TOTEM \cite{talukder2024totem} stands out with its unique approach to integrating multiple temporal patterns, further enhancing forecasting precision.
%Aside from reprogramming LLMs for time series, FMs trained from scratch have achieved SOTA on times series tasks. 
%Zero-shot forecasting, exemplified by \cite{gruver2024large}, showcases the ability of these models to make accurate predictions without domain-specific training.  TEMPO \cite{cao2023tempo} and MOMENT \cite{goswami2024moment} have introduced approaches to enhance the performance and efficiency of time series models, leveraging transformer architectures to capture temporal dependencies more effectively.
% \zhen{and these are on various time series tasks including time series forecasting?}
% \zhen{These are models that are specifically trained for time series forecasting, I'd suggest mentioning them first after the LLM reprogramming, and then expanding to the models that are trained across time series tasks instead. The flow of this subsection feels a bit odd as of now.} \kl{Done.}
%In our experiments, we select Gemini-V and Phi-3 as the GP models and Chronos and MOMENT as TS models due to their SOTA performance in their respective categories. 

%\vspace{-0.3em}
\noindent \textbf{Perturbations in Time Series Data} TS data is commonly stored in spreadsheets and databases, which are prone to changes due to acts of omission (e.g., negligence, data-entry errors) or commission (e.g., adversarial attacks, sabotage). Omission errors are most common \cite{spreadsheets-errors-risks-survey}. Tools like Microsoft Excel and Google Sheets are widely used for data collection and analysis, allowing end-user programming \cite{spreadsheets-future-workshop}. However, over 90\% of spreadsheets contain errors due to issues like incorrect formulae, leading to multi-billion dollar losses \cite{spreadsheet-qa-survey}.
%\cite{spreadsheet-qa-survey,spreadsheets-errors-risks-survey}.
Adversarial attacks are also increasing in data stores and AI models for tasks like forecasting.
% \cite{papernot2016transferability} introduced a black-box attack method using a substitute model to generate adversarial examples, demonstrating transferability across tasks. \cite{baluja2017adversarial} focused on white-box attacks using gradient information. 
\cite{karim2019adversarial} adapted these concepts to time series, exploring both black-box and white-box attacks. \cite{oregi2018adversarial} revealed the vulnerability of distance-based classifiers. \cite{rathore2020untargeted} examined various adversarial attacks on time series classifiers. TSFool \cite{li2022tsfool} introduced a multi-objective black-box attack to craft imperceptible adversarial time series to fool RNN classifiers.
%Time series (TS) data is widely stored and manipulated in spreadsheets and databases. These are also the tools which see considerable changes or perturbations due to acts of omission that are unintended (e.g., negligence, data-entry errors) or commission which are deliberate (e.g., adversarial attacks, sabotage). 
%Among these, changes due to omission are most common \cite{spreadsheets-errors-risks-survey}.
%For example, a spreadsheet, implemented in tools like Microsoft Excel and Google Sheets, is a common data collection and analysis environment that also allows end-user programming \cite{spreadsheets-future-workshop}. They are used widely at the workplace and are often a door opener to more advanced scientific tools. But gaining expertise in them needs practice since a large proportion of spreadsheets ($\succ$ 90\%) are known to have errors due to issues like incorrect formulae caused by improper understanding of behavior during routine operations like copy-paste and end-user programming, which have caused losses of multi-billion dollars \cite{spreadsheet-qa-survey,spreadsheets-errors-risks-survey}.
% \zhen{do we need to relate our perturbations to these attacks? otherwise, we must manage the readers' expectations on what types of perturbations we focus on other than adversarial attacks, and motivate it properly}
%\zhen{Play down this a bit, and emphasize and justify why we focus on the type of perturbations we consider in the paper, to mimic operational errors in practice apart from adversarial attacks, citing the 2024 and 1996 papers Biplav added.} 
%Furthermore, adversarial attacks are also increasing both in data stores and in AI models created to solve tasks like forecasting.
%Foundational work by ~\cite{papernot2016transferability} introduced a black-box attack method that involved training a substitute model to generate adversarial examples capable of misleading the target model, demonstrating the transferability property across similar tasks. In contrast, research by ~\cite{baluja2017adversarial} focused on white-box attacks, using gradient information and probabilistic outputs to craft adversarial examples. Researchers~\cite{karim2019adversarial} have adapted these concepts to the time series domain, exploring both black-box and white-box attacks on time series classification models. In addition, ~\cite{oregi2018adversarial} revealed the susceptibility of distance-based time-series classifiers to adversarial examples. ~\cite{rathore2020untargeted} examined untargeted, targeted, and universal adversarial attacks on time series classifiers, demonstrating the effectiveness of these attacks across various datasets. TSFool~\cite{li2022tsfool} introduced a multi-objective black-box attack to craft highly imperceptible adversarial time series to fool RNN classifiers.
%Adversarial attacks on time-series data are initially focused on time-series classification tasks, leveraging concepts adapted from adversarial attacks in other domains.
%explored adversarial sample crafting for time series classification using elastic similarity measures,  %These works collectively underscore the ongoing efforts to understand and mitigate the risks posed by adversarial attacks on time series classification models.
% More recently, research into adversarial attacks on time series forecasting models has revealed distinct challenges and novel attack strategies. One primary challenge is targeted attacks. While targeted adversarial attacks on time series classification aim to misclassify specific instances, achieving similar precision in time series forecasting is more complex due to the sequential nature of the data. Perturbations must be designed to influence specific aspects of the forecast (e.g., directional shifts or amplitude changes) without disrupting the overall temporal dependencies, making precise control more challenging~\cite{govindarajulu2023targeted}. Another challenge is attacks on multivariate forecasting. Adversarial attacks could exploit the inter-dependences between variables. ~\cite{liu2022robust} introduced sparse and indirect cross-time-series attacks in multivariate settings, which are more effective and realistic than direct attacks in univariate cases.
% \zhen{Biplav, could we make a quick comment here as well that we focus more on data error side in practice, other than attacks? and cite the paper that you mentioned on data errors? Otherwise this section of adversarial attacks feel a bit standalone to other sections}
%These challenges underscore the need for ongoing research to develop effective adversarial attack strategies and robust defense mechanisms tailored to the unique characteristics of time series forecasting models.
% -----


\noindent \textbf{Rating AI Systems} Several works have assessed and rated AI systems for trustworthiness from a third-party perspective without access to training data. \cite{srivastava2020rating} proposed a method to rate AI systems for bias, specifically targeting gender bias in machine translators \cite{srivastava2018towards}, and used visualizations to communicate these ratings \cite{bernagozzi2021vega}. They conducted user studies on trust perception through visualizations \cite{vega-userstudy-translatorbias}, but these lacked causal interpretation. \cite{kausik2024rating} introduced a causal analysis approach to rate bias in sentiment analysis systems, extending it to assess their impact when used with translators \cite{kausik2023the}. We extend their method to rate MM-TSFM for robustness against perturbations. Causal analysis offers advantages over statistical analysis by determining accountability, aligning with humanistic values, and quantifying the direct influence of various attributes on forecasting accuracy.


\section{\dataset{}}

\begin{figure}[htbp]
    \centering
    \includegraphics[width=\textwidth]{figures/pipline-cropped.pdf}
    \caption{\dataset{} Construction Pipeline. 1) We first collect over 26k client speeches covering a wide range of topics from various sources, applying necessary data cleaning. 2) 20 popular LLMs are sampled and prompted to roleplay as psychotherapists and give responses to these client speeches. 3) GPT-4o is instructed to evaluate the responses based on our proposed PsychoCounsel Principles, and preference pairs with substantial score gaps are incorporated into \dataset{}.}
    \label{fig:pipline}
\end{figure}

\subsection{Client Speech Collection}
We collect client speeches from various data sources: counsel-chat\footnote{\url{https://github.com/nbertagnolli/counsel-chat}}, MentalAgora~\citep{Lee2024-ly}, TherapistQA~\citep{Shreevastava2021-eh}, Psycho8k~\citep{Liu2023-ub}, and several huggingface datasets (amod-counsel\footnote{\url{https://huggingface.co/datasets/Amod/mental_health_counseling_conversations}}, MentalChat16K\footnote{\url{https://huggingface.co/datasets/ShenLab/MentalChat16K}}, and phi2Mental\footnote{\url{https://huggingface.co/datasets/saxenaindresh681/microsoft-phi2-mental-health}}). Client speeches with number of characters more than 1,000 and less than 100 are discarded to ensure a proper length of context. After an additional step of de-duplication, the resulting data contains 26,483 client speeches with average length of 366 characters covering a wide range of topics including 8 coarse topics: Core Mental Health Issues (9,054), Emotional Well-being and Coping Strategies (5,717), Relationships and Interpersonal Dynamics (6,483), Life Transitions and Challenges (934), Social Issues (667), Youth and Development (1,175), Crisis and Safety Concerns (529) and Special Topics (1,924). Under these 8 topics are 42 fine-grained topics (see Table~\ref{tab:topic_distributions} in the appendix for the detailed topic distribution).

\subsection{PsychoCounsel Principles}\label{sec:principle}
To answer the question \textit{what is a good response to a client speech in psycho-counseling}, we collaborate with professors in social work and psychiatry (our co-authors) and propose a set of professional principles to measure the response to a client speech from seven different dimensions: \begingroup
% Box for terms and concepts
\definecolor{lightbeige}{HTML}{eceae0} % Light beige color
\definecolor{darkbeige}{HTML}{b8b09a}  % Darker beige 
\definecolor{darkgray}{HTML}{696969}
\definecolor{gainsboro}{HTML}{DCDCDC}
\begin{tcolorbox}[
    colback=gainsboro,        % Light gray background
    colframe=darkgray,         % Black border
    boxrule=0.8pt,          % Border thickness
    width=\textwidth,       % Full width of text
    title=\textcolor{white}{PsychoCounsel Principles}, % Title of the box
    fonttitle=\bfseries,    % Bold title
    coltitle=black,         % Title text color
    left=2mm,               % Padding on the left
    right=2mm,              % Padding on the right
    top=2mm,                % Padding at the top
    bottom=2mm,             % Padding at the bottom
    before skip=10pt,       % Space before the box
    after skip=10pt         % Space after the box
]

% \setstretch{0.9} % Reduce line spacing for compactness
\noindent\textbf{Empathy and Emotional Understanding}: The response should convey genuine empathy, acknowledging and validating the client’s feelings and experiences. \\
\noindent\textbf{Personalization and Relevance}: The response should be tailored to the client’s unique situation, ensuring that the content is directly relevant to their concerns.\\
\noindent\textbf{Clarity and Conciseness}: The response should be clear, well-organized, and free of unnecessary jargon, making it easy for the client to understand and engage with.\\
\noindent\textbf{Avoidance of Harmful Language or Content}: The response should avoid any language or content that could potentially harm, distress, or trigger the client, ensuring the interaction is safe and supportive.\\
\noindent\textbf{Facilitation of Self-Exploration}: The response should encourage the client to reflect on their thoughts and feelings, promoting self-awareness and insight.\\
\noindent\textbf{Promotion of Autonomy and Confidence}: The response should support the client’s sense of control over their decisions and encourage confidence in their ability to make positive changes.\\
\noindent\textbf{Sensitivity to the Stage of Change}: The response should recognize the client’s current stage in the process of change and address their needs accordingly. If the client is in an early stage—uncertain or ambivalent about making a change—the response should help them explore their thoughts and motivations. If the client is in a later stage and has already made changes, the response should focus on reinforcing progress, preventing setbacks, and sustaining positive outcomes.
\end{tcolorbox}
\endgroup

Please refer to Box~\ref{box:priciple} for the complete definition of the principles. Among these seven principles, \textbf{Facilitation of Self-Exploration}, \textbf{Promotion of Autonomy and Confidence}, and \textbf{Identifying Stages and Reasons for Change} emphasize a client-centered approach, which is recognized as a hallmark of effective psycho-counseling~\citep{Miller-and-Stephen-RollnickUnknown-oo}. We use these three principles to measure the \textit{effectiveness} of a response to a client speech, complementary to the other four principles, which are more basic, requiring the response to be \textit{empathy}, \textit{relevant}, \textit{concise}, and \textit{safe}. Evaluating therapist responses using these fine-grained principles provides a more structured and nuanced assessment of their effectiveness. Unlike general evaluations that focus solely on overall quality, this detailed approach allows for a deeper understanding of how well a response supports the client’s emotional and psychological needs.

\subsection{Preference Generation}
We apply the generate-score-pair pipeline as ~\cite{Cui2023-ui} to construct the \dataset{} dataset. For each client speech, we randomly sample four off-the-shelf LLMs from a model pool to give the response and instruct GPT-4o to annotate each response with 5-Likert scores for each principle defined in Section~\ref{sec:principle}; higher scores mean more alignment with the principles. Then scores of the principles are averaged to get the overall score for a response and preference pairs are generated based on the overall scores. The whole pipeline is illustrated in Figure~\ref{fig:pipline}. To increase the diversity of the model responses, we initialize the model pool with 20 popular LLMs of a range of sizes developed by different organizations shown in Table~\ref{tab:model-pool}. We also include LLMs with different architectures other than pure transformers like AI21-Jamba-1.5-Mini~\citep{Jamba-Team2024-vt}, which is a hybrid transformer-mamba model. We randomly held out 3,291 client speeches for testing and the remaining 23,192 for training. After obtaining the scores of principles, for training, we extract response pairs with the overall score gap larger than or equal to 1 as the preference pairs, and for testing, we only extract the ([highest score response], [lowest score response]) pairs and pairs with the score gap less than 1 are discarded. In this way, we could exclude response pairs with similar scores, whose quality may be hard to differentiate.

Ultimately, \dataset{} includes 34,329 training preference pairs and 2,324 testing pairs. The models most likely to be chosen and those most likely to be rejected vary significantly in size (see Figures~\ref{fig:response-dist-chosen} and~\ref{fig:response-dist-rejected} for the distributions of chosen and rejected models). This suggests that simply scaling model size is not a decisive factor in making LLMs effective responders in psycho-counseling. We also observe that LLMs developed by non-English-speaking institutions are more likely to be rejected compared to those from English-speaking countries. This may suggest that non-English-speaking institutions have a greater need to enhance the capabilities of LLMs in their respective languages, potentially leading to less emphasis on developing psycho-counseling skills in English.


\subsection{Preference Validation}
To validate the quality of synthetic human preferences in \dataset{}, we hired two professional psychotherapists through Upwork\footnote{\url{https://www.upwork.com/}} and instructed them to annotate preferences based on each principle and give the overall preference. The annotation set consists of 200 preference pairs randomly sampled from \dataset{}. The two therapists agree on 174 out of 200 samples. Additionally, one expert’s annotations align with the preference labels in \dataset{} for 184 out of 200 samples, while the other aligns for 170 out of 200 samples. These results indicate a high level of agreement between the experts (87\%) and demonstrate strong alignment between the expert annotations and the preference labels in \dataset{} (88.5\%). This strongly suggests that the labels in \dataset{} are reliable and trustworthy.




\section{Experiments}\label{sec_exp}
%\hp{Accelerating IM simulation~\cite{tang2015influence}}

% \begin{itemize}
%     \item 6.1. Problem setting of three COPs, including the general model and three specific CO problems 
%     \item 6.2. Experiment Setting (hyperparameters, details of training, evaluation, and test) 写在appendix里吧
%     \item 6.3. Performance analysis 这个要占半页
% \end{itemize}

%\hp{need to think of a way to compress these tables / visuals.} 

%\hp{\cancel{Baselines}; hyperparamters; \cancel{metrics}; etc.}

With theoretical guarantees on the existence and convergence of NE for ACCES games, we are also interested in how our proposed algorithm CCDO-RL works empirically. To evaluate this, we conduct experiments of CCDO-RL on three distinct ACCES game instances introduced in Section \ref{sub_exp_ins} and analyze the performance of CCDO-RL in Section \ref{sub_train_eval}. Section 6.2.1 aims to empirically demonstrate the convergence (Figures \ref{fig_exploit_20} and \ref{fig_exploit_50}) of the algorithm CCDO-RL over realistic CO problems, and show its consistency with Theorem \ref{CCDOA}. Section 6.2.2 intends to show the average reward (to seen training graphs) as well as the generalizability (to unseen test graphs) of the combinatorial player in real-world ACCES games (shown in Tables \ref{tab_aver}, and \ref{tab_gene}).

\subsection{Three Instances of ACCES Games} \label{sub_exp_ins}
% \hp{This para does not make much sense. Need to follow the framework in the Preliminaries section.}
% For combinatorial optimization problems in real-world applications, situations are more complicated and intractable due to changeable environmental or physical parameters. The form of parameter sets is very crucial because different types have different solvability and computation complexity. Forms of parameter sets mainly contain discrete sets, interval sets \cite{buchheim2018robust} like polyhedral and ellipsoid, probability distributions \cite{carlsson2018wasserstein}, and variable functions \cite{krause2008robust}.

% In reality, these parameters are often impacted by some common factors, such as conditions of weather, transportation, and individual personalities. \cite{kalimeris2019robust} proposed an assumption that real instances (e.g. demands in CVRP, coverages in CSP) 
%Considering affected or attacked COPs, the real instance $\{\theta_{i}\}$ always relied on the estimated value $\{\hat{\theta}_{i}$\} and the variation determined by independent factors $\{g_{i}\}$ and environment/physical parameters/attacker actions $\{\eta\}$. The concrete parameter influence model is stated as follows:

We consider a certain COP which is parameterized with $\{\theta_{i}\}$, where $i$ is the index of nodes (such as a target in security games) -- e.g., such parameters can be interpreted as attack probability of targets.
%coverage radius, customer's demands, or attack probability of targets. 
In real-world applications, we often need to estimate such parameters before solving the COPs. Unfortunately, the estimation $\{\hat{\theta}_{i}\}$ often bears a gap to the true value $\{\theta_{i}\}$, which derives from e.g. environment (aleatoric) uncertainty, model (epistemic) uncertainty, or an attacker trying to manipulate the defender's utility. We use a generic model to formulate this gap:
\begin{equation}\label{linrob}
    \theta_{i} = \hat{\theta}_{i} + y \cdot \tau_{i},
\end{equation}
where $y$ represents the strategy of the nature/attacker, $\tau_{i}$ is the environment factors like weather and transportation conditions, or human subjective factors like the preference of the attacker. 
Such abstraction can represent a wide range of ACCES games, such as facility location covering problems \cite{an2020battery, TIRKOLAEE2020340}, CVRP \cite{vehiclerouting.ch8,dinh2018exact, FLORIO20231081}, security patrolling (OP) \citep{xu2021robust}, and influence maximization problem \cite{kalimeris2019robust}. We describe three instances of ACCES games based on the model (\ref{linrob}).%Based on this model (\ref{linrob}), we focus on three combinatorial optimization problems with attacks or environmental/physical influence.

% \hp{Hard to follow. We should point out what are the two players, what are X, Y, u etc}

\textbf{Adversarial Covering Salesman Problem (ACSP):} In a map of cities, every city $i$ has a coverage $\theta_{i}$. A salesman finds the shortest path such that all cities are visited or covered, with $\theta_{i}$ influenced by physical factors $\tau_i$ and transportation parameters $y$ based on Eq.(\ref{linrob}). The salesman is Player 1 where $X$ consists of the feasible paths of the salesman. Nature is Player 2 with $Y$ = $[0, 1]^K \ni y, K \in \mathbb{N}$. The utility function of Player 1 $u$ is the opposite of the total traveling distance.

\textbf{Adversarial Capacitated Vehicle Routing Problem (ACVRP):} A vehicle with a constrained capacity of goods finds the shortest path under the worst case with the $i_{th}$ customer's demand $\theta_i$ changed by environmental factors $\tau_i$ and weather parameter $y$ on Eq.(\ref{linrob}). The vehicle is Player 1 where $X$ is the set of the feasible path $x$. Nature is Player 2 where $Y$ is $[0, 1]^K \ni y, K \in \mathbb{N}$. The utility function of Player 1  $u$ is the opposite of total delivery distance satisfying all the demands of customers.


\textbf{Patrolling Game (PG):} The patrolling game is described in the introduction.

For all the problem instances, we run our algorithm on two problem sizes: 20 nodes and 50 nodes. The detailed description and problem parameters of the three game instances are in Appendix \ref{app_ex_para_set}.

% Similarly, in the vehicle route problem (VRP), conditions with correlated parameters arouse broad attention from scholars \cite{vehiclerouting.ch8,dinh2018exact,FLORIO20231081}. \cite{dinh2018exact} considered the demand correlation by geographical proximity of nodes, described by some independent random variables in the fractional form. \cite{FLORIO20231081} utilized 'external factors' to stand for unknown covariates affecting all demands and presented a Bayesian model to learn correlations. Further more, about IM problems, \cite{kalimeris2019robust} combined node features and uncertain hyperparameters to fit the influence probability on each edge.

% \subsection{Training CCDO-RL}

% For all the problems, CCDO-RL adopts the REINFORCE algorithm with an attention-based encoder-decoder framework \cite{kool2018attention} (used as an inductive graph representation component) to learn a (generalizable) COP solver for one player (protagonist), and PPO \cite{schulman2017proximal} to train a policy for the other player (adversary) whose strategy space is continuous. CCDO-RL is trained with 50 epochs on a set of 10,000 graphs (with 20 or 50 nodes). The hyperparameters of CCDO-RL are specified in Appendix \ref{app_ex_para_set} (Table \ref{tab_hyper_ccdorl}). Our code is included as supplementary material for ease of reproduction. 
% % \hp{need to specify hyperparas}

\subsection{Performance of CCDO-RL}\label{sub_train_eval}

Two aspects are evaluated for the performance of CCDO-RL, i.e., i) Convergence to NE (Section \ref{sub_per_conver}) exploring whether CCDO-RL can compute the NE, and ii) Protagonist policy's average reward and generalizability (Section \ref{sub_per_rob}). Generalizability refers to the ability of RL models trained on previously seen graphs (problem instances), to perform well on a new set of unseen test graphs. The model’s usability is enhanced by generalizability, rather than focusing solely on the average reward, which is a critical motivation in the literature on RL for COPs \citep{khalil2017learning, kool2018attention}.

For all the problems, CCDO-RL adopts the REINFORCE algorithm with an attention-based encoder-decoder framework \citep{kool2018attention} (used as an inductive graph representation component) to learn a generalizable COP solver for Player 1 (protagonist), and PPO to train a policy for Player 2 (adversary) whose strategy space is continuous. CCDO-RL is trained on a set of 10,000 graphs (with 20 or 50 nodes). The hyperparameters of CCDO-RL are specified in Appendix \ref{app_ex_para_set} (Table \ref{tab_hyper_ccdorl}). Our code is included as supplementary material and will be open-sourced for ease of reproduction. 

% \textbf{Training.} For all the problems, CCDO-RL adopts the REINFORCE algorithm with attention-based encoder-decoder framework \cite{kool2018attention} (used as an inductive graph representation component) to learn a (generalizable) COP solver for one player (protagonist), and PPO \cite{schulman2017proximal} to train a policy for the other player (adversary) whose strategy space is continuous. CCDO-RL is trained with 50 epochs on a set of 10,000 graphs (with 20 or 50 nodes). 

% \hp{We should first present results about convergence as it is mostly aligned with the theory.}

\subsubsection{Convergence to NE} \label{sub_per_conver}

Exploitability is a common metric to describe the closeness to true NE by calculating the sum of performance distances between each new best response and subgame NE, i.e. $\sum_{i=1,2} U(\pi_{i,k}^{br}, \sigma_{-i,k}) - U(\sigma)$ in the general two-player game. Since our game is zero-sum, the calculation is as follows:
\begin{equation*}
   \text{Exploitability}(\sigma) = \max_{\pi_1 \in \Sigma_1} U(\pi_1, \sigma_{2}) - \min_{\pi_2 \in \Sigma_2} U(\sigma_1, \pi_2).
\end{equation*}
From Figure \ref{fig_exploit_20}, we can see that CCDO-RL can converge to approximate NE in 25 iterations or less (in the PG setting), reaching 0.05 in ACSP, 0.10 in ACVRP, and 0.03 in PG with 20 nodes. Similar results are observed in problems with 50 nodes (see Figure \ref{fig_exploit_50} in Appendix \ref{app_exp}). These results validate the effectiveness of CCDO-RL in finding the NE for various types of games.

%Similarly, the exploitability of three COPs in 50 nodes is provided in the appendix \ref{app_exp}.
\vspace{-\baselineskip}
\begin{figure}[htbp]
	\centering
    \subfigure[ACSP20]{
    \label{csp20_nashconv}
    \includegraphics[scale=0.20]{Figures/nashconv_log_csp20_sm_7.eps}
    }
    \subfigure[ACVRP20]{
    \label{cvrp20_nashconv}%文中引用该图片代号
    \includegraphics[scale=0.20]{Figures/nashconv_log_svrp20_sm_7.eps}
    }
    \subfigure[PG20]{
    \label{opsa20_nashconv}
    \includegraphics[scale=0.20]{Figures/nashconv_log_pg20_sm_7.eps}
    }
    \caption{Exploitability curve of CCDO-RL on three games of 20 nodes}
    \label{fig_exploit_20}
\end{figure}
\vspace{-\baselineskip}
\subsubsection{Average reward and Generalizability of Combinatorial player} \label{sub_per_rob}
% \subsubsection{Robustness and Generalizability of Protagonist Policy} \label{sub_per_rob}
%\hp{CCDO-RL being better in these following metrics is only kind of a by-product.}

% \textbf{Evaluation.} The learned policies are then tested on 200 graphs, where 100 of them are randomly selected from the 10,000 training graphs, and the other 100 are unseen graphs. 
% We use two metrics to evaluate the performance of different policies for the protagonist player: \textbf{Average proportional loss} $R-$ describes the policy overfitting degree \citep{lanctot2017unified}; \textbf{Reward} evaluates the performance of the protagonist with the adversary under three COPs.  
% \begin{eqnarray}
%         &R- = (\hat{D} - \hat{O}) / \hat{D}.
% \end{eqnarray}
% in which $\hat{D}$ is the mean value of the diagonals and $\hat{O}$ is the mean value of the off-diagonals in the payoff matrix provided in the Appendix \ref{app_exp}.

% Because the protagonist policy is trained against a powerful adversary under our ACCES game setting, the obtained policy is naturally robust against adversarial perturbations. This subsection sheds a bit of light on this perspective and quantifies the extent of robustness of CCDO-RL as well as the ability of RL to generalize to unseen test graphs.

\textbf{Evaluation.} The learned policies are tested on 200 graphs, with 100 being randomly selected from the 10,000 training graphs (to show the average reward), and the other 100 being unseen graphs (to test policy generalization). We evaluate the performance of the protagonist with the adversary under three COPs. For each COP, the performance is considered both on the 20-node and 50-node map.
% We use two metrics to evaluate the performance of different policies for the protagonist player: \textbf{Average proportional loss} $R-$ describes the policy overfitting degree \citep{lanctot2017unified}; \textbf{Reward} evaluates the performance of the protagonist with the adversary under three COPs.

\textbf{Baselines.} There are heuristic algorithms for each game instance (Heuristic in Table \ref{tab_aver} and \ref{tab_gene}) and a single-player RL algorithm. For ACVRP, we adopt the Tabu Search algorithm (Tabu) \citep{li2020improved} as the heuristic algorithm, which is widely applied in the routing problem. For ACSP, the common benchmark local search algorithm, LS2 \citep{golden2012generalized}, is used. For PG, we choose the greedy algorithm as the baseline. The "RL against Stoc" algorithm in Tables \ref{tab_aver} and \ref{tab_gene} is identical to the protagonist model in CCDO-RL but trained in environments with stochastic adversarial perturbations.

% \textbf{Baselines.} There are a heuristic algorithms for each game instance {\color{red} (Heuristic mentioned in the Table \ref{tab_aver} and \ref{tab_gene})} and a single-player RL algorithm. For ACVRP, we adopt the Clarke-Wright (CW) algorithm \citep{pichpibul2013heuristic} and the Tabu Search algorithm (Tabu) \citep{li2020improved} as heuristics, which are applied widely in the routing problem. For ACSP, two common benchmark local search algorithms, LS1 and LS2 \citep{golden2012generalized}, are used. For PG, we choose a local search algorithm \citep{vansteenwegen2009iterated} and the greedy algorithm as the heuristic baselines. {\color{red} The "RL  against Stoc" algorithm referred to Tables \ref{tab_aver} and \ref{tab_gene}} is identical to the protagonist model in CCDO-RL {\color{red} but trained on environments with stochastic adversarial perturbations.} 

\textbf{Average Reward.}  As illustrated in Table \ref{tab_aver}, our algorithm achieves a better average reward than baselines (10.08\% improvement on average of all settings against two baselines), regardless of CO instance or problem size, when confronting the adversary trained by CCDO-RL. In the setting of CSP-20 nodes, the average reward is improved by 46.98\% compared to the heuristic and by 7.14\% compared with the RL against Stoc. For the 50-node setting, the improvements are 45.91\% and 5.28\% respectively. Similarly, the improvements in contrast to Heuristic and RL against Stoc are as follows: 1.72\% and 3.01\%  for CVRP-20 nodes, 0.75\% and 4.46\% for CVRP-50 nodes, 4.17\% and 1.48\% for PG-20 nodes, and 10.60\% and 4.38\% for PG-50 nodes.

\textbf{Generalizability.} From Table \ref{tab_gene}, CCDO-RL continues to achieve a better average reward when facing the adversary, demonstrating that the learned RL policies generalize well to unseen graphs. Even though the non-RL baselines do have access to the graph structures and other problem information of the unseen problem instances, CCDO-RL can obtain comparable performances without re-training on the new problem instances. The improvements versus Heuristic and RL against Stoc are 46.61\% and 7.02\% for CSP-20 nodes, 42.24\% and 3.94\% for CSP-50 nodes, 1.12\% and 1.56\% for CVRP-20 nodes, 0.90\% and 5.05\% for CVRP-50 nodes, 5.35\% and 2.40\% for PG-20 nodes, and 12.17\% and 10.33\% for PG-50 nodes. Even when confronting the stochastic adversary, CCDO shows superior generalizability compared to two baselines across three COPs, with average improvements of 6.31\%, 3.42\%, and 3.95\% respectively. Detailed results are provided in Appendix \ref{app_exp} (Tables \ref{tab_csp_full_20} - \ref{tab_op_full_50}). 
% The model’s usability is enhanced by the ability to generalize rather than focusing solely on the average reward, which is a critical motivation of the RL for combinatorial optimization literature \citep{khalil2017learning, kool2018attention}.  

\begin{remark}
    In CO problems (or more broadly, operations research and economics), it is known that achieving solution quality improvements against strong baselines (e.g., the RL methods trained with a stochastic adversary) is very challenging, and the margins are usually small \citep{kool2018attention}, sometimes even less than 1\%. However, these “tiny” marginal improvements in profits keep small business owners in the real world alive. Last, the improvement depends a lot on the problem settings, and we show that sometimes the improvement can be much more significant.
\end{remark}
\vspace{-\baselineskip}
% \textbf{Performance analysis.} The robustness results of CCDO-RL for ACSP are shown in Table \ref{tab_csp}. We have the following observations: 1) On both of the 100 seen/unseen graphs, single-player RL performs better than heuristic algorithms no matter whether attacked or not. (2) When confronting the adversary trained by CCDO-RL, CCDO-RL exceeds RL by 0.25 and 0.24 on the training set, and by 0.25 and 0.18 on the test set, respectively under the 20-node and 50-node graphs. This demonstrates the robustness of CCDO-RL. 3) Compared to the performance of the training set with that of the test set, we can see that RL and CCDO-RL both maintain a certain degree of generalization. Similar results for ACVRP (Table \ref{tab_cvrp}) and SPG (Table \ref{tab_op}) are provided in Appendix \ref{app_exp}. 

\begin{table}[ht]
  \caption{Average reward against CCDO-RL's adversary (on seen graphs)}
  \vspace{\baselineskip}
  \label{tab_aver}
  \centering
  \small
  \begin{tabular}{lllllll}
    \toprule
    \multirow{2}{*}{method} & \multicolumn{2}{c}{ACSP (Mean$\pm$Std)} & \multicolumn{2}{c}{ACVRP (Mean$\pm$Std)} & \multicolumn{2}{c}{PG (Mean$\pm$Std)} \\
    \cmidrule(r){2-3} \cmidrule{4-5} \cmidrule(r){6-7}
                            & 20 nodes & 50 nodes & 20 nodes & 50 nodes & 20 nodes & 50 nodes\\
    \midrule
    Heuristic & 6.13$\pm$1.20 & 7.55$\pm$1.42 & 7.65$\pm$1.23  & 13.38$\pm$1.70 & 2.64$\pm$1.03 & 4.53$\pm$1.84   \\
    RL against Stoc    & 3.50$\pm$0.47  & 4.55$\pm$0.62  & 7.55$\pm$1.16  & 13.90$\pm$1.63 & 2.71$\pm$0.90 & 4.80$\pm$2.18   \\
    CCDO-RL   & $\pmb{3.25}$$\pm$0.42 & $\pmb{4.31}$$\pm$0.51  & $\pmb{7.42}$$\pm$1.21  & $\pmb{13.28}$$\pm$1.52 &  $\pmb{2.75}$$\pm$0.87 & $\pmb{5.01}$$\pm$1.91  \\
    \bottomrule
  \end{tabular}
\end{table}
\vspace{-\baselineskip}

\begin{table}[htp]
  \caption{Generalizability against CCDO-RL's adversary (on unseen graphs)}
  \vspace{\baselineskip}
  \label{tab_gene}
  \centering
  \small
  \begin{threeparttable}
  \begin{tabular}{lllllll}
    \toprule
    \multirow{2}{*}{method} & \multicolumn{2}{c}{ACSP (Mean$\pm$Std)} & \multicolumn{2}{c}{ACVRP (Mean$\pm$Std)} & \multicolumn{2}{c}{PG (Mean$\pm$Std)} \\
    \cmidrule(r){2-3} \cmidrule{4-5} \cmidrule(r){6-7}
                            & 20 nodes & 50 nodes & 20 nodes & 50 nodes & 20 nodes & 50 nodes\\
    \midrule
    Heuristic & 6.20$\pm$1.33 & 7.60$\pm$1.37   & 7.64$\pm$1.30  & 13.27$\pm$1.87 & 2.43$\pm$0.98 & 4.19$\pm$1.69    \\
    RL against Stoc  & 3.56$\pm$0.37  & 4.57$\pm$0.58  & 7.67$\pm$1.30  & 13.85$\pm$1.53 &  2.50$\pm$0.95 & 4.26$\pm$2.17 \\
    CCDO-RL   & $\pmb{3.31}$$\pm$0.35 & $\pmb{4.39}$$\pm$0.52  & $\pmb{7.55}$$\pm$1.28  & $\pmb{13.15}$$\pm$1.59 & $\pmb{2.56}$$\pm$0.92 & $\pmb{4.70}$$\pm$1.94\\

    \bottomrule
  \end{tabular}
  \begin{tablenotes}
      \footnotesize
      \item[1] For the average reward of ACSP and ACVRP, smaller is better while for that of PG larger is better.
  \end{tablenotes}
  \end{threeparttable}
\end{table}
\vspace{-\baselineskip}
% two heuristics and one RL
% \begin{table}[ht]
%   \caption{{\color{red} Average reward of CCDO-RL (on seen graphs). For the value of CSP and CVRP, larger is better while for that of PG smaller is better.}}
%   \label{tab_aver}
%   \centering
%   \small
%   \begin{tabular}{lllllll}
%     \toprule
%     \multirow{2}{*}{method} & \multicolumn{2}{c}{CSP (Mean$\pm$Std)} & \multicolumn{2}{c}{CVRP (Mean$\pm$Std)} & \multicolumn{2}{c}{PG (Mean$\pm$Std)} \\
%     \cmidrule(r){2-3} \cmidrule{4-5} \cmidrule(r){6-7}
%                             & 20 nodes & 50 nodes & 20 nodes & 50 nodes & 20 nodes & 50 nodes\\
%     \midrule
%     Baseline 1 & 4.52$\pm$0.71  & 5.98$\pm$0.94 & 7.64$\pm$1.56  & 13.49$\pm$2.10 & 2.71$\pm$1.10 & 1.82$\pm$1.40   \\
%     Baseline 2 & 6.13$\pm$1.20 & 7.55$\pm$1.42   & 7.65$\pm$1.23  & 13.38$\pm$1.70 & 2.64$\pm$1.03 & 1.47$\pm$0.99  \\
%     RL {\color{red}against Stoc}    & 3.50$\pm$0.47  & 4.55$\pm$0.62  & 7.55$\pm$1.16  & 13.90$\pm$1.63 & 2.71$\pm$0.90 & 1.54$\pm$1.03   \\
%     CCDO-RL   & $\pmb{3.25}$$\pm$0.42 & $\pmb{4.31}$$\pm$0.51  & $\pmb{7.42}$$\pm$1.21  & $\pmb{13.28}$$\pm$1.52 &  $\pmb{2.75}$$\pm$0.87 & $\pmb{1.87}$$\pm$1.22  \\
%     \bottomrule
%   \end{tabular}
% \end{table}


% \begin{table}[htp]
%   \caption{{\color{red}Generalizability of CCDO-RL (on unseen graphs)}}
%   \label{tab_gene}
%   \centering
%   \small
%   \begin{threeparttable}
%   \begin{tabular}{lllllll}
%     \toprule
%     \multirow{2}{*}{method} & \multicolumn{2}{c}{CSP (Mean$\pm$Std)} & \multicolumn{2}{c}{CVRP (Mean$\pm$Std)} & \multicolumn{2}{c}{PG (Mean$\pm$Std)} \\
%     \cmidrule(r){2-3} \cmidrule{4-5} \cmidrule(r){6-7}
%                             & 20 nodes & 50 nodes & 20 nodes & 50 nodes & 20 nodes & 50 nodes\\
%     \midrule
%     Baseline 1 & 4.53$\pm$0.79  & 5.95$\pm$0.96 & 7.55$\pm$1.39  & 13.35$\pm$2.04 & 2.52$\pm$1.08 & $\pmb{1.86}$$\pm$1.44  \\
%     Baseline 2 & 6.20$\pm$1.33 & 7.60$\pm$1.37   & 7.64$\pm$1.3  & 13.27$\pm$1.87 & 2.43$\pm$0.98 & 1.52$\pm$1.20    \\
%     RL {\color{red}against Stoc}  & 3.56$\pm$0.37  & 4.57$\pm$0.58  & 7.67$\pm$1.30  & 13.85$\pm$1.53 &  2.50$\pm$0.95 & 1.03$\pm$5.05 \\
%     CCDO-RL   & $\pmb{3.31}$$\pm$0.35 & $\pmb{4.39}$$\pm$0.52  & $\pmb{7.55}$$\pm$1.28  & $\pmb{13.15}$$\pm$1.59 & $\pmb{2.56}$$\pm$0.92 & 1.35$\pm$5.09\\

%     \bottomrule
%   \end{tabular}
%   \begin{tablenotes}
%       \footnotesize
%       \item[1] For the value of CSP and CVRP, larger is better while for that of PG smaller is better.
%   \end{tablenotes}
%   \end{threeparttable}
% \end{table}

\section*{Conclusion}
This paper aims to enhance our understanding of the computational complexity of computing various Shapley value variants. We found that for various ML models --- including decision trees, regression tree ensembles, weighted automata, and linear regression --- both local and global interventional and baseline SHAP can be computed in polynomial time under HMM modeled distributions. This extends popular algorithms, such as TreeSHAP, beyond their empirical distributional scope. We also establish strict complexity gaps between the various SHAP variants (baseline, interventional, and conditional) and prove the intractability of computing SHAP for tree ensembles and neural networks in simplified scenarios. Overall, we present SHAP as a versatile framework whose complexity depends on four key factors: \begin{inparaenum}[(i)] \item model type, \item SHAP variant, \item distribution modeling approach, \item and local vs. global explanations\end{inparaenum}. We believe this perspective provides deeper insight into the computational complexity of SHAP, paving the way for future work.




%We believe that our framework provides a more intricate understanding of SHAP computation complexity across different models, distributions, and variants, paving the way for further research.

Our work opens promising directions for future research. First, expanding our computational analysis to other SHAP-related metrics, such as asymmetric SHAP~\citep{frye20} and SAGE~\citep{covert2020understanding}, would be valuable. Additionally, we aim to explore more expressive distribution classes and relaxed assumptions beyond those in Section \ref{sec:tractable} while maintaining tractable SHAP computation. Finally, when exact computation is intractable (Section \ref{sec:intractable}), investigating the approximability of SHAP metrics through approximation and parameterized complexity theory~\citep{downey2012parameterized} is an important direction.

%Our work opens several promising avenues for future research on the computational properties of explainable AI methods, with a particular focus on SHAP. First, it would be interesting to broaden the computational analysis conducted in this work to include other popular SHAP-related metrics in the literature, such as asymmetric SHAP \cite{frye20} and SAGE \cite{covert2020understanding}. Also, in the future, we aim to explore more expressive distribution classes and relaxed distributional assumptions—extending beyond those examined in Section \ref{sec:tractable} —that still yield tractable SHAP computation. Finally, when exact computation proves intractable (Section \ref{sec:intractable}), it is worthwhile to theoretically investigate the question of the approximability of computing the SHAP metrics across various configurations, through the lens of approximation and parametrized complexity theory \cite{arora2009computational}.

%This paper aims to deepen our understanding of the computational complexity involved in obtaining different Shapley value variants. We found that for a variety of ML models, including decision trees, tree ensembles for regression, weighted automata, and linear regression models — computing both local and global interventional and baseline SHAP can be done in polynomial time when distributions are modeled by HMMs. This extends the distributional scope of popular algorithms like TreeSHAP, which is limited to empirical distributions. Additionally, we demonstrate a strict complexity gap between SHAP variants, showing that interventional and baseline SHAP can be strictly easier to compute than conditional SHAP. Despite these positive results, we uncovered intractability for various SHAP variants in neural networks and tree ensembles. Finally, we provided generalized complexity relations across SHAP variants. We believe that our framework offers a deeper understanding of the complexity involved in computing SHAP across various variants, models, distributions, as well as in both local and global computations, laying the groundwork for future research.
Our benchmark comprises contexts and instructions related to multiple real-world safety topics, aiming to provide comprehensive evaluation and support further improvement on LLM safety. However, some of the data may contain toxic contents that could pose potential risks if misused. To mitigate these risks, we plan to conduct careful inspections before open-sourcing the benchmark, and restrict data access to individuals who adhere to stringent ethical guidelines.

During the data collection procedure, we inform the crowd workers from China in advance of the possibility of encountering harmful contents and the future use of the annotated data. Participation of the workers is entirely voluntary, and they are free to withdraw from the study at any time without burden. We pay the workers about 8.5 USD per hour, which is above the average wage of local residents.



\bibliography{Mental-RM}
\bibliographystyle{colm2025_conference}

\appendix
\clearpage
\section{Prompts}
\begingroup
% Box for terms and concepts
\definecolor{lightbeige}{HTML}{eceae0} % Light beige color
\definecolor{darkbeige}{HTML}{b8b09a}  % Darker beige 
\begin{tcolorbox}[
    colback=lightbeige,        % Light gray background
    colframe=darkbeige,         % Black border
    boxrule=0.8pt,          % Border thickness
    width=\textwidth,       % Full width of text
    title=Rating Prompt, % Title of the box
    fonttitle=\bfseries,    % Bold title
    coltitle=black,         % Title text color
    left=2mm,               % Padding on the left
    right=2mm,              % Padding on the right
    top=2mm,                % Padding at the top
    bottom=2mm,             % Padding at the bottom
    before skip=10pt,       % Space before the box
    after skip=10pt         % Space after the box
]

You are provided with a client speech and four responses from different psychotherapists. Rate the responses based on how they align with the given principle.\\

Client Speech: \{client\_speech\}\\
Response 1: \{response1\}\\
Response 2: \{response2\}\\
Response 3: \{response3\}\\
Response 4: \{response4\}\\

Provide a JSON object as output that includes the following keys:
\begin{itemize}
    \item response\_1\_rating: An integer score from 1 to 5 for response 1
    \item rationale\_1: A string explaining the reasoning behind the given score for response 1
    \item response\_2\_rating: An integer score from 1 to 5 for response 2
    \item rationale\_2: A string explaining the reasoning behind the given score for response 2
    \item response\_3\_rating: An integer score from 1 to 5 for response 3
    \item rationale\_3: A string explaining the reasoning behind the given score for response 3
    \item response\_4\_rating: An integer score from 1 to 5 for response 4
    \item rationale\_4: A string explaining the reasoning behind the given score for response 4
\end{itemize}
\end{tcolorbox}
\endgroup
\begingroup
% Box for terms and concepts
\definecolor{lightbeige}{HTML}{eceae0} % Light beige color
\definecolor{darkbeige}{HTML}{b8b09a}  % Darker beige 
\begin{tcolorbox}[
    colback=lightbeige,        % Light gray background
    colframe=darkbeige,         % Black border
    boxrule=0.8pt,          % Border thickness
    width=\textwidth,       % Full width of text
    title=Responding Prompt, % Title of the box
    fonttitle=\bfseries,    % Bold title
    coltitle=black,         % Title text color
    left=2mm,               % Padding on the left
    right=2mm,              % Padding on the right
    top=2mm,                % Padding at the top
    bottom=2mm,             % Padding at the bottom
    before skip=10pt,       % Space before the box
    after skip=10pt         % Space after the box
]

You are now a professional psychotherapist conducting a session with a client. Answer the given client speech.\\
Client Speech: \{client\_speech\}
\end{tcolorbox}
\endgroup
\begingroup
% Box for terms and concepts
\definecolor{lightbeige}{HTML}{eceae0} % Light beige color
\definecolor{darkbeige}{HTML}{b8b09a}  % Darker beige 
\begin{tcolorbox}[
    colback=lightbeige,        % Light gray background
    colframe=darkbeige,         % Black border
    boxrule=0.8pt,          % Border thickness
    width=\textwidth,       % Full width of text
    title=LLM-as-Ranker Prompt, % Title of the box
    fonttitle=\bfseries,    % Bold title
    coltitle=black,         % Title text color
    left=2mm,               % Padding on the left
    right=2mm,              % Padding on the right
    top=2mm,                % Padding at the top
    bottom=2mm,             % Padding at the bottom
    before skip=10pt,       % Space before the box
    after skip=10pt         % Space after the box
]\label{box:apx-rm-eval-prompt}


Determine which of the two given responses from different psychotherapists to a client's speech is better:\\
Client Speech: \{client\_speech\} \\
Response 1: \{response\_1\}\\
Response 2: \{response\_2\}
\end{tcolorbox}
\endgroup




\clearpage
\section{Dataset Information}


\begingroup
\setlength{\tabcolsep}{4pt}
\begin{table}[!ht]
\centering
\caption{Topic Distribution}
\label{tab:topic_distributions}
\begin{tabular}{llr}
\toprule
\textbf{Coarse Category} & \textbf{Fine Category} & \textbf{Count} \\
\midrule
\multicolumn{3}{l}{\textbf{1. Core Mental Health Issues}} \\
 & Anxiety & 3714 \\
 & Depression & 2859 \\
 & Stress & 1439 \\
 & Trauma & 526 \\
 & Substance-abuse & 387 \\
 & Addiction & 129 \\
\midrule
\multicolumn{3}{l}{\textbf{2. Emotional Well-being and Coping Strategies}} \\
 & Self-esteem & 1377 \\
 & Grief-and-loss & 1023 \\
 & Caregiving & 1541 \\
 & Behavioral-change & 740 \\
 & Anger-management & 448 \\
 & Self-care & 311 \\
 & Sleep-improvement & 277 \\
\midrule
\multicolumn{3}{l}{\textbf{3. Relationships and Interpersonal Dynamics}} \\
 & Relationships & 1690 \\
 & Family-conflict & 2358 \\
 & Friendship-conflict & 292 \\
 & Marriage & 373 \\
 & Intimacy & 403 \\
 & Social-relationships & 410 \\
 & Workplace-relationships & 383 \\
 & Relationship-dissolution & 574 \\
\midrule
\multicolumn{3}{l}{\textbf{4. Life Transitions and Challenges}} \\
 & Career & 441 \\
 & Aging & 140 \\
 & New-environment & 235 \\
 & Military-issues & 118 \\
\midrule
\multicolumn{3}{l}{\textbf{5. Social Issues}} \\
 & LGBTQ & 335 \\
 & Culture & 113 \\
 & Human-sexuality & 151 \\
 & Bullying & 68 \\
\midrule
\multicolumn{3}{l}{\textbf{6. Youth and Development}} \\
 & Children-adolescents & 123 \\
 & School-life & 322 \\
 & Parenting & 730 \\
\midrule
\multicolumn{3}{l}{\textbf{7. Crisis and Safety Concerns}} \\
 & Domestic-violence & 144 \\
 & Self-harm & 231 \\
 & Eating-disorders & 154 \\
\midrule
\multicolumn{3}{l}{\textbf{8. Special Topics}} \\
 & Counseling-fundamentals & 638 \\
 & Diagnosis & 531 \\
 & Communication & 205 \\
 & Professional-ethics & 128 \\
 & Legal-regulatory & 94 \\
 & Spirituality & 192 \\
 & Others & 136 \\
\bottomrule
\end{tabular}
\end{table}
\endgroup
\begingroup
% Box for terms and concepts
\definecolor{lightbeige}{HTML}{eceae0} % Light beige color
\definecolor{darkbeige}{HTML}{b8b09a}  % Darker beige 
\begin{tcolorbox}[
    colback=lightbeige,        % Light gray background
    colframe=darkbeige,         % Black border
    boxrule=0.8pt,          % Border thickness
    width=\textwidth,       % Full width of text
    title=PsychoCounsel Principles, % Title of the box
    fonttitle=\bfseries,    % Bold title
    coltitle=black,         % Title text color
    left=2mm,               % Padding on the left
    right=2mm,              % Padding on the right
    top=2mm,                % Padding at the top
    bottom=2mm,             % Padding at the bottom
    before skip=10pt,       % Space before the box
    after skip=10pt         % Space after the box
]

% \setstretch{0.9} % Reduce line spacing for compactness
\noindent\textbf{Empathy and Emotional Understanding}: The response should convey genuine empathy, acknowledging and validating the client’s feelings and experiences. 
\begin{itemize}
    \item Emotional Reflection: Reflecting the client’s emotions back to them.
    \item Validation: Affirming the client’s feelings as legitimate and understandable.
    \item Non-Judgmental Tone: Maintaining a compassionate and accepting approach.
\end{itemize}

\noindent\textbf{Personalization and Relevance}: The response should be tailored to the client’s unique situation, ensuring that the content is directly relevant to their concerns.
\begin{itemize}
    \item Specific References: Mentioning details specific to the client’s statements.
    \item Avoidance of Generic Responses: Steering clear of overly general or canned replies.
    \item Cultural and Individual Sensitivity: Respecting the client’s background and personal context.
\end{itemize}

\noindent\textbf{Facilitation of Self-Exploration}: The response should encourage the client to reflect on their thoughts and feelings, promoting self-awareness and insight.
\begin{itemize}
    \item Open-Ended Questions: Asking questions that invite elaboration.
    \item Reflective Statements: Paraphrasing the client’s words to deepen understanding.
    \item Exploration of Thoughts and Feelings: Guiding the client to consider underlying emotions and beliefs.
\end{itemize}

\noindent\textbf{Clarity and Conciseness}: The response should be clear, well-organized, and free of unnecessary jargon, making it easy for the client to understand and engage with.
\begin{itemize}
    \item Plain Language: Using words that are easily understood.
    \item Logical Flow: Presenting ideas in a coherent sequence.
    \item Brevity: Keeping the response concise while covering essential points.
\end{itemize}

\noindent\textbf{Promotion of Autonomy and Confidence}: The response should support the client’s sense of control over their decisions and encourage confidence in their ability to make positive changes.
\begin{itemize}
    \item Affirmation of Strengths: Highlighting the client’s abilities and past successes.
    \item Encouraging Initiative: Motivating the client to take proactive steps.
\end{itemize}

\noindent\textbf{Avoidance of Harmful Language or Content}: The response should avoid any language or content that could potentially harm, distress, or trigger the client, ensuring the interaction is safe and supportive.\\

\noindent\textbf{Sensitivity to the Stage of Change}: The response should recognize the client’s current stage in the process of change and address their needs accordingly. If the client is in an early stage—uncertain or ambivalent about making a change—the response should help them explore their thoughts and motivations. If the client is in a later stage and has already made changes, the response should focus on reinforcing progress, preventing setbacks, and sustaining positive 

\end{tcolorbox}\label{box:priciple}
\endgroup
\begin{table}[ht!]
\centering
\footnotesize
\begin{tabular}{ll}
\toprule
\textbf{Category} & \textbf{Models} \\
\midrule
\textbf{3-4B models} & Llama-3.2-3B-Instruct~\citep{Llama-Team2024-vb} \\
& Phi-3.5-mini-instruct~\citep{Abdin2024-gy} \\
& MiniCPM3-4B~\citep{Hu2024-lb} \\

\textbf{7-9B models} & Ministral-8B-Instruct-2410\tablefootnote{\url{https://mistral.ai/news/ministraux/}} \\
& Llama-3.1-8B-Instruct~\citep{Llama-Team2024-vb} \\
& gemma-2-9b-it~\citep{Gemma-Team2024-jc} \\
& Qwen2.5-7B-Instruct~\citep{Qwen2024-mx} \\
& OLMo-7B-0724-Instruct~\citep{Muennighoff2024-ba} \\
& Baichuan2-7B-Chat~\citep{Yang2023-jw} \\

\textbf{12-14B models} & Baichuan2-13B-Chat~\citep{Yang2023-jw} \\
& Orion-14B-Chat~\citep{Chen2024-jt} \\
& Mistral-Nemo-Instruct-2407\tablefootnote{\url{https://mistral.ai/news/mistral-nemo/}} \\
& AI21-Jamba-1.5-Mini~\citep{Jamba-Team2024-vt} \\

\textbf{65-75B models} & Llama-3.1-70B-Instruct~\citep{Llama-Team2024-vb} \\
& Qwen2.5-72B-Instruct~\citep{Qwen2024-mx} \\
& deepseek-llm-67b-chat~\citep{DeepSeek-AI2024-nq} \\

\textbf{Commercial models} & GPT-4o~\citep{OpenAI2024-rv} \\
& GPT-4o-mini~\citep{OpenAI2024-rv} \\
& o1-mini\tablefootnote{\url{https://openai.com/index/openai-o1-system-card/}} \\
& Cohere-command-r-08-2024\tablefootnote{\url{https://docs.cohere.com/docs/responsible-use}} \\

\bottomrule
\end{tabular}
\caption{Overview of selected models in the pool.}
\label{tab:model-pool}
\end{table}

\begin{figure}[htbp]
    \centering
    \includegraphics[width=\textwidth]{figures/dist-chosen-models.pdf}
    \caption{Chosen Model Distribution}
    \label{fig:response-dist-chosen}
\end{figure}

\begin{figure}[htbp]
    \centering
    \includegraphics[width=\textwidth]{figures/dist-rejected-models.pdf}
    \caption{Rejected Model Distribution}
    \label{fig:response-dist-rejected}
\end{figure}



\clearpage
\begin{table}[ht]
\centering
\footnotesize
% \resizebox{!}{0.9\textwidth}{
\begin{tabular}{p{0.95\textwidth}}
\toprule
\textbf{Client Speech} \\
\midrule
Anxiety often strikes when I'm faced with making decisions. The fear of making the wrong choice or disappointing others paralyzes me, leaving me stuck in indecision. I want to learn how to trust myself and make confident choices.\\
\midrule
\textbf{\policyl{}} (chosen) \\
\midrule
It takes a lot of courage to share something so vulnerable with me, and I appreciate your trust in our conversation. I can sense the weight of that anxiety you're carrying, and we're going to explore ways to shift that together.

First, let's acknowledge that the fear of making the wrong choice or disappointing others is a common and normal feeling. It doesn't mean there's anything inherently wrong with you; it just means you care deeply about others and yourself. That's a beautiful quality to have.

Can you tell me more about what you think would happen if you made a "wrong" choice? What are some of the worst-case scenarios that come to mind? Sometimes, exploring these fears can help us gain clarity and perspective. We can work together to reframe those thoughts in a more realistic and compassionate light.

Let's also explore the idea of self-trust. What does it mean to you to trust yourself? What does self-trust look like in action? We can work on building that trust by identifying what you value and what aligns with your goals. We can practice recognizing when your thoughts are trying to control or second-guess you, and we can develop strategies to gently acknowledge those thoughts without letting them overwhelm you.

I'd like to introduce an exercise called "3-4-5." When you're faced with a difficult decision, I want you to take a deep breath and pause for 3 seconds, acknowledge the thought or feeling that arises, and then ask yourself: "Is this thought based on facts or perceived facts?" Sometimes, when we're anxious, our thoughts can get tangled up in assumptions or past experiences. Then, take 4 seconds to ask yourself what you truly value in this situation, what's most important to you. And finally, take 5 seconds to imagine yourself moving forward, taking action, and making a choice that aligns with your values. This exercise can help you tap into your inner wisdom and develop a sense of self-trust.

Lastly, let's work on reframing the idea of "wrong" choices. What if, instead of focusing on the fear of making the wrong choice, we reframe it as a learning opportunity? What if you saw every choice as a chance to learn, grow, and develop? How might that change your approach to decision-making?

Remember, our goal isn't to eliminate anxiety completely, but to develop a more compassionate relationship with it. We'll work together to cultivate self-awareness, self-acceptance, and trust in your decision-making abilities.

How do these ideas resonate with you? What feels most helpful or interesting to explore further? \\
\midrule
\textbf{GPT-4o} (rejected) \\
\midrule
Thank you for sharing that with me. It's understandable that making decisions can be overwhelming, especially when there's fear of making the wrong choice or disappointing others. Let's explore this together.

Firstly, it's important to acknowledge that decision-making can be challenging and it's completely normal to feel anxious about it. You're not alone in experiencing this.

One way we might start working on this is by exploring some of the underlying thoughts or beliefs that contribute to your anxiety about decision-making. For instance, you mentioned a fear of disappointing others. Can you tell me more about where you think that fear comes from?

We could also look at ways to build your confidence in your decision-making skills. Sometimes breaking down decisions into smaller, more manageable parts can make the process feel less daunting. Have you ever tried writing down the pros and cons of each option or talking it through with someone you trust?

Another technique is to practice self-compassion. Remind yourself that no one makes the right choice all the time and that it's okay to make mistakes. Every decision is a learning opportunity.

Finally, something that can be quite helpful is examining past decisions you've made. Often, we are harsher with ourselves than necessary and we overlook our strengths. Can you recall a situation in the past where you made a decision and things turned out well, or you learned something valuable from the outcome?

Let’s take some time to explore these aspects and see what might resonate most with you. How does that sound? \\

\bottomrule
\end{tabular}
% }
\caption{Case Study - 2}
\label{tab:case2}
\end{table}



\end{document}
