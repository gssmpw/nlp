
\documentclass{article} % For LaTeX2e
\usepackage[preprint]{colm2025_conference}

\usepackage{microtype}
\usepackage{hyperref}
\usepackage{url}
\usepackage{booktabs}

\usepackage{lineno}
\usepackage{graphicx}

\usepackage{multirow}
\usepackage{xspace}
\usepackage{adjustbox}
\usepackage{pifont}
\usepackage{caption}
\usepackage{makecell}
\usepackage{subcaption}
\usepackage{hyperref}
\usepackage{tcolorbox}
\usepackage{lipsum} % for placeholder text
\usepackage{amsmath}
\usepackage{tablefootnote}

\definecolor{darkblue}{rgb}{0, 0, 0.5}
\hypersetup{colorlinks=true, citecolor=darkblue, linkcolor=darkblue, urlcolor=darkblue}

\newcommand{\zhiyu}[1]{\textcolor{red}{[zhiyu: #1]}}
\newcommand{\mian}[1]{\textcolor{blue}{[Mian: #1]}}

\title{Preference Learning Unlocks LLMs' Psycho-Counseling Skills}

% Aligning LLMs to Respond in Psychotherapy
% Authors must not appear in the submitted version. They should be hidden
% as long as the \colmfinalcopy macro remains commented out below.
% Non-anonymous submissions will be rejected without review.

\author{Mian Zhang$^{\alpha}$, Shaun M. Eack$^{\beta}$, Zhiyu Zoey Chen$^{\alpha}$ \\
$^{\alpha}$Department of Computer Science, University of Texas at Dallas\\ $^{\beta}$School of Social Work, University of Pittsburgh \\
\texttt{\{mian.zhang, zhiyu.chen2\}@utdallas.edu}
% Antiquus S.~Hippocampus, Natalia Cerebro \& Amelie P. Amygdale \thanks{ Use footnote for providing further information
% about author (webpage, alternative address)---\emph{not} for acknowledging
% funding agencies.  Funding acknowledgements go at the end of the paper.} \\
% Department of Computer Science\\
% Cranberry-Lemon University\\
% Pittsburgh, PA 15213, USA \\
% \texttt{\{hippo,brain,jen\}@cs.cranberry-lemon.edu} \\
% \And
% Ji Q. Ren \& Yevgeny LeNet \\
% Department of Computational Neuroscience \\
% University of the Witwatersrand \\
% Joburg, South Africa \\
% \texttt{\{robot,net\}@wits.ac.za} \\
% \And 
}

% The \author macro works with any number of authors. There are two commands
% used to separate the names and addresses of multiple authors: \And and \AND.
%
% Using \And between authors leaves it to \LaTeX{} to determine where to break
% the lines. Using \AND forces a linebreak at that point. So, if \LaTeX{}
% puts 3 of 4 authors names on the first line, and the last on the second
% line, try using \AND instead of \And before the third author name.

\newcommand{\fix}{\marginpar{FIX}}
\newcommand{\new}{\marginpar{NEW}}
\newcommand{\huggingface}{\raisebox{-1.5pt}{\includegraphics[height=1.05em]{figures/hf-logo.pdf}}\xspace}
\newcommand{\dataset}{\texttt{PsychoCounsel-Preference}}
\newcommand{\rewards}{\texttt{PsychoCounsel-Llama3-3B-Reward}}
\newcommand{\rewardl}{\texttt{PsychoCounsel-Llama3-8B-Reward}}
\newcommand{\policys}{\texttt{PsychoCounsel-Llama3-3B}}
\newcommand{\policyl}{\texttt{PsychoCounsel-Llama3-8B}}

\definecolor{darkred}{RGB}{139,0,0}
\newcommand{\red}[1]{\textcolor{red}{#1}}
\newcommand{\blue}[1]{\textcolor{blue}{#1}}
\newcommand{\darkred}[1]{\textcolor{darkred}{#1}}


\begin{document}
\ifcolmsubmission
\linenumbers
\fi
\maketitle

\begin{abstract}
Applying large language models (LLMs) to assist in psycho-counseling is an emerging and meaningful approach, driven by the significant gap between patient needs and the availability of mental health support. However, current LLMs struggle to consistently provide effective responses to client speeches, largely due to the lack of supervision from high-quality real psycho-counseling data, whose content is typically inaccessible due to client privacy concerns. Furthermore, the quality of therapists’ responses in available sessions can vary significantly based on their professional training and experience. Assessing the quality of therapists’ responses remains an open challenge.
In this work, we address these challenges by first proposing a set of professional and comprehensive principles to evaluate therapists’ responses to client speeches. Using these principles, we create a preference dataset, \dataset{}, which contains 36k high-quality preference comparison pairs. This dataset aligns with the preferences of professional psychotherapists, providing a robust foundation for evaluating and improving LLMs in psycho-counseling.
Experiments on reward modeling and preference learning demonstrate that \dataset{} is an excellent resource for LLMs to acquire essential skills for responding to clients in a counseling session. Our best-aligned model, \policyl{}, achieves an impressive win rate of 87\% against GPT-4o. We release \dataset{}, \policyl{} and the reward model \rewardl{} to facilitate the research of psycho-counseling with LLMs at:



\end{abstract}

% \vspace{6pt}
\begin{center}
\begin{tabular}{cll}
\huggingface & \textbf{Dataset \& Models} & \url{https://hf.co/Psychotherapy-LLM}

\end{tabular}
\end{center}


\section{Introduction}
\label{sec:introduction}
The business processes of organizations are experiencing ever-increasing complexity due to the large amount of data, high number of users, and high-tech devices involved \cite{martin2021pmopportunitieschallenges, beerepoot2023biggestbpmproblems}. This complexity may cause business processes to deviate from normal control flow due to unforeseen and disruptive anomalies \cite{adams2023proceddsriftdetection}. These control-flow anomalies manifest as unknown, skipped, and wrongly-ordered activities in the traces of event logs monitored from the execution of business processes \cite{ko2023adsystematicreview}. For the sake of clarity, let us consider an illustrative example of such anomalies. Figure \ref{FP_ANOMALIES} shows a so-called event log footprint, which captures the control flow relations of four activities of a hypothetical event log. In particular, this footprint captures the control-flow relations between activities \texttt{a}, \texttt{b}, \texttt{c} and \texttt{d}. These are the causal ($\rightarrow$) relation, concurrent ($\parallel$) relation, and other ($\#$) relations such as exclusivity or non-local dependency \cite{aalst2022pmhandbook}. In addition, on the right are six traces, of which five exhibit skipped, wrongly-ordered and unknown control-flow anomalies. For example, $\langle$\texttt{a b d}$\rangle$ has a skipped activity, which is \texttt{c}. Because of this skipped activity, the control-flow relation \texttt{b}$\,\#\,$\texttt{d} is violated, since \texttt{d} directly follows \texttt{b} in the anomalous trace.
\begin{figure}[!t]
\centering
\includegraphics[width=0.9\columnwidth]{images/FP_ANOMALIES.png}
\caption{An example event log footprint with six traces, of which five exhibit control-flow anomalies.}
\label{FP_ANOMALIES}
\end{figure}

\subsection{Control-flow anomaly detection}
Control-flow anomaly detection techniques aim to characterize the normal control flow from event logs and verify whether these deviations occur in new event logs \cite{ko2023adsystematicreview}. To develop control-flow anomaly detection techniques, \revision{process mining} has seen widespread adoption owing to process discovery and \revision{conformance checking}. On the one hand, process discovery is a set of algorithms that encode control-flow relations as a set of model elements and constraints according to a given modeling formalism \cite{aalst2022pmhandbook}; hereafter, we refer to the Petri net, a widespread modeling formalism. On the other hand, \revision{conformance checking} is an explainable set of algorithms that allows linking any deviations with the reference Petri net and providing the fitness measure, namely a measure of how much the Petri net fits the new event log \cite{aalst2022pmhandbook}. Many control-flow anomaly detection techniques based on \revision{conformance checking} (hereafter, \revision{conformance checking}-based techniques) use the fitness measure to determine whether an event log is anomalous \cite{bezerra2009pmad, bezerra2013adlogspais, myers2018icsadpm, pecchia2020applicationfailuresanalysispm}. 

The scientific literature also includes many \revision{conformance checking}-independent techniques for control-flow anomaly detection that combine specific types of trace encodings with machine/deep learning \cite{ko2023adsystematicreview, tavares2023pmtraceencoding}. Whereas these techniques are very effective, their explainability is challenging due to both the type of trace encoding employed and the machine/deep learning model used \cite{rawal2022trustworthyaiadvances,li2023explainablead}. Hence, in the following, we focus on the shortcomings of \revision{conformance checking}-based techniques to investigate whether it is possible to support the development of competitive control-flow anomaly detection techniques while maintaining the explainable nature of \revision{conformance checking}.
\begin{figure}[!t]
\centering
\includegraphics[width=\columnwidth]{images/HIGH_LEVEL_VIEW.png}
\caption{A high-level view of the proposed framework for combining \revision{process mining}-based feature extraction with dimensionality reduction for control-flow anomaly detection.}
\label{HIGH_LEVEL_VIEW}
\end{figure}

\subsection{Shortcomings of \revision{conformance checking}-based techniques}
Unfortunately, the detection effectiveness of \revision{conformance checking}-based techniques is affected by noisy data and low-quality Petri nets, which may be due to human errors in the modeling process or representational bias of process discovery algorithms \cite{bezerra2013adlogspais, pecchia2020applicationfailuresanalysispm, aalst2016pm}. Specifically, on the one hand, noisy data may introduce infrequent and deceptive control-flow relations that may result in inconsistent fitness measures, whereas, on the other hand, checking event logs against a low-quality Petri net could lead to an unreliable distribution of fitness measures. Nonetheless, such Petri nets can still be used as references to obtain insightful information for \revision{process mining}-based feature extraction, supporting the development of competitive and explainable \revision{conformance checking}-based techniques for control-flow anomaly detection despite the problems above. For example, a few works outline that token-based \revision{conformance checking} can be used for \revision{process mining}-based feature extraction to build tabular data and develop effective \revision{conformance checking}-based techniques for control-flow anomaly detection \cite{singh2022lapmsh, debenedictis2023dtadiiot}. However, to the best of our knowledge, the scientific literature lacks a structured proposal for \revision{process mining}-based feature extraction using the state-of-the-art \revision{conformance checking} variant, namely alignment-based \revision{conformance checking}.

\subsection{Contributions}
We propose a novel \revision{process mining}-based feature extraction approach with alignment-based \revision{conformance checking}. This variant aligns the deviating control flow with a reference Petri net; the resulting alignment can be inspected to extract additional statistics such as the number of times a given activity caused mismatches \cite{aalst2022pmhandbook}. We integrate this approach into a flexible and explainable framework for developing techniques for control-flow anomaly detection. The framework combines \revision{process mining}-based feature extraction and dimensionality reduction to handle high-dimensional feature sets, achieve detection effectiveness, and support explainability. Notably, in addition to our proposed \revision{process mining}-based feature extraction approach, the framework allows employing other approaches, enabling a fair comparison of multiple \revision{conformance checking}-based and \revision{conformance checking}-independent techniques for control-flow anomaly detection. Figure \ref{HIGH_LEVEL_VIEW} shows a high-level view of the framework. Business processes are monitored, and event logs obtained from the database of information systems. Subsequently, \revision{process mining}-based feature extraction is applied to these event logs and tabular data input to dimensionality reduction to identify control-flow anomalies. We apply several \revision{conformance checking}-based and \revision{conformance checking}-independent framework techniques to publicly available datasets, simulated data of a case study from railways, and real-world data of a case study from healthcare. We show that the framework techniques implementing our approach outperform the baseline \revision{conformance checking}-based techniques while maintaining the explainable nature of \revision{conformance checking}.

In summary, the contributions of this paper are as follows.
\begin{itemize}
    \item{
        A novel \revision{process mining}-based feature extraction approach to support the development of competitive and explainable \revision{conformance checking}-based techniques for control-flow anomaly detection.
    }
    \item{
        A flexible and explainable framework for developing techniques for control-flow anomaly detection using \revision{process mining}-based feature extraction and dimensionality reduction.
    }
    \item{
        Application to synthetic and real-world datasets of several \revision{conformance checking}-based and \revision{conformance checking}-independent framework techniques, evaluating their detection effectiveness and explainability.
    }
\end{itemize}

The rest of the paper is organized as follows.
\begin{itemize}
    \item Section \ref{sec:related_work} reviews the existing techniques for control-flow anomaly detection, categorizing them into \revision{conformance checking}-based and \revision{conformance checking}-independent techniques.
    \item Section \ref{sec:abccfe} provides the preliminaries of \revision{process mining} to establish the notation used throughout the paper, and delves into the details of the proposed \revision{process mining}-based feature extraction approach with alignment-based \revision{conformance checking}.
    \item Section \ref{sec:framework} describes the framework for developing \revision{conformance checking}-based and \revision{conformance checking}-independent techniques for control-flow anomaly detection that combine \revision{process mining}-based feature extraction and dimensionality reduction.
    \item Section \ref{sec:evaluation} presents the experiments conducted with multiple framework and baseline techniques using data from publicly available datasets and case studies.
    \item Section \ref{sec:conclusions} draws the conclusions and presents future work.
\end{itemize}

%\section{Related Work}
%\label{sec:related-work}

%\subsection{Background}

%Defect detection is critical to ensure the yield of integrated circuit manufacturing lines and reduce faults. Previous research has primarily focused on wafer map data, which engineers produce by marking faulty chips with different colors based on test results. The specific spatial distribution of defects on a wafer can provide insights into the causes, thereby helping to determine which stage of the manufacturing process is responsible for the issues. Although such research is relatively mature, the continual miniaturization of integrated circuits and the increasing complexity and density of chip components have made chip-level detection more challenging, leading to potential risks\cite{ma2023review}. Consequently, there is a need to combine this approach with magnified imaging of the wafer surface using scanning electron microscopes (SEMs) to detect, classify, and analyze specific microscopic defects, thus helping to identify the particular process steps where defects originate.

%Previously, wafer surface defect classification and detection were primarily conducted by experienced engineers. However, this method relies heavily on the engineers' expertise and involves significant time expenditure and subjectivity, lacking uniform standards. With the ongoing development of artificial intelligence, deep learning methods using multi-layer neural networks to extract and learn target features have proven highly effective for this task\cite{gao2022review}.

%In the task of defect classification, it is typical to use a model structure that initially extracts features through convolutional and pooling layers, followed by classification via fully connected layers. Researchers have recently developed numerous classification model structures tailored to specific problems. These models primarily focus on how to extract defect features effectively. For instance, Chen et al. presented a defect recognition and classification algorithm rooted in PCA and classification SVM\cite{chen2008defect}. Chang et al. utilized SVM, drawing on features like smoothness and texture intricacy, for classifying high-intensity defect images\cite{chang2013hybrid}. The classification of defect images requires the formulation of numerous classifiers tailored for myriad inspection steps and an Abundance of accurately labeled data, making data acquisition challenging. Cheon et al. proposed a single CNN model adept at feature extraction\cite{cheon2019convolutional}. They achieved a granular classification of wafer surface defects by recognizing misclassified images and employing a k-nearest neighbors (k-NN) classifier algorithm to gauge the aggregate squared distance between each image feature vector and its k-neighbors within the same category. However, when applied to new or unseen defects, such models necessitate retraining, incurring computational overheads. Moreover, with escalating CNN complexity, the computational demands surge.

%Segmentation of defects is necessary to locate defect positions and gather information such as the size of defects. Unlike classification networks, segmentation networks often use classic encoder-decoder structures such as UNet\cite{ronneberger2015u} and SegNet\cite{badrinarayanan2017segnet}, which focus on effectively leveraging both local and global feature information. Han Hui et al. proposed integrating a Region Proposal Network (RPN) with a UNet architecture to suggest defect areas before conducting defect segmentation \cite{han2020polycrystalline}. This approach enables the segmentation of various defects in wafers with only a limited set of roughly labeled images, enhancing the efficiency of training and application in environments where detailed annotations are scarce. Subhrajit Nag et al. introduced a new network structure, WaferSegClassNet, which extracts multi-scale local features in the encoder and performs classification and segmentation tasks in the decoder \cite{nag2022wafersegclassnet}. This model represents the first detection system capable of simultaneously classifying and segmenting surface defects on wafers. However, it relies on extensive data training and annotation for high accuracy and reliability. 

%Recently, Vic De Ridder et al. introduced a novel approach for defect segmentation using diffusion models\cite{de2023semi}. This approach treats the instance segmentation task as a denoising process from noise to a filter, utilizing diffusion models to predict and reconstruct instance masks for semiconductor defects. This method achieves high precision and improved defect classification and segmentation detection performance. However, the complex network structure and the computational process of the diffusion model require substantial computational resources. Moreover, the performance of this model heavily relies on high-quality and large amounts of training data. These issues make it less suitable for industrial applications. Additionally, the model has only been applied to detecting and segmenting a single type of defect(bridges) following a specific manufacturing process step, limiting its practical utility in diverse industrial scenarios.

%\subsection{Few-shot Anomaly Detection}
%Traditional anomaly detection techniques typically rely on extensive training data to train models for identifying and locating anomalies. However, these methods often face limitations in rapidly changing production environments and diverse anomaly types. Recent research has started exploring effective anomaly detection using few or zero samples to address these challenges.

%Huang et al. developed the anomaly detection method RegAD, based on image registration technology. This method pre-trains an object-agnostic registration network with various images to establish the normality of unseen objects. It identifies anomalies by aligning image features and has achieved promising results. Despite these advancements, implementing few-shot settings in anomaly detection remains an area ripe for further exploration. Recent studies show that pre-trained vision-language models such as CLIP and MiniGPT can significantly enhance performance in anomaly detection tasks.

%Dong et al. introduced the MaskCLIP framework, which employs masked self-distillation to enhance contrastive language-image pretraining\cite{zhou2022maskclip}. This approach strengthens the visual encoder's learning of local image patches and uses indirect language supervision to enhance semantic understanding. It significantly improves transferability and pretraining outcomes across various visual tasks, although it requires substantial computational resources.
%Jeong et al. crafted the WinCLIP framework by integrating state words and prompt templates to characterize normal and anomalous states more accurately\cite{Jeong_2023_CVPR}. This framework introduces a novel window-based technique for extracting and aggregating multi-scale spatial features, significantly boosting the anomaly detection performance of the pre-trained CLIP model.
%Subsequently, Li et al. have further contributed to the field by creating a new expansive multimodal model named Myriad\cite{li2023myriad}. This model, which incorporates a pre-trained Industrial Anomaly Detection (IAD) model to act as a vision expert, embeds anomaly images as tokens interpretable by the language model, thus providing both detailed descriptions and accurate anomaly detection capabilities.
%Recently, Chen et al. introduced CLIP-AD\cite{chen2023clip}, and Li et al. proposed PromptAD\cite{li2024promptad}, both employing language-guided, tiered dual-path model structures and feature manipulation strategies. These approaches effectively address issues encountered when directly calculating anomaly maps using the CLIP model, such as reversed predictions and highlighting irrelevant areas. Specifically, CLIP-AD optimizes the utilization of multi-layer features, corrects feature misalignment, and enhances model performance through additional linear layer fine-tuning. PromptAD connects normal prompts with anomaly suffixes to form anomaly prompts, enabling contrastive learning in a single-class setting.

%These studies extend the boundaries of traditional anomaly detection techniques and demonstrate how to effectively address rapidly changing and sample-scarce production environments through the synergy of few-shot learning and deep learning models. Building on this foundation, our research further explores wafer surface defect detection based on the CLIP model, especially focusing on achieving efficient and accurate anomaly detection in the highly specialized and variable semiconductor manufacturing process using a minimal amount of labeled data.

\section{\dataset{}}

\begin{figure}[htbp]
    \centering
    \includegraphics[width=\textwidth]{figures/pipline-cropped.pdf}
    \caption{\dataset{} Construction Pipeline. 1) We first collect over 26k client speeches covering a wide range of topics from various sources, applying necessary data cleaning. 2) 20 popular LLMs are sampled and prompted to roleplay as psychotherapists and give responses to these client speeches. 3) GPT-4o is instructed to evaluate the responses based on our proposed PsychoCounsel Principles, and preference pairs with substantial score gaps are incorporated into \dataset{}.}
    \label{fig:pipline}
\end{figure}

\subsection{Client Speech Collection}
We collect client speeches from various data sources: counsel-chat\footnote{\url{https://github.com/nbertagnolli/counsel-chat}}, MentalAgora~\citep{Lee2024-ly}, TherapistQA~\citep{Shreevastava2021-eh}, Psycho8k~\citep{Liu2023-ub}, and several huggingface datasets (amod-counsel\footnote{\url{https://huggingface.co/datasets/Amod/mental_health_counseling_conversations}}, MentalChat16K\footnote{\url{https://huggingface.co/datasets/ShenLab/MentalChat16K}}, and phi2Mental\footnote{\url{https://huggingface.co/datasets/saxenaindresh681/microsoft-phi2-mental-health}}). Client speeches with number of characters more than 1,000 and less than 100 are discarded to ensure a proper length of context. After an additional step of de-duplication, the resulting data contains 26,483 client speeches with average length of 366 characters covering a wide range of topics including 8 coarse topics: Core Mental Health Issues (9,054), Emotional Well-being and Coping Strategies (5,717), Relationships and Interpersonal Dynamics (6,483), Life Transitions and Challenges (934), Social Issues (667), Youth and Development (1,175), Crisis and Safety Concerns (529) and Special Topics (1,924). Under these 8 topics are 42 fine-grained topics (see Table~\ref{tab:topic_distributions} in the appendix for the detailed topic distribution).

\subsection{PsychoCounsel Principles}\label{sec:principle}
To answer the question \textit{what is a good response to a client speech in psycho-counseling}, we collaborate with professors in social work and psychiatry (our co-authors) and propose a set of professional principles to measure the response to a client speech from seven different dimensions: \begingroup
% Box for terms and concepts
\definecolor{lightbeige}{HTML}{eceae0} % Light beige color
\definecolor{darkbeige}{HTML}{b8b09a}  % Darker beige 
\definecolor{darkgray}{HTML}{696969}
\definecolor{gainsboro}{HTML}{DCDCDC}
\begin{tcolorbox}[
    colback=gainsboro,        % Light gray background
    colframe=darkgray,         % Black border
    boxrule=0.8pt,          % Border thickness
    width=\textwidth,       % Full width of text
    title=\textcolor{white}{PsychoCounsel Principles}, % Title of the box
    fonttitle=\bfseries,    % Bold title
    coltitle=black,         % Title text color
    left=2mm,               % Padding on the left
    right=2mm,              % Padding on the right
    top=2mm,                % Padding at the top
    bottom=2mm,             % Padding at the bottom
    before skip=10pt,       % Space before the box
    after skip=10pt         % Space after the box
]

% \setstretch{0.9} % Reduce line spacing for compactness
\noindent\textbf{Empathy and Emotional Understanding}: The response should convey genuine empathy, acknowledging and validating the client’s feelings and experiences. \\
\noindent\textbf{Personalization and Relevance}: The response should be tailored to the client’s unique situation, ensuring that the content is directly relevant to their concerns.\\
\noindent\textbf{Clarity and Conciseness}: The response should be clear, well-organized, and free of unnecessary jargon, making it easy for the client to understand and engage with.\\
\noindent\textbf{Avoidance of Harmful Language or Content}: The response should avoid any language or content that could potentially harm, distress, or trigger the client, ensuring the interaction is safe and supportive.\\
\noindent\textbf{Facilitation of Self-Exploration}: The response should encourage the client to reflect on their thoughts and feelings, promoting self-awareness and insight.\\
\noindent\textbf{Promotion of Autonomy and Confidence}: The response should support the client’s sense of control over their decisions and encourage confidence in their ability to make positive changes.\\
\noindent\textbf{Sensitivity to the Stage of Change}: The response should recognize the client’s current stage in the process of change and address their needs accordingly. If the client is in an early stage—uncertain or ambivalent about making a change—the response should help them explore their thoughts and motivations. If the client is in a later stage and has already made changes, the response should focus on reinforcing progress, preventing setbacks, and sustaining positive outcomes.
\end{tcolorbox}
\endgroup

Please refer to Box~\ref{box:priciple} for the complete definition of the principles. Among these seven principles, \textbf{Facilitation of Self-Exploration}, \textbf{Promotion of Autonomy and Confidence}, and \textbf{Identifying Stages and Reasons for Change} emphasize a client-centered approach, which is recognized as a hallmark of effective psycho-counseling~\citep{Miller-and-Stephen-RollnickUnknown-oo}. We use these three principles to measure the \textit{effectiveness} of a response to a client speech, complementary to the other four principles, which are more basic, requiring the response to be \textit{empathy}, \textit{relevant}, \textit{concise}, and \textit{safe}. Evaluating therapist responses using these fine-grained principles provides a more structured and nuanced assessment of their effectiveness. Unlike general evaluations that focus solely on overall quality, this detailed approach allows for a deeper understanding of how well a response supports the client’s emotional and psychological needs.

\subsection{Preference Generation}
We apply the generate-score-pair pipeline as ~\cite{Cui2023-ui} to construct the \dataset{} dataset. For each client speech, we randomly sample four off-the-shelf LLMs from a model pool to give the response and instruct GPT-4o to annotate each response with 5-Likert scores for each principle defined in Section~\ref{sec:principle}; higher scores mean more alignment with the principles. Then scores of the principles are averaged to get the overall score for a response and preference pairs are generated based on the overall scores. The whole pipeline is illustrated in Figure~\ref{fig:pipline}. To increase the diversity of the model responses, we initialize the model pool with 20 popular LLMs of a range of sizes developed by different organizations shown in Table~\ref{tab:model-pool}. We also include LLMs with different architectures other than pure transformers like AI21-Jamba-1.5-Mini~\citep{Jamba-Team2024-vt}, which is a hybrid transformer-mamba model. We randomly held out 3,291 client speeches for testing and the remaining 23,192 for training. After obtaining the scores of principles, for training, we extract response pairs with the overall score gap larger than or equal to 1 as the preference pairs, and for testing, we only extract the ([highest score response], [lowest score response]) pairs and pairs with the score gap less than 1 are discarded. In this way, we could exclude response pairs with similar scores, whose quality may be hard to differentiate.

Ultimately, \dataset{} includes 34,329 training preference pairs and 2,324 testing pairs. The models most likely to be chosen and those most likely to be rejected vary significantly in size (see Figures~\ref{fig:response-dist-chosen} and~\ref{fig:response-dist-rejected} for the distributions of chosen and rejected models). This suggests that simply scaling model size is not a decisive factor in making LLMs effective responders in psycho-counseling. We also observe that LLMs developed by non-English-speaking institutions are more likely to be rejected compared to those from English-speaking countries. This may suggest that non-English-speaking institutions have a greater need to enhance the capabilities of LLMs in their respective languages, potentially leading to less emphasis on developing psycho-counseling skills in English.


\subsection{Preference Validation}
To validate the quality of synthetic human preferences in \dataset{}, we hired two professional psychotherapists through Upwork\footnote{\url{https://www.upwork.com/}} and instructed them to annotate preferences based on each principle and give the overall preference. The annotation set consists of 200 preference pairs randomly sampled from \dataset{}. The two therapists agree on 174 out of 200 samples. Additionally, one expert’s annotations align with the preference labels in \dataset{} for 184 out of 200 samples, while the other aligns for 170 out of 200 samples. These results indicate a high level of agreement between the experts (87\%) and demonstrate strong alignment between the expert annotations and the preference labels in \dataset{} (88.5\%). This strongly suggests that the labels in \dataset{} are reliable and trustworthy.




\section{Experiments}
\label{sec:exp}
Following the settings in Section \ref{sec:existing}, we evaluate \textit{NovelSum}'s correlation with the fine-tuned model performance across 53 IT datasets and compare it with previous diversity metrics. Additionally, we conduct a correlation analysis using Qwen-2.5-7B \cite{yang2024qwen2} as the backbone model, alongside previous LLaMA-3-8B experiments, to further demonstrate the metric's effectiveness across different scenarios. Qwen is used for both instruction tuning and deriving semantic embeddings. Due to resource constraints, we run each strategy on Qwen for two rounds, resulting in 25 datasets. 

\subsection{Main Results}

\begin{table*}[!t]
    \centering
    \resizebox{\linewidth}{!}{
    \begin{tabular}{lcccccccccc}
    \toprule
    \multirow{3}*{\textbf{Diversity Metrics}} & \multicolumn{10}{c}{\textbf{Data Selection Strategies}} \\
    \cmidrule(lr){2-11}
    & \multirow{2}*{\textbf{K-means}} & \multirow{2}*{\vtop{\hbox{\textbf{K-Center}}\vspace{1mm}\hbox{\textbf{-Greedy}}}}  & \multirow{2}*{\textbf{QDIT}} & \multirow{2}*{\vtop{\hbox{\textbf{Repr}}\vspace{1mm}\hbox{\textbf{Filter}}}} & \multicolumn{5}{c}{\textbf{Random}} & \multirow{2}{*}{\textbf{Duplicate}} \\ 
    \cmidrule(lr){6-10}
    & & & & & \textbf{$\mathcal{X}^{all}$} & ShareGPT & WizardLM & Alpaca & Dolly &  \\
    \midrule
    \rowcolor{gray!15} \multicolumn{11}{c}{\textit{LLaMA-3-8B}} \\
    Facility Loc. $_{\times10^5}$ & \cellcolor{BLUE!40} 2.99 & \cellcolor{ORANGE!10} 2.73 & \cellcolor{BLUE!40} 2.99 & \cellcolor{BLUE!20} 2.86 & \cellcolor{BLUE!40} 2.99 & \cellcolor{BLUE!0} 2.83 & \cellcolor{BLUE!30} 2.88 & \cellcolor{BLUE!0} 2.83 & \cellcolor{ORANGE!20} 2.59 & \cellcolor{ORANGE!30} 2.52 \\    
    DistSum$_{cosine}$  & \cellcolor{BLUE!30} 0.648 & \cellcolor{BLUE!60} 0.746 & \cellcolor{BLUE!0} 0.629 & \cellcolor{BLUE!50} 0.703 & \cellcolor{BLUE!10} 0.634 & \cellcolor{BLUE!40} 0.656 & \cellcolor{ORANGE!30} 0.578 & \cellcolor{ORANGE!10} 0.605 & \cellcolor{ORANGE!20} 0.603 & \cellcolor{BLUE!10} 0.634 \\
    Vendi Score $_{\times10^7}$ & \cellcolor{BLUE!30} 1.70 & \cellcolor{BLUE!60} 2.53 & \cellcolor{BLUE!10} 1.59 & \cellcolor{BLUE!50} 2.23 & \cellcolor{BLUE!20} 1.61 & \cellcolor{BLUE!30} 1.70 & \cellcolor{ORANGE!10} 1.44 & \cellcolor{ORANGE!20} 1.32 & \cellcolor{ORANGE!10} 1.44 & \cellcolor{ORANGE!30} 0.05 \\
    \textbf{NovelSum (Ours)} & \cellcolor{BLUE!60} 0.693 & \cellcolor{BLUE!50} 0.687 & \cellcolor{BLUE!30} 0.673 & \cellcolor{BLUE!20} 0.671 & \cellcolor{BLUE!40} 0.675 & \cellcolor{BLUE!10} 0.628 & \cellcolor{BLUE!0} 0.591 & \cellcolor{ORANGE!10} 0.572 & \cellcolor{ORANGE!20} 0.50 & \cellcolor{ORANGE!30} 0.461 \\
    \midrule    
    \textbf{Model Performance} & \cellcolor{BLUE!60}1.32 & \cellcolor{BLUE!50}1.31 & \cellcolor{BLUE!40}1.25 & \cellcolor{BLUE!30}1.05 & \cellcolor{BLUE!20}1.20 & \cellcolor{BLUE!10}0.83 & \cellcolor{BLUE!0}0.72 & \cellcolor{ORANGE!10}0.07 & \cellcolor{ORANGE!20}-0.14 & \cellcolor{ORANGE!30}-1.35 \\
    \midrule
    \midrule
    \rowcolor{gray!15} \multicolumn{11}{c}{\textit{Qwen-2.5-7B}} \\
    Facility Loc. $_{\times10^5}$ & \cellcolor{BLUE!40} 3.54 & \cellcolor{ORANGE!30} 3.42 & \cellcolor{BLUE!40} 3.54 & \cellcolor{ORANGE!20} 3.46 & \cellcolor{BLUE!40} 3.54 & \cellcolor{BLUE!30} 3.51 & \cellcolor{BLUE!10} 3.50 & \cellcolor{BLUE!10} 3.50 & \cellcolor{ORANGE!20} 3.46 & \cellcolor{BLUE!0} 3.48 \\ 
    DistSum$_{cosine}$ & \cellcolor{BLUE!30} 0.260 & \cellcolor{BLUE!60} 0.440 & \cellcolor{BLUE!0} 0.223 & \cellcolor{BLUE!50} 0.421 & \cellcolor{BLUE!10} 0.230 & \cellcolor{BLUE!40} 0.285 & \cellcolor{ORANGE!20} 0.211 & \cellcolor{ORANGE!30} 0.189 & \cellcolor{ORANGE!10} 0.221 & \cellcolor{BLUE!20} 0.243 \\
    Vendi Score $_{\times10^6}$ & \cellcolor{ORANGE!10} 1.60 & \cellcolor{BLUE!40} 3.09 & \cellcolor{BLUE!10} 2.60 & \cellcolor{BLUE!60} 7.15 & \cellcolor{ORANGE!20} 1.41 & \cellcolor{BLUE!50} 3.36 & \cellcolor{BLUE!20} 2.65 & \cellcolor{BLUE!0} 1.89 & \cellcolor{BLUE!30} 3.04 & \cellcolor{ORANGE!30} 0.20 \\
    \textbf{NovelSum (Ours)}  & \cellcolor{BLUE!40} 0.440 & \cellcolor{BLUE!60} 0.505 & \cellcolor{BLUE!20} 0.403 & \cellcolor{BLUE!50} 0.495 & \cellcolor{BLUE!30} 0.408 & \cellcolor{BLUE!10} 0.392 & \cellcolor{BLUE!0} 0.349 & \cellcolor{ORANGE!10} 0.336 & \cellcolor{ORANGE!20} 0.320 & \cellcolor{ORANGE!30} 0.309 \\
    \midrule
    \textbf{Model Performance} & \cellcolor{BLUE!30} 1.06 & \cellcolor{BLUE!60} 1.45 & \cellcolor{BLUE!40} 1.23 & \cellcolor{BLUE!50} 1.35 & \cellcolor{BLUE!20} 0.87 & \cellcolor{BLUE!10} 0.07 & \cellcolor{BLUE!0} -0.08 & \cellcolor{ORANGE!10} -0.38 & \cellcolor{ORANGE!30} -0.49 & \cellcolor{ORANGE!20} -0.43 \\
    \bottomrule
    \end{tabular}
    }
    \caption{Measuring the diversity of datasets selected by different strategies using \textit{NovelSum} and baseline metrics. Fine-tuned model performances (Eq. \ref{eq:perf}), based on MT-bench and AlpacaEval, are also included for cross reference. Darker \colorbox{BLUE!60}{blue} shades indicate higher values for each metric, while darker \colorbox{ORANGE!30}{orange} shades indicate lower values. While data selection strategies vary in performance on LLaMA-3-8B and Qwen-2.5-7B, \textit{NovelSum} consistently shows a stronger correlation with model performance than other metrics. More results are provided in Appendix \ref{app:results}.}
    \label{tbl:main}
    \vspace{-4mm}
\end{table*}


\begin{table}[t!]
\centering
\resizebox{\linewidth}{!}{
\begin{tabular}{lcccc}
\toprule
\multirow{2}*{\textbf{Diversity Metrics}} & \multicolumn{3}{c}{\textbf{LLaMA}} & \textbf{Qwen}\\
\cmidrule(lr){2-4} \cmidrule(lr){5-5} 
& \textbf{Pearson} & \textbf{Spearman} & \textbf{Avg.} & \textbf{Avg.} \\
\midrule
TTR & -0.38 & -0.16 & -0.27 & -0.30 \\
vocd-D & -0.43 & -0.17 & -0.30 & -0.31 \\
\midrule
Facility Loc. & 0.86 & 0.69 & 0.77 & 0.08 \\
Entropy & 0.93 & 0.80 & 0.86 & 0.63 \\
\midrule
LDD & 0.61 & 0.75 & 0.68 & 0.60 \\
KNN Distance & 0.59 & 0.80 & 0.70 & 0.67 \\
DistSum$_{cosine}$ & 0.85 & 0.67 & 0.76 & 0.51 \\
Vendi Score & 0.70 & 0.85 & 0.78 & 0.60 \\
DistSum$_{L2}$ & 0.86 & 0.76 & 0.81 & 0.51 \\
Cluster Inertia & 0.81 & 0.85 & 0.83 & 0.76 \\
Radius & 0.87 & 0.81 & 0.84 & 0.48 \\
\midrule
NovelSum & \textbf{0.98} & \textbf{0.95} & \textbf{0.97} & \textbf{0.90} \\
\bottomrule
\end{tabular}
}
\caption{Correlations between different metrics and model performance on LLaMA-3-8B and Qwen-2.5-7B.  “Avg.” denotes the average correlation (Eq. \ref{eq:cor}).}
\label{tbl:correlations}
\vspace{-2mm}
\end{table}

\paragraph{\textit{NovelSum} consistently achieves state-of-the-art correlation with model performance across various data selection strategies, backbone LLMs, and correlation measures.}
Table \ref{tbl:main} presents diversity measurement results on datasets constructed by mainstream data selection methods (based on $\mathcal{X}^{all}$), random selection from various sources, and duplicated samples (with only $m=100$ unique samples). 
Results from multiple runs are averaged for each strategy.
Although these strategies yield varying performance rankings across base models, \textit{NovelSum} consistently tracks changes in IT performance by accurately measuring dataset diversity. For instance, K-means achieves the best performance on LLaMA with the highest NovelSum score, while K-Center-Greedy excels on Qwen, also correlating with the highest NovelSum. Table \ref{tbl:correlations} shows the correlation coefficients between various metrics and model performance for both LLaMA and Qwen experiments, where \textit{NovelSum} achieves state-of-the-art correlation across different models and measures.

\paragraph{\textit{NovelSum} can provide valuable guidance for data engineering practices.}
As a reliable indicator of data diversity, \textit{NovelSum} can assess diversity at both the dataset and sample levels, directly guiding data selection and construction decisions. For example, Table \ref{tbl:main} shows that the combined data source $\mathcal{X}^{all}$ is a better choice for sampling diverse IT data than other sources. Moreover, \textit{NovelSum} can offer insights through comparative analyses, such as: (1) ShareGPT, which collects data from real internet users, exhibits greater diversity than Dolly, which relies on company employees, suggesting that IT samples from diverse sources enhance dataset diversity \cite{wang2024diversity-logD}; (2) In LLaMA experiments, random selection can outperform some mainstream strategies, aligning with prior work \cite{xia2024rethinking,diddee2024chasing}, highlighting gaps in current data selection methods for optimizing diversity.



\subsection{Ablation Study}


\textit{NovelSum} involves several flexible hyperparameters and variations. In our main experiments, \textit{NovelSum} uses cosine distance to compute $d(x_i, x_j)$ in Eq. \ref{eq:dad}. We set $\alpha = 1$, $\beta = 0.5$, and $K = 10$ nearest neighbors in Eq. \ref{eq:pws} and \ref{eq:dad}. Here, we conduct an ablation study to investigate the impact of these settings based on LLaMA-3-8B.

\begin{table}[ht!]
\centering
\resizebox{\linewidth}{!}{
\begin{tabular}{lccc}
\toprule
\textbf{Variants} & \textbf{Pearson} & \textbf{Spearman} & \textbf{Avg.} \\
\midrule
NovelSum & 0.98 & 0.96 & 0.97 \\
\midrule
\hspace{0.10cm} - Use $L2$ distance & 0.97 & 0.83 & 0.90\textsubscript{↓ 0.08} \\
\hspace{0.10cm} - $K=20$ & 0.98 & 0.96 & 0.97\textsubscript{↓ 0.00} \\
\hspace{0.10cm} - $\alpha=0$ (w/o proximity) & 0.79 & 0.31 & 0.55\textsubscript{↓ 0.42} \\
\hspace{0.10cm} - $\alpha=2$ & 0.73 & 0.88 & 0.81\textsubscript{↓ 0.16} \\
\hspace{0.10cm} - $\beta=0$ (w/o density) & 0.92 & 0.89 & 0.91\textsubscript{↓ 0.07} \\
\hspace{0.10cm} - $\beta=1$ & 0.90 & 0.62 & 0.76\textsubscript{↓ 0.21} \\
\bottomrule
\end{tabular}
}
\caption{Ablation Study for \textit{NovelSum}.}
\label{tbl:ablation}
\vspace{-2mm}
\end{table}

In Table \ref{tbl:ablation}, $\alpha=0$ removes the proximity weights, and $\beta=0$ eliminates the density multiplier. We observe that both $\alpha=0$ and $\beta=0$ significantly weaken the correlation, validating the benefits of the proximity-weighted sum and density-aware distance. Additionally, improper values for $\alpha$ and $\beta$ greatly reduce the metric's reliability, highlighting that \textit{NovelSum} strikes a delicate balance between distances and distribution. Replacing cosine distance with Euclidean distance and using more neighbors for density approximation have minimal impact, particularly on Pearson's correlation, demonstrating \textit{NovelSum}'s robustness to different distance measures.






\section{Conclusion}
In this work, we propose a simple yet effective approach, called SMILE, for graph few-shot learning with fewer tasks. Specifically, we introduce a novel dual-level mixup strategy, including within-task and across-task mixup, for enriching the diversity of nodes within each task and the diversity of tasks. Also, we incorporate the degree-based prior information to learn expressive node embeddings. Theoretically, we prove that SMILE effectively enhances the model's generalization performance. Empirically, we conduct extensive experiments on multiple benchmarks and the results suggest that SMILE significantly outperforms other baselines, including both in-domain and cross-domain few-shot settings.
Our benchmark comprises contexts and instructions related to multiple real-world safety topics, aiming to provide comprehensive evaluation and support further improvement on LLM safety. However, some of the data may contain toxic contents that could pose potential risks if misused. To mitigate these risks, we plan to conduct careful inspections before open-sourcing the benchmark, and restrict data access to individuals who adhere to stringent ethical guidelines.

During the data collection procedure, we inform the crowd workers from China in advance of the possibility of encountering harmful contents and the future use of the annotated data. Participation of the workers is entirely voluntary, and they are free to withdraw from the study at any time without burden. We pay the workers about 8.5 USD per hour, which is above the average wage of local residents.



\bibliography{Mental-RM}
\bibliographystyle{colm2025_conference}

\appendix
\clearpage
\section{Prompts}
\begingroup
% Box for terms and concepts
\definecolor{lightbeige}{HTML}{eceae0} % Light beige color
\definecolor{darkbeige}{HTML}{b8b09a}  % Darker beige 
\begin{tcolorbox}[
    colback=lightbeige,        % Light gray background
    colframe=darkbeige,         % Black border
    boxrule=0.8pt,          % Border thickness
    width=\textwidth,       % Full width of text
    title=Rating Prompt, % Title of the box
    fonttitle=\bfseries,    % Bold title
    coltitle=black,         % Title text color
    left=2mm,               % Padding on the left
    right=2mm,              % Padding on the right
    top=2mm,                % Padding at the top
    bottom=2mm,             % Padding at the bottom
    before skip=10pt,       % Space before the box
    after skip=10pt         % Space after the box
]

You are provided with a client speech and four responses from different psychotherapists. Rate the responses based on how they align with the given principle.\\

Client Speech: \{client\_speech\}\\
Response 1: \{response1\}\\
Response 2: \{response2\}\\
Response 3: \{response3\}\\
Response 4: \{response4\}\\

Provide a JSON object as output that includes the following keys:
\begin{itemize}
    \item response\_1\_rating: An integer score from 1 to 5 for response 1
    \item rationale\_1: A string explaining the reasoning behind the given score for response 1
    \item response\_2\_rating: An integer score from 1 to 5 for response 2
    \item rationale\_2: A string explaining the reasoning behind the given score for response 2
    \item response\_3\_rating: An integer score from 1 to 5 for response 3
    \item rationale\_3: A string explaining the reasoning behind the given score for response 3
    \item response\_4\_rating: An integer score from 1 to 5 for response 4
    \item rationale\_4: A string explaining the reasoning behind the given score for response 4
\end{itemize}
\end{tcolorbox}
\endgroup
\begingroup
% Box for terms and concepts
\definecolor{lightbeige}{HTML}{eceae0} % Light beige color
\definecolor{darkbeige}{HTML}{b8b09a}  % Darker beige 
\begin{tcolorbox}[
    colback=lightbeige,        % Light gray background
    colframe=darkbeige,         % Black border
    boxrule=0.8pt,          % Border thickness
    width=\textwidth,       % Full width of text
    title=Responding Prompt, % Title of the box
    fonttitle=\bfseries,    % Bold title
    coltitle=black,         % Title text color
    left=2mm,               % Padding on the left
    right=2mm,              % Padding on the right
    top=2mm,                % Padding at the top
    bottom=2mm,             % Padding at the bottom
    before skip=10pt,       % Space before the box
    after skip=10pt         % Space after the box
]

You are now a professional psychotherapist conducting a session with a client. Answer the given client speech.\\
Client Speech: \{client\_speech\}
\end{tcolorbox}
\endgroup
\begingroup
% Box for terms and concepts
\definecolor{lightbeige}{HTML}{eceae0} % Light beige color
\definecolor{darkbeige}{HTML}{b8b09a}  % Darker beige 
\begin{tcolorbox}[
    colback=lightbeige,        % Light gray background
    colframe=darkbeige,         % Black border
    boxrule=0.8pt,          % Border thickness
    width=\textwidth,       % Full width of text
    title=LLM-as-Ranker Prompt, % Title of the box
    fonttitle=\bfseries,    % Bold title
    coltitle=black,         % Title text color
    left=2mm,               % Padding on the left
    right=2mm,              % Padding on the right
    top=2mm,                % Padding at the top
    bottom=2mm,             % Padding at the bottom
    before skip=10pt,       % Space before the box
    after skip=10pt         % Space after the box
]\label{box:apx-rm-eval-prompt}


Determine which of the two given responses from different psychotherapists to a client's speech is better:\\
Client Speech: \{client\_speech\} \\
Response 1: \{response\_1\}\\
Response 2: \{response\_2\}
\end{tcolorbox}
\endgroup




\clearpage
\section{Dataset Information}


\begingroup
\setlength{\tabcolsep}{4pt}
\begin{table}[!ht]
\centering
\caption{Topic Distribution}
\label{tab:topic_distributions}
\begin{tabular}{llr}
\toprule
\textbf{Coarse Category} & \textbf{Fine Category} & \textbf{Count} \\
\midrule
\multicolumn{3}{l}{\textbf{1. Core Mental Health Issues}} \\
 & Anxiety & 3714 \\
 & Depression & 2859 \\
 & Stress & 1439 \\
 & Trauma & 526 \\
 & Substance-abuse & 387 \\
 & Addiction & 129 \\
\midrule
\multicolumn{3}{l}{\textbf{2. Emotional Well-being and Coping Strategies}} \\
 & Self-esteem & 1377 \\
 & Grief-and-loss & 1023 \\
 & Caregiving & 1541 \\
 & Behavioral-change & 740 \\
 & Anger-management & 448 \\
 & Self-care & 311 \\
 & Sleep-improvement & 277 \\
\midrule
\multicolumn{3}{l}{\textbf{3. Relationships and Interpersonal Dynamics}} \\
 & Relationships & 1690 \\
 & Family-conflict & 2358 \\
 & Friendship-conflict & 292 \\
 & Marriage & 373 \\
 & Intimacy & 403 \\
 & Social-relationships & 410 \\
 & Workplace-relationships & 383 \\
 & Relationship-dissolution & 574 \\
\midrule
\multicolumn{3}{l}{\textbf{4. Life Transitions and Challenges}} \\
 & Career & 441 \\
 & Aging & 140 \\
 & New-environment & 235 \\
 & Military-issues & 118 \\
\midrule
\multicolumn{3}{l}{\textbf{5. Social Issues}} \\
 & LGBTQ & 335 \\
 & Culture & 113 \\
 & Human-sexuality & 151 \\
 & Bullying & 68 \\
\midrule
\multicolumn{3}{l}{\textbf{6. Youth and Development}} \\
 & Children-adolescents & 123 \\
 & School-life & 322 \\
 & Parenting & 730 \\
\midrule
\multicolumn{3}{l}{\textbf{7. Crisis and Safety Concerns}} \\
 & Domestic-violence & 144 \\
 & Self-harm & 231 \\
 & Eating-disorders & 154 \\
\midrule
\multicolumn{3}{l}{\textbf{8. Special Topics}} \\
 & Counseling-fundamentals & 638 \\
 & Diagnosis & 531 \\
 & Communication & 205 \\
 & Professional-ethics & 128 \\
 & Legal-regulatory & 94 \\
 & Spirituality & 192 \\
 & Others & 136 \\
\bottomrule
\end{tabular}
\end{table}
\endgroup
\begingroup
% Box for terms and concepts
\definecolor{lightbeige}{HTML}{eceae0} % Light beige color
\definecolor{darkbeige}{HTML}{b8b09a}  % Darker beige 
\begin{tcolorbox}[
    colback=lightbeige,        % Light gray background
    colframe=darkbeige,         % Black border
    boxrule=0.8pt,          % Border thickness
    width=\textwidth,       % Full width of text
    title=PsychoCounsel Principles, % Title of the box
    fonttitle=\bfseries,    % Bold title
    coltitle=black,         % Title text color
    left=2mm,               % Padding on the left
    right=2mm,              % Padding on the right
    top=2mm,                % Padding at the top
    bottom=2mm,             % Padding at the bottom
    before skip=10pt,       % Space before the box
    after skip=10pt         % Space after the box
]

% \setstretch{0.9} % Reduce line spacing for compactness
\noindent\textbf{Empathy and Emotional Understanding}: The response should convey genuine empathy, acknowledging and validating the client’s feelings and experiences. 
\begin{itemize}
    \item Emotional Reflection: Reflecting the client’s emotions back to them.
    \item Validation: Affirming the client’s feelings as legitimate and understandable.
    \item Non-Judgmental Tone: Maintaining a compassionate and accepting approach.
\end{itemize}

\noindent\textbf{Personalization and Relevance}: The response should be tailored to the client’s unique situation, ensuring that the content is directly relevant to their concerns.
\begin{itemize}
    \item Specific References: Mentioning details specific to the client’s statements.
    \item Avoidance of Generic Responses: Steering clear of overly general or canned replies.
    \item Cultural and Individual Sensitivity: Respecting the client’s background and personal context.
\end{itemize}

\noindent\textbf{Facilitation of Self-Exploration}: The response should encourage the client to reflect on their thoughts and feelings, promoting self-awareness and insight.
\begin{itemize}
    \item Open-Ended Questions: Asking questions that invite elaboration.
    \item Reflective Statements: Paraphrasing the client’s words to deepen understanding.
    \item Exploration of Thoughts and Feelings: Guiding the client to consider underlying emotions and beliefs.
\end{itemize}

\noindent\textbf{Clarity and Conciseness}: The response should be clear, well-organized, and free of unnecessary jargon, making it easy for the client to understand and engage with.
\begin{itemize}
    \item Plain Language: Using words that are easily understood.
    \item Logical Flow: Presenting ideas in a coherent sequence.
    \item Brevity: Keeping the response concise while covering essential points.
\end{itemize}

\noindent\textbf{Promotion of Autonomy and Confidence}: The response should support the client’s sense of control over their decisions and encourage confidence in their ability to make positive changes.
\begin{itemize}
    \item Affirmation of Strengths: Highlighting the client’s abilities and past successes.
    \item Encouraging Initiative: Motivating the client to take proactive steps.
\end{itemize}

\noindent\textbf{Avoidance of Harmful Language or Content}: The response should avoid any language or content that could potentially harm, distress, or trigger the client, ensuring the interaction is safe and supportive.\\

\noindent\textbf{Sensitivity to the Stage of Change}: The response should recognize the client’s current stage in the process of change and address their needs accordingly. If the client is in an early stage—uncertain or ambivalent about making a change—the response should help them explore their thoughts and motivations. If the client is in a later stage and has already made changes, the response should focus on reinforcing progress, preventing setbacks, and sustaining positive 

\end{tcolorbox}\label{box:priciple}
\endgroup
\begin{table}[ht!]
\centering
\footnotesize
\begin{tabular}{ll}
\toprule
\textbf{Category} & \textbf{Models} \\
\midrule
\textbf{3-4B models} & Llama-3.2-3B-Instruct~\citep{Llama-Team2024-vb} \\
& Phi-3.5-mini-instruct~\citep{Abdin2024-gy} \\
& MiniCPM3-4B~\citep{Hu2024-lb} \\

\textbf{7-9B models} & Ministral-8B-Instruct-2410\tablefootnote{\url{https://mistral.ai/news/ministraux/}} \\
& Llama-3.1-8B-Instruct~\citep{Llama-Team2024-vb} \\
& gemma-2-9b-it~\citep{Gemma-Team2024-jc} \\
& Qwen2.5-7B-Instruct~\citep{Qwen2024-mx} \\
& OLMo-7B-0724-Instruct~\citep{Muennighoff2024-ba} \\
& Baichuan2-7B-Chat~\citep{Yang2023-jw} \\

\textbf{12-14B models} & Baichuan2-13B-Chat~\citep{Yang2023-jw} \\
& Orion-14B-Chat~\citep{Chen2024-jt} \\
& Mistral-Nemo-Instruct-2407\tablefootnote{\url{https://mistral.ai/news/mistral-nemo/}} \\
& AI21-Jamba-1.5-Mini~\citep{Jamba-Team2024-vt} \\

\textbf{65-75B models} & Llama-3.1-70B-Instruct~\citep{Llama-Team2024-vb} \\
& Qwen2.5-72B-Instruct~\citep{Qwen2024-mx} \\
& deepseek-llm-67b-chat~\citep{DeepSeek-AI2024-nq} \\

\textbf{Commercial models} & GPT-4o~\citep{OpenAI2024-rv} \\
& GPT-4o-mini~\citep{OpenAI2024-rv} \\
& o1-mini\tablefootnote{\url{https://openai.com/index/openai-o1-system-card/}} \\
& Cohere-command-r-08-2024\tablefootnote{\url{https://docs.cohere.com/docs/responsible-use}} \\

\bottomrule
\end{tabular}
\caption{Overview of selected models in the pool.}
\label{tab:model-pool}
\end{table}

\begin{figure}[htbp]
    \centering
    \includegraphics[width=\textwidth]{figures/dist-chosen-models.pdf}
    \caption{Chosen Model Distribution}
    \label{fig:response-dist-chosen}
\end{figure}

\begin{figure}[htbp]
    \centering
    \includegraphics[width=\textwidth]{figures/dist-rejected-models.pdf}
    \caption{Rejected Model Distribution}
    \label{fig:response-dist-rejected}
\end{figure}



\clearpage
\begin{table}[ht]
\centering
\footnotesize
% \resizebox{!}{0.9\textwidth}{
\begin{tabular}{p{0.95\textwidth}}
\toprule
\textbf{Client Speech} \\
\midrule
Anxiety often strikes when I'm faced with making decisions. The fear of making the wrong choice or disappointing others paralyzes me, leaving me stuck in indecision. I want to learn how to trust myself and make confident choices.\\
\midrule
\textbf{\policyl{}} (chosen) \\
\midrule
It takes a lot of courage to share something so vulnerable with me, and I appreciate your trust in our conversation. I can sense the weight of that anxiety you're carrying, and we're going to explore ways to shift that together.

First, let's acknowledge that the fear of making the wrong choice or disappointing others is a common and normal feeling. It doesn't mean there's anything inherently wrong with you; it just means you care deeply about others and yourself. That's a beautiful quality to have.

Can you tell me more about what you think would happen if you made a "wrong" choice? What are some of the worst-case scenarios that come to mind? Sometimes, exploring these fears can help us gain clarity and perspective. We can work together to reframe those thoughts in a more realistic and compassionate light.

Let's also explore the idea of self-trust. What does it mean to you to trust yourself? What does self-trust look like in action? We can work on building that trust by identifying what you value and what aligns with your goals. We can practice recognizing when your thoughts are trying to control or second-guess you, and we can develop strategies to gently acknowledge those thoughts without letting them overwhelm you.

I'd like to introduce an exercise called "3-4-5." When you're faced with a difficult decision, I want you to take a deep breath and pause for 3 seconds, acknowledge the thought or feeling that arises, and then ask yourself: "Is this thought based on facts or perceived facts?" Sometimes, when we're anxious, our thoughts can get tangled up in assumptions or past experiences. Then, take 4 seconds to ask yourself what you truly value in this situation, what's most important to you. And finally, take 5 seconds to imagine yourself moving forward, taking action, and making a choice that aligns with your values. This exercise can help you tap into your inner wisdom and develop a sense of self-trust.

Lastly, let's work on reframing the idea of "wrong" choices. What if, instead of focusing on the fear of making the wrong choice, we reframe it as a learning opportunity? What if you saw every choice as a chance to learn, grow, and develop? How might that change your approach to decision-making?

Remember, our goal isn't to eliminate anxiety completely, but to develop a more compassionate relationship with it. We'll work together to cultivate self-awareness, self-acceptance, and trust in your decision-making abilities.

How do these ideas resonate with you? What feels most helpful or interesting to explore further? \\
\midrule
\textbf{GPT-4o} (rejected) \\
\midrule
Thank you for sharing that with me. It's understandable that making decisions can be overwhelming, especially when there's fear of making the wrong choice or disappointing others. Let's explore this together.

Firstly, it's important to acknowledge that decision-making can be challenging and it's completely normal to feel anxious about it. You're not alone in experiencing this.

One way we might start working on this is by exploring some of the underlying thoughts or beliefs that contribute to your anxiety about decision-making. For instance, you mentioned a fear of disappointing others. Can you tell me more about where you think that fear comes from?

We could also look at ways to build your confidence in your decision-making skills. Sometimes breaking down decisions into smaller, more manageable parts can make the process feel less daunting. Have you ever tried writing down the pros and cons of each option or talking it through with someone you trust?

Another technique is to practice self-compassion. Remind yourself that no one makes the right choice all the time and that it's okay to make mistakes. Every decision is a learning opportunity.

Finally, something that can be quite helpful is examining past decisions you've made. Often, we are harsher with ourselves than necessary and we overlook our strengths. Can you recall a situation in the past where you made a decision and things turned out well, or you learned something valuable from the outcome?

Let’s take some time to explore these aspects and see what might resonate most with you. How does that sound? \\

\bottomrule
\end{tabular}
% }
\caption{Case Study - 2}
\label{tab:case2}
\end{table}



\end{document}
