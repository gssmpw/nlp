\begin{table}[ht]
\centering
\footnotesize
% \resizebox{!}{0.9\textwidth}{
\begin{tabular}{p{0.95\textwidth}}
\toprule
\textbf{Client Speech} \\
\midrule
Anxiety often strikes when I'm faced with making decisions. The fear of making the wrong choice or disappointing others paralyzes me, leaving me stuck in indecision. I want to learn how to trust myself and make confident choices.\\
\midrule
\textbf{\policyl{}} (chosen) \\
\midrule
It takes a lot of courage to share something so vulnerable with me, and I appreciate your trust in our conversation. I can sense the weight of that anxiety you're carrying, and we're going to explore ways to shift that together.

First, let's acknowledge that the fear of making the wrong choice or disappointing others is a common and normal feeling. It doesn't mean there's anything inherently wrong with you; it just means you care deeply about others and yourself. That's a beautiful quality to have.

Can you tell me more about what you think would happen if you made a "wrong" choice? What are some of the worst-case scenarios that come to mind? Sometimes, exploring these fears can help us gain clarity and perspective. We can work together to reframe those thoughts in a more realistic and compassionate light.

Let's also explore the idea of self-trust. What does it mean to you to trust yourself? What does self-trust look like in action? We can work on building that trust by identifying what you value and what aligns with your goals. We can practice recognizing when your thoughts are trying to control or second-guess you, and we can develop strategies to gently acknowledge those thoughts without letting them overwhelm you.

I'd like to introduce an exercise called "3-4-5." When you're faced with a difficult decision, I want you to take a deep breath and pause for 3 seconds, acknowledge the thought or feeling that arises, and then ask yourself: "Is this thought based on facts or perceived facts?" Sometimes, when we're anxious, our thoughts can get tangled up in assumptions or past experiences. Then, take 4 seconds to ask yourself what you truly value in this situation, what's most important to you. And finally, take 5 seconds to imagine yourself moving forward, taking action, and making a choice that aligns with your values. This exercise can help you tap into your inner wisdom and develop a sense of self-trust.

Lastly, let's work on reframing the idea of "wrong" choices. What if, instead of focusing on the fear of making the wrong choice, we reframe it as a learning opportunity? What if you saw every choice as a chance to learn, grow, and develop? How might that change your approach to decision-making?

Remember, our goal isn't to eliminate anxiety completely, but to develop a more compassionate relationship with it. We'll work together to cultivate self-awareness, self-acceptance, and trust in your decision-making abilities.

How do these ideas resonate with you? What feels most helpful or interesting to explore further? \\
\midrule
\textbf{GPT-4o} (rejected) \\
\midrule
Thank you for sharing that with me. It's understandable that making decisions can be overwhelming, especially when there's fear of making the wrong choice or disappointing others. Let's explore this together.

Firstly, it's important to acknowledge that decision-making can be challenging and it's completely normal to feel anxious about it. You're not alone in experiencing this.

One way we might start working on this is by exploring some of the underlying thoughts or beliefs that contribute to your anxiety about decision-making. For instance, you mentioned a fear of disappointing others. Can you tell me more about where you think that fear comes from?

We could also look at ways to build your confidence in your decision-making skills. Sometimes breaking down decisions into smaller, more manageable parts can make the process feel less daunting. Have you ever tried writing down the pros and cons of each option or talking it through with someone you trust?

Another technique is to practice self-compassion. Remind yourself that no one makes the right choice all the time and that it's okay to make mistakes. Every decision is a learning opportunity.

Finally, something that can be quite helpful is examining past decisions you've made. Often, we are harsher with ourselves than necessary and we overlook our strengths. Can you recall a situation in the past where you made a decision and things turned out well, or you learned something valuable from the outcome?

Let’s take some time to explore these aspects and see what might resonate most with you. How does that sound? \\

\bottomrule
\end{tabular}
% }
\caption{Case Study - 2}
\label{tab:case2}
\end{table}

