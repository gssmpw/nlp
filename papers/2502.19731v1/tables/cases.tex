\begin{table}[ht]
\centering
\scriptsize
% \resizebox{!}{0.9\textwidth}{
\begin{tabular}{p{0.95\textwidth}}
\toprule
\textbf{Client Speech} \\
\midrule
I’ve been having emotional issues for a few years. Nonetheless, these have been somewhat manageable. However, I became increasingly paranoid this winter. I thought that people were trying to poison me, I feared that family members were going to kill me. I was aware that these thoughts were illogical, but I couldn’t shake the fear that they caused me. I would get panic attacks thinking that I drank from a poisoned water fountain. I thought that someone had put small rips in my ice cream sandwich packaging because there was only one that didn’t have tiny rips in it. I refused to eat this one because I assumed some reverse psychology was being pulled and the ones without the rips were fine. I thought that this was mostly caused by anxiety but there are other symptoms that lead me to believe it is something more.\\
\midrule
\textbf{\policyl{}} (chosen) \\
\midrule
I can sense the immense distress and fear that you've been experiencing, and I'm here to create a safe space for you to explore these feelings. It takes a lot of courage to acknowledge and share these thoughts with me. I want to acknowledge that it sounds like you've been carrying a heavy burden, and it's understandable that you felt the need to try to make sense of these experiences, even if they seemed illogical at times. \\
\darkred{$ \Rightarrow $ (detailed acknowledgment of the client's feelings tends to create a stronger emotional connection, making the client feel more profoundly understood)}

I'd like to gently explore this further. You mentioned that you recognized these thoughts as anxiety-driven, but there are other symptoms that make you question whether it's more than that. Can you tell me more about what specifically makes you think it's not just anxiety? What are those other symptoms that have led you to consider this? Let's work together to tease out what might be contributing to these feelings, and see if we can identify any patterns or triggers that might be at play. We can also explore ways to help you manage these fears and develop more grounding techniques to reduce your anxiety. How does that sound? \\
\blue{$ \Rightarrow $ (engage the client in a detailed, collaborative exploration of her experiences)} \\
\midrule
\textbf{GPT-4o} (rejected) \\
\midrule

Thank you for sharing all of this with me. It sounds like you've been going through a particularly difficult time with these intense feelings and thoughts. It must be very distressing to experience such fears and the resulting panic attacks. I want to make sure we explore this in a way that feels safe for you.\\
\red{$ \Rightarrow $ (more general in validating the client’s feelings)}

You mentioned that while you've considered it might be anxiety, there are additional symptoms that make you think there might be more to it. Could you tell me more about these other symptoms? Understanding the full picture can help us figure out what might be happening and how best to support you. \\
\red{$ \Rightarrow $ (lacks the specific, collaborative invitation to explore the triggers)}

Also, it's important to acknowledge the awareness you've shown about the thoughts being illogical at times, yet still powerful. This awareness is a valuable strength as we work through these experiences together. Let's take it one step at a time and see if we can find patterns or triggers for these thoughts and discuss some strategies that might help manage them.\\
\red{$ \Rightarrow $ (general and less detailed)}
 \\

\bottomrule
\end{tabular}
% }
\caption{Case Study}
\label{tab:case1}
\end{table}