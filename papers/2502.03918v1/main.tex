%%%%%%%%%%%%%%%%%%%%%%%%%%%%%%%%%%%%%%%%%%%%%%%%%%%%%%%%%%%%%%%%%%%%%%%%%%%%%%%%
%2345678901234567890123456789012345678901234567890123456789012345678901234567890
%        1         2         3         4         5         6         7         8

\documentclass[letterpaper, 10 pt, conference]{ieeeconf}  % Comment this line out if you need a4paper

%\documentclass[a4paper, 10pt, conference]{ieeeconf}      % Use this line for a4 paper

\IEEEoverridecommandlockouts                              % This command is only needed if 
                                                          % you want to use the \thanks command

\overrideIEEEmargins                                      % Needed to meet printer requirements.

%In case you encounter the following error:
%Error 1010 The PDF file may be corrupt (unable to open PDF file) OR
%Error 1000 An error occurred while parsing a contents stream. Unable to analyze the PDF file.
%This is a known problem with pdfLaTeX conversion filter. The file cannot be opened with acrobat reader
%Please use one of the alternatives below to circumvent this error by uncommenting one or the other
%\pdfobjcompresslevel=0
%\pdfminorversion=4

% See the \addtolength command later in the file to balance the column lengths
% on the last page of the document

% The following packages can be found on http:\\www.ctan.org
% The following packages can be found on http:\\www.ctan.org
\usepackage{graphics} % for pdf, bitmapped graphics files
\usepackage{epsfig} % for postscript graphics files
\usepackage{mathptmx} % assumes new font selection scheme installed
\usepackage{times} % assumes new font selection scheme installed
\usepackage{amsmath} % assumes amsmath package installed
\usepackage{amssymb}  % assumes amsmath package installed
\usepackage{xcolor}
\usepackage{colortbl}
\usepackage{url}
\usepackage{gensymb}
% \usepackage{algorithm2e}
\usepackage{algorithm}
\usepackage{algpseudocode}
\usepackage{listings}
\usepackage{caption}

\usepackage{hyperref}
\hypersetup{
    colorlinks=true,
    linkcolor=blue,
    filecolor=magenta,      
    urlcolor=blue
}
\urlstyle{same}

\hyphenation{op-tical net-works semi-conduc-tor IEEE-Xplore}
\def\BibTeX{{\rm B\kern-.05em{\sc i\kern-.025em b}\kern-.08em
    T\kern-.1667em\lower.7ex\hbox{E}\kern-.125emX}}
\usepackage{balance}

\newcommand{\real}[0]{\ensuremath{\mathbb{R}}}
\newcommand{\commenting}[1]{}
\newcommand{\todo}[1]{\textbf{TODO: #1}}
\newcommand{\newAddition}[1]{{\color{magenta} NEW ADDITION: #1}}
\newcommand{\toShorten}[1]{{\color{blue} #1}}
\newcommand{\myHighlight}[1]{{\color{red} #1}}

\newcommand{\skill}[0]{\textit{Skill}}
\newcommand{\skills}[0]{\textit{Skills}}
\newcommand{\action}[0]{\textit{Action}}
\newcommand{\actions}[0]{\textit{Actions}}

\definecolor{Gray}{gray}{0.85}
\definecolor{LightCyan}{rgb}{0.88,1,1}
\definecolor{White}{rgb}{1,1,1}

\newcolumntype{g}{>{\columncolor{Gray}}c}

% THESE ARE COMMANDS TO SHRINK THE OUTPUT PDF SIZE

\setlength\abovedisplayskip{2pt}
\setlength\belowdisplayskip{2pt}
\setlength\abovedisplayshortskip{2pt}
\setlength\belowdisplayshortskip{2pt}

\captionsetup[figure]{font=small,skip=4pt}

\setlength{\dbltextfloatsep}{0pt}
\setlength{\textfloatsep}{1pt}
\setlength{\skip\footins}{0.3ex}

\makeatletter%change heading spacings
\renewcommand{\section}{\@startsection{section}{1}{\z@}{1.0ex plus 1.0ex minus 0.5ex}%
	{0.5ex plus 1ex minus 0ex}{\normalfont\normalsize\centering\scshape}}%
\renewcommand{\subsection}{\@startsection{subsection}{2}{\z@}{1.0ex plus 1.0ex minus 0.5ex}%
	{0.5ex plus 1ex minus 0ex}{\normalfont\normalsize\itshape}}%
\makeatother

\title{\LARGE \bf
% Representing Task Goal States as Variations of Environment States
Adaptation of Task Goal States from Prior Knowledge
}


\author{Andrei Costinescu and Darius Burschka$^{1}$% <-this % stops a space
\thanks{*This work was supported by the Lighthouse Initiative Geriatronics by StMWi Bayern (Project X, grant no. 5140951).}% <-this % stops a space
\thanks{$^{1}$All authors are with the School of Computation, Information and Technology at the Technical University of Munich. 
        {\tt\small \{andrei.costinescu, burschka\}@tum.de}}%
}


\begin{document}



\maketitle
\thispagestyle{empty}
\pagestyle{empty}


%%%%%%%%%%%%%%%%%%%%%%%%%%%%%%%%%%%%%%%%%%%%%%%%%%%%%%%%%%%%%%%%%%%%%%%%%%%%%%%%
\begin{abstract}

This paper presents a framework to define a task with freedom and variability in its goal state. A robot could use this to observe the execution of a task and target a different goal from the observed one; a goal that is still compatible with the task description but would be easier for the robot to execute. We define the model of an environment state and an environment \textbf{variation}, and present experiments on how to interactively create the \textbf{variation} from a single task demonstration and how to use this \textbf{variation} to create an execution plan for bringing any environment into the goal state.
\end{abstract}


%%%%%%%%%%%%%%%%%%%%%%%%%%%%%%%%%%%%%%%%%%%%%%%%%%%%%%%%%%%%%%%%%%%%%%%%%%%%%%%%
\section{Introduction}

Video generation has garnered significant attention owing to its transformative potential across a wide range of applications, such media content creation~\citep{polyak2024movie}, advertising~\citep{zhang2024virbo,bacher2021advert}, video games~\citep{yang2024playable,valevski2024diffusion, oasis2024}, and world model simulators~\citep{ha2018world, videoworldsimulators2024, agarwal2025cosmos}. Benefiting from advanced generative algorithms~\citep{goodfellow2014generative, ho2020denoising, liu2023flow, lipman2023flow}, scalable model architectures~\citep{vaswani2017attention, peebles2023scalable}, vast amounts of internet-sourced data~\citep{chen2024panda, nan2024openvid, ju2024miradata}, and ongoing expansion of computing capabilities~\citep{nvidia2022h100, nvidia2023dgxgh200, nvidia2024h200nvl}, remarkable advancements have been achieved in the field of video generation~\citep{ho2022video, ho2022imagen, singer2023makeavideo, blattmann2023align, videoworldsimulators2024, kuaishou2024klingai, yang2024cogvideox, jin2024pyramidal, polyak2024movie, kong2024hunyuanvideo, ji2024prompt}.


In this work, we present \textbf{\ours}, a family of rectified flow~\citep{lipman2023flow, liu2023flow} transformer models designed for joint image and video generation, establishing a pathway toward industry-grade performance. This report centers on four key components: data curation, model architecture design, flow formulation, and training infrastructure optimization—each rigorously refined to meet the demands of high-quality, large-scale video generation.


\begin{figure}[ht]
    \centering
    \begin{subfigure}[b]{0.82\linewidth}
        \centering
        \includegraphics[width=\linewidth]{figures/t2i_1024.pdf}
        \caption{Text-to-Image Samples}\label{fig:main-demo-t2i}
    \end{subfigure}
    \vfill
    \begin{subfigure}[b]{0.82\linewidth}
        \centering
        \includegraphics[width=\linewidth]{figures/t2v_samples.pdf}
        \caption{Text-to-Video Samples}\label{fig:main-demo-t2v}
    \end{subfigure}
\caption{\textbf{Generated samples from \ours.} Key components are highlighted in \textcolor{red}{\textbf{RED}}.}\label{fig:main-demo}
\end{figure}


First, we present a comprehensive data processing pipeline designed to construct large-scale, high-quality image and video-text datasets. The pipeline integrates multiple advanced techniques, including video and image filtering based on aesthetic scores, OCR-driven content analysis, and subjective evaluations, to ensure exceptional visual and contextual quality. Furthermore, we employ multimodal large language models~(MLLMs)~\citep{yuan2025tarsier2} to generate dense and contextually aligned captions, which are subsequently refined using an additional large language model~(LLM)~\citep{yang2024qwen2} to enhance their accuracy, fluency, and descriptive richness. As a result, we have curated a robust training dataset comprising approximately 36M video-text pairs and 160M image-text pairs, which are proven sufficient for training industry-level generative models.

Secondly, we take a pioneering step by applying rectified flow formulation~\citep{lipman2023flow} for joint image and video generation, implemented through the \ours model family, which comprises Transformer architectures with 2B and 8B parameters. At its core, the \ours framework employs a 3D joint image-video variational autoencoder (VAE) to compress image and video inputs into a shared latent space, facilitating unified representation. This shared latent space is coupled with a full-attention~\citep{vaswani2017attention} mechanism, enabling seamless joint training of image and video. This architecture delivers high-quality, coherent outputs across both images and videos, establishing a unified framework for visual generation tasks.


Furthermore, to support the training of \ours at scale, we have developed a robust infrastructure tailored for large-scale model training. Our approach incorporates advanced parallelism strategies~\citep{jacobs2023deepspeed, pytorch_fsdp} to manage memory efficiently during long-context training. Additionally, we employ ByteCheckpoint~\citep{wan2024bytecheckpoint} for high-performance checkpointing and integrate fault-tolerant mechanisms from MegaScale~\citep{jiang2024megascale} to ensure stability and scalability across large GPU clusters. These optimizations enable \ours to handle the computational and data challenges of generative modeling with exceptional efficiency and reliability.


We evaluate \ours on both text-to-image and text-to-video benchmarks to highlight its competitive advantages. For text-to-image generation, \ours-T2I demonstrates strong performance across multiple benchmarks, including T2I-CompBench~\citep{huang2023t2i-compbench}, GenEval~\citep{ghosh2024geneval}, and DPG-Bench~\citep{hu2024ella_dbgbench}, excelling in both visual quality and text-image alignment. In text-to-video benchmarks, \ours-T2V achieves state-of-the-art performance on the UCF-101~\citep{ucf101} zero-shot generation task. Additionally, \ours-T2V attains an impressive score of \textbf{84.85} on VBench~\citep{huang2024vbench}, securing the top position on the leaderboard (as of 2025-01-25) and surpassing several leading commercial text-to-video models. Qualitative results, illustrated in \Cref{fig:main-demo}, further demonstrate the superior quality of the generated media samples. These findings underscore \ours's effectiveness in multi-modal generation and its potential as a high-performing solution for both research and commercial applications.
\section{Methodology}
\label{sec:approach}

In this section, we present the \textbf{N}eurosymbolic \textbf{P}rogram \textbf{C}omprehension (\framework) framework, which leverages SHAP values (refer to \secref{sec:background}) to interpret and guide model predictions. We first describe our approach to identifying patterns in SHAP values for input features. Next, we explain how these patterns are transformed into symbolic rules to improve model performance, particularly in scenarios with low prediction confidence.

%%%%%%%%%%%%%%%%%%% NPC PIPELINE  
\begin{figure}[ht]
		\centering
  \vspace{-1.5em}
  \includegraphics[width=0.45\textwidth]{images/pipeline.pdf}
		\caption{Description of \framework framework as a sequence of steps.}
    \label{fig:npc_pipeline}
    \vspace{-1em}
\end{figure}
%%%%%%%%%%%%%%%%%%

%%%%%%% RESULTS TABLE 

\begin{table*}[t]
\centering
\fontsize{11pt}{11pt}\selectfont
\begin{tabular}{lllllllllllll}
\toprule
\multicolumn{1}{c}{\textbf{task}} & \multicolumn{2}{c}{\textbf{Mir}} & \multicolumn{2}{c}{\textbf{Lai}} & \multicolumn{2}{c}{\textbf{Ziegen.}} & \multicolumn{2}{c}{\textbf{Cao}} & \multicolumn{2}{c}{\textbf{Alva-Man.}} & \multicolumn{1}{c}{\textbf{avg.}} & \textbf{\begin{tabular}[c]{@{}l@{}}avg.\\ rank\end{tabular}} \\
\multicolumn{1}{c}{\textbf{metrics}} & \multicolumn{1}{c}{\textbf{cor.}} & \multicolumn{1}{c}{\textbf{p-v.}} & \multicolumn{1}{c}{\textbf{cor.}} & \multicolumn{1}{c}{\textbf{p-v.}} & \multicolumn{1}{c}{\textbf{cor.}} & \multicolumn{1}{c}{\textbf{p-v.}} & \multicolumn{1}{c}{\textbf{cor.}} & \multicolumn{1}{c}{\textbf{p-v.}} & \multicolumn{1}{c}{\textbf{cor.}} & \multicolumn{1}{c}{\textbf{p-v.}} &  &  \\ \midrule
\textbf{S-Bleu} & 0.50 & 0.0 & 0.47 & 0.0 & 0.59 & 0.0 & 0.58 & 0.0 & 0.68 & 0.0 & 0.57 & 5.8 \\
\textbf{R-Bleu} & -- & -- & 0.27 & 0.0 & 0.30 & 0.0 & -- & -- & -- & -- & - &  \\
\textbf{S-Meteor} & 0.49 & 0.0 & 0.48 & 0.0 & 0.61 & 0.0 & 0.57 & 0.0 & 0.64 & 0.0 & 0.56 & 6.1 \\
\textbf{R-Meteor} & -- & -- & 0.34 & 0.0 & 0.26 & 0.0 & -- & -- & -- & -- & - &  \\
\textbf{S-Bertscore} & \textbf{0.53} & 0.0 & {\ul 0.80} & 0.0 & \textbf{0.70} & 0.0 & {\ul 0.66} & 0.0 & {\ul0.78} & 0.0 & \textbf{0.69} & \textbf{1.7} \\
\textbf{R-Bertscore} & -- & -- & 0.51 & 0.0 & 0.38 & 0.0 & -- & -- & -- & -- & - &  \\
\textbf{S-Bleurt} & {\ul 0.52} & 0.0 & {\ul 0.80} & 0.0 & 0.60 & 0.0 & \textbf{0.70} & 0.0 & \textbf{0.80} & 0.0 & {\ul 0.68} & {\ul 2.3} \\
\textbf{R-Bleurt} & -- & -- & 0.59 & 0.0 & -0.05 & 0.13 & -- & -- & -- & -- & - &  \\
\textbf{S-Cosine} & 0.51 & 0.0 & 0.69 & 0.0 & {\ul 0.62} & 0.0 & 0.61 & 0.0 & 0.65 & 0.0 & 0.62 & 4.4 \\
\textbf{R-Cosine} & -- & -- & 0.40 & 0.0 & 0.29 & 0.0 & -- & -- & -- & -- & - & \\ \midrule
\textbf{QuestEval} & 0.23 & 0.0 & 0.25 & 0.0 & 0.49 & 0.0 & 0.47 & 0.0 & 0.62 & 0.0 & 0.41 & 9.0 \\
\textbf{LLaMa3} & 0.36 & 0.0 & \textbf{0.84} & 0.0 & {\ul{0.62}} & 0.0 & 0.61 & 0.0 &  0.76 & 0.0 & 0.64 & 3.6 \\
\textbf{our (3b)} & 0.49 & 0.0 & 0.73 & 0.0 & 0.54 & 0.0 & 0.53 & 0.0 & 0.7 & 0.0 & 0.60 & 5.8 \\
\textbf{our (8b)} & 0.48 & 0.0 & 0.73 & 0.0 & 0.52 & 0.0 & 0.53 & 0.0 & 0.7 & 0.0 & 0.59 & 6.3 \\  \bottomrule
\end{tabular}
\caption{Pearson correlation on human evaluation on system output. `R-': reference-based. `S-': source-based.}
\label{tab:sys}
\end{table*}



\begin{table}%[]
\centering
\fontsize{11pt}{11pt}\selectfont
\begin{tabular}{llllll}
\toprule
\multicolumn{1}{c}{\textbf{task}} & \multicolumn{1}{c}{\textbf{Lai}} & \multicolumn{1}{c}{\textbf{Zei.}} & \multicolumn{1}{c}{\textbf{Scia.}} & \textbf{} & \textbf{} \\ 
\multicolumn{1}{c}{\textbf{metrics}} & \multicolumn{1}{c}{\textbf{cor.}} & \multicolumn{1}{c}{\textbf{cor.}} & \multicolumn{1}{c}{\textbf{cor.}} & \textbf{avg.} & \textbf{\begin{tabular}[c]{@{}l@{}}avg.\\ rank\end{tabular}} \\ \midrule
\textbf{S-Bleu} & 0.40 & 0.40 & 0.19* & 0.33 & 7.67 \\
\textbf{S-Meteor} & 0.41 & 0.42 & 0.16* & 0.33 & 7.33 \\
\textbf{S-BertS.} & {\ul0.58} & 0.47 & 0.31 & 0.45 & 3.67 \\
\textbf{S-Bleurt} & 0.45 & {\ul 0.54} & {\ul 0.37} & 0.45 & {\ul 3.33} \\
\textbf{S-Cosine} & 0.56 & 0.52 & 0.3 & {\ul 0.46} & {\ul 3.33} \\ \midrule
\textbf{QuestE.} & 0.27 & 0.35 & 0.06* & 0.23 & 9.00 \\
\textbf{LlaMA3} & \textbf{0.6} & \textbf{0.67} & \textbf{0.51} & \textbf{0.59} & \textbf{1.0} \\
\textbf{Our (3b)} & 0.51 & 0.49 & 0.23* & 0.39 & 4.83 \\
\textbf{Our (8b)} & 0.52 & 0.49 & 0.22* & 0.43 & 4.83 \\ \bottomrule
\end{tabular}
\caption{Pearson correlation on human ratings on reference output. *not significant; we cannot reject the null hypothesis of zero correlation}
\label{tab:ref}
\end{table}


\begin{table*}%[]
\centering
\fontsize{11pt}{11pt}\selectfont
\begin{tabular}{lllllllll}
\toprule
\textbf{task} & \multicolumn{1}{c}{\textbf{ALL}} & \multicolumn{1}{c}{\textbf{sentiment}} & \multicolumn{1}{c}{\textbf{detoxify}} & \multicolumn{1}{c}{\textbf{catchy}} & \multicolumn{1}{c}{\textbf{polite}} & \multicolumn{1}{c}{\textbf{persuasive}} & \multicolumn{1}{c}{\textbf{formal}} & \textbf{\begin{tabular}[c]{@{}l@{}}avg. \\ rank\end{tabular}} \\
\textbf{metrics} & \multicolumn{1}{c}{\textbf{cor.}} & \multicolumn{1}{c}{\textbf{cor.}} & \multicolumn{1}{c}{\textbf{cor.}} & \multicolumn{1}{c}{\textbf{cor.}} & \multicolumn{1}{c}{\textbf{cor.}} & \multicolumn{1}{c}{\textbf{cor.}} & \multicolumn{1}{c}{\textbf{cor.}} &  \\ \midrule
\textbf{S-Bleu} & -0.17 & -0.82 & -0.45 & -0.12* & -0.1* & -0.05 & -0.21 & 8.42 \\
\textbf{R-Bleu} & - & -0.5 & -0.45 &  &  &  &  &  \\
\textbf{S-Meteor} & -0.07* & -0.55 & -0.4 & -0.01* & 0.1* & -0.16 & -0.04* & 7.67 \\
\textbf{R-Meteor} & - & -0.17* & -0.39 & - & - & - & - & - \\
\textbf{S-BertScore} & 0.11 & -0.38 & -0.07* & -0.17* & 0.28 & 0.12 & 0.25 & 6.0 \\
\textbf{R-BertScore} & - & -0.02* & -0.21* & - & - & - & - & - \\
\textbf{S-Bleurt} & 0.29 & 0.05* & 0.45 & 0.06* & 0.29 & 0.23 & 0.46 & 4.2 \\
\textbf{R-Bleurt} & - &  0.21 & 0.38 & - & - & - & - & - \\
\textbf{S-Cosine} & 0.01* & -0.5 & -0.13* & -0.19* & 0.05* & -0.05* & 0.15* & 7.42 \\
\textbf{R-Cosine} & - & -0.11* & -0.16* & - & - & - & - & - \\ \midrule
\textbf{QuestEval} & 0.21 & {\ul{0.29}} & 0.23 & 0.37 & 0.19* & 0.35 & 0.14* & 4.67 \\
\textbf{LlaMA3} & \textbf{0.82} & \textbf{0.80} & \textbf{0.72} & \textbf{0.84} & \textbf{0.84} & \textbf{0.90} & \textbf{0.88} & \textbf{1.00} \\
\textbf{Our (3b)} & 0.47 & -0.11* & 0.37 & 0.61 & 0.53 & 0.54 & 0.66 & 3.5 \\
\textbf{Our (8b)} & {\ul{0.57}} & 0.09* & {\ul 0.49} & {\ul 0.72} & {\ul 0.64} & {\ul 0.62} & {\ul 0.67} & {\ul 2.17} \\ \bottomrule
\end{tabular}
\caption{Pearson correlation on human ratings on our constructed test set. 'R-': reference-based. 'S-': source-based. *not significant; we cannot reject the null hypothesis of zero correlation}
\label{tab:con}
\end{table*}

\section{Results}
We benchmark the different metrics on the different datasets using correlation to human judgement. For content preservation, we show results split on data with system output, reference output and our constructed test set: we show that the data source for evaluation leads to different conclusions on the metrics. In addition, we examine whether the metrics can rank style transfer systems similar to humans. On style strength, we likewise show correlations between human judgment and zero-shot evaluation approaches. When applicable, we summarize results by reporting the average correlation. And the average ranking of the metric per dataset (by ranking which metric obtains the highest correlation to human judgement per dataset). 

\subsection{Content preservation}
\paragraph{How do data sources affect the conclusion on best metric?}
The conclusions about the metrics' performance change radically depending on whether we use system output data, reference output, or our constructed test set. Ideally, a good metric correlates highly with humans on any data source. Ideally, for meta-evaluation, a metric should correlate consistently across all data sources, but the following shows that the correlations indicate different things, and the conclusion on the best metric should be drawn carefully.

Looking at the metrics correlations with humans on the data source with system output (Table~\ref{tab:sys}), we see a relatively high correlation for many of the metrics on many tasks. The overall best metrics are S-BertScore and S-BLEURT (avg+avg rank). We see no notable difference in our method of using the 3B or 8B model as the backbone.

Examining the average correlations based on data with reference output (Table~\ref{tab:ref}), now the zero-shoot prompting with LlaMA3 70B is the best-performing approach ($0.59$ avg). Tied for second place are source-based cosine embedding ($0.46$ avg), BLEURT ($0.45$ avg) and BertScore ($0.45$ avg). Our method follows on a 5. place: here, the 8b version (($0.43$ avg)) shows a bit stronger results than 3b ($0.39$ avg). The fact that the conclusions change, whether looking at reference or system output, confirms the observations made by \citet{scialom-etal-2021-questeval} on simplicity transfer.   

Now consider the results on our test set (Table~\ref{tab:con}): Several metrics show low or no correlation; we even see a significantly negative correlation for some metrics on ALL (BLEU) and for specific subparts of our test set for BLEU, Meteor, BertScore, Cosine. On the other end, LlaMA3 70B is again performing best, showing strong results ($0.82$ in ALL). The runner-up is now our 8B method, with a gap to the 3B version ($0.57$ vs $0.47$ in ALL). Note our method still shows zero correlation for the sentiment task. After, ranks BLEURT ($0.29$), QuestEval ($0.21$), BertScore ($0.11$), Cosine ($0.01$).  

On our test set, we find that some metrics that correlate relatively well on the other datasets, now exhibit low correlation. Hence, with our test set, we can now support the logical reasoning with data evidence: Evaluation of content preservation for style transfer needs to take the style shift into account. This conclusion could not be drawn using the existing data sources: We hypothesise that for the data with system-based output, successful output happens to be very similar to the source sentence and vice versa, and reference-based output might not contain server mistakes as they are gold references. Thus, none of the existing data sources tests the limits of the metrics.  


\paragraph{How do reference-based metrics compare to source-based ones?} Reference-based metrics show a lower correlation than the source-based counterpart for all metrics on both datasets with ratings on references (Table~\ref{tab:sys}). As discussed previously, reference-based metrics for style transfer have the drawback that many different good solutions on a rewrite might exist and not only one similar to a reference.


\paragraph{How well can the metrics rank the performance of style transfer methods?}
We compare the metrics' ability to judge the best style transfer methods w.r.t. the human annotations: Several of the data sources contain samples from different style transfer systems. In order to use metrics to assess the quality of the style transfer system, metrics should correctly find the best-performing system. Hence, we evaluate whether the metrics for content preservation provide the same system ranking as human evaluators. We take the mean of the score for every output on each system and the mean of the human annotations; we compare the systems using the Kendall's Tau correlation. 

We find only the evaluation using the dataset Mir, Lai, and Ziegen to result in significant correlations, probably because of sparsity in a number of system tests (App.~\ref{app:dataset}). Our method (8b) is the only metric providing a perfect ranking of the style transfer system on the Lai data, and Llama3 70B the only one on the Ziegen data. Results in App.~\ref{app:results}. 


\subsection{Style strength results}
%Evaluating style strengths is a challenging task. 
Llama3 70B shows better overall results than our method. However, our method scores higher than Llama3 70B on 2 out of 6 datasets, but it also exhibits zero correlation on one task (Table~\ref{tab:styleresults}).%More work i s needed on evaluating style strengths. 
 
\begin{table}%[]
\fontsize{11pt}{11pt}\selectfont
\begin{tabular}{lccc}
\toprule
\multicolumn{1}{c}{\textbf{}} & \textbf{LlaMA3} & \textbf{Our (3b)} & \textbf{Our (8b)} \\ \midrule
\textbf{Mir} & 0.46 & 0.54 & \textbf{0.57} \\
\textbf{Lai} & \textbf{0.57} & 0.18 & 0.19 \\
\textbf{Ziegen.} & 0.25 & 0.27 & \textbf{0.32} \\
\textbf{Alva-M.} & \textbf{0.59} & 0.03* & 0.02* \\
\textbf{Scialom} & \textbf{0.62} & 0.45 & 0.44 \\
\textbf{\begin{tabular}[c]{@{}l@{}}Our Test\end{tabular}} & \textbf{0.63} & 0.46 & 0.48 \\ \bottomrule
\end{tabular}
\caption{Style strength: Pearson correlation to human ratings. *not significant; we cannot reject the null hypothesis of zero corelation}
\label{tab:styleresults}
\end{table}

\subsection{Ablation}
We conduct several runs of the methods using LLMs with variations in instructions/prompts (App.~\ref{app:method}). We observe that the lower the correlation on a task, the higher the variation between the different runs. For our method, we only observe low variance between the runs.
None of the variations leads to different conclusions of the meta-evaluation. Results in App.~\ref{app:results}.
%%%%%%%%

\subsection{Pattern Identification}
\label{sec:npc_pattern_identification}

Drawing inspiration from probing classifier techniques widely used in NLP \cite{hewitt_designing_2019} and SE \cite{lopez_ast-probe_2022, troshin_probing_2022}, our framework leverages supervised machine learning techniques to identify patterns in the SHAP values computed for specific predictions in classification tasks. Probing techniques work by examining the latent representations of a model to determine the extent to which specific types of information are encoded. Specifically, a supervised model (\eg classifier) is trained to predict properties of interest from the neural network's hidden representations \cite{belinkov_probing_2021}. In the context of our framework, we propose training classifiers to predict target classes from SHAP value distributions enabling the formulation of symbolic rules, as illustrated in \figref{fig:npc_pipeline}.

First, given a set of inputs $\mathbb{X}$ that the \lcm predicts as belonging to a specific class $y \in Y$ (\eg Secure/Insecure), we compute SHAP values ($\phi$) for each input $x \in \mathbb{X}$. The SHAP values are calculated relative to the expected predicted class: $y = \mathbb{E}[f(\mathbb{X})]$. Inspired by syntax decomposition \cite{syntax_capabilities, palacio_towards_2024, docode}, we apply an alignment function $\delta(w_i): w_i \to \mu \in \mathbb{M}$ to tag tokens $w_i \in x$ with meaningful AST types $\mathbb{M}$, defined by the programming language grammar. This process produces a SHAP tensor for each target class: ${(i, w_i, \phi_i, \mu_i)}$, where $i$ is the position, $w_i$ is the token, $\phi_i$ is the SHAP value, and $\mu_i$ is the associated AST type. The entire process is depicted in region \circled{1} of \figref{fig:npc_pipeline}.

After computing the SHAP tensors for each target class in $Y$, we merge them and group the $\phi$ values by the AST tag associated with their corresponding tokens. We define position ranges as $[a, b], \quad 0 \leq a \leq b \leq \max{|x|: x \in \mathbb{X}}$. For each range, we train a supervised model (\eg logistic regression, decision tree, random forest) to identify curves that best capture the relationship between $\phi$ values and feature positions. Curves with an accuracy exceeding $60\%$ and a well-defined decision boundary for the target class (\ie intersection with the x-axis) provide evidence of patterns in specific AST type positions where SHAP values influence the model's decisions. The computed curves allow us to identify regions and position ranges where a feature’s $\phi$ value (\ie SHAP value corresponding to a specific AST node) consistently influences the overall prediction of the expected outputs either positively or negatively.

\subsection{Symbolic Rules}
From the identified patterns in SHAP value distributions, we derive symbolic rules encapsulating feature structures that align with expected model predictions. These rules consist of two parts: (i) configurations positively correlated with the predicted label, forming symbolic rules for correctly predicted patterns, and; (ii) complementary rules for configurations linked to lower prediction reliability, enabling targeted model adjustments in uncertain cases. We derive these rules by grouping SHAP-influential features within each type $\mu \in \mathbb{M}$ and formulating conditions based on both feature presence and SHAP value contributions. For instance, if a feature linked to an AST node consistently shows high SHAP values for insecure code at the input's start, it may represent a necessary condition for an \textbf{\textit{insecure}} prediction in the rule. As illustrated in region \circled{3} of \figref{fig:npc_pipeline}, the derived symbolic rules can be applied during the post-training stage of an ML pipeline, for instance, in supervised fine-tuning and knowledge distillation to facilitate knowledge transfer between models.
\section{RESULTS}

\begin{figure*}[t!]
    %\vspace{-0.5cm}
    \centering
    \includegraphics[width=1\linewidth]{images/SystemArchitecture_2.png}
    \caption{From a single user demonstration, the system extracts the desired task goal state with the help of user interaction to solve ambiguities. Using the created environment variation, the system computes a task execution plan to bring new environments into the goal state. It sends the plan to agents in the environment to execute.} \label{fig:system_architecture}
\end{figure*}

Figure \ref{fig:system_architecture} shows our proposed framework to define a task goal, i.e. an environment goals state, and to turn a given environment into this goal state. The system visually observes a task execution by a user and segments this \underline{single} demonstration into \skills. \actions\ and \skills\ are defined in \ref{ssec:actions_skills}. The demonstration changed one or several properties of entities in the environment; environment which is now in the goal state. This information and the differences in entity properties from the start and end environment states are used to represent the task goal state. More on that in \ref{ssec:exp_model_def}. To turn a new environment into the defined goal state, a planning problem must be solved. This entails computing the differences between the environment's current state and the goal state, finding \actions\ that solve these differences, instantiating \skills\ that implement the \actions\ in the environment, selecting the \skills\ to execute by minimizing a given metric, and finally, sending the \skills\ to the agents in the environment to execute. This process is detailed in \ref{ssec:exp_model_use}.

% To prove the usability of our model, we present experiments to create a new goal state and turn the current environment into an (already-defined) goal state.

\subsection{Actions and Skills}\label{ssec:actions_skills}
A change in the environment is modeled using \actions, i.e. \textbf{what} has happened, and \skills, i.e. \textbf{how} did the change happen \cite{conceptHierarchyGeriatronicsSummit24}. Like in STRIPS \cite{strips} and PDDL \cite{pddl}, we represent \actions\ by their effects on entity properties and \skills\ by their preconditions and effects. \actions\ do not need preconditions because they only describe the \textbf{what} part of a change, not which conditions must be satisfied to perform the change. Besides preconditions and effects, \skills\ have a list of checks that tell our system if the \skill\ is executed in the environment. These checks allow the creation of a \skill\ recognition program, like the one presented in \cite{conceptHierarchyGeriatronicsSummit24}.
%\todo{citation of Geriatronics summit paper or the journal/unsubmitted paper?}
Using the \skill\ recognition output, we capture the changes from a task demonstration.

A \skill\ is thus the physical enactment of an abstract \action\ in an environment. Hence, \skills\ are correlated with \actions\ via their effects. A \skill\ can have more effects than a corresponding \action. For example, the \skill\ of scooping jam from a jar with a spoon implements the \action\ of \textit{TransferringContents}, but it also \textit{Dirties} the spoon.

\subsection{How To Parameterize The Model}\label{ssec:exp_model_def}
Creating a new goal state should be easier than manually specifying all variations wanted from the goal state. Doing so requires programming knowledge, which should not be needed to define goal states. One can let the system, which knows how to represent goal states, question the user about the desired state of the environment. However, this tedious process requires many questions from the system, also leading to decreased system usability.

Therefore, our approach is to let the user turn a given environment into a desired goal state and analyze the differences between the initial and final environment state to create the goal state representation. This single demonstration highlights the entity property values that were not in the desired goal state before being changed by the user.

We capture the demonstration via an Intel Realsense 3D camera \cite{realsense}, analyze the human skeleton via the OpenPose human pose estimation method \cite{openpose}, and determine the 3d pose of objects with AprilTag markers \cite{aprilTag}.

One demonstration contains the initial environment, not in the task goal state, and the final environment, in the goal state. The final environment state alone is not enough to create the environment variation. Thus, additional questions, guided by the differences between the two environment values, are posed by the system to the user to determine the desired variation in the environment state.

In a demonstration in which the user pours milk into a bowl, as shown in the top of Figure \ref{fig:system_architecture}, the initial question posed to the user is which entities that have changed properties are relevant for the goal state. If the goal state is to have more milk in the bowl, the milk carton is irrelevant; it is a means to achieve the goal state but not relevant to the goal itself. The bowl is thus selected as a relevant entity. 

Next, the list of relevant modified properties must also be determined for each relevant entity. It could have happened that during pouring of the milk into the bowl, the bowl's location also changed, e.g. touched accidentally by the user. Thus, not all modified properties could be relevant to the task. After selecting the relevant properties, the system knows from the knowledge base \cite{conceptHierarchyGeriatronicsSummit24} their \textit{ValueDomain} and the list of implemented \textbf{variations} for that \textit{ValueDomain}. Thus, the user parametrizes a selected \textbf{variation} from the list: choosing either a fixed value, a \textit{ValueDomain}-specific \textbf{RangeVariation} that must be parametrized, a conjunction or disjunction of \textbf{RangeVariations}, or the whole \textit{ValueDomain}.

In the example above, the user chooses the \textit{contentLevel} property as relevant. The system knows this property's defined set of values: a non-negative real number, and the possible range variation types: an open interval, a closed interval, an open-closed or closed-open interval, an intersection or union of intervals, etc. The user chooses a closed interval of $[0.28, 0.32]$ around the final \textit{contentLevel} value of $0.3L$. The user also specifies a variation for the entity's concept. It is generalized from that specific bowl instance to a \textit{LiquidContainer}.

After each modified property of each entity has a represented \textbf{variation}, the system automatically collects the entities into a variation of type $A$, see \ref{ssec:variations}, which is the assigned \textbf{variation} for the collection of entities in the environment.

Thus, the environment variation is determined in $\mathcal{O}\left(n\times m \times p\right)$ questions to the user, where $n$ is the number of entities in the environment, $m$ is the maximal number of properties that an entity can have, and $p$ is the maximal number of parameters that a \textbf{RangeVariation} needs to be represented. In the example above, $10$ questions were necessary to determine the task goal state shown in Figure \ref{fig:system_architecture} of a \textit{LiquidContainer} with \textit{contentLevel} between $0.28$ and $0.32L$. Figure \ref{fig:task_goal_state} shows the internal JSON-like representation of the goal state as the environment variation.
% 1 question which entities are relevant -> just bowl
% 1 question which properties are relevant -> contentLevel and concept
% 1 question about concept values being the same; should create variation?
% 1 question which ConceptValue-variation to select -> ConceptValue in Environment
% 1 question: which generalized concept?
% 1 question -> add other range-variation
% 1 question which Number-variation to select -> Interval
% 1 question: min-bound?
% 1 question: max-bound?
% 1 question -> add other range-variation

\begin{figure}[t!]
    %\vspace{-0.5cm}
    \centering
    \includegraphics[width=1\linewidth]{images/TaskDefinition_7.png}
    \caption{The goal state is a \textbf{RangeVariation} of the environment, of type EnvironmentDataRangeEntityVariation, which contains a \textbf{variation} of entities. This sub-variation is a \textbf{RangeVariation} of type MapRangeInstanceSubset (\textbf{variation} of type $A$, see \ref{ssec:variations}) and contains one instance \textbf{RangeVariation} of type InstanceRangePropertiesVariation. It defines the instance's concept \textbf{RangeVariation}, a \textit{LiquidContainer} to be found in the environment, and the \textit{contentLevel} property \textbf{RangeVariation}, the closed interval $\left[0.28, 0.32\right]$.} \label{fig:task_goal_state}
\end{figure}

\subsection{How To Use The Model}\label{ssec:exp_model_use}
Assuming the representation of a task's goal state is given, i.e. an environment variation, we detail our procedure (see Figure \ref{fig:experiment_description}) to turn the current environment into the goal state.

First, a Comparison between the environment and the goal variation is computed. This leads, as described in \ref{ssec:comparisons}, to a list of reasons why the environment is not in the variation. These reasons, i.e. differences $\delta$ of concept properties $p$, must be fixed to turn the environment into the goal state.

% Computing the differences between an EnvironmentData and an EnvironmentData-Variation, that has a Collection-Variation of type $A$, see \ref{ssec:variations}, is done via a maximal matching algorithm, where an edge between an entity $e$ an an entity variation $v_e$ means $e \in v_e$. 
For an EnvironmentData-Variation $v_{env}$ that defines a Collection-RangeVariation of type $A$, see \ref{ssec:variations}, computing the Comparison between an EnvironmentData $env$ and this target $v_{env}$ leads to a list of reasons for each entity $e_{env}$ in the entity collection of $env$, why $e_{env} \not\in v, \forall v \in A$. This can be seen in Figure \ref{fig:experiment_description}, where for each entity of \textit{LiquidContainer} concept in the environment, there is a list of differences, i.e. Comparisons, created for why the respective entity does not match the defined variation on the top-right.

\begin{figure}[t!]
    %\vspace{-0.5cm}
    \centering
    \includegraphics[width=1\linewidth]{images/Experiment_DescriptionUsingVariations_2.png}
    \caption{The procedure to turn an environment into its goal state is divided into 5 steps: computing differences, finding abstract solutions (i.e. \actions), computing practical solutions for the abstract ones (i.e. \actions\ $\rightarrow$ \skills), selecting the best practical solution, and executing the solution.} \label{fig:experiment_description}
\end{figure}
The second step of the procedure is to turn the list of differences into a list of \actions\ that can fix them. In notation, \action\ $A_x$ solves a difference in the concept property $p_x$. The system knows which properties \actions\ modify by analyzing the definition of their effects. Thus, \actions\ are created (parametrized) to fix the differences in entity properties.

% Because multiple instances can fit the instance variation, the third step is to match instances with the variations. Our matching optimization criterion is to minimize the amount of \textit{Actions} needed to fix the instances' property differences. \todo{continue!}

In the third step, each \action\ $A_x$ is converted into an execution plan $P_x$ that implements solving the difference $\delta_{p_x}$ in the environment. It is also possible that there is no possibility to implement the \action\ $A_x$ in the environment; this is represented as an execution plan $P_x = \emptyset$. An execution plan $P_x$ is otherwise, in its simplest form, a set of \skill\ alternatives $\left\{S_y\right\}$, where the \skill\ $S_y$ implements the \action\ $A_x$. There is the case to consider that the \skill\ $S_y$ has preconditions that are not met. And so, before executing the skill $S_y$, a different execution plan $P_{S_y}$ has to be computed and executed to allow the \skill\ $S_y$ to solve the property difference $\delta_{p_x}$. It is also possible that one single \skill\  $S_y$ is not enough to implement the \action\ $A_x$. Consider the case where the environment contains three cups with $0.1L$ of water, and the goal is to have one cup with $0.3L$ of content. One single \textit{Pouring} \skill\ is not enough to fulfill the goal; two \textit{Pouring} \skills\ must be executed. Thus, in the most general form, an execution plan $P_x = \left[\left\{ S_{iy}, P_{S_{iy}} \right\}_i\right]$ is a list of skill alternatives $\left\{ S_{iy}, P_{S_{iy}} \right\}_i$, that possibly contain other execution plans $P_{S_{iy}}$ to solve the skill's preconditions.

Our procedure to parameterize the \skills\ $S_y$ that implement the \action\ $A_x$ is a custom solution for each property $p_x$. One could backtrack through all possible parameter values of all possible skills to create a general solution that works for all properties. Another idea is to invert \skill\ effects and thus guide the \skill\ parameter search from the target variation to the value. However, both approaches would be computationally intense and would not create execution plans in a reasonable time. 
% reinforcement learning with policy for each property

The procedure to solve an entity $e$'s \underline{contentLevel} property difference searches for other \textit{Container} object instances in the environment, sorts them according to their content volume, and iterates through them in ascending order if $e.contentLevel \le target.contentLevel$; otherwise, in descending order. If a \skill\ $S$ can be executed with the two objects, that reduces the difference between $e.contentLevel$ and $target.contentLevel$, the \skill\ is added to the execution plan. If, after checking all objects, $e.contentLevel \not\in target.contentLevel$, there is no solution to solve this property difference.

Thus, the result of the third step is an execution plan $P_x$ for each entity property difference.

Fourth, after having the execution plans $P_x$ per entity-variation and entity, a \underline{solution selector} scores all solutions according to defined metrics and then, via a maximal matching algorithm, selects the solutions to execute to satisfy all variations of the Collection-RangeVariation of type $A$. The edges in the maximal matching have the cost of the solution score. For this paper, the scoring metric by the \underline{solution selector} is the number of steps of the execution plan.

The fifth and final step is to pass the execution plan to the agent(s) to execute in the environment. Figure \ref{fig:data_flow} presents the flow of data through the five steps.
We have used the Franka Emika Panda robot in CoppeliaSim \cite{coppeliaSim} to perform the computed execution plan.
% Note that the approach is independent of the used robot; only when instantiating \skills\ must the robot's abilities, manipulability region, and workspace be considered. How the \skills\ are executed in the environment is separated from the modeling of what must be done.

\begin{figure}[t!]
    % \vspace{-0.2cm}
    \centering
    \includegraphics[width=1\linewidth]{images/Experiment_DescriptionUsingVariations_DataFlow_2.png}
    \caption{Data flow when transforming an environment into a given goal state. $\Delta$ are differences of entity properties $p$, $A$ are \actions, $P$ is an execution plan and $S$ are \skills.} \label{fig:data_flow}
\end{figure}

The experiments aim to compute solution plans for solving the difference of the \textbf{contentLevel} property of \textit{Container} objects. For this, we consider the following criteria. $C1$: \textbf{variation} type = $\left\{\text{fixed},\text{interval},\text{interval union}\right\}$. $C2$: target relative to content = \{$\left\{t < cL \le cV \right\}$, $\left\{cL < t < cV \right\}$, $\left\{cL < t \ni cV \right\}$, $\left\{cL \le cV < t \right\}$\}, where $t$ is the \textbf{variation} value and $cL$ and $cV$ are the \textit{contentLevel} and \textit{contentVolume} properties respectively. $C3$: achievable in environment $ = \left\{\text{yes}, \text{no}\right\}$. Figure \ref{fig:experiment_table} presents planning results for different environments and the criteria described above. The lower table shows cases where the computed solution does not match the actual solution. This only happens when multiple instance variations are defined. The reason is that the implemented procedure to turn the list of differences into an execution plan treats each difference independently. Thus, dependencies between two variations are not accurately solved.

In the upper table of Figure \ref{fig:experiment_table}, there are two solutions for $C1.3$, $C2.3$, $C3.1$: one with the bowl $B$ as the instance in the \textbf{variation} $V1$, the other with $M$. The solution when $B$ is the matched instance has three steps: 1) pouring $0.1L$ from $M$ into $B$, 2) pouring $0.1L$ from $C1$ into $B$, and, finally, 3) pouring  $0.02L$ from $C2$ into $B$. This plan is sent to the robot in simulation and is executed as shown in Figure \ref{fig:robot_plan_execution}.

% \begin{figure}[t!]
%     % \vspace{-0.2cm}
%     \centering
%     \includegraphics[width=1\linewidth]{images/Experiment_Table_1Variation_compressed.png}
%     \caption{$B$ is a bowl with $0.5L$ \textit{contentVolume}, $M$ is a milk carton with $1.0L$ \textit{contentVolume}, $C1$ and $C2$ are cups with $0.3L$ \textit{contentVolume} each. Times, in seconds, averaged across 10 runs. Criteria $C2.4$ and $C3.1$ are mutually exclusive (a solution does not exist to let a container have more \textit{contentLevel} than its \textit{contentVolume}); thus, they are not included in the table.} \label{fig:experiment_table}
% \end{figure}
\begin{figure}[t!]
    % \vspace{-0.2cm}
    \centering
    \includegraphics[width=1\linewidth]{images/Experiment_Table_Results.png}
    \caption{$B$ is a bowl with $0.5L$ \textit{contentVolume}, $M$ is a milk carton with $1.0L$ \textit{contentVolume}, $C1$ and $C2$ are cups with $0.3L$ \textit{contentVolume} each. Times, in seconds, averaged across 10 runs. Criteria $C2.4$ and $C3.1$ are mutually exclusive (a solution does not exist to let a container have more \textit{contentLevel} than its \textit{contentVolume}); thus, they are not included in the upper table. The lower table presents results for open intervals and multiple variations in the environment.} \label{fig:experiment_table}
\end{figure}

\begin{figure}[t!]
    %\vspace{-0.1cm}
    \centering
    \includegraphics[width=1\linewidth]{images/Robot_PouringInBowl_M_PC1_PC2.png}
    \caption{Robot executing plan to bring $B$, the bowl, into the goal state. Because no liquids were simulated, the pouring amount was associated with the pouring time via: $t_{pour} = 10 * amount_{pour}$.} \label{fig:robot_plan_execution}
\end{figure}
\section{Conclusion}
\label{sec:conclu}

In this study, we propose a retrieval-augmented approach to extend LLM-based TabICL from zero-shot and few-shot settings to any-shot scenarios.
This approach explores the potential of using text representations for tabular data learning, enables the creation of unique decision boundaries, and achieves highly competitive prediction performance across most tabular datasets.

Despite the unique strengths and promising potentials, we also acknowledge the limitations of this approach at the current stage, such as the absence of a universally effective retrieval policy, challenges in handling certain long-tail data distributions, and sub-optimal performance in several scenarios.
Given the demonstrated strengths of this approach, we believe that the potential of LLM-based TabICL is still in its early stages, and these limitations present valuable opportunities for future research and development.

\bibliographystyle{IEEEtran}
\bibliography{references}

\end{document}
