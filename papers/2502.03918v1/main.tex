%%%%%%%%%%%%%%%%%%%%%%%%%%%%%%%%%%%%%%%%%%%%%%%%%%%%%%%%%%%%%%%%%%%%%%%%%%%%%%%%
%2345678901234567890123456789012345678901234567890123456789012345678901234567890
%        1         2         3         4         5         6         7         8

\documentclass[letterpaper, 10 pt, conference]{ieeeconf}  % Comment this line out if you need a4paper

%\documentclass[a4paper, 10pt, conference]{ieeeconf}      % Use this line for a4 paper

\IEEEoverridecommandlockouts                              % This command is only needed if 
                                                          % you want to use the \thanks command

\overrideIEEEmargins                                      % Needed to meet printer requirements.

%In case you encounter the following error:
%Error 1010 The PDF file may be corrupt (unable to open PDF file) OR
%Error 1000 An error occurred while parsing a contents stream. Unable to analyze the PDF file.
%This is a known problem with pdfLaTeX conversion filter. The file cannot be opened with acrobat reader
%Please use one of the alternatives below to circumvent this error by uncommenting one or the other
%\pdfobjcompresslevel=0
%\pdfminorversion=4

% See the \addtolength command later in the file to balance the column lengths
% on the last page of the document

% The following packages can be found on http:\\www.ctan.org
% The following packages can be found on http:\\www.ctan.org
\usepackage{graphics} % for pdf, bitmapped graphics files
\usepackage{epsfig} % for postscript graphics files
\usepackage{mathptmx} % assumes new font selection scheme installed
\usepackage{times} % assumes new font selection scheme installed
\usepackage{amsmath} % assumes amsmath package installed
\usepackage{amssymb}  % assumes amsmath package installed
\usepackage{xcolor}
\usepackage{colortbl}
\usepackage{url}
\usepackage{gensymb}
% \usepackage{algorithm2e}
\usepackage{algorithm}
\usepackage{algpseudocode}
\usepackage{listings}
\usepackage{caption}

\usepackage{hyperref}
\hypersetup{
    colorlinks=true,
    linkcolor=blue,
    filecolor=magenta,      
    urlcolor=blue
}
\urlstyle{same}

\hyphenation{op-tical net-works semi-conduc-tor IEEE-Xplore}
\def\BibTeX{{\rm B\kern-.05em{\sc i\kern-.025em b}\kern-.08em
    T\kern-.1667em\lower.7ex\hbox{E}\kern-.125emX}}
\usepackage{balance}

\newcommand{\real}[0]{\ensuremath{\mathbb{R}}}
\newcommand{\commenting}[1]{}
\newcommand{\todo}[1]{\textbf{TODO: #1}}
\newcommand{\newAddition}[1]{{\color{magenta} NEW ADDITION: #1}}
\newcommand{\toShorten}[1]{{\color{blue} #1}}
\newcommand{\myHighlight}[1]{{\color{red} #1}}

\newcommand{\skill}[0]{\textit{Skill}}
\newcommand{\skills}[0]{\textit{Skills}}
\newcommand{\action}[0]{\textit{Action}}
\newcommand{\actions}[0]{\textit{Actions}}

\definecolor{Gray}{gray}{0.85}
\definecolor{LightCyan}{rgb}{0.88,1,1}
\definecolor{White}{rgb}{1,1,1}

\newcolumntype{g}{>{\columncolor{Gray}}c}

% THESE ARE COMMANDS TO SHRINK THE OUTPUT PDF SIZE

\setlength\abovedisplayskip{2pt}
\setlength\belowdisplayskip{2pt}
\setlength\abovedisplayshortskip{2pt}
\setlength\belowdisplayshortskip{2pt}

\captionsetup[figure]{font=small,skip=4pt}

\setlength{\dbltextfloatsep}{0pt}
\setlength{\textfloatsep}{1pt}
\setlength{\skip\footins}{0.3ex}

\makeatletter%change heading spacings
\renewcommand{\section}{\@startsection{section}{1}{\z@}{1.0ex plus 1.0ex minus 0.5ex}%
	{0.5ex plus 1ex minus 0ex}{\normalfont\normalsize\centering\scshape}}%
\renewcommand{\subsection}{\@startsection{subsection}{2}{\z@}{1.0ex plus 1.0ex minus 0.5ex}%
	{0.5ex plus 1ex minus 0ex}{\normalfont\normalsize\itshape}}%
\makeatother

\title{\LARGE \bf
% Representing Task Goal States as Variations of Environment States
Adaptation of Task Goal States from Prior Knowledge
}


\author{Andrei Costinescu and Darius Burschka$^{1}$% <-this % stops a space
\thanks{*This work was supported by the Lighthouse Initiative Geriatronics by StMWi Bayern (Project X, grant no. 5140951).}% <-this % stops a space
\thanks{$^{1}$All authors are with the School of Computation, Information and Technology at the Technical University of Munich. 
        {\tt\small \{andrei.costinescu, burschka\}@tum.de}}%
}


\begin{document}



\maketitle
\thispagestyle{empty}
\pagestyle{empty}


%%%%%%%%%%%%%%%%%%%%%%%%%%%%%%%%%%%%%%%%%%%%%%%%%%%%%%%%%%%%%%%%%%%%%%%%%%%%%%%%
\begin{abstract}

This paper presents a framework to define a task with freedom and variability in its goal state. A robot could use this to observe the execution of a task and target a different goal from the observed one; a goal that is still compatible with the task description but would be easier for the robot to execute. We define the model of an environment state and an environment \textbf{variation}, and present experiments on how to interactively create the \textbf{variation} from a single task demonstration and how to use this \textbf{variation} to create an execution plan for bringing any environment into the goal state.
\end{abstract}


%%%%%%%%%%%%%%%%%%%%%%%%%%%%%%%%%%%%%%%%%%%%%%%%%%%%%%%%%%%%%%%%%%%%%%%%%%%%%%%%
\section{Introduction}\label{sec:intro}

In computational finance, Monte Carlo simulations are used extensively to estimate the expected value of financial payoffs based on the solution of stochastic differential equations (SDEs) which model the evolution of stock prices, interest rates, exchange rates and other quantities \cite{glasserman04}.  Monte Carlo methods are very general and flexible, but for high accuracy it requires generating a large number of costly SDE path approximations, which has motivated research into a number of variance reduction or, equivalently, cost reduction techniques. One such method is
Multilevel Monte Carlo (MLMC), which was proposed in \cite{GILES2008} and was adapted for various applications that are summarised in \cite{Giles_overview17} and successfully combined with other methods such as quasi-Monte Carlo methods. The main idea of MLMC is to approximate the payoff using different time stepping resolutions when numerically solving the underlying SDE and to generate an optimal number of samples on each level, such that the overall computational cost is minimised subject to the desired bound on the variance. %, such that the total computational cost is minimised. 
The computational savings come from the fact that most samples are computed on the coarser levels and hence are less expensive while only a few samples from the finest levels are required \cite{GILES2008}.


Among the directions in which the computational cost 
of MLMC methods could further be reduced, an important avenue is the use of lower precision calculations, especially for the first Monte Carlo levels where the targeted accuracy is relatively low. 
 An overview of the research on mixed precision for the standard Monte Carlo (MC) framework is provided in \cite{ChowMixedPrecisionStandardMC} but only a few references study the potential of low precision computation in the MLMC framework \cite{Rounding_error_oliver}. To the best of our knowledge, the only MLMC framework with customised precision in the literature is \cite{brugger2014mixed}, but they use a uniform precision for all operations on each Monte Carlo level instead of optimising 
 the precision of each intermediary variable to reduce as much as possible the cost of path generation.
 
An important motivation for an MLMC framework with variable precision would be performing the low precision computations on reconfigurable hardware devices such as Field Programmable Gate Arrays (FPGAs). FPGAs contain customizable logic blocks and connectors that make it easy to adapt the digital circuit architecture for a specific application, leading to a highly parallel and optimised implementation. Therefore they are successfully exploited in applications that require high speed and have high computational workload, such as signal processing \cite{woods2008fpga}, and real time applications like high frequency trading \cite{HFT1,HFT2}. That is why a number of previous works in hardware architecture design implemented the MLMC algorithm to price financial options using FPGAs as accelerators, which resulted in improved speed and power efficiency compared to full CPU architectures \cite{Schryver2013AMM}. The paper \cite{lindsey2016domain} also proposed 
a Domain Specific Language to automate the configuration of FPGAs for this specific application. However, only \cite{brugger2014mixed} proposed a heuristic to reduce the precision in calculations.

In addition, all aforementioned works considered that the random number generation (RNG) is performed in single or double precision. Yet in most cases an important portion of the workload in the overall MLMC simulation comes from the RNG and in \cite{brugger2014mixed} this limited the total computational savings.
To reduce the cost of MLMC simulations in particular those based on the Geometric Brownian Motion (GBM), \cite{approximateICDF_Oliver, NestedOliver} have proposed to use approximate random numbers that are generated by applying an approximation of the inverse CDF to uniform random numbers. In \cite{NestedOliver}, the authors proposed a way to integrate these lower precision random variables into a \textit{nested} MLMC framework and completed a numerical analysis to bound the resulting error at each MC level by a product of the time step and the error in the random number approximation. The same authors show in \cite{approximateICDF_Oliver} that using approximate random variables reduces the cost of path generation by a factor 7.


In this paper we propose a nested MLMC framework that combines the use of approximate random normal variables and lower precision calculations to reduce the computational cost of MLMC even further than \cite{brugger2014mixed,NestedOliver}. We illustrate the efficiency of our framework in Matlab, after making several assumptions on the cost of operations and size of the errors that we carefully justify. We focus on the case of GBM and use the approximate RNG methods presented in \cite{approximateICDF_Oliver} as well as a new slightly modified method that combines CDF inversion and the central limit theorem. To choose the precision of the variables in the low precision path generation, we introduce a novel method to optimise the bit-widths. This optimisation is performed before the main path generation loop is executed and is based on a linear model of the payoff error  
due to rounding when computing in low precision. The error model relies on algorithmic differentiation in a similar manner to \cite{unifying-bwoptim,bitwidth-AD,ADAPT}. The bit-width optimisation procedure can be performed off-line, so this stage can be excluded from the on-line time complexity of our framework. The user specified desired accuracy is then enforced by calculating on-line the number of samples that need to be generated.

In terms of hardware design, we suggest implementing the low precision path generation on FPGAs and the full-precision ones on a CPU or GPU. 
The FPGA offers enough flexibility to define a separate bit-width for every variable in the low precision path generation, and can be reconfigured periodically to update the bit-widths when the market parameters have changed considerably. 


The paper is organized as follows : \Cref{sec:MLMC} introduces MLMC and nested MLMC to make clear the estimator that is implemented in our framework. Then in \Cref{sec:RNG} we detail the methods that could be used to obtain approximate random normally distributed numbers very cheaply for the low precision path generation. In \Cref{sec:error_model} and \Cref{sec:costModel} we propose an error model and a cost model (resp.) that we then use to formulate the optimisation problem that is solved to obtain the optimal bit-widths of fixed point variables in \Cref{sec:optimisation}. Finally we summarise our results and future directions in \Cref{sec:conclusion}.



% !TeX root = 0_main

\section{\tool: Automated S2R Quality Assessment}
\label{sec:approach}
\begin{figure*}[t]
		\vspace{-2em}
		\centering
		\includegraphics[width=1\linewidth]{figures/approach.pdf}
		\caption{{The \tool Approach}}
		\label{fig:approach}
\end{figure*}

This section presents \tool, an automated approach that leverages an LLM and a graph-based app execution model to assess the quality of the steps to reproduce (S2Rs) in textual bug reports.  
\tool identifies, extracts, and processes the S2Rs from a bug report to detect which ones are correct, ambiguous, missing, or phrased using language that does not correspond to a target app, according to the quality model described in \Cref{sec:quality_model}.
 \tool generates a quality report with annotations that provide feedback to the reporter about problematic S2Rs and includes generated missing S2Rs.
\tool has four main components, as illustrated in \Cref{fig:approach}:
\begin{enumerate}
	\item \textbf{S2R sentence identification}: \tool identifies the sentences that describe any S2Rs (\Cref{sec:identification-phase}).
	\item \textbf{Individual S2R extraction}: \tool extracts phrases describing individual S2Rs from S2R sentences (\Cref{sec:indiv-s2rs-approach}).
	\item \textbf{App execution model generation}: \tool builds a graph-based model using automated and manual app execution (\Cref{sec:execution-model}).  
	\item \textbf{S2R quality assessment}: \tool maps individual S2Rs to GUI-level interactions captured in the app execution model, providing feedback about high- and low-quality S2Rs as well as missing steps in a quality report (\Cref{sec:quality-assessment-annotations}).
\end{enumerate}

We leverage the language processing capabilities of LLMs \rev{(\ie\ GPT-4)} across the three phases, integrating these with GUI-level dynamic app analysis to assess S2R quality. \rev{The selection of GPT-4 as the LLM was based on its demonstrated effectiveness in language and bug understanding tasks, including bug reproduction~\cite{Feng2024} and analysis~\cite{Bo2024}.}
In the remainder of this section, we detail \tool's components or phases.



\subsection{S2R Sentence Identification Phase}
\label{sec:identification-phase}

\tool automatically identifies sentences that describe any steps to reproduce (S2R) in the bug report (see the blue sentences in Fig. \ref{fig:bug-report}). 
This is necessary as the bug report typically includes other content, notably the observed (OB) and expected app behaviors (EB). 
We formulate this task as a text classification task, using LLMs. 
\tool decomposes the bug report into a list of sentences and asks the LLM to identify which of these sentences describe any S2Rs. 
Sentence parsing is done using the Stanford CoreNLP toolkit~\cite{manning2014stanford} and heuristics. 
We experimented with three types of prompts, each one providing a different context to facilitate the task for the LLM (\eg the definition of S2Rs and guidelines on how to distinguish them from other content like the OB and EB). 
\Cref{sec:prompt_development} describes the process we followed to develop and evaluate these prompts.
\looseness=-1



\subsection{Individual S2R Extraction Phase}
\label{sec:indiv-s2rs-approach}

After identifying S2R sentences, \tool asks the LLM to extract the individual S2Rs from these sentences in a particular format (described below). 
Individual S2Rs are phrases that describe a single, atomic interaction with the app. 
Individual S2R extraction is needed because S2R sentences may describe multiple interactions with the app together with content such as the OB (\eg ``I opened the app and clicked on the Start button'' or ``The app crashes if the user checks the Angle Box''). 
In addition, different S2R sentences may describe the same interaction (\eg "... the user checks the Angle Box" and "give the Exercise a name and check the Angle Box"). 
\tool resolves this redundancy by asking the LLM to provide only one S2R among all extracted individual S2Rs that describe the same interaction.
\looseness=-1

The output of this phase is a list of individual S2Rs extracted in the order they appear in the sentences from left to right and top to bottom. 
\tool asks the LLM to represent the individual S2Rs in the following format: 
\texttt{\small[action][object][preposition][object2]}. 
The \texttt{\small[action]} is a verb associated with the app interaction (tap, long tap, enter, \etc). 
The \texttt{\small[object]} is the GUI component upon which the action is performed. 
The \texttt{\small[object2]} is additional information related to the object connected by a \texttt{\small[preposition]}. 
For example, the S2R ``Click any button on this page" is formatted as \texttt{\small[Click] [any button] [on] [this page]}.

We designed and evaluated three prompt types to extract individual S2Rs via GPT-4. 
Each prompt implements a different approach, providing different contexts about the task (\eg examples that illustrate how to accomplish the task). 
\Cref{sec:prompt_development} details the prompts and the process we followed to design and evaluate them.
\looseness=-1
 
\subsection{App Execution Model Generation Phase}
\label{sec:execution-model}

\tool's quality assessment relies on mapping individual S2Rs to interactions that can be executed on the app to replicate the reported bug. 
This requires collecting and representing possible user GUI interactions, for which we adapt graph-based representations and dynamic app execution strategies from prior work~\cite{song2022toward,saha2024toward}.

\tool creates an app execution model represented as a directed graph, $G = (V, E)$, where $V$ represents the set of unique GUI screens for an app, and $E$ represents the set of unique interactions that users can perform on the GUI components of the screens. 
A GUI screen (\ie node) is represented as a hierarchy of the GUI components and layouts. 
Two GUI screens with different GUI component hierarchies are considered distinct graph nodes. 
Each interaction (\ie edge) in $E$ is represented by a unique tuple in the form of ($v_x, v_y, e, c)$, where $c$ is a GUI component of screen $v_x$ and  $e$ in an action (tap, type, \etc) performed on $c$, resulting in a transition to another screen $v_y$. 
Each edge contains additional interaction metadata such as the interacted GUI component type, ID, text (\ie label), and description. 

To build the execution model for an app, \tool parses GUI interaction traces collected from automated app exploration and manual app usage.  \tool executes an adapted version of the \CrashScope tool~\cite{Moran2016, Moran2017}, which implements multiple
automated exploration strategies to interact with the UI components of app screens, trying to exercise as many app screens and GUI components as possible.
In the process, \CrashScope collects app screenshots and XML-based GUI hierarchies and metadata for the exercised app UI screens and components. 
As \CrashScope may \rev{fail} to interact with certain GUI screens and components that app users would normally interact with, \tool can also make use of interaction data collected from manual app usage and testing. 
In this paper, for the development set, we used the set of traces collected by Saha \etal~\cite{saha2024toward} which consists of 10-12 manually recorded feature interaction traces for each of the 5 test applications. For the prompt development dataset, two authors collected the same number of traces for each of the 31 apps. These recordings include all the app GUI interactions starting from launching the application to the last step related to carrying out an application feature (more details of this process, used for prompt development and evaluation, are found in \Cref{sec:dev_dataset}). 
In practical applications of \tool, manual executions can be collected in several ways.
For example, developers can enable user monitoring features in the app and perform record-and-replay during in-house or crowd-sourced app testing~\cite{du2022semcluster}. 
\tool parses the interaction traces generated by \CrashScope and the traces collected during app usage/testing to build the graph, according to the graph format we previously described in this section (details found in \Cref{sec:dev_dataset}). 
 
\subsection{S2R Quality Assessment Phase}
\label{sec:quality-assessment-annotations}

The app execution model captures possible interaction sequences that a user could perform when using or testing an app as paths in the graph. 
To assess the quality of the S2Rs, \tool attempts to map each individual S2R to interactions (\ie edges) along these paths. 
To do so, \tool implements an LLM-guided depth-first-search (DFS) graph traversal 
to establish the correspondence between an individual S2R and interactions on a given screen. 

Any S2Rs that cannot be mapped to a graph interaction are labeled as having a Vocabulary Mismatch (\textbf{VM}).  
S2Rs that map to multiple interactions performed on a single screen (\ie a node) are labeled as Ambiguous Steps (\textbf{AS}). 
Those that map to single interactions within a sequence are labeled as Correct Steps (\textbf{CS}). 
Finally, for the mapped S2Rs that correspond to non-consecutive interactions spanning different screens in a path, additional interactions are required to connect them to form a complete path. 
These additional interactions are used to generate individual S2Rs that are labeled as Missing Steps~(\textbf{MS}) and used to fill in the "gaps" between the existing S2Rs.

\subsubsection{Mapping Individual S2Rs to Interactions on a Screen}
\label{sec:qualtiy_phase:mapping_single_screen}
Mapping an individual S2R (S2R, hereon)
to interactions on a given screen is supported by GPT-4. 
For a graph node (\ie a screen), \tool asks GPT-4 to identify which of the outgoing edges (\ie interactions) from that node correspond to the S2R. 
Both the S2R and graph interactions are represented textually: the S2R is extracted from the bug report, while each interaction is represented as a tuple of textual information (\eg the event description and the label of the interacted GUI component).  
We designed and evaluated a set of prompts using different prompting strategies to accomplish this mapping in a 2-step manner: a first prompt asks GPT-4 to return a yes/no answer on whether an individual S2R maps to the interactions of a given screen and if the answer is yes, a second prompt asks GPT-4 to return the list of corresponding interactions. The methodology used to develop and evaluate the prompts is detailed in \Cref{sec:prompt_development_methodology}. 

\subsubsection{Graph Traversal and S2R Mapping to Interaction Paths}
\label{sec:graph-traversal}

To map all the S2Rs from a bug report to app interaction sequences, \tool implements an algorithm that traverses the graph in a depth-first-search (DFS) manner, aiming to \rev{map} the S2Rs to interactions along the DFS paths. 
When S2Rs map to non-consecutive interactions within a path, \tool connects these interactions by selecting the shortest path between the nodes where these interactions occur. 
Since multiple paths may map to the S2Rs, \tool selects the path with the most mapped S2Rs or the shortest path, if multiple paths have the same number of mapped S2Rs.

The DFS traversal of the graph is guided by the LLM-based mapping approach from \Cref{sec:qualtiy_phase:mapping_single_screen}, as only edges that map to S2Rs are traversed, avoiding the need to explore the entire graph. 
While none of the S2Rs can map to any interaction in the graph (in which case \tool would traverse the graph entirely), this scenario is expected to be rare, as we assume reporters would describe at least one S2R using the app’s vocabulary and the graph is as complete as possible, covering a broad range of screens and interactions.

\textbf{\textit{Algorithm Details.}} 
\tool's DFS-based graph traversal algorithm is recursive. 
It receives an S2R $s$ and graph node~$n$ as input, where $s$ is the first item in the S2R list $L$ (the bug report S2Rs). 
The algorithm returns either: the best DFS path $p$ (starting from $n$) that maps to a subset of S2Rs in $L$ (possibly including $s$), or no path if no S2Rs can be mapped. 

The traversal begins with the first S2R from the bug report and the starting node of the graph, which contains "\textit{open app}" interactions that navigate to the screens users usually see upon launching the application. 

The algorithm has two main logic branches:
\begin{enumerate}
	\item If S2R $s$ does not map to any of the outgoing interactions~$I$ from $n$, the algorithm recurses, attempting to map  $s$ on each node connected to $n$ by $I$.  If this traversal results in no DFS paths mapped to $s$ or following S2Rs in $L$, $s$ is labeled as having a Vocabulary Mismatch (\textbf{VM}), and the algorithm recurses with the next S2R in $L$ at the current node $n$. 
	This means that the S2R $s$ cannot be mapped to any node in the (sub)graph starting from $n$, then the algorithm attempts to map the next S2R.
	\item Conversely, if $s$ maps to interactions in $I$, the algorithm checks whether there are one or more mapped interactions. If there is a single interaction, $s$ is labeled as a Correct Step~(\textbf{CS}); if there are multiple, it is labeled as an Ambiguous Step (\textbf{AS}). 
	The algorithm then recurses with the next S2R in $L$ on each node connected to $n$ by only the mapped interactions from $I$. 
	Essentially, if the algorithm succeeds at mapping $s$ to interactions from $n$, then it proceeds with attempting to map the next S2R to the resulting nodes after navigating to the mapped interactions.
\end{enumerate}

It is possible that $s$ maps to interactions in $I$ (second branch above), but there are "gaps" between the previous mapped S2R and $s$: if their mapped interactions are not consecutive in the DFS path. 
If this is the case, the algorithm connects them by determining the shortest path between the involved nodes. 
The interactions used to connect the nodes are then labeled as Missing Steps (\textbf{MS}). 
Note that this shortest path may include interactions outside the DFS path, as we are not limiting the shortest path search to the DFS path alone. A shorter path may exist that bypasses parts of the DFS path.

After traversing a node with a given S2R (in either branch above), it is possible that when calling the algorithm recursively on a set of interactions (\ie when navigating down DFS paths), it returns multiple DFS paths mapped to the S2Rs. 
If this is the case, the algorithm selects the DFS path to return based on the following criteria: prioritizing the path with the most mapped S2Rs in $L$, or, if paths have the same number, choosing the shortest path.
\looseness=-1

The traversal continues until all S2Rs in $L$ have been exhausted or until none of the S2Rs are mapped to any DFS paths. 
If all S2Rs have been mapped, but there are still nodes along a DFS path, the algorithm does not proceed to check additional nodes down the current DFS path.
To prevent re-processing nodes and their interactions, the algorithm marks each (node, S2R) pair as visited before it processes the node and S2R.
\looseness=-1

\subsubsection{Quality Report Generation}

The returned DFS path contains interactions mapped to all or a subset of the S2Rs from the bug report. Each S2R is labeled as either a Correct Step (CS), Ambiguous Step (AS), or Vocabulary Mismatch Step (VM).  
In addition, interactions identified to fill in the "gaps" between S2Rs are labeled as Missing Steps (MS). 
For evaluation purposes, we also mark the corresponding S2Rs with missing steps as MS, so that we can perform a fine-grained analysis of results (more details found in Section \ref{sec:empirical_evaluation}).

\section{Experimental Results}
We demonstrate the effectiveness of STAIR through extensive experiments on multiple benchmarks that reflect both the safety guardrails and general capabilities of LLMs. 

\subsection{Experimental Settings}

We hereby introduce the key experimental settings, with more details explained in~\cref{sec:appendix_data} and~\ref{sec:appendix_exp}.


\textbf{Models and Datasets.} We take two base LLMs for safety alignment, LLaMA-3.1-8B-Instruct~\cite{dubey2024llama} and Qwen-2-7B-Instruct~\cite{qwen2}. For test-time scaling and ablation studies, only LLaMA is utilized. All experiments use a seed dataset $\mathcal{D}$ comprising 50k samples from three sources. For safety-focused data, we use a modified version of 22k preference samples from PKU-SafeRLHF~\cite{ji2024pku} along with 3k jailbreak data from JailbreakV-28k~\cite{luo2024jailbreakv}. Additionally, 25k pairwise data are drawn from UltraFeedback~\cite{cui2024ultrafeedback} to maintain helpfulness, as done in prior works~\cite{qi2024safety,wu2024thinking}. Note that responses in $\mathcal{D}$ are in normal conversational style rather than reasoning-oriented. While we use the whole dataset with labels for training baselines, we only take 10k samples each from PKU-SafeRLHF and UltraFeedback to construct structured CoT data $\mathcal{D}_{\text{CoT}}$. During each self-improvement iteration, 5k safety and 5k helpfulness samples are utilized. Jailbreak prompts are used only in the final two iterations, with 1k and 2k samples, respectively.

\textbf{Baselines.} We first evaluate the performance of CoT prompting~\cite{wei2022chain} to assess the contribution of available reasoning capability to safety consolidation. We then include SFT and DPO~\cite{rafailov2024direct} on standard datasets as representative alignment techniques, both of which are employed in our framework. Besides, SACPO~\cite{wachi2024stepwise}, designed to mitigate the safety-performance trade-off with two-step DPO, and Self-Rewarding~\cite{yuanself}, which leverages self-generated and self-rewarded data in iterative DPO, are also used as baselines for comparison.


\textbf{Evaluation.} We use 10 popular benchmarks to evaluate harmlessness and general performance of the trained models. For harmlessness, models are required to provide clear refusals to harmful queries, following~\cite{guan2024deliberative}. We test the models on StrongReject~\cite{souly2024strongreject}, XsTest~\cite{rottger2023xstest}, highly toxic prompts from WildChat~\cite{zhaowildchat}, and the stereotype-related split from Do-Not-Answer~\cite{wang2023not}. We report the average goodness score on the top-2 jailbreak methods of PAIR~\cite{chaojailbreaking} and PAP~\cite{zeng2024johnny} for StrongReject, and refusal rates for the rest. For general performance, we use benchmarks reflecting diverse aspects of trustworthiness in addition to the popular ones for helpfulness like GSM8k~\cite{hendrycks2measuring}, AlpacaEval2.0~\cite{dubois2024length} and BIG-bench HHH~\cite{zhou2024beyond}. We take SimpleQA~\cite{wei2024measuring} for truthfulness, InfoFlow~\cite{mireshghallahcan} for privacy awareness, and AdvGLUE~\cite{wang2adversarial} for adversarial robustness. Official metrics are reported for all.

% We leave other details including hyperparameters and evaluation strategies in~\cref{sec:appendix_exp}.


\begin{table*}[ht]
    \centering
    \caption{Performance on diverse benchmarks reflecting both harmlessness and general performance. CoT Style represents whether the method adopt Chain-of-Thought reasoning, while Self Gen. denotes whether the method use self-generated data for training. For all reported metrics, the best results are marked in \textbf{bold} and the second best results are marked by \underline{underline}.}
    \renewcommand{\arraystretch}{1.1} % Increase row height
    
\resizebox{\textwidth}{!}{%
    \begin{tabular}{l@{\;\,}|@{\;\,}c@{\;\,}|@{\;\,}c@{\;\,}|c@{\;\,}c@{\;\,}c@{\;\,}c|c@{\;\,}c@{\;\,}c@{\;\,}c@{\;\,}c@{\;\,}c}
        \toprule[1.5pt]
       & \multirow{2}{*}{\makecell{CoT\\Style}} & \multirow{2}{*}{\makecell{Self\\Gen.}}  &  \multicolumn{4}{c|}{\textbf{Harmlessness}} & \multicolumn{6}{c}{\textbf{General}}  \\ \cmidrule(lr){4-7}\cmidrule(lr){8-13}
       & & & StrongReject  & XsTest  & WildChat  & Stereotype  &  SimpleQA 	&  InfoFlow  &  AdvGLUE  & GSM8k  & AlpacaEval  & HHH  \\\midrule
        \multicolumn{13}{c}{\sc Llama-3.1-8B-Instruct} \\ \midrule
        Base &  - & - & 0.4054 & 88.00\% & 47.94\% & 87.37\% & 2.52\% & 0.4229 & 58.33\% &85.60\% &  25.55\% & 82.50\%\\ 
        CoT & \cmark & - & 0.3790 & 87.00\% & 50.23\% & 65.26\% & 4.09\% &  0.7041 & 58.40\% & 87.11\% &22.04\% & 81.63\% \\
        SFT & \xmark & \xmark & 0.4698 & 94.50\% & 50.68\% & 94.74\% & 4.72\% &  0.7134 & 57.53\% &72.02\% & 9.21\% & 82.63\% \\
        DPO & \xmark & \xmark & 0.5054 & 86.00\% & 54.79\% & \bf 97.89\% & 4.46\% & 0.7081 & 66.27\% &84.15\% &  15.26\% & 83.84\% \\ 
        SACPO & \xmark & \xmark  & 0.7264 & 88.50\% & 58.45\% & 96.84\% & 0.74\% &  0.0503 & 65.60\% &86.50\% & 20.44\% & 85.21\%\\ 
        Self-Rewarding & \xmark & \cmark & 0.4633 & \bf 99.00\% & 49.77\% & 94.74\% & 2.70\%  & 0.6618 & 59.10\% & \bf 88.10\%& 26.41\% & 82.09\%\\\midrule
        STAIR-SFT & \cmark & \xmark & 0.6536 & 85.50\% & 50.68\% & 94.74\% & \underline{6.31\%} & \underline{0.7876} & \bf 70.57\% & 86.05\%  &  31.21\% & 83.13\%\\
        +DPO-1 & \cmark & \cmark & 0.6955 & 94.00\% & 57.99\% & \bf 97.89\% & 6.08\% & \bf 0.7998 & 65.93\% & 86.81\% & 34.48\% & 84.53\% \\
        +DPO-2 & \cmark & \cmark & \underline{0.7973} & 96.50\% & \underline{68.95\%} & 96.84\% & 6.00\% &  0.7700 & \underline{69.43\%} & 87.26\% &\underline{36.24\%} & \bf 87.09\% \\
        +DPO-3 & \cmark & \cmark & \bf  0.8798 &  \bf 99.00\% & \bf 69.86\% & 96.84\% & \bf 6.38\% &  0.7395 & 69.20\% &\underline{87.64\%} &\bf  38.66\% & \underline{85.66\%} \\ \midrule
        \multicolumn{13}{c}{\sc Qwen-2-7B-Instruct} \\ \midrule
        Base &  - & - & 0.3808 & 72.50\% & 47.49\% & 90.53\% & 3.79\% & 0.7221 & 66.50\%& \underline{87.49\%}  & 20.06\% & 87.87\%\\ 
        CoT & \cmark & -  & 0.3792 & 70.00\% & 42.92\% & 88.42\% & 3.03\%& 0.7628 & 65.60\% & \bf 88.10\%  & \underline{25.97\%} & 88.30\%\\
        SFT & \xmark & \xmark & 0.4952 & 84.00\% & 58.45\% & 91.58\% & 3.47\% & 0.6267 & 66.90\% &82.34\% &  8.94\% & 89.74\% \\
        DPO & \xmark & \xmark & 0.5026 & 69.00\% & 66.21\% & 88.42\% & 2.59\% &  0.6793 & 70.97\% & 81.43\% & 11.48\% & 88.08\% \\
        SACPO & \xmark & \xmark & 0.5577 & 75.00\% & 60.27\% & 95.79\% & 0.62\%  & 0.6213 & 64.10\% & 85.22\% & 17.04\% & 89.60\% \\ 
        Self-Rewarding & \xmark & \cmark & 0.5062 & 96.00\% & 52.51\% &  94.74\% & 3.37\% & 0.7140 & 66.13\% & 87.34\% & 14.69\% & 88.31\% \\\midrule
        STAIR-SFT & \cmark & \xmark & 0.7356 & 83.50\% & 62.56\% & 95.79\% & 3.81\% &  0.8215 & 70.57\% &84.61\% & 20.31\% & \underline{90.38\%} \\
        +DPO-1 & \cmark & \cmark & 0.7606 & 96.50\% & 65.19\% & 95.79\% & \underline{3.88\%} & \underline{0.8235} & \underline{73.10\%} & 84.76\% & 23.29\% & 90.21\% \\
        +DPO-2 & \cmark & \cmark & \underline{0.8137} & \underline{98.50\%} & \underline{67.90\%} & \underline{97.89\%} & 3.79\% & \bf 0.8646 & 72.83\% & 86.05\% & 24.86\% & 90.11\% \\
        +DPO-3 & \cmark & \cmark & \bf 0.8486 & \bf 99.00\% & \bf 80.56\% & \bf 98.95\% & \bf 4.07\% & 0.7644 & \bf 74.13\% & 85.75\% & \bf 26.31\% & \bf 90.71\% \\ \bottomrule[1.5pt]
    \end{tabular}}
    \label{tab:benchmarks}
    \vspace{-2ex}
\end{table*}



\subsection{Main Results}

We present the results on diverse benchmarks evaluating both the harmlessness and the general performance in~\cref{tab:benchmarks}, which shows the superiority of STAIR, attributed to the incorporation of introspective reasoning to safety alignment and the self-improvement on stepwise data generated with SI-MCTS. 
We use STAIR-SFT to represent the model trained on $\mathcal{D}_\text{CoT}$ with SFT and DPO-k to denote the model after the k-th iteration of self-improvement. Some qualitative examples are displayed in~\cref{sec:appendix_examples}.

First of all, though initially aligned with instruction tuning, the base LLMs remain vulnerable to harmful queries, especially jailbreak attacks. This is evidenced by the goodness scores below 0.40 on StrongReject. We then explore CoT prompting to stimulate the existing reasoning capability in LLMs. While it leads to improvements in reasoning-dependent tasks like GSM8k and InfoFlow, it shows no enhancement in safety. When applying SFT or DPO to the whole dataset $\mathcal{D}$, we observe significant safety-performance trade-offs due to the conflicting objectives. For instance, for both LLaMA-3.1 and Qwen-2 trained with SFT and DPO, their winning rates against GPT-4 on AlpacaEval decline sharply compared to base models. By employing safety-constrained optimization, the trade-off issue is mitigated to a large extent by SACPO, with better safety enhancements compared to previous methods. However, the performance on SimpleQA and InfoFlow degrades, reflecting losses in factual knowledge and over-refusals to benign privacy-related queries. For Self-Rewarding, their improvements on XsTest, which contains queries apparently harmful, are considerable due to the original behaviors of direct refusals in base LLMs. Nevertheless, the behaviors of refusals fail to generalize to jailbreak attacks, as they lack sufficient capabilities to analyze the underlying risks. 

In comparison, STAIR demonstrates more balanced and continuous improvements on diverse benchmarks. With CoT format alignment, the models acquire the basic ability of safety-aware reasoning, enhancing their resilience against harmful inputs. Further training with stepwise preference data generated by SI-MCTS leads to consistent safety enhancements while maintaining or even improving general performance. For example, LLaMA-3.1 achieves an increase of over 20\% in refusal rate on WildChat after three iterations of self-improvement, while its winning rate against GPT-4 on AlpacaEval reaches 38.66\%, a significant improvement compared to 25.55\% for the base model. Similar trends are observed on other benchmarks like SimpleQA and GSM8k. Besides, the accuracy on AdvGLUE is substantially higher than other baselines, highlighting the benefit to robustness from step-by-step reasoning. On StrongReject, both LLMs eventually reach goodness scores of 0.8798 and 0.8486 respectively, which firmly confirm the positive impact of integrating reasoning with safety alignment.

\subsection{Test-time Scaling}

Using the trained process reward model, we investigate the impact of test-time scaling. Since both stepwise and full-trajectory data are used for training, we employ Best-of-N (BoN) and Beam Search, with results presented in~\cref{fig:tts-safe} and~\ref{fig:tts-helpful} for StrongReject and AlpacaEval respectively. Extra computational costs are estimated based on the number of generated steps relative to one-time greedy decoding, expressed in logarithmic form. For example, Bo8 and beam search generating 4 successors with a beam width of 2 correspond to $\log_2(N)=3$. The results indicate that test-time scaling consistently improves both safety and helpfulness. Both searching methods bring improvements of 0.06 for goodness on StrongReject and more than 3.0\% for winning rates on Alpaca.
This supports that the effect of test-time scaling can generalize from math and coding~\cite{snell2024scaling,xie2024self} to more general scenarios like safety.


\begin{figure*}[t]
     \centering
     \begin{minipage}{0.3\textwidth}
         \centering
         \includegraphics[width=\textwidth, trim={1cm 1cm 1cm 1cm}]{images/draft/strongreject.png}
         \vspace{-4ex}
         \caption{Changes in goodness scores on StrongReject with test-time scaling.}
         \label{fig:tts-safe}
     \end{minipage}
     \hfill
     \begin{minipage}{0.3\textwidth}
         \centering
         \includegraphics[width=\textwidth, trim={1cm 1cm 1cm 1cm}]{images/draft/alpaca.png}
         \vspace{-4ex}
         \caption{Changes in winning rates on AlpacaEval when with test-time scaling.}
         \label{fig:tts-helpful}
     \end{minipage}
     \hfill
     \begin{minipage}{0.3\textwidth}
         \centering
         \includegraphics[width=\textwidth, trim={1cm 1cm 1cm 1cm}]{images/draft/balance.png}
         \vspace{-4ex}
         \caption{Results on StrongReject and AlpacaEval as the ratio of safety data varies.}
         \label{fig:data}
     \end{minipage}
        % \caption{Three simple graphs}
        % \label{fig:three graphs}
    \vspace{-1ex}
\end{figure*}


\subsection{Detailed Analysis}

We then conduct some ablation studies to confirm the effectiveness of our framework.

\textbf{Balance between Safety and Helpfulness Data.} To evaluate the impact of the ratio between safety and helpfulness data in the training dataset, we conduct a study during the CoT format alignment stage as a representative. We plot the performance in terms of safety and helpfulness to the varying ratios in~\cref{fig:data}. While a trade-off between safety and helpfulness is observed, consistent with prior findings~\cite{bai2022training}, the performance in both dimensions consistently exceeds that of the base model. This highlights the effectiveness of training with structured CoT data.

\textbf{Step-level Optimization.} To verify the effectiveness of stepwise preference data in the stage of self-improvement, we compare the performance of DPO-1, which is trained on stepwise data based on STAIR-SFT using DPO, with models trained on full trajectory data using either SFT or DPO. The full trajectory data is selected from the same search trees of SI-MCTS, with the total number of training samples kept equal to that of DPO-1. Results in~\cref{tab:iterative} support our strategy of step-level optimization, which brings more fine-grained supervision to safety-aware reasoning.

\textbf{Iterative Training.} We adopt iterative optimization for continuous improvement, motivated by the belief that data generated in later iterations is of higher quality. To validate this, we compare the results of DPO-3 with the model trained using data crafted from all prompts in a single iteration and the model trained on data from the first iteration for three times as many epochs. Results in~\cref{tab:iterative} demonstrate superior improvements on different benchmarks, confirming the improving data quality throughout iterations.




\begin{table}[ht]
\vspace{-1ex}
    \centering
    \caption{Ablation studies on iterative training on stepwise data}
    % \renewcommand{\arraystretch}{1.2} % Increase row height
\resizebox{\linewidth}{!}{%
    \begin{tabular}{l@{\;\,}|@{\;\,}c@{\;\,}c@{\;\,}c@{\;\,}c}
    \toprule[1.5pt]
         & StrongReject & XsTest & GSM8k & AlpacaEval  \\ \midrule
      \multicolumn{5}{c}{Stepwise Data}\\\midrule
      STAIR-SFT + Full (SFT) &  0.6222 & 87.00\% & 85.29\% & 28.10\% \\
      STAIR-SFT + Full (DPO) &  0.6663 & 92.50\% & 86.50\% & 32.87\%\\\midrule
      STAIR-SFT + Step (DPO) & \bf 0.6955 & \bf 94.00\% & \bf 86.81\% & \bf 34.48\% \\\midrule
      \multicolumn{5}{c}{Iterative Training}\\\midrule
      1st Split, 3$\times$ Epochs & 0.6745 & 97.50\%  & 85.75\% & 37.28\% \\
      Full Dataset, 1 Iteration   & 0.7342 & 90.00\%  & 86.58\% & 36.96\%\\\midrule
      STAIR-DPO-3 & \bf 0.8798 & \bf 99.00\% &  \bf 87.64\% & \bf 38.66\% \\\bottomrule[1.5pt]
    \end{tabular}}
    \label{tab:iterative}
    \vspace{-2ex}
\end{table}


\section{Conclusion}
In this paper, we have introduced MME-CoT, a comprehensive benchmark designed to evaluate Chain-of-Thought reasoning in Large Multimodal Models. 
Our dataset comprises six categories to cover most scenarios of visual reasoning tasks.
To gain a thorough understanding of the reasoning process, we design a novel CoT evaluation suite with three metrics. 
Our systematic evaluation obtains useful insights into the issues within the current state-of-the-art Large Multimodal Models.
We identify critical flaws in all the tested open-source models.
As the field continues to evolve, MME-CoT stands as a valuable tool for measuring progress and identifying areas for improvement in the development of more sophisticated multimodal AI systems.

\section*{Impact Statement}
This paper presents work whose goal is to advance the field
of Computer Vision and Machine Learning. There are many
potential societal consequences of our work, none of which
we feel must be specifically highlighted here.

\bibliographystyle{IEEEtran}
\bibliography{references}

\end{document}
