\section{MODEL}
Task goal representations in household environments can specify fixed values for entity properties, a range of allowed values, or be indifferent to the property value. In the example of Figure \ref{fig:teaser}, the goal state could be that the drinking mug must be half full, and its accepted location should be close to its current location on the front table. In this goal example, there is a fixed value for the amount of content of the drinking mug, a range of values for its location, and any possible value for the table's location, because it was not specified in the goal state. A model of a task's goal must represent all these possibilities. Our approach is to introduce \textbf{variations} of values, described in \ref{ssec:variations}.

To turn an environment into a desired goal state, the system must first assess the current \underline{state} of an environment, described in \ref{ssec:environment_state}. In the environment state, values of entity properties and the whole range of allowed values are represented. This range of allowed values is the basis of \textbf{variations}, which can be seen as subsets of the whole domain of values.

In the second step, the differences between the current and goal environment states must be computed, see \ref{ssec:comparisons}, and solved to get to the desired task goal state.


\begin{figure}[t!]
    %\vspace{-0.5cm}
    \centering
    \includegraphics[width=1\linewidth]{images/TaskModelling_Values_vs_Variations.png}
    \caption{Variations can be used to express a desired range of values for environment states. On the left are the contents of an environment state. The right part shows RangeVariations of the \textit{ValueDomains} on the left. $A \rightarrow B$ means $A$ is a subtype/-concept of $B$. \textbf{Bold types} are variations.} \label{fig:objects_and_tasks_domain}
\end{figure}

\subsection{Model of the world: What is an Environment State?}\label{ssec:environment_state}
We model the state of an environment as the collection of states of entity instances (i.e. agents and objects) in the environment, see the left side of Figure \ref{fig:objects_and_tasks_domain}. In the environment on the left side of Figure \ref{fig:teaser}, there are four entity instances: the person doing the demonstration, the drinking mug that has changed location, and the two tables, one in the back and the other in the front, on which the mug is now located.

The state of an entity instance comprises the concepts of the entity and the collection of properties defined by the entity's concepts \cite{conceptHierarchyGeriatronicsSummit24}. Concepts also define their properties' allowed range of values (incl. value type). Subconcepts inherit properties. Instances, not concepts, define the values of their properties. The state of the drinking mug thus comprises all parent concepts, e.g. mug, liquid container, container, object, physical entity, concept, and all properties defined by these concepts, including the location, mass, maximal volume that can be contained (\textit{contentVolume}), volume currently contained (\textit{contentLevel}), contained instances, color, etc.

Thus, the environment state contains all the property values of its entity instances.

The property values are elements of a \textbf{\textit{ValueDomain}}, i.e. a set of values. For example, the set of values of the \textit{contentVolume} property is a non-negative real number, and the set of values of the \textit{location} property is a tuple of the reference entity and the pose delta to that reference's origin. To represent \underline{subsets} of \textit{ValueDomains}, we define \textbf{variations}.

\subsection{What is a Variation?}\label{ssec:variations}
A \textbf{variation} of a \textit{ValueDomain} represents a subset of values from that \textit{ValueDomain}. Thus, a variation can be empty, it can be a fixed value, it can be a range of values (\textbf{RangeVariation}), it can be a conjunction and/or a disjunction of variations, or it can be the whole \textit{ValueDomain} itself.

For example, the variation of an \textit{Integer} value could be the empty set, the number $4$, the set of prime numbers, the union of the integer intervals $\left[2; 5\right]$ and $\left[36; 42\right]$, the intersection of the integer intervals $\left[2; 5\right]$ and $\left[4; 9\right]$, or $\mathbb{Z}$-itself.

Similarly, the variation of a \textit{ConceptValue} could be the empty set, the concept \textit{Object}, the \textit{Object} concept including all its subconcepts, the \textit{RedObject} and/or \textit{GreenObject} concepts, or the set of all concepts.

An example of a meaningful RangeVariation of an entity \textit{Instance} is the variation of the entity's \textit{ConceptValue} and the variation of all properties that are defined by the \textit{ConceptValue} variation.

A meaningful RangeVariation of a collection, e.g. the collection of entity instances in an environment, is a set of variations for the collection's elements. Not all collection elements must satisfy a variation, but all variations must be satisfied by (different) elements. For referencing, we name this variation type as $A \equiv \{v \in \text{\textbf{Variation}}\left<\text{\textit{CollectionType}}\right>\}$. Given $x \in \text{Collection}\left<\text{CollectionType}\right>$, $x \in A \Leftrightarrow \forall v_e \in A,$\\$\exists e \in x: e \in v_e$.

Finally, a meaningful RangeVariation of an environment state is the variation of the collection of entity instances, see the right side of Figure \ref{fig:objects_and_tasks_domain}.

\subsection{Comparison between ValueDomains}\label{ssec:comparisons}
In our work, a \textbf{Comparison} represents the contrast between two \textit{ValueDomains}. It contains a target and a value to be checked for equality with the target.

Values and targets of different \textit{ValueDomains} are different. Values and targets of the same ValueDomain are compared for equality. If different and the \textit{ValueDomain} has sub-data, Comparisons of the sub-data are also created. For example, as per Figure \ref{fig:objects_and_tasks_domain}, a \textit{Location} is represented as a \textit{Pose} delta to a reference \textit{Instance} entity. Thus, the \textit{Location} has a \textit{Pose} and an \textit{Instance} as sub-data. If the \textit{Location} target and value are different, Comparisons of the \textit{Pose} and \textit{Instances} are also created. This helps pinpoint the exact reason for the values being different. Similarly, \textit{EnvironmentData} has a \textit{Collection} of \textit{Instances} as sub-data. \textit{Collections} have an \textit{Integer} size and elements as sub-data. 
\textit{Instances} have a \textit{ConceptValue} and \textit{ConceptParameters} as sub-data.

Comparisons of sub-data are saved as additional information in the parent data Comparison so that, e.g., the \textit{Location} Comparison knows there is a difference between the \textit{Pose} sub-data of the \textit{Location}.

Our system can also model Comparisons between a \textit{ValueDomain} and a \textbf{variation} of the same \textit{ValueDomain}. In this case, the Comparison is equivalent to checking if the value is inside the \textbf{variation}. If not, the additional information stored in the comparison are the reasons why the value is not in the target \textbf{variation}. In the example of an \textit{Integer}-Comparison, if the target \textbf{variation} is the interval $\left[6, 10\right]$ and the value is $4$, the reason is the \textit{Boolean}-Comparison of the \textit{LessEqual} function between the interval's lower bound $6$ and the value $4$, which is false but is supposed to be true when the value is inside the interval.

\begin{figure}[t!]
    %\vspace{-0.5cm}
    \centering
    \includegraphics[width=1\linewidth]{images/ConceptParameterDifference_v3.png}
    \caption{The \textit{Container} concept defines the \textit{contentLevel} property as a \textit{Number}. This property has a value of $0.45$ in the \textit{WhiteMugInstance}. One (or more) \skill(s) must be executed to bring the current level to the desired level inside the defined variation on the right.} \label{fig:concept_property_difference}
\end{figure}

The third Comparison type is when the value stems from an \textit{Instance}'s concept properties. Besides the additional info when the value is different from the target, this Comparison type has information about the instance, which the system can use to determine which \actions\ and \skills\ (see \ref{ssec:actions_skills}) should be used to change the value of this concept property. Figure \ref{fig:concept_property_difference} illustrates a solution to bring the \textit{contentLevel} property of the \textit{WhiteMugInstance} inside the variation range.
