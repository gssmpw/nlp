\section{CONCLUSIONS}
In this paper, we presented a model to represent the desired variation in a task's goal, i.e., the variation in an environment state. We also showed how to build this model from a single user demonstration of the task and how to use the model of the task's variation to create an execution plan to change a given environment into the goal state or determine if the goal state is unattainable with the \skills\ than an agent possesses.

With such a model, an agent would not need to imitate observed task execution trajectories, but could optimize the execution plan to its kinematic structure and attained \skills.

\subsection{Limitations}
The procedure detailed in \ref{ssec:exp_model_use} works for one instance variation. When multiple instance variations are defined in the environment variation, the procedure does not consider that to solve one instance variation, other instances will be modified. And thus, determined solution plans do not bring the whole environment into the goal state; just parts of it. This limitation is overcome by improving the planning procedure, e.g., by using a PDDL solver.

% The computation of execution plans treats the difference in entity properties $\delta_{p_x}$ as independent and, thus, there are cases where, e.g. entity $e_1$ fulfills the variation $v_1$ but for $e_2$ to fulfill variation $v_2$, a \skill\ $S_y$ is executed/selected, that modifies the property of $e_1$ which makes $e_1 \not\in v_1$ anymore.

Another limitation of the framework is the missing procedure to fix differences in \skill-preconditions, so that, e.g., if the milk carton is closed, the solution to open it and pour from it is determined by the system. The solution is to make the difference-solving procedure recursive and parameterize it with the differences found in \skill\ preconditions.

\subsection{Future Work}
Immediate future work is overcoming the limitations as presented above, followed by improving the scoring of \skills\ when selecting execution plans to consider the abilities of agents and, e.g., energy cost or path distance or time to completion of \skills. Furthermore, we plan to extend the variations to \textit{exclude} certain ranges of values. Including negations of ranges besides unions and intersections, would give the variations full expressiveness over the \textit{ValueDomains}.

