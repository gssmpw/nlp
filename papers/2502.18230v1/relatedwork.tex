\section{Related Work}
There are many studies that focused on preoperative planning using the da Vinci system, assuming surgeries such as laparoscopic and cardiac surgeries\cite{davincipreoperative1,davincipreoperative2,davincipreoperative3,davincipreoperative4,davincipreoperative5,davincipreoperative6,davincipreoperative7}. These studies have proposed methods for optimizing trocar placement and the initial posture of the robot. Based on the information from the CT scan, these optimizations are proposed by considering the accessibility of the surgical robot, the endoscopic field of view, arm dexterity, arm-to-arm interference, and even obstacles such as internal organs. Other studies have used general-purpose robotic arms to optimize trocar positioning and surgical instrument mounting\cite{trocaroptimization1,trocaroptimization2,trocaroptimization3}.
However, to the best of our knowledge, there are no studies that focus on the subject of preoperative planning in the field of vitreoretinal surgery. The major difference between vitreoretinal surgery and the mentioned surgeries is that the posture of the eye can be changed, which changes the field of view and accessibility. 
%Moreover, obstacles need not be taken into account in vitreoretinal surgery because there is no organ to be avoided. 

As for the eye surgery, there are several studies on eye rotation.
Smits et al. developed a fixture mechanism for repositioning RCM points~\cite {KUlevensetupandmethod}. Koyama et al.~\cite{Koyama} and the group of~\cite{weidesign,weiperformance,haorandesign} have developed algorithms to perform orbital manipulation within an eyeball utlizing dual-arm robot. However, their focus was more on RCM repositioning or rotating eyeballs rather than optimizing the visible-accessible area.
Therefore, this paper is the first to address presurgical planning for robot-assisted vitreoretinal surgery.