\section{Related Work}
\label{sec:related_work}
The field of monocular pose estimation is mostly researched in the context of robot localization and approaches can be categorized into three families:
\begin{enumerate}
    \item standard cameras with fiducial markers (e.g. Aruco) 
    \item standard cameras with active point markers (e.g. LEDs) 
    \item event-cameras with active markers (e.g. blinking LEDs) 
\end{enumerate}

\subsection{Pose Estimation with Standard Cameras}
Most works on monocular pose estimation for mobile robots use standard cameras because those are readily available on most robots. With a standard camera, the common approach~\cite{breitenmoser2011monocular, benligiray2019stagstablefiducialmarker, kalaitzakis2021fiducial} is to use fiducial markers (e.g. AprilTag~\cite{olson2011apriltag}, Aruco~\cite{garrido2014aruco}) as such markers can robustly identified in RGB and grayscale images. 

Fiducial markers have a minimum size because the camera must still be able to detect the structure of the marker. Therefore, using infrared LEDs has been proposed as an alternative solution~\cite{faessler14monocular} to further miniaturize the system. The LEDs can be detected by thresholding an IR-filtered image, but in this setup the points can't be uniquely identified as all LEDs look the same to the camera. The pose estimation algorithm thus has to exhaustively search through all possible combinations.

Independent of the marker type, standard cameras are not well-suited for a low-latency monocular motion capture system as the latency of the overall system is strongly limited by the framerate of the camera. 

\subsection{Pose Estimation with Event Cameras}
In contrast to a frame-based camera an event-camera is inherently low-latency as events are output with microsecond latency. A major challenge is to design algorithms in a way that they are able to make use of this high update rate.

The first work~\cite{censi2013activeled} using event-cameras for localization\textemdash in a sense a direct predecessor of this work\textemdash uses blinking LEDs as active markers that can be identified based on their frequency. However, the pipeline is not able to estimate the location along the optical axis of the event-camera. Furthermore, it has noise levels on the order of \unit[10]{cm}, and thus is unsuitable for closed-loop control.

More recent monocular event-camera pose-estimation pipelines~\cite{salah2022neuromorphicvisionbased, ebmer2024realtime6dof} rely on very large active markers, up to $\unit[60\times 60]{cm}$ to achieve accurate localization. The markers are also containing multiple LEDs such that their identification is not just via frequency-detection, but similar to fiducial markers. 

% We contribute the first monocular event-camera motion capture system that 

