\documentclass[letterpaper, 10 pt, conference]{ieeeconf}  
\IEEEoverridecommandlockouts                            

% Graphics
\usepackage{graphics} % for pdf, bitmapped graphics files
\usepackage{epsfig} % for postscript graphics files
\usepackage{graphicx}
\usepackage{xcolor} % for text in colors
\usepackage{capt-of}

% Path
\usepackage{amsmath} % assumes amsmath package installed
\usepackage{amssymb}  % assumes amsmath package installed
\usepackage{amsfonts}
\usepackage{mathtools}% Loads amsmath
\usepackage{units}

% Tables
\usepackage{booktabs}
\usepackage{multirow}
\usepackage{tabularx}
\usepackage{multirow}

% Utils
\usepackage{hyperref}
\usepackage{balance}
\usepackage{cite}

% Tikz + Plotting
\usepackage{tikz}
\usepackage{pgfplots}
\usepackage{pgfplotstable}
\usepackage{circuitikz}
\usetikzlibrary{bending, plotmarks, shapes, pgfplots.statistics, pgfplots.colorbrewer, positioning}

\pgfplotsset{scaled y ticks=false}
\pgfplotsset{compat=1.16,
    width = 7cm,
    height = 3.5cm,
    title style = {yshift=-6pt},
    xlabel shift = -3pt,
    ylabel shift = -0pt,
    cycle list={{matlab1},{matlab2},{matlab3},{matlab9},{matlab4},{matlab5},{matlab6},{matlab7},{matlab8}},
    legend columns = 1,
    legend cell align={left},
    legend style ={
        draw = gray,
        fill opacity=0.8,
        text opacity=1.0,
        draw opacity=1.0,
    },
    xmajorgrids,
    ymajorgrids,
    scale only axis,
} 
\usepackage{xfrac}
% \tikzset{external/force remake} % DONT DELETE THIS LINE

\newcommand{\cmark}{\ding{51}}%
\newcommand{\xmark}{\ding{55}}%

% Typical frame defs
\newcommand{\wfr}[0]{\ensuremath{\mathcal{W}}} % world frame
\newcommand{\bfr}[0]{\ensuremath{\mathcal{B}}} % body frame
\newcommand{\cfr}[0]{\ensuremath{\mathcal{C}}} % C-frame
\newcommand{\pfr}[0]{\ensuremath{\mathcal{P}}}

% 3D Tikz stuff
\newcommand{\rotateRPY}[3]% roll, pitch, yaw
{   \pgfmathsetmacro{\rollangle}{#1}
    \pgfmathsetmacro{\pitchangle}{#2}
    \pgfmathsetmacro{\yawangle}{#3}

    % to what vector is the x unit vector transformed, and which 2D vector is this?
    \pgfmathsetmacro{\newxx}{cos(\yawangle)*cos(\pitchangle)}
    \pgfmathsetmacro{\newxy}{sin(\yawangle)*cos(\pitchangle)}
    \pgfmathsetmacro{\newxz}{-sin(\pitchangle)}
    \path (\newxx,\newxy,\newxz);
    \pgfgetlastxy{\nxx}{\nxy};

    % to what vector is the y unit vector transformed, and which 2D vector is this?
    \pgfmathsetmacro{\newyx}{cos(\yawangle)*sin(\pitchangle)*sin(\rollangle)-sin(\yawangle)*cos(\rollangle)}
    \pgfmathsetmacro{\newyy}{sin(\yawangle)*sin(\pitchangle)*sin(\rollangle)+ cos(\yawangle)*cos(\rollangle)}
    \pgfmathsetmacro{\newyz}{cos(\pitchangle)*sin(\rollangle)}
    \path (\newyx,\newyy,\newyz);
    \pgfgetlastxy{\nyx}{\nyy};

    % to what vector is the z unit vector transformed, and which 2D vector is this?
    \pgfmathsetmacro{\newzx}{cos(\yawangle)*sin(\pitchangle)*cos(\rollangle)+ sin(\yawangle)*sin(\rollangle)}
    \pgfmathsetmacro{\newzy}{sin(\yawangle)*sin(\pitchangle)*cos(\rollangle)-cos(\yawangle)*sin(\rollangle)}
    \pgfmathsetmacro{\newzz}{cos(\pitchangle)*cos(\rollangle)}
    \path (\newzx,\newzy,\newzz);
    \pgfgetlastxy{\nzx}{\nzy};
}

\DeclareMathOperator*{\argmin}{arg\,min}

\newcolumntype{C}{>{\centering\arraybackslash}X}
\newcolumntype{x}[1]{>{\centering\let\newline\\\arraybackslash\hspace{0pt}}p{#1}}
\definecolor{matlab1}{rgb}{0.00000,0.44700,0.74100}
\definecolor{matlab2}{rgb}{0.85000,0.32500,0.09800}
\definecolor{matlab3}{rgb}{0.92900,0.69400,0.12500}
\definecolor{matlab4}{rgb}{0.49400,0.18400,0.55600}
\definecolor{matlab5}{rgb}{0.4660, 0.6740, 0.1880}
\definecolor{matlab6}{rgb}{0.3010, 0.7450, 0.9330}
\definecolor{matlab7}{rgb}{0.6350, 0.0780, 0.1840}
\definecolor{matlab8}{rgb}{0.8, 0.8, 0}
\definecolor{matlab9}{rgb}{0.6, 0.6, 0.6}
\definecolor{verylightgray}{rgb}{0.98,0.98,0.98}

\pgfdeclarelayer{bg}   

\makeatletter
\g@addto@macro\@maketitle{
    \setcounter{figure}{0}
    \vspace*{12pt}
    \centering
    \includegraphics[width=17.6cm]{media/Figure1_Wide.pdf}
    \vspace*{-3pt}
    \captionof{figure}{Overview of the monocular event-camera motion capture system: the object to track (e.g. a small quadrotor) is equipped with $N>=4$ infrared LED markers that blink at different frequencies. A single, calibrated event-camera is used to detect all markers and estimate the pose of the object by solving the perspective-n-points problem (PnP) with SqPnP~\cite{terzakis2020sqpnp}. The state estimate from this motion-capture system can then be used for closed-loop control.}
    \label{fig:overview}
    \vspace*{-12pt}
}


\title{\LARGE \bf
A Monocular Event-Camera Motion Capture System
}

\author{Leonard Bauersfeld and Davide Scaramuzza\\ 
Robotics and Perception Group, University of Zurich, Switzerland\\
\thanks{This work was supported by the European Union’s Horizon Europe Research and Innovation Programme under grant agreement No. 101120732 (AUTOASSESS), the European Research Council (ERC) under grant agreement No. 864042 (AGILEFLIGHT) and the Swiss National Science Foundation under project No. 200021\_212065.}
}


\begin{document}

\makeatletter
\maketitle
\thispagestyle{empty}

\begin{abstract}\label{00_Abstract}
Research in the field of automated vehicles, or more generally cognitive cyber-physical systems that operate in the real world, is leading to increasingly complex systems. Among other things, artificial intelligence enables an ever-increasing degree of autonomy. In this context, the V-model, which has served for decades as a process reference model of the system development lifecycle is reaching its limits. To the contrary, innovative processes and frameworks have been developed that take into account the characteristics of emerging autonomous systems. To bridge the gap and merge the different methodologies, we present an extension of the V-model for iterative data-based development processes that harmonizes and formalizes the existing methods towards a generic framework. The iterative approach allows for seamless integration of continuous system refinement. While the data-based approach constitutes the consideration of data-based development processes and formalizes the use of synthetic and real world data. In this way, formalizing the process of development, verification, validation, and continuous integration contributes to ensuring the safety of emerging complex systems that incorporate AI. 
\end{abstract}


\begin{IEEEkeywords}
	Process Reference Model, V-Model, Continuous Integration, AI Systems, Autonomy Technology, Safety Assurance
\end{IEEEkeywords}

\section{Introduction}

% \textcolor{red}{Still on working}

% \textcolor{red}{add label for each section}


Robot learning relies on diverse and high-quality data to learn complex behaviors \cite{aldaco2024aloha, wang2024dexcap}.
Recent studies highlight that models trained on datasets with greater complexity and variation in the domain tend to generalize more effectively across broader scenarios \cite{mann2020language, radford2021learning, gao2024efficient}.
% However, creating such diverse datasets in the real world presents significant challenges.
% Modifying physical environments and adjusting robot hardware settings require considerable time, effort, and financial resources.
% In contrast, simulation environments offer a flexible and efficient alternative.
% Simulations allow for the creation and modification of digital environments with a wide range of object shapes, weights, materials, lighting, textures, friction coefficients, and so on to incorporate domain randomization,
% which helps improve the robustness of models when deployed in real-world conditions.
% These environments can be easily adjusted and reset, enabling faster iterations and data collection.
% Additionally, simulations provide the ability to consistently reproduce scenarios, which is essential for benchmarking and model evaluation.
% Another advantage of simulations is their flexibility in sensor integration. Sensors such as cameras, LiDARs, and tactile sensors can be added or repositioned without the physical limitations present in real-world setups. Simulations also eliminate the risk of damaging expensive hardware during edge-case experiments, making them an ideal platform for testing rare or dangerous scenarios that are impractical to explore in real life.
By leveraging immersive perspectives and interactions, Extended Reality\footnote{Extended Reality is an umbrella term to refer to Augmented Reality, Mixed Reality, and Virtual Reality \cite{wikipediaExtendedReality}}
(XR)
is a promising candidate for efficient and intuitive large scale data collection \cite{jiang2024comprehensive, arcade}
% With the demand for collecting data, XR provides a promising approach for humans to teach robots by offering users an immersive experience.
in simulation \cite{jiang2024comprehensive, arcade, dexhub-park} and real-world scenarios \cite{openteach, opentelevision}.
However, reusing and reproducing current XR approaches for robot data collection for new settings and scenarios is complicated and requires significant effort.
% are difficult to reuse and reproduce system makes it hard to reuse and reproduce in another data collection pipeline.
This bottleneck arises from three main limitations of current XR data collection and interaction frameworks: \textit{asset limitation}, \textit{simulator limitation}, and \textit{device limitation}.
% \textcolor{red}{ASSIGN THESE CITATION PROPERLY:}
% \textcolor{red}{list them by time order???}
% of collecting data by using XR have three main limitations.
Current approaches suffering from \textit{asset limitation} \cite{arclfd, jiang2024comprehensive, arcade, george2025openvr, vicarios}
% Firstly, recent works \cite{jiang2024comprehensive, arcade, dexhub-park}
can only use predefined robot models and task scenes. Configuring new tasks requires significant effort, since each new object or model must be specifically integrated into the XR application.
% and it takes too much effort to configure new tasks in their systems since they cannot spawn arbitrary models in the XR application.
The vast majority of application are developed for specific simulators or real-world scenarios. This \textit{simulator limitation} \cite{mosbach2022accelerating, lipton2017baxter, dexhub-park, arcade}
% Secondly, existing systems are limited to a single simulation platform or real-world scenarios.
significantly reduces reusability and makes adaptation to new simulation platforms challenging.
Additionally, most current XR frameworks are designed for a specific version of a single XR headset, leading to a \textit{device limitation} 
\cite{lipton2017baxter, armada, openteach, meng2023virtual}.
% and there is no work working on the extendability of transferring to a new headsets as far as we know.
To the best of our knowledge, no existing work has explored the extensibility or transferability of their framework to different headsets.
These limitations hamper reproducibility and broader contributions of XR based data collection and interaction to the research community.
% as each research group typically has its own data collection pipeline.
% In addition to these main limitations, existing XR systems are not well suited for managing multiple robot systems,
% as they are often designed for single-operator use.

In addition to these main limitations, existing XR systems are often designed for single-operator use, prohibiting collaborative data collection.
At the same time, controlling multiple robots at once can be very difficult for a single operator,
making data collection in multi-robot scenarios particularly challenging \cite{orun2019effect}.
Although there are some works using collaborative data collection in the context of tele-operation \cite{tung2021learning, Qin2023AnyTeleopAG},
there is no XR-based data collection system supporting collaborative data collection.
This limitation highlights the need for more advanced XR solutions that can better support multi-robot and multi-user scenarios.
% \textcolor{red}{more papers about collaborative data collection}

To address all of these issues, we propose \textbf{IRIS},
an \textbf{I}mmersive \textbf{R}obot \textbf{I}nteraction \textbf{S}ystem.
This general system supports various simulators, benchmarks and real-world scenarios.
It is easily extensible to new simulators and XR headsets.
IRIS achieves generalization across six dimensions:
% \begin{itemize}
%     \item \textit{Cross-scene} : diverse object models;
%     \item \textit{Cross-embodiment}: diverse robot models;
%     \item \textit{Cross-simulator}: 
%     \item \textit{Cross-reality}: fd
%     \item \textit{Cross-platform}: fd
%     \item \textit{Cross-users}: fd
% \end{itemize}
\textbf{Cross-Scene}, \textbf{Cross-Embodiment}, \textbf{Cross-Simulator}, \textbf{Cross-Reality}, \textbf{Cross-Platform}, and \textbf{Cross-User}.

\textbf{Cross-Scene} and \textbf{Cross-Embodiment} allow the system to handle arbitrary objects and robots in the simulation,
eliminating restrictions about predefined models in XR applications.
IRIS achieves these generalizations by introducing a unified scene specification, representing all objects,
including robots, as data structures with meshes, materials, and textures.
The unified scene specification is transmitted to the XR application to create and visualize an identical scene.
By treating robots as standard objects, the system simplifies XR integration,
allowing researchers to work with various robots without special robot-specific configurations.
\textbf{Cross-Simulator} ensures compatibility with various simulation engines.
IRIS simplifies adaptation by parsing simulated scenes into the unified scene specification, eliminating the need for XR application modifications when switching simulators.
New simulators can be integrated by creating a parser to convert their scenes into the unified format.
This flexibility is demonstrated by IRIS’ support for Mujoco \cite{todorov2012mujoco}, IsaacSim \cite{mittal2023orbit}, CoppeliaSim \cite{coppeliaSim}, and even the recent Genesis \cite{Genesis} simulator.
\textbf{Cross-Reality} enables the system to function seamlessly in both virtual simulations and real-world applications.
IRIS enables real-world data collection through camera-based point cloud visualization.
\textbf{Cross-Platform} allows for compatibility across various XR devices.
Since XR device APIs differ significantly, making a single codebase impractical, IRIS XR application decouples its modules to maximize code reuse.
This application, developed by Unity \cite{unity3dUnityManual}, separates scene visualization and interaction, allowing developers to integrate new headsets by reusing the visualization code and only implementing input handling for hand, head, and motion controller tracking.
IRIS provides an implementation of the XR application in the Unity framework, allowing for a straightforward deployment to any device that supports Unity. 
So far, IRIS was successfully deployed to the Meta Quest 3 and HoloLens 2.
Finally, the \textbf{Cross-User} ability allows multiple users to interact within a shared scene.
IRIS achieves this ability by introducing a protocol to establish the communication between multiple XR headsets and the simulation or real-world scenarios.
Additionally, IRIS leverages spatial anchors to support the alignment of virtual scenes from all deployed XR headsets.
% To make an seamless user experience for robot learning data collection,
% IRIS also tested in three different robot control interface
% Furthermore, to demonstrate the extensibility of our approach, we have implemented a robot-world pipeline for real robot data collection, ensuring that the system can be used in both simulated and real-world environments.
The Immersive Robot Interaction System makes the following contributions\\
\textbf{(1) A unified scene specification} that is compatible with multiple robot simulators. It enables various XR headsets to visualize and interact with simulated objects and robots, providing an immersive experience while ensuring straightforward reusability and reproducibility.\\
\textbf{(2) A collaborative data collection framework} designed for XR environments. The framework facilitates enhanced robot data acquisition.\\
\textbf{(3) A user study} demonstrating that IRIS significantly improves data collection efficiency and intuitiveness compared to the LIBERO baseline.

% \begin{table*}[t]
%     \centering
%     \begin{tabular}{lccccccc}
%         \toprule
%         & \makecell{Physical\\Interaction}
%         & \makecell{XR\\Enabled}
%         & \makecell{Free\\View}
%         & \makecell{Multiple\\Robots}
%         & \makecell{Robot\\Control}
%         % Force Feedback???
%         & \makecell{Soft Object\\Supported}
%         & \makecell{Collaborative\\Data} \\
%         \midrule
%         ARC-LfD \cite{arclfd}                              & Real        & \cmark & \xmark & \xmark & Joint              & \xmark & \xmark \\
%         DART \cite{dexhub-park}                            & Sim         & \cmark & \cmark & \cmark & Cartesian          & \xmark & \xmark \\
%         \citet{jiang2024comprehensive}                     & Sim         & \cmark & \xmark & \xmark & Joint \& Cartesian & \xmark & \xmark \\
%         \citet{mosbach2022accelerating}                    & Sim         & \cmark & \cmark & \xmark & Cartesian          & \xmark & \xmark \\
%         ARCADE \cite{arcade}                               & Real        & \cmark & \cmark & \xmark & Cartesian          & \xmark & \xmark \\
%         Holo-Dex \cite{holodex}                            & Real        & \cmark & \xmark & \cmark & Cartesian          & \cmark & \xmark \\
%         ARMADA \cite{armada}                               & Real        & \cmark & \xmark & \cmark & Cartesian          & \cmark & \xmark \\
%         Open-TeleVision \cite{opentelevision}              & Real        & \cmark & \cmark & \cmark & Cartesian          & \cmark & \xmark \\
%         OPEN TEACH \cite{openteach}                        & Real        & \cmark & \xmark & \cmark & Cartesian          & \cmark & \cmark \\
%         GELLO \cite{wu2023gello}                           & Real        & \xmark & \cmark & \cmark & Joint              & \cmark & \xmark \\
%         DexCap \cite{wang2024dexcap}                       & Real        & \xmark & \cmark & \xmark & Cartesian          & \cmark & \xmark \\
%         AnyTeleop \cite{Qin2023AnyTeleopAG}                & Real        & \xmark & \xmark & \cmark & Cartesian          & \cmark & \cmark \\
%         Vicarios \cite{vicarios}                           & Real        & \cmark & \xmark & \xmark & Cartesian          & \cmark & \xmark \\     
%         Augmented Visual Cues \cite{augmentedvisualcues}   & Real        & \cmark & \cmark & \xmark & Cartesian          & \xmark & \xmark \\ 
%         \citet{wang2024robotic}                            & Real        & \cmark & \cmark & \xmark & Cartesian          & \cmark & \xmark \\
%         Bunny-VisionPro \cite{bunnyvisionpro}              & Real        & \cmark & \cmark & \cmark & Cartesian          & \cmark & \xmark \\
%         IMMERTWIN \cite{immertwin}                         & Real        & \cmark & \cmark & \cmark & Cartesian          & \xmark & \xmark \\
%         \citet{meng2023virtual}                            & Sim \& Real & \cmark & \cmark & \xmark & Cartesian          & \xmark & \xmark \\
%         Shared Control Framework \cite{sharedctlframework} & Real        & \cmark & \cmark & \cmark & Cartesian          & \xmark & \xmark \\
%         OpenVR \cite{openvr}                               & Real        & \cmark & \cmark & \xmark & Cartesian          & \xmark & \xmark \\
%         \citet{digitaltwinmr}                              & Real        & \cmark & \cmark & \xmark & Cartesian          & \cmark & \xmark \\
        
%         \midrule
%         \textbf{Ours} & Sim \& Real & \cmark & \cmark & \cmark & Joint \& Cartesian  & \cmark & \cmark \\
%         \bottomrule
%     \end{tabular}
%     \caption{This is a cross-column table with automatic line breaking.}
%     \label{tab:cross-column}
% \end{table*}

% \begin{table*}[t]
%     \centering
%     \begin{tabular}{lccccccc}
%         \toprule
%         & \makecell{Cross-Embodiment}
%         & \makecell{Cross-Scene}
%         & \makecell{Cross-Simulator}
%         & \makecell{Cross-Reality}
%         & \makecell{Cross-Platform}
%         & \makecell{Cross-User} \\
%         \midrule
%         ARC-LfD \cite{arclfd}                              & \xmark & \xmark & \xmark & \xmark & \xmark & \xmark \\
%         DART \cite{dexhub-park}                            & \cmark & \cmark & \xmark & \xmark & \xmark & \xmark \\
%         \citet{jiang2024comprehensive}                     & \xmark & \cmark & \xmark & \xmark & \xmark & \xmark \\
%         \citet{mosbach2022accelerating}                    & \xmark & \cmark & \xmark & \xmark & \xmark & \xmark \\
%         ARCADE \cite{arcade}                               & \xmark & \xmark & \xmark & \xmark & \xmark & \xmark \\
%         Holo-Dex \cite{holodex}                            & \cmark & \xmark & \xmark & \xmark & \xmark & \xmark \\
%         ARMADA \cite{armada}                               & \cmark & \xmark & \xmark & \xmark & \xmark & \xmark \\
%         Open-TeleVision \cite{opentelevision}              & \cmark & \xmark & \xmark & \xmark & \cmark & \xmark \\
%         OPEN TEACH \cite{openteach}                        & \cmark & \xmark & \xmark & \xmark & \xmark & \cmark \\
%         GELLO \cite{wu2023gello}                           & \cmark & \xmark & \xmark & \xmark & \xmark & \xmark \\
%         DexCap \cite{wang2024dexcap}                       & \xmark & \xmark & \xmark & \xmark & \xmark & \xmark \\
%         AnyTeleop \cite{Qin2023AnyTeleopAG}                & \cmark & \cmark & \cmark & \cmark & \xmark & \cmark \\
%         Vicarios \cite{vicarios}                           & \xmark & \xmark & \xmark & \xmark & \xmark & \xmark \\     
%         Augmented Visual Cues \cite{augmentedvisualcues}   & \xmark & \xmark & \xmark & \xmark & \xmark & \xmark \\ 
%         \citet{wang2024robotic}                            & \xmark & \xmark & \xmark & \xmark & \xmark & \xmark \\
%         Bunny-VisionPro \cite{bunnyvisionpro}              & \cmark & \xmark & \xmark & \xmark & \xmark & \xmark \\
%         IMMERTWIN \cite{immertwin}                         & \cmark & \xmark & \xmark & \xmark & \xmark & \xmark \\
%         \citet{meng2023virtual}                            & \xmark & \cmark & \xmark & \cmark & \xmark & \xmark \\
%         \citet{sharedctlframework}                         & \cmark & \xmark & \xmark & \xmark & \xmark & \xmark \\
%         OpenVR \cite{george2025openvr}                               & \xmark & \xmark & \xmark & \xmark & \xmark & \xmark \\
%         \citet{digitaltwinmr}                              & \xmark & \xmark & \xmark & \xmark & \xmark & \xmark \\
        
%         \midrule
%         \textbf{Ours} & \cmark & \cmark & \cmark & \cmark & \cmark & \cmark \\
%         \bottomrule
%     \end{tabular}
%     \caption{This is a cross-column table with automatic line breaking.}
% \end{table*}

% \begin{table*}[t]
%     \centering
%     \begin{tabular}{lccccccc}
%         \toprule
%         & \makecell{Cross-Scene}
%         & \makecell{Cross-Embodiment}
%         & \makecell{Cross-Simulator}
%         & \makecell{Cross-Reality}
%         & \makecell{Cross-Platform}
%         & \makecell{Cross-User}
%         & \makecell{Control Space} \\
%         \midrule
%         % Vicarios \cite{vicarios}                           & \xmark & \xmark & \xmark & \xmark & \xmark & \xmark \\     
%         % Augmented Visual Cues \cite{augmentedvisualcues}   & \xmark & \xmark & \xmark & \xmark & \xmark & \xmark \\ 
%         % OpenVR \cite{george2025openvr}                     & \xmark & \xmark & \xmark & \xmark & \xmark & \xmark \\
%         \citet{digitaltwinmr}                              & \xmark & \xmark & \xmark & \xmark & \xmark & \xmark &  \\
%         ARC-LfD \cite{arclfd}                              & \xmark & \xmark & \xmark & \xmark & \xmark & \xmark &  \\
%         \citet{sharedctlframework}                         & \cmark & \xmark & \xmark & \xmark & \xmark & \xmark &  \\
%         \citet{jiang2024comprehensive}                     & \cmark & \xmark & \xmark & \xmark & \xmark & \xmark &  \\
%         \citet{mosbach2022accelerating}                    & \cmark & \xmark & \xmark & \xmark & \xmark & \xmark & \\
%         Holo-Dex \cite{holodex}                            & \cmark & \xmark & \xmark & \xmark & \xmark & \xmark & \\
%         ARCADE \cite{arcade}                               & \cmark & \cmark & \xmark & \xmark & \xmark & \xmark & \\
%         DART \cite{dexhub-park}                            & Limited & Limited & Mujoco & Sim & Vision Pro & \xmark &  Cartesian\\
%         ARMADA \cite{armada}                               & \cmark & \cmark & \xmark & \xmark & \xmark & \xmark & \\
%         \citet{meng2023virtual}                            & \cmark & \cmark & \xmark & \cmark & \xmark & \xmark & \\
%         % GELLO \cite{wu2023gello}                           & \cmark & \xmark & \xmark & \xmark & \xmark & \xmark \\
%         % DexCap \cite{wang2024dexcap}                       & \xmark & \xmark & \xmark & \xmark & \xmark & \xmark \\
%         % AnyTeleop \cite{Qin2023AnyTeleopAG}                & \cmark & \cmark & \cmark & \cmark & \xmark & \cmark \\
%         % \citet{wang2024robotic}                            & \xmark & \xmark & \xmark & \xmark & \xmark & \xmark \\
%         Bunny-VisionPro \cite{bunnyvisionpro}              & \cmark & \cmark & \xmark & \xmark & \xmark & \xmark & \\
%         IMMERTWIN \cite{immertwin}                         & \cmark & \cmark & \xmark & \xmark & \xmark & \xmark & \\
%         Open-TeleVision \cite{opentelevision}              & \cmark & \cmark & \xmark & \xmark & \cmark & \xmark & \\
%         \citet{szczurek2023multimodal}                     & \xmark & \xmark & \xmark & Real & \xmark & \cmark & \\
%         OPEN TEACH \cite{openteach}                        & \cmark & \cmark & \xmark & \xmark & \xmark & \cmark & \\
%         \midrule
%         \textbf{Ours} & \cmark & \cmark & \cmark & \cmark & \cmark & \cmark \\
%         \bottomrule
%     \end{tabular}
%     \caption{TODO, Bruce: this table can be further optimized.}
% \end{table*}

\definecolor{goodgreen}{HTML}{228833}
\definecolor{goodred}{HTML}{EE6677}
\definecolor{goodgray}{HTML}{BBBBBB}

\begin{table*}[t]
    \centering
    \begin{adjustbox}{max width=\textwidth}
    \renewcommand{\arraystretch}{1.2}    
    \begin{tabular}{lccccccc}
        \toprule
        & \makecell{Cross-Scene}
        & \makecell{Cross-Embodiment}
        & \makecell{Cross-Simulator}
        & \makecell{Cross-Reality}
        & \makecell{Cross-Platform}
        & \makecell{Cross-User}
        & \makecell{Control Space} \\
        \midrule
        % Vicarios \cite{vicarios}                           & \xmark & \xmark & \xmark & \xmark & \xmark & \xmark \\     
        % Augmented Visual Cues \cite{augmentedvisualcues}   & \xmark & \xmark & \xmark & \xmark & \xmark & \xmark \\ 
        % OpenVR \cite{george2025openvr}                     & \xmark & \xmark & \xmark & \xmark & \xmark & \xmark \\
        \citet{digitaltwinmr}                              & \textcolor{goodred}{Limited}     & \textcolor{goodred}{Single Robot} & \textcolor{goodred}{Unity}    & \textcolor{goodred}{Real}          & \textcolor{goodred}{Meta Quest 2} & \textcolor{goodgray}{N/A} & \textcolor{goodred}{Cartesian} \\
        ARC-LfD \cite{arclfd}                              & \textcolor{goodgray}{N/A}        & \textcolor{goodred}{Single Robot} & \textcolor{goodgray}{N/A}     & \textcolor{goodred}{Real}          & \textcolor{goodred}{HoloLens}     & \textcolor{goodgray}{N/A} & \textcolor{goodred}{Cartesian} \\
        \citet{sharedctlframework}                         & \textcolor{goodred}{Limited}     & \textcolor{goodred}{Single Robot} & \textcolor{goodgray}{N/A}     & \textcolor{goodred}{Real}          & \textcolor{goodred}{HTC Vive Pro} & \textcolor{goodgray}{N/A} & \textcolor{goodred}{Cartesian} \\
        \citet{jiang2024comprehensive}                     & \textcolor{goodred}{Limited}     & \textcolor{goodred}{Single Robot} & \textcolor{goodgray}{N/A}     & \textcolor{goodred}{Real}          & \textcolor{goodred}{HoloLens 2}   & \textcolor{goodgray}{N/A} & \textcolor{goodgreen}{Joint \& Cartesian} \\
        \citet{mosbach2022accelerating}                    & \textcolor{goodgreen}{Available} & \textcolor{goodred}{Single Robot} & \textcolor{goodred}{IsaacGym} & \textcolor{goodred}{Sim}           & \textcolor{goodred}{Vive}         & \textcolor{goodgray}{N/A} & \textcolor{goodgreen}{Joint \& Cartesian} \\
        Holo-Dex \cite{holodex}                            & \textcolor{goodgray}{N/A}        & \textcolor{goodred}{Single Robot} & \textcolor{goodgray}{N/A}     & \textcolor{goodred}{Real}          & \textcolor{goodred}{Meta Quest 2} & \textcolor{goodgray}{N/A} & \textcolor{goodred}{Joint} \\
        ARCADE \cite{arcade}                               & \textcolor{goodgray}{N/A}        & \textcolor{goodred}{Single Robot} & \textcolor{goodgray}{N/A}     & \textcolor{goodred}{Real}          & \textcolor{goodred}{HoloLens 2}   & \textcolor{goodgray}{N/A} & \textcolor{goodred}{Cartesian} \\
        DART \cite{dexhub-park}                            & \textcolor{goodred}{Limited}     & \textcolor{goodred}{Limited}      & \textcolor{goodred}{Mujoco}   & \textcolor{goodred}{Sim}           & \textcolor{goodred}{Vision Pro}   & \textcolor{goodgray}{N/A} & \textcolor{goodred}{Cartesian} \\
        ARMADA \cite{armada}                               & \textcolor{goodgray}{N/A}        & \textcolor{goodred}{Limited}      & \textcolor{goodgray}{N/A}     & \textcolor{goodred}{Real}          & \textcolor{goodred}{Vision Pro}   & \textcolor{goodgray}{N/A} & \textcolor{goodred}{Cartesian} \\
        \citet{meng2023virtual}                            & \textcolor{goodred}{Limited}     & \textcolor{goodred}{Single Robot} & \textcolor{goodred}{PhysX}   & \textcolor{goodgreen}{Sim \& Real} & \textcolor{goodred}{HoloLens 2}   & \textcolor{goodgray}{N/A} & \textcolor{goodred}{Cartesian} \\
        % GELLO \cite{wu2023gello}                           & \cmark & \xmark & \xmark & \xmark & \xmark & \xmark \\
        % DexCap \cite{wang2024dexcap}                       & \xmark & \xmark & \xmark & \xmark & \xmark & \xmark \\
        % AnyTeleop \cite{Qin2023AnyTeleopAG}                & \cmark & \cmark & \cmark & \cmark & \xmark & \cmark \\
        % \citet{wang2024robotic}                            & \xmark & \xmark & \xmark & \xmark & \xmark & \xmark \\
        Bunny-VisionPro \cite{bunnyvisionpro}              & \textcolor{goodgray}{N/A}        & \textcolor{goodred}{Single Robot} & \textcolor{goodgray}{N/A}     & \textcolor{goodred}{Real}          & \textcolor{goodred}{Vision Pro}   & \textcolor{goodgray}{N/A} & \textcolor{goodred}{Cartesian} \\
        IMMERTWIN \cite{immertwin}                         & \textcolor{goodgray}{N/A}        & \textcolor{goodred}{Limited}      & \textcolor{goodgray}{N/A}     & \textcolor{goodred}{Real}          & \textcolor{goodred}{HTC Vive}     & \textcolor{goodgray}{N/A} & \textcolor{goodred}{Cartesian} \\
        Open-TeleVision \cite{opentelevision}              & \textcolor{goodgray}{N/A}        & \textcolor{goodred}{Limited}      & \textcolor{goodgray}{N/A}     & \textcolor{goodred}{Real}          & \textcolor{goodgreen}{Meta Quest, Vision Pro} & \textcolor{goodgray}{N/A} & \textcolor{goodred}{Cartesian} \\
        \citet{szczurek2023multimodal}                     & \textcolor{goodgray}{N/A}        & \textcolor{goodred}{Limited}      & \textcolor{goodgray}{N/A}     & \textcolor{goodred}{Real}          & \textcolor{goodred}{HoloLens 2}   & \textcolor{goodgreen}{Available} & \textcolor{goodred}{Joint \& Cartesian} \\
        OPEN TEACH \cite{openteach}                        & \textcolor{goodgray}{N/A}        & \textcolor{goodgreen}{Available}  & \textcolor{goodgray}{N/A}     & \textcolor{goodred}{Real}          & \textcolor{goodred}{Meta Quest 3} & \textcolor{goodred}{N/A} & \textcolor{goodgreen}{Joint \& Cartesian} \\
        \midrule
        \textbf{Ours}                                      & \textcolor{goodgreen}{Available} & \textcolor{goodgreen}{Available}  & \textcolor{goodgreen}{Mujoco, CoppeliaSim, IsaacSim} & \textcolor{goodgreen}{Sim \& Real} & \textcolor{goodgreen}{Meta Quest 3, HoloLens 2} & \textcolor{goodgreen}{Available} & \textcolor{goodgreen}{Joint \& Cartesian} \\
        \bottomrule
        \end{tabular}
    \end{adjustbox}
    \caption{Comparison of XR-based system for robots. IRIS is compared with related works in different dimensions.}
\end{table*}


%
We position CST with respect to current frameworks for discrimination testing along the goals of actionability and meaningfulness.
Later in Section~\ref{sec:CausalKnowledge} we discuss the role of causality for conceiving discrimination.
For a broader, multidisciplinary view on discrimination testing, we refer to the survey by~\textcite{Romei2014MultiSurveyDiscrimination}. 
For a recent survey of the fair ML testing literature, see \textcite{DBLP:journals/tosem/ChenZHHS24}.

Regarding actionability, it is important when proving discrimination to insure that the framework accounts for sources of randomness in the decision-making process. Popular non-algorithmic frameworks---such as natural \parencite{Godin2000Orchestra} and field \parencite{Bertrand2017_FieldExperimentDiscrimination} experiments, audit \parencite{Fix&Struyk1993_ClearConvincingEvidence} and correspondence \parencite{Bertrand2004_EmilyAndGreg, Rooth2021} studies---address this issue by using multiple observations to build inferential statistics. Similar statistics are asked in court for proving discrimination \parencite[Section 6.3]{EU2018_NonDiscriminationLaw}. 
Few algorithmic frameworks address this issue due to model complexity preventing formal inference \parencite{Athey2019MachineLearningForEconomists}. An exception are data mining frameworks for discrimination discovery \parencite{DBLP:conf/kdd/PedreschiRT08, DBLP:journals/tkdd/RuggieriPT10} that operationalize the non-algorithmic notions, including situation testing \parencite{Thanh_KnnSituationTesting2011, Zhang_CausalSituationTesting_2016}.
These frameworks \parencite{TR-DBLP:conf/sigsoft/GalhotraBM17, TR-DBLP:journals/corr/abs-1809-03260, DBLP:journals/jiis/QureshiKKRP20} keep the focus on comparing multiple control-test instances for making individual claims, providing evidence similar to that produced by the quantitative tools used in court.
It remains unclear if the same can be said about existing causal fair machine learning methods
% \parencite{DBLP:journals/jlap/MakhloufZP24} 
as these have yet to be used beyond academic circles.
The suitability of algorithmic fairness methods for testing discrimination, be it or not ADM, remains an ongoing discussion \parencite{DBLP:conf/fat/WeertsXTOP23}.

Regarding meaningfulness, situation testing and the other methods previously mentioned have been criticized for their handling of the counterfactual question behind the causal model of discrimination \parencite{Kohler2018CausalEddie, Hu_facct_sex_20, Kasirzadeh2021UseMisuse}. In particular, these actionable methods take for granted the influence of the protected attribute on all other attributes. This point can be seen, e.g.,~in how situation testing constructs the test group, which is equivalent to changing the protected attribute while keeping everything else equal. Such an approach goes against how most social scientists interpret the protected attribute and its role as a social construct when proving discrimination \parencite{Bonilla1997_RethinkingRace, rose_constructivist_2022, Sen2016_RaceABundle, Hanna2020_CriticalRace}. 
It is in that regard where structural causal models \parencite{PearlCausality2009} and their ability to generate counterfactuals via the abduction, action, and prediction steps (e.g.,~\textcite{Chiappa2019_PathCF, Yang2021_CausalIntersectionality}), including counterfactual fairness \parencite{Kusner2017CF}, have an advantage.
This advantage is overlooked by critics of counterfactual reasoning \parencite{Kasirzadeh2021UseMisuse, Hu_facct_sex_20}: generating counterfactuals, as long as the structural causal model is properly specified, accounts for the effects of changing the protected attribute on all other attributes. 
Hence, a framework like counterfactual fairness, relative to situation testing and other discrimination discovery methods, is more meaningful in its handling of protected attributes. 

CST bridges these two lines of work, borrowing the actionability aspects from frameworks like situation testing and meaningful aspects from frameworks like counterfactual fairness. 
% Intuitively, 
Counterfactual generation allows to create a comparator for the complainant that accounts for the influence of the protected attribute on the other attributes, departing from the idealized comparison.
It is not far, conceptually, from the broader ML problem of learning fair representations \parencite{Zemel2013LearningFairRepresentations} since we wish to learn (read, map) a new representation of the complainant that reflects where it would have been had it belonged to the non-protected group. 
It is a normative claim on what a non-protected instance similar to the complainant looks like.
Beyond counterfactual fairness and derivatives (e.g., \textcite{Chiappa2019_PathCF}), other works address this problem of deriving such a pair for the complainant.
For instance, \textcite{Plevcko2020FairDataAdaptation} use a quantile regression approach while \textcite{DBLP:journals/corr/abs-2307-12797} use a residual-based approach for generating the pair.
Both works rely on having access to a structural causal model, but do not exploit the abduction, action, and prediction steps for generating counterfactual distributions.
\textcite{BlackYF20_FlipTest}, instead, propose the FlipTest, a non-causal approach that uses an optimal transport mapping to derive the pair for the complainant.
These three works exemplify ML methods that use counterfactual reasoning to operationalize different interpretations of individual similarity. 
With CST we align with these and similar efforts to propose an alternative to the idealized comparison often used in discrimination testing.
% \sr{Isn't propensity score another form of representation learning?}

%
% EOS
%

\section{System Design}
\label{sec:system_design}

\begin{figure}
    \centering
    \includegraphics{media/main-figure0.pdf}
    \caption{Definition of the camera frame ($z_\cfr$ is aligned with the optical axis), the body frame $\bfr$ and the world frame $\wfr$. The position of the active markers is defined in body-frame and must be known. The transform $T_{\cfr \bfr}$ is estimated through PnP and the pose of the camera in the world frame is assumed to be known.}
    \label{fig:coordinate_systems}
\end{figure}

This section gives a brief overview of the developed monocular event-camera motion-capture system. Details regarding the implementation of the blinking LED detection are given in Sec.~\ref{sec:implementation_details}. The blinking LED circuit itself is discussed in Sec.~\ref{sec:blinked_led_circuit}.

\subsection{Prerequisites}
In order to obtain accurate 3D pose estimation of the tracked objects, the event-camera must be calibrated and the location of the blinking LEDs must be known. Furthermore, the transform $T_{\mathcal{CW}}$ between the camera frame $\mathcal{C}$ and the world frame $\mathcal{W}$ must be known. The coordinate systems are illustrated in Fig.~\ref{fig:coordinate_systems}.

To calibrate the camera, we follow the approach from~\cite{muglikar2021calibrate} where the calibration is performed by first converting the event stream into event frames and then calibrating these using Kalibr~\cite{rehder2016extending}. The calibration also estimates the lens distortion and in this work we rely on a double-sphere distortion model because of its accuracy and computation efficiency~\cite{usenko2018doublesphere}. The event-camera is mounted horizontally on a stable tripod such that the transform from camera-frame $\mathcal{C}$ to world-frame $\mathcal{W}$ is fixed and known.

To ensure that the locations of the blinking LEDs are known and fixed, a 3D-printed LED holder is used. The locations of the LEDs are then directly known from the CAD model. The blinking frequency of each LED is also known, see Sec.~\ref{sec:blinked_led_circuit} for details on the circuit design.

\subsection{LED detection}
Robustly detecting blinking LEDs in an event-stream in real-time is the critical component of the motion-capture system. Events are processed in batches between \unit[1]{ms} (\unit[1]{kHz}) and \unit[2.5]{ms} (\unit[400]{Hz}), depending on the desired pose update rate. In a first step, all pixels whose event rate is below a threshold are discarded. The threshold is calculated based on the assumption that each LED period at least triggers two events and that a transition is detected with a given probability (e.g. \unit[80]{\%}).

For all pixels with a sufficient event rate, the average period and standard deviation is calculated. For details on this process, see Sec.~\ref{sec:implementation_details}. After identifying the average period, neighboring pixels with similar periods are clustered together. If the average period of the cluster is closer than $\unit[25]{\mu s}$ to one of the expected blinking frequencies from the LED, it is matched to that LED. The centroid of each LED is tracked with a constant-velocity particle filter.

\subsection{Pose Estimation}
To estimate the pose from the detected LED centroids, the centroids are undistorted first. Then, the PnP-problem is solved with SQPnP~\cite{terzakis2020sqpnp}. We chose this algorithm over other well-known algorithms such as EPnP~\cite{lepetit2009epnp} because SQPnP is fast enough and globally optimal~\cite{terzakis2020sqpnp}. 

The pose-estimate in the camera-frame is finally transformed into the world-frame and published via ROS. 
\section{Experimental Results}
\label{sec:experimental_results}
Often a motion capture systems are chosen because of their low noise levels and low latency which is ideally suited for robot control. Therefore, we demonstrate the performance of the developed system, by flying the small drone depicted in Fig.~\ref{fig:overview} in closed-loop with the event-camera motion capture being the only source of state estimation. We use a Prophesee Gen 3.1 event-camera ($\unit[640\times 480]{px}$, \unit[3/4]{inch} sensor) with either a \unit[25]{mm} or a \unit[50]{mm} lens resulting in a horizontal FoV (field of view) of \unit[22]{deg} and \unit[11]{deg}, respectively.

\subsection{Pose Estimation Noise}
In a first experiment the drone is rigidly placed at various distances in front of the camera. Then, 10 seconds of data are recorded with the event-camera and we calculate the standard deviation of the pose estimate. Since the drone is static, all deviations from the mean are only due to noise. The results of this analysis summarized in Fig.~\ref{fig:pose_noise}. A few interesting observations can be made which are subsequently discussed in detail:
\begin{enumerate}
    \item The noise levels along the $z_\cfr$ axis are much larger compared to the $x_\cfr$ and $y_\cfr$ axis.
    \item The SqPnP~\cite{terzakis2020sqpnp} algorithm achieves a much better performance than EPnP~\cite{lepetit2009epnp}.
    \item For SqPnP the noise levels in position scale quadratically with the distance from the camera, while the orientation noise scales linearly.
\end{enumerate}

\begin{figure}
    \centering
    \includegraphics{media/main-figure1.pdf}
    \caption{The object is placed at distances between \unit[70]{cm} and \unit[5]{m} statically in front of the camera (with the \unit[25]{mm} lens). The plots show the standard deviation in the position measurement ($z_\cfr$ and $x_\cfr$, $y_\cfr$) as well as the orientation measurements. We can clearly see that SqPnP~\cite{terzakis2020sqpnp} outperforms EPnP~\cite{lepetit2009epnp} by a large margin.}
    \label{fig:pose_noise}
\end{figure}

The $z_\cfr$-axis is the optical axis of the camera and hence the $z_\cfr$ coordinate can only be inferred from the scale of the object. Assuming that elongation of the object along the optical axis is small compared to the distance to the camera (i.e. the object is nearly flat), a well-known result from stereo-vision applies~\cite{zisserman2004multipleview}: for a given inter-marker distance $d$, focal length $f$ and a marker detection with uncertainty $\sigma_u$ (in pixels) the depth uncertainty $\sigma_{pz}$ scales with the square of the distance $z$ as
\begin{equation}
    \sigma_{pz} = \frac{\partial z_\cfr}{\partial u} \cdot \sigma_u = \frac{b \cdot f}{z_\cfr^2} \cdot \sigma_u \sim \frac{1}{z^2}\;.
\end{equation}
For the positional errors in $x_\cfr$ and $y_\cfr$ we observe a similar quadratic dependency on the camera-object distance, however with much less noise. Intuitively, this makes sense as the translation along $x_\cfr$ and $y_\cfr$ are directly observable from each marker and thus the estimate is much more accurate.

The position noise plots also highlight the superior performance of SqPnP for this task: the optimization-based approach is able to estimate the position with much less variance given the same input data. This discrepancy becomes even larger when considering the orientation estimation shown at the bottom plot of Fig.~\ref{fig:pose_noise}. SqPnP dramatically outperforms EPnP which performs between two and four times worse. Interestingly, we observe that EPnP shows a large but nearly constant orientation uncertainty after \unit[2]{m}, whereas SqPnP shows a linear increase in the noise standard deviation. Note that we do not compare against EPnP with nonlinear refinement as the OpenCV implementation requires at least six points for iterative refinement.

\subsection{Closed-Loop Deployment}

\begin{figure}[t]
    \centering
    \includegraphics{media/main-figure2.pdf}
    \vspace*{-18pt}
    \caption{Closed-loop experiments: the drone flies a rectangular pattern starting at (2,0.1) and then lands at a distance of \unit[2.5]{m}. The event-camera is located at the origin of the coordinate system at $(x_\wfr, y_\wfr) = (0,0)$ and the optical axis of the \unit[50]{mm} lens is aligned with the $x_\wfr$ direction. The bottom plot shows the roll and pitch angle measurements during the flight.}
    \label{fig:closed_loop}
\end{figure}

In this experiment we fly the small drone shown in Fig.~\ref{fig:overview} in closed-loop and the event-camera motion capture runs at \unit[400]{Hz} leading to a system latency of \unit[2.5]{ms}. The monocular event-camera motion capture system is the only source of state-estimation for the MPC controller~\cite{foehn2022agilicious} of the quadrotor. The drone is tasked to fly a rectangular pattern, hover and then land. The trajectory is shown in the top plot Fig.~\ref{fig:closed_loop} and the roll and pitch angle measurements are shown at the bottom. When the drone is further away at {$x_\wfr=\unit[3]{m}$} the position and orientation estimate become more noisy but overall we find that the system is able to safely fly the small drone, thereby demonstrating the performance of the developed system.


\section{Implementation Details}
\label{sec:implementation_details}
To ensure real-time operation of the event-camera motion capture system the delay in the processing pipeline must be kept to a minimum. This is achieved through an efficient, multi-threaded implementation in combination with filtering data early on in the pipeline. This section gives an overview of the most important concepts, specifically the data representation, the filtering and the multi-threading.

\subsection{Event Data Representation}
\label{subsec:event_data_representation}
A single event as supplied by the camera is given as a four-tuple, consisting of an $x$ and a $y$ coordinate (both \texttt{uint16\_t}), a polarity $p$ which is either -1 or 1 (\texttt{int8\_t}), and a timestamp $t$ (\texttt{uint64\_t}). Additionally, 3 bytes of padding are included for 16-byte alignment. 
The event camera supplies a stream of such raw events. For the following discussion of different data representations for blinking LED detection, an event stream containing $k$ events over a time period $T$ coming from an event camera with image width $W$, height $H$ and $N = W \times H$ pixels is considered.

\subsubsection*{1D Representations}
The \emph{event stream} is the most basic and raw representation of events and has recently gained some attention in combination with spiking neural networks~\cite{gehrig2020eventbasedangular}. For LED detection with classical CPU architectures however this representation is completely unsuitable. To extract any spatial information, the entire event stream must be searched for pixels with matching coordinates. Furthermore, having a memory layout where each event is stored serially is not efficiently using the cache: if we search for a given x coordinate, the remaining 14 bytes of the raw event representation are unused and just occupy cache space.

\subsubsection*{2D Representations}
In the \emph{event frame} representation the events are stored as a 2D grid by either summing the polarity or by counting the number of events. The accumulation is done for the time window of length $t$ which represents the equivalent of the exposure time. The conversion from an event stream to an event frame is fast as can be done in linear time $\mathcal{O}(k)$ by iterating once over the stream. This representation is suitable for filtering out which pixels have a sufficiently high number of events to be candidates for a blinking LED, however it does not include any time information which would allow robust frequency detection.

A \emph{time-surface} representation is also a 2D image, but each pixel in this 2D grid is assigned the value of the latest timestamp. For detecting blinking LEDs this representation is unsuitable as it contains no information related to periodic on-off transitions of a pixels.

\subsubsection*{3D Representations}
In an \emph{event volume}, events are stored as a 3D grid in a form that can be thought of as a stack of multiple event frames. This representation also includes time information and, given a sufficiently fine binning in the time domain, could be used to detect the frequency of a blinking LED. However, to accurately detect the frequency the binning would have to be very fine, yielding a huge memory footprint. For an accuracy of \unit[5]{$\mu s$}, a window length $t = \unit[2]{ms}$, VGA resolution and \texttt{uint8\_t} storage, the event volume would occupy \unit[117]{MB} of memory. This size exceeds all levels of the processor cache, potentially affecting runtime adversely.

\subsubsection*{Signed Delta-Time Volume}
To get past the shortcomings of those widely used event representation we propose a data representation that is ideally suited for the task of blinking LED detection: the \emph{signed delta-time volume (SDTV)}. It is a 3D volume of size {$W \times H \times D$} where $D$ is the stack depth. For each pixel the time difference to the last event is stored and the polarity of the event is encoded in the sign of this time difference. This is possible because time must be monotonically increasing, so we can re-purpose the sign-bit for polarity encoding. The idea is illustrated in Fig.~\ref{fig:sdtv} for a single pixel stack of the SDTV.
\begin{figure}
    \centering
    \includegraphics{media/main-figure3.pdf}
    \vspace*{-18pt}
    \caption{Illustration on the construction of the \emph{Signed Delta-Time Volume (SDTV)} from an event stream. \textbf{a)} The LED is blinking with a period of $\unit[300]{\mu s}$ with a duty cycle of \unit[10]{\%}. \textbf{b)} A single pixel of the event camera records a noisy signal of this blinking LED. False double events (e.g. at $t = \unit[150]{\mu s}, \unit[165]{\mu s}$) and spurious events (e.g. at $t = \unit[630]{\mu s}$) are included. \textbf{c)} Construction of the SDTV illustrated before processing the latest time window and after processing the time window. \textbf{d)} Periods robustly identified from the SDTV by summing up absolute time differences between negative $\rightarrow$ positive transitions (the first positive value is included). All events until the first positive $\rightarrow$ negative transition are discarded.}
    \label{fig:sdtv}
\end{figure}

Because of the fast blinking of the LEDs the time differences (in microseconds) between consecutive events are always within \texttt{int16\_t} range, making storage compact. As most operations are done per pixel-stack, the memory layout is such that the $D$-dimension is consecutive. Similar to the other representations, converting an event stream to SDTV is linear in the number of events.
The signed delta-time volume is not computed per window of length $T$ but updated as a cyclic buffer. This increases the accuracy of the frequency detection for LEDs blinking at a lower frequency than $f_\text{max}$ since the amount of LED periods available for frequency identification is independent of the frequency.

The minimal depth $D$ can be calculated based on the window length $T$ and the frequency of the fastest LED $f_\text{max}$ as $D = 2\,T\,f_\text{max}$ because every LED should trigger two events (once on and once off) per period. Typical values of $D$ are between 4 and 16, reducing the memory footprint by a factor of 25 to 100 compared to a event volume.


\subsection{Filtering}
When computing the signed delta-time volume representation from the event stream for a time window $t$ of events, we also compute an event frame based on the event count of each pixel. Only pixels with more than {$\beta \cdot 2 t f_\text{min}$} events are considered further where $\beta$ is the probability that a transition triggers an event. We use {$\beta = 0.8$} to purposely underestimate the detection probability.

For all selected pixels the SDTV is used to calculate mean, median and standard deviation of the period. The period is defined as the time between two on-events with at least one off-event in between as illustrated in Fig.~\ref{fig:sdtv}d). In agreement with~\cite{censi2013activeled} we find that this is a robust measure. After rejecting pixels with a too-large standard deviation in the period, pixels are clustered together. Too small and too large clusters are rejected as the expected size of an LED is roughly known a priori. Clusters are then assigned to the individual LEDs by matching the measured average period in a cluster with the blinking frequencies of the LEDs. Each LED is tracked by a particle filter that gets the assigned clusters for each LED as an input.

\subsection{Multi-Theading}
\label{subsec:multi_threading}
In a real-time application like this, relying on generic multi-threading tools such as OpenMP can be problematic. For this reason, the threading is manually implemented to ensure optimal performance. Each thread in the pipeline shares its memory with the next thread in the pipeline. To ensure threads do not block each other, each thread allocates the required memory two times. During operation, the thread writes to one of its allocations while the other memory chunk is processed by the next pipeline step. Subsequently, the memory pointers are swapped and the newly filled batch processed. 

The pipeline primarily consists of three threads. They
\begin{enumerate}
    \item copy events from event camera driver into a buffer,
    \item convert a linear event buffer into the optimized SDTV representation described in Section~\ref{subsec:event_data_representation}, and
    \item process the accumulated data to detect the LEDs, assign the LED clusters and solve the PnP (perspective-n-points) problem.
\end{enumerate}

This design makes it possible to run the pipeline at hight speeds on a modern laptop. Speeds exceeding \unit[1]{kHz} are possible, but due the slowest LED blinking at \unit[1700]{Hz} increasing the processing frequency beyond \unit[800]{Hz} might degrade robustness.
\section{Blinking LED Circuit}
\label{sec:blinked_led_circuit}
%
\begin{figure*}
    \centering
    \includegraphics{media/Fig_CircuitDiagram.pdf}
    \caption{Circuit diagram of the complete blinking LED circuit. For simplicity we only show one LED driver circuit (area enclosed in dotted line), which is then replicated multiple times to drive multiple LEDs. The input voltage range for the power supply is $V_\text{DD}$ is \unit[8-35]{V}. The resistor and capacitor values $R_A$, $R_B$, and $C$ for the A-stable NE555 operation at different frequencies are given in Tab.~\ref{tab:resistor_capacitor_values}. For improved clarity the standard \unit[100]{nF} ceramic bypass capacitors to stabilize VCC for the NE555 and ADG802 have been omitted in the circuit diagram above.}
    \label{fig:circuit_diagram}
\end{figure*}
%
\subsection{Choice of LEDs}
\label{subsec:choice_of_leds}
For the accuracy of the event-based mocap system it is important that the center of the blinking LED can be detected easily by the event camera. To achieve this, the LED should be small, very bright and have a short switching time to produce a well-defined rising and falling edge. These requirements are ideally met by LEDs optimized for pulsed operation, such as infrared LEDs for data transmission.

In this work, the Osram SFH4350~\cite{osram4350} IR-LED \unit[3]{mm} was selected as it features an extremely short switching time of \unit[12]{ns}. Being designed for opto-electronics it also permits up to \unit[1]{A} pulsed forward-current for pulses shorter than \unit[100]{$\mu$s} if the duty cycle is below \unit[2]{\%}. The LED has its peak emission around \unit[850]{nm} with a spread of \unit[50]{nm} which is within the sensitivity of most CMOS-based imaging sensors.

\subsection{Circuit Design}
\label{subsec:circuit_design}

The design of the LED driver circuit is tightly coupled with the entire event-based motion capture system:
\begin{enumerate}
    \item Faster blinking frequencies increase the reactiveness of the system as at least one full period must be detected for identifying an LED. For robustness, a more conservative approach to detect at least two periods is better. Consequently, if the LEDs blink at \unit[1]{kHz} the overall system is limited to \unit[500]{Hz} output rate.
    \label{item:lower_limit}
    \item If the LEDs blink too fast, measuring the signal becomes difficult. Typically event cameras perform very well at measuring signals with frequencies up to \unit[2-3]{kHz}~\cite{wang2024towardshighspeed, crabtree2023refactoryperiod, lichtsteiner2008a128x128dvs} and the events are timestamped with \unit[1]{$\mu$s} time resolution.
    \label{item:upper_limit}
    \item At least four LEDs are neccessary to yield a unique solution to the PnP problem.
    \label{item:number_leds}
    \item the individual frequencies should not alias into each other. This means that, ideally, all LEDs have blinking frequencies within a factor of two.
    \label{item:aliasing}
    \item Due to the limitations of the LED a duty cycle of \unit[2]{\%} can not be exceeded.
    \label{item:duty}
\end{enumerate}

To control the blinking LED either a microcontroller or an analog circuit can be used. Because the high LED forward current of \unit[1]{A} necessitates an analog output stage, we opted for a fully analog design using NE555~\cite{ne555} precision timers. To generate the signal for the LEDs, the NE555 is operated in A-stable mode (c.f. Sec. 8.3.2~\cite{ne555}). 

The current output of the precision timer is limited to \unit[200]{mA}, but its performance significantly degrades if the output current exceeds \unit[10]{\%} of the maximum value (c.f. Figure 3 of~\cite{ne555}). Therefore an SPST (single pole, single throw) digital switch is used. We selected an ADG802 as it features close to \unit[1]{A} pulsed current and has typical switching times around \unit[55]{ns}~\cite{adg802}. While considerably slower than the LED, it is still fast enough for the given application.

\begin{table}[t]
    \centering
    \caption{\vspace*{6pt}\parbox{1\linewidth}{\textnormal{Resistor and capacitor values (see Fig.~\ref{fig:circuit_diagram}) for the different LEDs. The calculated periods as well as the measured frequencies $f$ and duty-cycles $\alpha$ are listed.}}}
    \vspace*{-6pt}
    \label{tab:resistor_capacitor_values}
    \begin{tabularx}{1\linewidth}{CCC|CCC|CC}
\toprule
\multicolumn{3}{c|}{Part Specification} & \multicolumn{3}{c|}{Calc. from Sec. 8.3.2 \cite{ne555} } & \multicolumn{2}{c}{Measured} \\[2pt]
$R_A$ \newline [$\unit{k\Omega}$] 
& $R_B$ \newline [$\unit{k\Omega}$] 
& $C$ \newline [\unit{nF}] 
& $t_\text{on}$ \newline [$\unit{\mu s}$] 
& $t_\text{off}$ \newline [$\unit{\mu s}$] 
& $f$ \newline [\unit{kHz}]
& $f_\text{meas}$ \newline [\unit{kHz}] 
& $\alpha_\text{meas}$ \newline [\%]\\ \midrule %
%
68.1 & 0.39 & 10 & 2.7 & 477 & 2.094 & 1.73 & 0.66 \\
59.0 & 0.39 & 10 & 2.7 & 415 & 2.413 & 1.98 & 0.75 \\
51.1 & 0.39 & 10 & 2.7 & 359 & 2.781 & 2.29 & 0.87 \\
44.2 & 0.39 & 10 & 2.7 & 312 & 3.207 & 2.61 & 0.99 \\
40.2 & 0.39 & 10 & 2.7 & 284 & 3.520 & 2.86 & 1.09 \\
\bottomrule
\end{tabularx}
\end{table}


The LEDs, precision timers and switches are all supplied with a single LM7805 voltage regulator. This is possible because the time-averaged load is well below the design limit of the voltage regulator. Each NE555 draws {$I_\text{NE555} = \unit[3]{mA}$} of supply current. The time averaged current $\bar{I}_\text{LED}$ for an LED pulsed with a duty cycle $\alpha = t_\text{on} / t_\text{period}$ with a pulse current {$I_p = \unit[1]{A}$} is given by
\begin{equation}
    \bar{I}_\text{LED} = \alpha \cdot I_p
\end{equation}
Assuming there is $N=5$ LEDs and they are operated at an average duty cycle of \unit[1]{\%} this leads to a total, time averaged current of 
\begin{equation}
    \bar{I} = N \cdot \left(\bar{I}_\text{LED} + I_\text{NE555} \right) = \unit[65]{mA}
\end{equation}
which is within specifications for an LM7805~\cite{lm7805} without any additional cooling (given the TO220 package and a \unit[12]{V} supply). Therefore, a sufficiently large decoupling capacitor $C_{DC}$ charged through $R_{DC}$ is used to supply the LEDs with power, effectively shielding the LM7805 from all current spikes caused by the LEDs. Based on all the above considerations, the circuit shown in Fig.~\ref{fig:circuit_diagram} has been designed and subsequently manufactured into a PCB with SMD version of the NE555 and the ADG802. The dotted line marks the components in the circuit that are replicated for each LED.

\begin{figure}
    \centering
    \includegraphics{media/main-figure4.pdf}
    \vspace*{-6pt}
    \caption{Current through an LED blinking at the lowest frequency of \unit[1.73]{kHz}. The upper plot shows the entire period $t = t_\text{on} + t_\text{off}$, whereas the lower plot shows only the pulse through where the LED is on for $\unit[3.8]{\mu s}$. From the plot, we get a switch-on time constant $\tau_\text{on}$ of the LED of \unit[84]{ns} (time to reach \unit[63]{\%} of the steady-state value).}
    \label{fig:led_current}
\end{figure}

The resistor and capacitor values used in the NE555 timer circuit are listed in Tab.~\ref{tab:resistor_capacitor_values}. The values have been calculated such that the frequencies of all 5 LEDs follow the points \ref{item:upper_limit} to \ref{item:duty}. After building and manufacturing of the PCB, the measured frequencies are also listed in Tab.~\ref{tab:resistor_capacitor_values}. Note that the mismatch w.r.t the calculated values is about \unit[20]{\%}. This mismatch is consistent across 5 identical copies of the board and of no concern for the practical applications as it is straightforward to measure the blinking frequency with an oscilloscope. Exemplarily, Fig.~\ref{fig:led_current} shows the current through one LED.



%
In this work, we presented counterfactual situation testing (CST), a new actionable and meaningful framework for detecting individual discrimination in a dataset of classifier decisions.
We studied both single and multidimensional discrimination, focusing on the indirect setting.
For the latter kind, we compared its multiple and intersectional forms and provided the first evidence for the need to recognize intersectional discrimination as separate from multiple discrimination under non-discrimination law.
Compared to other methods, such as situation testing (ST) and counterfactual fairness (CF), CST uncovered more cases even when the classifier was counterfactually fair and after accounting for statistical significance.
For CF, in particular, we showed how CST equips it with confidence intervals, extending how we understand the robustness of this popular causal fairness definition. 

The decision-making settings tackled in this work are intended to showcase the CST framework and, importantly, to illustrate why it is necessary to draw a distinction between idealized and fairness given the difference comparisons when testing for individual discrimination. 
We hope the results motivate the adoption of the \textit{mutatis mutandis} manipulation over the \textit{ceteris paribus} manipulation.
We are aware that the experimental setting could be pushed further by considering higher dimensions or more complex causal structures. 
We leave this for future work.
%
Further,
extensions of CST should consider the impact of using different distance functions for measuring individual similarity \parencite{WilsonM97_HeteroDistanceFunctions}, and should explore a purely data-driven setup in which the running parameters and auxiliary causal knowledge are derived from the dataset \parencite{Cohen2013StatisticalPower, Peters2017_CausalInference}.
%
Furthermore,
extensions of CST should study settings in which the protected attribute goes beyond the binary, such as a high-cardinality categorical or an ordinal protected attribute \parencite{DBLP:journals/tkde/CerdaV22}. 
The setting in which the protected attribute is continuous is also of interest, though, in that case we could discretize it \parencite{DBLP:journals/tkde/GarciaLSLH13} and treat it as binary (the current setting) or as a high-cardinality categorical attribute.

Multidimensional discrimination testing is largely understudied \parencite{DBLP:conf/fat/0001HN23, WangRR22}. 
% Here, 
We have set a foundation for exploring the tension between multiple and intersectional discrimination, but future work should further study the problem of dealing with multiple protected attributes and their intersection.
It is of interest, for instance, formalizing the case in which one protected attribute dominates the others and the case in which the impact of each protected attribute varies based on individual characteristics.
% Formalizing the case in which one protected attribute dominates over the others as well as the case in which the effect of each protected attribute varies by individual characteristics are of interest.
While interaction terms and heterogeneous effects are understudied within SCM, both topics enjoy a well established literature in fields like economics \parencite{Wooldridge2015IntroductoryEconometrics}, which should enable future work.
% 
We hope these extensions and, overall, the fairness given the difference powering the CST framework motivate new work on algorithmic discrimination testing.

%
% EOS
%



{\small
\bibliographystyle{IEEEtran}
\bibliography{references}
\balance
}

\end{document}