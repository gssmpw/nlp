\documentclass[letterpaper, 10 pt, conference]{ieeeconf}  
\IEEEoverridecommandlockouts                            

% Graphics
\usepackage{graphics} % for pdf, bitmapped graphics files
\usepackage{epsfig} % for postscript graphics files
\usepackage{graphicx}
\usepackage{xcolor} % for text in colors
\usepackage{capt-of}

% Path
\usepackage{amsmath} % assumes amsmath package installed
\usepackage{amssymb}  % assumes amsmath package installed
\usepackage{amsfonts}
\usepackage{mathtools}% Loads amsmath
\usepackage{units}

% Tables
\usepackage{booktabs}
\usepackage{multirow}
\usepackage{tabularx}
\usepackage{multirow}

% Utils
\usepackage{hyperref}
\usepackage{balance}
\usepackage{cite}

% Tikz + Plotting
\usepackage{tikz}
\usepackage{pgfplots}
\usepackage{pgfplotstable}
\usepackage{circuitikz}
\usetikzlibrary{bending, plotmarks, shapes, pgfplots.statistics, pgfplots.colorbrewer, positioning}

\pgfplotsset{scaled y ticks=false}
\pgfplotsset{compat=1.16,
    width = 7cm,
    height = 3.5cm,
    title style = {yshift=-6pt},
    xlabel shift = -3pt,
    ylabel shift = -0pt,
    cycle list={{matlab1},{matlab2},{matlab3},{matlab9},{matlab4},{matlab5},{matlab6},{matlab7},{matlab8}},
    legend columns = 1,
    legend cell align={left},
    legend style ={
        draw = gray,
        fill opacity=0.8,
        text opacity=1.0,
        draw opacity=1.0,
    },
    xmajorgrids,
    ymajorgrids,
    scale only axis,
} 
\usepackage{xfrac}
% \tikzset{external/force remake} % DONT DELETE THIS LINE

\newcommand{\cmark}{\ding{51}}%
\newcommand{\xmark}{\ding{55}}%

% Typical frame defs
\newcommand{\wfr}[0]{\ensuremath{\mathcal{W}}} % world frame
\newcommand{\bfr}[0]{\ensuremath{\mathcal{B}}} % body frame
\newcommand{\cfr}[0]{\ensuremath{\mathcal{C}}} % C-frame
\newcommand{\pfr}[0]{\ensuremath{\mathcal{P}}}

% 3D Tikz stuff
\newcommand{\rotateRPY}[3]% roll, pitch, yaw
{   \pgfmathsetmacro{\rollangle}{#1}
    \pgfmathsetmacro{\pitchangle}{#2}
    \pgfmathsetmacro{\yawangle}{#3}

    % to what vector is the x unit vector transformed, and which 2D vector is this?
    \pgfmathsetmacro{\newxx}{cos(\yawangle)*cos(\pitchangle)}
    \pgfmathsetmacro{\newxy}{sin(\yawangle)*cos(\pitchangle)}
    \pgfmathsetmacro{\newxz}{-sin(\pitchangle)}
    \path (\newxx,\newxy,\newxz);
    \pgfgetlastxy{\nxx}{\nxy};

    % to what vector is the y unit vector transformed, and which 2D vector is this?
    \pgfmathsetmacro{\newyx}{cos(\yawangle)*sin(\pitchangle)*sin(\rollangle)-sin(\yawangle)*cos(\rollangle)}
    \pgfmathsetmacro{\newyy}{sin(\yawangle)*sin(\pitchangle)*sin(\rollangle)+ cos(\yawangle)*cos(\rollangle)}
    \pgfmathsetmacro{\newyz}{cos(\pitchangle)*sin(\rollangle)}
    \path (\newyx,\newyy,\newyz);
    \pgfgetlastxy{\nyx}{\nyy};

    % to what vector is the z unit vector transformed, and which 2D vector is this?
    \pgfmathsetmacro{\newzx}{cos(\yawangle)*sin(\pitchangle)*cos(\rollangle)+ sin(\yawangle)*sin(\rollangle)}
    \pgfmathsetmacro{\newzy}{sin(\yawangle)*sin(\pitchangle)*cos(\rollangle)-cos(\yawangle)*sin(\rollangle)}
    \pgfmathsetmacro{\newzz}{cos(\pitchangle)*cos(\rollangle)}
    \path (\newzx,\newzy,\newzz);
    \pgfgetlastxy{\nzx}{\nzy};
}

\DeclareMathOperator*{\argmin}{arg\,min}

\newcolumntype{C}{>{\centering\arraybackslash}X}
\newcolumntype{x}[1]{>{\centering\let\newline\\\arraybackslash\hspace{0pt}}p{#1}}
\definecolor{matlab1}{rgb}{0.00000,0.44700,0.74100}
\definecolor{matlab2}{rgb}{0.85000,0.32500,0.09800}
\definecolor{matlab3}{rgb}{0.92900,0.69400,0.12500}
\definecolor{matlab4}{rgb}{0.49400,0.18400,0.55600}
\definecolor{matlab5}{rgb}{0.4660, 0.6740, 0.1880}
\definecolor{matlab6}{rgb}{0.3010, 0.7450, 0.9330}
\definecolor{matlab7}{rgb}{0.6350, 0.0780, 0.1840}
\definecolor{matlab8}{rgb}{0.8, 0.8, 0}
\definecolor{matlab9}{rgb}{0.6, 0.6, 0.6}
\definecolor{verylightgray}{rgb}{0.98,0.98,0.98}

\pgfdeclarelayer{bg}   

\makeatletter
\g@addto@macro\@maketitle{
    \setcounter{figure}{0}
    \vspace*{12pt}
    \centering
    \includegraphics[width=17.6cm]{media/Figure1_Wide.pdf}
    \vspace*{-3pt}
    \captionof{figure}{Overview of the monocular event-camera motion capture system: the object to track (e.g. a small quadrotor) is equipped with $N>=4$ infrared LED markers that blink at different frequencies. A single, calibrated event-camera is used to detect all markers and estimate the pose of the object by solving the perspective-n-points problem (PnP) with SqPnP~\cite{terzakis2020sqpnp}. The state estimate from this motion-capture system can then be used for closed-loop control.}
    \label{fig:overview}
    \vspace*{-12pt}
}


\title{\LARGE \bf
A Monocular Event-Camera Motion Capture System
}

\author{Leonard Bauersfeld and Davide Scaramuzza\\ 
Robotics and Perception Group, University of Zurich, Switzerland\\
\thanks{This work was supported by the European Union’s Horizon Europe Research and Innovation Programme under grant agreement No. 101120732 (AUTOASSESS), the European Research Council (ERC) under grant agreement No. 864042 (AGILEFLIGHT) and the Swiss National Science Foundation under project No. 200021\_212065.}
}


\begin{document}

\makeatletter
\maketitle
\thispagestyle{empty}

% 150words
% Replying to workplace emails that are typically long and require politeness is time-consuming and cognitively demanding.
% Replying to lengthy and polite workplace emails is often time-consuming and cognitively demanding.
% takes time to understand and reply
\red{Replying to formal emails is time-consuming and cognitively demanding, as it requires crafting polite phrasing and providing an adequate response to the sender's demands.}
% \red{Replying to formal emails, which often takes time to understand and require polite phrasing, is time-consuming and cognitively demanding.}
Although systems with Large Language Models (LLM) were designed to simplify the email replying process, users still need to provide detailed prompts to obtain the expected output.
Therefore, we proposed and evaluated an \red{LLM-powered question-and-answer (QA)-based approach} for users to reply to emails by answering a set of simple and short questions generated from the incoming email.
We developed a prototype system, \textit{ResQ}, and conducted controlled and field experiments with 12 and \red{8} participants.
Our results demonstrated that \red{the QA-based approach} improves the efficiency of replying to emails and reduces workload while maintaining email quality, compared to a conventional prompt-based approach that requires users to craft appropriate prompts to obtain email drafts.
We discuss how \red{the QA-based approach} influences the email reply process and interpersonal relationship dynamics, as well as the opportunities and challenges associated with using a QA-based approach in AI-mediated communication.

% original
% Replying to lengthy and polite workplace emails is often time-consuming and cognitively demanding.
% Although systems with Large Language Models were designed to simplify the email replying process, users still needed to provide detailed prompts to obtain the expected output.
% Therefore, we proposed and evaluated a question-and-answer-based approach for users to reply to emails by answering a set of simple and short questions generated from the incoming email.
% We developed a prototype system, \textit{ResQ}, and conducted both controlled and field experiments with 12 and 9 participants.
% Our results demonstrated that ResQ improves the efficiency of replying to emails and reduces workload while maintaining email quality compared to a conventional prompt-based approach that requires users to craft appropriate prompts to obtain email drafts.
% We discuss how ResQ influences the email reply process and interpersonal relationship dynamics, as well as the opportunities and challenges associated with using a QA-based approach in AI-mediated communication.

\section{Introduction}

\begin{figure*}
    \centering
    \includegraphics[width=\textwidth]{figures/Introduction.pdf}
    \caption{Showing the novel problem statement applied to traffic prediction use case. Multiple unstructured observations from the past are used to reconstruct a hidden traffic state from which a full traffic state is forecast with a set of query locations. }
    \label{fig:intro}
\end{figure*}

% Was sagen denn die anderen warum Traffic Prediction gut ist? 
Forecasting the traffic in the near future is an important task for city management.
Data from the near past is used to predict future traffic states with spatio-temporal Graph Neural Networks \cite{bui22}.
Accurate prediction provides the opportunity to optimize traffic flow, reduce traffic jams and increase air quality \cite{Po19}.

% Wieso ist Sparsity in allen Dimensionen wichtig.
While traffic prediction relies on the availability of data from traffic sensors, there exists a plethora of reasons why sensors may stop working temporarily, such as simple errors, energy saving, or overloaded communication systems.
Considering small- or medium-sized cities, the coverage of sensors may be low because the sensors are too expensive or not available.
Also, the sensors are typically static and do not adapt to changes in the traffic flow (e.g. caused by a construction site), which motivates moving sensors that for example could be mounted on cars. 
However, both missing and moving sensors introduce sparsity, since measurements may not be available for all locations at all times.
This sparsity must be explicitly addressed in traffic prediction for a realistic application scenario, which is illustrated in figure \ref{fig:intro}.
From one hour of data on Sunday morning, only few observations of the traffic state are available at each timestep.
The number of observations may differ throughout the observed time and the observation itself can be distributed arbitrarily in the city. 
We assume a relatively low number of sensors to account for resource saving and sensor failure in our proposed framework SUSTeR.
The task is to predict the dense traffic state one timestep after the observations at all possible sensor locations.
We study this problem on the traffic dataset Metr-LA and PEMS-BAY to test our assumption that only a fraction of the sensor values would be enough for good predictions.
By modifying an existing traffic dataset, we are able to compare our results from very sparse observations to the bottom line with all information available.
A successful study will provide insights in how sensors in new cities can be reduced before installing them and further mobile sensors would save more resources and are able to adapt to new traffic situations.
We argue that in order to be adaptable to other cities and changes in traffic flows, prior information like the road network should be neglected and just the sparse observations considered.
This comes with the added benefit of making our solution applicable in regions where no openly available road network is maintained or pathways change frequently (e.g. flood areas, animal observations). 


The aforementioned problem is novel and more challenging than the commonly considered traffic prediction problem, since there exist very few observations in each input sample.
Current works for the traffic prediction problem do not consider any missing values. \cite{Li2021, Shao22}
A common method among state of the art approaches is the usage of Graph Neural Networks on graphs that model the sensor network \cite{bui22}.
The values of a sensor are applied to the same graph node for each timestep which prohibits any non-stationary sensors . 
With fixed sensor locations, the resulting sensor network is highly correlated with the road network.
Streets connecting two intersections with sensors should be also an interesting point for correlations in the sensor network.
However, variable observations and high temporal sparsity rates can not be modeled adequately in a static network.
We show in our experiments that the road network has only a small influence on the traffic predictions.

Besides the traffic prediction for future timesteps, some works explore the field of traffic speed imputation \cite{Cini22, Cuza22} where missing sensor values are predicted.
But the amount of missing values is assumed to be at most 80\%, which on average are still over 40 given sensors in each timestep in the Metr-LA dataset with a total of 207 sensors.
We consider up to 99.9\% missing values which are on average 2.4 observations in each timestep that are used as input.
Such high sparsity rates drastically decrease the chance that multiple values are present in one input sample from the same sensor location, which makes it challenging to recognize and learn temporal correlations for each location on its own.

High sparsity rates (>95\%) result in few sensor values, but if a reconstruction of the traffic state would be possible, we question if spatio-temporal graphs require nodes for each sensor.
In SUSTeR we utilize only a small amount of graph nodes for the encoding of information and do not relate such nodes to the sensor network.
We call this the hidden graph (see figure \ref{fig:intro}), which is still able to reconstruct the complete traffic state.
Due to the reduced number of nodes SUSTeR achieves faster runtimes, as shown in the experiments.
This hidden graph is not embedded directly in the spatial domain, which is why the assignment of observations, as well as the querying of the future traffic, is done with an encoder and a decoder, implemented as neural networks.
The decoding from the hidden graph to future values depends on a set of query locations.
Figure \ref{fig:intro} shows the query locations as given from outside and in combination with the reconstructed traffic state the future values are predicted.

To construct the hidden graph we encode observations from each timestep into from multiple graphs, one for each timestep. 
The graphs are created in a residual style and information is added to the node embeddings from the previous timesteps.
We choose this method to incorporate all timesteps equally into the hidden state because the redundant information along the past is non-existing for high sparsity rates.
From the sequence of graphs where our framework inserted the observations step by step we apply STGCN \cite{Yu18}, an algorithm for traffic prediction to find and learn the spatio-temporal correlations on our small number of graph nodes.
The first future timestep of the STGCN is our hidden graph in which the traffic state is reconstructed. 

% Recent work has an implicit embedding of the graph nodes into the spatial domain as the assignment from the sensor to graph node is fixed one by one.
% Because the graph has the same structure as the road network spatio-temporal correlations can be learned between those sensors.
% We reduce the number of nodes and use a non-linear assignment learned data-driven from the observations.

We find in the experiments that SUSTeR outperforms the plain STGCN and modern traffic prediction frameworks like D2STGNN for high sparsity rates $(\geq 99\%)$.
This is equivalent to only $0.2$ to $2.4$ observation for each timestep on average.
SUSTeR uses fewer parameters than the baselines and can train faster and with less training data.
Our main contributions can be summarized as follows:
\begin{itemize}
    \item We introduce a sparse and unstructured variant of the traffic prediction problem with sparsity in all dimensions. The sensors report only a fraction of their values and are arbitrarily distributed in the spatial domain.
    \item We propose SUSTeR, a framework around the STGCN architecture, which maps sparse observations onto a dense hidden graph to reconstruct the complete traffic state.
    Our code is available at github.\footnote{https://github.com/ywoelker/SUSTeR}
    \item We conducts experiments that show that SUSTeR outperforms the baselines in very sparse situations ($\geq 95\%$) and has a competitive performance in low sparsity rates.
    % \item SUSTeR trains a third faster than the next competitor.
\end{itemize}

\section{Related Work}

Recent advancements in LLMs have significantly expanded their application across diverse academic fields such as social science \citep{aher_using_2023, gao_s3_2023}, behavioral economics \citep{horton_large_2023}, and human-computer interaction \citep{hamalainen_evaluating_2023}, primarily through the creation and deployment of diverse agent personas. These LLM-created personas aim to simulate complex human and social behaviors \citep{park_social_2022, park_generative_2023}, enabling the development of increasingly personalized applications such as recommendation systems \citep{wang_user_2023}.

Existing frameworks for persona creation primarily emphasize isolated human traits, focusing mainly on socio-demographic characteristics \citep{chen_empathy_2024, zhang_speechagents_2024, chuang-etal-2024-simulating} for modeling specific human subpopulations \citep{argyle_out_2023}. Although researchers have expanded these frameworks to incorporate other dimensions such as personality traits \citep{jiang-etal-2024-personallm, liu2024skepticism, xie_human_2024, yuan_evaluating_2024} and value systems \citep{zhou_sotopia_2024, xie2024largelanguagemodelagents, kang_values_2023}, their approaches often yield biased or incomplete representations, as evidenced by homogeneous depictions of socially underrepresented groups \citep{petrov_limited_2024, gupta_bias_2023, deusex_2024, cheng_compost_2023, lee_large_2024}.

Research in social psychology demonstrates that an individual's self-concept emerges from the dynamic interplay of multiple identity dimensions, including personal traits, social interactions, and lived experiences \citep{mead1934mind}. This fundamental understanding highlights the critical importance of incorporating such multidimensional aspects in developing LLM agents that authentically reflect real-world individuals and their behavioral and thought patterns \citep{xiao_how_2023}.

To address these limitations, we introduce the SPeCtrum framework (see Figure \ref{fig:1}), which enables structured and authentic persona representations through (a) identifying essential elements of multidimensional self-concept based on social science theories and research methodologies and (b) developing systematic pipelines for integrating diverse identity sources. Moving beyond the dominant focus on isolated traits, our framework emphasizes the dynamic interactions between identity components to create LLM-based agents that capture the rich complexity of real-world individuals.


\section{System Design}
\label{sec:system_design}

\begin{figure}
    \centering
    \includegraphics{media/main-figure0.pdf}
    \caption{Definition of the camera frame ($z_\cfr$ is aligned with the optical axis), the body frame $\bfr$ and the world frame $\wfr$. The position of the active markers is defined in body-frame and must be known. The transform $T_{\cfr \bfr}$ is estimated through PnP and the pose of the camera in the world frame is assumed to be known.}
    \label{fig:coordinate_systems}
\end{figure}

This section gives a brief overview of the developed monocular event-camera motion-capture system. Details regarding the implementation of the blinking LED detection are given in Sec.~\ref{sec:implementation_details}. The blinking LED circuit itself is discussed in Sec.~\ref{sec:blinked_led_circuit}.

\subsection{Prerequisites}
In order to obtain accurate 3D pose estimation of the tracked objects, the event-camera must be calibrated and the location of the blinking LEDs must be known. Furthermore, the transform $T_{\mathcal{CW}}$ between the camera frame $\mathcal{C}$ and the world frame $\mathcal{W}$ must be known. The coordinate systems are illustrated in Fig.~\ref{fig:coordinate_systems}.

To calibrate the camera, we follow the approach from~\cite{muglikar2021calibrate} where the calibration is performed by first converting the event stream into event frames and then calibrating these using Kalibr~\cite{rehder2016extending}. The calibration also estimates the lens distortion and in this work we rely on a double-sphere distortion model because of its accuracy and computation efficiency~\cite{usenko2018doublesphere}. The event-camera is mounted horizontally on a stable tripod such that the transform from camera-frame $\mathcal{C}$ to world-frame $\mathcal{W}$ is fixed and known.

To ensure that the locations of the blinking LEDs are known and fixed, a 3D-printed LED holder is used. The locations of the LEDs are then directly known from the CAD model. The blinking frequency of each LED is also known, see Sec.~\ref{sec:blinked_led_circuit} for details on the circuit design.

\subsection{LED detection}
Robustly detecting blinking LEDs in an event-stream in real-time is the critical component of the motion-capture system. Events are processed in batches between \unit[1]{ms} (\unit[1]{kHz}) and \unit[2.5]{ms} (\unit[400]{Hz}), depending on the desired pose update rate. In a first step, all pixels whose event rate is below a threshold are discarded. The threshold is calculated based on the assumption that each LED period at least triggers two events and that a transition is detected with a given probability (e.g. \unit[80]{\%}).

For all pixels with a sufficient event rate, the average period and standard deviation is calculated. For details on this process, see Sec.~\ref{sec:implementation_details}. After identifying the average period, neighboring pixels with similar periods are clustered together. If the average period of the cluster is closer than $\unit[25]{\mu s}$ to one of the expected blinking frequencies from the LED, it is matched to that LED. The centroid of each LED is tracked with a constant-velocity particle filter.

\subsection{Pose Estimation}
To estimate the pose from the detected LED centroids, the centroids are undistorted first. Then, the PnP-problem is solved with SQPnP~\cite{terzakis2020sqpnp}. We chose this algorithm over other well-known algorithms such as EPnP~\cite{lepetit2009epnp} because SQPnP is fast enough and globally optimal~\cite{terzakis2020sqpnp}. 

The pose-estimate in the camera-frame is finally transformed into the world-frame and published via ROS. 
\section{Experimental Results}
\label{sec:experimental_results}
Often a motion capture systems are chosen because of their low noise levels and low latency which is ideally suited for robot control. Therefore, we demonstrate the performance of the developed system, by flying the small drone depicted in Fig.~\ref{fig:overview} in closed-loop with the event-camera motion capture being the only source of state estimation. We use a Prophesee Gen 3.1 event-camera ($\unit[640\times 480]{px}$, \unit[3/4]{inch} sensor) with either a \unit[25]{mm} or a \unit[50]{mm} lens resulting in a horizontal FoV (field of view) of \unit[22]{deg} and \unit[11]{deg}, respectively.

\subsection{Pose Estimation Noise}
In a first experiment the drone is rigidly placed at various distances in front of the camera. Then, 10 seconds of data are recorded with the event-camera and we calculate the standard deviation of the pose estimate. Since the drone is static, all deviations from the mean are only due to noise. The results of this analysis summarized in Fig.~\ref{fig:pose_noise}. A few interesting observations can be made which are subsequently discussed in detail:
\begin{enumerate}
    \item The noise levels along the $z_\cfr$ axis are much larger compared to the $x_\cfr$ and $y_\cfr$ axis.
    \item The SqPnP~\cite{terzakis2020sqpnp} algorithm achieves a much better performance than EPnP~\cite{lepetit2009epnp}.
    \item For SqPnP the noise levels in position scale quadratically with the distance from the camera, while the orientation noise scales linearly.
\end{enumerate}

\begin{figure}
    \centering
    \includegraphics{media/main-figure1.pdf}
    \caption{The object is placed at distances between \unit[70]{cm} and \unit[5]{m} statically in front of the camera (with the \unit[25]{mm} lens). The plots show the standard deviation in the position measurement ($z_\cfr$ and $x_\cfr$, $y_\cfr$) as well as the orientation measurements. We can clearly see that SqPnP~\cite{terzakis2020sqpnp} outperforms EPnP~\cite{lepetit2009epnp} by a large margin.}
    \label{fig:pose_noise}
\end{figure}

The $z_\cfr$-axis is the optical axis of the camera and hence the $z_\cfr$ coordinate can only be inferred from the scale of the object. Assuming that elongation of the object along the optical axis is small compared to the distance to the camera (i.e. the object is nearly flat), a well-known result from stereo-vision applies~\cite{zisserman2004multipleview}: for a given inter-marker distance $d$, focal length $f$ and a marker detection with uncertainty $\sigma_u$ (in pixels) the depth uncertainty $\sigma_{pz}$ scales with the square of the distance $z$ as
\begin{equation}
    \sigma_{pz} = \frac{\partial z_\cfr}{\partial u} \cdot \sigma_u = \frac{b \cdot f}{z_\cfr^2} \cdot \sigma_u \sim \frac{1}{z^2}\;.
\end{equation}
For the positional errors in $x_\cfr$ and $y_\cfr$ we observe a similar quadratic dependency on the camera-object distance, however with much less noise. Intuitively, this makes sense as the translation along $x_\cfr$ and $y_\cfr$ are directly observable from each marker and thus the estimate is much more accurate.

The position noise plots also highlight the superior performance of SqPnP for this task: the optimization-based approach is able to estimate the position with much less variance given the same input data. This discrepancy becomes even larger when considering the orientation estimation shown at the bottom plot of Fig.~\ref{fig:pose_noise}. SqPnP dramatically outperforms EPnP which performs between two and four times worse. Interestingly, we observe that EPnP shows a large but nearly constant orientation uncertainty after \unit[2]{m}, whereas SqPnP shows a linear increase in the noise standard deviation. Note that we do not compare against EPnP with nonlinear refinement as the OpenCV implementation requires at least six points for iterative refinement.

\subsection{Closed-Loop Deployment}

\begin{figure}[t]
    \centering
    \includegraphics{media/main-figure2.pdf}
    \vspace*{-18pt}
    \caption{Closed-loop experiments: the drone flies a rectangular pattern starting at (2,0.1) and then lands at a distance of \unit[2.5]{m}. The event-camera is located at the origin of the coordinate system at $(x_\wfr, y_\wfr) = (0,0)$ and the optical axis of the \unit[50]{mm} lens is aligned with the $x_\wfr$ direction. The bottom plot shows the roll and pitch angle measurements during the flight.}
    \label{fig:closed_loop}
\end{figure}

In this experiment we fly the small drone shown in Fig.~\ref{fig:overview} in closed-loop and the event-camera motion capture runs at \unit[400]{Hz} leading to a system latency of \unit[2.5]{ms}. The monocular event-camera motion capture system is the only source of state-estimation for the MPC controller~\cite{foehn2022agilicious} of the quadrotor. The drone is tasked to fly a rectangular pattern, hover and then land. The trajectory is shown in the top plot Fig.~\ref{fig:closed_loop} and the roll and pitch angle measurements are shown at the bottom. When the drone is further away at {$x_\wfr=\unit[3]{m}$} the position and orientation estimate become more noisy but overall we find that the system is able to safely fly the small drone, thereby demonstrating the performance of the developed system.


\section{Implementation Details}
\label{sec:implementation_details}
To ensure real-time operation of the event-camera motion capture system the delay in the processing pipeline must be kept to a minimum. This is achieved through an efficient, multi-threaded implementation in combination with filtering data early on in the pipeline. This section gives an overview of the most important concepts, specifically the data representation, the filtering and the multi-threading.

\subsection{Event Data Representation}
\label{subsec:event_data_representation}
A single event as supplied by the camera is given as a four-tuple, consisting of an $x$ and a $y$ coordinate (both \texttt{uint16\_t}), a polarity $p$ which is either -1 or 1 (\texttt{int8\_t}), and a timestamp $t$ (\texttt{uint64\_t}). Additionally, 3 bytes of padding are included for 16-byte alignment. 
The event camera supplies a stream of such raw events. For the following discussion of different data representations for blinking LED detection, an event stream containing $k$ events over a time period $T$ coming from an event camera with image width $W$, height $H$ and $N = W \times H$ pixels is considered.

\subsubsection*{1D Representations}
The \emph{event stream} is the most basic and raw representation of events and has recently gained some attention in combination with spiking neural networks~\cite{gehrig2020eventbasedangular}. For LED detection with classical CPU architectures however this representation is completely unsuitable. To extract any spatial information, the entire event stream must be searched for pixels with matching coordinates. Furthermore, having a memory layout where each event is stored serially is not efficiently using the cache: if we search for a given x coordinate, the remaining 14 bytes of the raw event representation are unused and just occupy cache space.

\subsubsection*{2D Representations}
In the \emph{event frame} representation the events are stored as a 2D grid by either summing the polarity or by counting the number of events. The accumulation is done for the time window of length $t$ which represents the equivalent of the exposure time. The conversion from an event stream to an event frame is fast as can be done in linear time $\mathcal{O}(k)$ by iterating once over the stream. This representation is suitable for filtering out which pixels have a sufficiently high number of events to be candidates for a blinking LED, however it does not include any time information which would allow robust frequency detection.

A \emph{time-surface} representation is also a 2D image, but each pixel in this 2D grid is assigned the value of the latest timestamp. For detecting blinking LEDs this representation is unsuitable as it contains no information related to periodic on-off transitions of a pixels.

\subsubsection*{3D Representations}
In an \emph{event volume}, events are stored as a 3D grid in a form that can be thought of as a stack of multiple event frames. This representation also includes time information and, given a sufficiently fine binning in the time domain, could be used to detect the frequency of a blinking LED. However, to accurately detect the frequency the binning would have to be very fine, yielding a huge memory footprint. For an accuracy of \unit[5]{$\mu s$}, a window length $t = \unit[2]{ms}$, VGA resolution and \texttt{uint8\_t} storage, the event volume would occupy \unit[117]{MB} of memory. This size exceeds all levels of the processor cache, potentially affecting runtime adversely.

\subsubsection*{Signed Delta-Time Volume}
To get past the shortcomings of those widely used event representation we propose a data representation that is ideally suited for the task of blinking LED detection: the \emph{signed delta-time volume (SDTV)}. It is a 3D volume of size {$W \times H \times D$} where $D$ is the stack depth. For each pixel the time difference to the last event is stored and the polarity of the event is encoded in the sign of this time difference. This is possible because time must be monotonically increasing, so we can re-purpose the sign-bit for polarity encoding. The idea is illustrated in Fig.~\ref{fig:sdtv} for a single pixel stack of the SDTV.
\begin{figure}
    \centering
    \includegraphics{media/main-figure3.pdf}
    \vspace*{-18pt}
    \caption{Illustration on the construction of the \emph{Signed Delta-Time Volume (SDTV)} from an event stream. \textbf{a)} The LED is blinking with a period of $\unit[300]{\mu s}$ with a duty cycle of \unit[10]{\%}. \textbf{b)} A single pixel of the event camera records a noisy signal of this blinking LED. False double events (e.g. at $t = \unit[150]{\mu s}, \unit[165]{\mu s}$) and spurious events (e.g. at $t = \unit[630]{\mu s}$) are included. \textbf{c)} Construction of the SDTV illustrated before processing the latest time window and after processing the time window. \textbf{d)} Periods robustly identified from the SDTV by summing up absolute time differences between negative $\rightarrow$ positive transitions (the first positive value is included). All events until the first positive $\rightarrow$ negative transition are discarded.}
    \label{fig:sdtv}
\end{figure}

Because of the fast blinking of the LEDs the time differences (in microseconds) between consecutive events are always within \texttt{int16\_t} range, making storage compact. As most operations are done per pixel-stack, the memory layout is such that the $D$-dimension is consecutive. Similar to the other representations, converting an event stream to SDTV is linear in the number of events.
The signed delta-time volume is not computed per window of length $T$ but updated as a cyclic buffer. This increases the accuracy of the frequency detection for LEDs blinking at a lower frequency than $f_\text{max}$ since the amount of LED periods available for frequency identification is independent of the frequency.

The minimal depth $D$ can be calculated based on the window length $T$ and the frequency of the fastest LED $f_\text{max}$ as $D = 2\,T\,f_\text{max}$ because every LED should trigger two events (once on and once off) per period. Typical values of $D$ are between 4 and 16, reducing the memory footprint by a factor of 25 to 100 compared to a event volume.


\subsection{Filtering}
When computing the signed delta-time volume representation from the event stream for a time window $t$ of events, we also compute an event frame based on the event count of each pixel. Only pixels with more than {$\beta \cdot 2 t f_\text{min}$} events are considered further where $\beta$ is the probability that a transition triggers an event. We use {$\beta = 0.8$} to purposely underestimate the detection probability.

For all selected pixels the SDTV is used to calculate mean, median and standard deviation of the period. The period is defined as the time between two on-events with at least one off-event in between as illustrated in Fig.~\ref{fig:sdtv}d). In agreement with~\cite{censi2013activeled} we find that this is a robust measure. After rejecting pixels with a too-large standard deviation in the period, pixels are clustered together. Too small and too large clusters are rejected as the expected size of an LED is roughly known a priori. Clusters are then assigned to the individual LEDs by matching the measured average period in a cluster with the blinking frequencies of the LEDs. Each LED is tracked by a particle filter that gets the assigned clusters for each LED as an input.

\subsection{Multi-Theading}
\label{subsec:multi_threading}
In a real-time application like this, relying on generic multi-threading tools such as OpenMP can be problematic. For this reason, the threading is manually implemented to ensure optimal performance. Each thread in the pipeline shares its memory with the next thread in the pipeline. To ensure threads do not block each other, each thread allocates the required memory two times. During operation, the thread writes to one of its allocations while the other memory chunk is processed by the next pipeline step. Subsequently, the memory pointers are swapped and the newly filled batch processed. 

The pipeline primarily consists of three threads. They
\begin{enumerate}
    \item copy events from event camera driver into a buffer,
    \item convert a linear event buffer into the optimized SDTV representation described in Section~\ref{subsec:event_data_representation}, and
    \item process the accumulated data to detect the LEDs, assign the LED clusters and solve the PnP (perspective-n-points) problem.
\end{enumerate}

This design makes it possible to run the pipeline at hight speeds on a modern laptop. Speeds exceeding \unit[1]{kHz} are possible, but due the slowest LED blinking at \unit[1700]{Hz} increasing the processing frequency beyond \unit[800]{Hz} might degrade robustness.
\section{Blinking LED Circuit}
\label{sec:blinked_led_circuit}
%
\begin{figure*}
    \centering
    \includegraphics{media/Fig_CircuitDiagram.pdf}
    \caption{Circuit diagram of the complete blinking LED circuit. For simplicity we only show one LED driver circuit (area enclosed in dotted line), which is then replicated multiple times to drive multiple LEDs. The input voltage range for the power supply is $V_\text{DD}$ is \unit[8-35]{V}. The resistor and capacitor values $R_A$, $R_B$, and $C$ for the A-stable NE555 operation at different frequencies are given in Tab.~\ref{tab:resistor_capacitor_values}. For improved clarity the standard \unit[100]{nF} ceramic bypass capacitors to stabilize VCC for the NE555 and ADG802 have been omitted in the circuit diagram above.}
    \label{fig:circuit_diagram}
\end{figure*}
%
\subsection{Choice of LEDs}
\label{subsec:choice_of_leds}
For the accuracy of the event-based mocap system it is important that the center of the blinking LED can be detected easily by the event camera. To achieve this, the LED should be small, very bright and have a short switching time to produce a well-defined rising and falling edge. These requirements are ideally met by LEDs optimized for pulsed operation, such as infrared LEDs for data transmission.

In this work, the Osram SFH4350~\cite{osram4350} IR-LED \unit[3]{mm} was selected as it features an extremely short switching time of \unit[12]{ns}. Being designed for opto-electronics it also permits up to \unit[1]{A} pulsed forward-current for pulses shorter than \unit[100]{$\mu$s} if the duty cycle is below \unit[2]{\%}. The LED has its peak emission around \unit[850]{nm} with a spread of \unit[50]{nm} which is within the sensitivity of most CMOS-based imaging sensors.

\subsection{Circuit Design}
\label{subsec:circuit_design}

The design of the LED driver circuit is tightly coupled with the entire event-based motion capture system:
\begin{enumerate}
    \item Faster blinking frequencies increase the reactiveness of the system as at least one full period must be detected for identifying an LED. For robustness, a more conservative approach to detect at least two periods is better. Consequently, if the LEDs blink at \unit[1]{kHz} the overall system is limited to \unit[500]{Hz} output rate.
    \label{item:lower_limit}
    \item If the LEDs blink too fast, measuring the signal becomes difficult. Typically event cameras perform very well at measuring signals with frequencies up to \unit[2-3]{kHz}~\cite{wang2024towardshighspeed, crabtree2023refactoryperiod, lichtsteiner2008a128x128dvs} and the events are timestamped with \unit[1]{$\mu$s} time resolution.
    \label{item:upper_limit}
    \item At least four LEDs are neccessary to yield a unique solution to the PnP problem.
    \label{item:number_leds}
    \item the individual frequencies should not alias into each other. This means that, ideally, all LEDs have blinking frequencies within a factor of two.
    \label{item:aliasing}
    \item Due to the limitations of the LED a duty cycle of \unit[2]{\%} can not be exceeded.
    \label{item:duty}
\end{enumerate}

To control the blinking LED either a microcontroller or an analog circuit can be used. Because the high LED forward current of \unit[1]{A} necessitates an analog output stage, we opted for a fully analog design using NE555~\cite{ne555} precision timers. To generate the signal for the LEDs, the NE555 is operated in A-stable mode (c.f. Sec. 8.3.2~\cite{ne555}). 

The current output of the precision timer is limited to \unit[200]{mA}, but its performance significantly degrades if the output current exceeds \unit[10]{\%} of the maximum value (c.f. Figure 3 of~\cite{ne555}). Therefore an SPST (single pole, single throw) digital switch is used. We selected an ADG802 as it features close to \unit[1]{A} pulsed current and has typical switching times around \unit[55]{ns}~\cite{adg802}. While considerably slower than the LED, it is still fast enough for the given application.

\begin{table}[t]
    \centering
    \caption{\vspace*{6pt}\parbox{1\linewidth}{\textnormal{Resistor and capacitor values (see Fig.~\ref{fig:circuit_diagram}) for the different LEDs. The calculated periods as well as the measured frequencies $f$ and duty-cycles $\alpha$ are listed.}}}
    \vspace*{-6pt}
    \label{tab:resistor_capacitor_values}
    \begin{tabularx}{1\linewidth}{CCC|CCC|CC}
\toprule
\multicolumn{3}{c|}{Part Specification} & \multicolumn{3}{c|}{Calc. from Sec. 8.3.2 \cite{ne555} } & \multicolumn{2}{c}{Measured} \\[2pt]
$R_A$ \newline [$\unit{k\Omega}$] 
& $R_B$ \newline [$\unit{k\Omega}$] 
& $C$ \newline [\unit{nF}] 
& $t_\text{on}$ \newline [$\unit{\mu s}$] 
& $t_\text{off}$ \newline [$\unit{\mu s}$] 
& $f$ \newline [\unit{kHz}]
& $f_\text{meas}$ \newline [\unit{kHz}] 
& $\alpha_\text{meas}$ \newline [\%]\\ \midrule %
%
68.1 & 0.39 & 10 & 2.7 & 477 & 2.094 & 1.73 & 0.66 \\
59.0 & 0.39 & 10 & 2.7 & 415 & 2.413 & 1.98 & 0.75 \\
51.1 & 0.39 & 10 & 2.7 & 359 & 2.781 & 2.29 & 0.87 \\
44.2 & 0.39 & 10 & 2.7 & 312 & 3.207 & 2.61 & 0.99 \\
40.2 & 0.39 & 10 & 2.7 & 284 & 3.520 & 2.86 & 1.09 \\
\bottomrule
\end{tabularx}
\end{table}


The LEDs, precision timers and switches are all supplied with a single LM7805 voltage regulator. This is possible because the time-averaged load is well below the design limit of the voltage regulator. Each NE555 draws {$I_\text{NE555} = \unit[3]{mA}$} of supply current. The time averaged current $\bar{I}_\text{LED}$ for an LED pulsed with a duty cycle $\alpha = t_\text{on} / t_\text{period}$ with a pulse current {$I_p = \unit[1]{A}$} is given by
\begin{equation}
    \bar{I}_\text{LED} = \alpha \cdot I_p
\end{equation}
Assuming there is $N=5$ LEDs and they are operated at an average duty cycle of \unit[1]{\%} this leads to a total, time averaged current of 
\begin{equation}
    \bar{I} = N \cdot \left(\bar{I}_\text{LED} + I_\text{NE555} \right) = \unit[65]{mA}
\end{equation}
which is within specifications for an LM7805~\cite{lm7805} without any additional cooling (given the TO220 package and a \unit[12]{V} supply). Therefore, a sufficiently large decoupling capacitor $C_{DC}$ charged through $R_{DC}$ is used to supply the LEDs with power, effectively shielding the LM7805 from all current spikes caused by the LEDs. Based on all the above considerations, the circuit shown in Fig.~\ref{fig:circuit_diagram} has been designed and subsequently manufactured into a PCB with SMD version of the NE555 and the ADG802. The dotted line marks the components in the circuit that are replicated for each LED.

\begin{figure}
    \centering
    \includegraphics{media/main-figure4.pdf}
    \vspace*{-6pt}
    \caption{Current through an LED blinking at the lowest frequency of \unit[1.73]{kHz}. The upper plot shows the entire period $t = t_\text{on} + t_\text{off}$, whereas the lower plot shows only the pulse through where the LED is on for $\unit[3.8]{\mu s}$. From the plot, we get a switch-on time constant $\tau_\text{on}$ of the LED of \unit[84]{ns} (time to reach \unit[63]{\%} of the steady-state value).}
    \label{fig:led_current}
\end{figure}

The resistor and capacitor values used in the NE555 timer circuit are listed in Tab.~\ref{tab:resistor_capacitor_values}. The values have been calculated such that the frequencies of all 5 LEDs follow the points \ref{item:upper_limit} to \ref{item:duty}. After building and manufacturing of the PCB, the measured frequencies are also listed in Tab.~\ref{tab:resistor_capacitor_values}. Note that the mismatch w.r.t the calculated values is about \unit[20]{\%}. This mismatch is consistent across 5 identical copies of the board and of no concern for the practical applications as it is straightforward to measure the blinking frequency with an oscilloscope. Exemplarily, Fig.~\ref{fig:led_current} shows the current through one LED.



Game-based approaches have shown great promise as tools for inoculating individuals against the tactics commonly used to spread misinformation. Most existing games in this domain are single-player games which offer players limited, predefined choices. While this design reduces cognitive load, it often results in interactions which feel less natural and engaging. In response, we designed a two-player, PvP game that pits a misinformation creator against a misinformation stopper. By integrating LLM-powered personas to evaluate player outputs and provide real-time feedback, we created a more open-ended and immersive experience.
We found that the game we developed effectively improved players’ media literacy. Participants demonstrated an enhanced ability to evaluate and analyze media content, identify unreliable or misleading information, and employ effective counter-misinformation strategies. Moreover, the game's engaging mechanics, combined with the competitive element, motivated players to learn from both their own strategies and those of their opponents.
These findings suggest that integrating dynamic feedback systems and competitive gameplay elements into misinformation education games offers a compelling method to deepen users' engagement, while also improving their critical media skills. Future research can build on these insights to explore other forms of interactive learning environments, focusing on diverse player experiences and varying misinformation challenges.


{\small
\bibliographystyle{IEEEtran}
\bibliography{references}
\balance
}

\end{document}