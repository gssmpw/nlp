\section{Introduction} \label{s:intro}
Trusted Execution Environment (TEE) has emerged as an essential component in enhancing the security of mobile applications and cloud services~\cite{10738357, 10.1145/2541940.2541949, 9693116}.
In a typical TEE application, code is usually divided into secure and non-secure parts. The non-secure code operates in the non-secure environment, also known as the \textit{normal world}, while the secure code runs within TEE.
% often referred to as the \textit{secure world}. 
This mechanism provides significant security advantages by isolating the secure code from the normal world, preventing attackers from accessing sensitive data or disrupting secure functions, such as key management~\cite{8359163, 9024053}, cryptography~\cite{10745433, mark2021rvtee}, and memory operations~\cite{DBLP:conf/ndss/ZhaoM19, 10.1145/3620665.3640378}.
To achieve this isolation, TEE leverages hardware-based features like memory partitioning and access control enforced by the CPU~\cite{10646815, 10.1002/cpe.4130}.
When the code in the normal world needs to access secure functions or data, it must interact with TEE using specific interfaces.
Therefore, partitioning strategies of TEE applications that incorporate accurate isolation~\cite{10.1145/3319535.3363205, 10.1145/2382196.2382214, DBLP:conf/ndss/KimKCGL0X18} and access control~\cite{9925586, 10.1145/2382196.2382214} mechanisms can be helpful in reducing the attack surface.
For instance, Zhao \etal~\cite{10.1145/3319535.3363205} designed SecTEE, a solution that integrates computing primitives (\eg, integrity measurement, remote attestation, and life cycle management) into TEE.
This design enables SecTEE to resist privileged host software attacks from the normal world through isolation.
Liu \etal~\cite{9925586} proposed a multi-owner access control scheme for access authorization in Intel SGX, which can prevent unauthorized data disclosure.
The above research has shown that well-designed partitions of TEE applications can help mitigate risks by establishing clear boundaries and secure communication protocols between the normal world and TEE.

However, TEE applications still face some threats. For example, Khandaker \etal~\cite{10.1145/3373376.3378486} summarized four attacks that can be implemented in Intel SGX: \textit{Concurrency}, \textit{Order}, \textit{Inputs}, and \textit{Nested} (COIN attacks).
In these attacks, malicious TEE users can exploit the lax detection vulnerabilities of non-secure inputs (\eg, buffer and value with incorrect size) in Intel SGX to interfere with executing programs in TEE.
Cerdeira \etal~\cite{9152801} conducted a security investigation on some popular systems based on TrustZone.
They found that input validation weaknesses and other bugs in TrustZone-assisted systems can be used to obtain permissions of TEE. For example, Huawei TEE allows a TEE application to dump its stack trace to the normal-world memory through debugging channels, leaking sensitive information.

\begin{figure*}[t]
    \centering
    \includegraphics[width=0.8\textwidth]{figures/Fig_1.drawio.pdf}
    \caption{A bad partitioning case of input validation weaknesses.}
    \label{fig:bpc}
\end{figure*}

Through analysis of these cases, we found that vulnerabilities in TEE applications often arise from insecure data communication between the normal world and TEE~\cite{10.1145/3407023.3407072, 8365755}, which usually relies on three types of parameters: input/output and shared memory~\cite{s20041090}.
These parameters are typically managed by the normal world, making them susceptible to unauthorized access.
Hence, the usage of these parameters without isolation and access control (\eg, \textit{unencrypted data output}, \textit{input validation weaknesses}, and \textit{direct usage of shared memory}) can introduce \textbf{bad partitioning} issues to TEE applications.
For example, there is a bad partitioning case in Fig.~\ref{fig:bpc}. The normal-world code validates the size of the input buffer before invoking TEE, but this validation step is missing inside the TEE code.
While attackers who gain control over the normal world have the ability to intercept sensitive data or gain unauthorized access to secure operations~\cite{279916}, they can tamper with the size of the input buffer before the TEE invocation and disrupt the execution of the TEE application through malicious input.
Such scenarios highlight the critical importance of ensuring that partition boundaries are well-defined and communication channels are fortified against potential breaches.
Therefore, this paper discusses TEE bad partitioning issues based on two key aspects: improperly ported or unported code into TEE, and the lack of in-TEE code which securely interfaces the normal world and TEE.
Unfortunately, there are currently no tools to analyze and identify TEE partitioning issues.

To address these challenges and enhance TEE security, we formulated three rules for detecting these issues. 
Following these rules, we designed a data-flow analyzer to identify bad partitioning in TEE applications by determining whether input/output and shared memory parameters passed between the normal world and TEE violate the defined rules.
This is the first automatic tool that can identify bad partitioning issues in TEE applications.
Our contributions of this paper include:
\begin{itemize}
    \item \textbf{The first systematic summary of partitioning issues.} 
    With the real-world vulnerable code where input, output, and shared memory parameters are handling in sensitive functions and data of TEE, we summarized that the bad partitioning issues in TEE are often caused by the improper usage of these parameters.
    Unlike previous research~\cite{10.1145/3373376.3378486, 9152801}, which primarily focused on malicious inputs to TEE, our approach addresses the potential vulnerabilities brought by three parameters within TEE, offering a more comprehensive solution to bad partitioning issues.
    \item \textbf{Proposing a parameter data-flow analyzer to identify partitioning issues.} 
    % We designed an approach to automatically identify the types of parameters passed between TEE and the normal world, and construct their data-flows.
    Based on parameter types, we established a set of rules that these parameters must adhere to as they flow through sensitive functions and and interact with sensitive data.
    By applying data-flow analysis to determine whether these parameters violate the established rules caused by partitioning issues, we can effectively identify instances of bad partitioning in TEE applications.
    We named it \textbf{\ccSysName}, a fantastic beast has the ability to distinguish between good and evil from Chinese Mythology \textit{Journey to the West}.
    \item \textbf{Creating the benchmark for bad partitioning detection.} Due to the lacking of datasets containing TEE partitioning issues, we created the first \textbf{Partitioning} \textbf{E}rrors \textbf{Bench}mark named \textbf{\ccBenchName}, which consists of 110 test cases, and 90 cases of them cover 3 types of bad partitioning vulnerabilities. Experimental results demonstrate that \ccSysName achieves an F1 score of 0.90 in identifying bad partitioning issues. Using \ccSysName, we also find 55 issues in the real-world programs and are confirming them with the developers.
    \item The source code of \ccSysName and \ccBenchName are available on \url{https://github.com/CharlieMCY/PartitioningE-in-TEE}.
\end{itemize}

The rest of the paper is organized as follows. We first describe the data interaction model and bad partitioning issues of TEE applications in Section~\ref{s:bg}. Next, Section~\ref{s:design} presents the system design of \ccSysName and Section~\ref{s:eva} evaluates the system. Subsequently, we discuss the threats to validity and related works in Section~\ref{s:val} and Section~\ref{s:related}, respectively. Finally, we conclude the paper and mention future work in Section~\ref{s:con}.
