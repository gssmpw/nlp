\section{Threats to Validity} \label{s:val}

\subsection{Internal Validity}
We use the \ccBenchName dataset, which includes manually injected vulnerabilities, to evaluate \ccSysName.
This could lead to a discrepancy between the distribution of bad partitioning issues in the dataset and those in real-world TEE projects, thereby affecting the evaluation of detection accuracy.
It is possible that we incorporate more vulnerabilities from real-world TEE projects into \ccBenchName to reduce biases introduced by manual injection.

The performance of \ccSysName also heavily relies on the underlying static analysis tools.
However, most tools (\eg, CodeQL and Frame-C~\cite{framec}) typically necessitate successful compilation and expert customization.
This dependency limits their applicability to analyzing incomplete or uncompilable code.
If these tools fail to accurately generate data-flows, the detection capabilities of \ccSysName could be compromised.
LLMDFA~\cite{wang2024llmdfa} proposed by Wang \etal makes up for this defect.
Combined with the large language model, LLMDFA effectively improves the accuracy of identifying data-flow paths.

\subsection{External Validity}
The evaluations focus on TEEs based on ARM TrustZone and Intel SGX architectures. The detection performance of \ccSysName on other TEE architectures, such as AMD SEV~\cite{10.1145/3623392, 10.1145/3214292.3214301} or RISC-V-based TEE~\cite{10.1145/3433210.3453112, 9343170}, needs further evaluation.
In addition, the evaluated projects are relatively small in scale and do not cover complex industrial-level projects.
Future studies should evaluate \ccSysName on a broader range of TEE architectures and large-scale real-world projects, thereby improving the external applicability of the experimental results.

\subsection{Construct Validity}
Our study defines three types of rules (encrypting before output, checking value before in-TEE use, and using deep copy of shared memory) to identify bad partitioning issues.
However, in practical development, there may be certain security assumptions that violate these rules without actually compromising security, potentially leading to false positives.
For example, directly outputting a static string (\eg, a software version number, user id, etc.) that does not contain any sensitive data is considered secure and does not require encryption.
Therefore, if these definitions are not accurate or comprehensive enough, it may lead to biased detection results.
We need to update the rules based on the newly discovered bad partitioning issues in actual TEE applications to expand the detection capabilities of \ccSysName.
