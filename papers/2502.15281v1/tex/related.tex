\section{Related Work} \label{s:related}
\subsection{Vulnerabilities in TEE}
TEE has become integral to security-critical applications by providing isolated execution environments to protect sensitive data and computations. However, TEE applications are still affected by some vulnerabilities, as demonstrated by extensive research across various TEE technologies.
Cerdeira \etal~\cite{9152801} conducted a study on vulnerabilities in TrustZone-based systems. They have revealed critical implementation bugs, such as buffer overflows and inadequate memory protection mechanisms. Their work also underscores the role of hardware-level issues like microarchitectural side-channels in compromising TEE security.
Fei \etal~\cite{10.1145/3456631} provided a taxonomy of secure risks in Intel SGX and discussed attack vectors, such as hardware side-channel exploits and software code vulnerabilities.
Similarly, Khandaker \etal~\cite{10.1145/3373376.3378486} introduced the concept of COIN attacks against SGX, and they expose the insecurity of untrusted interfaces in TEE projects.
Bulck \etal~\cite{10.1145/3319535.3363206} assessed the vulnerabilities at the level of the Application Binary Interface (ABI) and the Application Programming Interface (API) that can lead to memory and side-channel attacks in TEE runtime libraries.

To mitigate the risk of attacks on TEE projects, some research has designed tools to detect vulnerabilities within TEE.
Briongos \etal~\cite{10.1145/3627106.3627187} present CloneBuster for detecting cloning attacks on Intel SGX applications. CloneBuster can find whether there are same TEE application binaries running on one platform, which may roll back the TEE applications to the previous state. 
Ghaniyoun \etal~\cite{10.1145/3579371.3589070} designed TEESec, a pre-silicon framework for discovering microarchitectural vulnerabilities in TEE. TEESec can test underlying microarchitecture against data and metadata leakage across isolation boundaries.

However, most of these studies focus on the security of TEE hardware architecture and development kits that may have side-channel attack vulnerabilities. Research addressing bad partitioning issues in TEE projects remains limited, despite it being an equally critical concern.
Poorly designed TEE code partitioning can result in severe security risks, such as the leakage of sensitive data or the injection of untrusted data, which could influence the integrity of TEE application execution.
Our work addresses this gap by comprehensively detecting security vulnerabilities caused by bad partitioning in TEE code, particularly in scenarios involving data interactions between the normal world and TEE with input, output, and shared memory. 

\subsection{TEE Application Partitioning}
Some research has attempted to mitigate vulnerabilities arising from bad partitioning by porting entire programs into TEE. 
For example, CryptSQLite~\cite{8946540} is a database system that places the entire database engine into TEE. 
This approach prevents attackers from stealing or tampering with critical data computed in TEE from the normal world, effectively protecting the confidentiality and integrity of data during database retrieval.
RT-TEE~\cite{9833604} is also a cyber-physical system that puts the whole flight control code into TrustZone. This can protect the stability of equipment like drones.
However, this way significantly increases Trusted Computing Base (TCB) of the applications, contradicting the principle of maintaining a small TCB for trusted code~\cite{DBLP:conf/ndss/ShindeTTS17, 10.1145/3313808.3313810}.

To address this, some other research made efforts to identify sensitive data and functions that require protection and place them inside TEE, ensuring a balance between security and efficiency.
Rubinov \etal~\cite{10.1145/2884781.2884817} developed an automatic approach for partitioning Android applications into critical code running in TEE and client code in the normal world.
They also reduce the overhead due to transitions between the two worlds by optimizing the granularity of TEE code.
Sun \etal~\cite{10577323} proposed dTEE that enables developers to declare tiered-sensitive variables and functions of existing IoT applications. What's more, dTEE can automatically transform device drivers into trusted ones.
Moreover, DarkneTZ~\cite{10.1145/3386901.3388946} is a framework that limits the attack surface of the model against Deep Neural Networks with the help of Trusted Execution Environment in an edge device.
They partitioned the model into sensitive layers inside TEE, and other layers executed in the normal world. 
This will provide reliable model privacy and defend against membership inference attacks.

The above TEE program partitioning schemes ensure software security while also maintaining the efficiency of program execution.
However, as discussed in Section~\ref{s:bp}, improper TEE partitioning can still result in vulnerabilities. Therefore, a tool capable of analyzing and identifying bad partitioning issues in TEE implementations is essential.
