\usepackage{color}
\usepackage{graphicx}
\usepackage{booktabs}
\usepackage{pifont}
\usepackage{listings}
\usepackage{graphicx}
\usepackage{caption}
\usepackage{subcaption}
\usepackage{xcolor}
\usepackage{multirow}
\usepackage{makecell}
\usepackage[linesnumbered,ruled,vlined]{algorithm2e}
\usepackage{amsmath}
\usepackage{enumitem}
\usepackage{tcolorbox}
\usepackage{threeparttable}
\usepackage{balance}
% \usepackage{cite}
\usepackage{url}
\usepackage{hyperref}
\usepackage{wasysym}

\newcommand{\mnote}[1]{\colorbox{yellow}{#1}}

\newcommand{\etal}{et al.}
\newcommand{\eg}{e.g.}

\newcommand{\ccSysName}[0]{DITING\xspace}
\newcommand{\ccBenchName}[0]{PartitioningE-Bench\xspace}
\newcommand{\ccTitle}[0]{\ccSysName: A Static Analyzer for Identifying Bad Partitioning Issues in TEE Applications}

\newcommand{\whiteding}[1]{\ding{\numexpr171+#1\relax}}

\lstset{
  language=C++,
  breaklines,
  backgroundcolor=\color[RGB]{245,245,244},
  % extendedchars=true,
  basicstyle=\footnotesize\ttfamily,
  frame=single,
  showstringspaces=false,
  showspaces=false,
  numbers=left,
  numberstyle=\footnotesize\ttfamily,
  numbersep=5pt,
  tabsize=2,
  breaklines=true,
  showtabs=false,
  keywordstyle = \color[RGB]{40,40,255}\bfseries,
  commentstyle = \color[RGB]{0,96,96}\rmfamily\itshape,
  stringstyle  = \color{red}\ttfamily,
  xleftmargin=0.45cm,
  xrightmargin=0.5em,
  captionpos=b,
  aboveskip=15pt,
  belowskip=-1pt
}
