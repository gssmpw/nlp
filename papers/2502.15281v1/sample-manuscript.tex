%%
%% This is file `sample-manuscript.tex',
%% generated with the docstrip utility.
%%
%% The original source files were:
%%
%% samples.dtx  (with options: `all,proceedings,bibtex,manuscript')
%% 
%% IMPORTANT NOTICE:
%% 
%% For the copyright see the source file.
%% 
%% Any modified versions of this file must be renamed
%% with new filenames distinct from sample-manuscript.tex.
%% 
%% For distribution of the original source see the terms
%% for copying and modification in the file samples.dtx.
%% 
%% This generated file may be distributed as long as the
%% original source files, as listed above, are part of the
%% same distribution. (The sources need not necessarily be
%% in the same archive or directory.)
%%
%%
%% Commands for TeXCount
%TC:macro \cite [option:text,text]
%TC:macro \citep [option:text,text]
%TC:macro \citet [option:text,text]
%TC:envir table 0 1
%TC:envir table* 0 1
%TC:envir tabular [ignore] word
%TC:envir displaymath 0 word
%TC:envir math 0 word
%TC:envir comment 0 0
%%
%% The first command in your LaTeX source must be the \documentclass
%% command.
%%
%% For submission and review of your manuscript please change the
%% command to \documentclass[manuscript, screen, review]{acmart}.
%%
%% When submitting camera ready or to TAPS, please change the command
%% to \documentclass[sigconf]{acmart} or whichever template is required
%% for your publication.
%%
%%
\documentclass[acmsmall,screen]{acmart}
%%
%% \BibTeX command to typeset BibTeX logo in the docs
\AtBeginDocument{%
  \providecommand\BibTeX{{%
    Bib\TeX}}}

%% Rights management information.  This information is sent to you
%% when you complete the rights form.  These commands have SAMPLE
%% values in them; it is your responsibility as an author to replace
%% the commands and values with those provided to you when you
%% complete the rights form.
\setcopyright{acmlicensed}
\copyrightyear{2025}
\acmYear{2025}
\acmDOI{XXXXXXX.XXXXXXX}
\acmJournal{TOSEM}
%% These commands are for a PROCEEDINGS abstract or paper.
% \acmConference[Conference acronym 'XX]{Make sure to enter the correct
%   conference title from your rights confirmation email}{June 03--05,
%   2018}{Woodstock, NY}
%%
%%  Uncomment \acmBooktitle if the title of the proceedings is different
%%  from ``Proceedings of ...''!
%%
%%\acmBooktitle{Woodstock '18: ACM Symposium on Neural Gaze Detection,
%%  June 03--05, 2018, Woodstock, NY}
\acmISBN{978-1-4503-XXXX-X/2018/06}


%%
%% Submission ID.
%% Use this when submitting an article to a sponsored event. You'll
%% receive a unique submission ID from the organizers
%% of the event, and this ID should be used as the parameter to this command.
%%\acmSubmissionID{123-A56-BU3}

%%
%% For managing citations, it is recommended to use bibliography
%% files in BibTeX format.
%%
%% You can then either use BibTeX with the ACM-Reference-Format style,
%% or BibLaTeX with the acmnumeric or acmauthoryear sytles, that include
%% support for advanced citation of software artefact from the
%% biblatex-software package, also separately available on CTAN.
%%
%% Look at the sample-*-biblatex.tex files for templates showcasing
%% the biblatex styles.
%%

%%
%% The majority of ACM publications use numbered citations and
%% references.  The command \citestyle{authoryear} switches to the
%% "author year" style.
%%
%% If you are preparing content for an event
%% sponsored by ACM SIGGRAPH, you must use the "author year" style of
%% citations and references.
%% Uncommenting
%% the next command will enable that style.
%%\citestyle{acmauthoryear}

\usepackage[american]{babel}
\usepackage{natbib}
    \bibliographystyle{plainnat}
    \renewcommand{\bibsection}{\subsubsection*{References}}
\usepackage{mathtools}
\usepackage{booktabs}
\usepackage{tikz}

% \usepackage[utf8]{inputenc} % allow utf-8 input
\usepackage[T1]{fontenc}    % use 8-bit T1 fonts
\usepackage{hyperref}       % hyperlinks
\usepackage{url}            % simple URL typesetting
\usepackage{amsfonts}       % blackboard math symbols
\usepackage{nicefrac}       % compact symbols for 1/2, etc.
\usepackage{microtype}      % microtypography
\usepackage{svg}
\usepackage{amsmath}
\usepackage{multirow}
\usepackage{centernot}
\usepackage{amsthm}
\usepackage{thmtools}
\usepackage{thm-restate}
\usepackage{cleveref}

\newtheorem{theorem}{Theorem}[section]
\newtheorem{corollary}{Corollary}[section]
\newtheorem{lemma}{Lemma}[section]
\newtheorem{proposition}{Proposition}[section]
\newtheorem{definition}{Definition}[section]
\newtheorem{example}{Example}[section]
\newenvironment{proofsketch}{%
  \renewcommand{\proofname}{Proof Sketch}\proof}{\endproof}

% Algorithms
\usepackage{algorithmicx}
\usepackage[noend]{algpseudocode}
\usepackage{algorithm}
% \usepackage[linesnumbered]{algorithm2e}

\usepackage{comment}

\usepackage{cases}

\usepackage{xcolor}
\usepackage{hyperref}
% \usepackage{graphicx}
\usepackage{subcaption}
\usepackage{paralist}
\usepackage{comment}

% CI
\newcommand{\indep}{\perp\kern-6pt\perp}
\newcommand{\dep}{\centernot{\perp\kern-6pt\perp}}
% Orientations
\newcommand{\starleft}{*\kern-5pt}
\newcommand{\starright}{\kern-5pt*}

\usepackage{acronym}
\acrodef{AID}{adjustment identification distance}
\acrodef{ATE}{average treatment effect}
\acrodef{CI}{conditional independence}
\acrodef{CPDAG}{complete partially directed acyclic graph}
\acrodef{DAG}{directed acyclic graph}
\acrodef{KCI}{Kernel-based conditional independence}
\acrodef{LCD}{local causal discovery}
\acrodef{MB}{Markov blanket}
\acrodef{MEC}{Markov equivalence class}
\acrodef{SHD}{structural Hamming distance}
\acrodef{SNAP}{Sequential Non-Ancestor Pruning}

%%
%% end of the preamble, start of the body of the document source.
\begin{document}

%%
%% The "title" command has an optional parameter,
%% allowing the author to define a "short title" to be used in page headers.
\title{\ccTitle}

%%
%% The "author" command and its associated commands are used to define
%% the authors and their affiliations.
%% Of note is the shared affiliation of the first two authors, and the
%% "authornote" and "authornotemark" commands
%% used to denote shared contribution to the research.
\author{Chengyan Ma}
% \authornote{Both authors contributed equally to this research.}
\email{chengyanma@smu.edu.sg}
\orcid{0000-0001-9256-6930}
% \author{G.K.M. Tobin}
\author{Ruidong Han}
\email{rdhan@smu.edu.sg}
% \authornotemark[1]
% \email{webmaster@marysville-ohio.com}
\author{Ye Liu}
\email{yeliu@smu.edu.sg}
\author{Yuqing Niu}
\email{yuqingniu@smu.edu.sg}
\affiliation{%
  \institution{Singapore Management University}
  % \city{Dublin}
  % \state{Ohio}
  \country{Singapore}
}

% \author{Ruidong Han}
% \affiliation{%
%   \institution{Singapore Management University}
%   % \city{Hekla}
%   \country{Singapore}}
% \email{rdhan@smu.edu.sg}

% \author{Ye Liu}
% \affiliation{%
%   \institution{Singapore Management University}
%   % \city{Rocquencourt}
%   \country{Singapore}
% }
% \email{yeliu@smu.edu.sg}

% \author{Yuqing Niu}
% \affiliation{%
%  \institution{Singapore Management University}
%  % \city{Doimukh}
%  % \state{Arunachal Pradesh}
%  \country{Singapore}}
% \email{yuqingniu@smu.edu.sg}

\author{Di Lu}
\email{dlu@xidian.edu.cn}
\author{Chuang Tian}
\email{tianchuang@xidian.edu.cn}
\author{Jianfeng Ma}
\email{jfma@mail.xidian.edu.cn}
\affiliation{%
  \institution{Xidian University}
  \city{Xi'an}
  \state{Shaanxi}
  \country{China}}

% \author{Chuang Tian}
% \affiliation{%
%   \institution{Xidian University}
%   \city{Xi'an}
%   \state{Shaanxi}
%   \country{China}}
% \email{tianchuang@xidian.edu.cn}

% \author{Jianfeng Ma}
% \affiliation{%
%   \institution{Xidian University}
%   \city{Xi'an}
%   \state{Shaanxi}
%   \country{China}}
% \email{jfma@mail.xidian.edu.cn}

\author{Debin Gao}
\email{dbgao@smu.edu.sg}
\author{David Lo}
\email{davidlo@smu.edu.sg}
\affiliation{%
  \institution{Singapore Management University}
  % \city{San Antonio}
  % \state{Texas}
  \country{Singapore}}

% \author{David Lo}
% \affiliation{%
%   \institution{Singapore Management University}
%   % \city{San Antonio}
%   % \state{Texas}
%   \country{Singapore}}
% \email{davidlo@smu.edu.sg}

% \author{Julius P. Kumquat}
% \affiliation{%
%   \institution{The Kumquat Consortium}
%   \city{New York}
%   \country{USA}}
% \email{jpkumquat@consortium.net}

%%
%% By default, the full list of authors will be used in the page
%% headers. Often, this list is too long, and will overlap
%% other information printed in the page headers. This command allows
%% the author to define a more concise list
%% of authors' names for this purpose.
\renewcommand{\shortauthors}{C. Ma et al.}

%%
%% The abstract is a short summary of the work to be presented in the
%% article.
\begin{abstract}
  Trusted Execution Environment (TEE) enhances the security of mobile applications and cloud services by isolating sensitive code in the secure world from the non-secure normal world. However, TEE applications are still confronted with vulnerabilities stemming from bad partitioning. Bad partitioning can lead to critical security problems of TEE, such as leaking sensitive data to the normal world or being adversely affected by malicious inputs from the normal world.

  To address this, we propose an approach to detect partitioning issues in TEE applications. First, we conducted a survey of TEE vulnerabilities caused by bad partitioning and found that the parameters exchanged between the secure and normal worlds often contain insecure usage with bad partitioning implementation. Second, we developed a tool named \ccSysName that can analyze data-flows of these parameters and identify their violations of security rules we defined to find bad partitioning issues. Different from existing research that only focuses on malicious input to TEE, we assess the partitioning issues more comprehensively through input/output and shared memory. Finally, we created the first benchmark targeting bad partitioning, consisting of 110 test cases. Experiments demonstrate the \ccSysName achieves an F1 score of 0.90 in identifying bad partitioning issues.
\end{abstract}

%%
%% The code below is generated by the tool at http://dl.acm.org/ccs.cfm.
%% Please copy and paste the code instead of the example below.
%%
\begin{CCSXML}
<ccs2012>
   <concept>
       <concept_id>10002978.10003022</concept_id>
       <concept_desc>Security and privacy~Software and application security</concept_desc>
       <concept_significance>500</concept_significance>
       </concept>
   <concept>
       <concept_id>10002978.10003022.10003023</concept_id>
       <concept_desc>Security and privacy~Software security engineering</concept_desc>
       <concept_significance>300</concept_significance>
       </concept>
   <concept>
       <concept_id>10011007.10011074.10011111.10011113</concept_id>
       <concept_desc>Software and its engineering~Software evolution</concept_desc>
       <concept_significance>500</concept_significance>
       </concept>
 </ccs2012>
\end{CCSXML}

\ccsdesc[500]{Security and privacy~Software and application security}
\ccsdesc[300]{Security and privacy~Software security engineering}
\ccsdesc[500]{Software and its engineering~Software evolution}

%%
%% Keywords. The author(s) should pick words that accurately describe
%% the work being presented. Separate the keywords with commas.
\keywords{Trusted Execution Environment, bad partitioning, static analysis, data-flow, shared memory}

% \received{20 February 2007}
% \received[revised]{12 March 2009}
% \received[accepted]{5 June 2009}

%%
%% This command processes the author and affiliation and title
%% information and builds the first part of the formatted document.
\maketitle

\section{Introduction}


\begin{figure}[t]
\centering
\includegraphics[width=0.6\columnwidth]{figures/evaluation_desiderata_V5.pdf}
\vspace{-0.5cm}
\caption{\systemName is a platform for conducting realistic evaluations of code LLMs, collecting human preferences of coding models with real users, real tasks, and in realistic environments, aimed at addressing the limitations of existing evaluations.
}
\label{fig:motivation}
\end{figure}

\begin{figure*}[t]
\centering
\includegraphics[width=\textwidth]{figures/system_design_v2.png}
\caption{We introduce \systemName, a VSCode extension to collect human preferences of code directly in a developer's IDE. \systemName enables developers to use code completions from various models. The system comprises a) the interface in the user's IDE which presents paired completions to users (left), b) a sampling strategy that picks model pairs to reduce latency (right, top), and c) a prompting scheme that allows diverse LLMs to perform code completions with high fidelity.
Users can select between the top completion (green box) using \texttt{tab} or the bottom completion (blue box) using \texttt{shift+tab}.}
\label{fig:overview}
\end{figure*}

As model capabilities improve, large language models (LLMs) are increasingly integrated into user environments and workflows.
For example, software developers code with AI in integrated developer environments (IDEs)~\citep{peng2023impact}, doctors rely on notes generated through ambient listening~\citep{oberst2024science}, and lawyers consider case evidence identified by electronic discovery systems~\citep{yang2024beyond}.
Increasing deployment of models in productivity tools demands evaluation that more closely reflects real-world circumstances~\citep{hutchinson2022evaluation, saxon2024benchmarks, kapoor2024ai}.
While newer benchmarks and live platforms incorporate human feedback to capture real-world usage, they almost exclusively focus on evaluating LLMs in chat conversations~\citep{zheng2023judging,dubois2023alpacafarm,chiang2024chatbot, kirk2024the}.
Model evaluation must move beyond chat-based interactions and into specialized user environments.



 

In this work, we focus on evaluating LLM-based coding assistants. 
Despite the popularity of these tools---millions of developers use Github Copilot~\citep{Copilot}---existing
evaluations of the coding capabilities of new models exhibit multiple limitations (Figure~\ref{fig:motivation}, bottom).
Traditional ML benchmarks evaluate LLM capabilities by measuring how well a model can complete static, interview-style coding tasks~\citep{chen2021evaluating,austin2021program,jain2024livecodebench, white2024livebench} and lack \emph{real users}. 
User studies recruit real users to evaluate the effectiveness of LLMs as coding assistants, but are often limited to simple programming tasks as opposed to \emph{real tasks}~\citep{vaithilingam2022expectation,ross2023programmer, mozannar2024realhumaneval}.
Recent efforts to collect human feedback such as Chatbot Arena~\citep{chiang2024chatbot} are still removed from a \emph{realistic environment}, resulting in users and data that deviate from typical software development processes.
We introduce \systemName to address these limitations (Figure~\ref{fig:motivation}, top), and we describe our three main contributions below.


\textbf{We deploy \systemName in-the-wild to collect human preferences on code.} 
\systemName is a Visual Studio Code extension, collecting preferences directly in a developer's IDE within their actual workflow (Figure~\ref{fig:overview}).
\systemName provides developers with code completions, akin to the type of support provided by Github Copilot~\citep{Copilot}. 
Over the past 3 months, \systemName has served over~\completions suggestions from 10 state-of-the-art LLMs, 
gathering \sampleCount~votes from \userCount~users.
To collect user preferences,
\systemName presents a novel interface that shows users paired code completions from two different LLMs, which are determined based on a sampling strategy that aims to 
mitigate latency while preserving coverage across model comparisons.
Additionally, we devise a prompting scheme that allows a diverse set of models to perform code completions with high fidelity.
See Section~\ref{sec:system} and Section~\ref{sec:deployment} for details about system design and deployment respectively.



\textbf{We construct a leaderboard of user preferences and find notable differences from existing static benchmarks and human preference leaderboards.}
In general, we observe that smaller models seem to overperform in static benchmarks compared to our leaderboard, while performance among larger models is mixed (Section~\ref{sec:leaderboard_calculation}).
We attribute these differences to the fact that \systemName is exposed to users and tasks that differ drastically from code evaluations in the past. 
Our data spans 103 programming languages and 24 natural languages as well as a variety of real-world applications and code structures, while static benchmarks tend to focus on a specific programming and natural language and task (e.g. coding competition problems).
Additionally, while all of \systemName interactions contain code contexts and the majority involve infilling tasks, a much smaller fraction of Chatbot Arena's coding tasks contain code context, with infilling tasks appearing even more rarely. 
We analyze our data in depth in Section~\ref{subsec:comparison}.



\textbf{We derive new insights into user preferences of code by analyzing \systemName's diverse and distinct data distribution.}
We compare user preferences across different stratifications of input data (e.g., common versus rare languages) and observe which affect observed preferences most (Section~\ref{sec:analysis}).
For example, while user preferences stay relatively consistent across various programming languages, they differ drastically between different task categories (e.g. frontend/backend versus algorithm design).
We also observe variations in user preference due to different features related to code structure 
(e.g., context length and completion patterns).
We open-source \systemName and release a curated subset of code contexts.
Altogether, our results highlight the necessity of model evaluation in realistic and domain-specific settings.





\section{Background and Motivation} \label{s:bg}
\subsection{TEE Data Interaction} \label{s:params}
As shown in Fig.~\ref{fig:datacom}, the normal world and TEE are two independent environments, separated to ensure the security of sensitive functions and data. In this architecture, the communication between TEE and the normal world involves three types of parameters: input, output, and shared memory~\cite{s20041090}.

\begin{figure}[t]
    \centering
    \includegraphics[width=0.5\linewidth]{figures/Fig_2.drawio.pdf}
    \caption{Data communication between TEE and the normal world.}
    \label{fig:datacom}
\end{figure}

Input parameters are used to transfer data from the normal side to TEE, while outputs handle the results or send processed data back. Inputs and outputs are simple mechanisms that allow users to temporarily transfer small amounts of data between the normal world and TEE. However, temporary inputs/outputs realize data transfer through memory copying, slightly lowering the performance of TEE applications due to additional memory copy. They are mainly used to transmit lightweight data, such as user commands and data for cryptographic operations.

Shared memory provides a zero-copy memory block to exchange larger data sets (\eg, multimedia files or bulk data) between two sides~\cite{optee}. It allows both sides to access the same memory space efficiently, which avoids frequent memory copying. Shared memory can also remain valid across different TEE invocation sessions, making it suitable for scenarios that require data to be reused multiple times.

Moreover, an SDK is responsible for managing these parameters and communication in the normal side. For example, TrustZone-based OP-TEE uses \texttt{TEEC\_InvokeCommand()} function to execute the in-TEE code, enabling the shared memory or temporary buffers to transfer data between the two sides~\cite{8684292}. 
Then, OP-TEE can handle these interactions through its internal APIs, which process incoming requests, perform secure computations, and return responses to the normal world.
Similarly, Intel SGX utilizes the ECALL and OCALL interfaces to achieve these functionalities~\cite{10632129}. ECALLs allow the normal world to securely invoke functions within the SGX enclave, passing data into the trusted environment for processing, while OCALLs enable the enclave to request services or share results with the untrusted normal world.

Table~\ref{tbl:comp_params} illustrates some differences between input/output parameters and shared memory. It is important to note that, while the normal side cannot directly access the memory copies of input and output parameters in TEE, an attacker with the permissions of the normal world can still tamper with the inputs before they are transmitted to TEE or intercept and read the outputs after they are returned from TEE~\cite{9925569, 10477533}.
Additionally, since shared memory relies on address-based data transfer between the two sides, any modifications made to the data on one side will be instantly mirrored on the other opposing side.
Therefore, data interactions controlled by the normal world code are the root cause of the bad partitioning issues in TEE applications.

\begin{table}[t]
    \caption{Comparison of input/output parameters and shared memory.}
    \label{tbl:comp_params}
    % \renewcommand{\arraystretch}{1.3}
    % \footnotesize
    \setlength{\tabcolsep}{3mm}
    \centering
	\begin{tabular}{lp{4.5cm}p{4.5cm}}
		\toprule
		\textbf{Feature} & \textbf{Input/Output} & \textbf{Shared Memory} \\
		\midrule
            Data Size & Small & Big  \\
            Efficiency & Low, memory copying required & High, no need of additional memory copy\\
            Life-time & Temporary, valid for a single TEE invocation & Valid for a long time and shared among several TEE invocations \\
            Privilege & The normal world cannot reach memory copy in TEE & Both the normal world and TEE can access at the same time \\
		\bottomrule
	\end{tabular}
\end{table}

\subsection{Bad Partitioning in TEE Projects} \label{s:bp}
By analyzing 10 TEE projects from \textit{github.com}, we identified three primary types of bad partitioning issues: \whiteding{1} unencrypted data output, \whiteding{2} input validation weaknesses, and \whiteding{3} the direct usage of shared memory. We list some projects and issues they face in Table~\ref{tbl:project_list}, and discuss each of these issues in detail below.
In this paper, we use projects developed for ARM TrustZone as examples. The issues we identify and the proposed solutions work for other platforms as well.

% \begin{table*}[t]
%     \caption{Surveys of bad partitioning issues in real-world TEE projects.}
%     \label{tbl:project_list}
%     % \renewcommand{\arraystretch}{1.3}
%     % \footnotesize
%     \setlength{\tabcolsep}{1mm}
%     \resizebox{\linewidth}{!}{
%     \centering
% 	\begin{tabular}{ccccccc}
% 		\toprule
% 		\textbf{Bad Partitioning Issues} & \textbf{optee-sdp} & \textbf{optee-fiovb} & \textbf{basicAlg\_use} & \textbf{external\_rk\_tee\_user} & \textbf{secvideo\_demo} & \textbf{darknetz} \\
% 		\midrule
%             Unencrypted Data Output & \ding{108} & \ding{108} & \ding{108} & \ding{108} & & \\
%             Input Validation Weaknesses & \ding{108} & \ding{108} & \ding{108} & \ding{108} & & \\
%             Direct Usage of Shared Memory & & & & \ding{108} & \ding{108} & \ding{108} \\
% 		\bottomrule
% 	\end{tabular}
%     }
% \end{table*}

\begin{table*}[t]
    \caption{Surveys of bad partitioning issues in real-world TEE projects. \CIRCLE~means that the project contains the corresponding issue, and \Circle~indicates not containing the issue.}
    \label{tbl:project_list}
    % \renewcommand{\arraystretch}{1.3}
    % \footnotesize
    % \setlength{\tabcolsep}{2.5mm}
    % \resizebox{\linewidth}{!}{
    \centering
	\begin{tabular}{lp{2.7cm}<{\centering}p{2.7cm}<{\centering}p{2.7cm}<{\centering}}
		\toprule
            \multirow{2}{*}{\textbf{Projects}} & Unencrypted Data Output & Input Validation Weaknesses & Direct Usage of Shared Memory \\
		\midrule
            optee-sdp & \CIRCLE & \CIRCLE & \Circle \\
            optee-fiovb & \CIRCLE & \CIRCLE & \Circle\\
            basicAlg\_use & \CIRCLE & \CIRCLE & \Circle\\
            external\_rk\_tee\_user & \CIRCLE & \CIRCLE & \CIRCLE \\
            secvideo\_demo & \Circle & \Circle & \CIRCLE \\
            darknetz & \Circle & \Circle & \CIRCLE \\
            acipher & \CIRCLE & \Circle & \Circle \\
            Lenet5\_in\_OPTEE & \CIRCLE & \CIRCLE & \Circle \\
            hotp & \CIRCLE & \Circle & \Circle \\
            random & \CIRCLE & \Circle & \Circle \\
		\bottomrule
	\end{tabular}
    % }
\end{table*}

\subsubsection{\textbf{Unencrypted Data Output}} \label{s:udo}
Since a TEE application can transfer data from TEE to the normal side using output parameters or shared memory, any unencrypted data is at risk of exposure to attackers during the output process, potentially compromising the confidentiality and security of sensitive data.

\begin{figure}[t]
  \centering
  \begin{subfigure}[b]{\linewidth}
    \begin{lstlisting}[language=c++]
char dump[MAX_DUMP_SIZE];
op.params[0].tmpref.buffer = (void *)dump;
op.params[0].tmpref.size = MAX_DUMP_SIZE - 1;
res = TEEC_InvokeCommand(&session, TA_SDP_DUMP_STATUS, &op, &err_origin);
    \end{lstlisting}
    \caption{The array \texttt{dump} is an output parameter in normal-world Code.}
  \end{subfigure}
  \hfill
  \begin{subfigure}[b]{\linewidth}
        \begin{lstlisting}[language=c++]
// params[0].memref is a memory copy of op.params[0].tmpref in the normal world
snprintf(params[0].memref.buffer, params[0].memref.size, "delta (decoder) refcount %d\n", delta_refcount);
// params[0].memref will be copied to op.params[0].tmpref after return
return TEE_SUCCESS;
    \end{lstlisting}
    \caption{TEE data \texttt{delta\_refcount} is directly copied to the output buffer.}
  \end{subfigure}
  \caption{The example of unencrypted data output from optee-sdp.}
  \label{fig:bp1code}
\end{figure}

\begin{figure}[t]
  \centering
  \begin{subfigure}[b]{\linewidth}
    \begin{lstlisting}[language=c++]
char g_AesOutpUT[256] = {0};
// op.params[1].tmpref.buffer is an input buffer larger than test buffer in TEE
op.params[1].tmpref.buffer = g_AesOutpUT;
res = l_CryptoVerifyCa_SendCommand(&op, &session, CMD_AES_OPER);
    \end{lstlisting}
    \caption{The input buffer and its size are defined outside TEE.}
  \end{subfigure}
  \hfill
  \begin{subfigure}[b]{\linewidth}
        \begin{lstlisting}[language=c++]
char test[] = {0x01, 0x02, 0x03, 0x04, 0x05};
// params[1].memref.buffer is the copy of op.params[1].tmpref.buffer in the normal world
TEE_MemMove(params[1].memref.buffer, test, sizeof(test));
    \end{lstlisting}
    \caption{\texttt{TEE\_MemMove} is called without verifying the size of input buffer.}
  \end{subfigure}
  \caption{The example of input validation weaknesses from basicAlg\_use.}
  \label{fig:bp2code}
\end{figure}

\begin{figure}[t]
  \centering
  \begin{subfigure}[b]{\linewidth}
    \begin{lstlisting}[language=c++]
unsigned char membuf_input[] = "from_CA_to_TA";
// definition of shared memory
TEEC_SharedMemory shm = {
    .size = sizeof(membuf_input),
};
TEEC_AllocateSharedMemory(&ctx, &shm);
// shared memory can be changed outside TEE
memcpy(shm.buffer, membuf_input, shm.size);
op.params[2].memref.parent = &shm;
res = TEEC_InvokeCommand(&sess, TA_TEST_APP_FILL_MEM_BUF, &op, &err_origin);
    \end{lstlisting}
    \caption{Shared memory is managed in the normal world.}
  \end{subfigure}
  \hfill
  \begin{subfigure}[b]{\linewidth}
        \begin{lstlisting}[language=c++]
// params[2].memref.buffer is shared memory
void *buf = params[2].memref.buffer;
unsigned int sz = params[2].memref.size;
// params[2].memref.buffer may have been changed outside TEE
if (!TEE_MemCompare(buf, "from_CA_to_TA", sz)) {
    IMSG("membuf test : Pass!\n");
}
    \end{lstlisting}
    \caption{\texttt{TEE\_MemCompare} is invoked without checking the buffer contents.}
  \end{subfigure}
  \caption{The example of the direct usage of shared memory from external\_rk\_tee\_user.}
  \label{fig:bp3code}
\end{figure}

Fig.~\ref{fig:bp1code} shows a scenario where if the TEE data is copied to an output buffer without being encrypted, then the attacker can directly obtain the plaintext from \texttt{dump} in the normal world. Therefore, implementing proper security measures is essential to prevent unauthorized access or data leakage during the TEE output.

\subsubsection{\textbf{Input Validation Weaknesses}} \label{s:ivw}
As mentioned in Section~\ref{s:params}, an attacker with normal-world permissions has the ability to tamper with any data intended for input into TEE. Fig.~\ref{fig:bp2code} illustrates that even though the normal-world code defines a much larger buffer than \texttt{text} buffer used in TEE, it is still susceptible to buffer overflow attacks that in-TEE code directly performs memory copy operations in the TEE code without validating the size of the input buffer \texttt{params[1].memref.buffer}. This occurs because the TEE code may incorrectly assume that the input buffer is secure and reliable without explicitly checking its size. If an attacker maliciously alters the input buffer size, memory copy operations within TEE could exceed the boundaries of the inputs, leading to severe consequences such as crashes and data corruption. Similar precautions should be taken when using input values as indices for array access in TEE. Moreover, checking the input value treated as the \textit{size} argument of the memory request function \texttt{TEE\_Malloc(}\textit{size}\texttt{)} can effectively reduce the probability of memory allocation failure.

To mitigate this risk, it is necessary for the TEE code to validate all inputs before performing any memory operations, regardless of whether input checking is partitioned to the normal side. This essential validation ensures that any tampering or unexpected changes to the input data in the normal world do not compromise the security of TEE.

\subsubsection{\textbf{Direct Usage of Shared Memory}} \label{s:dusm}
Shared memory, as a zero-copy memory block, provides the same physical memory space accessible to both the secure and non-secure partitions. Since this memory is maintained by the normal-world code, bad partitioning can introduce significant vulnerabilities. If the TEE code directly uses shared memory without validating its contents, an attacker could manipulate the shared memory data and influence the execution of TEE applications.

Fig.~\ref{fig:bp3code} shows that variable \texttt{buf} is initialized by a shallow copy of the shared memory. Therefore, the contents of \texttt{buf} can be directly modified from outside TEE. When \texttt{buf} is passed into \texttt{TEE\_MemCompare} function, the TEE application may not return the expected results. In order to solve this problem, shared memory should be deep copied to the in-TEE buffer, ensuring that the data used in TEE is isolated from external manipulations. 
% Additionally, integrity validate should be performed to protect the consistency of the shared memory data.

\section{The \search\ Search Algorithm}
\label{sec:search}

%In traditional ML, structure changes and step (operator) changes are performed before model training, \ie, fixed to the training process, and weights are updated with SGD, because weights are continous, differentiable values, and there are significantly more weights than structure and operator changes. In workflow autotuning, all three types of cogs can be chosen with a unified search-based approach, because all of them are non-differentiable configurations and the number of cogs in different types are all small.
%Thus, \sysname\ only needs to navigate the search space of combination of cogs as the search space to produce its workflow optimization results.

%We propose, \textit{\textbf{\search}}, an adaptive hierarchical search algorithm that autotunes gen-AI workflows based on observed end-to-end workflow results. In each search iteration, \search\ selects a combination of cogs to apply to the workflow and executes the resulting workflow with user-provided training inputs. \search\ evaluates the final generation quality using the user-specified evaluator and measures the execution time and cost for each training input. These results are aggregated and serve as BO observations and pruning criteria.
%the optimizer can condition on and propose better configurations in later trials. The optimizer will also be informed about the violation of any user-specified metric thresholds. More details of this mechanism can be found in Appendix ~\ref{appdx:TPE}.

With our insights in Section~\ref{sec:theory}, we believe that search methods based on Bayesian Optimizer (BO) can work for all types of cogs in gen-AI workflow autotuning because of BO's efficiency in searching discrete search space.
A key challenge in designing a BO-based search is the limited search budgets that need to be used to search a high-dimensional cog space. 
For example, for 4 cogs each with 4 options and a workflow of 3 LLM steps, the search space is $4^{12}$. Suppose each search uses GPT-4o and has 1000 output tokens, the entire space needs around \$168K to go through. A user search budget of \$100 can cover only 0.06\% of the search space. A traditional BO approach cannot find good results with such small budgets.
%The entire search space grows exponentially with the number of cogs and the number of steps in a workflow. Moreover, different cogs and different combinations of cogs can have varying impacts on different workflows. 
%Without prior knowledge, it is difficult to determine the amount of budget to give to each cog.

To confront this challenge, we propose \textit{\textbf{\search}}, an adaptive hierarchical search algorithm that efficiently assigns search budget across cogs based on budget size and observed workflow evaluation results, as defined in Algorithms~\ref{alg:main} and \ref{alg:outer} and described below.
%autotunes gen-AI workflows based on observed end-to-end workflow results.
%\search\ includes a search layer partitioning method, a search budget initial assignment method, an evaluation-guided budget re-allocation mechanism, and a convergence-based early-exiting strategy. We discuss them in details below.

%\zijian{\search\ allows users to specify the optimization budget allowed in terms of the maximum number of search iterations. Based on the relationship between the complexity of the search space and the available budget, we will separate all tunable parameters into different layers each optimized by independent Bayesian optimization routines. Then we will decide the maximum budget each layer can get with a bottom-up partition strategy. Besides search space and resource partition, we also employ a novel allocation algorithm that integrates successive halving~\cite{successivehalving} and a convergence-based early exiting strategy to facilitate efficient usage of assigned budget.}


% The outermost layer searches and selects structures for a workflow; the middle layer searches and selects step options under the workflow structure selected in the outermost layer; the innermost layer searches and selects weights with the given workflow structure and steps. 

\begin{algorithm}[h]
    \caption{\search\ Algorithm}
    \label{alg:main}
      \small
\begin{algorithmic}[1]
\STATE \textbf{Global Value:} $R = \emptyset$ \COMMENT{Global result set}
%\STATE \textbf{Global Value:} $F = \emptyset$ \COMMENT{Global observation set}

%Reduct factor $\eta > 1$, explore width $W$
\STATE \textbf{Input:} User-specified Total Budget $TB$
\STATE \textbf{Input:} Cog set $C = \{c_{11},c_{12},...\}, \{c_{21},c_{22},...\}, \{c_{31},c_{32},...\}$

    \STATE
%\FOR{$i = 1,2,3$}
    %\COMMENT{$\alpha$ is a configurable value default to 1.1}
%\ENDFOR
%\STATE
%    \STATE \{$B_1,B_2,B_3$\} = LayerPartition($C$) \COMMENT{Calculate ideal layer budget}
    %\STATE \textbf{Glob}.budgets = budgets
%    \STATE opt\_layers = init\_opt\_routines() \COMMENT{A list of optimize routine each layer will use for search}
%\STATE
%    \FOR{$i \in L, \dots, 1$}
%        \IF{$i == L$}
 %           \STATE opt\_layers[L] = InnerLayerOpt
  %      \ELSE
   %         \STATE opt\_layers[i] = OuterLayerOpt
            %\STATE opt\_layers[i].next\_layer\_budgets = B[i+1]
            %\STATE opt\_layers[i].next\_layer\_routine = opt\_layers[i+1]
    %    \ENDIF
    %\ENDFOR
%\STATE opt\_layers[1].invoke($\emptyset$, B[1])
\STATE $U = 0$ \COMMENT{Used budget so far, initialize to 0}

\STATE \COMMENT{Perform search with 1 to 3 layers until budget runs out}
\FOR{$L = 1,2,3$} 
        \IF{$L=1$}
            \STATE $C_1 = C_1 \cup C_2 \cup C_3$ \COMMENT{Merge all cogs into a single layer}
        \ENDIF
        \IF{$L==2$}
            \STATE $C_1 = C_1 \cup C_2$ \COMMENT{Merge step and weight cogs}
            \STATE $C_2 = C_3$ \COMMENT{Architecture cog becomes the second layer}
        \ENDIF
        \STATE
    \FOR{$i = 1,..,L$}
    \STATE $NC_i = |C_i|$ \COMMENT{Total number of cogs in layer $L$} 
%    NO_i &= \sum_{L} \{\text{number of possible options in cog } c_{ij}\} \\
    \STATE $S_i = NC_i^\alpha$ \COMMENT{Estimated expected search size in layer $i$}
    \ENDFOR
    \STATE $E_L = \prod\limits_{i=1}^{L}S_i$ \COMMENT{Expected total search size in the current round}
    \STATE $E = TB - U > E_L$ ? $E_L$ : $(TB - U)$ \COMMENT{Consider insufficient budget} 
    \IF{$L==3$ and $(TB - U)$ > $E_L$}
         \STATE $E = TB - U$ \COMMENT{Spend all remaining budget if at 3 layer}
    \ENDIF
    %\STATE$TL = |N|$ \COMMENT{number of layers}
    \FOR{$i = 1,..,L$}
        \STATE $B_i =  \lfloor S_i \times \sqrt[L]{\frac{E}{E_L}}\rfloor$
        %$B$ = BudgetAssign($N$, $TL$, $TB$)
        \COMMENT{Assign budget proportionally to $S_i$}
    \ENDFOR
    \STATE
\STATE \texttt{LayerSearch} ($\emptyset$, $B$, $L$, $B_L$) \COMMENT{Hierarchical search from layer $L$}
\STATE
\STATE $U = U + E$
\IF{$U \geq TB$}
\STATE break \COMMENT{Stop search when using up all user budget}
\ENDIF
\ENDFOR
%\STATE
%\STATE $O$ = \texttt{SelectBestConfigs} ($R$)
%\IF{$L == 1$}
%    \STATE InnerLayerOpt($\emptyset$, B[1])
%\ELSE
%    \STATE OuterLayerOpt($\emptyset$, B[1], 1)
%\ENDIF
\STATE
\STATE \textbf{Output:} $O$ = \texttt{SelectBestConfigs} ($R$) \COMMENT{Return best optimizations}
\end{algorithmic}
\end{algorithm}

\subsection{Hierarchical Layer and Budget Partition}
\label{sec:ssp}

%We motivate \search's adaptive hierarchical search 
A non-hierarchical search has all cog options in a single-layer search space for an optimizer like BO to search, an approach taken by prior workflow optimizers~\cite{dspy-2-2024,gptswarm}.
With small budgets, a single-layer hierarchy allows BO-like search to spend the budget on dimensions that could potentially generate some improvements.
%While given enough budget, the single-layer space can be extensively searched to find global optimal, with little budget, 
However, a major issue with a single-layer search space is that a search algorithm like BO can be stuck at a local optimum even when budgets increase.
% (unless the budget is close to covering a very large space across dimensions).
To mitigate this issue, our idea is to perform a hierarchical search that works by choosing configurations in the outermost layer first, then under each chosen configuration, choosing the next layer's configurations until the innermost layer. 
With such a hierarchy, a search algorithm could force each layer to sample some values. Given enough budget, each dimension will receive some sampling points, allowing better coverage in the entire search space. However, with high dimensionality (\ie, many types of cogs) and insufficient budget, a hierarchical search may not be able to perform enough local search to find any good optimizations.

To support different user-specified budgets and to get the best of both approaches, we propose an adaptive hierarchical search approach, as shown in Algorithm~\ref{alg:main}.
\search\ starts the search by combining all cogs into one layer ($L=1$, line 9 in Algorithm~\ref{alg:main}) and estimating the expected search budget of this single layer to be the total number of cogs to the power of $\alpha$ (lines 16-19, by default $\alpha = 1.1$). This budget is then passed to the \texttt{LayerSearch} function (Algorithm~\ref{alg:outer}) to perform the actual cog search. When the user-defined budget is no larger than this estimated budget, we expect the single-layer, non-hierarchical search to work better than hierarchical search.
%as the budget for this single layer.

If the user-defined budget is larger, \search\ continues the search with two layers ($L=2$), combining step and weight cogs into the inner layer and architecture cogs as the outer layer (lines 11-14).
\search\ estimates the total search budget for this round as the product of the number of cogs in each of the two layers to the power of $\alpha$ (lines 16-20). It then distributes the estimated search budget between the two layers proportionally to each layer's complexity (lines 22-24) and calls the upper layer's \texttt{LayerSearch} function. Afterward, if there is still budget left, \search\ performs a last round of search using three layers and the remaining budget in a similar way as described above but with three separate layers (architecture as the outermost, step as the middle, and weight cogs as the innermost layer). Two or three layers work better for larger user-defined budgets, as they allow for a larger coverage of the high-dimensional search space.

Finally, \search\ combines all the search results to select the best configurations based on user-defined metrics (line 34).

%\search\ organizes cogs by having architecture cogs in the outer-most search layer, step cogs in the middle layer, and weight cogs in the inner-most layer (line 4 in Algorithm~\ref{alg:main}).
%This is because step cogs' input and output format are dependent on the workflow structure, and the effectiveness of weights (\eg, prompting) are dependent on steps (\eg, LLM model). 

% increases the number of layers until hitting the user-specified total search budget, $TB$

%Thus, the first step of \search\ is to determine the number of layers in its hierarchy and what cogs to include in a layer.
%Intuitively, structure cogs should be placed in the outer-most search layer to be determined first before exploring other cogs. This is because other cogs change node and edge values, and it is easier for 
%However, instead of a fixed number of layers in the hierarchy, we adapt the cog layering according to user-specified total search budgets, $TB$, and the complexity of each layer, using Algorithm~\ref{alg:main}.

% the following \texttt{LayerPartition} method.
%We begin by modeling the relationship between the expected number of evaluations and the number of cogs as well as the number of options in each layer:

%We first consider the identity of each cog in the search space. All structure-cogs will be placed in the outer-most search layer exclusively, which is similar to non-differentiable NAS in traditional ML training. This layer will fix the workflow graph and pass it to the following layer, allowing a stabilized search space for faster convergence.

%Since step-cogs will not create a changing search space, the partition of step-cogs and weight-cogs is conditioned on the search space complexity and the given total budget. Separating step-cogs out can benefit from a more flexible budget allocation strategy and broader exploration for local search at weight-cogs but performs poorly when the given budget is more constrained, in that case, we will optimize them jointly in the same layer.


%\small
%\begin{align*}
%    C &= \{c_{11},c_{12},...\}, \{c_{21},c_{22},...\}, \{c_{31},c_{32},...\} \\
%    NC_i &= \text{total number of cogs in layer i} \\
%    NO_i &= \sum_{j} \{\text{number of possible options in cog } c_{ij}\} \\
%    N_i &= max(NC_i^\alpha,NO_i) \\
%    N_i &= \sum_{j} \{\text{number of possible options in } C_{ij}\} \\
%    N_i &= max(|C_i|^\alpha, N_i) \\
%    B_j &= \prod\limits_{i=1}^{j}N_i, j \in \{1,2,3\}
%\end{align*}

%\normalsize
%where $L$ represents the total number of layers and can be 1, 2, or 3. 
%$C$ represents the entire cog search space, with each row $c_{i*}$ being one of the three types of cogs and lower layers having lower-numbered rows (\eg, $c_{1*}$ being weight cogs). $NC_i$ is the number of cogs in layer $i$, and $NO_i$ is the total number of options across all cogs in layer $i$. $N_i$ is our estimation of the complexity of layer $i$ based on $NC_i$ and $NO_i$ ($\alpha$ is a configurable weight to control the importance between $NC_i$ and $NO_i$; by default $\alpha = 1.1$). 
%$\alpha$ stands for a control parameter, setting the intensity of this scaling behavior w.r.t the number of cogs, we found that $\alpha = 1.2$ is empirically sufficient and efficient for optimizing real workloads. 
%$B_j$ is the expected total number of workflow evaluations for all the lower $j$ layers.
%After calculating $B_1$, $B_2$, and $B_3$, we compare the total budget $TB$ with them.
%When $TB \geq B_3$, we set the total number of layers, $TL$, to 3. When $B_2 \leq TB < B_3$, we set the total number of layers to 2 and merge the step and weight cogs into one layer. When $TB < B_1$, we put all cogs in one layer.
%We only create a separate layer for step-cogs when the given budget $TB$ is greater or equal to the total expected budget for three layers.

%\subsection{Seach Budget Partition}
%\label{sec:sbp}
%After determining cog layers, we distribute the total budget, $TB$, across the layers proportionally to each layer's expected budget $N_i$: , which is the \texttt{BudgetAssign} function.
%We follow a bottom-up partition strategy, where lower layers will try to greedily take the expected budget. This stems from two simple heuristics: (1) feedback to the upper layer is more accurate when the succeeding layer is trained with enough iterations, and (2) the effectiveness of a structure change depends on the setting of individual steps in the workflow (\eg, majority voting is more powerful when each LLM-agent is embedded with diverse few-shot examples or reasoning styles). In cases where the given resource exceeds the total expected budget, 
%We assign $TB$ across layers proportionally to their expected budget $N_i$. 
%The budget assigned at each layer $B_i$ given the total available number of evaluations $TB$ is obtained as follows:

%\small
%\begin{align}
%B_i &=  \lfloor N_i \times \sqrt[L]{\frac{TB}{B^*}}\rfloor
%    B_L &= \begin{cases}
%        min(N_L, TB) & TB < B^* \\
%        \lfloor N_L \times \sqrt[L]{\frac{TB}{B^*}}\rfloor & TB \geq B^*
%    \end{cases}
%    \\
%    B_i &= \begin{cases}
%        min(N_i, \lfloor\frac{TB}{\prod_{j=i+1}^L B_j}\rfloor) & TB < B^* \\
%        \lfloor N_i \times \sqrt[L]{\frac{TB}{B^*}}\rfloor & TB \geq B^*
%    \end{cases}
%\end{align}

%\normalsize


\subsection{Recursive Layer-Wise Search Algorithm}
%The calculation above pre-assigns cogs to layers and search budgets to each layer. 
We now introduce how \search\ performs the actual search in a recursive manner until the inner-most layer is searched, as presented in Algorithm~\ref{alg:outer} \texttt{LayerSearch}. 
Our overall goal is to ensure strong cog option coverage within each layer while quickly directing budgets to more promising cog options based on evaluation results.
%So far, we have determined the optimization layer structure and the maximum allowed search iteration each layer will get. Next, we introduce how the budget is consumed in each layer. The inner-most layer, where weight-cogs, and potentially step-cogs, reside, follows the conventional Bayesian optimization process, exhausting all budgets unless an early stop signal is sent. This signal will be triggered when the current optimizer witnesses $p$ consecutive iterations without any improvements above the threshold. All optimization layers use early stopping to avoid budget waste.
%Algorithm~\ref{alg:inner} describes the search happening at the inner-most (bottom) layer, and 
Specifically, every layer's search is under a chosen set of cog configurations from its upper layers ($C_{chosen}$) and is given a budget $b$. 
In the inner-most layer (lines 7-20), \search\ samples $b$ configurations and evaluates the workflow for each of them together with the configurations from all upper layers ($C_{chosen}$). The evaluation results are added to the feedback set $F$ as the return of this layer.

\begin{algorithm}[h]
  %\algsetup{linenosize=\tiny}
  \small
    \caption{\texttt{LayerSearch} Function}
    \label{alg:outer}
\begin{algorithmic}[1]
%\STATE \textbf{Global Config:} Reduct factor $\eta > 1$, explore width $W$
\STATE \textbf{Global Value:} $R$ \COMMENT{Global result set}
%\STATE \textbf{Global Value:} $F$ \COMMENT{Global observation set}
\STATE \textbf{Input:} $C_{chosen}$: configs chosen in upper layers
\STATE \textbf{Input:} $B$: Array storing assigned budgets to different layers
\STATE \textbf{Input:} $curr\_layer$: this layer's level
\STATE \textbf{Input:} $curr\_b$: this layer's assigned budget
%\STATE
%\FUNCTION{LayerSearch\hspace{0.4em}($C_{chosen}$, $B$, $curr\_layer$, $curr\_b$)}

    \STATE
    \STATE \COMMENT{Search for inner-most layer}
    \IF{curr\_layer == 1}
        \STATE $F = \emptyset$ \COMMENT{Init this layer's feedback set to empty}
        %\STATE $F^{\prime} = match(C_{chosen}, F)$ \COMMENT{Local feedback set}
        \FOR{$k = 0, \dots, curr\_b$}
            \STATE $\lambda$ = \texttt{TPESample} (1) \COMMENT{Sample one configuration using TPE}
            \STATE $f = $ \texttt{EvaluateWorkflow} ($C_{chosen} \cup \lambda$)
            \STATE $R = R \cup \{C_{chosen} \cup \lambda\}$ \COMMENT{Add configuration to global $R$}
            \IF{\texttt{EarlyStop} (f)}
            \STATE break
            \ENDIF
            \STATE $F = F \cup \{f\}$ \COMMENT{Add evaluate result to feedback $F$}
        \ENDFOR
        %\STATE $F = F \cup F^{\prime}$
        \STATE \textbf{Return} $F$
    \ENDIF
    \STATE
    \STATE \COMMENT{Search for non-inner-most layer}
    %\STATE $K = \lfloor \frac{b}{W} \rfloor$, 
    \STATE $b\_used = 0$, $TF = \emptyset$ \COMMENT{Init this layer's used budget and feedback set}
    \STATE $R = \lceil\frac{curr\_b}{\eta}\rceil$, $S = \lfloor\frac{curr\_b}{R}\rfloor$ \COMMENT{Set $R$ and $S$ based on $curr\_b$}
    \STATE
    \WHILE{$b\_{used}$ $\leq$ $curr\_b$}
        \STATE \COMMENT{Sample $W$ configs at a time until running out of $curr\_b$}
        \STATE $n = (curr\_b - b_{used})$ > $W$ ? $W$ : $(curr\_b - b_{used})$
        %\IF{$b - b_{used} < W$}
        %    \STATE $n = b_l - b_{used}$
        %\ELSE
         %   \STATE $n=W$
        %\ENDIF
        \STATE $b\_used$ += $n$
        %\STATE $n = \text{min}(W,\ b_l - kW)$ \COMMENT{Propose $W$ configs and meet $b_l$ constraint}
        \STATE $\Theta = $ \texttt{TPESample} ($n$) \COMMENT{Sample a chunk of $n$ configs in the layer} 
        %\STATE $F^{\prime} = match(C_{chosen}, F)$ \COMMENT{Per-chunk feedback set}
        \STATE $F = \emptyset$ \COMMENT{Init this layer's feedback set to empty}
        \STATE
        \FOR{$s = 0, 1, \dots, S$}
            \STATE $r_s = R\cdot \eta^s$
            \FOR{$\theta \in \Theta$}
                %\IF{$curr\_layer < max\_layer$}
                    \STATE $f =$ \texttt{LayerSearch} ($C_{chosen} \cup \{\theta\}$, $B$, curr\_layer$-1$, $r_s$)
                %\ELSE
                %    \STATE $f =$ InnerOpt($\gamma \cup \{\theta\}$, $r_s$)
                %\STATE $f$ = $opt\_layers[current\_layer+1](\gamma \cup \{\theta\}, r_s)$ \{Optimize the current config at the next layer with $r_s$ budget \}
                %\ENDIF
                \STATE $F = F \cup f$ \COMMENT{Add evaluate result to feedback}
                \IF{\texttt{EarlyStop} ($f$)}
                    \STATE $\Theta = \Theta - \{\theta\}$ \COMMENT{Skip converged configs}
                \ENDIF
            \ENDFOR
            \STATE $\Theta$ = Select top $\lfloor \frac{|\Theta|}{\eta}\rfloor$ configs from $F$ based on user-specified metrics
        \ENDFOR
        \STATE
        \IF{\texttt{EarlyStop} ($F$)}
            \STATE break \COMMENT{Skip remaining search if results converged}
        \ENDIF
        \STATE $TF = TF \cup F$
    \ENDWHILE
    %\STATE $F = F \cup TF$
        \STATE \textbf{Return} $TF$

%\ENDFUNCTION

%\STATE \textbf{Output:} Best metrics in all trials
\end{algorithmic}
\end{algorithm}

% consumption\_nextlayer\_bucket = WSR

% for s in 0, 1,...S do
%     w = W*\eta^{s}
%     r = R*\eta^{-s}

% total budget at next layer = b_l / W * WSR = b_l * SR

% b_l * SR <= b_l * B_l+1

% S = B_{l+1} / R



For a non-inner-most layer, \search\ samples a chunk ($W$) of points at a time using the TPE BO algorithm~\cite{bergstra2011tpe} until all this layer's pre-assigned budget is exhausted (lines 27-30). Within a chunk, \search\ uses a successive-halving-like approach to iteratively direct the search budget to more promising configurations within the chunk (the dynamically changing set, $\Theta$). In each iteration, \search\ calls the next-level search function for each sampled configuration in $\Theta$ with a budget of $r_s$ and adds the evaluation observations from lower layers to the feedback set $F$ for later TPE sampling to use (lines 35-37).
In the first iteration ($s=0$), $r_s$ is set to $R\cdot \eta^0=R$ (line 34). After the inner layers use this budget to search, \search\ filters out configurations with lower performance and only keeps the top $\lfloor \frac{|\Theta|}{\eta}\rfloor$ configurations as the new $\Theta$ to explore in the next iteration (line 42). In each next iteration, \search\ increases $r_s$ by $\eta$ times (line 34), essentially giving more search budget to the better configurations from the previous iteration.

The successive halving method effectively distributes the search budget to more promising configurations, while the chunk-based sampling approach allows for evaluation feedback to accumulate quickly so that later rounds of TPE can get more feedback (compared to no chunking and sampling all $b$ configurations at the same time). To further improve the search efficiency, we adopt an {\em early stop} approach where we stop a chunk or a layer's search when we find its latest few searches do not improve workflow results by more than a threshold, indicating convergence (lines 14,38,45).

%algorithm takes as input other cog settings from previous layers and the assigned budget at the current layer. It tiles the search loop into fixed-size blocks (line 4), each runs the SuccessiveHalving (SH) subroutine in the inner loop (line 7-15). In each SH iteration, only top-$1/\eta$-quantile configurations in $\Theta$ will continue in the next round with $\eta$ times larger budget consumption. As a result, exponentially more trials will be performed by more promising configurations. 

%On average, \textit{Outer-layer search} will create $K$ brackets, each granting approximately $WRS$ budget to the next layer. $R$ represents the smallest amount of resource allocated to any configurations in $\Theta$. 

% layer - 1: budget = 4
% K * W <= b\_current layer
% layer -1: itear 0: propose 2

%     SH:
%     2 config -> R
%     1 config -> 2R

%     iteration 1: propose 2 = W
%     SH:
%     2 config -> R
%     1 config -> 2R

% W configs; each has R resource

% W / eta configs; each has R * eta resource

% R -> least resource one config can get = B2 - smth
% R + R*eta + ... + R*eta\^s -> most promising = B2 + smth


% $L2$ is the middle layer where structure-cogs and step-cogs may be placed exclusively. We employ hyperband for its robustness in exploration and exploitation trace-off. If this layer exists, it will instruct $L1$ the number of search iterations to run in each invocation. Specifically, in each iteration at line 4, \sysname will propose $n$ configurations and run SuccessiveHalving (SH) subroutine (line 8-15). SH will optimize each proposal and use the search results from $L1$ to rank their performance. Each time only the top-performing $n \cdot \eta^{-i}$ can continue in the next round with a larger budget. With this strategy, exponentially more search budgets are allocated to more promising configs at $L2$.

% \input{algo-l2-search}

% $L3$ is the outer-most layer for structure-cogs only when $L2$ is created. For this layer, we employ plain SH without hyperband because of its predictable convergence behavior. This is mainly due to two factors: (1) structure change to the workflow is more significant thus different configurations are more likely to deviate after training with the following layers. (2) with the search space partition strategy in Sec ~\ref{sec:ssp}, we can assume the available budget at each layer is substantial when $L3$ exists. Given this prior knowledge, we can avoid grid searching control parameter $n$ as in the hyperband but adopt a more aggressive allocation scheme to bias towards better proposals and moderate search wastes.



%\subsubsection{Runtime Budget Adaptation}
%Using static estimation of the expected budget for each layer is not enough, we also adjust the assignment during the optimization based on real convergence behavior. Specifically, for layer $i$, we record the number of configurations evaluated in each optimize routine. We set the convergence indicator $C_{ij}$ of $j^{th}$ routine with this number if the search early exits before reaching the budget limit, otherwise 2\x of its assigned resource. Then we update $E_i$ with $\frac{\sum_{j}^M C_{ij}}{M}$. \sysname\ will update the budget partition according to Sec~\ref{sec:sbp} for any newly spawned optimizer routines. Besides controlling the proportion of budgets across layers, a smaller/larger $B_{l+1}$ will also guide the SH in Alg~\ref{alg:outer} to shrink/extend the budget $R$ for differentiating config performance.


\section{\sysname\ Design}
\label{sec:cognify}

We build \sysname, an extensible gen-AI workflow autotuning platform based on the \search\ algorithm. The input to \sysname\ is the user-written gen-AI workflow (we currently support LangChain \cite{langchain-repo}, DSPy \cite{khattab2024dspy}, and our own programming model), a user-provided workflow training set, a user-chosen evaluator, and a user-specified total search budget. \sysname\ currently supports three autotuning objectives: generation quality (defined by the user evaluator), total workflow execution cost, and total workflow execution latency. Users can choose one or more of these objectives and set thresholds for them or the remaining metrics (\eg, optimize cost and latency while ensuring quality to be at least 5\% better than the original workflow). 
\sysname\ uses the \search\ algorithm to search through the cog space.
When given multiple optimization objectives, \sysname\ maintains a sorted optimization queue for each objective and performs its pruning and final result selection from all the sorted queues (possibly with different weighted numbers).
To speed up the search process, we employ parallel execution, where a user-configurable number of optimizers, each taking a chunk of search load, work together in parallel. %Below, we introduce each type of cogs in more details.
\sysname\ returns multiple autotuned workflow versions based on user-specified objectives.
\sysname\ also allows users to continue the auto-tuning from a previous optimization result with more budgets so that users can gradually increase their search budget without prior knowledge of what budget is sufficient.
Appendix~\ref{sec:apdx-example} shows an example of \sysname-tuned workflow outputs. 
\sysname\ currently supports six cogs in three categories, as discussed below. 

%In \sysname, we call every workflow optimization technique a {\em cog}, including structure-changing cogs like task decomposition, step-changing cogs like model selection, and weight-changing cogs like adding few-shot examples to prompts. 
%\sysname\ places structure-changing cogs in the outermost layer, step cogs in the middle layer, and weight cogs in the innermost layer, because \fixme{TODO}.


\subsection{Architecture Cogs}
\label{sec:structure-cog}
%Changing the structure of a workflow can potentially improve its generation quality (\eg, by using multiple steps to attempt at a task in parallel or in chain) or reduce its execution cost and latency (\eg, by merging or removing steps).
\sysname\ currently supports two architecture cogs: task decomposition and task ensemble.
Task decomposition~\cite{khot2023decomposed} breaks a workflow step into multiple sub-steps and can potentially improve generation quality and lower execution costs, as decomposed tasks are easier to solve even with a small (cheaper) model.
There are numerous ways to perform task decomposition in a workflow. 
%, as all LM steps can potentially be decomposed and into different numbers of sub-steps in different ways. Throwing all options to the Bayesian Optimizer would drastically increase the search space for \sysname. 
To reduce the search space, we propose several ways to narrow down task decomposition options. Even though we present these techniques in the setting of task decomposition, they generalize to many other structure-changing tuning techniques.

%First, we narrow down a selected set of steps in a workflow to decompose. 
Intuitively, complex tasks are the hardest to solve and worth decomposition the most. We use a combination of LLM-as-a-judge \cite{vicuna_share_gpt} and static graph (program) analysis to identify complex steps. We instruct an LLM to give a rating of the complexity of each step in a workflow. We then analyze the relationship between steps in a workflow and find the number of out-edges of each step (\ie, the number of subsequent steps getting this step's output). More out-edges imply that a step is likely performing more tasks at the same time and is thus more complex. We multiply the LLM-produced rating and the number of out-edges for each step and pick the modules with scores above a learnable threshold as the target for task decomposition. We then instruct an LLM to propose a decomposition (\ie, generate the submodules and their prompts) for each candidate step. %We provide the LLM with few-shot examples for what proposed modules for a separate task could look like. We also add a refinement step that validates whether the proposition decomposition maintains the semantics of the original trajectory. Once candidate decompositions are generated, those are used for the entirety of the optimization.

{
\begin{figure*}[t!]
\begin{center}
\centerline{\includegraphics[width=0.95\textwidth]{Figures/big_grid.pdf}}
\vspace{-0.1in}
\mycap{Generation Quality vs Cost/Latency.}{Dashed lines show the Pareto frontier (upper left is better). Cost shown as model API dollar cost for every 1000 requests. Cognify selects models
from GPT-4o-mini and Llama-8B. DSPy and Trace do not support model selection and are given GPT-4o-mini for all steps. Trace results for Text-2-SQL and FinRobot have 0 quality and are not included.} 
\Description{Eight graphs with different shapes representing baselines compared to points on a Pareto frontier.}
\label{fig-biggrid}
\end{center}
\end{figure*}
}


The second structure-changing cog that \sysname\ supports is task ensembling. This cog spawns multiple parallel steps (or samplers) for a single step in the original workflow, as well as an aggregator step that returns the best output (or combination of outputs). By introducing parallel steps, \sysname\ can optimize these independently with step and weight cogs. This provides the aggregator with a diverse set of outputs to choose from. 
%The aggregator is prompted with the role of the samplers, as well as the inputs to each. It also receives a criteria by which it should make a decision. We choose to prompt it with a qualitative description of how it should resolve discrepancies between outputs. 


\subsection{Step Cogs}
We currently support two step-changing cogs: model selection for language-model (LM) steps and code rewriting for code steps.

For model selection, to reduce its search space, we identify ``important'' LM steps---steps that most critically impact the final workflow output to reduce the set \search\ performs TPE sampling on. Our approach is to test each step in isolation by freezing other steps with the cheapest model and trying different models on the step under testing. 
We then calculate the difference between the model yielding the best and worst workflow results as the importance of the step under testing. %For each model, we get the workflow output quality score using sampled user-supplied inputs and user-specific evaluator. We then calculate the difference between the highest and lowest scores as this module's importance. 
After testing all the steps, we choose the steps with the highest K\% importance as the ones for TPE to sample from.
%, where K is determined based on user-chosen stop criteria. We then initialize the Bayesian optimization to start with the state where important modules use the largest model and all other modules use the cheapest model. We set the TPE optimization bandwidth of each module to be the inverse of importance, \ie, the more important a module is the more TPE spends on optimizing.

The second step cog \sysname\ supports is code rewriting, where it automatically changes code steps to use better implementation. To rewrite a code step, \sysname\ finds the $k$ worst- and best-performing training data points and feeds their corresponding input and output pairs of this code step to an LLM. We let the LLM propose $n$ new candidate code pieces for the step at a time.
%in parallel to generate a set of $n$ candidates.
In subsequent trials, the optimizer dynamically updates the candidate set using feedback from the evaluator.


\subsection{Weight Cogs}
\sysname\ currently supports two weight-changing cogs: reasoning and few-shot examples.
First, \sysname\ supports adding reasoning capability to the user's original prompt, with two options: zero-shot Chain-of-Thought \cite{wei2022chain} (\ie, ``think step-by-step...'') and dynamic planning \cite{huang2022language} (\ie, ``break down the task into simpler sub-tasks...''). These prompts are appended to the user's prompt. In the case where the original module relies on structured output, we support a reason-then-format option that injects reasoning text into the prompt while maintaining the original output schema.

Second, \sysname\ supports dynamically adding few-shot examples to a prompt. At the end of each iteration, we choose the top-$k$-performing examples for an LM step in the training data and use their corresponding input-output pairs of the LM step as the few-shot examples to be appended to the original prompt to the LM step for later iterations' TPE sampling. As such, the set of few-shot examples is constantly evolving during the optimization process based on the workflow's evaluation results. 
%Few-shot examples are available to all modules, even intermediate steps in the workflow. We use the full trajectory of each request to generate examples for the intermediate steps. Furthermore, we automatically filter out examples that do not meet a user-specified threshold. 



\section{Evaluations} \label{s:eva}
We have conducted several experiments to prove the effectiveness of \ccSysName. In particular, our evaluation focuses on addressing the following research questions (RQs):

\begin{enumerate}[label=\textbf{RQ\arabic*.}]
    \item How effective is \ccSysName?
    \item How efficient is \ccSysName?
    \item How \ccSysName performs for the real-world TEE projects?
\end{enumerate}

\subsection{Environment Setup}
\subsubsection{Benchmark for Evaluation} \label{s:bench}
% Our dataset consists of \mnote{X} TEE-related projects sourced from GitHub, supplemented with synthetic code to expand its scale.
% The dataset contains \mnote{XXX} lines of code and includes \mnote{XX} ground truths (GT) related to bad partitioning.
% Specifically, there are \mnote{XX} instances of bad partitioning type \whiteding{1}, \mnote{XX} instances of type \whiteding{2}, and \mnote{XX} instances of type \whiteding{3}, providing a comprehensive foundation for evaluating the capabilities of \ccSysName.

% In order to cover a wide range of bad partitioning instances in TEE projects, it is necessary to create datasets by manually injecting vulnerabilities into TEE code~\cite{10.1145/3540250.3549128, 8901573}.
Prior studies~\cite{10.1145/3540250.3549128, 8901573} have created benchmarks for various static analysis tools that involve injecting vulnerabilities into code.
However, none of the existing studies have created a benchmark for TEE partitioning issues.
By referencing the construction methodology of the popular API-misuse dataset CryptoAPI-Bench~\cite{8901573, 9721567, 10225251}, we designed the benchmark \ccBenchName for evaluating the abilities of \ccSysName in bad partitioning detection.
We will describe its creation steps below.

\begin{figure}[t]
  \centering
  \begin{subfigure}[b]{\linewidth}
    \begin{lstlisting}[language=c++]
int function(TEE_Param params[4])
{...
    params[0].value.a = var;
    ...
}
    \end{lstlisting}
    \caption{Example code of the in-procedure cases. In this case, the critical data \texttt{params} may be passed to another in-TEE function instead of being used directly after entering TEE, making it more challenging to track its data-flow.}
    \label{f:eic}
  \end{subfigure}
  \hfill
  \begin{subfigure}[b]{\linewidth}
        \begin{lstlisting}[language=c++]
snprintf(params[1].memref.buffer, params[1].memref.size, "%s-%s-%d", key, vi, var);
    \end{lstlisting}
    \caption{Example code of the variadic function case. In this case, \texttt{snprintf} may have multiple arguments that need to be copied to the output buffer. Each argument must be checked to ensure it is encrypted.}
    \label{f:efs}
  \end{subfigure}
  \begin{subfigure}[b]{\linewidth}
        \begin{lstlisting}[language=c++]
if(params[2].memref.size > size) {
    memcpy(params[2].memref.buffer, buf, size);
}
    \end{lstlisting}
    \caption{Example code of control-flow-based cases. In this case, the memory operation \texttt{memcpy} is inside a conditional statement that checks the buffer size, so this operation is secure.}
    \label{f:eps}
  \end{subfigure}
  \caption{Examples of test cases in \ccBenchName.}
  \label{fig:cases}
\end{figure}

We divide the test cases into 5 parts, including basic cases, in-procedure cases, variadic function cases, control-flow-based cases, and combined cases. These parts will cover various bad partitioning issues.
\begin{enumerate}
    \item \textbf{Basic cases} are some simple instances described in in Section~\ref{s:bp}. These vulnerabilities typically occur in statements where parameters are directly used after invoking TEE.
    \item \textbf{In-procedure cases} mean that the parameter may be passed into other procedures or methods as an argument, which is shown in Fig.~\ref{f:eic}.
    \item Fig.~\ref{f:efs} gives a \textbf{variadic function case}, in which \texttt{snprintf()} may have more than one argument copied to the output buffer. We need to perform data flow analysis on each of them to find the issues of unencrypted data output.
    \item \textbf{Control-flow-based cases} will provide some \textbf{no-issue cases}, which can evaluate the precision of \ccSysName. For example, in Fig.~\ref{f:eps}, line 2 should not be identified as a bad partitioning issues, because it is in an \texttt{if} block, which checks the buffer size.
    \item \textbf{Combined cases} will introduce at least two of the above cases.
\end{enumerate}

As illustrated in Table~\ref{tbl:gttn}, we totally provide 110 cases for \ccBenchName, which are written in C programming language and developed for ARM TrustZone. Among these cases, 90 cases contain bad partitioning issues and 20 cases have no issues.

% \begin{table*}[t]
%     \caption{Summary of test cases in \ccBenchName. Details are given in Section~\ref{s:bench}.}
%     \label{tbl:benchmark}
%     \renewcommand{\arraystretch}{1.3}
%     \footnotesize
%     \centering
%     \setlength{\tabcolsep}{2.2mm}
% 	\begin{tabular}{c c p{1.3cm}<{\centering} p{1.3cm}<{\centering} p{1.3cm}<{\centering} p{1.3cm}<{\centering} p{1.3cm}<{\centering} p{1.6cm}<{\centering}}
% 		\toprule
% 		\multirow{2}{*}{\textbf{Bad Partitioning Issues}} & \multirow{2}{*}{\textbf{Basic Cases}} & \textbf{Once-Enterproce.} & \textbf{Twice-Enterproce.} & \textbf{Field Sensitive} & \textbf{Path Sensitive} & \textbf{Combined Cases} & \textbf{Total Cases per Issues}\\ 
%             \midrule
%             \makecell[l]{Unencrypted Data Output} & 5 & 6 & 4 & 6 & 6 & 16 & 43 \\
%             \makecell[l]{Input Validation Weaknesses} & 7 & 7 & 5 & 2 & 5 & 8 & 34 \\
%             \makecell[l]{Direct Usage of Shared Memory} & 6 & 7 & 3 & 6 & 2 & 9 & 33 \\
%             \midrule
%             \textbf{Total Cases pre Group} & 18 & 20 & 12 & 14 & 13 & 33 & 110\\
% 		\bottomrule
% 	\end{tabular}
% \end{table*}

% \begin{table}[t]
%     \caption{The number of bad partitioning issues and no-issue cases in \ccBenchName.}
%     \label{tbl:gttn}
%     % \renewcommand{\arraystretch}{1.3}
%     % \footnotesize
%     \setlength{\tabcolsep}{3mm}
%     \centering
% 	\begin{tabular}{lcc}
% 		\toprule
% 		\makecell[c]{\textbf{Bad Partitioning Issues}} & \textbf{Issue cases} & \textbf{No-issue Cases}\\ 
%             \midrule
%             Unencrypted Data Output & 35 & 8 \\
%             Input Validation Weaknesses & 29 & 5 \\
%             Direct Usage of Shared Memory & 26 & 7 \\
%             \midrule
%             \makecell[c]{\textbf{Total Cases}} & 90 & 20 \\
% 		\bottomrule
% 	\end{tabular}
% \end{table}

\begin{table}[t]
    \caption{The number of bad partitioning issues and no-issue cases in \ccBenchName.}
    \label{tbl:gttn}
    % \renewcommand{\arraystretch}{1.3}
    % \footnotesize
    \setlength{\tabcolsep}{3mm}
    \centering
	\begin{tabular}{lc}
		\toprule
		\makecell[c]{\textbf{Bad Partitioning Issues}} & \textbf{Number of Cases}\\ 
            \midrule
            Unencrypted Data Output & 35 \\
            Input Validation Weaknesses & 29 \\
            Direct Usage of Shared Memory & 26 \\
            \midrule
            \makecell[c]{\textbf{No-issue Cases}} & 20 \\
            \midrule
            \makecell[c]{\textbf{Total Cases}} & 110 \\
		\bottomrule
	\end{tabular}
\end{table}

\subsubsection{Experimental Platform}
\ccSysName is implemented by Python in combination with the static analysis tool CodeQL. CodeQL is utilized to generate the data-flow for each parameter based on sources and sinks in Tables~\ref{tbl:out_df} to~\ref{tbl:shm}, and Algorithms~\ref{algo:odf} to~\ref{algo:mdf} for data-flow generation. The Python code is responsible for retrieving these data-flows and identifying the locations of bad partitioning issues with the help of Algorithms~\ref{algo:bp1} to~\ref{algo:bp3}.
All experiments were conducted on a machine running Ubuntu 24.04, equipped with a 48-core 2.3GHz AMD EPYC 7643 processor, and 512GB RAM.

% \begin{table*}[t]
%     \caption{The performance of \ccSysName in detecting three kinds of bad partitioning issues on benchmark. GT is the ground truth number of bad partitioning issues in each TEE project, TP is the number of true positives, and FP is the number of false positives. P, R, and F1 indicate the precision, recall, and F1 score, respectively\textsuperscript{*}.
%     }
%     \label{tbl:expr_res}
%     \renewcommand{\arraystretch}{1.3}
%     \footnotesize
%     \centering
%     \setlength{\tabcolsep}{2mm}
%     \begin{threeparttable} 
% 	\begin{tabular}{lccccccc|cccccc|cccccc}
% 		\toprule
% 		\multirow{2}{*}{\textbf{Projects}} & 
%             \multirow{2}{*}{\textbf{LoC}} & 
%             \multicolumn{6}{c}{Unencrypted Data Output} & 
%             \multicolumn{6}{c}{Input Validation Weaknesses} & \multicolumn{6}{c}{Direct Usage of Shared Memory} \\
%             \cline{3-20}
%              & & GT & TP & FP & P(\%) & R(\%) & F1 & 
%                  GT & TP & FP & P(\%) & R(\%) & F1 & 
%                  GT & TP & FP & P(\%) & R(\%) & F1 \\
% 		\hline
%             synthesis & 515 & 9 & 8 & 0 & 100 & 88.89 & 0.94 & 
%                         7 & 7 & 1 & 83.33 & 85.71 & 0.85 & 
%                         5 & 5 & 1 & 80 & 100 & 0.89 \\
%             optee-sdp & 468 & 12 & 10  & 0 & 100 & 83.33 & 0.91 & 
%                       --- & --- & --- & --- & --- & --- & 
%                       --- & --- & --- & --- & --- & --- \\
%             optee-fiovb & 399 & 1 & 1 & 1 & 100 & 100 & 1 & 
%                         &  &  &  &  &  & 
%                       --- & --- & --- & --- & --- & --- \\
%             basicAlg &  & & & & & & & 
%                        & & & & & & 
%                        --- & --- & --- & --- & --- & --- \\
%             rk\_tee\_user & & 2 & & & & & & 
%                        1 & 1 & --- & --- & --- & --- & 
%                        1 & 1 & & & & \\
%             secvideo & & --- & --- & --- & --- & --- & --- & 
%                       --- & --- & --- & --- & --- & --- & 
%                       & & & & & \\
%             darknetz & & --- & --- & --- & --- & --- & --- & 
%                       --- & --- & --- & --- & --- & --- & 
%                       & & & & & \\
%             \midrule
%             Total & & & & & & & & 
%                       & & & & & & 
%                       & & & & & \\
% 		\bottomrule
% 	\end{tabular}
%     \begin{tablenotes}
%         \footnotesize
%         \item[*] Note: $P(\%) = TP / (TP + FP) * 100$; $R(\%) = TP / GT * 100$; $F1 = 2 * P * R / (P + R)$.
%     \end{tablenotes}
%     \end{threeparttable} 
% \end{table*}

\subsection{RQ1: Precision and Recall Evaluation}

\begin{table}[t]
    \caption{The detection precision and recall of \ccSysName on \ccBenchName. NI is the number of partitioning issues, N is the number of detection results, and TP is the number of true positives. P, R and F1 indicates the precision, recall, and F1 score, respectively\textsuperscript{*}.
    }
    \label{tbl:recall_res}
    % \renewcommand{\arraystretch}{1.3}
    % \footnotesize
    \setlength{\tabcolsep}{3mm}
    \centering
    \begin{threeparttable}
        \begin{tabular}{lcccccc}
        \toprule
        \makecell[c]{\textbf{Bad Partitioning Issues}} & \textbf{NI} & \textbf{N} & \textbf{TP} & \textbf{P(\%)} & \textbf{R(\%)} & \textbf{F1}\\
        \midrule
        Unencrypted Data Output & 35 & 34 & 33 & 97.06 & 94.29 & 0.96 \\
        Input Validation Weaknesses & 29 & 27 & 23 & 85.19 & 79.31 & 0.82 \\
        Direct Usage of Shared Memory & 26 & 30 & 25 & 83.33 & 96.15 & 0.89 \\
        \midrule
        \makecell[c]{\textbf{Total}} & 90 & 91 & 81 & 89.01 & 90 & 0.90\\
        \bottomrule
        \end{tabular}
    \begin{tablenotes}
        % \footnotesize
        \item[*] Note: $P(\%) = TP / N \times 100$, $R(\%) = TP / NI \times 100$.
    \end{tablenotes}
    \end{threeparttable}
\end{table}

Table~\ref{tbl:recall_res} lists the ground truth count of bad partitioning instances in these TEE projects, alongside the number of issues reported by \ccSysName. It can be seen that \ccSysName achieves the highest precision of 97.06\% in detecting bad partitioning of unencrypted data output. However, for the other two issues, more FPs are observed.

Through analysing the examples of misdiagnosed code shown in Fig.~\ref{fig:fp}, we categorize the FPs into the following three types:

\begin{figure}[t]
  \centering
    \begin{subfigure}[b]{\linewidth}
    \begin{lstlisting}[language=c++]
char buf[] = "aabbcc";
...
params[0].memref.size = strlen(buf);
// params[1].value.a is the output, while params[2].value.a is the input
params[1].value.a = params[2].value.a;
    \end{lstlisting}
    \caption{The length of \texttt{buf} and \texttt{params[2].value.a} are not the sensitive data.}
    \label{fig:fp1}
  \end{subfigure}
  \begin{subfigure}[b]{\linewidth}
    \begin{lstlisting}[language=c++]
unsigned int size = params[0].memref.size;
// obtain the third last character in the input buffer
char c = params[0].memref.buffer[size - 3];
    \end{lstlisting}
    \caption{Line 3 is the FP of input validation weaknesses, where \texttt{size} is an input parameter as array index.}
    \label{fig:fp2}
  \end{subfigure}
  \hfill
  \begin{subfigure}[b]{\linewidth}
    \begin{lstlisting}[language=c++]
// variable assignment by shared memory value
unsigned int size = params[3].memref.size;
// shallow copy of shared memory buffer
void *buf = params[3].memref.buffer;
    \end{lstlisting}
    \caption{Line 2 is the FP of direct usage of shared memory, where \texttt{params[0].memref.size} is a shared memory parameter.}
    \label{fig:fp3}
  \end{subfigure}
  \caption{Examples of FPs in \ccSysName reports.}
  \label{fig:fp}
\end{figure}

\begin{enumerate}
    \item \textbf{Non-sensitive data is assigned to the output:} Some data can be assigned to the output without encryption, and this will not compromise the confidentiality of TEE. For example, in Fig.~\ref{fig:fp1}, it is acceptable that we pass the length of the output buffer to the outside or copy the input directly to the output.
    \item \textbf{Indexing for Input Buffer with Its Size:} As shown in Fig.~\ref{fig:fp2}, although the index of array access is influenced by the input value \texttt{params[0].memref.size}, TEE can ensure that this input will not lead to buffer overflow. This is because that the input \texttt{params[0].memref.buffer} in TEE is actually a memory copy of the normal-side data, which is allocated with a size based on the above input value, and this input value is usually the size of the input data outside TEE. It means that the size of \texttt{params[0].memref.buffer} must not be smaller than the input value \texttt{params[0].memref.size}. Therefore, it is secure to directly use \texttt{params[0].memref.size} as an array index, even if no checks are performed on it which violates Rule 2.
    \item \textbf{Variable Assignment Using Shared Memory:} Similar to the input buffer, shared memory also transmits memory address and memory size, from the normal side to TEE by \texttt{params[3].memref}. According to Algorithm~\ref{algo:param_type}, both of these parameters are treated as the type of shared memory. Meanwhile, any assignment statement containing shared memory parameters is not allowed in Rule 3. However, the variable assignment in line 1 of Fig.~\ref{fig:fp3} can be acceptable, as it differs from the insecure shallow copy of shared memory in line 2.
\end{enumerate}

According to the Recall rate in Table~\ref{tbl:recall_res}, \ccSysName can give more comprehensive coverage of bad partitioning related to the direct use of shared memory than other issues.
% In order to analyze the missed detection, we get FNs by excluding true positives (TP) from GTs.
We also give two examples of FNs in Fig.~\ref{fig:fn} and discuss them as follow:

\begin{figure}[t]
  \centering
  \begin{subfigure}[b]{\linewidth}
    \begin{lstlisting}[language=c++]
void test(char *dest, char *src)
{
    TEE_MemMove(dest, src, strlen(src));
}

void TA_InvokeCommandEntryPoint()
{
    char key[] = "123456";
    ...
    char *str = TEE_Malloc(strlen(key) + 1, 0);
    // case 1
    test(str, key);
    ...
    // case 2
    test(params[1].memref.buffer, key);
}
    \end{lstlisting}
    \caption{Line 15 is the FN of unencrypted data output, where \texttt{params[1].memref.buffer} is an output parameter.}
    \label{fig:fn1}
  \end{subfigure}
  \hfill
  \begin{subfigure}[b]{\linewidth}
    \begin{lstlisting}[language=c++]
char *str[1024] = {0};
...
for (int i = 0; i < params[2].memref.size; i++) {
    str[i] = params[2].memref.buffer[i];
}
    \end{lstlisting}
    \caption{Line 4 is the FN of input validation weaknesses, where \texttt{i} is affected by the input \texttt{params[2].memref.size}.}
    \label{fig:fn2}
  \end{subfigure}
  \caption{Examples of FNs in \ccSysName reports.}
  \label{fig:fn}
\end{figure}

\begin{enumerate}
    \item \textbf{Using Memory Operation Functions In Procedure:} The destination buffer \texttt{dest} of the function \texttt{TEE\_MemMove} in line 3 of Fig.~\ref{fig:fn1} is the formal parameter of function \texttt{test}.
    Therefore, \ccSysName recognizes the type of \texttt{dest} as a character array.
    However, as illustrated in lines 10 to 13, there are two instances of using \texttt{test}, one involves copying data to the in-TEE buffer \texttt{str}, while the other will copy data to an output.
    Since \ccSysName does not identify the actual parameter type of \texttt{dest} as an output, it also fails to report the above second instance as bad partitioning, which violates Rule 1.
    \item \textbf{Array Access Affected by Input:} Lines 3 to 5 give a \texttt{for} loop block, which is responsible for copying an buffer byte by byte.
    At this point, the input value \texttt{params[2].memref.size} does not directly serve as an index for array access but is instead used in the condition of \texttt{for} statement to constrain the value of \texttt{i}.
    Therefore, Algorithm~\ref{algo:idf} does not recognize this type of array access statement, that has broken Rule 2.
\end{enumerate}

In summary, if we formulate additional filters for the reported issues in the future, it is possible to effectively reduce FPs that deviate from the predefined rules but do not actually introduce security risks. What's more, we can refine the rules by incorporating the semantic context of the code, and FNs can be further reduced.

\begin{tcolorbox} [colback=gray!20!white]
\textbf{Answer to RQ1:}
(1) \ccSysName provides the highest F1 score of 0.96 in the detection of unencrypted data output (both precision and recall exceed 90\%).
(2) The detection of direct usage of shared memory achieves the best recall (above 95\%), but with relatively low precision (below 85\%).
(3) The detection of input validation weaknesses has an F1 score of 0.82. While its precision is above 85\%, the recall remains slightly lower at approximately 79\%.
\end{tcolorbox}

\subsection{RQ2: Efficiency Evaluation}

\begin{figure}
    \centering
    \includegraphics[width=0.5\linewidth]{figures/Fig_10.pdf}
    \caption{Time cost of data-flow construction and analysis under different scales of LoC in \ccSysName.}
    \label{fig:effi}
\end{figure}

On our benchmark tests, we recorded the time taken by \ccSysName to generate data-flows for three types of parameters and match these data-flows against the corresponding rules.

Algorithms~\ref{algo:odf} to \ref{algo:mdf}, which are responsible for constructing data-flows for parameters, involve $m$ parameters that need to be tracked and $n$ sensitive functions and data in TEE, resulting in the algorithmic complexity of them being $O(mn)$.
As shown in Fig.~\ref{fig:effi}, the time consumed by constructing data-flows increases from 21.53 seconds to 34.95 seconds as the lines of code (LoC) grow. 
Then, \ccSysName focuses only on critical functions and data during data-flow analysis, such as \texttt{TEE\_MemMove} and array access operations. The time spent on data-flow analysis remains relatively low and does not grow significantly with the changes of LoC (from 3.96 milliseconds to 7.13 milliseconds).

\begin{tcolorbox} [colback=gray!20!white]
\textbf{Answer to RQ2:}
As the code scale increases from 322 LoC to 5162 LoC, the time cost of the data-flow construction for bad partitioning analysis increases from 21.53 seconds to 34.95 seconds.
However, this time consumption also depends on the performance of the underlying code analysis tool.
\ccSysName still exhibits high efficiency for detecting bad partitioning issues in TEE projects (around 5.56 milliseconds).
\end{tcolorbox}

\subsection{RQ3: In-the-Wild Study}

% \begin{table*}[t]
%     \caption{The precision of \ccSysName in detecting three kinds of bad partitioning issues on real-world TEE projects. 
%     % N is the number of detection results, TP is the number of true positives, FP is the number of false positives, and P indicates the precision\textsuperscript{*}.
%     }
%     \label{tbl:expr_res}
%     \renewcommand{\arraystretch}{1.3}
%     \footnotesize
%     \centering
%     \setlength{\tabcolsep}{2.2mm}
%     \begin{threeparttable} 
% 	\begin{tabular}{lrp{0.6cm}<{\centering} p{0.6cm}<{\centering} p{0.6cm}<{\centering} p{0.6cm}<{\centering}|p{0.6cm}<{\centering} p{0.6cm}<{\centering} p{0.6cm}<{\centering} p{0.6cm}<{\centering}|p{0.6cm}<{\centering} p{0.6cm}<{\centering} p{0.6cm}<{\centering} p{0.6cm}<{\centering}}
% 		\toprule
% 		\multirow{2}{*}{\textbf{Projects}} & 
%             \multirow{2}{*}{\textbf{LoC}} & 
%             \multicolumn{4}{c}{Unencrypted Data Output} & 
%             \multicolumn{4}{c}{Input Validation Weaknesses} & \multicolumn{4}{c}{Direct Usage of Shared Memory} \\
%             \cline{3-14}
%              & & N & TP & FP & P(\%) & 
%                  N & TP & FP & P(\%) & 
%                  N & TP & FP & P(\%) \\
% 		\hline
%             % synthesis & 515 & 8 & 8 & 0 & 100 & 
%             %             6 & 5 & 1 & 83.33 & 
%             %             6 & 5 & 1 & 83.33 \\
%             optee-sdp & 830 & 11 & 11 & 0 & 100 & 
%                                7 &  7 & 0 & 100 & 
%                                0 &  0 & 0 & --- \\
%             optee-fiovb & 776 & 2 & 1 & 1 & 50 & 
%                                 2 & 2 & 0 & 100 & 
%                                 0 & 0 & 0 & --- \\
%             basicAlg\_use & 5,162 & & & & & 
%                                & & & & 
%                              0 & 0 & 0 & --- \\
%             external\_rk\_tee\_user & 716 & 4 & 2 & 2 & 50 & 
%                                   2 & 2 & 0 & 100 & 
%                                   1 & 1 & 0 & 100 \\
%             secvideo\_demo & 645 & 0 & 0 & 0 & --- & 
%                              0 & 0 & 0 & --- & 
%                              7 & 4 & 3 & 57.14 \\
%             darknetz & 41,608 & 0 & 0 & 0 & --- & 
%                                 0 & 0 & 0 & --- & 
%                                 1 & 1 & 0 & 100 \\
%             acipher & 338 & 1 & 0 & 0 & --- & 
%                             0 & 0 & 0 & --- & 
%                             0 & 0 & 0 & --- \\
%             Lenet5\_in\_OPTEE & 779 & 1 & 1 & 0 & 100 & 
%                                       3 & 2 & 1 & 66.67 & 
%                                       0 & 0 & 0 & --- \\
%             \midrule
%             Total & & & & & & 
%                       & & & & 
%                       & & & \\
% 		\bottomrule
% 	\end{tabular}
%     % \begin{tablenotes}
%     %     \footnotesize
%     %     \item[*] Note: $FP = N - TP$, $P(\%) = TP / N * 100$.
%     % \end{tablenotes}
%     \end{threeparttable} 
% \end{table*}

We searched for TEE-related projects on GitHub using API keywords associated with TEE. 
After excluding TEE SDK projects, we selected the top 16 projects with the highest star counts to evaluate the performance of \ccSysName on real-world programs, as shown in Table~\ref{tbl:expr_res}. 
These projects include 10 projects that have been manually analyzed in Section~\ref{s:bp}, along with 6 additional projects.

In these projects, \ccSysName reports 68 issues caused by bad partitioning and 55 of them are confirmed by ourselves.
For example, in \textit{optee-sdp} project, the log functionality implemented via \texttt{snprintf} copies a large amount of register data into the output parameters, which are accessible only to TEE. This compromises the confidentiality of TEE.
Additionally, in projects such as \textit{Lenet5\_in\_OPTEE},  \textit{external\_rk\_tee\_user}, there are instances where input buffers are directly used without validating their length.
Attackers could exploit this bad partitioning to cause buffer overflows in TEE applications.
What's more, in \textit{secvideo} project, encryption and decryption operations are directly performed on video data in shared memory, undermining the isolation provided by TEE.

\begin{table*}[t]
    \caption{The performance of \ccSysName in detecting three kinds of bad partitioning issues on real-world TEE projects. 
    % N is the number of detection results, TP is the number of true positives, FP is the number of false positives, and P indicates the precision\textsuperscript{*}.
    }
    \label{tbl:expr_res}
    % \renewcommand{\arraystretch}{1.3}
    % \footnotesize
    \setlength{\tabcolsep}{1.5mm}
    \resizebox{\linewidth}{!}{
    \centering
	\begin{tabular}{lr cc | cc | cc}
		\toprule
		\multirow{2}{*}{\textbf{Projects}} & 
            \multirow{2}{*}{\textbf{LoC}} & 
            \multicolumn{2}{c}{Unencrypted Data Output} & 
            \multicolumn{2}{c}{Input Validation Weaknesses} & \multicolumn{2}{c}{Direct Usage of Shared Memory} \\
            \cline{3-8}
             & & Report Number & True Positive & 
                 Report Number & True Positive & 
                 Report Number & True Positive \\
		\hline
            % synthesis & 515 & 8 & 8 & 0 & 100 & 
            %             6 & 5 & 1 & 83.33 & 
            %             6 & 5 & 1 & 83.33 \\
            optee-sdp & 830 & 11 & 11 &
                               7 &  7 & 
                             --- &  --- \\
            optee-fiovb & 776 & 2 & 1 & 
                                2 & 2 & 
                              --- & --- \\
            basicAlg\_use & 5,162 & 2 & 2 &  
                                    1 & 1 &  
                                  --- & --- \\
            external\_rk\_tee\_user & 716 & 4 & 2 & 
                                            2 & 2 & 
                                            1 & 1 \\
            secvideo\_demo & 645 & --- & --- & 
                                   --- & --- &
                                     7 & 4 \\
            darknetz & 41,608 & --- & --- &
                                --- & --- &
                                 1 & 1 \\
            acipher & 338 & 1 & 0 &
                          --- & --- &
                          --- & --- \\
            Lenet5\_in\_OPTEE & 1,610 & 1 & 1 & 
                                        3 & 2 &
                                      --- & --- \\
            hotp & 428 & 1 & 1 & 
                       --- & --- &
                       --- & --- \\
            random & 322 & 1 & 1 & 
                       --- & --- &
                       --- & --- \\
            read\_key & 404 & 2 & 1 & 
                       --- & --- &
                       --- & --- \\
            save\_key & 529 & 2 & 1 & 
                       --- & --- &
                       --- & --- \\
            socket-benchmark & 1,209 & 5 & 4 & 
                       --- & --- &
                       --- & --- \\
            socket-throughput & 1,100 & 5 & 5 & 
                       --- & --- &
                       --- & --- \\
            % tcp2ext & 322 & 5 & 1 & 
            %            --- & --- &
            %            --- & --- \\
            tcp\_server & 750 & 2 & 2 & 
                       --- & --- &
                       --- & --- \\
            threaded-socket & 1,059 & 5 & 3 & 
                       --- & --- &
                       --- & --- \\
            % \midrule
            % Total & & & & 
            %           & &  
            %           & \\
		\bottomrule
	\end{tabular}
    }
\end{table*}

\begin{tcolorbox} [colback=gray!20!white]
\textbf{Answer to RQ3:} 
\ccSysName is capable of identifying bad partitioning issues in real-world projects.
Through a review of 68 issues reported by \ccSysName in the wild, we have confirmed that 55 of them were true alarms.
% Additionally, we are actively reporting these issues to developers on GitHub.
\end{tcolorbox}

\section{Research Trustworthiness, Limitations and Trade-offs}\label{sec:trust}

\subsection{Trustworthiness}

\noindent We implemented several techniques to address the requirements of research trustworthiness \citep{miles2014qualitative}. We reported \textit{Saturation} (Phase I), \textit{Member checking} (Phase I), and \emph{Feedback session} (Phase II) in Sect. \ref{sec:methods}.

\textit{Triangulation}: We triangulated data sources, including interviews, focus groups, and participant feedback sessions. This exercise allowed us to ensure that our findings are corroborated across different data sources and contexts.

\textit{Peer debriefing}: Although the analysis was primarily conducted by the first author, the second and third authors reviewed the proposed codes, and the results were continuously discussed and scrutinized by the other two authors in several meetings throughout the analysis process. The participation of two authors in the coding process helped minimize researcher biases \citep{miles2014qualitative}. This approach is grounded in our epistemological stance, constructivism, which posits that knowledge is socially constructed and that collective intellectual engagement can lead to more reliable understandings of the data \citep{fosnot2013constructivism}. 

\textit{Thick description}: We endeavored to provide a detailed explanation of our research process and the decisions we have made throughout (see Sect. \ref{sec:methods}). In addition, we assembled a comprehensive replication package (see Sect. \ref{sec:replication}).

\subsection{Limitations and Trade-offs}\label{sec:limit}

\textit{Homogeneous sample}: Our sample is composed only of software engineers. In line with roles theory \citep{katz1978social,frink2004advancing}, we limited our sample to the software engineer role to mitigate the potential for variations that may arise by the inclusion of multiple roles. Roles theory suggests that individuals' accountability is closely linked to roles \citep{katz1978social,frink2004advancing}. This narrow focus strengthens the internal validity of our study and allows for role-centric conclusions.

\textit{Focus on intrinsic drivers}: By focusing primarily on intrinsic drivers and their influence on accountability, we may have inadvertently undermined other factors. For example, in our previous work, we identified institutional factors, such as financial incentives or denial of promotions, that also promote accountability in SE environments \citep{alami2024understanding}. 

\textit{Limited variation in the focus group design}: Another tradeoff is the limited number of variations in the focus group configurations, and the code snippets we used were not of industrial caliber. The consistency across the four groups, shown in the collected data and findings, suggests that additional configurations might not have significantly altered the results. In addition, we prioritized in-depth discussions, which may have been diluted by overly complicated configurations and complex code.

Another tradeoff for this study design is with more realistic, complex, and context-aware code. However, we felt this would greatly limit the accessibility of the focus groups. A future study, examining the contextual intricacies of a proprietary codebase, would shed insight on the role of context in this setting.

We conducted focus groups synchronously and online. Often code reviews, in particular on GitHub and similar sites, are asynchronous and text-based. Open source projects have different dynamics than the ones we discuss here. Hence, our findings our findings may not fully transferable to asynchronous or open-source code reviews. Furthermore, the online setting may have influenced participants' behavior differently than an in-person setup. 

The implementation of a pre-focus group questionnaire to mitigate the risk of social desirability bias \citep{furnham1986response} and self-censorship \citep{yanos2008false} during the focus group discussions carries the risk of priming participants. To mitigate this risk, we avoided the explicit use of the word ``accountability'' in the questions. In addition, during the discussions, we asked participants to provide concrete examples to anchor their responses in their personal experiences, thereby avoiding generic or socially desired answers.


\putsec{related}{Related Work}

\noindent \textbf{Efficient Radiance Field Rendering.}
%
The introduction of Neural Radiance Fields (NeRF)~\cite{mil:sri20} has
generated significant interest in efficient 3D scene representation and
rendering for radiance fields.
%
Over the past years, there has been a large amount of research aimed at
accelerating NeRFs through algorithmic or software
optimizations~\cite{mul:eva22,fri:yu22,che:fun23,sun:sun22}, and the
development of hardware
accelerators~\cite{lee:cho23,li:li23,son:wen23,mub:kan23,fen:liu24}.
%
The state-of-the-art method, 3D Gaussian splatting~\cite{ker:kop23}, has
further fueled interest in accelerating radiance field
rendering~\cite{rad:ste24,lee:lee24,nie:stu24,lee:rho24,ham:mel24} as it
employs rasterization primitives that can be rendered much faster than NeRFs.
%
However, previous research focused on software graphics rendering on
programmable cores or building dedicated hardware accelerators. In contrast,
\name{} investigates the potential of efficient radiance field rendering while
utilizing fixed-function units in graphics hardware.
%
To our knowledge, this is the first work that assesses the performance
implications of rendering Gaussian-based radiance fields on the hardware
graphics pipeline with software and hardware optimizations.

%%%%%%%%%%%%%%%%%%%%%%%%%%%%%%%%%%%%%%%%%%%%%%%%%%%%%%%%%%%%%%%%%%%%%%%%%%
\myparagraph{Enhancing Graphics Rendering Hardware.}
%
The performance advantage of executing graphics rendering on either
programmable shader cores or fixed-function units varies depending on the
rendering methods and hardware designs.
%
Previous studies have explored the performance implication of graphics hardware
design by developing simulation infrastructures for graphics
workloads~\cite{bar:gon06,gub:aam19,tin:sax23,arn:par13}.
%
Additionally, several studies have aimed to improve the performance of
special-purpose hardware such as ray tracing units in graphics
hardware~\cite{cho:now23,liu:cha21} and proposed hardware accelerators for
graphics applications~\cite{lu:hua17,ram:gri09}.
%
In contrast to these works, which primarily evaluate traditional graphics
workloads, our work focuses on improving the performance of volume rendering
workloads, such as Gaussian splatting, which require blending a huge number of
fragments per pixel.

%%%%%%%%%%%%%%%%%%%%%%%%%%%%%%%%%%%%%%%%%%%%%%%%%%%%%%%%%%%%%%%%%%%%%%%%%%
%
In the context of multi-sample anti-aliasing, prior work proposed reducing the
amount of redundant shading by merging fragments from adjacent triangles in a
mesh at the quad granularity~\cite{fat:bou10}.
%
While both our work and quad-fragment merging (QFM)~\cite{fat:bou10} aim to
reduce operations by merging quads, our proposed technique differs from QFM in
many aspects.
%
Our method aims to blend \emph{overlapping primitives} along the depth
direction and applies to quads from any primitive. In contrast, QFM merges quad
fragments from small (e.g., pixel-sized) triangles that \emph{share} an edge
(i.e., \emph{connected}, \emph{non-overlapping} triangles).
%
As such, QFM is not applicable to the scenes consisting of a number of
unconnected transparent triangles, such as those in 3D Gaussian splatting.
%
In addition, our method computes the \emph{exact} color for each pixel by
offloading blending operations from ROPs to shader units, whereas QFM
\emph{approximates} pixel colors by using the color from one triangle when
multiple triangles are merged into a single quad.


\section{Conclusion}
In this work, we propose a simple yet effective approach, called SMILE, for graph few-shot learning with fewer tasks. Specifically, we introduce a novel dual-level mixup strategy, including within-task and across-task mixup, for enriching the diversity of nodes within each task and the diversity of tasks. Also, we incorporate the degree-based prior information to learn expressive node embeddings. Theoretically, we prove that SMILE effectively enhances the model's generalization performance. Empirically, we conduct extensive experiments on multiple benchmarks and the results suggest that SMILE significantly outperforms other baselines, including both in-domain and cross-domain few-shot settings.

% \section{Introduction}
% ACM's consolidated article template, introduced in 2017, provides a
% consistent \LaTeX\ style for use across ACM publications, and
% incorporates accessibility and metadata-extraction functionality
% necessary for future Digital Library endeavors. Numerous ACM and
% SIG-specific \LaTeX\ templates have been examined, and their unique
% features incorporated into this single new template.

% If you are new to publishing with ACM, this document is a valuable
% guide to the process of preparing your work for publication. If you
% have published with ACM before, this document provides insight and
% instruction into more recent changes to the article template.

% The ``\verb|acmart|'' document class can be used to prepare articles
% for any ACM publication --- conference or journal, and for any stage
% of publication, from review to final ``camera-ready'' copy, to the
% author's own version, with {\itshape very} few changes to the source.

% \section{Template Overview}
% As noted in the introduction, the ``\verb|acmart|'' document class can
% be used to prepare many different kinds of documentation --- a
% double-anonymous initial submission of a full-length technical paper, a
% two-page SIGGRAPH Emerging Technologies abstract, a ``camera-ready''
% journal article, a SIGCHI Extended Abstract, and more --- all by
% selecting the appropriate {\itshape template style} and {\itshape
%   template parameters}.

% This document will explain the major features of the document
% class. For further information, the {\itshape \LaTeX\ User's Guide} is
% available from
% \url{https://www.acm.org/publications/proceedings-template}.

% \subsection{Template Styles}

% The primary parameter given to the ``\verb|acmart|'' document class is
% the {\itshape template style} which corresponds to the kind of publication
% or SIG publishing the work. This parameter is enclosed in square
% brackets and is a part of the {\verb|documentclass|} command:
% \begin{verbatim}
%   \documentclass[STYLE]{acmart}
% \end{verbatim}

% Journals use one of three template styles. All but three ACM journals
% use the {\verb|acmsmall|} template style:
% \begin{itemize}
% \item {\texttt{acmsmall}}: The default journal template style.
% \item {\texttt{acmlarge}}: Used by JOCCH and TAP.
% \item {\texttt{acmtog}}: Used by TOG.
% \end{itemize}

% The majority of conference proceedings documentation will use the {\verb|acmconf|} template style.
% \begin{itemize}
% \item {\texttt{sigconf}}: The default proceedings template style.
% \item{\texttt{sigchi}}: Used for SIGCHI conference articles.
% \item{\texttt{sigplan}}: Used for SIGPLAN conference articles.
% \end{itemize}

% \subsection{Template Parameters}

% In addition to specifying the {\itshape template style} to be used in
% formatting your work, there are a number of {\itshape template parameters}
% which modify some part of the applied template style. A complete list
% of these parameters can be found in the {\itshape \LaTeX\ User's Guide.}

% Frequently-used parameters, or combinations of parameters, include:
% \begin{itemize}
% \item {\texttt{anonymous,review}}: Suitable for a ``double-anonymous''
%   conference submission. Anonymizes the work and includes line
%   numbers. Use with the \texttt{\string\acmSubmissionID} command to print the
%   submission's unique ID on each page of the work.
% \item{\texttt{authorversion}}: Produces a version of the work suitable
%   for posting by the author.
% \item{\texttt{screen}}: Produces colored hyperlinks.
% \end{itemize}

% This document uses the following string as the first command in the
% source file:
% \begin{verbatim}
% \documentclass[manuscript,screen,review]{acmart}
% \end{verbatim}

% \section{Modifications}

% Modifying the template --- including but not limited to: adjusting
% margins, typeface sizes, line spacing, paragraph and list definitions,
% and the use of the \verb|\vspace| command to manually adjust the
% vertical spacing between elements of your work --- is not allowed.

% {\bfseries Your document will be returned to you for revision if
%   modifications are discovered.}

% \section{Typefaces}

% The ``\verb|acmart|'' document class requires the use of the
% ``Libertine'' typeface family. Your \TeX\ installation should include
% this set of packages. Please do not substitute other typefaces. The
% ``\verb|lmodern|'' and ``\verb|ltimes|'' packages should not be used,
% as they will override the built-in typeface families.

% \section{Title Information}

% The title of your work should use capital letters appropriately -
% \url{https://capitalizemytitle.com/} has useful rules for
% capitalization. Use the {\verb|title|} command to define the title of
% your work. If your work has a subtitle, define it with the
% {\verb|subtitle|} command.  Do not insert line breaks in your title.

% If your title is lengthy, you must define a short version to be used
% in the page headers, to prevent overlapping text. The \verb|title|
% command has a ``short title'' parameter:
% \begin{verbatim}
%   \title[short title]{full title}
% \end{verbatim}

% \section{Authors and Affiliations}

% Each author must be defined separately for accurate metadata
% identification.  As an exception, multiple authors may share one
% affiliation. Authors' names should not be abbreviated; use full first
% names wherever possible. Include authors' e-mail addresses whenever
% possible.

% Grouping authors' names or e-mail addresses, or providing an ``e-mail
% alias,'' as shown below, is not acceptable:
% \begin{verbatim}
%   \author{Brooke Aster, David Mehldau}
%   \email{dave,judy,steve@university.edu}
%   \email{firstname.lastname@phillips.org}
% \end{verbatim}

% The \verb|authornote| and \verb|authornotemark| commands allow a note
% to apply to multiple authors --- for example, if the first two authors
% of an article contributed equally to the work.

% If your author list is lengthy, you must define a shortened version of
% the list of authors to be used in the page headers, to prevent
% overlapping text. The following command should be placed just after
% the last \verb|\author{}| definition:
% \begin{verbatim}
%   \renewcommand{\shortauthors}{McCartney, et al.}
% \end{verbatim}
% Omitting this command will force the use of a concatenated list of all
% of the authors' names, which may result in overlapping text in the
% page headers.

% The article template's documentation, available at
% \url{https://www.acm.org/publications/proceedings-template}, has a
% complete explanation of these commands and tips for their effective
% use.

% Note that authors' addresses are mandatory for journal articles.

% \section{Rights Information}

% Authors of any work published by ACM will need to complete a rights
% form. Depending on the kind of work, and the rights management choice
% made by the author, this may be copyright transfer, permission,
% license, or an OA (open access) agreement.

% Regardless of the rights management choice, the author will receive a
% copy of the completed rights form once it has been submitted. This
% form contains \LaTeX\ commands that must be copied into the source
% document. When the document source is compiled, these commands and
% their parameters add formatted text to several areas of the final
% document:
% \begin{itemize}
% \item the ``ACM Reference Format'' text on the first page.
% \item the ``rights management'' text on the first page.
% \item the conference information in the page header(s).
% \end{itemize}

% Rights information is unique to the work; if you are preparing several
% works for an event, make sure to use the correct set of commands with
% each of the works.

% The ACM Reference Format text is required for all articles over one
% page in length, and is optional for one-page articles (abstracts).

% \section{CCS Concepts and User-Defined Keywords}

% Two elements of the ``acmart'' document class provide powerful
% taxonomic tools for you to help readers find your work in an online
% search.

% The ACM Computing Classification System ---
% \url{https://www.acm.org/publications/class-2012} --- is a set of
% classifiers and concepts that describe the computing
% discipline. Authors can select entries from this classification
% system, via \url{https://dl.acm.org/ccs/ccs.cfm}, and generate the
% commands to be included in the \LaTeX\ source.

% User-defined keywords are a comma-separated list of words and phrases
% of the authors' choosing, providing a more flexible way of describing
% the research being presented.

% CCS concepts and user-defined keywords are required for for all
% articles over two pages in length, and are optional for one- and
% two-page articles (or abstracts).

% \section{Sectioning Commands}

% Your work should use standard \LaTeX\ sectioning commands:
% \verb|\section|, \verb|\subsection|, \verb|\subsubsection|,
% \verb|\paragraph|, and \verb|\subparagraph|. The sectioning levels up to
% \verb|\subsusection| should be numbered; do not remove the numbering
% from the commands.

% Simulating a sectioning command by setting the first word or words of
% a paragraph in boldface or italicized text is {\bfseries not allowed.}

% Below are examples of sectioning commands.

% \subsection{Subsection}
% \label{sec:subsection}

% This is a subsection.

% \subsubsection{Subsubsection}
% \label{sec:subsubsection}

% This is a subsubsection.

% \paragraph{Paragraph}

% This is a paragraph.

% \subparagraph{Subparagraph}

% This is a subparagraph.

% \section{Tables}

% The ``\verb|acmart|'' document class includes the ``\verb|booktabs|''
% package --- \url{https://ctan.org/pkg/booktabs} --- for preparing
% high-quality tables.

% Table captions are placed {\itshape above} the table.

% Because tables cannot be split across pages, the best placement for
% them is typically the top of the page nearest their initial cite.  To
% ensure this proper ``floating'' placement of tables, use the
% environment \textbf{table} to enclose the table's contents and the
% table caption.  The contents of the table itself must go in the
% \textbf{tabular} environment, to be aligned properly in rows and
% columns, with the desired horizontal and vertical rules.  Again,
% detailed instructions on \textbf{tabular} material are found in the
% \textit{\LaTeX\ User's Guide}.

% Immediately following this sentence is the point at which
% Table~\ref{tab:freq} is included in the input file; compare the
% placement of the table here with the table in the printed output of
% this document.

% \begin{table}
%   \caption{Frequency of Special Characters}
%   \label{tab:freq}
%   \begin{tabular}{ccl}
%     \toprule
%     Non-English or Math&Frequency&Comments\\
%     \midrule
%     \O & 1 in 1,000& For Swedish names\\
%     $\pi$ & 1 in 5& Common in math\\
%     \$ & 4 in 5 & Used in business\\
%     $\Psi^2_1$ & 1 in 40,000& Unexplained usage\\
%   \bottomrule
% \end{tabular}
% \end{table}

% To set a wider table, which takes up the whole width of the page's
% live area, use the environment \textbf{table*} to enclose the table's
% contents and the table caption.  As with a single-column table, this
% wide table will ``float'' to a location deemed more
% desirable. Immediately following this sentence is the point at which
% Table~\ref{tab:commands} is included in the input file; again, it is
% instructive to compare the placement of the table here with the table
% in the printed output of this document.

% \begin{table*}
%   \caption{Some Typical Commands}
%   \label{tab:commands}
%   \begin{tabular}{ccl}
%     \toprule
%     Command &A Number & Comments\\
%     \midrule
%     \texttt{{\char'134}author} & 100& Author \\
%     \texttt{{\char'134}table}& 300 & For tables\\
%     \texttt{{\char'134}table*}& 400& For wider tables\\
%     \bottomrule
%   \end{tabular}
% \end{table*}

% Always use midrule to separate table header rows from data rows, and
% use it only for this purpose. This enables assistive technologies to
% recognise table headers and support their users in navigating tables
% more easily.

% \section{Math Equations}
% You may want to display math equations in three distinct styles:
% inline, numbered or non-numbered display.  Each of the three are
% discussed in the next sections.

% \subsection{Inline (In-text) Equations}
% A formula that appears in the running text is called an inline or
% in-text formula.  It is produced by the \textbf{math} environment,
% which can be invoked with the usual
% \texttt{{\char'134}begin\,\ldots{\char'134}end} construction or with
% the short form \texttt{\$\,\ldots\$}. You can use any of the symbols
% and structures, from $\alpha$ to $\omega$, available in
% \LaTeX~\cite{Lamport:LaTeX}; this section will simply show a few
% examples of in-text equations in context. Notice how this equation:
% \begin{math}
%   \lim_{n\rightarrow \infty}x=0
% \end{math},
% set here in in-line math style, looks slightly different when
% set in display style.  (See next section).

% \subsection{Display Equations}
% A numbered display equation---one set off by vertical space from the
% text and centered horizontally---is produced by the \textbf{equation}
% environment. An unnumbered display equation is produced by the
% \textbf{displaymath} environment.

% Again, in either environment, you can use any of the symbols and
% structures available in \LaTeX\@; this section will just give a couple
% of examples of display equations in context.  First, consider the
% equation, shown as an inline equation above:
% \begin{equation}
%   \lim_{n\rightarrow \infty}x=0
% \end{equation}
% Notice how it is formatted somewhat differently in
% the \textbf{displaymath}
% environment.  Now, we'll enter an unnumbered equation:
% \begin{displaymath}
%   \sum_{i=0}^{\infty} x + 1
% \end{displaymath}
% and follow it with another numbered equation:
% \begin{equation}
%   \sum_{i=0}^{\infty}x_i=\int_{0}^{\pi+2} f
% \end{equation}
% just to demonstrate \LaTeX's able handling of numbering.

% \section{Figures}

% The ``\verb|figure|'' environment should be used for figures. One or
% more images can be placed within a figure. If your figure contains
% third-party material, you must clearly identify it as such, as shown
% in the example below.
% \begin{figure}[h]
%   \centering
%   \includegraphics[width=\linewidth]{sample-franklin}
%   \caption{1907 Franklin Model D roadster. Photograph by Harris \&
%     Ewing, Inc. [Public domain], via Wikimedia
%     Commons. (\url{https://goo.gl/VLCRBB}).}
%   \Description{A woman and a girl in white dresses sit in an open car.}
% \end{figure}

% Your figures should contain a caption which describes the figure to
% the reader.

% Figure captions are placed {\itshape below} the figure.

% Every figure should also have a figure description unless it is purely
% decorative. These descriptions convey what’s in the image to someone
% who cannot see it. They are also used by search engine crawlers for
% indexing images, and when images cannot be loaded.

% A figure description must be unformatted plain text less than 2000
% characters long (including spaces).  {\bfseries Figure descriptions
%   should not repeat the figure caption – their purpose is to capture
%   important information that is not already provided in the caption or
%   the main text of the paper.} For figures that convey important and
% complex new information, a short text description may not be
% adequate. More complex alternative descriptions can be placed in an
% appendix and referenced in a short figure description. For example,
% provide a data table capturing the information in a bar chart, or a
% structured list representing a graph.  For additional information
% regarding how best to write figure descriptions and why doing this is
% so important, please see
% \url{https://www.acm.org/publications/taps/describing-figures/}.

% \subsection{The ``Teaser Figure''}

% A ``teaser figure'' is an image, or set of images in one figure, that
% are placed after all author and affiliation information, and before
% the body of the article, spanning the page. If you wish to have such a
% figure in your article, place the command immediately before the
% \verb|\maketitle| command:
% \begin{verbatim}
%   \begin{teaserfigure}
%     \includegraphics[width=\textwidth]{sampleteaser}
%     \caption{figure caption}
%     \Description{figure description}
%   \end{teaserfigure}
% \end{verbatim}

% \section{Citations and Bibliographies}

% The use of \BibTeX\ for the preparation and formatting of one's
% references is strongly recommended. Authors' names should be complete
% --- use full first names (``Donald E. Knuth'') not initials
% (``D. E. Knuth'') --- and the salient identifying features of a
% reference should be included: title, year, volume, number, pages,
% article DOI, etc.

% The bibliography is included in your source document with these two
% commands, placed just before the \verb|\end{document}| command:
% \begin{verbatim}
%   \bibliographystyle{ACM-Reference-Format}
%   \bibliography{bibfile}
% \end{verbatim}
% where ``\verb|bibfile|'' is the name, without the ``\verb|.bib|''
% suffix, of the \BibTeX\ file.

% Citations and references are numbered by default. A small number of
% ACM publications have citations and references formatted in the
% ``author year'' style; for these exceptions, please include this
% command in the {\bfseries preamble} (before the command
% ``\verb|\begin{document}|'') of your \LaTeX\ source:
% \begin{verbatim}
%   \citestyle{acmauthoryear}
% \end{verbatim}


%   Some examples.  A paginated journal article \cite{Abril07}, an
%   enumerated journal article \cite{Cohen07}, a reference to an entire
%   issue \cite{JCohen96}, a monograph (whole book) \cite{Kosiur01}, a
%   monograph/whole book in a series (see 2a in spec. document)
%   \cite{Harel79}, a divisible-book such as an anthology or compilation
%   \cite{Editor00} followed by the same example, however we only output
%   the series if the volume number is given \cite{Editor00a} (so
%   Editor00a's series should NOT be present since it has no vol. no.),
%   a chapter in a divisible book \cite{Spector90}, a chapter in a
%   divisible book in a series \cite{Douglass98}, a multi-volume work as
%   book \cite{Knuth97}, a couple of articles in a proceedings (of a
%   conference, symposium, workshop for example) (paginated proceedings
%   article) \cite{Andler79, Hagerup1993}, a proceedings article with
%   all possible elements \cite{Smith10}, an example of an enumerated
%   proceedings article \cite{VanGundy07}, an informally published work
%   \cite{Harel78}, a couple of preprints \cite{Bornmann2019,
%     AnzarootPBM14}, a doctoral dissertation \cite{Clarkson85}, a
%   master's thesis: \cite{anisi03}, an online document / world wide web
%   resource \cite{Thornburg01, Ablamowicz07, Poker06}, a video game
%   (Case 1) \cite{Obama08} and (Case 2) \cite{Novak03} and \cite{Lee05}
%   and (Case 3) a patent \cite{JoeScientist001}, work accepted for
%   publication \cite{rous08}, 'YYYYb'-test for prolific author
%   \cite{SaeediMEJ10} and \cite{SaeediJETC10}. Other cites might
%   contain 'duplicate' DOI and URLs (some SIAM articles)
%   \cite{Kirschmer:2010:AEI:1958016.1958018}. Boris / Barbara Beeton:
%   multi-volume works as books \cite{MR781536} and \cite{MR781537}. A
%   couple of citations with DOIs:
%   \cite{2004:ITE:1009386.1010128,Kirschmer:2010:AEI:1958016.1958018}. Online
%   citations: \cite{TUGInstmem, Thornburg01, CTANacmart}.
%   Artifacts: \cite{R} and \cite{UMassCitations}.

% \section{Acknowledgments}

% Identification of funding sources and other support, and thanks to
% individuals and groups that assisted in the research and the
% preparation of the work should be included in an acknowledgment
% section, which is placed just before the reference section in your
% document.

% This section has a special environment:
% \begin{verbatim}
%   \begin{acks}
%   ...
%   \end{acks}
% \end{verbatim}
% so that the information contained therein can be more easily collected
% during the article metadata extraction phase, and to ensure
% consistency in the spelling of the section heading.

% Authors should not prepare this section as a numbered or unnumbered {\verb|\section|}; please use the ``{\verb|acks|}'' environment.

% \section{Appendices}

% If your work needs an appendix, add it before the
% ``\verb|\end{document}|'' command at the conclusion of your source
% document.

% Start the appendix with the ``\verb|appendix|'' command:
% \begin{verbatim}
%   \appendix
% \end{verbatim}
% and note that in the appendix, sections are lettered, not
% numbered. This document has two appendices, demonstrating the section
% and subsection identification method.

% \section{Multi-language papers}

% Papers may be written in languages other than English or include
% titles, subtitles, keywords and abstracts in different languages (as a
% rule, a paper in a language other than English should include an
% English title and an English abstract).  Use \verb|language=...| for
% every language used in the paper.  The last language indicated is the
% main language of the paper.  For example, a French paper with
% additional titles and abstracts in English and German may start with
% the following command
% \begin{verbatim}
% \documentclass[sigconf, language=english, language=german,
%                language=french]{acmart}
% \end{verbatim}

% The title, subtitle, keywords and abstract will be typeset in the main
% language of the paper.  The commands \verb|\translatedXXX|, \verb|XXX|
% begin title, subtitle and keywords, can be used to set these elements
% in the other languages.  The environment \verb|translatedabstract| is
% used to set the translation of the abstract.  These commands and
% environment have a mandatory first argument: the language of the
% second argument.  See \verb|sample-sigconf-i13n.tex| file for examples
% of their usage.

% \section{SIGCHI Extended Abstracts}

% The ``\verb|sigchi-a|'' template style (available only in \LaTeX\ and
% not in Word) produces a landscape-orientation formatted article, with
% a wide left margin. Three environments are available for use with the
% ``\verb|sigchi-a|'' template style, and produce formatted output in
% the margin:
% \begin{description}
% \item[\texttt{sidebar}:]  Place formatted text in the margin.
% \item[\texttt{marginfigure}:] Place a figure in the margin.
% \item[\texttt{margintable}:] Place a table in the margin.
% \end{description}

%%
%% The acknowledgments section is defined using the "acks" environment
%% (and NOT an unnumbered section). This ensures the proper
%% identification of the section in the article metadata, and the
%% consistent spelling of the heading.
\begin{acks}
This research is supported by the National Research Foundation, Singapore, and the Cyber Security Agency of Singapore under its National Cybersecurity R\&D Programme (Proposal ID: NCR25-DeSCEmT-SMU). Any opinions, findings and conclusions or recommendations expressed in this material are those of the author(s) and do not reflect the views of the National Research Foundation, Singapore, and the Cyber Security Agency of Singapore.
\end{acks}

%%
%% The next two lines define the bibliography style to be used, and
%% the bibliography file.
\bibliographystyle{ACM-Reference-Format}
\bibliography{sample-base}


%%
%% If your work has an appendix, this is the place to put it.
% \appendix

% \section{Research Methods}

% \subsection{Part One}

% Lorem ipsum dolor sit amet, consectetur adipiscing elit. Morbi
% malesuada, quam in pulvinar varius, metus nunc fermentum urna, id
% sollicitudin purus odio sit amet enim. Aliquam ullamcorper eu ipsum
% vel mollis. Curabitur quis dictum nisl. Phasellus vel semper risus, et
% lacinia dolor. Integer ultricies commodo sem nec semper.

% \subsection{Part Two}

% Etiam commodo feugiat nisl pulvinar pellentesque. Etiam auctor sodales
% ligula, non varius nibh pulvinar semper. Suspendisse nec lectus non
% ipsum convallis congue hendrerit vitae sapien. Donec at laoreet
% eros. Vivamus non purus placerat, scelerisque diam eu, cursus
% ante. Etiam aliquam tortor auctor efficitur mattis.

% \section{Online Resources}

% Nam id fermentum dui. Suspendisse sagittis tortor a nulla mollis, in
% pulvinar ex pretium. Sed interdum orci quis metus euismod, et sagittis
% enim maximus. Vestibulum gravida massa ut felis suscipit
% congue. Quisque mattis elit a risus ultrices commodo venenatis eget
% dui. Etiam sagittis eleifend elementum.

% Nam interdum magna at lectus dignissim, ac dignissim lorem
% rhoncus. Maecenas eu arcu ac neque placerat aliquam. Nunc pulvinar
% massa et mattis lacinia.

\end{document}
\endinput
%%
%% End of file `sample-manuscript.tex'.
