\begin{figure}
    \centering
    \begin{subfigure}[b]{0.24\textwidth}
        \centering
        \includegraphics[width=\textwidth]{figures/observation/observation3/qk_direction_plot.pdf}
        \caption{Query-Key Similarity}
        \label{fig:qk_sim}
    \end{subfigure}
    \hfill
    \begin{subfigure}[b]{0.24\textwidth}
        \centering
        \includegraphics[width=\textwidth]{figures/observation/observation3/k_norm_plot.pdf}
        \caption{Key Magnitude}
        \label{fig:norm_k}
    \end{subfigure}
    \hfill
    \begin{subfigure}[b]{0.24\textwidth}
        \centering
        \includegraphics[width=\textwidth]{figures/observation/observation3/v_norm_plot.pdf}
        \caption{Value Magnitude}
        \label{fig:norm_v}
    \end{subfigure}
    \hfill
    \begin{subfigure}[b]{0.24\textwidth}
        \centering
        \includegraphics[width=\textwidth]{figures/observation/observation3/attn_score.pdf}
        \caption{Query-Key Product}
        \label{fig:attn_score}
    \end{subfigure}
    \caption{\textbf{Visualization of Different Components in Attention.} \textbf{(a)} 
The cosine similarity between query and key states increases as the distance between their positions decreases. \textbf{(b)} The magnitudes of key states show a slowly upward trend as position increases. \textbf{(c)} The magnitude of value states remain constant across positions. \textbf{(d)} Query-key dot products keep consistently low values except at initial and recent positions. A red dashed line marks the anomalous region for the first two tokens in all figures. The X-axis shows positions of KV states on a log scale. Results are measured with the \textsc{LLaMA-3-8B-Instruct} model. Visualizations and analyses for more base models are provided in Appendix~\ref{app:obs2}.}
    \label{fig:norm}
\end{figure}