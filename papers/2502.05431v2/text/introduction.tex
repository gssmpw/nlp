\section{Introduction}
Recent advances in context-augmented generation (CAG) techniques, particularly retrieval-augmented generation (RAG)~\citep{gupta2024rag, gao2023retrieval} and in-context learning (ICL)~\citep{dong2022survey, wei2022emergent}, have been widely adopted in large language models (LLMs)~\citep{llama3, achiam2023gpt}, improving their ability to generalize to unseen tasks with contextual information, as demonstrated in Figure~\ref{fig:intro} (top).
%These data includes the latest knowledge~\citep{jiang2023active}, illustrative examples~\citep{liu2021makes}, and user-specific information~\citep{shanahan2023role}. 
These techniques employ a \textit{sequential encoding} process to ground LLM inputs with knowledge from external sources: concatenating the retrieved texts into one sequence, and encoding the sequence into key-value (KV) states as the context for subsequent queries. While this new, significantly longer input improves performance, the increased latency in context prefilling becomes a bottleneck in tasks that require long inputs but generate short outputs~\citep{bai2023longbench, agarwal2024many, jiang2024longrag}. For example, prefilling a 128K context takes 17 seconds, whereas generating 256 tokens requires only 6 seconds. This discrepancy leaves significant room to improve the practical efficiency of CAG systems in real-world deployments~\citep{Liu_LlamaIndex_2022, Chase_Longchain_2022}.

Since texts for CAG are typically stored independently in external databases~\citep{Qdrant, douze2024faiss}, pre-caching all these texts for direct loading during inference offers a brute-force approach to accelerate CAG. However, for autoregressive LLMs, the KV states are inherently context-dependent. This dependency makes naive pre-caching impractical, as it would require caching all possible context permutations, leading to factorial growth in memory requirements as the database size increases. For instance, caching all permutations of just ten 256-token text chunks for the \textsc{LLaMA-3-8B} model would demand an impractical 22 PB of memory.

To address this issue, \textit{parallel encoding}~\citep{ratner2022parallel, yen2024long, li2024focusllm, Sun2024BlockAttentionFE} is introduced to encode each context into KV states separately, ensuring that tokens from different contexts cannot attend to each other during encoding. Next, the on-the-fly generation starts by prefilling user queries, which can attend to the cached KV states from all contexts without re-encoding, offering two benefits:

\textbf{Pre-caching Contexts for Fast Inference:} Texts from external sources can be pre-computed and cached into KV states, which serve as contexts for direct loading during inference. Additionally, this approach allows for cost-free manipulation of contexts, including operations like insertion, deletion, replacement, and swapping.

\textbf{Re-using Positions for Long Context:} Contexts can be inserted into the same range of positions in an LLM's context window, allowing for more and longer context chunks. It also mitigates the problem of ``lost in the middle" in context ordering~\citep{liu2024lost}, as each context is equally ``close'' to the generated tokens. 

\section{Overview}

\revision{In this section, we first explain the foundational concept of Hausdorff distance-based penetration depth algorithms, which are essential for understanding our method (Sec.~\ref{sec:preliminary}).
We then provide a brief overview of our proposed RT-based penetration depth algorithm (Sec.~\ref{subsec:algo_overview}).}



\section{Preliminaries }
\label{sec:Preliminaries}

% Before we introduce our method, we first overview the important basics of 3D dynamic human modeling with Gaussian splatting. Then, we discuss the diffusion-based 3d generation techniques, and how they can be applied to human modeling.
% \ZY{I stopp here. TBC.}
% \subsection{Dynamic human modeling with Gaussian splatting}
\subsection{3D Gaussian Splatting}
3D Gaussian splatting~\cite{kerbl3Dgaussians} is an explicit scene representation that allows high-quality real-time rendering. The given scene is represented by a set of static 3D Gaussians, which are parameterized as follows: Gaussian center $x\in {\mathbb{R}^3}$, color $c\in {\mathbb{R}^3}$, opacity $\alpha\in {\mathbb{R}}$, spatial rotation in the form of quaternion $q\in {\mathbb{R}^4}$, and scaling factor $s\in {\mathbb{R}^3}$. Given these properties, the rendering process is represented as:
\begin{equation}
  I = Splatting(x, c, s, \alpha, q, r),
  \label{eq:splattingGA}
\end{equation}
where $I$ is the rendered image, $r$ is a set of query rays crossing the scene, and $Splatting(\cdot)$ is a differentiable rendering process. We refer readers to Kerbl et al.'s paper~\cite{kerbl3Dgaussians} for the details of Gaussian splatting. 



% \ZY{I would suggest move this part to the method part.}
% GaissianAvatar is a dynamic human generation model based on Gaussian splitting. Given a sequence of RGB images, this method utilizes fitted SMPLs and sampled points on its surface to obtain a pose-dependent feature map by a pose encoder. The pose-dependent features and a geometry feature are fed in a Gaussian decoder, which is employed to establish a functional mapping from the underlying geometry of the human form to diverse attributes of 3D Gaussians on the canonical surfaces. The parameter prediction process is articulated as follows:
% \begin{equation}
%   (\Delta x,c,s)=G_{\theta}(S+P),
%   \label{eq:gaussiandecoder}
% \end{equation}
%  where $G_{\theta}$ represents the Gaussian decoder, and $(S+P)$ is the multiplication of geometry feature S and pose feature P. Instead of optimizing all attributes of Gaussian, this decoder predicts 3D positional offset $\Delta{x} \in {\mathbb{R}^3}$, color $c\in\mathbb{R}^3$, and 3D scaling factor $ s\in\mathbb{R}^3$. To enhance geometry reconstruction accuracy, the opacity $\alpha$ and 3D rotation $q$ are set to fixed values of $1$ and $(1,0,0,0)$ respectively.
 
%  To render the canonical avatar in observation space, we seamlessly combine the Linear Blend Skinning function with the Gaussian Splatting~\cite{kerbl3Dgaussians} rendering process: 
% \begin{equation}
%   I_{\theta}=Splatting(x_o,Q,d),
%   \label{eq:splatting}
% \end{equation}
% \begin{equation}
%   x_o = T_{lbs}(x_c,p,w),
%   \label{eq:LBS}
% \end{equation}
% where $I_{\theta}$ represents the final rendered image, and the canonical Gaussian position $x_c$ is the sum of the initial position $x$ and the predicted offset $\Delta x$. The LBS function $T_{lbs}$ applies the SMPL skeleton pose $p$ and blending weights $w$ to deform $x_c$ into observation space as $x_o$. $Q$ denotes the remaining attributes of the Gaussians. With the rendering process, they can now reposition these canonical 3D Gaussians into the observation space.



\subsection{Score Distillation Sampling}
Score Distillation Sampling (SDS)~\cite{poole2022dreamfusion} builds a bridge between diffusion models and 3D representations. In SDS, the noised input is denoised in one time-step, and the difference between added noise and predicted noise is considered SDS loss, expressed as:

% \begin{equation}
%   \mathcal{L}_{SDS}(I_{\Phi}) \triangleq E_{t,\epsilon}[w(t)(\epsilon_{\phi}(z_t,y,t)-\epsilon)\frac{\partial I_{\Phi}}{\partial\Phi}],
%   \label{eq:SDSObserv}
% \end{equation}
\begin{equation}
    \mathcal{L}_{\text{SDS}}(I_{\Phi}) \triangleq \mathbb{E}_{t,\epsilon} \left[ w(t) \left( \epsilon_{\phi}(z_t, y, t) - \epsilon \right) \frac{\partial I_{\Phi}}{\partial \Phi} \right],
  \label{eq:SDSObservGA}
\end{equation}
where the input $I_{\Phi}$ represents a rendered image from a 3D representation, such as 3D Gaussians, with optimizable parameters $\Phi$. $\epsilon_{\phi}$ corresponds to the predicted noise of diffusion networks, which is produced by incorporating the noise image $z_t$ as input and conditioning it with a text or image $y$ at timestep $t$. The noise image $z_t$ is derived by introducing noise $\epsilon$ into $I_{\Phi}$ at timestep $t$. The loss is weighted by the diffusion scheduler $w(t)$. 
% \vspace{-3mm}

\subsection{Overview of the RTPD Algorithm}\label{subsec:algo_overview}
Fig.~\ref{fig:Overview} presents an overview of our RTPD algorithm.
It is grounded in the Hausdorff distance-based penetration depth calculation method (Sec.~\ref{sec:preliminary}).
%, similar to that of Tang et al.~\shortcite{SIG09HIST}.
The process consists of two primary phases: penetration surface extraction and Hausdorff distance calculation.
We leverage the RTX platform's capabilities to accelerate both of these steps.

\begin{figure*}[t]
    \centering
    \includegraphics[width=0.8\textwidth]{Image/overview.pdf}
    \caption{The overview of RT-based penetration depth calculation algorithm overview}
    \label{fig:Overview}
\end{figure*}

The penetration surface extraction phase focuses on identifying the overlapped region between two objects.
\revision{The penetration surface is defined as a set of polygons from one object, where at least one of its vertices lies within the other object. 
Note that in our work, we focus on triangles rather than general polygons, as they are processed most efficiently on the RTX platform.}
To facilitate this extraction, we introduce a ray-tracing-based \revision{Point-in-Polyhedron} test (RT-PIP), significantly accelerated through the use of RT cores (Sec.~\ref{sec:RT-PIP}).
This test capitalizes on the ray-surface intersection capabilities of the RTX platform.
%
Initially, a Geometry Acceleration Structure (GAS) is generated for each object, as required by the RTX platform.
The RT-PIP module takes the GAS of one object (e.g., $GAS_{A}$) and the point set of the other object (e.g., $P_{B}$).
It outputs a set of points (e.g., $P_{\partial B}$) representing the penetration region, indicating their location inside the opposing object.
Subsequently, a penetration surface (e.g., $\partial B$) is constructed using this point set (e.g., $P_{\partial B}$) (Sec.~\ref{subsec:surfaceGen}).
%
The generated penetration surfaces (e.g., $\partial A$ and $\partial B$) are then forwarded to the next step. 

The Hausdorff distance calculation phase utilizes the ray-surface intersection test of the RTX platform (Sec.~\ref{sec:RT-Hausdorff}) to compute the Hausdorff distance between two objects.
We introduce a novel Ray-Tracing-based Hausdorff DISTance algorithm, RT-HDIST.
It begins by generating GAS for the two penetration surfaces, $P_{\partial A}$ and $P_{\partial B}$, derived from the preceding step.
RT-HDIST processes the GAS of a penetration surface (e.g., $GAS_{\partial A}$) alongside the point set of the other penetration surface (e.g., $P_{\partial B}$) to compute the penetration depth between them.
The algorithm operates bidirectionally, considering both directions ($\partial A \to \partial B$ and $\partial B \to \partial A$).
The final penetration depth between the two objects, A and B, is determined by selecting the larger value from these two directional computations.

%In the Hausdorff distance calculation step, we compute the Hausdorff distance between given two objects using a ray-surface-intersection test. (Sec.~\ref{sec:RT-Hausdorff}) Initially, we construct the GAS for both $\partial A$ and $\partial B$ to utilize the RT-core effectively. The RT-based Hausdorff distance algorithms then determine the Hausdorff distance by processing the GAS of one object (e.g. $GAS_{\partial A}$) and set of the vertices of the other (e.g. $P_{\partial B}$). Following the Hausdorff distance definition (Eq.~\ref{equation:hausdorff_definition}), we compute the Hausdorff distance to both directions ($\partial A \to \partial B$) and ($\partial B \to \partial A$). As a result, the bigger one is the final Hausdorff distance, and also it is the penetration depth between input object $A$ and $B$.


%the proposed RT-based penetration depth calculation pipeline.
%Our proposed methods adopt Tang's Hausdorff-based penetration depth methods~\cite{SIG09HIST}. The pipeline is divided into the penetration surface extraction step and the Hausdorff distance calculation between the penetration surface steps. However, since Tang's approach is not suitable for the RT platform in detail, we modified and applied it with appropriate methods.

%The penetration surface extraction step is extracting overlapped surfaces on other objects. To utilize the RT core, we use the ray-intersection-based PIP(Point-In-Polygon) algorithms instead of collision detection between two objects which Tang et al.~\cite{SIG09HIST} used. (Sec.~\ref{sec:RT-PIP})
%RT core-based PIP test uses a ray-surface intersection test. For purpose this, we generate the GAS(Geometry Acceleration Structure) for each object. RT core-based PIP test takes the GAS of one object (e.g. $GAS_{A}$) and a set of vertex of another one (e.g. $P_{B}$). Then this computes the penetrated vertex set of another one (e.g. $P_{\partial B}$). To calculate the Hausdorff distance, these vertex sets change to objects constructed by penetrated surface (e.g. $\partial B$). Finally, the two generated overlapped surface objects $\partial A$ and $\partial B$ are used in the Hausdorff distance calculation step.

Despite these advantages, parallel encoding leads to significant performance degradation across multiple RAG and ICL scenarios, as shown in Figure \ref{fig:obseravtion1}, with average declines of 4.9\% (despite using 2-10$\times$ more contexts) and 49.0\%, respectively. While prior works~\citep{Sun2024BlockAttentionFE, yen2024long} have attempted to correct this with fine-tuning, these methods continue to exhibit reduced accuracy in reasoning tasks (e.g., GSM8K). This decrease arises from the limited generalization capability of models fine-tuned on simple tasks to complex ones.

However, our results in Figure~\ref{fig:obseravtion1} also reveal that parallel encoding holds promise, as LLMs can still generate reasonable responses due to their inherent alignments with sequential encoding. Based on this observation, we aim to strengthen these alignments while addressing the remaining discrepancies to achieve more accurate parallel encoding. Our insight from Figure~\ref{fig:observation2} and Figure~\ref{fig:norm} is that \textit{KV states from independent contexts can be naturally merged into one sequence due to their similarity in direction and magnitude, attributed to the presence of an attention sink}~\citep{xiao2023efficient}. This observation reduces our challenge to addressing residual misalignments, which manifest as anomalous distributions at the initial and recent positions within each context.

Motivated by this, we propose \underline{A}daptive \underline{P}arallel \underline{E}ncoding (\textbf{APE}) to align the distribution between sequential and parallel encoding, which enables accurate and fast CAG (see Figure~\ref{fig:intro} (Bottom)). Our contributions involve:

\begin{itemize}
[itemsep=0.0pt,topsep=0pt,leftmargin=*]
\item We systematically analyze the distribution properties of attention weights in parallel encoding, focusing on the magnitude and direction of KV states across various samples and positions. Our observations identify major alignments and minor misalignments between parallel and sequential encoding for further improvement.
\item We propose APE to recover the accuracy of parallel encoding with three alignment steps: (i) Prepend a shared prefix to avoid the duplication of abnormal distribution of initial tokens. (ii) Adjust a lower attention temperature to sharpen the distribution, focusing on contextually important tokens. (iii) Apply a scaling factor to offset the increase in the magnitude of the LogSumExp value of attention scores from the context.
\item  We empirically show that (i) APE maintains 98\% and 93\% of the sequential encoding performance in RAG and ICL tasks, respectively. (ii) APE outperforms parallel encoding in RAG and ICL, yielding improvements of 3.6\% and 7.9\%, respectively. (iii) APE scales to handle hundreds of contexts in parallel, matching or exceeding sequential encoding in many-shot scenarios. (iv) APE accelerates long-context generation, achieving up to 4.5$\times$ speedup through a 28$\times$ reduction in prefilling time for a context including 128K tokens.
\end{itemize} 