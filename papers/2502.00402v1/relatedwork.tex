\section{Related Work}
Accurate accident detection is crucial for improving road safety and emergency response times. However, existing methods have not been extensively tested on real traffic data from designated test sites. Real-world accident datasets \cite{xu2022tad} are rare and often lack sufficient data to train deep learning models effectively \cite{liu2024survey}. In this section, we categorize accident detection approaches into different groups based on their methodologies and data sources.

\subsection{Rule-Based Accident Detection}
Rule-based methods \cite{pirdavani_application_2015} rely on predefined traffic rules and heuristics to identify accidents. These approaches typically use sensor readings, such as abrupt changes in speed or sudden braking, to infer collisions. While effective in controlled environments, rule-based systems struggle with unseen accident scenarios and complex traffic settings.

\subsection{Learning-Based Accident Detection}
Machine learning models, particularly deep learning approaches, have been increasingly used for accident detection. These methods use large datasets to learn patterns associated with crashes \cite{zahid2024datadriven, adewopo2024smart}. Supervised learning techniques train models on labeled accident and non-accident data, whereas unsupervised approaches \cite{yao2023dota} detect anomalies that may indicate collisions. Despite their effectiveness, these models require extensive labeled data, which is often difficult to obtain.

\subsection{Trajectory-Based Accident Detection}
Trajectory-based methods \cite{yang_freeway_2021} analyze the motion patterns of vehicles to detect unusual behavior that may lead to accidents. These techniques use sensor data to track vehicles in 2D or 3D and identify sudden deviations, abrupt stops, or erratic lane changes. While useful for predicting potential accidents, trajectory-based methods may struggle with false positives and ID switches in highly occluded traffic scenarios.

\subsection{Vision-Based Accident Detection}
Vision-based approaches \cite{fang_vision-based_2024} utilize camera footage to detect accidents by identifying visual cues such as vehicle deformations, smoke, or fire. These methods often employ convolutional neural networks (CNNs) to classify accident and non-accident scenarios. Although vision-based detection can provide valuable real-time insights, it is sensitive to lighting conditions, occlusions, and camera angles.

\subsection{Video-Based Accident Detection}
Video-based methods \cite{maaloul_adaptive_2017} extend image-based approaches by analyzing temporal information in video streams. These techniques apply deep learning models such as recurrent neural networks (RNNs) or transformers to track accident progression over time. Video-based detection enhances accuracy by considering motion dynamics but requires substantial computational resources and well-annotated datasets.

\subsection{Accident Detection Using Synthetic Data}
To overcome the scarcity of real-world accident data, synthetic datasets have been developed for training and evaluating accident detection models. One notable example is DeepAccident \cite{wang2024deepaccident}, which contains 691 synthetic accident scenarios generated in the CARLA simulator \cite{dosovitskiy_carla_2017}. These accidents are based on crash reports from the National Highway Traffic Safety Administration (NHTSA) and include labeled data from four vehicles and one roadside infrastructure camera. However, a key limitation of DeepAccident is its reliance on simulated environments, which may not fully capture the complexities of real-world crashes. Addressing the sim-to-real gap is essential to improve the generalization of perception models trained on synthetic data.

\subsection{Accident Detection Using Real Data}
Real-world accident detection datasets are limited due to the difficulty of capturing and annotating crash events. Some existing datasets contain dashcam footage, roadside surveillance videos, or event logs from connected vehicles. However, many of these datasets lack sufficient labeled data to train deep learning models effectively. Furthermore, variations in road conditions, weather, and camera perspectives introduce additional challenges in real-world accident detection.

In summary, while significant progress has been made in accident detection, the scarcity of real-world datasets remains a major challenge. 
Related approaches may be used to detect both accidents and general traffic anomalies \cite{greer2024towards, greer2023pedestrian}. 
Existing methods often struggle with generalization due to limited real-world training data or the reliance on synthetic environments. To address this gap, our accident detection framework integrates both rule-based and learning-based approaches to reliably identify accidents in 3D. By leveraging real-world roadside sensor data and combining explicit traffic rules with deep learning models, our framework enhances detection accuracy, improves robustness across diverse traffic conditions, and enables faster emergency response times.