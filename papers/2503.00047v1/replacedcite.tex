\section{Related Work}
In this section, we review relevant work on point cloud attribute quality enhancement, focusing on two approaches: traditional methods and deep learning-based methods.
\subsection{Traditional methods for quality enhancement}
Traditional methods for quality enhancement can be roughly divided into two classes: traditional filter-based algorithms and graph signal processing ____-based algorithms.

Traditional filter-based algorithms: Wang et al. ____ introduced the Kalman filter for the reconstructed attributes in G-PCC, where the PredLift configuration is activated for attribute compression. This not only improves the quality of the reconstructed attributes but also effectively enhances prediction accuracy. However, the Kalman filter is highly sensitive to signal stationarity, making it primarily effective for chrominance components (i.e., Cb and Cr). To address this, Wiener filter-based methods ____ were subsequently proposed to effectively mitigate distortion accumulation during the coding process and enhance reconstruction quality. While Wiener filters offer certain advantages, their effectiveness is limited by the assumption of linear distortion. 

Graph signal processing algorithms: Yamamoto et al. ____ proposed a deblurring algorithm for point cloud attributes, inspired by the multi-Wiener SURE-LET deconvolution method ____ used in image processing. This algorithm models blurred textures as graph signals, where the deblurring process begins with Wiener-like filtering, followed by sub-band decomposition and thresholding, to enhance the quality of deblurred point clouds. Dinesh et al. ____ proposed two algorithms for 3D point cloud color denoising: one based on graph Laplacian regularization (GLR) and the other on graph total variation (GTV) priors, formulating the denoising problem as a maximum a posteriori estimation and solving it using conjugate gradient for GLR and alternating direction method of multipliers with proximal gradient descent for GTV. Watanabe et al. ____ proposed a point cloud color denoising method based on 3D patch similarity, which generates a more robust graph structure by calculating the similarity of 3D patches around connected points, and designed a low-pass filter that automatically selects the frequency response based on the estimated noise level. Later, they ____ proposed a fast graph-based denoising method that achieves efficient denoising on large-scale point clouds by employing scan-line neighborhood search for high-speed graph construction, covariance matrix eigenvalues for fast noise estimation, and low-cost filter selection. However, the time complexity and the difficulty of constructing an appropriate graph remain significant challenges for the aforementioned methods.
\vspace{-0.4cm}
\subsection{Deep learning-based methods for quality enhancement}
Deep learning-based approaches for improving the attribute quality of distorted point clouds can be broadly categorized into three methods: graph convolution-based, sparse convolution-based, and projection-based techniques. 

Graph convolution-based methods: Sheng et al. ____ proposed a multi-scale graph attention network to remove artifacts in point cloud attributes compressed by G-PCC. They constructed a graph based on geometry coordinates and applied Chebyshev graph convolutions to extract feature information from point cloud attributes. Xing et al. ____ introduced a graph-based quality enhancement network that uses geometry information as an auxiliary input. By using graph convolution, the network efficiently extracts local features and can handle point clouds with varying levels of distortion using a single pre-trained model. 

Sparse convolution-based methods: Sparse 3D convolution has proven effective for improving computational efficiency and reducing memory consumption. Liu et al. ____ proposed a quality enhancement method for dynamic point clouds based on inter-frame motion prediction. Their method includes an inter-frame motion prediction module with relative position encoding and motion consistency to align the current frame with the reference frame. Ding et al. ____ developed a learning-based adaptive in-loop filter for G-PCC. Zhang et al. ____ proposed a method to improve the reconstruction quality of both geometry and attributes. This approach first uses linear interpolation to densify the decoded geometry, creating a continuous surface. Then, a Gaussian distance-weighted mapping is used to recolor the enhanced geometry, which is further refined by an attribute enhancement network. Later, they ____ proposed a fully data-driven method and a rules-unrolling-based optimization to restore G-PCC compressed point cloud attributes.

Projection-based methods: Gao et al. ____ proposed an occupancy-assisted compression artifact reduction method for V-PCC, which takes the occupancy information as a prior knowledge to guide the network to focus on learning the attribute distortions of the occupied regions. Xing et al. ____ proposed a U-Net-based quality enhancement method for color attributes of dense 3D point clouds. To mitigate the hole artifacts caused by trisoup-based geometry compression in G-PCC, Tao et al. ____ proposed a multi-view projection-based joint geometry and color hole repairing method. In this approach, irregular points are converted into regular pixels on 2D planes through multi-view projection. They also design a multi-view projection-based triangular hole detection scheme based on depth distribution to effectively repair the holes in both geometry and color.

However, the aforementioned methods primarily focus on point-to-point data fidelity and overlook the perceptual quality of human visual system. As a result, these methods often yield suboptimal performance in terms of providing high-quality visual reconstruction. To address this challenge, we propose an attribute quality enhancement algorithm based on optimal transport theory. Our approach aims to find the optimal solution that simultaneously enhances both data fidelity and subjective quality.

\begin{figure}
\centering
\includegraphics[width=3.2in]{FIG1.png}
\caption{Problem description. For an original point cloud \(\bm{P}\) and an initial reconstructed point cloud \(\hat{\bm{P}}\) with severe distortion caused by compression, we aim to obtain an enhanced point cloud \(\tilde{\bm{P}}\) from \(\hat{\bm{P}}\) through an enhancement network \(e(\cdot)\).}
\label{FIG1}
\end{figure}