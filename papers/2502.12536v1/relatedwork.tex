\section{Related work}
In neural decoding, some studies often use supervised algorithms to achieve good decoding predictions, such as finger position decoding \cite{wallisch2014matlab,wu2019neural} and a rat's movement trajectory decoding \cite{Glaser2020machine}. Due to the temporal characteristics of neural decoding signals, temporal decoding algorithms have received much attention. Firstly, state-space models (SSMs) with associations between the current state and the previous state have become more developed \cite{gilja2012a,wallisch2014matlab,wu2019neural}. 
In addition, deep learning has been the focus of numerous studies, such as LSTM (Long Short-Term Memory) \cite{elango2016sequence,pan2018rapid}. Further, the source of the recorded neural activity can change from day to day, e.g., due to a slight movement of the implanted electrodes. The proposed multiplicative RNN (Recurrent Neural Network) allows mappings from the neural input to the motor output to partially change from neural activity \cite{sussillo2018making}. The decoding performed by these methods is more accurate than that of the traditional methods. 

In recent years, some researchers have proposed a novel weakly supervised method different from previous studies. This method is mainly based on a symmetric pattern discovered from decoding brain neural data between the unsupervised decoding positions and the ground-truth positions. This method has achieved decoding predictions that are much higher than those of unsupervised methods and close to those of supervised methods \cite{feng2018neural,feng2020weakly}. Based on these studies, a general framework for refining weakly supervised system has been proposed, which has been algorithmically validated \cite{feng2021vif} and theoretically justified from machine learning, neuroscience, cognitive science, and more \cite{feng2023grid}. Due to the complexity and difficulty in understanding this framework, further research is needed to explore the generation of symmetric mechanisms and the processing details of data in this system for the development of this framework.