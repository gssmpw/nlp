%!TEX root = ../root.tex

\let\labelindent\relax

\usepackage{multicol}

\makeatletter
\let\NAT@parse\undefined
\makeatother

% My includes
\usepackage{comment}
	\usepackage{amsmath}
	\usepackage{verbatim} % multi-line comments
	\usepackage{amsfonts}
	\usepackage{amssymb}
	\usepackage{ragged2e}
	\usepackage{graphicx}
	\usepackage{cancel}
	\usepackage{mathtools}
	\usepackage{tabularx}
     \usepackage{adjustbox} % vertical alignment of figures
	\usepackage{arydshln}
	\usepackage{tensor}
	\usepackage{array}
	\usepackage[dvipsnames]{xcolor}
	\usepackage{listings}
	\usepackage{textcomp}
	% \usepackage[pdf,tmpdir,singlefile]{graphviz}
	\usepackage{mathrsfs}
	\usepackage{bbm}
	\usepackage{tikz}
	\usepackage{tikz-cd}
	\usepackage{enumitem}
	\usepackage{arydshln}
	\usepackage{relsize}
	\usepackage{multirow}
	\usepackage{scalerel}
	\usepackage{upgreek}
	\usepackage{ifthen}
	\usepackage{yhmath}
	\usepackage{blkarray}
	\usepackage{dashrule}
	\usepackage{subcaption}
	\usepackage{overpic}
    \usepackage[commandnameprefix=ifneeded]{changes}

% My commands
	% General Math
		\newcommand{\inv}{^{-1}}
		\newcommand{\abs}[1]{\left|#1\right|}
		\newcommand{\ceil}[1]{\left\lceil{}#1\right\rceil{}}
		\newcommand{\floor}[1]{\left\lfloor{}#1\right\rfloor{}}
		\newcommand{\conj}[1]{\overline{#1}}
		\newcommand{\of}{\circ}
		\newcommand{\tri}{\triangle}
		\newcommand{\inj}{\hookrightarrow}
		\newcommand{\surj}{\twoheadrightarrow}
		\newcommand{\restr}[1]{\left.#1\right|}
		\newcommand{\bigand}{\bigwedge}
		\newcommand{\bigor}{\bigvee}
	% Set Theory
		\newcommand{\card}[1]{\left|#1\right|}
		\newcommand{\set}[1]{\left\{#1\right\}}
		\newcommand{\setmid}{\;\middle|\;}
		\newcommand{\naturals}{\mathbb{N}}
		\newcommand{\N}{\naturals}
		\newcommand{\integers}{\mathbb{Z}}
		\newcommand{\Z}{\integers}
		\newcommand{\rationals}{\mathbb{Q}}
		\newcommand{\Q}{\rationals}
		\newcommand{\reals}{\mathbb{R}}
		\newcommand{\R}{\reals}
		\newcommand{\complex}{\mathbb{C}}
		\newcommand{\C}{\complex}
		\newcommand{\comp}{^{\complement}}
		\newcommand{\cut}{\setminus}
	% Linear Algebra
		\DeclareMathOperator{\Id}{Id}
		\DeclareMathOperator{\im}{im}
		\newcommand{\norm}[1]{\abs{\abs{#1}}}
		\newcommand{\tpose}{^{T}}
		\newcommand{\iprod}[1]{\left<#1\right>}
		\DeclareMathOperator{\trace}{tr}
		\DeclareMathOperator{\GL}{GL}
		\DeclareMathOperator{\vspan}{span}
		\DeclareMathOperator{\rank}{rank}
		\DeclareMathOperator{\proj}{proj}
		\DeclareMathOperator{\compProj}{comp}
		\newcommand{\smallPMatrix}[1]{\paren{\begin{smallmatrix}#1\end{smallmatrix}}}
		\newcommand{\smallBMatrix}[1]{\brack{\begin{smallmatrix}#1\end{smallmatrix}}}
		\newcommand{\pmat}[1]{\begin{pmatrix}#1\end{pmatrix}}
		\newcommand{\bmat}[1]{\begin{bmatrix}#1\end{bmatrix}}
		\newcommand{\dual}{^{*}}
		\newcommand{\pinv}{^{\dagger}}
		\newcommand{\horizontalMatrixLine}{\ptxt{\rotatebox[origin=c]{-90}{$|$}}}
	% Topology
		\newcommand{\closure}[1]{\overline{#1}}
		\DeclareMathOperator{\Int}{Int}
		\DeclareMathOperator{\Ext}{Ext}
		\DeclareMathOperator{\Bd}{Bd}
		\DeclareMathOperator{\rInt}{rInt}
	% Optimization
            \DeclareMathOperator*{\minimize}{minimize}
		\DeclareMathOperator{\subto}{subject\,to}
		\DeclareMathOperator{\argmin}{argmin}
		\DeclareMathOperator{\argmax}{argmax}
		\DeclareMathOperator{\argsup}{argsup}
		\DeclareMathOperator{\arginf}{arginf}
		\DeclareMathOperator{\piecewise}{pw}
	% Proofs
		\newcommand{\unique}{!}
		\newcommand{\st}{s.t.}
		\newcommand{\eqVertical}{\rotatebox[origin=c]{90}{=}}
		\newcommand{\mapsfrom}{\mathrel{\reflectbox{\ensuremath{\mapsto}}}}
		\newcommand{\from}{\!\mathrel{\reflectbox{\ensuremath{\to}}}}
	% Brackets
		\newcommand{\paren}[1]{\left(#1\right)}
		\renewcommand{\brack}[1]{\left[#1\right]}
		\renewcommand{\brace}[1]{\left\{#1\right\}}
		\newcommand{\ang}[1]{\left<#1\right>}
	% Shorthand
		\newcommand{\ptxt}[1]{\textrm{\textnormal{#1}}}
		\newcommand{\mf}[1]{\mathfrak{#1}}
		\newcommand{\mc}[1]{\mathcal{#1}}
		\newcommand{\ms}[1]{\mathscr{#1}}
		\newcommand{\mb}[1]{\mathbb{#1}}
	% Theorems
		\newtheorem{theorem}{Theorem}
		\newtheorem{assumption}{Assumption}
		\newtheorem{proposition}{Proposition}
		\newtheorem{lemma}{Lemma}
		\newtheorem{corollary}{Corollary}
	% Editing
		%\newcommand{\todo}[1]{\textbf{\textcolor{blue}{TODO: #1}}}

% Fix vdots and ddots
	\usepackage{letltxmacro}
	\LetLtxMacro\orgvdots\vdots
	\LetLtxMacro\orgddots\ddots

	\makeatletter
	\DeclareRobustCommand\vdots{%
		\mathpalette\@vdots{}%
	}
	\newcommand*{\@vdots}[2]{%
		% #1: math style
		% #2: unused
		\sbox0{$#1\cdotp\cdotp\cdotp\m@th$}%
		\sbox2{$#1.\m@th$}%
		\vbox{%
			\dimen@=\wd0 %
			\advance\dimen@ -3\ht2 %
			\kern.5\dimen@
			% remove side bearings
			\dimen@=\wd2 %
			\advance\dimen@ -\ht2 %
			\dimen2=\wd0 %
			\advance\dimen2 -\dimen@
			\vbox to \dimen2{%
				\offinterlineskip
				\copy2 \vfill\copy2 \vfill\copy2 %
			}%
		}%
	}
	\DeclareRobustCommand\ddots{%
		\mathinner{%
			\mathpalette\@ddots{}%
			\mkern\thinmuskip
		}%
	}
	\newcommand*{\@ddots}[2]{%
		% #1: math style
		% #2: unused
		\sbox0{$#1\cdotp\cdotp\cdotp\m@th$}%
		\sbox2{$#1.\m@th$}%
		\vbox{%
			\dimen@=\wd0 %
			\advance\dimen@ -3\ht2 %
			\kern.5\dimen@
			% remove side bearings
			\dimen@=\wd2 %
			\advance\dimen@ -\ht2 %
			\dimen2=\wd0 %
			\advance\dimen2 -\dimen@
			\vbox to \dimen2{%
				\offinterlineskip
				\hbox{$#1\mathpunct{.}\m@th$}%
				\vfill
				\hbox{$#1\mathpunct{\kern\wd2}\mathpunct{.}\m@th$}%
				\vfill
				\hbox{$#1\mathpunct{\kern\wd2}\mathpunct{\kern\wd2}\mathpunct{.}\m@th$}%
			}%
		}%
	}
	\makeatother

% Allow custom symbols as arrows in tikz-cd
	\tikzset{
	  symbol/.style={
		draw=none,
		every to/.append style={
		  edge node={node [sloped, allow upside down, auto=false]{$#1$}}}
	  }
	}
	% Example: \arrow[r,symbol=\cong]
	% https://tex.stackexchange.com/questions/394154/how-to-include-inclusion-subgroup-relationship-in-tikz-cd-diagram

\usepackage[bookmarks=true]{hyperref}

\usepackage{cleveref}
\Crefname{figure}{Figure}{Figures}
\Crefname{table}{Table}{Tables}
\Crefname{equation}{Eq.}{Eqs.}
\Crefname{section}{Section}{Sections}
\Crefname{subsection}{Subsection}{Subsections}
\Crefname{appendix}{Appendix}{Appendices}

\captionsetup[subfigure]{subrefformat=simple,labelformat=simple}
    \renewcommand\thesubfigure{ (\alph{subfigure})}

\DeclareMathOperator{\SE}{SE}
\DeclareMathOperator{\id}{id}
\DeclareMathOperator{\VALID}{VALID}
\DeclareMathOperator{\FREE}{FREE}

\newcommand{\toitself}{\mathbin{\scalebox{.85}{%
    \lefteqn{\scalebox{.5}{$\blacktriangleleft$}}\raisebox{.34ex}{$\supset$}}}}

% Algorithm considerations
% See: https://tex.stackexchange.com/questions/656872/modify-algorithm-to-compatible-with-ieeetran
\usepackage[commentColor=black]{algpseudocodex}
\tikzset{algpxIndentLine/.style={draw=black}}
\algrenewcommand{\alglinenumber}[1]{\bfseries\footnotesize #1}
\algrenewcommand{\textproc}{}
\algrenewcommand{\algorithmicrequire}{\textbf{Input:}}
\algrenewcommand{\algorithmicensure}{\textbf{Output:}}
\usepackage{algorithm}
\floatplacement{algorithm}{tbp}
\makeatletter
\newcommand{\algorithmname}{\ALG@name}
\renewcommand{\floatc@ruled}[2]{{\@fs@cfont #1:} #2\par}
\makeatother

\captionsetup[algorithm]{labelsep=colon}

\newtheorem{definition}{Definition}


\usetikzlibrary{shapes.geometric}
\usetikzlibrary{angles, quotes, calc}
\usetikzlibrary{arrows.meta}



