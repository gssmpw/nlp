\section{Conclusion}

In this work, we present a novel, cost-effective pipeline that combines physics-based simulations, human demonstrations, and model-based planning to address data scarcity in contact-rich robotic manipulation tasks. A key insight of our approach is that human demonstrations—even when collected on a different morphology—offer valuable global task information that model-based planners often struggle to discover independently due to the high-dimensional search space and complex contact dynamics. By leveraging these demonstrations as a global prior, our method refines and augments them through kinematic retargeting and trajectory optimization, resulting in large datasets of dynamically feasible trajectories across a range of physical parameters, initial conditions, and embodiments. Our framework significantly reduces the reliance on costly, hardware-specific data collection while offering the potential to reuse legacy datasets collected with outdated hardware or configurations. We demonstrate its effectiveness across multiple robotic systems in simulation and successfully zero-shot deploy policies trained on the augmented dataset to a bimanual iiwa hardware setup.