%!TEX root = ../root.tex

\begin{enumerate}
    \item Robot foundation model is hot. But robotics suffers from lack of data, especially for contact-rich tasks. 
    \item Imitation from human demonstration is just not scalable, and this approach uses data in a limited way by copying over the demonstrations directly. 
    \item On the other hand we can automatically generate data with RL, but this is not always reliable due to i) reward tuning, and ii) sparse exploration.
    \item So we explore the possibility of using local control to augment data to a distribution of tasks around the nominal demonstration, that generalizes to different embodiments, objects, etc. This enables us to 
    \begin{itemize}
        \item Bypass difficult exploration since demonstration solves away the 'combinatorics'
        \item No need for reward tuning
        \item Maximize the utlity of a demonstration by covering a wide range of distribution of objects / embodiments from a single demonstration.  
    \end{itemize}
    This covers the weakness of RL while lessening the burden of demonstration. 
\end{enumerate}

\subsection{Key Messages}
\begin{enumerate}
    \item Predictive control is only a function of the model and allows fast computation across model changes.
    \item We can leverage this capability to find a middle ground between pure demonstration / pure RL. 
    \item One demonstration, multiple data that is physically consistent. 
\end{enumerate}
\subsection{Cross-Embodiment Generalization}
Previous methods can only handle robots with similar morphologies: e.g. Franka $\rightarrow$ UR5 \cite{chen2024rovi}, and did not focus on contact-rich tasks. \cite{wang2024cross} learns a latent representation to enable pick-and-place type of manipulation skill transfer among robots with similar morphologies. \cite{patel2024get} is aware of a robot hand's missing fingers for in-hand reorientation tasks. 
invariant representation / feature space -- ``analogy making"
\cite{shankar2022translating} ``Our insight and premise is that agents
with different embodiments use similar strategies
(high-level skill sequences) to solve similar
tasks."