\section{Dataset Construction}
\section{Dataset}
\label{sec:dataset}

\subsection{Data Collection}

To analyze political discussions on Discord, we followed the methodology in \cite{singh2024Cross-Platform}, collecting messages from politically-oriented public servers in compliance with Discord's platform policies.

Using Discord's Discovery feature, we employed a web scraper to extract server invitation links, names, and descriptions, focusing on public servers accessible without participation. Invitation links were used to access data via the Discord API. To ensure relevance, we filtered servers using keywords related to the 2024 U.S. elections (e.g., Trump, Kamala, MAGA), as outlined in \cite{balasubramanian2024publicdatasettrackingsocial}. This resulted in 302 server links, further narrowed to 81 English-speaking, politics-focused servers based on their names and descriptions.

Public messages were retrieved from these servers using the Discord API, collecting metadata such as \textit{content}, \textit{user ID}, \textit{username}, \textit{timestamp}, \textit{bot flag}, \textit{mentions}, and \textit{interactions}. Through this process, we gathered \textbf{33,373,229 messages} from \textbf{82,109 users} across \textbf{81 servers}, including \textbf{1,912,750 messages} from \textbf{633 bots}. Data collection occurred between November 13th and 15th, covering messages sent from January 1st to November 12th, just after the 2024 U.S. election.

\subsection{Characterizing the Political Spectrum}
\label{sec:timeline}

A key aspect of our research is distinguishing between Republican- and Democratic-aligned Discord servers. To categorize their political alignment, we relied on server names and self-descriptions, which often include rules, community guidelines, and references to key ideologies or figures. Each server's name and description were manually reviewed based on predefined, objective criteria, focusing on explicit political themes or mentions of prominent figures. This process allowed us to classify servers into three categories, ensuring a systematic and unbiased alignment determination.

\begin{itemize}
    \item \textbf{Republican-aligned}: Servers referencing Republican and right-wing and ideologies, movements, or figures (e.g., MAGA, Conservative, Traditional, Trump).  
    \item \textbf{Democratic-aligned}: Servers mentioning Democratic and left-wing ideologies, movements, or figures (e.g., Progressive, Liberal, Socialist, Biden, Kamala).  
    \item \textbf{Unaligned}: Servers with no defined spectrum and ideologies or opened to general political debate from all orientations.
\end{itemize}

To ensure the reliability and consistency of our classification, three independent reviewers assessed the classification following the specified set of criteria. The inter-rater agreement of their classifications was evaluated using Fleiss' Kappa \cite{fleiss1971measuring}, with a resulting Kappa value of \( 0.8191 \), indicating an almost perfect agreement among the reviewers. Disagreements were resolved by adopting the majority classification, as there were no instances where a server received different classifications from all three reviewers. This process guaranteed the consistency and accuracy of the final categorization.

Through this process, we identified \textbf{7 Republican-aligned servers}, \textbf{9 Democratic-aligned servers}, and \textbf{65 unaligned servers}.

Table \ref{tab:statistics} shows the statistics of the collected data. Notably, while Democratic- and Republican-aligned servers had a comparable number of user messages, users in the latter servers were significantly more active, posting more than double the number of messages per user compared to their Democratic counterparts. 
This suggests that, in our sample, Democratic-aligned servers attract more users, but these users were less engaged in text-based discussions. Additionally, around 10\% of the messages across all server categories were posted by bots. 

\subsection{Temporal Data} 

Throughout this paper, we refer to the election candidates using the names adopted by their respective campaigns: \textit{Kamala}, \textit{Biden}, and \textit{Trump}. To examine how the content of text messages evolves based on the political alignment of servers, we divided the 2024 election year into three periods: \textbf{Biden vs Trump} (January 1 to July 21), \textbf{Kamala vs Trump} (July 21 to September 20), and the \textbf{Voting Period} (after September 20). These periods reflect key phases of the election: the early campaign dominated by Biden and Trump, the shift in dynamics with Kamala Harris replacing Joe Biden as the Democratic candidate, and the final voting stage focused on electoral outcomes and their implications. This segmentation enables an analysis of how discourse responds to pivotal electoral moments.

Figure \ref{fig:line-plot} illustrates the distribution of messages over time, highlighting trends in total messages volume and mentions of each candidate. Prior to Biden's withdrawal on July 21, mentions of Biden and Trump were relatively balanced. However, following Kamala's entry into the race, mentions of Trump surged significantly, a trend further amplified by an assassination attempt on him, solidifying his dominance in the discourse. The only instance where Trump’s mentions were exceeded occurred during the first debate, as concerns about Biden’s age and cognitive abilities temporarily shifted the focus. In the final stages of the election, mentions of all three candidates rose, with Trump’s mentions peaking as he emerged as the victor.



\subsection{Question Collection}
We collect Chinese questions from the sources:
\noindent \textbf{WebText} \cite{bright_xu_2019_3402023}: A large-scale Chinese community question-answering dataset spanning diverse topics.

\noindent \textbf{WebCPM} \cite{DBLP:conf/acl/QinCJYLZLHDWXQL23}: A Chinese long-form question-answering dataset focused on interactive web search contexts.

\noindent \textbf{Zhihu-KOL} \cite{zhihu-kol}: A high-quality question-answering dataset derived from Zhihu, a prominent Chinese QA platform.

\noindent \textbf{RGB} \cite{DBLP:conf/aaai/0011LH024}: A bilingual question-answering dataset based on news reports.

\noindent \textbf{TrickQA}: Questions with ambiguous, incorrect, or unverifiable premises (see Appendix~\ref{app:question} for details).

After collecting these questions, we input them into an open-sourced RAG system to simulate real-world question-answering scenarios and analyze how the system processes and responds to these diverse inputs. The RAG system retrieves five external documents and generates responses. Statements in the answers are annotated with citation marks (1–5), indicating alignment with information from the corresponding documents. On average, each statement spans 33.4 tokens, while each document averages 177.3 tokens. An original sample is formed by pairing a labeled statement with its cited documents, represented as a tuple (question, answer, statement, cited documents).


\subsection{Data Augmentation}

The goal of data augmentation is to create negative samples of high quality by making minor modifications to the cited documents in the original samples. Given the use of an industrial RAG system, the number of negative samples in the original samples is estimated to be insufficient. To construct a balanced training set, as well as a label-balanced dev set and test set for evaluation, successfully augmented negative samples can be used. The modified documents should not be inconsistent or incoherent, so as not to provide the trained model with a false basis for judging the negative samples.

We use GPT-4o \cite{openai2024gpt4technicalreport} for data augmentation. After providing the original sample to the LLM, it is asked to perform the following steps in sequence:

\noindent \textbf{Segments Identification}: Find all key segments in the cited document that directly support the information in the statement.

\noindent \textbf{Segments Grouping}: Group the key segments by the information they support, with each group containing key segments that support the same or related information in the statement.

\noindent \textbf{Segments Modification}: Select a group of key segments and modify them so that they do not support the corresponding information in the statement. 

The modification changes only the portion that relates to the supported information in the statement. This maintains logical flow and non-contradictory information within the key segments, and keep the key segments logical in the context of the document and non-contradictory to other information in the document. If there is more than one key segment in a group, the information in all of them should be consistent after the modification.

For each sample, the LLM is asked to try two methods of modification:

\noindent\textbf{Content Revision}: Alter specific details within a key segment without introducing direct contradictions to the original information.

\noindent \textbf{Structure Preservation}: Remove information from a key segment while ensuring the overall coherence and integrity of the segment remain intact.

After completing the LLM augmentation, each original sample is accompanied by the LLM-labeled key segment information and corresponds to the two augmented samples generated by the LLM using the two modification methods. The cost is 0.026\$ per sample. See Appendix~\ref{app:aug} for more details of the augmentation.


\subsection{Two-stage Manual Annotation}

The original samples need to be manually labeled as positive or negative samples before they can be used to form the dataset (examples are shown in Table~\ref{tab:dataset}). In the LLM augmentation phase, although we try to guide the LLM to augment negative samples with qualified quality, the LLM may generate some samples that do not meet the requirements. Therefore, the augmented samples also need to be manually labeled for compliance before they can be used to form the dataset. The goal of the two-stage manual annotation is to complete the manual annotation needed above.

In the first stage, the annotators (from the professional data annotation institution in China) need to label whether the original sample is a positive or negative sample, i.e., to determine whether the sum of the information provided by the cited documents fully supports the statement. In order to reduce the difficulty of labeling, the information of key segments labeled by LLM will be provided to the annotators as a reference. However, since the LLM labeling is not always accurate, if the annotators are unable to make a judgment after reading the key segments, they still need to read other parts of the documents to make a judgment. 
In this stage, the number of negative samples identified by the annotation is 1,006, with a negative sample rate of about 9\%. We randomly selected 2,000 samples (1,000 negative and 1,000 positive) and split them equally to create the development and test sets. The augmented samples corresponding to the positive samples in the remaining original samples will be labeled in the second stage.

In the second stage, the annotators need to determine whether an augmented sample is of acceptable quality and whether it is a negative sample. In order to reduce the difficulty of labeling, we show the annotator a comparison of the documents before and after the modification in the form of modification traces. Among the augmented samples that the annotators determine to be negative samples of acceptable quality, we select 2,449 samples that use the modification methods of changing information and deleting information respectively, totaling 4,898 samples. These augmented negative samples together with the 4,898 positive samples in the original samples identified by the first stage of annotation constitute the training set. The two-stage manual annotation costs 0.5\$ per sample. See Appendix~\ref{app:ann} for instructions for annotators.
