\section{Experiments}
In our experiments, we evaluate the dataset using two approaches. First, we assess the zero-shot performance of state-of-the-art LLMs on the development and test sets. This aims to highlight the challenge posed by the test samples. Second, due to resource constraints, we conduct parameter-efficient fine-tuning on smaller models using the training data. This focuses on demonstrating the effectiveness of the training samples.

\subsection{Settings}


State-of-the-art LLMs that we use for zero-shot performance tests include GPT-4o \cite{openai2024gpt4technicalreport}, Qwen2.5-Plus \cite{qwen2024qwen25technicalreport}, and DeepSeek-v3 \cite{deepseekai2024deepseekv3technicalreport}. We provide the sample to the LLMs and ask for their judgment. The relatively small language models we use for training include Llama-3.1-8B \cite{grattafiori2024llama3herdmodels}, Mistral-7B \cite{jiang2023mistral7b}, and Qwen2.5-7B \cite{qwen2024qwen25technicalreport}. The parameter-efficient fine-tuning method we use is LoRA \cite{DBLP:conf/iclr/HuSWALWWC22}. See Appendix~\ref{app:detail} for training details. We use accuracy as the metric. Since there are equal numbers of positive and negative samples, the accuracy is equivalent to the commonly used balanced accuracy \cite{DBLP:journals/corr/abs-2303-15621}, which is the average of the accuracy on positive and negative samples. We also report the accuracy of positive and negative samples separately.





\begin{table*}[h!]
\scriptsize
% \footnotesize
\centering
\caption{
    Success rate
}
% \vspace{-0.5em}
\label{tab:exp_alg_res_sim_large}
\begin{tabular}{cc ccccc ccccc c}
\toprule
\multicolumn{1}{c}{} &
\multicolumn{1}{c}{Planner} &
\multicolumn{1}{c}{Traj 1} &
\multicolumn{1}{c}{Traj 2} &
\multicolumn{1}{c}{Traj 3} &
\multicolumn{1}{c}{Traj 4} &
\multicolumn{1}{c}{Traj 5} &
\multicolumn{1}{c}{Traj 6} &
\multicolumn{1}{c}{Traj 7} &
\multicolumn{1}{c}{Traj 8} &
\multicolumn{1}{c}{Traj 9} &
\multicolumn{1}{c}{Traj 10} &
\multicolumn{1}{c}{\textbf{Average}} \\ 
\midrule
% & No prediction & 0.15 & 0.10 & 0.05 & 0.30 & 0.04 & 0.21 & 0.20 & 0.28 & 0.10 & 0.21 & \textbf{0.16} \\ 
\multirow{3}*{\shortstack[c]{No\\prediction}} 
& No policy      & 0.11 & 0.05 & 0.13 & 0.27 & 0.03 & 0.17 & 0.15 & 0.20 & 0.10 & 0.18 & \textbf{0.14} \\
& Pretrained     & 0.28 & 0.43 & 0.40 & 0.39 & 0.39 & 0.29 & 0.37 & 0.32 & 0.33 & 0.60 & \textbf{0.38} \\
& MFRL           & 0.39 & 0.57 & 0.47 & 0.56 & 0.64 & 0.49 & 0.52 & 0.50 & 0.65 & 0.94 & \textbf{0.57} \\
% [23.04 30.63 26.19]
\midrule
\multirow{3}*{\shortstack[c]{Ground\\truth}} 
& No policy      & 0.20 & 0.41 & 0.51 & 0.46 & 0.30 & 0.17 & 0.51 & 0.70 & 0.27 & 0.43 & \textbf{0.40} \\
& Pretrained     & 0.70 & 0.88 & 0.83 & 0.76 & 0.73 & 0.69 & 0.83 & 0.74 & 0.77 & 0.91 & \textbf{0.78} \\
& MFRL           & 0.76 & 0.92 & 0.89 & 0.81 & 0.96 & 0.88 & 0.94 & 0.99 & 0.81 & 1.00 & \textbf{0.90} \\
% [23.77 33.87 28.6 ]
\midrule
\multirow{3}*{\shortstack[c]{Ground\\truth\\($\sigma\texttt{=}0.01 \si[]{m}$)}}
& No policy      & 0.36 & 0.42 & 0.41 & 0.39 & 0.31 & 0.25 & 0.44 & 0.61 & 0.32 & 0.47 & \textbf{0.40} \\
& Pretrained     & 0.72 & 0.97 & 0.75 & 0.75 & 0.61 & 0.71 & 0.76 & 0.79 & 0.69 & 0.85 & \textbf{0.76} \\
& MFRL           & 0.81 & 0.96 & 0.96 & 0.88 & 0.72 & 0.84 & 0.79 & 0.98 & 0.76 & 0.79 & \textbf{0.85} \\
% [22.41 34.51 30.06]
\midrule
\multirow{3}*{\shortstack[c]{Ground\\truth\\($\sigma\texttt{=}0.05 \si[]{m}$)}}
& No policy      & 0.28 & 0.34 & 0.35 & 0.02 & 0.23 & 0.20 & 0.31 & 0.46 & 0.38 & 0.42 & \textbf{0.30} \\
& Pretrained     & 0.46 & 0.67 & 0.48 & 0.36 & 0.70 & 0.71 & 0.73 & 0.84 & 0.76 & 0.88 & \textbf{0.66} \\
& MFRL           & 0.54 & 0.71 & 0.76 & 0.63 & 0.73 & 0.71 & 0.93 & 0.93 & 0.85 & 0.85 & \textbf{0.76} \\
% [21.81 36.73 30.12]
\midrule
\multirow{3}*{\shortstack[c]{GMM\\prediction}} 
& No policy      & 0.34 & 0.26 & 0.31 & 0.30 & 0.31 & 0.21 & 0.40 & 0.64 & 0.27 & 0.54 & \textbf{0.36} \\
& Pretrained     & 0.75 & 0.75 & 0.71 & 0.51 & 0.63 & 0.54 & 0.71 & 0.86 & 0.71 & 0.67 & \textbf{0.68} \\
& MFRL           & 0.81 & 0.94 & 0.76 & 0.67 & 0.74 & 0.67 & 0.92 & 0.99 & 0.81 & 0.87 & \textbf{0.82} \\
% [24.15 34.84 30.96]
\hline
\end{tabular}
% \vspace{-0.5em}
\end{table*}

% MFRL 2100 buf idx 0.1 
% Catch ratio: Pretrained, MFRL [16.55 18.66 26.76 23.52]
% vmax 0.5 ~ 4.5 m/s, start delay 5 ~ 25 * 0.1 sec (0.5~2.5)
\subsection{Results}

Table~\ref{tab:res} reveals significant differences in performance between LLMs tested under zero-shot conditions and smaller models fine-tuned with parameter-efficient methods. Among the zero-shot LLMs, GPT-4o achieved the highest overall accuracy, outperforming Qwen2.5-Plus and DeepSeek-v3. However, even GPT-4o struggled with negative samples, achieving only 70.4\% accuracy on the dev set and 71.6\% on the test set. This limitation highlights a persistent challenge in distinguishing negative cases, which significantly impacts overall accuracy. DeepSeek-v3, while demonstrating near-perfect accuracy on positive samples, performed poorly on negative samples (39.6\% dev, 39.4\% test), indicating a clear trade-off between the two categories.

In contrast, smaller models fine-tuned with the training set achieved remarkable improvements, particularly in handling negative samples. Llama-3.1-8B stood out as the top performer, achieving 91.4\% accuracy on the dev set and 90.6\% on the test set, while maintaining a strong balance between positive and negative samples. These results suggest that the training data effectively addressed the challenges posed by negative samples, enabling the fine-tuned models to achieve significantly higher overall accuracy. Overall, the results underscore the effectiveness of fine-tuning in improving model robustness, particularly for negative samples. The dataset’s training data appears to play a crucial role in enhancing model performance, as evidenced by the fine-tuned models’ ability to achieve high accuracy across both positive and negative samples. These insights suggest that tailored training strategies and targeted fine-tuning can significantly enhance model capabilities, even for smaller models.
