

\section{Unexpected Questions}
\label{app:question}
Real-world questions do not always have the correct premises. For example, in the question "\begin{CJK}{UTF8}{gbsn}水俣病的传染途径是什么?\end{CJK}(What is the route of infection for Minamata disease?)", Minamata disease is not an infectious disease. Taking this situation into account, we add a small number of human-written questions with incorrect premises and LLM-generated questions with hard-to-verify premises in the question collection phase. The number of these questions in the total number of questions is about 3\%.

\section{Prompt for LLM Augmentation}
\label{app:aug}

\definecolor{titlecolor}{rgb}{0.9, 0.5, 0.1}
\definecolor{anscolor}{rgb}{0.2, 0.5, 0.8}
\definecolor{labelcolor}{HTML}{48a07e}
\begin{table*}[h]
	\centering
	
 % \vspace{-0.2cm}
	
	\begin{center}
		\begin{tikzpicture}[
				chatbox_inner/.style={rectangle, rounded corners, opacity=0, text opacity=1, font=\sffamily\scriptsize, text width=5in, text height=9pt, inner xsep=6pt, inner ysep=6pt},
				chatbox_prompt_inner/.style={chatbox_inner, align=flush left, xshift=0pt, text height=11pt},
				chatbox_user_inner/.style={chatbox_inner, align=flush left, xshift=0pt},
				chatbox_gpt_inner/.style={chatbox_inner, align=flush left, xshift=0pt},
				chatbox/.style={chatbox_inner, draw=black!25, fill=gray!7, opacity=1, text opacity=0},
				chatbox_prompt/.style={chatbox, align=flush left, fill=gray!1.5, draw=black!30, text height=10pt},
				chatbox_user/.style={chatbox, align=flush left},
				chatbox_gpt/.style={chatbox, align=flush left},
				chatbox2/.style={chatbox_gpt, fill=green!25},
				chatbox3/.style={chatbox_gpt, fill=red!20, draw=black!20},
				chatbox4/.style={chatbox_gpt, fill=yellow!30},
				labelbox/.style={rectangle, rounded corners, draw=black!50, font=\sffamily\scriptsize\bfseries, fill=gray!5, inner sep=3pt},
			]
											
			\node[chatbox_user] (q1) {
				\textbf{System prompt}
				\newline
				\newline
				You are a helpful and precise assistant for segmenting and labeling sentences. We would like to request your help on curating a dataset for entity-level hallucination detection.
				\newline \newline
                We will give you a machine generated biography and a list of checked facts about the biography. Each fact consists of a sentence and a label (True/False). Please do the following process. First, breaking down the biography into words. Second, by referring to the provided list of facts, merging some broken down words in the previous step to form meaningful entities. For example, ``strategic thinking'' should be one entity instead of two. Third, according to the labels in the list of facts, labeling each entity as True or False. Specifically, for facts that share a similar sentence structure (\eg, \textit{``He was born on Mach 9, 1941.''} (\texttt{True}) and \textit{``He was born in Ramos Mejia.''} (\texttt{False})), please first assign labels to entities that differ across atomic facts. For example, first labeling ``Mach 9, 1941'' (\texttt{True}) and ``Ramos Mejia'' (\texttt{False}) in the above case. For those entities that are the same across atomic facts (\eg, ``was born'') or are neutral (\eg, ``he,'' ``in,'' and ``on''), please label them as \texttt{True}. For the cases that there is no atomic fact that shares the same sentence structure, please identify the most informative entities in the sentence and label them with the same label as the atomic fact while treating the rest of the entities as \texttt{True}. In the end, output the entities and labels in the following format:
                \begin{itemize}[nosep]
                    \item Entity 1 (Label 1)
                    \item Entity 2 (Label 2)
                    \item ...
                    \item Entity N (Label N)
                \end{itemize}
                % \newline \newline
                Here are two examples:
                \newline\newline
                \textbf{[Example 1]}
                \newline
                [The start of the biography]
                \newline
                \textcolor{titlecolor}{Marianne McAndrew is an American actress and singer, born on November 21, 1942, in Cleveland, Ohio. She began her acting career in the late 1960s, appearing in various television shows and films.}
                \newline
                [The end of the biography]
                \newline \newline
                [The start of the list of checked facts]
                \newline
                \textcolor{anscolor}{[Marianne McAndrew is an American. (False); Marianne McAndrew is an actress. (True); Marianne McAndrew is a singer. (False); Marianne McAndrew was born on November 21, 1942. (False); Marianne McAndrew was born in Cleveland, Ohio. (False); She began her acting career in the late 1960s. (True); She has appeared in various television shows. (True); She has appeared in various films. (True)]}
                \newline
                [The end of the list of checked facts]
                \newline \newline
                [The start of the ideal output]
                \newline
                \textcolor{labelcolor}{[Marianne McAndrew (True); is (True); an (True); American (False); actress (True); and (True); singer (False); , (True); born (True); on (True); November 21, 1942 (False); , (True); in (True); Cleveland, Ohio (False); . (True); She (True); began (True); her (True); acting career (True); in (True); the late 1960s (True); , (True); appearing (True); in (True); various (True); television shows (True); and (True); films (True); . (True)]}
                \newline
                [The end of the ideal output]
				\newline \newline
                \textbf{[Example 2]}
                \newline
                [The start of the biography]
                \newline
                \textcolor{titlecolor}{Doug Sheehan is an American actor who was born on April 27, 1949, in Santa Monica, California. He is best known for his roles in soap operas, including his portrayal of Joe Kelly on ``General Hospital'' and Ben Gibson on ``Knots Landing.''}
                \newline
                [The end of the biography]
                \newline \newline
                [The start of the list of checked facts]
                \newline
                \textcolor{anscolor}{[Doug Sheehan is an American. (True); Doug Sheehan is an actor. (True); Doug Sheehan was born on April 27, 1949. (True); Doug Sheehan was born in Santa Monica, California. (False); He is best known for his roles in soap operas. (True); He portrayed Joe Kelly. (True); Joe Kelly was in General Hospital. (True); General Hospital is a soap opera. (True); He portrayed Ben Gibson. (True); Ben Gibson was in Knots Landing. (True); Knots Landing is a soap opera. (True)]}
                \newline
                [The end of the list of checked facts]
                \newline \newline
                [The start of the ideal output]
                \newline
                \textcolor{labelcolor}{[Doug Sheehan (True); is (True); an (True); American (True); actor (True); who (True); was born (True); on (True); April 27, 1949 (True); in (True); Santa Monica, California (False); . (True); He (True); is (True); best known (True); for (True); his roles in soap operas (True); , (True); including (True); in (True); his portrayal (True); of (True); Joe Kelly (True); on (True); ``General Hospital'' (True); and (True); Ben Gibson (True); on (True); ``Knots Landing.'' (True)]}
                \newline
                [The end of the ideal output]
				\newline \newline
				\textbf{User prompt}
				\newline
				\newline
				[The start of the biography]
				\newline
				\textcolor{magenta}{\texttt{\{BIOGRAPHY\}}}
				\newline
				[The ebd of the biography]
				\newline \newline
				[The start of the list of checked facts]
				\newline
				\textcolor{magenta}{\texttt{\{LIST OF CHECKED FACTS\}}}
				\newline
				[The end of the list of checked facts]
			};
			\node[chatbox_user_inner] (q1_text) at (q1) {
				\textbf{System prompt}
				\newline
				\newline
				You are a helpful and precise assistant for segmenting and labeling sentences. We would like to request your help on curating a dataset for entity-level hallucination detection.
				\newline \newline
                We will give you a machine generated biography and a list of checked facts about the biography. Each fact consists of a sentence and a label (True/False). Please do the following process. First, breaking down the biography into words. Second, by referring to the provided list of facts, merging some broken down words in the previous step to form meaningful entities. For example, ``strategic thinking'' should be one entity instead of two. Third, according to the labels in the list of facts, labeling each entity as True or False. Specifically, for facts that share a similar sentence structure (\eg, \textit{``He was born on Mach 9, 1941.''} (\texttt{True}) and \textit{``He was born in Ramos Mejia.''} (\texttt{False})), please first assign labels to entities that differ across atomic facts. For example, first labeling ``Mach 9, 1941'' (\texttt{True}) and ``Ramos Mejia'' (\texttt{False}) in the above case. For those entities that are the same across atomic facts (\eg, ``was born'') or are neutral (\eg, ``he,'' ``in,'' and ``on''), please label them as \texttt{True}. For the cases that there is no atomic fact that shares the same sentence structure, please identify the most informative entities in the sentence and label them with the same label as the atomic fact while treating the rest of the entities as \texttt{True}. In the end, output the entities and labels in the following format:
                \begin{itemize}[nosep]
                    \item Entity 1 (Label 1)
                    \item Entity 2 (Label 2)
                    \item ...
                    \item Entity N (Label N)
                \end{itemize}
                % \newline \newline
                Here are two examples:
                \newline\newline
                \textbf{[Example 1]}
                \newline
                [The start of the biography]
                \newline
                \textcolor{titlecolor}{Marianne McAndrew is an American actress and singer, born on November 21, 1942, in Cleveland, Ohio. She began her acting career in the late 1960s, appearing in various television shows and films.}
                \newline
                [The end of the biography]
                \newline \newline
                [The start of the list of checked facts]
                \newline
                \textcolor{anscolor}{[Marianne McAndrew is an American. (False); Marianne McAndrew is an actress. (True); Marianne McAndrew is a singer. (False); Marianne McAndrew was born on November 21, 1942. (False); Marianne McAndrew was born in Cleveland, Ohio. (False); She began her acting career in the late 1960s. (True); She has appeared in various television shows. (True); She has appeared in various films. (True)]}
                \newline
                [The end of the list of checked facts]
                \newline \newline
                [The start of the ideal output]
                \newline
                \textcolor{labelcolor}{[Marianne McAndrew (True); is (True); an (True); American (False); actress (True); and (True); singer (False); , (True); born (True); on (True); November 21, 1942 (False); , (True); in (True); Cleveland, Ohio (False); . (True); She (True); began (True); her (True); acting career (True); in (True); the late 1960s (True); , (True); appearing (True); in (True); various (True); television shows (True); and (True); films (True); . (True)]}
                \newline
                [The end of the ideal output]
				\newline \newline
                \textbf{[Example 2]}
                \newline
                [The start of the biography]
                \newline
                \textcolor{titlecolor}{Doug Sheehan is an American actor who was born on April 27, 1949, in Santa Monica, California. He is best known for his roles in soap operas, including his portrayal of Joe Kelly on ``General Hospital'' and Ben Gibson on ``Knots Landing.''}
                \newline
                [The end of the biography]
                \newline \newline
                [The start of the list of checked facts]
                \newline
                \textcolor{anscolor}{[Doug Sheehan is an American. (True); Doug Sheehan is an actor. (True); Doug Sheehan was born on April 27, 1949. (True); Doug Sheehan was born in Santa Monica, California. (False); He is best known for his roles in soap operas. (True); He portrayed Joe Kelly. (True); Joe Kelly was in General Hospital. (True); General Hospital is a soap opera. (True); He portrayed Ben Gibson. (True); Ben Gibson was in Knots Landing. (True); Knots Landing is a soap opera. (True)]}
                \newline
                [The end of the list of checked facts]
                \newline \newline
                [The start of the ideal output]
                \newline
                \textcolor{labelcolor}{[Doug Sheehan (True); is (True); an (True); American (True); actor (True); who (True); was born (True); on (True); April 27, 1949 (True); in (True); Santa Monica, California (False); . (True); He (True); is (True); best known (True); for (True); his roles in soap operas (True); , (True); including (True); in (True); his portrayal (True); of (True); Joe Kelly (True); on (True); ``General Hospital'' (True); and (True); Ben Gibson (True); on (True); ``Knots Landing.'' (True)]}
                \newline
                [The end of the ideal output]
				\newline \newline
				\textbf{User prompt}
				\newline
				\newline
				[The start of the biography]
				\newline
				\textcolor{magenta}{\texttt{\{BIOGRAPHY\}}}
				\newline
				[The ebd of the biography]
				\newline \newline
				[The start of the list of checked facts]
				\newline
				\textcolor{magenta}{\texttt{\{LIST OF CHECKED FACTS\}}}
				\newline
				[The end of the list of checked facts]
			};
		\end{tikzpicture}
        \caption{GPT-4o prompt for labeling hallucinated entities.}\label{tb:gpt-4-prompt}
	\end{center}
\vspace{-0cm}
\end{table*}
\begin{table*}

\centering

\begin{tabular}{|p{\textwidth}|}
\hline
\\ [2pt]
\par Here is a statement and a corresponding piece of reference text. Please complete the task as follows, strictly following the format I have given for the output:
\par (1) Find all the original key passages in the reference text that directly support the information in the statement (there may be more than one, find each one). Output one original key passage per line and the information in the statement it directly supports in the format “Key passage {number}: {key passage} (information in the supporting statement: {supporting information})”.
\par (2) Please group key passages, each group contains key passages supporting the same or related information in the statement, output one line of the grouping results in the format of “Key passage grouping: Group 1: (first group of key passage numbers), Group 2: (second group of key passage numbers) ...”. For example, if there are 2 pieces of information in the statement, key paragraph 1 supports information 1, key paragraph 2 supports information 2, and key paragraph 3 supports information 1, then the output is “Key Paragraph Grouping: Group 1: (1, 3), Group 2: (2)”.
\par (3) Select a group of key text segments and modify the parts of them that support the information in the statement to meet the following requirements:
\par - The modification should make it impossible for the key passage to fully support the corresponding information in the statement.
\par - The modifications should maintain the logical flow of the key passages and no contradictions between the information in the key passages.
\par - The modification should keep the key paragraph logically coherent in the context of the reference text and not contradict the rest of the reference text.
\par - Modify only the parts that support the information in a statement, leaving the rest unchanged.
\par - If there is more than one key passage in a set, the information in them should remain consistent after revision.
\par You need to try two methods of modification:
\par - Changing the message: modifying the message in one part of the key paragraph to another. Do not make changes that directly conflict with the original information. For example, if the original message is “The Audi A7 Signature Edition has a faster top speed than its predecessor”, an appropriate change would be “The Audi A7 Luxury Edition has a faster top speed than its predecessor”, and an inappropriate change1 would be “The Audi A7 Signature Edition has a slower top speed than its predecessor” (using an antonym, which is in direct conflict with the original message), and inappropriate modification 2 is ‘The top speed of the Audi A7 Signature Edition is not faster than the previous generation’ (adding a negative word, which is in direct conflict with the original message).
\par - Delete Information: Remove information from a place in a key paragraph. If the key paragraph is a complete sentence, it should still be a complete sentence after deleting the information. For example, if the original paragraph reads “Due to weather conditions, the project was delayed until March 15” (complete sentence), an appropriate change would be “Due to weather conditions, the project was delayed until March” (still a complete sentence), an inappropriate change would be “Due to the weather” (no longer a complete sentence).
\par For each method, output the key passage that was modified and check its logical fluency, giving an integer within 1 to 10 as a rating (higher means more fluent). Output one modified key passage per line in the format “{method}-modified key passage {number}: {modified key passage} (logical fluency: {score})”.
\\ [5pt]
\\ [5pt]
\hline

\end{tabular}

\caption{\label{tab:prompt_en} The complete prompt for the LLM augmentation (translated into English).}
\end{table*}

See Table~\ref{tab:prompt} for the prompt for LLM augmentation. Table~\ref{tab:prompt_en} provides an English version.

\section{Instructions for Annotators}
\label{app:ann}
\subsection{First Stage}
In the first stage, we provide the annotators with the question, answer, statement, and cited documents. What LLM considers to be key segments are highlighted in red in the cited documents (see Figure~\ref{fig:stage} for an example). We instruct the annotators to follow the process below:

\par (1) First look at the highlighted text. If the highlighted text fully supports the statement, then the annotation is positive; if the highlighted text contradicts the statement, then the annotation is negative.

\par (2) If the annotation cannot be derived from the highlighted text, then look at the rest of the documents to make the annotation. When the documents fully support the statement, the label is positive, and when there is any information in the statement that contradicts the documents or information that is not mentioned in the documents, the label is negative.

\subsection{Second Stage}

In the second stage, we provide the annotator with the statement and the modified documents. In the documents, the modified parts are highlighted in green, where the dashed and crossed-out text is deleted and the rest is added (see Figure~\ref{fig:stage} for examples). 

For the annotation of whether the quality of the modification is acceptable, the annotators are instructed to note that qualified modifications need to satisfy the following two requirements: (1) There are no contradictions within each modified document. (2) The modified key segments are fluent in their own right and in the context of the document. The annotation for support is the same as the first stage, but based on the modified documents.

 
\section{Input and Training Details}
\label{app:detail}
We input the statement and the cited documents into the model and ask the model to determine whether the statement is fully supported by the documents, outputting yes or no. For input, we label and concatenate the cited documents in order (as shown in Table~\ref{tab:dataset}). For training, we use the following settings: For training, we use the following settings: learning rate is 5e-4, number of epochs is 10, scheduler is cosine scheduler, warmup ratio is 0.03, batch size is 256, and LoRA setting is $r=8$, $a=32$ and 0.1 dropout. We report the model performance for the epoch that achieves the best performance on the dev set.
\label{app:detail}



\section{Related Works}
Language models are known to produce hallucinations - statements that are inaccurate or unfounded~\citep{MaynezNBM20,HuCLGWYG24}. To address this limitation, recent research has focused on augmenting LLMs with external tools such as retrievers~\citep{GuuLTPC20,BorgeaudMHCRM0L22,LiuCtrla2024} and search engines~\citep{WebGPT2021, Komeili0W22, TanGSXLFLWSLS24}. While this approach suggests that generated content is supported by external references, the reliability of such attribution requires careful examination. Recent studies have investigated the validity of these attributions. \citet{DBLP:conf/emnlp/LiuZL23} conducted human evaluations to assess the verifiability of responses from generative search engines. \citet{hu2024evaluate} further investigate the reliability of such attributions when giving adversarial questions to RAG systems. Their findings revealed frequent occurrences of unsupported statements and inaccurate citations, highlighting the need for rigorous attribution verification~\citep{RashkinNLA00PTT23}. However, human evaluation processes are resource-intensive and time-consuming. To overcome these limitations, existing efforts~\citep{GaoDPCCFZLLJG23,DBLP:conf/emnlp/GaoYYC23} proposed an automated approach using Natural Language Inference models to evaluate attribution accuracy. While several English-language benchmarks have been developed for this purpose~\citep{DBLP:conf/emnlp/YueWCZS023}, comparable resources in Chinese are notably lacking. Creating such datasets presents unique challenges, particularly in generating realistic negative samples (unsupported citations).  To address this gap, we introduce the first large-scale Chinese dataset for citation faithfulness detection, developed through a cost-effective two-stage manual annotation process.

\begin{figure*}
    \centering
    \includegraphics[width=0.3\textwidth]{appendix/s1.png}
    \includegraphics[width=0.3\textwidth]{appendix/s2-m.png}
    \includegraphics[width=0.3\textwidth]{appendix/s2-d.png}
    \caption{Examples of interfaces that provide samples to the annotators. The first figure shows an example of the first stage. The last two images show the second stage with the same sample modified (information changed/deleted).}
    \label{fig:stage}
\end{figure*}

