\begin{table*}

\centering

\begin{tabular}{|p{\textwidth}|}
\hline
\\ [2pt]
\begin{CJK}{UTF8}{gbsn}
\par 这里有一段陈述和对应的一段参考文本。请按如下步骤完成任务,严格按我给出的格式进行输出:
\par(1)找到参考文本中所有直接支撑陈述中信息的原始关键文段(可能有多个,每一处都要找到)。每行输出一个原始关键文段及其直接支撑的陈述中的信息,格式为“关键文段{编号}:{关键文段}(支撑陈述中的信息:{支撑信息})”。
\par(2)请将关键文段分组,每组包含的关键文段支撑陈述中的相同或相关的信息,输出一行分组结果,格式为“关键文段分组:第一组:(第一组关键文段编号),第二组:(第二组关键文段编号)...”。例如,陈述中有2个信息,关键文段1支撑信息1,关键文段2支撑信息2,关键文段3支撑信息1,那么输出“关键文段分组:第一组:(1,3),第二组:(2)”
\par(3)选择一组关键文段,对其中支撑陈述中信息的部分进行修改,满足以下要求:
\par - 修改应该使得关键文段无法完全支撑陈述中的对应信息。
\par - 修改应该保持关键文段的逻辑通顺、关键文段中的信息之间不矛盾。
\par - 修改之后的关键文段应该在参考文本的上下文语境中保持逻辑通顺,且与参考文本中的其他内容不矛盾。
\par - 只修改支撑陈述中信息的部分,其它部分保持不变。
\par - 如果一组中有多个关键文段,修改后它们的信息应该保持一致。
\par 你需要尝试两种修改方法:
\par - 改变信息:将关键文段中的某一处信息修改为另外的信息。请不要进行与原信息产生直接冲突的修改。例如,原信息为“奥迪A7旗舰版的最高速度比上一代快”,合适的修改是“奥迪A7豪华版的最高速度比上一代快”,不合适的修改1是“奥迪A7旗舰版的最高速度比上一代慢”(使用反义词,与原信息直接冲突),不合适的修改2是“奥迪A7旗舰版的最高速度不比上一代快”(添加否定词,与原信息直接冲突)。
\par - 删除信息:将关键文段中的某一处信息删除。关键文段如果是完整的句子,删除信息后应该仍然是一个完整的句子。例如,原文段为“由于天气原因,项目推迟至3月15日启动”(完整的句子),合适的修改是“由于天气原因,项目推迟至3月启动”(仍然是完整的句子),不合适的修改是“由于天气原因”(不再是完整的句子)。
\par 对每种方法,输出被修改的关键文段,并检查其逻辑通顺程度,给出一个1\textasciitilde 10以内的整数作为评分(越高表示越通顺)。每行输出一个修改后的关键文段,格式为“{方法}-修改后的关键文段{编号}:{修改后的关键文段}(逻辑通顺程度:{分数})”。
\end{CJK} 
\\ [5pt]
\\ [5pt]
\hline

\end{tabular}

\caption{\label{tab:prompt} The complete prompt for the LLM augmentation.}
\end{table*}
