\documentclass[11pt,a4paper]{article}
\usepackage[lmargin=1.0in,rmargin=1.0in,bottom=1.0in,top=1.0in,twoside=False]{geometry}

\usepackage{inconsolata}
\usepackage{libertine}

\usepackage{amssymb,amsmath,amsthm}
\usepackage{mathtools}
\usepackage{graphicx}%,algorithmic,algorithm}%, verbatim}
\usepackage{enumerate}
\usepackage{tikz}
\usetikzlibrary{shapes}

\usepackage[T1]{fontenc}

\usepackage{enumitem}

\usepackage{microtype}
\usepackage{amsfonts}
\usepackage{comment}
\usepackage{mathrsfs}
\setlength{\marginparwidth}{2cm}
\usepackage{todonotes}
\newcommand{\mpil}[2][]{\todo[color=magenta!80,#1]{{\textbf{MaPi:} #2}}}
\newcommand{\mipil}[2][]{\todo[color=orange!30,#1]{\small {\textbf{MiPi:} #2}}}
\newcommand{\gs}[2][]{\todo[color=blue!20,#1]{\small {\textbf{GS:} #2}}}
\newcommand{\kk}[2][]{\todo[color=green!20,#1]{\small {\textbf{KK:} #2}}}

\definecolor{blue}{rgb}{0.1,0.2,0.5}
\definecolor{brown}{rgb}{0.6,0.6,0.2}
\usepackage[ocgcolorlinks, linkcolor={blue}, citecolor={brown}]{hyperref}
\usepackage{cleveref}

\usepackage{enumerate}
\usepackage{latexsym}

\usepackage{soul}

\newtheorem{lemma}{Lemma}[section]
\newtheorem{observation}{Observation}[section]
\newtheorem{claim}{Claim}[section]
\newtheorem{theorem}[lemma]{Theorem}
\newtheorem{corollary}[lemma]{Corollary}
\newtheorem{conjecture}[lemma]{Conjecture}

\newcommand{\Oh}{\mathcal{O}}

\newcommand{\argmax}{\text{argmax}}
\newcommand{\argmin}{\text{argmin}}

\newcommand{\distprofile}[3]{\mathrm{prof}_{#1,#2}[#3]}
\newcommand{\tildeprofile}[3]{\widetilde{\mathrm{prof}}_{#1,#2}[#3]}
\newcommand{\barprofile}[3]{\overline{\mathrm{prof}}_{#1,#2}[#3]}

\newcommand{\hatprofile}[3]{\widehat{\mathrm{prof}}_{#1,#2}[#3]}
\newcommand{\diststarprofile}[3]{\mathrm{prof}^\star_{#1,#2}[#3]}
\newcommand{\dist}{\mathrm{dist}}
\newcommand{\ecc}{\mathrm{ecc}}

\renewcommand{\leq}{\leqslant}
\renewcommand{\geq}{\geqslant}
\renewcommand{\le}{\leqslant}
\renewcommand{\ge}{\geqslant}
\renewcommand{\setminus}{-}

\usepackage{marvosym}

\newcommand{\cutgraph}{\textrm{\CutLeft}}

\usepackage{textpos}


\begin{document}

\author{
  Kacper Kluk% 
  \thanks{Institute of Informatics, University of Warsaw, Poland. \texttt{k.kluk@uw.edu.pl}}
  \and
  Marcin Pilipczuk%
  \thanks{Institute of Informatics, University of Warsaw, Poland. \texttt{m.pilipczuk@uw.edu.pl}}
  \and
  Micha\l{} Pilipczuk%
  \thanks{Institute of Informatics, University of Warsaw, Poland. \texttt{michal.pilipczuk@mimuw.edu.pl}}
  \and
  Giannos Stamoulis%
  \thanks{IRIF, Université Paris Cité, CNRS, Paris, France. \texttt{giannos.stamoulis@irif.fr}}
}


\title{Faster diameter computation in graphs of bounded Euler genus%
  \thanks{
  K.K. and Ma.P. are supported by Polish National Science Centre SONATA BIS-12 grant number 2022/46/E/ST6/00143.
  This work is a part of project BOBR (Mi.P., G.S.) that has received funding from the European Research Council (ERC) 
under the European Union's Horizon 2020 research and innovation programme (grant agreement No.~948057). In particular, a majority of work on this manuscript was done while G.S. was affiliated with University of Warsaw.}}

\date{}

\maketitle

\begin{abstract}
We show that for any fixed integer $k \geq 0$, there exists an algorithm
that computes the diameter and the eccentricies of all vertices of an input
unweighted, undirected $n$-vertex graph of Euler genus at most $k$ in time
 \[ \Oh_k(n^{2-\frac{1}{25}}). \]
Furthermore, for the more general class of graphs
that can be constructed
by clique-sums from graphs that are of Euler genus at most $k$ after deletion
of at most $k$ vertices, we show an algorithm for the same task that achieves the running time bound
 \[ \Oh_k(n^{2-\frac{1}{356}} \log^{6k} n). \]
Up to today, the only known subquadratic algorithms for computing the diameter in those graph classes
are that of [Ducoffe, Habib, Viennot; SICOMP 2022], [Le, Wulff-Nilsen; SODA 2024],
and [Duraj, Konieczny, Pot\k{e}pa; ESA 2024]. These algorithms work
in the more general setting of $K_h$-minor-free graphs,
but the running time bound is $\Oh_h(n^{2-c_h})$ for some constant $c_h > 0$ depending
on $h$. 
That is, our savings in the exponent, as compared to the naive quadratic algorithm, 
are independent of the parameter $k$.

The main technical ingredient of our work is an improved bound on the number of distance
profiles, as defined in [Le, Wulff-Nilsen; SODA 2024], in graphs of bounded Euler genus.
\end{abstract}


\begin{textblock}{20}(11.8,4.35)
  \includegraphics[width=60px]{ERC2.jpg}
\end{textblock}


\section{Introduction}
\section{Introduction}
\label{sec:intro}

\begin{figure*}[tb]
    \centering
    \includegraphics[width=0.848\linewidth]{figs/circuitnn.pdf} 
    \caption{Illustration of differentiable CircuitNN. CircuitNN is designed based on differentiable NAND gates. After DAS is guided by PI and PO pairs of the truth table, CircuitNN can get the precise circuit architecture logic equivalent to the truth table.}
    \label{fig:circuitnn}
\end{figure*}

% 1. Describe the importance of logic synthesis
% 2. Existing Problems
% (a) Neural Architecture Search: Unstable, Predefined Setting, etc.
% (b) Circuit Generation: Probabilistic Model, Logic Equivalence

With the rapid advancement of technology, the scale of integrated circuits (ICs) has expanded exponentially. 
This expansion has introduced significant challenges in chip manufacturing, particularly concerning power and area metrics.
A primary objective in IC design is achieving the same circuit function with fewer transistors, thereby reducing power usage and area occupancy.

Logic synthesis~\cite{hachtel2005logicsynth}, a critical step in electronic design automation (EDA), transforms behavioral-level circuit designs into optimized gate-level circuits, ultimately yielding the final IC layout. 
The primary goal of logic synthesis is to identify the physical implementation with the fewest gates for a given circuit function. 
This task constitutes a challenging NP-hard combinatorial optimization problem. 
Current logic synthesis tools~\cite{brayton2010abc, wolf2013yosys} rely on human-designed heuristics, often leading to sub-optimal outcomes.

Differentiable architecture search (DAS) techniques~\cite{liu2018darts, chu2020darts} offer novel perspectives on addressing challenges in this problem.
Circuit functions can be represented through truth tables, which map binary inputs to their corresponding outputs. 
Truth tables provide a precise representation of input-output relationships, ensuring the design of functionally equivalent circuits.
Inspired by this, researchers~\cite{deepmind2024ai4sys, wang2024tnet} have begun exploring the application of DAS to synthesize circuits directly from truth tables.
Specifically, \citet{deepmind2024ai4sys} proposed CircuitNN, a framework that learns differentiable connection structures with logic gates, enabling the automatic generation of logic circuits from truth tables.
This approach significantly reduces the complexity of traditional circuit generation. 
Building on this, \citet{wang2024tnet} introduced T-Net, a triangle-shaped variant of CircuitNN, incorporating regularization techniques to enhance the efficiency of DAS.

Despite these advancements, several challenges remain. 
The computational complexity of DAS grows quadratically with the number of gates, posing scalability issues.
Although triangle-shaped architecture~\cite{wang2024tnet} partially mitigates this problem, redundancy persists. 
%Additionally, DAS is susceptible to converging to local optima, limiting the ability to search architectures that satisfy the given truth tables~\cite{liu2018darts}. 
%Furthermore, hyperparameters (network depth and layer width) require extensive searches, introducing complexity and prolonging the synthesis process. 
Additionally, DAS is susceptible to converging to local optima~\cite{liu2018darts} and hyperparameters (network depth and layer width) require extensive searches. 
The challenges arise from the vast search space in DAS. 
% Even with predefined settings for CircuitNN, finding a configuration that meets the truth table requires extensive trial and error during the DAS process. 
Intuitively, limiting the search space through predefined parameters (network depth, gates per layer, and connection probabilities) can significantly reduce the complexity.

Recent advances~\cite{openai2023gpt4, abramson2024alphafold3, esser2024sd3, li2024mar} in conditional generative models have demonstrated remarkable performance across language, vision, and graph generation tasks. 
Motivated by these developments, we propose a novel approach to circuit generation that generates preliminary circuit structures to guide DAS in generating refined circuits matching specified truth tables. 
Firstly, we introduce CircuitVQ, a tokenizer with a discrete codebook for circuit tokenization. 
Built upon our Circuit AutoEncoder framework~\cite{hou2022graphmae,li2023maskgae,wu2025mgvga}, CircuitVQ is trained through a circuit reconstruction task. 
Specifically, the CircuitVQ encoder encodes input circuits into discrete tokens using a learnable codebook, while the decoder reconstructs the circuit adjacency matrix based on these tokens.
Subsequently, the CircuitVQ encoder serves as a circuit tokenizer for CircuitAR pretraining, which employs a masked autoregressive modeling paradigm~\cite{chang2022maskgit, li2023mage}. 
In this process, the discrete codes function as supervision signals. 
After training, CircuitAR can generate discrete tokens progressively, which can be decoded into initial circuit structures by the decoder of the CircuitVQ. 
These prior insights can guide DAS in producing refined circuits that match the target truth tables precisely.

Our key contributions can be summarized as follows:
\begin{itemize}
\item We introduce CircuitVQ, a circuit tokenizer that facilitates graph autoregressive modeling for circuit generation, based on our Circuit AutoEncoder framework;
\item Develop CircuitAR, a model trained using masked autoregressive modeling, which generates initial circuit structures conditioned on given truth tables;
\item Propose a refinement framework that integrates differentiable architecture search to produce functionally equivalent circuits guided by target truth tables;
\item Comprehensive experiments demonstrating the scalability and capability emergence of our CircuitAR and the superior performance of the proposed circuit generation approach.
\end{itemize}

% Motivation
% (a) Diffusion (Vision, Graph), Autoregressive (Language, Vision)
% (b) Circuit Generation for Predefined Setting
% (c) Neural Architecture Search for Strict Logic Equivalence

% Contribution
% (a) Circuit Tokenizer (new transformer arch, training strategy)
% (b) CircuitAR (train and gen strategies, post-ar strategy)
% (c) Extensive Evaluation including BitD (Bit Distance) for Scalability


\section{Preliminaries}\label{sec:prelims}

\section{Notation and Preliminaries}\label{sec:prelims}
This section fixes the notation and relevant notions for fair division of goods; the notation specific to division of chores is relegated to Section \ref{sec:chores}. 
 
\paragraph{Fair Division Instances.} A {fair division instance} is given by a tuple $\langle [n], [m], \{v_i\}_{i=1}^n \rangle$, where $[n]=\{1,2,.\dots,n\}$ is the set of $n\in\mathbb{Z}_+$ agents, $[m]=\{1,2, \dots, m\}$ the set of $m\in \mathbb{Z}_+$ indivisible goods, and for each agent $i\in[n]$, the set function $v_i: 2^{[m]} \to \mathbb{R}_+$ denotes the valuation of agent $i$ over subsets of goods. Specifically, $v_i(S) \in \mathbb{R}_+$ denotes the value that agent $i$ derives from the subset $S \subseteq [m]$ of goods. For subsets $S \subseteq [m]$ and $g \in [m]$, we will write $S + g$ to denote the union $S \cup \{ g\}$. 

A valuation $v_i$ is said to be monotone if the inclusion of goods into any subset does not decrease its value, under $v_i$, i.e., $v_i(S)\leq v_i(T)$ for every pair of subsets $S \subseteq T \subseteq[m]$. We will assume throughout that the agents' valuations are monotone and normalized: $v_i(\emptyset)=0$ for all agents $i$. 

We will additionally consider instances with identically ordered valuations. Here, we have an indexing of the $m$ goods, $\{g_1, \ldots g_m\}$, such that for each pair of goods $g_s, g_t$, with index $s < t$, and all agents $i \in [n]$, the inequality $v_i(S + g_s) \geq  v_i(S + g_t)$ holds for each subset $S \subset [m]$ that does not contain $g_s$ and $g_t$; see Example \ref{ex:sqrt-ordered} in Section \ref{subsec:additive-ordered}. 

This work also establishes improved bounds for the specific case of additive valuations. Recall that a valuation $v_i$ is said to be additive if, for every subset $S\subseteq[m]$ of goods, $v_i(S)=\sum_{g\in S} v_i(\{g\})$. We will use the shorthand $v_i(g)$---instead of $v_i(\{g\}) \in \mathbb{R}_+$---to denote agent $i$'s value for any good $g \in [m]$.  


\paragraph{Allocations and Multi-Allocations.} An allocation $\calB=(B_1,B_2,\ldots, B_n)$ of the goods among the $n$ agents is a partition of $[m]$ into $n$ pairwise disjoint subsets $B_1,\ldots, B_n \subseteq [m]$. Here, the subset of goods $B_i$ is assigned to agent $i \in [n]$ and is referred to as $i$'s bundle. In addition, write $\Pi_n([m])$ to denote the collection of all $n$-partitions of $[m]$. Note that for any allocation $\calB =(B_1,\ldots, B_n)$ we have, by definition, $\cup_{i=1}^n B_i = [m]$ and $B_i \cap B_j = \emptyset$, for all $i \neq j$, and hence $\calB \in \Pi_n([m])$.

 
A \textit{multi-allocation} is a tuple $\calA=(A_1,A_2\dots,A_n)$ of $n$ subsets, wherein subset $A_i \subseteq [m]$ denotes the bundle assigned to agent $i$. In contrast to allocations, in a multi-allocation, we do not require that the assigned bundles $A_i$ are pairwise disjoint and that they partition $[m]$.\footnote{Note that $A_i$s are still subsets of goods and not multisets.} Hence, in a multi-allocation, a single good may be present in multiple bundles or even in none. 

Though, when in a multi-allocation $\calA$, each good $g$ is assigned to exactly one agent, we refer to $\calA$ as an {\it exact allocation}; this is to emphasize that the bundles of such a multi-allocation do partition $[m]$. 

We associate with each bundle $A_i \subseteq [m]$ an $m$-dimensional characteristic vector $\rmchar(A_i) \in \{0,1\}^m$. For each good $g\in [m]$, the $g$th component of the characteristic vector---denoted as $\rmchar(A_i)_g$---is equal to one if $g \in A_i$, otherwise the $g$th component is zero. That is, 
\begin{align*}
\rmchar(A_i)_g \coloneqq \begin{cases}
    1 & \text{if } g\in A_i \\
    0 & \text{otherwise}.
\end{cases}
\end{align*}

For any multi-allocation $\calA=(A_1, \ldots, A_n)$, we will use $\chi^\calA \in \mathbb{Z}^m_+$ to denote the vector sum of the characteristic vectors of its bundles, $\chi^\calA \coloneqq \sum_{i=1}^n\rmchar(A_i)$. We will refer to $\chi^\calA$ as the \textit{characteristic vector} of the multi-allocation $\calA$. When there is no ambiguity, we will omit the notational dependence in the superscript and simply write $\chi$ for $\chi^\calA$.

Note that for any good $g\in [m]$ and multi-allocation $\calA$, the $g^{th}$ component of the characteristic vector $\chi^\calA_g$ is equal to the number of bundles in $\calA$ that contain $g$. Conceptually, we think of this setting as one in which $\chi^A_g$ identical copies of the good $g$ are assigned among different agents. 

Write $\ellone{\chi^\calA}$ and $\ellinfty{\chi^\calA}$ to denote the $\ell_1$ and $\ell_\infty$ norm, respectively, of the characteristic vector. Hence,  $\ellone{\chi^\calA} = \sum_{g=1}^m \chi^\calA_g$ and $\ellinfty{\chi^\calA} = \max_{g\in[m]} \chi^\calA_g$. It is relevant to note that $\ellone{\chi^\calA}$ captures the total number of goods, with copies, assigned among the agents,  $\ellone{\chi^\calA} = \sum_{i=1}^n |A_i|$. Further, $\ellinfty{\chi^\calA}$ captures the maximum number of copies of any one good $g$ assigned under $\calA$.

In particular, if $\calA$ is an {\it exact} allocation, then $\chi^\calA$ is equal to the all-ones vector and we have $\ellone{\chi^\calA} =m$ and $\ellinfty{\chi^\calA} =1$.
 
\noindent
The shared-based fairness criterion we study in this work is defined using maximin shares; these shares are defined next.
\begin{definition}[Maximin Share (MMS)]\label{def:mms}
    Given any fair division instance $\langle [n], [m], \{v_i\}_{i=1}^n \rangle$ with goods, the {maximin share}, $\mu_i \in \mathbb{R}_+$, of each agent $i \in [n]$ is defined as 
    \begin{align*}
    \mu_i \coloneqq  \max_{(X_1,\dots, X_n) \in \Pi_n([m])} \ \ \min_{j\in[n]} v_i(X_{j}).
    \end{align*}
Further, for each agent $i$, let $\calM^i=(M^i_1, M^i_2, \ldots, M^i_n) \in \Pi_n([m])$ denote an {MMS-inducing partition}:
\begin{align*}
\calM^i \in \argmax_{(X_1,\dots, X_n) \in \Pi_n([m])} \ \ \min_{j\in[n]} v_i(X_{j})
\end{align*}
\end{definition}

Note that in Definition \ref{def:mms} the maximum is taken over all $n$-partitions of $[m]$. Also, by definition, the partition $\calM^i =(M^i_1, \ldots, M^i_n)$ satisfies $v_i(M^i_j) \geq \mu_i$, for each index $j \in [n]$. 

\paragraph{Fair Multi-Allocations.} A multi-allocation $\calA=(A_1,\dots,A_n)$ is said to be an \emph{MMS multi-allocation} (i.e., it is deemed to be fair) if under it each agent receives a bundle of value at least its maximin share:  $v_i(A_i)\geq \mu_i$ for all agents $i \in [n]$.
 
To establish existential guarantees for MMS multi-allocations $\calA$, we will assume that, for all the agents, we are given the MMS-inducing partitions $\calM^i$, which in turn are guaranteed to exist (see Definition \ref{def:mms}).  

\section{Distance profiles in graphs of bounded Euler genus}\label{sec:distprofiles}


In this section we prove~\Cref{thm:distprofiles}. Our argument consists of a reduction to the planar case, where we can use the constant bound on the VC-dimension of the set system given by the distance profiles due to Le and Wulff-Nilsen~\cite{LeW24}.
The main idea behind the reduction is to consider certain notions of ``extended'' profiles, where the extension is built along a collections of shortest paths. These shortest paths can be chosen in such a way that by cutting the graph along these paths we obtain a plane graph. Then a bound on the number of the extended profiles in the obtained plane graph translates to a bound on the number of (standard) distance profiles in the original graph.

Preliminary definitions and results needed for defining profiles with respect to shortest paths are given in~\Cref{subsec:milestones}.
These extended profiles are then defined in~\Cref{subsec:mdprofiles}. There, we also prove that a fundamental lemma that equality of extended profiles entails equality of (standard) distance profiles.
The main reduction providing the proof of~\Cref{thm:distprofiles} is given at the end of this section.

\subsection{Milestones}
\label{subsec:milestones}

Let $G$ be a graph, $R$ be a subset of $V(G)$,
$v_0$ be a vertex in $V(G)$, and $P$ be a shortest path from $v_0$ to $R$. Let $x$ be the unique vertex in $V(P)\cap R$. Further, let $\le_P$ be the linear ordering of the vertices traversed by $P$: for two vertices $v,u\in V(P)$, we have $v\le_P u$ if $u$ belongs to $P[v,x]$.
We say that a vertex $v\in V(P)$ is a \emph{milestone of $P$}
if either $v=x$ or we have $\distprofile{R}{x}{v}\neq\distprofile{R}{x}{u}$, where $u$ is the successor of $v$ in $\le_P$.
We denote by $M_{R}(P)$ the set of all milestones of $P$.
Given a milestone $v\in M_{R}(P)$,
the \emph{neutral prefix of $v$ in $P$} is defined as the vertex set of the maximal subpath $Q$ of $P[v_0,v]$ satisfying the following: $v$ is the only milestone of $P$ that belongs to $Q$.

The next lemma shows that minimum-length paths towards $R$ that contain a vertex in the neutral prefix of a milestone can be assumed to pass through that milestone vertex.

\begin{lemma}
  \label{lem:dispref}
  Let $G$ be a graph, $R$ be a subset of $V(G)$, $v_0$ be a vertex in $V(G)$ and $P$ be a shortest path from $v_0$ to $R$.
  Then for every $v\in M_R(P)$, every $u$ in the neutral prefix of $v$, and every $y\in R$, it holds that
  $\dist(u,y)=|P[u,v]|+\dist(v,y)$. 
\end{lemma}
\begin{proof}
  Let $x$ be the unique vertex of $V(P)\cap R$. Note that, by definition, $\distprofile{R}{x}{v}=\distprofile{R}{x}{u}$.
  Also, $\dist(u,x)=\dist(u,v)+\dist(v,x)$ and $\dist(u,v)=|P[u,v]|$. Therefore, $\dist(u,y)=|P[u,v]|+\dist(v,y)$ for every $y\in R$.
\end{proof}

We also give an upper bound on the number of milestones.

\begin{lemma}
  \label{lem:mile}
  Let $G$ be a graph, $R$ be a connected subset of $V(G)$, $v_0$ be a vertex of $G$, and $P$ be a shortest path from $v_0$ to $R$. Then the number of milestones of $P$ is at most $|R|^2+1$.
\end{lemma}
\begin{proof}
  Let $x$ be the unique vertex of $V(P)\cap R$.
  First observe that since $P$ is a shortest path from $v_0$ to~$R$, we have $\dist(v,y)\ge \dist(v,x)$ for every $v\in V(P)$ and every $y\in R$; hence $\distprofile{R}{x}{v}(y)\ge 0$.
  Also, since $R$ is connected, for every $y\in R$ we have $\distprofile{R}{x}{x}(y)\le |R|$.
  To conclude the proof, it suffices to prove that for all $v_1,v_2\in V(P)$ with $v_1\leq_P v_2$, we have \begin{equation}\label{eq:wydra}\distprofile{R}{x}{v_1}(y)\le \distprofile{R}{x}{v_2}(y)\qquad\textrm{for all }y\in R.\end{equation}
  Indeed, \eqref{eq:wydra} together with the previous observations shows that all the distinct distance profiles of the form $\distprofile{R}{x}{v}$ for $v\in V(P)$ can be treated as vectors of length $|R|$ with entries in $\{0,\ldots,|R|\}$, and they all have distinct sums $\sum_{y\in R} \distprofile{R}{x}{v}(y)$. Since these sums range between $0$ and $|R|^2$, the total number of distinct profiles is at most $|R|^2+1$, implying the same bound on the number of milestones.

  To see why \eqref{eq:wydra} holds, note that $\dist(v_1,y)\le \dist(v_1,v_2)+\dist(v_2,y)$ implies that $$\dist(v_1,y)\le \dist(v_1,v_2)+\distprofile{R}{x}{v_2}(y)+\dist(v_2,x)=\distprofile{R}{x}{v_2}(y) + \dist(v_1,x);$$
  the last equality follows from $P$ being a shortest path containing $v_1,v_2,$ and $x$ (in this order). This in turn implies that
  $\distprofile{R}{x}{v_1}(y)=\dist(v_1,y)-\dist(v_1,x)\le \distprofile{R}{x}{v_2}(y)$, as claimed.
\end{proof}


\subsection{Anchor-distance profiles}
\label{subsec:mdprofiles}

\paragraph{Shortest path collections.}
Let $G$ be a graph and $R$ be a subset of vertices of $G$.
We say that a collection $\mathcal{P}$ of paths in $G$
is an \emph{$R$-shortest path collection} if
\begin{itemize}[nosep]
  \item every $P \in \mathcal{P}$ is a shortest path from some $v^P \in V(G)$ to $R$, i.e., $|P|=\dist(v^P,R)$; and
  \item $R \subseteq \bigcup_{P \in \mathcal{P}} V(P)$.
\end{itemize}

For each $P \in \mathcal{P}$, we denote by $x^P$ the endpoint of $P$ in $R$.
Note that the collection $\mathcal{P}$ obtained by taking, for every $y\in R$, the zero-length path from $y$ to $y$, is an $R$-shortest path collection. 

We say that an $R$-shortest path collection is \emph{consistent} if, for every $P_1,P_2 \in \mathcal{P}$
and $v \in V(P_1) \cap V(P_2)$ the paths $P_1[v,x^{P_1}]$ and $P_2[v,x^{P_2}]$ are equal. That is, 
once two paths intersect, they continue together towards $R$. 

The following statement is a direct consequence of the definition of an $R$-shortest path collection.
\begin{observation}\label{obs:short}
  Let $G$ be a graph, $R$ be a subset of vertices of $G$, and $\mathcal{P}$ be an $R$-shortest path collection. Then for every two paths $P_1,P_2\in \mathcal{P}$ and every $v\in V(P_1)\cap V(P_2)$, we have $|P_1[v,x^{P_1}]|=|P_2[v,x^{P_2}]|$.
\end{observation}

\paragraph{Anchor vertices and their prefixes.}
Let $G$ be a graph, $R$ be a subset of $V(G)$, and $\mathcal{P}$ be an $R$-shortest path collection. We denote by $H_{\mathcal{P}}$ the union of the paths in $\mathcal{P}$, i.e., the graph $(\bigcup_{P\in\mathcal{P}}V(P),\bigcup_{P\in\mathcal{P}}E(P))$.
We say that a vertex is an \emph{anchor vertex} if either it has degree more than two in $H_{\mathcal{P}}$
or it is a milestone of a path $P \in \mathcal{P}$.
We denote by $A_R(P)$ the set of all anchor vertices lying on a path $P \in \mathcal{P}$
and by $A_R(\mathcal{P})$ the set of all anchor vertices for $\mathcal{P}$.
Given a path $P\in\mathcal{P}$ with endpoints $v_0$ and $y\in R$, and an anchor vertex $w\in A_R(P)$,
the \emph{prefix of $w$ in $P$} is the vertex set of the maximal subpath $Q$ of
$P[v_0,v]$ satisfying the following: $v$ is the only anchor vertex of $P$ that belongs to $Q$.
Note that for every anchor $w\in V(P)$ there is a milestone $w'$ of $P$ (possibly $w=w'$)
such that the prefix of $w$ in $P$ is a subset of the neutral prefix of $w'$ in $P$.
Finally, for an anchor vertex $w$, the \emph{tail} of $w$, 
denoted $\mathrm{tail}(w)$, is the subgraph of $G$ consisting
of the union of all prefixes of $w$ in $P$ over all paths $P \in \mathcal{P}$ that contain $w$.

\paragraph{Hat-distances.}
Let $G$ be a graph, $R$ be a subset of vertices of $G$, and $\mathcal{P}$ be an $R$-shortest path collection.
We denote by $$U_{\mathcal{P}}\coloneqq V(G)-\bigcup_{P\in\mathcal{P}}V(P).$$
For every $u\in U_{\mathcal{P}}$,
and every anchor vertex $w\in A_R(\mathcal{P})$,
we set 
\[ \widehat{\dist}(u,w)\coloneqq \min\{|Q_{u,z}|+|P[z,w]|\colon
P \in \mathcal{P} \wedge w \in V(P) \wedge 
z\text{ is in the prefix of $w$ in $P$}\},\] 
where $Q_{u,z}$ is a shortest path from $u$ to $z$ with all its internal vertices in $U_{\mathcal{P}}$.
If such $Q_{u,z}$ does not exist for any $z \in V(\mathrm{tail}(w))$,
we set $\widehat{\dist}(u, w) \coloneqq \infty$. 

The following statement is a direct consequence of the definition of $\widehat{\dist}(\cdot,\cdot)$.

\begin{observation}
  \label{obs:dist}
  Let $G$ be a graph, $R$ be a subset of vertices of $G$, and $\mathcal{P}$ be
  an $R$-shortest path collection.
  Then for every $u\in U_{\mathcal{P}}$, we have that $$\dist(u,R)=\min \left\{\widehat{\dist}(u,w)+ \dist(w,R)
  \colon w\in A_R(\mathcal{P})\right\}.$$
\end{observation}

\paragraph{Anchor-distance profiles.}
Let $G$ be a graph, $R$ be a subset of vertices of $G$,  and $\mathcal{P}$
be an $R$-shortest path collection.
The \emph{anchor-distance profile} of a vertex $u\in U_{\mathcal{P}}$ to $R$ with respect to $\mathcal{P}$
is a function $\diststarprofile{R}{\mathcal{P}}{u}$ mapping each $w \in A_R(\mathcal{P})$ to
\[\diststarprofile{R}{\mathcal{P}}{u}(w) \coloneqq \widehat{\dist}(u,w) + \dist(w,R)
 - \dist(u,R).\]
\Cref{obs:dist} implies that we always have $\diststarprofile{R}{\mathcal{P}}{u}(w) \geq 0$.
We set
\[\hatprofile{R}{\mathcal{P}}{u}(w)\coloneqq\min\{\diststarprofile{R}{\mathcal{P}}{u}(w),|R|+1\}.\]

\begin{lemma}\label{lem:hat-to-normal}
  Let $G$ be a graph, let $R$ be a connected subset of vertices of $G$, and $s_R\in R$. Also, let $\mathcal{P}$ be an $R$-shortest path collection.
  Then for all $u_1,u_2\in U_{\mathcal{P}}$,
  $$\hatprofile{R}{\mathcal{P}}{u_1}=\hatprofile{R}{\mathcal{P}}{u_2}\qquad\textrm{implies}\qquad \distprofile{R}{s_R}{u_1}=\distprofile{R}{s_R}{u_2}.$$
\end{lemma}
\begin{proof}
  Fix $u_1,u_2\in U_{\mathcal{P}}$ with
  $\hatprofile{R}{\mathcal{P}}{u_1}=\hatprofile{R}{\mathcal{P}}{u_2}$.
  We start by proving the following.

  \begin{claim}\label{cl:bnd}
    Let $u\in U_{\mathcal{P}}$ and $y\in R$. There is an anchor $w \in A_R(\mathcal{P})$
    such that
    \begin{itemize}[nosep]
      \item $\widehat{\dist}(u,w)+\dist(w,y) = \dist(u,y)$ and
      \item $\hatprofile{R}{\mathcal{P}}{u}(w)\le |R|$.
    \end{itemize}
  \end{claim}
  \begin{proof}
    Let $Q$ be a shortest path from $u$ to $y$ and let $P\in\mathcal{P}$ be the path which $Q$ first intersects
    (if the first vertex of $Q$ in $\bigcup_{P \in \mathcal{P}} V(P)$ belongs to more than one paths in~$\mathcal{P}$,
    we choose $P$ arbitrarily among these paths). 
    Also, let $u'$ be the first vertex of $Q$ (when ordering from $u$ to $y$) in $V(P)$
    and $w$ be the anchor of $P$ that contains $u'$ in its prefix (in $P$).
    Note that $u' \in V(\mathrm{tail}(w))$.

    We first show that 
    \begin{equation}\label{eq:bnd:1}\widehat{\dist}(u,w)+\dist(w,y)= \dist(u,y).\end{equation}
    By~\Cref{lem:dispref} and the fact that $|Q[u',y]| = \dist(u',y)$, we have
    \begin{equation}\label{eq:I}
      \dist(w,y) = |Q[u',y]|-|P[u',w]|.
    \end{equation}
    Also, by definition, we have
    \begin{equation}\label{eq:II}
      \widehat{\dist}(u,w)\le |Q[u,u']|+|P[u',w]|.
    \end{equation}
    By~\eqref{eq:I} and~\eqref{eq:II}, we get that $\widehat{\dist}(u,w)+\dist(w,y)\le |Q|$.
    Moreover, since $Q$ is a shortest path from $u$ to $y$ and $\widehat{dist}(u,w) \geq \dist(u,w)$,
    we have
    \[ |Q| = \dist(u,y) \leq \dist(u,w) + \dist(w,y) \leq \widehat{dist}(u,w) + dist(w,y).\]
    This proves~\eqref{eq:bnd:1}.
 
    Next, we show that $\hatprofile{R}{\mathcal{P}}{u}(w)\le |R|$. Note that
    \begin{align*}
      \diststarprofile{R}{\mathcal{P}}{u}(w) + \dist(u,R) & = \widehat{\dist}(u,w)+\dist(w,R)\\
      & \le \widehat{\dist}(u,w)+\dist(w,y)= \dist(u,y).
    \end{align*}
    The connectivity of $R$ implies that $\dist(u,y)\le \dist(u,R)+|R|$, which gives
    $\diststarprofile{R}{\mathcal{P}}{u}(w)\le |R|$, and the claim follows.
  \end{proof}

  We next show that
  there is an integer $c$ such that for every $y\in R$, we have
  $$\dist(u_1,y)=\dist(u_2,y)+c.$$
  Note that this will immediately imply that
  $\distprofile{R}{s_R}{u_1}=\distprofile{R}{s_R}{u_2}$.

  By Observation~\ref{obs:dist}, for every $h \in \{1,2\}$, there is an anchor $w_h \in A_R(\mathcal{P})$
  such that $\dist(u_h,R) = \widehat{\dist}(u_h,w_h) + \dist(w_h, R)$, which is equivalent to
  $\diststarprofile{R}{\mathcal{P}}{u_h}(w_h) = 0$. 
  If $w_h$ lies on $P_h \in \mathcal{P}$, then $\dist(u_h,R) = \dist(u_h,x^{P_h})$. 
  Therefore, as $\diststarprofile{R}{\mathcal{P}}{u_1}=\diststarprofile{R}{\mathcal{P}}{u_2}$,
  we can choose $w_1 = w_2$ and $P_1 = P_2$, hence $x^{P_1} = x^{P_2}$. 
  In other words, there exists $x\in R$ such that $\dist(u_1,R) = \dist(u_1,x)$ and $\dist(u_2,R) = \dist(u_2,x)$.
  We set $c\coloneqq \dist(u_1,x)-\dist(u_2,x)=\dist(u_1,R)-\dist(u_2,R)$.

  Now, fix $y\in R$.
  Let $w_1\in A_R(P)$ be the anchor from~\Cref{cl:bnd} (applied for $u_1$ and $y$).
  As $\hatprofile{R}{\mathcal{P}}{u_1} = \hatprofile{R}{\mathcal{P}}{u_2}$ and
  $\diststarprofile{R}{\mathcal{P}}{u_1}(w_1)\le |R|$, we have
   $\diststarprofile{R}{\mathcal{P}}{u_1}(w_1) = \diststarprofile{R}{\mathcal{P}}{u_2}(w_1)$, i.e.,
  \[\widehat{\dist}(u_1,w_1) + \dist(w_1,R) - \dist(u_1,R) = \widehat{\dist}(u_2,w_1) + \dist(w_1,R)- \dist(u_2,R).\]
  Therefore,
  \begin{align*}
    \dist(u_1,y) &= \widehat{\dist}(u_1,w_1) + \dist(w_1,y)\\
      & =\widehat{\dist}(u_2,w_1) + \dist(w_1,y)+c\geq \dist(u_2,y)+c;
  \end{align*}
  the first equality follows from~\Cref{cl:bnd}.
  Thus $\dist(u_2,y)+c\le \dist(u_1,y)$.
  A symmetric reasoning shows that also $\dist(u_1,y)-c\le\dist(u_2,y)$.
  Therefore we get $\dist(u_1,y) = \dist(u_2,y)+c$, as required.
\end{proof}

\subsection{Reduction from bounded genus graphs to planar graphs}


We next recall several definitions related to embeddings of graphs on surfaces.
Our basic terminology follows~\cite{MoharT01grap}.
We say that a graph $H$ embedded in a surface $\Sigma$ is a {\em{simple cut-graph}} of $\Sigma$ if $H$ has a single face that is also homeomorphic to an open disk; equivalently, $H$ has a single facial walk.
Given a surface $\Sigma$ and a simple cut-graph $H$ on $\Sigma$, we denote by $\Sigma\cutgraph H$ the surface obtained by cutting $\Sigma$ along~$H$. Note that, provided $H$ is a simple cut-graph, $\Sigma\cutgraph H$ is always a disk.

Suppose now that a graph $G$ embedded on  $\Sigma$ and $H$ is a subgraph of $G$ that is a simple cut-graph of $H$.
We define $G\cutgraph H$ to be the graph embedded on $\Sigma\cutgraph H$ obtained from $G$ as follows.
First, let $\sigma$ be the (unique) facial walk of $H$ and note that each edge $e$ of $H$ is contained exactly twice in $\sigma$ and each vertex $v$ of $H$ is contained in $\sigma$ as many times as the degree of $v$ in $H$.
To obtain $G\cutgraph H$, we replace $H$ with a simple cycle $C_\sigma$ whose vertex set is the set of copies of vertices of $H$ and its edge set is the set of copies of edges of $H$ in the obvious way. Notice that $\sigma$ also prescribes for every edge $uv$ of $G$ between a vertex $u\in V(G)\setminus V(H)$ and a vertex $v\in V(H)$, to which copy of $v$ in $G\cutgraph H$ the vertex $u$ should be adjacent to (in $G\cutgraph H$).
See~\Cref{fig:cutopen} for an illustration.

\begin{figure}[ht]
  \centering
  \includegraphics[width=0.8\textwidth]{cut}
  \caption{Left: A graph $G$ embedded on a surface $\Sigma$ and a subgraph $H$ of $G$ (in blue) that is a simple cut-graph of $\Sigma$. Right: The graph $G\cutgraph H$ embedded on the surface $\Sigma\cutgraph H$ (which is homeomorphic to a disk); the blue vertices/edges are copies of the vertices/edges of $H$.}
  \label{fig:cutopen}
\end{figure}

We will use the following well-known result which appears in the literature under different formulations; see e.g.~\cite{BorradaileDT14,CabelloCL12algo,EricksonW05}.

\begin{lemma}\label{lem:genuscut}
  For every integer $k\ge 1$ and for every edge-weighted connected graph $G$ embedded on a surface $\Sigma$ of Euler genus at most $k$ and every vertex $u\in V(G)$, there is a subgraph $H$ of $G$ with the following properties:
  \begin{itemize}[nosep]
    \item  $H$ is a simple cut-graph of $\Sigma$, and
    \item  $V(H)$ is the union of the vertex sets of $\mathcal{O}(k)$ shortest paths in $G$ that have $u$ as a common endpoint.
  \end{itemize}
  \end{lemma}

We are now ready to proceed to the proof of~\Cref{thm:distprofiles}.
Employing~\Cref{lem:hat-to-normal},
we aim at bouding the VC-dimension of the set system defined by the anchor-distance profiles.
This is can be done by a suitable reduction to the planar setting using~\Cref{lem:genuscut}.



\begin{proof}[Proof of~\Cref{thm:distprofiles}]
  We assume that $G$ is connected -- the distance profiles of all vertices that are not connected to $R$ are equal.
  Let $T_R$ be a spanning tree of $G[R]$ and let $G_0$ be the graph obtained from $G$ after contracting $T_R$ into a single vertex $v_R$. 
  Consider an embedding of $G_0$ on a surface $\Sigma$ of Euler genus at most $k$.
  By~\Cref{lem:genuscut}, there is a subgraph $H_0$ of $G_0$ that is a simple cut-graph of $\Sigma$ and a family $\mathcal{P}_0$ of $\mathcal{O}(k)$ shortest
  paths in $G_0$, each with $v_R$ as an endpoint, such that $V(H_0)=\bigcup_{P\in\mathcal{P}_0}V(P)$.
  Furthermore, as Lemma~\ref{lem:genuscut} handles edge weights, we can slightly perturb
  the weights so that shortest paths in $G_0$ are unique and, in particular, 
  all shortest paths with one endpoint in $v_R$ form a tree.
  Since $H_0$ is a simple cut-graph of $\Sigma$, $G_0\cutgraph H_0$ is disk-embedded.
  Uncontracting $T_R$, we get a subgraph $H$ of $G$ such that $G\cutgraph H$ is disk-embedded.
  Let $\mathcal{P}$ be the family of projections of the paths of $\mathcal{P}_0$
  onto $G$ plus, for every $y \in R$, a zero-length path from $y$ to $y$. 
  Hence, $\mathcal{P}$ is an $R$-shortest paths collection of size $\mathcal{O}(k)$
  with $V(\mathcal{P}) = V(H)$.
  Furthermore, since in $G_0$ the paths of $\mathcal{P}_0$ formed a tree rooted
  at $v_R$, $\mathcal{P}$ is consistent.

  Note that due to~\Cref{lem:mile} we have that $\sum_{P\in\mathcal{P}} |M_R(P)|\le \mathcal{O}_k(|R|^2)$.
  Also, since $\mathcal{P}$ is consistent, 
  if $B$ are the vertices that are not in $R$
  (recall that vertices in $R$ are milestones) and have degree more than two in the graph obtained by the union of the paths in $\mathcal{P}$, then $|B|\leq |\mathcal{P}|-1$.
  Hence,
  \begin{equation}
    \sum_{P\in\mathcal{P}} |A_R(P)|\le \mathcal{O}_k(|R|^2).\label{eq:mile}
  \end{equation}
  We set $\mathcal{T}$ be the set of all vertices of $G\cutgraph H$
  that are copies of the anchor vertices $A_R(\mathcal{P})$.
  Every anchor vertex has $\mathcal{O}_k(1)$ copies in $\mathcal{Q}$
  and therefore, due to~\eqref{eq:mile},
  \begin{equation}
    |\mathcal{T}|=\mathcal{O}_k(|R|^2).\label{eq:term_size}
  \end{equation}
  For $s \in \mathcal{T}$, let $w(s) \in A_R(\mathcal{R})$ be the anchor vertex
  whose copy (in $G \cutgraph H$) is $s$. In the other direction, for $w \in A_R(\mathcal{R})$, let $S(w)$
  be the set of copies of $w$ in $G \cutgraph H$. 
  
  Let $U$ be the set of vertices of $G\cutgraph H$ that are \textsl{not} copies of vertices from $H$ (i.e., $U=V(G)\setminus V(H)$).
  We set $E_{\mathsf{out}}$ be the set of all edges $uv$ of $G\cutgraph H$ where $u\in U$ and $v$ is a copy of a vertex from $H$,
  i.e., $v\in V(G\cutgraph H)\setminus U$.
  We also set $E_{\mathsf{next}}$ be the set of all edges $uv$ of $G\cutgraph H$ where $u$ is a copy
  of an anchor vertex $w\in A_R(P)$ for some $P\in\mathcal{P}$ and $v$ is a copy of the neighbor of $w$
  in $P$ that \textsl{is not} in the prefix of $w$ in $P$.
  
  Let now $\widehat{G}$ be the graph obtained from $G\cutgraph H$ after the following modifications:
  \begin{itemize}[nosep]
    \item we subdivide $|V(G)|$-many times each edge in $E_{\mathsf{out}}\cup E_{\mathsf{next}}$,
    \item we introduce a new vertex $t$ and add, for every $s\in\mathcal{T}$, a path between $t$
    and $s$ of length $$d_{w(s),t}\coloneqq |V(G)| + \dist_G(w(s),R).$$
  \end{itemize}
  See~\Cref{fig:hatG}.
  Observe that since $G\cutgraph H$ is disk-embedded, $\widehat{G}$ is planar,
  because we may embed $t$ together with all the added paths outside of the disk containing $G\cutgraph H$.

  \begin{figure}[ht]
    \centering
    \includegraphics[width=0.3\textwidth]{hatG}
    \caption{An illustration of (a part of) the construction of the graph $\widehat{G}$. The squared vertices are copies of anchor vertices. The marked squared vertex is also a copy of a vertex in $R$. The highlighted edges are copies of edges of $H$ in $G\cutgraph H$, while the paths obtained by subdividing the edges of $E_{\mathsf{out}}\cup E_{\mathsf{next}}$ are depicted with dashed edges. Edges adjacent to $t$ correspond to paths of appropriate length.}
    \label{fig:hatG}
  \end{figure}

  For every $u\in U$, we define a function $\pi[u]$, mapping every $w\in A_R(\mathcal{P})$ to
  \[\pi[u](w)\coloneqq\min\{\dist_{\widehat{G}}(u,s): s \in S(w)\}+d_{w,t}-\dist_{\widehat{G}}(u,t).\]
  Also, we set $\widehat{\mathcal{X}}\coloneqq\{\widehat{X}_u\mid u\in U\}$,
  where for $u\in U$,
  $$\widehat{X}_u\coloneqq\left\{(w,i) \in A_R(\mathcal{P}) \times \{0,\ldots,|R|+1\}~|~i \leq \pi[u](w)\right\}.$$

  \begin{claim}\label{cl:vcd}
    The set system $\widehat{\mathcal{X}}$ has size $\mathcal{O}_k(|R|^{12})$.
  \end{claim}

  \begin{proof}
  We set $\mathcal{T}^+\coloneqq \mathcal{T}\cup \{t\}$.
  We start with the set system $\mathcal{C}^1\coloneqq\left\{C_u^1~\colon~u \in U\right\}$, where
  $$C_u^1\coloneqq\left\{(s,i) \in \mathcal{T}^+ \times \mathbb{Z}~|~i \leq \dist_{\widehat{G}}(u,s)-\dist_{\widehat{G}}(u,t)\right\}.$$
  As $\widehat{G}$ is planar, by~\Cref{thm:LW} we infer that $\mathcal{C}$ has VC-dimension at most 4.

  We now ``shift columns'' of $\mathcal{C}^1$. That is, define $\mathcal{C}^2 \coloneqq \left\{C_u^2~\colon~u \in U\right\}$, where
  $$C_u^2\coloneqq\left\{(s,i) \in \mathcal{T}^+ \times \mathbb{Z}~|~i \leq \dist_{\widehat{G}}(u,s)+d_{w(s),t}-\dist_{\widehat{G}}(u,t)\right\}.$$
  Clearly, the VC-dimension of $\mathcal{C}^1$ and $\mathcal{C}^2$ are equal: a set $Z \subseteq \mathcal{T}^+ \times \mathbb{Z}$
  shatters $\mathcal{C}^1$ if and only if the set $\{(s,d_{w(s),t}+i)~\colon~(s,i) \in Z\}$ shatters $\mathcal{C}^2$. 

  Now, let $\mathcal{C}^3$ be ``cropped'' $\mathcal{C}^2$:
  $\mathcal{C}^3 \coloneqq \left\{ C_u^3~\colon~u \in U\right\}$, where
  \[ C_u^3 \coloneqq C_u^2 \cap \left(\mathcal{T}^+ \times \{0,\ldots, |R|+1\}\right). \]
  Since restricting to a smaller universe cannot increase VC-dimension, $\mathcal{C}^3$ has VC-dimension at most $4$. 
  Since $|\mathcal{T}^+| = \Oh_k(|R|^2)$, by Sauer-Shelah lemma (Lemma~\ref{lem:sauer-shelah})
  we have $|\mathcal{C}^3| = \Oh_k(|R|^{12})$. 

  Now observe that for every $u_1,u_2 \in U$
  \begin{equation}\label{eq:CtoX}
   C^3_{u_1} = C^3_{u_2} \qquad \mathrm{implies} \qquad \widehat{X}_{u_1} = \widehat{X}_{u_2}. 
  \end{equation}
  Indeed, the assumption $C^3_{u_1} = C^3_{u_2}$ implies that
  for every $w \in A_R(\mathcal{P})$ and $s \in S(W)$ we have
  \begin{align*}
  & \max(0, \min(|R|+1, \dist_{\widehat{G}}(u_1,s)+d_{w,t}-\dist_{\widehat{G}}(u_1,t))) \\ 
  &= \max(0, \min(|R|+1, \dist_{\widehat{G}}(u_2,s)+d_{w,t}-\dist_{\widehat{G}}(u_2,t))).
\end{align*} 
  For fixed $w \in A_R(\mathcal{P})$, we take a minimum of the above expression over all $s \in S(w)$, obtaining:
  \begin{align*}
    & \max(0, \min(|R|+1, \min\{\dist_{\widehat{G}}(u_1,s)~\colon~s \in S(w)\}+d_{w,t}-\dist_{\widehat{G}}(u_1,t))) \\ 
    &= \max(0, \min(|R|+1, \min\{\dist_{\widehat{G}}(u_2,s)~\colon~s \in S(w)\}+d_{w,t}-\dist_{\widehat{G}}(u_2,t))).
  \end{align*} 
  This proves~\eqref{eq:CtoX}.
  From~\eqref{eq:CtoX}, we infer $|\widehat{\mathcal{X}}| \leq |\mathcal{C}^3| = \Oh_k(|R|^{12})$, as desired.
  \end{proof}

  We next relate the distance from a vertex $u\in U$ to $R$ (in $G$) and to $t$ (in $\widehat{G}$).
  \begin{claim}\label{cl:dist}
    For every $u\in U$, $\dist_G(u,R)=\dist_{\widehat{G}}(u,t) - 2|V(G)|$.
  \end{claim}
  \begin{proof}
    Fix $u\in U$.
    We first show that $\dist_G(u,R)\le \dist_{\widehat{G}}(u,t) - 2|V(G)|$.
    For this, consider a shortest path $\widehat{Q}$ in $\widehat{G}$ from $u$ to $t$.
    Observe that there is a vertex $s\in\mathcal{T}$ that is a copy of an anchor vertex $w$,
    such that $\widehat{Q}[s,t]$ is the path from $s$ to $t$ of length $d_{w,t}$
    added in the construction of $\widehat{G}$ from $G\cutgraph H$. Recall that $d_{w,t} =\dist_G(w,R)+|V(G)|$.
    Also, observe that $\widehat{Q}[u,s]$ contains at least one subdivided edge of $E_{\mathsf{out}}$, as it starts
    in $U$ and ends outside $U$, and otherwise corresponds to a walk from $u$ to $w$ in $G$.
    Therefore, we have
    \begin{align*}
      \dist_{\widehat{G}}(u,t)  = |\widehat{Q}| & = |\widehat{Q}[u,s]|+|\widehat{Q}[s,t]|\\ 
        & = |\widehat{Q}[u,s]| + \dist_G(w,R)+|V(G)|\\
        & \ge |V(G)| + \dist_G(u, w) + \dist_G(w,R) + |V(G)|\\
        & \ge \dist_G(u,R) + 2|V(G)|.
    \end{align*}
    
    We next show that $\dist_G(u,R)\ge \dist_{\widehat{G}}(u,t) - 2|V(G)|$.
    For this, consider a shortest path $Q$ in $G$ from $u$ to $R$. Let $y\in R$ be the unique vertex in $R\cap V(Q)$.
    Also, let $z$ be the first vertex of $Q$ (when ordering from $u$ to $y$)
    in $\bigcup_{P\in\mathcal{P}}V(P)$ and let $P\in\mathcal{P}$ be the path that $z$ is contained
    (if $z$ is contained to more than one paths, pick one of them arbitrarily).
    Also, let $w$ be the first vertex of $P[z,x^P]$ (when ordering from $z$ to $x^P$) that is an anchor vertex.
    Observe that $Q[u,z]$ corresponds to a path in $\widehat{G}$ from $u$ to a copy $s'$ of $z$
    that contains exactly one subdivided edge of $E_{\mathsf{out}}$ (and no edge of $E_{\mathsf{next}}$)
    and there is a copy of $P[z,w]$ in $\widehat{G}$ from $s'$ to a copy $s$ of $w$ 
    that contains no edge of $E_{\mathsf{out}} \cup E_{\mathsf{next}}$. 
    Therefore,
    \begin{align*}
      |Q| & = |Q[u,z]|+|Q[z,y]|& \\
      & = |Q[u,z]|+ |P[z,x^P]| & \text{\!\!\!($Q[z,y]$ and $P[z,x^P]$ being shortest paths from $z$ to $R$)}\\
      & = |Q[u,z]|+ |P[z,w]| + |P[w,x^P]|& \\
      & = |Q[u,z]|+ |P[z,w]| + \dist_G(w,R) & \text{($P$ being shortest path from a vertex $v^P$ to $R$)}\\
      & \ge \dist_{\widehat{G}}(u,s) - |V(G)| + d_{w,t} - |V(G)|& \\
      & \ge \dist_{\widehat{G}}(u,t) - 2|V(G)|. &  
    \end{align*} 
    Thus, we have $\dist_G(u,R) = |Q| \ge \dist_{\widehat{G}}(u,t)-2|V(G)|$, as desired.
  \end{proof}

  \begin{claim}\label{cl:dist2}
    For every $u \in U$ and $w \in A_R(\mathcal{P})$, it holds that
    \begin{align*}
    \widehat{\dist}(u,w) < \infty &\quad\mathrm{if\ and\ only\ if} \quad \widehat{\dist}(u,w) = \min\left\{\dist_{\widehat{G}}(u,s)~\colon~s \in S(w)\right\}-|V(G)|,\ \mathrm{and}\\
    \widehat{\dist}(u,w) = \infty &\quad\mathrm{if\ and\ only\ if} \quad \min\left\{\dist_{\widehat{G}}(u,s)~\colon~s \in S(w)\right\} > 2|V(G)|.
    \end{align*}
  \end{claim}
  \begin{proof}
    We first show that if $\widehat{\dist}(u,w) < \infty$, then 
    there exists $s \in S(w)$ with $\dist_{\widehat{G}}(u,s) \leq |V(G)| + \widehat{\dist}(u,w)$. 
    To this end, let $Q$ be a path from $u$ to $w$ in $G$ of length $\widehat{\dist}(u,w)$, as in the definition
    of $\widehat{\dist}(u,w)$. There exists $P \in \mathcal{P}$ with $w \in A_R(P)$ and a vertex $z \in V(P) \cap V(Q)$
    such that $Q$ decomposes into $Q[u,z]$ and $Q[z,w] = P[z,w]$, with all internal vertices of $Q[u,z]$ in $U$. 
    Then, $\widehat{G}$ contains a copy $s'$ of $z$ such that $Q[u,z]$ projects to a path from $u$ to $s'$
    with one subdivided edge of $E_{\mathsf{out}}$ (and no edge of $E_{\mathsf{next}}$) and also a copy of $P[z,w]$ from $s'$
    to a copy $s$ of $w$ with no subdivided edge of $E_{\mathsf{out}} \cup E_{\mathsf{next}}$.
    The concatenation of these two paths witness that $\dist_{\widehat{G}}(u,s) \leq |V(G)| + \widehat{\dist}(u,w)$, as desired.

    To finish the proof of the claim, it suffices to show that if there exists $s \in S(w)$ with
    $\dist_{\widehat{G}}(u,s) \leq 2|V(G)|$, then $\widehat{\dist}(u,w) \leq \dist_{\widehat{G}}(u,s) - |V(G)|$
    (in particular, $\widehat{\dist}(u,w) \neq \infty$).
    To this end, let $\widehat{Q}$ be a path in $\widehat{G}$ from $u$ to $s$ of minimum length. 
    Since $u \in U$ but $s \notin U$, $\widehat{Q}$ necessarily contains at least one subdivided edge of $E_{\mathsf{out}}$. 
    Since $|\widehat{Q}| \leq 2|V(G)|$, $\widehat{Q}$ contains exactly one edge of $E_{\mathsf{out}}$, no edge of 
    $E_{\mathsf{next}}$, and no edge incident with $t$. Consequently, there exists a vertex $s'$ on $\widehat{Q}$
    which is a copy of a vertex $z$ that lies in the prefix of $w$ on some path $P \in \mathcal{P}$ such that 
    $\widehat{Q}$ decomposes as $\widehat{Q}[u,s']$, which has all internal vertices in $U$, and $\widehat{Q}[s',s]$
    going along a copy of $P[z,w]$ to $s \in S(w)$. Hence, $\widehat{Q}$ corresponds to a path $Q$ in $G$
    from $u$ to $w$ that satisfies the requirements of the definition of $\widehat{\dist}(u,w)$
    and witnesses $\widehat{\dist}(u,w) \leq |\widehat{Q}| - |V(G)|$, as desired. 

    This finishes the proof of the claim.
  \end{proof}
  
  Using the two previous claims, we infer that for every $u \in U$ and $w \in A_R(\mathcal{P})$ it holds that
  \begin{equation}\label{eq:prof-to-dist}
    \hatprofile{R}{\mathcal{P}}{u}(w) = \min(|R|+1, \pi[u](w)).
  \end{equation}
  Indeed, 
  \begin{align*}
  \min\left(|R|+1, \pi[u](w)\right) &= \min\left(|R|+1,\min\left\{\dist_{\widehat{G}}(u,s)~\colon~s\in S(w)\right\}+d_{w,t}-\dist_{\widehat{G}}(u,t)\right)\\
  &=\min\big(|R|+1, \min\left\{\dist_{\widehat{G}}(u,s)~\colon~s\in S(w)\right\} - |V(G)| &\\
  &\qquad\qquad\qquad\qquad + \dist_G(w,R)-\dist_G(u,R)\big) &\text{by Claim~\ref{cl:dist}}\\
  &=\min\left(|R|+1, \widehat{\dist}(u,w) + \dist_G(w,R)-\dist_G(u,R)\right)&\text{by Claim~\ref{cl:dist2}}\\
  &=\hatprofile{R}{\mathcal{P}}{u}(w).
  \end{align*}
  Here, in the third step we used the estimate $\dist_G(u,R) - \dist_G(w,R) \leq |U| \leq |V(G)|-|R|$, 
  so if $\min\left\{\dist_{\widehat{G}}(u,s)~\colon~s\in S(w)\right\} > 2|V(G)|$
  (which is equivalent to $\widehat{\dist}(u,w) = \infty$ by Claim~\ref{cl:dist2}),
  then the minimum is attained at value $|R|+1$.

  For every $u\in U$, we set
    $$B_u\coloneqq\left\{(w,i) \in A_R(\mathcal{P}) \times \mathbb{Z}~|~i \leq \hatprofile{G}{R}{u}(w)\right\}.$$
  Claim~\ref{cl:vcd} and~\eqref{eq:prof-to-dist} imply that the set system $\{B_u~\colon~u\in U\}$
    has size $\mathcal{O}_k(|R|^{12})$.

  
  
  Now, for every $v\in V(G)$, we set $$S_v\coloneqq\left\{(s,i) \in R \times \{-|R|,\ldots,|R|\}~|~i \leq \distprofile{R}{s_R}{v}(s)\right\}.$$
  The bound on the size of the set system $\{B_u~\colon~u \in U\}$ and~\Cref{lem:hat-to-normal}
  imply that the size of $\left\{S_u~\colon~u \in U \right\}$ is bounded by $\Oh_k(|R|^{12})$.
  We conclude the proof of the lemma by bounding the size of $\left\{S_u~\colon~u \in V(G)\setminus U \right\}$. For this, note that every vertex $v\in V(G)\setminus U$ is either a milestone for some path $P\in\mathcal{P}$ or a vertex in the neutral prefix of a milestone.
  In the latter case, there is a path $P\in\mathcal{P}$ and a milestone $w\in M_R(P)$ such that $S_v=S_w$. Therefore, we have
  $$|\left\{S_u~\colon~u \in V(G)\setminus U \right\}|\le \sum_{P\in\mathcal{P}} |M_R(P)|\le \mathcal{O}_k(|R|^2),$$
  where the second inequality follows from~\eqref{eq:mile}.
  Hence, the size of $\left\{S_v~\colon~v \in V(G)\right\}$ is at most $$|\left\{S_u~\colon~u \in U\right\}|+|\left\{S_u~\colon~u \in V(G)\setminus U\right\}|=\Oh_k(|R|^{12}).$$
  This finishes the proof of Theorem~\ref{thm:distprofiles}.
\end{proof}


\section{Bounded Euler genus graphs with apices: proof of Theorem~\ref{thm:main-apices}}\label{sec:algo-genus}

In this section we prove Theorem~\ref{thm:main-apices}.
(Note that Theorem~\ref{thm:main-genus} is a special case of Theorem~\ref{thm:main-apices}
 for $k=1$.)
We start by deriving the following corollary from \Cref{t:orth_query}.

\begin{corollary}\label{l:max_min_query}
Let $V$ be a set of $n$ points in $\mathbb{R}^d$.
There is a data structure that uses $\Oh \left( d n \log^{d - 2} n \right)$ preprocessing time, $\Oh \left( d n \log^{d - 2} n \right)$ memory and answers the following queries in time $\Oh \left( d \log^{d - 2} n \right)$: given $r_1, \dots, r_d \in \mathbb{R}$, find $\max_{v \in V} \min_{i \in [d]} (v_i + r_i)$, where $v_i$ denotes the $i$th coordinate of $v$.
\end{corollary}

\begin{proof}
    Fix query parameters $r_1, \dots, r_d \in \mathbb{R}$.
    Let $\lambda \coloneqq \max_{v \in V} \min_{i \in [d]} (v_i + r_i)$ denote the answer we want to find.
    
    We say that a pair $(v,i) \in V \times [d]$ is \emph{good} if
    for every $j \in [d]$, it holds that $v_j - v_i \geq r_i -r_j$.
    Let
    $$
    		\lambda' = \max \left\{ v_i + r_i \colon i\in [d], v\in V,\textrm{ and }(v,i)\textrm{ is good} \right\}.
    $$
    We claim that 
    \begin{equation}
    \label{eq:query-lambda}
    \lambda = \lambda'.
   \end{equation}
   Let $v' = \argmax_{v \in V} (\min_{i \in [d]} v_i + r_i)$ and let $i' = \argmin_{i \in [d]} v'_i + r_i$. By the choice of $i'$, for each $j$ we have $v'_j+r_j\geq v'_{i'}+r_{i'}$, implying $v'_j - v'_{i'} \geq r_{i'} - r_j$. Hence $(v',i')$ is good, so $\lambda' \geq v'_{i'} + r_{i'} = \lambda$.
    
    On the other hand, consider a good pair $(v', i')$ maximizing $v'_{i'} + r_{i'}$.
    The goodness of $(v',i')$ implies that $i' = \argmin_{i \in [d]} v'_i + r_i$, hence $\lambda \geq \min_{i \in [d]} v'_i + r_i = v'_{i'} + r_{i'} = \lambda'$. This proves~\eqref{eq:query-lambda}.
    
    For every $i \in [d]$, we set $V_i$ to be the set
    $$ 
    		\{ (v_1 - v_i, v_2 - v_i, \dots, v_{i - 1} - v_i, v_{i + 1} - v_i, \dots, v_d - v_i) : v \in V \}\subseteq \mathbb{R}^{d-1},
    	$$
    	and set $w_i(v) \coloneqq v_i$. Let $\mathbb{D}_i$ be the data structure obtained by applying \Cref{t:orth_query} to $V_i$ and $w_i$. Consider the suffix range
    $$
    		R_i\coloneqq \mathsf{Range}(r_i - r_1, r_i-r_2, \dots, r_i - r_{i - 1}, r_i - r_{i + 1}, \dots,  r_i - r_d)\subseteq \mathbb{R}^{d-1}.
    $$
    Now, by \eqref{eq:query-lambda} we have that
    $$
    		\lambda = \max \left\{ r_i + \max \{ w_i(v) : v \in V_i \cap R_i \}\colon i\in [d] \right\}.
    $$
    This value can be computed by asking $d$ queries to the data structures $\mathbb{D}_i$, for $i\in [d]$. This gives us a data structure satisfying the conditions given in the lemma statement.
\end{proof}

The main work in the proof of Theorem~\ref{thm:main-apices} will be done in the following lemma,
which provides a fast computation of eccentricities once a suitable division is provided on input.
We adopt the notation for divisions introduced in the statement of \Cref{t:r_division}.

\begin{lemma}\label{l:main_ecc}
Fix constants $0 < \alpha, \gamma, \rho < 1$ and $k \in \mathbb{N}$. Assume we are given a connected graph $G$ on $n$ vertices with $O(n)$ edges with positive integer weights, a subset of vertices $X$, a subset of apices $A \subseteq V(G)$ of size at most $k$, and a family $\mathcal{R}$ with $V(G) \setminus A = \bigcup \mathcal{R} $ such that the following conditions are satisfied:
\begin{itemize}[nosep]
	\item $\sum_{R \in \mathcal{R}} |\partial R| \leq \Oh(n^\gamma)$;
	\item for every $R \in \mathcal{R}$, $|R| \leq \Oh(n^\rho)$ and $G[R]$ is connected and contains $O(|R|)$ edges; and
	\item for every $R \in \mathcal{R}$, the number of distance profiles in $G-A$ on $\partial R$ is of $\Oh(n^\alpha)$.
\end{itemize}
Then, we can compute $X$-eccentricity of every vertex of $G$ in time $\Oh(n^{\gamma + 2\rho} \log n + (n^{1 + \gamma} + n^{1 + \alpha}) \log^{k - 1} n)$.
\end{lemma}

\begin{proof}
    Let $G' \coloneqq G - A$ and $X' \coloneqq X \cap V(G')$.    Denote $A \coloneqq \{a_1,a_2,\ldots,a_k\}$. We first describe the procedure, and then discuss its time complexity.

    For~every $a \in A$ and $u \in V(G)$, we compute distance between $a$ and $u$ denoted $d_A(a, u)$.
    
    \medskip
    \emph{Step 1.} \ 
    We start by precomputing the following information for every region $R\in \mathcal{R}$.
    For all $u, v \in R$, we compute the distance between $u$ and $v$ in $G'[R]$, denoted $d_R(u, v)$.
    For all $s \in \partial R, u \in V(G')$, we compute the distance between $s$ and $u$ in $G'$, denoted $d_{\partial R}(u,s)$.
    We arbitrarily pick a pivot vertex $s_R \in \partial R$, and for brevity denote $p_R[u] \coloneqq \distprofile{\partial R}{s_R}{u}$, where the profile is considered in $G'$. That is, $p_R[u]$ is the $(\partial R)$-profile of $u$ with respect to $s_R$:
    $$p_R[u](s) = d_{\partial R}(u, s) - d_{\partial R}(u, s_R),\quad \textrm{for all }u \in V(G')\textrm{ and } s \in \partial R.$$
    We define $P_R \coloneqq \{ p_R[u] \colon u \in V(G')\}$. By our assumption, we have $|P_R| \leq \Oh(n^\alpha)$. Finally, for every profile $p \in P_R$, we list all vertices $v \in X' \setminus R$ such that $p_R[v] = p$ and set up the data structure of \Cref{l:max_min_query} for the points $(d_A(a_1, v), \dots, d_A(a_k, v), d_{\partial R_u}(s_R, v))$; denote it by $\mathbb{D}_{R, p}$.
    
    \medskip
    \emph{Step 2.} \ 
    For every $u \in V(G)$, we compute $\ecc_X(u)$ as follows. If $u \in A$, the answer is $\max_{v \in X} d_A(u, v)$. Hence, we may assume $u \not\in A$.
    Let $R_u$ denote any region of $\mathcal{R}$ containing $u$. For every $v \in R_u$, the shortest path from $u$ to $v$ in $G$ either:
    \begin{itemize}[nosep]
        \item goes through an apex, in which case its length is $\min_{a \in A} d_A(a, u) + d_A(a, v)$; or
        \item is disjoint from $A$ and intersects $\partial R_u$, in which case its length is $\min_{s \in \partial R_u} d_{\partial R_u}(s, u) + d_{\partial R_u}(s, v)$;~or
        \item is contained entirely in $R_u$, in which case its length is $d_{R_u}(u, v)$.
    \end{itemize}
    The length of this path is therefore the minimum among the three quantities.
    Using the above observation, we compute $\dist_G(u, v)$ explicitly for each $v \in R_u$, and define $\Delta^{R_u}_u \coloneqq \max_{v \in R_u \cap X} \dist_G(u, v)$.
    
    For every $v \in V(G) \setminus (A \cup R_u)$, the shortest path between $u$ and $v$ either crosses $A$ or $\partial R_u$. The length of such path avoiding $A$ is
    $$ \min_{s \in \partial R_u} d_{\partial R_u}(s, u) + d_{\partial R_u}(s, v) = 
       d_{\partial R_u}(s_R, v) + \min_{s \in \partial R_u} \left( d_{\partial R_u}(s, u) + p_{R_u}[v](s) \right). $$

We partition the vertices $v$ by their profile $p_{R_u}[v]$ and for every $p \in P_{R_u}$, we compute the maximum distance to a vertex with profile $p$ separately. Let $V_p = \{ v \in X' \setminus R_u \mid p_{R_u}[v] = p \}$. For every $v \in V_p$, we have
    $$ \dist_G(u, v) = \min \left( \min_{a \in A} d_A(a, u) + d_A(a, v), d_{\partial R_u}(s_R, v) + \min_{s \in \partial R_u} \left( d_{\partial R_u}(s, u) + p(s) \right) \right). $$
    We set $r_i \coloneqq d_A(a, u)$ for $i \in [k]$, and $r_{k + 1} \coloneqq \min_{s \in \partial R_u}  \left( d_{\partial R_u}(s, u) + p(s) \right)$. Now,
    $$ \max_{v \in V_p} \dist_G(u, v) = \max_{v \in V_p} \min(r_1 + d_A(a_1, v), \dots, r_k + d_A(a_k, v), r_{k + 1} + d_{\partial R_u}(s_R, v)).$$
    This value can be computed by querying $r_1, \dots, r_{k + 1}$ to the data structure $\mathbb{D}_{R_u, p}$. We define $\Delta^{V(G) \setminus (A \cup R_u)}_u$ as the maximum of such values over all $p \in P_{R_u}$.
    
    Finally, we set $\Delta^{A}_u \coloneqq \max_{a \in A \cap X} d_A(a, u)$, and report $\ecc_X(u) = \max \left(\Delta^{A}_u, \Delta^{R_u}_u, \Delta^{V(G) \setminus (A \cup R_u)}_u \right)$.
    
    It remains to argue that this algorithm can be implemented in the desired running time. For any source $u \in V(G)$, distance from $u$ to all vertices of $G$ can be calculated in time $\Oh((|V(G)| + |E(G)|) \log |V(G)|)$ using Dijkstra's algorithm. Therefore:
    \begin{itemize}[nosep]
        \item computing $d_A(a,\cdot)$ for all $a$ can be done in time $\Oh (n \log n)$,
        \item computing $d_{\partial R}(\cdot,\cdot)$ for all $R$ can be done in time $\Oh (n\log n \cdot \sum_{R \in \mathcal{R}} |\partial R| ) \leq \Oh (n^{1 + \gamma} \log n)$,
        \item computing $d_R(\cdot,\cdot)$ for all $R \in \mathcal{R}$ can be done in time $\Oh (|\mathcal{R}| n^{2\rho} \log n) \leq \Oh (n^{\gamma + 2\rho} \log n)$; constructing $G[R]$ takes $\Oh(|R|^2 \log n) = \Oh(n^{2\rho} \log n)$ time and calculating all pairs shortest paths can be done in time $\Oh(|R||E(G[R])| \log n) = \Oh(n^{2\rho} \log n)$.
    \end{itemize}
    Finally, the total size of the data structures $\mathbb{D}_{R, p}$ over all $R, p$ is $\Oh(|\mathcal{R}| n) = \Oh (n^{1 + \gamma})$, hence we can construct them in time $\Oh (n^{1 + \gamma} \log^{k - 1} n)$.
    
    Consider $u \in V(G) \setminus A$ fixed in step 2. Computing $\Delta^{R_u}_u$ takes $\Oh (|R| \cdot |\partial R_u|)$ time. Computing $\Delta^{A}_u$ can be done in constant time. Computing $\Delta^{V(G) \setminus (A \cup R_u)}_u$ requires asking $|P_{R_u}|$ queries to some $\mathbb{D}_{R, p}$, which takes $\Oh (n^{\alpha} \log^{k - 1} n)$ time in total.
    In total, step 2 for all vertices $u$ can be done in time $\Oh (n^{1 + \alpha} \log^{k - 1} n + n^\rho \cdot \sum_{u \in V(G) \setminus A} |\partial R_u|) = \Oh (n^{1 + \alpha} \log^{k - 1} n + n^{\gamma + 2\rho})$.
    
    We conclude that the total running time is $\Oh(n^{\gamma + 2\rho} \log n + (n^{1 + \gamma} + n^{1 + \alpha}) \log^{k - 1} n)$.
\end{proof}

The next statement is a reformulation of Theorem~\ref{thm:main-apices}. 

\begin{theorem}\label{t:main_bdgenus_apices}
Fix constants $k, g \in \mathbb{N}$. Let $\mathcal{C}$ denote the class of all graphs that can be obtained by taking a graph $G$ of Euler genus bounded by $g$, and adding $k$ apices adjacent arbitrarily to the rest of $G$ and to each other. Then there is an algorithm that given an unweighted graph $G$ belonging to $\mathcal{C}$, together with its set of apices $A$, computes the eccentricity of every vertex in time $\Oh_{k,g} \left( n^{1 + \frac{24}{25}} \log^{k - 1} n \right)$.
\end{theorem}

\begin{proof}
Let $A = \{a_1, \dots, a_k\}$ denote the set of apices and let $G' = G - A$. Fix $\rho\coloneqq \frac{2}{25}$.
Since graphs of bounded genus exclude some fixed clique as a minor,
by \Cref{t:r_division} (with $\varepsilon = \rho/2$)
we can find an $\Oh(n^\rho)$-division $\mathcal{R}$ of $G'$
satisfying $\sum_{R \in \mathcal{R}} |\partial R| = \Oh (n^{1 - \frac{\rho}{2}})$
in time $\Oh(n^{1 + \rho})$ .
By Theorem~\ref{thm:distprofiles}, the graph $G'$ has a degree $12$ polynomial bound on the number of distance profiles. In particular, the number of profiles on every $\partial R$ is of $\Oh(|R|^{12}) = \Oh(n^{12\rho})$. Let $X \coloneqq V(G)$, $\gamma \coloneqq 1 - \frac{\rho}{2} = \frac{24}{25}$ and $\alpha \coloneqq 12\rho = \frac{24}{25}$. Now,  applying \Cref{l:main_ecc} gives us an algorithm computing all eccentricities in time $\Oh(n^{1 + \frac{24}{25}} \log^{k - 1} n)$.
\end{proof}


\section{The general case: Proof of \texorpdfstring{\Cref{thm:main-decomp}}{Theorem 1.6}}\label{sec:algo}

First, we show that data structure of \Cref{l:max_min_query} can be used to compute distances witnessed by shortest paths that pass through a constant-size separator.

\begin{lemma}\label{l:single_adhesion}
Fix a constant $k \in \mathbb{N}$. There exists an algorithm which as the input receives an edge-weighted graph $G$ on $n$ vertices and $m$ edges together with a partition of its vertices into three sets $A, B, C$ such that $|B| \leq k$ and there are no edges between $A$ and $C$, and as the output computes $\max_{c \in C} \dist(a, c)$ for every $a \in A$. The running time is $\Oh(m \log n + n \log^{k - 1} n)$.
\end{lemma}

\begin{proof}
Let $B = \{b_1, \ldots, b_k\}$. For any $a \in A, c \in C$, we have $\dist(a, c) = \min_{i \in [k]} \dist(a, b_i) + \dist(c, b_i)$. First, we run Dijkstra's algorithm from every vertex in $B$ to find $\dist(v, b_i)$ for every $v \in V(G)$ and $i \in [k]$. Next, we use \Cref{l:max_min_query} to construct a data structure $\mathbb{D}$ for the point set $\{(\dist(c, b_1), \dots, \dist(c, b_k))\colon c\in C\}\subseteq \mathbb{R}^k$. Now, the value $\max_{c \in C} \dist(a, c)$ for any given $a$ is equal to the answer of $\mathbb{D}$ to the query with argument $(\dist(a, b_1), \dots, \dist(a, b_k))$.
\end{proof}

After computing the distances over a constant-size separator, we will use the following observation to simplify one of the sides of the separation.

\begin{lemma}\label{l:inserting_paths}
Let $G$ be a edge-weighted connected graph and let $A, B, C$ be a partition of its vertices such that there are no edges between $A$ and $C$. For every pair of vertices $u, v \in B$, let $P_{u, v}$ be any shortest path from $u$ to $v$ with all internal vertices in $C$ (assuming such a path exists).

Let $G'$ denote a graph obtained from $G[A \cup B]$ by adding an edge from $u$ to $v$ of weight equal to the length of $P_{u, v}$, for all $u, v \in B$ for which $P_{u, v}$ exists. Then,  $$\dist_G(s, t) = \dist_{G'}(s, t)\qquad\textrm{for all }s,t\in A\cup B.$$
\end{lemma}
\begin{proof}
Let $G''$ be the graph obtained by adding new edges of $G'$ to $G$.
Fix any $s, t \in A \cup B$ and let $P$ denote the shortest path from $s$ to $t$ in $G''$ which minimizes the number of vertices from $C$ visited. Naturally, the weight of $P$ is equal $\dist_G(s, t)$. Assume that such path visits at least one vertex of $C$. Then, the path $P$ is of the form $s \xrightarrow{P_1} x \xrightarrow{P_2} y \xrightarrow{P_3} t$, where $x, y \in B$ and all the internal vertices of $P_2$ are in $C$. By the construction of $G'$, $P_2$ can be replaced with a direct edge from $x$ to $y$ of the same weight. We obtain a same weight path with a smaller number of vertices of $C$ visited, which is a contradiction. Therefore, $P$ is entirely contained in $A \cup B$, hence it exists in $G'$. This shows that $\dist_G(s, t) = \dist_{G'}(s, t)$.
\end{proof}


The next lemma encapsulates the main algorithmic content of the proof of \Cref{thm:main-decomp}. The algorithm will split the tree decomposition provided on input into smaller parts for which the eccentricities are easier to calculate. We use the following lemma to handle a single such part.
\begin{lemma}\label{l:star}
Fix constants $k, g \in \mathbb{N}, 0 < \delta < \frac{1}{54}$. Assume we are given $n \in \mathbb{N}$, an edge-weighted graph $G$ on at most $n$ vertices with a weight function $w \colon E(G) \to \mathbb{N}$, a vertex subset $A$ and a collection of non-empty vertex subsets $V_0, V_1, \dots, V_\ell$ satisfying the following conditions:
\begin{itemize}[nosep]
	\item The sum of weights of all the edges in $G$ is bounded by $\Oh(n)$.
	\item $V(G) \setminus A = V_0 \cup V_1 \cup \dots \cup V_\ell$.
	\item $|A| \leq k$.
	\item For every $i \in [\ell]$, $G[V_i \setminus V_0]$ is connected, $N_G(V_i \setminus V_0) = V_i \cap V_0$, $|V_i| = \Oh(n^\delta)$, and $|V_0 \cap V_i| \leq 4$.
	\item For all $i, j \in [\ell], i \neq j$, $V_i \setminus V_0$ and $V_j \setminus V_0$ are disjoint and non-adjacent in $G$.
	\item Every edge $uv \in E(G)$ with $u, v \not\in A$ is contained in $G[V_i]$ for some $i\in \{0,1,\ldots,\ell\}$.
	\item The graph obtained by taking $G[V_0]$ and adding a clique on $V_0 \cap V_i$ for every $i \in [\ell]$ has Euler genus bounded by $g$.
\end{itemize}
Then, we can compute the eccentricity of every vertex of $G$ in time $\Oh \left( n^{1 + \frac{150 + 54 \delta}{151}} \log^k n \right)$.
\end{lemma}

\begin{proof}
Fix $\delta' = \frac{1 + 97 \delta}{151}$; we have $\delta' - \delta = \frac{1 - 54\delta}{151} > 0$.
Let $E_i$ denote the set of edges with one endpoint in $V_i$ and the other endpoint in $V_i \setminus V_0$. For $i \in [\ell]$, we shall say that $V_i$ is {\em{heavy}} if the sum of weights of $E_i$ is larger than $n^{\delta'}$. Since the sets $E_i$ are pairwise disjoint and the total sum of weights of all the edges is bounded by $\Oh(n)$, the number of heavy subsets is bounded by $\Oh(n^{1 - \delta'})$. Without loss of generality, we may assume that $V_{\ell' + 1}, \dots, V_\ell$ are heavy and $V_1, \dots, V_{\ell'}$ are not, for some $\ell'\in \{0,\ldots,\ell\}$.


For any source vertex $s$, we can calculate distances from $s$ to every vertex of $G$  using breadth first search in time $\Oh(\sum_{e \in E(G)} w(e)) = \Oh(n)$.
In particular, for every $\ell' < i \leq \ell$, we can compute the distances from every vertex of $V_i$ to every vertex of $G$ in total time $\Oh(n^{2 - \delta' + \delta})$, because $$|V_{\ell'+1}\cup \ldots\cup V_{\ell}|\leq n^{1-\delta'}\cdot \Oh(n^\delta)=\Oh(n^{1-\delta'+
\delta}).$$
Additionally, we calculate distances $\dist_G(a, v)$ for every $a \in A, v \in V(G)$ in time $O(n)$.

For every $i \in [\ell]$ and $u,v \in V_0 \cap V_i$, there exists a shortest path $P_{i,u,v}$ from $u$ to $v$ with all internal vertices belonging to $V_i - V_0$ due to the assumption that $G[V_i - V_0]$ is connected and $N_G(V_i - V_0) = V_i \cap V_0$. Therefore, the distance from $u$ to $v$ is bounded by the sum of weights of edges in $E_i$. In particular, for $i \in [\ell']$, $\dist_G(u, v) \leq n^{\delta'}$.

We define $\widetilde{G}$ to be the graph obtained by taking $G[A \cup V_0 \cup \dots \cup V_{\ell'}]$ and applying the following operation for every $i \in \{\ell' + 1, \dots, \ell\}$:
for each pair of vertices $u, v \in A \cup (V_0 \cap V_i)$, add an edge in $\widetilde{G}$ between $u$ and $v$ with weight equal to the total weight of $P_{i,u,v}$. For a fixed $i, u$, we can find $P_{i, u, v}$ for all $v$ using breadth first search in time $\Oh(n)$. Taking a sum over all $i, u$, we get that $\tilde{G}$ can be computed in total time $\Oh(n^{2 - \delta'})$.


\begin{claim}\label{cl:wG}
The sum of the edge weights in $\widetilde{G}$ is $\Oh(n)$. Moreover, for all $u, v \in V(\widetilde{G})$, we have $\dist_{\widetilde{G}}(u, v) = \dist_{G}(u, v)$.
\end{claim}

\begin{proof}
Consider $i \in \{\ell' + 1, \dots, \ell\}$ and any $u, v \in A \cup (V_0 \cap V_i)$ for which we added an edge. Its weight is bounded by the sum of weights of edges in $E_i$. Therefore, the total weight of all edges added is at most
$$
\sum_{i \in \{\ell' + 1, \dots, \ell\}} \left( |A \cup (V_0 \cap V_i)|^2 \sum_{e \in E_i} w(e) \right) \leq (4 + k)^2 \sum_{e \in E(G)} w(e) = \Oh(n).
$$
This proves the first part of the claim.

For the second part of the claim, consider any $i \in \{\ell' + 1, \dots, \ell \}$ and observe that by our assumptions, $A \cup (V_0 \cap V_i)$ separates $(V_0 \cup \dots \cup V_{\ell'} \cup V_{i + 1} \cup \dots \cup V_\ell) \setminus V_i$ from $V_i \setminus V_0$. Hence it suffices to repeatedly apply \Cref{l:inserting_paths}.
\end{proof}

For every $u \in V(\widetilde{G})$, we have $\ecc_G(u) = \max(\ecc_{\widetilde{G}}(v), \max_{v \in V(G) \setminus V(\widetilde{G})} \dist_G(u, v))$. Note, that we already know all the distances $\dist_G(u, v)$ for $v \in V(G) \setminus V(\widetilde{G})$. Similarly, we can already compute $\ecc_G(u)$ for every $u \in V(G) \setminus V(\widetilde{G})$. Therefore, it remains to compute $\ecc_{\widetilde{G}}(v)$ for each $v \in V(\widetilde{G})$. Our goal is to show that this can be done efficiently using \Cref{l:main_ecc}.

Now, let $G'$ be the graph obtained from $\tilde{G}$ by replacing every edge $e$ non-indicent to $A$ with $w(e)\geq 2$ with a path of length $w(e)$ consisting of unit-weight edges. This operation again preserves the distances. Since the sum of edge weights in $\tilde{G}$ is of $\Oh(n)$, the total number of vertices in $G'$ is of $\Oh(n)$. For $0 \leq i \leq \ell'$, we write $V'_i$ to denote the set $V_i$ together with all the vertices added as a part of a path between two endpoints in $V_i$.
As $V_i$ is not heavy for each $i\in [\ell']$, we have
$$
|V'_i \setminus V'_0| \leq |V_i| + \sum_{e \in E_i} w(e) = \Oh(n^{\delta'})\qquad \textrm{for all }i\in [\ell'].
$$

Let $G_0$ denote the graph $G'[V'_0]$ and let $G_0^*$ denote the graph $G'- A$ with $V'_i - V'_0$ contracted to a single vertex $v_i^*$, for each $i \in [\ell']$; note that, all edges of $G_0$ and $G_0^*$ have unit weight.

\begin{claim}
	The graph $G_0^*$ is does not contain $K_{t}$ as a minor, where $t = \Oh(\sqrt{g})$.
\end{claim}

\begin{proof}
Let $\bar{G}_0$ denote the graph obtained by taking $G_0$ and adding a clique on $V_0 \cap V_i$ for every $i \in [\ell']$.
By lemma assumptions and the fact that subdividing edges does not increase the Euler genus, $\bar{G}_0$ has Euler genus at most $g$. In particular, $\bar{G}_0$ is $K_{t'}$-minor-free for some $t' = \Oh(\sqrt{g})$, because the Euler genus of $K_{t'}$ is $\Omega({t'}^2)$.

Similarly, let $\bar{G}_0^*$ be the graph obtained by taking $G_0^*$ and adding a clique on each $V_0 \cap V_i$.
Note, that $\bar{G}_0^* - \{v_1^*, \dots, v_{\ell'}^*\}$ is precisely $\bar{G}_0$. Let $t = \max(t', 6)$.
Recall that a minor model of a clique $K_t$ consists of $t$ pairwise vertex-disjoint connected subgraphs, called
branch sets, such that there is at least one edge between each pair of the branch sets.
Consider a minor model $\varphi$ of $K_{t}$ in $\bar{G}^*_0$.
Note that $\varphi$ cannot contain any singleton branch set of the form $\{v^*_i\}$, for the degree of $v^*_i$ in $\bar{G}^*_0$ is at most $4 < t - 1$. Furthermore, since $N_{\bar{G}^*_0}(v^*_i) = V_0 \cap V_i$, any branch set containing $v^*_i$ and at least one other vertex contains some $u \in V_0 \cap V_i$, and $N_{\bar{G}^*_0}(v^*_i)\subseteq N_{\bar{G}^*_0}(u)$, hence removing $v^*_i$ from this branch set preserves the model. Therefore, we can assume without loss of generality that all branch sets of $\varphi$ are disjoint from $\{v^*_1, \dots, v^*_{\ell'}\}$, hence $\varphi$ is a minor model of $K_{t}$ in $\bar{G}_0$. This is a contradiction, as $t \geq t'$ and $\bar{G}_0$ is $K_{t'}$-minor-free. Therefore, $\bar{G}_0^*$ is $K_t$-minor-free, hence $G_0^*$ also.
\end{proof}

Let $\rho' = \frac{2 - 108 \delta}{151} > 0$. The graph $G^*_0$ is a unit-weight graph and is $K_{t}$-minor-free.
Hence, by applying \Cref{t:r_division} to $G^*_0$ (with $\varepsilon = \rho'/2$)
we obtain an $n^{\rho'}$-division $\mathcal{R}_0$ in time $\Oh(n^{1 + \rho'})$.
We extend it to $G' - A$ by mapping every contracted vertex $v^*_i$ to $N_{G' - A}[V'_i - V'_0] = (V'_i - V'_0) \cup (V_0 \cap V_i)$. Formally, we put $V''_i \coloneqq N_{G' - A}[V'_i - V'_0]$ and 
$$
\mathcal{R} \coloneqq \left\{ (R_0 \cap V'_0) \cup \bigcup_{i \colon v^*_i \in R_0} V''_i \colon R_0 \in \mathcal{R}_0 \right\}.
$$

Now, we argue that $\mathcal{R}$ is a reasonable division of $G' - A$. Clearly, all sets $R \in \mathcal{R}$ are connected in $G' - A$. Pick any $R \in \mathcal{R}$ and let $R_0$ be its corresponding set in $\mathcal{R}_0$.
Every vertex $v^*_i$ is mapped to a set of size $\Oh(n^{\delta'})$, therefore
$$|R| \leq |R_0| \cdot \Oh(n^{\delta'}) = \Oh(n^{\rho' + \delta'}).$$

By our construction, for every $i \in [\ell']$, $R$ is either disjoint from $V'_i - V'_0$ or contains whole $N_{G' - A}[V'_i - V'_0]$. This means that no vertex belonging to any $V'_i - V'_0$ can be in $\partial R$, hence $\partial R \subseteq V'_0$.

Pick any $u \in \partial R \cap R_0$. Assume that $u \not\in \partial R_0$. Then every vertex of $N_{G_0^*}(u)$ must be in $R_0$, hence $N_{G - A'}(u) \subseteq R$, which is a contradiction. This means that $\partial R \cap R_0 \subseteq \partial R_0$.

Pick any $u \in \partial R - R_0$. Then, $u \in V_0 \cap V_i$ for some $i \in [\ell']$ such that $v_i^* \in R_0$. Moreover, $v_i^* \in \partial R_0$ and is adjacent to $u$ in $G_0^*$. The number of such $u$ is bounded by $4 |\partial R_0 \cap \{ v_1^*, \dots, v_{\ell'}^* \}|$.

Putting two cases together, we obtain:
$$
\sum_{R \in \mathcal{R}} |\partial R| = \sum_{R \in \mathcal{R}} \left(|\partial R \cap R_0| + |\partial R - R_0|\right) \leq \sum_{R_0 \in \mathcal{R}_0} \left(|\partial R_0| + 4 |\partial R_0 \cap \{ v_1^*, \dots, v_{\ell'}^* \}|\right) = \Oh(n^{1 - \frac{1}{2}\rho'}).
$$

It remains to show the following claim.

\begin{claim}
Pick any $R \in \mathcal{R}, s_R \in R$. The number of different distance profiles on $R$ relative to $s_R$ in $G' - A$ is of $\Oh(n^{48\rho' + 54\delta'})$.
\end{claim}
\begin{proof}
We look at every vertex $v \in V(G') \setminus A$ and consider three cases: $v \in R$, $v \in V'_0$, and $v \in V'_i \setminus (V'_0 \cup R)$ for some $i \in [\ell']$. By our construction, $R \cap V'_0$ is non-empty, hence w.l.o.g. we can assume that $s_R \in V'_0$ as whether two vertices have the same profile on $R$ is independent of the choice of the pivot vertex.

In the first case, there are at most $|R| = \Oh(n^{\rho' + \delta'})$ such vertices, hence they realise at most that many profiles.

In the second case, we want to observe that profile of any vertex $v \in V'_0$ on $R$ depends only on its profile on $R \cap V'_0$ (relative to $s_R$). Pick any $t \in R - V'_0$. Then $t \in V'_i - V'_0$ for some $i \in [\ell']$, $V_i \cap V_0 \subseteq R \cap V'_0$, and every path from $v$ to $t$ intersects $V_i \cap V_0$. In particular, distances from $v$ to vertices of $V_i \cap V_0$ determine its distance to $t$, which proves the observation.

Let $\tilde{G}_0$ denote the graph obtained by taking $G'[V'_0]$ and for every $i \in [\ell'], u, v \in V_0 \cap V_i$ adding a disjoint path from $u$ to $v$ of length $\dist(u, v)$. Let $P_i$ denote the vertex set of paths added between $V_0 \cap V_i$. For every $t \in V'_0$ we have $\dist_{G' - A}(v, t) = \dist_{\tilde{G}_0}(v, t)$, so it suffices to bound the number of profiles on $R \cap V'_0$ in $\tilde{G}_0$. By our assumptions, $\tilde{G}_0$ has Euler genus bounded by $g$ and all $P_i$ are of size $\Oh(n^{\delta'})$.

Let $R_0$ be the set of $\mathcal{R}_0$ corresponding to $R$. Let $\tilde{R}_0$ denote the set $(R \cap V'_0) \cup \bigcup_{i : v^*_i \in R_0} P_i$. Such set is connected in $\tilde{G}_0$. Moreover, similarly to $R$, its size is $\Oh(n^{\rho' + \delta'})$. Applying \Cref{thm:distprofiles}, we get that the number of distance profiles on $\tilde{R}_0$ in $\tilde{G}_0$ is $\Oh(n^{12(\rho' + \delta')})$, which also bounds the number of profiles on $R$ in $G' - A$ realised by $V'_0$.

For the third case, assume $v \in V'_i \setminus (V'_0 \cup R)$ for some $i\in [\ell']$. Every path from $v$ to any vertex of $R$ in $G' - A$ intersects $V_i \cap V_0$. Let $v_1, \dots v_p$ be the vertices of $V_i \cap V_0$, where $p \leq 4$. The profile of $v$ on $R$ is then determined by the following:
\begin{itemize}[nosep]
 \item[(a)] the profile of each $v_j$ on $R$,
 \item[(b)] $\dist_{G' - A}(v, v_j) - \dist_{G' - A}(v, v_1)$ for each $2 \leq j \leq p$, and
 \item[(c)] $\dist_{G' - A}(s_R, v_j) - \dist_{G' - A}(s_R, v_1)$ for each $2 \leq j \leq p$ where $s_R$ is some pivot vertex of $R$.
\end{itemize}
By the previous case, the number of distance profiles of each $v_j$ is $\Oh(n^{12(\rho' + \delta')})$. The distances between $v$ and $v_j$ are bounded by $|V'_i|$, hence each quantity described in (b) can take $\Oh(n^{\delta'})$ different possible values. Similarly, since $v_1$ and $v_j$ are connected via $V'_i$, $|\dist_{G' - A}(s_R, v_j) - \dist_{G' - A}(s_R, v_1)| \leq \Oh(n^{\delta'})$. The number of different possible profiles of such $v$ is therefore bounded by $\Oh(n^{48(\rho' + \delta') + 6\delta'}) = \Oh(n^{48\rho' + 54\delta'})$. This finishes the proof of the claim.
\end{proof}

Now we can apply \Cref{l:main_ecc} to graph $G'$ with apex set $A$, $X = V(\widetilde{G})$, and the following constants: $$\rho = \rho' + \delta',\qquad \gamma = 1 - \frac{1}{2}\rho',\quad \textrm{and}\quad \alpha = 48\rho' + 54 \delta'.$$ This allows us to calculate all $V(\widetilde{G})$-eccentricities in $G'$ in time
$$
\Oh \left( \left(
	n^{ 2 - \frac{1}{2} \rho' } +
	n^{ 1 + 48\rho' + 54 \delta' }
\right) \log^k n \right) =
\Oh \left( n^{1 + \frac{150 + 54 \delta}{151}} \log^k n \right).
$$
Since for each $v\in V(\widetilde{G})$ we have $\ecc_{\widetilde{G}}(v) = \max_{u \in V(\widetilde{G})} \dist_{\widetilde{G}}(v, u) = \max_{u \in V(\widetilde{G})} \dist_{G'}(v, u)$, this means that we have successfully computed all the eccentricities in $\widetilde{G}$; and as we argued, this is enough to compute all the eccentricities in $G$ as well.

Finally, the total running time of the algorithm is
$$
\Oh \left( n^{1 + \frac{150 + 54 \delta}{151}} \log^k n + n^{2 - \delta' + \delta} \right) =
\Oh \left( n^{1 + \frac{150 + 54 \delta}{151}} \log^k n \right).
$$\qedhere
\end{proof}


\begin{lemma}\label{l:star2}
Fix constants $k, g \in \mathbb{N}, 0 < \delta < \frac{1}{54}$. Assume we are given $n \in \mathbb{N}$, an edge-weighted graph $G$ on at most $n$ vertices with a weight function $w \colon E(G) \to \mathbb{N}$, a vertex subset $A$ and a collection of non-empty vertex subsets $V_0, V_1, \dots, V_\ell$ satisfying the same conditions as in \Cref{l:star} with the following differences:
\begin{itemize}
	\item we don't require $G[V_i - V_0]$ to be connected and $V_i - V_0$ to be adjacent to whole $V_i \cap V_0$;
	\item instead of $|V_0 \cap V_i| \leq 4$, we require $|V_0 \cap V_i| \leq k$.
\end{itemize}
Then, we can compute the eccentricity of every vertex of $G$ in time $\Oh \left( n^{1 + \frac{150 + 54 \delta}{151}} \log^{k + 5g} n \right)$.
\end{lemma}

\begin{proof}
We will reduce our input to one which will satisfy the conditions of \Cref{l:star}. We start by addressing the adhesions $V_0 \cap V_i$ containing too many vertices.

Let $G_0$ denote the graph $G[V_0]$ with cliques placed at $V_0 \cap V_i$ for every $i \in [\ell]$.
For every $i \in [\ell]$ we repeat the following procedure: while $|V_0 \cap V_i| > 4$,
remove arbitrary $5$ vertices from $V_0 \cap V_i$. Since $|V_0 \cap V_i| \leq k$ for each $i\in [\ell]$,
this procedure can be implemented in total time $\Oh(n)$. As a result, at the end we have $|V_0 \cap V_i| \leq 4$ for all $i \in [\ell]$. Let $M$ be the set of all the removed vertices. By our assumptions, $G_0$ has Euler genus bounded by $g$, hence it cannot contain $g + 1$ pairwise disjoint copies of $K_5$
(as the Euler genus of a graph is the sum of the Euler genera of its 2-connected components~\cite{StahlB77} and $K_5$ is not planar). Each removed quintiple of vertices induces a $K_5$ in $G_0$, hence we have $|M| \leq 5g$. We set $A' = A \cup M$ and may thus assume that $V_i$ is disjoint from $A'$ for all $0 \leq i \leq \ell$.

Now, fix $i \in [\ell]$. Let $C^i_1, \dots, C^i_{r_i}$ denote the connected components of $V_i - V_0$ in $G - A'$. We define $W^i_j := N_{G - A'}[C^i_j]$ for every $j \in [r_i]$. Clearly, all $W^i_j$ induce a connected subgraph of $G$ and satisfy $N_{G - A'}(W^i_j - V_0) = W^i_j \cap V_0$. We put $V'_0 := V_0$ and enumerate
$$
\{V'_1, V'_2, \dots V'_{\ell'}\} := \{ W^i_j \colon i \in [\ell], j \in [r_i] \}.
$$
It is easy to verify that the sets $A'$ and $V'_0, V'_1, \dots, V'_{\ell'}$ satisfy the conditions of \Cref{l:star}. We apply said lemma to calculate the eccentricity of every vertex of $G$ in the desired time.
\end{proof}



The next statement is a reformulation of \Cref{thm:main-decomp}.

\begin{theorem}
Fix constants $k, g \in \mathbb{N}$. Assume we are given a graph $G$ on $n$ vertices together with its tree decomposition $(T, \beta)$ and a set of private apices $A_t \subseteq \beta(t)$ for each node $t\in V(T)$ such that the following conditions hold:
\begin{itemize}[nosep]
 \item For every node $t \in V(T)$, we have $|A_t| \leq k$.
 \item For every edge $st \in E(T)$,  we have $|\beta(v) \cap \beta(u)|\leq k$.
 \item For every node $t \in V(T)$, graph obtained by taking $G[\beta(t)] - A_t$ and turning  $(\beta(t) \cap \beta(s))\setminus A_t$ into a clique for every edge $st \in E(T)$ has Euler genus bounded by $g$.
\end{itemize}
Then, we can compute the eccentricity of every vertex of $G$ in time $\Oh \left( n^{1 + \frac{355}{356}} \log^{k + 5g} n \right)$.
\end{theorem}

\begin{proof}
We may assume that $|V(T)|\leq n$, for every tree decomposition with no two bags comparable by inclusion has this property; and adjacent comparable bags can be merged by contracting the edge between them.

For a node $t\in V(T)$, by the {\em{weight}} of $t$ we mean the size of the corresponding bag, that is, $|\beta(t)|$. For any subset of nodes $S \subseteq V(T)$, we define $\beta(S) \coloneqq \bigcup_{t \in S} \beta(t)$ By the {\em{weight}} of $S$, we mean the total weight of the elements of $S$, that is, $\sum_{t\in S} |\beta(t)|$. 

\begin{claim}\label{cl:weight-T}
The weight of $V(T)$ is of $\Oh(n)$.
\end{claim}

\begin{proof}
The sets $\beta'(t) := \beta(t) - \bigcup_{s \in N_T(t)} \beta(s)$ are pairwise disjoint. We have
$$
\sum_{t \in V(T)} |\beta(t)| =
\sum_{t \in V(T)} |\beta'(t)| + 2 \cdot \sum_{st \in E(T)} |\beta(s) \cap \beta(t)| \leq
|V(T)| + 2k|E(T)| = \Oh(n).
$$
\end{proof}

Since every bag induces a graph of bounded Euler genus, the number of edges contained in a bag is linear in its size. In particular, this implies that the total number of edges of $G$ is also bounded by $\Oh(n)$.

We set $$\delta \coloneqq \frac{1}{356}\qquad\textrm{and}\qquad \Delta \coloneqq \frac{355}{356}.$$ Root the tree $T$ in an arbitrarily chosen node; this naturally imposes an ancestor-descendant relation in $T$ (for convenience, every node is considered its own ancestor and descendant).

We start by partitioning $T$ into connected subtrees using the following procedure.
We proceed bottom-up over $T$, processing nodes in any order so that a node is processed after all its strict descendants have been processed. Along the way, we mark some nodes and split the edges of $T$ into heavy and light. Let $t \in V(T)$ be the currently processed non-root node of $T$ and let $e \in E(T)$ be the edge connecting $t$ with its parent. If the total weight of all the unmarked nodes that are descendants of $t$ is at least $n^\delta$ (recall that this includes $t$ itself as well), then we declare $e$ heavy and mark all the descendants of $t$ that were unmarked so far. Otherwise, the edge $e$ is declared light and the procedure proceeds to further nodes of $T$.

Observe that
removing all heavy edges splits $T$ into connected subtrees, say $T'_1, \cdots T'_m$. All of the subtrees, except for possibly the subtree containing the root node, are of weight at least $n^\delta$. In particular, the number of subtrees $m$, and therefore the number of heavy edges, is  bounded by $\Oh(n^{1 - \delta})$. Moreover, in every subtree $T'_i$, removing the node closest to the root splits $T'_i$ into smaller components, each of weight less than $n^\delta$.

Fix a heavy edge $e$ and let $T^e_1$ and $T^e_2$ be the two subtrees into which $T$ splits after removing~$e$. Let $X^e_i = \beta(T^e_i)$ for $i \in \{1, 2\}$. Put $A_e = X^e_1 \setminus X^e_2$, $C_e = X^e_2 \setminus X^e_1$, and $B_e = X^e_1 \cap X^e_2$. By the properties of tree decompositions, such choice of $A_e, B_e, C_e$ satisfies the conditions of \Cref{l:single_adhesion}, hence in time $\Oh(n \log^{k - 1} n)$ we can compute $\max_{v \in X^e_2} \dist_G(u,v)$ for every $u \in X^e_1$, and $\max_{u \in X^e_1} \dist_G(u,v)$ for every $v \in X^e_2$. Computing this for every heavy edge $e$ takes total time $\Oh(n^{2 - \delta} \log^{k - 1} n)$.

Fix any subtree $T'=T'_j$. Let $e_1 = t^{e_1}_1t^{e_1}_2, e_2 = t^{e_2}_1 t^{e_2}_2, \dots, e_\ell = t^{e_\ell}_1 t^{e_\ell}_2$ denote the heavy edges incident to $T'$, where $t^{e_i}_1 \in V(T')$ and $V(T') \subseteq V(T_1^{e_i})$ for every $i \in [\ell]$.
For a vertex $v \in \beta(T')$, let
$$d_0(v) = \max_{u \in \beta(T')} \dist_G(v, u)\qquad\textrm{and}\qquad d_i(v) = \max_{u \in X_2^{e_i}}\dist_G(v,u),\quad\textrm{for } i \in [\ell].$$ We have $\ecc(v) = \max \{ d_i(v)\colon i\in \{0,1,\ldots,\ell\}\}$.The values of $d_i(v)$ are already calculated for all $i\in [\ell]$, hence it remains to compute $d_0(v)$.

For every $i \in [\ell]$ and every pair of vertices $u, v \in \beta(t^{e_i}_1) \cap \beta(t^{e_i}_2)$ we find a shortest path between $u$ and $v$ with all internal vertices inside $X^{e_i}_2$ (or determine that it doesn't exist). For a fixed $u, v$ this can be done in time $\Oh(n)$. Since in total we perform this step at most $2k^2$ times per heavy edge, it takes $\Oh(n^{2 - \delta})$ time in total. Let $P_{i, u, v}$ denote such path, assuming it exists.

Let $G'$ denote the graph obtained from $G[\beta(T')]$ by taking every $i, u, v$ for which $P_{i, u, v}$ exists and adding an edge between $u$ and $v$ of weight equal to the total weight of $P_{i, u, v}$.
The weight of every edge inserted in $\beta(t^{e_i}_1) \cap \beta(t^{e_i}_2)$ is bounded by $|X^{e_i}_2|+1$. The total weight of all edges inserted is therefore at most
$$
\sum_{i \in [\ell]} |\beta(t^{e_i}_1) \cap \beta(t^{e_i}_2)|^2 \cdot (|X^{e_i}_2|+1) \leq
k^2 \sum_{i \in [\ell]} (|X^{e_i}_2|+1) = \Oh(n),
$$
where the last equality follows from the fact that all the trees $T^{e_i}_2$ are pairwise disjoint.
By \Cref{l:inserting_paths}, we have $\dist_{G'}(u, v) = \dist_G(u, v)$ for each $u, v \in \beta(T')$. Hence, computing $d_0(v)$ for every $v \in \beta(T')$ is equivalent to computing the eccentricity of every vertex in $G'$.

If the size of $\beta(T')$ is smaller than $n^\Delta$, we compute the eccentricities naively in time $\Oh(|\beta(T')|^2)$, 
noting that $G'$ has $\Oh(|\beta(T')|)$ edges (thanks to Claim~\ref{cl:weight-T} and bounded genus assumption 
of the last bullet of the theorem statement). Otherwise, we argue that we can use the algorithm in \Cref{l:star} as follows.

Let $t$ be the node of $T'$ closest to the root. Let $s_1, \dots, s_p$ be the children of $t$ in $T$ and let $T''_i$ denote the connected component of $T' - \{t\}$ containing $s_i$. Set $V_0 = \beta(t)$ and $V_i = \beta(T''_i)$ for $i \in [p]$.

It is now easy to verify that $G'$ and sets $A, \{V_i\colon 0\leq i\leq p\}$ selected this way satisfy the assumptions of \Cref{l:star2}. This allows us to use it to compute the eccentricities in $G'$ in time
$$
\Oh \left( n^{1 + \frac{150 + 54\delta}{151}} \log^{k + 5g} n \right) =
\Oh \left( n^{1 + \frac{354}{356}} \log^{k + 5g} n \right).
$$
As we argued, from these eccentricities, we may easily compute all the eccentricities in $G$.

Now, let us analyse the total running time of the whole algorithm. We invoke \Cref{l:star} $\Oh(n^{1 - \Delta})$ times, since we apply it only to subtrees $T'_i$ of size at least $n^\Delta$. The total running time of those applications is hence
$$
\Oh \left( n^{2 + \frac{354}{356} - \Delta} \log^{k + 5g} n \right) =
\Oh \left( n^{1 + \frac{355}{356}} \log^{k + 5g} n \right).
$$
We compute the eccentricities naively for subtrees smaller than $n^\Delta$, hence the total running time of this computation is
$$
\sum_{i \in [m] \colon |\beta(T'_i)| \leq n^\Delta} |\beta(T'_i)|^2 \leq
n^\Delta \cdot \sum_{i \in m} |\beta(T'_i)| = \Oh(n^{1 + \Delta})=\Oh\left(n^{1+\frac{355}{356}}\right).
$$
The rest of computation can be done in $\Oh(n^{2 - \delta} \log^k n)$. Therefore, the whole algorithm runs in time $\Oh \left( n^{1 + \frac{355}{356}} \log^{k + 5g} n \right)$.
\end{proof}


\section*{Acknowledgements}
Marcin thanks Jacob Holm, Eva Rotenberg, and Erik Jan van Leeuwen
for many discussions on subquadratic algorithms for diameter while his stay on sabbatical
in Copenhagen.

\bibliographystyle{plain}
\begin{thebibliography}{10}

  \bibitem{BorradaileDT14}
  Glencora Borradaile, Erik~D. Demaine, and Siamak Tazari.
  \newblock Polynomial-time approximation schemes for subset-connectivity problems in bounded-genus graphs.
  \newblock {\em Algorithmica}, 68(2):287--311, 2014.
  
  \bibitem{Cabello19}
  Sergio Cabello.
  \newblock Subquadratic algorithms for the diameter and the sum of pairwise distances in planar graphs.
  \newblock {\em {ACM} Transactions on Algorithms}, 15(2):21:1--21:38, 2019.
  
  \bibitem{CabelloK09}
  Sergio Cabello and Christian Knauer.
  \newblock Algorithms for graphs of bounded treewidth via orthogonal range searching.
  \newblock {\em Computational Geometry}, 42(9):815--824, 2009.
  
  \bibitem{CabelloCL12algo}
  Sergio Cabello, Éric {Colin de Verdière}, and Francis Lazarus.
  \newblock Algorithms for the edge-width of an embedded graph.
  \newblock {\em Computational Geometry}, 45(5):215--224, 2012.
  \newblock Special issue: 26th Annual Symposium on Computation Geometry at Snowbird, Utah, USA.
  
  \bibitem{DucoffeHV22}
  Guillaume Ducoffe, Michel Habib, and Laurent Viennot.
  \newblock Diameter, eccentricities and distance oracle computations on ${H}$-minor free graphs and graphs of bounded (distance) {V}apnik-{C}hervonenkis dimension.
  \newblock {\em {SIAM} Journal on Computing}, 51(5):1506--1534, 2022.
  
  \bibitem{DurajKP23}
  Lech Duraj, Filip Konieczny, and Krzysztof Pot\k{e}pa.
  \newblock Better diameter algorithms for bounded {VC}-dimension graphs and geometric intersection graphs.
  \newblock In {\em 32nd Annual European Symposium on Algorithms, {ESA} 2024}, volume 308 of {\em LIPIcs}, pages 51:1--51:18. Schloss Dagstuhl --- Leibniz-Zentrum f{\"{u}}r Informatik, 2024.
  
  \bibitem{EricksonW05}
  Jeff Erickson and Kim Whittlesey.
  \newblock Greedy optimal homotopy and homology generators.
  \newblock In {\em Proc. of the 16th Annual {ACM-SIAM} Symposium on Discrete Algorithms, {SODA} 2005}, pages 1038--1046. {SIAM}, 2005.
  
  \bibitem{GawrychowskiKMS21}
  Pawe\l{} Gawrychowski, Haim Kaplan, Shay Mozes, Micha Sharir, and Oren Weimann.
  \newblock Voronoi diagrams on planar graphs, and computing the diameter in deterministic $\widetilde{O}(n^{5/3})$ time.
  \newblock {\em {SIAM} Journal on Computing}, 50(2):509--554, 2021.
  
  \bibitem{KorhonenPS24}
  Tuukka Korhonen, Micha\l{} Pilipczuk, and Giannos Stamoulis.
  \newblock Minor {C}ontainment and {D}isjoint {P}aths in almost-linear time.
  \newblock {\em CoRR}, abs/2404.03958, 2024.
  
  \bibitem{KorhonenPST24priv}
  Tuukka Korhonen, Micha\l{} Pilipczuk, Giannos Stamoulis, and Dimitrios Thilikos, 2024.
  \newblock Private communication.
  
  \bibitem{LeW24}
  Hung Le and Christian Wulff{-}Nilsen.
  \newblock {VC} set systems in minor-free (di)graphs and applications.
  \newblock In {\em 2024 {ACM-SIAM} Symposium on Discrete Algorithms, {SODA} 2024}, pages 5332--5360. {SIAM}, 2024.
  
  \bibitem{MoharT01grap}
  Bojan Mohar and Carsten Thomassen.
  \newblock {\em Graphs on Surfaces}.
  \newblock Johns Hopkins series in the mathematical sciences. Johns Hopkins University Press, 2001.
  
  \bibitem{RobertsonS03a}
  Neil Robertson and Paul~D. Seymour.
  \newblock Graph {M}inors. {XVI.} {E}xcluding a non-planar graph.
  \newblock {\em Journal of Combinatorial Theory, Series {B}}, 89(1):43--76, 2003.
  
  \bibitem{RodittyW13}
  Liam Roditty and Virginia~Vassilevska Williams.
  \newblock Fast approximation algorithms for the diameter and radius of sparse graphs.
  \newblock In {\em 45th Symposium on Theory of Computing Conference, STOC 2013}, pages 515--524. {ACM}, 2013.
  
  \bibitem{Sauer72}
  Norbert Sauer.
  \newblock On the density of families of sets.
  \newblock {\em Journal of Combinatorial Theory, Series A}, 13(1):145--147, 1972.
  
  \bibitem{Shelah72}
  Saharon Shelah.
  \newblock A combinatorial problem; stability and order for models and theories in infinitary languages.
  \newblock {\em Pacific Journal of Mathematics}, 41(1):247 -- 261, 1972.
  
  \bibitem{StahlB77}
  Saul Stahl and Lowell~W. Beineke.
  \newblock Blocks and the nonorientable genus of graphs.
  \newblock {\em Journal of Graph Theory}, 1(1):75--78, 1977.
  
  \bibitem{ThilikosW22}
  Dimitrios~M. Thilikos and Sebastian Wiederrecht.
  \newblock Killing a vortex.
  \newblock In {\em 63rd {IEEE} Annual Symposium on Foundations of Computer Science, {FOCS} 2022}, pages 1069--1080. {IEEE}, 2022.
  
  \bibitem{VapnikC71}
  V.~N. Vapnik and A.~Ya. Chervonenkis.
  \newblock On the uniform convergence of relative frequencies of events to their probabilities.
  \newblock {\em Theory of Probability \& Its Applications}, 16(2):264--280, 1971.
  
  \bibitem{Willard85}
  Dan~E. Willard.
  \newblock New data structures for orthogonal range queries.
  \newblock {\em SIAM Journal on Computing}, 14(1):232--253, 1985.
  
  \bibitem{WulffNilsen11}
  Christian Wulff{-}Nilsen.
  \newblock Separator theorems for minor-free and shallow minor-free graphs with applications.
  \newblock {\em CoRR}, abs/1107.1292, 2011.
  
  \end{thebibliography}

\end{document}
