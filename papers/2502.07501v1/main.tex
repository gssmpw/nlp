\documentclass[11pt,a4paper]{article}
\usepackage[lmargin=1.0in,rmargin=1.0in,bottom=1.0in,top=1.0in,twoside=False]{geometry}

\usepackage{inconsolata}
\usepackage{libertine}

\usepackage{amssymb,amsmath,amsthm}
\usepackage{mathtools}
\usepackage{graphicx}%,algorithmic,algorithm}%, verbatim}
\usepackage{enumerate}
\usepackage{tikz}
\usetikzlibrary{shapes}

\usepackage[T1]{fontenc}

\usepackage{enumitem}

\usepackage{microtype}
\usepackage{amsfonts}
\usepackage{comment}
\usepackage{mathrsfs}
\setlength{\marginparwidth}{2cm}
\usepackage{todonotes}
\newcommand{\mpil}[2][]{\todo[color=magenta!80,#1]{{\textbf{MaPi:} #2}}}
\newcommand{\mipil}[2][]{\todo[color=orange!30,#1]{\small {\textbf{MiPi:} #2}}}
\newcommand{\gs}[2][]{\todo[color=blue!20,#1]{\small {\textbf{GS:} #2}}}
\newcommand{\kk}[2][]{\todo[color=green!20,#1]{\small {\textbf{KK:} #2}}}

\definecolor{blue}{rgb}{0.1,0.2,0.5}
\definecolor{brown}{rgb}{0.6,0.6,0.2}
\usepackage[ocgcolorlinks, linkcolor={blue}, citecolor={brown}]{hyperref}
\usepackage{cleveref}

\usepackage{enumerate}
\usepackage{latexsym}

\usepackage{soul}

\newtheorem{lemma}{Lemma}[section]
\newtheorem{observation}{Observation}[section]
\newtheorem{claim}{Claim}[section]
\newtheorem{theorem}[lemma]{Theorem}
\newtheorem{corollary}[lemma]{Corollary}
\newtheorem{conjecture}[lemma]{Conjecture}

\newcommand{\Oh}{\mathcal{O}}

\newcommand{\argmax}{\text{argmax}}
\newcommand{\argmin}{\text{argmin}}

\newcommand{\distprofile}[3]{\mathrm{prof}_{#1,#2}[#3]}
\newcommand{\tildeprofile}[3]{\widetilde{\mathrm{prof}}_{#1,#2}[#3]}
\newcommand{\barprofile}[3]{\overline{\mathrm{prof}}_{#1,#2}[#3]}

\newcommand{\hatprofile}[3]{\widehat{\mathrm{prof}}_{#1,#2}[#3]}
\newcommand{\diststarprofile}[3]{\mathrm{prof}^\star_{#1,#2}[#3]}
\newcommand{\dist}{\mathrm{dist}}
\newcommand{\ecc}{\mathrm{ecc}}

\renewcommand{\leq}{\leqslant}
\renewcommand{\geq}{\geqslant}
\renewcommand{\le}{\leqslant}
\renewcommand{\ge}{\geqslant}
\renewcommand{\setminus}{-}

\usepackage{marvosym}

\newcommand{\cutgraph}{\textrm{\CutLeft}}

\usepackage{textpos}


\begin{document}

\author{
  Kacper Kluk% 
  \thanks{Institute of Informatics, University of Warsaw, Poland. \texttt{k.kluk@uw.edu.pl}}
  \and
  Marcin Pilipczuk%
  \thanks{Institute of Informatics, University of Warsaw, Poland. \texttt{m.pilipczuk@uw.edu.pl}}
  \and
  Micha\l{} Pilipczuk%
  \thanks{Institute of Informatics, University of Warsaw, Poland. \texttt{michal.pilipczuk@mimuw.edu.pl}}
  \and
  Giannos Stamoulis%
  \thanks{IRIF, Université Paris Cité, CNRS, Paris, France. \texttt{giannos.stamoulis@irif.fr}}
}


\title{Faster diameter computation in graphs of bounded Euler genus%
  \thanks{
  K.K. and Ma.P. are supported by Polish National Science Centre SONATA BIS-12 grant number 2022/46/E/ST6/00143.
  This work is a part of project BOBR (Mi.P., G.S.) that has received funding from the European Research Council (ERC) 
under the European Union's Horizon 2020 research and innovation programme (grant agreement No.~948057). In particular, a majority of work on this manuscript was done while G.S. was affiliated with University of Warsaw.}}

\date{}

\maketitle

\begin{abstract}
We show that for any fixed integer $k \geq 0$, there exists an algorithm
that computes the diameter and the eccentricies of all vertices of an input
unweighted, undirected $n$-vertex graph of Euler genus at most $k$ in time
 \[ \Oh_k(n^{2-\frac{1}{25}}). \]
Furthermore, for the more general class of graphs
that can be constructed
by clique-sums from graphs that are of Euler genus at most $k$ after deletion
of at most $k$ vertices, we show an algorithm for the same task that achieves the running time bound
 \[ \Oh_k(n^{2-\frac{1}{356}} \log^{6k} n). \]
Up to today, the only known subquadratic algorithms for computing the diameter in those graph classes
are that of [Ducoffe, Habib, Viennot; SICOMP 2022], [Le, Wulff-Nilsen; SODA 2024],
and [Duraj, Konieczny, Pot\k{e}pa; ESA 2024]. These algorithms work
in the more general setting of $K_h$-minor-free graphs,
but the running time bound is $\Oh_h(n^{2-c_h})$ for some constant $c_h > 0$ depending
on $h$. 
That is, our savings in the exponent, as compared to the naive quadratic algorithm, 
are independent of the parameter $k$.

The main technical ingredient of our work is an improved bound on the number of distance
profiles, as defined in [Le, Wulff-Nilsen; SODA 2024], in graphs of bounded Euler genus.
\end{abstract}


\begin{textblock}{20}(11.8,4.35)
  \includegraphics[width=60px]{ERC2.jpg}
\end{textblock}


\section{Introduction}
\section{Introduction}


\begin{figure}[t]
\centering
\includegraphics[width=0.6\columnwidth]{figures/evaluation_desiderata_V5.pdf}
\vspace{-0.5cm}
\caption{\systemName is a platform for conducting realistic evaluations of code LLMs, collecting human preferences of coding models with real users, real tasks, and in realistic environments, aimed at addressing the limitations of existing evaluations.
}
\label{fig:motivation}
\end{figure}

\begin{figure*}[t]
\centering
\includegraphics[width=\textwidth]{figures/system_design_v2.png}
\caption{We introduce \systemName, a VSCode extension to collect human preferences of code directly in a developer's IDE. \systemName enables developers to use code completions from various models. The system comprises a) the interface in the user's IDE which presents paired completions to users (left), b) a sampling strategy that picks model pairs to reduce latency (right, top), and c) a prompting scheme that allows diverse LLMs to perform code completions with high fidelity.
Users can select between the top completion (green box) using \texttt{tab} or the bottom completion (blue box) using \texttt{shift+tab}.}
\label{fig:overview}
\end{figure*}

As model capabilities improve, large language models (LLMs) are increasingly integrated into user environments and workflows.
For example, software developers code with AI in integrated developer environments (IDEs)~\citep{peng2023impact}, doctors rely on notes generated through ambient listening~\citep{oberst2024science}, and lawyers consider case evidence identified by electronic discovery systems~\citep{yang2024beyond}.
Increasing deployment of models in productivity tools demands evaluation that more closely reflects real-world circumstances~\citep{hutchinson2022evaluation, saxon2024benchmarks, kapoor2024ai}.
While newer benchmarks and live platforms incorporate human feedback to capture real-world usage, they almost exclusively focus on evaluating LLMs in chat conversations~\citep{zheng2023judging,dubois2023alpacafarm,chiang2024chatbot, kirk2024the}.
Model evaluation must move beyond chat-based interactions and into specialized user environments.



 

In this work, we focus on evaluating LLM-based coding assistants. 
Despite the popularity of these tools---millions of developers use Github Copilot~\citep{Copilot}---existing
evaluations of the coding capabilities of new models exhibit multiple limitations (Figure~\ref{fig:motivation}, bottom).
Traditional ML benchmarks evaluate LLM capabilities by measuring how well a model can complete static, interview-style coding tasks~\citep{chen2021evaluating,austin2021program,jain2024livecodebench, white2024livebench} and lack \emph{real users}. 
User studies recruit real users to evaluate the effectiveness of LLMs as coding assistants, but are often limited to simple programming tasks as opposed to \emph{real tasks}~\citep{vaithilingam2022expectation,ross2023programmer, mozannar2024realhumaneval}.
Recent efforts to collect human feedback such as Chatbot Arena~\citep{chiang2024chatbot} are still removed from a \emph{realistic environment}, resulting in users and data that deviate from typical software development processes.
We introduce \systemName to address these limitations (Figure~\ref{fig:motivation}, top), and we describe our three main contributions below.


\textbf{We deploy \systemName in-the-wild to collect human preferences on code.} 
\systemName is a Visual Studio Code extension, collecting preferences directly in a developer's IDE within their actual workflow (Figure~\ref{fig:overview}).
\systemName provides developers with code completions, akin to the type of support provided by Github Copilot~\citep{Copilot}. 
Over the past 3 months, \systemName has served over~\completions suggestions from 10 state-of-the-art LLMs, 
gathering \sampleCount~votes from \userCount~users.
To collect user preferences,
\systemName presents a novel interface that shows users paired code completions from two different LLMs, which are determined based on a sampling strategy that aims to 
mitigate latency while preserving coverage across model comparisons.
Additionally, we devise a prompting scheme that allows a diverse set of models to perform code completions with high fidelity.
See Section~\ref{sec:system} and Section~\ref{sec:deployment} for details about system design and deployment respectively.



\textbf{We construct a leaderboard of user preferences and find notable differences from existing static benchmarks and human preference leaderboards.}
In general, we observe that smaller models seem to overperform in static benchmarks compared to our leaderboard, while performance among larger models is mixed (Section~\ref{sec:leaderboard_calculation}).
We attribute these differences to the fact that \systemName is exposed to users and tasks that differ drastically from code evaluations in the past. 
Our data spans 103 programming languages and 24 natural languages as well as a variety of real-world applications and code structures, while static benchmarks tend to focus on a specific programming and natural language and task (e.g. coding competition problems).
Additionally, while all of \systemName interactions contain code contexts and the majority involve infilling tasks, a much smaller fraction of Chatbot Arena's coding tasks contain code context, with infilling tasks appearing even more rarely. 
We analyze our data in depth in Section~\ref{subsec:comparison}.



\textbf{We derive new insights into user preferences of code by analyzing \systemName's diverse and distinct data distribution.}
We compare user preferences across different stratifications of input data (e.g., common versus rare languages) and observe which affect observed preferences most (Section~\ref{sec:analysis}).
For example, while user preferences stay relatively consistent across various programming languages, they differ drastically between different task categories (e.g. frontend/backend versus algorithm design).
We also observe variations in user preference due to different features related to code structure 
(e.g., context length and completion patterns).
We open-source \systemName and release a curated subset of code contexts.
Altogether, our results highlight the necessity of model evaluation in realistic and domain-specific settings.






\section{Preliminaries}\label{sec:prelims}

\section{Preliminaries} \label{sec:prelims}
Before diving into the technical results, we state the basic graph notations used throughout the paper and recap the new non-standard definitions we have introduced throughout \Cref{sec:overview}.

\paragraph{Graphs.}
Throughout we consider directed simple graphs $G = (V, E)$, where $E \subseteq V^2$, with $n = |V|$ nodes and $m = |E|$ edges. The edges of the graph can be associated with some value: a length $\ell(e)$ or a capacity/cost $c(e)$, all of which we require to be positive. For any $U \subseteq V$, we write $\overline U = V \setminus U$. Let $G[U]$ be the subgraph induced by $U$. We denote with $\delta^{+}(U)$ the set of edges that have their starting point in $U$ and endpoint in~$\overline U$. We define $\delta^{-}(U)$ symmetrically. We also sometimes write $c(S) = \sum_{e \in S} c(e)$ (for a set of edges $S$) or $c(U, W) = \sum_{e \in E \cap (U \times W)} c(e)$ and $c(U) = c(U, U)$ (for sets of nodes $U, W$).

The distance between two nodes $v$ and $u$ is written $d_G(v,u)$ (throughout we consider only the \emph{length} functions to be relevant for distances). We may omit the subscript if it is clear from the context. The diameter of the graph is the maximum distance between any pair of nodes. For a subgraph $G'$ of $G$ we occasionally say that~$G'$ has \emph{weak diameter} $D$ if for all pairs of nodes $u, v$ in~$G'$, we have $d_G(u, v), d_G(v, u) \leq D$. A strongly connected component in a directed graph $G$ is a subgraph where for every pair of nodes $v,u$ there is a path from $v$ to $u$ and vise versa. Finally, for a radius $r \geq 0$ we write $B^+(v, r) = \set{x \in V : d_G(v, x) \leq r}$ and $B^-(v, r) = \set{y \in V : d_G(y, v) \leq r}$.


\paragraph{Polynomial Bounds.}
For graphs with edge lengths (or capacities), we assume that they are positive and the maximum edge length is bounded by $\poly(n)$. This is only for the sake of simplicity in \cref{sec:ldd-expander,sec:ldd-deterministic} (where in the more general case that all edge lengths are bounded by some threshold $W$ some logarithmic factors in $n$ become $\log (nW)$ instead), and is not necessary for our strongest LDD developed in \cref{sec:ldd-fast}.

\paragraph{Expander Graphs.}
Let $G = (V, E, \ell, c)$ be a directed graph with positive edge capacities $c$ and positive unit edge lengths $\ell$. We define the \emph{volume $\vol(U)$} by
\begin{equation*}
	\vol(U) = c(U, V) = \sum_{e \in E \cap (U \times V)} c(e),
\end{equation*}
and set $\minvol(U) = \min\set{\vol(U), \vol(\overline U)}$ where $\overline U = V \setminus U$. A node set $U$ naturally corresponds to a cut $(U, \overline U)$. The \emph{sparsity} (or \emph{conductance}) of $U$ is defined by
\begin{equation*}
	\phi(U) = \frac{c(U, \overline U)}{\minvol(U)}.
\end{equation*}
In the special cases that $U = \emptyset$ we set $\phi(U) = 1$ and in the special case that $U \neq \emptyset$ but $\vol(U) = 0$, we set $\phi(U) = 0$.
We say that $U$ is \emph{$\phi$-sparse} if $\phi(U) \leq \phi$. We say that a directed graph is a $\phi$-expander if it does not contain a $\phi$-sparse cut $U \subseteq V$. 
We define the \emph{lopsided sparsity} of $U$ as
\begin{equation*}
	\psi(U) = \frac{c(U, \overline U)}{\minvol(U) \cdot \log \frac{\vol(V)}{\minvol(U)}},
\end{equation*}
(with similar special cases), and we similarly say that $U$ is \emph{$\psi$-lopsided sparse} if $\psi(U) \leq \psi$. Finally, we call a graph a \emph{$\psi$-lopsided expander} if it does not contain a $\psi$-lopsided sparse cut $U \subseteq V$.


\section{Distance profiles in graphs of bounded Euler genus}\label{sec:distprofiles}


In this section we prove~\Cref{thm:distprofiles}. Our argument consists of a reduction to the planar case, where we can use the constant bound on the VC-dimension of the set system given by the distance profiles due to Le and Wulff-Nilsen~\cite{LeW24}.
The main idea behind the reduction is to consider certain notions of ``extended'' profiles, where the extension is built along a collections of shortest paths. These shortest paths can be chosen in such a way that by cutting the graph along these paths we obtain a plane graph. Then a bound on the number of the extended profiles in the obtained plane graph translates to a bound on the number of (standard) distance profiles in the original graph.

Preliminary definitions and results needed for defining profiles with respect to shortest paths are given in~\Cref{subsec:milestones}.
These extended profiles are then defined in~\Cref{subsec:mdprofiles}. There, we also prove that a fundamental lemma that equality of extended profiles entails equality of (standard) distance profiles.
The main reduction providing the proof of~\Cref{thm:distprofiles} is given at the end of this section.

\subsection{Milestones}
\label{subsec:milestones}

Let $G$ be a graph, $R$ be a subset of $V(G)$,
$v_0$ be a vertex in $V(G)$, and $P$ be a shortest path from $v_0$ to $R$. Let $x$ be the unique vertex in $V(P)\cap R$. Further, let $\le_P$ be the linear ordering of the vertices traversed by $P$: for two vertices $v,u\in V(P)$, we have $v\le_P u$ if $u$ belongs to $P[v,x]$.
We say that a vertex $v\in V(P)$ is a \emph{milestone of $P$}
if either $v=x$ or we have $\distprofile{R}{x}{v}\neq\distprofile{R}{x}{u}$, where $u$ is the successor of $v$ in $\le_P$.
We denote by $M_{R}(P)$ the set of all milestones of $P$.
Given a milestone $v\in M_{R}(P)$,
the \emph{neutral prefix of $v$ in $P$} is defined as the vertex set of the maximal subpath $Q$ of $P[v_0,v]$ satisfying the following: $v$ is the only milestone of $P$ that belongs to $Q$.

The next lemma shows that minimum-length paths towards $R$ that contain a vertex in the neutral prefix of a milestone can be assumed to pass through that milestone vertex.

\begin{lemma}
  \label{lem:dispref}
  Let $G$ be a graph, $R$ be a subset of $V(G)$, $v_0$ be a vertex in $V(G)$ and $P$ be a shortest path from $v_0$ to $R$.
  Then for every $v\in M_R(P)$, every $u$ in the neutral prefix of $v$, and every $y\in R$, it holds that
  $\dist(u,y)=|P[u,v]|+\dist(v,y)$. 
\end{lemma}
\begin{proof}
  Let $x$ be the unique vertex of $V(P)\cap R$. Note that, by definition, $\distprofile{R}{x}{v}=\distprofile{R}{x}{u}$.
  Also, $\dist(u,x)=\dist(u,v)+\dist(v,x)$ and $\dist(u,v)=|P[u,v]|$. Therefore, $\dist(u,y)=|P[u,v]|+\dist(v,y)$ for every $y\in R$.
\end{proof}

We also give an upper bound on the number of milestones.

\begin{lemma}
  \label{lem:mile}
  Let $G$ be a graph, $R$ be a connected subset of $V(G)$, $v_0$ be a vertex of $G$, and $P$ be a shortest path from $v_0$ to $R$. Then the number of milestones of $P$ is at most $|R|^2+1$.
\end{lemma}
\begin{proof}
  Let $x$ be the unique vertex of $V(P)\cap R$.
  First observe that since $P$ is a shortest path from $v_0$ to~$R$, we have $\dist(v,y)\ge \dist(v,x)$ for every $v\in V(P)$ and every $y\in R$; hence $\distprofile{R}{x}{v}(y)\ge 0$.
  Also, since $R$ is connected, for every $y\in R$ we have $\distprofile{R}{x}{x}(y)\le |R|$.
  To conclude the proof, it suffices to prove that for all $v_1,v_2\in V(P)$ with $v_1\leq_P v_2$, we have \begin{equation}\label{eq:wydra}\distprofile{R}{x}{v_1}(y)\le \distprofile{R}{x}{v_2}(y)\qquad\textrm{for all }y\in R.\end{equation}
  Indeed, \eqref{eq:wydra} together with the previous observations shows that all the distinct distance profiles of the form $\distprofile{R}{x}{v}$ for $v\in V(P)$ can be treated as vectors of length $|R|$ with entries in $\{0,\ldots,|R|\}$, and they all have distinct sums $\sum_{y\in R} \distprofile{R}{x}{v}(y)$. Since these sums range between $0$ and $|R|^2$, the total number of distinct profiles is at most $|R|^2+1$, implying the same bound on the number of milestones.

  To see why \eqref{eq:wydra} holds, note that $\dist(v_1,y)\le \dist(v_1,v_2)+\dist(v_2,y)$ implies that $$\dist(v_1,y)\le \dist(v_1,v_2)+\distprofile{R}{x}{v_2}(y)+\dist(v_2,x)=\distprofile{R}{x}{v_2}(y) + \dist(v_1,x);$$
  the last equality follows from $P$ being a shortest path containing $v_1,v_2,$ and $x$ (in this order). This in turn implies that
  $\distprofile{R}{x}{v_1}(y)=\dist(v_1,y)-\dist(v_1,x)\le \distprofile{R}{x}{v_2}(y)$, as claimed.
\end{proof}


\subsection{Anchor-distance profiles}
\label{subsec:mdprofiles}

\paragraph{Shortest path collections.}
Let $G$ be a graph and $R$ be a subset of vertices of $G$.
We say that a collection $\mathcal{P}$ of paths in $G$
is an \emph{$R$-shortest path collection} if
\begin{itemize}[nosep]
  \item every $P \in \mathcal{P}$ is a shortest path from some $v^P \in V(G)$ to $R$, i.e., $|P|=\dist(v^P,R)$; and
  \item $R \subseteq \bigcup_{P \in \mathcal{P}} V(P)$.
\end{itemize}

For each $P \in \mathcal{P}$, we denote by $x^P$ the endpoint of $P$ in $R$.
Note that the collection $\mathcal{P}$ obtained by taking, for every $y\in R$, the zero-length path from $y$ to $y$, is an $R$-shortest path collection. 

We say that an $R$-shortest path collection is \emph{consistent} if, for every $P_1,P_2 \in \mathcal{P}$
and $v \in V(P_1) \cap V(P_2)$ the paths $P_1[v,x^{P_1}]$ and $P_2[v,x^{P_2}]$ are equal. That is, 
once two paths intersect, they continue together towards $R$. 

The following statement is a direct consequence of the definition of an $R$-shortest path collection.
\begin{observation}\label{obs:short}
  Let $G$ be a graph, $R$ be a subset of vertices of $G$, and $\mathcal{P}$ be an $R$-shortest path collection. Then for every two paths $P_1,P_2\in \mathcal{P}$ and every $v\in V(P_1)\cap V(P_2)$, we have $|P_1[v,x^{P_1}]|=|P_2[v,x^{P_2}]|$.
\end{observation}

\paragraph{Anchor vertices and their prefixes.}
Let $G$ be a graph, $R$ be a subset of $V(G)$, and $\mathcal{P}$ be an $R$-shortest path collection. We denote by $H_{\mathcal{P}}$ the union of the paths in $\mathcal{P}$, i.e., the graph $(\bigcup_{P\in\mathcal{P}}V(P),\bigcup_{P\in\mathcal{P}}E(P))$.
We say that a vertex is an \emph{anchor vertex} if either it has degree more than two in $H_{\mathcal{P}}$
or it is a milestone of a path $P \in \mathcal{P}$.
We denote by $A_R(P)$ the set of all anchor vertices lying on a path $P \in \mathcal{P}$
and by $A_R(\mathcal{P})$ the set of all anchor vertices for $\mathcal{P}$.
Given a path $P\in\mathcal{P}$ with endpoints $v_0$ and $y\in R$, and an anchor vertex $w\in A_R(P)$,
the \emph{prefix of $w$ in $P$} is the vertex set of the maximal subpath $Q$ of
$P[v_0,v]$ satisfying the following: $v$ is the only anchor vertex of $P$ that belongs to $Q$.
Note that for every anchor $w\in V(P)$ there is a milestone $w'$ of $P$ (possibly $w=w'$)
such that the prefix of $w$ in $P$ is a subset of the neutral prefix of $w'$ in $P$.
Finally, for an anchor vertex $w$, the \emph{tail} of $w$, 
denoted $\mathrm{tail}(w)$, is the subgraph of $G$ consisting
of the union of all prefixes of $w$ in $P$ over all paths $P \in \mathcal{P}$ that contain $w$.

\paragraph{Hat-distances.}
Let $G$ be a graph, $R$ be a subset of vertices of $G$, and $\mathcal{P}$ be an $R$-shortest path collection.
We denote by $$U_{\mathcal{P}}\coloneqq V(G)-\bigcup_{P\in\mathcal{P}}V(P).$$
For every $u\in U_{\mathcal{P}}$,
and every anchor vertex $w\in A_R(\mathcal{P})$,
we set 
\[ \widehat{\dist}(u,w)\coloneqq \min\{|Q_{u,z}|+|P[z,w]|\colon
P \in \mathcal{P} \wedge w \in V(P) \wedge 
z\text{ is in the prefix of $w$ in $P$}\},\] 
where $Q_{u,z}$ is a shortest path from $u$ to $z$ with all its internal vertices in $U_{\mathcal{P}}$.
If such $Q_{u,z}$ does not exist for any $z \in V(\mathrm{tail}(w))$,
we set $\widehat{\dist}(u, w) \coloneqq \infty$. 

The following statement is a direct consequence of the definition of $\widehat{\dist}(\cdot,\cdot)$.

\begin{observation}
  \label{obs:dist}
  Let $G$ be a graph, $R$ be a subset of vertices of $G$, and $\mathcal{P}$ be
  an $R$-shortest path collection.
  Then for every $u\in U_{\mathcal{P}}$, we have that $$\dist(u,R)=\min \left\{\widehat{\dist}(u,w)+ \dist(w,R)
  \colon w\in A_R(\mathcal{P})\right\}.$$
\end{observation}

\paragraph{Anchor-distance profiles.}
Let $G$ be a graph, $R$ be a subset of vertices of $G$,  and $\mathcal{P}$
be an $R$-shortest path collection.
The \emph{anchor-distance profile} of a vertex $u\in U_{\mathcal{P}}$ to $R$ with respect to $\mathcal{P}$
is a function $\diststarprofile{R}{\mathcal{P}}{u}$ mapping each $w \in A_R(\mathcal{P})$ to
\[\diststarprofile{R}{\mathcal{P}}{u}(w) \coloneqq \widehat{\dist}(u,w) + \dist(w,R)
 - \dist(u,R).\]
\Cref{obs:dist} implies that we always have $\diststarprofile{R}{\mathcal{P}}{u}(w) \geq 0$.
We set
\[\hatprofile{R}{\mathcal{P}}{u}(w)\coloneqq\min\{\diststarprofile{R}{\mathcal{P}}{u}(w),|R|+1\}.\]

\begin{lemma}\label{lem:hat-to-normal}
  Let $G$ be a graph, let $R$ be a connected subset of vertices of $G$, and $s_R\in R$. Also, let $\mathcal{P}$ be an $R$-shortest path collection.
  Then for all $u_1,u_2\in U_{\mathcal{P}}$,
  $$\hatprofile{R}{\mathcal{P}}{u_1}=\hatprofile{R}{\mathcal{P}}{u_2}\qquad\textrm{implies}\qquad \distprofile{R}{s_R}{u_1}=\distprofile{R}{s_R}{u_2}.$$
\end{lemma}
\begin{proof}
  Fix $u_1,u_2\in U_{\mathcal{P}}$ with
  $\hatprofile{R}{\mathcal{P}}{u_1}=\hatprofile{R}{\mathcal{P}}{u_2}$.
  We start by proving the following.

  \begin{claim}\label{cl:bnd}
    Let $u\in U_{\mathcal{P}}$ and $y\in R$. There is an anchor $w \in A_R(\mathcal{P})$
    such that
    \begin{itemize}[nosep]
      \item $\widehat{\dist}(u,w)+\dist(w,y) = \dist(u,y)$ and
      \item $\hatprofile{R}{\mathcal{P}}{u}(w)\le |R|$.
    \end{itemize}
  \end{claim}
  \begin{proof}
    Let $Q$ be a shortest path from $u$ to $y$ and let $P\in\mathcal{P}$ be the path which $Q$ first intersects
    (if the first vertex of $Q$ in $\bigcup_{P \in \mathcal{P}} V(P)$ belongs to more than one paths in~$\mathcal{P}$,
    we choose $P$ arbitrarily among these paths). 
    Also, let $u'$ be the first vertex of $Q$ (when ordering from $u$ to $y$) in $V(P)$
    and $w$ be the anchor of $P$ that contains $u'$ in its prefix (in $P$).
    Note that $u' \in V(\mathrm{tail}(w))$.

    We first show that 
    \begin{equation}\label{eq:bnd:1}\widehat{\dist}(u,w)+\dist(w,y)= \dist(u,y).\end{equation}
    By~\Cref{lem:dispref} and the fact that $|Q[u',y]| = \dist(u',y)$, we have
    \begin{equation}\label{eq:I}
      \dist(w,y) = |Q[u',y]|-|P[u',w]|.
    \end{equation}
    Also, by definition, we have
    \begin{equation}\label{eq:II}
      \widehat{\dist}(u,w)\le |Q[u,u']|+|P[u',w]|.
    \end{equation}
    By~\eqref{eq:I} and~\eqref{eq:II}, we get that $\widehat{\dist}(u,w)+\dist(w,y)\le |Q|$.
    Moreover, since $Q$ is a shortest path from $u$ to $y$ and $\widehat{dist}(u,w) \geq \dist(u,w)$,
    we have
    \[ |Q| = \dist(u,y) \leq \dist(u,w) + \dist(w,y) \leq \widehat{dist}(u,w) + dist(w,y).\]
    This proves~\eqref{eq:bnd:1}.
 
    Next, we show that $\hatprofile{R}{\mathcal{P}}{u}(w)\le |R|$. Note that
    \begin{align*}
      \diststarprofile{R}{\mathcal{P}}{u}(w) + \dist(u,R) & = \widehat{\dist}(u,w)+\dist(w,R)\\
      & \le \widehat{\dist}(u,w)+\dist(w,y)= \dist(u,y).
    \end{align*}
    The connectivity of $R$ implies that $\dist(u,y)\le \dist(u,R)+|R|$, which gives
    $\diststarprofile{R}{\mathcal{P}}{u}(w)\le |R|$, and the claim follows.
  \end{proof}

  We next show that
  there is an integer $c$ such that for every $y\in R$, we have
  $$\dist(u_1,y)=\dist(u_2,y)+c.$$
  Note that this will immediately imply that
  $\distprofile{R}{s_R}{u_1}=\distprofile{R}{s_R}{u_2}$.

  By Observation~\ref{obs:dist}, for every $h \in \{1,2\}$, there is an anchor $w_h \in A_R(\mathcal{P})$
  such that $\dist(u_h,R) = \widehat{\dist}(u_h,w_h) + \dist(w_h, R)$, which is equivalent to
  $\diststarprofile{R}{\mathcal{P}}{u_h}(w_h) = 0$. 
  If $w_h$ lies on $P_h \in \mathcal{P}$, then $\dist(u_h,R) = \dist(u_h,x^{P_h})$. 
  Therefore, as $\diststarprofile{R}{\mathcal{P}}{u_1}=\diststarprofile{R}{\mathcal{P}}{u_2}$,
  we can choose $w_1 = w_2$ and $P_1 = P_2$, hence $x^{P_1} = x^{P_2}$. 
  In other words, there exists $x\in R$ such that $\dist(u_1,R) = \dist(u_1,x)$ and $\dist(u_2,R) = \dist(u_2,x)$.
  We set $c\coloneqq \dist(u_1,x)-\dist(u_2,x)=\dist(u_1,R)-\dist(u_2,R)$.

  Now, fix $y\in R$.
  Let $w_1\in A_R(P)$ be the anchor from~\Cref{cl:bnd} (applied for $u_1$ and $y$).
  As $\hatprofile{R}{\mathcal{P}}{u_1} = \hatprofile{R}{\mathcal{P}}{u_2}$ and
  $\diststarprofile{R}{\mathcal{P}}{u_1}(w_1)\le |R|$, we have
   $\diststarprofile{R}{\mathcal{P}}{u_1}(w_1) = \diststarprofile{R}{\mathcal{P}}{u_2}(w_1)$, i.e.,
  \[\widehat{\dist}(u_1,w_1) + \dist(w_1,R) - \dist(u_1,R) = \widehat{\dist}(u_2,w_1) + \dist(w_1,R)- \dist(u_2,R).\]
  Therefore,
  \begin{align*}
    \dist(u_1,y) &= \widehat{\dist}(u_1,w_1) + \dist(w_1,y)\\
      & =\widehat{\dist}(u_2,w_1) + \dist(w_1,y)+c\geq \dist(u_2,y)+c;
  \end{align*}
  the first equality follows from~\Cref{cl:bnd}.
  Thus $\dist(u_2,y)+c\le \dist(u_1,y)$.
  A symmetric reasoning shows that also $\dist(u_1,y)-c\le\dist(u_2,y)$.
  Therefore we get $\dist(u_1,y) = \dist(u_2,y)+c$, as required.
\end{proof}

\subsection{Reduction from bounded genus graphs to planar graphs}


We next recall several definitions related to embeddings of graphs on surfaces.
Our basic terminology follows~\cite{MoharT01grap}.
We say that a graph $H$ embedded in a surface $\Sigma$ is a {\em{simple cut-graph}} of $\Sigma$ if $H$ has a single face that is also homeomorphic to an open disk; equivalently, $H$ has a single facial walk.
Given a surface $\Sigma$ and a simple cut-graph $H$ on $\Sigma$, we denote by $\Sigma\cutgraph H$ the surface obtained by cutting $\Sigma$ along~$H$. Note that, provided $H$ is a simple cut-graph, $\Sigma\cutgraph H$ is always a disk.

Suppose now that a graph $G$ embedded on  $\Sigma$ and $H$ is a subgraph of $G$ that is a simple cut-graph of $H$.
We define $G\cutgraph H$ to be the graph embedded on $\Sigma\cutgraph H$ obtained from $G$ as follows.
First, let $\sigma$ be the (unique) facial walk of $H$ and note that each edge $e$ of $H$ is contained exactly twice in $\sigma$ and each vertex $v$ of $H$ is contained in $\sigma$ as many times as the degree of $v$ in $H$.
To obtain $G\cutgraph H$, we replace $H$ with a simple cycle $C_\sigma$ whose vertex set is the set of copies of vertices of $H$ and its edge set is the set of copies of edges of $H$ in the obvious way. Notice that $\sigma$ also prescribes for every edge $uv$ of $G$ between a vertex $u\in V(G)\setminus V(H)$ and a vertex $v\in V(H)$, to which copy of $v$ in $G\cutgraph H$ the vertex $u$ should be adjacent to (in $G\cutgraph H$).
See~\Cref{fig:cutopen} for an illustration.

\begin{figure}[ht]
  \centering
  \includegraphics[width=0.8\textwidth]{cut}
  \caption{Left: A graph $G$ embedded on a surface $\Sigma$ and a subgraph $H$ of $G$ (in blue) that is a simple cut-graph of $\Sigma$. Right: The graph $G\cutgraph H$ embedded on the surface $\Sigma\cutgraph H$ (which is homeomorphic to a disk); the blue vertices/edges are copies of the vertices/edges of $H$.}
  \label{fig:cutopen}
\end{figure}

We will use the following well-known result which appears in the literature under different formulations; see e.g.~\cite{BorradaileDT14,CabelloCL12algo,EricksonW05}.

\begin{lemma}\label{lem:genuscut}
  For every integer $k\ge 1$ and for every edge-weighted connected graph $G$ embedded on a surface $\Sigma$ of Euler genus at most $k$ and every vertex $u\in V(G)$, there is a subgraph $H$ of $G$ with the following properties:
  \begin{itemize}[nosep]
    \item  $H$ is a simple cut-graph of $\Sigma$, and
    \item  $V(H)$ is the union of the vertex sets of $\mathcal{O}(k)$ shortest paths in $G$ that have $u$ as a common endpoint.
  \end{itemize}
  \end{lemma}

We are now ready to proceed to the proof of~\Cref{thm:distprofiles}.
Employing~\Cref{lem:hat-to-normal},
we aim at bouding the VC-dimension of the set system defined by the anchor-distance profiles.
This is can be done by a suitable reduction to the planar setting using~\Cref{lem:genuscut}.



\begin{proof}[Proof of~\Cref{thm:distprofiles}]
  We assume that $G$ is connected -- the distance profiles of all vertices that are not connected to $R$ are equal.
  Let $T_R$ be a spanning tree of $G[R]$ and let $G_0$ be the graph obtained from $G$ after contracting $T_R$ into a single vertex $v_R$. 
  Consider an embedding of $G_0$ on a surface $\Sigma$ of Euler genus at most $k$.
  By~\Cref{lem:genuscut}, there is a subgraph $H_0$ of $G_0$ that is a simple cut-graph of $\Sigma$ and a family $\mathcal{P}_0$ of $\mathcal{O}(k)$ shortest
  paths in $G_0$, each with $v_R$ as an endpoint, such that $V(H_0)=\bigcup_{P\in\mathcal{P}_0}V(P)$.
  Furthermore, as Lemma~\ref{lem:genuscut} handles edge weights, we can slightly perturb
  the weights so that shortest paths in $G_0$ are unique and, in particular, 
  all shortest paths with one endpoint in $v_R$ form a tree.
  Since $H_0$ is a simple cut-graph of $\Sigma$, $G_0\cutgraph H_0$ is disk-embedded.
  Uncontracting $T_R$, we get a subgraph $H$ of $G$ such that $G\cutgraph H$ is disk-embedded.
  Let $\mathcal{P}$ be the family of projections of the paths of $\mathcal{P}_0$
  onto $G$ plus, for every $y \in R$, a zero-length path from $y$ to $y$. 
  Hence, $\mathcal{P}$ is an $R$-shortest paths collection of size $\mathcal{O}(k)$
  with $V(\mathcal{P}) = V(H)$.
  Furthermore, since in $G_0$ the paths of $\mathcal{P}_0$ formed a tree rooted
  at $v_R$, $\mathcal{P}$ is consistent.

  Note that due to~\Cref{lem:mile} we have that $\sum_{P\in\mathcal{P}} |M_R(P)|\le \mathcal{O}_k(|R|^2)$.
  Also, since $\mathcal{P}$ is consistent, 
  if $B$ are the vertices that are not in $R$
  (recall that vertices in $R$ are milestones) and have degree more than two in the graph obtained by the union of the paths in $\mathcal{P}$, then $|B|\leq |\mathcal{P}|-1$.
  Hence,
  \begin{equation}
    \sum_{P\in\mathcal{P}} |A_R(P)|\le \mathcal{O}_k(|R|^2).\label{eq:mile}
  \end{equation}
  We set $\mathcal{T}$ be the set of all vertices of $G\cutgraph H$
  that are copies of the anchor vertices $A_R(\mathcal{P})$.
  Every anchor vertex has $\mathcal{O}_k(1)$ copies in $\mathcal{Q}$
  and therefore, due to~\eqref{eq:mile},
  \begin{equation}
    |\mathcal{T}|=\mathcal{O}_k(|R|^2).\label{eq:term_size}
  \end{equation}
  For $s \in \mathcal{T}$, let $w(s) \in A_R(\mathcal{R})$ be the anchor vertex
  whose copy (in $G \cutgraph H$) is $s$. In the other direction, for $w \in A_R(\mathcal{R})$, let $S(w)$
  be the set of copies of $w$ in $G \cutgraph H$. 
  
  Let $U$ be the set of vertices of $G\cutgraph H$ that are \textsl{not} copies of vertices from $H$ (i.e., $U=V(G)\setminus V(H)$).
  We set $E_{\mathsf{out}}$ be the set of all edges $uv$ of $G\cutgraph H$ where $u\in U$ and $v$ is a copy of a vertex from $H$,
  i.e., $v\in V(G\cutgraph H)\setminus U$.
  We also set $E_{\mathsf{next}}$ be the set of all edges $uv$ of $G\cutgraph H$ where $u$ is a copy
  of an anchor vertex $w\in A_R(P)$ for some $P\in\mathcal{P}$ and $v$ is a copy of the neighbor of $w$
  in $P$ that \textsl{is not} in the prefix of $w$ in $P$.
  
  Let now $\widehat{G}$ be the graph obtained from $G\cutgraph H$ after the following modifications:
  \begin{itemize}[nosep]
    \item we subdivide $|V(G)|$-many times each edge in $E_{\mathsf{out}}\cup E_{\mathsf{next}}$,
    \item we introduce a new vertex $t$ and add, for every $s\in\mathcal{T}$, a path between $t$
    and $s$ of length $$d_{w(s),t}\coloneqq |V(G)| + \dist_G(w(s),R).$$
  \end{itemize}
  See~\Cref{fig:hatG}.
  Observe that since $G\cutgraph H$ is disk-embedded, $\widehat{G}$ is planar,
  because we may embed $t$ together with all the added paths outside of the disk containing $G\cutgraph H$.

  \begin{figure}[ht]
    \centering
    \includegraphics[width=0.3\textwidth]{hatG}
    \caption{An illustration of (a part of) the construction of the graph $\widehat{G}$. The squared vertices are copies of anchor vertices. The marked squared vertex is also a copy of a vertex in $R$. The highlighted edges are copies of edges of $H$ in $G\cutgraph H$, while the paths obtained by subdividing the edges of $E_{\mathsf{out}}\cup E_{\mathsf{next}}$ are depicted with dashed edges. Edges adjacent to $t$ correspond to paths of appropriate length.}
    \label{fig:hatG}
  \end{figure}

  For every $u\in U$, we define a function $\pi[u]$, mapping every $w\in A_R(\mathcal{P})$ to
  \[\pi[u](w)\coloneqq\min\{\dist_{\widehat{G}}(u,s): s \in S(w)\}+d_{w,t}-\dist_{\widehat{G}}(u,t).\]
  Also, we set $\widehat{\mathcal{X}}\coloneqq\{\widehat{X}_u\mid u\in U\}$,
  where for $u\in U$,
  $$\widehat{X}_u\coloneqq\left\{(w,i) \in A_R(\mathcal{P}) \times \{0,\ldots,|R|+1\}~|~i \leq \pi[u](w)\right\}.$$

  \begin{claim}\label{cl:vcd}
    The set system $\widehat{\mathcal{X}}$ has size $\mathcal{O}_k(|R|^{12})$.
  \end{claim}

  \begin{proof}
  We set $\mathcal{T}^+\coloneqq \mathcal{T}\cup \{t\}$.
  We start with the set system $\mathcal{C}^1\coloneqq\left\{C_u^1~\colon~u \in U\right\}$, where
  $$C_u^1\coloneqq\left\{(s,i) \in \mathcal{T}^+ \times \mathbb{Z}~|~i \leq \dist_{\widehat{G}}(u,s)-\dist_{\widehat{G}}(u,t)\right\}.$$
  As $\widehat{G}$ is planar, by~\Cref{thm:LW} we infer that $\mathcal{C}$ has VC-dimension at most 4.

  We now ``shift columns'' of $\mathcal{C}^1$. That is, define $\mathcal{C}^2 \coloneqq \left\{C_u^2~\colon~u \in U\right\}$, where
  $$C_u^2\coloneqq\left\{(s,i) \in \mathcal{T}^+ \times \mathbb{Z}~|~i \leq \dist_{\widehat{G}}(u,s)+d_{w(s),t}-\dist_{\widehat{G}}(u,t)\right\}.$$
  Clearly, the VC-dimension of $\mathcal{C}^1$ and $\mathcal{C}^2$ are equal: a set $Z \subseteq \mathcal{T}^+ \times \mathbb{Z}$
  shatters $\mathcal{C}^1$ if and only if the set $\{(s,d_{w(s),t}+i)~\colon~(s,i) \in Z\}$ shatters $\mathcal{C}^2$. 

  Now, let $\mathcal{C}^3$ be ``cropped'' $\mathcal{C}^2$:
  $\mathcal{C}^3 \coloneqq \left\{ C_u^3~\colon~u \in U\right\}$, where
  \[ C_u^3 \coloneqq C_u^2 \cap \left(\mathcal{T}^+ \times \{0,\ldots, |R|+1\}\right). \]
  Since restricting to a smaller universe cannot increase VC-dimension, $\mathcal{C}^3$ has VC-dimension at most $4$. 
  Since $|\mathcal{T}^+| = \Oh_k(|R|^2)$, by Sauer-Shelah lemma (Lemma~\ref{lem:sauer-shelah})
  we have $|\mathcal{C}^3| = \Oh_k(|R|^{12})$. 

  Now observe that for every $u_1,u_2 \in U$
  \begin{equation}\label{eq:CtoX}
   C^3_{u_1} = C^3_{u_2} \qquad \mathrm{implies} \qquad \widehat{X}_{u_1} = \widehat{X}_{u_2}. 
  \end{equation}
  Indeed, the assumption $C^3_{u_1} = C^3_{u_2}$ implies that
  for every $w \in A_R(\mathcal{P})$ and $s \in S(W)$ we have
  \begin{align*}
  & \max(0, \min(|R|+1, \dist_{\widehat{G}}(u_1,s)+d_{w,t}-\dist_{\widehat{G}}(u_1,t))) \\ 
  &= \max(0, \min(|R|+1, \dist_{\widehat{G}}(u_2,s)+d_{w,t}-\dist_{\widehat{G}}(u_2,t))).
\end{align*} 
  For fixed $w \in A_R(\mathcal{P})$, we take a minimum of the above expression over all $s \in S(w)$, obtaining:
  \begin{align*}
    & \max(0, \min(|R|+1, \min\{\dist_{\widehat{G}}(u_1,s)~\colon~s \in S(w)\}+d_{w,t}-\dist_{\widehat{G}}(u_1,t))) \\ 
    &= \max(0, \min(|R|+1, \min\{\dist_{\widehat{G}}(u_2,s)~\colon~s \in S(w)\}+d_{w,t}-\dist_{\widehat{G}}(u_2,t))).
  \end{align*} 
  This proves~\eqref{eq:CtoX}.
  From~\eqref{eq:CtoX}, we infer $|\widehat{\mathcal{X}}| \leq |\mathcal{C}^3| = \Oh_k(|R|^{12})$, as desired.
  \end{proof}

  We next relate the distance from a vertex $u\in U$ to $R$ (in $G$) and to $t$ (in $\widehat{G}$).
  \begin{claim}\label{cl:dist}
    For every $u\in U$, $\dist_G(u,R)=\dist_{\widehat{G}}(u,t) - 2|V(G)|$.
  \end{claim}
  \begin{proof}
    Fix $u\in U$.
    We first show that $\dist_G(u,R)\le \dist_{\widehat{G}}(u,t) - 2|V(G)|$.
    For this, consider a shortest path $\widehat{Q}$ in $\widehat{G}$ from $u$ to $t$.
    Observe that there is a vertex $s\in\mathcal{T}$ that is a copy of an anchor vertex $w$,
    such that $\widehat{Q}[s,t]$ is the path from $s$ to $t$ of length $d_{w,t}$
    added in the construction of $\widehat{G}$ from $G\cutgraph H$. Recall that $d_{w,t} =\dist_G(w,R)+|V(G)|$.
    Also, observe that $\widehat{Q}[u,s]$ contains at least one subdivided edge of $E_{\mathsf{out}}$, as it starts
    in $U$ and ends outside $U$, and otherwise corresponds to a walk from $u$ to $w$ in $G$.
    Therefore, we have
    \begin{align*}
      \dist_{\widehat{G}}(u,t)  = |\widehat{Q}| & = |\widehat{Q}[u,s]|+|\widehat{Q}[s,t]|\\ 
        & = |\widehat{Q}[u,s]| + \dist_G(w,R)+|V(G)|\\
        & \ge |V(G)| + \dist_G(u, w) + \dist_G(w,R) + |V(G)|\\
        & \ge \dist_G(u,R) + 2|V(G)|.
    \end{align*}
    
    We next show that $\dist_G(u,R)\ge \dist_{\widehat{G}}(u,t) - 2|V(G)|$.
    For this, consider a shortest path $Q$ in $G$ from $u$ to $R$. Let $y\in R$ be the unique vertex in $R\cap V(Q)$.
    Also, let $z$ be the first vertex of $Q$ (when ordering from $u$ to $y$)
    in $\bigcup_{P\in\mathcal{P}}V(P)$ and let $P\in\mathcal{P}$ be the path that $z$ is contained
    (if $z$ is contained to more than one paths, pick one of them arbitrarily).
    Also, let $w$ be the first vertex of $P[z,x^P]$ (when ordering from $z$ to $x^P$) that is an anchor vertex.
    Observe that $Q[u,z]$ corresponds to a path in $\widehat{G}$ from $u$ to a copy $s'$ of $z$
    that contains exactly one subdivided edge of $E_{\mathsf{out}}$ (and no edge of $E_{\mathsf{next}}$)
    and there is a copy of $P[z,w]$ in $\widehat{G}$ from $s'$ to a copy $s$ of $w$ 
    that contains no edge of $E_{\mathsf{out}} \cup E_{\mathsf{next}}$. 
    Therefore,
    \begin{align*}
      |Q| & = |Q[u,z]|+|Q[z,y]|& \\
      & = |Q[u,z]|+ |P[z,x^P]| & \text{\!\!\!($Q[z,y]$ and $P[z,x^P]$ being shortest paths from $z$ to $R$)}\\
      & = |Q[u,z]|+ |P[z,w]| + |P[w,x^P]|& \\
      & = |Q[u,z]|+ |P[z,w]| + \dist_G(w,R) & \text{($P$ being shortest path from a vertex $v^P$ to $R$)}\\
      & \ge \dist_{\widehat{G}}(u,s) - |V(G)| + d_{w,t} - |V(G)|& \\
      & \ge \dist_{\widehat{G}}(u,t) - 2|V(G)|. &  
    \end{align*} 
    Thus, we have $\dist_G(u,R) = |Q| \ge \dist_{\widehat{G}}(u,t)-2|V(G)|$, as desired.
  \end{proof}

  \begin{claim}\label{cl:dist2}
    For every $u \in U$ and $w \in A_R(\mathcal{P})$, it holds that
    \begin{align*}
    \widehat{\dist}(u,w) < \infty &\quad\mathrm{if\ and\ only\ if} \quad \widehat{\dist}(u,w) = \min\left\{\dist_{\widehat{G}}(u,s)~\colon~s \in S(w)\right\}-|V(G)|,\ \mathrm{and}\\
    \widehat{\dist}(u,w) = \infty &\quad\mathrm{if\ and\ only\ if} \quad \min\left\{\dist_{\widehat{G}}(u,s)~\colon~s \in S(w)\right\} > 2|V(G)|.
    \end{align*}
  \end{claim}
  \begin{proof}
    We first show that if $\widehat{\dist}(u,w) < \infty$, then 
    there exists $s \in S(w)$ with $\dist_{\widehat{G}}(u,s) \leq |V(G)| + \widehat{\dist}(u,w)$. 
    To this end, let $Q$ be a path from $u$ to $w$ in $G$ of length $\widehat{\dist}(u,w)$, as in the definition
    of $\widehat{\dist}(u,w)$. There exists $P \in \mathcal{P}$ with $w \in A_R(P)$ and a vertex $z \in V(P) \cap V(Q)$
    such that $Q$ decomposes into $Q[u,z]$ and $Q[z,w] = P[z,w]$, with all internal vertices of $Q[u,z]$ in $U$. 
    Then, $\widehat{G}$ contains a copy $s'$ of $z$ such that $Q[u,z]$ projects to a path from $u$ to $s'$
    with one subdivided edge of $E_{\mathsf{out}}$ (and no edge of $E_{\mathsf{next}}$) and also a copy of $P[z,w]$ from $s'$
    to a copy $s$ of $w$ with no subdivided edge of $E_{\mathsf{out}} \cup E_{\mathsf{next}}$.
    The concatenation of these two paths witness that $\dist_{\widehat{G}}(u,s) \leq |V(G)| + \widehat{\dist}(u,w)$, as desired.

    To finish the proof of the claim, it suffices to show that if there exists $s \in S(w)$ with
    $\dist_{\widehat{G}}(u,s) \leq 2|V(G)|$, then $\widehat{\dist}(u,w) \leq \dist_{\widehat{G}}(u,s) - |V(G)|$
    (in particular, $\widehat{\dist}(u,w) \neq \infty$).
    To this end, let $\widehat{Q}$ be a path in $\widehat{G}$ from $u$ to $s$ of minimum length. 
    Since $u \in U$ but $s \notin U$, $\widehat{Q}$ necessarily contains at least one subdivided edge of $E_{\mathsf{out}}$. 
    Since $|\widehat{Q}| \leq 2|V(G)|$, $\widehat{Q}$ contains exactly one edge of $E_{\mathsf{out}}$, no edge of 
    $E_{\mathsf{next}}$, and no edge incident with $t$. Consequently, there exists a vertex $s'$ on $\widehat{Q}$
    which is a copy of a vertex $z$ that lies in the prefix of $w$ on some path $P \in \mathcal{P}$ such that 
    $\widehat{Q}$ decomposes as $\widehat{Q}[u,s']$, which has all internal vertices in $U$, and $\widehat{Q}[s',s]$
    going along a copy of $P[z,w]$ to $s \in S(w)$. Hence, $\widehat{Q}$ corresponds to a path $Q$ in $G$
    from $u$ to $w$ that satisfies the requirements of the definition of $\widehat{\dist}(u,w)$
    and witnesses $\widehat{\dist}(u,w) \leq |\widehat{Q}| - |V(G)|$, as desired. 

    This finishes the proof of the claim.
  \end{proof}
  
  Using the two previous claims, we infer that for every $u \in U$ and $w \in A_R(\mathcal{P})$ it holds that
  \begin{equation}\label{eq:prof-to-dist}
    \hatprofile{R}{\mathcal{P}}{u}(w) = \min(|R|+1, \pi[u](w)).
  \end{equation}
  Indeed, 
  \begin{align*}
  \min\left(|R|+1, \pi[u](w)\right) &= \min\left(|R|+1,\min\left\{\dist_{\widehat{G}}(u,s)~\colon~s\in S(w)\right\}+d_{w,t}-\dist_{\widehat{G}}(u,t)\right)\\
  &=\min\big(|R|+1, \min\left\{\dist_{\widehat{G}}(u,s)~\colon~s\in S(w)\right\} - |V(G)| &\\
  &\qquad\qquad\qquad\qquad + \dist_G(w,R)-\dist_G(u,R)\big) &\text{by Claim~\ref{cl:dist}}\\
  &=\min\left(|R|+1, \widehat{\dist}(u,w) + \dist_G(w,R)-\dist_G(u,R)\right)&\text{by Claim~\ref{cl:dist2}}\\
  &=\hatprofile{R}{\mathcal{P}}{u}(w).
  \end{align*}
  Here, in the third step we used the estimate $\dist_G(u,R) - \dist_G(w,R) \leq |U| \leq |V(G)|-|R|$, 
  so if $\min\left\{\dist_{\widehat{G}}(u,s)~\colon~s\in S(w)\right\} > 2|V(G)|$
  (which is equivalent to $\widehat{\dist}(u,w) = \infty$ by Claim~\ref{cl:dist2}),
  then the minimum is attained at value $|R|+1$.

  For every $u\in U$, we set
    $$B_u\coloneqq\left\{(w,i) \in A_R(\mathcal{P}) \times \mathbb{Z}~|~i \leq \hatprofile{G}{R}{u}(w)\right\}.$$
  Claim~\ref{cl:vcd} and~\eqref{eq:prof-to-dist} imply that the set system $\{B_u~\colon~u\in U\}$
    has size $\mathcal{O}_k(|R|^{12})$.

  
  
  Now, for every $v\in V(G)$, we set $$S_v\coloneqq\left\{(s,i) \in R \times \{-|R|,\ldots,|R|\}~|~i \leq \distprofile{R}{s_R}{v}(s)\right\}.$$
  The bound on the size of the set system $\{B_u~\colon~u \in U\}$ and~\Cref{lem:hat-to-normal}
  imply that the size of $\left\{S_u~\colon~u \in U \right\}$ is bounded by $\Oh_k(|R|^{12})$.
  We conclude the proof of the lemma by bounding the size of $\left\{S_u~\colon~u \in V(G)\setminus U \right\}$. For this, note that every vertex $v\in V(G)\setminus U$ is either a milestone for some path $P\in\mathcal{P}$ or a vertex in the neutral prefix of a milestone.
  In the latter case, there is a path $P\in\mathcal{P}$ and a milestone $w\in M_R(P)$ such that $S_v=S_w$. Therefore, we have
  $$|\left\{S_u~\colon~u \in V(G)\setminus U \right\}|\le \sum_{P\in\mathcal{P}} |M_R(P)|\le \mathcal{O}_k(|R|^2),$$
  where the second inequality follows from~\eqref{eq:mile}.
  Hence, the size of $\left\{S_v~\colon~v \in V(G)\right\}$ is at most $$|\left\{S_u~\colon~u \in U\right\}|+|\left\{S_u~\colon~u \in V(G)\setminus U\right\}|=\Oh_k(|R|^{12}).$$
  This finishes the proof of Theorem~\ref{thm:distprofiles}.
\end{proof}


\section{Bounded Euler genus graphs with apices: proof of Theorem~\ref{thm:main-apices}}\label{sec:algo-genus}

In this section we prove Theorem~\ref{thm:main-apices}.
(Note that Theorem~\ref{thm:main-genus} is a special case of Theorem~\ref{thm:main-apices}
 for $k=1$.)
We start by deriving the following corollary from \Cref{t:orth_query}.

\begin{corollary}\label{l:max_min_query}
Let $V$ be a set of $n$ points in $\mathbb{R}^d$.
There is a data structure that uses $\Oh \left( d n \log^{d - 2} n \right)$ preprocessing time, $\Oh \left( d n \log^{d - 2} n \right)$ memory and answers the following queries in time $\Oh \left( d \log^{d - 2} n \right)$: given $r_1, \dots, r_d \in \mathbb{R}$, find $\max_{v \in V} \min_{i \in [d]} (v_i + r_i)$, where $v_i$ denotes the $i$th coordinate of $v$.
\end{corollary}

\begin{proof}
    Fix query parameters $r_1, \dots, r_d \in \mathbb{R}$.
    Let $\lambda \coloneqq \max_{v \in V} \min_{i \in [d]} (v_i + r_i)$ denote the answer we want to find.
    
    We say that a pair $(v,i) \in V \times [d]$ is \emph{good} if
    for every $j \in [d]$, it holds that $v_j - v_i \geq r_i -r_j$.
    Let
    $$
    		\lambda' = \max \left\{ v_i + r_i \colon i\in [d], v\in V,\textrm{ and }(v,i)\textrm{ is good} \right\}.
    $$
    We claim that 
    \begin{equation}
    \label{eq:query-lambda}
    \lambda = \lambda'.
   \end{equation}
   Let $v' = \argmax_{v \in V} (\min_{i \in [d]} v_i + r_i)$ and let $i' = \argmin_{i \in [d]} v'_i + r_i$. By the choice of $i'$, for each $j$ we have $v'_j+r_j\geq v'_{i'}+r_{i'}$, implying $v'_j - v'_{i'} \geq r_{i'} - r_j$. Hence $(v',i')$ is good, so $\lambda' \geq v'_{i'} + r_{i'} = \lambda$.
    
    On the other hand, consider a good pair $(v', i')$ maximizing $v'_{i'} + r_{i'}$.
    The goodness of $(v',i')$ implies that $i' = \argmin_{i \in [d]} v'_i + r_i$, hence $\lambda \geq \min_{i \in [d]} v'_i + r_i = v'_{i'} + r_{i'} = \lambda'$. This proves~\eqref{eq:query-lambda}.
    
    For every $i \in [d]$, we set $V_i$ to be the set
    $$ 
    		\{ (v_1 - v_i, v_2 - v_i, \dots, v_{i - 1} - v_i, v_{i + 1} - v_i, \dots, v_d - v_i) : v \in V \}\subseteq \mathbb{R}^{d-1},
    	$$
    	and set $w_i(v) \coloneqq v_i$. Let $\mathbb{D}_i$ be the data structure obtained by applying \Cref{t:orth_query} to $V_i$ and $w_i$. Consider the suffix range
    $$
    		R_i\coloneqq \mathsf{Range}(r_i - r_1, r_i-r_2, \dots, r_i - r_{i - 1}, r_i - r_{i + 1}, \dots,  r_i - r_d)\subseteq \mathbb{R}^{d-1}.
    $$
    Now, by \eqref{eq:query-lambda} we have that
    $$
    		\lambda = \max \left\{ r_i + \max \{ w_i(v) : v \in V_i \cap R_i \}\colon i\in [d] \right\}.
    $$
    This value can be computed by asking $d$ queries to the data structures $\mathbb{D}_i$, for $i\in [d]$. This gives us a data structure satisfying the conditions given in the lemma statement.
\end{proof}

The main work in the proof of Theorem~\ref{thm:main-apices} will be done in the following lemma,
which provides a fast computation of eccentricities once a suitable division is provided on input.
We adopt the notation for divisions introduced in the statement of \Cref{t:r_division}.

\begin{lemma}\label{l:main_ecc}
Fix constants $0 < \alpha, \gamma, \rho < 1$ and $k \in \mathbb{N}$. Assume we are given a connected graph $G$ on $n$ vertices with $O(n)$ edges with positive integer weights, a subset of vertices $X$, a subset of apices $A \subseteq V(G)$ of size at most $k$, and a family $\mathcal{R}$ with $V(G) \setminus A = \bigcup \mathcal{R} $ such that the following conditions are satisfied:
\begin{itemize}[nosep]
	\item $\sum_{R \in \mathcal{R}} |\partial R| \leq \Oh(n^\gamma)$;
	\item for every $R \in \mathcal{R}$, $|R| \leq \Oh(n^\rho)$ and $G[R]$ is connected and contains $O(|R|)$ edges; and
	\item for every $R \in \mathcal{R}$, the number of distance profiles in $G-A$ on $\partial R$ is of $\Oh(n^\alpha)$.
\end{itemize}
Then, we can compute $X$-eccentricity of every vertex of $G$ in time $\Oh(n^{\gamma + 2\rho} \log n + (n^{1 + \gamma} + n^{1 + \alpha}) \log^{k - 1} n)$.
\end{lemma}

\begin{proof}
    Let $G' \coloneqq G - A$ and $X' \coloneqq X \cap V(G')$.    Denote $A \coloneqq \{a_1,a_2,\ldots,a_k\}$. We first describe the procedure, and then discuss its time complexity.

    For~every $a \in A$ and $u \in V(G)$, we compute distance between $a$ and $u$ denoted $d_A(a, u)$.
    
    \medskip
    \emph{Step 1.} \ 
    We start by precomputing the following information for every region $R\in \mathcal{R}$.
    For all $u, v \in R$, we compute the distance between $u$ and $v$ in $G'[R]$, denoted $d_R(u, v)$.
    For all $s \in \partial R, u \in V(G')$, we compute the distance between $s$ and $u$ in $G'$, denoted $d_{\partial R}(u,s)$.
    We arbitrarily pick a pivot vertex $s_R \in \partial R$, and for brevity denote $p_R[u] \coloneqq \distprofile{\partial R}{s_R}{u}$, where the profile is considered in $G'$. That is, $p_R[u]$ is the $(\partial R)$-profile of $u$ with respect to $s_R$:
    $$p_R[u](s) = d_{\partial R}(u, s) - d_{\partial R}(u, s_R),\quad \textrm{for all }u \in V(G')\textrm{ and } s \in \partial R.$$
    We define $P_R \coloneqq \{ p_R[u] \colon u \in V(G')\}$. By our assumption, we have $|P_R| \leq \Oh(n^\alpha)$. Finally, for every profile $p \in P_R$, we list all vertices $v \in X' \setminus R$ such that $p_R[v] = p$ and set up the data structure of \Cref{l:max_min_query} for the points $(d_A(a_1, v), \dots, d_A(a_k, v), d_{\partial R_u}(s_R, v))$; denote it by $\mathbb{D}_{R, p}$.
    
    \medskip
    \emph{Step 2.} \ 
    For every $u \in V(G)$, we compute $\ecc_X(u)$ as follows. If $u \in A$, the answer is $\max_{v \in X} d_A(u, v)$. Hence, we may assume $u \not\in A$.
    Let $R_u$ denote any region of $\mathcal{R}$ containing $u$. For every $v \in R_u$, the shortest path from $u$ to $v$ in $G$ either:
    \begin{itemize}[nosep]
        \item goes through an apex, in which case its length is $\min_{a \in A} d_A(a, u) + d_A(a, v)$; or
        \item is disjoint from $A$ and intersects $\partial R_u$, in which case its length is $\min_{s \in \partial R_u} d_{\partial R_u}(s, u) + d_{\partial R_u}(s, v)$;~or
        \item is contained entirely in $R_u$, in which case its length is $d_{R_u}(u, v)$.
    \end{itemize}
    The length of this path is therefore the minimum among the three quantities.
    Using the above observation, we compute $\dist_G(u, v)$ explicitly for each $v \in R_u$, and define $\Delta^{R_u}_u \coloneqq \max_{v \in R_u \cap X} \dist_G(u, v)$.
    
    For every $v \in V(G) \setminus (A \cup R_u)$, the shortest path between $u$ and $v$ either crosses $A$ or $\partial R_u$. The length of such path avoiding $A$ is
    $$ \min_{s \in \partial R_u} d_{\partial R_u}(s, u) + d_{\partial R_u}(s, v) = 
       d_{\partial R_u}(s_R, v) + \min_{s \in \partial R_u} \left( d_{\partial R_u}(s, u) + p_{R_u}[v](s) \right). $$

We partition the vertices $v$ by their profile $p_{R_u}[v]$ and for every $p \in P_{R_u}$, we compute the maximum distance to a vertex with profile $p$ separately. Let $V_p = \{ v \in X' \setminus R_u \mid p_{R_u}[v] = p \}$. For every $v \in V_p$, we have
    $$ \dist_G(u, v) = \min \left( \min_{a \in A} d_A(a, u) + d_A(a, v), d_{\partial R_u}(s_R, v) + \min_{s \in \partial R_u} \left( d_{\partial R_u}(s, u) + p(s) \right) \right). $$
    We set $r_i \coloneqq d_A(a, u)$ for $i \in [k]$, and $r_{k + 1} \coloneqq \min_{s \in \partial R_u}  \left( d_{\partial R_u}(s, u) + p(s) \right)$. Now,
    $$ \max_{v \in V_p} \dist_G(u, v) = \max_{v \in V_p} \min(r_1 + d_A(a_1, v), \dots, r_k + d_A(a_k, v), r_{k + 1} + d_{\partial R_u}(s_R, v)).$$
    This value can be computed by querying $r_1, \dots, r_{k + 1}$ to the data structure $\mathbb{D}_{R_u, p}$. We define $\Delta^{V(G) \setminus (A \cup R_u)}_u$ as the maximum of such values over all $p \in P_{R_u}$.
    
    Finally, we set $\Delta^{A}_u \coloneqq \max_{a \in A \cap X} d_A(a, u)$, and report $\ecc_X(u) = \max \left(\Delta^{A}_u, \Delta^{R_u}_u, \Delta^{V(G) \setminus (A \cup R_u)}_u \right)$.
    
    It remains to argue that this algorithm can be implemented in the desired running time. For any source $u \in V(G)$, distance from $u$ to all vertices of $G$ can be calculated in time $\Oh((|V(G)| + |E(G)|) \log |V(G)|)$ using Dijkstra's algorithm. Therefore:
    \begin{itemize}[nosep]
        \item computing $d_A(a,\cdot)$ for all $a$ can be done in time $\Oh (n \log n)$,
        \item computing $d_{\partial R}(\cdot,\cdot)$ for all $R$ can be done in time $\Oh (n\log n \cdot \sum_{R \in \mathcal{R}} |\partial R| ) \leq \Oh (n^{1 + \gamma} \log n)$,
        \item computing $d_R(\cdot,\cdot)$ for all $R \in \mathcal{R}$ can be done in time $\Oh (|\mathcal{R}| n^{2\rho} \log n) \leq \Oh (n^{\gamma + 2\rho} \log n)$; constructing $G[R]$ takes $\Oh(|R|^2 \log n) = \Oh(n^{2\rho} \log n)$ time and calculating all pairs shortest paths can be done in time $\Oh(|R||E(G[R])| \log n) = \Oh(n^{2\rho} \log n)$.
    \end{itemize}
    Finally, the total size of the data structures $\mathbb{D}_{R, p}$ over all $R, p$ is $\Oh(|\mathcal{R}| n) = \Oh (n^{1 + \gamma})$, hence we can construct them in time $\Oh (n^{1 + \gamma} \log^{k - 1} n)$.
    
    Consider $u \in V(G) \setminus A$ fixed in step 2. Computing $\Delta^{R_u}_u$ takes $\Oh (|R| \cdot |\partial R_u|)$ time. Computing $\Delta^{A}_u$ can be done in constant time. Computing $\Delta^{V(G) \setminus (A \cup R_u)}_u$ requires asking $|P_{R_u}|$ queries to some $\mathbb{D}_{R, p}$, which takes $\Oh (n^{\alpha} \log^{k - 1} n)$ time in total.
    In total, step 2 for all vertices $u$ can be done in time $\Oh (n^{1 + \alpha} \log^{k - 1} n + n^\rho \cdot \sum_{u \in V(G) \setminus A} |\partial R_u|) = \Oh (n^{1 + \alpha} \log^{k - 1} n + n^{\gamma + 2\rho})$.
    
    We conclude that the total running time is $\Oh(n^{\gamma + 2\rho} \log n + (n^{1 + \gamma} + n^{1 + \alpha}) \log^{k - 1} n)$.
\end{proof}

The next statement is a reformulation of Theorem~\ref{thm:main-apices}. 

\begin{theorem}\label{t:main_bdgenus_apices}
Fix constants $k, g \in \mathbb{N}$. Let $\mathcal{C}$ denote the class of all graphs that can be obtained by taking a graph $G$ of Euler genus bounded by $g$, and adding $k$ apices adjacent arbitrarily to the rest of $G$ and to each other. Then there is an algorithm that given an unweighted graph $G$ belonging to $\mathcal{C}$, together with its set of apices $A$, computes the eccentricity of every vertex in time $\Oh_{k,g} \left( n^{1 + \frac{24}{25}} \log^{k - 1} n \right)$.
\end{theorem}

\begin{proof}
Let $A = \{a_1, \dots, a_k\}$ denote the set of apices and let $G' = G - A$. Fix $\rho\coloneqq \frac{2}{25}$.
Since graphs of bounded genus exclude some fixed clique as a minor,
by \Cref{t:r_division} (with $\varepsilon = \rho/2$)
we can find an $\Oh(n^\rho)$-division $\mathcal{R}$ of $G'$
satisfying $\sum_{R \in \mathcal{R}} |\partial R| = \Oh (n^{1 - \frac{\rho}{2}})$
in time $\Oh(n^{1 + \rho})$ .
By Theorem~\ref{thm:distprofiles}, the graph $G'$ has a degree $12$ polynomial bound on the number of distance profiles. In particular, the number of profiles on every $\partial R$ is of $\Oh(|R|^{12}) = \Oh(n^{12\rho})$. Let $X \coloneqq V(G)$, $\gamma \coloneqq 1 - \frac{\rho}{2} = \frac{24}{25}$ and $\alpha \coloneqq 12\rho = \frac{24}{25}$. Now,  applying \Cref{l:main_ecc} gives us an algorithm computing all eccentricities in time $\Oh(n^{1 + \frac{24}{25}} \log^{k - 1} n)$.
\end{proof}


\begin{algorithm}[ht!]
\caption{\textit{NovelSelect}}
\label{alg:novelselect}
\begin{algorithmic}[1]
\State \textbf{Input:} Data pool $\mathcal{X}^{all}$, data budget $n$
\State Initialize an empty dataset, $\mathcal{X} \gets \emptyset$
\While{$|\mathcal{X}| < n$}
    \State $x^{new} \gets \arg\max_{x \in \mathcal{X}^{all}} v(x)$
    \State $\mathcal{X} \gets \mathcal{X} \cup \{x^{new}\}$
    \State $\mathcal{X}^{all} \gets \mathcal{X}^{all} \setminus \{x^{new}\}$
\EndWhile
\State \textbf{return} $\mathcal{X}$
\end{algorithmic}
\end{algorithm}


\section*{Acknowledgements}
Marcin thanks Jacob Holm, Eva Rotenberg, and Erik Jan van Leeuwen
for many discussions on subquadratic algorithms for diameter while his stay on sabbatical
in Copenhagen.

\bibliographystyle{plain}
\begin{thebibliography}{10}

  \bibitem{BorradaileDT14}
  Glencora Borradaile, Erik~D. Demaine, and Siamak Tazari.
  \newblock Polynomial-time approximation schemes for subset-connectivity problems in bounded-genus graphs.
  \newblock {\em Algorithmica}, 68(2):287--311, 2014.
  
  \bibitem{Cabello19}
  Sergio Cabello.
  \newblock Subquadratic algorithms for the diameter and the sum of pairwise distances in planar graphs.
  \newblock {\em {ACM} Transactions on Algorithms}, 15(2):21:1--21:38, 2019.
  
  \bibitem{CabelloK09}
  Sergio Cabello and Christian Knauer.
  \newblock Algorithms for graphs of bounded treewidth via orthogonal range searching.
  \newblock {\em Computational Geometry}, 42(9):815--824, 2009.
  
  \bibitem{CabelloCL12algo}
  Sergio Cabello, Éric {Colin de Verdière}, and Francis Lazarus.
  \newblock Algorithms for the edge-width of an embedded graph.
  \newblock {\em Computational Geometry}, 45(5):215--224, 2012.
  \newblock Special issue: 26th Annual Symposium on Computation Geometry at Snowbird, Utah, USA.
  
  \bibitem{DucoffeHV22}
  Guillaume Ducoffe, Michel Habib, and Laurent Viennot.
  \newblock Diameter, eccentricities and distance oracle computations on ${H}$-minor free graphs and graphs of bounded (distance) {V}apnik-{C}hervonenkis dimension.
  \newblock {\em {SIAM} Journal on Computing}, 51(5):1506--1534, 2022.
  
  \bibitem{DurajKP23}
  Lech Duraj, Filip Konieczny, and Krzysztof Pot\k{e}pa.
  \newblock Better diameter algorithms for bounded {VC}-dimension graphs and geometric intersection graphs.
  \newblock In {\em 32nd Annual European Symposium on Algorithms, {ESA} 2024}, volume 308 of {\em LIPIcs}, pages 51:1--51:18. Schloss Dagstuhl --- Leibniz-Zentrum f{\"{u}}r Informatik, 2024.
  
  \bibitem{EricksonW05}
  Jeff Erickson and Kim Whittlesey.
  \newblock Greedy optimal homotopy and homology generators.
  \newblock In {\em Proc. of the 16th Annual {ACM-SIAM} Symposium on Discrete Algorithms, {SODA} 2005}, pages 1038--1046. {SIAM}, 2005.
  
  \bibitem{GawrychowskiKMS21}
  Pawe\l{} Gawrychowski, Haim Kaplan, Shay Mozes, Micha Sharir, and Oren Weimann.
  \newblock Voronoi diagrams on planar graphs, and computing the diameter in deterministic $\widetilde{O}(n^{5/3})$ time.
  \newblock {\em {SIAM} Journal on Computing}, 50(2):509--554, 2021.
  
  \bibitem{KorhonenPS24}
  Tuukka Korhonen, Micha\l{} Pilipczuk, and Giannos Stamoulis.
  \newblock Minor {C}ontainment and {D}isjoint {P}aths in almost-linear time.
  \newblock {\em CoRR}, abs/2404.03958, 2024.
  
  \bibitem{KorhonenPST24priv}
  Tuukka Korhonen, Micha\l{} Pilipczuk, Giannos Stamoulis, and Dimitrios Thilikos, 2024.
  \newblock Private communication.
  
  \bibitem{LeW24}
  Hung Le and Christian Wulff{-}Nilsen.
  \newblock {VC} set systems in minor-free (di)graphs and applications.
  \newblock In {\em 2024 {ACM-SIAM} Symposium on Discrete Algorithms, {SODA} 2024}, pages 5332--5360. {SIAM}, 2024.
  
  \bibitem{MoharT01grap}
  Bojan Mohar and Carsten Thomassen.
  \newblock {\em Graphs on Surfaces}.
  \newblock Johns Hopkins series in the mathematical sciences. Johns Hopkins University Press, 2001.
  
  \bibitem{RobertsonS03a}
  Neil Robertson and Paul~D. Seymour.
  \newblock Graph {M}inors. {XVI.} {E}xcluding a non-planar graph.
  \newblock {\em Journal of Combinatorial Theory, Series {B}}, 89(1):43--76, 2003.
  
  \bibitem{RodittyW13}
  Liam Roditty and Virginia~Vassilevska Williams.
  \newblock Fast approximation algorithms for the diameter and radius of sparse graphs.
  \newblock In {\em 45th Symposium on Theory of Computing Conference, STOC 2013}, pages 515--524. {ACM}, 2013.
  
  \bibitem{Sauer72}
  Norbert Sauer.
  \newblock On the density of families of sets.
  \newblock {\em Journal of Combinatorial Theory, Series A}, 13(1):145--147, 1972.
  
  \bibitem{Shelah72}
  Saharon Shelah.
  \newblock A combinatorial problem; stability and order for models and theories in infinitary languages.
  \newblock {\em Pacific Journal of Mathematics}, 41(1):247 -- 261, 1972.
  
  \bibitem{StahlB77}
  Saul Stahl and Lowell~W. Beineke.
  \newblock Blocks and the nonorientable genus of graphs.
  \newblock {\em Journal of Graph Theory}, 1(1):75--78, 1977.
  
  \bibitem{ThilikosW22}
  Dimitrios~M. Thilikos and Sebastian Wiederrecht.
  \newblock Killing a vortex.
  \newblock In {\em 63rd {IEEE} Annual Symposium on Foundations of Computer Science, {FOCS} 2022}, pages 1069--1080. {IEEE}, 2022.
  
  \bibitem{VapnikC71}
  V.~N. Vapnik and A.~Ya. Chervonenkis.
  \newblock On the uniform convergence of relative frequencies of events to their probabilities.
  \newblock {\em Theory of Probability \& Its Applications}, 16(2):264--280, 1971.
  
  \bibitem{Willard85}
  Dan~E. Willard.
  \newblock New data structures for orthogonal range queries.
  \newblock {\em SIAM Journal on Computing}, 14(1):232--253, 1985.
  
  \bibitem{WulffNilsen11}
  Christian Wulff{-}Nilsen.
  \newblock Separator theorems for minor-free and shallow minor-free graphs with applications.
  \newblock {\em CoRR}, abs/1107.1292, 2011.
  
  \end{thebibliography}

\end{document}
