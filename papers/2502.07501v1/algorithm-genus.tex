\section{Bounded Euler genus graphs with apices: proof of Theorem~\ref{thm:main-apices}}\label{sec:algo-genus}

In this section we prove Theorem~\ref{thm:main-apices}.
(Note that Theorem~\ref{thm:main-genus} is a special case of Theorem~\ref{thm:main-apices}
 for $k=1$.)
We start by deriving the following corollary from \Cref{t:orth_query}.

\begin{corollary}\label{l:max_min_query}
Let $V$ be a set of $n$ points in $\mathbb{R}^d$.
There is a data structure that uses $\Oh \left( d n \log^{d - 2} n \right)$ preprocessing time, $\Oh \left( d n \log^{d - 2} n \right)$ memory and answers the following queries in time $\Oh \left( d \log^{d - 2} n \right)$: given $r_1, \dots, r_d \in \mathbb{R}$, find $\max_{v \in V} \min_{i \in [d]} (v_i + r_i)$, where $v_i$ denotes the $i$th coordinate of $v$.
\end{corollary}

\begin{proof}
    Fix query parameters $r_1, \dots, r_d \in \mathbb{R}$.
    Let $\lambda \coloneqq \max_{v \in V} \min_{i \in [d]} (v_i + r_i)$ denote the answer we want to find.
    
    We say that a pair $(v,i) \in V \times [d]$ is \emph{good} if
    for every $j \in [d]$, it holds that $v_j - v_i \geq r_i -r_j$.
    Let
    $$
    		\lambda' = \max \left\{ v_i + r_i \colon i\in [d], v\in V,\textrm{ and }(v,i)\textrm{ is good} \right\}.
    $$
    We claim that 
    \begin{equation}
    \label{eq:query-lambda}
    \lambda = \lambda'.
   \end{equation}
   Let $v' = \argmax_{v \in V} (\min_{i \in [d]} v_i + r_i)$ and let $i' = \argmin_{i \in [d]} v'_i + r_i$. By the choice of $i'$, for each $j$ we have $v'_j+r_j\geq v'_{i'}+r_{i'}$, implying $v'_j - v'_{i'} \geq r_{i'} - r_j$. Hence $(v',i')$ is good, so $\lambda' \geq v'_{i'} + r_{i'} = \lambda$.
    
    On the other hand, consider a good pair $(v', i')$ maximizing $v'_{i'} + r_{i'}$.
    The goodness of $(v',i')$ implies that $i' = \argmin_{i \in [d]} v'_i + r_i$, hence $\lambda \geq \min_{i \in [d]} v'_i + r_i = v'_{i'} + r_{i'} = \lambda'$. This proves~\eqref{eq:query-lambda}.
    
    For every $i \in [d]$, we set $V_i$ to be the set
    $$ 
    		\{ (v_1 - v_i, v_2 - v_i, \dots, v_{i - 1} - v_i, v_{i + 1} - v_i, \dots, v_d - v_i) : v \in V \}\subseteq \mathbb{R}^{d-1},
    	$$
    	and set $w_i(v) \coloneqq v_i$. Let $\mathbb{D}_i$ be the data structure obtained by applying \Cref{t:orth_query} to $V_i$ and $w_i$. Consider the suffix range
    $$
    		R_i\coloneqq \mathsf{Range}(r_i - r_1, r_i-r_2, \dots, r_i - r_{i - 1}, r_i - r_{i + 1}, \dots,  r_i - r_d)\subseteq \mathbb{R}^{d-1}.
    $$
    Now, by \eqref{eq:query-lambda} we have that
    $$
    		\lambda = \max \left\{ r_i + \max \{ w_i(v) : v \in V_i \cap R_i \}\colon i\in [d] \right\}.
    $$
    This value can be computed by asking $d$ queries to the data structures $\mathbb{D}_i$, for $i\in [d]$. This gives us a data structure satisfying the conditions given in the lemma statement.
\end{proof}

The main work in the proof of Theorem~\ref{thm:main-apices} will be done in the following lemma,
which provides a fast computation of eccentricities once a suitable division is provided on input.
We adopt the notation for divisions introduced in the statement of \Cref{t:r_division}.

\begin{lemma}\label{l:main_ecc}
Fix constants $0 < \alpha, \gamma, \rho < 1$ and $k \in \mathbb{N}$. Assume we are given a connected graph $G$ on $n$ vertices with $O(n)$ edges with positive integer weights, a subset of vertices $X$, a subset of apices $A \subseteq V(G)$ of size at most $k$, and a family $\mathcal{R}$ with $V(G) \setminus A = \bigcup \mathcal{R} $ such that the following conditions are satisfied:
\begin{itemize}[nosep]
	\item $\sum_{R \in \mathcal{R}} |\partial R| \leq \Oh(n^\gamma)$;
	\item for every $R \in \mathcal{R}$, $|R| \leq \Oh(n^\rho)$ and $G[R]$ is connected and contains $O(|R|)$ edges; and
	\item for every $R \in \mathcal{R}$, the number of distance profiles in $G-A$ on $\partial R$ is of $\Oh(n^\alpha)$.
\end{itemize}
Then, we can compute $X$-eccentricity of every vertex of $G$ in time $\Oh(n^{\gamma + 2\rho} \log n + (n^{1 + \gamma} + n^{1 + \alpha}) \log^{k - 1} n)$.
\end{lemma}

\begin{proof}
    Let $G' \coloneqq G - A$ and $X' \coloneqq X \cap V(G')$.    Denote $A \coloneqq \{a_1,a_2,\ldots,a_k\}$. We first describe the procedure, and then discuss its time complexity.

    For~every $a \in A$ and $u \in V(G)$, we compute distance between $a$ and $u$ denoted $d_A(a, u)$.
    
    \medskip
    \emph{Step 1.} \ 
    We start by precomputing the following information for every region $R\in \mathcal{R}$.
    For all $u, v \in R$, we compute the distance between $u$ and $v$ in $G'[R]$, denoted $d_R(u, v)$.
    For all $s \in \partial R, u \in V(G')$, we compute the distance between $s$ and $u$ in $G'$, denoted $d_{\partial R}(u,s)$.
    We arbitrarily pick a pivot vertex $s_R \in \partial R$, and for brevity denote $p_R[u] \coloneqq \distprofile{\partial R}{s_R}{u}$, where the profile is considered in $G'$. That is, $p_R[u]$ is the $(\partial R)$-profile of $u$ with respect to $s_R$:
    $$p_R[u](s) = d_{\partial R}(u, s) - d_{\partial R}(u, s_R),\quad \textrm{for all }u \in V(G')\textrm{ and } s \in \partial R.$$
    We define $P_R \coloneqq \{ p_R[u] \colon u \in V(G')\}$. By our assumption, we have $|P_R| \leq \Oh(n^\alpha)$. Finally, for every profile $p \in P_R$, we list all vertices $v \in X' \setminus R$ such that $p_R[v] = p$ and set up the data structure of \Cref{l:max_min_query} for the points $(d_A(a_1, v), \dots, d_A(a_k, v), d_{\partial R_u}(s_R, v))$; denote it by $\mathbb{D}_{R, p}$.
    
    \medskip
    \emph{Step 2.} \ 
    For every $u \in V(G)$, we compute $\ecc_X(u)$ as follows. If $u \in A$, the answer is $\max_{v \in X} d_A(u, v)$. Hence, we may assume $u \not\in A$.
    Let $R_u$ denote any region of $\mathcal{R}$ containing $u$. For every $v \in R_u$, the shortest path from $u$ to $v$ in $G$ either:
    \begin{itemize}[nosep]
        \item goes through an apex, in which case its length is $\min_{a \in A} d_A(a, u) + d_A(a, v)$; or
        \item is disjoint from $A$ and intersects $\partial R_u$, in which case its length is $\min_{s \in \partial R_u} d_{\partial R_u}(s, u) + d_{\partial R_u}(s, v)$;~or
        \item is contained entirely in $R_u$, in which case its length is $d_{R_u}(u, v)$.
    \end{itemize}
    The length of this path is therefore the minimum among the three quantities.
    Using the above observation, we compute $\dist_G(u, v)$ explicitly for each $v \in R_u$, and define $\Delta^{R_u}_u \coloneqq \max_{v \in R_u \cap X} \dist_G(u, v)$.
    
    For every $v \in V(G) \setminus (A \cup R_u)$, the shortest path between $u$ and $v$ either crosses $A$ or $\partial R_u$. The length of such path avoiding $A$ is
    $$ \min_{s \in \partial R_u} d_{\partial R_u}(s, u) + d_{\partial R_u}(s, v) = 
       d_{\partial R_u}(s_R, v) + \min_{s \in \partial R_u} \left( d_{\partial R_u}(s, u) + p_{R_u}[v](s) \right). $$

We partition the vertices $v$ by their profile $p_{R_u}[v]$ and for every $p \in P_{R_u}$, we compute the maximum distance to a vertex with profile $p$ separately. Let $V_p = \{ v \in X' \setminus R_u \mid p_{R_u}[v] = p \}$. For every $v \in V_p$, we have
    $$ \dist_G(u, v) = \min \left( \min_{a \in A} d_A(a, u) + d_A(a, v), d_{\partial R_u}(s_R, v) + \min_{s \in \partial R_u} \left( d_{\partial R_u}(s, u) + p(s) \right) \right). $$
    We set $r_i \coloneqq d_A(a, u)$ for $i \in [k]$, and $r_{k + 1} \coloneqq \min_{s \in \partial R_u}  \left( d_{\partial R_u}(s, u) + p(s) \right)$. Now,
    $$ \max_{v \in V_p} \dist_G(u, v) = \max_{v \in V_p} \min(r_1 + d_A(a_1, v), \dots, r_k + d_A(a_k, v), r_{k + 1} + d_{\partial R_u}(s_R, v)).$$
    This value can be computed by querying $r_1, \dots, r_{k + 1}$ to the data structure $\mathbb{D}_{R_u, p}$. We define $\Delta^{V(G) \setminus (A \cup R_u)}_u$ as the maximum of such values over all $p \in P_{R_u}$.
    
    Finally, we set $\Delta^{A}_u \coloneqq \max_{a \in A \cap X} d_A(a, u)$, and report $\ecc_X(u) = \max \left(\Delta^{A}_u, \Delta^{R_u}_u, \Delta^{V(G) \setminus (A \cup R_u)}_u \right)$.
    
    It remains to argue that this algorithm can be implemented in the desired running time. For any source $u \in V(G)$, distance from $u$ to all vertices of $G$ can be calculated in time $\Oh((|V(G)| + |E(G)|) \log |V(G)|)$ using Dijkstra's algorithm. Therefore:
    \begin{itemize}[nosep]
        \item computing $d_A(a,\cdot)$ for all $a$ can be done in time $\Oh (n \log n)$,
        \item computing $d_{\partial R}(\cdot,\cdot)$ for all $R$ can be done in time $\Oh (n\log n \cdot \sum_{R \in \mathcal{R}} |\partial R| ) \leq \Oh (n^{1 + \gamma} \log n)$,
        \item computing $d_R(\cdot,\cdot)$ for all $R \in \mathcal{R}$ can be done in time $\Oh (|\mathcal{R}| n^{2\rho} \log n) \leq \Oh (n^{\gamma + 2\rho} \log n)$; constructing $G[R]$ takes $\Oh(|R|^2 \log n) = \Oh(n^{2\rho} \log n)$ time and calculating all pairs shortest paths can be done in time $\Oh(|R||E(G[R])| \log n) = \Oh(n^{2\rho} \log n)$.
    \end{itemize}
    Finally, the total size of the data structures $\mathbb{D}_{R, p}$ over all $R, p$ is $\Oh(|\mathcal{R}| n) = \Oh (n^{1 + \gamma})$, hence we can construct them in time $\Oh (n^{1 + \gamma} \log^{k - 1} n)$.
    
    Consider $u \in V(G) \setminus A$ fixed in step 2. Computing $\Delta^{R_u}_u$ takes $\Oh (|R| \cdot |\partial R_u|)$ time. Computing $\Delta^{A}_u$ can be done in constant time. Computing $\Delta^{V(G) \setminus (A \cup R_u)}_u$ requires asking $|P_{R_u}|$ queries to some $\mathbb{D}_{R, p}$, which takes $\Oh (n^{\alpha} \log^{k - 1} n)$ time in total.
    In total, step 2 for all vertices $u$ can be done in time $\Oh (n^{1 + \alpha} \log^{k - 1} n + n^\rho \cdot \sum_{u \in V(G) \setminus A} |\partial R_u|) = \Oh (n^{1 + \alpha} \log^{k - 1} n + n^{\gamma + 2\rho})$.
    
    We conclude that the total running time is $\Oh(n^{\gamma + 2\rho} \log n + (n^{1 + \gamma} + n^{1 + \alpha}) \log^{k - 1} n)$.
\end{proof}

The next statement is a reformulation of Theorem~\ref{thm:main-apices}. 

\begin{theorem}\label{t:main_bdgenus_apices}
Fix constants $k, g \in \mathbb{N}$. Let $\mathcal{C}$ denote the class of all graphs that can be obtained by taking a graph $G$ of Euler genus bounded by $g$, and adding $k$ apices adjacent arbitrarily to the rest of $G$ and to each other. Then there is an algorithm that given an unweighted graph $G$ belonging to $\mathcal{C}$, together with its set of apices $A$, computes the eccentricity of every vertex in time $\Oh_{k,g} \left( n^{1 + \frac{24}{25}} \log^{k - 1} n \right)$.
\end{theorem}

\begin{proof}
Let $A = \{a_1, \dots, a_k\}$ denote the set of apices and let $G' = G - A$. Fix $\rho\coloneqq \frac{2}{25}$.
Since graphs of bounded genus exclude some fixed clique as a minor,
by \Cref{t:r_division} (with $\varepsilon = \rho/2$)
we can find an $\Oh(n^\rho)$-division $\mathcal{R}$ of $G'$
satisfying $\sum_{R \in \mathcal{R}} |\partial R| = \Oh (n^{1 - \frac{\rho}{2}})$
in time $\Oh(n^{1 + \rho})$ .
By Theorem~\ref{thm:distprofiles}, the graph $G'$ has a degree $12$ polynomial bound on the number of distance profiles. In particular, the number of profiles on every $\partial R$ is of $\Oh(|R|^{12}) = \Oh(n^{12\rho})$. Let $X \coloneqq V(G)$, $\gamma \coloneqq 1 - \frac{\rho}{2} = \frac{24}{25}$ and $\alpha \coloneqq 12\rho = \frac{24}{25}$. Now,  applying \Cref{l:main_ecc} gives us an algorithm computing all eccentricities in time $\Oh(n^{1 + \frac{24}{25}} \log^{k - 1} n)$.
\end{proof}
