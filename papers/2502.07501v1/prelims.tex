\paragraph{Set systems and VC-dimension.}
A \emph{set system} is a collection $\mathcal{F}$ of subsets of a given set $A$, which we call \emph{ground set} of $\mathcal{F}$.
We say that a subset $Y\subseteq A$ is \emph{shattered by} $\mathcal{F}$
if $\{Y \cap S : S \in \mathcal{F}\} = 2^Y$, that is, the intersections of $Y$ and the sets in $\mathcal{F}$ contain every subset of $Y$.
The \emph{VC-dimension} of a set system $\mathcal{F}$ with ground set $A$
is the size of the largest subset $Y\subseteq A$ shattered by $\mathcal{F}$. The notion of VC-dimension was introduced by Vapnik and Chervonenkis~\cite{VapnikC71}.

We will use the following well-known Sauer-Shelah Lemma~\cite{Sauer72,Shelah72}, which gives a polynomial upper bound on the size of a set system of bounded VC-dimension.

\begin{lemma}[Sauer-Shelah Lemma]\label{lem:sauer-shelah}
  Let $\mathcal{F}$ be a set system with ground set $A$.
  If the VC-dimension of $\mathcal{F}$ is at most $k$, then $|\mathcal{F}|=\mathcal{O}(|A|^k)$.
\end{lemma}

\paragraph{Basic graph notation.} All our graphs are undirected.
For a graph $G$, the neighborhood of a vertex $u$ is defined as $N_G(u) = \{v~\colon~uv \in E(G)\}$
and for $X \subseteq V(G)$ we have $N_G(X) = \bigcup_{u \in X} N_G(u) \setminus X$. 

The {\em{length}} of a path $P$, denoted $|P|$, is the number of edges of $P$.
For two vertices $u,v$ of a graph $G$, the {\em{distance}} between $u$ and $v$,
denoted $\dist_G(u,v)$, is defined as the minimum length of a path in $G$ with endpoints $u$ and $v$.
For every $v\in V(G)$ and set $R \subseteq V(G)$, we set $\dist_G(v,R)\coloneqq \min\{\dist_G(v,y)\colon y\in R\}.$
For vertices $x,y$ appearing on a path $P$, by $P[x,y]$ we denote the subpath of $P$ with endpoints $x$ and $y$.
The set of vertices traversed by a path $P$ is denoted by $V(P)$.
In all above notation, we sometimes drop the subscript if the graph is clear from the context.

For a nonnegative integer $q$, we use the shorthand $[q]\coloneqq\{1,\ldots,q\}$.
For a vertex $v \in V(G)$ and a set $X \subseteq V(G)$, we define the \emph{$X$-eccentricity} of $v$ as
$\ecc_X(v) \coloneqq \max_{x \in X} \dist(v,x)$.
Thus, the eccentricity of $v$ in $G$ is the same as its $V(G)$-eccentricity.

The \emph{Euler genus} of a graph $G$ is the minimum Euler characteristic of a surface, where $G$ is embeddable.
We refer to the textbook of Mohar and Thomassen for more on surfaces and embedded graphs~\cite{MoharT01grap}.

We will use the following result of Le and Wulff-Nilsen~\cite[Theorem 1.3]{LeW24} for planar graphs. Note that the set $R$ is not necessarily connected.

\begin{theorem}\label{thm:LW}
  Let $h\ge 1$ be an integer, $G$ be a $K_h$-minor-free (unweighted, undirected) graph, $R$ be a subset of $V(G)$, and $s_R\in R$. Then the set system $$\left\{\left\{(s,i) \in R \times \mathbb{Z}~|~i \leq \dist_G(u,s)-\dist_G(u,s_R)\right\}~\colon~u \in V(G)\right\}$$
  has VC-dimension at most $h-1$.
\end{theorem}

\paragraph{Algorithmic tools.}
All our algorithms assume the word RAM model.

To cope with apices, we will need the following classic data structure due to Willard~\cite{Willard85}.
\begin{theorem}[\cite{Willard85}]\label{t:orth_query}
Let $V$ be a set of $n$ points in $\mathbb{R}^d$ and let $w\colon V \to \mathbb{R}$ be a weight function. By a \emph{suffix range}, we mean any set of the form
$$\mathsf{Range}(r_1,\ldots,r_d)\coloneqq \{ (x_1, \dots, x_d) \in \mathbb{R}^d\,\mid\,x_i \geq r_i\textrm{ for all }i\in [d] \}$$ for some range parameters $r_1,\ldots,r_d\in \mathbb{R}$.

There is a data structure that uses $\Oh\left(n \log^{d - 1} n \right)$ preprocessing time, $\Oh\left( n \log^{d - 1} n \right)$ memory and answers the following suffix range queries in time $\Oh \left( \log^{d - 1} n \right)$:
given a tuple $(r_i)_{i \in [d]}$, find the maximum value of $w(v)$ over all
$v \in V\cap \mathsf{Range}(r_1,\ldots,r_d)$.
\end{theorem}

We will also need the following standard statement about $r$-divisions.

\begin{theorem}[\cite{WulffNilsen11}]\label{t:r_division}
Let $G$ be a $K_t$-minor-free graph on $n$ vertices. For any fixed constant $\varepsilon > 0$, and for any parameter $r$ with $C t^2 \log n\leq r\leq  n$, where $C$ is some absolute constant, we can construct in time $\Oh \left( n^{1 + \varepsilon} \sqrt{r} \right)$ an $r$-division of $G$, that is, a collection $\mathcal{R}$ of connected subsets of vertices of $G$ such that:
\begin{itemize}[nosep]
\item $\bigcup \mathcal{R}=V(G)$,
\item $|R| \leq r$ for every $R \in \mathcal{R}$, and
\item $\sum_{R \in \mathcal{R}} |\partial R| \leq \Oh(nt / \sqrt{r})$,
where $\partial R = R \cap N_G(V(G) \setminus R)$.
\end{itemize}
\end{theorem}


