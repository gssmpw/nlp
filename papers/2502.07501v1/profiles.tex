
In this section we prove~\Cref{thm:distprofiles}. Our argument consists of a reduction to the planar case, where we can use the constant bound on the VC-dimension of the set system given by the distance profiles due to Le and Wulff-Nilsen~\cite{LeW24}.
The main idea behind the reduction is to consider certain notions of ``extended'' profiles, where the extension is built along a collections of shortest paths. These shortest paths can be chosen in such a way that by cutting the graph along these paths we obtain a plane graph. Then a bound on the number of the extended profiles in the obtained plane graph translates to a bound on the number of (standard) distance profiles in the original graph.

Preliminary definitions and results needed for defining profiles with respect to shortest paths are given in~\Cref{subsec:milestones}.
These extended profiles are then defined in~\Cref{subsec:mdprofiles}. There, we also prove that a fundamental lemma that equality of extended profiles entails equality of (standard) distance profiles.
The main reduction providing the proof of~\Cref{thm:distprofiles} is given at the end of this section.

\subsection{Milestones}
\label{subsec:milestones}

Let $G$ be a graph, $R$ be a subset of $V(G)$,
$v_0$ be a vertex in $V(G)$, and $P$ be a shortest path from $v_0$ to $R$. Let $x$ be the unique vertex in $V(P)\cap R$. Further, let $\le_P$ be the linear ordering of the vertices traversed by $P$: for two vertices $v,u\in V(P)$, we have $v\le_P u$ if $u$ belongs to $P[v,x]$.
We say that a vertex $v\in V(P)$ is a \emph{milestone of $P$}
if either $v=x$ or we have $\distprofile{R}{x}{v}\neq\distprofile{R}{x}{u}$, where $u$ is the successor of $v$ in $\le_P$.
We denote by $M_{R}(P)$ the set of all milestones of $P$.
Given a milestone $v\in M_{R}(P)$,
the \emph{neutral prefix of $v$ in $P$} is defined as the vertex set of the maximal subpath $Q$ of $P[v_0,v]$ satisfying the following: $v$ is the only milestone of $P$ that belongs to $Q$.

The next lemma shows that minimum-length paths towards $R$ that contain a vertex in the neutral prefix of a milestone can be assumed to pass through that milestone vertex.

\begin{lemma}
  \label{lem:dispref}
  Let $G$ be a graph, $R$ be a subset of $V(G)$, $v_0$ be a vertex in $V(G)$ and $P$ be a shortest path from $v_0$ to $R$.
  Then for every $v\in M_R(P)$, every $u$ in the neutral prefix of $v$, and every $y\in R$, it holds that
  $\dist(u,y)=|P[u,v]|+\dist(v,y)$. 
\end{lemma}
\begin{proof}
  Let $x$ be the unique vertex of $V(P)\cap R$. Note that, by definition, $\distprofile{R}{x}{v}=\distprofile{R}{x}{u}$.
  Also, $\dist(u,x)=\dist(u,v)+\dist(v,x)$ and $\dist(u,v)=|P[u,v]|$. Therefore, $\dist(u,y)=|P[u,v]|+\dist(v,y)$ for every $y\in R$.
\end{proof}

We also give an upper bound on the number of milestones.

\begin{lemma}
  \label{lem:mile}
  Let $G$ be a graph, $R$ be a connected subset of $V(G)$, $v_0$ be a vertex of $G$, and $P$ be a shortest path from $v_0$ to $R$. Then the number of milestones of $P$ is at most $|R|^2+1$.
\end{lemma}
\begin{proof}
  Let $x$ be the unique vertex of $V(P)\cap R$.
  First observe that since $P$ is a shortest path from $v_0$ to~$R$, we have $\dist(v,y)\ge \dist(v,x)$ for every $v\in V(P)$ and every $y\in R$; hence $\distprofile{R}{x}{v}(y)\ge 0$.
  Also, since $R$ is connected, for every $y\in R$ we have $\distprofile{R}{x}{x}(y)\le |R|$.
  To conclude the proof, it suffices to prove that for all $v_1,v_2\in V(P)$ with $v_1\leq_P v_2$, we have \begin{equation}\label{eq:wydra}\distprofile{R}{x}{v_1}(y)\le \distprofile{R}{x}{v_2}(y)\qquad\textrm{for all }y\in R.\end{equation}
  Indeed, \eqref{eq:wydra} together with the previous observations shows that all the distinct distance profiles of the form $\distprofile{R}{x}{v}$ for $v\in V(P)$ can be treated as vectors of length $|R|$ with entries in $\{0,\ldots,|R|\}$, and they all have distinct sums $\sum_{y\in R} \distprofile{R}{x}{v}(y)$. Since these sums range between $0$ and $|R|^2$, the total number of distinct profiles is at most $|R|^2+1$, implying the same bound on the number of milestones.

  To see why \eqref{eq:wydra} holds, note that $\dist(v_1,y)\le \dist(v_1,v_2)+\dist(v_2,y)$ implies that $$\dist(v_1,y)\le \dist(v_1,v_2)+\distprofile{R}{x}{v_2}(y)+\dist(v_2,x)=\distprofile{R}{x}{v_2}(y) + \dist(v_1,x);$$
  the last equality follows from $P$ being a shortest path containing $v_1,v_2,$ and $x$ (in this order). This in turn implies that
  $\distprofile{R}{x}{v_1}(y)=\dist(v_1,y)-\dist(v_1,x)\le \distprofile{R}{x}{v_2}(y)$, as claimed.
\end{proof}


\subsection{Anchor-distance profiles}
\label{subsec:mdprofiles}

\paragraph{Shortest path collections.}
Let $G$ be a graph and $R$ be a subset of vertices of $G$.
We say that a collection $\mathcal{P}$ of paths in $G$
is an \emph{$R$-shortest path collection} if
\begin{itemize}[nosep]
  \item every $P \in \mathcal{P}$ is a shortest path from some $v^P \in V(G)$ to $R$, i.e., $|P|=\dist(v^P,R)$; and
  \item $R \subseteq \bigcup_{P \in \mathcal{P}} V(P)$.
\end{itemize}

For each $P \in \mathcal{P}$, we denote by $x^P$ the endpoint of $P$ in $R$.
Note that the collection $\mathcal{P}$ obtained by taking, for every $y\in R$, the zero-length path from $y$ to $y$, is an $R$-shortest path collection. 

We say that an $R$-shortest path collection is \emph{consistent} if, for every $P_1,P_2 \in \mathcal{P}$
and $v \in V(P_1) \cap V(P_2)$ the paths $P_1[v,x^{P_1}]$ and $P_2[v,x^{P_2}]$ are equal. That is, 
once two paths intersect, they continue together towards $R$. 

The following statement is a direct consequence of the definition of an $R$-shortest path collection.
\begin{observation}\label{obs:short}
  Let $G$ be a graph, $R$ be a subset of vertices of $G$, and $\mathcal{P}$ be an $R$-shortest path collection. Then for every two paths $P_1,P_2\in \mathcal{P}$ and every $v\in V(P_1)\cap V(P_2)$, we have $|P_1[v,x^{P_1}]|=|P_2[v,x^{P_2}]|$.
\end{observation}

\paragraph{Anchor vertices and their prefixes.}
Let $G$ be a graph, $R$ be a subset of $V(G)$, and $\mathcal{P}$ be an $R$-shortest path collection. We denote by $H_{\mathcal{P}}$ the union of the paths in $\mathcal{P}$, i.e., the graph $(\bigcup_{P\in\mathcal{P}}V(P),\bigcup_{P\in\mathcal{P}}E(P))$.
We say that a vertex is an \emph{anchor vertex} if either it has degree more than two in $H_{\mathcal{P}}$
or it is a milestone of a path $P \in \mathcal{P}$.
We denote by $A_R(P)$ the set of all anchor vertices lying on a path $P \in \mathcal{P}$
and by $A_R(\mathcal{P})$ the set of all anchor vertices for $\mathcal{P}$.
Given a path $P\in\mathcal{P}$ with endpoints $v_0$ and $y\in R$, and an anchor vertex $w\in A_R(P)$,
the \emph{prefix of $w$ in $P$} is the vertex set of the maximal subpath $Q$ of
$P[v_0,v]$ satisfying the following: $v$ is the only anchor vertex of $P$ that belongs to $Q$.
Note that for every anchor $w\in V(P)$ there is a milestone $w'$ of $P$ (possibly $w=w'$)
such that the prefix of $w$ in $P$ is a subset of the neutral prefix of $w'$ in $P$.
Finally, for an anchor vertex $w$, the \emph{tail} of $w$, 
denoted $\mathrm{tail}(w)$, is the subgraph of $G$ consisting
of the union of all prefixes of $w$ in $P$ over all paths $P \in \mathcal{P}$ that contain $w$.

\paragraph{Hat-distances.}
Let $G$ be a graph, $R$ be a subset of vertices of $G$, and $\mathcal{P}$ be an $R$-shortest path collection.
We denote by $$U_{\mathcal{P}}\coloneqq V(G)-\bigcup_{P\in\mathcal{P}}V(P).$$
For every $u\in U_{\mathcal{P}}$,
and every anchor vertex $w\in A_R(\mathcal{P})$,
we set 
\[ \widehat{\dist}(u,w)\coloneqq \min\{|Q_{u,z}|+|P[z,w]|\colon
P \in \mathcal{P} \wedge w \in V(P) \wedge 
z\text{ is in the prefix of $w$ in $P$}\},\] 
where $Q_{u,z}$ is a shortest path from $u$ to $z$ with all its internal vertices in $U_{\mathcal{P}}$.
If such $Q_{u,z}$ does not exist for any $z \in V(\mathrm{tail}(w))$,
we set $\widehat{\dist}(u, w) \coloneqq \infty$. 

The following statement is a direct consequence of the definition of $\widehat{\dist}(\cdot,\cdot)$.

\begin{observation}
  \label{obs:dist}
  Let $G$ be a graph, $R$ be a subset of vertices of $G$, and $\mathcal{P}$ be
  an $R$-shortest path collection.
  Then for every $u\in U_{\mathcal{P}}$, we have that $$\dist(u,R)=\min \left\{\widehat{\dist}(u,w)+ \dist(w,R)
  \colon w\in A_R(\mathcal{P})\right\}.$$
\end{observation}

\paragraph{Anchor-distance profiles.}
Let $G$ be a graph, $R$ be a subset of vertices of $G$,  and $\mathcal{P}$
be an $R$-shortest path collection.
The \emph{anchor-distance profile} of a vertex $u\in U_{\mathcal{P}}$ to $R$ with respect to $\mathcal{P}$
is a function $\diststarprofile{R}{\mathcal{P}}{u}$ mapping each $w \in A_R(\mathcal{P})$ to
\[\diststarprofile{R}{\mathcal{P}}{u}(w) \coloneqq \widehat{\dist}(u,w) + \dist(w,R)
 - \dist(u,R).\]
\Cref{obs:dist} implies that we always have $\diststarprofile{R}{\mathcal{P}}{u}(w) \geq 0$.
We set
\[\hatprofile{R}{\mathcal{P}}{u}(w)\coloneqq\min\{\diststarprofile{R}{\mathcal{P}}{u}(w),|R|+1\}.\]

\begin{lemma}\label{lem:hat-to-normal}
  Let $G$ be a graph, let $R$ be a connected subset of vertices of $G$, and $s_R\in R$. Also, let $\mathcal{P}$ be an $R$-shortest path collection.
  Then for all $u_1,u_2\in U_{\mathcal{P}}$,
  $$\hatprofile{R}{\mathcal{P}}{u_1}=\hatprofile{R}{\mathcal{P}}{u_2}\qquad\textrm{implies}\qquad \distprofile{R}{s_R}{u_1}=\distprofile{R}{s_R}{u_2}.$$
\end{lemma}
\begin{proof}
  Fix $u_1,u_2\in U_{\mathcal{P}}$ with
  $\hatprofile{R}{\mathcal{P}}{u_1}=\hatprofile{R}{\mathcal{P}}{u_2}$.
  We start by proving the following.

  \begin{claim}\label{cl:bnd}
    Let $u\in U_{\mathcal{P}}$ and $y\in R$. There is an anchor $w \in A_R(\mathcal{P})$
    such that
    \begin{itemize}[nosep]
      \item $\widehat{\dist}(u,w)+\dist(w,y) = \dist(u,y)$ and
      \item $\hatprofile{R}{\mathcal{P}}{u}(w)\le |R|$.
    \end{itemize}
  \end{claim}
  \begin{proof}
    Let $Q$ be a shortest path from $u$ to $y$ and let $P\in\mathcal{P}$ be the path which $Q$ first intersects
    (if the first vertex of $Q$ in $\bigcup_{P \in \mathcal{P}} V(P)$ belongs to more than one paths in~$\mathcal{P}$,
    we choose $P$ arbitrarily among these paths). 
    Also, let $u'$ be the first vertex of $Q$ (when ordering from $u$ to $y$) in $V(P)$
    and $w$ be the anchor of $P$ that contains $u'$ in its prefix (in $P$).
    Note that $u' \in V(\mathrm{tail}(w))$.

    We first show that 
    \begin{equation}\label{eq:bnd:1}\widehat{\dist}(u,w)+\dist(w,y)= \dist(u,y).\end{equation}
    By~\Cref{lem:dispref} and the fact that $|Q[u',y]| = \dist(u',y)$, we have
    \begin{equation}\label{eq:I}
      \dist(w,y) = |Q[u',y]|-|P[u',w]|.
    \end{equation}
    Also, by definition, we have
    \begin{equation}\label{eq:II}
      \widehat{\dist}(u,w)\le |Q[u,u']|+|P[u',w]|.
    \end{equation}
    By~\eqref{eq:I} and~\eqref{eq:II}, we get that $\widehat{\dist}(u,w)+\dist(w,y)\le |Q|$.
    Moreover, since $Q$ is a shortest path from $u$ to $y$ and $\widehat{dist}(u,w) \geq \dist(u,w)$,
    we have
    \[ |Q| = \dist(u,y) \leq \dist(u,w) + \dist(w,y) \leq \widehat{dist}(u,w) + dist(w,y).\]
    This proves~\eqref{eq:bnd:1}.
 
    Next, we show that $\hatprofile{R}{\mathcal{P}}{u}(w)\le |R|$. Note that
    \begin{align*}
      \diststarprofile{R}{\mathcal{P}}{u}(w) + \dist(u,R) & = \widehat{\dist}(u,w)+\dist(w,R)\\
      & \le \widehat{\dist}(u,w)+\dist(w,y)= \dist(u,y).
    \end{align*}
    The connectivity of $R$ implies that $\dist(u,y)\le \dist(u,R)+|R|$, which gives
    $\diststarprofile{R}{\mathcal{P}}{u}(w)\le |R|$, and the claim follows.
  \end{proof}

  We next show that
  there is an integer $c$ such that for every $y\in R$, we have
  $$\dist(u_1,y)=\dist(u_2,y)+c.$$
  Note that this will immediately imply that
  $\distprofile{R}{s_R}{u_1}=\distprofile{R}{s_R}{u_2}$.

  By Observation~\ref{obs:dist}, for every $h \in \{1,2\}$, there is an anchor $w_h \in A_R(\mathcal{P})$
  such that $\dist(u_h,R) = \widehat{\dist}(u_h,w_h) + \dist(w_h, R)$, which is equivalent to
  $\diststarprofile{R}{\mathcal{P}}{u_h}(w_h) = 0$. 
  If $w_h$ lies on $P_h \in \mathcal{P}$, then $\dist(u_h,R) = \dist(u_h,x^{P_h})$. 
  Therefore, as $\diststarprofile{R}{\mathcal{P}}{u_1}=\diststarprofile{R}{\mathcal{P}}{u_2}$,
  we can choose $w_1 = w_2$ and $P_1 = P_2$, hence $x^{P_1} = x^{P_2}$. 
  In other words, there exists $x\in R$ such that $\dist(u_1,R) = \dist(u_1,x)$ and $\dist(u_2,R) = \dist(u_2,x)$.
  We set $c\coloneqq \dist(u_1,x)-\dist(u_2,x)=\dist(u_1,R)-\dist(u_2,R)$.

  Now, fix $y\in R$.
  Let $w_1\in A_R(P)$ be the anchor from~\Cref{cl:bnd} (applied for $u_1$ and $y$).
  As $\hatprofile{R}{\mathcal{P}}{u_1} = \hatprofile{R}{\mathcal{P}}{u_2}$ and
  $\diststarprofile{R}{\mathcal{P}}{u_1}(w_1)\le |R|$, we have
   $\diststarprofile{R}{\mathcal{P}}{u_1}(w_1) = \diststarprofile{R}{\mathcal{P}}{u_2}(w_1)$, i.e.,
  \[\widehat{\dist}(u_1,w_1) + \dist(w_1,R) - \dist(u_1,R) = \widehat{\dist}(u_2,w_1) + \dist(w_1,R)- \dist(u_2,R).\]
  Therefore,
  \begin{align*}
    \dist(u_1,y) &= \widehat{\dist}(u_1,w_1) + \dist(w_1,y)\\
      & =\widehat{\dist}(u_2,w_1) + \dist(w_1,y)+c\geq \dist(u_2,y)+c;
  \end{align*}
  the first equality follows from~\Cref{cl:bnd}.
  Thus $\dist(u_2,y)+c\le \dist(u_1,y)$.
  A symmetric reasoning shows that also $\dist(u_1,y)-c\le\dist(u_2,y)$.
  Therefore we get $\dist(u_1,y) = \dist(u_2,y)+c$, as required.
\end{proof}

\subsection{Reduction from bounded genus graphs to planar graphs}


We next recall several definitions related to embeddings of graphs on surfaces.
Our basic terminology follows~\cite{MoharT01grap}.
We say that a graph $H$ embedded in a surface $\Sigma$ is a {\em{simple cut-graph}} of $\Sigma$ if $H$ has a single face that is also homeomorphic to an open disk; equivalently, $H$ has a single facial walk.
Given a surface $\Sigma$ and a simple cut-graph $H$ on $\Sigma$, we denote by $\Sigma\cutgraph H$ the surface obtained by cutting $\Sigma$ along~$H$. Note that, provided $H$ is a simple cut-graph, $\Sigma\cutgraph H$ is always a disk.

Suppose now that a graph $G$ embedded on  $\Sigma$ and $H$ is a subgraph of $G$ that is a simple cut-graph of $H$.
We define $G\cutgraph H$ to be the graph embedded on $\Sigma\cutgraph H$ obtained from $G$ as follows.
First, let $\sigma$ be the (unique) facial walk of $H$ and note that each edge $e$ of $H$ is contained exactly twice in $\sigma$ and each vertex $v$ of $H$ is contained in $\sigma$ as many times as the degree of $v$ in $H$.
To obtain $G\cutgraph H$, we replace $H$ with a simple cycle $C_\sigma$ whose vertex set is the set of copies of vertices of $H$ and its edge set is the set of copies of edges of $H$ in the obvious way. Notice that $\sigma$ also prescribes for every edge $uv$ of $G$ between a vertex $u\in V(G)\setminus V(H)$ and a vertex $v\in V(H)$, to which copy of $v$ in $G\cutgraph H$ the vertex $u$ should be adjacent to (in $G\cutgraph H$).
See~\Cref{fig:cutopen} for an illustration.

\begin{figure}[ht]
  \centering
  \includegraphics[width=0.8\textwidth]{cut}
  \caption{Left: A graph $G$ embedded on a surface $\Sigma$ and a subgraph $H$ of $G$ (in blue) that is a simple cut-graph of $\Sigma$. Right: The graph $G\cutgraph H$ embedded on the surface $\Sigma\cutgraph H$ (which is homeomorphic to a disk); the blue vertices/edges are copies of the vertices/edges of $H$.}
  \label{fig:cutopen}
\end{figure}

We will use the following well-known result which appears in the literature under different formulations; see e.g.~\cite{BorradaileDT14,CabelloCL12algo,EricksonW05}.

\begin{lemma}\label{lem:genuscut}
  For every integer $k\ge 1$ and for every edge-weighted connected graph $G$ embedded on a surface $\Sigma$ of Euler genus at most $k$ and every vertex $u\in V(G)$, there is a subgraph $H$ of $G$ with the following properties:
  \begin{itemize}[nosep]
    \item  $H$ is a simple cut-graph of $\Sigma$, and
    \item  $V(H)$ is the union of the vertex sets of $\mathcal{O}(k)$ shortest paths in $G$ that have $u$ as a common endpoint.
  \end{itemize}
  \end{lemma}

We are now ready to proceed to the proof of~\Cref{thm:distprofiles}.
Employing~\Cref{lem:hat-to-normal},
we aim at bouding the VC-dimension of the set system defined by the anchor-distance profiles.
This is can be done by a suitable reduction to the planar setting using~\Cref{lem:genuscut}.



\begin{proof}[Proof of~\Cref{thm:distprofiles}]
  We assume that $G$ is connected -- the distance profiles of all vertices that are not connected to $R$ are equal.
  Let $T_R$ be a spanning tree of $G[R]$ and let $G_0$ be the graph obtained from $G$ after contracting $T_R$ into a single vertex $v_R$. 
  Consider an embedding of $G_0$ on a surface $\Sigma$ of Euler genus at most $k$.
  By~\Cref{lem:genuscut}, there is a subgraph $H_0$ of $G_0$ that is a simple cut-graph of $\Sigma$ and a family $\mathcal{P}_0$ of $\mathcal{O}(k)$ shortest
  paths in $G_0$, each with $v_R$ as an endpoint, such that $V(H_0)=\bigcup_{P\in\mathcal{P}_0}V(P)$.
  Furthermore, as Lemma~\ref{lem:genuscut} handles edge weights, we can slightly perturb
  the weights so that shortest paths in $G_0$ are unique and, in particular, 
  all shortest paths with one endpoint in $v_R$ form a tree.
  Since $H_0$ is a simple cut-graph of $\Sigma$, $G_0\cutgraph H_0$ is disk-embedded.
  Uncontracting $T_R$, we get a subgraph $H$ of $G$ such that $G\cutgraph H$ is disk-embedded.
  Let $\mathcal{P}$ be the family of projections of the paths of $\mathcal{P}_0$
  onto $G$ plus, for every $y \in R$, a zero-length path from $y$ to $y$. 
  Hence, $\mathcal{P}$ is an $R$-shortest paths collection of size $\mathcal{O}(k)$
  with $V(\mathcal{P}) = V(H)$.
  Furthermore, since in $G_0$ the paths of $\mathcal{P}_0$ formed a tree rooted
  at $v_R$, $\mathcal{P}$ is consistent.

  Note that due to~\Cref{lem:mile} we have that $\sum_{P\in\mathcal{P}} |M_R(P)|\le \mathcal{O}_k(|R|^2)$.
  Also, since $\mathcal{P}$ is consistent, 
  if $B$ are the vertices that are not in $R$
  (recall that vertices in $R$ are milestones) and have degree more than two in the graph obtained by the union of the paths in $\mathcal{P}$, then $|B|\leq |\mathcal{P}|-1$.
  Hence,
  \begin{equation}
    \sum_{P\in\mathcal{P}} |A_R(P)|\le \mathcal{O}_k(|R|^2).\label{eq:mile}
  \end{equation}
  We set $\mathcal{T}$ be the set of all vertices of $G\cutgraph H$
  that are copies of the anchor vertices $A_R(\mathcal{P})$.
  Every anchor vertex has $\mathcal{O}_k(1)$ copies in $\mathcal{Q}$
  and therefore, due to~\eqref{eq:mile},
  \begin{equation}
    |\mathcal{T}|=\mathcal{O}_k(|R|^2).\label{eq:term_size}
  \end{equation}
  For $s \in \mathcal{T}$, let $w(s) \in A_R(\mathcal{R})$ be the anchor vertex
  whose copy (in $G \cutgraph H$) is $s$. In the other direction, for $w \in A_R(\mathcal{R})$, let $S(w)$
  be the set of copies of $w$ in $G \cutgraph H$. 
  
  Let $U$ be the set of vertices of $G\cutgraph H$ that are \textsl{not} copies of vertices from $H$ (i.e., $U=V(G)\setminus V(H)$).
  We set $E_{\mathsf{out}}$ be the set of all edges $uv$ of $G\cutgraph H$ where $u\in U$ and $v$ is a copy of a vertex from $H$,
  i.e., $v\in V(G\cutgraph H)\setminus U$.
  We also set $E_{\mathsf{next}}$ be the set of all edges $uv$ of $G\cutgraph H$ where $u$ is a copy
  of an anchor vertex $w\in A_R(P)$ for some $P\in\mathcal{P}$ and $v$ is a copy of the neighbor of $w$
  in $P$ that \textsl{is not} in the prefix of $w$ in $P$.
  
  Let now $\widehat{G}$ be the graph obtained from $G\cutgraph H$ after the following modifications:
  \begin{itemize}[nosep]
    \item we subdivide $|V(G)|$-many times each edge in $E_{\mathsf{out}}\cup E_{\mathsf{next}}$,
    \item we introduce a new vertex $t$ and add, for every $s\in\mathcal{T}$, a path between $t$
    and $s$ of length $$d_{w(s),t}\coloneqq |V(G)| + \dist_G(w(s),R).$$
  \end{itemize}
  See~\Cref{fig:hatG}.
  Observe that since $G\cutgraph H$ is disk-embedded, $\widehat{G}$ is planar,
  because we may embed $t$ together with all the added paths outside of the disk containing $G\cutgraph H$.

  \begin{figure}[ht]
    \centering
    \includegraphics[width=0.3\textwidth]{hatG}
    \caption{An illustration of (a part of) the construction of the graph $\widehat{G}$. The squared vertices are copies of anchor vertices. The marked squared vertex is also a copy of a vertex in $R$. The highlighted edges are copies of edges of $H$ in $G\cutgraph H$, while the paths obtained by subdividing the edges of $E_{\mathsf{out}}\cup E_{\mathsf{next}}$ are depicted with dashed edges. Edges adjacent to $t$ correspond to paths of appropriate length.}
    \label{fig:hatG}
  \end{figure}

  For every $u\in U$, we define a function $\pi[u]$, mapping every $w\in A_R(\mathcal{P})$ to
  \[\pi[u](w)\coloneqq\min\{\dist_{\widehat{G}}(u,s): s \in S(w)\}+d_{w,t}-\dist_{\widehat{G}}(u,t).\]
  Also, we set $\widehat{\mathcal{X}}\coloneqq\{\widehat{X}_u\mid u\in U\}$,
  where for $u\in U$,
  $$\widehat{X}_u\coloneqq\left\{(w,i) \in A_R(\mathcal{P}) \times \{0,\ldots,|R|+1\}~|~i \leq \pi[u](w)\right\}.$$

  \begin{claim}\label{cl:vcd}
    The set system $\widehat{\mathcal{X}}$ has size $\mathcal{O}_k(|R|^{12})$.
  \end{claim}

  \begin{proof}
  We set $\mathcal{T}^+\coloneqq \mathcal{T}\cup \{t\}$.
  We start with the set system $\mathcal{C}^1\coloneqq\left\{C_u^1~\colon~u \in U\right\}$, where
  $$C_u^1\coloneqq\left\{(s,i) \in \mathcal{T}^+ \times \mathbb{Z}~|~i \leq \dist_{\widehat{G}}(u,s)-\dist_{\widehat{G}}(u,t)\right\}.$$
  As $\widehat{G}$ is planar, by~\Cref{thm:LW} we infer that $\mathcal{C}$ has VC-dimension at most 4.

  We now ``shift columns'' of $\mathcal{C}^1$. That is, define $\mathcal{C}^2 \coloneqq \left\{C_u^2~\colon~u \in U\right\}$, where
  $$C_u^2\coloneqq\left\{(s,i) \in \mathcal{T}^+ \times \mathbb{Z}~|~i \leq \dist_{\widehat{G}}(u,s)+d_{w(s),t}-\dist_{\widehat{G}}(u,t)\right\}.$$
  Clearly, the VC-dimension of $\mathcal{C}^1$ and $\mathcal{C}^2$ are equal: a set $Z \subseteq \mathcal{T}^+ \times \mathbb{Z}$
  shatters $\mathcal{C}^1$ if and only if the set $\{(s,d_{w(s),t}+i)~\colon~(s,i) \in Z\}$ shatters $\mathcal{C}^2$. 

  Now, let $\mathcal{C}^3$ be ``cropped'' $\mathcal{C}^2$:
  $\mathcal{C}^3 \coloneqq \left\{ C_u^3~\colon~u \in U\right\}$, where
  \[ C_u^3 \coloneqq C_u^2 \cap \left(\mathcal{T}^+ \times \{0,\ldots, |R|+1\}\right). \]
  Since restricting to a smaller universe cannot increase VC-dimension, $\mathcal{C}^3$ has VC-dimension at most $4$. 
  Since $|\mathcal{T}^+| = \Oh_k(|R|^2)$, by Sauer-Shelah lemma (Lemma~\ref{lem:sauer-shelah})
  we have $|\mathcal{C}^3| = \Oh_k(|R|^{12})$. 

  Now observe that for every $u_1,u_2 \in U$
  \begin{equation}\label{eq:CtoX}
   C^3_{u_1} = C^3_{u_2} \qquad \mathrm{implies} \qquad \widehat{X}_{u_1} = \widehat{X}_{u_2}. 
  \end{equation}
  Indeed, the assumption $C^3_{u_1} = C^3_{u_2}$ implies that
  for every $w \in A_R(\mathcal{P})$ and $s \in S(W)$ we have
  \begin{align*}
  & \max(0, \min(|R|+1, \dist_{\widehat{G}}(u_1,s)+d_{w,t}-\dist_{\widehat{G}}(u_1,t))) \\ 
  &= \max(0, \min(|R|+1, \dist_{\widehat{G}}(u_2,s)+d_{w,t}-\dist_{\widehat{G}}(u_2,t))).
\end{align*} 
  For fixed $w \in A_R(\mathcal{P})$, we take a minimum of the above expression over all $s \in S(w)$, obtaining:
  \begin{align*}
    & \max(0, \min(|R|+1, \min\{\dist_{\widehat{G}}(u_1,s)~\colon~s \in S(w)\}+d_{w,t}-\dist_{\widehat{G}}(u_1,t))) \\ 
    &= \max(0, \min(|R|+1, \min\{\dist_{\widehat{G}}(u_2,s)~\colon~s \in S(w)\}+d_{w,t}-\dist_{\widehat{G}}(u_2,t))).
  \end{align*} 
  This proves~\eqref{eq:CtoX}.
  From~\eqref{eq:CtoX}, we infer $|\widehat{\mathcal{X}}| \leq |\mathcal{C}^3| = \Oh_k(|R|^{12})$, as desired.
  \end{proof}

  We next relate the distance from a vertex $u\in U$ to $R$ (in $G$) and to $t$ (in $\widehat{G}$).
  \begin{claim}\label{cl:dist}
    For every $u\in U$, $\dist_G(u,R)=\dist_{\widehat{G}}(u,t) - 2|V(G)|$.
  \end{claim}
  \begin{proof}
    Fix $u\in U$.
    We first show that $\dist_G(u,R)\le \dist_{\widehat{G}}(u,t) - 2|V(G)|$.
    For this, consider a shortest path $\widehat{Q}$ in $\widehat{G}$ from $u$ to $t$.
    Observe that there is a vertex $s\in\mathcal{T}$ that is a copy of an anchor vertex $w$,
    such that $\widehat{Q}[s,t]$ is the path from $s$ to $t$ of length $d_{w,t}$
    added in the construction of $\widehat{G}$ from $G\cutgraph H$. Recall that $d_{w,t} =\dist_G(w,R)+|V(G)|$.
    Also, observe that $\widehat{Q}[u,s]$ contains at least one subdivided edge of $E_{\mathsf{out}}$, as it starts
    in $U$ and ends outside $U$, and otherwise corresponds to a walk from $u$ to $w$ in $G$.
    Therefore, we have
    \begin{align*}
      \dist_{\widehat{G}}(u,t)  = |\widehat{Q}| & = |\widehat{Q}[u,s]|+|\widehat{Q}[s,t]|\\ 
        & = |\widehat{Q}[u,s]| + \dist_G(w,R)+|V(G)|\\
        & \ge |V(G)| + \dist_G(u, w) + \dist_G(w,R) + |V(G)|\\
        & \ge \dist_G(u,R) + 2|V(G)|.
    \end{align*}
    
    We next show that $\dist_G(u,R)\ge \dist_{\widehat{G}}(u,t) - 2|V(G)|$.
    For this, consider a shortest path $Q$ in $G$ from $u$ to $R$. Let $y\in R$ be the unique vertex in $R\cap V(Q)$.
    Also, let $z$ be the first vertex of $Q$ (when ordering from $u$ to $y$)
    in $\bigcup_{P\in\mathcal{P}}V(P)$ and let $P\in\mathcal{P}$ be the path that $z$ is contained
    (if $z$ is contained to more than one paths, pick one of them arbitrarily).
    Also, let $w$ be the first vertex of $P[z,x^P]$ (when ordering from $z$ to $x^P$) that is an anchor vertex.
    Observe that $Q[u,z]$ corresponds to a path in $\widehat{G}$ from $u$ to a copy $s'$ of $z$
    that contains exactly one subdivided edge of $E_{\mathsf{out}}$ (and no edge of $E_{\mathsf{next}}$)
    and there is a copy of $P[z,w]$ in $\widehat{G}$ from $s'$ to a copy $s$ of $w$ 
    that contains no edge of $E_{\mathsf{out}} \cup E_{\mathsf{next}}$. 
    Therefore,
    \begin{align*}
      |Q| & = |Q[u,z]|+|Q[z,y]|& \\
      & = |Q[u,z]|+ |P[z,x^P]| & \text{\!\!\!($Q[z,y]$ and $P[z,x^P]$ being shortest paths from $z$ to $R$)}\\
      & = |Q[u,z]|+ |P[z,w]| + |P[w,x^P]|& \\
      & = |Q[u,z]|+ |P[z,w]| + \dist_G(w,R) & \text{($P$ being shortest path from a vertex $v^P$ to $R$)}\\
      & \ge \dist_{\widehat{G}}(u,s) - |V(G)| + d_{w,t} - |V(G)|& \\
      & \ge \dist_{\widehat{G}}(u,t) - 2|V(G)|. &  
    \end{align*} 
    Thus, we have $\dist_G(u,R) = |Q| \ge \dist_{\widehat{G}}(u,t)-2|V(G)|$, as desired.
  \end{proof}

  \begin{claim}\label{cl:dist2}
    For every $u \in U$ and $w \in A_R(\mathcal{P})$, it holds that
    \begin{align*}
    \widehat{\dist}(u,w) < \infty &\quad\mathrm{if\ and\ only\ if} \quad \widehat{\dist}(u,w) = \min\left\{\dist_{\widehat{G}}(u,s)~\colon~s \in S(w)\right\}-|V(G)|,\ \mathrm{and}\\
    \widehat{\dist}(u,w) = \infty &\quad\mathrm{if\ and\ only\ if} \quad \min\left\{\dist_{\widehat{G}}(u,s)~\colon~s \in S(w)\right\} > 2|V(G)|.
    \end{align*}
  \end{claim}
  \begin{proof}
    We first show that if $\widehat{\dist}(u,w) < \infty$, then 
    there exists $s \in S(w)$ with $\dist_{\widehat{G}}(u,s) \leq |V(G)| + \widehat{\dist}(u,w)$. 
    To this end, let $Q$ be a path from $u$ to $w$ in $G$ of length $\widehat{\dist}(u,w)$, as in the definition
    of $\widehat{\dist}(u,w)$. There exists $P \in \mathcal{P}$ with $w \in A_R(P)$ and a vertex $z \in V(P) \cap V(Q)$
    such that $Q$ decomposes into $Q[u,z]$ and $Q[z,w] = P[z,w]$, with all internal vertices of $Q[u,z]$ in $U$. 
    Then, $\widehat{G}$ contains a copy $s'$ of $z$ such that $Q[u,z]$ projects to a path from $u$ to $s'$
    with one subdivided edge of $E_{\mathsf{out}}$ (and no edge of $E_{\mathsf{next}}$) and also a copy of $P[z,w]$ from $s'$
    to a copy $s$ of $w$ with no subdivided edge of $E_{\mathsf{out}} \cup E_{\mathsf{next}}$.
    The concatenation of these two paths witness that $\dist_{\widehat{G}}(u,s) \leq |V(G)| + \widehat{\dist}(u,w)$, as desired.

    To finish the proof of the claim, it suffices to show that if there exists $s \in S(w)$ with
    $\dist_{\widehat{G}}(u,s) \leq 2|V(G)|$, then $\widehat{\dist}(u,w) \leq \dist_{\widehat{G}}(u,s) - |V(G)|$
    (in particular, $\widehat{\dist}(u,w) \neq \infty$).
    To this end, let $\widehat{Q}$ be a path in $\widehat{G}$ from $u$ to $s$ of minimum length. 
    Since $u \in U$ but $s \notin U$, $\widehat{Q}$ necessarily contains at least one subdivided edge of $E_{\mathsf{out}}$. 
    Since $|\widehat{Q}| \leq 2|V(G)|$, $\widehat{Q}$ contains exactly one edge of $E_{\mathsf{out}}$, no edge of 
    $E_{\mathsf{next}}$, and no edge incident with $t$. Consequently, there exists a vertex $s'$ on $\widehat{Q}$
    which is a copy of a vertex $z$ that lies in the prefix of $w$ on some path $P \in \mathcal{P}$ such that 
    $\widehat{Q}$ decomposes as $\widehat{Q}[u,s']$, which has all internal vertices in $U$, and $\widehat{Q}[s',s]$
    going along a copy of $P[z,w]$ to $s \in S(w)$. Hence, $\widehat{Q}$ corresponds to a path $Q$ in $G$
    from $u$ to $w$ that satisfies the requirements of the definition of $\widehat{\dist}(u,w)$
    and witnesses $\widehat{\dist}(u,w) \leq |\widehat{Q}| - |V(G)|$, as desired. 

    This finishes the proof of the claim.
  \end{proof}
  
  Using the two previous claims, we infer that for every $u \in U$ and $w \in A_R(\mathcal{P})$ it holds that
  \begin{equation}\label{eq:prof-to-dist}
    \hatprofile{R}{\mathcal{P}}{u}(w) = \min(|R|+1, \pi[u](w)).
  \end{equation}
  Indeed, 
  \begin{align*}
  \min\left(|R|+1, \pi[u](w)\right) &= \min\left(|R|+1,\min\left\{\dist_{\widehat{G}}(u,s)~\colon~s\in S(w)\right\}+d_{w,t}-\dist_{\widehat{G}}(u,t)\right)\\
  &=\min\big(|R|+1, \min\left\{\dist_{\widehat{G}}(u,s)~\colon~s\in S(w)\right\} - |V(G)| &\\
  &\qquad\qquad\qquad\qquad + \dist_G(w,R)-\dist_G(u,R)\big) &\text{by Claim~\ref{cl:dist}}\\
  &=\min\left(|R|+1, \widehat{\dist}(u,w) + \dist_G(w,R)-\dist_G(u,R)\right)&\text{by Claim~\ref{cl:dist2}}\\
  &=\hatprofile{R}{\mathcal{P}}{u}(w).
  \end{align*}
  Here, in the third step we used the estimate $\dist_G(u,R) - \dist_G(w,R) \leq |U| \leq |V(G)|-|R|$, 
  so if $\min\left\{\dist_{\widehat{G}}(u,s)~\colon~s\in S(w)\right\} > 2|V(G)|$
  (which is equivalent to $\widehat{\dist}(u,w) = \infty$ by Claim~\ref{cl:dist2}),
  then the minimum is attained at value $|R|+1$.

  For every $u\in U$, we set
    $$B_u\coloneqq\left\{(w,i) \in A_R(\mathcal{P}) \times \mathbb{Z}~|~i \leq \hatprofile{G}{R}{u}(w)\right\}.$$
  Claim~\ref{cl:vcd} and~\eqref{eq:prof-to-dist} imply that the set system $\{B_u~\colon~u\in U\}$
    has size $\mathcal{O}_k(|R|^{12})$.

  
  
  Now, for every $v\in V(G)$, we set $$S_v\coloneqq\left\{(s,i) \in R \times \{-|R|,\ldots,|R|\}~|~i \leq \distprofile{R}{s_R}{v}(s)\right\}.$$
  The bound on the size of the set system $\{B_u~\colon~u \in U\}$ and~\Cref{lem:hat-to-normal}
  imply that the size of $\left\{S_u~\colon~u \in U \right\}$ is bounded by $\Oh_k(|R|^{12})$.
  We conclude the proof of the lemma by bounding the size of $\left\{S_u~\colon~u \in V(G)\setminus U \right\}$. For this, note that every vertex $v\in V(G)\setminus U$ is either a milestone for some path $P\in\mathcal{P}$ or a vertex in the neutral prefix of a milestone.
  In the latter case, there is a path $P\in\mathcal{P}$ and a milestone $w\in M_R(P)$ such that $S_v=S_w$. Therefore, we have
  $$|\left\{S_u~\colon~u \in V(G)\setminus U \right\}|\le \sum_{P\in\mathcal{P}} |M_R(P)|\le \mathcal{O}_k(|R|^2),$$
  where the second inequality follows from~\eqref{eq:mile}.
  Hence, the size of $\left\{S_v~\colon~v \in V(G)\right\}$ is at most $$|\left\{S_u~\colon~u \in U\right\}|+|\left\{S_u~\colon~u \in V(G)\setminus U\right\}|=\Oh_k(|R|^{12}).$$
  This finishes the proof of Theorem~\ref{thm:distprofiles}.
\end{proof}
