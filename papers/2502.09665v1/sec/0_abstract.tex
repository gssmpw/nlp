\begin{abstract}
Identifying subtle phenotypic variations in cellular images is critical for advancing biological research and accelerating drug discovery. These variations are often masked by the inherent cellular heterogeneity, making it challenging to distinguish differences between experimental conditions. Recent advancements in deep generative models have demonstrated significant potential for revealing these nuanced phenotypes through image translation, opening new frontiers in cellular and molecular biology as well as the identification of novel biomarkers. Among these generative models, diffusion models stand out for their ability to produce high-quality, realistic images. However, training diffusion models typically requires large datasets and substantial computational resources, both of which can be limited in biological research. In this work, we propose a novel approach that leverages pre-trained latent diffusion models to uncover subtle phenotypic changes. We validate our approach qualitatively and quantitatively on several small datasets of microscopy images. Our findings reveal that our approach enables effective detection of phenotypic variations, capturing both visually apparent and imperceptible differences. Ultimately, our results highlight the promising potential of this approach for phenotype detection, especially in contexts constrained by limited data and computational capacity.
\end{abstract}