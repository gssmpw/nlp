\section{Hardness results}
\label{sec:hardness}


We will first present two hardness results 
for this problem.
Our first theorem, states that it is not possible to design a nontrivial dominant-strategy incentive-compatible mechanism for this problem.
To state this recall that $\strati$ is said
to be dominant strategy for agent $i$ if
for all strategies $\stratmi$ of other agents, and all alternative strategies $\strati'$ for agent $i$
we have $\utilci(\mech, \strat_i^{\star}, \strat_{-i}) \geq  \utilci(\mech, \strat_i, \strat_{-i})$. 



\begin{theorem}
    \label{thm:dsic}
    % Let $\mech$ be a mechanism for this problem.
    % If $\strati=(\nni, \subfunci)$ is a dominant strategy for agent $i$, then $\nni=0$ with probability $1$.
    If $\strat=\cbr{(\nni, \subfunci)}_{i\in\contributors}$ is a dominant strategy profile under any mechanism $\mech$, then $\forall i\in\contributors$, $\nni=0$.
\end{theorem}


Our next theorem provides an upper bound on the maximum profit achievable in any Nash equilibrium of any mechanism.
In~\S\ref{sec:mechanism}, we will design a mechanism whose profit is arbitrarily close to this upper bound.

\begin{theorem}
    \label{thm:neup}
    Let $\blprofit$ be as defined in~$\blprofit$~\eqref{eqn:blprofitone}.
    Let $M$ be any mechanism satisfying EFB and IRB properties, and let $s$ be a Nash equilibrium of this mechanism.
    Then $\profit(M, s) \leq \blprofit - (\cost_2 - \cost_1)$.
\end{theorem}


\subsection{Proofs of Theorems~\ref{thm:dsic} and~\ref{thm:neup}}
\label{sec:hardnessproofs}

The following lemma is the key technical ingredient in proving both results above.
It states that when other agents are not collecting any data, then, regardless of their submission functions, the best response for an agent is to also not collect any data.
Note that even when an agent has not collected any data, an untruthful submission functions may actually report some data.


\begin{lemma}
    \label{lem:otherscollectzero}
        Let $M$ be any mechanism and let
        $\stratmi = \{(0, \subfuncj)\}_{j\neq i}$ be a strategy profile for all agents except $i$, where
        $\{\subfuncj\}_{j\neq i}$ are arbitrary submission functions.
        If $\strati = (\nni, \subfunci)$ is the best response of agent $i$ to $\stratmi$, then $\nni=0$.
\end{lemma}

We will prove Lemma~\ref{lem:otherscollectzero} in~\S\ref{sec:otherscollectzero}.
We first prove Theorem~\ref{thm:dsic} and Theorem~\ref{thm:neup}
%Lemma~\ref{lem:nic-under-single-collector}
using Lemma~\ref{lem:otherscollectzero}.

\begin{proof}[Proof of Theorem~\ref{thm:dsic}]
    Let $\stratmi' = \{(0, \subfuncj')\}_{j\neq i}$ be a strategy profile for all agents except $i$, where $\cbr{\subfuncj}_{j\neq i}$ are arbitrary submission functions. Since $s$ is a dominant strategy profile we know that $\forall \strati'$
    \begin{align*}
        \utilci\rbr{M, \rbr{\strati, \stratmi}}
        \geq 
        \utilci\rbr{M,\rbr{\strati', \stratmi}}
    \end{align*}
    i.e. $\strati$ is the best response to $\stratmi$. By Lemma~\ref{lem:otherscollectzero} we conclude that $\nni=0$.
\end{proof}

% \begin{proof}[Proof of Lemma~\ref{lem:nic-under-single-collector}]
%     Since $\strat$ is a NE under $\mech$, we have that $\forall \strati'$
%     \begin{align*}
%         \utilci\rbr{M, \rbr{\strati, \stratmi}}
%         \geq 
%         \utilci\rbr{M,\rbr{\strati', \stratmi}}
%     \end{align*}
%     i.e. $\strati$ is the best response to $\stratmi$. 
%     By hypothesis we also know that $\stratmi=\cbr{(0,\subfuncj)}_{j\neq i}$. Therefore, we can apply Lemma~\ref{lem:otherscollectzero} to conclude $\nni=0$.
% \end{proof}
We now present and prove a short claim that will be used in the proof of Theorem~\ref{thm:neup} that follows.

\begin{claim}
    \label{lem:nic-under-single-collector}
    If $\strat=\cbr{(\nni, \subfunci)}_{i\in\contributors}$ is a NE under $\mech$ and $\forall j\neq i$, $\nnj=0$, then $\nni=0$.
\end{claim}


\begin{proof}
Since $\strat$ is a NE under $\mech$, $\forall \strati'$ we have that 
% \begin{align*}%
    $\utilci\rbr{M, \rbr{\strati, \stratmi}}
    \geq 
    \utilci\rbr{M,\rbr{\strati', \stratmi}}$,
% \end{align*}
i.e. $\strati$ is the best response to $\stratmi$. 
By hypothesis we also know that $\stratmi=\cbr{(0,\subfuncj)}_{j\neq i}$. Therefore, we can apply Lemma~\ref{lem:otherscollectzero} to conclude $\nni=0$. 
\end{proof}

\begin{proof}[Proof of Theorem~\ref{thm:neup}]
%We begin with a short claim which demonstrates why we cannot ask a single agent to collect all of the data.
    % 
    % \emph{Claim: }
    %     If $\strat=\cbr{(\nni, \subfunci)}_{i\in\contributors}$ is a NE under $\mech$ and $\forall j\neq i$, $\nnj=0$, then $\nni=0$.
        
    % \emph{Proof. } Since $\strat$ is a NE under $\mech$, we have that $\forall \strati'$
    % \begin{align*}
    %     \utilci\rbr{M, \rbr{\strati, \stratmi}}
    %     \geq 
    %     \utilci\rbr{M,\rbr{\strati', \stratmi}}
    % \end{align*}
    % i.e. $\strati$ is the best response to $\stratmi$. 
    % By hypothesis we also know that $\stratmi=\cbr{(0,\subfuncj)}_{j\neq i}$. Therefore, we can apply Lemma~\ref{lem:otherscollectzero} to conclude $\nni=0$.
    Claim~\ref{lem:nic-under-single-collector}
    tells us that under any Nash equilibrium $s=\cbr{(\nni, \subfunci)}_{i\in\contributors}$, at least two contributors must collect data. Assuming that $\cost_1 \leq \cost_2 \leq \dots \leq \cost_{|\buyers|}$, we have that the cheapest way for at least two contributors to collect a total of $\numdata            \defeq \sum_{i} \nni$ points is to have agent 1 collect $\numdata-1$ points and agent 2 collect 1 point. Combining this with the fact that 
    $\OPTREVmech\rbr{ \numdata}$ is the best revenue achievable when selling $\numdata$ data points under IRB and EFB, we get
    \begin{align*}
        &~\profit(M, s)  \leq \;  
        \OPTREVmech\rbr{ 
            \numdata 
        } 
        - \sum_{i\in\contributors} \cost_i N_i \leq \;  
        \OPTREVmech\rbr{ 
            \numdata 
        } 
        -  \cost_1 \rbr{\numdata -1} -  \cost_2 
        \\ 
        = &  \; \OPTREVmech
        \rbr{ 
            \numdata 
        } 
        -  \cost_1 \rbr{\numdata} -  (\cost_2 -\cost_1)   \leq  \; \max_N \cbr{ \OPTREVmech\left(N\right) - \costone N} -  (\cost_2 -\cost_1) 
        \\ 
        = &  \; \blprofit - (\cost_2 - \cost_1) .
    \end{align*}
    % Here, the first step is because $\OPTREV\rbr{ \sum_{i} n_i}$ is the best revenue achievable when selling $\sum_{i} n_i$ data points. The second step is by the fact that at least two contributors are collecting data, and that $\cost_1 \leq \cost_2 \leq \dots \leq \cost_{|\buyers|}$. 
    The last step follows from equation (\ref{eqn:blprofittwo}), the definition of $\blprofit$. 
\end{proof}



% \begin{proof}[Proof of Theorem~\ref{thm:neup}]
%     foobar
% \end{proof}

\subsection{Proof of Lemma~\ref{lem:otherscollectzero}}
\label{sec:otherscollectzero}

% \kkcomment{Begin the proof by explaining why this should be  intuitively true.}

At a high level, Lemma~\ref{lem:otherscollectzero} claims that when the other agents do not collect data, it is best for agent $i$ to not collect data. This is because the mechanism has no knowledge of the true value of $\mu\in\RR$ and thus cannot penalize agent $i$ for submitting fabricated data since the others to do not provide data to validate against. Therefore, agent $i$ has the ability to forgo the cost of data collection without decreasing the payment they receive. Hence, agents $i$'s best response involves collecting no data. 

\begin{proof}
    Suppose for a contradiction that $\nni>0$. Define the strategy $\strat_{i,\mu'}=\rbr{\nni,\subfunc_{i,\mu'}}$
    where $\subfunc_{i,\mu'}$ is the submission function which disregards $\initdatai$ and instead samples and submits $\nni$ points from $N(\mu',\sigma^2)$ using $\subfunci$, i.e. $\subfunc_{i,\mu'}(\cdot):=\subfunci(Z_i),~Z_i\sim N(\mu',\sigma^2)^{\nni}$.  
    Applying the definition of contributor $i$'s utility we see that
    \begin{align*}
        \inf_{\mu\in\RR}
        \EEV{
            \initdatai\sim N(\mu,\sigma^2)^{\nni}
        }
        \sbr{
            \pay_i\rbr{
                \subfunci(\initdatai), \rbr{\subfunc_j(\varnothing)}_{j\neq i}
            }
        }
        -\costi\nni
        &\geq 
        \inf_{\mu\in\RR}
        \EEV{
            Z_i\sim N(\mu',\sigma^2)^{\nni}
        }
        \sbr{
            \pay_i\rbr{
                \subfunci(Z_i), \rbr{\subfunc_j(\varnothing)}_{j\neq i}
            }
        }
        -\costi\nni
        \\
        \Rightarrow
        \inf_{\mu\in\RR}
        \EEV{
            \initdatai\sim N(\mu,\sigma^2)^{\nni}
        }
        \sbr{
            \pay_i\rbr{
                \subfunci(\initdatai), \rbr{\subfunc_j(\varnothing)}_{j\neq i}
            }
        }
        &\geq 
        \inf_{\mu\in\RR}
        \EEV{
            Z_i\sim N(\mu',\sigma^2)^{\nni}
        }
        \sbr{
            \pay_i\rbr{
                \subfunci(Z_i), \rbr{\subfunc_j(\varnothing)}_{j\neq i}
            }
        } .
    \end{align*}
    For clarity we omit writing the dependence on $M$ and $\strat_{-i}$ under the expectation as it is not relevant to the proof. Now notice that since all of the other agents are not collecting data we have
    \begin{align*}
        \inf_{\mu\in\RR}
        \EEV{
            Z_i\sim N(\mu',\sigma^2)^{\nni}
        }
        \sbr{
            \pay_i\rbr{
                \subfunci(Z_i), \rbr{\subfunc_j(\varnothing)}_{j\neq i}
            }
        }
        &=
        \EEV{
            Z_i\sim N(\mu',\sigma^2)^{\nni}
        }
        \sbr{
            \pay_i\rbr{
                \subfunci(Z_i), \rbr{\subfunc_j(\varnothing)}_{j\neq i}
            }
        }
        \\
        &\geq
        \inf_{\mu'\in\RR}
        \EEV{
            Z_i\sim N(\mu',\sigma^2)^{\nni}
        }
        \sbr{
            \pay_i\rbr{
                \subfunci(Z_i), \rbr{\subfunc_j(\varnothing)}_{j\neq i}
            }
        }
        \\
        &=
        \inf_{\mu\in\RR}
        \EEV{
            \initdatai\sim N(\mu,\sigma^2)^{\nni}
        }
        \sbr{
            \pay_i\rbr{
                \subfunci(\initdatai), \rbr{\subfunc_j(\varnothing)}_{j\neq i}
            }
        } .
    \end{align*}
    Because we have both matching upper and lower bounds, we conclude that $\forall \mu'\in\RR$
    \begin{align*}
        \inf_{\mu\in\RR}
        \EEV{
            \initdatai\sim N(\mu,\sigma^2)^{\nni}
        }
        \sbr{
            \pay_i\rbr{
                \subfunci(\initdatai), \rbr{\subfunc_j(\varnothing)}_{j\neq i}
            }
        }
        = 
        \EEV{
            Z_i\sim N(\mu',\sigma^2)^{\nni}
        }
        \sbr{
            \pay_i\rbr{
                \subfunci(Z_i), \rbr{\subfunc_j(\varnothing)}_{j\neq i}
            }
        } .
    \end{align*}
    Intuitively this says that the expected payment an agent receives is independent of which normal distribution they submit data from. But this means that agent $i$ can avoid the cost of data collection by submitting fabricated data with no reduction in payment. More formally, for any $\mu'\in\RR$, consider the strategy $\Tilde{\strat}_i:=
    (0,\subfunc_{i,\mu'})$ . 
    % (0,\Tilde{\subfunc}_{i,\mu'})$ 
    % where $\Tilde{\subfunc}_{i,\mu'}(\cdot):=Z_i\sim N(\mu',\sigma^2)^{\datanum_i^\star}$. 
    Under this strategy agent $i$'s utility is 
    \begin{align*}
        \utilci\rbr{M, \rbr{\Tilde{\strat}_i, \strat_{-i}}}
        &=
        \inf_{\mu\in\RR}
        \EEV{
            Z_i\sim N(\mu',\sigma^2)^{\nni}
        }
        \sbr{
            \pay_i\rbr{
                \subfunci(Z_i), \rbr{\subfunc_j(\varnothing)}_{j\neq i}
            }
        }
        \\
        &=
        \inf_{\mu\in\RR}
        \EEV{
            \initdatai\sim N(\mu,\sigma^2)^{\nni}
        }
        \sbr{
            \pay_i\rbr{
                \subfunci(\initdatai), \rbr{\subfunc_j(\varnothing)}_{j\neq i}
            }
        }
        \\
        &>
        \inf_{\mu\in\RR}
        \EEV{
            \initdatai\sim N(\mu,\sigma^2)^{\nni}
        }
        \sbr{
            \pay_i\rbr{
                \subfunci(\initdatai), \rbr{\subfunc_j(\varnothing)}_{j\neq i}
            }
        }
        -\costi\nni
        % \\
        % &
        \,=
        \utilci\rbr{M, \rbr{\strat_i, \strat_{-i}}} .
    \end{align*}
    But this contradicts that $\strati$ is the best response to $\strat_{-i}$. Therefore, $\nni=0$.
\end{proof}

% \kkcomment{Some comments: \\
% 1) I think it is easy to do the proof without the revelation principle. If so, lets do it that way as not all readers may know what it is. \\
% 2) Let us not use $\stratopt$, $\nnopt$ notation. Just use $\strati, \nni$ etc.}

% In section~\ref{sec:mechanism}, we provided a mechanism $M$ and proved that it was NIC, meaning that it is best for each agent to follow the recommended strategy $\strat^\star$ assuming the other agents are doing so. A priori we may ask if we can design a mechanism with a stronger guarantee, where following $\strat^\star$ is best for each agent \emph{regardless of the strategies employed by the other agents}. We now formally define this type of guarantee in Definition~\ref{def:dsic} and in Theorem~\ref{thm: DSIC} show that there does not exist a non-trivial mechanism satisfying it.


% we provide an incentive-compatible mechanism and prove that no dominant strategy incentive-compatible (DSIC) solution exists. 


% \begin{definition}
%     \label{def:dsic}
%     (\emph{Dominant strategy profile}). A strategy profile $\strat^{\star}$ is a dominant strategy profile under a mechanism $M$ if $\forall i\in\contributors$ and $\forall \strat_i,\strat_{-i}\ac{\in?}$ 
%     \begin{align*}
%          \utilci(\mech, \strat_i^{\star}, \strat_{-i}) \geq  \utilci(\mech, \strat_i, \strat_{-i}).
%     \end{align*} 
%     If there exists a dominant strategy profile under $M$, $M$ is said to be \emph{Dominant Strategy Incentive Compatible (DSIC)}.
%     % A strategy $\strat_i^{\star}$ is dominant if it yields the highest utility for a contributor, no matter what strategies others use.  For any contributor $i\in\contributors$, and any $\strat_i, \; \strat_{-i}$, we have 
%     % \begin{equation*}
%     %      \utilci(\mech, \strat_i^{\star}, \strat_{-i}) \geq  \utilci(\mech, \strat_i, \strat_{-i}).
%     % \end{equation*}
% \end{definition}

% \begin{definition}
%     (DSIC). A mechanism is \emph{Dominant Strategy Incentive Compatible (DSIC)}, if there exists a dominant strategy $\strat_i^{\star}$ for each contributor, regardless of the strategies chosen by others.
% \end{definition}
% The following theorem shows that whenever we have a mechanism with a dominant strategy profile, under that profile, none of the agents collect data.

% \begin{theorem} \label{thm: DSIC}
%     There is no mechanism $M$ under which $\strat^\star:=\{(\nni^\star, \subfunci^\star)\}_{i\in\contributors}$
%     is a dominant strategy profile,
%     where $\nni^\star>0$ for some $i\in\contributors$.
%     % There does not exist a DSIC mechanism for contributors that incentivize truthful submission. 
%     % %(2) Individually rationality for both buyers and sellers.
%     % \ac{as written isn't this wrong? There are DSIC mechanisms but they are bad ones?}
% \end{theorem}





% \begin{proof}
%     Suppose for a contradiction that there does exist such a mechanism $M$ and a corresponding strategy profile $\strat^\star$. Without loss of generality we can assume by the revelation principle that $\strat^\star=\{(\nni^\star, \identity)\}_{i\in\contributors}$. Define the strategy $\strat_{i,\mu'}=\rbr{\datanum_i^{\star},\subfunc_{i,\mu'}}$
%     where $\subfunc_{i,\mu'}$ is the submission function which disregards $\initdatai$ and instead samples and submits $\nni^\star$ points from $N(\mu',\sigma^2)$, i.e. $\subfunc_{i,\mu'}(\cdot):=Z_i\sim N(\mu',\sigma^2)^{\datanum_i^\star}$. 
%     Also let $s_{-i}=\cbr{(0,\subfunc_j)}_{j\neq i}$ be a strategy profile where the other agents collect 0 data.
%     % \ac{forgot to define that $s_{-i}$ has others submitting 0}
%     Since $\strat^\star$ is a dominant strategy profile we have that $\forall\mu'\in\RR$
%     \begin{align}
%         \utilci\rbr{M, \rbr{\strat_i^{\star}, \strat_{-i}}}
%         \geq 
%         \utilci\rbr{M,\rbr{\strat_{i,\mu'}, \strat_{-i}}} .
%     \end{align}
%     Applying the definition of contributor $i$'s utility we see that
%     \begin{align*}
%         \inf_{\mu\in\RR}
%         \EEV{
%             \initdatai\sim N(\mu,\sigma^2)^{\nni^\star}
%         }
%         \sbr{
%             \pay_i\rbr{
%                 \initdatai, \rbr{\subfunc_j(\varnothing)}_{j\neq i}
%             }
%         }
%         -\costi\nni^\star
%         \geq 
%         \inf_{\mu\in\RR}
%         \EEV{
%             Z_i\sim N(\mu',\sigma^2)^{\nni^\star}
%         }
%         \sbr{
%             \pay_i\rbr{
%                 Z_i, \rbr{\subfunc_j(\varnothing)}_{j\neq i}
%             }
%         }
%         -\costi\nni^\star .
%     \end{align*}
%     For clarity we omit writing the dependence on $M$ and $\strat_{-i}$ under the expectation as it is not relevant to the proof. Now notice that since all of the other agents are not collecting data we have
%     \begin{align*}
%         \inf_{\mu\in\RR}
%         \EEV{
%             Z_i\sim N(\mu',\sigma^2)^{\nni^\star}
%         }
%         \sbr{
%             \pay_i\rbr{
%                 Z_i, \rbr{\subfunc_j(\varnothing)}_{j\neq i}
%             }
%         }
%         -\costi\nni^\star 
%         % \\
%         &=
%         \EEV{
%             Z_i\sim N(\mu',\sigma^2)^{\nni^\star}
%         }
%         \sbr{
%             \pay_i\rbr{
%                 Z_i, \rbr{\subfunc_j(\varnothing)}_{j\neq i}
%             }
%         }
%         -\costi\nni^\star 
%         \\
%         &\geq
%         \inf_{\mu'\in\RR}
%         \EEV{
%             Z_i\sim N(\mu',\sigma^2)^{\nni^\star}
%         }
%         \sbr{
%             \pay_i\rbr{
%                 Z_i, \rbr{\subfunc_j(\varnothing)}_{j\neq i}
%             }
%         }
%         -\costi\nni^\star 
%         \\
%         &=
%         \inf_{\mu\in\RR}
%         \EEV{
%             \initdatai\sim N(\mu,\sigma^2)^{\nni^\star}
%         }
%         \sbr{
%             \pay_i\rbr{
%                 \initdatai, \rbr{\subfunc_j(\varnothing)}_{j\neq i}
%             }
%         }
%         -\costi\nni^\star .
%         % \numberthis
%         % \label{eq:nodsic-lb}
%     \end{align*}
%     Because we have both matching upper and lower bounds, ignoring the $-\costi\nni^\star$ term, we conclude that $\forall \mu'\in\RR$
%     \begin{align*}
%         \inf_{\mu\in\RR}
%         \EEV{
%             \initdatai\sim N(\mu,\sigma^2)^{\nni^\star}
%         }
%         \sbr{
%             \pay_i\rbr{
%                 \initdatai, \rbr{\subfunc_j(\varnothing)}_{j\neq i}
%             }
%         }
%         = 
%         \EEV{
%             Z_i\sim N(\mu',\sigma^2)^{\nni^\star}
%         }
%         \sbr{
%             \pay_i\rbr{
%                 Z_i, \rbr{\subfunc_j(\varnothing)}_{j\neq i}
%             }
%         } .
%     \end{align*}
%     Intuitively this says that the expected payment an agent receives is independent of which normal distribution they submit data from. But this means that agent $i$ can avoid the cost of data collection by submitting fabricated data with no reduction in payment. More formally, for any $\mu'\in\RR$, consider the strategy $\Tilde{\strat}_i:=
%     (0,\subfunc_{i,\mu'})$ . 
%     % (0,\Tilde{\subfunc}_{i,\mu'})$ 
%     % where $\Tilde{\subfunc}_{i,\mu'}(\cdot):=Z_i\sim N(\mu',\sigma^2)^{\datanum_i^\star}$. 
%     Under this strategy agent $i$'s utility is 
%     \begin{align*}
%         \utilci\rbr{M, \rbr{\Tilde{\strat}_i, \strat_{-i}}}
%         &=
%         \inf_{\mu\in\RR}
%         \EEV{
%             Z_i\sim N(\mu',\sigma^2)^{\nni^\star}
%         }
%         \sbr{
%             \pay_i\rbr{
%                 Z_i, \rbr{\subfunc_j(\varnothing)}_{j\neq i}
%             }
%         }
%         \\
%         &=
%         \inf_{\mu\in\RR}
%         \EEV{
%             \initdatai\sim N(\mu,\sigma^2)^{\nni^\star}
%         }
%         \sbr{
%             \pay_i\rbr{
%                 \initdatai, \rbr{\subfunc_j(\varnothing)}_{j\neq i}
%             }
%         }
%         \\
%         &>
%         \inf_{\mu\in\RR}
%         \EEV{
%             \initdatai\sim N(\mu,\sigma^2)^{\nni^\star}
%         }
%         \sbr{
%             \pay_i\rbr{
%                 \initdatai, \rbr{\subfunc_j(\varnothing)}_{j\neq i}
%             }
%         }
%         -\costi\nni^\star 
%         \\
%         &=
%         \utilci\rbr{M, \rbr{\strat_i^{\star}, \strat_{-i}}} .
%     \end{align*}
%     But this contradicts that $\strat^\star$ is a dominant strategy profile. Thus we conclude there does not exist such a DSIC mechanism.
% \end{proof}

% \begin{proof}
    
% We prove this by contradiction. Suppose there exists a DSIC mechanism that incentivizes truthful submission. For each contributor, we assume the dominant strategy under such mechanism is $\strat_i^{\star} = \rbr{\datanum_i^{\star}, \mathbf{I}}$. 
% By definition of DSIC, for any $\strat_{-i} $ and  $\strat_{i} $, we have \[ \utilci(M,\strat_i^{\star}, \strat_{-i}) \geq \utilci(M,\strat_i, \strat_{-i}).   \]


% We begin by noting that at least one of the $\datanum_i^{\star}$ is positive. If $\datanum_i^{\star}= 0$ for all contributors $i \in \contributors$, then the dominant strategy is meaningless. Therefore, we assume that $\datanum_i^{\star}$ is positive for some contributor. 


% We consider the scenario where $\mu$ is the true mean. For contributors $j \neq i$, we fix their strategy $\strat_{-i}$:  $\strat_j = (\datanum_j =0,f_j = \mathbf{I})$, i.e., they do not collect any data points. 


% Contributor $i$'s dominant strategy $\strat_i^{\star}$ involves collecting $\datanum_i^{\star}$ samples from a distribution $N(\mu, \sigma^2)$ and submitting this data truthfully. However, consider an alternative strategy $\strat_i$, where contributor $i$ collects $\datanum_i^{\star}$ i.i.d. data from $N(\mu, \sigma^2)$ but instead submits a fake dataset $\subdata_i$ containing $\datanum_i^{\star}$ i.i.d. points from another normal distribution  $N(\mu',\sigma^2)$. Formally, this fake strategy is denoted by:
% \[
% \strat_i = (\datanum_i^{\star}, f_i), \; \text{where} \; \subdata_i = f_i(X_i)\sim N(\mu', \sigma^2), \; \text{and} \; f_i(X_i) \; \bot \; X_i .
% \]



% In this case, contributor $i$ fabricates data independently from their original data collection process, effectively manipulating the reported mean to $\mu’$. We then analyze the effect on the expected payment under this fake strategy. We define a function $ g: \mathbb{R} \to \mathbb{R}_{+} $. The expected payment for contributor $ i $ under strategy $\strat = (\strat_i,\strat_{-i})$ is denoted as $g(\mu')$:
% \[
% \mathbb{E}\left[ \pay_i(M, \strat) \right] = g( \mu'),
% \]  
% where the expectation is taken over the randomness in the mechanism, and submitted dataset $\subdata$. By definition of DSIC, we have 
% \begin{align*}
%     & \; \utilci(M,\strat_i^{\star}, \strat_{-i})  = \inf_\mu \mathbb{E}\left[ \pay_i(M, \strat_i^{\star}, \strat_{-i})  \right] -\cost_i \datanum_i^{\star} = \inf_\mu g(\mu) -\cost_i \datanum_i^{\star} \\ \geq  & \; \utilci(M,\strat_i, \strat_{-i}) = \inf_\mu  g( \mu') -\cost_i \datanum_i^{\star}= g( \mu')-\cost_i \datanum_i^{\star}
% \end{align*}

% The above inequality implies that $\inf_\mu g(\mu)  \geq g( \mu') $ for any $\mu'$. By definition of infimum, $\forall \mu',\; g( \mu') \geq \inf_\mu g(\mu)$, and it follows that
% \[ g( \mu') \geq \inf_\mu g(\mu) \geq g( \mu'), \]
% therefore further implies that $g( \mu)$ is a constant function on $\mathbb{R}$.

% Consider another strategy $\strat'_i$ for contributor $i$: collect nothing but directly fabricates $\datanum_i^{\star}$ i.i.d. data points from distribution $N(\mu',\sigma^2)$. Under this strategy, the expected payment remains the same as $\strat_i$: $\mathbb{E}\left[ \pay_i(M,\strat'_i, \strat_{-i}) \right] = g( \mu')$. We then prove that following strategy $\strat'_i$ will lead to a higher  utility for contributor $i$ than following the dominant strategy $\strat_i^{\star}$: 

% \begin{align*}
%     \utilci(M,\strat'_i, \strat_{-i}) &  = \underset{\mu}{\inf}\  g({\mu}') = \underset{\mu}{\inf}\  g(\mu)=   \underset{\mu} {\inf} \; \mathbb{E}\left[ \pay_i(M, \strat_i^{\star}, \strat_{-i}) \right] \\ &  > \underset{\mu} {\inf} \; \mathbb{E}\left[ \pay_i(M, \strat_i^{\star}, \strat_{-i}) \right] - \cost_i\datanum_i^{\star} =  \utilci(M,\strat_i^{\star}, \strat_{-i})   .
% \end{align*}

% The inequality above indicates that if contributor $i$ fabricates data without actually collecting any, she attains a higher utility than by following the dominant strategy $	{\strat}_i^{\star}$. This contradicts the fact that this DSIC mechanism guarantees truthful submission.


% \end{proof}

% \ac{make comment about how proofs are technically very similar}

% \ac{add $\strat^\star\in -$ and $M\in -$ to the theorem hypothesis}
% \begin{theorem} \label{thm:nash_atleasttwo}
%     If $\strat^\star$ is a NE under a mechanism $\mech$ and $\forall j\neq i$, $\datanum_{j}^{\star}=0$, then $\datanum_{i}^{\star}=0$.
% \end{theorem}
% \begin{proof}
%     Suppose for a contradiction that there exists a strategy profile which is a NE under some mechanism $M$ and that $\forall j\neq i$, $\datanum_{j}^{\star}=0$, but $\datanum_{i}^{\star}>0$. Without loss of generality we can assume by the revelation principle that $\strat^\star=\{(\nni^\star, \identity)\}_{i\in\contributors}$. Define the strategy $\strat_{i,\mu'}=\rbr{\datanum_i^{\star},\subfunc_{i,\mu'}}$
%     where $\subfunc_{i,\mu'}$ is the submission function which disregards $\initdatai$ and instead samples and submits $\nni^\star$ points from $N(\mu',\sigma^2)$, i.e. $\subfunc_{i,\mu'}(\cdot):=Z_i\sim N(\mu',\sigma^2)^{\datanum_i^\star}$. 
%     % Also let $s_{-i}=\cbr{(0,\subfunc_j)}_{j\neq i}$ be a strategy profile where the other agents collect 0 data.
%     Since $\strat^\star$ is a NE we have that $\forall \mu'\in\RR$
%     \begin{align}
%         \utilci\rbr{M, \rbr{\strat_i^{\star}, \strat_{-i}^\star}}
%         \geq 
%         \utilci\rbr{M,\rbr{\strat_{i,\mu'}, \strat_{-i}^\star}} .
%     \end{align}
%     Applying the definition of contributor $i$'s utility we see that
%     \begin{align*}
%         \inf_{\mu\in\RR}
%         \EEV{
%             \initdatai\sim N(\mu,\sigma^2)^{\nni^\star}
%         }
%         \sbr{
%             \pay_i\rbr{
%                 \initdatai, 
%                 % \rbr{\subfunc_j(\varnothing)}_{j\neq i}
%                 \cbr{\varnothing}_{j\neq i}
%             }
%         }
%         -\costi\nni^\star
%         \geq 
%         \inf_{\mu\in\RR}
%         \EEV{
%             Z_i\sim N(\mu',\sigma^2)^{\nni^\star}
%         }
%         \sbr{
%             \pay_i\rbr{
%                 Z_i, 
%                 % \rbr{\subfunc_j(\varnothing)}_{j\neq i}
%                 \cbr{\varnothing}_{j\neq i}
%             }
%         }
%         -\costi\nni^\star .
%     \end{align*}
%     For clarity we omit writing the dependence on $M$ and $\strat_{-i}$ under the expectation as it is not relevant to the proof. Now notice that since all of the other agents are not collecting data we have
%     \begin{align*}
%         \inf_{\mu\in\RR}
%         \EEV{
%             Z_i\sim N(\mu',\sigma^2)^{\nni^\star}
%         }
%         \sbr{
%             \pay_i\rbr{
%                 Z_i, 
%                 \cbr{\varnothing}_{j\neq i}
%             }
%         }
%         -\costi\nni^\star 
%         % \\
%         &=
%         \EEV{
%             Z_i\sim N(\mu',\sigma^2)^{\nni^\star}
%         }
%         \sbr{
%             \pay_i\rbr{
%                 Z_i, 
%                 \cbr{\varnothing}_{j\neq i}
%             }
%         }
%         -\costi\nni^\star 
%         \\
%         &\geq
%         \inf_{\mu'\in\RR}
%         \EEV{
%             Z_i\sim N(\mu',\sigma^2)^{\nni^\star}
%         }
%         \sbr{
%             \pay_i\rbr{
%                 Z_i, 
%                 \cbr{\varnothing}_{j\neq i}
%             }
%         }
%         -\costi\nni^\star 
%         \\
%         &=
%         \inf_{\mu\in\RR}
%         \EEV{
%             \initdatai\sim N(\mu,\sigma^2)^{\nni^\star}
%         }
%         \sbr{
%             \pay_i\rbr{
%                 \initdatai, 
%                 \cbr{\varnothing}_{j\neq i}
%             }
%         }
%         -\costi\nni^\star .
%         % \numberthis
%         % \label{eq:nodsic-lb}
%     \end{align*}
%     Because we have both matching upper and lower bounds, ignoring the $-\costi\nni^\star$ term, we conclude that $\forall \mu'\in\RR$
%     \begin{align*}
%         \inf_{\mu\in\RR}
%         \EEV{
%             \initdatai\sim N(\mu,\sigma^2)^{\nni^\star}
%         }
%         \sbr{
%             \pay_i\rbr{
%                 \initdatai, 
%                 \cbr{\varnothing}_{j\neq i}
%             }
%         }
%         = 
%         \EEV{
%             Z_i\sim N(\mu',\sigma^2)^{\nni^\star}
%         }
%         \sbr{
%             \pay_i\rbr{
%                 Z_i,
%                 \cbr{\varnothing}_{j\neq i}
%             }
%         } .
%     \end{align*}
%     Intuitively this says that the expected payment an agent receives is independent of which normal distribution they submit data from. But this means that agent $i$ can avoid the cost of data collection by submitting fabricated data with no reduction in payment. More formally, for any $\mu'\in\RR$, consider the strategy $\Tilde{\strat}_i:=
%     (0,\subfunc_{i,\mu'})$ . 
%     Under this strategy agent $i$'s utility is 
%     \begin{align*}
%         \utilci\rbr{M, \rbr{\Tilde{\strat}_i, \strat_{-i}^\star}}
%         &=
%         \inf_{\mu\in\RR}
%         \EEV{
%             Z_i\sim N(\mu',\sigma^2)^{\nni^\star}
%         }
%         \sbr{
%             \pay_i\rbr{
%                 Z_i, 
%                 \cbr{\varnothing}_{j\neq i}
%             }
%         }
%         \\
%         &=
%         \inf_{\mu\in\RR}
%         \EEV{
%             \initdatai\sim N(\mu,\sigma^2)^{\nni^\star}
%         }
%         \sbr{
%             \pay_i\rbr{
%                 \initdatai, 
%                 \cbr{\varnothing}_{j\neq i}
%             }
%         }
%         \\
%         &>
%         \inf_{\mu\in\RR}
%         \EEV{
%             \initdatai\sim N(\mu,\sigma^2)^{\nni^\star}
%         }
%         \sbr{
%             \pay_i\rbr{
%                 \initdatai, 
%                 \cbr{\varnothing}_{j\neq i}
%             }
%         }
%         -\costi\nni^\star 
%         \\
%         &=
%         \utilci\rbr{M, \rbr{\strat_i^{\star}, \strat_{-i}^\star}} .
%     \end{align*}
%     But this contradicts that $\strat^\star$ is a NE.
%     Thus we conclude that $\nni^\star=0$.
    
%     % a dominant strategy profile. Thus we conclude there does not exist such a DSIC mechanism.
% \end{proof}

% In Theorem~\ref{thm:upper_bound}, we demonstrate that at least two contributors are necessary to encourage truthful submissions. A single contributor’s data cannot be verified for authenticity. Additionally, we show that our mechanism achieves near-optimal profit for contributors compared to all mechanisms that ensure truthfulness.



% \begin{theorem} \label{thm:upper_bound}
%     \textbf{(Upper bound)}. For any mechanism that ensures \ac{precisely say what ``ensures'' means} truthful submission, the best expected profit achievable is no greater than
%     % \[
%     $
%     \mathrm{profit^{\star}}-\cost_1 + \cost_2 .
%     $ \ac{add ``see equation (--)'' for $\text{profit}^\star$ definition}
%     % \]  
%     \begin{equation*}
%         \ac{\text{insert expression for ``best expected profit achievable''} \geq \text{profit}^\star-\cost_1 + \cost_2 }
%     \end{equation*}
    
% \end{theorem}

% \begin{proof}
%     We first prove that guaranteeing truthfulness in the mechanism requires the participation of at least two contributors.
    
%     Assume, for the sake of contradiction, that only one contributor provides data to the broker. Then \ac{ by definition?} the optimal profit is \[ \mathrm{profit^{\star} } =  \max_\nn  \rbr{\mathrm{rev}^{\star} 
%     \rbr{ 
%     \buyers,\nn } - \cost_1 \nn  }. \]
    
    
%     However, by \ac{this no longer formally interfaces with Thm 5.2?} Theorem~\ref{thm: DSIC}, the contributor can fabricate data without actually collecting any real data points, thus violating the truthfulness requirement.

%     Therefore, to ensure truthfulness, at least two contributors must participate \ac{need to argue you can make this work with 2 or more agents}. In this case, to minimize the cost of data collection, contributor 2 collects one data point, while contributor 1 collects the remaining data points. Therefore, the maximum achievable profit in this setup is $(\mathrm{profit^{\star}}-\cost_1 + \cost_2)$. This result aligns with Theorem~\ref{thm:multi_thm}, confirming that among all mechanisms that ensure truthfulness, our mechanism achieves a near-optimal profit within an error of $\eps$.

    
% \end{proof}