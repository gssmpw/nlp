\section{Profit-optimal mechanisms with strategic contributors}

\label{sec:mechanism}
\newcommand{\insertalgotab}{\hspace{0.6cm}}

%\insertAlgoMain
\begin{algorithm}[H] 
    \caption{\label{alg:mechanism}}
    \begin{algorithmic}[1]
        \State
        {\bfseries Input:} Buyer valuations $\{\valj\}_{j\in\buyers}$, contributor costs $\{\costi\}_{i\in\contributors}$, algorithm for ordered item pricing $A$, approximation
            parameter $\eps$. %an $\eps$-revenue optimal algorithm $A$.
        \State
        \textbf{The broker: }
        \State \insertalgotab $\numdataA$, $\{\mmjA\}_{j\in\buyers}$, $\{\pricejA\}_{j\in\buyers} \leftarrow$ \text{Execute Algorithm~\ref{alg:oipalgo} with } $\{\valj\}_{j\in\buyers}$, $\{\costi\}_{i\in\contributors}$, $A$, $\eps$.   
        \State \insertalgotab Instruct contributors to collect $\{\nnopti\}_{i\in\contributors}$ where,
        \[
                \nnopt_1 = \numdataA-1, \quad \nnopt_2 = 1, \quad \nnopt_j = 0 \text{ for all } j\in \{3,\dots,|\contributors|\}.
        \]
      
        \State
        \textbf{Each contributor $i \in \contributors$:}
        \State
        \insertalgotab
         Chooses a strategy $\strati = (\colamnti, \subfunci)$. 
        \State 
        \insertalgotab
         Collects $\nni$ data points $\initdatai$, and submits $\subdatai = \subfunci(\initdatai)$ . 
        \State
        \textbf{The broker:}
        \State 
        \insertalgotab   Receive  $\{\subdatai\}_{i\in\contributors}$ from contributors.
        \State \insertalgotab Let   $Y \leftarrow \subdata_1 \cup \subdata_2$ be the combined dataset of the first two contributors.
       % \Comment{The combined dataset}
        \State
        \insertalgotab Allocate  $\mmoptj$ ($=\mmjA$) randomly chosen points from $Y$ to each buyer $j \in \buyers$.
        \State \insertalgotab Charge $\priceoptj$ from each buyer $j\in\buyers$ (see~\eqref{eq:buyer_pay}).

        \State
        \insertalgotab Pay $\payopti$ to each contributor $i \in \contributors$ (see~\eqref{eq:contrib_pay}).
   \end{algorithmic}   
\end{algorithm}

We have outlined our mechanism in Algorithm~\ref{alg:mechanism}.
We will first describe it procedurally, specify our choices
for $\mechopt = (\{\nnopti\}_{i\in\contributors}, \{\mmoptj\}_{j\in\buyers},
\{\priceoptj\}_{j\in\buyers}, \{\payopti\}_{i\in\contributors}\}$,
and then motivate our design choices.

Algorithm~\ref{alg:mechanism} takes as inputs the buyer valuations, contributor costs, an algorithm for ordered item pricing, and an approximation parameter.
First, we execute Algorithm~\ref{alg:oipalgo} 
to obtain $\numdataA$, $\{\mmjA\}_{j\in\buyers}$, $\{\pricejA\}_{j\in\buyers}$.
We then set $\{\nnopti\}_{i\in\contributors}, \{\mmoptj\}_{j\in\buyers}$ as shown below.
%We have
\begin{align*}
    \reqamnt_1^{\star} =\numdataA-1,\quad \reqamnt_2^{\star} =1 , \quad \reqamnti^{\star} = 0 \text{ for } i\geq 3.
    \hspace{0.5in}
    \mmoptj = \mmjA \text{ for } j\in\buyers
    \numberthis \label{eqn:reqamnt}
\end{align*}
That is, the broker instructs  contributor 1 to collect $\nnopt_1$ points, contributor 2 
to collect $\nnopt_2=1$ point, and the rest to collect none.
Then these two contributors (and  possibly others), collect $\nni$ points each to collect a dataset $\initdatai$, and submit $\subdatai = \subfunci(\initdatai)$.

Let $\subdata = \subdata_1 \cup \subdata_2$ be the union of the first two contributors' submissions (the broker ignores the others' submissions).
She then allocates $\mmoptj$ points to buyer $j$ and charges them prices $\priceoptj$ as shown below
in~\eqref{eq:buyer_pay}.
She then issues payments $\payopti$ to the contributors as shown below in~\eqref{eq:contrib_pay}.

%To state the prices and payments, 
%\emph{Prices and payments.}
For $i\in\{1,2\}$,
%recall that $\numdataA$ a prices returned from Algorithm~\ref{alg:oipalgo}.
let $Y_{-i}$ denote the data submitted by the agent other than $i$,
let $\nnopt_{-i} = \numdataA - \nnopti$ be the amount of data that the other agent should collect,
and let $d_i = \cost_i \cdot (\nnopti)^2 / \sigma^2$.
Denote $W = \sum_{j\in\buyers} \pricejA -\cost_1 \rbr{\numdataA-1} -\cost_2 $, where, recall $\{\pricejA\}_j$ are the prices computed in Algorithm~\ref{alg:oipalgo}.
%be the profit of contributors when following $\stratopt$. %\kkcomment{@Keran fill in. Also, is there an intutive way to think about $d_i$?} %\rcomment{since $d_i$ is revalent with $\cost_i$ and $\nnopti$, and maybe more the larger $\nnopti \& $$\cost_i$ is, the more responsibility for the contributor to build a truthful marketplace, therefore,  if the contributor fabricate data, she will get a greater penalty in the payment(since $d_i$ is large) }
As we will show, $W$ will be the expected profit of our mechanism.
%\frac{\cost_i \cdot (\nnopti)^2}{\sigma^2} $ 
%\kkcomment{Repeated quantiy in $\priceoptj$}
We can now state $\priceoptj$ as follows:
  \begin{align*} 
    \label{eq:buyer_pay}
    &\priceoptj\left(\{\subdatai\}_i \right)   
    % & 
    = 
      \sum_{i \in \{1,2\}} \mathbb{I}\left( \left| Y_i \right| = \reqamnt_i^{\star} \right) 
    \bigg( 
    \pricejA \frac{\reqamnt_i^{\star}}{\numdataA} + \frac{d_i}{|\buyers|} \frac{\sigma^2}{\reqamnt_{-i}^{\star}} + \frac{d_i}{|\buyers|} \frac{\sigma^2}{ \reqamnt_i^{\star}} 
    \bigg) 
    -\hspace{-0.05in} \sum_{i \in \{1,2\}} \frac{d_i}{|\buyers|} \bigg( \hat{\mu}(Y_i) - \hat{\mu}(Y_{-i}) \bigg)^2. \numberthis
    \end{align*}
    Next the payments $\payopti\left(\{\subdatai\}_{i}\right)=0$ for all $i\in\{3,\dots,|\contributors|\}$. For agents $i\in\{1, 2\}$ we have,
    \begin{align*} 
\label{eq:contrib_pay}
% & 
&\payopti \left(\{\subdatai\}_{i}\right) 
% \\
% & 
=   \mathbb{I}\left( | Y_i | = \reqamnt_i^{\star} \right) 
\bigg( 
\profi \frac{\reqamnt_i^{\star}}{\numdataA} + \cost_i \reqamnt_i^{\star} + d_i \frac{\sigma^2}{\reqamnt_{-i}^{\star}} + d_i \frac{\sigma^2}{ \reqamnt_i^{\star}} 
\bigg) 
% \\
% & 
- d_i \bigg( \hat{\mu}(Y_i) - \hat{\mu}(Y_{-i}) \bigg)^2  . \numberthis 
\end{align*}


\parahead{Design choices}
Here, the payments to buyers are designed so that the \emph{expected payment} matches the value $\pricejA$ obtained from Algorithm~\ref{alg:oipalgo}. Similarly, the payments to contributors ensure that cumulative profit is distributed in proportion to their contributions. Specifically, contributor $i$'s \emph{expected payment} is given by $\costi\nnopti + W \nnopti/\numdataA$.

However, the actual prices and payments are further adjusted based on the difference between the means of the datasets provided by the two contributors. If this difference is large, buyers pay less; if it is small, they pay more. These adjustments are then passed on to the two contributors.
This design allows us to achieve three requirements: ICC: each contributor should report truthfully when the other does so, to minimize this difference;
IRB: buyers pay less if there are significant discrepancies in contributors' submissions, which may indicate unreliable data;
BB: the market must remain feasible for every random realization of data, \ie with probability 1.

Finally, we set $\mmoptj=\mmjA$ since Algorithm~\ref{alg:oipalgo} is designed to maximize profit. However, if contributor 1 were to collect all the data, she could fabricate it. Fortunately, as we prove in Theorem~\ref{thm:multi_thm}, using just one data point from contributor 2 is sufficient to incentivize contributor 1 to follow the rules.


% The broker combines the dataset from only the first two contributors
% to produce $\subdata = \subdata_1\cup\subdata_2$.


% % In~\S\ref{sec:mechanism}, we design mechanisms that satisfy the six requirements outlined earlier.  
% % While collecting all data from one agent is not ICC, we show that using two agents suffices.  
% % Intuitively, the broker can verify truthfulness by comparing their reported data with each other.  
% % Thus, to maximize profit, the mechanism assigns data collection to the two cheapest agents.  
% % Moreover, the cheapest contributor collects almost all of the points. The second cheapest agent collects only a minimal amount (just one for normal mean estimation), as her primarily role is to help the broker verify the reliability of the cheapest contributor's submissions.

% % We employ the profit-optimal envy-free pricing algorithm from~\S\ref{sec:buyerside} to determine the total data collection amount, buyers' dataset sizes, and expected prices.  
% % Buyers' actual prices also include a term dependent on the difference
% % between the means of the two contributors' reported datasets, with this variation passed on to the contributors.
% % This design is motivated by three of our requirements:
% % ICC: each contributor should report truthfully when the other is doing so in order to minimize this difference.
% % IRB: buyers will pay less if there are significant discrepancies contributors' submissions, which may indicate unreliable data.
% % BB (with probability 1): the market must be feasible for every random realization of data.


% We outlined our mechanism in Algorithm~\ref{alg:mechanism}. The mechanism takes as input the population of buyers $(\buyers, \val)$, and an $\eps$-revenue optimal algorithm $A$, which helps determine profit-maximizing strategies. The broker selects the optimal number of data points $\datanum_i^{\star}$ for each contributor $i$, which maximizes the contributors' profit, defined by 
% \kkcomment{Why is this written this way? Have you guys looked at Section 2.2? We only need to ask the first contributor to collect almost all of the data, and the second contributor to collect 1 more data point.}
% \begin{align}
%     \reqamnt_1^{\star} =\numdataA-1,\; \reqamnt_2^{\star} =1 , \; \reqamnti^{\star} = 0 \text{ for } i\geq 3.
% \end{align}
% We denote the total profit of contributors as $\profi \defeq  \ExpectREV(\numdataA, \itemprice_{\numdataA}) -\cost_1 \rbr{\numdataA-1} -\cost_2 $.

% % $\mmjA_{j\in\buyers}$, $\pricejA_{j\in\buyers}$

% Each contributor $i$ selects their own strategy $\strat_i=(\datanum_i,f_i)$. They collect $\datanum_i$ data points $X_i$ and submit $\subdatai = f_i(X_i)$ to the broker. 

% \textbf{Broker's processing.} The broker aggregates the submitted data from all contributors to form a combined dataset $Y=\bigcup_{i\in\contributors}\subdatai$. For each buyer $j \in \buyers$, the broker's dataset size selection rule is $\mm(Y) = \mmj'$. The broker randomly samples $\mmj'$ data points from $Y$ for the buyer, and charges her a price determined by 
% % \ac{isn't the LHS a number and the RHS a random variable? is there a missing expectation because this doesn't make sense?}
%   \begin{align*} 
%     \label{eq:buyer_pay}
%     \priceoptj(\mech,\strat)   
%     % & 
%     = 
%     \;  \sum_{i \in \contributors} \mathbb{I}\left( \left| Y_i \right| = \reqamnt_i^{\star} \right) 
%     \bigg( 
%     \pricejA \frac{\reqamnt_i^{\star}}{\numdataA} + \frac{d_i}{|\buyers|} \frac{\sigma^2}{\reqamnt_{-i}^{\star}} + \frac{d_i}{|\buyers|} \frac{\sigma^2}{ \reqamnt_i^{\star}} 
%     \bigg) 
%     % \\ 
%     % &  
%     - \sum_{i \in \contributors} \frac{d_i}{|\buyers|} \bigg( \hat{\mu}(Y_i) - \hat{\mu}(Y_{-i}) \bigg)^2 , \numberthis 
% \end{align*}
% where $d_i = \frac{\cost_i (\nnopti)^2}{\sigma^2} $ is a constant. The broker then redistributes the total revenue $\sum_{j \in \buyers} \price_j$ among contributors via $\pay_{i \in \contributors}$: 



% \begin{align*} 
% \label{eq:contrib_pay}
% % & 
% \payopti (\mech,\strat) 
% % \\
% % & 
% =   \mathbb{I}\left( | Y_i | = \reqamnt_i^{\star} \right) 
% \bigg( 
% \profi \frac{\reqamnt_i^{\star}}{\numdataA} + \cost_i \reqamnt_i^{\star} + d_i \frac{\sigma^2}{\reqamnt_{-i}^{\star}} + d_i \frac{\sigma^2}{ \reqamnt_i^{\star}} 
% \bigg) 
% % \\
% % & 
% - d_i \bigg( \hat{\mu}(Y_i) - \hat{\mu}(Y_{-i}) \bigg)^2  . \numberthis 
% \end{align*}



% The expected price for each buyer under the recommended strategy is $
% \mathbb{E}  \left[ \price_j(M,\strat^{\star})  \right] = \pricejA
% $, which aligns with our result in ordered item pricing problem in Eq~\ref{eq:buyer_allocate}.

\vspace{0.06in}
\noindent
The following theorem summarizes the key properties of our mechanism. It satisfies all the required constraints while achieving a profit arbitrarily close to the upper bound in Theorem~\ref{thm:neup}, demonstrating that the mechanism is essentially unimprovable.  

\begin{restatable}{thm}{thmNIC} \label{thm:multi_thm}
    Let $A$ be an algorithm for ordered item pricing which takes in an approximation parameter $\eps$ and returns a $|\buyers|\bigO(\eps)$ approximation to the optimal revenue.
    Let $\mechopt$ denote the mechanism in Algorithm~\ref{alg:mechanism} when executed with $A$.
    Let $\stratopt = \{(\nnopti, \identity)\}_{i\in \contributors}$ denote the strategy profile where contributors are collecting the amounts specified in~\eqref{eqn:reqamnt} and reporting it truthfully.
    Then $\mechopt$ satisfies IRB, EFB, IRC, ICC, and BB.
    Moroever, its expected profit is
    \[
    \profit(\mechopt, \stratopt) \geq \blprofit - (\cost_2 - \cost_1) - |\buyers|\bigO(\eps),
    \]
    % Our Mechanism $M$ is Nash Incentive Compatible, and the strategy profile $\strat^{\star}$ satisfies: (1) $\strat^{\star}$ is Nash equilibrium. (2) The mechanism is individually rational at $\strat^{\star}$ for both buyers and sellers. (3) Budget balance. (3) Envy-free for the buyers (5) Under strategy $\strat^{\star}$, the expected profit of buyers approximates the optimal profit $ \mathrm{profit}^{\star}$ within an additive error $\eps$. 
\end{restatable}

The proof  is given in Appendix~\ref{app:mainproof}.
We have outlined the key ideas of our ICC proof in~\S\ref{sec:iccsketch}.

\parahead{Discussion}
We highlight two observations.
% We provide two observations before providing a proof sketch of the ICC argument which contains the most interesting ideas in Theorem~\ref{thm:multi_thm}.
\emph{1) Multiple minimum cost agents.}  
%\subparahead{1) Multiple minimum cost agents}  
If there are multiple agents with the same minimum cost, we can distribute the data collection evenly among them to achieve $\blprofit$.
This also reduces the variance of the comparison terms in~\eqref{eq:buyer_pay} and~\eqref{eq:contrib_pay}, which in turn lowers the variance in prices and payments.
This may be desirable in practical settings.  

\subparahead{2) Beyond mean estimation}  
We believe many of our insights extend to learning tasks beyond mean estimation. For instance, we expect that no nontrivial DSIC mechanism exists and that two contributors would still suffice to induce a profit-optimal Nash equilibrium. However, the requirement of only one sample from the second contributor is specific to normal mean estimation, where there is only a single learnable parameter. More generally, we expect the number of required samples from the second contributor to scale with the model's degrees of freedom, as their primary role is to help the broker verify the reliability of the first contributor's submissions.  

\subsection{Proof sketch of ICC in Theorem~\ref{thm:multi_thm}}
\label{sec:iccsketch}

% \kkcomment{Should we mention somewhere that we only need to consider the first two agents. Since the others are not receiving any payments, they won't really participate at all.}

To prove ICC we must show that $\stratopt$ is a Nash equilibrium of $\mechopt$. 
That is, we will show, $\forall i\in\contributors$, 
    $\forall\rbr{\nni,\subfunci}$,
    % \ac{unsure if $\subfuncspace$ has been previously defined?}
    \begin{equation*}
        \utilci\rbr{
            \mechopt, \rbr{
                (\nni,\subfunci),
                \strat_{-i}^{\star}
            }
        }
        \leq
        \utilci\rbr{
            \mechopt, \rbr{
                \strat_i^\star,
                \strat_{-i}^{\star}
            }
        } . 
    \end{equation*}

% \noindent
Note that because we are only asking the two agents with the lowest collection costs to collect data, we can focus on $\cbr{1,2}\in\contributors$ as the above inequality is immediately satisfied for all other agents. We now prove the inequality for these agents in three steps, each with a corresponding lemma.

% \noindent 
\textbf{Step 1.}
Let $(\nni,\subfunci)$ be given.
First we show that a contributor can improve her utility by switching their strategy from $(\nni,\subfunci)$ to $(\nni,f_a)$, where $f_a$ is a function that always submits exactly $\nnopti$ points while maintaining the same sample mean as $\subfunci$. This is true as our mechanism determines a contributors penalty based on the sample mean of their submitted data and whether or not they submitted exactly $\nnopti$ points. This is stated formally in the lemma below.

\textbf{(\text{Lemma}~\ref{lem:NIC-part1}):}
Fix any $(\nni,\subfunci)$. Let $f_a:\dataspace\rightarrow\RR^{\nnopti}$ be a function such that $\forall \initdatai\in\dataspace$, $\muhat\rbr{f_a(\initdatai)}=\muhat\rbr{\subfunci(\initdatai)}$. Then,
\begin{align*}
    \utilci \rbr{M, \rbr{(\nni,\subfunci),\strat_{-i}^{\star} }}
    \leq
    \utilci \rbr{M, \rbr{(\nni,f_a),\strat_{-i}^{\star} }} .
\end{align*}

\textbf{Step 2.} 
The second step is showing that a contributor can improve their utility by switching their strategy from $(\nni,f_a)$ to $(\nni,f_b)$, where $f_b$ is a function that always submits exactly $\nnopti$ points and whose sample mean agrees with the sample mean of the original data $\initdatai$.
We will show that this is a result of a standard fact from normal mean estimation that the sample mean is minimax optimal~\citep{lehmann2006theory}. This is stated formally in the lemma below.

\textbf{(\text{Lemma}~\ref{lem:NIC-part2}):}
Fix any $\nni$. Consider any $f_a:\dataspace\rightarrow\RR^{\nnopti}$ and let $f_b:\dataspace\rightarrow\RR^{\nnopti}$ be a function such that 
$\forall \initdatai\in\dataspace$, $\muhat\rbr{f_b(\initdatai)}=\muhat\rbr{\initdatai}$. Then,
\begin{align*}
    \utilci \rbr{M, \rbr{(\nni,f_a),\strat_{-i}^{\star} }}
    % \qquad
    \leq
    \utilci \rbr{M, \rbr{(\nni,f_b),\strat_{-i}^{\star} }} .
\end{align*}

\textbf{Step 3.}
The final step is showing a contributor can improve their utility by switching their strategy from $(\nni,f_b)$ to $(\nnopti,f_b)$. We will show that when a contributor is using $f_b$ their utility is a concave function in $\nni$ which is maximized at $(\nnopti,f_b)$. We then show the utility under $(\nnopti,f_b)$ is the same as under $(\nnopti,\identity)=\strati^\star$.

\textbf{(\text{Lemma}~\ref{lem:NIC-part3}):}
Fix any $\nni$. Let $f_b:\dataspace\rightarrow\RR^{\nnopti}$ be a function such that
$\forall \initdatai\in\dataspace$, $\muhat\rbr{f_b(\initdatai)}=\muhat\rbr{\initdatai}$.
Then,
\begin{align*}
    \utilci \rbr{M, \rbr{(\nni,f_b),\strat_{-i}^{\star} }}
    \leq
    \utilci \rbr{M, \rbr{\strat_{i}^{\star},\strat_{-i}^{\star} }} .
\end{align*}

These three steps now combine to prove ICC as follows. Fix any $(\nni,\subfunci)$,
    %Let $f_a:\dataspace\rightarrow\RR^{\nnopti}$ be a function such that $\forall \initdatai\in\dataspace$, $\muhat\rbr{f_a(\initdatai)}=\muhat\rbr{\subfunci(\initdatai)}$. Let $f_b:\dataspace\rightarrow\RR^{\nnopti}$ be a function such that $\forall \initdatai\in\dataspace$, $\muhat\rbr{f_b(\initdatai)}=\muhat\rbr{\initdatai}$.
    we then have that
    \begin{align*}
        \utilci \rbr{M, \rbr{(\nni,\subfunci),\strat_{-i}^{\star} }}
        &\leq
        \utilci \rbr{M, \rbr{(\nni,f_a),\strat_{-i}^{\star} }}
        % \qquad
        &
        (\text{Lemma}~\ref{lem:NIC-part1})
        \\
        &\leq
        \utilci \rbr{M, \rbr{(\nni,f_b),\strat_{-i}^{\star} }}
        % \qquad
        &
        (\text{Lemma}~\ref{lem:NIC-part2})
        \\
        &\leq
        % \utilci \rbr{M, (\datanum_i^{\star},f_b),\strat_{-i}^{\star} }
        % =
        \utilci \rbr{M, \rbr{\strat_{i}^{\star},\strat_{-i}^{\star} }} .
        &
        % \qquad
        \hspace{-1cm}
        (\text{Lemma}~\ref{lem:NIC-part3})
    \end{align*}



% \begin{lemma}
%     \label{lem:main_prop5}
%     Under $M$, when following the recommended strategy $\strat^\star $, the expected profit of buyers approximates the optimal profit, $\mathrm{profit}^{\star}$, with additive error $\epsilon$.
%     Formally, 
%     \begin{equation*}
%         \mathrm{profit}_A( \datanum_1^{\star},\dots,\datanum_d^{\star}) \geq \mathrm{profit}^{\star} - \eps .       
%     \end{equation*}
% \end{lemma}
% \begin{proof}
%     We assume that 
%     \[\left\{ \tilde{\datanum}_{i} \right\}_{i\in \contributors} = \underset{\datanum_1,\dots,\datanum_d}{\arg \max}\rbr{\mathrm{rev}^{\star}  \rbr{ \buyers,\sum_i \datanum_i} - \sum_{i=1}^{d} \cost_i \datanum_i  },\]
%     then we have
%     \begin{align*}
%       \mathrm{profit}^{\star} &  = \underset{\datanum_1,\dots,\datanum_d}{\max}\rbr{\mathrm{rev}^{\star} 
%         \rbr{ B,\sum_i \datanum_i} - \sum_{i=1}^{d} \cost_i \datanum_i  } \\ & = \mathrm{rev}^{\star} 
%         \rbr{ B,\sum_i \tilde{\datanum}_i} - \sum_{i=1}^{d} \cost_i \tilde{\datanum}_i \\ & \leq \mathrm{rev}_A
%         \rbr{ B,\sum_i \tilde{\datanum}_i} - \sum_{i=1}^{d} \cost_i \tilde{\datanum}_i  + \eps  \\ & \leq  \mathrm{profit}_A(\datanum_1^{\star},\dots,\datanum_d^{\star}) + \eps.
%     \end{align*}
%     Here the first inequality is by the definition of $\eps$-revenue optimal algorithm: \[\mathrm{rev}_A
%         \rbr{ B,\sum_i \tilde{\datanum}_i} \geq \mathrm{rev}^{\star} 
%         \rbr{ B,\sum_i \tilde{\datanum}_i}-\eps,\]
%     and the second inequality is by equation~\eqref{eq:n_i_star}.
% \end{proof}


% \textbf{Proof sketch: }For any $(\nni,\subfunci)$, we introduce two functions,

% \begin{itemize}
%         \item[-] $f_a(\initdatai) $ modifies the data output by  $\subfunci$ without changing the resulting sample mean, i.e., $  \muhat(f_a(\initdatai)) = \muhat(\subfunci(\initdatai))  $, but change the size of dataset to $\datanum_i^{\star}$.
%         \item[-] $ f_b(\initdatai) $ outputs data that preserves the mean of the original dataset, i.e., $ \muhat(f_b(\initdatai)) = \muhat(\initdatai)$.
%     \end{itemize}

  
%     We apply three lemmas in Appendix~\ref{app:tech} to bound the utility at each step:
%     \begin{itemize}
%         \item[-] \textbf{Step 1:} By applying \textit{Lemma~\ref{lem:NIC-part1}}, we show that replacing \( \subfunci \) with \( f_a \) does not increase contributor \( i \)'s utility:
%         \[
%         \utilci \rbr{M, \rbr{(\nni, \subfunci), \strat_{-i}^{\star} }} \leq \utilci \rbr{M, \rbr{(\nni, f_a), \strat_{-i}^{\star} }}.
%         \]

%         \item[-] \textbf{Step 2:} By applying \textit{Lemma~\ref{lem:NIC-part2}}, we further show that replacing \( f_a \) with \( f_b \) also does not increase the utility:
%         \[
%         \utilci \rbr{M, \rbr{(\nni, f_a), \strat_{-i}^{\star} }} \leq \utilci \rbr{M, \rbr{(\nni, f_b), \strat_{-i}^{\star} }}.
%         \]

%         \item[-] \textbf{Step 3:} Finally, \textit{Lemma~\ref{lem:NIC-part3}} establishes that the utility is maximized when contributor \( i \) adheres to the dominant strategy \( \strat_{i}^{\star} \):
%         \[
%         \utilci \rbr{M, \rbr{(\nni, f_b), \strat_{-i}^{\star} }} \leq \utilci \rbr{M, \rbr{\strat_{i}^{\star}, \strat_{-i}^{\star} }}.
%         \]
%     \end{itemize}





