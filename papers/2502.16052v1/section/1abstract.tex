

\begin{abstract}

We study a data marketplace where a broker facilitates transactions between buyers and contributors. Buyers seek to estimate the mean \(\mu\) of an unknown normal distribution \(\mathcal{N}(\mu, \sigma^2)\), but have varying valuations based on their estimation error. Contributors, each with different data collection costs, gather samples from this distribution and report them (not necessarily truthfully) to the broker. The broker then sells subsets of the combined dataset to buyers at varying prices and redistributes the revenue to contributors. 
%
We formalize this as a mechanism design problem aimed at maximizing profit (total revenue minus data costs) while satisfying key market constraints: individual rationality (buyers and contributors benefit from participation), envy-freeness (no buyer prefers another’s allocation), budget balance (total payments match revenue), and incentive-compatibility (contributors are incentivized to collect a sufficient amount of data report it truthfully).  

We design a mechanism which satisfies these requirements.
We first establish a connection between envy-free data pricing and ordered-item pricing (OIP) for unit-demand buyers and leverage OIP algorithms to determine the optimal data allocation and expected prices for buyers. The actual prices paid by buyers are centered around these expected prices, but also vary based on discrepancies in contributors' reported data. This variation is then passed on to the contributors via their payments.
This scheme results in a Nash equilibrium (NE) where only the two lowest-cost contributors collect all the data and report it truthfully.
% We also show that this NE is individually rational for buyers, satisfies budget balance, and maximi
Including this variation in the prices also helps us achieve individual rationality for buyers, as buyers will pay less if there are significant discrepancies in contributors' datasets.


To complement these findings, we prove a nearly matching upper bound on the maximum possible profit achievable in any NE of any mechanism, thus proving that our mechanism is essentially unimprovable.
We also show that no nontrivial dominant-strategy incentive-compatible mechanism exists in this problem. 


\end{abstract}