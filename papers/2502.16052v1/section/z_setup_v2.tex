\section{Problem set up}

We now formally define our problem. 
First, we will briefly review mean estimation to motivate our modeling choices.

\parahead{Review of mean estimation}
%Let $\Pcal$ be the class of all one-dimensional distributions whose variance is at most $\sigma^2$.
We wish to estimate the mean $\mu(P) = \EE_{Z\sim P}[X]$ of some \emph{unknown} distribution $P\in\Pcal$, via 
an i.i.d dataset of $n\in\NN$ points collected from $P$.
The sample mean $\muhat(X) = \frac{1}{|X|}\sum_{x \in X} x$ achieves 
$L_2$ error
\begin{align}
    \label{eqn:err}
    \err(n; P) = %\inf_{\muhat}
         \EE_{X\sim \genprocess(P)}\left[
        (\muhat(X) - \mu(P))^2
    \right]      
    = \frac{\Var(P)}{n},
\end{align}
where $\Var(P)$ is the variance of $P$.
We will focus on the sample mean estimator since it is the most natural, 
and it is well-known that $\muhat$ is minimax optimal\footnote{%
That is, let $\Pcal_{\sigma^2}$ denote all distributions with variance at most $\sigma^2$.
Then, for any $\sigma^2$, among all possible estimators $h$, $\muhatsm$ minimizes the maximum
$L_2$ error in $\Pcal_{\sigma^2}$,
i.e
$\sup_{P\in\Pcal_{\sigma^2}} \EE_{X\sim \genprocess(P)}\left[
        (\muhat(X) - \mu(P))^2
    \right]
    = \inf_{h} \sup_{P\in\Pcal_{\sigma^2}} \EE_{X\sim \genprocess(P)}\left[
        (h(X) - \mu(P))^2
    \right]$.
}~\citep{stein1956inadmissibility}.



\parahead{Marketplace}
A marketplace for mean estimation involves three key actors:  
\emph{(i)} a finite set of data \emph{contributors} $\contributors$,
\emph{(ii)} a finite set of \emph{buyers} $\buyers$,
and
\emph{(iii)} a \emph{trusted broker}, who facilitates transactions between contributors and buyers.  


Each contributor draws i.i.d samples from the same but \emph{unknown} distribution $P$, and submits it (potentially misreported) to the broker.
Buyers wish to estimate $\mu(P)$, but have different valuations $\{\vali\}_{i\in\buyers}$ for how well they have estimated it (we will define $\vali$ shortly).




Once contributors have submitted datasets, the broker sells subsets of the combined dataset to buyers, and redistributes the revenue among contributors. These interactions proceed as follows:
\vspace{-0.15in}
\begin{enumerate} [leftmargin=0.16in]
    \item \emph{Mechanism design}:
    \label{itm:mechanismdesign}
    The broker designs and publishes a mechanism  to evaluate data submissions from
contributors, allocate subsets and set prices for buyers, and redistribute the revenue back to contributors.
\vspace{-0.10in}
    \item \emph{Data collection}:
    \label{itm:datacollection}
Each contributor $i \in \contributors$ collects $\nni \in \NN$ data points at cost $\cost_i \nni$, to produce a dataset $\initdatai \in \RR^{\nni}$.
\vspace{-0.1in}
    \item \emph{Data submission}:
    \label{itm:datasubmission}
    Each contributor submits $\subdatai = f_i(\initdatai) \in \bigcup_{\ell=0}^\infty \RR^\ell$, where the submission function $f_i$ may modify the collected dataset. This enables strategic
alterations (\eg fabrication) to maximize personal benefit by reducing costs and/or
increasing their payment. \hspace{-0.5in}
\vspace{-0.26in}

    \item \emph{Data allocation and purchases}:
    \label{itm:pricing}
     The broker allocates independent subsets of size
     $\{\mmj'\}_{j\in\buyers} \in \NN^{|\buyers|}$ of
     all contributors' submissions $\bigcup_{i\in\contributors}\subdatai$
     %where $\dsmapj' \in \bigcup_{\ell=0}^\infty \RR^\ell $
     to each buyer, and charges them \emph{prices} $\{\pricej'\}_{j \in \buyers}\in\RR^{|\buyers|}$.
    
     % The broker allocates subsets  
     % $\{\dsmapj'\}_{j\in\buyers}$ of
     % all contributors' submissions $\bigcup_{i\in\contributors}\subdatai$
     % where $\dsmapj' \in \bigcup_{\ell=0}^\infty \RR^\ell $
     % for each buyer, and charges them \emph{prices} $\{\pricej'\}_{j \in \buyers}\in\RR^{|\buyers|}$.
     %It generates total revenue $\sum_{j\in\buyers}\pricej'$.

\vspace{-0.1in}
    \item \emph{Revenue redistribution}: \hspace{-0.05in}
    \label{itm:revenueredistribution}
    The broker disburses the revenue $\sum_{j \in \buyers} \pricej'$ via \emph{payments} $\{\payi'\}_{i \in \contributors} \in \RR^{|\contributors|}$  to contributors.
\end{enumerate}


\subparahead{Verbiage}
As noted above, the amount charged to a buyer for a dataset is referred to as the \emph{price}, while the amount given to a contributor for their work is called a \emph{payment}.  




\subparahead{Problem instance}
We will assume that the variance of $P$ is \emph{at most} $\sigma^2$.
Let  $\cost \defeq \{\costi\}_{i \in \contributors} \in \RR_+^{|\contributors|}$ denote all contributor costs, and $\val \defeq \{\valj\}_{j \in \buyers}$ denote all buyer valuations.
An instance of this market design problem is specified by the tuple $(\contributors, \buyers, \cost, \val, \sigma^2)$,  which we will assume is \emph{public information},
\ie known to the broker and all contributors and buyers.
No party---the contributors, buyers, or broker---has any additional information (\eg a prior on $P$).




\parahead{Mechanism}
The broker's mechanism determines buyers' dataset sizes, buyers' prices, and contributors' payments based on the  datasets reported by contributors.
To define this formally, let $\dataspace = \bigcup_{\ell=0}^\infty \RR^\ell$
be the space of datasets that each contributor may submit.
A \emph{deterministic mechanism} is specified by  
$M = \left(\{\mmj\}_{j \in \buyers}, \{\pricej\}_{j \in \buyers}, \{\payi\}_{i \in \contributors}\right)
$,
where $\mmj : \dataspace^{|\contributors|} \to \NN$
%\bigcup_{\ell=0}^\infty \RR^\ell$
is the dataset size selection rule for buyer $j$,  
$\pricej : \dataspace^{|\contributors|} \to \RR$ is the pricing rule for buyer $j$, and  
$\payi : \dataspace^{|\contributors|} \to \RR$ is the payment rule for contributor $i$. 
All three functions could also depend on the public information $(\contributors, \buyers, \cost,\val, \sigma^2)$, but we have suppressed this dependence as it will remain fixed.
A \emph{randomized mechanism} $M$ is a distribution over such tuples, from which a sample is drawn and executed at runtime.

Due to the stochastic nature of data collection and the mechanism's randomness, prices and payments will also be random. We allow \emph{ex post} (\ie stochastic) negative prices (where the broker pays a buyer) and negative payments (where a contributor pays the broker). Although unconventional, this enables the design of mechanisms with strong guarantees. Notably, in our mechanism, the \emph{ex-ante} (\ie expected) prices and payments are always non-negative.


% The broker's mechanism should determine the buyers' estimates, the buyers' prices, and contributors' payments, based on the 
% datasets reported by the contributors.
% Hence, a \emph{deterministic} mechanism for this problem can be specified via 
% $\left(\{\muhatj\}_{j\in\buyers}, \{\pricej\}_{j\in\buyers}, \{\payi\}_{i\in\contributors}\right)$.
% Here, $\muhatj:\Dcal^{|\contributors|}\rightarrow \RR$ is the estimator
% for buyer $j$,
% $\pricej:\Dcal^{|\contributors|}\rightarrow \RR$ is the pricing rule for buyer $j$,
% and $\payi:\Dcal^{|\contributors|}\rightarrow \RR$ is the payment rule for contributor $j$.
% A randomized mechanism $M$ is a distribution over all possible values for $\left(\{\muhatj\}_{j\in\buyers}, \{\pricej\}_{j\in\buyers}, \{\payi\}_{i\in\contributors}\right)$, from which a sample is drawn and executed at run time.


\parahead{Contributors}
After the mechanism is published, each contributor decides how much data to collect and what to submit to the broker. A \emph{deterministic strategy} for contributor $i$ is a pair $\strati = (\nni, \subfunci)$. First, the contributor samples $\nni \in \NN$ points to collect an initial dataset $\initdatai = \{x_{i,j}\}_{j=1}^{\nni}$, incurring a cost of $\costi \nni$. Here, each $x_{i,j}$ is drawn independently from $P$. She then submits $\subdatai = \{y_{i,j}\}_j = \subfunci(\initdatai)$, where $\subfunci : \Dcal \to \Dcal$ maps the collected dataset to a possibly altered dataset of potentially different size.  
A \emph{randomized (mixed) strategy} $\strati$ is a distribution over $(\nni, \subfunci)$ pairs, from which a sample is drawn and executed at runtime.


%\subparahead{Strategy profiles and generating process}
Let $\strat = \{\stratj\}_{j \in \contributors}$ denote the strategy profile of all contributors, and $\stratmi = \{\stratj\}_{j \in \contributors \setminus \{i\}}$ the strategies of all contributors except $i$.
% We use $\genstrat$ to represent the generating process for the combined dataset $\subdata \defeq \{\subdatai\}_{i \in \contributors}$ received by the broker (see~\eqref{eqn:err}). This process accounts for each contributor sampling $\nni$ i.i.d. points $\initdatai$ and submitting $\subdatai = \subfunci(\initdatai)$.
% In particular, $\subdata$ may not be i.i.d. if any contributor uses $\subfunci \neq \identity$.

\subparahead{(Un)truthful submissions}
To provide reliable mean estimates for buyers, 
we wish to incentivize agents to submit data truthfully, \ie 
$\subfunci=\identity$, where $\identity$ is the identity function which maps the data to itself.
In naive mechanisms, contributors may find it beneficial to use $\subfunci\neq\identity$, to fabricate, falsify, or otherwise misreport their data.
Note that our formalism allows contributors to alter their submission based on the collected dataset $\initdatai$. For example, she may collect a small dataset, fit a distribution to it, and fabricate additional points by sampling from this fitted distribution.





\subparahead{Contributor utility} \hspace{-0.1in}
If a contributor samples $\nni$ points and receives payment $\payi' \in \RR$, her \emph{ex post} utility is $\payi' - \costi \nni$.
As these quantities depend on the mechanism and the contributors' strategies,
her \emph{ex ante} utility under the \emph{unknown} distribution $P$ is,
% and possibly the distribution $P \in \Pcal$ from which data is drawn, her \emph{ex ante} utility can be written as:
\begin{align*}
    % \utilci(\mech, \strat)
    %     &= \inf_{P\in\Pcal} \utilci(\mech, \strat; P)
    %     \hspace{0.3in}\text{where, }\hspace{0.1in}
    %       \numberthis  \label{eqn:utilc}
    %     \\
    \utilci(\mech, \strat; P)
        &= \EE_{M, \sigma, P}\left[ \payi\big(\{\subfunci(\initdatai)\}_{i\in\contributors}\big)
                - \costi \nni\right].
          \numberthis  \label{eqn:utilc}
    % \utilci(\mech, \strat)
    %     &= \inf_{P\in\Pcal}  \EE_{M, \sigma, P}\left[ \payi\big(\{\subfunci(\initdatai)\}_{i\in\contributors}\big)
    %             - \costi \nni\right]
    %        \numberthis  \label{eqn:utilc}
\end{align*}
The expectation $\EE_{M, \sigma, P}$ is taken over the randomness in the mechanism, the contributor strategies $\strati = (\nni, \subfunci)_{i \in \contributors}$, and the stochasticity of i.i.d. datasets $\{\initdatai\}_{i \in \contributors}$ drawn from $P$.
%Modeling this distribution under the unknown distribution $P$
%xplicitly modeling this dependence on $P$

% The $\inf_{P \in \Pcal}$ serves a similar purpose as the $\sup_{P \in \Pcal}$ in~\eqref{eqn:err}, ensuring the quantity is well-defined in the frequentist setting; it reflects the fact that $P$ is unknown and requires the mechanism to perform well for all $P \in \Pcal$ under reasonable strategies $\strat$. Notably, in our mechanism, $\utilci(\mech,\strat; P)$ will be constant with respect to $P$ when agents report truthfully, \ie independent of $P$.
% \kkcomment{@Keran: check if this is this correct?}


\parahead{Buyers}
Buyers are not strategic players in our marketplace.
However, each buyer may individually choose whether to participate, and expects fair treatment relative to other buyers.
Buyers wish to estimate $\mu(P)$, \ie achieve small error, as defined in~\eqref{eqn:err}.
This error depends on the dataset provided by the mechanism $M$,
the contributors' strategies $\strat$, and the \emph{unknown} distribution $P$.
% and the generating process $\genstrat$ induced by the contributors' strategies $\strat$ (see paragraph titled "Strategy profiles and generating process").
Moreover, buyers may have differing valuations for varying errors, modeled by buyer-specific non-increasing functions $\valfunci: \RR_+ \rightarrow \RR_+$.
%This is illustrated in Fig.~\ref{fig:buyervalfuncs}.
A buyer's \emph{valuation} $\vali$ is thus given by:
\begin{align}
    \vali(M, \strat; P) \defeq \valfunci(\err(, \genstrat)).
    \label{eqn:buyerval}
\end{align}
Finally, a buyer's \emph{utility} $\utili$ is her valuation minus her payment,
and is defined as follows:
%We have,
\begin{align}
\hspace{-0.07in}
    \utilbi(M, \strat) \defeq \EE_{M, \sigma, P}\left[ \vali(\muhatj, \strat) - \pricei\big(\{\subfunci(\initdatai)\}_{i\in\contributors}\big) \right].
    \hspace{-0.03in}
    \label{eqn:buyerutil}
\end{align}

\textbf{The mechanism's objectives}

We aim to design a mechanism that satisfies the following properties under the recommended strategy $\strategy^{\star}_{i \in \contributors}$.



1. IR for the contributors: \[\utilci (M,\strategy) = \underset{\distrifamily}{\inf} \ \mathbb{E}\left[  p_i(M,\strategy)\right]-\cost_i \datanum_i \geq 0 , \; i \in \contributors \] 

2. IR for the buyer: \[\utilityb_j(M,\strategy  ) \geq 0 , \; j \in \buyers  \] 

3. The assignment is \emph{envy-free} for the buyer, i.e, no buyer has a higher utility for the assignment offered to another buyer.

4. Collecting the recommended amount of data and truthful submission is NIC: when others follow the best strategy by the mechanism, for any $\strategy_i \neq \strategy_i^{\star} $,
\[ \utilci (M,\strategy_i,\strategy_{-i}^{\star}) \leq \utilci (M,\strategy_i^{\star},\strategy_{-i}^{\star})  \]


5. Efficiency: Maximize contributors' total profit. 

6. Budget balance: \[  \sum_{j=1}^{\numtypes} \pricej(M,\strategy) = \sum_{i=1}^{\numcontributors} \pay_i(M,\strategy) \] 




\textcolor{red}{contents below are not needed}

\textbf{Contributors: }There are $\numcontributors$ contributors, each capable of collecting a dataset $\data_i$ containing data points drawn independently and identically from a distribution $\distri \in \distrifamily$. Each contributor $i$ incurs a cost $\cost_i$ per data point collected.


\textbf{Contributors’ strategy space: } Under this mechanism, contributors collect $\datanum_i$ data points, denoted as $X_i$, and then submit $\subdata_i = \function_i(X_i)$ to the mechanism, where $\function_i : \bigcup_{n\geq 0}^{}\mathbb{R}^n \to \bigcup_{n\geq 0}^{}\mathbb{R}^n$ is a function specific to the contributor that maps the collected dataset to a potentially fabricated dataset of a different size. The strategy space for each contributor is represented by $\strategy_i = (\datanum_i, f_i)$, indicating how many data points to collect and how to report their data. 
%Buyers then purchase data from the mechanism to estimate the true mean value $\mu$. 
We write the strategy
space of a contributor as $\mathcal{S}= \mathbb{N} \times \mathcal{F}$, where $ \mathbb{N}$ represents the space for number of data collected, and $\mathcal{F}$ is the space of functions mapping the dataset collected to the
dataset submitted. 

In the case of \textbf{Truthful submission}, contributors submit their data unaltered (“i.e., $f_i = \mathbf{I}$”).






\textbf{Buyers: } The buyer population $\buyers$ consists of $\numtypes$ types of buyers. Each buyer $j \in \buyers$ has a non-decreasing valuation curve $v_j$, representing their value for the data points received. The valuation curves will be formally defined in Equation \ref{eq:buyervalue}.

\textbf{Mechanism: } The mechanism takes as input the submitted data \( \subdata_1, \dots, \subdata_d \), and outputs:

\begin{itemize}
    \item An estimator $\hat{\mu}$ for the distribution mean. (\ref{itm:pricing})
    \item Payments $\pi$ for buyers. (\ref{itm:pricing})
    \item Revenue redistribution among contributors $p$. (\ref{itm:revenueredistribution})
\end{itemize}

The mechanism space is defined as:
\[
\mathcal{M} = \left\{ M = (p,  \pi, \hat{\mu}) \right\},
\]
where $p : \left(\bigcup_{n \geq 0} \mathbb{R}^n\right)^d \to \mathbb{R}^d$, $\pi : \left(\bigcup_{n \geq 0} \mathbb{R}^n\right)^d \to \mathbb{R}^\numtypes$, and $\hat{\mu}:  \left(\bigcup_{n \geq 0} \mathbb{R}^n\right)^d \to  \mathbb{R} $.

\textbf{Buyer's ex-post valuation and utility: } The valuation of buyer $j$ is based on estimation error. Given contributors' strategy $\strategy=(\strategy_1,\dots, \strategy_\numcontributors)$ and a mechanism $M$, buyer $j$'s estimation \emph{error} is:

\begin{align}
   \mathrm{error} =  \underset{p \in \distrifamily}{\sup} \; \mathbb{E}\left[ \rbr{\hat{\mu}_j-\mu }^2 \right] ,
\end{align}

and buyer $j$'s \emph{valuation} is a decreasing function of error:
\begin{align} \label{eq:buyervalue}
     \tilde{v}_j(M,s) = \phi_j(\mathrm{error}).
\end{align}

Where $\phi : R_{+}\to [0,1]$ is a decreasing function on $[0,\infty]$.

For any buyer $j \in \buyers$, the \emph{utility} is defined as 
\begin{align*}
    \utilityb_j (M,s) = \tilde{v}_j(M,s) - \mathbb{E}\sbr{\pi_j(M,s)} .
\end{align*}

\textbf{Buyer's ex-ante valuation of dataset: }  Let $Z=(Z_1,\dots,Z_n)$ be a sample drawn from the distribution $\distri^n$. We denote  $\hat{\mu}(Z)$ as sample mean, i.e, $\hat{\mu}(Z) \overset{d}{=} \frac{\sum_i Z_i}{n}$. Buyer $j$’s valuation under truthful data is defined as: 

\begin{align}
   \mathrm{error} =  \underset{p \in \distrifamily}{\sup} \; \mathbb{E}\left[ \rbr{\hat{\mu}(Z)-\mu }^2 \right] ,
\end{align}

and buyer $j$'s valuation is a decreasing function of error:
\begin{align} \label{eq:buyervalue}
     v_j(n) = \phi_j(\mathrm{error}).
\end{align}

\textbf{Contributors' profit: }Contributors $i$'s profit
\[ u_i(M,s) = \underset{\distrifamily}{\inf}\; \mathbb{E}\sbr{p_i(M,s)}- c\datanum_i . \]


\textbf{Recommended strategies for the contributors: } The broker recommend strategy $\strategy^{\star}_{i \in C}$ to each contributor to (\emph{i}) incentivize truthful contribution.




%\[n^* =\max \sbr{ \arg\max_{n \in [N]} \left[ v(n) - \price(n) \right]}\]


\begin{definition}
    Envy Free pricing problem: Given the input $\left\{ v_j \right\}_{i \in \buyers} $, and all buyers agree to participate in the mechanism, an Envy Free pricing scheme outputs a set of $(\hat{\mu}_j,\pricej), \; \forall j \in \buyers$ maximizing the contributors' revenue. Here,  $\left\{(\hat{\mu}_j,\pricej)\right\}_{j \in \buyers}$ represents the assignment: an estimator and a price offered to each buyer. It satisfies the following properties:

    \emph{(i)} Individual Rationality (IR): for any $j \in \buyers$, \[ \utilityb_j(M,\strategy)\geq 0 .\] 

    \emph{(ii)} Envy Free (EF): for any $j$ and any $k \neq j$, we have 
    \[ v_j(M, \hat{\mu}_j) -\mathbb{E}\sbr{\price_j} \geq   v_j(M, \hat{\mu}_k) -\mathbb{E}\sbr{\price_k}. \]    

\textcolor{red}{need to be fixed}
     \begin{align*}
         \EE_{M, \sigma, P}\left[ \valj(\muhatj, \strat) - \pricej \right]  \\ \geq  \EE_{M, \sigma, P}\left[ \valj(\muhatk, \strat) - \pricek \right] 
     \end{align*}
     
\end{definition}



We demonstrate that the Envy-free pricing problem is equivalent to the Revenue-optimal data pricing problem \ref{def:rev_optpricing}. Thus, any solution to the Envy-free pricing problem is also a solution to the Revenue-optimal data pricing problem. Therefore, we convert our problem (\ref{itm:pricing}) into finding an optimal price curve that maximizes the contributors' revenue: \textcolor{red}{proof of equivalence is in Appendix Lemma}

    \begin{align*}
        n^{\star}_i(\pricecurv) = \max \rbr{\underset{n}{\arg\max} \rbr{v_i(n)-\pricecurv(n) }}
    \end{align*}

    \begin{align*}
       \mathrm{rev}(\pricecurv^{\star}) = \underset{\pricecurv}{\max} \rbr{\sum_{i \in \buyers} \pricecurv \rbr{ n^{\star}_i(\pricecurv) }}
    \end{align*}

