\section{Problem set up with strategic contributors}
\label{sec:setup}

In~\S\ref{sec:buyerside}, we modeled the data marketplace based on the quantity of data transacted.
However, this framework breaks down in the presence of strategic contributors. 
We will now study strategic contributor behavior. 
Our goal will be to approximate the baseline in~\S\ref{sec:profitbaseline} while accounting for strategic contributor behavior. 
We study such strategic considerations in the restricted, yet nontrivial setting of normal mean estimation.


% We now formally define our problem when the contributors are strategic and buyers wish to estimate
% the mean of an unknown normal distribution.

In~\S\ref{sec:env}, we describe the environment, including the mechanism, the contributors' strategy spaces, and contributor and buyer utilities.
\S\ref{sec:mechdesignproblem} formulates the mechanism design problem, aiming to maximize profit while satisfying constraints on buyer and seller incentives and market feasibility.  
In~\S\ref{sec:modelingdiscussion}, we discuss our modeling choices.  
We begin by briefly describing normal mean estimation.  




\parahead{Normal mean estimation}
A finite dataset $X=\{x_i\}_{i}$ consists of i.i.d points drawn from 
a normal distribution $\Ncal(\mu,\sigma^2)$, where the variance $\sigma^2$ is known, but the mean $\mu$ is not.
A finite dataset $Y=g(X) = \{y_i\}_i$
%\{y_1,\dots,y_{m}\}$ of size $m$
is reported to a learner,
where the (possibly random) \emph{reporting function} $g$ maps the collected dataset $X$ to a
dataset $Y$.
%\ac{is this function $g$ typical in normal mean estimation problems? is this strange to have in a review of normal mean estimation?}
The learner estimates $\mu$ via the sample mean $\muhatsm(Y) = \frac{1}{|Y|}\sum_{y \in Y} y$
%\ac{it seems like we are saying the sample mean is always used, wouldn't this not be the case if $g$ is something interesting?}.
Conventionally, one studies the \emph{loss} the learner incurs as a function of the estimation error \( |\hat{\mu}(Y) - \mu| \), where the loss increases with error.  
Here, we instead model the \emph{value} the learner derives as a decreasing function \( \valfunc: \mathbb{R}_+ \to [0,1] \), where $\valfuncnob(e')$ is her instantaneous value if her error is $e'$. %\ac{why is value between 0 and 1?}.
We can write the learner's valuation $\val$ for a reporting function $g$ as:
%\vspace{-0.05in}
\begin{align}
    \label{eqn:valerr}
    \val(g) = \inf_{\mu\in\RR} \EE_{g,\mu}\left[
        \valfuncnob\left(\left|\,\muhat(g(X)) - \mu\,\right|\right)
    \right].
\end{align}
The expectation $\EE_{g, \mu}$ is over the randomness in generating
data $X$ from the \emph{unknown} distribution $\Ncal(\mu, \sigma^2)$, and possibly the reporting function $g$.
The $\inf_\mu$ accounts for the fact that the $\mu$ is unknown
and a learner wishes for a good reporting function to yield good estimates for all unknown distributions.
To illustrate, let $\mutilde\in\RR$ be some arbitrary value. A pathological reporting function $g(\cdot) = (\mutilde, \dots, \mutilde)$ which always reports a vector of repeated $\mutilde$ values will achieve maximum
value when the true mean is $\mutilde$, but does poorly elsewhere.

Typically, \( g \) is either the identity function or selects a random subset of \( X \), in which case \( Y \) remains an i.i.d.\ dataset.  
In a marketplace, although data is sampled i.i.d., strategic contributors  may fabricate or misreport data to maximize their personal benefit.  
This strategic behavior can result in situations where the reported data is no longer i.i.d  samples from $\Ncal(\mu,\sigma^2)$.

% An \emph{estimator} $\muhat$ is a function which maps this \emph{reported dataset} to a scalar, to approximate $\mu$.
% Below, we define $\errgen(\muhat, f, m; \mu)$ to be
% the $L_2$ error of an estimator $\muhat$ when data is drawn from $\Ncal(\mu, \sigma^2)$ and reported according to $g$ (note that $\mu$ is the only unknown).
% With a slight abuse of notation, we also define
% $\errgen(\muhat, g, m)$ to be the worst-case risk.
% We have:
% \begin{align}
%     \label{eqn:err}
%     \errgen(\muhat, g, m; \mu) \defeq %\inf_{\muhat}
%         \EE_{X\sim \Ncal(\mu, \sigma^2)^m}\left[
%         (\muhat(g(X)) - \mu(P))^2
%     \right],
%     \hspace{0.3in}
%     \err(\muhat, g, m) = \sup_{\mu\in\RR} \errgen(\muhat, g, m; \mu).
% \end{align}
% Here, the supremum accounts for the fact that $\mu$ is unknown and that an estimator should do well for all values of $\mu$.
% For instance, consider any $\mutilde\in\RR$.
% The (nonsensical) estimator which always outputs $\mutilde$,
% \ie $\muhat(Y)=\mutilde$ achieves zero error when $\mu=\mutilde$, but does poorly for other values of $\mu$.
% The supremum accounts for the fact that
% a sensible estimator should do well on all values of $\mu$.

% \subparahead{Truthfully reported data and estimation via sample mean}
% When $g$ simply chooses an independent subset of size $m$ from $X$, then $Y$ is also an i.i.d dataset.
% Then, taking the sample mean $\muhatsm(Y) = \frac{1}{|Y|}\sum_{y \in Y} y$,
% is known to be optimal, \ie for all $m$, $\muhatsm$ minimizes $\err(\muhat, \identity, m)$ among all possible estimators $\muhat$,
% with error
% $\errgen(\muhatsm, \identity, m; \mu) %\inf_{\muhat}
%          \;= \EE\left[
%         (\muhatsm(X) - \mu(P))^2
%     \right]      
%     = \frac{\sigma^2}{m}$.
    %for all $m$
% \footnote{%
% Generally speaking
% the performance of an estimator depends on the specific
% unknown distribution $\Ncal(\mu, \sigma^2)$.
% For instance, 
% the (nonsensical) estimator $\muhat(Y)=1.23$ achieves zero error when $\mu=12.3$, but does poorly for other values of $\mu$.
% The sample mean $\muhatsm$ minimizes the maximum
% $L_2$ error in among all normal distributions with unknown mean, \ie $\{\Ncal(\mu,\sigma^2); \mu\in\RR\}$.
% Precisely,
% $\sup_{\mu\in\RR} \EE_{X\sim \Ncal(\mu,\sigma^2)^m}\left[
%         (\muhatsm(X) - \mu(P))^2
%     \right]
%     = \inf_{\muhat} \EE_{X\sim \Ncal(\mu,\sigma^2)^m}\left[
%         (\muhat(X) - \mu(P))^2 \right] = \sigma^2/m.
%         $
% }
    %which does not depend on $\mu$~\citep{stein1956inadmissibility}.
% The simplest case occurs when the generating process $\genprocess = \geniid$ draws i.i.d. samples from $P$. Here, the sample mean $\muhatsm(X) = \frac{1}{|X|}\sum_{x \in X} x$ achieves 
% \begin{align}
%     \label{eqn:err}
%     &\err(m; P) \defeq \errgen(\muhatsm, \geniid, m; P) %\inf_{\muhat}
%          \\
%          &\;= \EE_{X\sim P^m}\left[
%         (\muhatsm(X) - \mu(P))^2
%     \right]      
%     = \frac{\Var(P)}{m},
% \end{align}
% where $\Var(P)$ is the variance of $P$.
% It is well-known that $\muhat$ is minimax optimal for I.I.D data~\citep{stein1956inadmissibility}.




\subsection{Description of the environment}
\label{sec:env}


\parahead{Data marketplace for normal mean estimation}
%The market sho
% We now describe 
% A marketplace for mean estimation  involves three key actors:  
% \emph{(i)} a finite set of data \emph{contributors} $\contributors$,
% \emph{(ii)} a finite set of \emph{buyers} $\buyers$,
% and
% \emph{(iii)} a \emph{trusted broker}, who facilitates transactions between contributors and buyers. 
In the data marketplace,
each contributor draws i.i.d. samples from the same but \emph{unknown} distribution $\Ncal(\mu,\sigma^2)$, and submits a dataset (potentially misreported) to the broker.
Buyers wish to estimate $\mu$, but have different valuations $\{\valfunci\}_{i\in\buyers}$ based on their estimation errors.
Once contributors have submitted datasets, the broker sells subsets of the combined dataset to buyers, and redistributes the revenue among contributors.
These interactions proceed in the following order follows:
%\vspace{-0.15in}
\begin{enumerate} [leftmargin=0.2in]
    \item \emph{Mechanism design}:
    \label{itm:mechanismdesign}
    The broker designs and publishes a mechanism  to collect data from
contributors, allocate subsets and set prices for buyers, and redistribute the revenue back to contributors.
She also instructs contributors to
collect amounts $\{\reqamnti\}_{i\in\contributors}\in\NN^{|\contributors|}$.
%\vspace{-0.10in}
    \item \emph{Data collection}:
    \label{itm:datacollection}
Each contributor $i \in \contributors$ collects $\colamnti \in \NN$ i.i.d. data points at cost $\cost_i \nni$, to produce a dataset $\initdatai \in \RR^{\colamnti}$.
Here $\colamnti$ is not necessarily equal to $\reqamnti$.
%\vspace{-0.1in}
    \item \emph{Data submission}:
    \label{itm:datasubmission}
    Each contributor submits $\subdatai = f_i(\initdatai) \in \bigcup_{\ell=0}^\infty \RR^\ell$, where the submission function $f_i$ may modify the collected dataset. This enables strategic
alterations (e.g. fabrication) to maximize personal benefit by reducing costs and/or
increasing their payment. \hspace{-0.5in}
%\vspace{-0.26in}

    \item \emph{Data allocation and purchases}:
    \label{itm:pricing}
     The broker allocates independent subsets of size
     $\{\mmj\}_{j\in\buyers} \in \NN^{|\buyers|}$ of
     all contributors' submissions $\bigcup_{i\in\contributors}\subdatai$
     %where $\dsmapj' \in \bigcup_{\ell=0}^\infty \RR^\ell $
     to each buyer, and charges them \emph{prices} $\{\pricej'\}_{j \in \buyers}\in\RR^{|\buyers|}$.
    
     % The broker allocates subsets  
     % $\{\dsmapj'\}_{j\in\buyers}$ of
     % all contributors' submissions $\bigcup_{i\in\contributors}\subdatai$
     % where $\dsmapj' \in \bigcup_{\ell=0}^\infty \RR^\ell $
     % for each buyer, and charges them \emph{prices} $\{\pricej'\}_{j \in \buyers}\in\RR^{|\buyers|}$.
     %It generates total revenue $\sum_{j\in\buyers}\pricej'$.

%\vspace{-0.1in}
    \item \emph{Revenue redistribution}: \hspace{-0.05in}
    \label{itm:revenueredistribution}
    The broker redistributes the total revenue $\sum_{j \in \buyers} \pricej'$ from buyers via \emph{payments} $\{\payi'\}_{i \in \contributors} \in \RR^{|\contributors|}$  to contributors.
    %\ac{why does the notation for (4) and (5) involve ``primes''? to not conflict with function notation under ``mechanism'' section?}
\end{enumerate}




\subparahead{Problem instance}
%We will assume that the variance of $P$ is \emph{at most} $\sigma^2$.
Let  $\cost \defeq \{\costi\}_{i \in \contributors} \in \RR_+^{|\contributors|}$ denote all contributor costs, and $\valfunc \defeq \{\valfuncj\}_{j \in \buyers}$ denote all buyer valuations.
An instance of this market design problem is specified by the tuple $(\contributors, \buyers, \cost, \valfunc, \sigma^2)$,  which we assume is \emph{public information},
i.e. known to the broker and all contributors and buyers.
Neither the contributors, buyers, or broker has any additional information (e.g. a prior on $\mu$).



\subparahead{Notational convention}
We will use primed quantities to denote specific realizations of quantities that are otherwise functions of the reported dataset.
For instance, a contributor's payment $\payj$ will be a function of the reported data, but above $\payj'\in\RR$ denotes the specific amount paid to the contributor.
The amount charged to a buyer for a dataset is referred to as the \emph{price} denoted $\pricej,\pricej'$, while the amount given to a contributor for their work is called a \emph{payment}, denoted $\payj, \payj'$.

We will define three types of (related) valuation functions for buyers.
First the non-increasing function $\valfuncj(e')$ is buyer $j$'s valuation as a function of her estimation \emph{e}rror $e'$.
Second, $\valmechj(M,s)$ is $j$'s valuation as a function of the \emph{m}echanism $M$ and contributor \emph{s}trategies $s$---we will define this in~\eqref{eqn:valb}.
Third, $\valdataj(m)$ is $j$'s valuation as a function of the \emph{d}ataset size $m$ when the reported data is \emph{truthful}---we have defined this in~\eqref{eqn:valdataj};
in particular, $\valdataj$ is related to the valuations in~\S\ref{sec:buyerside}.

Finally, we will denote the contributor and buyer utilities by $\utilci(M, s)$ and $\utilbi(M,s)$ 
respectively, in a mechanism $M$ under contributor strategies $s$---we will define them in~\eqref{eqn:utilc} and~\eqref{eqn:utilbone} respectively.
%We will also use male pronouns for the broker




\parahead{Mechanism}
A broker's mechanism determines how much data each contributor should collect, 
the buyers' dataset sizes, buyers' prices, and contributors' payments.
To define this, let $\dataspace = \bigcup_{\ell=0}^\infty \RR^\ell$
be the space of datasets that each contributor may submit.
A mechanism is specified by the following tuple of (possibly randomized) quantities
$M = \left(\{\reqamnti\}_{i\in\contributors}, \{\mmj\}_{j \in \buyers}, \{\pricej\}_{j \in \buyers}, \{\payi\}_{i \in \contributors}\right)
$.
Here $\reqamnti\in\NN$ is an \emph{specification} to contributor $i$ on how much data she should collect (contributors need not follow this specification).
Next,  $\mmj \in \NN$
is the dataset size for buyer $j$,  which satisfies $\mmj\leq\sum_{i\in\contributors}\reqamnti$.
Finally,
$\;\pricej : \dataspace^{|\contributors|} \to \RR$ is the pricing rule for buyer $j$, and  
$\payi : \dataspace^{|\contributors|} \to \RR$ is the payment rule for contributor $i$.
These quantities may depend on the public information $(\contributors, \buyers, \cost,\val, \sigma^2)$, but we have suppressed this dependence.

To incentivize truthful reporting from contributors, we would like to pay them based on the quality of the data they have submitted  and not just the amount of data they report;
if the payment only depended on the size of the reported datasets, contributors would find it advantageous to fabricate data.
Hence contributor payments $\payi$ should depend on the reported datasets.
To  ensure the market remains feasible, we should also allow the prices $\pricej$ to also depend on the reported datasets.
%\ac{contributors can have different collection costs so we allow for different payment rules $\pi_i$; but do buyers have different problem parameters? if not wouldn't they share the same selection and pricing functions?}. 

% In addition to designing a mechanism, the broker will also specify data collection amounts $\{\nnopti\}_{i\in\contributors}$
% for each contributor to maximize profit, based on the public information $(\contributors, \buyers, \cost,\val, \sigma^2)$.

Due to the stochastic nature of data collection and the mechanism's randomness, prices and payments will also be random. We allow \emph{ex post} (i.e. stochastic) negative prices (where the broker pays a buyer) and negative payments (where a contributor pays the broker). Although unconventional, this enables the design of mechanisms with strong guarantees. Notably, in our mechanism, the \emph{ex-ante} (i.e. expected) prices and payments are always non-negative.


% The broker's mechanism should determine the buyers' estimates, the buyers' prices, and contributors' payments, based on the 
% datasets reported by the contributors.
% Hence, a \emph{deterministic} mechanism for this problem can be specified via 
% $\left(\{\muhatj\}_{j\in\buyers}, \{\pricej\}_{j\in\buyers}, \{\payi\}_{i\in\contributors}\right)$.
% Here, $\muhatj:\Dcal^{|\contributors|}\rightarrow \RR$ is the estimator
% for buyer $j$,
% $\pricej:\Dcal^{|\contributors|}\rightarrow \RR$ is the pricing rule for buyer $j$,
% and $\payi:\Dcal^{|\contributors|}\rightarrow \RR$ is the payment rule for contributor $j$.
% A randomized mechanism $M$ is a distribution over all possible values for $\left(\{\muhatj\}_{j\in\buyers}, \{\pricej\}_{j\in\buyers}, \{\payi\}_{i\in\contributors}\right)$, from which a sample is drawn and executed at run time.


\parahead{Contributors}
After the mechanism is published, each contributor decides how much data to collect and what to submit to the broker. A contributor $i$'s strategy is a possibly randomized pair $\strati = (\colamnti, \subfunci)$. First, the contributor samples $\colamnti \in \NN$ points to collect an initial dataset $\initdatai = \{x_{i,j}\}_{j=1}^{\nni}$, incurring a cost of $\costi \nni$. Here, each $x_{i,j}$ is drawn independently from the unknown normal distribution
$\Ncal(\mu,\sigma^2)$.
She then submits $\subdatai = \{y_{i,j}\}_j = \subfunci(\initdatai)$, where $\subfunci : \dataspace \to \dataspace$ maps the collected dataset to a possibly altered dataset of potentially different size.  
Let $\strat = \{\stratj\}_{j \in \contributors}$ denote the strategy profile of all contributors, and $\stratmi = \{\stratj\}_{j \in \contributors \setminus \{i\}}$ the strategies of all contributors except $i$.
%A \emph{randomized (mixed) strategy} $\strati$ is a distribution over $(\nni, \subfunci)$ pairs, from which a sample is drawn and executed at runtime.
%\subparahead{Strategy profiles and generating process}
%Let $\strat = \{\stratj\}_{j \in \contributors}$ denote the strategy profile of all contributors, and $\stratmi = \{\stratj\}_{j \in \contributors \setminus \{i\}}$ the strategies of all contributors except $i$.


\subparahead{Strategic behaviors}
A strategic contributor may not collect the amount specified by the broker, i.e. $\colamnti$ may differ from $\reqamnti$.
She may under-collect to reduce costs.
To provide reliable mean estimates for buyers, 
the broker wishes to incentivize contributors to submit data truthfully, \ie 
use the identity function $\subfunci=\identity$ which maps the dataset to itself.
However, in naive mechanisms, contributors may benefit from misreporting, i.e., using $\subfunci \neq \identity$ to fabricate or falsify data. 
Our formalism allows contributors to modify their submission based on their collected dataset $\initdatai$. For instance, a contributor could collect a small dataset ($\colamnti \ll \reqamnti$), fit a normal distribution to it, and sample additional synthetic points to mislead the broker into believing a larger dataset was collected.  




% \subparahead{Strategy profiles and generating process}
% Let $\strat = \{\stratj\}_{j \in \contributors}$ denote the strategy profile of all contributors, and $\stratmi = \{\stratj\}_{j \in \contributors \setminus \{i\}}$ the strategies of all contributors except $i$. We use $\genstrat$ to represent the generating process for the combined dataset $\subdata \defeq \{\subdatai\}_{i \in \contributors}$ received by the broker (see~\eqref{eqn:err}). This process accounts for each contributor sampling $\nni$ i.i.d. points $\initdatai$ and submitting $\subdatai = \subfunci(\initdatai)$.
% In particular, $\subdata$ may not be i.i.d. if any contributor uses $\subfunci \neq \identity$.


\subparahead{Contributor utility} 
If a contributor samples $\nni$ points and receives payment $\payi' \in \RR$, her \emph{ex post} utility is $\payi' - \costi \nni$.
As these quantities depend on the mechanism $M$ and the contributors' strategies $\strat$,
her \emph{ex ante} utility $\utilci$
% is defined as (with a slight abuse of notation),
% \begin{align*}
%     \utilci(\mech, \strat; \mu)
%         \defeq \EE_{M, \strat, \mu}\left[ \payi\big(\{\subfunci(\initdatai)\}_{i\in\contributors}\big)
%                 - \costi \nni\right],
% \hspace{0.4in}
%     \utilci(\mech, \strat) \defeq \inf_{\mu\in\RR} \utilci(\mech, \strat; \mu)
%           \numberthis  \label{eqn:utilc}
% \end{align*}
is defined as,
\begin{align*}
    \utilci(\mech, \strat) \defeq \inf_{\mu\in\RR} \EE_{M, \strat, \mu}\left[ \payi\big(\{\subfunci(\initdatai)\}_{i\in\contributors}\big)
                - \costi \nni\right].
                \numberthis\label{eqn:utilc}
\end{align*}
Here, the expectation $\EE_{M, \strat, \mu}$ is taken over the randomness in the mechanism, the contributor strategies  $\strat = \{(\nni, \subfunci)\}_{i \in \contributors}$, and the stochasticity of i.i.d. datasets $\{\initdatai\}_{i \in \contributors}$ drawn from $\Ncal(\mu, \sigma^2)$.  
We consider the infimum for a similar reason to~\eqref{eqn:valerr}. Since the true
distribution $\Ncal(\mu,\sigma^2)$ is unknown to the contributor \emph{a priori}, a contributor's strategy should perform well for all values of $\mu$. %Without this, the problem is poorly defined.  
To illustrate further consider an arbitrary $\mutilde\in\RR$. Suppose a contributor chooses to not collect any data (incurring zero cost) and report a large vector consisting of multiple copies of $\mutilde$, i.e.  
$(\mutilde, \dots, \mutilde)$. This strategy benefits the buyers, as they could obtain arbitrarily good
approximations of $\mu$ using the sample mean.
%\rcomment{Do you mean buyers get better estimator than sample mean}
As a result, buyers will be willing to pay more, which in turn increases the contributor's expected payment, without incurring any cost. 
However, such a strategy is only feasible if the contributor \emph{a priori} knew that $\mu = \mutilde$. Taking the infimum over all $\mu$ accounts for the fact that $\mu$ is unknown, ensuring the problem is well-defined\footnote{An alternative approach to account for uncertainty of $\mu$ 
is to consider a prior over $\mu$ and take expectation with respect to this prior. While we do not study this Bayesian setting here, our high-level techniques can also be applied in that context.}.
%That said, we will see that for our mechanism $\mechopt$, under a set of recommened contributor strategies $\stratopt$, $ \utilci(\mechopt, \stratopt; \mu)$ will not
%depend on $\mu$.
%\rcomment{Do you mean $ \utilci(\mechopt, \stratopt)$ does not depend on $\mu$}





\parahead{Buyers}
Buyers receive data from the mechanism, pay the specified price to the broker, and then estimate $\mu$ via the sample mean of the dataset they received.
%Buyers wish to achieve small error, as defined in~\eqref{eqn:err}.
Different buyers may have differing valuations for varying errors, modeled by buyer-specific non-increasing functions $\valfuncj: \RR_+ \rightarrow [0,1]$,
where $\valfuncj(e')$ is her \emph{ex-post} value if her estimation error is $e'$.
Her value $\valfuncj$ in a mechanism $M$ under a strategy profile $\strat$ can be written as (with slight abuse of notation),
\begin{align}
\hspace{-0.07in}
    &\valmechj(M, \strat) \defeq \valmechj(\mmj, s) \defeq 
        \inf_{\mu\in\RR} \EE_{M, \strat, \mu}\left[ \valfuncj\left( 
       \left|\,\muhat\big(\genstratmj( \{\initdatai\}_{i\in\contributors} \big) - \mu\,\right| \right)
       % \mmj\left(\textstyle\bigcup_{i\in\contributors} \subfunci(\initdatai)\right), \genstrat \right)
       % - \pricej\left( \{ \subfunci(\initdatai)\}_{i\in\contributors}\right)
       \right].
    \hspace{-0.03in}
    \label{eqn:valb}
\end{align}
Here, $\genstrat$ is the ``reporting function'' (see~\eqref{eqn:valerr})
induced by the
contributor strategies $\strat = \{(\nni, \subfunci)\}_{i\in\contributors}$ and the mechanism $M = \left(\{\mmj\}_{j \in \buyers}, \{\pricej\}_{j \in \buyers}, \{\payi\}_{i \in \contributors}\right)$;
this accounts for each contributor sampling $\nni$ i.i.d. points $\initdatai$, submitting $\subdatai = \subfunci(\initdatai)$, and
the mechanism then sampling an independent subset of size
$\mmj\left( \{ \subfunci(\initdatai)\}_{i\in\contributors}\right)$ from the
the combined dataset of all contributors.
Similar to~\eqref{eqn:valerr} and~\eqref{eqn:utilc}, the infimum is necessary for $\valj$ to be well-defined as $\mu$ is unknown.

\subparahead{Buyer utility}
Defining a buyer's utility is a bit tricky in our setting and infimum and expectations do not commute.
For this, we may consider two options shown below in~\eqref{eqn:utilbone} and~\eqref{eqn:utilbtwo}:
%\ac{using $g$ and $f$?}
\begin{align}
\hspace{-0.07in}
    &\utilbj(M, \strat) \defeq
        \utilbj(\mmj, \pricej, \strat) \defeq
        \valmechj(\mmj, \strat) - \sup_{\mu\in\RR} \EE_{M, \strat, \mu}\left[ 
       \pricej\left( \{ \subfunci(\initdatai)\}_{i\in\contributors}\right)
       \right]. 
    \hspace{-0.03in}
    \label{eqn:utilbone} \\
    &\utilbj(M, \strat) \defeq
        \utilbj(\mmj, \pricej, \strat) \defeq
        \inf_{\mu\in\RR} \EE_{M, \strat, \mu}\left[ \valfuncj\left( 
       \left|\,\muhat\big(\genstratmj( \{\initdatai\}_{i\in\contributors} \big) - \mu\,\right| \right)
       % \mmj\left(\textstyle\bigcup_{i\in\contributors} \subfunci(\initdatai)\right), \genstrat \right)
       - \pricej\left( \{ \subfunci(\initdatai)\}_{i\in\contributors}\right)
       \right].
    \hspace{-0.03in}
    \label{eqn:utilbtwo}
\end{align}
Here,~\eqref{eqn:utilbone} captures the difference between the buyer's worst-case expected value (see~\eqref{eqn:valb}) and worst-case expected price, while~\eqref{eqn:utilbtwo} considers her \emph{ex-post} utility, $\valfuncj(e') - \pricej'$, for error $e'$ and price $\pricej'$, and then takes the worst-case expectation.  
Clearly,~\eqref{eqn:utilbone} $\leq$ \eqref{eqn:utilbtwo}, so guarantees on~\eqref{eqn:utilbone} extend to~\eqref{eqn:utilbtwo}. Thus, we adopt~\eqref{eqn:utilbone} as the utility measure.  
That said in our mechanism, \emph{both quantities are equal}, as the expected price is independent of the underlying distribution $\Ncal(\mu, \sigma^2)$.  

\subparahead{Buyer valuation for i.i.d data}
For what follows, we define the buyer's valuation $\valdataj$ as a function of the amount of i.i.d data she receives. We have
\begin{align}
    \label{eqn:valdataj}
    \valdataj(m) = \inf_{\mu\in\RR} \EE_{X\sim\Ncal(\mu,\sigma^2)^m}\left[
        \valfunc\left(\left|\,\muhat(X) - \mu\,\right|\right)
    \right].
\end{align}
Comparing this with~\eqref{eqn:valb}, we see that for any mechanism 
$M = \left(\{\reqamnti\}_{i\in\contributors}, \{\mmj\}_{j \in \buyers}, \{\pricej\}_{j \in \buyers}, \{\payi\}_{i \in \contributors}\right)
$, we have
$\valmechj(M, \{(\reqamnti, \identity)\}) = \valdataj(\mmj)$.
That is, buyers achieve $\valdata(\mmj)$ when contributors collect the instructed amounts and report truthfully (recall that the mechanism always chooses $\mmj\leq \sum_{i\in\contributors} \reqamnti$.)
In particular, note that $\valdataj$ is the same as the buyers' quantity-based utility we studied in~\S\ref{sec:buyerside}.

% Here,~\eqref{eqn:utilbone} considers the difference between the buyer's worst-case expected value (see~\eqref{eqn:valb}) and worst-case expected price,
% while~\eqref{eqn:utilbtwo} considers her
% \emph{ex-post} utility $\valfuncj(e') - \pricej'$ for errors $e'$ and price $\pricej'$, and
% considers the worst-case expected utility.
% Clearly,~\eqref{eqn:utilbone}$\leq$\eqref{eqn:utilbtwo}, and hence any guarantees on~\eqref{eqn:utilbone} translate to~\eqref{eqn:utilbtwo}. For this reason, we will take the utility to be~\eqref{eqn:utilbone}.
% That said, in our mechanism \emph{both quantities will be equal}, since the expected price
% will not depend on the underlying distribution $\Ncal(\mu, \sigma^2)$.


% If the buyer's estimation error is $e'\in\RR_+$ and her price is $\pricej'\in\RR$, her
% \emph{ex-post} utility is $\valfuncj(e') - \pricej'$.
% These quantities depend on the dataset size selection rule $\mmj$ and pricing rule $\pricej$ the mechanism $M$ has chosen for the buyer $j$,
% and the contributors' strategies $\strat$.
% Therefore, we can write
% her expected utility as (with a slight abuse of notation), 
% \begin{align}
% \hspace{-0.07in}
%     &\utilbj(M, \strat) \defeq
%         \utilbj(\mmj, \pricej, \strat) \defeq
%         \inf_{\mu\in\RR} \EE_{M, \strat, \mu}\left[ \valfuncj\left( 
%        \left|\,\muhat\big(\genstratmj( \{\initdatai\}_{i\in\contributors} \big) - \mu\,\right| \right)
%        % \mmj\left(\textstyle\bigcup_{i\in\contributors} \subfunci(\initdatai)\right), \genstrat \right)
%        - \pricej\left( \{ \subfunci(\initdatai)\}_{i\in\contributors}\right)
%        \right].
%     \hspace{-0.03in}
%     \label{eqn:utilb}
% \end{align}

% \ac{why is $g$ called a reporting function? Is $g$ analogous to the allocation function in Yiding's paper?}.
% Similar to~\eqref{eqn:utilc},
% the infimum is necessary for the utility to be well-defined:
% the reported data, and consequently the prices and dataset sizes a buyer receives may depend on the underlying unknown distribution $\Ncal(\mu,\sigma^2)$.
% We wish to design a mechanism $M$ and incentivize strategies $s$ which achieve high buyer utilities on all possible distributions.
% As in~\eqref{eqn:utilc}, in our mechanism, a buyer's utility will not depend on $\mu$.

% we have

% A buyer's \emph{utility} $\utili$ is her value derived from
% her eserror minus her payment.


% A buyer $j$'s estimation error depends on the size of the dataset $\mmj'$
% provided by the mechanism, the contributors' strategies $\strat = \{(\nni, \subfunci)\}_{i\in\contributors}$ who decide how much data to collect and what to submit, as well as the estimator she uses.
% Hence, a buyer $j$'s \emph{valuation} $\valj$ is
% $\valfuncj(\errgen(\muhatj, \genstrat, \mmj'))$.
% % \begin{align}
% %     \valj(\mmj', \genstrat) \defeq
% %     \valfuncj(\errgen(\muhatj, \genstrat, \mmj')).
% %     \label{eqn:buyerval}
% % \end{align}

% % In particular, $\subdata$ may not be i.i.d. if any contributor uses $\subfunci \neq \identity$.

% \subparahead{Buyer utility}
% A buyer's \emph{utility} $\utili$ is her valuation minus her payment.
% This depends on the dataset size selection rule $\mmj$ and pricing rule $\pricej$ the mechanism $M$ has chosen for the buyer $j$,
% the contributors' strategies $\strat$,
% and the buyer's estimator $\muhatj$.
% %and possibly the distribution $P$.
% We can write this as
% %We have,
% \begin{align}
% \hspace{-0.07in}
%     &\utilbj(M, \strat, \muhatj) \defeq
%         \utilbj(\mmj, \pricej, \strat, \muhatj) \defeq
%         \sup_{\mu\in\RR} \EE_{M, \strat, \mu}\left[ \valfuncj(\mmj\left(\textstyle\bigcup_{i\in\contributors} \subfunci(\initdatai)\right), \genstrat; P) - \pricej\left(\textstyle\bigcup_{i\in\contributors} \subfunci(\initdatai)\right) \right].
%     \hspace{-0.03in}
%     \label{eqn:buyerutil}
% \end{align}
% Note that the contributors' strategies affect both, how much data each buyer
% will get via $\mmj$ and the actual dataset they receive, which is captured via the generating process $\genstrat$.

% In particular, if the contributors are submitting data truthfully,
% the RHS is equal to $\valfunci(\Var(P)/\mmj')$.

% \kkcomment{@KK: to see if the above can be simplified. }


% For reasons that we will explain shortly, we will assume that buyers will use the sample mean estimator.
% The following assumption, which we will make throughout this paper, states
% that untruthful reporting on the part of the contributors will not improve buyers' estimates.
% Recall the definition of $\errgen$ from~\ref{eqn:errgen} and~\eqref{eqn:err}.

% \begin{assumption}[Untruthful reporting can only hurt buyers]
% Let $\genprocess$ be the generating process when contributors
% follow some (possibly untruthful) strategy $\{(\nni, \subfunci)\}_{i\in\contributors}$,
% and let $\genprocess'$ be the generating process when they collect the same amount of data but submit it truthfully, \ie strategies $\{(\nni, \identity)\}_{i\in\contributors}$.
% Then, for all $m\in\NN$ and all distributions $P$ we have
% $\errgen(\muhatsm, \genprocess, m;P) \geq \errgen(\muhatsm, \genprocess, m;P) = \Var(P)/m $.
% \end{assumption}

\parahead{Cumulative profit}
If the contributors collect data in amounts
$\{\nni\}_{i\in\contributors} \in\NN^{|\contributors|}$ and the
buyers are charged prices $\{\pricej'\}_{j\in\buyers}\in\RR^{|\buyers|}$, the 
cumulative profit for contributors is $\sum_{j\in\buyers} \pricej' - \sum_{i\in\contributors}\costi\nni$.
Hence, the expected cumulative profit is given by,
\begin{align*}
   \hspace{-0.2in} \profit(M, s; \mu) = 
    \EE_{M, \strat, \mu}\Bigg[
        \sum_{j\in\buyers} \pricej\left( \{\subfunci(\initdatai)\}_{i\in\contributors}\right) -
            \sum_{i\in\contributors}\costi\nni
        \Bigg],
        \hspace{0.2in}
        \profit(M, s) = \inf_{\mu\in\RR} \profit(M, s; \mu).
        \hspace{-0.15in}
        \numberthis
        \label{eqn:profit}
        % \\
        % \ac{\text{should } \bigcup_{i\in\contributors} \subfunci(\initdatai) \rightarrow \cbr{f_i(X_i)}_{i\in\contributors}? \text{ o.w. we lose track of who submitted what?}}
\end{align*}
As before, the infimum accounts for the fact that $\mu$ is unknown. For an arbitrary mechanism, profit may depend on reported data, which itself depends on the actual distribution.
We wish for a mechanism to achieve high profit for all unknown distributions $\{\Ncal(\mu,\sigma^2): \mu\in\RR\}$, and not just do well for some specific distributions.
However, we will see that for our mechanism and contributor strategies, the expected cumulative profit $\profit(\mechopt, \stratopt; \mu)$ is independent of $\mu$.


\subsection{The mechanism design problem}
\label{sec:mechdesignproblem}


%\subparahead{Market constraints}
Our goal is to design a  mechanism $\mechopt = (\{\nnopti\}_{i\in\contributors}, \{\mmoptj\}_{j\in\buyers},
\{\priceoptj\}_{j\in\buyers}, \{\payopti\}_{i\in\contributors}\}$ to maximize the cumulative profit for contributors, while ensuring that all buyers and contributors do not lose by participation, charge fair prices to buyers, fully redistribute revenue from buyers to contributors,
and incentivize contributors to collect the instructed amounts $(\nni=\nnopti)$ and report it truthfully ($\subfunci=\identity$).
To state these requirements formally, let
 $\stratopt = \{(\nnopti, \identity)\}_{i\in\contributors}$ denote the ``well-behaved''
 contributor strategy profile.
 % agents collect the instructed amounts $\{\nnopti\}_{i\in\contributors}$ and report it truthfully.
The mechanism should satisfy the following requirements:
\begin{enumerate} [leftmargin=0.2in]
\label{mech:requirements}

\item \emph{Individually rational for buyers (IRB):}
$\mechopt$ is \emph{ex ante}  individually rational for the buyers if
$\utilbj(\mechopt, \stratopt) \geq 0$ for all buyers $j\in\buyers$.

\item \emph{Envy-free for the buyers (EFB):}
$\mechopt$ is envy-free for the buyers if no buyer prefers another buyer's dataset size selection rule and pricing rule to her own, i.e.
  $\utilbj(\mmoptj, \priceoptj, \stratopt) 
    \geq \utilbj(\mmoptk, \priceoptk, \stratopt)$ for all buyers $j,k\in\buyers$. %for all $k\in\buyers$.
   % \ac{see earlier comment about having different functions for the buyers}
    
\item \emph{Individually rational for contributors (IRC):}
$M$ is \emph{ex-ante} individually rational for the contributors if $\utilci(\mechopt, \stratopt) \geq 0$ for all contributors $i\in\contributors$.

\item \emph{Incentive-compatible for contributors (ICC):}
$\mechopt$ is incentive-compatible for contributors if $\stratopt$ is a Nash equilibrium. That is,  for all contributors $i\in\contributors$, and  for every alternative strategy $s_i$ for  $i$, we have $\utilci(\mechopt, \stratopt) \geq \utilci(\mechopt, (s_i, \stratoptmi))$.
(In~\S\ref{sec:hardness}, we show that no DSIC mechanisms are possible).


\item \emph{Budget balance (BB):}
$\mechopt$ satisfies budget balance if the total revenue 
is equal to total payments with probability 1,
i.e. for \emph{every realization of the data}
$\{\initdatak\}_{k\in\contributors}$ 
we have $\sum_{i\in\contributors} \payopti\left(\{\initdatak\}_{k\in\contributors}\right)
=
\sum_{j\in\buyers} \priceoptj\left(\{\initdatak\}_{k\in\contributors}\right)
$. (Note that contributors submit truthfully, i.e. $\subfunck=\identity$, in $\stratopt$).
%\ac{should be $\pi_i(M,s)? and p_j(M,s)$ instead?}

\item \emph{Profit optimality (PO):}
Let $\blprofit$ in~\eqref{eqn:blprofitone} and~\eqref{eqn:blprofittwo} be the optimum
envy-free revenue with non-strategic contributors, when using the valuation functions $\{\valdataj\}_{j\in\buyers}$ for i.i.d. data, as defined in~\eqref{eqn:valdataj}.
We say $\mechopt$ is $\epsilon$-PO if 
$\profit(\mechopt, \stratopt) \geq \blprofit - \epsilon$.
    
\end{enumerate}
In BB, we require total payments to equal total revenue. Payments cannot exceed revenue to ensure market feasibility.
We additionally require equality to ensure that trade benefits are fully distributed to contributors, leaving no unallocated excess revenue.
Unlike the other five requirements, BB must hold always and not just in expectation, as the market must be feasible for every realization of data.

\vspace{0.04in}
\noindent
We note that the baseline mechanism achieving OPT in~\eqref{eqn:blprofitone} satisfies IRB and EFB by design, as well as IRC and BB, since all revenue can be allocated to contributor 1. However, this mechanism is not ICC.
Intuitively, if a single agent collects all the data, she can fabricate it without detection, as the broker has no basis for verification.  
We prove this rigorously in Claim~\ref{lem:nic-under-single-collector}. 


% $\valdata(m) = \valmech(m, \{(\nni, \identity)\}_{i\in\contributors})$  whenever the total amount of data collected by the
% contributors is at least $m$, \ie $\sum_i\nni \geq m$.



% Next, we define a baseline $(M, s)$ for profit maximization, which our mechanism should approximate as closely as possible while satisfying the five requirements above.
% To define this baseline, first note that without assumptions on buyer behavior, profit could be arbitrarily large if the broker simply ask buyers to ``pay more'', even if this is not individually rational for buyers. To prevent this, we require that the baseline $(M, s)$ satisfy IRB.
% Second, if the baseline were not envy-free, it would sell all the buyers all the data collected by the contributors and charge them the maximum each buyer is willing to pay;
% while this satisfies IRB, it would not be fair (envy-free), as different buyers could pay different amounts for the same dataset.
% As an envy-free $(\mechopt,\stratopt)$ cannot compete with a non-envy-free baseline, we also require $(M, s)$ to satisfy EFB.
% Finally, one naturally expects buyers to obtain the most reliable estimates when contributors report truthfully, leading them to pay more. Hence, we assume contributors report truthfully in the baseline $(M, s)$, i.e.
% $s$ can be written as $\{(\nni, \identity)\}_{i\in\contributors}$ for some $\{\nni\}_i$ values.

% Let $\Theta$ be the set of all $(M, \{\nni\}_{i\in\contributors})$ tuples such that  
% $(M, \{(\nni, \identity)\}_{i\in\contributors})$
% satisfies IRB and EFB. The baseline profit $\blprofit$ is the maximum profit (see~\eqref{eqn:profit}) within this set:
% \kkcomment{explain more}
% \begin{align*}
%     % &\Theta = \big\{(M, (\{\nni\}_{i\in\contributors}); \;
%     %     (M, \{(\nni, \identity)\}_{i\in\contributors}) \text{ is IRB and EFB}
%     %     \big\},
%     %     \hspace{0.2in}
%     %     \\
%     &\blprofit \,=\, \max_{M, \{\nni\}_{i\in\contributors} \in \Theta}
%         \profit(M, \{(\nni, \identity)\}_{i\in\contributors})
%         \numberthis
%         \label{eqn:blprofit}
% \end{align*}



\subsection{Discussion of modeling choices}
\label{sec:modelingdiscussion}

% In~\S\ref{sec:buyerside}, we modeled the data marketplace in a similar fashion to how it is modeled commonly in the literature, by pricing the quantity of data (\eg~\citep{chen2023,chen2024learning,karimireddy2022mechanisms,agarwal2019marketplace}).
% However, this approach fails in the presence of strategic contributors.
% Our goal in~\S\ref{sec:setup} is to frame it as approximating this data-quantity-based market when
% contributors are strategic.

We conclude this section with a few observations on our modeling choices.


% \begin{itemize}[leftmargin=0.2in]
%     \item \emph{Knowledge of buyer valuations and contributor costs.}
    \subparahead{Knowledge of buyer valuations and contributor costs}
We assume buyer valuations $\val = \{\valj\}_{j\in\buyers}$ and contributor costs $\{\costi\}_{i\in\contributors}$ are public information. This allows us to focus on our key challenges: incentivizing truthful data contribution while adhering to market constraints, and analyzing its impact on buyer incentives. Extending our model to cases where these are unknown but can be elicited via bids or learned in repeated markets is left for future work.


%     % \item \emph{Nash vs dominant strategies}
%     \subparahead{Nash vs dominant strategies}
% The ICC condition requires that $\stratopt$ forms a Nash equilibrium (NE) in $\mechopt$. A natural question is whether a dominant-strategy incentive-compatible (DSIC) solution exists. However, as we show in Theorem~\ref{thm:dsic}, no mechanism has a dominant strategy equilibrium in which any agent is collecting non-zero data.

    % \item \emph{Better but untruthful Nash equilibria.}
    \subparahead{Better but untruthful Nash equilibria}
Our requirements in~\S\ref{sec:mechdesignproblem} assume that $\stratopt$ is a truthfully reporting NE. This is without loss of generality---if a non-truthful NE achieves higher profit, then one can obtain truth-telling NE with the same profit via the revelation principle.

    % \item \emph{Buyers using the sample mean.}
    \subparahead{Buyers using the sample mean}
    We assume that buyers estimate $\mu$ using the sample mean. While it is possible that other estimators could be optimal for different valuations $\vali$, general techniques for constructing optimal estimators based on specific loss/valuation functions remains an open problem in the statistics literature and are beyond the scope of this work.
    %Using the sample mean, which is the most natural estimator, 

    
    % \item \emph{Broker returning biased or corrupted data to buyers.}
    \subparahead{Broker returning biased or corrupted data to buyers}
    In~\S\ref{sec:env}, we assume that the broker sells an \emph{independent} subset of contributor submissions to buyers, with the sizes of the subsets given by $\{\mmj\}_{j\in\buyers}$ values. One might ask whether brokers could extract more revenue by biasing the selection or even corrupting the data. While this is technically possible, %we exclude this for two reasons:  
     % From a practical standpoint,
     such practices could undermine trust in the market if buyers suspect data manipulation by a broker. Returning an independent subset is less controversial.  
    % (ii) From a technical perspective, optimizing biasing or corruption strategies would require complex analysis beyond the scope of this work.

    % \item \emph{Impact of (un)truthful reporting on buyer valuations:}  
    \subparahead{Impact of (un)truthful reporting on buyer valuations} 
    We do not explicitly model the fact that truthfully reported data from contributors maximizes buyer valuations, even though this is generally expected in practice. Modeling how different types of untruthful behaviors could affect buyer valuations will be unwieldy, as it would require additional assumptions on $\vali$ and an analysis of optimal estimators for different $\vali$.
    Our approach instead is to define a profit-optimal baseline where 
    \emph{non-strategic} contributors \emph{always} follow the mechanism’s data collection and truthful reporting rules.
    %Our baseline maximizes profit under these constraints.
    Instead of explicitly modeling how untruthful reporting can help or hurt buyers, we frame our goal
    as designing a mechanism which \emph{incentivizes} strategic contributors to follow the broker's rules, while
    still  approximating the profit in the baseline.
    % Our goal is to design a mechanism to approximate this baseline even when contributors are allowed to act strategically.

% \end{itemize}





% \textcolor{red}{contents below are not needed}

% \textbf{Contributors: }There are $\numcontributors$ contributors, each capable of collecting a dataset $\data_i$ containing data points drawn independently and identically from a distribution $\distri \in \distrifamily$. Each contributor $i$ incurs a cost $\cost_i$ per data point collected.


% \textbf{Contributors’ strategy space: } Under this mechanism, contributors collect $\datanum_i$ data points, denoted as $X_i$, and then submit $\subdata_i = \function_i(X_i)$ to the mechanism, where $\function_i : \bigcup_{n\geq 0}^{}\mathbb{R}^n \to \bigcup_{n\geq 0}^{}\mathbb{R}^n$ is a function specific to the contributor that maps the collected dataset to a potentially fabricated dataset of a different size. The strategy space for each contributor is represented by $\strat_i = (\datanum_i, f_i)$, indicating how many data points to collect and how to report their data. 
% %Buyers then purchase data from the mechanism to estimate the true mean value $\mu$. 
% We write the strategy
% space of a contributor as $\mathcal{S}= \mathbb{N} \times \mathcal{F}$, where $ \mathbb{N}$ represents the space for number of data collected, and $\mathcal{F}$ is the space of functions mapping the dataset collected to the
% dataset submitted. 

% In the case of \textbf{Truthful submission}, contributors submit their data unaltered (“i.e., $f_i = \mathbf{I}$”).






% \textbf{Buyers: } The buyer population $\buyers$ consists of $\numtypes$ types of buyers. Each buyer $j \in \buyers$ has a non-decreasing valuation curve $v_j$, representing their value for the data points received. The valuation curves will be formally defined in Equation \ref{eq:buyervalue}.

% \textbf{Mechanism: } The mechanism takes as input the submitted data \( \subdata_1, \dots, \subdata_d \), and outputs:

% \begin{itemize}
%     \item An estimator $\hat{\mu}$ for the distribution mean. (\ref{itm:pricing})
%     \item Payments $\pi$ for buyers. (\ref{itm:pricing})
%     \item Revenue redistribution among contributors $p$. (\ref{itm:revenueredistribution})
% \end{itemize}

% The mechanism space is defined as:
% \[
% \mathcal{M} = \left\{ M = (p,  \pi, \hat{\mu}) \right\},
% \]
% where $p : \left(\bigcup_{n \geq 0} \mathbb{R}^n\right)^d \to \mathbb{R}^d$, $\pi : \left(\bigcup_{n \geq 0} \mathbb{R}^n\right)^d \to \mathbb{R}^\numtypes$, and $\hat{\mu}:  \left(\bigcup_{n \geq 0} \mathbb{R}^n\right)^d \to  \mathbb{R} $.

% \textbf{Buyer's ex-post valuation and utility: } The valuation of buyer $j$ is based on estimation error. Given contributors' strategy $\strat=(\strat_1,\dots, \strat_\numcontributors)$ and a mechanism $M$, buyer $j$'s estimation \emph{error} is:

% \begin{align}
%    \mathrm{error} =  \underset{p \in \distrifamily}{\sup} \; \mathbb{E}\left[ \rbr{\hat{\mu}_j-\mu }^2 \right] ,
% \end{align}

% and buyer $j$'s \emph{valuation} is a decreasing function of error:
% \begin{align} \label{eq:buyervalue}
%      \tilde{v}_j(M,s) = \phi_j(\mathrm{error}).
% \end{align}

% Where $\phi : R_{+}\to [0,1]$ is a decreasing function on $[0,\infty]$.

% For any buyer $j \in \buyers$, the \emph{utility} is defined as 
% \begin{align*}
%     \utilityb_j (M,s) = \tilde{v}_j(M,s) - \mathbb{E}\sbr{\pi_j(M,s)} .
% \end{align*}

% \textbf{Buyer's ex-ante valuation of dataset: }  Let $Z=(Z_1,\dots,Z_n)$ be a sample drawn from the distribution $\distri^n$. We denote  $\hat{\mu}(Z)$ as sample mean, i.e, $\hat{\mu}(Z) \overset{d}{=} \frac{\sum_i Z_i}{n}$. Buyer $j$’s valuation under truthful data is defined as: 

% \begin{align}
%    \mathrm{error} =  \underset{p \in \distrifamily}{\sup} \; \mathbb{E}\left[ \rbr{\hat{\mu}(Z)-\mu }^2 \right] ,
% \end{align}

% and buyer $j$'s valuation is a decreasing function of error:
% \begin{align} \label{eq:buyervalue}
%      v_j(n) = \phi_j(\mathrm{error}).
% \end{align}

% \textbf{Contributors' profit: }Contributors $i$'s profit
% \[ u_i(M,s) = \underset{\distrifamily}{\inf}\; \mathbb{E}\sbr{p_i(M,s)}- c\datanum_i . \]


% \textbf{Recommended strategies for the contributors: } The broker recommend strategy $\strat^{\star}_{i \in C}$ to each contributor to (\emph{i}) incentivize truthful contribution.




% %\[n^* =\max \sbr{ \arg\max_{n \in [N]} \left[ v(n) - \price(n) \right]}\]


