\section{Envy-free Profit Maximization with Non-strategic Contributors}
\label{sec:buyerside}

We first study a simplified setting where contributors are not strategic. Our goal is twofold:  
first, in~\S\ref{sec:profitbaseline}, we establish a profit-maximization baseline that considers only buyer incentives, ignoring contributor incentives.
In~\S\ref{sec:setup}, we will frame our mechanism design problem as approximating this baseline while accounting for strategic contributor behavior.  
Second, in~\S\ref{sec:nonstrategicalgo}, we design algoithms for profit maximization with non-strategic contributors. Later in~\S\ref{sec:mechanism}, we will build on these techniques when developing mechanisms for the strategic setting.  
Our ideas in~\S\ref{sec:buyerside} are not specific to mean estimation
and apply to general data marketplaces.  
We begin by describing the environment.  



\parahead{Data Marketplace}  
A data marketplace consists of three key actors:  
\emph{(i)} a finite set of data \emph{contributors} $\contributors = \{1,\dots,|\contributors|\}$,  
\emph{(ii)} a finite set of \emph{buyers} $\buyers$, and  
\emph{(iii)} a \emph{trusted broker} who facilitates transactions between contributors and buyers.  

\subparahead{Contributors}  
Contributors collect samples from the same but \emph{unknown} distribution. Contributor $i$ incurs a cost of $\costi$ to collect each sample.  
Without loss of generality, we assume that costs are ordered as $\cost_1 \leq \cost_2 \leq \dots \leq \cost_{|\contributors|}$.  
In this section, we assume contributors are non-strategic, meaning they
will collect the  amount of data requested by the broker and submit it truthfully.  


\subparahead{Buyers}  
Each buyer $i$ has a valuation function $\vali: \mathbb{N} \to [0,1]$, where $\vali(m)$ represents her value for $m$ data points.  
Since more data is preferable, $\vali(m)$ is non-decreasing in $m$.  
We will assume $\vali(0)=0$.
If a buyer pays price $\pricej \in [0,1]$ for $\mmj \in \mathbb{N}$ data points, her
utility is given by  
$\utilbj(\mmj, \pricej) = \valj(\mmj) - \pricej$.

\subparahead{Broker}  
The broker requests each contributor $i\in\contributors$ to collect $\reqamnti \in \mathbb{N}$ data points.  
Let $\reqtotal = \sum_{i\in\contributors} \reqamnti$ denote the total data collected.  
The broker then allocates
%\ac{maybe different word in place of allocate since buyers can receive overlapping sets?}
subsets of size $\{\mmj\}_{j\in\buyers} \in \{0,\dots,\reqtotal\}^{|\buyers|}$ from all contributors' submissions to each buyer, and charges them prices $\{\pricej\}_{j \in \buyers} \in [0,1]^{|\buyers|}$.  
We denote the broker's mechanism\footnote{To minimize jargon, we use the term \emph{mechanism} to describe the broker's design of $\big(\{\reqamnti\}_{i\in\contributors}, \{\mmj\}_{j\in\buyers}, \{\pricej\}_{j \in \buyers}\big)$, even though neither buyers nor contributors require incentives to report truthfully in this setting. Later, we will use the term mechanism again when incentivizing truthful reporting from contributors becomes necessary.  
} as  
$\efscheme = \big(\{\reqamnti\}_{i\in\contributors}, \{\mmj\}_{j\in\buyers}, \{\pricej\}_{j \in \buyers}\big)$.
% \subparahead{Cumulative profit}
The cumulative profit of this mechanism is given by,
$\efprofit(\efscheme) = \sum_{j \in \buyers} \pricej - \sum_{i \in \contributors} \costi\reqamnti$.
%\ac{why are valuations and payments $\in[0,1]$?}
%The cumulative profit of this mechanism is
% \begin{align*}
%     \efprofit(\efscheme) &= sum_{j\in\buyers} \pricej - \sum_{i\in\contributors} \costi \reqamnti.
%     \numberthis
%     \label{eqn:nonstrategicprofit}
%     %= \ExpectREVmech\left(\sum_{i\in\contributors} \reqamnti, \; \left(\{\mmj\}_{j\in\buyers}, \{\pricej\}_{j \in \buyers}\right) \right) \;-\; \sum_{i\in\contributors} \costi \reqamnti 
% \end{align*}



\subsection{A baseline for profit maximization}
\label{sec:profitbaseline}
%\label{sec:efrevenueoptimal}


\parahead{Envy-Free Profit Maximization}  
Given a set of buyers $\buyers$ along with their valuations $\{\valj\}_{j\in\buyers}$, and a set of contributors $\contributors$ along with their costs $\{\costi\}_{i\in\contributors}$,
the broker aims to design a mechanism $\efscheme$,
that maximizes profit.
Without any constraints this problem is poorly defined, as the broker can simply ask all buyers to ``pay more'' to maximize profit. We will require the buyers' prices $\{\pricej\}_{j\in\buyers}$ and dataset sizes $\{\mmj\}_{j\in\buyers}$ to satisfy the following two constraints:  
\begin{enumerate}[leftmargin=0.2in]
    \item \emph{Individually rational for buyers (IRB):}  
    $\efscheme$ is IRB if $\utilbj(\mmj,\pricej) \geq 0$ for all buyers $j \in \buyers$.
    \item \emph{Envy-free for buyers (EFB):}  
    $\efscheme$ is EFB if no buyer prefers another's allocation and price over her own, i.e.
    $\utilbj(\mmj, \pricej) \geq \utilbj(\mmk, \pricek) \quad \forall j,k \in \buyers$.
\end{enumerate}  
The first constraint ensures that buyers benefit from the market, as they would otherwise choose not to participate.  
The second constraint ensures fairness, as otherwise the broker could sell all $\numdata$ points collected to all buyers and charge each buyer $\pricej = \valitemj(\numdata)$.
This maximizes profit while satisfying IRB;
however, in this scheme different buyers would pay different prices for the same dataset, which would be unfair to buyers who have to pay more.  

We will refer to all $\{\pricej,\mmj\}_{j\in\buyers}$ values which satisfy
IRB and EFB as \emph{envy-free pricing schemes}.
Let $\efclass(N)$ denote all envy-free pricing schemes where the maximum amount of data sold is at most $N$, \ie $\max_j \mmj \leq N$, and let $\efclass = \bigcup_{N\geq 0}\efclass(N)$.

\subparahead{Optimal baseline profit}
The optimal baseline profit $\blprofit$, subject to the IRB and EFB constraints, can therefore be written as shown in~\eqref{eqn:blprofitone}.
Here, $\blprofit$ depends on the buyer valuations $\{\valj\}_{j\in\buyers}$ and contributor costs $\{\costi\}_{i\in\contributors}$, but we have suppressed this dependence for simplicity.
We have,
\vspace{-0.05in}
\begin{align*}
\numberthis \label{eqn:blprofitone}
    \blprofit &= \max_{\{\mmj, \pricej\}_{j} \in\efclass, \; \{\reqamnti\}_{i} \in\NN^{|\contributors|}  } \bigg(\; \sum_{j \in \buyers} \pricej - \sum_{i \in \contributors} \costi \reqamnti\;\bigg) 
\end{align*}
\vspace{-0.1in}

\subparahead{Revenue-optimal envy-free pricing schemes}
To better understand $\blprofit$, we first note that as buyer valuations only depend on the amount of data they receive (and not which contributors collected the data), the broker's decisions
for buyers' dataset sizes $\{\mmj\}_{j\in\contributors}$ and prices $\{\pricej\}_{j\in\contributors}$ need to depend on $\{\reqamnti\}_{i\in\contributors}$ 
only through their sum $\numdata = \sum_{i\in\contributors}\reqamnti$, i.e. the total amount of data collected.
%Hence, let us assume that the seller has received $N$ total data points.
The revenue $\ExpectREVmech$ for specific prices $\{\pricej\}_{j\in\buyers}$,
and the optimal revenue $\OPTREVmech$ under IRB, EFB constraints when the seller has received $\numdata$ total
data points can be written as shown below:
% in~\eqref{eqn:revoptmech}.
% Here, the prices $\{\pricej\}_j$ depend on $\{\mmj\}_j$  and $\numdata$ due to the IRB and EFB constraints.
%\ac{maybe we make the dependencies a bit more explicit?}
% in~\eqref{eqn:revoptmech}.
% Both $\ExpectREVmech$ and $\OPTREVmech$ depends on $\buyers$ and their valuations $\{\valj\}_{j\in\buyers}$, but we have suppressed this dependence for simplicity.
% We have:
\begin{align*}
    % \ExpectREVmech\left(\numdata, \{\mmj, \pricej\}_{j\in\buyers}\right) = {\sum_{j\in\buyers} \pricej},
    \ExpectREVmech\left(\{\pricej\}_{j\in\buyers}\right) = {\sum_{j\in\buyers} \pricej},
    \hspace{0.4in}
    \OPTREVmech(N) = \max_{\{\mmj, \pricej\}_{j\in\buyers} \in \efclass(\numdata)} \ExpectREVmech\left(\{ \pricej\}_{j\in\buyers}\right).
    \numberthis
    \label{eqn:revoptmech}
\end{align*}
\vspace{-0.1in}


\subparahead{From optimal revenue to optimal profit}
The following straightforward calculations lead to the following expression for the optimal envy-free profit $\blprofit$~\eqref{eqn:blprofitone}:
\begingroup
\allowdisplaybreaks
\begin{align*}
    \blprofit &= \max_{\{\mmj, \pricej\}_{j\in\buyers}\in\efclass, \{\reqamnti\}_{i}  } \left(\;\ExpectREVmech\left( \{\pricej\}_{j\in\buyers} \right) \;-\; \sum_{i\in\contributors} \costi \reqamnti  \;\right) \\
    &= \max_{\{\reqamnti\}_{i}  } \left(\max_{\{\mmj, \pricej\}_{j\in\buyers}\in\efclass(\sum_i \reqamnti)  }\;\ExpectREVmech\left(\{ \pricej\}_{j\in\buyers} \right) \;-\; \sum_{i\in\contributors} \costi \reqamnti  \;\right) 
    \\
    &= \max_{\{\reqamnti\}_{i} } \left(\; \OPTREVmech\left(\sum_{i\in\contributors} \reqamnti\right) 
    \;-\; \sum_{i\in\contributors} \costi \reqamnti \; \right)
    \; = \;\OPTREVmech\left(\optnone\right) - \costone\optnone. 
    \numberthis
    \label{eqn:blprofittwo}
\end{align*}
\endgroup
where $\optnone = \argmax_{N\in\NN} \OPTREVmech(N) - \costone N$.
The last step uses the observation that agent $1$ has the smallest cost.
Intuitively, the broker should ask the cheapest agent to collect all the data.

%Going forward, we will aim to approximate the baseline in~\eqref{eqn:blprofittwo} when contributors are strategic.
As we will see, we will not be able to achieve, or even be arbitrarily close to, $\blprofit$
when contributors are strategic and can misreport report data.
In particular, if the broker relies on a single agent to collect all the data, the agent can simply fabricate it, and the broker has no way to verify authenticity.
We formalize this rigorously through our hardness results in~\S\ref{sec:hardness}. 

% As we will see, achieving—or even approximating—$\blprofit$ is impossible when contributors are strategic and can misreport data.  
% If the broker relies on a single agent to collect all data, the agent can fabricate it, and the broker has no way to verify authenticity.  
% We formalize this rigorously through our hardness results in~\S\ref{sec:hardness}.  



% In the remainder of \S\ref{sec:buyerside}, we develop envy-free mechanisms to maximize profit.  
% In \S\ref{sec:efrevenueoptimal}, we assume the broker has collected $N$ points and design envy-free revenue-optimal mechanism.  
% Then, in \S\ref{sec:efprofitoptimal}, we build on this to develop envy-free profit-optimal mechanisms.



% Our first goal is to develop a mechanism for pricing (\ie without accountinf for contributor collection amounts) $\efrevscheme = (\{\mmj\}_{j \in \buyers}, \{\pricej\}_{j \in \buyers})$, that maximizes revenue by selling the data to buyers. 
% Letting $\Mcal$ denote all mechanisms which satisfy IRB and EFB (as defined above), 
% we can write the revenue of a mechanism $\ExpectREV$ and the optimal revenue $\OPTREVmech$ as 






\subsection{Designing envy-free profit-optimal mechanisms with non-strategic Contributors}
\label{sec:nonstrategicalgo}

We outline an algorithm for designing an approximately profit-optimal mechanism $M$ with non-strategic contributors. 
In~\S\ref{sec:mechanism}, we will build on the methods here with strategic contributors.



\parahead{Designing revenue-optimal envy-free pricing schemes}
Our first key insight here is that designing an envy-free pricing scheme reduces to constructing a revenue-optimal pricing curve (Lemma~\ref{lem:equal_two_problem}).
% \ac{move up the lemma?}. 
Here, the seller posts a pricing curve $\itemprice: \{0,\dots,N\} \to [0,1]$ with $\itemprice(0) = 0$, and buyers choose their purchase quantities to maximize utility  
(in contrast, in our setting, the broker directly determines dataset sizes and prices for each buyer).  
When buyers have monotonic valuations, revenue-optimal pricing aligns with the ordered item pricing problem~\citep{chawla2022pricing}, allowing us to leverage existing algorithms.  
We first define the ordered item pricing problem below.
To simplify the exposition, we do so in the context of data pricing\footnote{In~\citet{chawla2022pricing}, the seller has $\numdata \in \mathbb{N}$ items, and a set of \emph{unit-demand} buyers, i.e. wish to purchase only a single item, with non-decreasing valuations $\valitemi: [\numdata] \to [0,1]$. While buyers may have different valuations for the same good, their preference ranking is the same. The seller should choose a pricing function $\itemprice: [\numdata] \to \mathbb{R}_+$ to maximize revenue.}.


%\parahead{Ordered item pricing\textnormal{~\citep{chawla2022pricing}}}
\begin{definition}\label{def:Ordered_item_problem}(Ordered item pricing~\citep{chawla2022pricing})
Suppose the broker has $\numdata \in \mathbb{N}$ data points and posts a pricing curve $\itemprice$.  
Buyers then decide how much to purchase based on $\itemprice$.  
A utility-maximizing buyer $j$ selects $m \in \{0, \dots, \numdata\}$ to maximize utility, $\valitemj(m) - \itemprice(m)$.  
If multiple values of $m$ yield the same maximum utility, the buyer chooses the largest dataset.  
Thus, buyer $j$ purchases $\mjp(\numdata, \itemprice)$ data points, where  
\begin{align*}
% \mjpset(\numdata, \itemprice)
\mjp(\numdata, \itemprice) \defeq
% \argmax_{m\in[\numdata]}
\max\left\{\argmax_{m\leq \numdata}
\big(\,\valitemj(m) - \itemprice(m) \,\big)\right\} .
% \hspace{0.4in}
% \mjp(\numdata, \itemprice) \defeq \argmax_{m\in\mjpset} m.
    \numberthis
    \label{eqn:buyerpurchasemodel}
\end{align*}
It follows that the revenue from buyer $j$ is $\itemprice(\mjp(\numdata, \itemprice))$.
Hence, the total revenue $\ExpectREV(\numdata, \itemprice)$ of a pricing curve
$\itemprice$ is as defined below.
The broker wishes  to find a pricing
function $\optitemprice$ which  maximizes the
cumulative revenue.
The optimal revenue $\OPTREV$ can be defined as:
% \vspace{-0.075in}
\begin{align*}
    \ExpectREV(\numdata, \itemprice) \defeq
        \textstyle\sum_{j\in\buyers} \itemprice\left(\mjp(\numdata, \itemprice)\right),
     %   \ac{p\rightarrow q?, m_i\rightarrow\Tilde{m_i}?}
    \hspace{0.5in}
    \OPTREV(\numdata) \defeq \max_\itemprice \;\ExpectREV(\numdata, \itemprice).
    \numberthis
    \label{eqn:revenue}
\end{align*}
\end{definition}
\subparahead{From ordered item pricing to revenue-optimal envy-free mechanisms}
Given a pricing curve $\itemprice$ with $\numdata$ points, it is straightforward to construct an envy-free pricing scheme $\{\mmj,\pricej\}_{j\in\buyers} \in\efclass(\numdata)$  while achieving the same revenue as $\itemprice$, i.e.  
$\ExpectREV(\numdata, \itemprice) = \ExpectREVmech( \{\pricej\}_{j\in\buyers})$ (see~\eqref{eqn:revoptmech} and~\eqref{eqn:revenue}).  

To do so, we can set %$\efrevscheme = (\{\mmj\}_{j \in \buyers}, \{\pricej\}_{j \in \buyers})$ with 
$\mmj = \mjp(\numdata, \itemprice)$ and $\pricej = \itemprice(\mmj)$ (see~\eqref{eqn:buyerpurchasemodel}).  
Since each buyer selects data to maximize utility in~\eqref{eqn:buyerpurchasemodel}, it follows that  
$\utilbj(\mmj, \pricej) \geq \utilbj(\mm, \itemprice(\mm))$ for every other $\mm\in[N]$,
and in particular, for all other buyers' dataset sizes $\{\mmk\}_{k\neq j}$. This ensures EFB.  
Moreover, as buying no data ($\mmj = 0$ at $\pricej = 0$) yields zero utility, IRB is also satisfied. 

This also implies that the revenue of the optimal envy-free pricing scheme is at least as large as the revenue of the optimal pricing curve, i.e. $\OPTREVmech(\numdata) \geq \OPTREV(\numdata)$ (see~\eqref{eqn:revoptmech} and~\eqref{eqn:revenue}).  

Our next lemma shows the converse, thus establishing an equivalence between optimizing ordered item pricing curves and envy-free pricing schemes.  
Thus, we have $\OPTREVmech(\numdata) = \OPTREV(\numdata)$.  
The proof  is provided in~\S\ref{sec:proofequaltwoproblem}.  


\begin{lemma}%[ordered item pricing to revenue-optimal envy-free mechanisms] 
\label{lem:equal_two_problem}
Given the set of buyers $\buyers$, along with their valuations $\{\vali\}_{i\in\buyers}$.
Let $\numdata$ be the amount of data available to the broker.
For every envy-free pricing scheme $(\{\mmj\}_{j \in \buyers}, \{\pricej\}_{j \in \buyers}) \in\efclass(\numdata)$, there exists a 
pricing curve $\itemprice$ 
which achieves a higher revenue, \ie $\ExpectREV(\numdata, \itemprice) \geq \ExpectREVmech(\{\pricej\}_{j\in\buyers})$.
\end{lemma}
% \subparaheadwnogap{Designing mechanisms for revenue-optimal envy-free pricing}
% The above observations suggest the following procedure to design a mechanism:  
% Let $\algprice$ be the pricing curve returned by an ordered pricing algorithm with inputs $\buyers$, $N$, and approximation parameter $\eps$.  
% Define $\efrevscheme^A$ as the mechanism constructed using $\algprice$ as described above.  
% Then, $\ExpectREVmech(N, \efrevscheme^A) \geq \OPTREVmech(N) - \eps$.  

\insertTableApproxResults
\subparahead{Algorithms for ordered item pricing}
The ordered item pricing problem is known to be \textsf{NP}-hard~\citep{chawla2007algorithmic,hartline2005near}.  
As a result, prior work has developed approximation algorithms.  
An ordered item pricing algorithm takes as input the buyer valuations $\{\valj\}_{j\in\buyers}$, the total amount of data $\numdata$, and an approximation parameter $\eps$.  
It outputs a pricing curve $\itemprice$ such that the cumulative revenue satisfies  
$\ExpectREV(\numdata, \itemprice) \geq \OPTREV(\numdata) - |\buyers|\bigO(\epsilon)$.
Table~\ref{tb:approxresults} summarizes algorithms from prior work under various assumptions on buyer valuations.  

\vspace{0.5 cm}
\insertOIPAlgo
\parahead{Designing profit-optimal envy-free mechanisms}
The above observations suggest the following procedure to design an approximately profit-optimal mechanism, by leveraging an algorithm for ordered item pricing.  
For any given $\numdata\in\NN$, we apply $A$ to determine an approximately optimal pricing curve.  
To maximize profit, we assign all data collection to contributor 1 and optimize over $\numdata$ to maximize revenue minus $\cost_1 \numdata$.  

Algorithm~\ref{alg:oipalgo} outlines this procedure.  
It takes as input buyer valuations $\{\valj\}_{j\in\buyers}$, contributor costs $\{\costi\}_{i\in\contributors}$, an ordered item pricing algorithm $A$, and an approximation parameter $\eps$.  
It returns the optimal data quantity $\numdataA$, the corresponding pricing curve $\itemprice_{\numdataA}$, and dataset allocations $\{\mm_{\numdataA, j}\}_{j\in\buyers}$.  
Since buyer valuations are bounded in $[0,1]$, the maximum total buyer value is $|\buyers|$.  
Thus, it suffices to search for $\numdata \leq |\buyers|/\cost_1$.  
%
While our brute-force approach is admittedly inefficient;
we leave it to future work to develop more computationally efficient procedures to find the optimal $\numdata$.
The following lemma lower bounds the profit of Algorithm~\ref{alg:oipalgo}, with the proof given in Appendix~\ref{app:tech}.


\begin{restatable}{lemma}{Lempropfive}
    \label{lem:main_prop5}
    Suppose we execute Algorithm~\ref{alg:oipalgo} with an algorithm $A$ for ordered item pricing.
    Let $\numdataA$, $\itempriceA$, and $\{\mm_{\numdataA, j}\}_{j\in\buyers}$
    be the returned values. %\ac{inconsistent notation from what Alg 1 returns?}.
    Consider the envy-free mechanism $\efscheme = (\{\reqamnti\}_{i\in\contributors}, \{\mmj\}_{j\in\buyers},$ $ \{\pricej\}_{j\in\buyers})$ obtained by choosing
    $\reqamnt_1=\numdataA$, $\reqamnt_i=0$ for all $i\in\contributors\backslash\{1\}$,
    $\mmj = \mm_{\numdataA, j}$ for all $j\in\buyers$,
    and $\pricej = \itemprice_{\numdataA}(\mmj)$.
    We then have $\efprofit(\efscheme) \geq \blprofit-|\buyers|\bigO(\epsilon)$.
\end{restatable}


% \begin{proof}
% Let $\$
% We have,
% \begin{align*}
%     \efprofit(\efscheme) = \ExpectREVmech(\numdataA, \{\mmj, \pricej\}_{j\in\buyers}) - \costone\numdataA
% \end{align*}

%     We assume that 
%     \[\left\{ \tilde{\datanum}_{i} \right\}_{i\in \contributors} = \underset{\datanum_1,\dots,\datanum_d}{\arg \max}\rbr{\mathrm{rev}^{\star}  \rbr{ \buyers,\sum_i \datanum_i} - \sum_{i=1}^{d} \cost_i \datanum_i  },\]
%     then we have
%     \begin{align*}
%       \mathrm{profit}^{\star} &  = \underset{\datanum_1,\dots,\datanum_d}{\max}\rbr{\mathrm{rev}^{\star} 
%         \rbr{ B,\sum_i \datanum_i} - \sum_{i=1}^{d} \cost_i \datanum_i  } \\ & = \mathrm{rev}^{\star} 
%         \rbr{ B,\sum_i \tilde{\datanum}_i} - \sum_{i=1}^{d} \cost_i \tilde{\datanum}_i \\ & \leq \mathrm{rev}_A
%         \rbr{ B,\sum_i \tilde{\datanum}_i} - \sum_{i=1}^{d} \cost_i \tilde{\datanum}_i  + \eps  \\ & \leq  \mathrm{profit}_A(\datanum_1^{\star},\dots,\datanum_d^{\star}) + \eps.
%     \end{align*}
%     Here the first inequality is by the definition of $\eps$-revenue optimal algorithm: \[\mathrm{rev}_A
%         \rbr{ B,\sum_i \tilde{\datanum}_i} \geq \mathrm{rev}^{\star} 
%         \rbr{ B,\sum_i \tilde{\datanum}_i}-\eps,\]
%     and the second inequality is by equation~\eqref{eq:n_i_star}.
% \end{proof}

% and returns a mechanism $\efscheme$ such that
% $\efprofit(\efscheme) \geq \blprofit - |\buyers|\bigO(\epsilon)$.



%\begin{definition}[Ordered item pricing~\citep{chawla2022pricing}]



\subsection{Proof of Lemma~\ref{lem:equal_two_problem}}
\label{sec:proofequaltwoproblem}


We need to show that for any envy-free pricing scheme \(\left\{(\mm_j, \price_j)\right\}_{j \in \buyers}\), there exists a non-decreasing price curve \(\itemprice : \{0,1,\dots,\numdata\} \rightarrow [0, 1]\) that yields a revenue of at least $\sum_{j \in \buyers} \price_j$.
%
Without lose of generality, we assume $\mm_1 \leq \mm_2 \leq \dots \leq \mm_{|\buyers|}$. We define \(\pricecurv\) as follows:
\begin{align*}
    {\pricecurv}(\mm) \defeq
    \begin{cases}
        \price_{1}, & \text{if } \mm \leq \mm_1 \\
        \price_{2}, & \text{if } \mm_1 < \mm \leq \mm_2 \\
        \vdots \\
        %\pricecurv_{|\buyers| - 1}, & \text{if } \mm_{|\buyers| - 2} < \nn \leq \mm_{|\buyers| - 1} \\
        \price_{|\buyers|}, & \text{if } \mm_{|\buyers| - 1} < \mm \leq \numdata
    \end{cases}
\end{align*}
This price function \(\pricecurv\) is non-decreasing and has at most $|\buyers|$ steps. By the purchase model of ordered item pricing \eqref{eqn:buyerpurchasemodel}, each buyer $j$ would purchase $\mjp(\numdata, \itemprice)$ data points under price curve $\pricecurv$, where
\[\mjp(\numdata, \itemprice) \defeq
% \argmax_{m\in[\numdata]}
\max\left\{\argmax_{m\leq \numdata}
\big(\,\valitemj(m) - \itemprice(m) \,\big)\right\}\]


It is sufficient to show that $\mjp(\numdata, \itemprice) \geq \mmj$  holds for every buyer $j$. 
Then, as the price curve is non-decreasing,
the revenue from each buyer when using the pricing curve $q$ would be larger than her price in the envy-free pricing scheme. Therefore,
\[
\ExpectREV(\numdata, \itemprice) = \sum_{j \in \buyers}  \pricecurv(\mjp(\numdata, \itemprice)) \geq
\sum_{j \in \buyers}\pricecurv(\mmj) = \sum_{j \in \buyers}\price_j   = \ExpectREVmech(\{\price_j\}_{j\in\buyers}).
\]
which will complete the proof.

To show that $\mjp(\numdata, \itemprice) \geq \mmj$,
let us first consider an arbitrary \(\mm \leq \mm_{|\buyers|} \). For any \(\mm \leq \mm_{|\buyers|} \), let \(k \in [|\buyers|]\) be such that \(\mm_{k-1} < \mm \leq \mm_{k}\). 
As valuations are non-decreasing we have,
\begin{align*}
    v_j(\mm) - \pricecurv(\mm) &\leq v_j(\mm_{k}) - \pricecurv(\mm)
    = v_j(\mm_{k}) - \pricecurv(\mm_{k}) 
    \leq v_j(\mm_{j}) - \pricecurv(\mm_{j}).
\end{align*}
This implies that $\mm_j \in \underset{\mm \leq \mm_{|\buyers|}  }{\arg\max} \left( v_j(\mm) - \pricecurv(\mm) \right) $. 
%\kkcomment{Explain this last step a bit more. I had to think hard to understand what yo uwere saying.}
Finally, as $\mm_{|\buyers|} \leq \numdata$,  we have
\vspace{-0.1in}
\[
\mm_j \leq \max\Big(\underset{\mm \leq \mm_{|\buyers|}  }{\arg\max} \left( v_j(\mm) - \pricecurv(\mm) \right)\Big) \leq \max \Big(\underset{\mm \leq \numdata}{\arg\max} \left( v_j(\mm) - \pricecurv(\mm) \right) \Big)
= \mjp(\numdata, \itemprice).
\]

\vspace{-0.1in}
%By definition of $\mjp(\numdata, \itemprice)$, it further implies that $\mm_j \leq \mjp(\numdata, \itemprice)$.
\qed






%In~\S\ref{sec:contributorside}, we extend this approach to construct profit-optimal mechanisms when contributors are strategic.  



% \subsection{Profit-optimal Envy-free Mechanisms}

% Let us now return to the original question posed at the begireqamnting of~\S\ref{sec:buyerside},
% one how we may design a mechanism $\efscheme = \big(\{\mmj\}_{j\in\buyers}, \{\pricej\}_{j \in \buyers}, \{\reqamnti\}_{i\in\contributors}\big)$, including contributors data collection amounts,
% to maximize buyer profit.
% Recalling the definition of $\ExpectREVmech(\buyers,N,\efrevscheme)$ from~\eqref{eqn:revoptmech},
% we can write out the profit of an envy-free mechanisms $\efscheme$ with a set of buyers $\buyers$
% as follows,
% \begin{align*}
%     \efprofit(\efscheme) &= \sum_{j\in\buyers} \pricej - \sum_{i\in\contributors} \costi \reqamnti            = \ExpectREVmech\left(\sum_{i\in\contributors} \reqamnti, \; \left(\{\mmj\}_{j\in\buyers}, \{\pricej\}_{j \in \buyers}\right) \right) \;-\; \sum_{i\in\contributors} \costi \reqamnti 
% \end{align*}



%In~\S\ref{sec}


%\parahead{An algorithm for approximate Envy-free profit maximization}



% For this, let $A$ be an algorithm for revenue-optimal pricing 


% In the previous section, we focused on the buyers’ side of the problem, defining their purchase model under truthful data submission and discussing algorithms to achieve near-optimal revenue for the brokers. 
% In this section, we remove the truthfulness assumption on the contributors and design a mechanism for incentivizing them to truthfully submit their collected data.
% % contributors to truthfully submit their data.
% Our objective is to create a mechanism 
% that satisfies the propeties~\ref{mech:requirements}


% ensures, under the recommended strategy, the expected profit of buyers approximates the optimal profit, $ \mathrm{profit}^{\star}$, within an additive error $ \epsilon $. Here $ \mathrm{profit}^{\star} $ represents the maximum profit attainable by the contributors and is given by
% \begin{equation*}
%     \mathrm{profit}^{\star} = 
%     \underset{\datanum_1,\dots,\datanum_d}{\max}
%     \rbr{
%         \mathrm{rev}^{\star} 
%         \rbr{ 
%             \buyers,\sum_{i \in \contributors}\datanum_i
%         }
%         - \sum_{i=1}^{d} \cost_i \datanum_i  
%     }.
% \end{equation*}
% The broker will utilize the $\eps$-revenue optimal price algorithm to approximate the contributors' profit. We let $\mathrm{profit}_A( \datanum_1^{\star},\dots,\datanum_{|\contributors|}^{\star})$ denote the profit of contributors when contributor $i$ collects $\datanum_i$ points,
% \begin{equation} 
%     \label{eq:profit_A}
%     \mathrm{profit}_A(  \datanum_1,\dots, \datanum_{|\contributors|})
%     =
%     \mathrm{rev}_A \rbr{
%         \buyers, \sum_{i \in \contributors}\datanum_i
%     } 
%     -\sum_{i=1}^{d} \cost_i \datanum_i .
% \end{equation}
% Then the broker selects the set $\left\{ \datanum_i \right\}_{i \in \contributors}$ which maximizes the function $\mathrm{profit}_A(  \datanum_1,\dots, \datanum_{|\contributors|})$, i.e. the best profit achievable under the pricing algorithm $A$:
% \begin{align} \label{eq:n_i_star}
%     \{ \datanum_i^{\star} \}_{i=1}^d = \underset{\datanum_1,\dots,\datanum_d}{\arg\max}\ \mathrm{profit}_A( \datanum_1,\dots,\datanum_{|\contributors|}),
% \end{align}
         
% In Lemma 
% ~\ref{lem:main_prop5}, 
% % ~\ref{lem:profit_appro},
% we show that $\mathrm{profit}_A(  \datanum_1^{\star},\dots, \datanum_{|\contributors|}^{\star})$ approximates $\mathrm{profit}^{\star}$ within an additive error $\eps$,    




% The $\eps$-reveue optimal price algorithm $A$ takes as input  a population $buyers$ of buyers, a dataset of size $\sum_i \datanum_i^{\star}$, and an error
% margin $\eps$. The algorithm outputs a price curve $\pricecurv$, along with a set of \( \mm'_1, \mm'_2, \ldots, \mm'_{|\buyers|} \), which represents the amount of data buyer \( i \) would purchase at the price curve $\pricecurv$. 

% To be specific, $\pricecurv$ is the price curve that achieves the revenue of $\mathrm{rev}_A
% \rbr{ \buyers,\sum_i {\datanum}_i^{\star}}$, and let $\pricecurv_j$ be defined as the price that a buyer with valuation $\valitemj$ pays at the price curve $\pricecurv$. For each buyer $j$, the optimal data size $\mmj'$ is determined by the following expression:

% \begin{align} \label{eq:buyer_allocate}
% \mmj'
% % & 
% = \max \left( \underset{n}{\arg\max} \left( \valitemj(n) - \pricecurv(n) \right) \right), 
% % \\
% \quad
% \pricecurv_j 
% % & 
% = \pricecurv \left( \mmj' \right).
% \end{align}

% Let $\profi$ denote the profit of contributors, 
% \begin{align}
%     \profi \defeq \mathrm{profit}_A(  \datanum_1^{\star},\dots, \datanum_{|\contributors|}^{\star})= \mathrm{rev}_A
% \rbr{ \buyers,\sum_{i \in \contributors} {\datanum}_i^{\star}}- \sum_{i \in \contributors} \cost_i \datanum_i^{\star}
% \end{align}