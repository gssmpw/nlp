% \section{Proof multiple buyer}
\section{Proof Theorem~\ref{thm:multi_thm}}
\label{app:main_theorem}
 
To prove Theorem~\ref{thm:multi_thm} we show that $\rbr{M,\strat^{\star}}$ satisfies the following five properties.
\begin{enumerate}
    \item Under $M$, $\strat^\star$ is a Nash Equilibrium for contributors. See Lemma~\ref{lem:main_prop1}.
    
    \item Under $M$, $\strat^\star$ is individually rational for both contributors and buyers. See Lemma~\ref{lem:main_prop2}.
    
    \item The mechanism $M$ satisfies the budget balance property. See Lemma~\ref{lem:main_prop3}.
    
    \item $M$ is envy-free for buyers. See Lemma~\ref{lem:main_prop4}.
    
    
    \item Under $M$, when the contributors follow following $\strat^\star$, the expected profit of buyers approximates the optimal profit, $\mathrm{profit}^{\star}$, with additive error $\epsilon$. See Lemma~\ref{lem:main_prop5}.
\end{enumerate}

\begin{lemma}
    \label{lem:main_prop1}
    Under $M$, $\strat^\star$ is a Nash Equilibrium for contributors. Formally, $\forall i\in\contributors$, 
    $\forall\rbr{\nni,\subfunci}\in$\ac{fill?}
    % \ac{unsure if $\subfuncspace$ has been previously defined?}
    \begin{equation*}
        \utilci\rbr{
            M, \rbr{
                (\nni,\subfunci),
                \strat_{-i}^{\star}
            }
        }
        \leq
        \utilci\rbr{
            M, \rbr{
                \strat_i^\star,
                \strat_{-i}^{\star}
            }
        } . 
    \end{equation*}
\end{lemma}
\begin{proof}
    Fix any $(\nni,\subfunci)$. Let $f_a:\dataspace\rightarrow\RR^{\datanum_i^{\star}}$ be a function such that $\forall \initdatai\in\dataspace$, $\muhat\rbr{f_a(\initdatai)}=\muhat\rbr{\subfunci(\initdatai)}$. Let $f_b:\dataspace:\rightarrow\RR^{\datanum_i^{\star}}$ be a function such that $\forall \initdatai\in\dataspace$, $\muhat\rbr{f_b(\initdatai)}=\muhat\rbr{\initdatai}$.
    % Fix any $(\nni,\subfunci)$. Let $f_a:\dataspace\rightarrow\dataspace$ be a function such that $\forall \initdatai\in\dataspace$, $\abr{f_a(\initdatai)}=\abr{\subfunci(\initdatai)}$ and $\muhat\rbr{f_a(\initdatai)}=\muhat\rbr{\initdatai}$. Let $f_b:\dataspace\rightarrow\RR^{\datanum_i^{\star}}$ be a function such that
    % $\forall \initdatai\in\dataspace$, $\muhat\rbr{f_b(\initdatai)}=\muhat\rbr{f_a(\initdatai)}$.
    We then have that
    \begin{align*}
        \utilci \rbr{M, \rbr{(\nni,\subfunci),\strat_{-i}^{\star} }}
        &\leq
        \utilci \rbr{M, \rbr{(\nni,f_a),\strat_{-i}^{\star} }}
        % \qquad
        &
        (\text{Lemma}~\ref{lem:NIC-part1})
        \\
        &\leq
        \utilci \rbr{M, \rbr{(\nni,f_b),\strat_{-i}^{\star} }}
        % \qquad
        &
        (\text{Lemma}~\ref{lem:NIC-part2})
        \\
        &\leq
        % \utilci \rbr{M, (\datanum_i^{\star},f_b),\strat_{-i}^{\star} }
        % =
        \utilci \rbr{M, \rbr{\strat_{i}^{\star},\strat_{-i}^{\star} }} .
        &
        % \qquad
        \hspace{-1cm}
        (\text{Lemma}~\ref{lem:NIC-part3})
    \end{align*}
\end{proof}

% \begin{lemma}
%     \label{lem:main_prop1-old}
%     Under $M$, $\strat^\star$ is a Nash Equilibrium for contributors.
% \end{lemma}
% \begin{proof}
%     Recall that contributor $i$'s payment is
%     \begin{align*}
%     & \pay_i (\mech,\strat_i,\strat_{-i}^{\star})  =   \mathbb{I}\left( | Y_i | = \datanum_i^{\star} \right) 
%     \bigg( 
%     \profi \frac{\datanum_i^{\star}}{N} + \cost_i \datanum_i  + d_i \frac{\sigma^2}{\datanum_{-i}^{\star}} + d_i \frac{\sigma^2}{ \datanum_i^{\star}} 
%     \bigg)  - d_i \bigg( \hat{\mu}(Y_i) - \hat{\mu}(Y_{-i}) \bigg)^2  
%     % \numberthis 
%     \end{align*}
%     where $d_i= \frac{\cost_i (\datanum_i^{\star})^2}{ \sigma^2}$. We prove the Nash equilibrium in three steps.
%     Step (1) shows that regardless of the amount of data collected and the size of the dataset submitted, it is best to have $\hat{\mu}(Y_i)=\hat{\mu}(X_i)$. Step (2) shows that regardless of the amount of data collected, the optimal number of points to submit is $\datanum_i^{\star}$. Step (3) Shows that when fixing the submission size to be $\datanum_i^{\star}$, it is best to collect $\datanum_i^{\star}$ points.
%     \ac{should this be reorganized into three lemmas which fit together (similar to )}

%     \ac{step 1. doesn't cover functions that submit variable amounts, shouldn't this be changed to regardless of the amount of data submitted (which could depend on $X_i$) its best to submit something that agrees with the sample mean?}
    
%     \textbf{Step 1}: Fix the number of data collected $n_i$ and the number of data submitted $\left| Y_i \right|$ by contributor $i$. According to Lemmas~\ref{lem:minimax} and~\ref{lem:same_minimax}, $ \hat{\mu}\left(X_i\right)$ is minimax estimator of $\mathbb{E}\sbr{\rbr{\hat{\mu}(Y_i)-\hat{\mu}(Y_{-i})}^2 \;\middle|\; P} $, i.e. 
%     \begin{align*}
%       \hat{\mu}(X_i) = \underset{\mu}{\arg\min} \sbr{\sup _\distrifamily \mathbb{E}\left[\sum_{j=1}^{m} d_i \rbr{ \hat{\mu}(Y_i) - \hat{\mu}(Y_{-i})}^2 \;\middle|\;  P \right] },
%     \end{align*}
%     \ac{why is there a sum in the expectation?}
    
%     As $n_i$ and $\left| Y_i \right|$ are fixed, this implies contributor $i$ maximizes their utility by selecting $\hat{\mu}(Y_i) = \hat{\mu}(X_i)$. 
    
%     \textbf{Step 2}: When fixing $n_i$, and assuming that contributor $i$ submit $\hat{\mu}(Y_i)=\hat{\mu}(X_i)$,  the optimal number of data points to submit is $\left| Y_i \right| = \datanum_i^{\star}$. Because the payment 
%     $\pay_i $ will be negative if $\left| Y_i \right| \neq \datanum_i^{\star}$. Therefore, for any $n_i$, we combine the first and second step:
%     \begin{align*}
%     & \utilci \rbr{\mech, (n_i,f_i \text{ s.t. } \hat{\mu}(Y_i)=\hat{\mu}(X_i), \left| Y_i \right| =\datanum_i^{\star}),\strat_{-i}^{\star} }  \geq  \utilci \rbr{M, (n_i,f_i),\strat_{-i}^{\star} }.
%     \end{align*}
    
    
    
%     \textbf{Step 3}: We choose constant $d_i= \frac{\cost_i (\datanum_i^{\star})^2}{ \sigma^2}$. When fixing $\hat{\mu}(Y_i)=\hat{\mu}(X_i)$ and $\left| Y_i \right| = \datanum_i^{\star}$, collecting $n_i = \datanum_i^{\star}$ amount of data maximize the contributor $i$'s utility. To see this, we write the contributor's utility given $\hat{\mu}(Y_i)=\hat{\mu}(X_i)$ and $\left| Y_i \right| = \datanum_i^{\star}$ as:
    
%     \begin{align*}
%         & \utilci \rbr{\mech, (n_i, f_i \text{ s.t }\hat{\mu}(Y_i)=\hat{\mu}(X_i), \left| Y_i \right| = \datanum_i^{\star}),\strat_{-i}^{\star} }  
%         \\ 
%         = & \rbr{w \frac{\datanum_i^{\star}}{N} + \cost_i \datanum_{i}^{\star}  + d_i \frac{\sigma^2}{\datanum_{-i}^{\star}} + d_i \frac{\sigma^2}{ \datanum_i^{\star}} } -    d_i \rbr{ \frac{\sigma^2}{\datanum_{-i}^{\star}} +\frac{\sigma^2}{\datanum_i} } - \cost_i n_i ,
%     \end{align*}
    
    
%     Plug in the value $d_i= \frac{\cost_i (\datanum_i^{\star})^2}{\sigma^2}$, it follows that $ d_i \frac{\sigma^2}{\datanum_i} + \cost_i \datanum_i \geq 2\cost_i \datanum_i^{\star}$, and this inequality takes equality if and only if  $\datanum_i = \datanum_i^{\star}$. Therefore, the utility is maximized when contributor $i$ collects $\datanum_i = \datanum_i^{\star}$ data points,
%     \begin{align*}
%         &\;\utilci \rbr{\mech, \strat_{i}^{\star} ,\strat_{-i}^{\star} }\\  = & \;\utilci \rbr{\mech, (\datanum_i^{\star},\mathbf{I}),\strat_{-i}^{\star} }\\  = & \;\utilci \rbr{\mech, (\datanum_i^{\star}, f_i \text{ s.t }\hat{\mu}(Y_i)=\hat{\mu}(X_i), \left| Y_i \right|=\datanum_i^{\star}),\strat_{-i}^{\star} }  \\ \geq &   \; \utilci \rbr{\mech, \strat_{i} ,\strat_{-i}^{\star}} .
%     \end{align*}
% \end{proof}

\begin{lemma}
    \label{lem:main_prop2}
    Under $M$, $\strat^\star$ is individually rational for both contributors and buyers.
\end{lemma}
\begin{proof}
    Under the recommended strategy, the utility for contributor $i$ is 
    \begin{align*}
        \utilci(M,\strat^{\star})
        &=
        \inf_{\mu\in\RR}
        \EE\Bigg[
            \mathbb{I}\left( \left| Y_i \right| = \datanum_i^{\star} \right) 
            \rbr{
                \profi \frac{\datanum_i^{\star}}{N} + \cost_i \datanum_i^{\star}  +  \frac{d_i\sigma^2}{\datanum_{-i}^{\star}} +  \frac{d_i\sigma^2}{ \datanum_i^{\star}} 
            }
            -  d_i\rbr{
                \muhat(\subfunci(\initdatai))-\muhat(\subdatami)
            }^2
            -\costi\nni
        \Bigg]
        \\
        &=
            \rbr{
                \profi \frac{\datanum_i^{\star}}{N} + \cost_i \datanum_i^{\star}  +  \frac{d_i\sigma^2}{\datanum_{-i}^{\star}} +  \frac{d_i\sigma^2}{ \datanum_i^{\star}} 
            }
            -\costi\nni
        -d_i \sup_{\mu\in\RR}
        \EE\Bigg[
          d_i\rbr{
            \muhat(\initdatai)-\muhat(\subdatami)
        }^2
        \Bigg]
        \\
        &=
            \rbr{
                \profi \frac{\datanum_i^{\star}}{N} + \cost_i \datanum_i^{\star}  +  \frac{d_i\sigma^2}{\datanum_{-i}^{\star}} +  \frac{d_i\sigma^2}{ \datanum_i^{\star}} 
            }
            -\costi\nni
        -\rbr{
            \frac{d_i\sigma^2}{\datanum_{-i}^{\star}} +  \frac{d_i\sigma^2}{ \datanum_i^{\star}} 
        }
        \\
        &=
        \profi \frac{\datanum_i^{\star}}{N} 
        \\
        &\geq 0 .
    \end{align*}

    Under the recommended strategy, the expected payment for buyer $i$ is
    \begin{align*}
        \mathbb{E}\sbr{\price_j(\mech,\strat^{\star})}   & = \;  \sum_{i \in \contributors} \mathbb{I}\left( \left| Y_i \right| = \datanum_i^{\star} \right) 
    \bigg( 
    \pricecurv_j \frac{\datanum_i^{\star}}{N} + \frac{d_i}{|\buyers|} \frac{\sigma^2}{\datanum_{-i}^{\star}} + \frac{d_i}{|\buyers|} \frac{\sigma^2}{ \datanum_i^{\star}} 
    \bigg)   - \sum_{i \in \contributors} \frac{d_i}{|\buyers|} \mathbb{E}\sbr{\bigg( \hat{\mu}(Y_i) - \hat{\mu}(Y_{-i}) \bigg)^2}  \\ & = \;    
    \bigg( 
    \pricecurv_j  + \sum_{i \in \contributors}  \frac{d_i}{|\buyers|} \frac{\sigma^2}{\datanum_{-i}^{\star}} +\sum_{i \in \contributors}  \frac{d_i}{|\buyers|} \frac{\sigma^2}{ \datanum_i^{\star}} 
    \bigg)   - \sum_{i \in \contributors} \frac{d_i}{|\buyers|} \rbr{ \frac{\sigma^2}{\datanum_i^{\star}} + \frac{\sigma^2}{\datanum_{-i}^{\star}} }  \\ & = \; \pricecurv_j .
    \end{align*}
        
    Therefore, buyer $j$'s utility is 
    \ac{connection to being non negative needs to be clearer, come back to this, wait for definitions to be written today?}\ac{keran said she'd do}
    \begin{align*}
         \mathbb{E}\sbr{\valj(\mech,\strat^{\star})} - \mathbb{E}\sbr{\price_j(\mech,\strat^{\star})}  
         &=   \;\mathbb{E}\sbr{\valj(\mech,\strat^{\star})} - \pricecurv_j + \pricecurv_j -\mathbb{E}\sbr{\price_j(\mech,\strat^{\star})} \\ &=  \; \mathbb{E}\sbr{\valj(\mech,\strat^{\star})} - \pricecurv_j   \\ &\geq  \; 0 .
    \end{align*}
\end{proof}

% \begin{proof}
%     Under the recommended strategy, the expected payment for contributor $i$ is

%     \begin{align*}
%          \mathbb{E} \sbr{\pay_i }  = 
%           \mathbb{I}\left( \left| Y_i \right| = \datanum_i^{\star} \right) 
%     \bigg(\profi \frac{\datanum_i^{\star}}{N} + \cost_i \datanum_i^{\star}  + d_i \frac{\sigma^2}{\datanum_{-i}^{\star}} + d_i \frac{\sigma^2}{ \datanum_i^{\star}} 
%     \bigg)   -  d_i \bigg( \frac{\sigma^2}{\datanum_i^{\star}} + \frac{\sigma^2}{\datanum_{-i}^{\star}} \bigg) = \profi \frac{\datanum_i^{\star}}{N}+ \cost_i \datanum_i^{\star}.
%     \end{align*}
    
%     We also have contributor $i$'s utility is
%     \begin{align*}
%           \utilci(M,\strat^{\star}) = \mathbb{E} [\pay_i] - \cost_i \datanum_i^{\star} =  \profi \frac{\datanum_i^{\star}}{N} \geq 0 .
%     \end{align*}
    
    
%     Under the recommended strategy, the expected payment for buyer $i$ is
%     \begin{align*}
%         \mathbb{E}\sbr{\price_j(\mech,\strat)}   & = \;  \sum_{i \in \contributors} \mathbb{I}\left( \left| Y_i \right| = \datanum_i^{\star} \right) 
%     \bigg( 
%     \pricecurv_j \frac{\datanum_i^{\star}}{N} + \frac{d_i}{|\buyers|} \frac{\sigma^2}{\datanum_{-i}^{\star}} + \frac{d_i}{|\buyers|} \frac{\sigma^2}{ \datanum_i^{\star}} 
%     \bigg)   - \sum_{i \in \contributors} \frac{d_i}{|\buyers|} \mathbb{E}\sbr{\bigg( \hat{\mu}(Y_i) - \hat{\mu}(Y_{-i}) \bigg)^2}  \\ & = \;    
%     \bigg( 
%     \pricecurv_j  + \sum_{i \in \contributors}  \frac{d_i}{|\buyers|} \frac{\sigma^2}{\datanum_{-i}^{\star}} +\sum_{i \in \contributors}  \frac{d_i}{|\buyers|} \frac{\sigma^2}{ \datanum_i^{\star}} 
%     \bigg)   - \sum_{i \in \contributors} \frac{d_i}{|\buyers|} \rbr{ \frac{\sigma^2}{\datanum_i^{\star}} + \frac{\sigma^2}{\datanum_{-i}^{\star}} }  \\ & = \; \pricecurv_j .
%     \end{align*}
        
%     Therefore, buyer $j$'s utility is 
%     \begin{align*}
%          \mathbb{E}\sbr{\valj(\mech,\strat^{\star})} - \mathbb{E}\sbr{\price_j(\mech,\strat^{\star})}  
%          &=   \;\mathbb{E}\sbr{\valj(\mech,\strat^{\star})} - \pricecurv_j + \pricecurv_j -\mathbb{E}\sbr{\price_j(\mech,\strat^{\star})} \\ &=  \; \mathbb{E}\sbr{\valj(\mech,\strat^{\star})} - \pricecurv_j   \\ &\geq  \; 0 .
%     \end{align*}
    
% \end{proof}

\begin{lemma}
    \label{lem:main_prop3}
    The mechanism $M$ satisfies the budget balance property.
\end{lemma}
\begin{proof}
    % Budget balance under recommended strategy:
    Summing the contributor payments, we get that
    \ac{keran said  missing definitions of $N:=\sum n^\star_i$, and how $W$ and $q_j$ are related; check def 2.5}
    \ac{keran said she'd do}

    \begin{align*}
        \sum_{i\in\contributors}\pay_i (M, \strat^{\star})
        & = \profi +  \sum_{i \in \contributors} \cost_i \datanum_i^{\star} +  \sum_{i \in \contributors} \rbr{ d_i \frac{\sigma^2}{\datanum_{-i}^{\star}} + d_i \frac{\sigma^2}{ \datanum_i^{\star}} } - \sum_{i \in \contributors} d_i \bigg( \hat{\mu}(Y_i) - \hat{\mu}(Y_{-i}) \bigg)^2 
        \\ 
        & = \sum_{j \in \buyers} \pricecurv_j + \sum_{j \in \buyers}  \sum_{i \in \contributors}  \frac{d_i}{|\buyers|} \frac{\sigma^2}{\datanum_{-i}^{\star}} +\sum_{j \in \buyers} 
         \sum_{i \in \contributors}  \frac{d_i}{|\buyers|} \frac{\sigma^2}{ \datanum_i^{\star}} - \sum_{j \in \buyers}  \sum_{i \in \contributors} \frac{d_i}{|\buyers|} \rbr{ \frac{\sigma^2}{\datanum_i^{\star}} + \frac{\sigma^2}{\datanum_{-i}^{\star}} }  
         \\ 
         & =  \sum_{j \in \buyers}  \price_j .
    \end{align*}
\end{proof}

\begin{lemma}
    \label{lem:main_prop4}
    Under $M$, following $\strat^\star$ is envy-free for buyers.
\end{lemma}
\begin{proof}
    By definition. \ac{check def} \ac{keran said she'd do}
\end{proof}

\begin{lemma}
    \label{lem:main_prop5}
    Under $M$, when following the recommended strategy $\strat^\star $, the expected profit of buyers approximates the optimal profit, $\mathrm{profit}^{\star}$, with additive error $\epsilon$.
    Formally, 
    \begin{equation*}
        \mathrm{profit}_A( \datanum_1^{\star},\dots,\datanum_d^{\star}) \geq \mathrm{profit}^{\star} - \eps .       
    \end{equation*}
\end{lemma}
\begin{proof}
    We assume that 
    \[\left\{ \tilde{\datanum}_{i} \right\}_{i\in \contributors} = \underset{\datanum_1,\dots,\datanum_d}{\arg \max}\rbr{\mathrm{rev}^{\star}  \rbr{ \buyers,\sum_i \datanum_i} - \sum_{i=1}^{d} \cost_i \datanum_i  },\]
    then we have
    \begin{align*}
      \mathrm{profit}^{\star} &  = \underset{\datanum_1,\dots,\datanum_d}{\max}\rbr{\mathrm{rev}^{\star} 
        \rbr{ B,\sum_i \datanum_i} - \sum_{i=1}^{d} \cost_i \datanum_i  } \\ & = \mathrm{rev}^{\star} 
        \rbr{ B,\sum_i \tilde{\datanum}_i} - \sum_{i=1}^{d} \cost_i \tilde{\datanum}_i \\ & \leq \mathrm{rev}_A
        \rbr{ B,\sum_i \tilde{\datanum}_i} - \sum_{i=1}^{d} \cost_i \tilde{\datanum}_i  + \eps  \\ & \leq  \mathrm{profit}_A(\datanum_1^{\star},\dots,\datanum_d^{\star}) + \eps.
    \end{align*}
    Here the first inequality is by the definition of $\eps$-revenue optimal algorithm: \[\mathrm{rev}_A
        \rbr{ B,\sum_i \tilde{\datanum}_i} \geq \mathrm{rev}^{\star} 
        \rbr{ B,\sum_i \tilde{\datanum}_i}-\eps,\]
    and the second inequality is by equation~\eqref{eq:n_i_star}.
\end{proof}




% \thmNIC*
% \begin{proof}[Proof of Theorem~\ref{thm:multi_thm}]



% %Therefore, the optimal strategy for contributor $i$ is to submit a set with size $\datanum_i^{\star}$, where each element is equal to $\hat{\mu}(X_i)$.
% We have shown that $\strat^{\star}$ is a Nash Equilibrium in Lemma~\ref{lem:nash}. We will now prove properties 2 through 5 as follows.

% (2) Under the recommended strategy, the expected payment for contributor $i$ is

% \begin{align*}
%      \mathbb{E} \sbr{\pay_i }  = 
%       \mathbb{I}\left( \left| Y_i \right| = \datanum_i^{\star} \right) 
% \bigg(w \frac{\datanum_i^{\star}}{N} + \cost_i \datanum_i^{\star}  + d_i \frac{\sigma^2}{\datanum_{-i}^{\star}} + d_i \frac{\sigma^2}{ \datanum_i^{\star}} 
% \bigg)   -  d_i \bigg( \frac{\sigma^2}{\datanum_i^{\star}} + \frac{\sigma^2}{\datanum_{-i}^{\star}} \bigg) = w \frac{\datanum_i^{\star}}{N}+ \cost_i \datanum_i^{\star}.
% \end{align*}

% And contributor $i$'s utility 
% \begin{align*}
%       \utilci(M,\strat^{\star}) = \mathbb{E} [\pay_i] - \cost_i \datanum_i^{\star} =  w \frac{\datanum_i^{\star}}{N} \geq 0 .
% \end{align*}


% Under the recommended strategy, the expected payment for buyer $i$ is
% \begin{align*}
%     \mathbb{E}\sbr{\price_j(\mech,\strat)}   & = \;  \sum_{i \in \contributors} \mathbb{I}\left( \left| Y_i \right| = \datanum_i^{\star} \right) 
% \bigg( 
% \pricecurv_j \frac{\datanum_i^{\star}}{N} + \frac{d_i}{|\buyers|} \frac{\sigma^2}{\datanum_{-i}^{\star}} + \frac{d_i}{|\buyers|} \frac{\sigma^2}{ \datanum_i^{\star}} 
% \bigg)   - \sum_{i \in \contributors} \frac{d_i}{|\buyers|} \mathbb{E}\sbr{\bigg( \hat{\mu}(Y_i) - \hat{\mu}(Y_{-i}) \bigg)^2}  \\ & = \;    
% \bigg( 
% \pricecurv_j  + \sum_{i \in \contributors}  \frac{d_i}{|\buyers|} \frac{\sigma^2}{\datanum_{-i}^{\star}} +\sum_{i \in \contributors}  \frac{d_i}{|\buyers|} \frac{\sigma^2}{ \datanum_i^{\star}} 
% \bigg)   - \sum_{i \in \contributors} \frac{d_i}{|\buyers|} \rbr{ \frac{\sigma^2}{\datanum_i^{\star}} + \frac{\sigma^2}{\datanum_{-i}^{\star}} }  \\ & = \; \pricecurv_j
% \end{align*}
    
% Thereofere, buyer $j$'s utility is 
% \[\mathbb{E}\sbr{\valj(\mech,\strat^{\star})} - \mathbb{E}\sbr{\price_j(\mech,\strat^{\star})} \geq 0 \]


% (3) Budget balance under recommended strategy:

% \begin{align*}
%     \sum_i \pay_i & = w +  \sum_{i \in \contributors} \cost_i \datanum_i^{\star} +  \sum_{i \in \contributors} \rbr{ d_i \frac{\sigma^2}{\datanum_{-i}^{\star}} + d_i \frac{\sigma^2}{ \datanum_i^{\star}} } - \sum_{i \in \contributors} d_i \bigg( \hat{\mu}(Y_i) - \hat{\mu}(Y_{-i}) \bigg)^2 \\ & = \sum_{j \in \buyers} \pricecurv_j + \sum_{j \in \buyers}  \sum_{i \in \contributors}  \frac{d_i}{|\buyers|} \frac{\sigma^2}{\datanum_{-i}^{\star}} +\sum_{j \in \buyers} 
%  \sum_{i \in \contributors}  \frac{d_i}{|\buyers|} \frac{\sigma^2}{ \datanum_i^{\star}} - \sum_{j \in \buyers}  \sum_{i \in \contributors} \frac{d_i}{|\buyers|} \rbr{ \frac{\sigma^2}{\datanum_i^{\star}} + \frac{\sigma^2}{\datanum_{-i}^{\star}} }  \\ & = \; \sum_{j \in \buyers}  \price_j
% \end{align*}

% (4) By definition.

% (5) By Lemma~\ref{lem:profit_appro}.

% From each dataset $Y_i$ submitted by contributors, we randomly select $|Y_i|\frac{\nn^b_j}{N}$ data points among $Y_i$ and then give them to buyer $j$. Therefore, each buyer get $\nn^b_j$ data points. When every contributor follow the recommended strategy, the utility of each buyer is non-negative.

% \end{proof}

%Under the truthful submission, $\left\{\datanum_i, \hat{\mu}(Y_i)\right\}_{i=i}^{d}$, we calculate the $\hat{\mu}_j(Y)$ for buyer $j$ as follows. Let $\datanum_{i,j} = m_i \frac{\datanum_i}{N} $, the amount of data contributor $i$ gives to buyer $j$. \begin{align*}   \hat{\mu}(Y) = \frac{\sum_{i=1}^d  \sum_{k=1}^{\datanum_{i,j}}  \rbr{\hat{\mu}(Y_i) + \eps_{i,j}} } {\sum_{i=1}^d \datanum_{i,j}} \end{align*} Where $\eps_{i,j} \sim N(0, \sigma_{i,j}^2)$, where \[\sigma_{i,j}^2 =\sigma^2 \rbr{\frac{1}{n_{i,j}} - \frac{1}{n_i}}\] so that $Var(\hat{\mu}(Y) ) = \frac{\sigma^2}{m_i}$, and $\hat{\mu}(Y) \sim N\rbr{\mu,\frac{\sigma^2}{m_i}}$.


% \begin{lemma} \label{lem:profit_appro}
%     For $\left\{ \datanum_i \right\}_{i \in \contributors}$, we have the following properties:
%     \begin{align*}
%     \mathrm{profit}_A( \datanum_1^{\star},\dots,\datanum_d^{\star}) \geq \mathrm{profit}^{\star} - \eps .
% \end{align*}
% \end{lemma}

% We then prove that 
% \begin{proof}
%     We assume that 
%     \[\left\{ \tilde{\datanum}_{i} \right\}_{i\in \contributors}= \underset{\datanum_1,\dots,\datanum_d}{\max}\rbr{\mathrm{rev}^{\star}  \rbr{ \buyers,\sum_i \datanum_i} - \sum_{i=1}^{d} \cost_i \datanum_i  },\]
%     then we have
%     \begin{align*}
%       \mathrm{profit}^{\star} &  = \underset{\datanum_1,\dots,\datanum_d}{\max}\rbr{\mathrm{rev}^{\star} 
%     \rbr{ B,\sum_i \datanum_i} - \sum_{i=1}^{d} \cost_i \datanum_i  } \\ & = \mathrm{rev}^{\star} 
%     \rbr{ B,\sum_i \tilde{\datanum}_i} - \sum_{i=1}^{d} \cost_i \tilde{\datanum}_i \\ & \leq \mathrm{rev}_A
%     \rbr{ B,\sum_i \tilde{\datanum}_i} - \sum_{i=1}^{d} \cost_i \tilde{\datanum}_i  + \eps  \\ & \leq  \mathrm{profit}_A(\datanum_1^{\star},\dots,\datanum_d^{\star}) + \eps.
% \end{align*}
% The first inequality is by the definition of $\eps$-revenue optimal algorithm: \[\mathrm{rev}_A
%     \rbr{ B,\sum_i \tilde{\datanum}_i} \geq \mathrm{rev}^{\star} 
%     \rbr{ B,\sum_i \tilde{\datanum}_i}-\eps,\]
% the second inequality is by equation~\ref{eq:n_i_star}.
% \end{proof}