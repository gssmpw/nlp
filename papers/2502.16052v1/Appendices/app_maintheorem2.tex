\section{Proof of Theorem~\ref{thm:multi_thm}}
\label{app:mainproof}

% \kkcomment{Some thoughts on organizing this proof. \\
% - Don't use lemmas without stating them. Perhaps state lemmas A.1-A.3 and in the bod of the paper and prove it in the appendix? You should also tell what each lemma is saying and perhaps some technical challenge in rpoving them (if necessary). \\
% - Right now, it just reads very math-heavy, and all the intersting work in proving NE is in the appendix. 
% - We can get rid of the 6 bullets in the beginning. They are not really doing much. \\
% - I don't think we need/should put lemmas for each property. We can just state what we are proving and then prove it. See example below. (You should use lemmas for intermediate results - right now, each lemma is just part of the theorem) \\
% - If we are over 18 pages, we can move some parts of Theroem 5.1 to the appendix. I think ICC is the most important? We can also move the proof of Lemma 2.2 to the appendix. \\
% - I have not read the proof of 5.1 and probably won't have time to before the deadline.}

% \begin{proof}
% We prove the five properties of Theorem~\ref{thm:multi_thm} in the following five lemmas.
% % we show that $\rbr{M,\strat^{\star}}$ satisfies the following five properties.
% \begin{enumerate}
%     \item Under $M$, $\strat^\star$ is a Nash Equilibrium for contributors. See Lemma~\ref{lem:main_prop1}.
    
%     \item Under $M$, $\strat^\star$ is individually rational for contributors. See Lemma~\ref{lem:main_prop2}.
%     % \kkcomment{Break into two.}

%     \item Under $M$, $\strat^\star$ is individually rational for buyers. See Lemma~\ref{lem:main_prop3}.
    
%     \item The mechanism $M$ satisfies the budget balance property. See Lemma~\ref{lem:main_prop4}.
    
%     \item $M$ is envy-free for buyers. See Lemma~\ref{lem:main_prop5}.
    
    
%     \item Under $M$, when the contributors follow $\strat^\star$, the expected profit of buyers approximates the profit upper bound, $\blprofit - (\cost_2-\cost_1)$, with additive error $|\buyers|\bigO(\eps)$. See Lemma
%     ~\ref{lem:main_prop6}.
% \end{enumerate}
% \end{proof}

% \parahead{
\noindent
\textbf{(ICC):} Under $\mech$, $\strat^\star$ is a Nash equilibrium for contributors.
% }
\begin{proof}
We will show that $\stratopt$ is a Nash equilibrium of $\mechopt$. 
That is, we will show, $\forall i\in\contributors$, 
    $\forall\rbr{\nni,\subfunci}$,
    % \ac{unsure if $\subfuncspace$ has been previously defined?}
    \begin{equation*}
        \utilci\rbr{
            \mechopt, \rbr{
                (\nni,\subfunci),
                \strat_{-i}^{\star}
            }
        }
        \leq
        \utilci\rbr{
            \mechopt, \rbr{
                \strat_i^\star,
                \strat_{-i}^{\star}
            }
        } . 
    \end{equation*}

% \begin{lemma}
%     \label{lem:main_prop1}
%     Under $M$, $\strat^\star$ is a Nash Equilibrium for contributors. Formally, $\forall i\in\contributors$, 
%     $\forall\rbr{\nni,\subfunci}$,
%     % \ac{unsure if $\subfuncspace$ has been previously defined?}
%     \begin{equation*}
%         \utilci\rbr{
%             M, \rbr{
%                 (\nni,\subfunci),
%                 \strat_{-i}^{\star}
%             }
%         }
%         \leq
%         \utilci\rbr{
%             M, \rbr{
%                 \strat_i^\star,
%                 \strat_{-i}^{\star}
%             }
%         } . 
%     \end{equation*}
% \end{lemma}
    Fix any $(\nni,\subfunci)$. Let $f_a:\dataspace\rightarrow\RR^{\nnopti}$ be a function such that $\forall \initdatai\in\dataspace$, $\muhat\rbr{f_a(\initdatai)}=\muhat\rbr{\subfunci(\initdatai)}$. Let $f_b:\dataspace\rightarrow\RR^{\nnopti}$ be a function such that $\forall \initdatai\in\dataspace$, $\muhat\rbr{f_b(\initdatai)}=\muhat\rbr{\initdatai}$.
    % Fix any $(\nni,\subfunci)$. Let $f_a:\dataspace\rightarrow\dataspace$ be a function such that $\forall \initdatai\in\dataspace$, $\abr{f_a(\initdatai)}=\abr{\subfunci(\initdatai)}$ and $\muhat\rbr{f_a(\initdatai)}=\muhat\rbr{\initdatai}$. Let $f_b:\dataspace\rightarrow\RR^{\datanum_i^{\star}}$ be a function such that
    % $\forall \initdatai\in\dataspace$, $\muhat\rbr{f_b(\initdatai)}=\muhat\rbr{f_a(\initdatai)}$.
    We then have that
    \begin{align*}
        \utilci \rbr{M, \rbr{(\nni,\subfunci),\strat_{-i}^{\star} }}
        &\leq
        \utilci \rbr{M, \rbr{(\nni,f_a),\strat_{-i}^{\star} }}
        % \qquad
        &
        (\text{Lemma}~\ref{lem:NIC-part1})
        \\
        &\leq
        \utilci \rbr{M, \rbr{(\nni,f_b),\strat_{-i}^{\star} }}
        % \qquad
        &
        (\text{Lemma}~\ref{lem:NIC-part2})
        \\
        &\leq
        % \utilci \rbr{M, (\datanum_i^{\star},f_b),\strat_{-i}^{\star} }
        % =
        \utilci \rbr{M, \rbr{\strat_{i}^{\star},\strat_{-i}^{\star} }} .
        &
        % \qquad
        \hspace{-1cm}
        (\text{Lemma}~\ref{lem:NIC-part3})
    \end{align*}
\end{proof}

% \begin{lemma}
%     \label{lem:main_prop1-old}
%     Under $M$, $\strat^\star$ is a Nash Equilibrium for contributors.
% \end{lemma}
% \begin{proof}
%     Recall that contributor $i$'s payment is
%     \begin{align*}
%     & \pay_i (\mech,\strat_i,\strat_{-i}^{\star})  =   \mathbb{I}\left( | Y_i | = \datanum_i^{\star} \right) 
%     \bigg( 
%     \profi \frac{\datanum_i^{\star}}{N} + \cost_i \datanum_i  + d_i \frac{\sigma^2}{\datanum_{-i}^{\star}} + d_i \frac{\sigma^2}{ \datanum_i^{\star}} 
%     \bigg)  - d_i \bigg( \hat{\mu}(Y_i) - \hat{\mu}(Y_{-i}) \bigg)^2  
%     % \numberthis 
%     \end{align*}
%     where $d_i= \frac{\cost_i (\datanum_i^{\star})^2}{ \sigma^2}$. We prove the Nash equilibrium in three steps.
%     Step (1) shows that regardless of the amount of data collected and the size of the dataset submitted, it is best to have $\hat{\mu}(Y_i)=\hat{\mu}(X_i)$. Step (2) shows that regardless of the amount of data collected, the optimal number of points to submit is $\datanum_i^{\star}$. Step (3) Shows that when fixing the submission size to be $\datanum_i^{\star}$, it is best to collect $\datanum_i^{\star}$ points.
%     \ac{should this be reorganized into three lemmas which fit together (similar to )}

%     \ac{step 1. doesn't cover functions that submit variable amounts, shouldn't this be changed to regardless of the amount of data submitted (which could depend on $X_i$) its best to submit something that agrees with the sample mean?}
    
%     \textbf{Step 1}: Fix the number of data collected $n_i$ and the number of data submitted $\left| Y_i \right|$ by contributor $i$. According to Lemmas~\ref{lem:minimax} and~\ref{lem:same_minimax}, $ \hat{\mu}\left(X_i\right)$ is minimax estimator of $\mathbb{E}\sbr{\rbr{\hat{\mu}(Y_i)-\hat{\mu}(Y_{-i})}^2 \;\middle|\; P} $, i.e. 
%     \begin{align*}
%       \hat{\mu}(X_i) = \underset{\mu}{\arg\min} \sbr{\sup _\distrifamily \mathbb{E}\left[\sum_{j=1}^{m} d_i \rbr{ \hat{\mu}(Y_i) - \hat{\mu}(Y_{-i})}^2 \;\middle|\;  P \right] },
%     \end{align*}
%     \ac{why is there a sum in the expectation?}
    
%     As $n_i$ and $\left| Y_i \right|$ are fixed, this implies contributor $i$ maximizes their utility by selecting $\hat{\mu}(Y_i) = \hat{\mu}(X_i)$. 
    
%     \textbf{Step 2}: When fixing $n_i$, and assuming that contributor $i$ submit $\hat{\mu}(Y_i)=\hat{\mu}(X_i)$,  the optimal number of data points to submit is $\left| Y_i \right| = \datanum_i^{\star}$. Because the payment 
%     $\pay_i $ will be negative if $\left| Y_i \right| \neq \datanum_i^{\star}$. Therefore, for any $n_i$, we combine the first and second step:
%     \begin{align*}
%     & \utilci \rbr{\mech, (n_i,f_i \text{ s.t. } \hat{\mu}(Y_i)=\hat{\mu}(X_i), \left| Y_i \right| =\datanum_i^{\star}),\strat_{-i}^{\star} }  \geq  \utilci \rbr{M, (n_i,f_i),\strat_{-i}^{\star} }.
%     \end{align*}
    
    
    
%     \textbf{Step 3}: We choose constant $d_i= \frac{\cost_i (\datanum_i^{\star})^2}{ \sigma^2}$. When fixing $\hat{\mu}(Y_i)=\hat{\mu}(X_i)$ and $\left| Y_i \right| = \datanum_i^{\star}$, collecting $n_i = \datanum_i^{\star}$ amount of data maximize the contributor $i$'s utility. To see this, we write the contributor's utility given $\hat{\mu}(Y_i)=\hat{\mu}(X_i)$ and $\left| Y_i \right| = \datanum_i^{\star}$ as:
    
%     \begin{align*}
%         & \utilci \rbr{\mech, (n_i, f_i \text{ s.t }\hat{\mu}(Y_i)=\hat{\mu}(X_i), \left| Y_i \right| = \datanum_i^{\star}),\strat_{-i}^{\star} }  
%         \\ 
%         = & \rbr{w \frac{\datanum_i^{\star}}{N} + \cost_i \datanum_{i}^{\star}  + d_i \frac{\sigma^2}{\datanum_{-i}^{\star}} + d_i \frac{\sigma^2}{ \datanum_i^{\star}} } -    d_i \rbr{ \frac{\sigma^2}{\datanum_{-i}^{\star}} +\frac{\sigma^2}{\datanum_i} } - \cost_i n_i ,
%     \end{align*}
    
    
%     Plug in the value $d_i= \frac{\cost_i (\datanum_i^{\star})^2}{\sigma^2}$, it follows that $ d_i \frac{\sigma^2}{\datanum_i} + \cost_i \datanum_i \geq 2\cost_i \datanum_i^{\star}$, and this inequality takes equality if and only if  $\datanum_i = \datanum_i^{\star}$. Therefore, the utility is maximized when contributor $i$ collects $\datanum_i = \datanum_i^{\star}$ data points,
%     \begin{align*}%         &\;\utilci \rbr{\mech, \strat_{i}^{\star} ,\strat_{-i}^{\star} }\\  = & \;\utilci \rbr{\mech, (\datanum_i^{\star},\mathbf{I}),\strat_{-i}^{\star} }\\  = & \;\utilci \rbr{\mech, (\datanum_i^{\star}, f_i \text{ s.t }\hat{\mu}(Y_i)=\hat{\mu}(X_i), \left| Y_i \right|=\datanum_i^{\star}),\strat_{-i}^{\star} }  \\ \geq &   \; \utilci \rbr{\mech, \strat_{i} ,\strat_{-i}^{\star}} .
%     \end{align*}
% \end{proof}

% \begin{lemma}
    % \label{lem:main_prop2}
    % Under $M$, $\strat^\star$ is individually rational for contributors.
% \end{lemma}
% \parahead{
\noindent
\textbf{(IRC):} Under $M$, $\strat^\star$ is individually rational for contributors.
% }
\begin{proof}
    Under the recommended strategy, the utility for contributor $i$ is 
    \begin{align*}
        \utilci(M,\strat^{\star})
        &=
        \inf_{\mu\in\RR}
        \EE\Bigg[
            \mathbb{I}\left( \left| Y_i \right| = \reqamnt_i^{\star} \right) 
            \rbr{
                \profi \frac{\reqamnt_i^{\star}}{\numdataA} + \cost_i \reqamnt_i^{\star}  +  \frac{d_i\sigma^2}{\reqamnt_{-i}^{\star}} +  \frac{d_i\sigma^2}{ \reqamnt_i^{\star}} 
            }
            -  d_i\rbr{
                \muhat(\subfunci(\initdatai))-\muhat(\subdatami)
            }^2
            -\costi\nnopti
        \Bigg]
        \\
        &=
            \rbr{
                \profi \frac{\reqamnt_i^{\star}}{\numdataA} + \cost_i \reqamnt_i^{\star}  +  \frac{d_i\sigma^2}{\reqamnt_{-i}^{\star}} +  \frac{d_i\sigma^2}{ \reqamnt_i^{\star}} 
            }
            -\costi\nnopti
        -d_i \sup_{\mu\in\RR}
        \EE\Bigg[
          d_i\rbr{
            \muhat(\initdatai)-\muhat(\subdatami)
        }^2
        \Bigg]
        \\
        &=
            \rbr{
                \profi \frac{\reqamnt_i^{\star}}{\numdataA} + \cost_i \reqamnt_i^{\star}  +  \frac{d_i\sigma^2}{\reqamnt_{-i}^{\star}} +  \frac{d_i\sigma^2}{ \reqamnt_i^{\star}} 
            }
            -\costi\nnopti
        -\rbr{
            \frac{d_i\sigma^2}{\reqamnt_{-i}^{\star}} +  \frac{d_i\sigma^2}{ \reqamnt_i^{\star}} 
        }
        \\
        &=
        \profi \frac{\reqamnt_i^{\star}}{\numdataA} 
        \geq 0 .
    \end{align*}

    
\end{proof}

% \begin{lemma}
%     \label{lem:main_prop3}
%     Under $M$, $\strat^\star$ is individually rational for buyers.
% \end{lemma}
% \parahead{
\noindent
\textbf{(IRB):} Under $M$, $\strat^\star$ is individually rational for buyers.
% }
\begin{proof}
    Under the recommended strategy, the expected payment for buyer $i$ is
    \begin{align*}
        \mathbb{E}\sbr{\priceoptj(\mech,\strat^{\star})}   & =  \sum_{i \in  \{1,2\} }\mathbb{I}\left( \left| Y_i \right| = \reqamnt_i^{\star} \right) 
    \bigg( 
    \pricejA \frac{\reqamnt_i^{\star}}{\numdataA} + \frac{d_i}{|\buyers|} \frac{\sigma^2}{\reqamnt_{-i}^{\star}} + \frac{d_i}{|\buyers|} \frac{\sigma^2}{ \reqamnt_i^{\star}} 
    \bigg)   - \sum_{i  \in \{1,2\} } \frac{d_i}{|\buyers|} \mathbb{E}\sbr{\bigg( \hat{\mu}(Y_i) - \hat{\mu}(Y_{-i}) \bigg)^2}  \\ & = \;    
    \bigg( 
     \pricejA  + \sum_{i \in  \{1,2\}}  \frac{d_i}{|\buyers|} \frac{\sigma^2}{\reqamnt_{-i}^{\star}} +\sum_{i \in \contributors}  \frac{d_i}{|\buyers|} \frac{\sigma^2}{ \reqamnt_i^{\star}} 
    \bigg)   - \sum_{i \in  \{1,2\}} \frac{d_i}{|\buyers|} \rbr{ \frac{\sigma^2}{\reqamnt_i^{\star}} + \frac{\sigma^2}{\reqamnt_{-i}^{\star}} }  \\ & = \; \pricejA . \numberthis \label{eqn:IRB}
    \end{align*}
        
    Therefore, combing with definition of buyer's utility (\ref{eqn:utilbone}), and buyers's value for i.i.d. data (\ref{eqn:valdataj}), buyer $j$'s utility under recommended strategy is 
       \begin{align*} \label{eqn:utilityj_stratopt}
         \utilbj(M, \stratopt)  \defeq & \; 
        \utilbj(\mmoptj, \priceoptj, \stratopt)  \defeq
        \inf_{\mu\in\RR} \EE_{M, \stratopt, \mu}\left[ \valfuncj\left( 
       \left|\,\muhat\big(g_{\stratopt, \mmoptj}( \{\initdatai\}_{i\in\contributors} \big) - \mu\,\right| \right) -\priceoptj\left( \{ \subfunci(\initdatai)\}_{i\in\contributors}\right)\right] \\  =&  \; \valdataj(\mmoptj)-  \sup_{\mu\in\RR}\EE_{M, \stratopt, \mu}\cbr{\priceoptj\left( \{ \subfunci(\initdatai)\}_{i\in\contributors}\right)  } = \; \valdataj(\mmoptj)- \pricejA = \; \valdataj(\mmjA)- \pricejA \geq 0. \numberthis
    \end{align*}
    In the last step, recall that  $(\mmjA,\pricejA)_{j \in \buyers}$ are generated by Algorithm~\ref{alg:oipalgo}, combing with the buyer's purchase model in (\ref{eqn:buyerpurchasemodel}), we have $\valdataj(\mmjA)- \pricejA \geq 0$ .   
    
 
\end{proof}

% \begin{proof}
%     Under the recommended strategy, the expected payment for contributor $i$ is

%     \begin{align*}
%          \mathbb{E} \sbr{\pay_i }  = 
%           \mathbb{I}\left( \left| Y_i \right| = \reqamnt_i^{\star} \right) 
%     \bigg(\profi \frac{\datanum_i^{\star}}{N} + \cost_i \datanum_i^{\star}  + d_i \frac{\sigma^2}{\datanum_{-i}^{\star}} + d_i \frac{\sigma^2}{ \datanum_i^{\star}} 
%     \bigg)   -  d_i \bigg( \frac{\sigma^2}{\datanum_i^{\star}} + \frac{\sigma^2}{\datanum_{-i}^{\star}} \bigg) = \profi \frac{\datanum_i^{\star}}{N}+ \cost_i \datanum_i^{\star}.
%     \end{align*}
    
%     We also have contributor $i$'s utility is
%     \begin{align*}
%           \utilci(M,\strat^{\star}) = \mathbb{E} [\pay_i] - \cost_i \datanum_i^{\star} =  \profi \frac{\datanum_i^{\star}}{N} \geq 0 .
%     \end{align*}
    
    
%     Under the recommended strategy, the expected payment for buyer $i$ is
%     \begin{align*}
%         \mathbb{E}\sbr{\price_j(\mech,\strat)}   & = \;  \sum_{i \in \contributors} \mathbb{I}\left( \left| Y_i \right| = \datanum_i^{\star} \right) 
%     \bigg( 
%     \pricecurv_j \frac{\datanum_i^{\star}}{N} + \frac{d_i}{|\buyers|} \frac{\sigma^2}{\datanum_{-i}^{\star}} + \frac{d_i}{|\buyers|} \frac{\sigma^2}{ \datanum_i^{\star}} 
%     \bigg)   - \sum_{i \in \contributors} \frac{d_i}{|\buyers|} \mathbb{E}\sbr{\bigg( \hat{\mu}(Y_i) - \hat{\mu}(Y_{-i}) \bigg)^2}  \\ & = \;    
%     \bigg( 
%     \pricecurv_j  + \sum_{i \in \contributors}  \frac{d_i}{|\buyers|} \frac{\sigma^2}{\datanum_{-i}^{\star}} +\sum_{i \in \contributors}  \frac{d_i}{|\buyers|} \frac{\sigma^2}{ \datanum_i^{\star}} 
%     \bigg)   - \sum_{i \in \contributors} \frac{d_i}{|\buyers|} \rbr{ \frac{\sigma^2}{\datanum_i^{\star}} + \frac{\sigma^2}{\datanum_{-i}^{\star}} }  \\ & = \; \pricecurv_j .
%     \end{align*}
        
%     Therefore, buyer $j$'s utility is 
%     \begin{align*}
%          \mathbb{E}\sbr{\valj(\mech,\strat^{\star})} - \mathbb{E}\sbr{\price_j(\mech,\strat^{\star})}  
%          &=   \;\mathbb{E}\sbr{\valj(\mech,\strat^{\star})} - \pricecurv_j + \pricecurv_j -\mathbb{E}\sbr{\price_j(\mech,\strat^{\star})} \\ &=  \; \mathbb{E}\sbr{\valj(\mech,\strat^{\star})} - \pricecurv_j   \\ &\geq  \; 0 .
%     \end{align*}
    
% \end{proof}

% \begin{lemma}
%     \label{lem:main_prop4}
%     The mechanism $M$ satisfies the budget balance property.
% \end{lemma}
% \parahead{
\noindent
\textbf{(BB):} The mechanism $M$ satisfies the budget balance property.
% }
\begin{proof}
    From the definition of the payment rule in $M$ and a sequence of algebraic manipulations we have
    \begin{align*}
        \sum_{i\in\contributors}\payopti (M, \strat^{\star})
        &=
        \sum_{i\in\contributors}
        \rbr{
            \rbr{
                W\frac{\nnopti}{\numdataA}
                +
                \costi\nnopti
                +
                \frac{d_i\sigma^2}{\nnopti}
                +
                \frac{d_i\sigma^2}{\nnopt_{-i}}
            }
            -d_i\rbr{
                \muhat(\initdatai)-\muhat(X_{-i}) 
            }^2
        }
        \\
        &=
        \frac{W}{\numdataA} \sum_{i\in\contributors} \nnopti
        +
        \sum_{i\in\contributors} \costi\nnopti
        +
        \sum_{i\in\contributors}
        \rbr{
            \frac{d_i\sigma^2}{\nnopt_{-i}}
            +
            \frac{d_i\sigma^2}{\nnopti}
            -d_i\rbr{
                \muhat(\initdatai)-\muhat(X_{-i}) 
            }^2
        }
        \\
        &=
        W 
        +
        \sum_{i\in\contributors} \costi\nnopti
        +
        \sum_{j\in\buyers}\sum_{i\in\contributors}
        \rbr{
            \frac{d_i\sigma^2}{\abr{\buyers}\nnopt_{-i}}
            +
            \frac{d_i\sigma^2}{\abr{\buyers}\nnopti}
            -
            \frac{d_i}{\abr{\buyers}}
            \rbr{
                \muhat(\initdatai)-\muhat(X_{-i}) 
            }^2
        }
        \\
        &=
        \sum_{j\in\buyers} \pricejA
        +
        \sum_{j\in\buyers}\sum_{i\in\contributors}
        \rbr{
            \frac{d_i\sigma^2}{\abr{\buyers}\nnopt_{-i}}
            +
            \frac{d_i\sigma^2}{\abr{\buyers}\nnopt_{i}}
            -
            \frac{d_i}{\abr{\buyers}}
            \rbr{
                \muhat(\initdatai)-\muhat(X_{-i}) 
            }^2
        }
        \\
        &=
        \sum_{j\in\buyers}
        \rbr{
            \sum_{i\in\contributors}
            \rbr{
                \frac{\pricejA \nnopti}{\numdataA}
                +
                \frac{d_i\sigma^2}{\abr{\buyers}\nnopt_{-i}}
                +
                \frac{d_i\sigma^2}{\abr{\buyers}\nnopti}
                -
                \frac{d_i}{\abr{\buyers}}
                \rbr{
                    \muhat(\initdatai)-\muhat(X_{-i}) 
                }^2
            }
        }
        \\
        &=
        \sum_{j\in\buyers}
        \priceoptj (M, \strat^{\star}) .
    \end{align*}
    Therefore, $M$ satisfies the budget balance property.
    
\end{proof}
% \begin{proof}
%     % Budget balance under recommended strategy:
%     Summing the contributor payments, we get that
%     \ac{keran said  missing definitions of $N:=\sum n^\star_i$, and how $W$ and $q_j$ are related; check def 2.5}
%     \ac{keran said she'd do}

%     \begin{align*}
%         \sum_{i\in\contributors}\pay_i (M, \strat^{\star})
%         & = \profi +  \sum_{i \in \contributors} \cost_i \nnopti_i^{\star} +  \sum_{i \in \contributors} \rbr{ d_i \frac{\sigma^2}{\nnopti_{-i}^{\star}} + d_i \frac{\sigma^2}{ \datanum_i^{\star}} } - \sum_{i \in \contributors} d_i \bigg( \hat{\mu}(Y_i) - \hat{\mu}(Y_{-i}) \bigg)^2 
%         \\ 
%         & = \sum_{j \in \buyers} \pricecurv_j + \sum_{j \in \buyers}  \sum_{i \in \contributors}  \frac{d_i}{|\buyers|} \frac{\sigma^2}{\datanum_{-i}^{\star}} +\sum_{j \in \buyers} 
%          \sum_{i \in \contributors}  \frac{d_i}{|\buyers|} \frac{\sigma^2}{ \datanum_i^{\star}} - \sum_{j \in \buyers}  \sum_{i \in \contributors} \frac{d_i}{|\buyers|} \rbr{ \frac{\sigma^2}{\datanum_i^{\star}} + \frac{\sigma^2}{\datanum_{-i}^{\star}} }  
%          \\ 
%          & =  \sum_{j \in \buyers}  \price_j .
%     \end{align*}
% \end{proof}

% \begin{lemma}
%     \label{lem:main_prop5}
%     Under $M$, following $\strat^\star$ is envy-free for buyers.
% \end{lemma}
% \parahead{
\noindent
    \textbf{(EFB):}
    Under $M$, following $\strat^\star$ is envy-free for buyers.
% }
\begin{proof}
    To show that when contributors follow $\strat^\star$, $\mech$ is envy-free for buyers, we must prove that $\utilbj(\mmoptj, \priceoptj, \stratopt) 
    \geq \utilbj(\mmoptk, \priceoptk, \stratopt)$, $\forall j,k\in\buyers$. As demonstrated in~\eqref{eqn:utilityj_stratopt}, we know $\utilbj(M, \stratopt)  \defeq \; 
        \utilbj(\mmoptj, \priceoptj, \stratopt) = \valdataj(\mmjA)- \pricejA \geq 0$. 
 Similarly, by the definition of buyers utility (\ref{eqn:utilbone}) and buyer's valuation for i.i.d. data (\ref{eqn:valdataj}), we have
       \begin{align*}           
       & \utilbj(\mmoptk, \priceoptk, \stratopt)  \defeq 
        \inf_{\mu\in\RR} \EE_{M, \stratopt, \mu}\left[ \valfuncj\left( 
       \left|\,\muhat\big(g_{\stratopt, \mmoptk}( \{\initdatai\}_{i\in\contributors} \big) - \mu\,\right| \right) -\priceoptk\left( \{ \subfunci(\initdatai)\}_{i\in\contributors}\right)\right] \\  =  \; & \valdataj(\mmoptk)-  \sup_{\mu\in\RR}\EE_{M, \stratopt, \mu}\cbr{\priceoptk\left( \{ \subfunci(\initdatai)\}_{i\in\contributors}\right)  } = \; \valdataj(\mmoptk)- \pricekA = \; \valdataj(\mmkA)- \pricekA 
 \leq \; \valdataj(\mmjA)- \pricejA .
    \end{align*}
The last step is by the definition of $\{q^+_j\}_{j \in \buyers}$ in Algorithm~\ref{alg:oipalgo}, and buyer's purchase model (\ref{eqn:buyerpurchasemodel}).
\end{proof} 

% \begin{lemma}
%     \label{lem:main_prop6}
%     Under $M$, when the contributors follow following $\strat^\star$, the expected profit of buyers approximates the optimal profit upper bound (defined in Theorem~\ref{thm:neup}) with additive error $|\buyers|\bigO(\eps)$. 
% \end{lemma}
% \parahead{
\noindent
    \textbf{Approximately optimal profit:}
    % \emph{
    Under $M$, when the contributors follow following $\strat^\star$, the expected profit of buyers approximates the optimal profit upper bound (defined in Theorem~\ref{thm:neup}) with additive error $|\buyers|\bigO(\eps)$.
    % }
% }
\begin{proof}
    By definition of contributors' cumulative profit (\ref{eqn:profit}), 
    \begin{align*}
     \profit(M, \stratopt) & = \inf_{\mu\in\RR} \EE_{M, \stratopt, \mu}\Bigg[
        \sum_{j\in\buyers} \priceoptj\left( \{\subfunci(\initdatai)\}_{i\in\contributors}\right) -
            \sum_{i\in\contributors}\costi\nni
        \Bigg]    \\ & =  \inf_{\mu\in\RR} \EE_{M, \stratopt, \mu}\Bigg[
        \sum_{j\in\buyers} \priceoptj\left( \{\subfunci(\initdatai)\}_{i\in\contributors}\right) 
        \Bigg]  -\sum_{i\in\contributors}\costi\nnopti  \\ & =   \sum_{j\in \buyers} \pricejA -\sum_{i\in\contributors}\costi\nnopti \tag*{(By proof of \textbf{IRB}, equation (\ref{eqn:IRB}))}
\end{align*}
Recall that $\ExpectREV(\numdataA, \itemprice_{\numdataA}) = \sum_{j\in \buyers} \pricejA$ and that $\numdataA = \sum_i {\nnopti} $, $\nnopt_1 = \numdataA-1$, $\nnopt_2 = 1$, we have 
   \begin{align*}
        ~\profit(M, \stratopt)  =& \;  
        \ExpectREV(\numdataA, \itemprice_{\numdataA}) -\cost_1 \rbr{\numdataA} -(\cost_2 - \cost_1)
         \geq  \; \ExpectREV(\numdata, \itemprice_{\numdata}) -\cost_1 \rbr{\numdata} -(\cost_2 - \cost_1)
       \\  \geq &  \; 
       \OPTREV
        \rbr{ 
            \numdata 
        } 
        -  \cost_1 \rbr{\numdata} - |\buyers|\bigO(\eps)-  (\cost_2 -\cost_1)   
        =   \; \blprofit - (\cost_2 - \cost_1)- |\buyers|\bigO(\eps),
    \end{align*}
where the first inequality is by the optimality of $\numdataA$ in Algorithm~\ref{alg:oipalgo}, the second inequality is because algorithm $A$ returns a 
$|\buyers|\bigO(\eps)$ approximation.

\end{proof}