\section{Related Work}
{\bf Related Work}: {\it LPV control}: In LPV control framework \cite{Shamma92gain-CSM,Rugh00gs_survey, Moh12LPVBook}, a nonlinear system is first approximated by an LPV model, $\dot x = A(\theta(t))x+ B(\theta(t))u$, whether $\theta(t)$ is the vector of scheduling parameters that can be measured or estimated online 
 and characterize the variation of system dynamics. 
 Given the LPV model, an LPV controller can be designed to ensure stability and performance of the closed-loop system, e.g., using linear matrix inequalities \cite{Wu96Induced,Apk98}. A major issue with the LPV framework is that the LPV model used for control design is either {\it only locally valid} when Jacobian linearization is used to obtain the LPV model, or is an {\it over-approximation} of the real dynamics when the quasi-LPV approach is used, leading to conservative performance \cite{Rugh00gs_survey}. 
 %In contrast, the proposed research {\it fully accounts for the nonlinear dynamics} without using linearization or over-approximation. 
 Additionally,  in the LPV framework, the control law is limited to be linear (although it can be parameter-dependent). 


% {\it Nonlinear control synthesis using SOS optimization}:  Leveraging a linear-like representation, \cite{prajna2004nonlinear-sos} proposed to search for a state-dependent Lyapunov function (LF) and a controller jointly using SOS programming. %A well-recognized problem in Lyapunov-based nonlinear synthesis is that nonlinear terms involving the product of LF coefficients and controller gain may appear in the synthesis conditions, leading to non-convex problems. 
% In \cite{prajna2004nonlinear-sos},  nonlinear terms involving the product of LF coefficients and controller gain are avoided by constraining the LF to depend on states whose derivatives are not directly affected by any control input. Recently, \cite{fu2018hinf-npv,fu2019exponential-npv,lu2020domain-npv,yu2024hinf-npv} extended the approach in \cite{prajna2004nonlinear-sos} to design nonlinear PD controllers for NPV systems. However, all these existing works are based on the Lyapunov function approach instead of a CLF. Moreover, none of them provides an approach to maximize the PD region of attraction or PD region of stabilization in controller synthesis, although PD region of attraction was studied in \cite{lu2020domain-npv}.

{\it Verification and synthesis of CLF and control barrier functions (CBFs)}: Inspired by CLFs, CBFs are proposed to design control inputs to ensure set-invariance of nonlinear systems without resorting to a specific control law \cite{ames2016cbf-tac}. Due to the similarity between a CLF and a CBF, approaches developed for verifying and synthesizing one of them can be readily applied to the other.  \cite{tan2004searching-clf-sos} proposed to iteratively search for CLFs via SOS optimization without searching for a controller. %\cite{dai2022convex-clf-cbf} developed an SOS optimization-based approach to verify and synthesis a CLF or CBF in the presence of input constraints. Additionally, 
SOS optimization based CBF synthesis has been studied in   \cite{wang2018permissive-sos, clark2021verification} without considering input constraints, and in \cite{dai2022convex-clf-cbf,zhao2023convex-cbf} with consideration of input constraints. 
%linearization \cite{wabersich2022predictive}, machine learning \cite{robey2020learning-cbf-expert,srinivasan2020synthesis-learning-cbf} and Hamilton-Jacobi reachability analysis \cite{choi2021robust-cbvf,tonkens2022refining-CBF}.  
However, all these existing works are for nonlinear autonomous systems, and cannot handle the NPV systems considered in this manuscript. 
%an initial valid LF is needed for the synthesis procedure.}
%Finally, there are machine learning-based methods \cite{robey2020learning-cbf-expert,srinivasan2020synthesis-learning-cbf} that seek to learn CBFs using data generated by experts and require a large amount of data to provide reasonable safety guarantees. Finally, a method was proposed in \cite{tonkens2022refining-CBF}  to refine given CBF candidates using Hamilton-Jacobi reachability analysis. % and was guaranteed to produce a valid CBF upon convergence. 
%However, due to the computational complexity of the dynamic programming-based recursion, the approach is applicable to low-dimensional systems only. 

%