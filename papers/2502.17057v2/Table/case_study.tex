\begin{table*}
\centering
% \resizebox{\textwidth}{!}{
% \captionsetup{font=small,labelfont=bf}
% 
\small
\begin{tabular}{>{\raggedright\arraybackslash}p{0.96\textwidth}}
\toprule
% \textbf{Content} \\
% \midrule
% \multicolumn{1}{c}{\textbf{Case 1}}  \\
% \midrule
\textbf{Query:} How does the coronavirus respond to changes in the weather? \\
\textbf{Golden Pagssage:} Epidemics ... occur during the winter months. ... Two major contributing \sethlcolor{lightblue}\hl{factors} are the \sethlcolor{lightyellow}\hl{changes} in \sethlcolor{lightblue}\hl{environmental} parameters and \sethlcolor{lightblue}\hl{human behavior}. Studies have revealed the effect of \sethlcolor{lightgreen}\hl{temperature} and \sethlcolor{lightgreen}\hl{humidity} on respiratory \sethlcolor{lightblue}\hl{virus stability} and \sethlcolor{lightblue}\hl{transmission} rates. More recent research highlights the importance of the \sethlcolor{lightblue}\hl{environmental factors}, especially \sethlcolor{lightgreen}\hl{temperature} and \sethlcolor{lightgreen}\hl{humidity} ...   \\
\midrule
\multicolumn{1}{c}{\textbf{Raw Query}} \\ 
\textbf{nDCG@10:} \textcolor{darkgray}{22.01\%} \\
\textbf{Original Query}: How does the \sethlcolor{lightyellow}\hl{coronavirus} respond to \sethlcolor{lightyellow}\hl{changes} in the weather? \\
\midrule
\multicolumn{1}{c}{\textbf{Vanilla LLM}}  \\
\textbf{nDCG@10:} \textcolor{darkgray}{76.63\%} \\
\textbf{Expanded Query}: The \sethlcolor{lightgreen}\hl{coronavirus}, like many other \sethlcolor{lightgreen}\hl{viruses}, has been observed to respond to \sethlcolor{lightgreen}\hl{changes} in the weather by experiencing fluctuations in \sethlcolor{lightgreen}\hl{transmission} and spread. ... the \sethlcolor{lightgreen}\hl{virus} tends to thrive in environments with high \sethlcolor{lightgreen}\hl{humidity}, typically above 40\%, and a \sethlcolor{lightgreen}\hl{temperature} range of 37°C to 46°C. ... studies have found that the \sethlcolor{lightgreen}\hl{virus} can survive on surfaces for longer periods at lower \sethlcolor{lightgreen}\hl{temperatures} and \sethlcolor{lightgreen}\hl{humidity} levels, ... \\
\midrule

\multicolumn{1}{c}{\textbf{LLM-QE}} \\ 
\textbf{nDCG@10:} \textcolor{red}{100.00\%} \\
\textbf{Expanded Query}: The  \sethlcolor{lightblue}\hl{coronavirus} responds to \sethlcolor{lightblue}\hl{changes} in the weather by adapting its \sethlcolor{lightblue}\hl{transmission} and spread patterns. This is because \sethlcolor{lightblue}\hl{temperature}, \sethlcolor{lightblue}\hl{humidity}, and other \sethlcolor{lightblue}\hl{environmental factors} can affect the \sethlcolor{lightblue}\hl{stability} and survival of the \sethlcolor{lightblue}\hl{virus} on surfaces, ... research suggests that the \sethlcolor{lightblue}\hl{virus} may thrive in cooler and more humid environments, ... such as air circulation, ventilation, and \sethlcolor{lightblue}\hl{human behavior}. \\

\bottomrule
\end{tabular}
% 
\caption{Case Study. All experiments are conducted based on the Contriever model under the zero-shot setting. To facilitate evaluation, we highlight the potential matching phrases between the golden passage and both the original and expanded queries. Different colors are used to annotate these matched phrases for each method: \sethlcolor{lightyellow}\hl{Green} for Direct Retrieval, \sethlcolor{lightgreen}\hl{Red} for Vanilla LLM, and \sethlcolor{lightblue}\hl{Blue} for LLM-QE.}
\label{tab:case_study}
\end{table*}



% \definecolor{co5}{HTML}{53589A}
% \definecolor{co6}{HTML}{F8DF70}

% \begin{table*}
% \centering
% % \resizebox{\textwidth}{!}{
% % \captionsetup{font=small,labelfont=bf}
% % 
% \small
% \begin{tabular}{>{\raggedright\arraybackslash}p{0.96\textwidth}}
% \toprule
% % \textbf{Content} \\
% % \midrule
% % \multicolumn{1}{c}{\textbf{Case 1}}  \\
% % \midrule
% \textbf{Query:} How does the coronavirus respond to changes in the weather? \\
% \textbf{Golden Pagssage:} Epidemics ... occur during the winter months. ... Two major contributing \textcolor{red}{{factors}} are the \textcolor{red}{{changes}} in \textcolor{red}{{environmental}} parameters and \textcolor{red}{{human behavior}}. Studies have revealed the effect of \textcolor{red}{{temperature}} and \textcolor{red}{{humidity}} on respiratory \textcolor{red}{{virus}} \textcolor{red}{{stability}} and \textcolor{red}{{transmission}} rates. More recent research highlights the importance of the \textcolor{red}{{environmental factors}}, especially \textcolor{red}{{temperature}} and \textcolor{red}{{humidity}} ...   \\
% \midrule
% \multicolumn{1}{c}{\textbf{Direct Retrieval}} \\ 
% \textbf{nDCG@10:} \textcolor{darkgray}{22.01\%} \\
% How does the \textcolor{blue}{{coronavirus}} respond to \textcolor{blue}{{changes}} in the weather? \\
% \midrule
% \multicolumn{1}{c}{\textbf{Zero-shot Expansion}}  \\
% \textbf{nDCG@10:} \textcolor{darkgray}{76.63\%} \\
% The \textcolor{blue}{{coronavirus}}, like many other \textcolor{blue}{{viruses}}, has been observed to respond to \textcolor{blue}{{changes}} in the weather by experiencing fluctuations in \textcolor{blue}{{transmission}} and spread. ... the \textcolor{blue}{{virus}} tends to thrive in environments with high \textcolor{blue}{{humidity}}, typically above 40\%, and a \textcolor{blue}{{temperature}} range of 37°C to 46°C. ... studies have found that the \textcolor{blue}{{virus}} can survive on surfaces for longer periods at lower \textcolor{blue}{{temperatures}} and \textcolor{blue}{{humidity}} levels, ... \\
% \midrule

% \multicolumn{1}{c}{\textbf{Unsupervised LLM-QE Expansion}} \\ 
% \textbf{nDCG@10:} \textcolor{red}{93.37\%} \\
% The  \textcolor{blue}{{coronavirus}} responds to \textcolor{blue}{{changes}} in the weather by adapting its \textcolor{blue}{{transmission}} and spread patterns. This is because \textcolor{blue}{{temperature}}, \textcolor{blue}{{humidity}}, and other \textcolor{blue}{{environmental factors}} can affect the \textcolor{blue}{{stability}} and survival of the \textcolor{blue}{{virus}} on surfaces, ... research suggests that the \textcolor{blue}{{virus}} may thrive in cooler and more humid environments, ... such as air circulation, ventilation, and \textcolor{blue}{\textbf{human behavior}}. \\
% \midrule
% \multicolumn{1}{c}{\textbf{Supervised LLM-QE Expansion}} \\ 
% \textbf{nDCG@10:} \textcolor{red}{\textbf{100.00\%}} \\
% \bottomrule
% \end{tabular}
% % 
% \caption{Case Study.}
% \label{tab:case_study}
% \end{table*}