
\section{Methodology}
As illustrated in Figure~\ref{fig:model_pipline}, this section describes our LLM based query expansion method, LLM-QE. First, we introduce our query-expanded dense retrieval method (Sec.~\ref{model:preliminary}). Next, we use the DPO method to optimize LLMs for query expansion (Sec.~\ref{model:dpo-training}). Finally, we build a reward model to train LLM-QE by modeling the ranking preferences of both dense retrievers and LLMs (Sec.~\ref{model:reward-model}).



% 3.1
\subsection{Query Expanded Dense Retrieval}\label{model:preliminary}

Given a query $q$ and a document collection $\mathcal{D} = \{d_\text{1}, ..., d_\text{k}\}$, dense retrieval models~\cite{karpukhin2020dense,xiong2021approximate,cocondenser} first encode the query $q$ and the $i$-th document $d_i$ into vector representations $\vec{q}$ and $\vec{d}_i$ using PLMs, such as BERT~\cite{devlin2019bert}:
\begin{equation}\label{eq:encode}
\small
    \vec{q} = \text{BERT}_q (q),  \quad  \vec{d}_i = \text{BERT}_d (d_i).
\end{equation}
Then the score $S(q, d_i)$ is calculated to estimate the relevance between $q$ and $d_i$, followed by a KNN search~\cite{douze2024faiss} to retrieve the top-ranked documents to satisfy the user needs. 

Different from existing dense retrieval models, we introduce our query expanded dense retrieval method. Firstly, we prompt the LLM ($\mathcal{M}$) to produce a document-based query expansion:
\begin{equation}
\small
d^{\text{exp}} = \mathcal{M}(\text{Instruct}_\text{q2d}, q),
\end{equation}
where $\text{Instruct}_\text{q2d}$ denotes the instruction that asks LLMs to generate document-like query expansion outcomes~\cite{jagerman2023query}.
The representation of expanded query $\vec{q}^{\; \text{exp}}$ is calculated by averaging the representations of the raw query $q$ and the document-based query expansion $d^{\text{exp}}$ to enhance both unsupervised and supervised dense retrievers:
% $q$ and $d^{\text{exp}}$:
\begin{equation}\label{eq:qexp}
\small
    \vec{q}^{\; \text{exp}} = \frac{\vec{q} + \vec{d}^{\; \text{exp}}}{2}.
\end{equation}


To illustrate equilibria and dynamics of performative prediction games, we focus on a scenario in which a \emph{duopoly} of mortgage companies, i.e. banks, compete to sell loans to customers.

\paragraph{Customer Model:} In our game, each bank is trying to attract customers from a given population $\mathcal{P}$. We model this population as comprised of individuals with a single-dimensional type: we denote individual $j$'s type as $y_j \in [0,1]$. For simplicity, we assume that \(y\) represents the customer’s probability of repaying the loan\footnote{In practice, a customer's (normalized) credit score can be interpreted as a noisy observation of $y_j$. This also corresponds to credit scores being \emph{calibrated}.}, i.e., $y_j := \P[Y_j = 1]$, where $Y_j$ is a random variable such that $Y_j = 0$ means that $j$ defaults on their loan, and $Y_j = 1$ means they repay their loan. Customer types in the population are drawn from a known distribution $D_y$ supported on $[0,1]$. 

\paragraph{Game between Banks:} Each Bank \(i \in \{1, 2\}\) selects two parameters \( (\tau_i, \gamma_i) := \theta_i\), where:
\begin{itemize}
    \item \(\tau_i \in \{\tau_l,\tau_h\}\) is the credit score threshold for approving a customer\footnote{We restrict the bank to only pick between two thresholds, $\tau_l$ and $\tau_h$. However, we highlight how our results are affected when we expand the strategy space to $n > 2$ actions in our experiments of Appendix \ref{app:3gamma}.}. Specifically, a customer $j$ with credit score \(y_j\) is approved by Bank $i$ if and only if \(y_j \geq \tau_i\);
    \item \(\gamma_i \in \{\gamma_l, \gamma_h\}\) is the interest rate offered to approved customers.
\end{itemize}
We denote as shorthand the space of allowable thresholds by $\Gamma := [0,1]$ and allowable interests rates by $\Lambda := [0,1]$. %The latter is set without loss of generality---we simply normalize the rates to be at most $1$. 
% {\color{red} Vidya: just thinking about this but is it natural to restrict interest rate to $1$? I don't think it would affect the equilibrium structure of the game but theoretically I think the interest rate could be anything in $[0,\infty)$.} {\color{green} Guanghui: Could we say something like this is without loss of generality} \gua{changed.}\juba{I think we repeated this twice, the next sentence already had this}
The loan amount is normalized to $1$ in the entire paper, without loss of generality; in this case, if a customer chooses Bank $i$, and the customer is approved by the bank at an interest rate of $\gamma_i$, the expected utility for the bank is equal to
\[
(1+\gamma_i)\cdot \P[Y_i = 1]-\P[Y_i = 0] = (1+\gamma_i)y_i-(1-y_i).
\]


%In practice, the credit score \(y\) serves as a noisy observation of the true likelihood of the customer's repayment. 

\paragraph{Banks' Utilities:} For given parameter choices \(\theta_1 = (\tau_1, \gamma_1)\) by Bank 1 and \(\theta_2 = (\tau_2, \gamma_2)\) by Bank 2, a \emph{rational} customer with credit score $y$ acts as follows:

\begin{enumerate}
    \item \textbf{Qualified for a single bank}: 
        \begin{itemize}
        \item If \(\tau_1 \leq y < \tau_2\), the customer goes to Bank 1, as the score qualifies for Bank 1 but not Bank 2. Conversely, if \(\tau_2 \leq y < \tau_1\), the customer chooses Bank 2.
    \end{itemize}
    \item \textbf{Qualified for both banks}:
     \begin{itemize}
        \item If \(\tau_1, \tau_2 \leq y\) and \(\gamma_1 < \gamma_2\), the customer selects Bank 1 for its lower interest rate. Conversely, if \(\gamma_1 > \gamma_2\), the customer chooses Bank 2.
        \item If \(\gamma_1 = \gamma_2\), the customer picks each bank with probability $1/2$. 
    \end{itemize}
    \item \textbf{Unqualified for both banks}:
    \begin{itemize}
        \item If \(y < \tau_1\) and \(y < \tau_2\), the customer is rejected by both banks.
    \end{itemize}
\end{enumerate}

The expected reward for Bank 1, denoted as \(u_1(\theta_1, \theta_2)\), can then be expressed as:
\begin{align}\label{eq:utility}
    u_1(\theta_1, \theta_2) 
    &=  \mathbb{E}_{y \sim D_y} \left[ \mathbb{I}\{\underbrace{\tau_1 \leq y < \tau_2 \ \cup \ (\tau_1, \tau_2 \leq y \ \cap \ \gamma_1 < \gamma_2)}_{\text{accepted by Bank 1}}\} \cdot \big((1+\gamma_1)y - (1-y)\big) \right] \nonumber\\
    & + \frac{1}{2} \mathbb{E}_{y \sim D_y} \left[ \mathbb{I}\{\underbrace{\tau_1, \tau_2 \leq y \ \cap \ \gamma_1 = \gamma_2}_{\text{accepted by both Banks}}\} \cdot \big((1+\gamma_1)y - (1-y)\big) \right].
\end{align}
Note that the problem is \emph{symmetric}, i.e., the utility function for Bank 2 can be derived by swapping the roles of \(\theta_1\) and \(\theta_2\). I.e., $u_2(\theta_1, \theta_2) = u_1(\theta_2, \theta_1)$. 

% If a bank only attracts customers between thresholds $\tau_a$ and $\tau_b$, for $\tau_a<\tau_b$, we call $[\tau_a,\tau_b]$ the \emph{threshold} range for that bank. For example, if Bank $1$ sets a threshold of $\tau_1$, Bank $2$ a threshold of $\tau_2 > \tau_1$, and $\gamma_1 > \gamma_2$, then Bank 1 has a threshold range of $[\tau_1,\tau_2]$, while bank $2$ has a threshold range of $[\tau_2,1]$.
% Note that the parameters set by \emph{both} banks, i.e. $(\theta_1,\theta_2)$ both influence the threshold range for each of Bank 1 and 2.  If $\tau_1>\tau_2$, $\gamma_1>\gamma_2$, then $\tau_a>\tau_b$, and the bank does not attract any customers. 
% {\color{red} is it possible for $\tau_a > \tau_b$, leading to the bank never attracting customers?} \gua{if $\gamma_1>\gamma_2$, $\tau_1>\tau_2$, then it gets no customer. I think it also makes sense.}\juba{I think we said we wanted to delete the discussion of the threshold range, no?}

% \noindent \textbf{Discrete Model}   
% We now present the discrete version of our model, where the interest rates and thresholds are selected from finite sets \(\Gamma\) and \(\Lambda\), respectively, with $\tau\in[0,1], \gamma\in[0,1]$,  for all $\tau\in\Lambda$ and $\gamma\in\Gamma$, \(|\Gamma| = n\) and \(|\Lambda| = m\). Let \(p_1, p_2 \in \Delta(\Gamma \times \Lambda)\) represent the mixed strategies of the two banks, where \(\Delta(\Gamma \times \Lambda)\) denotes the set of probability distributions over the discrete decision space \(\Gamma \times \Lambda\).


% \begin{Remark}
%    Note that our proposed problem can be reformulated as a standard multi-player performative prediction problem \citep{narang2023multiplayer}. However, in our problem, the data distribution faced by each learner breaks the Lipschitzness assumption of previous work~\citep{hardt2023performative,narang2023multiplayer}. A small modification in one of the learner's thresholds can completely change how demand is allocated across both learners, as is often the case in Bertrand-style games. 
% \end{Remark} 

% \gua{I made some changes to Remark 1, please have a look}
\begin{Remark}
   Previous works in multi-learner performative prediction~\citep{narang2023multiplayer} resort to an insensitivity assumption, i.e., the data distribution faced by each player can only changes slightly when the parameters also change slightly; formally, the data distribution faced by each player is Lipschitz in their decisions. This is immediately not true in our setting: the bank slightly changing its parameters can completely changes the demand distribution of customers it faces. Intuitively, this is because of Bertrand-competition-style effects, where if two banks have similar rates, one bank that lowers their rate by a small amount suddenly captures the entire customer demand that is eligible for that rate.%\juba{made further light edits adding intuition}
   
   In Appendix \ref{Appendix:refumulation}, we discuss this problem more carefully by reformulating our problem in the standard multi-learner performative prediction form given by~\citep{narang2023multiplayer}. We show the distribution is not Lipschitz with respect to the parameters, and thus does not satisfy the insensitivity assumption. 
%Prior work~\citep{hardt2023performative,narang2023multiplayer} showed that, for a general multi-agent performative prediction framework to work, insensitivity assumptions are needed: in the \textbf{worst case}, they can construct settings where the insensitivity assumption does not hold and simple dynamics do not converge anymore. We add nuance to this picture. We will show that our dynamics often converge, even absent insensitivity assumptions, highlighting that while the impossibility results of previous work hold in the worst case, they may not hold in the ``average case'' and especially not in problems motivated by applications. In particular, we will show convergence to a variety of equilibria of our game, and often to symmetric Nash equilibria where insensitivity is immediately violated.
     
\end{Remark}



% \paragraph{Relationship to Performative Prediction} A central point of our work is to highlight that \textcolor{red}{needs writing from intro}. We highlight how our work specifically ties to ``Performative Prediction'' below:


%\textcolor{red}{needs a definition environment}



%Here, \(\E_{\theta_1, \theta_2}\) represents the expected utility of the banks over their respective strategies \((\theta_1, \theta_2)\). These inequalities ensure that neither bank can unilaterally improve its expected utility by deviating from its mixed strategy in the equilibrium.



%and  for all $\tau\in\Gamma$, we have $\tau\in\Lambda$, $(\tau,\gamma)\in[0,1]^2$. Let $\Gamma\times\Lambda$
%In this paper, we focus on the most fundamental case, where there are two choices for each parameter: $0\leq\tau_{\ell}<\tau_{h}\leq 1$, and $0\leq \gamma_{\ell}< \gamma_{h}\leq 1$. In this case, the utility for each pair of decisions forms a $4\times4$ matrix (given in Table \ref{tab:my-table}). We consider the canonical case where $\tau_{\ell}=\frac{1}{2+\gamma_{h}}$, and $\tau_{h}=\frac{1}{2+\gamma_{\ell}}.$ Note that these are natural choices for the thresholds, in the sense that, if there is only one bank and the interest rate is set to be $\gamma$, then $\frac{1}{2+\gamma}$ is the optimal threshold corresponding to the fixed $\gamma$.


%and the thresholds are chosen in $\Lambda=\{\tau^{(1)},\dots,\tau^{(m)}\}$. Here, we only assume that, for each $\gamma\in\Gamma$, there at least exist one $\tau\in\Lambda$ such that $f(\gamma,\tau,1)>0$. Note that this is a very minor assumption, in the sense that, if for a $\gamma$ such that $f(\gamma,\tau,1)<0$ for all $\tau\in\Lambda$, then adopting this decision will lead to negative utility regardless of the opponent's decision, and thus is not an interesting case. 

%\textcolor{red}{The model section is missing the dynamic version of the game. We should clearly define the one-shot and the dynamic game}
% we only considered one-shot case in our paper






\textbf{Unsupervised Dense Retrieval.} The unsupervised dense retrieval models, such as Contriever~\cite{izacard2021unsupervised}, are pre-trained to match text segments that share the same semantic information~\cite{fang2020cert,wu2020clear} and do not use the query-document relevance label during training~\cite{bajaj2016ms}.


Following~\citet{gao2023precise}, we can directly use Eq.~\ref{eq:qexp} to represent the expanded query by incorporating the information of raw query $q$ and the document-based query expansion $d^{\text{exp}}$. Then the relevance score $S(q, d_i)$ between the query $q$ and the candidate document $d_i$ can be calculated:
\begin{equation}
\small
    S(q, d_i) = \text{sim} (\vec{q}^{\; \text{exp}}, \vec{d}_i),
\end{equation}
where $\text{sim} (\cdot)$ is the similarity estimation function that conducts the dot product operation.

\textbf{Supervised Dense Retrieval.} Then we describe the details of training and inference processes in the supervised dense retrieval scenario.

\textit{Training.} During training the dense retriever augmented with query expansions, we first regard the expansion result $d^{\text{exp}}$ as the query and then calculate the similarity score $S(d^{\text{exp}}, d_i)$ between $d^{\text{exp}}$ and $d_i$:
\begin{equation}
\small
    S(d^{\text{exp}}, d_i) = \text{sim} (\vec{d}^{\; \text{exp}}, \vec{d}_i).
\end{equation}
Then we can contrastively train the encoder to learn matching signals from both $d^{\text{exp}}$ and the query-related document $d_*$ using the training loss $\mathcal{L}_\text{DR}$:
\begin{equation}
\small
 \mathcal{L}_\text{DR} = -\log\frac{e^{S(d^{\text{exp}},d_*)}}
    {e^{S(d^{\text{exp}},d_*)} + \sum_{d_-\in \mathcal{D}^-}{e^{S(d^{\text{exp}},d_-)}}},
\end{equation}
where $\mathcal{D}^-$ represents the set of negative documents, which are sampled from in-batch negatives~\cite{karpukhin2020dense}. 

\textit{Inference.} During retrieval, we can also represent the expanded query using Eq.~\ref{eq:qexp} and then calculate the relevance score:
\begin{equation}
\small
    S(q, d_i) = \text{sim} (\vec{q}^{\; \text{exp}}, \vec{d}_i).
\end{equation}




% 3.2
\subsection{Expanding Queries by Training LLMs with Preference Optimization}\label{model:dpo-training}
Existing query expansion methods~\cite{gao2023precise,li2024can} typically focus on directly prompting LLMs to generate various expansion results to mitigate hallucinations~\cite{brown2020language,thoppilan2022lamda}. To obtain more tailored and query-specific expansion results, we fine-tune the LLM to align with ranking preferences using the Direct Preference Optimization (DPO) method~\cite{amini2024direct}.

First, we prompt the LLM to generate multiple expansion documents by adjusting the temperature $\tau$ during sampling:
\begin{equation}
\small
d^{\text{exp}}_i \sim \mathcal{M}(\text{Instruct}_\text{q2d}, q).
\end{equation}
Thus, we can collect $k$ document-based query expansions $\mathcal{D}^q = \{d^{\text{exp}}_1, d^{\text{exp}}_2, \dots, d^{\text{exp}}_k\}$. 
Then we follow the DPO method to optimize the LLM ($\mathcal{M}$) using the loss function $\mathcal{L}(\mathcal{M}; \mathcal{M}^\text{Ref})$:
\begin{equation}
\small % 设置字体大小
\begin{aligned}
\label{eq:dpo}
& \mathcal{L}(\mathcal{M}; \mathcal{M}^\text{Ref}) = 
- \mathbb{E}_{(q, d^{\text{exp}}_+,d^{\text{exp}}_-) \sim \mathcal{P}} \Big[ \log \sigma \Big( \\
& \beta \log \frac{\mathcal{M}(d^{\text{exp}}_+ \mid q)}{\mathcal{M}^\text{Ref}(d^{\text{exp}}_+ \mid q)} - 
\beta \log \frac{\mathcal{M}(d^{\text{exp}}_- \mid q)}{\mathcal{M}^\text{Ref}(d^{\text{exp}}_- \mid q)} \Big) \Big],
\end{aligned}
\end{equation}
where $\sigma$ is the Sigmoid function and $\beta$ is a hyperparameter that controls the strength of the regulation from the reference model $\mathcal{M}^\text{Ref}$. $\mathcal{M}^\text{Ref}$ is frozen during DPO training. $\mathcal{P}$ is the dataset used for DPO training, which contains the triple ($q$, $d^{\text{exp}}_+$, $d^{\text{exp}}_-$). $d^{\text{exp}}_+$ and $d^{\text{exp}}_-$ are positive and negative responses, which are sampled from the document-based query expansions $\mathcal{D}^q$:
\begin{equation}
\small
    R(d^{\text{exp}}_+) > R(d^{\text{exp}}_-),
\end{equation}
where $R(d^{\text{exp}})$ is the reward model that is used to estimate the ranking preference of the document-based query expansion $d^{\text{exp}}$. The details of the ranking preference modeling are introduced in Sec.~\ref{model:reward-model}.

% 3.3
\subsection{Modeling Ranking Preferences for Rewarding Query Expansion Models}\label{model:reward-model}
For the given query $q$, the quality of the $i$-th document-based query expansion $d^{\text{exp}}_i$ is evaluated using the reward $R(d^{\text{exp}}_i)$ defined as:
\begin{equation}
\small
R(d^{\text{exp}}_i) = R_{\text{rank}}(d^{\text{exp}}_i) + R_{\text{ans}}(d^{\text{exp}}_i),
\end{equation}
where $d^{\text{exp}}_i \in \mathcal{D}^q$. $R_{\text{rank}}( d^{\text{exp}}_i)$ and $R_{\text{ans}}(d^{\text{exp}}_i)$ represent the rank-based reward score and the answer-based reward score, respectively. These scores are combined to model the ranking preference by capturing the relevance preference of the dense retriever and estimating the consistency with the question answering model.



\textbf{Rank-based Reward.} To assess the quality of the document-based query expansion $d^\text{exp}_i$, the most straightforward approach is to use a ranking score. Specifically, we calculate the Mean Reciprocal Rank (MRR) score by treating the ground truth document $d_*$ as the query and ranking the document-based query expansions $\mathcal{D}^q$:
\begin{equation}
\small
R_{\text{rank}}(d^{\text{exp}}_i) = \frac{1}{\text{Rank}(d_*, d^{\text{exp}}_i)},
\end{equation}
where $\text{Rank}(d_*, d^{\text{exp}}_i)$ represents the rank of $d^{\text{exp}}_i$ based on the relevant score $\text{sim}(d_*, d^{\text{exp}}_i)$. A higher reward score $R_{\text{rank}}(d^{\text{exp}}_i)$ indicates that the document-based query expansion $d^{\text{exp}}_i$ is more similar to the ground truth document $d_*$. 


\textbf{Answer-based Reward.} While the rank-based reward modeling accurately captures the preference of dense retrievers, it often leads to biases and fairness issues in reward modeling~\cite{dai2024bias}. Thus, relying solely on rank-based reward may cause the dense retrieval model to overfit to the inherent biases of dense retrievers. To address this challenge, we propose an answer-based reward modeling approach, which leverages the self-consistency of LLMs to calibrate the reward.

Specifically, we first instruct the LLM to generate an answer $y$ based on the given query $q$ and the ground truth document $d_*$:
\begin{equation}
\small
y = \mathcal{M}(\text{Instruct}_\text{q2a}, q, d_*),
\end{equation}
where $\text{Instruct}_\text{q2a}$ is the instruction that guides the LLM to generate a response to answer the query $q$. We then treat the LLM-generated answer $y$ as a query and rank the document-based query expansions $\mathcal{D}^q$ to compute the answer-based reward:  
\begin{equation}
\small
R_{\text{ans}}(d^{\text{exp}}_i) = \frac{1}{\text{Rank}(y, d^{\text{exp}}_i)},
\end{equation}
where $\text{Rank}(y, d^{\text{exp}}_i)$ denotes the rank of document $d^{\text{exp}}_i$ based on its relevance score $\text{sim}(y, d^{\text{exp}}_i)$.

Both $y$ and $d^{\text{exp}}_i$ are generated by the same LLM, using different instructions: $\text{Instruct}_\text{q2a}$ for generating the answer and $\text{Instruct}_\text{q2d}$ for generating the expansion documents. Higher-ranked documents indicate that $y$ and $d^{\text{exp}}_i$ share more semantic similarity, reflecting greater consistency and semantic agreement between $y$ and $d^{\text{exp}}_i$.






