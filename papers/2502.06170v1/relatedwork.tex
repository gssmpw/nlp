\section{Related Works}
Tabular machine learning methods have been widely applied in Earth science for tasks such as predicting environmental changes and understanding geographical phenomena. These methods, including popular algorithms like LightGBM \citep{ke2017lightgbm} and XGBoost \citep{chen2016xgboost}, are designed under the assumption that samples are independent and identically distributed, which limits their applicability in scenarios with spatial dependencies. While geographically weighted models, such as GWR \citep{fotheringham2009geographically} and GTWR \citep{fotheringham2015geographical}, have been introduced to address spatial heterogeneity by adjusting coefficients locally, they face significant limitations. GWR models fail to capture temporal evolution, and while GTWR extends this capability, both models struggle with nonlinearity and exhibit high computational complexity when applied to large datasets. GWR-RF \citep{wang2024geographically} combines GWR with Random Forest for nonlinearity but still suffers from local overfitting and dense weight matrices. GNNWR \citep{du2020geographically} balances global patterns and spatial variability through neural network-corrected coefficients but retains the complexity of dense spatial weights. GTNNWR \citep{wu2021geographically} further incorporates spatiotemporal heterogeneity but remains limited by the need to fit local variables and dense spatiotemporal weights, making these models prone to overfitting and computationally inefficient in large-scale applications.