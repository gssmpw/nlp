\section{Annotator's demographics}\label{app-demographics}

\Cref{tab:ann_demographics} shows the demographic information about the annotators. We additionally asked them to indicate their nationality and first language. 38 are Italian, 2 are Spanish, 2 are Chinese, and there is one person for each of the following nationalities: Italian-Argentinian, Italian-Romanian, Iranian, Greek, Russian, Kazakhstan, Indian, Moldovan, Persian, Dutch, Romanian. As regards their mother tongue: 38 chose Italian, 3 Spanish, Russian, Chinese, 2 Persian, Romanian and Greek, and 1 Hebrew, Hindi, Farsi and Dutch.

\begin{table}[]
    \centering\small
    \begin{tabular}{lc|c}
    \hline
          \multicolumn{2}{c}{Demographics}&  \#annotators\\
          \hline
          \multirow{4}{*}{Gender identity}&Female& 
    34\\
 & Male&20\\
 & Non-binary&2\\
 & Prefer not to say&1\\
 \hline
 \multirow{4}{*}{Generation}& Boomer&1\\
 & GenX&2\\
 & GenY&18\\
 & GenZ&36\\
 \hline
 \multirow{5}{*}{Ethnicity}& White&48\\
 & Asian&5\\
 & Mixed&2\\
 & Black&1\\
 & Caucasian&1\\
 \hline
 \multirow{4}{*}{Education level}& Bachelor degree&29\\
 & Master degree&18\\
 & Doctorate degree&5\\
 & High school diploma&5\\
  \hline
 \multirow{8}{*}{Employment}& Unemployed&16\\
 & Full-time&15\\
 & Part-time&10\\
 & Due to start a new job&3\\
 & Student&8\\
 & Not in paid work&2\\
 & Occasional work&1\\
 & Freelancer&2\\
 \hline
 \end{tabular}
    \caption{Sociodemographic information.}
    \label{tab:ann_demographics}
\end{table}

