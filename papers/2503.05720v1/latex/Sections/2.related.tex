\section{Related Work} \label{sec:related}
\textit{Canceling} is  ``an expression of agency, a choice to withdraw one’s attention from someone or something whose values, (in)action, or speech are so offensive, one no longer wishes to grace them with their presence, time and money'' \cite{Clark2020}. The phenomenon, which originated in Black Twitter \cite{Clark2015BlackTB, Ng2022}, has evolved into a mainstream practice within internet activism that shifted towards online censorship, silencing, and aggression with growing concerns about safety in social media platforms. Currently, there is a lack of NLP works on cancel culture with the exception of a dataset generated automatically by combining sentiment analysis and Named Entity Recognition \cite{erker-etal-2022-cancel}. Our work is the first attempt to develop a manually annotated corpus on this complex phenomenon.

The growing interest of the NLP community in moral pluralism \cite{graham2008moral,schwartz2012overview} resulted in the creation of resources \cite{araque2020moralstrength,hoover2020moral} aimed at analyzing the impact of morality in perceiving harmful contents \cite{stranisci2021expression,davani2024disentangling}. Moreover, there is an active line of research that studies the morality of LLMs \cite{abdulhai-etal-2024-moral,rottger-etal-2024-political}. Our work apply moral pluralism to understand canceling attitudes in social media platforms, where the morality of LLMs can have an impact in content moderation.