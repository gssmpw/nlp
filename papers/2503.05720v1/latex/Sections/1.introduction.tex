\section{Introduction}\label{introduction}
The recent interest of Natural Language Processing (NLP) scholars in morality is driven by the conviction that the moral stance of people shapes their view of the world \cite{forbes2020social} and motivates their behavior \cite{van-der-meer-etal-2023-differences}. In the context of social media interaction, the pluralistic nature of morality \cite{graham2008moral} is a proxy to understand their attitudes towards the increasing amount of polarizing events~\cite{falkenberg2024patterns} that generate an escalation of violence.

The so-called Cancel Culture \cite{Clark2020} is a representative example of how such a moral polarization works: celebrities' behaviors that are perceived as morally wrong by communities of users trigger violent reactions aimed at excluding them from the public sphere. Being able to automatically identify and mitigate this form of public shaming would be a crucial step in preserving the well-being of people in social media \cite{davani2024disentangling}. 

The main objective of our research is to provide the first study of canceling attitudes through the lens of people's different moral perspectives. To this aim, we developed the Canceling Attitudes DEtection (CADE) dataset: a corpus of canceling incidents gathered from YouTube. The corpus includes six videos regarding controversial events about celebrities and comments that have been annotated for the presence of canceling attitudes against them. Given the social relevance of the task, we involved three types of stakeholders (activists, researchers, and students) in a participatory annotation lab~\cite{Delgado2023} during which we constantly received feedback from them about the annotation process and the potential uses of CADE in downstream applications for content moderation. The moral perspectives of annotators are identified according to the Moral Foundations Theory (MFT) \cite{graham2008moral}, which enables the identification of people's moral profiles by ranking their attitudes towards five foundations. By developing CADE we investigate two research questions.

\textbf{RQ1: What is the impact of an individual's morality in evaluating canceling attitudes?}
CADE enables a systematic study of human disagreement that is not limited to individuals' sociodemographics \cite{sap2019risk,frenda-etal-2023-epic} but also considers their moral stance towards canceling attitudes. We clustered annotators on the basis of their moral profiles and observed how they evaluate YouTube comments on controversial events involving celebrities. The analysis shows that individuals' morality is event-focused: people with different moral profiles tend to evaluate differently the canceling attitudes against specific celebrities, independently of their demographic traits.%This dimension of variation appears to be independent from sociodemographic traits and must be considered to fully understand the nature of disagreement in controversial topics. 

\textbf{RQ 2: Do different LLMs align with different moral profiles in evaluating canceling attitudes?} We measured the moral stance towards canceling of $6$ LLMs, which has been compared with the morality of human annotators. We compared each LLM with annotations provided by people grouped along different axes: moral profile, type of stakeholder, and gender. Our results show that different LLMs exhibit different moral perspectives when they evaluate canceling attitudes. 

This paper is organized as follows. In \Cref{sec:related} we present relevant related work; in \Cref{sec:corpus} the creation of the corpus and in \Cref{sec:corpus-analysis} its analysis. In \Cref{sec:experiments} we present two experiments relative to our two RQs. \Cref{sec:discussion} summarizes our findings. 
