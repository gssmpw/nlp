\section{Conclusion}
In this paper, we presented the Canceling Attitude DEtection (CADE) dataset, a corpus specifically designed to investigate how people with different moral views of the world evaluate this phenomenon. The corpus, which includes six canceling incidents gathered from YouTube, has been annotated by annotators belonging to three categories of relevant stakeholders for this social issue (activists, researchers, and students) whose moral profiles have been obtained through the MFT questionnaires. 
%Results show that morality is an independent and significant source of disagreement among annotators, especially in the evaluation of specific events and celebrities. Therefore morality appears to be a crucial factor to understand how the ideological polarization of users operates in social media. LLMs also show significant variations in the classification of canceling attitudes that lead to their alignment with specific groups of annotators characterized by sociodemographic features and by their moral perspectives.

Future work will be devoted to expanding the study of canceling attitudes by including in the analysis different categories of targets (e.g., organizations). Additionally, we will adopt an intersectional approach to better understand how morality intersects with other sociodemographic factors in determining annotators' disagreement, and to which extent LLMs are able to grasp this complex issue.  