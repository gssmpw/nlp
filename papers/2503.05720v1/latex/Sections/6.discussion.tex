\section{Discussion} \label{sec:discussion}
%Findings of our research show that canceling attitudes represent a very subjective phenomenon that depends on a compound of two factors: the type of event that triggers canceling attitudes, the moral views of annotators and their sociodemographic features.

A first result that emerges from our study is that \textbf{canceling attitudes heavily depend on the types of controversial event that triggers them}. YouTube comments towards certain celebrities are systematically considered less acceptable by all annotators, regardless of their background. This demonstrates that the so-called Cancel Culture does not have the same impact on all its victims.%, showing that the characteristic of celebrities and their behavior play a crucial role in triggering violent reactions.  
%Regardless their background, annotators find that celebrities such as Lizzo and Ellen DeGenres are more prone to be attacked and to receive unacceptable YouTube comments, while celebrities like J.K. Rowling and Andrew Tate tend to be defended the most. Such a general behavior can be interpreted through different lenses: the socio-demographic characteristics of celebrities targeted by canceling; the political leaning of Newsrooms that report the event; the category of behavior that is condemned by online communities.

This event-based variation explains human disagreement in the evaluation of canceling attitudes. Rather than operating at a general level, \textbf{morality, gender, and stakeholder type drive the disagreement of annotators against specific celebrities}. In this context, morality appears to have a significant influence, especially in the evaluation of unacceptability. In $4$ cases out of $6$ annotators with different moral profiles show a statistically significant variation (\Cref{tab:stance-acc_all}) in judging unacceptability. This highlights the importance of developing resources that better represent the communicative context in which potentially harmful content is spread. % in order to have more in-depth insights on human disagreement   

%The analysis of annotations broken down by gender, stakeholder, and moral profile demonstrates that people morality represents an independent source of subjectivity on the evaluation of canceling attitudes. Especially in the recognition of stance, people belonging to different moral clusters tend to diverge the most. [...] 

The analysis of LLMs behavior in recognizing canceling attitudes shows similar patterns to the ones observed in the study of human disagreement. LLMs align with specific groups about certain celebrities. Additionally, %but this is not the only observed behavior. %For instance, all LLMs converge with men and people from a specific moral cluster in evaluating the unacceptability of comments towards Andrew Tate. 
\textbf{different models tend to align with different categories of annotators characterized by their sociodemographic characteristics and morality}. Controlling these patterns of alignment would be a significant step in the implementation of fairer technologies for reducing the ideological polarization between users on social media platforms.  

%The analysis of LLMs classifications shows that the canceling attitudes detection task is really challenging for these technologies that reach an F-score below $0.4$ in the recognition of stance and below $0.5$ in the recognition of acceptability. Models performance broken down by morality, gender, and stakeholders appear to be coherent with the analysis of human annotations: different models align differently along different axes. [...]