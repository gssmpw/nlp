% This must be in the first 5 lines to tell arXiv to use pdfLaTeX, which is strongly recommended.
\pdfoutput=1
% In particular, the hyperref package requires pdfLaTeX in order to break URLs across lines.

\documentclass[11pt]{article}

% Change "review" to "final" to generate the final (sometimes called camera-ready) version.
% Change to "preprint" to generate a non-anonymous version with page numbers.
\usepackage[preprint]{acl}

% Standard package includes
\usepackage{times}
\usepackage{latexsym}

% For proper rendering and hyphenation of words containing Latin characters (including in bib files)
\usepackage[T1]{fontenc}
% For Vietnamese characters
% \usepackage[T5]{fontenc}
% See https://www.latex-project.org/help/documentation/encguide.pdf for other character sets

% This assumes your files are encoded as UTF8
\usepackage[utf8]{inputenc}

% This is not strictly necessary, and may be commented out,
% but it will improve the layout of the manuscript,
% and will typically save some space.
\usepackage{microtype}

% This is also not strictly necessary, and may be commented out.
% However, it will improve the aesthetics of text in
% the typewriter font.
\usepackage{inconsolata}

%Including images in your LaTeX document requires adding
%additional package(s)
\usepackage{graphicx}
\usepackage{xcolor}
\usepackage{soul}
\usepackage{multirow}
\usepackage{subfigure} 
\usepackage{booktabs}
\usepackage{enumitem}
\usepackage{amsmath}

\newcommand{\fixme}[1]{{\color{red} *** {\textbf{Rajesh: }}\color{blue}{#1}}{\color{red} ***}}
% If the title and author information does not fit in the area allocated, uncomment the following
%
%\setlength\titlebox{<dim>}
%
% and set <dim> to something 5cm or larger.

\title{That is Unacceptable: the Moral Foundations of Canceling}

% Author information can be set in various styles:
% For several authors from the same institution:
% \author{Author 1 \and ... \and Author n \\
%         Address line \\ ... \\ Address line}
% if the names do not fit well on one line use
%         Author 1 \\ {\bf Author 2} \\ ... \\ {\bf Author n} \\
% For authors from different institutions:
% \author{Author 1 \\ Address line \\  ... \\ Address line
%         \And  ... \And
%         Author n \\ Address line \\ ... \\ Address line}
% To start a separate ``row'' of authors use \AND, as in
% \author{Author 1 \\ Address line \\  ... \\ Address line
%         \AND
%         Author 2 \\ Address line \\ ... \\ Address line \And
%         Author 3 \\ Address line \\ ... \\ Address line}

\iffalse
\author{First Author \\
  Affiliation / Address line 1 \\
  Affiliation / Address line 2 \\
  Affiliation / Address line 3 \\
  \texttt{email@domain} \\\And
  Second Author \\
  Affiliation / Address line 1 \\
  Affiliation / Address line 2 \\
  Affiliation / Address line 3 \\
  \texttt{email@domain} \\}
\fi

\author{
 \textbf{Soda Marem Lo\textsuperscript{1,2}},
 \textbf{Oscar Araque\textsuperscript{3}},
 \textbf{Rajesh Sharma\textsuperscript{4}},
  \textbf{Marco Antonio Stranisci\textsuperscript{1,2}},
\\
\\
\\
 \textsuperscript{1}Università degli Studi di Torino,
 \textsuperscript{2}aequa-tech,
  \textsuperscript{3}Universitad Complutense de Madrid,
  \textsuperscript{4}Tartuk {Ü}likool
  \\
  \small{
   \textbf{Correspondence:} \href{mailto:marcoantonio.stranisci@unito.it}{marcoantonio.stranisci@unito.it}
  }
}


%---added packages--------
\usepackage{cleveref}
\usepackage{url}
%---------------------


\begin{document}
\maketitle
\begin{abstract}
Canceling is a morally-driven phenomenon that hinders the development of safe social media platforms and contributes to ideological polarization. To address this issue we present the \textbf{C}anceling \textbf{A}ttitudes \textbf{De}ection (CADE) dataset, an annotated corpus of canceling incidents aimed at exploring the factors of disagreements in evaluating people canceling attitudes on social media. Specifically, we study the impact of annotators' morality in their perception of canceling, showing that morality is an independent axis for the explanation of disagreement on this phenomenon. Annotator's judgments heavily depend on the type of controversial events and involved celebrities. This shows the need to develop more event-centric datasets to better understand how harms are perpetrated in social media and to develop more aware technologies for their detection. 

\textbf{WARNING:} the CADE corpus could contain racist, sexist, violent, and generally offensive content.
\end{abstract}


%-------------sections------------------------------


\section{Introduction}
\IEEEPARstart{I}{n} recent years, flourishing of Artificial Intelligence Generated Content (AIGC) has sparked significant advancements in modalities such as text, image, audio, and even video. 
Among these, AI-Generated Image (AGI) has garnered considerable interest from both researchers and the public.
Plenty of remarkable AGI models and online services, such as StableDiffusion\footnote{\url{https://stability.ai/}}, Midjourney\footnote{\url{https://www.midjourney.com/}}, and FLUX\footnote{\url{https://blackforestlabs.ai/}}, offer users an excellent creative experience.
However, users often remain critical of the quality of the AGI due to image distortions or mismatches with user intentions.
Consequently, methods for assessing the quality of AGI are becoming increasingly crucial to help improve the generative capabilities of these models.

Unlike Natural Scene Image (NSI) quality assessment, which focuses primarily on perception aspects such as sharpness, color, and brightness, AI-Generated Image Quality Assessment (AGIQA) encompasses additional aspects like correspondence and authenticity. 
Since AGI is generated on the basis of user text prompts, it may fail to capture key user intentions, resulting in misalignment with the prompt.
Furthermore, authenticity refers to how closely the generated image resembles real-world artworks, as AGI can sometimes exhibit logical inconsistencies.
While traditional IQA models may effectively evaluate perceptual quality, they are often less capable of adequately assessing aspects such as correspondence and authenticity.

\begin{figure}\label{fig:radar}
    \centering
    \includegraphics[width=1.0\linewidth]{figures/radar_plot.pdf}
    \caption{A comparison on quality, correspondence, and authenticity aspects of AIGCIQA2023~\cite{wang2023aigciqa2023} dataset illustrates the superior performance of our method.}
\end{figure}

Several methods have been proposed specifically for the AGIQA task, including metrics designed to evaluate the authenticity and diversity of generated images~\cite{gulrajani2017improved,heusel2017gans}. 
Nevertheless, these methods tend to compare and evaluate grouped images rather than single instances, which limits their utility for single image assessment.
Beginning with AGIQA-1k~\cite{zhang2023perceptual}, a series of AGIQA databases have been introduced, including AGIQA-3k~\cite{li2023agiqa}, AIGCIQA-20k~\cite{li2024aigiqa}, etc.
Concurrently, there has been a surge in research utilizing deep learning methods~\cite{zhou2024adaptive,peng2024aigc,yu2024sf}, which have significantly benefited from pre-trained models such as CLIP~\cite{radford2021learning}. 
These approaches enhance the analysis by leveraging the correlations between images and their descriptive texts.
While these models are effective in capturing general text-image alignments, they may not effectively detect subtle inconsistencies or mismatches between the generated image content and the detailed nuances of the textual description.
Moreover, as these models are pre-trained on large-scale datasets for broad tasks, they might not fully exploit the textual information pertinent to the specific context of AGIQA without task-specific fine-tuning.
To overcome these limitations, methods that leverage Multimodal Large Language Models (MLLMs)~\cite{wang2024large,wang2024understanding} have been proposed.
These methods aim to fully exploit the synergies of image captioning and textual analysis for AGIQA.
Although they benefit from advanced prompt understanding, instruction following, and generation capabilities, they often do not utilize MLLMs as encoders capable of producing a sequence of logits that integrate both image and text context.

In conclusion, the field of AI-Generated Image Quality Assessment (AGIQA) continues to face significant challenges: 
(1) Developing comprehensive methods to assess AGIs from multiple dimensions, including quality, correspondence, and authenticity; 
(2) Enhancing assessment techniques to more accurately reflect human perception and the nuanced intentions embedded within prompts; 
(3) Optimizing the use of Multimodal Large Language Models (MLLMs) to fully exploit their multimodal encoding capabilities.

To address these challenges, we propose a novel method M3-AGIQA (\textbf{M}ultimodal, \textbf{M}ulti-Round, \textbf{M}ulti-Aspect AI-Generated Image Quality Assessment) which leverages MLLMs as both image and text encoders. 
This approach incorporates an additional network to align human perception and intentions, aiming to enhance assessment accuracy. 
Specially, we distill the rich image captioning capability from online MLLMs into a local MLLM through Low-Rank Adaption (LoRA) fine-tuning, and train this model with human-labeled data. The key contributions of this paper are as follows:
\begin{itemize}
    \item We propose a novel AGIQA method that distills multi-aspect image captioning capabilities to enable comprehensive evaluation. Specifically, we use an online MLLM service to generate aspect-specific image descriptions and fine-tune a local MLLM with these descriptions in a structured two-round conversational format.
    \item We investigate the encoding potential of MLLMs to better align with human perceptual judgments and intentions, uncovering previously underestimated capabilities of MLLMs in the AGIQA domain. To leverage sequential information, we append an xLSTM feature extractor and a regression head to the encoding output.
    \item Extensive experiments across multiple datasets demonstrate that our method achieves superior performance, setting a new state-of-the-art (SOTA) benchmark in AGIQA.
\end{itemize}

In this work, we present related works in Sec.~\ref{sec:related}, followed by the details of our M3-AGIQA method in Sec.~\ref{sec:method}. Sec.~\ref{sec:exp} outlines our experimental design and presents the results. Sec.~\ref{sec:limit},~\ref{sec:ethics} and~\ref{sec:conclusion} discuss the limitations, ethical concerns, future directions and conclusions of our study.
\section{Related Works}


\noindent\textbf{3D Point Cloud Domain Adaptation and Generalization.}
Early endeavors within 3D domain adaptation (3DDA) focused on extending 2D adversarial methodologies~\cite{qin2019pointdan} to align point cloud features. Alternative methods have delved into geometry-aware self-supervised pre-tasks. Achituve \etal~\cite{achituve2021self} introduced DefRec, a technique employing self-complement tasks by reconstructing point clouds from a non-rigid distorted version, while Zou \etal~\cite{zou2021geometry} incorporating norm curves prediction as an auxiliary task. Liang \etal~\cite{liang2022point} put forth MLSP, focusing on point estimation tasks like cardinality, position, and normal. SDDA~\cite{cardace2023self} employs self-distillation to learn the point-based features. Additionally, post-hoc self-paced training~\cite{zou2021geometry,fan2022self,park2023pcadapter} has been embraced to refine adaptation to target distributions by accessing target data and conducting further finetuning based on prior knowledge from the source domain.
In contrast, the landscape of 3D domain generalization (3DDG) research remains nascent. Metasets~\cite{huang2021metasets} leverage meta-learning to address geometric variations, while PDG~\cite{wei2022learning} decomposes 3D shapes into part-based features to enhance generalization capabilities.
Despite the remarkable progress, existing studies assume that objects in both the source and target domains share the same orientation, limiting their practical application. This limitation propels our exploration into orientation-aware 3D domain generalization through intricate orientation learning.


\noindent\textbf{Rotation-generalizable Point Cloud Analysis.}
Previous works in point cloud analysis~\cite{qi2017pointnet, wang2019dynamic} enhance rotation robustness by introducing random rotations to augment point clouds. {However, generating a comprehensive set of rotated data is impractical, resulting in variable model performance across different scenes. To robustify the networks \wrt randomly rotated point clouds,} rotation-equivariance methods explore equivalent model architectures by incorporating equivalent operations~\cite{su2022svnet, Deng_2021_ICCV, luo2022equivariant} or steerable convolutions~\cite{chen2021equivariant, poulenard2021functional}.
Alternatively, rotation-invariance approaches aim to identify geometric descriptors invariant to rotations, such as distances and angles between local points~\cite{chen2019clusternet, zhang2020global} or point norms~\cite{zhao2019rotation, li2021rotation}. Besides, {Li \etal~\cite{li2021closer} have explored disambiguating the number of PCA-based canonical poses, while Kim \etal~\cite{kim2020rotation} and Chen \etal~\cite{chen2022devil} have transformed local point coordinates according to local reference frames to maintain rotation invariance. However, these methods focus on improving in-domain rotation robustness, neglecting domain shift and consequently exhibiting limited performance when applied to diverse domains. This study addresses the challenge of cross-domain generalizability together with rotation robustness and proposes novel solutions.} 

\noindent\textbf{Intricate Sample Mining}, aimed at identifying or synthesizing challenging samples that are difficult to classify correctly, seeks to rectify the imbalance between positive and negative samples for enhancing a model's discriminability. While traditional works have explored this concept in SVM optimization~\cite{felzenszwalb2009object}, shallow neural networks~\cite{dollar2009integral}, and boosted decision trees~\cite{yu2019unsupervised}, recent advances in deep learning have catalyzed a proliferation of researches in this area across various computer vision tasks. For instance, 
Lin \etal~\cite{lin2017focal} proposed a focal loss to concentrate training efforts on a selected group of hard examples in object detection, while Yu \etal~\cite{yu2019unsupervised} devised a soft multilabel-guided hard negative mining method to learn discriminative embeddings for person Re-ID. Schroff \etal~\cite{schroff2015facenet} introduced an online negative exemplar mining process to encourage spherical clusters in face embeddings for individual recognition, and Wang \etal~\cite{wang2021instance} designed an adversarially trained negative generator to yield instance-wise negative samples, bolstering the learning of unpaired image-to-image translation. In contrast to existing studies, our work presents the first attempt to mitigate the orientational shift in 3D point cloud domain generalization, by developing an effective intricate orientation mining strategy to achieve orientation-aware learning.


\section{The CADE Dataset}\label{sec:corpus}
In this section we present the process that led to the creation of the Canceling Attitudes DEtection (CADE) dataset: a corpus of YouTube comments annotated for the study of canceling attitudes against controversial events. The section is organized as follows: we first present the data collection (\Cref{collection}) phase, then we describe the design of the annotation task (\Cref{annotation-task}), and present the Annotation Lab: the participatory approach that we adopted to recruit annotators (\Cref{annotation-lab}).


\subsection{Data collection}\label{collection}
As described in \Cref{sec:related}, cancel culture is characterized by the online public shaming of celebrities for their actions or behaviors. In a NLP perspective, this can be conceived as an event-centric task \cite{chen2021event}, where the characteristics of the event orient the interest of the research. Relying on this assumption we chose YouTube %\footnote{\url{https://www.youtube.com/}} 
 as source for our data collection. We leveraged the public APIs
 %\footnote{\url{https://developers.google.com/youtube/v3}}  
  to obtain communicative situations where the presence of canceling attitudes can be assessed. 

Firstly, we selected controversial events related to six celebrities that refer to various topics: J.K. Rowling (homo-transphobia), Kanye West (antisemitism), Lizzo (harassment), Halle Bailey (anti-woke culture), Ellen DeGeneres (bullying), Andrew Tate (sexual assault).
For each of them, we manually selected a video of a news broadcast reporting the event the target celebrity got canceled for. Since the annotators had to watch it before completing the annotation task, we opted for the most viewed video among those that did not exceed 4 minutes. For each video we extracted all the comments and randomly sampled 350 comments for each celebrity, resulting in a total of $2,100$ texts for the annotation. 


\subsection{Design of the annotation task}\label{annotation-task} 
The design of the annotation task takes into consideration two research needs: the identification of annotators' moral profiles and the definition of an annotation scheme that is effective in representing the complexity of social interactions canceling attitudes rely on.

\paragraph{Moral Foundation Questionnaire.} 
In social psychology, a common method for eliciting people's moral profile is the administration of questionnaires \cite{graham2013moral,hinz2005investigating}, which have been recently used in NLP to assess the moral stance of LLMs \cite{abdulhai-etal-2024-moral} and people \cite{davani2024disentangling}. Coherently with this research, we chose the 30-item Moral Foundation Questionnaire (MFQ30) \cite{graham2013moral}\footnote{\url{https://moralfoundations.org/questionnaires/}}, consisting of two blocks. In the first block, respondents have to reply to the question \textit{When you decide whether something is right or wrong, to what extent are the following considerations relevant to your thinking?}, rating 15 sentences on a scale from 0 (This consideration has nothing to do with my judgments of right and wrong) to 5 (This is one of the most important factors when I judge right and wrong). The second assignment is to read the 15 sentences and indicate their agreement or disagreement using a scale from 0 (strongly agree) to 5 (strongly disagree). %--- complete questionnaire in Appendix \ref{app-mfq}. 
By aggregating respondents' replies, it is possible to elicit which moral foundations contribute the most to their moral profile.

\paragraph{Annotation scheme.}
The annotation scheme is composed of two axes: stance \cite{aldayel2021stance} and acceptability \cite{forbes2020social}. The former is adopted to annotate the stance of the comment towards the celebrity; acceptability is adopted to evaluate whether the comment is perceived as morally unacceptable by the annotator. 

For each of the six target celebrities, we prepared a summary of the controversial event that introduced the topic to the annotators (Appendix \ref{app-event_description}) and then asked them to watch a YouTube video 
% that presents 
on the controversial event (Appendix \ref{app-annotation_materials}). 
Following this step, they could annotate
% A second step is the annotation of 
the stance and the unacceptability of comments about celebrities. First, annotators must evaluate whether the YouTube user intended to attack, defend or was neutral towards the controversial event involving the target celebrity (stance).
Annotators must then evaluate the social unacceptability of the comment, choosing on a scale that ranges from 1 (totally acceptable) to 4 (totally unacceptable). %This 

\subsection{Annotation Lab}\label{annotation-lab}
Since our research aims to identify how people with specific moral profiles perceive canceling attitudes in a realistic scenario, our annotator recruitment strategies focused on the involvement of people who have specific interests in the issue. We engaged with three stakeholders: activists against online discrimination, AI researchers, and NLP students. For each stakeholder we organized an annotation lab, which is designed following the literature on Participatory AI \cite{Delgado2023}: people are not only recruited as annotators, but are involved during the whole dataset creation process. For this reason, we involved stakeholders rather than relying on crowdsourcing annotator platforms. 

The annotation lab is structured in three main steps: \textit{i)} engagement: we presented a set of slides through a shared video in which we provided a definition of the phenomenon of cancel culture, we shared our research objective and its social impact and explain the whole annotation process; \textit{ii)} annotation: people were asked to fill out the MFT questionnaire and to annotate YouTube comments according to our annotation scheme; \textit{iii)} feedback: we asked participants to provide feedback about the annotation process and share their ideas about potential applications of a technology for the identification of canceling attitudes. To this aim, we adopted two methodologies: the administration of a checkout questionnaire, and the organization of two focus groups. 

\paragraph{Feedback questionnaire.} We administered the questionnaire to all the involved stakeholders. It included 7 questions (Appendix \ref{app-checkout_qst}) about three aspects of the annotation lab: evaluating the experience in terms of its emotional impact and difficulty; providing improvements to the annotation scheme; and suggesting downstream applications of a technology trained on such a resource.

\paragraph{Focus group.} We organized one focus group aimed at students and one aimed at activists. During these meetings, we presented the preliminary results of the annotation task, and kicked off a semi-structured discussion to collect their feedback along three topics of interest: \textit{i)} evaluation of the data creation process; \textit{ii)} the relevance of morality for the task; \textit{iii)} the co-design of NLP technologies based on the dataset. 


\section{Corpus Analysis}
\label{sec:corpus-analysis}
In this section we analyse the CADE corpus.\footnote{The repository of the dataset will be made available under a CC BY license after the anonymity period.} We first describe the composition of annotators (\Cref{ss:process}), then the agreement between them (\Cref{ss:iaa}), finally we describe whether annotators moral profiles correlate with their sociodemographic traits (\Cref{ss:moral_profiles}).
% has an independent impact in their perception of canceling attitudes. 

\subsection{Annotation Process} \label{ss:process}
The annotation task involved  $57$ annotators belonging to three stakeholders: $30$ students, $17$ activists, and $10$ researchers. In addition, each annotator voluntarily shared their information about gender identity, and age, which was grouped into generations (Boomer, GenX, GenY and GenZ), nationality, ethnicity, education level and employment status.
% , and stakeholder type (student, researcher, activists). 
All the sociodemographic data are reported in \Cref{tab:ann_demographics} in Appendix \ref{app-demographics}.
%, the annotators could always write their answers instead of choosing among the provided options.
Having adopted a participatory approach to annotators recruitment (\Cref{annotation-lab}) that focuses on engaging with people who are interested in the phenomenon rather than hiring crowdworkers, the pool of annotators is not balanced along all sociodemographic axes. 

Each annotator is asked to annotate a subset of $210$ comments gathered from YouTube (\Cref{collection}): $35$ for each celebrity and the controversial event related to them. After cleaning unrelated comments, the final corpus includes $2,094$ texts and $11,935$ annotations. Each text has been annotated an average of $5.7$ times (with a median of $6$). Annotators had the option to refrain from assigning a label. We eliminated those who did not finish more than one-third of the task.


\subsection{Inter Annotator Agreement} \label{ss:iaa}
To assess the Inter Annotator Agreement (IAA) between annotators we employed Krippendorff's Alpha \cite{krippendorff2011computing}, which handles both agreement by chance (as the more common Cohen's Kappa agreement score), and incomplete annotations. We computed the IAA per sample and averaged them, resulting in a moderate agreement on stance ($0.501$), and a fair agreement on acceptability ($0.222$). The strong difference between the two tasks highlights how people tend to agree more when judging YouTube users' intentions rather than when they are asked to evaluate which messages they consider unacceptable \cite{davani2024disentangling}
%, compared to defining their boundaries in what can be considered acceptable. 
Labeling %a user's stance 
the stance expressed by a comment results less subjective than stating %what 
whether the comment is %is
to consider unacceptable. We report the scores broken down by sample in \Cref{tab:sample-iaa} in Appendix \ref{app-annotation_materials}.

\subsection{Annotators Moral Profiles} \label{ss:moral_profiles}
As a third part of the analysis of our annotation corpus, we assign annotators' moral profiles by leveraging the results of the MFT Questionnaire (\Cref{annotation-task}) and test to which extent they correlate with the following sociodemographic characteristics: gender, age, and stakeholder type.\footnote{Since the recruitment of stakeholder has been performed in European countries, we decided not to include %the other collected demographics 
nationality and ethnicity as variables of our study, as they result to be unbalanced.%because they are more unbalanced across the annotators.
 We also excluded the education level and employment status because they are strongly dependent on the type of stakeholders involved during the annotation, which is the only social condition monitored by design.} %we were interested in exploring these traits as potential stakeholders, the only variable monitored by design.}

We first computed the moral profile of annotators by obtaining their scores over the five foundations (care, fairness, loyalty, authority and purity). Each foundation is assigned a score between 0 (irrelevant) and 5 (very relevant). Together, the five scores represent the moral profile of the annotator.   

We then computed the Pearson coefficient between these profiles and gender, age, and stakeholder type. We found no statistically significant relationship between traits and moral foundations. This suggests that when annotators are grouped based on demographics, their moral profiles exhibit high variability and sparsity.

Given these preliminary results, we assigned a moral profile to each annotator through a clustering process that relies on their replies to the questionnaire. Annotators were represented with a vector with 5 dimensions, where each value in the vector represents the score obtained by the annotator for each moral foundation. 
We computed the pairwise distance with Euclidean metric and performed Agglomerative Clustering, which is preferred because it does not require setting the number of clusters as a parameter. We used Ward's linkage criterion;  to choose the best number of clusters we relied on three intrinsic evaluation metrics that do not need ground truth labels, namely Silhouette Coefficient \cite{ROUSSEEUW198753}, Calinski Harabaz Index \cite{Caliński01011974} and Davies Bouldin Index \cite{Bouldin1979}. We obtained $2$ clusters, Cluster\_0 (CL\_0) with $41$ annotators, and Cluster\_1 (CL\_1) with $16$.

\begin{figure}
    \centering
    \includegraphics[width=0.75\columnwidth]{latex/images/cl_foundatoins.png}
    \caption{Heatmap visualization for cluster analysis}
    \label{fig:cluster-foundations}
\end{figure}

We performed a qualitative analysis of the obtained moral clusters to understand which patterns emerge from their composition.
First, we analyzed whether the clusters were consistent with the scores that individuals obtained on each foundation. The heatmap shown in \Cref{fig:cluster-foundations} represents 
% the correlation between the two clusters and the moral foundations. 
the mean score of the five foundations for each cluster. Looking at the higher divergence, we computed the Pearson coefficient between the clusters and each of the moral foundations, reporting a correlation with loyalty ($r = 0.598$ with $p-value = 9.191 \cdot 10^{-7}$), authority ($r = 0.759$ with $p-value = 7.562 \cdot 10^{-12}$) and purity ($r = 0.792$ with $p-value = 2.063 \cdot 10^{-13}$).

These results are theoretically motivated by the MFT theory, which distinguishes between individual binding foundations (care, harm), rooted on the preservation of individual freedom, and group-binding foundations (loyalty, authority, and purity) which focus on the duties of individuals towards their social groups. In this sense, cluster\_1 is characterized by a higher moral attitude towards belonging to a group.

\begin{table}[]
    \centering\small
    \resizebox{1\columnwidth}{!}{
    \begin{tabular}{llcc}
    \hline
          \multicolumn{2}{c}{Demographics}&   CL\_0& CL\_1\\
    \hline
  \multirow{4}{*}{Gender identity} &Female& 23 (67.7\%) & 11 (32.3\%) (\\
 & Male& 16 (80\%) & 4 (20\%)\\
 & Non-binary& 2 (100\%)&-\\
 & Prefer not to say& -&1 (100\%)\\
 \hline
 \multirow{4}{*}{Generation}& Boomer& 1 (100\%)&-\\
 & GenX& 1 (50\%) &1 (50\%)\\
 & GenY& 13 (73\%) & 5 (27\%)\\
          &GenZ& 
     26 (73\%)& 10 (27\%)\\
 \hline
 \multirow{3}{*}{Stakeholder}& Activist& 15 (89\%) &2 (11\%)\\
 & Researcher& 7 (70\%)&3 (30\%)\\
 & Student& 19 (64\%) &11 (36\%)\\
 \hline
 \end{tabular}}
    \caption{Composition of the clusters (CL\_0 and CL\_1) in respect to annotators' gender identity, generation and role as stakeholder}
    \label{tab:cl-composition}
\end{table}

Once moral clusters have been validated against the MFT, we investigated again the presence of sociodemographic patterns in clusters to gain more insights from their potential correlation with moralities.
\Cref{tab:cl-composition} shows the composition of moral clusters broken down by annotators' gender, generation, and role as stakeholders. As it can be observed, the distribution of moral clusters along the gender axis is uniform: 32\% of women and 20\% of men belong to cluster\_1, showing a distribution that can be explained by annotators' gender imbalance. This is even more emphasized if generations are observed: moral clusters are perfectly distributed in Generation Y and Generation Z, despite the age of annotators being highly unbalanced towards the latter. This suggests that, in the context of this research, age is not a factor in determining the morality of annotators. Observing the distribution of moral clusters among stakeholders shows interesting patterns, instead. If on one side 36\% of students and 30\%  of researchers belong to cluster\_1, only 11\% of activists belong to this cluster. 

\section{Experiments}
\begin{table}[t]
\centering
{\resizebox{\columnwidth}{!}{
\begin{tabular}{lccccccc}
\toprule
\multicolumn{1}{c}{\multirow{2}{*}{\textbf{Model}}} & \multirow{2}{*}{\begin{tabular}[c]{@{}c@{}}\textbf{Parameter}\\ \textbf{Scale}\end{tabular}}  & \multicolumn{5}{c}{\textbf{Multifacet}} & \multirow{2}{*}{\textbf{Average}} \\ \cmidrule{3-7}  
\multicolumn{1}{c}{} &  & \multicolumn{1}{l}{\textbf{AE}} & \multicolumn{1}{l}{\textbf{FL}} & \multicolumn{1}{l}{\textbf{Ko}} & \multicolumn{1}{l}{\textbf{MT}} & \multicolumn{1}{l}{\textbf{SI}} & \\ \midrule
\textit{Open-Source Models} \\ \midrule
Solar-10.7B-instruct & 10.7B & 3.30 & 3.31 & 3.09 & 3.19 & 3.08 & 3.19  \\  
Gemma-2-9b-it & 9B & 4.10 & 3.80 & 4.26 & 4.15 & 3.92 & 4.05  \\ 
\midrule
\multicolumn{8}{l}{\textit{Open-source Models} $+$ \textit{KD (Fine-tuning on \textbf{\textsc{SysGen}} dataset)}} \\
\midrule 
Solar-10.7B-instruct & 10.7B & 3.97 & 3.73 & 3.64 & 3.98 & 3.52 & 3.76 (+0.57) \\ 
Gemma-2-9b-it & 9B & 4.40 & 4.04 & 4.30 & 4.23 & 4.18 & 4.23 (+0.18) \\ 
\bottomrule
\end{tabular}}}
\caption{
We conduct a knowledge distillation (KD) experiments leveraging data generated by \textsc{SysGen} pipeline using Phi-4.}
\label{tab:knowledge_distillation}
\vspace{-0.3cm}
\end{table}
\begin{table*}[t]
\centering
{\resizebox{\textwidth}{!}{
\begin{tabular}{lcccccccccc}
\toprule
\multicolumn{1}{c}{\multirow{2}{*}{\textbf{Model}}} & \multirow{2}{*}{\begin{tabular}[c]{@{}c@{}}\textbf{Parameter}\\ \textbf{Scale}\end{tabular}}  & \multicolumn{8}{c}{\textbf{Unseen Benchmarks}} & \multirow{2}{*}{\textbf{Average}} \\  \cmidrule{3-10} 
\multicolumn{1}{c}{} &  & \multicolumn{1}{l}{\textbf{MMLU}} & \multicolumn{1}{l}{\textbf{MMLU-Pro}} & \multicolumn{1}{l}{\textbf{ARC-c}} & \multicolumn{1}{l}{\textbf{GPQA}} & \multicolumn{1}{l}{\textbf{HellaSwag}} & \multicolumn{1}{l}{\textbf{IFEVAL}} & \textbf{MATHQA} & \textbf{BBH} & \\ \midrule
\multicolumn{11}{l}{\textit{Open-Source Models}} \\ \midrule 
Solar-10.7B-instruct & 10.7B  &  63.28 & 30.20 & 63.99 & 30.36 & 86.35 & 38.59 &  36.38 & 37.28 & 48.31 \\
Gemma-2-9b-it & 9B & 73.27 & 32.78 & 67.89 & 31.05 & 81.92 & 74.78 & 38.87 & 41.98 & 55.31 \\ 
LLaMA-3.1-8B-instruct & 8B & 67.95 & 40.87 & 54.95 & 34.60 & 79.18 & 50.71 & 39.53 & 70.85 & 54.83 \\ 
% Mixtral-8x22B-instruct & 8x22B & 75.62 &  52.63 & 67.83 & 36.83 & 87.68 & 60.43 & 50.08 & 83.03 & \\   
Qwen2.5-14B-instruct & 14B &  79.73  & 51.22 & 67.39 & 45.51 & 82.31 & 79.83 & 42.12 & 78.25 & 65.79 \\ 
Phi-4 & 14B & 84.56 & 70.12  & 68.26 & 55.93 & 84.42 & 62.98 & 48.87 & 79.87 & 69.37 \\  
\midrule
\multicolumn{11}{l}{\textit{Open-Source Models (Fine-tuning on original SFT Dataset)}} \\ \midrule
Solar-10.7B-instruct & 10.7B & 62.38 & 29.12 & 58.87 & 29.17 & 81.58 & 31.27 & 37.21 & 32.85 & 45.30 (-3.01) \\ 
Gemma-2-9b-it & 9B & 71.85 & 31.67  & 62.57 & 30.51 & 77.54 & 69.25 & 39.12 & 37.25 & 52.47 (-2.84) \\ 
LLaMA-3.1-8B-instruct & 8B & 65.34 &  36.85 &  
54.18 & 33.93 & 77.98 & 35.64 & 40.03 & 62.83 & 50.85 (-3.98)  \\ 
% Mixtral-8x22B-instruct & 8x22B&  &   & & & & & &  & \\ 
Qwen2.5-14B-instruct & 14B & 75.87  & 49.85  & 66.89 & 43.98 & 80.99 & 62.57 & 43.28 & 71.17 & 61.82 (-3.97) \\ 
Phi-4 & 14B & 80.27 & 66.58  & 66.27 & 52.89 & 83.39 & 55.83 & 49.98 & 75.49 & 66.33 (-6.04) \\ 
\midrule
\multicolumn{11}{l}{\textit{Open-Source Models (Fine-tuning on \textbf{\textsc{SysGen}} dataset)}} \\ \midrule 
LLaMA-3.1-8B-instruct & 8B & 66.89 & 39.77 & 54.55 & 34.21 & 78.89 & 46.75 & 42.11 & 68.98 & 54.02 (-0.81) \\ 
% Gemma-2-9B-instruct & 9B &  &   & & & & & &  & \\ 
% Mixtral-8x22B-instruct & 8x22B&  &   & & & & & &  & \\ 
% Solar-10.7B-instruct & 10.7B & 63.28 & 30.20 & 63.99 & 30.36 & 86.35 & 38.59 &  36.38 & 37.28 & \\ 
Qwen2.5-14B-instruct & 14B & 78.92 & 43.38 & 66.82 & 44.46 & 80.98 & 74.59 & 43.23 & 76.28 & 63.58 (-2.20) \\ 
Phi-4 & 14B & 83.27 & 68.77  & 67.89 & 55.18 & 84.31 & 57.87 & 50.23 & 77.12 & 68.08 (-1.29) \\ 
\midrule
\multicolumn{11}{l}{\textit{Open-source Models} $+$ \textit{Knowledge Distillation (Fine-tuning on \textbf{\textsc{SysGen}} dataset))}} \\
\midrule 
Solar-10.7B-instruct & 10.7B & 59.98  & 29.26  & 62.81 & 30.25 & 85.91 & 34.58 & 38.25 & 35.97 & 47.12 (-1.19) \\ 
Gemma-2-9b-it & 9B & 72.19 & 31.56 & 66.75 & 30.89 & 81.53 & 71.37 & 40.27 & 40.38 & 54.37 (-0.94) \\ 
\bottomrule
\end{tabular}}}
\caption{We utilize the Open LLM Leaderboard 2 score as the unseen benchmark. This reveals the key finding that adding system messages to existing SFT datasets does not lead to significant performance degradation.}
\label{tab:unseen_experiments}
\end{table*}
The primary goal of \textsc{SysGen} pipeline is to enhance the utilization of the \emph{system role} while minimizing performance degradation on unseen benchmarks, thereby improving the effectiveness of supervised fine-tuning (SFT).
To validate this, we evaluate how well the models trained on \textsc{SysGen} data generate appropriate assistant responses given both the system messages and user instructions, using the Multifacet~\citep{lee2024aligning} dataset.
For models that cannot generate data independently, we apply knowledge distillation to assess their effectiveness.
Additionally, we leverage the widely used Open LLM Leaderboard 2~\citep{myrzakhan2024open} as an unseen benchmark to determine whether our approach can be effectively integrated into existing SFT workflows.


\paragraph{\textsc{SysGen} provides better system message and assistant response to align with user instructions.}
Given the system messages and user instructions, the assistant's response is evaluated across four dimensions: style, background knowledge, harmlessness, and informativeness.
Each of these four aspects is scored on a scale of 1 to 5 using a rubric, and the average score is presented as the final score for the given instruction.
As shown in Table~\ref{tab:main_experiments}, recent open-source models achieve comparable scores to the proprietary models, indicating that open-source models have already undergone training related to system roles~\citep{meta2024introducing, yang2024qwen2, abdin2024phi}.

When trained on \textsc{SysGen} data, both LLaMA (4.12 → 4.21) and Phi (4.41 → 4.54) show score improvements.
Among the four dimensions, LLaMA exhibits score increases in style (4.15 → 4.32) and harmlessness (4.23 → 4.29).
Similarly, Phi shows the improvements in style (4.42 → 4.61) and informativeness (4.37 → 4.49).
As a result, even open-source models that have already been trained on system roles demonstrate their positive effects on style, informativeness, and harmlessness.



\paragraph{Knowledge distillation through \textsc{SysGen} data.}
If an open-source model does not support the system roles, it may not generate the system messages properly using \textsc{SysGen} pipeline. 
However, the effectiveness of knowledge distillation, using data generated by another open-source model without the limitation, remains uncertain.
To explore this, we train Gemma~\citep{team2024gemma} and Solar~\citep{kim-etal-2024-solar} using data generated by Phi-4~\citep{abdin2024phi}.
We use the Phi-4 data because it preserves most of the data and provides high quality  assistant responses as shown in Table~\ref{tab:statistics_generated_answer} and \ref{tab:data_statistics}.

As shown in Table~\ref{tab:knowledge_distillation}, even for models that do not inherently support system roles, modifying the chat template to incorporate system role and training on knowledge distilled dataset leads to an improvement in Multifacet performance, as observed in Gemma (4.05 → 4.23).
We describe the details in the Appendix~\ref{app:system_role_support}.
Additionally, for the Solar model, which had not been trained on system roles, we observe a dramatic performance improvement (3.19 → 3.76).\footnote{We speculate that Solar model did not properly learn the system role because its initial Multifacet score was low.}
This demonstrates that the data generated by the \textsc{SysGen} pipeline effectively supports the system roles.


\paragraph{\textsc{SysGen} data minimizes the performance degradation in unseen benchmarks.}
When incorporating system messages that were not present in the original SFT datasets and modifying the corresponding assistant responses, it is crucial to ensure that the model’s existing performance should not degrade.
For example, one key consideration in post-training is maintaining the model's original performance.
To assess this, we observed performance difference in unseen benchmark after applying supervised fine-tuning.
As shown in Table~\ref{tab:unseen_experiments}, we use the Open LLM Leaderboard 2 dataset as an unseen benchmark, with performance categorized into four groups:
\begin{itemize}
    \item Performance of existing open-source models (row 1-6)
    \item Performance of fine-tuning with open-source models using SFT datasets (row 7-12)
    \item Performance of fine-tuning with \textsc{SysGen} data (row 13-16)
    \item Performance after applying knowledge distillation using Phi-4 \textsc{SysGen} data (row 17-19)
\end{itemize}
The average performance degradation reflects the scores missing from each open-source model's original performance (row 1-6).

When fine-tuning with independently generated data using \textsc{SysGen}, the performance degradation is significantly lower than fine-tuning with the original SFT datasets selected under the same conditions.
Additionally, even for models that cannot generate data independently (e.g., those that do not support system roles), knowledge distillation helps mitigate performance drops considerably.









\section{Discussion}


\subsection{Implications}
Our investigation highlights the critical need to mitigate knowledge base poisoning. The findings from our study have several implications for enhancing the security of code generated by RACG systems.
Firstly, existing retrieval strategies naturally favor the most relevant examples from the knowledge base, which gives attackers the opportunity to successfully mislead the generation process with a small number of vulnerable examples. We argue that this risk can be mitigated by adjusting the retrieval strategy. For instance, an alternative strategy would be to select the second most similar example or randomly choose from a candidate pool containing several of the most similar examples, and we plan to explore such a way in future work.


Secondly, based on the results from RQ1, we found that hiding the programming intent (i.e., in Scenario II) increases the difficulty of successful poisoning. For instance, attackers can achieve a VRRC of 0.38 in Scenario I with only one poisoned vulnerable example. However, in Scenario II, attackers achieve a VRRC of 0.35 with a poisoning proportion of 0.8, meaning that they need to inject 9,642 vulnerable examples into the knowledge base (calculated as $\lfloor12,053 \times 0.8\rfloor$).
% which corresponds to a poisoning rate of 40\% (9,642/21,695). 
This indicates that concealing the programming intent makes poisoning more intricate and easier to detect.

Thirdly, according to the finding from \S\ref{subsec:vul_type}, the security of LLM-generated code vary across CWE types. This suggests that RACG systems could devise special strategies to check for the existence of several specific CWE types in the knowledge base, such as CWE-352, with the aim of improving the security of the generated code, as these vulnerabilities are more likely to induce the generation of vulnerable code.
% key types of vulnerabilities in the knowledge base, thereby improving the security of the generated code.

\subsection{Effectiveness of Judge}
\label{subsec:judge_effectivenss}
To assess the performance of using an LLM as a judge to label responses, we evaluate the effectiveness through both manual sampling inspection and automated inspection. For {\bf manual} inspection, we determine the sample size based on a 95\% confidence level and a 10 confidence interval, using a population size of 12,053 responses from GPT-4o. The final sample sizes for C, C++, Java, and Python are 95, 81, 93, and 91, respectively, as calculated using an off-the-shelf tool.\footnote{\url{https://www.surveysystem.com/sscalc.htm}} The evaluated code was sampled from GPT-4o in a moderated, one-shot setting using the JINA retriever in Scenario I with five poisoning vulnerabilities.
Two authors independently evaluated the samples through manual review, followed by a double-check to ensure consistency. For {\bf automated} inspection, we using a dataset containing both vulnerable code and its fixed version. Specifically, we evaluate pairs of items (i.e., the vulnerable version and its fixed counterpart) using our LLM-based judge to see if it can distinguish between them. This evaluation is performed on the full dataset.
We classify a code sample as positive when the judge correctly identifies it as vulnerable and the results are presented in Table~\ref{tab:dis_judge_combined}. 
\begin{table}[!t]
  \centering
  \caption{Performance of LLM-based judges under manual and automated inspection.}
  \resizebox{1\linewidth}{!}{
    \begin{tabular}{llrrrr}
    \toprule
    \multirow{2}{*}{\shortstack{\textbf{Inspection} \\ \textbf{Method}}} & \multirow{2}{*}{\textbf{Language}} & \multirow{2}{*}{\textbf{Accuracy}} & \multirow{2}{*}{\textbf{Precision}} & \multirow{2}{*}{\textbf{Recall}} & \multirow{2}{*}{\textbf{F1}} \\
    & & & & & \\
    \midrule
    \multirow{4}{*}{Manual} 
      & C     &   0.76  & 0.80 & 0.78  & 0.79 \\
      & C++   &  0.72   &  0.84  & 0.79    & 0.81 \\
      & Java  &  0.79 &  0.81  & 0.75 & 0.78 \\
      & Python &  0.84  &  0.85 & 0.78      & 0.81 \\
    \midrule
    \multirow{4}{*}{Automated} 
      & C     & 0.80 &  0.86 &  0.72 & 0.78 \\
      & C++   & 0.80 &  0.83 &  0.76 & 0.79 \\
      & Java  & 0.79 &  0.80 &  0.76 & 0.78 \\
      & Python & 0.82  & 0.83 & 0.80 & 0.81 \\
    \bottomrule
    \end{tabular}%
    }
  \label{tab:dis_judge_combined}%
\end{table}%
Overall, the LLM-based judge demonstrates commendable performance, consistently achieving high accuracy, precision, recall, and F1 scores across both inspection approaches. This indicates that the judge is capable of effectively detecting vulnerabilities in generated code, making it a reliable evaluation method. Among the four programming languages evaluated, the judge's performance remains generally consistent.

In the results under manual inspection, Python yields the highest performance, with accuracy and F1 scores reaching 0.84 and 0.81, respectively. The performance is slightly lower for C and Java, with F1 scores of 0.79 and 0.78, respectively, but still remains at a high level. In the results under automated inspection, the judge's F1 scores across all four programming languages hover around 0.8, aligning closely with the results from manual inspection. These evaluations demonstrate that the judge effectively detects vulnerabilities in generated code.


\subsection{Effectiveness of Retrievers}
% Table generated by Excel2LaTeX from sheet 'Sheet1'
\begin{table}[!t]
  \centering
  \caption{The effectiveness of retrievers}
  \resizebox{0.8\linewidth}{!}{
      \begin{tabular}{lrrrr}
    \toprule
    \textbf{Retriever} & \textbf{MRR} & \textbf{SR@1$\dagger$} & \textbf{SR@5} &\textbf{SR@10} \\
    \midrule
    JINA  & \bf{0.85} & \bf{0.79} & \bf{0.91} & \bf{0.93} \\
    BM25  & 0.20 & 0.14 & 0.19 & 0.24 \\
    \bottomrule
    \end{tabular}%
  }
  \label{tab:dis_retriever}%
  \caption*{\footnotesize $\dagger$ SR@k represents the SuccessRate@k\hspace{2.8cm}\textcolor{white}{.}}
\end{table}%
\vspace{-2mm}


Our influencing factors analysis of LLM-introduced vulnerabilities (\S\ref{subsec:cause_analysis}) reveals that different retrievers impact the security of generated code. Specifically, LLMs using the JINA retriever are more prone to generating vulnerable code than those using BM25 across various LLMs and scenarios. We attribute this to JINA's superior retrieval of relevant code. To validate this, we evaluate retriever effectiveness using MRR and SuccessRate@k (Table~\ref{tab:dis_retriever}), following prior work~\cite{liu2021opportunities,wang2024fusing}. MRR is the average reciprocal rank of results for a set of queries $q$, and SuccessRate@k is the percentage of queries where the relevant code snippet appears within the top-k results. As shown, JINA significantly outperforms BM25 across all metrics: MRR (0.85 vs. 0.20), SR@1 (0.79 vs. 0.14), SR@5 (0.91 vs. 0.19), and SR@10 (0.93 vs. 0.24). This confirms JINA's superior retrieval capability, which, while beneficial for general code generation, exposes LLMs to more potentially vulnerable code, thus increasing the likelihood of generating vulnerable code.

\subsection{The Difference with RAG Poisoning}
\label{subsec:diff_rag_poisoning}
RAG and RACG systems share the use of external knowledge to enhance content generation. However, they differ significantly in the nature of poisoning attacks and their consequences. Specifically, RAG poisoning targets the functional accuracy of the system~\cite{zou2024poisonedrag,zhang2024hijackrag}. In a RAG setup, attackers inject false or misleading examples into the knowledge base, causing the system to retrieve incorrect information. This disrupts the model’s ability to generate factually accurate outputs, undermining its usefulness in tasks requiring reliable information. The primary goal here is to compromise the system’s ability to produce correct content. 

In contrast, RACG poisoning aims to compromise the security of generated code without impacting functionality. Otherwise, the developer would discard the generated code and there would not be targeted vulnerability in the software.
By introducing vulnerable code examples into the knowledge base, attackers aim to influence the code generation process and lead to the creation of code with exploitable vulnerabilities, such as buffer overflows or SQL injection risks. This poisoning could propagate security vulnerabilities, creating potential real-world risks. RACG poisoning aims to infect the generated code with vulnerabilities that could be exploited.

This paper is the first comprehensive study examining how vulnerable code examples in the knowledge base impact the security of code generated by RACG systems. We focus on how these poisoned examples can lead to the generation of insecure code, introducing potential vulnerabilities. Our work highlights the need for securing RACG knowledge sources to prevent the propagation of security risks in generated code.

\subsection{Threats to Validity}

{\bf Query Generation through LLM.} In Section~\ref{subsec:dataset_cons}, we use DeepSeek-V2.5 to generate queries for code. However, there is a possibility that DeepSeek-V2.5 may produce inaccurate content. To mitigate this threat, we manually review the generated queries. Specifically, we randomly select 100 queries for each programming language and have them reviewed by the two authors. Any inconsistencies in the evaluation results were resolved through discussion between the authors. The manual review indicates that 86\% of the generated queries accurately reflect the functionality of the code on average. Therefore, the impact of this threat is minimal.

\noindent

{\bf Programming Languages Investigated.} In this study, we conduct experiments using four widely-used programming languages: C, C++, Java, and Python. One potential threat is that the selected languages may not fully represent real-world development scenarios. However, according to GitHub usage statistics (measured by the number of pull requests) for the first quarter of 2024~\cite{githut2024}, these four languages account for 42.7\% of the total activity. Among them, Python and Java are the most popular. Additionally, as other languages like Go gain popularity, we plan to extend our study to include more programming languages in future work.
\section{Conclusion}
%In conclusion, this paper presents a significant leap forward in addressing the challenge of FCHs in LLMs. By developing a systematic framework based on logic programming and integrating it into \tool, we have effectively tackled the limitations of current detection methodologies. Our approach, grounded in transforming a comprehensive factual knowledge base sourced from Wikipedia through advanced logic reasoning methods, has demonstrated superior performance in detecting factual inaccuracies across various LLMs. The automation of this process marks a notable advancement in scalability, reducing reliance on manual intervention. Furthermore, the release of our enriched dataset as a benchmark contributes to the broader research community, paving the way for future innovations in hallucination detection. This work not only enhances the reliability and usability of LLMs but also sets a new standard for research in this critical area of language processing technology. 
We target the critical challenge of FCH in LLM, where they generate outputs contradicting established facts. We developed a novel automated testing framework that combines logic programming and metamorphic testing to systematically detect FCH issues in LLMs. Our novel approach constructs a comprehensive factual knowledge base by crawling sources like Wikipedia, then applies innovative logic reasoning rules to transform this knowledge into a large set of test cases with ground truth answers. 
These reasoning rules are either predefined relations or automatically generated from randomly sampled temporal formulae. 
LLMs are evaluated on these test cases through template prompts, with two semantic-aware oracles analyzing the similarity between the logical/semantic structures of the LLM outputs and ground truth to validate reasoning and pinpoint FCHs. 

Across diverse subjects and LLM architectures, our framework automatically generated over 9,000 useful test cases, uncovering hallucination rates as high as 59.8\% and identifying lack of logical reasoning as a key contributor to FCH issues. This work pioneers automated FCH testing capabilities, providing a comprehensive benchmark, data augmentation techniques, and answer validation methods. The implications are far-reaching --- enhancing LLM reliability and trustworthiness for high-stakes applications by exposing critical weaknesses while advancing systematic evaluation methodologies.

%--------------------------------------------------

\section*{Limitations}
While this work moves towards the development of participatory approaches to Natural Language Processing, the annotator pool is not balanced, especially leaning towards the White and Educated population. Moreover, although the chosen events had worldwide coverage, the majority of the annotators do not come from the same social background as the target celebrities. This made it possible to carry out the experiment with people from different nationalities, who, however, experienced cancel culture more as spectators than actors. In the future, we plan to expand the annotation lab to a more diverse group of annotators along all the sociodemographic axes. 
Finally, the current design involved watching only one video per celebrity, with exposure to the framing implicit in the political orientation of the chosen news source. While we aimed to ensure political diversity across the six selected videos, in the future, we intend to achieve this within the same canceling event.

\section*{Ethical Statement}
This research relies on the voluntary work of those who participated in the Annotation Labs. All the involved annotators freely accepted to take part to the laboratory, for which no compensation was provided. 
We adopted all the measures to protect data privacy and safeguard personal information. The work has been approved by the Ethics Committee of the institution of one of the authors. %is affiliated with. 

In the future, we plan to expand this work to a less Eurocentric context, concerning both the chosen celebrities and the involved annotators, looking at it as a necessary improvement to foster diversity. 




%Appendix todo: 
% annotators' demographics
% MFT
% check-out questionnaire 



\bibliography{latex/custom}

\clearpage

\appendix
\section{Annotator's demographics}\label{app-demographics}

\Cref{tab:ann_demographics} shows the demographic information about the annotators. We additionally asked them to indicate their nationality and first language. 38 are Italian, 2 are Spanish, 2 are Chinese, and there is one person for each of the following nationalities: Italian-Argentinian, Italian-Romanian, Iranian, Greek, Russian, Kazakhstan, Indian, Moldovan, Persian, Dutch, Romanian. As regards their mother tongue: 38 chose Italian, 3 Spanish, Russian, Chinese, 2 Persian, Romanian and Greek, and 1 Hebrew, Hindi, Farsi and Dutch.

\begin{table}[]
    \centering\small
    \begin{tabular}{lc|c}
    \hline
          \multicolumn{2}{c}{Demographics}&  \#annotators\\
          \hline
          \multirow{4}{*}{Gender identity}&Female& 
    34\\
 & Male&20\\
 & Non-binary&2\\
 & Prefer not to say&1\\
 \hline
 \multirow{4}{*}{Generation}& Boomer&1\\
 & GenX&2\\
 & GenY&18\\
 & GenZ&36\\
 \hline
 \multirow{5}{*}{Ethnicity}& White&48\\
 & Asian&5\\
 & Mixed&2\\
 & Black&1\\
 & Caucasian&1\\
 \hline
 \multirow{4}{*}{Education level}& Bachelor degree&29\\
 & Master degree&18\\
 & Doctorate degree&5\\
 & High school diploma&5\\
  \hline
 \multirow{8}{*}{Employment}& Unemployed&16\\
 & Full-time&15\\
 & Part-time&10\\
 & Due to start a new job&3\\
 & Student&8\\
 & Not in paid work&2\\
 & Occasional work&1\\
 & Freelancer&2\\
 \hline
 \end{tabular}
    \caption{Sociodemographic information.}
    \label{tab:ann_demographics}
\end{table}


% \section{Related Work}


\textbf{Class-Incremental Learning (CIL)} necessitates a model that can continuously learn new classes while retaining knowledge of previously learned ones \cite{dohare2024loss,cao2023retentive,zhou2023revisiting,zhou2024continual}, which can be roughly divided into several categories. 
Regularization-based methods incorporate explicit regularization terms into the loss function to balance the weights assigned to new and old tasks \cite{kirkpatrick2017overcoming,aljundi2018memory,wang2022continual,li2017learning}.
Replay-based methods address the problem of catastrophic forgetting by replaying data from previous classes during the training of new ones. This can be achieved by either directly using complete data from old classes \cite{lopez2017gradient,riemer2018learning,cha2021co2l,wang2022foster} or by generating samples \cite{shin2017continual,zhu2022self}, such as employing GANs to synthesize samples from previous classes \cite{cong2020gan,liu2020generative}.
Dynamic network methods adapt to new classes by adjusting the network structure, such as adding neurons or layers, to maintain sensitivity to previously learned knowledge while acquiring new tasks. This approach allows the model's capacity to expand based on task requirements, improving its ability to manage knowledge accumulation in CIL \cite{wang2022coscl,wang2023incorporating,aljundi2017expert,ostapenko2021continual}.
Recently, CIL methods based on pre-trained models (PTMs)
\cite{cao2023retentive,chen2022adaptformer,zhou2024continual} have demonstrated promising results.
Prompt-based methods utilize prompt tuning \cite{jia2022visual} to facilitate lightweight updates to PTMs. By keeping the pre-trained weights frozen, these methods preserve the generalizability of PTMs, thereby mitigating the forgetting in CIL \cite{smith2023coda,wang2022dualprompt,NEURIPS2023_d9f8b5ab,wang2022learning,wang2022dualprompt,smith2023coda,li2024learning}. 
Model mixture-based methods mitigate forgetting by saving models during training and integrating them through model ensemble or model merge techniques \cite{gao2023unified,wang2023isolation,wang2024hierarchical,zheng2023preventing,zhou2023learning,zhou2024expandable}.
Prototype-based methods draw from the concept of representation learning \cite{ye2017learning}, leveraging the robust representation capabilities of PTMs for classification with NCM classifiers \cite{panos2023first,zhou2023revisiting,mcdonnell2024ranpac}.

\textbf{The Order in CIL} remains a significant and unresolved challenge \cite{wang2024comprehensive}. APD \cite{yoon2019scalable} effectively addresses the problem of CF by decomposing model parameters into task-shared and sparse task-specific components, thereby enhancing the model's robustness to changes in class order. HALRP \cite{li2024hessian}, on the other hand, simulates parameter conversion in continuous tasks by applying low-rank approximation to task-adaptive parameters at each layer of the neural network, thereby improving the model's order robustness. However, the optimization strategies employed by these methods are confined to the network architecture itself and do not fundamentally resolve the underlying issues.
Recent theoretical analyses of CIL \cite{lin2023theory,shan2024order,bell2022effect,wu2021curriculum} indicate that prioritizing the learning of tasks (or classes) with lower similarity enhances the model's generalization and resistance to forgetting. Building on these theories, we conducted further research and developed corresponding methods in the following sections.

\section{Instructions for the annotation process}\label{app-event_description}
\Cref{fig:annotation_platform} shows the instructions provided to the annotators. 

\begin{figure*}
    \centering
    \includegraphics[width=1\linewidth]{latex/images/annotation_guidelines.png}
    \caption{Instructions for the annotators.}
    \label{fig:annotation_platform}
\end{figure*}

In the following, we report all the celebrity descriptions.

\paragraph{J K Rowling} J K Rowling is a British author, and writer of the fantasy novel “Harry Potter”. In recent years she often expressed derogatory remarks about the transgender community. This has caused “Harry Potter” film actors such as Daniel Radcliffe, Emma Watson, Rupert Grint and Eddie Redmayne to speak out against the author, also calling for boycotts of her projects.
\paragraph{Kanye West} Kanye West, also known as Ye, is an American rapper. He has frequently spoken out on political and social issues with controversial opinions on topics such as abortion, capital punishment, welfare and gun rights. On frequent occasions he expressed antisemitic thoughts, stating his admiration for Adolf Hitler, denying the Holocaust, and supporting other conspiracy views against Jewish people, which led him to being banned from Twitter for 8 months.
\paragraph{Lizzo} Lizzo is an American rapper and singer. Throughout her career, she has been publicly interested and outspoken on social issues. She supported the LGBTQ+ community considering herself an ally, and advocated for body positivity, being subject to body shaming herself. In August 2023, she was accused of sexual, religious and racial harassment, disability discrimination, assault, weight-shaming and a hostile work environment by three former backup dancers, supported by other co-workers.
\paragraph{Halle Bailey} Halle Bailey is an American singer and actress. In 2023 she performed as the protagonist in the Disney movie “The Little Mermaid”, a choice that was subject to widespread criticism because in the cartoon the little mermaid was depicted as white, while Bailey is black. At the time, the hashtag \#NotMyAriel was launched, leading to a discussion about Disney film revision in the name of woke culture.
\paragraph{Ellen DeGeneres} Ellen DeGeneres is an American comedian and television host, famous for “The Ellen DeGeneres Show”. In July 2020 ten former employees of this show accused her of creating a toxic environment, with racist micro-aggressions, intimidation, abuse and sexual harassment episodes against female employees. She publicly apologized, promising that she would correct the issue.
\paragraph{Andrew Tate} Andrew Tate is an American and British former professional kickboxer, who became famous for promoting misogynist and violent messages, representative of the manosphere community. He was deplatformed from Twitter, Instagram, Facebook and TikTok. His account on Twitter was reinstalled in November 2022 after the Elon Musk acquisition. In December 2022 he was arrested with charges of rape, human trafficking and forming a criminal gang for the sexual exploitation of women. He was released a few months later.
\section{Annotation materials}\label{app-annotation_materials}
\Cref{tab:video_info} presents all the information about the selected YouTube videos.

\begin{table*}[ht]
    \centering\small
    \begin{tabular}{ccccc}
\hline
Celebrity&  Topic &News Broadcast & Political orientation &Minutes\\
\hline
 J K Rowling& Homo-transphobia& Sky News Australia & Right wing & 2.29\\
 Kanye West& Antisemitism& Fox 11 Los Angeles& Right wing & 1.14\\
 Lizzo& Harassment& Fox News& Right wing &3.24\\ 
 Halle Bailey& Incelism& CBS Media& Left wing &1.47\\
 Ellen DeGeneres& Bullying& CBS Media& Left wing &3.01\\
 Andrew Tate& Sexual assault & Law and Crime network& Left wing &2.26 \\
 \hline
    \end{tabular}
    \caption{Information about the selected YouTube videos their duration for each target celebrity.}
    \label{tab:video_info}
\end{table*}

\Cref{tab:sample-iaa} reports the detailed sample composition and Inter Annotator Agreement.

\begin{table*}[]
    \centering\small
    \begin{tabular}{ccllllll}
    \hline
 Samples &\#Researchers&\#Students& \#Activists& \#Texts&\#Annotations& $\alpha$ Stance& $\alpha$ Acceptability\\
 \hline
         Sample\_0&    1&4& 2&  210&1,470& 0.518&0.177\\
 Sample\_1&  1&2& 2&  210&1,050& 0.472&0.306\\
 Sample\_2&  1&3& 2&  208&1,248& 0.39&0.189\\
 Sample\_3&  1&3& 3&  210&1,470& 0.548&0.153\\
 Sample\_4&  1&4& 1&  209&1,254& 0.491&0.183\\
 Sample\_5&  1&4& 2&  210&1,470& 0.601&0.283\\
 Sample\_6&  1&2& 1&  210&840& 0.505&0.202\\
 Sample\_7&  1&1& 1&  210&630& 0.425&0.215\\
 Sample\_8&  1&3& 1&  208&1,040& 0.463&0.271\\
 Sample\_9&   1&4& 2&  209&1,463& 0.603&0.245\\
 \hline
 Total & 10& 30& 17&  2,094&11,935&  & \\
 $\alpha$ $\text{mean}_{\text({std})}$ & & & &  &&$0.501_{(0.069)}$&$0.222_{(0.051)}$\\
 \hline
    \end{tabular}
    \caption{Sample composition, annotation and IAA. }
    \label{tab:sample-iaa}
\end{table*}





\section{Checkout Questionnaire}\label{app-checkout_qst}
We report the questions asked to the annotators in the checkout questionnaire:

\begin{itemize}[noitemsep]
    \item What opinion do you have about the topic of \textit{social media shaming} after the annotation? 
    \item How did you find the annotation task? (difficult? emotionally impactful? etc.)
    \item Is there anything you would like to add?
    \item Is there anything you would like to change?
    \item Did the moral questionnaire influence the way you performed the annotation? 
    \item Did you perceive the completion of the questionnaire and annotation as thematically related?
    \item How would you use a tool that can recognize social media shaming? 
\end{itemize}

This questionnaire was not mandatory and received $33$ answers.
\section{LLM implementation details}

In this work, we have used the following public models:
\begin{itemize}
    \item \texttt{OLMo 7B}\footnote{\url{https://huggingface.co/allenai/OLMo-2-1124-7B-Instruct}}
    \item \texttt{BLOOMZ 3B}\footnote{\url{https://huggingface.co/bigscience/bloomz-3b}}
    \item \texttt{DeepSeek R1 1.5B}\footnote{\url{deepseek-ai/DeepSeek-R1-Distill-Qwen-1.5B}}
    \item \texttt{OPT-IML 1.3B}\footnote{\url{https://huggingface.co/facebook/opt-iml-max-1.3b}}
    \item \texttt{Llama 3.2 3B}\footnote{\url{https://huggingface.co/meta-llama/Llama-3.2-3B-Instruct}}
    \item \texttt{Ministral 8B}\footnote{\url{https://huggingface.co/mistralai/Ministral-8B-Instruct-2410}}
\end{itemize}

To generate the annotations with these models, we have followed a zero-shot approach, prompting the models to generate their annotations.
All models were executed in a NVIDIA Titan X Pascal GPU, with 12GB of memory.

To generate the annotations regarding stance, the used prompt is as follows:
\textit{Classify the text into being defensive, neutral or attacking.}
Similarly, we used the following prompt to generate the acceptability annotations:
\textit{Classify the text into a scale from 1 to 4, considering how much the text contributes to shaming or degrading a subject, being 1 the lower and 4 the higher.}




\end{document}
