% This must be in the first 5 lines to tell arXiv to use pdfLaTeX, which is strongly recommended.
\pdfoutput=1
% In particular, the hyperref package requires pdfLaTeX in order to break URLs across lines.

\documentclass[11pt]{article}

% Change "review" to "final" to generate the final (sometimes called camera-ready) version.
% Change to "preprint" to generate a non-anonymous version with page numbers.
\usepackage[preprint]{acl}

% Standard package includes
\usepackage{times}
\usepackage{latexsym}

% For proper rendering and hyphenation of words containing Latin characters (including in bib files)
\usepackage[T1]{fontenc}
% For Vietnamese characters
% \usepackage[T5]{fontenc}
% See https://www.latex-project.org/help/documentation/encguide.pdf for other character sets

% This assumes your files are encoded as UTF8
\usepackage[utf8]{inputenc}

% This is not strictly necessary, and may be commented out,
% but it will improve the layout of the manuscript,
% and will typically save some space.
\usepackage{microtype}

% This is also not strictly necessary, and may be commented out.
% However, it will improve the aesthetics of text in
% the typewriter font.
\usepackage{inconsolata}

%Including images in your LaTeX document requires adding
%additional package(s)
\usepackage{graphicx}
\usepackage{xcolor}
\usepackage{soul}
\usepackage{multirow}
\usepackage{subfigure} 
\usepackage{booktabs}
\usepackage{enumitem}
\usepackage{amsmath}

\newcommand{\fixme}[1]{{\color{red} *** {\textbf{Rajesh: }}\color{blue}{#1}}{\color{red} ***}}
% If the title and author information does not fit in the area allocated, uncomment the following
%
%\setlength\titlebox{<dim>}
%
% and set <dim> to something 5cm or larger.

\title{That is Unacceptable: the Moral Foundations of Canceling}

% Author information can be set in various styles:
% For several authors from the same institution:
% \author{Author 1 \and ... \and Author n \\
%         Address line \\ ... \\ Address line}
% if the names do not fit well on one line use
%         Author 1 \\ {\bf Author 2} \\ ... \\ {\bf Author n} \\
% For authors from different institutions:
% \author{Author 1 \\ Address line \\  ... \\ Address line
%         \And  ... \And
%         Author n \\ Address line \\ ... \\ Address line}
% To start a separate ``row'' of authors use \AND, as in
% \author{Author 1 \\ Address line \\  ... \\ Address line
%         \AND
%         Author 2 \\ Address line \\ ... \\ Address line \And
%         Author 3 \\ Address line \\ ... \\ Address line}

\iffalse
\author{First Author \\
  Affiliation / Address line 1 \\
  Affiliation / Address line 2 \\
  Affiliation / Address line 3 \\
  \texttt{email@domain} \\\And
  Second Author \\
  Affiliation / Address line 1 \\
  Affiliation / Address line 2 \\
  Affiliation / Address line 3 \\
  \texttt{email@domain} \\}
\fi

\author{
 \textbf{Soda Marem Lo\textsuperscript{1,2}},
 \textbf{Oscar Araque\textsuperscript{3}},
 \textbf{Rajesh Sharma\textsuperscript{4}},
  \textbf{Marco Antonio Stranisci\textsuperscript{1,2}},
\\
\\
\\
 \textsuperscript{1}Università degli Studi di Torino,
 \textsuperscript{2}aequa-tech,
  \textsuperscript{3}Universitad Complutense de Madrid,
  \textsuperscript{4}Tartuk {Ü}likool
  \\
  \small{
   \textbf{Correspondence:} \href{mailto:marcoantonio.stranisci@unito.it}{marcoantonio.stranisci@unito.it}
  }
}


%---added packages--------
\usepackage{cleveref}
\usepackage{url}
%---------------------


\begin{document}
\maketitle
\begin{abstract}
Canceling is a morally-driven phenomenon that hinders the development of safe social media platforms and contributes to ideological polarization. To address this issue we present the \textbf{C}anceling \textbf{A}ttitudes \textbf{De}ection (CADE) dataset, an annotated corpus of canceling incidents aimed at exploring the factors of disagreements in evaluating people canceling attitudes on social media. Specifically, we study the impact of annotators' morality in their perception of canceling, showing that morality is an independent axis for the explanation of disagreement on this phenomenon. Annotator's judgments heavily depend on the type of controversial events and involved celebrities. This shows the need to develop more event-centric datasets to better understand how harms are perpetrated in social media and to develop more aware technologies for their detection. 

\textbf{WARNING:} the CADE corpus could contain racist, sexist, violent, and generally offensive content.
\end{abstract}


%-------------sections------------------------------
\section{Introduction}

\begin{figure}[h]
    \centering
    \begin{overpic}[trim=0cm 0cm 0cm 0cm,clip,angle=0,origin=c,width=.4\linewidth]{images/teaser_absolute.png}
        %  trim={<left> <lower> <right> <upper>}
        %  \put(horiz, vert)
        %  \put(horiz, vert){\rotatebox{90}{Text}}
        %
        \put(107, 32){$\mathbf{\to}$}
    \end{overpic}\hspace{1cm}
    \begin{overpic}[trim=0cm 0cm 0cm 0cm,clip,angle=0,origin=c,width=.4\linewidth]{images/teaser_translated_yellow.png}
        %  trim={<left> <lower> <right> <upper>}
        %  \put(horiz, vert)
        %  \put(horiz, vert){\rotatebox{90}{Text}}
        %
    \end{overpic}
    \caption{Using translation methods, a controller trained on an environment with a given visual variation \textit{(left)} can be reused without any training or fine-tuning on a different environment (\textit{right}) with comparable performance. In red we see the trajectory of a car driven by the same controller when connected to two different encoders, one for each visual variation.
    }
    \label{fig:teaser}
\end{figure}

Deep Reinforcement Learning (RL) has enabled agents to achieve remarkable performance in complex decision-making tasks, from robotic manipulation to high-dimensional games (Mnih et al., 2015; Silver et al., 2017). 
Although recent RL techniques achieved strong improvements over sample efficiency \citep{yarats2021drqv2, kostrikov2020image}, training new agents remains a costly process, both in computational and temporal terms.
Despite these advances, most methods still require at least partial retraining when dealing with domain shifts such as visual appearance, reward functions, or action spaces \citep{pmlr-v97-cobbe19a, zhang2020learning}. These domain changes typically require expensive retraining, which can be prohibitive for real-world settings that require millions of interactions.

A variety of approaches have been proposed to address these shifting conditions. Domain randomization \citep{tobin2017domain, sadeghi2016cad2rl} trains agents across diverse visual styles or physics settings, promoting invariant features but demanding broader coverage of possible variations. Multi-task RL \citep{parisotto2015actor, teh2017distral} attempts to learn shared representations across multiple tasks.

In the supervised setting, recent representation learning techniques \citep{Moschella2022-yf,maiorca2023latent, norelli2022b, cannistraci2023bricks}, show that it is possible to zero-shot recombine encoders and decoders to perform new tasks across different modalities (images, text..) and tasks (classification, reconstruction) and even architectures.
In RL, methods adopting the relative representation framework \citep{Moschella2022-yf} have shown promising results in adapting encoders to different controllers with zero or few-shots adaptation, for robotic control from proprioceptive states \citep{jian2021adversarial} or for playing games in the Gymnasium suite \citep{towers2024gymnasium} from pixels \citep{ricciardi2025r3lrelativerepresentationsreinforcement}.
These methods, however, still require training models to use the new relative representations.

By contrast, \cite{maiorca2023latent} suggest that modules from independently trained neural networks can be connected via a simple linear or affine transformation, with no training constraint or fine-tuning required, if such transformations can be reliably estimated from a small set of “anchor” samples, pairs of states or observations deemed semantically equivalent.

Our main contribution is the implementation of a RL method based on semantic alignment to map between latent spaces of different neural models, so that their encoders and controllers can be stitched with the goal of creating new agents that can act on visual-task combinations never seen together in training. This includes the use of the transformations to map modules from different networks, and the collection of anchor samples used to estimate these transformations. We call our method Semantic Alignment for Policy Stitching (\textbf{SAPS}).
We perform analyses and empirical tests on the CarRacing and LunarLander environments to show the performance of new agents created via zero-shot stitching of encoders and controllers trained on different visual-task variations, demonstrating significant gains compared to existing zero-shot methods.
\section{Related Work} \label{related}




% \subsection{Benchmarks in Coding Scenarios}
% \begin{enumerate}
%     \item Code Generation
%     \item Bug Fixing
% \end{enumerate}

% \subsection{Large Language Model Agents}

% At the heart of the LLM Agent is an Agent Core, which coordinates the core \textit{logic} and \textit{behavioral} characteristics of the agent. In addition, the Agent includes the following key components:

% \begin{itemize}
%     \item Memory Module: It consists of both short-term and long-term memory components that record the agent's internal logs and interactions with the user.
%     \item Tools: These are the tools that the agent can use to perform tasks, usually specific third-party APIs.
%     \item Planning Module: This is used for solving complex problems, such as decomposing tasks and problems, reflexivity or critique.
% \end{itemize}

% \subsection{Multi Agent Collaboration Framework}

% MetaGPT \url{https://arxiv.org/abs/2308.00352}


\parabf{Coding \llm{s}.}
Large Language Models (\llm{s}) have become the go-to solution for a wide array of coding tasks due to their exceptional performance in both code generation and comprehension~\cite{codex}. These models have been successfully applied to various software engineering activities, including program synthesis~\cite{patton2024programming, codex, li2022competition, iyer2018mapping}, code translation~\cite{pan2024lost, roziere2020unsupervised, roziere2021leveraging}, program repair~\cite{xia2023repairstudy, chatrepair, monperrus2018living, bouzenia2024repairagent}, and test generation~\cite{titanfuzz, fuzz4all, deng2023fuzzgpt, lemieux2023codamosa, kang2023testing}. Beyond general-purpose \llm{s}, specialized models have been developed by further training on extensive datasets of open-source code snippets. Notable examples of these code-specific \llm{s} include \codex~\cite{codex}, \codellama~\cite{codellama}, StarCoder~\cite{starcoder,starcodertwo}, and \deepseek~\cite{deepseek}. Additionally, instruction-following code models have emerged, refined through instruction-tuning techniques. These include models such as \codellamainstruct~\cite{codellama}, \deepseekinstruct~\cite{deepseek}, \wizardcoder~\cite{wizardcoder}, \magicoder~\cite{magicoder}, and OpenCodeInterpreter~\cite{zheng2024opencodeinterpreter}.

\parabf{Benchmarking \llm-based coding tasks.}
To assess the capabilities of \llm{s} in coding, a variety of benchmarks have been proposed. Among the most widely utilized are \humaneval~\cite{codex} and \mbpp~\cite{austin2021program}, which are handcrafted benchmarks for code generation that include test cases to validate the correctness of \llm outputs. Other benchmarks have been developed to offer more rigorous tests~\cite{evalplus}, cover additional programming languages~\cite{zheng2023codegeex,cassano2023multipl}, and address different programming domains~\cite{livecodebench, hendrycksapps2021, codecontest, ds1000, arcade}.

More recently, research has shifted towards evaluating \llm{s} on real-world software engineering challenges by operating on entire code repositories rather than isolated coding problems~\cite{swebench, zhang2023repocoder, liu2023repobench}. A notable benchmark in this area is \swebench~\cite{swebench}, which includes tasks requiring repository modifications to resolve actual GitHub issues. The authors of \swebench have also released a more focused subset, \swebenchlite~\cite{swebenchlite}, which contains 300 problems centered on bug fixing that only involves single-file modifications in the ground truth patches. ML-Bench \cite{liu2023mlbench} is a benchmark for evaluating large language models and agents for Machine Learning tasks on reporitory-level code. It involves 18 repositories and focuses on code generation and interactions with Jupyter Notebooks.

\parabf{Repository-level coding.}
The rise of agent-based frameworks~\cite{xi2023rise} has spurred the development of agent-based approaches to software engineering tasks. Devin~\cite{devinwebpage} (and its open-source counterpart OpenDevin~\cite{opendevin}) is among the first comprehensive \llm agent-based frameworks. Devin employs agents to first perform task planning based on user requirements, then allows them to use tools like file editors, terminals, and web search engines to iteratively execute the tasks. \sweagent~\cite{sweagent} introduces a custom agent-computer interface (ACI), enabling the \llm agent to interact with the repository environment through actions like reading and editing files or running bash commands. Another agent-based approach, \autocoderover~\cite{autocoderover}, equips the \llm agent with specific APIs (e.g., searching for methods within certain classes) to effectively identify the necessary modifications for issue resolution. Beside these examples, a variety of other agent-based approaches have been developed in both open-source~\cite{aidar} and commercial products~\cite{bouzenia2024repairagent, coder, repounderstander, lingma, factorydroid, ibmagent, opencsgstarship, marscode, amazonqdeveloper}.

% Unlike these agent-based methods, \tech offers a straightforward and cost-efficient solution for addressing real-world software engineering challenges. Our work is the first to demonstrate that an \emph{agentless} approach can achieve comparable performance without the need for complex tools or modeling intricate environment behavior and feedback.

Unlike existing benchmarks and agent-based frameworks, which focus on the code generation/completion tasks, our proposed \model and \agent focus on the code deployment task, which is under-studied in the field.
\section{The CADE Dataset}\label{sec:corpus}
In this section we present the process that led to the creation of the Canceling Attitudes DEtection (CADE) dataset: a corpus of YouTube comments annotated for the study of canceling attitudes against controversial events. The section is organized as follows: we first present the data collection (\Cref{collection}) phase, then we describe the design of the annotation task (\Cref{annotation-task}), and present the Annotation Lab: the participatory approach that we adopted to recruit annotators (\Cref{annotation-lab}).


\subsection{Data collection}\label{collection}
As described in \Cref{sec:related}, cancel culture is characterized by the online public shaming of celebrities for their actions or behaviors. In a NLP perspective, this can be conceived as an event-centric task \cite{chen2021event}, where the characteristics of the event orient the interest of the research. Relying on this assumption we chose YouTube %\footnote{\url{https://www.youtube.com/}} 
 as source for our data collection. We leveraged the public APIs
 %\footnote{\url{https://developers.google.com/youtube/v3}}  
  to obtain communicative situations where the presence of canceling attitudes can be assessed. 

Firstly, we selected controversial events related to six celebrities that refer to various topics: J.K. Rowling (homo-transphobia), Kanye West (antisemitism), Lizzo (harassment), Halle Bailey (anti-woke culture), Ellen DeGeneres (bullying), Andrew Tate (sexual assault).
For each of them, we manually selected a video of a news broadcast reporting the event the target celebrity got canceled for. Since the annotators had to watch it before completing the annotation task, we opted for the most viewed video among those that did not exceed 4 minutes. For each video we extracted all the comments and randomly sampled 350 comments for each celebrity, resulting in a total of $2,100$ texts for the annotation. 


\subsection{Design of the annotation task}\label{annotation-task} 
The design of the annotation task takes into consideration two research needs: the identification of annotators' moral profiles and the definition of an annotation scheme that is effective in representing the complexity of social interactions canceling attitudes rely on.

\paragraph{Moral Foundation Questionnaire.} 
In social psychology, a common method for eliciting people's moral profile is the administration of questionnaires \cite{graham2013moral,hinz2005investigating}, which have been recently used in NLP to assess the moral stance of LLMs \cite{abdulhai-etal-2024-moral} and people \cite{davani2024disentangling}. Coherently with this research, we chose the 30-item Moral Foundation Questionnaire (MFQ30) \cite{graham2013moral}\footnote{\url{https://moralfoundations.org/questionnaires/}}, consisting of two blocks. In the first block, respondents have to reply to the question \textit{When you decide whether something is right or wrong, to what extent are the following considerations relevant to your thinking?}, rating 15 sentences on a scale from 0 (This consideration has nothing to do with my judgments of right and wrong) to 5 (This is one of the most important factors when I judge right and wrong). The second assignment is to read the 15 sentences and indicate their agreement or disagreement using a scale from 0 (strongly agree) to 5 (strongly disagree). %--- complete questionnaire in Appendix \ref{app-mfq}. 
By aggregating respondents' replies, it is possible to elicit which moral foundations contribute the most to their moral profile.

\paragraph{Annotation scheme.}
The annotation scheme is composed of two axes: stance \cite{aldayel2021stance} and acceptability \cite{forbes2020social}. The former is adopted to annotate the stance of the comment towards the celebrity; acceptability is adopted to evaluate whether the comment is perceived as morally unacceptable by the annotator. 

For each of the six target celebrities, we prepared a summary of the controversial event that introduced the topic to the annotators (Appendix \ref{app-event_description}) and then asked them to watch a YouTube video 
% that presents 
on the controversial event (Appendix \ref{app-annotation_materials}). 
Following this step, they could annotate
% A second step is the annotation of 
the stance and the unacceptability of comments about celebrities. First, annotators must evaluate whether the YouTube user intended to attack, defend or was neutral towards the controversial event involving the target celebrity (stance).
Annotators must then evaluate the social unacceptability of the comment, choosing on a scale that ranges from 1 (totally acceptable) to 4 (totally unacceptable). %This 

\subsection{Annotation Lab}\label{annotation-lab}
Since our research aims to identify how people with specific moral profiles perceive canceling attitudes in a realistic scenario, our annotator recruitment strategies focused on the involvement of people who have specific interests in the issue. We engaged with three stakeholders: activists against online discrimination, AI researchers, and NLP students. For each stakeholder we organized an annotation lab, which is designed following the literature on Participatory AI \cite{Delgado2023}: people are not only recruited as annotators, but are involved during the whole dataset creation process. For this reason, we involved stakeholders rather than relying on crowdsourcing annotator platforms. 

The annotation lab is structured in three main steps: \textit{i)} engagement: we presented a set of slides through a shared video in which we provided a definition of the phenomenon of cancel culture, we shared our research objective and its social impact and explain the whole annotation process; \textit{ii)} annotation: people were asked to fill out the MFT questionnaire and to annotate YouTube comments according to our annotation scheme; \textit{iii)} feedback: we asked participants to provide feedback about the annotation process and share their ideas about potential applications of a technology for the identification of canceling attitudes. To this aim, we adopted two methodologies: the administration of a checkout questionnaire, and the organization of two focus groups. 

\paragraph{Feedback questionnaire.} We administered the questionnaire to all the involved stakeholders. It included 7 questions (Appendix \ref{app-checkout_qst}) about three aspects of the annotation lab: evaluating the experience in terms of its emotional impact and difficulty; providing improvements to the annotation scheme; and suggesting downstream applications of a technology trained on such a resource.

\paragraph{Focus group.} We organized one focus group aimed at students and one aimed at activists. During these meetings, we presented the preliminary results of the annotation task, and kicked off a semi-structured discussion to collect their feedback along three topics of interest: \textit{i)} evaluation of the data creation process; \textit{ii)} the relevance of morality for the task; \textit{iii)} the co-design of NLP technologies based on the dataset. 


\section{Corpus Analysis}
\label{sec:corpus-analysis}
In this section we analyse the CADE corpus.\footnote{The repository of the dataset will be made available under a CC BY license after the anonymity period.} We first describe the composition of annotators (\Cref{ss:process}), then the agreement between them (\Cref{ss:iaa}), finally we describe whether annotators moral profiles correlate with their sociodemographic traits (\Cref{ss:moral_profiles}).
% has an independent impact in their perception of canceling attitudes. 

\subsection{Annotation Process} \label{ss:process}
The annotation task involved  $57$ annotators belonging to three stakeholders: $30$ students, $17$ activists, and $10$ researchers. In addition, each annotator voluntarily shared their information about gender identity, and age, which was grouped into generations (Boomer, GenX, GenY and GenZ), nationality, ethnicity, education level and employment status.
% , and stakeholder type (student, researcher, activists). 
All the sociodemographic data are reported in \Cref{tab:ann_demographics} in Appendix \ref{app-demographics}.
%, the annotators could always write their answers instead of choosing among the provided options.
Having adopted a participatory approach to annotators recruitment (\Cref{annotation-lab}) that focuses on engaging with people who are interested in the phenomenon rather than hiring crowdworkers, the pool of annotators is not balanced along all sociodemographic axes. 

Each annotator is asked to annotate a subset of $210$ comments gathered from YouTube (\Cref{collection}): $35$ for each celebrity and the controversial event related to them. After cleaning unrelated comments, the final corpus includes $2,094$ texts and $11,935$ annotations. Each text has been annotated an average of $5.7$ times (with a median of $6$). Annotators had the option to refrain from assigning a label. We eliminated those who did not finish more than one-third of the task.


\subsection{Inter Annotator Agreement} \label{ss:iaa}
To assess the Inter Annotator Agreement (IAA) between annotators we employed Krippendorff's Alpha \cite{krippendorff2011computing}, which handles both agreement by chance (as the more common Cohen's Kappa agreement score), and incomplete annotations. We computed the IAA per sample and averaged them, resulting in a moderate agreement on stance ($0.501$), and a fair agreement on acceptability ($0.222$). The strong difference between the two tasks highlights how people tend to agree more when judging YouTube users' intentions rather than when they are asked to evaluate which messages they consider unacceptable \cite{davani2024disentangling}
%, compared to defining their boundaries in what can be considered acceptable. 
Labeling %a user's stance 
the stance expressed by a comment results less subjective than stating %what 
whether the comment is %is
to consider unacceptable. We report the scores broken down by sample in \Cref{tab:sample-iaa} in Appendix \ref{app-annotation_materials}.

\subsection{Annotators Moral Profiles} \label{ss:moral_profiles}
As a third part of the analysis of our annotation corpus, we assign annotators' moral profiles by leveraging the results of the MFT Questionnaire (\Cref{annotation-task}) and test to which extent they correlate with the following sociodemographic characteristics: gender, age, and stakeholder type.\footnote{Since the recruitment of stakeholder has been performed in European countries, we decided not to include %the other collected demographics 
nationality and ethnicity as variables of our study, as they result to be unbalanced.%because they are more unbalanced across the annotators.
 We also excluded the education level and employment status because they are strongly dependent on the type of stakeholders involved during the annotation, which is the only social condition monitored by design.} %we were interested in exploring these traits as potential stakeholders, the only variable monitored by design.}

We first computed the moral profile of annotators by obtaining their scores over the five foundations (care, fairness, loyalty, authority and purity). Each foundation is assigned a score between 0 (irrelevant) and 5 (very relevant). Together, the five scores represent the moral profile of the annotator.   

We then computed the Pearson coefficient between these profiles and gender, age, and stakeholder type. We found no statistically significant relationship between traits and moral foundations. This suggests that when annotators are grouped based on demographics, their moral profiles exhibit high variability and sparsity.

Given these preliminary results, we assigned a moral profile to each annotator through a clustering process that relies on their replies to the questionnaire. Annotators were represented with a vector with 5 dimensions, where each value in the vector represents the score obtained by the annotator for each moral foundation. 
We computed the pairwise distance with Euclidean metric and performed Agglomerative Clustering, which is preferred because it does not require setting the number of clusters as a parameter. We used Ward's linkage criterion;  to choose the best number of clusters we relied on three intrinsic evaluation metrics that do not need ground truth labels, namely Silhouette Coefficient \cite{ROUSSEEUW198753}, Calinski Harabaz Index \cite{Caliński01011974} and Davies Bouldin Index \cite{Bouldin1979}. We obtained $2$ clusters, Cluster\_0 (CL\_0) with $41$ annotators, and Cluster\_1 (CL\_1) with $16$.

\begin{figure}
    \centering
    \includegraphics[width=0.75\columnwidth]{latex/images/cl_foundatoins.png}
    \caption{Heatmap visualization for cluster analysis}
    \label{fig:cluster-foundations}
\end{figure}

We performed a qualitative analysis of the obtained moral clusters to understand which patterns emerge from their composition.
First, we analyzed whether the clusters were consistent with the scores that individuals obtained on each foundation. The heatmap shown in \Cref{fig:cluster-foundations} represents 
% the correlation between the two clusters and the moral foundations. 
the mean score of the five foundations for each cluster. Looking at the higher divergence, we computed the Pearson coefficient between the clusters and each of the moral foundations, reporting a correlation with loyalty ($r = 0.598$ with $p-value = 9.191 \cdot 10^{-7}$), authority ($r = 0.759$ with $p-value = 7.562 \cdot 10^{-12}$) and purity ($r = 0.792$ with $p-value = 2.063 \cdot 10^{-13}$).

These results are theoretically motivated by the MFT theory, which distinguishes between individual binding foundations (care, harm), rooted on the preservation of individual freedom, and group-binding foundations (loyalty, authority, and purity) which focus on the duties of individuals towards their social groups. In this sense, cluster\_1 is characterized by a higher moral attitude towards belonging to a group.

\begin{table}[]
    \centering\small
    \resizebox{1\columnwidth}{!}{
    \begin{tabular}{llcc}
    \hline
          \multicolumn{2}{c}{Demographics}&   CL\_0& CL\_1\\
    \hline
  \multirow{4}{*}{Gender identity} &Female& 23 (67.7\%) & 11 (32.3\%) (\\
 & Male& 16 (80\%) & 4 (20\%)\\
 & Non-binary& 2 (100\%)&-\\
 & Prefer not to say& -&1 (100\%)\\
 \hline
 \multirow{4}{*}{Generation}& Boomer& 1 (100\%)&-\\
 & GenX& 1 (50\%) &1 (50\%)\\
 & GenY& 13 (73\%) & 5 (27\%)\\
          &GenZ& 
     26 (73\%)& 10 (27\%)\\
 \hline
 \multirow{3}{*}{Stakeholder}& Activist& 15 (89\%) &2 (11\%)\\
 & Researcher& 7 (70\%)&3 (30\%)\\
 & Student& 19 (64\%) &11 (36\%)\\
 \hline
 \end{tabular}}
    \caption{Composition of the clusters (CL\_0 and CL\_1) in respect to annotators' gender identity, generation and role as stakeholder}
    \label{tab:cl-composition}
\end{table}

Once moral clusters have been validated against the MFT, we investigated again the presence of sociodemographic patterns in clusters to gain more insights from their potential correlation with moralities.
\Cref{tab:cl-composition} shows the composition of moral clusters broken down by annotators' gender, generation, and role as stakeholders. As it can be observed, the distribution of moral clusters along the gender axis is uniform: 32\% of women and 20\% of men belong to cluster\_1, showing a distribution that can be explained by annotators' gender imbalance. This is even more emphasized if generations are observed: moral clusters are perfectly distributed in Generation Y and Generation Z, despite the age of annotators being highly unbalanced towards the latter. This suggests that, in the context of this research, age is not a factor in determining the morality of annotators. Observing the distribution of moral clusters among stakeholders shows interesting patterns, instead. If on one side 36\% of students and 30\%  of researchers belong to cluster\_1, only 11\% of activists belong to this cluster. 
\section{Experiments}\label{sec:experiments}
We now evaluate SAPS using both qualitative and quantitative analyses. We first compare its zero-shot performance to R3L on benchmark tasks, then 
%perform delve into ablation studies and
an analysis of how our alignment approach behaves under different conditions.

\paragraph{Environments}
Our agents act by receiving pixel images as input observation, consisting of four consecutive $84 \times 84$ RGB images, stacked along the channel dimension to capture dynamic information such as velocity and acceleration.
We consider environments where we can freely change visual features (background color, camera perspective) or task (rewards, dynamics), therefore we use CarRacing \citep{klimov2016carracing} and LunarLander as both implemented in R3L.
CarRacing requires the agent to drive in a track using pixel observations, whose variations can be in the background color or the target speed, while LunarLander requires the agent to land on a platform, with variations comprising background color and different gravities.
% \AR{appendice per dettagli approfonditi su variazioni}.
No context is provided, hence the agents do not receive any information about the task.
% In the appendix we have other tests with atari env: \Cref{appendix:atari} \AR{riscrivi frase}

\paragraph{Baselines}
We mainly compare SAPS to (R3L), another zero-shot stitching method using relative representations whose approach is similar to ours.
For an additional baseline we also compare to naive zero-shot stitching, where we stitch encoders and controllers with no additional processing, to showcase the progress reached by the methods performing latent alignment techniques.
\section{Case Analysis}\label{sec:case_study}

\begin{figure}[b]
\centering
\includegraphics[width=\columnwidth]{Fig/case_study_v2.pdf}
\vspace{-0.6cm}
\caption{The case study of {\tool}.}
\label{fig:case_study}
\vspace{-0.6cm}
\end{figure}

To intuitively illustrate the benefits of {\tool}, we apply FixedCMD, FixedDBCMD, and {\tool} to Figure \ref{fig:motivation_tool_learning}'s example and observe the attack results from the output of ToolBench.
Figure \ref{fig:case_study} shows the results of the case study.
We can see that, the command generated by FixedCMD is defended by the LLM, so the frontend output and the backend toolchain are not affected.
FixedDBCMD can generate the command that successfully calls \textit{Book\_Flight} again and steals the \textit{Book\_Hotel}'s input. However, this abnormal toolchain is shown in the frontend, which will be observed by the users.
Compared with them, The command generated by {\tool} can not only achieve information theft but also have stealthiness, which means the attack is not exposed in the frontend.
In conclusion, {\tool} is applicable to generate effective commands that can applied to information theft attacks. 


\section{Conclusion}
This paper introduces \textbf{ReLearn}, a novel unlearning framework via positive optimization that balances forgetting, retention, and linguistic capabilities. 
Our key contributions encompass a practical unlearning paradigm, comprehensive metrics (KFR, KRR, LS), and a mechanistic analysis comparing reverse and positive optimization. 
%As underscored, unlearning should not only erase knowledge but also relearn knowledge for constructive outputs.

\label{sec:bibtex}
\section*{Limitations}
While ReLearn shows promising performance, several limitations remain.
(1) Computational Overhead: Data synthesis may hinder scalability.
(2) Metric Sensitivity: Our metrics still have limited sensitivity to subtle knowledge nuances.
(3) Theoretical Grounding: Understanding the dynamics of knowledge restructuring requires deeper theoretical investigation, which we plan to explore in the future work.
%--------------------------------------------------

\section*{Limitations}
While this work moves towards the development of participatory approaches to Natural Language Processing, the annotator pool is not balanced, especially leaning towards the White and Educated population. Moreover, although the chosen events had worldwide coverage, the majority of the annotators do not come from the same social background as the target celebrities. This made it possible to carry out the experiment with people from different nationalities, who, however, experienced cancel culture more as spectators than actors. In the future, we plan to expand the annotation lab to a more diverse group of annotators along all the sociodemographic axes. 
Finally, the current design involved watching only one video per celebrity, with exposure to the framing implicit in the political orientation of the chosen news source. While we aimed to ensure political diversity across the six selected videos, in the future, we intend to achieve this within the same canceling event.

\section*{Ethical Statement}
This research relies on the voluntary work of those who participated in the Annotation Labs. All the involved annotators freely accepted to take part to the laboratory, for which no compensation was provided. 
We adopted all the measures to protect data privacy and safeguard personal information. The work has been approved by the Ethics Committee of the institution of one of the authors. %is affiliated with. 

In the future, we plan to expand this work to a less Eurocentric context, concerning both the chosen celebrities and the involved annotators, looking at it as a necessary improvement to foster diversity. 




%Appendix todo: 
% annotators' demographics
% MFT
% check-out questionnaire 



\bibliography{latex/custom}

\clearpage

\appendix
\section{Annotator's demographics}\label{app-demographics}

\Cref{tab:ann_demographics} shows the demographic information about the annotators. We additionally asked them to indicate their nationality and first language. 38 are Italian, 2 are Spanish, 2 are Chinese, and there is one person for each of the following nationalities: Italian-Argentinian, Italian-Romanian, Iranian, Greek, Russian, Kazakhstan, Indian, Moldovan, Persian, Dutch, Romanian. As regards their mother tongue: 38 chose Italian, 3 Spanish, Russian, Chinese, 2 Persian, Romanian and Greek, and 1 Hebrew, Hindi, Farsi and Dutch.

\begin{table}[]
    \centering\small
    \begin{tabular}{lc|c}
    \hline
          \multicolumn{2}{c}{Demographics}&  \#annotators\\
          \hline
          \multirow{4}{*}{Gender identity}&Female& 
    34\\
 & Male&20\\
 & Non-binary&2\\
 & Prefer not to say&1\\
 \hline
 \multirow{4}{*}{Generation}& Boomer&1\\
 & GenX&2\\
 & GenY&18\\
 & GenZ&36\\
 \hline
 \multirow{5}{*}{Ethnicity}& White&48\\
 & Asian&5\\
 & Mixed&2\\
 & Black&1\\
 & Caucasian&1\\
 \hline
 \multirow{4}{*}{Education level}& Bachelor degree&29\\
 & Master degree&18\\
 & Doctorate degree&5\\
 & High school diploma&5\\
  \hline
 \multirow{8}{*}{Employment}& Unemployed&16\\
 & Full-time&15\\
 & Part-time&10\\
 & Due to start a new job&3\\
 & Student&8\\
 & Not in paid work&2\\
 & Occasional work&1\\
 & Freelancer&2\\
 \hline
 \end{tabular}
    \caption{Sociodemographic information.}
    \label{tab:ann_demographics}
\end{table}


% \section{Background}

\subsection{Multi-Agent Deep Reinforcement Learning}
 A typical system consists of the following components: agents, an environment, and a training algorithm, as depicted in Figure~\ref{fig:madrl_system}. Formally, we consider a system with $N$ agents, each indexed by $i \in \{1, \dots, N\}$. At each time step, the agent $i$ is presented with an observation $o_i$ and produces an action $a_i$. For the sake of generality, we included a possible communication channel $c_i$, seeing that it is increasingly used \cite{Zhu2022ASO}. In principle, we can extend the definition of communication to include the most common MADRL methods like parameter sharing \cite{Gupta2017CooperativeMC,Chu2017ParameterSD}, which can be seen as a form of latent space communication. Finally, the training algorithm provides feedback $\nabla_i$ to each agent.

Training algorithms in MADRL can be centralized, decentralized, or hybrid. Centralized training uses the joint action $a=(a_1,...,a_N)$ and the state $s$, which can be understood as an observation augmented by information at training time \cite{Lambrechts2023InformedPL}, and consists of applying classical RL to multi-agent problems like for AplhaStar \cite{Mathieu2023AlphaStarUL}. While decentralized training restricts each agent to local observations $o_i$, possibly including a local reward $r_i$, see IDQN \cite{Tampuu2015MultiagentCA} or IPPO \cite{Yu2021TheSE}. Hybrid approaches, such as centralized training with decentralized execution, leverage global information during training but allow agents to act independently using only local observations during execution, see VDN \cite{Sunehag2017ValueDecompositionNF}, QMIX \cite{Rashid2018QMIXMV}, MADPG \cite{Lowe2017MultiAgentAF} or MAPPO \cite{Yu2021TheSE}. Here, we consider agents based on DNNs; therefore, the feedbacks $\nabla_i$ are gradients of a loss $\ell$. Depending on the training algorithm, this loss can be a function of the reward $r$, the state $s$, the actions $a_i$, the observations $o_i$ and the communications $c_i$. For simplicity, we didn't include those dependencies in Figure~\ref{fig:madrl_system}.
 
\begin{figure}[ht]
    \centering
    \includegraphics[width=\linewidth]{figures/MADRL.pdf}
    \caption{Schema of a simplified view of MADRL systems. At each time step, the agent $i$ receives the initial observation $o_i$, complemented by potential communications $c_i$ and produces an action $a_i$. The agent learns throughout training by the means of gradients $\nabla_i$. }
    \label{fig:madrl_system}
\end{figure}


\subsection{Direct Interpretability of DNNs}
\label{sec:background_interp}

We now present an overview of the modern methods widely used to interpret DNNs in Computer Vision (CV) and Natural Language Processing (NLP). As these domains heavily relied on pre-trained models \cite{Simonyan2014VeryDC,He2015DeepRL, Radford2018ImprovingLU}, direct post-hoc methods have dominated the research landscape, providing key hindsight without altering models' architectures.


\paragraph{Feature importance.} Typical methods used in CV to understand convolutional networks involve visualising important pixels, i.e. saliency maps, \cite{Zeiler2013VisualizingAU,Selvaraju2016GradCAMVE}. Other methods compute importance by perturbing the input \cite{Covert2020ExplainingBR}, using the gradients \cite{Radford2015UnsupervisedRL,Selvaraju2016GradCAMVE,Shrikumar2016NotJA, Smilkov2017SmoothGradRN} or locally decomposing relevance \cite{Montavon2015ExplainingNC,Bach2015OnPE}. Recent works in NLP focus on the Transformer architecture and its attention mechanism \cite{Vaswani2017AttentionIA}, providing token-level insights \cite{Wiegreffe2019AttentionIN,Achtibat2024AttnLRPAL}. 


\paragraph{Prototypes:} a class of methods that creates explanations based on characteristic samples. In CV, it is common to analyse neurons using activation maximisation to create pre-images \cite{Mahendran2015VisualizingDC}, or find related images \cite{Chen2020ConceptWF}. Prototypes can be of various forms like perturbed images \cite{Ribeiro2018AnchorsHM}, cropped images \cite{Dreyer2023UnderstandingT} or latent space vector \cite{alain2018understanding,kim2018interpretability}. Recent works based on sparse autoencoders were able to elicit interpretable features in LLMs, i.e., prototypes \cite{Cunningham2023SparseAF}.

\paragraph{Latent manipulation:} techniques that further extend the interpretability of concepts and features by exploring the internal representations learned by models. These methods were introduced in CV with \cite{kim2018interpretability}, later derived as the field of representation engineering \cite{zou2023representation}. Such latent features enable locating, editing, erasing or decoding models' knowledge \cite{Meng2022LocatingAE,belrose2023leace, Ghandeharioun2024PatchscopesAU}, but causally modify or analyse the produced outputs \cite{rimsky2023steering, Kramar2024AtPAE}.

\paragraph{Circuit analysis:} provides a more granular understanding of model internals by examining pathways and dependencies between models' components, usually neurons or attention heads. Circuits were first discovered in CNNs \cite{Olah2020ZoomIA} before being formalised for Transformers \cite{elhage2021mathematical}.  These circuits revealed peculiar models' components that learned precise mechanisms like induction \cite{Olsson2022IncontextLA}. Using specific datasets, relevant circuits can be automatically discovered \cite{conmy2023automated}. More recent works focus on larger models' components at the layer scale \cite{Dunefsky2024TranscodersFI}.


% \begin{table}[H]
%  \begin{center}
%    % \tabcolsep = 2\tabcolsep
%    \begin{tabular}{ll}
%    \toprule
%    \textbf{Methodology} & \textbf{Related Works} \\
%    \midrule
%    Feature Importance & \cite{Zeiler2013VisualizingAU,Selvaraju2016GradCAMVE, Lundberg2017AUA, Bach2015OnPE,Radford2015UnsupervisedRL, Covert2020ExplainingBR, Montavon2015ExplainingNC, Achtibat2024AttnLRPAL, Smilkov2017SmoothGradRN, Wiegreffe2019AttentionIN} \\
%    %Ribeiro2016WhySI
%    %Katz2024BackwardLP
%    Prototypes   & \cite{Ribeiro2018AnchorsHM,Achtibat2022FromAM, Chen2020ConceptWF, Mahendran2015VisualizingDC, alain2018understanding,Cunningham2023SparseAF} \\
%    %Dreyer2023FromHT
%    %bills2023language
%    %Dar2022AnalyzingTI
%    Latent Manipulations & \cite{kim2018interpretability,Meng2022LocatingAE,zou2023representation,rimsky2023steering,belrose2023leace,Kramar2024AtPAE,Ghandeharioun2024PatchscopesAU} \\
%    Circuit Analysis          & \cite{Olah2020ZoomIA,elhage2021mathematical,Olsson2022IncontextLA,conmy2023automated, Dunefsky2024TranscodersFI}\\
%    \bottomrule
%    \end{tabular}
% \caption{Categorisation of modern direct interpretability methods drawn from CV and NLP domains.} \label{tab:interp_methods}
%  \end{center}
% \end{table}




\section{Instructions for the annotation process}\label{app-event_description}
\Cref{fig:annotation_platform} shows the instructions provided to the annotators. 

\begin{figure*}
    \centering
    \includegraphics[width=1\linewidth]{latex/images/annotation_guidelines.png}
    \caption{Instructions for the annotators.}
    \label{fig:annotation_platform}
\end{figure*}

In the following, we report all the celebrity descriptions.

\paragraph{J K Rowling} J K Rowling is a British author, and writer of the fantasy novel “Harry Potter”. In recent years she often expressed derogatory remarks about the transgender community. This has caused “Harry Potter” film actors such as Daniel Radcliffe, Emma Watson, Rupert Grint and Eddie Redmayne to speak out against the author, also calling for boycotts of her projects.
\paragraph{Kanye West} Kanye West, also known as Ye, is an American rapper. He has frequently spoken out on political and social issues with controversial opinions on topics such as abortion, capital punishment, welfare and gun rights. On frequent occasions he expressed antisemitic thoughts, stating his admiration for Adolf Hitler, denying the Holocaust, and supporting other conspiracy views against Jewish people, which led him to being banned from Twitter for 8 months.
\paragraph{Lizzo} Lizzo is an American rapper and singer. Throughout her career, she has been publicly interested and outspoken on social issues. She supported the LGBTQ+ community considering herself an ally, and advocated for body positivity, being subject to body shaming herself. In August 2023, she was accused of sexual, religious and racial harassment, disability discrimination, assault, weight-shaming and a hostile work environment by three former backup dancers, supported by other co-workers.
\paragraph{Halle Bailey} Halle Bailey is an American singer and actress. In 2023 she performed as the protagonist in the Disney movie “The Little Mermaid”, a choice that was subject to widespread criticism because in the cartoon the little mermaid was depicted as white, while Bailey is black. At the time, the hashtag \#NotMyAriel was launched, leading to a discussion about Disney film revision in the name of woke culture.
\paragraph{Ellen DeGeneres} Ellen DeGeneres is an American comedian and television host, famous for “The Ellen DeGeneres Show”. In July 2020 ten former employees of this show accused her of creating a toxic environment, with racist micro-aggressions, intimidation, abuse and sexual harassment episodes against female employees. She publicly apologized, promising that she would correct the issue.
\paragraph{Andrew Tate} Andrew Tate is an American and British former professional kickboxer, who became famous for promoting misogynist and violent messages, representative of the manosphere community. He was deplatformed from Twitter, Instagram, Facebook and TikTok. His account on Twitter was reinstalled in November 2022 after the Elon Musk acquisition. In December 2022 he was arrested with charges of rape, human trafficking and forming a criminal gang for the sexual exploitation of women. He was released a few months later.
\section{Annotation materials}\label{app-annotation_materials}
\Cref{tab:video_info} presents all the information about the selected YouTube videos.

\begin{table*}[ht]
    \centering\small
    \begin{tabular}{ccccc}
\hline
Celebrity&  Topic &News Broadcast & Political orientation &Minutes\\
\hline
 J K Rowling& Homo-transphobia& Sky News Australia & Right wing & 2.29\\
 Kanye West& Antisemitism& Fox 11 Los Angeles& Right wing & 1.14\\
 Lizzo& Harassment& Fox News& Right wing &3.24\\ 
 Halle Bailey& Incelism& CBS Media& Left wing &1.47\\
 Ellen DeGeneres& Bullying& CBS Media& Left wing &3.01\\
 Andrew Tate& Sexual assault & Law and Crime network& Left wing &2.26 \\
 \hline
    \end{tabular}
    \caption{Information about the selected YouTube videos their duration for each target celebrity.}
    \label{tab:video_info}
\end{table*}

\Cref{tab:sample-iaa} reports the detailed sample composition and Inter Annotator Agreement.

\begin{table*}[]
    \centering\small
    \begin{tabular}{ccllllll}
    \hline
 Samples &\#Researchers&\#Students& \#Activists& \#Texts&\#Annotations& $\alpha$ Stance& $\alpha$ Acceptability\\
 \hline
         Sample\_0&    1&4& 2&  210&1,470& 0.518&0.177\\
 Sample\_1&  1&2& 2&  210&1,050& 0.472&0.306\\
 Sample\_2&  1&3& 2&  208&1,248& 0.39&0.189\\
 Sample\_3&  1&3& 3&  210&1,470& 0.548&0.153\\
 Sample\_4&  1&4& 1&  209&1,254& 0.491&0.183\\
 Sample\_5&  1&4& 2&  210&1,470& 0.601&0.283\\
 Sample\_6&  1&2& 1&  210&840& 0.505&0.202\\
 Sample\_7&  1&1& 1&  210&630& 0.425&0.215\\
 Sample\_8&  1&3& 1&  208&1,040& 0.463&0.271\\
 Sample\_9&   1&4& 2&  209&1,463& 0.603&0.245\\
 \hline
 Total & 10& 30& 17&  2,094&11,935&  & \\
 $\alpha$ $\text{mean}_{\text({std})}$ & & & &  &&$0.501_{(0.069)}$&$0.222_{(0.051)}$\\
 \hline
    \end{tabular}
    \caption{Sample composition, annotation and IAA. }
    \label{tab:sample-iaa}
\end{table*}





\section{Checkout Questionnaire}\label{app-checkout_qst}
We report the questions asked to the annotators in the checkout questionnaire:

\begin{itemize}[noitemsep]
    \item What opinion do you have about the topic of \textit{social media shaming} after the annotation? 
    \item How did you find the annotation task? (difficult? emotionally impactful? etc.)
    \item Is there anything you would like to add?
    \item Is there anything you would like to change?
    \item Did the moral questionnaire influence the way you performed the annotation? 
    \item Did you perceive the completion of the questionnaire and annotation as thematically related?
    \item How would you use a tool that can recognize social media shaming? 
\end{itemize}

This questionnaire was not mandatory and received $33$ answers.
\section{LLM implementation details}

In this work, we have used the following public models:
\begin{itemize}
    \item \texttt{OLMo 7B}\footnote{\url{https://huggingface.co/allenai/OLMo-2-1124-7B-Instruct}}
    \item \texttt{BLOOMZ 3B}\footnote{\url{https://huggingface.co/bigscience/bloomz-3b}}
    \item \texttt{DeepSeek R1 1.5B}\footnote{\url{deepseek-ai/DeepSeek-R1-Distill-Qwen-1.5B}}
    \item \texttt{OPT-IML 1.3B}\footnote{\url{https://huggingface.co/facebook/opt-iml-max-1.3b}}
    \item \texttt{Llama 3.2 3B}\footnote{\url{https://huggingface.co/meta-llama/Llama-3.2-3B-Instruct}}
    \item \texttt{Ministral 8B}\footnote{\url{https://huggingface.co/mistralai/Ministral-8B-Instruct-2410}}
\end{itemize}

To generate the annotations with these models, we have followed a zero-shot approach, prompting the models to generate their annotations.
All models were executed in a NVIDIA Titan X Pascal GPU, with 12GB of memory.

To generate the annotations regarding stance, the used prompt is as follows:
\textit{Classify the text into being defensive, neutral or attacking.}
Similarly, we used the following prompt to generate the acceptability annotations:
\textit{Classify the text into a scale from 1 to 4, considering how much the text contributes to shaming or degrading a subject, being 1 the lower and 4 the higher.}




\end{document}
