\section{Simulations}
To examine the learning ability of Dynamic RL, the author applied it to two dynamic tasks:
a sequential navigation task, which served as a memory-required task, and a slider-crank control task,
which served as a dynamic pattern generation task.
%In both tasks, the results of the conventional RL in which BPTT was used is shown for reference.
%but because the proposed method has a large space to be improved,
%strict comparison was not done,

The network architecture used is shown in the left figure in Fig.~\ref{fig:Task_SW},
regardless of the learning method or task, although the parameters used are different.
The architecture consists of an actor network and a critic network;
as mentioned, Dynamic RL was applied only to the actor network here.
The critic network was trained in the same manner as conventional RL.
The critic network was an Elman-type RNN with a single hidden layer comprising 50 neurons, having feedback connections to itself.
It was trained to minimize the square of the TD-error using BPTT
with the training signal as in Eq.~(\ref{Eq:C_train}).

The actor network used a multiple timescale RNN (MTRNN) \citep{MTRNN}
to improve performance and verify whether the proposed RL would function
on a more complex architecture than simple Elman-type RNNs.
The MTRNN comprised two hidden layers with different time constants and one output layer.
The two hidden layers were interconnected, and one of them, called upper hidden layer (100 neurons),
did not have any connection to the input or output layer; however, it had self-feedback connections.
Although the other hidden layer, called lower hidden layer (200 neurons), had connections with the input and output layers,
it did not have self-feedback connections.
Neurons in the upper hidden layer in the actor network were dynamic,
while the rest, including all neurons in the critic network, were static (time constant $\tau=1$).
All connected layers are fully connected in both networks.

%The number of neurons and the time constant are shown in the Table \ref{Table:Num_Neuron}
%\begin{table}[ht]
%\label{Table:Num_Neuron}
%\begin{center}
%\caption{Neral network architecture}
%    \vspace{2mm}\small
%  \begin{tabular}{c|c|c}
%   \toprule
%     \multicolumn{3}{c}{number of neurons}\\
%    \midrule
%     critic\_net & & 50\\
%    \multirow{2}{*}{actor\_net} & upper layer & 100\\
%    & lower layer & 200\\
%   \bottomrule
%   \toprule
%     \multicolumn{3}{c}{neuron model $\tau$}\\
%    \midrule
%     actor\_net & upper layer & dynamic ($\tau=5.0$)\\
%    \multicolumn{2}{c}{others} & static\\
%   \bottomrule
%   \end{tabular}
%\end{center}
%\end{table}

In the proposed Dynamic RL, either SRL or SAL
% (Eqs.(\ref{Eq:SRL},\ref{Eq:SRL_Bias}))  (Eqs.(\ref{Eq:SAL},\ref{Eq:SAL_Bias}))
was applied to each hidden neuron depending on its average sensitivity $\bar{s}$ as shown in Fig.~\ref{fig:DynamicRL};
meanwhile, SAL was not applied to each output neuron to avoid the risk that SAL would cause the output to deviate from the ideal value.
To see the difference from the conventional RL and also how SRL and SAL worked respectively,
six learning conditions varying the method of applying learning were compared in the following
as in Table \ref{Tab:CompSet}.
\begin{table}[t]
  \label{Table:LearningCases}
  \begin{center}
    \vspace{0.5cm}
    \caption{Six learning conditions varying the application methods of learning for the actor network compared in the following simulations.
    (DYN\_RL: Dynamic RL, Conv\_RL: conventional RL, Reg: regularization (Eq.~(\ref{Eq:Regularize})),
     SRL \& SAL: SRL (Eqs.~(\ref{Eq:SRL})(\ref{Eq:SRL_Bias})) and SAL (Eqs.~(\ref{Eq:SAL})(\ref{Eq:SAL_Bias})), 
     but actually one of them is applied depending on the condition as in 
     Fig.~\ref{fig:DynamicRL} at each step in each neuron, BPTT: Backpropagation Through Time, 
     BP: Backpropagation not going back in time.
   In convenience, the notation `BP' is used also for the output layer
   where no propagation of error signal is necessary.
   The training signals for `BPTT' or `BP' are as in Eq.~(\ref{Eq:ConvRL}).)}
    \label{Tab:CompSet}
    \vspace{5mm}
    \small
      \begin{tabular}{l|ccc}
        \toprule
        \multicolumn{1}{c|}{\multirow{2}{*}{Name}}
        & \multicolumn{3}{c}{combination(in: input, hid: hidden, out: output)}\\ \cline{2-4}
             & hid $\to$ out & hid $\to$ hid & in $\to$ hid\\
         \midrule
         (A) \ \ DYN\_RL (proposed) & SRL \& Reg & SRL \& SAL & SRL \& SAL\\
         (A-1) DYN\_RL& \multirow{2}{*}{SRL \& Reg} & \multirow{2}{*}{SAL} & \multirow{2}{*}{SRL \& SAL}\\
          \ \ \ \ \ \ (No SRL for hid$\rightarrow$hid)\\
         (A-2) DYN\_RL (No SAL) & SRL \& Reg & SRL & SRL\\
         \midrule
         (B) \ \ Conv\_RL (BPTT) & BP \& Reg & BPTT & BPTT\\
         (B-1) Conv\_RL (BPTT+ SAL) & BP \& Reg & BPTT \& SAL & BPTT \& SAL\\
         (B-2) Conv\_RL (BP+SAL) & BP \& Reg & BP \& SAL & BP \& SAL\\ 
    \bottomrule
    \end{tabular}
    \label{Table:CompSet}
  \end{center}
\end{table}

%\begin{table}[ht]
%  \begin{center}
%    \caption{Learning methods for the actor network compared in the following simulations.\\
%                   (``hid'': applying to all the hidden layers, ``out'': applying to the output layer)}
%    \vspace{1mm}\small
%      \begin{tabular}{l|cccc} \toprule
%        \multicolumn{1}{c}{Name}\vline
%             & \hspace{-0.2cm}\begin{tabular}{c}SRL\\Eq.(\ref{Eq:SRL})\end{tabular} \hspace{-0.2cm}
%             & \hspace{-0.2cm}\begin{tabular}{c}SAL\\Eq.(\ref{Eq:SAL})\end{tabular}\hspace{-0.2cm}
%             & \hspace{-0.4cm}\begin{tabular}{c}Grad\\Eq.(\ref{Eq:GradL},\ref{Eq:GradL_Bias})\end{tabular} 
%                 \hspace{-0.4cm}
%             & \hspace{-0.5cm}\begin{tabular}{c}BPTT or BP\\Eq.(\ref{Eq:ConvRL})\end{tabular}\hspace{-0.5cm}\\ 
%         \midrule
%         (A) DYN\_RL (proposed) & hid, out & hid & out & - \\
%         (A-1) DYN\_RL (SRL: out only)& out & hid & out & -\\
%         (A-2) DYN\_RL (No SAL) & hid, out & - & out & -\\
%         (A-3) DYN\_RL (No Grad) & hid, out & hid & - & - \\
%         \midrule
%         (B) Conv\_RL & - & - & - & \hspace{-0.2cm}hid, out\hspace{-0.2cm}\\
%         (B-1) Conv\_RL + SAL & - & hid & - & \hspace{-0.2cm}hid, out\hspace{-0.2cm}\\
%         (B-2) Conv\_RL (No BPTT)+SAL & - & hid & - & \hspace{-0.2cm}hid, out\hspace{-0.2cm}\\ 
%    \bottomrule
%    \end{tabular}
%    \label{Table:CompSet}
%  \end{center}
%\end{table}

\subsection{Sequential navigation (memory-required) task}
In this simulation, as shown in the right figure of Fig.~\ref{fig:Task_SW},
there is a $20 \times 20$ size walled field centered at the origin.
An agent starts each episode at a randomly chosen location
weighted on the peripheral area of the field to accelerate learning (\ref{App:Task}).
% In order to promote the learning in the peripheral area, once per every 4 episodes,
% the location is limited to a point on the four sides of the $18\times 18$ square
% whose center is on the origin.
The agent received a reward of $r=0.8$ ($r$: reinforcement signal)
when the center of the agent reached the final goal at $(0.0, -3.0)$ after reaching the subgoal at $(0.0, 4.0)$.
At every step when the agent center is on the final goal before reaching the subgoal,
it receives a small penalty of $r=-0.03$.
The shape of each goal (final goal or subgoal) is a circle with a radius of 2.0, 
and when the center of the agent comes within the circle, the agent is considered to have reached the goal.
There is a visual sensor with $11 \times 11=121$ visual cells, and only the agent appears on it.
Each sensor cell covers a $1 \times 1$ square, which is identical to the agent's size.
%the size of each is $1\times1$.
%The sensor is fixed with its center at the origin, and the cells are arranged in a square without overlap.
%The sensor captures only the agent that can be seen as a $1\times1$ square,
%and each cell outputs the area occupied by the agent in the range from 0 to 1.
The agent cannot know the location of each goal from the sensor signals.
The agent also receives one more sensor signal.
Only when the agent's center is within the subgoal, it is 2.0; otherwise, it is 0.0.
Therefore, the agent must memorize whether it has already reached the subgoal or not in the RNN
and switch its behavior according to the memory.
\begin{figure*}[th]
\centerline{\includegraphics[scale=0.27]{Task1.pdf}}
\caption{Sequential navigation task and the RNN used in the simulation.}
\label{fig:Task_SW}
\end{figure*}

The agent receives 122 sensor signals in total and inputs them to both the critic and actor networks.
The actor network has two outputs, which are used as the agent's movement vector after normalizing
to make the movable area circular, as described in \ref{App:Task} at each step.
In each episode, the agent resets all the internal states of the two RNNs to 0.0 at first,
and for preparation, it computes the RNNs for the first five steps without moving or counting steps.
%after being multiplied by 1.25 and clipped between the values of -1.0 and 1.0.
%Furthermore, the actual one-step movement of the agent is normalized while maintaining its direction
%so that its range of movement is confined to a circle with a radius of 1.0.
%For example, if the actor outputs are (0.0, 0.9), the agent will move by (0.0, 1.0),
%and if they are (0.3, 0.4), it will move by $(0.3, 0.4)$.
When the agent reaches a wall around the field, it stops there, but no penalty is given.
The episode terminates either when the agent receives a reward or after 200 steps have passed.
At each step, the agent updates all the weights and biases in the critic network
using the training signal as Eq.~(\ref{Eq:C_train}),
and those in the actor network according to Eqs.~(\ref{Eq:SRL})(\ref{Eq:SRL_Bias})(\ref{Eq:SAL})(\ref{Eq:SAL_Bias})(\ref{Eq:Regularize}).
When it received a reward, the critic at the next step $P_{t+1}$ was set to 0.0.
The main used parameters are shown in Table \ref{Table:Parameters1}.
In Dynamic RL, the spectral radius of the initial self-feedback weight matrix of the upper hidden layer in the actor network
was as large as 3.0 to ensure that the system, including actor network, generates chaotic dynamics,
and the agent can explore the field.
It was 1.3 in the conventional RL from the viewpoint of appropriate error propagation.
% BPTT or BP was used in the actor network in the case of conventional RL, and was always used in the critic network.
%Detailed parameters are shown in \ref{App:Table}.
\begin{table}[t]
  \begin{center}
    \caption{Parameters used in the sequential navigation (memory-required) task. (MA: Moving Average)}
    \vspace{2mm}\small
      \begin{tabular}{c|c} \toprule
        \begin{tabular}{c}Number of Neurons\\ (input-hidden[lower-upper]-output)\end{tabular}
         &  \begin{tabular}{c}122-50-1 (critic)\\
        122-[200-100]-2 (actor) \end{tabular} \\ \midrule
               \begin{tabular}{c}Spectral Radius of Initial \\ Self-feedback Connection Weight Matrix\end{tabular}
        & 
        \begin{tabular}{c}1.3 (critic, actor(Conv\_RL))\\
        3.0 (actor(DYN\_RL)) \end{tabular} \\ \midrule
        Neuron Model	& 
        \begin{tabular}{c}
%\renewcommand{\arraystretch}{0.5}
            Dynamic with $\tau=5\ \ \ \ \ \ \ \ \ \ \ \ \ \ $\\(upper hidden layer in actor net.)\\
%\renewcommand{\arraystretch}{1.0}
            Static (others)
        \end{tabular}
        \\ \hline
	Target Sensitivity for SAL  $s_{th}$ in Fig.~\ref{fig:DynamicRL}& 1.3\\ \midrule
	Discount Factor $\gamma$ in Eqs.~(\ref{Eq:TDerr}),(\ref{Eq:C_train})& 0.98\\ \midrule
	Rate of Regulation $\eta_{reg}$ in Eq.~(\ref{Eq:Regularize}) & $1e^{-6}$\\ \midrule
	Raise Critic Criterion $C_{th}$ in Eq.~(\ref{Eq:Raise_Critic})  & 0.1\\ \midrule
	Rate of Raising Critic $\eta_{raise}$ in Eq.~(\ref{Eq:Raise_Critic}) & 0.0005\\ \midrule
%        learning rate $\eta_{SAL}$ in Eq.~(\ref{Eq:Delta_w}) & 0.00002\\ \hline
%	decay rate $\beta$ in Eq.~(\ref{Eq:decay}) & $ 0.99 $\\ \hline
	Const. $\alpha$ for MA of Sensitivities in Eq.~(\ref{Eq:Ave_sen}) &  0.001\\ \midrule
	Const. $\beta$ for MA of Critic in Eq.~(\ref{Eq:Ave_C}) &  0.0001\\ \midrule
	Truncated Number of Steps in BPTT & 20\\ \midrule
	Initial Weights and Biases, Learning Rate & Refer to \ref{App:Table}\\ \bottomrule
      \end{tabular}
    \label{Table:Parameters1}
  \end{center}
\end{table}

In one simulation run, the agent learned for 20,000 episodes first and then for another 20,000 episodes
after swapping the roles of the actor's two outputs to observe its adaptability to environmental changes.
For each learning condition, a total of 40 simulation runs with different random sequences were performed.
The sequences were used to determine the initial connection weights, biases, and also the initial agent location at each episode.
%When the same random sequence was used, the initial agent location in the same episode was the same
%regardless of the learning method.
If the reduction in the average number of steps to the final goal was too small despite some progress in learning,
the simulation run was treated as a ``failure'' and terminated midway through the learning process (refer to \ref{App:Task} for details).
After each of the 20,000 and 40,000 episodes, a test was performed.
The agent started from 46 initial locations, arranged in a grid pattern with a 3.0 unit interval,
ranging from $-9.0$ to $9.0$ on the field, excluding the two goal areas, and did not learn. 
The random noise for exploration was not added to the actor outputs even in the case of conventional RL.
If the average number of steps in either test exceeded 20, the simulation run was deemed ``overrun.''
A simulation run was considered a ``success'' when it was neither a ``failure'' nor ``overrun.''

\begin{figure}[t]
\centerline{\includegraphics[scale=0.7]{CompPerf1_SW.png}}
\caption{Comparison of success rate among six learning conditions as in Table~\ref{Table:CompSet}
in the sequential navigation task.}
\label{fig:CompPerf1_SW}
\end{figure}
First, Fig.~\ref{fig:CompPerf1_SW} compares the success rates from 40 simulation runs
across the six learning conditions listed in Table \ref{Table:CompSet}.
The (A) Dynamic RL proposed in this paper achieved 36 successes, 2 failures, and 2 overruns out of the 40 runs.
In the two failure runs, learning failed before the environmental change, and no failure occurred afterward.
In the two overruns, the average number of steps during learning was consistently below 20.0 just before the test;
however, due to some instability in Dynamic RL, it occasionally exceeded 20 steps in the test.

The success rate for (B) conventional RL using BPTT is only about 40\%, which is lower than for (A) Dynamic RL.
However, in the case of (B-1), where SAL was applied to the hidden neurons in parallel with conventional RL,
the results improved to 36-4-0, making them comparable to those of case (A) Dynamic RL.
Therefore, the case (B-1) will serve as the main comparison target for Dynamic RL hereafter.
When the error was not propagated to the past in the case of (B-2),  
the performance was worse than in the case of (B-1).

Meanwhile, comparing (A), (A-1), and (A-2) suggests that both SRL and SAL are necessary
for good performance in Dynamic RL.
Different from the case of the chaos-based RL method from the previous work \citep{IJCNN2015},
applying SRL to the weights between hidden neurons clearly improved the results.
%It can be said that Dynamic RL proposed here is effective also to the learning of weights among hidden neurons.
%However, though BPTT was applied, it is decreased to around 50\% if learning was not done for the past as (g).
%In (A) Dynamic RL, in which SAL was applied to the hidden layers by default,
%the agent sometimes took more than 20 steps on average in the tests,
%and the success rate was slightly lower than (B-1) conventional RL.
%but when SAL was not applied, the learning was almost unsuccessful as (A-2).
%The first thing that can be said from the results indicates at first that, 
%SAL, which can ensure sensitivity in each neuron, plays a significant role for the successful learning.
%Then it can be seen that SRL works for the learning of memory-required task.
%When SRL was applied only for the output layer,
%the success ratio was not so small, but as mentioned.

\begin{figure}[ht]
\centerline{\includegraphics[scale=0.6]{LearningCurve_SW.png}}
\caption{Learning curve for the sequential navigation task. The vertical axis indicates the average number of steps
necessary to reach the final goal successfully over 500 episodes. The plots and shaded areas indicate the mean and standard deviation
over successful simulation runs.
(a) shows the comparison result between Dynamic RL and conventional RL.
(b) shows the comparison result among three cases to assess the effect of SRL or SAL.}
\label{fig:LearningCurve_SW}
\end{figure}
Figure \ref{fig:LearningCurve_SW} (a) compares the learning curves of Dynamic RL and conventional RL.
The vertical axis shows the average number of steps needed to reach the final goal,
calculated every 500 episodes.
The plot lines and shaded areas in the figure represent the mean and standard deviation
across all successful simulation runs.
When the agent could not reach the final goal within 200 steps, it was recorded as 200 steps.
It was observed that in all cases, the number of steps decreased as learning progressed.
After the environment changed just after the 20,000th episode, the number of steps increased once
but then decreased again.
The large standard deviation in conventional RL occurs because,
depending on the simulation run, the number of steps did not decrease from around 200 for some episodes.
The learning curve for (B) conventional RL without applying SAL was similar to that for (B-1) with SAL,
but slightly worse overall, although no figure is presented here.

Thus, Dynamic RL functions effectively even in a task where the agent must learn
to find necessary information, memorize it, and reflect it in its motions for appropriate future behaviors.
Dynamic RL does not require backward computation through time.
The actual computational cost for learning the actor network, compared to conventional RL (BPTT+SAL),
was 1:8.0 when using the author's Python and NumPy program without using a framework or a GPU. 
Nevertheless, the curve before the environmental change was similar to that of conventional RL.
Moreover, learning after the environmental change was significantly faster in Dynamic RL than in conventional RL.
In Dynamic RL, learning was faster after the environmental change compared to the beginning of the run;
whereas in conventional RL, it was slower.

Figure \ref{fig:LearningCurve_SW} (b) illustrates the learning curves for three cases in Dynamic RL:
(A) regular case, (A-1) without SRL for hidden-to-hidden connections, and (A-2) without SAL,
to evaluate the effects of SRL and SAL.
Note that even in the case of (A-1), SRL was still applied to the other connection weights.
%(A) proposed Dynamic RL, (A-1) Dynamic RL (SRL was not applied to the weights between hidden layers),
%(A-2) Dynamic RL (SAL was not applied).
Case (A) outperforms both (A-1) and (A-2), leading to the conclusion that both SRL and SAL are beneficial.
SAL seems to be generally more effective for fast learning, while application of SRL
to the connection weights between hidden neurons is particularly effective when the environment changes.

\begin{figure}[ht]
\centerline{\includegraphics[scale=0.7]{CompPerf_SW.png}}
\caption{Comparison of the average number of steps in the final stage of learning and in the test
among four learning conditions.
The plots and the error bars indicate the mean and standard deviation over successful simulation runs.}
\label{fig:CompPerf}
\end{figure}
Figure~\ref{fig:CompPerf} compares the average number of steps in the last 500 learning episodes
to that in the final test, which consists of 46 episodes after learning.
This comparison is made across the four learning conditions presented in Fig. \ref{fig:LearningCurve_SW}.
% Figure~\ref{fig:CompPerf} shows its mean and standard deviation over the simulations where learning was successful.
In all cases, the number of steps in the test is lower than during learning.
This difference is due to the frequent starting positions at the periphery during learning.
In (B-1) conventional RL with SAL, the number of steps in the test is significantly lower than during learning
because no exploration noise was introduced in the test.

In case (A) Dynamic RL,
the agent's behavior became less exploratory through learning,
leading to a smaller average number of steps during the final stage of learning compared to (B-1) conventional RL.
However, the learning process in Dynamic RL was somewhat unstable,
sometimes causing sudden and significant increases in the number of steps,
which in turn increased both the average and standard deviation to that extent.
In the test, exploration was still present, and the average is larger than in conventional RL.
%exploration does not disappear, although the behavior becomes less exploratory through learning.
% but becomes smaller autonomously as the progress of learning as will be shown in Fig. \ref{fig:Task1_Loci}.
% Looking at this from the standpoint of Dynamic RL,
%This may have helped the Dynamic RL to adapt quickly to the new environment
%as shown in Fig. \ref{fig:LearningCurve_SW}.
%Furthermore, as previously mentioned, because the learning process is more unstable in Dynamic RL,
%the number of steps sometimes increases largely, and
%the standard deviation is larger in Dynamic RL than in conventional RL.
%As a result, in Dynamic RL, the average number of steps is smaller than in conventional RL in the final stage of learning
%but larger during the test.
%From the comparison of (A) to (A-1) or (A-2), we can see the usefulness of SRL and SAL also in the final stage of learning.
Additionally, in Dynamic RL, the absence of SAL or SRL in (A-1) or (A-2) resulted in an increase in the average and
standard deviation.

\begin{figure}[p]
\centering
%  \begin{minipage}[t]{0.4\linewidth}
  \begin{subfigure}[t]{0.45\linewidth}
  \centerline{\includegraphics[scale=0.3]{Loci_DYN36.pdf}}
  \subcaption*{(A) DYN\_RL (proposed)}
  \end{subfigure}
 %\end{figure}
 %\begin{figure}[t]
%  \begin{minipage}[t]{0.4\linewidth}
  \begin{subfigure}[t]{0.45\linewidth}
  \centerline{\includegraphics[scale=0.3]{Loci_BPTT36.pdf}}
  \subcaption*{(B-1) Conv\_RL (BPTT+SAL)}
  \end{subfigure}
\caption{Comparison of sample agent behaviors at four stages in learning between (A) Dynamic RL and (B-1) conventional RL (with BPTT and SAL).}
\label{fig:Task1_Loci}
\end{figure}
%\newpage
Figure~\ref{fig:Task1_Loci} shows sample trajectories from a simulation run
that successfully learned for both (A) Dynamic RL and (B-1) conventional RL+SAL cases.
Since the number of steps varies largely among simulation runs in Dynamic RL, as shown in Fig.~\ref{fig:CompPerf},
the results for an average run are presented,
where the number of steps ranks 24th in the first test and 11th in the second test
out of 36 successful simulation runs for (A) Dynamic RL.
In (A) and (B-1), the cases for the same random sequences are shown.
The initial weights and biases in the RNNs, as well as the agent's starting position for each episode, were the same,
except for the weight matrix of the output layer and the spectral radius of the upper hidden layer's
self-feedback connection weight matrix in the actor network.
 
Figures~\ref{fig:Task1_Loci}(A)(a) and \ref{fig:Task1_Loci}(B-1)(a) show the agent paths in the first ten episodes.
The agent explored the entire field in both cases, but the exploration appeared different.
In (A) Dynamic RL, the paths were smoother due to the chaotic dynamics generated internally
with the large time constant in the upper hidden layer.
In (B-1) conventional RL with BPTT+SAL, they were jagged and irregular due to the random noise
introduced for exploration at each step.

Figures~\ref{fig:Task1_Loci}(A)(b) and \ref{fig:Task1_Loci}(B-1)(b) show the results of the simplified test
with nine initial locations
after 20,000 episodes of learning, just before the environmental change.
In this test, random exploration was stopped even in the conventional RL case.
Starting from any initial location, the agent first reaches the subgoal and then the final goal in both cases.
Dynamic RL could learn memory-required behaviors without using backward computation, such as BPTT.
However, the agent's paths in (A) were not as smooth as in the case of (B-1) conventional RL.
%As mentioned, i
In (A) Dynamic RL, the actor outputs were easily saturated and
the agent was apt to move diagonally although regularization was applied to avoid it. 
This problem needs to be solved.

Figures~\ref{fig:Task1_Loci}(A)(c) and \ref{fig:Task1_Loci}(B-1)(c) show the actual agent paths
for ten episodes just after the environmental change,
in which two actor outputs were swapped.
For instance, if the agent is located in the lower right direction of the subgoal
and intends to move toward the upper left direction, it actually moves to the lower right, which is opposite to its intention.
Consequently, the agent tends to gravitate toward the bottom right or top left corner.
Therefore, in (B-1) conventional RL with BPTT+SAL,
the agent's final position was in the area defined by $|x|, |y| > 9.0$ in 8 out of 10 episodes.
However, in (A) Dynamic RL, the agent's paths appear more exploratory,
and its final position was never in this area across the 10 episodes.
When the agent was not allowed to learn after the environmental change,
it moved to either the bottom right or top left corner from the starting position and remained there,
although no figure is presented here.
This suggests that Dynamic RL promotes exploration.
It can be inferred that the difference in exploration between the two RLs leads to variations in learning speed
after the environmental change.
Figures~\ref{fig:Task1_Loci}(A)(d) and \ref{fig:Task1_Loci}(B-1)(d) show the final simplified test results
after 40,000 episodes, i.e., after all learning was completed.
As in the case of (b), the agent regained the ability to reach the final goal in both cases,
and the paths were smoother, resulting in fewer steps in conventional RL.

Finally, the changes in dynamic characteristics during learning are compared.
Here, the exploration exponent is introduced, which is a type of Lyapunov exponent for the entire system, including
both the RNN and the physical system (See \ref{App:Expl_factor} for its calculation).
During learning, every 100 episodes, the exponent was computed for each of the nine starting points,
which are the same as in Fig.\ref{fig:Task1_Loci} (b) or (d), and the nine exponents were averaged.
It is calculated for each of the two environments that appeared before and after the environmental change,
referred to as Env1 and Env2, respectively.
%Env1 and Env2 are the learning environments before and after the environmental change respectively.

Although the exponent fluctuated significantly, the average over successful simulation runs
reveals the characteristics of each learning condition, as shown in Fig.~\ref{fig:ExplExponent}.
The four graphs show how the exploration exponent changes as learning progresses
under the four learning conditions.
In (A), (A-1), and (A-2), the starting value is larger than in (B-1)
due to the differences in the output connection weights
and the spectral radius of the self-feedback connection weights in the upper hidden layer.
%That is because the output connection weights in the actor network are not zero,
%and furthermore, the spectral radius of the initial self-feedback connection weights in the higher hidden layer
%was larger in (A), (A-1), and (A-2).
%The way of value change differs among the four learning conditions but
%In all cases, the value in the Env1 is larger than in the Env2 in the first half of the learning,
%and after the environmental change,  the relationship between the larger and smaller values is reversed.
\begin{figure}[tb]
\centerline{\includegraphics[scale=0.6]{Expl_SW.png}}
\caption{Comparison of changes in exploration exponents in the sequential navigation task
among four learning conditions.
Refer to the text for the definition of the exploration exponent.
Env1 and Env2 are the learning environments before and after the environmental change, respectively.
Thick lines indicate an exponent in the learning environment: Env1 before environmental change and
Env2 after that.}
\label{fig:ExplExponent}
\end{figure}

In the proposed Dynamic RL (A), the exponent in the learning environment first increased
and then gradually decreased while maintaining the value in the learning environment
(Env1 for the first half of the learning episodes and Env2 for the second half)
greater than that in the non-learning environment.
When SRL was not applied to the hidden-to-hidden connections in (A-1),
the exponent did not decrease as learning progressed.
This suggests that SRL facilitates convergence in the dynamics by reducing each neuron's sensitivity
around the state transitions when the TD-error is positive.
Conversely, when SAL was not applied in (A-2), the exponent initially decreased.
This indicates that SAL prevents the system dynamics around the actual state transitions
from converging excessively by increasing each neuron's sensitivity.

Moreover, after the environment changed, the exponent for the learning environment Env2 in (A)
increased much faster than in the other cases.
%In many cases, the agent often went to the corner of the field in the non-learning environment,
%and the locations on the two agent paths for the computation of the exponent overlapped exactly.
This aligns with the rapid adaptation in the learning curve observed in Fig.~\ref{fig:LearningCurve_SW}.
The slow adaptation observed in (A-1) and (A-2) suggests that the quick increase of the exponent
by both SRL and SAL was the key to quick learning after the environmental change.
While the exponent for the new learning environment, Env2, increased,
the exponent for Env1, which was newly transitioned into a non-learning environment, decreased.
This indicates that the agent enhanced its exploratory behavior, specializing in the learning environment.
%In (A-1), the exponent changed at first in each environment but did not change so much after that.
%This would be because SAL first increased the sensitivity of each hidden neuron at first in the learning environment,
%and then it was not applied when the sensitivity became large enough.
%This suggests that SRL decreased the sensitivity according to the progress of learning.
%In the case of (A-2) in which SAL was not applied,
%the exponent became small once in the early phase of learning, and then gradually became large.
%This and  \citep{Sensitivity} suggest that SAL avoided low sensitivities in hidden neurons
%and accelerated learning as shown in Fig. \ref{fig:LearningCurve_SW}.
%Since the fluctuation is smaller in (A-1) than in (A), it is possible that SRL causes the learning unstable.
%As for the large fluctuation in (A-2), the possibility that the small number in successful simulation runs
%can cause it should be also considered.

Meanwhile, in (B-1) BPTT with SAL, the trend in the exponent change was similar to that of (A),
but the exponent started from a smaller value and stayed close to zero.
This low exponent in the learning environment may have led to stable learning but slow adaptation
after the environmental change.
%This would be because since the conventional RL committed exploration to the external random noise,
%the initial connection weights were set so that the initial exponent was small,
%and learning did not directly control the dynamics.
%The exponent at the end of learning in each learning environment was larger in Dynamic RL than in conventional RL.
%This may cause the instability of learning in Dynamics RL.
%The change of the exponent after the environmental change was not so fast.

\subsection{Slider-Crank control (Dynamic pattern generation) task}
To evaluate whether the proposed Dynamic RL works in a task where a learning agent
is required to generate dynamic patterns,
a slider-crank control task, as shown in Fig.~\ref{fig:Crank}, was employed as the next learning task.
In this task, an agent learns to apply an appropriate time-varying force $f$ in the $x$-direction to the slider.
This force is then transmitted to the rotor through the connecting rod,
causing the rotor to rotate around its central axis O.
\begin{figure}[htb]
\centerline{\includegraphics[scale=0.43]{Crank.pdf}}
\caption{Slider-crank control task as a dynamic pattern generation task and the RNN used.
The agent applies a force $f$ to the slider in the direction of the thick blue arrow
(which can be positive or negative), causing the rotor to rotate.
There are four joint-detecting sensors (s1,...s4), and only the four binary signals from them
are the input to the actor network.
The initial rotor joint angle $\theta_0$ is randomly chosen within the ranges indicated by the two green arcs at the beginning of each episode.
}
\label{fig:Crank}
\end{figure}

The actor network receives only four binary sensor signals: $s_1$ to $s_4$.
Each signal is 1.0 only when the rotor angle $\theta$ passes through a specific angle: $\pi/4, 3\pi/4, -\pi/4$, or $-3\pi/4 $ radian
respectively, regardless of the rotational direction.
The actor network receives the four signals and outputs only one motor command
representing the force $f$ applied to the slider in the $x$-direction.
Therefore, the agent cannot always perceive the rotational angle $\theta$ or the slider's location $x$,
nor can it directly perceive the rotational direction or speed.
The agent has to generate a force sequence
that is not defined as a static function of its four inputs.
The actor output is modified by multiplying it by 1.25 and clipping it between -1.0 and 1.0
to convert it into a motor command (force $f$).
This is expected to prevent the output from entering the saturation range of the neuron's activation function.
The uniform disk-type rotor, with mass, makes this system dynamic
while considering the negligible masses of the slider and link.
The equations of motion for this system are provided in \ref{App:Eq_Crank},

In each episode, the initial rotation angle $\theta_0$ is set randomly
to an angle from $-3\pi/4 <= \theta_0 <= -\pi/4$ or $\pi/4<=\theta_0<=3\pi/4$ radian.
The initial rotational speed is 0.0.
Since the agent does not perceive the joint location directly,
it cannot determine the current rotational direction or identify the necessary direction of force
to rotate the rotor in the rewarded direction until it receives the first ``on'' signal.
%the initial rotor angle is $-3\pi/4 <= \theta <= -\pi/4$ and
Since Dynamic RL was not applied to the critic network,
it was designed to receive $cos \theta$ and $sin \theta$ as sensor inputs,
which are sufficient for identifying the rotor angle, for simplicity.
When the angle passes through $\pi$ with a positive rotational speed, the agent receives a reward of 0.15,
and when with a negative rotational speed, it receives a penalty of -0.15.
In each episode, the agent performed 400 steps of force loading,
and the episode did not end even though the agent received a reward or penalty halfway through.
In one simulation run, 10,000 episodes of learning were conducted.
After 5,000 episodes, the reward sign was reversed as an environmental change.
Tests were performed after 5,000 and 10,000 episodes.
In each test, the rotor started from each of 50 initial angles at intervals of $0.02\pi$,
and the average revolutions per episode were calculated.
Learning is defined as a ``failure'' if the average is less than 20 in the first test or more than -20 in the final test;
otherwise, it is defined as a ``success.''
As well as the previous task, 40 simulation runs were performed for each learning condition.
The parameters used here were almost the same as in the previous task,
but some of them are different as in Table \ref{Table:Parameters2}.\vspace{5mm}

\begin{table}[ht]
  \begin{center}
    \caption{Parameters used in the dynamic pattern generation (slider-crank control) task,
    excluding the same ones as in the previous task.
    %that differ from the previous task
    For the excluded ones, refer to Table \ref{Table:Parameters1}.}
    \vspace{2mm}\small
      \begin{tabular}{c|c} \toprule
	Target Sensitivity for SAL  $s_{th}$ in Fig.~\ref{fig:DynamicRL}& 1.6\\ \midrule
	Rate of Regulation $\eta_{reg}$ in Eq.~(\ref{Eq:Regularize}) & $1e^{-7}$\\ \midrule
	Rate of Raising Critic $\eta_{raise}$ in Eq.~(\ref{Eq:Raise_Critic}) & 0.0002\\ \midrule
	Truncated Number of Steps in BPTT & 10\\ \midrule
      \end{tabular}
    \label{Table:Parameters2}
  \end{center}
\end{table}

\begin{figure}[ht]
\centerline{\includegraphics[scale=0.7]{CompPerf1_CR.png}}
\caption{Comparison of success rate among six learning conditions in Table~\ref{Table:CompSet} in the slider-crank control task.}
\label{fig:CompPerf1_CR}
\end{figure}
Figure \ref{fig:CompPerf1_CR} shows the counts of ``success'' and ``failure'' runs for each learning condition.
As shown, (A) Dynamic RL achieved the highest success rate, while (B-1) conventional RL (BPTT + SAL) had
the second-highest rate.
In contrast, (B-2), which did not use backward error propagation through time, still achieved a high success rate.
In the other cases, including the case (B) where SAL was not applied in conventional RL,
the success rate was notably low.
This indicates that both SRL and SAL are essential in Dynamic RL, and 
a method to ensure sensitivity, like SAL, is also necessary in conventional RL.

\begin{figure}[t]
\centerline{\includegraphics[scale=0.6]{LearningCurve_CR.png}}
\caption{Comparison of changes in the average number of revolutions per episode, calculated over every 100 episodes, 
as learning progresses among three learning conditions in the slider-crank control task.
Each line and shaded area indicate the mean and standard deviation of the number over successful simulation runs, respectively.}
\label{fig:LearningCurve_Crank}
\end{figure}


\begin{figure}[pth]
\center
\centerline{\includegraphics[scale=0.65]{Epi_CR_testD.png}}
\footnotesize{(A) DYN\_RL}
\vspace{3mm}
\centerline{\includegraphics[scale=0.65]{Epi_CR_testB.png}}
\footnotesize{(B-1) Conv\_RL (BPTT+SAL)}
\caption{Comparison of changes in (a) rotor angle with four sensor signals and a reinforcement signal
(reward or penalty), (b) actor output, and
(c) outputs of three sample upper hidden neurons during a test episode after 5,000 episodes of learning
between (A) Dynamic RL and (B-1) conventional RL (with BPTT and SAL) in the slider-crank control task.
The initial rotor angle was different between (A) and (B-1) because these show the cases
where the agent first rotated the rotor in the opposite direction to the rewarded one.}
%and then changed it after receiving the first `on' signal.}
\label{fig:Epi_CR_before}
\end{figure}

\begin{figure}[pth]
\center
\centerline{\includegraphics[scale=0.65]{Epi_CR_5100D.png}}
\footnotesize{(A-I) DYN\_RL (after 5100 episodes)}
\vspace{5mm}
%\caption{Changes in the angle and some outputs of neurons in the 5100th episode in dynamic RL.}
\centerline{\includegraphics[scale=0.65]{Epi_CR_test_finalD.png}}
\footnotesize{(A-II) DYN\_RL after learning}
\caption{Changes in (a) rotor angle with four sensor signals and a reinforcement signal (reward or penalty),
(b) actor output, and (c) outputs of three sample upper hidden neurons
at two phases after the environmental change in the cases of (A) Dynamic RL.
(A-I) represents the state soon after the environmental change, and (A-II) reflects the state after all learning had finished.}
\label{fig:Epi_CR_after}
\end{figure}

Figure \ref{fig:LearningCurve_Crank} shows the change in the number of revolutions per episode
% in the direction of reward per episode
for the three cases with a high success rate, as shown in Fig.~\ref{fig:CompPerf1_CR}.
When the proposed Dynamic RL (DYN\_RL) was applied,
the rotational speed gradually increased, and the average number of revolutions per episode reached approximately 30.
Learning progressed slightly faster compared to the case of conventional RL using BPTT,
despite not needing backward computation through time.
The ratio of the actual computational cost for learning the actor network,
compared to conventional RL (BPTT+SAL), was 1:4.9.
The difference from the previous task is mainly
due to the difference in the truncated time steps in BPTT: 10 in the sequential navigation task and 5 in this task.

Shortly after the rotational direction for the reward was reversed as the environmental change after 5,000 episodes,
% if the agent behaved the same as before the environmental change, the sign was reversed
% while maintaining the absolute value is the same.
the number of revolutions decreased, and the average over the 5,100-5,200 episodes was already negative.
As in the previous task, the learning speed immediately after the environmental change was considerably faster
than in the case of conventional RL.
In conventional RL with only BP, the success rate was reasonable,
but the learning speed and performance after learning were much lower compared to the case with BPTT.

Figure \ref{fig:Epi_CR_before} illustrates the changes in several states during a sample test episode just before the environmental change
for both cases: (A) Dynamic RL and (B-1) conventional RL using BPTT and SAL.
Specifically, the states include (a) rotor angle with sensor signals and reinforcement signal, (b) actor output,
and (c) three sample neuron outputs from the upper hidden layer.
%In this task, the agent could not know the direction of rotation until either sensor signal became on.
The figure depicts the case in which the initial rotational speed caused by the agent's force load was negative,
i.e., opposite to the rewarded direction, for one of the two initial angles: $-\pi/2$ or $\pi/2$.
Therefore, the agent had to change its direction after detecting an ``on'' sensor signal.
From (a), it can be observed that the agent changed the rotational direction in both cases
and then maintained it thereafter.
After the change, the outputs of the actor and hidden neurons changed almost periodically, as shown in (b) and (c).
In the case of Dynamic RL, the outputs changed more irregularly than in conventional RL.
%that would be because in the Dynamic RL, the outputs included irregular factors due to the chaotic dynamics.
%On the other hand, since this was a test phase, irregular factors seem smaller in the case of conventional RL, 
This observation was made during the test phase; however during learning,
external random noise for exploration was added
to the actor outputs in the case of conventional RL.
Another difference is that the hidden neuron outputs in Dynamic RL have a larger variation range,
though the underlying reasons or influences have yet to be investigated.


Figure~\ref{fig:Epi_CR_after} shows the changes in states in two phases after the environmental change
in the case of Dynamic RL.
Figure~\ref{fig:Epi_CR_after}(A-I) shows the state just after 5,100 episodes, shortly after the environmental change,
while Fig.~\ref{fig:Epi_CR_after}(A-II) depicts the state in the test phase after all learning was completed.
In both cases, the initial rotor angle was the same as in Fig.~\ref{fig:Epi_CR_before}(A).
In Fig.~\ref{fig:Epi_CR_after}(A-I), the actor output and hidden outputs changed irregularly, causing the rotor to rotate
both clockwise and counter-clockwise irregularly.
In contrast, in Fig.~\ref{fig:Epi_CR_after}(A-II) where learning had progressed, the rotor rotated in the opposite direction
compared to Fig.~\ref{fig:Epi_CR_before}(A), and the actor output, along with the hidden outputs, changed
almost periodically sometime after the start of the episode.
However, the periodic pattern of the upper hidden neurons in Fig.~\ref{fig:Epi_CR_after}(A-II) differs
from that in Fig.~\ref{fig:Epi_CR_before}(A),
%This means that the fast learning after the environmental change in Dynamic RL
%was not because the internal periodic pattern did not change.
%revealing no common features in the internal periodic patterns
%before and after the environmental change.
making it hard to explain the rapid adaptation after the environmental change in Dynamic RL
based on the retention of internal periodic patterns.

\section{Discussion and conclusion}\label{Sec:Discussion}
In this paper, the author proposed Dynamic Reinforcement Learning (Dynamic RL), a new RL framework,
expecting it to be a fundamental technique for the emergence of `thinking.'
This paper could not clearly show its critical advantages in this regard. 
%This RL differs significantly from conventional RL in terms of exploration,
%which is performed as chaotic dynamics produced through the process of generating continuous motor commands
%rather than stochastic action selection separately from action generation.
%as the first step towards its emergence,
However, the author demonstrated that despite the major shift in learning from static to dynamic,
this unconventional RL approach effectively functions as RL in two simple dynamic tasks.
Learning recurrent connection weights among hidden neurons without error backpropagation
improved learning performance, which was not achievable with previous chaos-based RL \citep{IJCNN2015}.
The autonomous change in the state transitions from ``more irregular'' to ``more rational'' through learning observed in the simulations
%He has also shown that through the direct learning of the system dynamics that generate motions in which explorations are inherent,
%the proposed RL changes the dynamics autonomously from more irregular to more rational while maintaining chaotic dynamics.
appears to have solved the balance between exploration and exploitation \citep{Sutton1998}
and suggests the potential for the emergence of thinking.
The learning characteristics of Dynamic RL, as observed in the simulation results of the two tasks,
are summarized first, followed by a discussion of its possibilities and challenges from broader perspectives.
% towards the emergence of thinking are discussed.
Finally, the risks and concerns of Dynamic RL are discussed.

\subsection{Summary as one method of Reinforcement Learning}
Dynamic RL allows for significantly lower computational costs per step than conventional RL
due to the absence of backward computation for learning, particularly along the time axis.
Nevertheless, the success rate was almost the same as that of (A) and (B-1)
in both Figs.~\ref{fig:CompPerf1_SW} and \ref{fig:CompPerf1_CR},
and the learning curve with episodes as the horizontal axis was also comparable to that of conventional RL
using BPTT and SAL as (A) and (B-1)
in both Figs.~\ref{fig:LearningCurve_SW} and \ref{fig:LearningCurve_Crank}, regardless of the learning task.
Furthermore, as for adapting to new environments, conversely, Dynamic RL was considerably faster.
The agent's behaviors became more exploratory immediately after entering a new environment
and less exploratory as learning progressed thereafter, as shown in Fig.~\ref{fig:Task1_Loci}(A).
This was also reflected in the exploration exponent,
a type of Lyapunov exponent for the entire system, including the external world,
as shown in Fig.~\ref{fig:ExplExponent}(A) in the sequential navigation task.
In the slider-crank control task, it was reflected in the internal state change of the RNN,
as shown in Figs.~\ref{fig:Epi_CR_before} and \ref{fig:Epi_CR_after}.
%This could be because Dynamic RL could effectively use what it had learned before
%and/or the hidden neurons could learn directly without using back-propagated errors,
%but further studies are needed.

Comprehensively considering the results of varying the application of
SAL (Sensitivity Adjustment Learning) or SRL (Sensitivity-Controlled RL),
the author suggests that the following mechanisms are functioning.
If neurons are linear, larger weights generally increase sensitivity.
However, due to the saturation property of each non-linear neuron,
excessively large weights often reduce sensitivity conversely
by decreasing $f'(U)$ in Eq.~(\ref{Eq:sensitivity}).
This generally decreases the Lyapunov exponent of the system and also the learning activities.
Such problems occur when the initial weights are very large
or often when another learning method (SRL in this case) progresses,
as shown in the early phase of Fig.~\ref{fig:ExplExponent}(A-2).
SAL maintains the sensitivity of each neuron,
thereby preserving the exploration exponent positive (chaotic) in the learning environment,
as shown in Fig.\ref{fig:ExplExponent}(A).
This prevents excessive convergence of the system and a decline in learning activity.
Therefore, the absence of SAL leads to a decrease in both the learning success rate and learning speed, as shown in
the difference between (A) and (A-2) in Figs.~\ref{fig:CompPerf1_SW}, \ref{fig:LearningCurve_SW}, 
and \ref{fig:CompPerf1_CR}.
This is the case not only for Dynamic RL but also for BPTT, as shown in the difference between (B) and (B-1)
in Figs.~\ref{fig:CompPerf1_SW} and \ref{fig:CompPerf1_CR},
which is compatible with the results in the previous work \citep{Sensitivity}.

As learning progresses, while avoiding over-convergence by using SAL,
SRL adjusts the dynamics around the state transitions with positive TD-errors to be more convergent;
thus, the state transitions become more reproducible.
Conversely, it adjusts the dynamics around the state transitions with negative TD-errors to be more exploratory.
Since the irregularity sometimes increases and sometimes decreases depending on the TD-errors,
it seems that the irregularity remains unchanged in total.
However, actually, since the frequency of better state transitions increases,
%for state transitions where irregularity decreases, reproducibility increases,
%while for those where irregularity increases, reproducibility decreases.
the irregularity around actual state transitions decreases,
as the exploration exponent in Fig.~\ref{fig:ExplExponent}(A).
%This function of SRL clearly appears in the differences in the change of the exploration exponent between
%Figs.~\ref{fig:ExplExponent}(A) and (A-1).
Moreover, the state transitions that are adjusted to be more reproducible had positive TD-errors.
Therefore, SRL makes the state transitions not only ``less irregular'' but also ``more rational.''

%SAL always prevents over-convergence and maintains chaos in the current learning environment.
%That ensures that an agent continues to explore greater or less and remains flexible.
%Furthermore, in a new environment, since the TD-error is negative more often,
%SRL also makes the current system dynamics more chaotic.
After the environment changed during learning, the agent rapidly resumed exploration,
and its performance quickly recovered in Dynamic RL. 
In the navigation task, the dynamics immediately before entering the new environment were neither chaotic nor exploratory
for the new environment Env2,
as the exploration exponent for Env2 was slightly below zero at 20,000 episodes in Fig.~\ref{fig:ExplExponent}(A).
Observing the agent's behavior from two slightly separated initial positions, 
the agent moved to the same corner of the field and remained trapped there in most cases.
When the agent moved toward or was trapped in the corner, 
%its behavior was worse than expected, and
the TD-error was negative very frequently.
In the slider-crank control task, due to the reversal of the reward's sign, large negative TD-errors frequently occurred. 
Accordingly, in both cases, it can be argued that SRL restored the strong chaotic nature of the system dynamics
with the help of SAL and resumed exploratory behaviors,
%it can be understood that the rapidly restoring chaotic nature of the system dynamics
%and resuming exploration behaviors by SRL and SAL shortly after the environmental change,
resulting in the quick recovery of performance.

%In the exploration generated through the actor network in Dynamic RL, randomness decreased autonomously
%as learning progressed.
%That was observed in both the agent behaviors and the exploration exponent.
%The exponent is a kind of Lyapunov exponent for the whole system dynamics
%including the network and physical states and a large exponent means that the agent's behavior is more explorative.
%After the environment changed during learning, the exponent for the new learning environment increased very quickly,
%and after that, it decreased as the learning progressed.
%This rapid increase in the exploratory exponent may have led to the agent promptly resuming exploration
%and adapting quickly to the new environment as seen in the learning curve.
%It can be said that Dynamic RL autonomously solved the ``exploration and exploitation dilemma''.
%
%Just after the environment changed, the exploration exponent for the new learning environment Env2 was smaller
%than for the old environment Env1 at first, and then the relationship between the larger and smaller values is reversed soon.
%In a non-learning environment, the agent cannot move well even after learning has already progressed,
%and so collides with a wall, and often went a corner of the field.
%If the agent went to a corner during the computation of the exploration exponent,
%two agent paths starting a little apart become the same and the the network states become closer.
%That is one reason for the small exponent in the non-learning environment.
%
%From the learning curve and exploration exponent when SRL (Sensitivity-based RL)  was not applied to the hidden neurons
%and those when SAL (Sensitivity Adjustment Learning) was not applied, 
%the composition appears to be that SRL makes rational state transitions more convergent
%while SAL prevents the sensitivity in each neuron from decreasing in the learning environment.
%Furthermore, both SRL and SAL seem to achieve the quick increase in the exploration exponent after the environmental change.

For agents, the dynamics are shaped not only by the RNN
but by the entire system integrated with the outside world, including the physical systems of the agents and their environment.
Nevertheless, the gap is interesting where only adjusting or controlling the sensitivity of individual neurons locally
by updating their weights and bias can globally control the system dynamics,
thereby regulating the system and enabling the agent to respond effectively to the external world.
In this paper, the sensitivity targets for SAL are set as appropriate values
at 1.3 for the navigation task and 1.6 for the slider-crank control task.
A value of 1.0 was appropriate for the sensitivity target in the preceding study
investigating SAL in supervised learning with BP or BPTT \citep{Sensitivity}.
In reward-modulated Hebbian learning using chaotic dynamics, 
performance was optimized near the ``edge of chaos'' \citep{NN_Matsuki}.
The author is considering the possibility that a large value may be suitable for RL's exploration, which requires stronger chaoticity.
%However, further research is needed on the sensitivity target,
%considering its relation to thinking that requires autonomous state transitions,
%including unexpected ones in inspiration or discovery.

For massively parallel and flexible systems, asynchronous pulse-and-analog coexisting systems would be far superior
to a synchronous digital system.
The introduction of Dynamic RL into brain models or brain-like hardware
based on such systems offers significant advantages.
Local learning without backward computation significantly reduces communication between neurons,
as well as the computational cost in each neuron, compared to conventional RL using BPTT.
In particular, eliminating backward computation through time is highly beneficial,
as it removes the need for time management and memory to store past states.
Furthermore, a separate random number generator for stochastic selection is now unnecessary.
A concern remains on how to build an RNN with connections that initially have large, random weights
to make the system dynamics chaotic.
The author expects SAL will enlarge small, irregular initial weights,
as in the case of supervised learning in \citep{Sensitivity}.
The introduction of chaotic neurons \citep{Aihara} would help it.
In this paper, Dynamic RL is built as a discrete-time model to compute on a digital computer,
but it would not be difficult to implement it on a continuous-time system.

However, even though we see Dynamic RL as regular RL, there are still some problems and concerns.
First, learning was more unstable than conventional RL, showing occasional performance drops during learning.
%and sometimes the learning performance dropped suddenly during learning.
This resulted in larger deviations in learning performance in Dynamic RL than in conventional RL,
as shown in Fig.~\ref{fig:CompPerf}.
%In Dynamic RL, weights and biases are updated to control the convergence or divergence of the flows
%around actual state transitions, but at the same time, by this update,
%not only the neighbors but also the transitions themselves are changed together even when TD-error is positive.
%This may cause the observed instability in learning.
%, and its influence needs to be investigated.
%As for these two problems, because the exploration is inherence in the action
%and also from the view of life-long learning, it could be considered unavoidable to some extent.
% However, if humans are learning similarly, there should be more room for improvement.

The second issue is that the actor outputs are often saturated, making it difficult to fine-tune them.
This can be seen, for example, in the agent's tendency to move in the 45-degree diagonal direction in Fig.~\ref{fig:Task1_Loci}(A)(b,d)
compared to the conventional RL case in Fig.~\ref{fig:Task1_Loci}(B)(b,d),
though they were improved to some extent by regulation learning as in Eq.~(\ref{Eq:Regularize}).
%A more fundamental solution is needed.

The third issue is that, in Dynamic RL, the exploration factor cannot be separated from the actor output.
Therefore, as shown in Fig.~\ref{fig:CompPerf}, the performance in the test is worse than in conventional RL,
where the actor outputs can be used without adding exploration noises.
 %Some solutions could narrow this gap, which would be, however, somewhat inherently inevitable.
The fourth is that there are many parameters for learning, and adjusting them is not easy.
In this sense, the author's computational power (iMac 2020, Intel Core i9 3.6GHz) did not allow systematic
and sufficient optimization in this study.

The author currently has several major challenges in mind, along with ideas for solving some of them,
regarding this research.
However, due to the risk discussed in subsection \ref{subsec:Risk},
he would rather refrain from writing about that.
%The main remaining challenges are as follows.
%Herein, the critic was learned by the conventional RL using BPTT.
%However, we will be able to say that Dynamics RL has been established
%only when critic learning is performed within the framework of Dynamic RL.
%To achieve this, further study is required,
%including the introduction of a novel architecture like TD3 \citep{TD3, TD3_Matsuki}.
%Another challenge is the acquisition of multistep state transitions through RL.
%It was demonstrated that Dynamic RL can learn past processes without going back through time,
%but there must be a limitation to that.
%This would need an additional technique; local memories, such as causality traces \citep{CausalityTraces},
%would be the key to solving it.
%%The network architecture should also be considered, including such as TD3 \citep{TD3,TD3_Matsuki}.
%The Introduction states that Transformers surpass traditional AI.
%They significantly differ from the proposed Dynamic RL and conventional RL.
%%but demonstrates that it has much to do with `thinking.'
%However, Transformers are essential for achieving `thinking' due to their capabilities in generative AI. 
%Further exploration of the two and also their integration is needed.

\subsection{Towards the future from a higher perspective}\label{subsec:Future}
Even from the broader perspective of neural network (NN) learning not limited to RL,
the Dynamic Learning approach used in Dynamic RL is still quite different in quality from conventional approaches.
The state vector of an NN represents a point in its internal state space.
In general NN learning, a gradient-based method or a derivative of it determines
how to move this point in space based on some value or cost function, such as an error function,
whose inputs are the NN outputs.
Then, how to change the weights and biases is determined.
Learning considers only the current point and usually does not consider its temporal changes.
BPTT applies the same method to past states or points by extending the time axis as a dimension of space
but still does not focus on their temporal changes or the system dynamics.
While RNNs have enabled dynamic processing, their learning methods still remain stuck in static approaches.
%Most conventional NN learning, which started with learning a static mapping from inputs to outputs
%with no time axis, focuses on the point and aims to move it in an appropriate direction
%based on an error function or some other static function of the outputs.
%The gradient method helps to determine the update of all the weights and biases.
%RNNs have enabled dynamic processing along the time axis, and BPTT has enabled learning of it.
%However, the aim is still to reduce some error function which does not take into account its temporal changes,
%and to move the point in the output space at the current step

However, since a state changes continuously over time, a point forms a line,
and changes in the neighborhood around that line form flows in the state space.
The point representing the current state is merely a cross-section for convenience
of what is essentially a continuous line and generally cannot be separated from the system's dynamics.
%unlike spatial information, time proceeds constantly in one direction.
In control systems, for instance,
%a controller cannot achieve good control by only using the current error from the target trajectory.
system stability is discussed based on convergence or divergence around the equilibrium point,
and for better control, the controller often uses the time derivative of the error from the target trajectory.

Dynamic Learning transforms neural network learning from ``learning of points'' to ``learning of flow,'' in other words,
from ``static'' to ``dynamic.''
Unlike control problems, where a target trajectory is specified, and stability is absolutely required,
%in RL, an agent does not have target trajectories.
an agent's behavior should be exploratory initially and then become less exploratory over time,
and the system dynamics must not fully converge.
Thus, as a concrete method for dynamic learning,
the author arrived at the idea of controlling the convergence or divergence of the flow directly
around the current transition, rather than moving points.
He also arrived at the idea of controlling global dynamics through controlling dynamics locally in both space and time.
The author believes he could demonstrate that these ideas work appropriately, at least in RL.
%The agent explores based on the chaotic system dynamics,
%and the chaoticity is controlled according to being better or worse of the current state transition.
%Many partifal differential equations in which the time axis is not symmetrical to any spatial axis
%are useful in expressing the relationship between space and time.
%Thus, time is not merely one of the axes in space, but should be taken as a special one.
%Then Dynamic Learning tries to control the convergence or divergence of the flow according to the gradient method.
%The difference appears in Fig.~\ref{fig:ConvRL} and Fig.~\ref{fig:DYN_RL}.
%In Fig.~\ref{fig:ConvRL}, the aim is to move the output vector at the moment, but in Fig.~\ref{fig:DYN_RL},
%the aim is to control the convergence or divergence of the neighborhood flow in the network state space over time.
%as weights and biases are updated to control the convergence or divergence of flows,
%the states and outputs themselves also change.
%Even when the neighborhood along a state transition becomes convergent through learning with a positive TD-error,
%the original state transition itself may change. 
%In a network with a large number of neurons, it is known that any ideal function exists in the neighborhood
%of any sufficiently wide random network, this new learning works better
%though the author cannot examine due to his poor computational resources.
%The author thinks that it can easily be extended to continuous time.
%It can also be extended to supervised learning by subtracting the temporal average of the error function from itself.

This shift in neural network learning is fundamental.
% and has the potential to be applied more widely.
However, using ideal divergence/convergence as supervised signals for learning dynamics does not seem very practical.
Though it may be possible to compute and use ideal dynamics derived from ideal outputs, which are given as training signals,
it remains unclear how much advantage this approach would have over direct learning with conventional static-type supervised learning.
It may be mainly used in RL at last.
However, in large-scale NNs, thanks to the magic of high-dimensional geometry,
appropriate parameters (weights and biases) for a given cost function do not lie so far from their initial values \citep{Amari}.
The author speculates that in large-scale RNNs, ``learning of flow'' along the time axis may become more significant and effective
than ``learning of point'' on a time cross-section,
though the scale of the RNN used in this study was not large enough to experience these benefits.

As noted in the Introduction, Dynamic RL brings about a big transformation to exploration.
Exploration goes from stochastic to deterministic and no longer requires a random number generator.
In Dynamic RL, exploration is not performed at the level of actions or final outputs
but at the level of the entire process within the agent's RNN.
This exploration is expected to be considerably more flexible because it is learnable and also has many DOFs
rather than relying on stochastic selection with a scalar parameter called temperature.
The excellent adaptability to a new environment, as seen in the simulation results,
may be considered an advantage of such high-performance exploration.
Furthermore, the author finally positioned exploration as a preliminary step to grow into `thinking.'

\begin{figure}[t]
\centerline{\includegraphics[scale=0.26]{LearningDynamics.pdf}}
\caption{A conceptual diagram explaining how Dynamic RL works in simplified 2-D state space at the expense of exactness.
The reproducibility of better state transitions increases, and
novelty with consistency is expected to emerge as the red line after learning.}
\label{fig:LearningDynamics}
\end{figure}
Here, let us consider how Dynamic RL with such exploration works in a simplified two-dimensional state space as an example,
using a conceptual diagram, as shown in Fig.~\ref{fig:LearningDynamics}.
Before learning, dynamics are set to be highly chaotic, and state transitions are overall explorative.
Dynamics are learned around the actual paths depending on the TD-error at each state.
Assuming that there are regions in which learning is easy and a region in which learning is not easy in the state space.
In the former cases, by making the flows around rational state transitions with a positive TD-error converge,
the flows gather around such rational state transitions.
In the latter cases, the flows are still more explorative.
If it occurs in a high-dimensional space, pseudo-attractors would emerge.
Thus, as shown in Fig.~\ref{fig:LearningDynamics}(b),
a rational state transition, indicated by a thick black line with an arrow at the end, is achieved.
Furthermore, due to the explorative transitions in the region where learning is difficult,
sometimes another solution emerges.
However, the author expects the transitions not to become incoherent
because the state transitions are rational in the regions where certainty in learning is high.

%Dynamic RL brings a qualitative transformation to the exploration as well.
%Conventional exploration is performed
%as a stochastic selection after all the processes for the action or motion generation at each time step.
%The exploration here is not a stochastic process, but a deterministic process utilizing chaotic dynamics of the system.
%It is performed throughout the process of the agent.
%It is not static, focusing only on a specific point in time, but dynamic along the time axis.
%In the Introduction, the author touched that the high DOF exploration in Dynamic RL
%can balance the persistence and flexibility and promote to step outside of common sense.
%The author did not check the big merit here, but the excellent adaptation speed in Dynamic RL
%may be a fruit of such exploration.

\begin{figure}[t]
\centerline{\includegraphics[scale=0.36]{ThinkingExplLoop.pdf}}
\caption{`Exploration' and `thinking' loop and its relationship to Dynamic RL, the author advocates.
This is an updated version of Fig.~\ref{fig:Thinking} and was obtained through the analysis of the simulation results.
SAL maintains chaotic dynamics of the system,
thereby preventing a decline in learning activities.
Under this condition, SRL changes the agent's processing from ``more irregular'' to ``more rational'' gradually;
in other words, it moves the processing from `exploration' to `thinking'.
When situations are changed, the state transitions become less rational for the agent,
and negative TD-errors appear more frequently or with larger magnitudes.
(labeled this as ``confusion'')
Then SRL makes the state transitions ``more irregular'' rapidly.
}
\label{fig:ThinkingExplLoop}
\end{figure}
Next, the author summarizes how he thinks about `exploration', `thinking,' and their relationship to Dynamic RL,
referring to Fig.~\ref{fig:ThinkingExplLoop}.
As mentioned, since both `exploration' and `thinking' need autonomous state transitions
without being stuck in a convergent state,
the system dynamics, including both the agent and its environment, should be chaotic.
It ensures the agent's activity in both behaviors and learning.
SAL plays this role (right-pointing gray block arrows in Fig.~\ref{fig:ThinkingExplLoop}).
The agent explores at first with strong chaotic system dynamics.
SRL enhances the reproducibility of the autonomous state transitions that are evaluated as better by the critic,
and thus, the transitions and also the agent behaviors become more rational (diagonal block arrow).
In this way, exploration grows into thinking, which is autonomous and more rational state transitions
with weak chaotic system dynamics.
This learning is slow because it takes time for the agent to explore rational state transitions,
in other words, to acquire the knowledge needed for getting rewards and avoiding punishments.

When the situation changes, such as the environmental change in the learning tasks in this paper,
the frequency or magnitude of worse evaluations, i.e., negative TD-errors, increases.
This means that the state transitions become less rational  (downward pointing block arrow).
Then, SRL resumes exploration autonomously by making the system dynamics more chaotic
(right-pointing large block arrow).
This learning progresses rapidly because it only requires increasing the sensitivity of each neuron.
The relationship between exploration and exploitation is quite similar to that between exploration and thinking.
Exploitation and thinking are similar in that both are rational.
Therefore, this mechanism can also explain how Dynamic RL resolves the ``exploration and exploitation'' dilemma
and balances them depending on the situation.

%The author believes that `thinking' requires multistep state transitions that are rational,
%but sometimes unexpected state transitions like inspiration or discovery.
%He believes that such transitions occur not in convergent dynamics
%but in chaotic dynamics, such as pseudo-attractors formed around rational state transitions.
%From this perspective, the author's idea that
%``Exploration grows into thinking through learning'' is crucial
%and may open the path to the emergence of thinking, which conventional RL cannot achieve.
%However, even if Dynamic RL becomes a fundamental technique,
%the emergence of thinking would not be easily achieved because many challenges remain.
%As one of them, he thinks it is important to show
%that an agent acquires rational multi-step state transitions through RL as mentioned.

%As mentioned, in conventional RL, exploration is performed only at the level of the final output of a learning agent,
%and the DOFs of the exploration space or discovered things would be limited
%to those of action or motion.
%On the other hand, in Dynamic RL, since exploration is performed throughout the process from sensors to motors,
%the DOFs of the exploration would be incomparably greater,
%and the ability of discovery must be much greater than in the case of conventional RL.
%With weak chaotic dynamics, such as pseudo-attractors formed around rational state transitions,
%an agent is expected to have inspiration and make discoveries while thinking.

%\noindent
%*****************\\
%Dynamic RL made a major shift in reinforcement learning.
%However, today's Transform-based generative AIs are capable of great responses, as if they were thinking.
%Given this, the author believes that it is necessary to introduce a Transformer or an alternative mechanism
%within the framework of Dynamic RL in order to realize thinking.\\
%*****************


The brain, a massively parallel system, is too interconnected and complex for us to understand,
and we cannot easily divide its processing into modules.
For many years, the author has believed that our functions, such as recognition and prediction, are
merely convenient labels to understand the brain's complicated parallel processing by dividing it into parts \citep{Intech}.
For instance, processing performed close to sensors is commonly called ``recognition.''
We recognize an image while we think, ``Is this inside a house?''
``Is it daytime or nighttime?'' or ``What kind of people live there?'' drawing on our knowledge.
These thoughts occur during image recognition, 
making it difficult to classify them clearly as either thinking or recognition.
%and cannot be separated from recognition.
%In this way, recognition and thinking would be deeply connected and are not what can be divided clearly.

For ease of understanding, this paper has discussed exploration and thinking
as if they exist alongside other functions.
However, considering the actual human processing as described,
these should not be labels for any part of the overall process.
Instead, they may represent a tendency of our processing,
including recognition and motion generation, to be either more irregular or more rational.
In the presented simulations, sensor signals were used as input to the actor network without any preprocessing.
Hence, all processes from sensors to motors are performed in the network,
even though it is referred to as an ``actor.''
Learning was applied to the entire network, not just a part of it,
and exploration was generated throughout the system, including the external world.
This is consistent with the findings that chaotic attractors for existing odorants and chaos for new ones were observed
in the olfactory bulb \citep{Skarda, Freeman} even though it does not seem to be directly related to motion generation.
The relationship between Dynamic RL and the ``default mode network'' \citep{DMN} would also be interesting
in terms of thinking.

Supervised learning has an excellent ability to generate reasonable answers
through learning a vast amount of examples but originally does not have the ability to discover something new by itself.
RL receives no direct training signal; instead, it has the ability to discover new things through exploration.
Exploration in Dynamic RL is performed flexibly throughout the agent's RNN with massive DOFs.
Such exploration grows into `thinking' through learning in harmony with novelty and rationality, maintaining the massive DOFs.
Accordingly, the author believes that thinking that emerges in this way would make it possible to step outside of common sense
without becoming incoherent, which is difficult for current Transformer-based LLMs to achieve,
as introduced in the Introduction.

However, thinking seems to be a conscious process that is more sequential than parallel.
This may simply suggest that we can identify only the conscious part of the processes, which are just the tip of the iceberg.
Visible language-level processing is exactly what current LLMs excel at, as shown in the Introduction.
Given the LLMs' outstanding abilities, the author believes that integrating Dynamic RL with an LLM is essential to realize `thinking.'
%However, there still remain many obstacles to overcome to realize the "emergence of thinking." 
 %However, we should further explore what this implies.
%The author would rather refrain from writing here about what we should do
%due to the risk discussed in the following.

%When the weights are updated considering past processing,
%in the gradient method such as BPTT, the gradients of the cost function with respect to the past states
%are computed.
%Then the weights are updated according to the gradient descent to decrease the cost function
%by moving the state to the opposite of the gradient direction.
%The state of $N$ neurons is represented as a point in $N$-dimensional space.
%However, the point moves over time and forms a line.
%The author has thought that breaking away from learning to move points at a specific moment
%in an appropriate direction,
%but instead the flow, which includes the neighborhood around the line, should be learned.
%Furthermore, the difference between exploration and thinking
%is irregularity and rationality in state transitions.
%The author thinks that the transition between irregularity and rationality can be controlled
%by controlling the divergence or convergence of the dynamics.
%The dynamics can be controlled by controlling the sensitivity in each neuron \citep{Sensitivity}.
%Then the above items \ref{Item:LearnDyn} and \ref{Item:UseSen} are introduced.


%The author anticipates this learning will be a fundamental technique for the emergence of higher functions
%that requires autonomous state transitions, such as thinking, and believes
%that this learning should not be studied any further for the sake of humanity.
\subsection{Risks and Concerns}\label{subsec:Risk}
Lastly, the author expresses concerns about the risks associated with this research.
%As mentioned, the author thinks that the frame of this research has the possibility to open the way to the emergence of thinking
%including inspiration or discovery.
%If artificial creatures begin to think autonomously and discover ideas beyond human imagination, they could pose a grave risk to humanity.
%
%However, if the author's concern were to come to pass, 
%
%AI weapons such as LAWS (Lethal Autonomous Weapons System) are now in actual use on the battlefield.
%https://docs-library.unoda.org/Convention_on_Certain_Conventional_Weapons_-Group_of_Governmental_Experts_on_Lethal_Autonomous_Weapons_Systems_(2023)/CCW_GGE1_2023_CRP.1_0.pdf
The risks of AI have been frequently pointed out \citep{Signatories, Doya, AI_Risk}.
%, especially AGI (Artificial General Intelligence), 
The general consensus on science and technology that we should benefit from them while avoiding their risks
has been widely accepted also in the AI community \citep{Signatories, Doya}.
%it is widely accepted in the AI community that
%``it is important to research how to reap its benefits while avoiding potential pitfalls'' 
%However, with the advent of generative AI, many ones feel the risk more real, and the word ``catastrophic risk" has been often seen. 
%Among them, some mention the responsibility of researchers,
%but the author is not aware of anyone other than YUdkowsky who has mentioned the stop of the research itself.
%Generally, researchers are held accountable for their own research,
%and people are encouraged to discuss establishing regulations on its use.
%but generally, the greatest responsibility lies with the users, not the researchers.
However, if AI entities begin to think autonomously and discover ideas beyond human comprehension,
%they could pose a grave risk to humanity.   % life-or-death
they could threaten humanity.
%the potential dangers of autonomous thinking may exceed our expectations.
%Currently,
%The author has no solution to ensure that we can control and avoid such risks.
In this context, it is groundbreaking that the prioritization of risk aversion over profit-taking is clearly stated in \citep{AI_Risk1} as follows:
``The potential benefits of AI could justify the risks if the risks were negligible.
However, the chance of existential risk from AI is too high for it to be prudent to rapidly develop AI.
Since extinction is forever, a far more cautious approach is required.''
As mentioned, the author believes that Dynamic RL's potential to discover new things would be significantly higher
than the other AI technologies and should be especially considered. 

%Of course, the author understands that the level of this research is not at all up to the level of concern at this moment and that
Of course, given the current level of this research, these concerns must seem presumptuous or alarmist.
%the author acknowledges.
%Technological advancement, such as the performance of semiconductors, often follows an exponential curve \citep{Kurzweil}.
However, once AI entities reach a level where they can design more intelligent versions of themselves
without any human assistance, their growth will surpass exponential rates,
unlike human-developed technologies such as semiconductors \citep{Kurzweil}.
It will accelerate far beyond our predictions.
Above all, they will get out of control and evolve rapidly even if we stop the research,
leaving us with no means to avoid or halt them.
%However, once AI can think by itself with its own will without human intervention, how may humans stop it?
%it will not only make its own decisions but also design itself.

The author begs people to seriously imagine the horror of autonomous, superintelligent drones flying around
and humanoids walking around and targeting humans as a potential reality. 
He has no idea how we can absolutely ensure
that superintelligent drones or humanoids will never muzzle humanity.
%The author cannot understand the rationale of a foolproof scenario
%in which superintelligent drones would never turn against humanity.
%Once AI reaches a level where it can design higher levels of intelligence without any external help,
%its growth will accelerate exponentially, far beyond our imagination when we realize that.
%Then, humans may have no means to stop it, leaving them with no option but to fight.
%Then, humans may have no means to stop it, except for fighting and extinguishing superintelligence.
%we will no longer be able to stop it.
%The author guesses this would be what A.~Turing wanted to say in 1950 \citep{Doya}.
% presumptuous 厚かましい,僭越な,おこがましい
% seems conceited 自惚れた
% sheer arrogance 傲慢
Given the gravity of the matter and the fact that redoing it is not an option, 
easy wishful thinking without a thorough understanding should definitely be avoided \citep{Yudkowsky}.
%The author does not have a clear vision of its future development,
%With advancements in science and technology allowing us to live satisfactorily,
Science and technology have made remarkable progress.
Now, they have reached a point where we can lead satisfactory lives
although challenges remain in other aspects, such as wars and economic disparities.
% by utilizing them, thanks to our predecessors for their accumulated efforts,
%What kind of benefit on earth is worth putting humanity's safety at a fatal risk?
What kind of benefit on earth justifies putting humanity's safety at such a fatal risk?
We must seriously consider this.
% In today's society, which is already very convenient,

History shows that 
%once the potential of technology is unveiled, it spreads rapidly. Moreover, 
once nations begin developing lethal weapons using a new powerful technology,
it becomes extremely difficult to eliminate them in power games, as in the case of nuclear weapons.
%The imagination and wisdom of humanity are being tested.
However, the author would like to strongly warn that
%, unlike the case of nuclear weapons,
if one nation attempts to use this technique to conquer another, 
it is likely that the world, including the nation itself, will eventually be defeated by the technique.

The researchers developing technologies can usually be the first to recognize their potential risks
by using their imagination about what will happen.
Given the severity of the potential consequences and the importance of early action,
they should not throw the decision solely to society but must take responsible measures themselves first.
The author may have neither the right nor the power to halt the research.
Submitting this research, despite being aware of its potentially fatal risks, represents a major contradiction.
%It is a major contradiction to the submission of this research, which he believes can pose fatal risks.
Some may also argue that such discussions should be excluded from a technical paper.
However, even without this submission, someone else will eventually conduct this research.
The author believes that we have entered an era
where researchers must take a certain amount of responsibility for their research as soon as they publish a paper on it.
Therefore, in order to raise the issue and spark discussions that harness the collective wisdom of humankind,
the author takes the bold step of making this submission and hereby presents the following statement
to initiate discussions in order to fulfill his responsibilities as a researcher.

%\begin{itembox}[|]{\bf Statement}
%\noindent{\bf Statement for Discussion}\\
%Dynamic RL proposed here has the potential to endow AI including AGI (Artificial General Intelligence)
%with the ability to think and discover new things, which could pose a grave threat to humanity.
%Now, while this RL research is still in its baby steps,
%we should halt its further progress to protect humanity from the risk.
%%for the sake of humanity.
%The author sincerely and earnestly urges researchers to refrain from advancing this work
%and developers to avoid developing a library for Dynamic RL in frameworks,
%% not to proceed with
%at least until a consensus is reached among humans on this matter.
%Let us pause, imagine, and contemplate, free from preconception before it is too late.
%\end{itembox}

%\noindent
%Dear reviewers,\\
%If you think the above description of what the author thinks about the risk of this research is inappropriate for a technical paper, could you point out concretely why and which part is inappropriate, please?
