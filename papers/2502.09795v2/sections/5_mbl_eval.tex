\section{Map-based Localization Evaluation}
\label{sec:mbl_eval}

We evaluated the performance of our MbL pipeline on multiple image datasets generated in MARTIAN from HiRISE maps and DTMs of the Jezero Crater. To ensure an unbiased assessment, none of the test data overlaps with the data used for training. We conducted several experiments to assess the robustness of these methods under changes in lighting (Sec. \ref{subsec:mbl_robustnbess_to_az_and_el}) and scale (Sec. \ref{subsec:mbl_robustnbess_to_scale}). Figure \ref{fig:test_samples} shows areas on orthographic maps sampled from the test sets and rendered with two different illumination conditions, accompanied by three example observations at different altitudes.
To further stress our MbL pipeline with challenging lighting in a real-case scenario, we tested it over a simulated Martian day on the Jezero Crater site, with aerial observations generated in MARTIAN at multiple simulated times of day from sunrise to sunset (Sec. \ref{subsec:mbl_martian_day}). Furthermore, an evaluation of our method's performance under varying terrain morphologies is presented in the supplementary material.

We compared our results to the original LoFTR model fine-tuned on our training dataset (\textit{Fine-tuned LoFTR}), and to the model pre-trained on the MegaDepth dataset (\textit{Pre-trained LoFTR}). As for comparison with state-of-the-art feature matching in planetary aerial mobility, we also tested SIFT that proved to be one of the most accurate handcrafted methods for absolute localization over simulated Mars terrain \cite{brockers2022}.
% In the SIFT-based MbL evaluation, up to 4000 SIFT features were extracted from each query image and matched against those from the map, limited to the relevant search area. Feature matching was performed using k-nearest neighbors (KNN) with k=2. To ensure robustness, a ratio test was applied, retaining matches only if the ratio of the distance of the closest match to the distance of the second-closest match was below 0.75.
In each experiment, we use the percentage of queries with localization error $\|\mathbf{t}_{WC_{\text{query}}} - \widetilde{\mathbf{t}}_{WC_{\text{query}}}\|$ below 1m (@1m) as our evaluation metric. Also, we plot the Cumulative Distribution Function (CDF) of the localization accuracy up to 10m.
%In each experiment, we quantified the MbL accuracy as percentage of queries with localization errors $\|\mathbf{t}_{WC_{\text{query}}} - \widetilde{\mathbf{t}}_{WC_{\text{query}}}\|$. We computed the Cumulative distribution Functions (CdFs) of the localization accuracy, and show the @1m precision level as our main evaluation metric.  

\begin{figure}
    \centering
     \centering
        \includegraphics[width=\linewidth]{Figures/light_and_scale_var_example_resized.png}
    \vspace{-10pt}
    \caption{\label{fig:test_samples}Tiles from orthographic maps at sun (AZ=0°, EL=5°) (\textit{left}) (AZ=180°, EL=40°) (\textit{center}) with three sampled query observations (\textit{right}).}
\end{figure}


\subsection{Robustness to Changing Solar Angles}
\label{subsec:mbl_robustnbess_to_az_and_el}
Robustness to challenging illumination variance is assessed through registering  query observations onto orthographic maps rendered with varying sun elevation and azimuth angles, the effects of which are evaluated separately. 

The dataset for the experiment addressing the robustness to sun elevation changes comprises orthographic maps rendered at EL=$\{2°, 5°, 10°, 40°, 60°, 90°\}$ and AZ=$180°$, along with 500 nadir-pointing query observations at fixed sun angles (AZ=$180°$, EL=$40°$). The queries were taken in the 64-200 m altitude range which encompasses the nominal operation of the MSH. The minimum altitude of 64 m is set by the best resolution achievable in MARTIAN, which coincides with the 0.25 cm / pixel resolution of the HiRISE ortho-image used as texture. It is worth to note that elevations below 30$^{\circ}$ were not encountered during the model training. During the Martian day when our reference HiRISE map was collected, the sun elevation varied in the 7.9-82.6$^{\circ}$ range,  from 6:00 to 17:00 Local Mean Solar Time (LMST), with sunrise and sunset occurring at 05:11 LMST and 17:32 LMST, respectively. Therefore, elevations below 10$^{\circ}$ occupy a very brief portion of sun's local trajectory, representing exceptional cases in the Martian surface operations. Nevertheless, we include these cases in our evaluation to assess the models' ability to generalize and perform effectively under challenging lighting conditions.

To evaluate sun azimuth effect we generated 500 nadir-pointing queries with sun AZ=0$^{\circ}$ and EL=10$^{\circ}$ to be registered onto maps with varying azimuth angles in the 0-360$^{\circ}$ and same elevation as the queries.\\


\noindent \textbf{Robustness to sun elevation.} Figure \ref{fig:cdf_sun_el_var} shows the CDFs of the test observations' localization error onto maps at four different sun elevations and fixed 0 azimuth offset. Geo-LoFTR outperforms all the other methods in localization accuracy across the entire range of sun elevation offsets.
% and at all the precision levels except for the 5m-Accuracy, where its performance is comparable to the fine-tuned model.
In the case of zero sun angles' offsets, Geo-LoFTR is 87\% @1m accurate, showing an improvement of +38\% over the fine-tuned model, and +63\% over SIFT. Below 10$^{\circ}$ EL of the map, the performance of all the methods is significantly impacted, with Geo-LoFTR being 17\% @1m accurate at the very challenging case EL=2$^{\circ}$, where the pre-trained model and SIFT completely fail.
% However, Fine-tuned LoFTR shows the best failure localization rate overall, with a 5.8\% and 46.8\% rates at the low map sun elevations of 2$^{\circ}$ and 10$^{\circ}$, respectively, compared to 7.0\% and 53.6\% for Geo-LoFTR. 
%% UNCOMMENT
\begin{figure}
\centering
\begin{minipage}[b]{0.49\linewidth}
    \centering
    EL = $40^{\circ}$
    \vspace{5pt} \\
    \includegraphics[width=\linewidth]{Figures/cdf_acc_180az_40el_map_180az_40el_obs.png}
\end{minipage}
\begin{minipage}[b]{0.49\linewidth}
    \centering
    EL = $10^{\circ}$
    \vspace{5pt} \\
    \includegraphics[width=\linewidth]{Figures/cdf_acc_180az_10el_map_180az_40el_obs.png}
\end{minipage}
\\
\vspace{10pt}
\begin{minipage}[b]{0.49\linewidth}
    \centering
    EL = $5^{\circ}$
    \vspace{5pt} \\
    \includegraphics[width=\linewidth]{Figures/cdf_acc_180az_5el_map_180az_40el_obs.png}
\end{minipage}
\begin{minipage}[b]{0.49\linewidth}
    \centering
    EL = $2^{\circ}$
    \vspace{5pt} \\
    \includegraphics[width=\linewidth]{Figures/cdf_acc_180az_2el_map_180az_40el_obs.png}
\end{minipage}
\vspace{-10pt}
\caption{\label{fig:cdf_sun_el_var}Cumulative distributions of the localization error of simulated Mars observations at sun AZ=180$^{\circ}$ and EL=40$^{\circ}$, registered onto maps at four different elevation angles and 0$^{\circ}$ azimuth offset.}
\end{figure}
% \begin{figure}
% \centering
% \begin{minipage}[b]{0.4\linewidth}
%     \centering
%     \subcaption{0.5m-Accuracy}
%     \includegraphics[width=\linewidth]{Figures/loc_acc_vs_el_0.5acc-level.eps}
% \end{minipage}
% \begin{minipage}[b]{0.4\linewidth}
%     \centering
%     \subcaption{1m-Accuracy}
%     \includegraphics[width=\linewidth]{Figures/loc_acc_vs_el_1acc-level.eps}
% \end{minipage}
% \\
% \begin{minipage}[b]{0.4\linewidth}
%     \centering
%     \subcaption{2m-Accuracy}
%     \includegraphics[width=\linewidth]{Figures/loc_acc_vs_el_2acc-level.eps} 
% \end{minipage}
% \begin{minipage}[b]{0.4\linewidth}
%     \centering
%     \subcaption{5m-Accuracy}
%     \includegraphics[width=\linewidth]{Figures/loc_acc_vs_el_5acc-level.eps}
% \end{minipage}
% \caption{\label{fig:mbl_loc_acc_vs_el_vs_precision}Variation of localization accuracy with map sun elevation, for the MbL of 500 nadir-pointing observations at reference sun EL=40$^{\circ}$ (\textit{dashed line}).}
% % The localization accuracy is shown at the precision level of 0.5m (a) , 1 m (b), 2 m (c) and 5 m  (d). Queries and maps are generated from the Jezero Crater site with the MARTIAN framework, with a consistent sun azimuth angle of 180$^{\circ}$.}
% \end{figure}

%% UNCOMMENT
% \begin{figure}
% \centering
% % \makebox[0.01\linewidth]{}
% \makebox[0.47\linewidth]{Geo-LoFTR}
% \makebox[0.47\linewidth]{SIFT}
% \vspace{0.1cm} 

% \begin{minipage}[b]{0.49\linewidth}
%     \centering
%     \subcaption{map EL = $40^{\circ}$}
%     \includegraphics[width=\linewidth]{Figures/geo_map_40_180_obsv_40_180_id80_inlier_matches.png}  
% \end{minipage}
% \begin{minipage}[b]{0.49\linewidth}
%     \centering
%     \subcaption{map EL = $40^{\circ}$}
%     \includegraphics[width=\linewidth]{Figures/sift_map_40_180_obsv_40_180_id80_inlier_matches.png}
% \end{minipage}


% \begin{minipage}[b]{0.49\linewidth}
%     \centering
%     \subcaption{map EL = $10^{\circ}$}
%     \includegraphics[width=\linewidth]{Figures/geo_map_10_180_obsv_40_180_id80_inlier_matches.png}
% \end{minipage}
% \begin{minipage}[b]{0.49\linewidth}
%     \centering
%     \subcaption{map EL = $10^{\circ}$}
%     \includegraphics[width=\linewidth]{Figures/sift_map_10_180_obsv_40_180_id80_matches.png}
% \end{minipage}

% \begin{minipage}[b]{0.49\linewidth}
%     \centering
%     \subcaption{map EL = $5^{\circ}$}
%     \includegraphics[width=\linewidth]{Figures/geo_map_5_180_obsv_40_180_id80_inlier_matches.png} 
% \end{minipage}
% \begin{minipage}[b]{0.49\linewidth}
%     \centering
%     \subcaption{map EL = $5^{\circ}$}
%     \includegraphics[width=\linewidth]{Figures/sift_map_5_180_obsv_40_180_id80_matches.png} 
% \end{minipage}

% \begin{minipage}[b]{0.49\linewidth}
%     \centering
%     \subcaption{map EL = $2^{\circ}$}
%     \includegraphics[width=\linewidth]{Figures/geo_map_2_180_obsv_40_180_id80_inlier_matches.png} 
% \end{minipage}
% \begin{minipage}[b]{0.49\linewidth}
%     \centering
%     \subcaption{map EL = $2^{\circ}$}
%     \includegraphics[width=\linewidth]{Figures/sift_map_2_180_40_180_id80_matches.png} 
% \end{minipage}

% \caption{\label{fig:geo_sift_matched_keypoints_vs_el}Geo-LoFTR and SIFT matched keypoints displayed for a sample query image (\textit{left side of each panel}) with (180$^{\circ}$ AZ, 40$^{\circ}$ EL) sun angles and a map search area image (\textit{right side of each panel}) under four different sun elevations and 0$^{\circ}$ azimuth offset. Match lines are color-coded by confidence score, with redder indicating higher confidence.}
% \end{figure}


% \begin{figure*}
% \centering
% % \makebox[0.01\linewidth]{}
% \makebox[0.15\linewidth]{Geo-LoFTR}
% \makebox[0.15\linewidth]{Pre-trained LoFTR}
% \makebox[0.15\linewidth]{SIFT}
% \makebox[0.15\linewidth]{Geo-LoFTR}
% \makebox[0.15\linewidth]{Pre-trained LoFTR}
% \makebox[0.15\linewidth]{SIFT}
% \vspace{0.1cm} 

% % ROW 1
% \begin{minipage}[b]{0.16\linewidth}
%     \centering
%     \subcaption{map EL = $40^{\circ}$}
%     \includegraphics[width=\linewidth]{Figures/geo_map_40_180_obsv_40_180_id80_inlier_matches.png}  
% \end{minipage}
% \begin{minipage}[b]{0.16\linewidth}
%     \centering
%     \subcaption{map EL = $40^{\circ}$}
%     \includegraphics[width=\linewidth]{Figures/pre_map_40_180_obsv_40_180_id80_inlier_matches.png}  
% \end{minipage}
% \begin{minipage}[b]{0.16\linewidth}
%     \centering
%     \subcaption{map EL = $40^{\circ}$}
%     \includegraphics[width=\linewidth]{Figures/sift_map_40_180_obsv_40_180_id80_inlier_matches.png}
% \end{minipage}
% \begin{minipage}[b]{0.16\linewidth}
%     \centering
%     \subcaption{map AZ = $0^{\circ}$}
%     \includegraphics[width=\linewidth]{Figures/geo_map_10_0_obsv_10_0_id118_inlier_matches.png}  
% \end{minipage}
% \begin{minipage}[b]{0.16\linewidth}
%     \centering
%     \subcaption{map AZ = $0^{\circ}$}
%     \includegraphics[width=\linewidth]{Figures/pre_map_10_0_obsv_10_0_id118_inlier_matches.png}   
% \end{minipage}
% \begin{minipage}[b]{0.16\linewidth}
%     \centering
%     \subcaption{map AZ = $0^{\circ}$}
%     \includegraphics[width=\linewidth]{Figures/sift_map_10_0_obsv_10_0_id118_inlier_matches.png}  
% \end{minipage}

% % ROW 2
% \begin{minipage}[b]{0.16\linewidth}
%     \centering
%     \subcaption{map EL = $10^{\circ}$}
%     \includegraphics[width=\linewidth]{Figures/geo_map_10_180_obsv_40_180_id80_inlier_matches.png}
% \end{minipage}
% \begin{minipage}[b]{0.16\linewidth}
%     \centering
%     \subcaption{map EL = $10^{\circ}$}
%     \includegraphics[width=\linewidth]{Figures/pre_map_10_180_obsv_40_180_id80_inlier_matches.png}
% \end{minipage}
% \begin{minipage}[b]{0.16\linewidth}
%     \centering
%     \subcaption{map EL = $10^{\circ}$}
%     \includegraphics[width=\linewidth]{Figures/sift_map_10_180_obsv_40_180_id80_matches.png}
% \end{minipage}
% \begin{minipage}[b]{0.16\linewidth}
%     \centering
%     \subcaption{map AZ = $90^{\circ}$}
%     \includegraphics[width=\linewidth]{Figures/geo_map_10_90_obsv_10_0_id118_inlier_matches.png}  
% \end{minipage}
% \begin{minipage}[b]{0.16\linewidth}
%     \centering
%     \subcaption{map AZ = $90^{\circ}$}
%     \includegraphics[width=\linewidth]{Figures/pre_map_10_90_obsv_10_0_id118_inlier_matches.png}   
% \end{minipage}
% \begin{minipage}[b]{0.16\linewidth}
%     \centering
%     \subcaption{map AZ = $90^{\circ}$}
%     \includegraphics[width=\linewidth]{Figures/sift_map_10_90_obsv_10_0_id118_inlier_matches.png}  
% \end{minipage}

% % ROW 3
% \begin{minipage}[b]{0.16\linewidth}
%     \centering
%     \subcaption{map EL = $5^{\circ}$}
%     \includegraphics[width=\linewidth]{Figures/geo_map_5_180_obsv_40_180_id80_inlier_matches.png}
% \end{minipage}
% \begin{minipage}[b]{0.16\linewidth}
%     \centering
%     \subcaption{map EL = $5^{\circ}$}
%     \includegraphics[width=\linewidth]{Figures/pre_map_5_180_obsv_40_180_id80_inlier_matches.png}
% \end{minipage}
% \begin{minipage}[b]{0.16\linewidth}
%     \centering
%     \subcaption{map EL = $5^{\circ}$}
%     \includegraphics[width=\linewidth]{Figures/sift_map_5_180_obsv_40_180_id80_matches.png}
% \end{minipage}
% \begin{minipage}[b]{0.16\linewidth}
%     \centering
%     \subcaption{map AZ = $180^{\circ}$}
%     \includegraphics[width=\linewidth]{Figures/geo_map_10_180_obsv_10_0_id118_inlier_matches.png}  
% \end{minipage}
% \begin{minipage}[b]{0.16\linewidth}
%     \centering
%     \subcaption{map AZ = $180^{\circ}$}
%     \includegraphics[width=\linewidth]{Figures/pre_map_10_180_obsv_10_0_id118_inlier_matches.png}   
% \end{minipage}
% \begin{minipage}[b]{0.16\linewidth}
%     \centering
%     \subcaption{map AZ = $180^{\circ}$}
%     \includegraphics[width=\linewidth]{Figures/sift_map_10_180_obsv_10_0_id118_inlier_matches.png}  
% \end{minipage}

% % ROW 4
% \begin{minipage}[b]{0.16\linewidth}
%     \centering
%     \subcaption{map EL = $2^{\circ}$}
%     \includegraphics[width=\linewidth]{Figures/geo_map_2_180_obsv_40_180_id80_inlier_matches.png}
% \end{minipage}
% \begin{minipage}[b]{0.16\linewidth}
%     \centering
%     \subcaption{map EL = $2^{\circ}$}
%     \includegraphics[width=\linewidth]{Figures/pre_map_2_180_obsv_40_180_id80_inlier_matches.png}
% \end{minipage}
% \begin{minipage}[b]{0.16\linewidth}
%     \centering
%     \subcaption{map EL = $2^{\circ}$}
%     \includegraphics[width=\linewidth]{Figures/sift_map_2_180_40_180_id80_matches.png}
% \end{minipage}
% \begin{minipage}[b]{0.16\linewidth}
%     \centering
%     \subcaption{map AZ = $270^{\circ}$}
%     \includegraphics[width=\linewidth]{Figures/geo_map_10_270_obsv_10_0_id118_inlier_matches.png}  
% \end{minipage}
% \begin{minipage}[b]{0.16\linewidth}
%     \centering
%     \subcaption{map AZ = $270^{\circ}$}
%     \includegraphics[width=\linewidth]{Figures/pre_map_10_270_obsv_10_0_id118_inlier_matches.png}   
% \end{minipage}
% \begin{minipage}[b]{0.16\linewidth}
%     \centering
%     \subcaption{map AZ = $270^{\circ}$}
%     \includegraphics[width=\linewidth]{Figures/sift_map_10_270_obsv_10_0_id118_inlier_matches.png}  
% \end{minipage}

% \caption{\label{fig:matched_keypoints_vs_el_and_az}Each panel shows matched keypoints by Geo-LoFTR, Pre-trained LoFTR, and SIFT between a query image (\textit{left}) and a map search area (\textit{right}), with match lines color-coded by confidence (redder indicates higher confidence). The left 4x3 group uses a query with (180$^{\circ}$ AZ, 40$^{\circ}$ EL) sun angles, and maps with varying elevations ans same azimuth as the query. The right 4x3 group uses a querywith (0$^{\circ}$ AZ, 10$^{\circ}$ EL) sun angles, and maps with varying azimuths (same elevation as the query).}
% \end{figure*}


\noindent \textbf{Robustness to sun azimuth.}
The sun azimuth effect on MbL performance is presented through the cumulative distributions  (Figure \ref{fig:cdf_sun_az_var}) of the localization error for the test observations registered onto maps at four different  azimuth angles. Also in this experiment, Geo-LoFTR proved to be the most accurate model with a @1m accuracy being bound to the 54-63 \% range in the entire map sun azimuth range, despite the relatively low elevation of 10$^{\circ}$. The number and quality of the SIFT matched keypoints between query and map (Figure~\ref{fig:geo_sift_matched_keypoints_vs_az}) decreases much faster than the LoFTR-based models as we depart from the zero azimuth offset case, with failure already at 90$^{\circ}$ offset.
%Figure \ref{fig:mbl_loc_acc_vs_az_vs_precision} shows the variation of localization accuracy with the sun azimuth of the map at the precision levels of 0.5 m, 1 m, 2 m and 5 m. The azimuth angles are reported in the range [-180$^{\circ}$, 180$^{\circ}$] for visual clarity. Also in this experiment, Geo-LoFTR outperforms all the other methods, except for the 5m-Accuracy, where Fine-tuned LoFTR performs slightly better. Compared to the effect of the sun elevation variation, the accuracy of the LoFTR models trained on the Mars synthetic data results more robust to map sun azimuth offsets from the reference observation, with the 2m-Accuracy varying in a bandwidth of 8.5\% for Geo-LoFTR and 12.5\% for Fine-Tuned LoFTR. Such robustness is not observed for the pre-trained LoFTR model and SIFT, with a 2m-Accuracy varying in a bandwidth of 12\% for the former and 38.5\% for the latter.
% The matched keypoints on a sample observation and map search area are displayed for Geo-LoFTR and SIFT in Figure \ref{fig:geo_sift_matched_keypoints_vs_az}.

%% UNCOMMENT
\begin{figure}
\centering
\begin{minipage}[b]{0.49\linewidth}
    \centering
    AZ = $0^{\circ}$
    \vspace{5pt} \\
    \includegraphics[width=\linewidth]{Figures/cdf_acc_0az_10el_map_0az_10el_obs.png}
\end{minipage}
\begin{minipage}[b]{0.49\linewidth}
    \centering
    AZ = $90^{\circ}$
    \vspace{5pt} \\
    \includegraphics[width=\linewidth]{Figures/cdf_acc_90az_10el_map_0az_10el_obs.png}
\end{minipage}
\\
\vspace{10pt}
\begin{minipage}[b]{0.49\linewidth}
    \centering
    AZ = $180^{\circ}$
    \vspace{5pt} \\
    \includegraphics[width=\linewidth]{Figures/cdf_acc_180az_10el_map_0az_10el_obs.png}
\end{minipage}
\begin{minipage}[b]{0.49\linewidth}
    \centering
    AZ = $270^{\circ}$
    \vspace{5pt} \\
    \includegraphics[width=\linewidth]{Figures/cdf_acc_270az_10el_map_180az_40el_obs.png}
\end{minipage}
\vspace{-10pt}
\caption{\label{fig:cdf_sun_az_var}Cumulative distributions of the localization error of simulated Mars observations at sun AZ=0$^{\circ}$ and EL=10$^{\circ}$, registered onto maps at four different azimuth angles and 0$^{\circ}$ elevation offset.}
\end{figure}

% UNCOMMENT
\begin{figure*}
\centering
\makebox[0.3\linewidth]{\textbf{Geo-LoFTR}}
\makebox[0.3\linewidth]{\textbf{Pre-trained LoFTR}}
\makebox[0.3\linewidth]{\textbf{SIFT}}
\vspace{0.5cm} 

\begin{minipage}[b]{0.3\linewidth}
    \centering
    AZ = $0^{\circ}$
    \vspace{5pt} \\
    \includegraphics[width=\linewidth]{Figures/geo_map_10_0_obsv_10_0_id118_inlier_matches.png}  
\end{minipage}
\begin{minipage}[b]{0.3\linewidth}
    \centering
    AZ = $0^{\circ}$
    \vspace{5pt} \\
    \includegraphics[width=\linewidth]{Figures/pre_map_10_0_obsv_10_0_id118_inlier_matches.png}   
\end{minipage}
\begin{minipage}[b]{0.3\linewidth}
    \centering
    AZ = $0^{\circ}$
    \vspace{5pt} \\
    \includegraphics[width=\linewidth]{Figures/sift_map_10_0_obsv_10_0_id118_inlier_matches.png}  
\end{minipage}
\begin{minipage}[b]{0.3\linewidth}
    \centering
     \vspace{5pt} 
    AZ = $90^{\circ}$
    \vspace{5pt} \\
    \includegraphics[width=\linewidth]{Figures/geo_map_10_90_obsv_10_0_id118_inlier_matches.png}
\end{minipage}
\begin{minipage}[b]{0.3\linewidth}
    \centering
     \vspace{5pt} 
    AZ = $90^{\circ}$
    \vspace{5pt} \\
    \includegraphics[width=\linewidth]{Figures/pre_map_10_90_obsv_10_0_id118_inlier_matches.png}   
\end{minipage}
\begin{minipage}[b]{0.3\linewidth}
    \vspace{5pt} 
    \centering   
    AZ = $90^{\circ}$
    \vspace{5pt} \\
    \includegraphics[width=\linewidth]{Figures/sift_map_10_90_obsv_10_0_id118_inlier_matches.png}
\end{minipage}
\begin{minipage}[b]{0.3\linewidth}
    \centering
     \vspace{5pt} 
    AZ = $180^{\circ}$
    \vspace{5pt} 
    \includegraphics[width=\linewidth]{Figures/geo_map_10_180_obsv_10_0_id118_inlier_matches.png}
\end{minipage}
\begin{minipage}[b]{0.3\linewidth}
    \centering
     \vspace{5pt} 
    AZ = $180^{\circ}$
    \vspace{5pt} 
    \includegraphics[width=\linewidth]{Figures/pre_map_10_180_obsv_10_0_id118_inlier_matches.png}   
\end{minipage}
\begin{minipage}[b]{0.3\linewidth}
    \centering
     \vspace{5pt} 
    AZ = $180^{\circ}$
    \vspace{5pt} 
    \includegraphics[width=\linewidth]{Figures/sift_map_10_180_obsv_10_0_id118_matches.png}
\end{minipage}

\caption{\label{fig:geo_sift_matched_keypoints_vs_az}Geo-LoFTR, Pre-trained LoFTR and SIFT matched keypoints displayed for a sample query image (\textit{left side of each panel}) with (0$^{\circ}$ AZ, 10$^{\circ}$ EL) sun angles and a map search area image (\textit{right side of each panel}) under three different sun elevations and 0$^{\circ}$ azimuth offset. Match lines are color-coded by confidence score, with redder indicating higher confidence. Despite still providing a localization solution in the 0-180° AZ range, the pre-trained LoFTR matches exhibit lower confidence with azimuth changes than Geo-LoFTR, resulting in a coarser localization.} 
\end{figure*}
% \DP{include explanation about the fact that despite pre-trained LoFTR still provide accurate matches, the accuracy is much lower than Geo- and Fine-tunded LoFTR}}


% \begin{figure}
% \centering

% \makebox[0.01\linewidth]{}
% \makebox[0.3\linewidth]{64 m altitude}
% \makebox[0.3\linewidth]{100 m altitude}
% \makebox[0.3\linewidth]{200 m altitude}
% \vspace{0.1cm} % Space between azimuth labels and images

% \begin{minipage}[b]{0.01\linewidth}
%     \raisebox{2.5em}[0pt][0pt]{\makebox[0pt][r]{Geo-LoFTR}} % Adjust the raise value as needed
% \end{minipage}
% \begin{minipage}[b]{0.3\linewidth}
%     \centering
%     \includegraphics[width=\linewidth]{Figures/geo_cliff_64m_id0_inlier_matches_light_offset.png}
% \end{minipage}
% \begin{minipage}[b]{0.3\linewidth}
%     \centering
%     \includegraphics[width=\linewidth]{Figures/geo_cliff_100m_id5_inlier_matches_light_offset.png}
% \end{minipage}
% \begin{minipage}[b]{0.3\linewidth}
%     \centering
%     \includegraphics[width=\linewidth]{Figures/geo_cliff_200m_id27_inlier_matches_light_offset.png}
% \end{minipage}
    

% \begin{minipage}[b]{0.01\linewidth}
%     \raisebox{2.5em}[0pt][0pt]{\makebox[0pt][r]{SIFT}} % Adjust the raise value as needed
% \end{minipage}
% \begin{minipage}[b]{0.3\linewidth}
%     \centering
%     \includegraphics[width=\linewidth]{Figures/sift_cliff_64m_id0_inlier_matches_light_offset.png}
% \end{minipage}
% \begin{minipage}[b]{0.3\linewidth}
%     \centering
%     \includegraphics[width=\linewidth]{Figures/sift_cliff_100m_id5_inlier_matches_light_offset.png}
% \end{minipage}
% \begin{minipage}[b]{0.3\linewidth}
%     \centering
%     \includegraphics[width=\linewidth]{Figures/sift_cliff_200m_id27_inlier_matches_light_offset.png}
% \end{minipage}

% \begin{minipage}[b]{0.01\linewidth}
%     \raisebox{2.5em}[0pt][0pt]{\makebox[0pt][r]{Geo-LoFTR}} % Adjust the raise value as needed
% \end{minipage}
% \begin{minipage}[b]{0.3\linewidth}
%     \centering
%     \includegraphics[width=\linewidth]{Figures/geo_dunes_64m_id134_inlier_matches_light_offset.png}
% \end{minipage}
% \begin{minipage}[b]{0.3\linewidth}
%     \centering
%     \includegraphics[width=\linewidth]{Figures/geo_dunes_100m_id5_inlier_matches_light_offset.png}
% \end{minipage}
% \begin{minipage}[b]{0.3\linewidth}
%     \centering
%     \includegraphics[width=\linewidth]{Figures/geo_dunes_200m_id0_inlier_matches_light_offset.png}
% \end{minipage}
    
% \begin{minipage}[b]{0.01\linewidth}
%     \raisebox{2.5em}[0pt][0pt]{\makebox[0pt][r]{SIFT}} % Adjust the raise value as needed
% \end{minipage}
% \begin{minipage}[b]{0.3\linewidth}
%     \centering
%     \includegraphics[width=\linewidth]{Figures/sift_dunes_64m_id134_inlier_matches_light_offset.png}
% \end{minipage}
% \begin{minipage}[b]{0.3\linewidth}
%     \centering
%     \includegraphics[width=\linewidth]{Figures/sift_dunes_100m_id5_inlier_matches_light_offset.png}
% \end{minipage}
% \begin{minipage}[b]{0.3\linewidth}
%     \centering
%     \includegraphics[width=\linewidth]{Figures/sift_dunes_200m_id0_inlier_matches_light_offset.png}
% \end{minipage}


% \caption{\label{fig:matches_vs_alt_rugged}Matched keypoints for rugged (\textit{top}) and dunal (\textit{bottom}) between observations (\textit{left, in each panel}) and map \textit{right, in each panel}) for the Jezero Crater HiRISE map, at altitudes of 64 m (\textit{left column)}, 100 m (\textit{middle column}) and 200 m (\textit{right column}) for Geo-LoFTR and SIFT. Observations are generated in MARTIAN framework with sun AZ=180$^{\circ}$ and EL=40$^{\circ}$. The map is rendered at sun AZ=0$^{\circ}$ and EL=5$^{\circ}$.}
% \end{figure}

\subsection{Robustness to Scale Variation}
\label{subsec:mbl_robustnbess_to_scale}

We split the test observations from Sec. \ref{subsec:mbl_robustnbess_to_az_and_el} in three different altitude sub-ranges, and registered them onto maps with zero sun angle offsets to assess the pipeline's performance under scale changes. Figure \ref{fig:cdf_scale_var} shows the CDF of the localization errors of observations taken within 64m-112m, 112m-155m, 155m-200m registered on a map with constant (AZ=180°, EL=40°) sun angles. With only a -7\% @1m accuracy drop across the entire altitude range, Geo-LoFTR proved to be more robust than the fine-tuned model (-33\% @1m). A similar degree of scale invariance is shown for the pre-trained model and SIFT, although being much less accurate.  

\begin{figure}
\centering
\begin{minipage}[b]{0.49\linewidth}
    \centering
    64-112 m
    \vspace{2pt} \\
    \includegraphics[width=\linewidth]{Figures/cdf_acc_180az_40el_map_180az_40el_obs_64-112m.png}
\end{minipage}
\begin{minipage}[b]{0.49\linewidth}
    \centering
    112-155 m
    \vspace{2pt} \\
    \includegraphics[width=\linewidth]{Figures/cdf_acc_180az_40el_map_180az_40el_obs_112-155m.png}
\end{minipage}
\begin{minipage}[b]{0.49\linewidth}
    \centering
    155-200 m
    \vspace{2pt} \\
    \includegraphics[width=\linewidth]{Figures/cdf_acc_180az_40el_map_180az_40el_obs_155-200m.png}
\end{minipage}
\caption{\label{fig:cdf_scale_var}Cumulative distributions of the localization error of simulated Mars observations at sun AZ=0$^{\circ}$ and EL=10$^{\circ}$, registered onto maps with the same illumination condition for three different altitude ranges.}
\end{figure}


\subsection{Robustness to Combined Illumination and Scale Changes}
\label{subsec:mbl_robustnbess_to_scale_ill}

Leveraging the test data in Sec \ref{subsec:mbl_robustnbess_to_az_and_el}, we performed a quantitative evaluation of the scale variation effects in conjunction with sun angle offsets. Figure \ref{fig:mbl_loc_acc_vs_el_and_az_vs_alt}, shows the @1m accuracy as a function of map sun EL and AZ for observations taken within three different altitude ranges.
%and fixed sun angles in the respective cases. 
Although localization accuracy declines sharply at relatively low sun elevation angles, Geo-LoFTR maintains consistent localization performance across altitude variations within the 10–90° EL range. In contrast, the fine-tuned model demonstrates poor robustness in the same range. A similar trend is observed for azimuth variations, where localization accuracy remains stable with changing azimuth but decreases with altitude.


% Figure \ref{fig:mbl_loc_acc_vs_el_and_az_vs_alt} show the accuracy of MbL at 1 m precision against sun angles offsets in elevation and azimuth, within three different altitude sub-ranges. Geo-LoFTR results more robust to altitude variations than the other methods, suggesting the benefits introduced to localization accuracy by depth data.
% Geo-LoFTR results more robust to altitude variations than the fine-tuned model over the entire set of sun elevation offsets. Picking a map sun elevation of 10$^{\circ}$ for instance, the drop in 1m-accuracy is only of -13\% between the 156-199 m and 64-113 m altitude ranges for Geo-LoFTR, compared to a -28\% accuracy drop for Fine-tuned LoFTR. The improved scale invariance of the geometry-aided model over the other methods is confirmed also under lighting variations in terms of sun azimuth offsets, thus providing yet another piece of evidence of the benefits introduced to localization accuracy by depth data.

%% UNCOMMENT
\begin{figure}
\centering
\begin{minipage}[b]{1.0\linewidth}
    \centering
    \includegraphics[width=\linewidth]{Figures/1m-acc_loc_vs_el_vs_alt.png}    
\end{minipage}
\vspace{5pt} \\
\begin{minipage}[b]{1.0\linewidth}
    \centering
    \includegraphics[width=\linewidth]{Figures/1m-acc_loc_vs_az_vs_alt.png}    
\end{minipage}
\vspace{-5pt} \\
\caption{Localization accuracy at 1m precision as a function of map sun elevation (\textit{top}) and azimuth (\textit{bottom}) for test observations across three altitude ranges. Sun azimuth angles are in the $[-180^{\circ}, 180^{\circ}]$ range. Map sun angles matching the observations are marked with a thick black vertical line.}
\label{fig:mbl_loc_acc_vs_el_and_az_vs_alt}
\end{figure}

\subsection{Localization Over a Simulated Martian Day}
\label{subsec:mbl_martian_day}
The MbL performance is investigated for observations taken at different LMSTs during a simulated Martian day on the Jezero Crater HiRISE map at coordinates (77.44$^{\circ}$E, 18.43$^{\circ}$N). We used the Mars24~\cite{mars24} software developed by NASA Goddard Institute for Space Studies to compute the sun's local trajectory for the selected site on a given date. The chosen date, 2031-05-10, ensures that the sun zenith is at a relatively high elevation angle of 86.7$^{\circ}$, allowing a broad range of elevation angles to be observed throughout the day (Figure \ref{fig:sun_profile}). 
% where the sun azimuth is measured clockwise from North in Mars24, differently from our MARTIAN framework, where it is defined counterclockwise from East. The sun elevation is still reported with the same convention adopted in the MARTIAN.

% UNCOMMENT
\begin{figure}
\setlength{\abovecaptionskip}{0pt}  % Removes space above caption
\setlength{\belowcaptionskip}{0pt}  % Removes space below caption
\centering
\includegraphics[width=0.98\linewidth]{Figures/sun_profile.png}
\caption{Sun trajectory on a local panorama from 77.44$^{\circ}$E longitude and 18.43$^{\circ}$N latitude on Mars, on 2031-05-10, with positions shown at four Local Mean Solar Times (LMSTs). Adapted from Mars24 \cite{mars24}.}
\label{fig:sun_profile}
\end{figure}

We generated nadir-pointing observations in MARTIAN at multiple times of the day from 5:30 to 17:00 LMST, with a total of 3000 queries collected across the 64-200 m altitude range (Figure \ref{fig:obsv_lmst}). We also rendered an orthographic map at 15:00 LMST, (AZ=175.1$^{\circ}$, 39.9$^{\circ}$ EL), serving as a HiRISE-like reference. 
Figure \ref{fig:loc_acc_vs_lmst_1acc_vs_alt} shows the @1m accuracy as function of LMTs within three different altitude sub-ranges. Geo-LoFTR outperformed the other methods for most of the Martian day, except at 5:00 LMST, where the fine-tuned model shows better accuracy. However, the fine-tuned LoFTR experienced significant performance degradation with altitude, in contrast with the other methods that exhibited a certain grade of scale invariance also in this experiment.

% LoFTR trained on the Mars datasets outperformed SIFT. SIFT proves more accurate than Pre-trained LoFTR in times of day between 8:00 and 15:00 LMST. However, its accuracy rapidly declines for observations taken earlier than 8:00 LMST underperforming the base LoFTR model, especially at lower altitudes. Geo-LoFTR results in being the best localization method at 1 m and 2 m precision levels for most of the Martian day, except when observations are taken at 5:30 LMST, where its performance significantly degrades. Despite the overall higher accuracy, Geo-LoFTR is also observed to be less stable than the fine-tuned model throughout the Martian day. However, it is more robust to altitude changes than its visual counterpart, providing yet another evidence of the contribution provided by geometric context in improving scale invariance. 
% UNCOMMENT
\begin{figure}
\centering

\makebox[0.01\linewidth]{}
% \makebox[0.25\linewidth]{\small 64 m}
% \makebox[0.25\linewidth]{\small 100 m}
% \makebox[0.25\linewidth]{\small 200 m}
% \vspace{0.1cm} % Space between azimuth labels and images

% \begin{minipage}[b]{0.01\linewidth}
%     \raisebox{3.5em}[0pt][0pt]{\makebox[0pt][r]{\small LMST:}} % Adjust the raise value as needed
% \end{minipage}
\begin{minipage}[b]{0.01\linewidth}
    \raisebox{2em}[0pt][0pt]{\makebox[0pt][r]{\small \shortstack{\textbf{LMST}: \\ 05:30}}} 
\end{minipage}
\begin{minipage}[b]{0.25\linewidth}
    \centering
    \includegraphics[width=\linewidth]{Figures/05_30_0479.png}
\end{minipage}
\begin{minipage}[b]{0.25\linewidth}
    \centering
    \includegraphics[width=\linewidth]{Figures/05_30_0460.png}
\end{minipage}
\begin{minipage}[b]{0.25\linewidth}
    \centering
    \includegraphics[width=\linewidth]{Figures/05_30_0443.png}
\end{minipage}

\begin{minipage}[b]{0.01\linewidth}
    \raisebox{2em}[0pt][0pt]{\makebox[0pt][r]{\small 06:00
    }} % Adjust the raise value as needed
\end{minipage}
\begin{minipage}[b]{0.25\linewidth}
    \centering
    \includegraphics[width=\linewidth]{Figures/06_00_0018.png}
\end{minipage}
\begin{minipage}[b]{0.25\linewidth}
    \centering
    \includegraphics[width=\linewidth]{Figures/06_00_0040.png}
\end{minipage}
\begin{minipage}[b]{0.25\linewidth}
    \centering
    \includegraphics[width=\linewidth]{Figures/06_00_0046.png}
\end{minipage}

\begin{minipage}[b]{0.01\linewidth}
    \raisebox{2em}[0pt][0pt]{\makebox[0pt][r]{\small 08:00
    }} % Adjust the raise value as needed
\end{minipage}
\begin{minipage}[b]{0.25\linewidth}
    \centering
    \includegraphics[width=\linewidth]{Figures/08_00_0000.png}
\end{minipage}
\begin{minipage}[b]{0.25\linewidth}
    \centering
    \includegraphics[width=\linewidth]{Figures/08_00_0005.png}
\end{minipage}
\begin{minipage}[b]{0.25\linewidth}
    \centering
    \includegraphics[width=\linewidth]{Figures/08_00_0023.png}
\end{minipage}

\begin{minipage}[b]{0.01\linewidth}
    \raisebox{2em}[0pt][0pt]{\makebox[0pt][r]{\small11:29
    }} % Adjust the raise value as needed
\end{minipage}
\begin{minipage}[b]{0.25\linewidth}
    \centering
    \includegraphics[width=\linewidth]{Figures/11_29_0000.png}
\end{minipage}
\begin{minipage}[b]{0.25\linewidth}
    \centering
    \includegraphics[width=\linewidth]{Figures/11_29_0002.png}
\end{minipage}
\begin{minipage}[b]{0.25\linewidth}
    \centering
    \includegraphics[width=\linewidth]{Figures/11_29_0006.png}
\end{minipage}

\begin{minipage}[b]{0.01\linewidth}
    \raisebox{2em}[0pt][0pt]{\makebox[0pt][r]{\small15:00
    }} % Adjust the raise value as needed
\end{minipage}
\begin{minipage}[b]{0.25\linewidth}
    \centering
    \includegraphics[width=\linewidth]{Figures/15_00_0004.png}
\end{minipage}
\begin{minipage}[b]{0.25\linewidth}
    \centering
    \includegraphics[width=\linewidth]{Figures/15_00_0008.png}
\end{minipage}
\begin{minipage}[b]{0.25\linewidth}
    \centering
    \includegraphics[width=\linewidth]{Figures/15_00_0007.png}
\end{minipage}

\begin{minipage}[b]{0.0255\linewidth}
    \raisebox{2em}[0pt][0pt]{\makebox[0pt][r]{\small17:00
    }} % Adjust the raise value as needed
\end{minipage}
\begin{minipage}[b]{0.25\linewidth}
    \centering
    \includegraphics[width=\linewidth]{Figures/17_00_0022.png}
\end{minipage}
\begin{minipage}[b]{0.25\linewidth}
    \centering
    \includegraphics[width=\linewidth]{Figures/17_00_0042.png}
\end{minipage}
\begin{minipage}[b]{0.25\linewidth}
    \centering
    \includegraphics[width=\linewidth]{Figures/17_00_0082.png}
\end{minipage}
\caption{Sample nadir-pointing observations rendered at different Local Mean Solar Times (LMSTs) taken on Mars on 2031-05-10. The reference HiRISE map is taken at 15:00 LMST. The sun Zenith is at 11:29 LMST.}
\label{fig:obsv_lmst}
% Data are generated from Jezero Crater using the MARTIAN framework. }
\end{figure}


% \begin{table}
% \centering
% \begin{tabular}{l c c}
% \textbf{LMST} & \textbf{sun EL} & \textbf{sun AZ} \\
% \hline
% 05:07 (rise)   & -0.2$^{\circ}$ & 16.6$^{\circ}$\\
% 05:30         & 5.0$^{\circ}$ & 14.8$^{\circ}$\\
% 06:00         & 12.1$^{\circ}$ & 12.6$^{\circ}$\\
% 08:00         & 40.1$^{\circ}$ & 5.0$^{\circ}$\\
% 11:29 (zenith) & 86.9$^{\circ}$ & 270.3$^{\circ}$\\
% 15:00 (HiRISE) & 39.9$^{\circ}$ & 175.1$^{\circ}$\\
% 17:00         & 11.6$^{\circ}$ & 167.4$^{\circ}$\\
% 17:52 (set)   &  -0.2$^{\circ}$ & 163.5$^{\circ}$\\
%   \hline \\
% \end{tabular}
% \caption{\label{tab:lmst_solar_profile}Specifications of points of interests on the solar local trajectory from (77.44$^{\circ}$E, 18.43$^{\circ}$N) coordinates on Mars, on 2031-05-10, with sun azimuth and elevation angles referred to the MARTIAN framework. Adapted from Mars24\cite{mars24}.} 
% % sun azimuth is measured clockwise from North in Mars24 framework; while it is defined counterclockwise from East in the MARTAIAN framework. Adapted from Mars24\cite{mars24}.}
% \end{table}

% \begin{figure}
% \centering
% \begin{minipage}[b]{0.3\linewidth}
%     \centering
%     \subcaption{5:30 LMST}
%     \includegraphics[width=\linewidth]{Figures/CDF_loc_acc_map_HiRISE_obsv_morning_5_30.eps}
% \end{minipage}
% \begin{minipage}[b]{0.3\linewidth}
%     \centering
%     \subcaption{6:00 LMST}
%     \includegraphics[width=\linewidth]{Figures/CDF_loc_acc_map_HiRISE_obsv_morning_6_00.eps}
% \end{minipage}
% \begin{minipage}[b]{0.3\linewidth}
%     \centering
%     \subcaption{8:00 LMST}
%     \includegraphics[width=\linewidth]{Figures/CDF_loc_acc_map_HiRISE_obsv_morning.eps}
% \end{minipage}
% \begin{minipage}[b]{0.3\linewidth}
%     \centering
%     \subcaption{11:29 LMST}
%     \includegraphics[width=\linewidth]{Figures/CDF_loc_acc_map_HiRISE_obsv_zenith.eps}
% \end{minipage}
% \begin{minipage}[b]{0.3\linewidth}
%     \centering
%     \subcaption{15:00 LMST}
%     \includegraphics[width=\linewidth]{Figures/CDF_loc_acc_map_40_180_obsv_40_180.eps} 
% \end{minipage}
% \begin{minipage}[b]{0.3\linewidth}
%     \centering
%     \subcaption{17:00 LMST}
%     \includegraphics[width=\linewidth]{Figures/CDF_loc_acc_map_HiRISE_obsv_set.eps}
% \end{minipage}

% \caption{\label{fig:mbl_cdf_martian_day}Cumulative distributions of the localization error for the MbL of 500 nadir-pointing observations (altitude 64-200m) taken at six different Local Mean Solar Times (LMSTs) on Mars, on 2031-05-10. The reference HiRISE map is taken at 15:00 LMST (c). The sun Zenith is at 11:29 LMST (b). Data are generated from Jezero Crater using the MARTIAN framework.}
% \end{figure}

% UNCOMMENT
\begin{figure}
\centering
\includegraphics[width=1\linewidth]{Figures/1m-acc_loc_vs_LMST_vs_alt_zoom.png} 
\caption{Localization accuracy (@1m) as a function of Local Mean Solar Time (LMST) of simulated test observations from the Jezero Crater on 2031-05-10 across the 64-200 m altitude range. The reference HiRISE-like map is taken at 15:00 LMST (\textit{dashed black line}). The sun Zenith is at 11:29 LMST.}
\vspace{-10pt}
\label{fig:loc_acc_vs_lmst_1acc_vs_alt}
\end{figure}
% Analyzing the 1m-accuracy, the LoFTR trained on the Mars datasets outperformed the pre-trained LoFTR model and SIFT, which performed in the 20-30\% accuracy range. SIFT proves more accurate than Pre-trained LoFTR in times of day between 8:00 LMST and 15:00 LMST. However, its accuracy rapidly declines for observations taken earlier than 8:00 LMST underperforming the base LoFTR model, especially at lower altitudes. Geo-LoFTR results in being the best localization method at 1 m and 2 m precision levels for most of the Martian day, except when observations are taken at 5:30 LMST, where its performance significantly degrades. Despite the overall higher accuracy, Geo-LoFTR is also observed to be less stable than the fine-tuned model throughout the Martian day. However, it is confirmed again to be more robust to altitude changes than its visual counterpart, providing yet another evidence of the contribution provided by geometric context in achieving scale invariance. 

\subsection{Discussion}
\label{subsec:discussion}
Geo-LoFTR demonstrated superior localization accuracy compared to other methods across a broad range of illumination conditions, indicating that incorporating depth information can mitigate degeneracies inherent to purely visual data. Robustness to sun elevation is maintained within a wide range of angles, except in extremely challenging cases (e.g., EL = 2°), where poor lighting and extensive shadow coverage might saturate the constraining power of the geometric information, leading to a rapid decline in localization accuracy. More stable is the behavior for changes in azimuth.
Geo-LoFTR also showed greater robustness to changing observation altitude than the fine-tuned model across multiple experiments, suggesting that providing a geometric context contributes to scale invariance. A possible explanation is that depth data constrains matches by providing consistent pixel-to-pixel depth relationships across altitudes, reflecting terrain elevation alone. This added layer of geometric consistency likely enhances Geo-LoFTR's ability to learn accurate matches by reducing ambiguity from appearance-based features alone.

