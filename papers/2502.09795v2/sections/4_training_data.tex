\section{Training dataset}
\label{sec:training_set}

We generated a training image dataset comprising 17 orthographic gray-scale maps rendered from combinations of Sun azimuth ($0^{\circ}-360^{\circ}, 45^{\circ} steps$) and elevation $(30^{\circ}, 60^{\circ}, 90^{\circ}$), one corresponding orthographic depth map, and 4500 nadir-pointing aerial observations at fixed Sun angles AZ=180$^{\circ}$ and EL=40$^{\circ}$ along with their corresponding depth images. The query observations were randomly sampled from the HiRISE Jezero Crater's DTM with uniform distribution in the (x,y) coordinates in the world frame and within altitude range [$64, 200$] m. The query camera intrinsics are given by a pinhole camera model characterized by a sensor width of 80 mm, a focal length of 32 mm, 0 lens shifts along the image width and height axes, and an unitary pixel aspect ratio. Camera extrinsics, along with altitude data, were stored for each observation and served as ground truth. Further details are reported in Table \ref{tab:dataset_params}. Figure \ref{fig:map_tile_w_obs} shows gray-scale and normalized depth images of a map tile, with three sampled observations. 
Training examples for Geo-LoFTR are created by forming triplets ($I_A$, $I_B$, $I_C$) out of query observations and map windows crops (gray-scale and depth images) for multiple combinations of Sun azimuth and elevation angles' offsets between queries and the source maps. For a given combination of query and map lighting, each query observation is paired with map windows with at least 25\% area overlap on the terrain. Therefore, the set of triplets is formed by different combination of image $I_A$ with the mop window tuple ($I_B$, $I_C$). The map window sizes are carefully chosen to introduce an appropriate level of scale variance within the altitude range of the observations, ensuring a balance between model generalization and training efficiency. 
% By choosing prefixed map windows size of 1320 $\times$ 990 pixels for observations within the [64, 132] m altitude range, and a size of 2000 $\times$ 1500 pixels for the (132, 200] m, the scale ratio between query images and map windows is ensured to vary between 1 and 3.
Geo-LoFTR has been fine-tuned from the original LoFTR pre-trained model on a total 
of 150,705 generated triplets. An independent validation set of 3,400 triplets has been used to regularly assess the model performance during training and prevent over-fitting. 

\begin{figure}
    \centering
    \begin{minipage}[b]{0.9\linewidth}
        \centering
        \includegraphics[width=\linewidth]{Figures/map_tile_w_3_obs_aligned.jpg}
    \end{minipage}
    \caption{\label{fig:map_tile_w_obs} Gray-scale (left) and normalized depth (middle) images of a rendered orthographic map tile with three observations (right) sampled from different locations under the same illumination conditions.}
\end{figure}


\begin{table}
\centering
% Review this table
\begin{tabular}{l l l}
\textbf{Parameters} & \textbf{Maps} & \textbf{Observations} \\ \hline 
N.o. gray images & 17 & 4500 \\
N.o. depth images & 1 & 4500 \\
Image size & 26,949 $\times$ 57,613  & 480 $\times$ 640  \\ 
Pixel resolution & 0.25 m / px  &  [0.25, 0.78] m / px \\
Projection type  & Orthographic & Perspective \\  
Orthographic scale & 6737 m & / \\
Focal length & / & 32 mm \\
Sensor width & / & 80 mm \\
Location in $W$ & (0,0) m &  uniform random \\
&  &  distribution \\
Orientation in $W$ & nadir-pointing & nadir-pointing \\
Altitude & 4000 m & [64, 200] m\\
Sun AZ   & [0, 360]$^{\circ}$ with 45$^{\circ}$ steps & 180$^{\circ}$ \\  
Sun EL & 30$^{\circ}$, 60$^{\circ}$, 90$^{\circ}$ \\  
\end{tabular}
\caption{\label{tab:dataset_params} Parameters for the maps and observations image dataset sourced for the generation of the pairs formed by query observations and map windows, used for LoFTR and Geo-LoFTR training.}
\end{table}