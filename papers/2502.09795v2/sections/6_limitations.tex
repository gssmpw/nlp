\section{Limitations}
\label{sec:limitations}
Since our main motivation was to investigate robustness to illumination and scale variations, all observations in our datasets are nadir-pointing (i.e., we do not add any variation in the camera's viewpoint). While LoFTR~\cite{loftr} has shown sufficient invariance to viewpoint changes in in-the-wild datasets~\cite{megadepth}, it is still not clear whether this would transfer in the Martian domain.
Furthermore, we do not use any CTX orbital map products (6m/pixel) to generate our data and focus only on HiRISE (0.25m/pixel) maps which are of better quality and higher resolution. There is interest by the Mars exploration community to utilize CTX maps due to their almost 99\% coverage of the planet. A separate investigation is warranted to determine whether current image matching methods can handle this large resolution difference. 
Finally, we did not focus on optimizing our pipeline in terms of computational efficiency as this was out-of-the-scope of this work.

%These findings set the stage for future research into multi-modal feature representation interpretability to deepen the understanding of the learned depth-visual relationships and further justify the indications that emerged from our MbL evaluation. Future efforts could include adopting contrastive learning techniques to refine the shared embeddings further, potentially improving localization performance. One limitation of this work consists in the nadir-pointing nature of the aerial observations. While this setup serves as an initial investigation into robustness against illumination changes, future work needs to address robustness analysis to viewpoint changes. New versions of LoFTR and Geo-LoFTR can be trained on datasets including observations taken with specified ranges in yaw, pitch and raw angles, to be evaluated on dedicated test sets. Future directions in validating our MbL pipeline should also include performance evaluations on lower-resolution orbital maps, such as CTX. Our method could register real observations collected during Mars 2020's Entry, Descent, and Landing (EDL) phase on CTX maps, benchmarking against state-of-the-art template matching methods. Lastly, LoFTR and Geo-LoFTR models fine-tuned on dedicated synthetic datasets simulating imagery from Ingenuity, can be used to evaluate our MbL pipeline's ability to register real Ingenuity images on HiRISE maps.