\section{Conclusion} 
\label{sec:conclusion}
In this paper, we presented a new map-based localization pipeline that uses Geo-LoFTR, a geometry-aided feature matching model to register onboard images to reference maps, and MARTIAN, a custom simulation framework that uses real Digital Terrain Models from Mars to generate large-scale datasets.
Our method has outperformed the baselines in terms of localization accuracy by a large margin, demonstrating that enhancing the feature matching with geometric context results in increased robustness to challenging environmental conditions. This robustness was witnessed across the board in terms of varying sun elevation and azimuth angles, along with altitude variation.

%We proposed a map-based localization pipeline for the long-range navigation of future Mars rotorcraft under challenging illumination conditions, which integrates Geo-LoFTR, a novel multi-modal deep-learning-based image matcher that incorporates map depth data using Digital Terrain Models (DTMs) obtained from orbital assets.  By adopting cross-attention mechanisms, Geo-LoFTR learns a shared feature embedding that merges depth and visual information, aiming to leverage geometric context to enhance the robustness of feature representations against illumination changes, facilitating reliable matches. To support model training and evaluation, we developed MARTIAN, a custom rendering framework in Blender, which allowed us to generate synthetic Mars datasets derived from HiRSE DTMs and ortho-images. We fine-tuned LoFTR and trained Geo-LoFTR models on query observation/map window pairs formed from combinations of 17 ortho-projected maps and 4,500 nadir-pointing aerial observations across diverse lighting conditions and scales, for a total of 150,000 training pairs. Over 4,000 test observations and 24 maps with controlled lighting, scale, and terrain morphology variations were further rendered to rigorously evaluate MbL performance under varying conditions and against established methods, such as SIFT and the pre-trained LoFTR model. Geo-LoFTR proved higher localization accuracy compared to other methods across a broad range of Sun elevation and azimuth angle offsets between maps and observations. Additionally, the integration of depth information in Geo-LoFTR appears to increase robustness to terrain morphology and scale variations beyond what visual feature representations alone can achieve. 
% This integrated geometric context seems to contribute to scale invariance, as depth data constrains matches with pixel-to-pixel depth relationships depending on DTM's elevations only, thus mitigating the effect of altitude changes. 

% \subsection*{Fixed Sun Azimuth and Elevation Offsets}
% MbL evaluations over observations with challenging offsets of 180$^{\circ}$ in Sun azimuth (AZ) and 35$^{\circ}$ in elevation (EL) form a reference map at (180$^{\circ}$ AZ, 5$^{\circ}$ EL) proved that Geo-LoFTR consistently outperformed other methods, particularly in 1-meter precision localization accuracy (1m-Accuracy). With a 29\% improvement over the fine-tuned model, a 53\% improvement over the pre-trained LoFTR and a 65\% improvement over SIFT, Geo-LoFTR’s integration of depth information leveraged the geometric consistency provided by DTMs, especially across altitude variations. This integration likely contributed to scale invariance, as depth data constrains matches with pixel-to-pixel depth relationships depending on DTM's elevations only, thus mitigating the effect of altitude changes.

% \subsection*{Terrain Morphology Robustness}
% Experiments across high-relief terrains and dune areas sampled from the Jezero Crater site have been conducted to assess MbL performance over different terrain morphologies. On dunal terrains, which contain repetitive patterns prone to ambiguous matches, all the methods show a decline in localization accuracy compared to rugged terrain. However, Geo-LoFTR exhibited greater robustness at 1m and 2m accuracy levels, outperforming both fine-tuned LoFTR and SIFT. This suggests that the inclusion of depth information in Geo-LoFTR helps to resolve ambiguities inherent to visual data in repetitive pattern scenarios by providing a geometric context that stabilizes matching.

% \subsection*{Lighting Variation (Sun Azimuth and Elevation)} Geo-LoFTR proved higher localization accuracy compared to other methods across a broad range of Sun elevation and azimuth angles, with the fine-tuned model performing comparably only at the 5m-Accuracy level. Specifically, Geo-LoFTR maintained a 1m-accuracy in a 78-88\% range over map Sun elevations between 10$^{\circ}$ and 90$^{\circ}$, with observations taken at 40$^{\circ}$. In contrast, the 1m-acuracy bounds are 45-50\% for fine-tuned LoFTR and 19-24\% for SIFT. MbL at challenging map elevations below 10$^{\circ}$ showed a significant decline in accuracy, but Geo-LoFTR still proved to be the most accurate overall.

% \subsection*{Martian Day Localization Performance}
% In an evaluation across a simulated Martian day on the Jezero Crater site, Geo-LoFTR outperformed other models, especially at 1m and 2m precision levels, though with slightly less stability across different times of day compared to the fine-tuned LoFTR. Geo-LoFTR struggles more than fine-tuned LoFTR for observations at 5:30 Local Mean Solar Time (LMST) corresponding to 2$^{\circ}$ in Sun elevation (sunrise is at 5:07 LMST for selected date and location), especially at lower altitudes. Nonetheless, Geo-LoFTR’s robustness to altitude variations reinforced the benefits of depth data, with significant accuracy improvements in observations from 6:00 LMST to 17:00 LMST, with sunset occurring at 17:52 LMST.
