\documentclass[conference]{IEEEtran}
\usepackage{times}

% numbers option provides compact numerical references in the text. 
\usepackage[numbers]{natbib}
\usepackage{multicol}
\usepackage[bookmarks=true]{hyperref}

% % Figures packages -------------------------------
\usepackage{graphicx}	      % allows figures
\usepackage{float}
% \usepackage{caption}
% \usepackage{subcaption}
\usepackage{xcolor}



% % Math Packages ----------------------------------
% \usepackage{amsmath}
\usepackage{newtxmath}        % font for math

% \pdfinfo{
%    /Author (Homer Simpson)
%    /Title  (Robots: Our new overlords)
%    /CreationDate (D:20101201120000)
%    /Subject (Robots)
%    /Keywords (Robots;Overlords)
% }
\newcommand{\GG}[1]{{\color{blue}{GG: #1}}}
\newcommand{\DP}[1]{{\color{green}{DP: #1}}}

\setlength{\belowcaptionskip}{0pt}

\begin{document}

% paper title
\title{Vision-based Geo-Localization of Future Mars Rotorcraft in Challenging Illumination Conditions}
%\title{Geometry-aided Map-based Localization of Future Mars Rotorcraft in Challenging Illumination Conditions}

% You will get a Paper-ID when submitting a pdf file to the conference system
%\author{Author Names Omitted for Anonymous Review. Paper-ID 636}

% \author{\authorblockN{Michael Shell}
% \authorblockA{School of Electrical and\\Computer Engineering\\
% Georgia Institute of Technology\\
% Atlanta, Georgia 30332--0250\\
% Email: mshell@ece.gatech.edu}
% \and
% \authorblockN{Homer Simpson}
% \authorblockA{Twentieth Century Fox\\
% Springfield, USA\\
% Email: homer@thesimpsons.com}
% \and
% \authorblockN{James Kirk\\ and Montgomery Scott}
% \authorblockA{Starfleet Academy\\
% San Francisco, California 96678-2391\\
% Telephone: (800) 555--1212\\
% Fax: (888) 555--1212}}

% \makeatletter
% \def\blfootnote{\xdef\@thefnmark{}\@footnotetext}
% \makeatother

% \author{
% \authorblockN{Dario Pisanti\authorrefmark{1},
% Robert Hewitt\authorrefmark{2}*,
% Roland Brockers\authorrefmark{1}
% Georgios Georgakis\authorrefmark{1}}
% \authorblockA{\authorrefmark{1}Jet Propulsion Lab, California Institute of Technology\\
% \authorblockA{\authorrefmark{2}Torc Robotics}}
% }

% \author{
% \authorblockN{Dario Pisanti\textsuperscript{1},
% Robert Hewitt\textsuperscript{2*},
% Roland Brockers\textsuperscript{1},
% Georgios Georgakis\textsuperscript{1}}
% \authorblockA{\textsuperscript{1}Jet Propulsion Lab, California Institute of Technology\\
% \authorblockA{\textsuperscript{2}Torc Robotics}
% }
% }

\author{
\authorblockN{Dario Pisanti\textsuperscript{1},
Robert Hewitt\textsuperscript{1},
Roland Brockers\textsuperscript{1},
Georgios Georgakis\textsuperscript{1}}
\authorblockA{\textsuperscript{1}Jet Propulsion Lab, California Institute of Technology\\
%\authorblockA{\textsuperscript{2}Torc Robotics}
}
}

% avoiding spaces at the end of the author lines is not a problem with
% conference papers because we don't use \thanks or \IEEEmembership


% for over three affiliations, or if they all won't fit within the width
% of the page, use this alternative format:
% 
%\author{\authorblockN{Michael Shell\authorrefmark{1},
%Homer Simpson\authorrefmark{2},
%James Kirk\authorrefmark{3}, 
%Montgomery Scott\authorrefmark{3} and
%Eldon Tyrell\authorrefmark{4}}
%\authorblockA{\authorrefmark{1}School of Electrical and Computer Engineering\\
%Georgia Institute of Technology,
%Atlanta, Georgia 30332--0250\\ Email: mshell@ece.gatech.edu}
%\authorblockA{\authorrefmark{2}Twentieth Century Fox, Springfield, USA\\
%Email: homer@thesimpsons.com}
%\authorblockA{\authorrefmark{3}Starfleet Academy, San Francisco, California 96678-2391\\
%Telephone: (800) 555--1212, Fax: (888) 555--1212}
%\authorblockA{\authorrefmark{4}Tyrell Inc., 123 Replicant Street, Los Angeles, California 90210--4321}}


\maketitle

\begin{abstract}
Planetary exploration using aerial assets has the potential for unprecedented scientific discoveries on Mars. While NASA's Mars helicopter Ingenuity proved flight in Martian atmosphere is possible, future Mars rotocrafts will require advanced navigation capabilities for long-range flights. One such critical capability is Map-based Localization (MbL) which registers an onboard image to a reference map during flight in order to mitigate cumulative drift from visual odometry. However, significant illumination differences between rotocraft observations and a reference map prove challenging for traditional MbL systems, restricting the operational window of the vehicle. 
In this work, we investigate a new MbL system and propose Geo-LoFTR, a geometry-aided deep learning model for image registration that is more robust under large illumination differences than prior models. The system is supported by a custom simulation framework that uses real orbital maps to produce large amounts of realistic images of the Martian terrain. Comprehensive evaluations show that our proposed system outperforms prior MbL efforts in terms of localization accuracy under significant lighting and scale variations. Furthermore, we demonstrate the validity of our approach across a simulated Martian day.

%The new aerial mobility dimension enabled by Ingenuity has unlocked unprecedented potential for groundbreaking science investigations in astrobiology, geology and climate on Mars. The next generation of Martian rotorcraft will need advanced navigation capabilities to conduct long-range flights over diverse and challenging terrains. Accurate geo-localization within a global reference frame is essential to mitigate cumulative drift from on-board visual odometry, ensuring precise navigation during extended traverses. Absolute visual localization can be achieved in a map-based approach by matching real-time images from the rotorcraft's navigation camera with pre-referenced orbital maps stored on-board. However, significant illumination differences between query observations and on-board maps can challenge visual geo-localization performance, restricting the mission's operation envelope to certain times of day. This work investigates deep-learning-based methods to perform robust Map-based Localization (MbL) in challenging lighting. We proposed a novel multi-modal deep-learning framework that utilizes cross-attention mechanisms to fuse visual and depth data from ortho-projected maps and Digital Terrain Models (DTMs) obtained from orbital assets, effectively leveraging geometric context to learn illumination and scale invariance. To support training and validation, we developed a custom rendering framework to generate a synthetic Mars dataset with HiRISE ortho-images and DTMs, simulating aerial observations under varying lighting and altitude. Comprehensive evaluations show that our multi-modal method improves localization accuracy and robustness to a wide range of lighting offsets between maps and observations compared to single-modality models, including both deep-learning-based and traditional methods. Additionally, the integration of depth information has been shown to provide a degree of scale invariance.
\end{abstract}

\IEEEpeerreviewmaketitle

\section{Introduction}\label{sec:intro}

In computational finance, Monte Carlo simulations are used extensively to estimate the expected value of financial payoffs based on the solution of stochastic differential equations (SDEs) which model the evolution of stock prices, interest rates, exchange rates and other quantities \cite{glasserman04}.  Monte Carlo methods are very general and flexible, but for high accuracy it requires generating a large number of costly SDE path approximations, which has motivated research into a number of variance reduction or, equivalently, cost reduction techniques. One such method is
Multilevel Monte Carlo (MLMC), which was proposed in \cite{GILES2008} and was adapted for various applications that are summarised in \cite{Giles_overview17} and successfully combined with other methods such as quasi-Monte Carlo methods. The main idea of MLMC is to approximate the payoff using different time stepping resolutions when numerically solving the underlying SDE and to generate an optimal number of samples on each level, such that the overall computational cost is minimised subject to the desired bound on the variance. %, such that the total computational cost is minimised. 
The computational savings come from the fact that most samples are computed on the coarser levels and hence are less expensive while only a few samples from the finest levels are required \cite{GILES2008}.


Among the directions in which the computational cost 
of MLMC methods could further be reduced, an important avenue is the use of lower precision calculations, especially for the first Monte Carlo levels where the targeted accuracy is relatively low. 
 An overview of the research on mixed precision for the standard Monte Carlo (MC) framework is provided in \cite{ChowMixedPrecisionStandardMC} but only a few references study the potential of low precision computation in the MLMC framework \cite{Rounding_error_oliver}. To the best of our knowledge, the only MLMC framework with customised precision in the literature is \cite{brugger2014mixed}, but they use a uniform precision for all operations on each Monte Carlo level instead of optimising 
 the precision of each intermediary variable to reduce as much as possible the cost of path generation.
 
An important motivation for an MLMC framework with variable precision would be performing the low precision computations on reconfigurable hardware devices such as Field Programmable Gate Arrays (FPGAs). FPGAs contain customizable logic blocks and connectors that make it easy to adapt the digital circuit architecture for a specific application, leading to a highly parallel and optimised implementation. Therefore they are successfully exploited in applications that require high speed and have high computational workload, such as signal processing \cite{woods2008fpga}, and real time applications like high frequency trading \cite{HFT1,HFT2}. That is why a number of previous works in hardware architecture design implemented the MLMC algorithm to price financial options using FPGAs as accelerators, which resulted in improved speed and power efficiency compared to full CPU architectures \cite{Schryver2013AMM}. The paper \cite{lindsey2016domain} also proposed 
a Domain Specific Language to automate the configuration of FPGAs for this specific application. However, only \cite{brugger2014mixed} proposed a heuristic to reduce the precision in calculations.

In addition, all aforementioned works considered that the random number generation (RNG) is performed in single or double precision. Yet in most cases an important portion of the workload in the overall MLMC simulation comes from the RNG and in \cite{brugger2014mixed} this limited the total computational savings.
To reduce the cost of MLMC simulations in particular those based on the Geometric Brownian Motion (GBM), \cite{approximateICDF_Oliver, NestedOliver} have proposed to use approximate random numbers that are generated by applying an approximation of the inverse CDF to uniform random numbers. In \cite{NestedOliver}, the authors proposed a way to integrate these lower precision random variables into a \textit{nested} MLMC framework and completed a numerical analysis to bound the resulting error at each MC level by a product of the time step and the error in the random number approximation. The same authors show in \cite{approximateICDF_Oliver} that using approximate random variables reduces the cost of path generation by a factor 7.


In this paper we propose a nested MLMC framework that combines the use of approximate random normal variables and lower precision calculations to reduce the computational cost of MLMC even further than \cite{brugger2014mixed,NestedOliver}. We illustrate the efficiency of our framework in Matlab, after making several assumptions on the cost of operations and size of the errors that we carefully justify. We focus on the case of GBM and use the approximate RNG methods presented in \cite{approximateICDF_Oliver} as well as a new slightly modified method that combines CDF inversion and the central limit theorem. To choose the precision of the variables in the low precision path generation, we introduce a novel method to optimise the bit-widths. This optimisation is performed before the main path generation loop is executed and is based on a linear model of the payoff error  
due to rounding when computing in low precision. The error model relies on algorithmic differentiation in a similar manner to \cite{unifying-bwoptim,bitwidth-AD,ADAPT}. The bit-width optimisation procedure can be performed off-line, so this stage can be excluded from the on-line time complexity of our framework. The user specified desired accuracy is then enforced by calculating on-line the number of samples that need to be generated.

In terms of hardware design, we suggest implementing the low precision path generation on FPGAs and the full-precision ones on a CPU or GPU. 
The FPGA offers enough flexibility to define a separate bit-width for every variable in the low precision path generation, and can be reconfigured periodically to update the bit-widths when the market parameters have changed considerably. 


The paper is organized as follows : \Cref{sec:MLMC} introduces MLMC and nested MLMC to make clear the estimator that is implemented in our framework. Then in \Cref{sec:RNG} we detail the methods that could be used to obtain approximate random normally distributed numbers very cheaply for the low precision path generation. In \Cref{sec:error_model} and \Cref{sec:costModel} we propose an error model and a cost model (resp.) that we then use to formulate the optimisation problem that is solved to obtain the optimal bit-widths of fixed point variables in \Cref{sec:optimisation}. Finally we summarise our results and future directions in \Cref{sec:conclusion}.



\section{Related Work}
\label{sec:related_work}

The original investigation \cite{gibson1979ecological} on the relationship between visual perception and human action defines \emph{affordance} as the opportunities for interaction with the surrounding environment. Behavioral studies on regular and cognitively impaired persons have shown evidence that perception results in both visual and motor signals in the human brain. An extended study \cite{anderson2002attentional} shows that visual attention to the spatial characteristics of the perceived objects initiates automatic motor signals for different actions. In computer vision, human affordance learning involves novel pose prediction such that the estimated pose represents a valid human action within the scene context. The task is fundamental to many problems requiring robust semantic reasoning about the environment, such as human motion synthesis \cite{wang2021scene} and scene-aware human pose generation \cite{wang2017binge, roy2016multi, zhang2022inpaint, yao2023scene}.

Earlier methods of affordance learning have explored knowledge mining \cite{zhu2014reasoning} and multimodal feature cues \cite{roy2016multi} to address the problem. In \cite{zhu2014reasoning}, the authors use a Markov Logic Network for constructing a knowledge base by extracting several object attributes from different image and metadata sources, which can perform various downstream visual inference tasks without any additional classifier, including zero-shot affordance prediction. In \cite{roy2016multi}, the authors use depth map, surface normals, and segmentation map as multimodal cues to train a multi-scale convolutional neural network (CNN) for scene-level semantic label assignment associated with specific human actions. In \cite{do2018affordancenet}, the authors design a multi-branch end-to-end CNN with two separate pathways for object detection and affordance label assignment to achieve high real-time inference throughput. Researchers \cite{chuang2018learning} have also explored socially imposed constraints for affordance learning. In \cite{chuang2018learning}, the authors propose a graph neural network (GNN) to propagate contextual scene information from egocentric views for action-object affordance reasoning.

Probabilistic modeling of scene-aware human motion generation also involves semantic reasoning of human interaction with the environment. Initial works on human motion synthesis have taken different architectural approaches, such as sequence-to-sequence models \cite{barsoum2018hp}, generative adversarial networks (GAN) \cite{barsoum2018hp, cai2018deep, yang2018pose}, graph convolutional networks (GCN) \cite{yan2019convolutional}, and variational autoencoders (VAE) \cite{guo2020action2motion}. However, these methods have mostly ignored the role of environmental semantics. Due to potential uncertainty in human motion, in a recent approach \cite{wang2021scene}, the authors address such motion synthesis with a GAN conditioned on scene attributes and motion trajectory to predict probable body pose dynamics.

One key challenge of human affordance generation in 2D scenes is the lack of large-scale datasets with rich pose annotations. In \cite{wang2017binge}, the authors compile the only public dataset of annotated human body poses in complex 2D indoor scenes by extracting frames from sitcom videos. Aiming to generate a contextually valid human affordance at a user-defined location, the authors propose sampling the scale and deformation parameters for an existing human pose template using a VAE conditioned on the localized image patches as scene context. In \cite{zhang2022inpaint}, the authors introduce a two-stage GAN architecture for achieving a similar goal by estimating the affine bounding box parameters to localize a probable human in the scene and then generating a potential body pose at that location. The method uses the input scene, corresponding depth, and segmentation maps as semantic guidance. In \cite{yao2023scene}, the authors propose a transformer-based approach with knowledge distillation for generating human affordances in 2D indoor scenes.


\section{Method}

\begin{figure*}[t]
    \centering
    \includegraphics[width=\linewidth]{figures/pipeline.png} \hfill

    \caption{An overview of our data synthesis pipeline. Starting from our seed data, we select a reference sample and collect \textsc{Reference-Level Feedback} on both the instruction and response. Instruction feedback is used to synthesize new instructions. We generate their corresponding responses, and then improve it using the response feedback.}
    \label{fig:pipeline}
\end{figure*}

In this section, we present our data synthesis pipeline that leverages \textsc{Reference-Level Feedback} to generate high-quality instruction-response pairs. An overview of the pipeline is presented in Figure \ref{fig:pipeline}, and the steps are detailed in the following subsections. Complete examples for each step can be found in Appendix \ref{sec:appendix_examples}, and the prompts used for each section can be found in Appendix \ref{sec:appendix_prompt_templates}.


\subsection{Feedback Collection}

Our pipeline begins with a seed dataset -- a small collection of carefully curated instruction-response pairs that serve as exemplars for synthesized data samples. It can be either manually crafted by human annotators or automatically selected using quality-based criteria. These reference samples are high-quality and exhibit desirable characteristics such as clarity and relevance, which we aim to replicate in our synthetic data. For \textsc{Reference-Level Feedback}, we systematically identify and capture such qualities through a framework that identifies the strength of each sample, as well as potential areas for improvement.

Unlike traditional approaches that collect feedback on generated responses at the sample-level, our method identifies the qualities that make reference samples high-quality and uses it for feedback. This feedback captures a richer signal than feedback collected at the sample-level, establishing higher quality standards for synthesis and providing more effective guidance for generating training data that exhibits similar properties to the reference samples.

For each reference sample in the seed dataset, we collect \textsc{Reference-Level Feedback} from both the instruction and the response:

\textbf{Instruction Feedback.} To collect feedback from a reference instruction and capture essential features that make it effective for training, we analyze key attributes (e.g., clarity and actionability). We also ensure comprehensive coverage along a wide breadth by collecting feedback along two dimensions: relevant subject areas (e.g. cellular biology, csv file manipulation, legislative processes) and relevant skills necessary to respond to the instruction (e.g. understanding of specific tools, knowledge of processes, analysis). This enables us to systematically identify desirable characteristics of instructions while maximizing the breadth of instruction types.

\textbf{Response Feedback.} When collecting feedback from a reference response, we identify key qualities that make it an effective response to the instruction. We evaluate along multiple critical dimensions, including factual accuracy, relevance to the instruction, and comprehensiveness. This feedback captures both the strengths of the reference response and specific areas where it can be improved upon.


\subsection{Data Synthesis}
Now, we use the collected \textsc{Reference-Level Feedback} from the previous stage to synthesize new data samples, while maintaining the quality standards established by our reference data. For each reference sample and its corresponding feedback, we employ a two-phase synthesis process, as illustrated in Figure \ref{fig:pipeline}:

\begin{enumerate}
    \item \textbf{Instruction Synthesis.} We provide an LLM the reference instruction as an example and the instruction feedback as guidelines to synthesize new instructions that maintain the qualities specified in the feedback. As depicted in Step 2 of Figure \ref{fig:pipeline}, we synthesize 10 new instructions for \textbf{subject-based} feedback, which produces instructions that align with the subject areas of the reference response. We also synthesize 10 new instructions for \textbf{skill-based} feedback, which produces instructions that align with the skills needed to respond to the reference instruction.
    
    \item \textbf{Response Synthesis and Refinement.} For each synthesized instruction, we first generate an initial response. We then enhance this response using the reference response feedback, instructing the language model to analyze the feedback and incorporate the relevant aspects. This process is shown in Step 3 of Figure \ref{fig:pipeline}.
    
    \paragraph{Note on relevance of response feedback.}
    Although the response feedback was originally collected for the reference response, many aspects of it can still remain applicable because of the shared characteristics between the reference and synthesized instructions. We acknowledge that not all feedback elements may transfer, and to account for this, we explicitly instruct the model to selectively apply only the relevant aspects of the feedback and ignore the irrelevant aspects. An example of this can be found in \ref{sec:appendix_examples}.
\end{enumerate}

This synthesis process enables us to synthesize new data, while systematically propagating the high-quality characteristics of reference samples.

\subsection{Theoretical Efficiency Analysis}
Our presented pipeline for data synthesis with \textsc{Reference-Level Feedback} is significantly more efficient than using traditional sample-level feedback methods, specifically in the frequency of feedback collection. While sample-level approaches require feedback for every synthesized sample, our method only requires feedback once for every reference sample. This translates to a reduction from $O(n)$ feedback collections, where $n$ represents the number of synthesized samples, to $O(1)$. However, it is also important to note that this efficiency gain comes with an initial fixed cost of collecting and curating seed data.
%\section{Training dataset}
\label{sec:training_set}

We generated a training image dataset comprising 17 orthographic gray-scale maps rendered from combinations of Sun azimuth ($0^{\circ}-360^{\circ}, 45^{\circ} steps$) and elevation $(30^{\circ}, 60^{\circ}, 90^{\circ}$), one corresponding orthographic depth map, and 4500 nadir-pointing aerial observations at fixed Sun angles AZ=180$^{\circ}$ and EL=40$^{\circ}$ along with their corresponding depth images. The query observations were randomly sampled from the HiRISE Jezero Crater's DTM with uniform distribution in the (x,y) coordinates in the world frame and within altitude range [$64, 200$] m. The query camera intrinsics are given by a pinhole camera model characterized by a sensor width of 80 mm, a focal length of 32 mm, 0 lens shifts along the image width and height axes, and an unitary pixel aspect ratio. Camera extrinsics, along with altitude data, were stored for each observation and served as ground truth. Further details are reported in Table \ref{tab:dataset_params}. Figure \ref{fig:map_tile_w_obs} shows gray-scale and normalized depth images of a map tile, with three sampled observations. 
Training examples for Geo-LoFTR are created by forming triplets ($I_A$, $I_B$, $I_C$) out of query observations and map windows crops (gray-scale and depth images) for multiple combinations of Sun azimuth and elevation angles' offsets between queries and the source maps. For a given combination of query and map lighting, each query observation is paired with map windows with at least 25\% area overlap on the terrain. Therefore, the set of triplets is formed by different combination of image $I_A$ with the mop window tuple ($I_B$, $I_C$). The map window sizes are carefully chosen to introduce an appropriate level of scale variance within the altitude range of the observations, ensuring a balance between model generalization and training efficiency. 
% By choosing prefixed map windows size of 1320 $\times$ 990 pixels for observations within the [64, 132] m altitude range, and a size of 2000 $\times$ 1500 pixels for the (132, 200] m, the scale ratio between query images and map windows is ensured to vary between 1 and 3.
Geo-LoFTR has been fine-tuned from the original LoFTR pre-trained model on a total 
of 150,705 generated triplets. An independent validation set of 3,400 triplets has been used to regularly assess the model performance during training and prevent over-fitting. 

\begin{figure}
    \centering
    \begin{minipage}[b]{0.9\linewidth}
        \centering
        \includegraphics[width=\linewidth]{Figures/map_tile_w_3_obs_aligned.jpg}
    \end{minipage}
    \caption{\label{fig:map_tile_w_obs} Gray-scale (left) and normalized depth (middle) images of a rendered orthographic map tile with three observations (right) sampled from different locations under the same illumination conditions.}
\end{figure}


\begin{table}
\centering
% Review this table
\begin{tabular}{l l l}
\textbf{Parameters} & \textbf{Maps} & \textbf{Observations} \\ \hline 
N.o. gray images & 17 & 4500 \\
N.o. depth images & 1 & 4500 \\
Image size & 26,949 $\times$ 57,613  & 480 $\times$ 640  \\ 
Pixel resolution & 0.25 m / px  &  [0.25, 0.78] m / px \\
Projection type  & Orthographic & Perspective \\  
Orthographic scale & 6737 m & / \\
Focal length & / & 32 mm \\
Sensor width & / & 80 mm \\
Location in $W$ & (0,0) m &  uniform random \\
&  &  distribution \\
Orientation in $W$ & nadir-pointing & nadir-pointing \\
Altitude & 4000 m & [64, 200] m\\
Sun AZ   & [0, 360]$^{\circ}$ with 45$^{\circ}$ steps & 180$^{\circ}$ \\  
Sun EL & 30$^{\circ}$, 60$^{\circ}$, 90$^{\circ}$ \\  
\end{tabular}
\caption{\label{tab:dataset_params} Parameters for the maps and observations image dataset sourced for the generation of the pairs formed by query observations and map windows, used for LoFTR and Geo-LoFTR training.}
\end{table}
\section{Map-based Localization Evaluation}
\label{sec:mbl_eval}

We evaluated the performance of our MbL pipeline on multiple image datasets generated in MARTIAN from HiRISE maps and DTMs of the Jezero Crater. To ensure an unbiased assessment, none of the test data overlaps with the data used for training. We conducted several experiments to assess the robustness of these methods under changes in lighting (Sec. \ref{subsec:mbl_robustnbess_to_az_and_el}) and scale (Sec. \ref{subsec:mbl_robustnbess_to_scale}). Figure \ref{fig:test_samples} shows areas on orthographic maps sampled from the test sets and rendered with two different illumination conditions, accompanied by three example observations at different altitudes.
To further stress our MbL pipeline with challenging lighting in a real-case scenario, we tested it over a simulated Martian day on the Jezero Crater site, with aerial observations generated in MARTIAN at multiple simulated times of day from sunrise to sunset (Sec. \ref{subsec:mbl_martian_day}). Furthermore, an evaluation of our method's performance under varying terrain morphologies is presented in the supplementary material.

We compared our results to the original LoFTR model fine-tuned on our training dataset (\textit{Fine-tuned LoFTR}), and to the model pre-trained on the MegaDepth dataset (\textit{Pre-trained LoFTR}). As for comparison with state-of-the-art feature matching in planetary aerial mobility, we also tested SIFT that proved to be one of the most accurate handcrafted methods for absolute localization over simulated Mars terrain \cite{brockers2022}.
% In the SIFT-based MbL evaluation, up to 4000 SIFT features were extracted from each query image and matched against those from the map, limited to the relevant search area. Feature matching was performed using k-nearest neighbors (KNN) with k=2. To ensure robustness, a ratio test was applied, retaining matches only if the ratio of the distance of the closest match to the distance of the second-closest match was below 0.75.
In each experiment, we use the percentage of queries with localization error $\|\mathbf{t}_{WC_{\text{query}}} - \widetilde{\mathbf{t}}_{WC_{\text{query}}}\|$ below 1m (@1m) as our evaluation metric. Also, we plot the Cumulative Distribution Function (CDF) of the localization accuracy up to 10m.
%In each experiment, we quantified the MbL accuracy as percentage of queries with localization errors $\|\mathbf{t}_{WC_{\text{query}}} - \widetilde{\mathbf{t}}_{WC_{\text{query}}}\|$. We computed the Cumulative distribution Functions (CdFs) of the localization accuracy, and show the @1m precision level as our main evaluation metric.  

\begin{figure}
    \centering
     \centering
        \includegraphics[width=\linewidth]{Figures/light_and_scale_var_example_resized.png}
    \vspace{-10pt}
    \caption{\label{fig:test_samples}Tiles from orthographic maps at sun (AZ=0°, EL=5°) (\textit{left}) (AZ=180°, EL=40°) (\textit{center}) with three sampled query observations (\textit{right}).}
\end{figure}


\subsection{Robustness to Changing Solar Angles}
\label{subsec:mbl_robustnbess_to_az_and_el}
Robustness to challenging illumination variance is assessed through registering  query observations onto orthographic maps rendered with varying sun elevation and azimuth angles, the effects of which are evaluated separately. 

The dataset for the experiment addressing the robustness to sun elevation changes comprises orthographic maps rendered at EL=$\{2°, 5°, 10°, 40°, 60°, 90°\}$ and AZ=$180°$, along with 500 nadir-pointing query observations at fixed sun angles (AZ=$180°$, EL=$40°$). The queries were taken in the 64-200 m altitude range which encompasses the nominal operation of the MSH. The minimum altitude of 64 m is set by the best resolution achievable in MARTIAN, which coincides with the 0.25 cm / pixel resolution of the HiRISE ortho-image used as texture. It is worth to note that elevations below 30$^{\circ}$ were not encountered during the model training. During the Martian day when our reference HiRISE map was collected, the sun elevation varied in the 7.9-82.6$^{\circ}$ range,  from 6:00 to 17:00 Local Mean Solar Time (LMST), with sunrise and sunset occurring at 05:11 LMST and 17:32 LMST, respectively. Therefore, elevations below 10$^{\circ}$ occupy a very brief portion of sun's local trajectory, representing exceptional cases in the Martian surface operations. Nevertheless, we include these cases in our evaluation to assess the models' ability to generalize and perform effectively under challenging lighting conditions.

To evaluate sun azimuth effect we generated 500 nadir-pointing queries with sun AZ=0$^{\circ}$ and EL=10$^{\circ}$ to be registered onto maps with varying azimuth angles in the 0-360$^{\circ}$ and same elevation as the queries.\\


\noindent \textbf{Robustness to sun elevation.} Figure \ref{fig:cdf_sun_el_var} shows the CDFs of the test observations' localization error onto maps at four different sun elevations and fixed 0 azimuth offset. Geo-LoFTR outperforms all the other methods in localization accuracy across the entire range of sun elevation offsets.
% and at all the precision levels except for the 5m-Accuracy, where its performance is comparable to the fine-tuned model.
In the case of zero sun angles' offsets, Geo-LoFTR is 87\% @1m accurate, showing an improvement of +38\% over the fine-tuned model, and +63\% over SIFT. Below 10$^{\circ}$ EL of the map, the performance of all the methods is significantly impacted, with Geo-LoFTR being 17\% @1m accurate at the very challenging case EL=2$^{\circ}$, where the pre-trained model and SIFT completely fail.
% However, Fine-tuned LoFTR shows the best failure localization rate overall, with a 5.8\% and 46.8\% rates at the low map sun elevations of 2$^{\circ}$ and 10$^{\circ}$, respectively, compared to 7.0\% and 53.6\% for Geo-LoFTR. 
%% UNCOMMENT
\begin{figure}
\centering
\begin{minipage}[b]{0.49\linewidth}
    \centering
    EL = $40^{\circ}$
    \vspace{5pt} \\
    \includegraphics[width=\linewidth]{Figures/cdf_acc_180az_40el_map_180az_40el_obs.png}
\end{minipage}
\begin{minipage}[b]{0.49\linewidth}
    \centering
    EL = $10^{\circ}$
    \vspace{5pt} \\
    \includegraphics[width=\linewidth]{Figures/cdf_acc_180az_10el_map_180az_40el_obs.png}
\end{minipage}
\\
\vspace{10pt}
\begin{minipage}[b]{0.49\linewidth}
    \centering
    EL = $5^{\circ}$
    \vspace{5pt} \\
    \includegraphics[width=\linewidth]{Figures/cdf_acc_180az_5el_map_180az_40el_obs.png}
\end{minipage}
\begin{minipage}[b]{0.49\linewidth}
    \centering
    EL = $2^{\circ}$
    \vspace{5pt} \\
    \includegraphics[width=\linewidth]{Figures/cdf_acc_180az_2el_map_180az_40el_obs.png}
\end{minipage}
\vspace{-10pt}
\caption{\label{fig:cdf_sun_el_var}Cumulative distributions of the localization error of simulated Mars observations at sun AZ=180$^{\circ}$ and EL=40$^{\circ}$, registered onto maps at four different elevation angles and 0$^{\circ}$ azimuth offset.}
\end{figure}
% \begin{figure}
% \centering
% \begin{minipage}[b]{0.4\linewidth}
%     \centering
%     \subcaption{0.5m-Accuracy}
%     \includegraphics[width=\linewidth]{Figures/loc_acc_vs_el_0.5acc-level.eps}
% \end{minipage}
% \begin{minipage}[b]{0.4\linewidth}
%     \centering
%     \subcaption{1m-Accuracy}
%     \includegraphics[width=\linewidth]{Figures/loc_acc_vs_el_1acc-level.eps}
% \end{minipage}
% \\
% \begin{minipage}[b]{0.4\linewidth}
%     \centering
%     \subcaption{2m-Accuracy}
%     \includegraphics[width=\linewidth]{Figures/loc_acc_vs_el_2acc-level.eps} 
% \end{minipage}
% \begin{minipage}[b]{0.4\linewidth}
%     \centering
%     \subcaption{5m-Accuracy}
%     \includegraphics[width=\linewidth]{Figures/loc_acc_vs_el_5acc-level.eps}
% \end{minipage}
% \caption{\label{fig:mbl_loc_acc_vs_el_vs_precision}Variation of localization accuracy with map sun elevation, for the MbL of 500 nadir-pointing observations at reference sun EL=40$^{\circ}$ (\textit{dashed line}).}
% % The localization accuracy is shown at the precision level of 0.5m (a) , 1 m (b), 2 m (c) and 5 m  (d). Queries and maps are generated from the Jezero Crater site with the MARTIAN framework, with a consistent sun azimuth angle of 180$^{\circ}$.}
% \end{figure}

%% UNCOMMENT
% \begin{figure}
% \centering
% % \makebox[0.01\linewidth]{}
% \makebox[0.47\linewidth]{Geo-LoFTR}
% \makebox[0.47\linewidth]{SIFT}
% \vspace{0.1cm} 

% \begin{minipage}[b]{0.49\linewidth}
%     \centering
%     \subcaption{map EL = $40^{\circ}$}
%     \includegraphics[width=\linewidth]{Figures/geo_map_40_180_obsv_40_180_id80_inlier_matches.png}  
% \end{minipage}
% \begin{minipage}[b]{0.49\linewidth}
%     \centering
%     \subcaption{map EL = $40^{\circ}$}
%     \includegraphics[width=\linewidth]{Figures/sift_map_40_180_obsv_40_180_id80_inlier_matches.png}
% \end{minipage}


% \begin{minipage}[b]{0.49\linewidth}
%     \centering
%     \subcaption{map EL = $10^{\circ}$}
%     \includegraphics[width=\linewidth]{Figures/geo_map_10_180_obsv_40_180_id80_inlier_matches.png}
% \end{minipage}
% \begin{minipage}[b]{0.49\linewidth}
%     \centering
%     \subcaption{map EL = $10^{\circ}$}
%     \includegraphics[width=\linewidth]{Figures/sift_map_10_180_obsv_40_180_id80_matches.png}
% \end{minipage}

% \begin{minipage}[b]{0.49\linewidth}
%     \centering
%     \subcaption{map EL = $5^{\circ}$}
%     \includegraphics[width=\linewidth]{Figures/geo_map_5_180_obsv_40_180_id80_inlier_matches.png} 
% \end{minipage}
% \begin{minipage}[b]{0.49\linewidth}
%     \centering
%     \subcaption{map EL = $5^{\circ}$}
%     \includegraphics[width=\linewidth]{Figures/sift_map_5_180_obsv_40_180_id80_matches.png} 
% \end{minipage}

% \begin{minipage}[b]{0.49\linewidth}
%     \centering
%     \subcaption{map EL = $2^{\circ}$}
%     \includegraphics[width=\linewidth]{Figures/geo_map_2_180_obsv_40_180_id80_inlier_matches.png} 
% \end{minipage}
% \begin{minipage}[b]{0.49\linewidth}
%     \centering
%     \subcaption{map EL = $2^{\circ}$}
%     \includegraphics[width=\linewidth]{Figures/sift_map_2_180_40_180_id80_matches.png} 
% \end{minipage}

% \caption{\label{fig:geo_sift_matched_keypoints_vs_el}Geo-LoFTR and SIFT matched keypoints displayed for a sample query image (\textit{left side of each panel}) with (180$^{\circ}$ AZ, 40$^{\circ}$ EL) sun angles and a map search area image (\textit{right side of each panel}) under four different sun elevations and 0$^{\circ}$ azimuth offset. Match lines are color-coded by confidence score, with redder indicating higher confidence.}
% \end{figure}


% \begin{figure*}
% \centering
% % \makebox[0.01\linewidth]{}
% \makebox[0.15\linewidth]{Geo-LoFTR}
% \makebox[0.15\linewidth]{Pre-trained LoFTR}
% \makebox[0.15\linewidth]{SIFT}
% \makebox[0.15\linewidth]{Geo-LoFTR}
% \makebox[0.15\linewidth]{Pre-trained LoFTR}
% \makebox[0.15\linewidth]{SIFT}
% \vspace{0.1cm} 

% % ROW 1
% \begin{minipage}[b]{0.16\linewidth}
%     \centering
%     \subcaption{map EL = $40^{\circ}$}
%     \includegraphics[width=\linewidth]{Figures/geo_map_40_180_obsv_40_180_id80_inlier_matches.png}  
% \end{minipage}
% \begin{minipage}[b]{0.16\linewidth}
%     \centering
%     \subcaption{map EL = $40^{\circ}$}
%     \includegraphics[width=\linewidth]{Figures/pre_map_40_180_obsv_40_180_id80_inlier_matches.png}  
% \end{minipage}
% \begin{minipage}[b]{0.16\linewidth}
%     \centering
%     \subcaption{map EL = $40^{\circ}$}
%     \includegraphics[width=\linewidth]{Figures/sift_map_40_180_obsv_40_180_id80_inlier_matches.png}
% \end{minipage}
% \begin{minipage}[b]{0.16\linewidth}
%     \centering
%     \subcaption{map AZ = $0^{\circ}$}
%     \includegraphics[width=\linewidth]{Figures/geo_map_10_0_obsv_10_0_id118_inlier_matches.png}  
% \end{minipage}
% \begin{minipage}[b]{0.16\linewidth}
%     \centering
%     \subcaption{map AZ = $0^{\circ}$}
%     \includegraphics[width=\linewidth]{Figures/pre_map_10_0_obsv_10_0_id118_inlier_matches.png}   
% \end{minipage}
% \begin{minipage}[b]{0.16\linewidth}
%     \centering
%     \subcaption{map AZ = $0^{\circ}$}
%     \includegraphics[width=\linewidth]{Figures/sift_map_10_0_obsv_10_0_id118_inlier_matches.png}  
% \end{minipage}

% % ROW 2
% \begin{minipage}[b]{0.16\linewidth}
%     \centering
%     \subcaption{map EL = $10^{\circ}$}
%     \includegraphics[width=\linewidth]{Figures/geo_map_10_180_obsv_40_180_id80_inlier_matches.png}
% \end{minipage}
% \begin{minipage}[b]{0.16\linewidth}
%     \centering
%     \subcaption{map EL = $10^{\circ}$}
%     \includegraphics[width=\linewidth]{Figures/pre_map_10_180_obsv_40_180_id80_inlier_matches.png}
% \end{minipage}
% \begin{minipage}[b]{0.16\linewidth}
%     \centering
%     \subcaption{map EL = $10^{\circ}$}
%     \includegraphics[width=\linewidth]{Figures/sift_map_10_180_obsv_40_180_id80_matches.png}
% \end{minipage}
% \begin{minipage}[b]{0.16\linewidth}
%     \centering
%     \subcaption{map AZ = $90^{\circ}$}
%     \includegraphics[width=\linewidth]{Figures/geo_map_10_90_obsv_10_0_id118_inlier_matches.png}  
% \end{minipage}
% \begin{minipage}[b]{0.16\linewidth}
%     \centering
%     \subcaption{map AZ = $90^{\circ}$}
%     \includegraphics[width=\linewidth]{Figures/pre_map_10_90_obsv_10_0_id118_inlier_matches.png}   
% \end{minipage}
% \begin{minipage}[b]{0.16\linewidth}
%     \centering
%     \subcaption{map AZ = $90^{\circ}$}
%     \includegraphics[width=\linewidth]{Figures/sift_map_10_90_obsv_10_0_id118_inlier_matches.png}  
% \end{minipage}

% % ROW 3
% \begin{minipage}[b]{0.16\linewidth}
%     \centering
%     \subcaption{map EL = $5^{\circ}$}
%     \includegraphics[width=\linewidth]{Figures/geo_map_5_180_obsv_40_180_id80_inlier_matches.png}
% \end{minipage}
% \begin{minipage}[b]{0.16\linewidth}
%     \centering
%     \subcaption{map EL = $5^{\circ}$}
%     \includegraphics[width=\linewidth]{Figures/pre_map_5_180_obsv_40_180_id80_inlier_matches.png}
% \end{minipage}
% \begin{minipage}[b]{0.16\linewidth}
%     \centering
%     \subcaption{map EL = $5^{\circ}$}
%     \includegraphics[width=\linewidth]{Figures/sift_map_5_180_obsv_40_180_id80_matches.png}
% \end{minipage}
% \begin{minipage}[b]{0.16\linewidth}
%     \centering
%     \subcaption{map AZ = $180^{\circ}$}
%     \includegraphics[width=\linewidth]{Figures/geo_map_10_180_obsv_10_0_id118_inlier_matches.png}  
% \end{minipage}
% \begin{minipage}[b]{0.16\linewidth}
%     \centering
%     \subcaption{map AZ = $180^{\circ}$}
%     \includegraphics[width=\linewidth]{Figures/pre_map_10_180_obsv_10_0_id118_inlier_matches.png}   
% \end{minipage}
% \begin{minipage}[b]{0.16\linewidth}
%     \centering
%     \subcaption{map AZ = $180^{\circ}$}
%     \includegraphics[width=\linewidth]{Figures/sift_map_10_180_obsv_10_0_id118_inlier_matches.png}  
% \end{minipage}

% % ROW 4
% \begin{minipage}[b]{0.16\linewidth}
%     \centering
%     \subcaption{map EL = $2^{\circ}$}
%     \includegraphics[width=\linewidth]{Figures/geo_map_2_180_obsv_40_180_id80_inlier_matches.png}
% \end{minipage}
% \begin{minipage}[b]{0.16\linewidth}
%     \centering
%     \subcaption{map EL = $2^{\circ}$}
%     \includegraphics[width=\linewidth]{Figures/pre_map_2_180_obsv_40_180_id80_inlier_matches.png}
% \end{minipage}
% \begin{minipage}[b]{0.16\linewidth}
%     \centering
%     \subcaption{map EL = $2^{\circ}$}
%     \includegraphics[width=\linewidth]{Figures/sift_map_2_180_40_180_id80_matches.png}
% \end{minipage}
% \begin{minipage}[b]{0.16\linewidth}
%     \centering
%     \subcaption{map AZ = $270^{\circ}$}
%     \includegraphics[width=\linewidth]{Figures/geo_map_10_270_obsv_10_0_id118_inlier_matches.png}  
% \end{minipage}
% \begin{minipage}[b]{0.16\linewidth}
%     \centering
%     \subcaption{map AZ = $270^{\circ}$}
%     \includegraphics[width=\linewidth]{Figures/pre_map_10_270_obsv_10_0_id118_inlier_matches.png}   
% \end{minipage}
% \begin{minipage}[b]{0.16\linewidth}
%     \centering
%     \subcaption{map AZ = $270^{\circ}$}
%     \includegraphics[width=\linewidth]{Figures/sift_map_10_270_obsv_10_0_id118_inlier_matches.png}  
% \end{minipage}

% \caption{\label{fig:matched_keypoints_vs_el_and_az}Each panel shows matched keypoints by Geo-LoFTR, Pre-trained LoFTR, and SIFT between a query image (\textit{left}) and a map search area (\textit{right}), with match lines color-coded by confidence (redder indicates higher confidence). The left 4x3 group uses a query with (180$^{\circ}$ AZ, 40$^{\circ}$ EL) sun angles, and maps with varying elevations ans same azimuth as the query. The right 4x3 group uses a querywith (0$^{\circ}$ AZ, 10$^{\circ}$ EL) sun angles, and maps with varying azimuths (same elevation as the query).}
% \end{figure*}


\noindent \textbf{Robustness to sun azimuth.}
The sun azimuth effect on MbL performance is presented through the cumulative distributions  (Figure \ref{fig:cdf_sun_az_var}) of the localization error for the test observations registered onto maps at four different  azimuth angles. Also in this experiment, Geo-LoFTR proved to be the most accurate model with a @1m accuracy being bound to the 54-63 \% range in the entire map sun azimuth range, despite the relatively low elevation of 10$^{\circ}$. The number and quality of the SIFT matched keypoints between query and map (Figure~\ref{fig:geo_sift_matched_keypoints_vs_az}) decreases much faster than the LoFTR-based models as we depart from the zero azimuth offset case, with failure already at 90$^{\circ}$ offset.
%Figure \ref{fig:mbl_loc_acc_vs_az_vs_precision} shows the variation of localization accuracy with the sun azimuth of the map at the precision levels of 0.5 m, 1 m, 2 m and 5 m. The azimuth angles are reported in the range [-180$^{\circ}$, 180$^{\circ}$] for visual clarity. Also in this experiment, Geo-LoFTR outperforms all the other methods, except for the 5m-Accuracy, where Fine-tuned LoFTR performs slightly better. Compared to the effect of the sun elevation variation, the accuracy of the LoFTR models trained on the Mars synthetic data results more robust to map sun azimuth offsets from the reference observation, with the 2m-Accuracy varying in a bandwidth of 8.5\% for Geo-LoFTR and 12.5\% for Fine-Tuned LoFTR. Such robustness is not observed for the pre-trained LoFTR model and SIFT, with a 2m-Accuracy varying in a bandwidth of 12\% for the former and 38.5\% for the latter.
% The matched keypoints on a sample observation and map search area are displayed for Geo-LoFTR and SIFT in Figure \ref{fig:geo_sift_matched_keypoints_vs_az}.

%% UNCOMMENT
\begin{figure}
\centering
\begin{minipage}[b]{0.49\linewidth}
    \centering
    AZ = $0^{\circ}$
    \vspace{5pt} \\
    \includegraphics[width=\linewidth]{Figures/cdf_acc_0az_10el_map_0az_10el_obs.png}
\end{minipage}
\begin{minipage}[b]{0.49\linewidth}
    \centering
    AZ = $90^{\circ}$
    \vspace{5pt} \\
    \includegraphics[width=\linewidth]{Figures/cdf_acc_90az_10el_map_0az_10el_obs.png}
\end{minipage}
\\
\vspace{10pt}
\begin{minipage}[b]{0.49\linewidth}
    \centering
    AZ = $180^{\circ}$
    \vspace{5pt} \\
    \includegraphics[width=\linewidth]{Figures/cdf_acc_180az_10el_map_0az_10el_obs.png}
\end{minipage}
\begin{minipage}[b]{0.49\linewidth}
    \centering
    AZ = $270^{\circ}$
    \vspace{5pt} \\
    \includegraphics[width=\linewidth]{Figures/cdf_acc_270az_10el_map_180az_40el_obs.png}
\end{minipage}
\vspace{-10pt}
\caption{\label{fig:cdf_sun_az_var}Cumulative distributions of the localization error of simulated Mars observations at sun AZ=0$^{\circ}$ and EL=10$^{\circ}$, registered onto maps at four different azimuth angles and 0$^{\circ}$ elevation offset.}
\end{figure}

% UNCOMMENT
\begin{figure*}
\centering
\makebox[0.3\linewidth]{\textbf{Geo-LoFTR}}
\makebox[0.3\linewidth]{\textbf{Pre-trained LoFTR}}
\makebox[0.3\linewidth]{\textbf{SIFT}}
\vspace{0.5cm} 

\begin{minipage}[b]{0.3\linewidth}
    \centering
    AZ = $0^{\circ}$
    \vspace{5pt} \\
    \includegraphics[width=\linewidth]{Figures/geo_map_10_0_obsv_10_0_id118_inlier_matches.png}  
\end{minipage}
\begin{minipage}[b]{0.3\linewidth}
    \centering
    AZ = $0^{\circ}$
    \vspace{5pt} \\
    \includegraphics[width=\linewidth]{Figures/pre_map_10_0_obsv_10_0_id118_inlier_matches.png}   
\end{minipage}
\begin{minipage}[b]{0.3\linewidth}
    \centering
    AZ = $0^{\circ}$
    \vspace{5pt} \\
    \includegraphics[width=\linewidth]{Figures/sift_map_10_0_obsv_10_0_id118_inlier_matches.png}  
\end{minipage}
\begin{minipage}[b]{0.3\linewidth}
    \centering
     \vspace{5pt} 
    AZ = $90^{\circ}$
    \vspace{5pt} \\
    \includegraphics[width=\linewidth]{Figures/geo_map_10_90_obsv_10_0_id118_inlier_matches.png}
\end{minipage}
\begin{minipage}[b]{0.3\linewidth}
    \centering
     \vspace{5pt} 
    AZ = $90^{\circ}$
    \vspace{5pt} \\
    \includegraphics[width=\linewidth]{Figures/pre_map_10_90_obsv_10_0_id118_inlier_matches.png}   
\end{minipage}
\begin{minipage}[b]{0.3\linewidth}
    \vspace{5pt} 
    \centering   
    AZ = $90^{\circ}$
    \vspace{5pt} \\
    \includegraphics[width=\linewidth]{Figures/sift_map_10_90_obsv_10_0_id118_inlier_matches.png}
\end{minipage}
\begin{minipage}[b]{0.3\linewidth}
    \centering
     \vspace{5pt} 
    AZ = $180^{\circ}$
    \vspace{5pt} 
    \includegraphics[width=\linewidth]{Figures/geo_map_10_180_obsv_10_0_id118_inlier_matches.png}
\end{minipage}
\begin{minipage}[b]{0.3\linewidth}
    \centering
     \vspace{5pt} 
    AZ = $180^{\circ}$
    \vspace{5pt} 
    \includegraphics[width=\linewidth]{Figures/pre_map_10_180_obsv_10_0_id118_inlier_matches.png}   
\end{minipage}
\begin{minipage}[b]{0.3\linewidth}
    \centering
     \vspace{5pt} 
    AZ = $180^{\circ}$
    \vspace{5pt} 
    \includegraphics[width=\linewidth]{Figures/sift_map_10_180_obsv_10_0_id118_matches.png}
\end{minipage}

\caption{\label{fig:geo_sift_matched_keypoints_vs_az}Geo-LoFTR, Pre-trained LoFTR and SIFT matched keypoints displayed for a sample query image (\textit{left side of each panel}) with (0$^{\circ}$ AZ, 10$^{\circ}$ EL) sun angles and a map search area image (\textit{right side of each panel}) under three different sun elevations and 0$^{\circ}$ azimuth offset. Match lines are color-coded by confidence score, with redder indicating higher confidence. Despite still providing a localization solution in the 0-180° AZ range, the pre-trained LoFTR matches exhibit lower confidence with azimuth changes than Geo-LoFTR, resulting in a coarser localization.} 
\end{figure*}
% \DP{include explanation about the fact that despite pre-trained LoFTR still provide accurate matches, the accuracy is much lower than Geo- and Fine-tunded LoFTR}}


% \begin{figure}
% \centering

% \makebox[0.01\linewidth]{}
% \makebox[0.3\linewidth]{64 m altitude}
% \makebox[0.3\linewidth]{100 m altitude}
% \makebox[0.3\linewidth]{200 m altitude}
% \vspace{0.1cm} % Space between azimuth labels and images

% \begin{minipage}[b]{0.01\linewidth}
%     \raisebox{2.5em}[0pt][0pt]{\makebox[0pt][r]{Geo-LoFTR}} % Adjust the raise value as needed
% \end{minipage}
% \begin{minipage}[b]{0.3\linewidth}
%     \centering
%     \includegraphics[width=\linewidth]{Figures/geo_cliff_64m_id0_inlier_matches_light_offset.png}
% \end{minipage}
% \begin{minipage}[b]{0.3\linewidth}
%     \centering
%     \includegraphics[width=\linewidth]{Figures/geo_cliff_100m_id5_inlier_matches_light_offset.png}
% \end{minipage}
% \begin{minipage}[b]{0.3\linewidth}
%     \centering
%     \includegraphics[width=\linewidth]{Figures/geo_cliff_200m_id27_inlier_matches_light_offset.png}
% \end{minipage}
    

% \begin{minipage}[b]{0.01\linewidth}
%     \raisebox{2.5em}[0pt][0pt]{\makebox[0pt][r]{SIFT}} % Adjust the raise value as needed
% \end{minipage}
% \begin{minipage}[b]{0.3\linewidth}
%     \centering
%     \includegraphics[width=\linewidth]{Figures/sift_cliff_64m_id0_inlier_matches_light_offset.png}
% \end{minipage}
% \begin{minipage}[b]{0.3\linewidth}
%     \centering
%     \includegraphics[width=\linewidth]{Figures/sift_cliff_100m_id5_inlier_matches_light_offset.png}
% \end{minipage}
% \begin{minipage}[b]{0.3\linewidth}
%     \centering
%     \includegraphics[width=\linewidth]{Figures/sift_cliff_200m_id27_inlier_matches_light_offset.png}
% \end{minipage}

% \begin{minipage}[b]{0.01\linewidth}
%     \raisebox{2.5em}[0pt][0pt]{\makebox[0pt][r]{Geo-LoFTR}} % Adjust the raise value as needed
% \end{minipage}
% \begin{minipage}[b]{0.3\linewidth}
%     \centering
%     \includegraphics[width=\linewidth]{Figures/geo_dunes_64m_id134_inlier_matches_light_offset.png}
% \end{minipage}
% \begin{minipage}[b]{0.3\linewidth}
%     \centering
%     \includegraphics[width=\linewidth]{Figures/geo_dunes_100m_id5_inlier_matches_light_offset.png}
% \end{minipage}
% \begin{minipage}[b]{0.3\linewidth}
%     \centering
%     \includegraphics[width=\linewidth]{Figures/geo_dunes_200m_id0_inlier_matches_light_offset.png}
% \end{minipage}
    
% \begin{minipage}[b]{0.01\linewidth}
%     \raisebox{2.5em}[0pt][0pt]{\makebox[0pt][r]{SIFT}} % Adjust the raise value as needed
% \end{minipage}
% \begin{minipage}[b]{0.3\linewidth}
%     \centering
%     \includegraphics[width=\linewidth]{Figures/sift_dunes_64m_id134_inlier_matches_light_offset.png}
% \end{minipage}
% \begin{minipage}[b]{0.3\linewidth}
%     \centering
%     \includegraphics[width=\linewidth]{Figures/sift_dunes_100m_id5_inlier_matches_light_offset.png}
% \end{minipage}
% \begin{minipage}[b]{0.3\linewidth}
%     \centering
%     \includegraphics[width=\linewidth]{Figures/sift_dunes_200m_id0_inlier_matches_light_offset.png}
% \end{minipage}


% \caption{\label{fig:matches_vs_alt_rugged}Matched keypoints for rugged (\textit{top}) and dunal (\textit{bottom}) between observations (\textit{left, in each panel}) and map \textit{right, in each panel}) for the Jezero Crater HiRISE map, at altitudes of 64 m (\textit{left column)}, 100 m (\textit{middle column}) and 200 m (\textit{right column}) for Geo-LoFTR and SIFT. Observations are generated in MARTIAN framework with sun AZ=180$^{\circ}$ and EL=40$^{\circ}$. The map is rendered at sun AZ=0$^{\circ}$ and EL=5$^{\circ}$.}
% \end{figure}

\subsection{Robustness to Scale Variation}
\label{subsec:mbl_robustnbess_to_scale}

We split the test observations from Sec. \ref{subsec:mbl_robustnbess_to_az_and_el} in three different altitude sub-ranges, and registered them onto maps with zero sun angle offsets to assess the pipeline's performance under scale changes. Figure \ref{fig:cdf_scale_var} shows the CDF of the localization errors of observations taken within 64m-112m, 112m-155m, 155m-200m registered on a map with constant (AZ=180°, EL=40°) sun angles. With only a -7\% @1m accuracy drop across the entire altitude range, Geo-LoFTR proved to be more robust than the fine-tuned model (-33\% @1m). A similar degree of scale invariance is shown for the pre-trained model and SIFT, although being much less accurate.  

\begin{figure}
\centering
\begin{minipage}[b]{0.49\linewidth}
    \centering
    64-112 m
    \vspace{2pt} \\
    \includegraphics[width=\linewidth]{Figures/cdf_acc_180az_40el_map_180az_40el_obs_64-112m.png}
\end{minipage}
\begin{minipage}[b]{0.49\linewidth}
    \centering
    112-155 m
    \vspace{2pt} \\
    \includegraphics[width=\linewidth]{Figures/cdf_acc_180az_40el_map_180az_40el_obs_112-155m.png}
\end{minipage}
\begin{minipage}[b]{0.49\linewidth}
    \centering
    155-200 m
    \vspace{2pt} \\
    \includegraphics[width=\linewidth]{Figures/cdf_acc_180az_40el_map_180az_40el_obs_155-200m.png}
\end{minipage}
\caption{\label{fig:cdf_scale_var}Cumulative distributions of the localization error of simulated Mars observations at sun AZ=0$^{\circ}$ and EL=10$^{\circ}$, registered onto maps with the same illumination condition for three different altitude ranges.}
\end{figure}


\subsection{Robustness to Combined Illumination and Scale Changes}
\label{subsec:mbl_robustnbess_to_scale_ill}

Leveraging the test data in Sec \ref{subsec:mbl_robustnbess_to_az_and_el}, we performed a quantitative evaluation of the scale variation effects in conjunction with sun angle offsets. Figure \ref{fig:mbl_loc_acc_vs_el_and_az_vs_alt}, shows the @1m accuracy as a function of map sun EL and AZ for observations taken within three different altitude ranges.
%and fixed sun angles in the respective cases. 
Although localization accuracy declines sharply at relatively low sun elevation angles, Geo-LoFTR maintains consistent localization performance across altitude variations within the 10–90° EL range. In contrast, the fine-tuned model demonstrates poor robustness in the same range. A similar trend is observed for azimuth variations, where localization accuracy remains stable with changing azimuth but decreases with altitude.


% Figure \ref{fig:mbl_loc_acc_vs_el_and_az_vs_alt} show the accuracy of MbL at 1 m precision against sun angles offsets in elevation and azimuth, within three different altitude sub-ranges. Geo-LoFTR results more robust to altitude variations than the other methods, suggesting the benefits introduced to localization accuracy by depth data.
% Geo-LoFTR results more robust to altitude variations than the fine-tuned model over the entire set of sun elevation offsets. Picking a map sun elevation of 10$^{\circ}$ for instance, the drop in 1m-accuracy is only of -13\% between the 156-199 m and 64-113 m altitude ranges for Geo-LoFTR, compared to a -28\% accuracy drop for Fine-tuned LoFTR. The improved scale invariance of the geometry-aided model over the other methods is confirmed also under lighting variations in terms of sun azimuth offsets, thus providing yet another piece of evidence of the benefits introduced to localization accuracy by depth data.

%% UNCOMMENT
\begin{figure}
\centering
\begin{minipage}[b]{1.0\linewidth}
    \centering
    \includegraphics[width=\linewidth]{Figures/1m-acc_loc_vs_el_vs_alt.png}    
\end{minipage}
\vspace{5pt} \\
\begin{minipage}[b]{1.0\linewidth}
    \centering
    \includegraphics[width=\linewidth]{Figures/1m-acc_loc_vs_az_vs_alt.png}    
\end{minipage}
\vspace{-5pt} \\
\caption{Localization accuracy at 1m precision as a function of map sun elevation (\textit{top}) and azimuth (\textit{bottom}) for test observations across three altitude ranges. Sun azimuth angles are in the $[-180^{\circ}, 180^{\circ}]$ range. Map sun angles matching the observations are marked with a thick black vertical line.}
\label{fig:mbl_loc_acc_vs_el_and_az_vs_alt}
\end{figure}

\subsection{Localization Over a Simulated Martian Day}
\label{subsec:mbl_martian_day}
The MbL performance is investigated for observations taken at different LMSTs during a simulated Martian day on the Jezero Crater HiRISE map at coordinates (77.44$^{\circ}$E, 18.43$^{\circ}$N). We used the Mars24~\cite{mars24} software developed by NASA Goddard Institute for Space Studies to compute the sun's local trajectory for the selected site on a given date. The chosen date, 2031-05-10, ensures that the sun zenith is at a relatively high elevation angle of 86.7$^{\circ}$, allowing a broad range of elevation angles to be observed throughout the day (Figure \ref{fig:sun_profile}). 
% where the sun azimuth is measured clockwise from North in Mars24, differently from our MARTIAN framework, where it is defined counterclockwise from East. The sun elevation is still reported with the same convention adopted in the MARTIAN.

% UNCOMMENT
\begin{figure}
\setlength{\abovecaptionskip}{0pt}  % Removes space above caption
\setlength{\belowcaptionskip}{0pt}  % Removes space below caption
\centering
\includegraphics[width=0.98\linewidth]{Figures/sun_profile.png}
\caption{Sun trajectory on a local panorama from 77.44$^{\circ}$E longitude and 18.43$^{\circ}$N latitude on Mars, on 2031-05-10, with positions shown at four Local Mean Solar Times (LMSTs). Adapted from Mars24 \cite{mars24}.}
\label{fig:sun_profile}
\end{figure}

We generated nadir-pointing observations in MARTIAN at multiple times of the day from 5:30 to 17:00 LMST, with a total of 3000 queries collected across the 64-200 m altitude range (Figure \ref{fig:obsv_lmst}). We also rendered an orthographic map at 15:00 LMST, (AZ=175.1$^{\circ}$, 39.9$^{\circ}$ EL), serving as a HiRISE-like reference. 
Figure \ref{fig:loc_acc_vs_lmst_1acc_vs_alt} shows the @1m accuracy as function of LMTs within three different altitude sub-ranges. Geo-LoFTR outperformed the other methods for most of the Martian day, except at 5:00 LMST, where the fine-tuned model shows better accuracy. However, the fine-tuned LoFTR experienced significant performance degradation with altitude, in contrast with the other methods that exhibited a certain grade of scale invariance also in this experiment.

% LoFTR trained on the Mars datasets outperformed SIFT. SIFT proves more accurate than Pre-trained LoFTR in times of day between 8:00 and 15:00 LMST. However, its accuracy rapidly declines for observations taken earlier than 8:00 LMST underperforming the base LoFTR model, especially at lower altitudes. Geo-LoFTR results in being the best localization method at 1 m and 2 m precision levels for most of the Martian day, except when observations are taken at 5:30 LMST, where its performance significantly degrades. Despite the overall higher accuracy, Geo-LoFTR is also observed to be less stable than the fine-tuned model throughout the Martian day. However, it is more robust to altitude changes than its visual counterpart, providing yet another evidence of the contribution provided by geometric context in improving scale invariance. 
% UNCOMMENT
\begin{figure}
\centering

\makebox[0.01\linewidth]{}
% \makebox[0.25\linewidth]{\small 64 m}
% \makebox[0.25\linewidth]{\small 100 m}
% \makebox[0.25\linewidth]{\small 200 m}
% \vspace{0.1cm} % Space between azimuth labels and images

% \begin{minipage}[b]{0.01\linewidth}
%     \raisebox{3.5em}[0pt][0pt]{\makebox[0pt][r]{\small LMST:}} % Adjust the raise value as needed
% \end{minipage}
\begin{minipage}[b]{0.01\linewidth}
    \raisebox{2em}[0pt][0pt]{\makebox[0pt][r]{\small \shortstack{\textbf{LMST}: \\ 05:30}}} 
\end{minipage}
\begin{minipage}[b]{0.25\linewidth}
    \centering
    \includegraphics[width=\linewidth]{Figures/05_30_0479.png}
\end{minipage}
\begin{minipage}[b]{0.25\linewidth}
    \centering
    \includegraphics[width=\linewidth]{Figures/05_30_0460.png}
\end{minipage}
\begin{minipage}[b]{0.25\linewidth}
    \centering
    \includegraphics[width=\linewidth]{Figures/05_30_0443.png}
\end{minipage}

\begin{minipage}[b]{0.01\linewidth}
    \raisebox{2em}[0pt][0pt]{\makebox[0pt][r]{\small 06:00
    }} % Adjust the raise value as needed
\end{minipage}
\begin{minipage}[b]{0.25\linewidth}
    \centering
    \includegraphics[width=\linewidth]{Figures/06_00_0018.png}
\end{minipage}
\begin{minipage}[b]{0.25\linewidth}
    \centering
    \includegraphics[width=\linewidth]{Figures/06_00_0040.png}
\end{minipage}
\begin{minipage}[b]{0.25\linewidth}
    \centering
    \includegraphics[width=\linewidth]{Figures/06_00_0046.png}
\end{minipage}

\begin{minipage}[b]{0.01\linewidth}
    \raisebox{2em}[0pt][0pt]{\makebox[0pt][r]{\small 08:00
    }} % Adjust the raise value as needed
\end{minipage}
\begin{minipage}[b]{0.25\linewidth}
    \centering
    \includegraphics[width=\linewidth]{Figures/08_00_0000.png}
\end{minipage}
\begin{minipage}[b]{0.25\linewidth}
    \centering
    \includegraphics[width=\linewidth]{Figures/08_00_0005.png}
\end{minipage}
\begin{minipage}[b]{0.25\linewidth}
    \centering
    \includegraphics[width=\linewidth]{Figures/08_00_0023.png}
\end{minipage}

\begin{minipage}[b]{0.01\linewidth}
    \raisebox{2em}[0pt][0pt]{\makebox[0pt][r]{\small11:29
    }} % Adjust the raise value as needed
\end{minipage}
\begin{minipage}[b]{0.25\linewidth}
    \centering
    \includegraphics[width=\linewidth]{Figures/11_29_0000.png}
\end{minipage}
\begin{minipage}[b]{0.25\linewidth}
    \centering
    \includegraphics[width=\linewidth]{Figures/11_29_0002.png}
\end{minipage}
\begin{minipage}[b]{0.25\linewidth}
    \centering
    \includegraphics[width=\linewidth]{Figures/11_29_0006.png}
\end{minipage}

\begin{minipage}[b]{0.01\linewidth}
    \raisebox{2em}[0pt][0pt]{\makebox[0pt][r]{\small15:00
    }} % Adjust the raise value as needed
\end{minipage}
\begin{minipage}[b]{0.25\linewidth}
    \centering
    \includegraphics[width=\linewidth]{Figures/15_00_0004.png}
\end{minipage}
\begin{minipage}[b]{0.25\linewidth}
    \centering
    \includegraphics[width=\linewidth]{Figures/15_00_0008.png}
\end{minipage}
\begin{minipage}[b]{0.25\linewidth}
    \centering
    \includegraphics[width=\linewidth]{Figures/15_00_0007.png}
\end{minipage}

\begin{minipage}[b]{0.0255\linewidth}
    \raisebox{2em}[0pt][0pt]{\makebox[0pt][r]{\small17:00
    }} % Adjust the raise value as needed
\end{minipage}
\begin{minipage}[b]{0.25\linewidth}
    \centering
    \includegraphics[width=\linewidth]{Figures/17_00_0022.png}
\end{minipage}
\begin{minipage}[b]{0.25\linewidth}
    \centering
    \includegraphics[width=\linewidth]{Figures/17_00_0042.png}
\end{minipage}
\begin{minipage}[b]{0.25\linewidth}
    \centering
    \includegraphics[width=\linewidth]{Figures/17_00_0082.png}
\end{minipage}
\caption{Sample nadir-pointing observations rendered at different Local Mean Solar Times (LMSTs) taken on Mars on 2031-05-10. The reference HiRISE map is taken at 15:00 LMST. The sun Zenith is at 11:29 LMST.}
\label{fig:obsv_lmst}
% Data are generated from Jezero Crater using the MARTIAN framework. }
\end{figure}


% \begin{table}
% \centering
% \begin{tabular}{l c c}
% \textbf{LMST} & \textbf{sun EL} & \textbf{sun AZ} \\
% \hline
% 05:07 (rise)   & -0.2$^{\circ}$ & 16.6$^{\circ}$\\
% 05:30         & 5.0$^{\circ}$ & 14.8$^{\circ}$\\
% 06:00         & 12.1$^{\circ}$ & 12.6$^{\circ}$\\
% 08:00         & 40.1$^{\circ}$ & 5.0$^{\circ}$\\
% 11:29 (zenith) & 86.9$^{\circ}$ & 270.3$^{\circ}$\\
% 15:00 (HiRISE) & 39.9$^{\circ}$ & 175.1$^{\circ}$\\
% 17:00         & 11.6$^{\circ}$ & 167.4$^{\circ}$\\
% 17:52 (set)   &  -0.2$^{\circ}$ & 163.5$^{\circ}$\\
%   \hline \\
% \end{tabular}
% \caption{\label{tab:lmst_solar_profile}Specifications of points of interests on the solar local trajectory from (77.44$^{\circ}$E, 18.43$^{\circ}$N) coordinates on Mars, on 2031-05-10, with sun azimuth and elevation angles referred to the MARTIAN framework. Adapted from Mars24\cite{mars24}.} 
% % sun azimuth is measured clockwise from North in Mars24 framework; while it is defined counterclockwise from East in the MARTAIAN framework. Adapted from Mars24\cite{mars24}.}
% \end{table}

% \begin{figure}
% \centering
% \begin{minipage}[b]{0.3\linewidth}
%     \centering
%     \subcaption{5:30 LMST}
%     \includegraphics[width=\linewidth]{Figures/CDF_loc_acc_map_HiRISE_obsv_morning_5_30.eps}
% \end{minipage}
% \begin{minipage}[b]{0.3\linewidth}
%     \centering
%     \subcaption{6:00 LMST}
%     \includegraphics[width=\linewidth]{Figures/CDF_loc_acc_map_HiRISE_obsv_morning_6_00.eps}
% \end{minipage}
% \begin{minipage}[b]{0.3\linewidth}
%     \centering
%     \subcaption{8:00 LMST}
%     \includegraphics[width=\linewidth]{Figures/CDF_loc_acc_map_HiRISE_obsv_morning.eps}
% \end{minipage}
% \begin{minipage}[b]{0.3\linewidth}
%     \centering
%     \subcaption{11:29 LMST}
%     \includegraphics[width=\linewidth]{Figures/CDF_loc_acc_map_HiRISE_obsv_zenith.eps}
% \end{minipage}
% \begin{minipage}[b]{0.3\linewidth}
%     \centering
%     \subcaption{15:00 LMST}
%     \includegraphics[width=\linewidth]{Figures/CDF_loc_acc_map_40_180_obsv_40_180.eps} 
% \end{minipage}
% \begin{minipage}[b]{0.3\linewidth}
%     \centering
%     \subcaption{17:00 LMST}
%     \includegraphics[width=\linewidth]{Figures/CDF_loc_acc_map_HiRISE_obsv_set.eps}
% \end{minipage}

% \caption{\label{fig:mbl_cdf_martian_day}Cumulative distributions of the localization error for the MbL of 500 nadir-pointing observations (altitude 64-200m) taken at six different Local Mean Solar Times (LMSTs) on Mars, on 2031-05-10. The reference HiRISE map is taken at 15:00 LMST (c). The sun Zenith is at 11:29 LMST (b). Data are generated from Jezero Crater using the MARTIAN framework.}
% \end{figure}

% UNCOMMENT
\begin{figure}
\centering
\includegraphics[width=1\linewidth]{Figures/1m-acc_loc_vs_LMST_vs_alt_zoom.png} 
\caption{Localization accuracy (@1m) as a function of Local Mean Solar Time (LMST) of simulated test observations from the Jezero Crater on 2031-05-10 across the 64-200 m altitude range. The reference HiRISE-like map is taken at 15:00 LMST (\textit{dashed black line}). The sun Zenith is at 11:29 LMST.}
\vspace{-10pt}
\label{fig:loc_acc_vs_lmst_1acc_vs_alt}
\end{figure}
% Analyzing the 1m-accuracy, the LoFTR trained on the Mars datasets outperformed the pre-trained LoFTR model and SIFT, which performed in the 20-30\% accuracy range. SIFT proves more accurate than Pre-trained LoFTR in times of day between 8:00 LMST and 15:00 LMST. However, its accuracy rapidly declines for observations taken earlier than 8:00 LMST underperforming the base LoFTR model, especially at lower altitudes. Geo-LoFTR results in being the best localization method at 1 m and 2 m precision levels for most of the Martian day, except when observations are taken at 5:30 LMST, where its performance significantly degrades. Despite the overall higher accuracy, Geo-LoFTR is also observed to be less stable than the fine-tuned model throughout the Martian day. However, it is confirmed again to be more robust to altitude changes than its visual counterpart, providing yet another evidence of the contribution provided by geometric context in achieving scale invariance. 

\subsection{Discussion}
\label{subsec:discussion}
Geo-LoFTR demonstrated superior localization accuracy compared to other methods across a broad range of illumination conditions, indicating that incorporating depth information can mitigate degeneracies inherent to purely visual data. Robustness to sun elevation is maintained within a wide range of angles, except in extremely challenging cases (e.g., EL = 2°), where poor lighting and extensive shadow coverage might saturate the constraining power of the geometric information, leading to a rapid decline in localization accuracy. More stable is the behavior for changes in azimuth.
Geo-LoFTR also showed greater robustness to changing observation altitude than the fine-tuned model across multiple experiments, suggesting that providing a geometric context contributes to scale invariance. A possible explanation is that depth data constrains matches by providing consistent pixel-to-pixel depth relationships across altitudes, reflecting terrain elevation alone. This added layer of geometric consistency likely enhances Geo-LoFTR's ability to learn accurate matches by reducing ambiguity from appearance-based features alone.


\section{Limitations}

Our method imposes certain constraints on its applicability to existing decoder-only large language models (LLMs) due to its reliance on parallel encoding/decoding capabilities during the pre-filling stage. This requirement limits its direct adoption in conventional autoregressive LLMs. However, it is worth noting that many high-performance language models with parallel encoding/decoding capabilities have already become standard choices in various Retrieval-Augmented Generation (RAG) systems, such as FiD~\cite{DBLP:conf/eacl/IzacardG21}, CEPE~\cite{DBLP:conf/acl/YenG024}, and Parallel Windows~\cite{DBLP:conf/acl/RatnerLBRMAKSLS23}. Furthermore, our approach requires such models only during the reranker training phase; once trained, the reranker itself is independent of any specific LLM and can be flexibly adapted to other decoder-only models. Therefore, our method primarily serves as a general training framework rather than imposing architectural constraints on the final inference model. Additionally, our approach introduces extra hyperparameters in the Gumbel-Softmax process, including the temperature parameter $\tau$ and the scaling factor $\kappa$, which require tuning to achieve optimal performance. However, through empirical studies, we find that $\tau=0.5$ and $\kappa=1.0$ provide robust and stable performance across different model architectures and datasets. We provide a further discussion on the effect of $\tau$ and $\kappa$ in \autoref{sec: Effect of hyper-parameters on the Training Process}.

\section{Ethical Considerations}
While our method aims to improve the accuracy of the RAG system, it does not eliminate the inherent risks of biased data or model outputs, as the performance of RAG systems still heavily depends on the quality of training data and underlying models. The potential for bias in the training data, particularly for domain-specific queries, can lead to the amplification of these biases in the retrieved results, which can impact downstream applications.
\section{Conclusion and future directions} \label{sec:conclusion}

In this paper we proposed a nested MLMC framework that offers important computational savings by performing most calculations in low precision and exploiting approximate random normal variables for the low precision path calculations. The low precision calculations could be performed in fixed precision on an FPGA for greater efficiency, and we suggested a procedure to optimise the bit-widths of every variable at each Monte Carlo level. This is an important improvement over previous mixed precision MLMC frameworks which held the lower precision fixed \cite{Rounding_error_oliver} or defined uniform bit-width at every level heuristically \cite{brugger2014mixed}. Our numerical results suggest that for the first levels our procedure reduces the cost at these levels by a factor 5 or 7. Hence the overall savings are significant since most paths are calculated on the first levels. Our approach would be even more efficient for the Milstein scheme because its higher order strong convergence leads to a greater proportion of the computational costs being on the coarsest levels.

The next stage of the research project will be to implement the RNG methods and the nested framework on FPGAs to determine the hardware requirements and confirm the extent of the computational savings. It would also be good to compare the performance benefits to using half-precision floating point arithmetic on GPUs or CPUs for the low-accuracy computations.



\section*{Acknowledgments}
{\textcopyright}2025 All rights reserved. The research described in this paper was carried out at the Jet Propulsion Laboratory, California Institute of Technology, under a contract with the National Aeronautics and Space Administration (80NM0018D0004).

% \section*{Acknowledgments}
% The research was carried out at the Jet Propulsion Laboratory, California Institute of Technology, under a contract with the National Aeronautics and Space Administration (80NM0018D0004). For this review version: \textcopyright2025. All rights reserved.

%% Use plainnat to work nicely with natbib. 

\bibliographystyle{plainnat}
%\bibliography{references}
\bibliography{paper}

\end{document}


