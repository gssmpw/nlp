\documentclass[conference]{IEEEtran}
\usepackage{times}

% numbers option provides compact numerical references in the text. 
\usepackage[numbers]{natbib}
\usepackage{multicol}
\usepackage[bookmarks=true]{hyperref}

% % Figures packages -------------------------------
\usepackage{graphicx}	      % allows figures
\usepackage{float}
% \usepackage{caption}
% \usepackage{subcaption}
\usepackage{xcolor}



% % Math Packages ----------------------------------
% \usepackage{amsmath}
\usepackage{newtxmath}        % font for math

% \pdfinfo{
%    /Author (Homer Simpson)
%    /Title  (Robots: Our new overlords)
%    /CreationDate (D:20101201120000)
%    /Subject (Robots)
%    /Keywords (Robots;Overlords)
% }
\newcommand{\GG}[1]{{\color{blue}{GG: #1}}}
\newcommand{\DP}[1]{{\color{green}{DP: #1}}}

\setlength{\belowcaptionskip}{0pt}

\begin{document}

% paper title
\title{Vision-based Geo-Localization of Future Mars Rotorcraft in Challenging Illumination Conditions}
%\title{Geometry-aided Map-based Localization of Future Mars Rotorcraft in Challenging Illumination Conditions}

% You will get a Paper-ID when submitting a pdf file to the conference system
%\author{Author Names Omitted for Anonymous Review. Paper-ID 636}

% \author{\authorblockN{Michael Shell}
% \authorblockA{School of Electrical and\\Computer Engineering\\
% Georgia Institute of Technology\\
% Atlanta, Georgia 30332--0250\\
% Email: mshell@ece.gatech.edu}
% \and
% \authorblockN{Homer Simpson}
% \authorblockA{Twentieth Century Fox\\
% Springfield, USA\\
% Email: homer@thesimpsons.com}
% \and
% \authorblockN{James Kirk\\ and Montgomery Scott}
% \authorblockA{Starfleet Academy\\
% San Francisco, California 96678-2391\\
% Telephone: (800) 555--1212\\
% Fax: (888) 555--1212}}

% \makeatletter
% \def\blfootnote{\xdef\@thefnmark{}\@footnotetext}
% \makeatother

% \author{
% \authorblockN{Dario Pisanti\authorrefmark{1},
% Robert Hewitt\authorrefmark{2}*,
% Roland Brockers\authorrefmark{1}
% Georgios Georgakis\authorrefmark{1}}
% \authorblockA{\authorrefmark{1}Jet Propulsion Lab, California Institute of Technology\\
% \authorblockA{\authorrefmark{2}Torc Robotics}}
% }

% \author{
% \authorblockN{Dario Pisanti\textsuperscript{1},
% Robert Hewitt\textsuperscript{2*},
% Roland Brockers\textsuperscript{1},
% Georgios Georgakis\textsuperscript{1}}
% \authorblockA{\textsuperscript{1}Jet Propulsion Lab, California Institute of Technology\\
% \authorblockA{\textsuperscript{2}Torc Robotics}
% }
% }

\author{
\authorblockN{Dario Pisanti\textsuperscript{1},
Robert Hewitt\textsuperscript{1},
Roland Brockers\textsuperscript{1},
Georgios Georgakis\textsuperscript{1}}
\authorblockA{\textsuperscript{1}Jet Propulsion Lab, California Institute of Technology\\
%\authorblockA{\textsuperscript{2}Torc Robotics}
}
}

% avoiding spaces at the end of the author lines is not a problem with
% conference papers because we don't use \thanks or \IEEEmembership


% for over three affiliations, or if they all won't fit within the width
% of the page, use this alternative format:
% 
%\author{\authorblockN{Michael Shell\authorrefmark{1},
%Homer Simpson\authorrefmark{2},
%James Kirk\authorrefmark{3}, 
%Montgomery Scott\authorrefmark{3} and
%Eldon Tyrell\authorrefmark{4}}
%\authorblockA{\authorrefmark{1}School of Electrical and Computer Engineering\\
%Georgia Institute of Technology,
%Atlanta, Georgia 30332--0250\\ Email: mshell@ece.gatech.edu}
%\authorblockA{\authorrefmark{2}Twentieth Century Fox, Springfield, USA\\
%Email: homer@thesimpsons.com}
%\authorblockA{\authorrefmark{3}Starfleet Academy, San Francisco, California 96678-2391\\
%Telephone: (800) 555--1212, Fax: (888) 555--1212}
%\authorblockA{\authorrefmark{4}Tyrell Inc., 123 Replicant Street, Los Angeles, California 90210--4321}}


\maketitle

\begin{abstract}
Planetary exploration using aerial assets has the potential for unprecedented scientific discoveries on Mars. While NASA's Mars helicopter Ingenuity proved flight in Martian atmosphere is possible, future Mars rotocrafts will require advanced navigation capabilities for long-range flights. One such critical capability is Map-based Localization (MbL) which registers an onboard image to a reference map during flight in order to mitigate cumulative drift from visual odometry. However, significant illumination differences between rotocraft observations and a reference map prove challenging for traditional MbL systems, restricting the operational window of the vehicle. 
In this work, we investigate a new MbL system and propose Geo-LoFTR, a geometry-aided deep learning model for image registration that is more robust under large illumination differences than prior models. The system is supported by a custom simulation framework that uses real orbital maps to produce large amounts of realistic images of the Martian terrain. Comprehensive evaluations show that our proposed system outperforms prior MbL efforts in terms of localization accuracy under significant lighting and scale variations. Furthermore, we demonstrate the validity of our approach across a simulated Martian day.

%The new aerial mobility dimension enabled by Ingenuity has unlocked unprecedented potential for groundbreaking science investigations in astrobiology, geology and climate on Mars. The next generation of Martian rotorcraft will need advanced navigation capabilities to conduct long-range flights over diverse and challenging terrains. Accurate geo-localization within a global reference frame is essential to mitigate cumulative drift from on-board visual odometry, ensuring precise navigation during extended traverses. Absolute visual localization can be achieved in a map-based approach by matching real-time images from the rotorcraft's navigation camera with pre-referenced orbital maps stored on-board. However, significant illumination differences between query observations and on-board maps can challenge visual geo-localization performance, restricting the mission's operation envelope to certain times of day. This work investigates deep-learning-based methods to perform robust Map-based Localization (MbL) in challenging lighting. We proposed a novel multi-modal deep-learning framework that utilizes cross-attention mechanisms to fuse visual and depth data from ortho-projected maps and Digital Terrain Models (DTMs) obtained from orbital assets, effectively leveraging geometric context to learn illumination and scale invariance. To support training and validation, we developed a custom rendering framework to generate a synthetic Mars dataset with HiRISE ortho-images and DTMs, simulating aerial observations under varying lighting and altitude. Comprehensive evaluations show that our multi-modal method improves localization accuracy and robustness to a wide range of lighting offsets between maps and observations compared to single-modality models, including both deep-learning-based and traditional methods. Additionally, the integration of depth information has been shown to provide a degree of scale invariance.
\end{abstract}

\IEEEpeerreviewmaketitle

\section{Introduction}

Video generation has garnered significant attention owing to its transformative potential across a wide range of applications, such media content creation~\citep{polyak2024movie}, advertising~\citep{zhang2024virbo,bacher2021advert}, video games~\citep{yang2024playable,valevski2024diffusion, oasis2024}, and world model simulators~\citep{ha2018world, videoworldsimulators2024, agarwal2025cosmos}. Benefiting from advanced generative algorithms~\citep{goodfellow2014generative, ho2020denoising, liu2023flow, lipman2023flow}, scalable model architectures~\citep{vaswani2017attention, peebles2023scalable}, vast amounts of internet-sourced data~\citep{chen2024panda, nan2024openvid, ju2024miradata}, and ongoing expansion of computing capabilities~\citep{nvidia2022h100, nvidia2023dgxgh200, nvidia2024h200nvl}, remarkable advancements have been achieved in the field of video generation~\citep{ho2022video, ho2022imagen, singer2023makeavideo, blattmann2023align, videoworldsimulators2024, kuaishou2024klingai, yang2024cogvideox, jin2024pyramidal, polyak2024movie, kong2024hunyuanvideo, ji2024prompt}.


In this work, we present \textbf{\ours}, a family of rectified flow~\citep{lipman2023flow, liu2023flow} transformer models designed for joint image and video generation, establishing a pathway toward industry-grade performance. This report centers on four key components: data curation, model architecture design, flow formulation, and training infrastructure optimization—each rigorously refined to meet the demands of high-quality, large-scale video generation.


\begin{figure}[ht]
    \centering
    \begin{subfigure}[b]{0.82\linewidth}
        \centering
        \includegraphics[width=\linewidth]{figures/t2i_1024.pdf}
        \caption{Text-to-Image Samples}\label{fig:main-demo-t2i}
    \end{subfigure}
    \vfill
    \begin{subfigure}[b]{0.82\linewidth}
        \centering
        \includegraphics[width=\linewidth]{figures/t2v_samples.pdf}
        \caption{Text-to-Video Samples}\label{fig:main-demo-t2v}
    \end{subfigure}
\caption{\textbf{Generated samples from \ours.} Key components are highlighted in \textcolor{red}{\textbf{RED}}.}\label{fig:main-demo}
\end{figure}


First, we present a comprehensive data processing pipeline designed to construct large-scale, high-quality image and video-text datasets. The pipeline integrates multiple advanced techniques, including video and image filtering based on aesthetic scores, OCR-driven content analysis, and subjective evaluations, to ensure exceptional visual and contextual quality. Furthermore, we employ multimodal large language models~(MLLMs)~\citep{yuan2025tarsier2} to generate dense and contextually aligned captions, which are subsequently refined using an additional large language model~(LLM)~\citep{yang2024qwen2} to enhance their accuracy, fluency, and descriptive richness. As a result, we have curated a robust training dataset comprising approximately 36M video-text pairs and 160M image-text pairs, which are proven sufficient for training industry-level generative models.

Secondly, we take a pioneering step by applying rectified flow formulation~\citep{lipman2023flow} for joint image and video generation, implemented through the \ours model family, which comprises Transformer architectures with 2B and 8B parameters. At its core, the \ours framework employs a 3D joint image-video variational autoencoder (VAE) to compress image and video inputs into a shared latent space, facilitating unified representation. This shared latent space is coupled with a full-attention~\citep{vaswani2017attention} mechanism, enabling seamless joint training of image and video. This architecture delivers high-quality, coherent outputs across both images and videos, establishing a unified framework for visual generation tasks.


Furthermore, to support the training of \ours at scale, we have developed a robust infrastructure tailored for large-scale model training. Our approach incorporates advanced parallelism strategies~\citep{jacobs2023deepspeed, pytorch_fsdp} to manage memory efficiently during long-context training. Additionally, we employ ByteCheckpoint~\citep{wan2024bytecheckpoint} for high-performance checkpointing and integrate fault-tolerant mechanisms from MegaScale~\citep{jiang2024megascale} to ensure stability and scalability across large GPU clusters. These optimizations enable \ours to handle the computational and data challenges of generative modeling with exceptional efficiency and reliability.


We evaluate \ours on both text-to-image and text-to-video benchmarks to highlight its competitive advantages. For text-to-image generation, \ours-T2I demonstrates strong performance across multiple benchmarks, including T2I-CompBench~\citep{huang2023t2i-compbench}, GenEval~\citep{ghosh2024geneval}, and DPG-Bench~\citep{hu2024ella_dbgbench}, excelling in both visual quality and text-image alignment. In text-to-video benchmarks, \ours-T2V achieves state-of-the-art performance on the UCF-101~\citep{ucf101} zero-shot generation task. Additionally, \ours-T2V attains an impressive score of \textbf{84.85} on VBench~\citep{huang2024vbench}, securing the top position on the leaderboard (as of 2025-01-25) and surpassing several leading commercial text-to-video models. Qualitative results, illustrated in \Cref{fig:main-demo}, further demonstrate the superior quality of the generated media samples. These findings underscore \ours's effectiveness in multi-modal generation and its potential as a high-performing solution for both research and commercial applications.
\section{Related Work}

\subsection{Large 3D Reconstruction Models}
Recently, generalized feed-forward models for 3D reconstruction from sparse input views have garnered considerable attention due to their applicability in heavily under-constrained scenarios. The Large Reconstruction Model (LRM)~\cite{hong2023lrm} uses a transformer-based encoder-decoder pipeline to infer a NeRF reconstruction from just a single image. Newer iterations have shifted the focus towards generating 3D Gaussian representations from four input images~\cite{tang2025lgm, xu2024grm, zhang2025gslrm, charatan2024pixelsplat, chen2025mvsplat, liu2025mvsgaussian}, showing remarkable novel view synthesis results. The paradigm of transformer-based sparse 3D reconstruction has also successfully been applied to lifting monocular videos to 4D~\cite{ren2024l4gm}. \\
Yet, none of the existing works in the domain have studied the use-case of inferring \textit{animatable} 3D representations from sparse input images, which is the focus of our work. To this end, we build on top of the Large Gaussian Reconstruction Model (GRM)~\cite{xu2024grm}.

\subsection{3D-aware Portrait Animation}
A different line of work focuses on animating portraits in a 3D-aware manner.
MegaPortraits~\cite{drobyshev2022megaportraits} builds a 3D Volume given a source and driving image, and renders the animated source actor via orthographic projection with subsequent 2D neural rendering.
3D morphable models (3DMMs)~\cite{blanz19993dmm} are extensively used to obtain more interpretable control over the portrait animation. For example, StyleRig~\cite{tewari2020stylerig} demonstrates how a 3DMM can be used to control the data generated from a pre-trained StyleGAN~\cite{karras2019stylegan} network. ROME~\cite{khakhulin2022rome} predicts vertex offsets and texture of a FLAME~\cite{li2017flame} mesh from the input image.
A TriPlane representation is inferred and animated via FLAME~\cite{li2017flame} in multiple methods like Portrait4D~\cite{deng2024portrait4d}, Portrait4D-v2~\cite{deng2024portrait4dv2}, and GPAvatar~\cite{chu2024gpavatar}.
Others, such as VOODOO 3D~\cite{tran2024voodoo3d} and VOODOO XP~\cite{tran2024voodooxp}, learn their own expression encoder to drive the source person in a more detailed manner. \\
All of the aforementioned methods require nothing more than a single image of a person to animate it. This allows them to train on large monocular video datasets to infer a very generic motion prior that even translates to paintings or cartoon characters. However, due to their task formulation, these methods mostly focus on image synthesis from a frontal camera, often trading 3D consistency for better image quality by using 2D screen-space neural renderers. In contrast, our work aims to produce a truthful and complete 3D avatar representation from the input images that can be viewed from any angle.  

\subsection{Photo-realistic 3D Face Models}
The increasing availability of large-scale multi-view face datasets~\cite{kirschstein2023nersemble, ava256, pan2024renderme360, yang2020facescape} has enabled building photo-realistic 3D face models that learn a detailed prior over both geometry and appearance of human faces. HeadNeRF~\cite{hong2022headnerf} conditions a Neural Radiance Field (NeRF)~\cite{mildenhall2021nerf} on identity, expression, albedo, and illumination codes. VRMM~\cite{yang2024vrmm} builds a high-quality and relightable 3D face model using volumetric primitives~\cite{lombardi2021mvp}. One2Avatar~\cite{yu2024one2avatar} extends a 3DMM by anchoring a radiance field to its surface. More recently, GPHM~\cite{xu2025gphm} and HeadGAP~\cite{zheng2024headgap} have adopted 3D Gaussians to build a photo-realistic 3D face model. \\
Photo-realistic 3D face models learn a powerful prior over human facial appearance and geometry, which can be fitted to a single or multiple images of a person, effectively inferring a 3D head avatar. However, the fitting procedure itself is non-trivial and often requires expensive test-time optimization, impeding casual use-cases on consumer-grade devices. While this limitation may be circumvented by learning a generalized encoder that maps images into the 3D face model's latent space, another fundamental limitation remains. Even with more multi-view face datasets being published, the number of available training subjects rarely exceeds the thousands, making it hard to truly learn the full distibution of human facial appearance. Instead, our approach avoids generalizing over the identity axis by conditioning on some images of a person, and only generalizes over the expression axis for which plenty of data is available. 

A similar motivation has inspired recent work on codec avatars where a generalized network infers an animatable 3D representation given a registered mesh of a person~\cite{cao2022authentic, li2024uravatar}.
The resulting avatars exhibit excellent quality at the cost of several minutes of video capture per subject and expensive test-time optimization.
For example, URAvatar~\cite{li2024uravatar} finetunes their network on the given video recording for 3 hours on 8 A100 GPUs, making inference on consumer-grade devices impossible. In contrast, our approach directly regresses the final 3D head avatar from just four input images without the need for expensive test-time fine-tuning.


\section{RoleMRC}
\label{sec:method}

In this section, we build RoleMRC. Figure\,\ref{fig:method} illustrates the overall pipeline of RoleMRC from top to bottom, which is divided into three steps.

\subsection{A Meta-pool of 10k Role Profiles}
\label{sec:meta_pool}
We first collect a meta-pool of 10k role profile using two open-source datasets, with Step 1 and 2.

\paragraph{Step 1: Persona Sampling.} We randomly sample 10.5k one-sentence demographic persona description from PersonaHub\,\cite{ge2024scaling}, such as ``\emph{A local business owner interested in economic trends}'', as shown at the top of Figure\,\ref{fig:method}. 

\paragraph{Step 2: Role Profile Standardization.} Next, we use a well-crafted prompt with gpt-4o\,\cite{gpt4o} to expand each sampled persona into a complete role profile, in reference to the 1-shot standardized example. Illustrated in the middle of Figure\,\ref{fig:method}, we require a standardized role profile consisting of seven components: \emph{Role Name and Brief Description}, \emph{Specific Abilities and Skills}, \emph{Speech Style}, \emph{Personality Characteristics}, \emph{Past Experience and Background}, \emph{Ability and Knowledge Boundaries} and \emph{Speech Examples}. %Setting standard specifications helps convert the generated role profiles into formatted records, which is beneficial for the post quality control. 
Standardizing these profiles ensures structured formatting, simplifying quality control. 
After manual checking and format filtering, we remove 333 invalid responses from gpt-4o, resulting in 10.2k final role profiles. We report complete persona-to-profile standardization prompt and structure tree of final role profiles in Appendix\,\ref{sec:app_prompt_1} and \,\ref{sec:app_tree}, respectively.

Machine Reading Comprehension (MRC) is one of the core tasks for LLMs to interact with human users. Consequently, we choose to synthesize fine-grained role-playing instruction-following data based on MRC. We first generate a retrieval pool containing 808.7k MRC data from the MSMARCO training set\,\cite{bajaj2016ms}. By leveraging SFR-Embedding\,\cite{SFR-embedding-2}, we perform an inner product search to identify the most relevant and least relevant MRC triplets (Passages, Question, Answer) for each role profile. For example, the middle part of Figure\,\ref{fig:method} shows that for the role \emph{Jessica Thompson, a resilient local business owner}, the most relevant question is about \emph{the skill of resiliency}, while the least relevant question is \emph{converting Fahrenheit to Celsius}. After review, we categorise the most relevant MRC triplet as within a role's knowledge boundary, and the least relevant MRC triplet as beyond their expertise.

\begin{figure}[t]
    \centering
    \includegraphics[width=1.0\linewidth]{figures/step3.png}
    \caption{The strategy of gradually synthesizing finer role-playing instructions in step 3 of Figure\,\ref{fig:method}.}
    \vspace{-1.0em}
    \label{fig:step3}
\end{figure}

\subsection{38k Role-playing Instructions}
Based on the role profiles, we then adopt \textbf{Step 3: Multi-stage Dialogue Synthesis} to generate 38k role-playing instructions, progressively increasing granularity across three categories %including three types with gradually finer granularity 
(Figure\,\ref{fig:step3}):
%\begin{itemize}
%[leftmargin=*,noitemsep,topsep=0pt]

\noindent \textbf{\underline{Free Chats.}} The simplest dialogues, free chats, are synthesized at first. Here, we ask gpt-4o to simulate and generate multi-turn open-domain conversations between the role and an imagined user based on the standardized role profile. When synthesizing the conversation, we additionally consider two factors: the \textbf{initial speaker} in the starting round of the conversation, and whether the role's speech has \textbf{a narration wrapped in brackets} at the beginning (e.g., \emph{(Aiden reviews the network logs, his eyes narrowing as he spots unusual activity) I found it!}). The narration refers to a short, vivid description of the role's speaking state from an omniscient perspective, which further strengthens the sense of role's depth and has been adopted in some role-playing datasets\,\cite{tu2024charactereval}. 

As shown on the left side of Figure\,\ref{fig:step3}, based on the aforementioned two factors, we synthesize four variations of Free Chats. In particular, when  narration is omitted, we deleted all the 
narration content in the speech examples from the role profile; %and for the case that 
when narration is allowed, we retain the narration content, and also add instructions to allow appropriate insertion of narration in the task prompt of gpt-4o. It worth to note that, in narration-allowed dialogues, not every response of the role has narration inserted to prevent overfitting. All categories of data in RoleMRC incorporate narration insertion and follow similar control mechanisms. The following sections will omit further details on narration.

\noindent \textbf{\underline{On-scene MRC Dialogues.}} The synthesis of on-scene MRC dialogues can be divided into two parts. The first part is similar to the free chats. As shown by the {\color{lightgreen}{green round rectangle}} in the upper part of Figure\,\ref{fig:step3}, we ask gpt-4o to synthesize a conversation (lower left corner of Figure\,\ref{fig:step3}) between the role and the user focusing on relevant passages. This part of the synthesis and the Free Chats share the entire meta-pool, so each consisting of 5k dialogues.

The remaining part forms eight types of single-turn role-playing Question Answering (QA). In the middle of Figure\,\ref{fig:step3}, we randomly select a group of roles and examined the most relevant MRCs they matched: if the question in the MRC is answerable, then the ground truth answer is stylized to match the role profile; otherwise, a seed script of ``unanswerable'' is randomly selected then stylized. The above process generates four groups of 1k data from type ``[1]'' to type``[4]''. According to the middle right side of Figure\,\ref{fig:step3}, we also select a group of roles and ensure that the least relevant MRCs they matched contain answerable QA pairs. Since the most irrelevant MRCs are outside the knowledge boundary of the roles, the role-playing responses to these questions are ``out-of-mind'' refusal or ``let-me-try'' attempt, thus synthesizing four groups of 1k data, from type ``[5]'' to type ``[8]''.

\noindent \textbf{\underline{Ruled Chats.}} We construct Ruled Chats by extending On-scene MRC Dialogues in categories ``[1]'' to ``[8]'' with incorporated three additional rules, as shown in the right bottom corner of Figure\,\ref{fig:step3}. For the \textbf{multi-turn rules}, we apply them to the four unanswerable scenarios ``[3]'', ``[4]'', ``[5]'', and ``[6]'', adding a user prompt that  forces the role to answer. Among them, data ``[3]'' and ``[4]'' maintain refusal since the questions in MRC are unanswerable; while ``[5]'' and ``[6]'' are transformed into attempts to answer despite knowledge limitations. For the \textbf{nested formatting rules}, we add new formatting instructions to the four categories of data ``[1]'', ``[2]'', ``[3]'', and ``[4]'', such as requiring emojis,  capitalization, specific punctuation marks, and controlling the total number of words, then modify the previous replies accordingly. For the last \textbf{prioritized rules}, we apply them to subsets ``[1]'' and ``[2]'' that contain normal stylized answers, inserting a  global refusal directive from the system, and thus creating a conflict between system instructions and the role's ability boundary.
%\end{itemize}

\begin{table}[t]
\resizebox{\columnwidth}{!}{%
  \begin{tabular}{c|c|c|c|c|c}
    \toprule
    & & \textbf{S*} & \textbf{P*} & \textbf{\#Turns} & \textbf{\#Words} \\ 
    \midrule
    \multirow{13.5}{*}{\textbf{RoleMRC}} 
    & \multicolumn{5}{c|}{\textbf{Free Chats}} \\ 
    \cmidrule(lr){2-6}
    & Chats & 5k & / & 9.47 & 38.62 \\ 
    \cmidrule(lr){2-6}
    & \multicolumn{5}{c|}{\textbf{On-scene MRC Dialogues}} \\ 
    \cmidrule(lr){2-6} 
    & On-scene Chats & 5k & / & 9.2 & 43.18 \\
    & Answer & 2k & 2k & 1 & 39.45 \\ 
    & No Answer & 2k & 2k & 1 & 47.09 \\ 
    & Refusal & 2k & 2k & 1 & 48.41 \\ 
    & Attempt & 2k & 2k & 1 & 47.92 \\ 
    \cmidrule(lr){2-6}
    & \multicolumn{5}{c|}{\textbf{Ruled Chats}} \\ 
    \cmidrule(lr){2-6}
    & Multi-turn & 2k & 2k & 2 & 42.47 \\ 
    & Nested & 1.6k & 1.6k & 1 & 46.17 \\ 
    & Prioritized & 2.4k & 2.4k & 1 & 42.65 \\ 
    \midrule
    & \textbf{Total} & 24k & 14k & 3.5 & 40.6 \\ 
    \midrule
    \multirow{3}{*}{\textbf{-mix}} 
    & RoleBench & 16k & / & 1 & 23.95 \\ 
    & RLHFlow & 40k & / & 1.39 & 111.79 \\ 
    & UltraFeedback & / & 14k & 1 & 199.28 \\ 
    \midrule
    & \textbf{Total} & 80k & 28k & 2 & 67.1 \\ 
    \bottomrule
  \end{tabular}}
  \vspace{-2mm}
  \caption{Statistics of RoleMRC. In particular, the column names S*, P*, \#Turns, and \#Words, stands for size of single-label data, size of pair-label data, average turns, and average number of words per reply, respectively. RoleMRC-mix expands RoleMRC by adding existing role-playing data.}
 \vspace{-3mm}
  \label{tab:roleMRC}
\end{table}

\subsection{Integration and Mix-up}
All the seed scripts and prioritized rules used for constructing On-scene Dialogues and Ruled Chats are reported in Appendix\,\ref{sec:app_scripts}. These raw responses are logically valid manual answers that remain unaffected by the roles' speaking styles, making them suitable as negative labels to contrast with the stylized answers. Thanks to these meticulous seed texts, we obtain high-quality synthetic data with stable output from gpt-4o. After integration, as shown in Table\,\ref{tab:roleMRC}, the final RoleMRC contains 24k single-label data for Supervised Fine-Tuning (SFT) and 14k pair-label data for Human Preference Optimization (HPO)\,\cite{ouyang2022training,rafailov2023direct,sampo,hong2024reference}. Considering that fine-tuning LLMs with relatively fixed data formats may lead to catastrophic forgetting\,\cite{kirkpatrick2017overcoming}, we create RoleMRC-mix as a robust version by incorporating external role-playing data (RoleBench\,\cite{wang2023rolellm}) and general instructions (RLHFlow\,\cite{dong2024rlhf}, UltraFeedback\,\cite{cui2023ultrafeedback}).

%\section{Training dataset}
\label{sec:training_set}

We generated a training image dataset comprising 17 orthographic gray-scale maps rendered from combinations of Sun azimuth ($0^{\circ}-360^{\circ}, 45^{\circ} steps$) and elevation $(30^{\circ}, 60^{\circ}, 90^{\circ}$), one corresponding orthographic depth map, and 4500 nadir-pointing aerial observations at fixed Sun angles AZ=180$^{\circ}$ and EL=40$^{\circ}$ along with their corresponding depth images. The query observations were randomly sampled from the HiRISE Jezero Crater's DTM with uniform distribution in the (x,y) coordinates in the world frame and within altitude range [$64, 200$] m. The query camera intrinsics are given by a pinhole camera model characterized by a sensor width of 80 mm, a focal length of 32 mm, 0 lens shifts along the image width and height axes, and an unitary pixel aspect ratio. Camera extrinsics, along with altitude data, were stored for each observation and served as ground truth. Further details are reported in Table \ref{tab:dataset_params}. Figure \ref{fig:map_tile_w_obs} shows gray-scale and normalized depth images of a map tile, with three sampled observations. 
Training examples for Geo-LoFTR are created by forming triplets ($I_A$, $I_B$, $I_C$) out of query observations and map windows crops (gray-scale and depth images) for multiple combinations of Sun azimuth and elevation angles' offsets between queries and the source maps. For a given combination of query and map lighting, each query observation is paired with map windows with at least 25\% area overlap on the terrain. Therefore, the set of triplets is formed by different combination of image $I_A$ with the mop window tuple ($I_B$, $I_C$). The map window sizes are carefully chosen to introduce an appropriate level of scale variance within the altitude range of the observations, ensuring a balance between model generalization and training efficiency. 
% By choosing prefixed map windows size of 1320 $\times$ 990 pixels for observations within the [64, 132] m altitude range, and a size of 2000 $\times$ 1500 pixels for the (132, 200] m, the scale ratio between query images and map windows is ensured to vary between 1 and 3.
Geo-LoFTR has been fine-tuned from the original LoFTR pre-trained model on a total 
of 150,705 generated triplets. An independent validation set of 3,400 triplets has been used to regularly assess the model performance during training and prevent over-fitting. 

\begin{figure}
    \centering
    \begin{minipage}[b]{0.9\linewidth}
        \centering
        \includegraphics[width=\linewidth]{Figures/map_tile_w_3_obs_aligned.jpg}
    \end{minipage}
    \caption{\label{fig:map_tile_w_obs} Gray-scale (left) and normalized depth (middle) images of a rendered orthographic map tile with three observations (right) sampled from different locations under the same illumination conditions.}
\end{figure}


\begin{table}
\centering
% Review this table
\begin{tabular}{l l l}
\textbf{Parameters} & \textbf{Maps} & \textbf{Observations} \\ \hline 
N.o. gray images & 17 & 4500 \\
N.o. depth images & 1 & 4500 \\
Image size & 26,949 $\times$ 57,613  & 480 $\times$ 640  \\ 
Pixel resolution & 0.25 m / px  &  [0.25, 0.78] m / px \\
Projection type  & Orthographic & Perspective \\  
Orthographic scale & 6737 m & / \\
Focal length & / & 32 mm \\
Sensor width & / & 80 mm \\
Location in $W$ & (0,0) m &  uniform random \\
&  &  distribution \\
Orientation in $W$ & nadir-pointing & nadir-pointing \\
Altitude & 4000 m & [64, 200] m\\
Sun AZ   & [0, 360]$^{\circ}$ with 45$^{\circ}$ steps & 180$^{\circ}$ \\  
Sun EL & 30$^{\circ}$, 60$^{\circ}$, 90$^{\circ}$ \\  
\end{tabular}
\caption{\label{tab:dataset_params} Parameters for the maps and observations image dataset sourced for the generation of the pairs formed by query observations and map windows, used for LoFTR and Geo-LoFTR training.}
\end{table}
\section{Map-based Localization Evaluation}
\label{sec:mbl_eval}

We evaluated the performance of our MbL pipeline on multiple image datasets generated in MARTIAN from HiRISE maps and DTMs of the Jezero Crater. To ensure an unbiased assessment, none of the test data overlaps with the data used for training. We conducted several experiments to assess the robustness of these methods under changes in lighting (Sec. \ref{subsec:mbl_robustnbess_to_az_and_el}) and scale (Sec. \ref{subsec:mbl_robustnbess_to_scale}). Figure \ref{fig:test_samples} shows areas on orthographic maps sampled from the test sets and rendered with two different illumination conditions, accompanied by three example observations at different altitudes.
To further stress our MbL pipeline with challenging lighting in a real-case scenario, we tested it over a simulated Martian day on the Jezero Crater site, with aerial observations generated in MARTIAN at multiple simulated times of day from sunrise to sunset (Sec. \ref{subsec:mbl_martian_day}). Furthermore, an evaluation of our method's performance under varying terrain morphologies is presented in the supplementary material.

We compared our results to the original LoFTR model fine-tuned on our training dataset (\textit{Fine-tuned LoFTR}), and to the model pre-trained on the MegaDepth dataset (\textit{Pre-trained LoFTR}). As for comparison with state-of-the-art feature matching in planetary aerial mobility, we also tested SIFT that proved to be one of the most accurate handcrafted methods for absolute localization over simulated Mars terrain \cite{brockers2022}.
% In the SIFT-based MbL evaluation, up to 4000 SIFT features were extracted from each query image and matched against those from the map, limited to the relevant search area. Feature matching was performed using k-nearest neighbors (KNN) with k=2. To ensure robustness, a ratio test was applied, retaining matches only if the ratio of the distance of the closest match to the distance of the second-closest match was below 0.75.
In each experiment, we use the percentage of queries with localization error $\|\mathbf{t}_{WC_{\text{query}}} - \widetilde{\mathbf{t}}_{WC_{\text{query}}}\|$ below 1m (@1m) as our evaluation metric. Also, we plot the Cumulative Distribution Function (CDF) of the localization accuracy up to 10m.
%In each experiment, we quantified the MbL accuracy as percentage of queries with localization errors $\|\mathbf{t}_{WC_{\text{query}}} - \widetilde{\mathbf{t}}_{WC_{\text{query}}}\|$. We computed the Cumulative distribution Functions (CdFs) of the localization accuracy, and show the @1m precision level as our main evaluation metric.  

\begin{figure}
    \centering
     \centering
        \includegraphics[width=\linewidth]{Figures/light_and_scale_var_example_resized.png}
    \vspace{-10pt}
    \caption{\label{fig:test_samples}Tiles from orthographic maps at sun (AZ=0°, EL=5°) (\textit{left}) (AZ=180°, EL=40°) (\textit{center}) with three sampled query observations (\textit{right}).}
\end{figure}


\subsection{Robustness to Changing Solar Angles}
\label{subsec:mbl_robustnbess_to_az_and_el}
Robustness to challenging illumination variance is assessed through registering  query observations onto orthographic maps rendered with varying sun elevation and azimuth angles, the effects of which are evaluated separately. 

The dataset for the experiment addressing the robustness to sun elevation changes comprises orthographic maps rendered at EL=$\{2°, 5°, 10°, 40°, 60°, 90°\}$ and AZ=$180°$, along with 500 nadir-pointing query observations at fixed sun angles (AZ=$180°$, EL=$40°$). The queries were taken in the 64-200 m altitude range which encompasses the nominal operation of the MSH. The minimum altitude of 64 m is set by the best resolution achievable in MARTIAN, which coincides with the 0.25 cm / pixel resolution of the HiRISE ortho-image used as texture. It is worth to note that elevations below 30$^{\circ}$ were not encountered during the model training. During the Martian day when our reference HiRISE map was collected, the sun elevation varied in the 7.9-82.6$^{\circ}$ range,  from 6:00 to 17:00 Local Mean Solar Time (LMST), with sunrise and sunset occurring at 05:11 LMST and 17:32 LMST, respectively. Therefore, elevations below 10$^{\circ}$ occupy a very brief portion of sun's local trajectory, representing exceptional cases in the Martian surface operations. Nevertheless, we include these cases in our evaluation to assess the models' ability to generalize and perform effectively under challenging lighting conditions.

To evaluate sun azimuth effect we generated 500 nadir-pointing queries with sun AZ=0$^{\circ}$ and EL=10$^{\circ}$ to be registered onto maps with varying azimuth angles in the 0-360$^{\circ}$ and same elevation as the queries.\\


\noindent \textbf{Robustness to sun elevation.} Figure \ref{fig:cdf_sun_el_var} shows the CDFs of the test observations' localization error onto maps at four different sun elevations and fixed 0 azimuth offset. Geo-LoFTR outperforms all the other methods in localization accuracy across the entire range of sun elevation offsets.
% and at all the precision levels except for the 5m-Accuracy, where its performance is comparable to the fine-tuned model.
In the case of zero sun angles' offsets, Geo-LoFTR is 87\% @1m accurate, showing an improvement of +38\% over the fine-tuned model, and +63\% over SIFT. Below 10$^{\circ}$ EL of the map, the performance of all the methods is significantly impacted, with Geo-LoFTR being 17\% @1m accurate at the very challenging case EL=2$^{\circ}$, where the pre-trained model and SIFT completely fail.
% However, Fine-tuned LoFTR shows the best failure localization rate overall, with a 5.8\% and 46.8\% rates at the low map sun elevations of 2$^{\circ}$ and 10$^{\circ}$, respectively, compared to 7.0\% and 53.6\% for Geo-LoFTR. 
%% UNCOMMENT
\begin{figure}
\centering
\begin{minipage}[b]{0.49\linewidth}
    \centering
    EL = $40^{\circ}$
    \vspace{5pt} \\
    \includegraphics[width=\linewidth]{Figures/cdf_acc_180az_40el_map_180az_40el_obs.png}
\end{minipage}
\begin{minipage}[b]{0.49\linewidth}
    \centering
    EL = $10^{\circ}$
    \vspace{5pt} \\
    \includegraphics[width=\linewidth]{Figures/cdf_acc_180az_10el_map_180az_40el_obs.png}
\end{minipage}
\\
\vspace{10pt}
\begin{minipage}[b]{0.49\linewidth}
    \centering
    EL = $5^{\circ}$
    \vspace{5pt} \\
    \includegraphics[width=\linewidth]{Figures/cdf_acc_180az_5el_map_180az_40el_obs.png}
\end{minipage}
\begin{minipage}[b]{0.49\linewidth}
    \centering
    EL = $2^{\circ}$
    \vspace{5pt} \\
    \includegraphics[width=\linewidth]{Figures/cdf_acc_180az_2el_map_180az_40el_obs.png}
\end{minipage}
\vspace{-10pt}
\caption{\label{fig:cdf_sun_el_var}Cumulative distributions of the localization error of simulated Mars observations at sun AZ=180$^{\circ}$ and EL=40$^{\circ}$, registered onto maps at four different elevation angles and 0$^{\circ}$ azimuth offset.}
\end{figure}
% \begin{figure}
% \centering
% \begin{minipage}[b]{0.4\linewidth}
%     \centering
%     \subcaption{0.5m-Accuracy}
%     \includegraphics[width=\linewidth]{Figures/loc_acc_vs_el_0.5acc-level.eps}
% \end{minipage}
% \begin{minipage}[b]{0.4\linewidth}
%     \centering
%     \subcaption{1m-Accuracy}
%     \includegraphics[width=\linewidth]{Figures/loc_acc_vs_el_1acc-level.eps}
% \end{minipage}
% \\
% \begin{minipage}[b]{0.4\linewidth}
%     \centering
%     \subcaption{2m-Accuracy}
%     \includegraphics[width=\linewidth]{Figures/loc_acc_vs_el_2acc-level.eps} 
% \end{minipage}
% \begin{minipage}[b]{0.4\linewidth}
%     \centering
%     \subcaption{5m-Accuracy}
%     \includegraphics[width=\linewidth]{Figures/loc_acc_vs_el_5acc-level.eps}
% \end{minipage}
% \caption{\label{fig:mbl_loc_acc_vs_el_vs_precision}Variation of localization accuracy with map sun elevation, for the MbL of 500 nadir-pointing observations at reference sun EL=40$^{\circ}$ (\textit{dashed line}).}
% % The localization accuracy is shown at the precision level of 0.5m (a) , 1 m (b), 2 m (c) and 5 m  (d). Queries and maps are generated from the Jezero Crater site with the MARTIAN framework, with a consistent sun azimuth angle of 180$^{\circ}$.}
% \end{figure}

%% UNCOMMENT
% \begin{figure}
% \centering
% % \makebox[0.01\linewidth]{}
% \makebox[0.47\linewidth]{Geo-LoFTR}
% \makebox[0.47\linewidth]{SIFT}
% \vspace{0.1cm} 

% \begin{minipage}[b]{0.49\linewidth}
%     \centering
%     \subcaption{map EL = $40^{\circ}$}
%     \includegraphics[width=\linewidth]{Figures/geo_map_40_180_obsv_40_180_id80_inlier_matches.png}  
% \end{minipage}
% \begin{minipage}[b]{0.49\linewidth}
%     \centering
%     \subcaption{map EL = $40^{\circ}$}
%     \includegraphics[width=\linewidth]{Figures/sift_map_40_180_obsv_40_180_id80_inlier_matches.png}
% \end{minipage}


% \begin{minipage}[b]{0.49\linewidth}
%     \centering
%     \subcaption{map EL = $10^{\circ}$}
%     \includegraphics[width=\linewidth]{Figures/geo_map_10_180_obsv_40_180_id80_inlier_matches.png}
% \end{minipage}
% \begin{minipage}[b]{0.49\linewidth}
%     \centering
%     \subcaption{map EL = $10^{\circ}$}
%     \includegraphics[width=\linewidth]{Figures/sift_map_10_180_obsv_40_180_id80_matches.png}
% \end{minipage}

% \begin{minipage}[b]{0.49\linewidth}
%     \centering
%     \subcaption{map EL = $5^{\circ}$}
%     \includegraphics[width=\linewidth]{Figures/geo_map_5_180_obsv_40_180_id80_inlier_matches.png} 
% \end{minipage}
% \begin{minipage}[b]{0.49\linewidth}
%     \centering
%     \subcaption{map EL = $5^{\circ}$}
%     \includegraphics[width=\linewidth]{Figures/sift_map_5_180_obsv_40_180_id80_matches.png} 
% \end{minipage}

% \begin{minipage}[b]{0.49\linewidth}
%     \centering
%     \subcaption{map EL = $2^{\circ}$}
%     \includegraphics[width=\linewidth]{Figures/geo_map_2_180_obsv_40_180_id80_inlier_matches.png} 
% \end{minipage}
% \begin{minipage}[b]{0.49\linewidth}
%     \centering
%     \subcaption{map EL = $2^{\circ}$}
%     \includegraphics[width=\linewidth]{Figures/sift_map_2_180_40_180_id80_matches.png} 
% \end{minipage}

% \caption{\label{fig:geo_sift_matched_keypoints_vs_el}Geo-LoFTR and SIFT matched keypoints displayed for a sample query image (\textit{left side of each panel}) with (180$^{\circ}$ AZ, 40$^{\circ}$ EL) sun angles and a map search area image (\textit{right side of each panel}) under four different sun elevations and 0$^{\circ}$ azimuth offset. Match lines are color-coded by confidence score, with redder indicating higher confidence.}
% \end{figure}


% \begin{figure*}
% \centering
% % \makebox[0.01\linewidth]{}
% \makebox[0.15\linewidth]{Geo-LoFTR}
% \makebox[0.15\linewidth]{Pre-trained LoFTR}
% \makebox[0.15\linewidth]{SIFT}
% \makebox[0.15\linewidth]{Geo-LoFTR}
% \makebox[0.15\linewidth]{Pre-trained LoFTR}
% \makebox[0.15\linewidth]{SIFT}
% \vspace{0.1cm} 

% % ROW 1
% \begin{minipage}[b]{0.16\linewidth}
%     \centering
%     \subcaption{map EL = $40^{\circ}$}
%     \includegraphics[width=\linewidth]{Figures/geo_map_40_180_obsv_40_180_id80_inlier_matches.png}  
% \end{minipage}
% \begin{minipage}[b]{0.16\linewidth}
%     \centering
%     \subcaption{map EL = $40^{\circ}$}
%     \includegraphics[width=\linewidth]{Figures/pre_map_40_180_obsv_40_180_id80_inlier_matches.png}  
% \end{minipage}
% \begin{minipage}[b]{0.16\linewidth}
%     \centering
%     \subcaption{map EL = $40^{\circ}$}
%     \includegraphics[width=\linewidth]{Figures/sift_map_40_180_obsv_40_180_id80_inlier_matches.png}
% \end{minipage}
% \begin{minipage}[b]{0.16\linewidth}
%     \centering
%     \subcaption{map AZ = $0^{\circ}$}
%     \includegraphics[width=\linewidth]{Figures/geo_map_10_0_obsv_10_0_id118_inlier_matches.png}  
% \end{minipage}
% \begin{minipage}[b]{0.16\linewidth}
%     \centering
%     \subcaption{map AZ = $0^{\circ}$}
%     \includegraphics[width=\linewidth]{Figures/pre_map_10_0_obsv_10_0_id118_inlier_matches.png}   
% \end{minipage}
% \begin{minipage}[b]{0.16\linewidth}
%     \centering
%     \subcaption{map AZ = $0^{\circ}$}
%     \includegraphics[width=\linewidth]{Figures/sift_map_10_0_obsv_10_0_id118_inlier_matches.png}  
% \end{minipage}

% % ROW 2
% \begin{minipage}[b]{0.16\linewidth}
%     \centering
%     \subcaption{map EL = $10^{\circ}$}
%     \includegraphics[width=\linewidth]{Figures/geo_map_10_180_obsv_40_180_id80_inlier_matches.png}
% \end{minipage}
% \begin{minipage}[b]{0.16\linewidth}
%     \centering
%     \subcaption{map EL = $10^{\circ}$}
%     \includegraphics[width=\linewidth]{Figures/pre_map_10_180_obsv_40_180_id80_inlier_matches.png}
% \end{minipage}
% \begin{minipage}[b]{0.16\linewidth}
%     \centering
%     \subcaption{map EL = $10^{\circ}$}
%     \includegraphics[width=\linewidth]{Figures/sift_map_10_180_obsv_40_180_id80_matches.png}
% \end{minipage}
% \begin{minipage}[b]{0.16\linewidth}
%     \centering
%     \subcaption{map AZ = $90^{\circ}$}
%     \includegraphics[width=\linewidth]{Figures/geo_map_10_90_obsv_10_0_id118_inlier_matches.png}  
% \end{minipage}
% \begin{minipage}[b]{0.16\linewidth}
%     \centering
%     \subcaption{map AZ = $90^{\circ}$}
%     \includegraphics[width=\linewidth]{Figures/pre_map_10_90_obsv_10_0_id118_inlier_matches.png}   
% \end{minipage}
% \begin{minipage}[b]{0.16\linewidth}
%     \centering
%     \subcaption{map AZ = $90^{\circ}$}
%     \includegraphics[width=\linewidth]{Figures/sift_map_10_90_obsv_10_0_id118_inlier_matches.png}  
% \end{minipage}

% % ROW 3
% \begin{minipage}[b]{0.16\linewidth}
%     \centering
%     \subcaption{map EL = $5^{\circ}$}
%     \includegraphics[width=\linewidth]{Figures/geo_map_5_180_obsv_40_180_id80_inlier_matches.png}
% \end{minipage}
% \begin{minipage}[b]{0.16\linewidth}
%     \centering
%     \subcaption{map EL = $5^{\circ}$}
%     \includegraphics[width=\linewidth]{Figures/pre_map_5_180_obsv_40_180_id80_inlier_matches.png}
% \end{minipage}
% \begin{minipage}[b]{0.16\linewidth}
%     \centering
%     \subcaption{map EL = $5^{\circ}$}
%     \includegraphics[width=\linewidth]{Figures/sift_map_5_180_obsv_40_180_id80_matches.png}
% \end{minipage}
% \begin{minipage}[b]{0.16\linewidth}
%     \centering
%     \subcaption{map AZ = $180^{\circ}$}
%     \includegraphics[width=\linewidth]{Figures/geo_map_10_180_obsv_10_0_id118_inlier_matches.png}  
% \end{minipage}
% \begin{minipage}[b]{0.16\linewidth}
%     \centering
%     \subcaption{map AZ = $180^{\circ}$}
%     \includegraphics[width=\linewidth]{Figures/pre_map_10_180_obsv_10_0_id118_inlier_matches.png}   
% \end{minipage}
% \begin{minipage}[b]{0.16\linewidth}
%     \centering
%     \subcaption{map AZ = $180^{\circ}$}
%     \includegraphics[width=\linewidth]{Figures/sift_map_10_180_obsv_10_0_id118_inlier_matches.png}  
% \end{minipage}

% % ROW 4
% \begin{minipage}[b]{0.16\linewidth}
%     \centering
%     \subcaption{map EL = $2^{\circ}$}
%     \includegraphics[width=\linewidth]{Figures/geo_map_2_180_obsv_40_180_id80_inlier_matches.png}
% \end{minipage}
% \begin{minipage}[b]{0.16\linewidth}
%     \centering
%     \subcaption{map EL = $2^{\circ}$}
%     \includegraphics[width=\linewidth]{Figures/pre_map_2_180_obsv_40_180_id80_inlier_matches.png}
% \end{minipage}
% \begin{minipage}[b]{0.16\linewidth}
%     \centering
%     \subcaption{map EL = $2^{\circ}$}
%     \includegraphics[width=\linewidth]{Figures/sift_map_2_180_40_180_id80_matches.png}
% \end{minipage}
% \begin{minipage}[b]{0.16\linewidth}
%     \centering
%     \subcaption{map AZ = $270^{\circ}$}
%     \includegraphics[width=\linewidth]{Figures/geo_map_10_270_obsv_10_0_id118_inlier_matches.png}  
% \end{minipage}
% \begin{minipage}[b]{0.16\linewidth}
%     \centering
%     \subcaption{map AZ = $270^{\circ}$}
%     \includegraphics[width=\linewidth]{Figures/pre_map_10_270_obsv_10_0_id118_inlier_matches.png}   
% \end{minipage}
% \begin{minipage}[b]{0.16\linewidth}
%     \centering
%     \subcaption{map AZ = $270^{\circ}$}
%     \includegraphics[width=\linewidth]{Figures/sift_map_10_270_obsv_10_0_id118_inlier_matches.png}  
% \end{minipage}

% \caption{\label{fig:matched_keypoints_vs_el_and_az}Each panel shows matched keypoints by Geo-LoFTR, Pre-trained LoFTR, and SIFT between a query image (\textit{left}) and a map search area (\textit{right}), with match lines color-coded by confidence (redder indicates higher confidence). The left 4x3 group uses a query with (180$^{\circ}$ AZ, 40$^{\circ}$ EL) sun angles, and maps with varying elevations ans same azimuth as the query. The right 4x3 group uses a querywith (0$^{\circ}$ AZ, 10$^{\circ}$ EL) sun angles, and maps with varying azimuths (same elevation as the query).}
% \end{figure*}


\noindent \textbf{Robustness to sun azimuth.}
The sun azimuth effect on MbL performance is presented through the cumulative distributions  (Figure \ref{fig:cdf_sun_az_var}) of the localization error for the test observations registered onto maps at four different  azimuth angles. Also in this experiment, Geo-LoFTR proved to be the most accurate model with a @1m accuracy being bound to the 54-63 \% range in the entire map sun azimuth range, despite the relatively low elevation of 10$^{\circ}$. The number and quality of the SIFT matched keypoints between query and map (Figure~\ref{fig:geo_sift_matched_keypoints_vs_az}) decreases much faster than the LoFTR-based models as we depart from the zero azimuth offset case, with failure already at 90$^{\circ}$ offset.
%Figure \ref{fig:mbl_loc_acc_vs_az_vs_precision} shows the variation of localization accuracy with the sun azimuth of the map at the precision levels of 0.5 m, 1 m, 2 m and 5 m. The azimuth angles are reported in the range [-180$^{\circ}$, 180$^{\circ}$] for visual clarity. Also in this experiment, Geo-LoFTR outperforms all the other methods, except for the 5m-Accuracy, where Fine-tuned LoFTR performs slightly better. Compared to the effect of the sun elevation variation, the accuracy of the LoFTR models trained on the Mars synthetic data results more robust to map sun azimuth offsets from the reference observation, with the 2m-Accuracy varying in a bandwidth of 8.5\% for Geo-LoFTR and 12.5\% for Fine-Tuned LoFTR. Such robustness is not observed for the pre-trained LoFTR model and SIFT, with a 2m-Accuracy varying in a bandwidth of 12\% for the former and 38.5\% for the latter.
% The matched keypoints on a sample observation and map search area are displayed for Geo-LoFTR and SIFT in Figure \ref{fig:geo_sift_matched_keypoints_vs_az}.

%% UNCOMMENT
\begin{figure}
\centering
\begin{minipage}[b]{0.49\linewidth}
    \centering
    AZ = $0^{\circ}$
    \vspace{5pt} \\
    \includegraphics[width=\linewidth]{Figures/cdf_acc_0az_10el_map_0az_10el_obs.png}
\end{minipage}
\begin{minipage}[b]{0.49\linewidth}
    \centering
    AZ = $90^{\circ}$
    \vspace{5pt} \\
    \includegraphics[width=\linewidth]{Figures/cdf_acc_90az_10el_map_0az_10el_obs.png}
\end{minipage}
\\
\vspace{10pt}
\begin{minipage}[b]{0.49\linewidth}
    \centering
    AZ = $180^{\circ}$
    \vspace{5pt} \\
    \includegraphics[width=\linewidth]{Figures/cdf_acc_180az_10el_map_0az_10el_obs.png}
\end{minipage}
\begin{minipage}[b]{0.49\linewidth}
    \centering
    AZ = $270^{\circ}$
    \vspace{5pt} \\
    \includegraphics[width=\linewidth]{Figures/cdf_acc_270az_10el_map_180az_40el_obs.png}
\end{minipage}
\vspace{-10pt}
\caption{\label{fig:cdf_sun_az_var}Cumulative distributions of the localization error of simulated Mars observations at sun AZ=0$^{\circ}$ and EL=10$^{\circ}$, registered onto maps at four different azimuth angles and 0$^{\circ}$ elevation offset.}
\end{figure}

% UNCOMMENT
\begin{figure*}
\centering
\makebox[0.3\linewidth]{\textbf{Geo-LoFTR}}
\makebox[0.3\linewidth]{\textbf{Pre-trained LoFTR}}
\makebox[0.3\linewidth]{\textbf{SIFT}}
\vspace{0.5cm} 

\begin{minipage}[b]{0.3\linewidth}
    \centering
    AZ = $0^{\circ}$
    \vspace{5pt} \\
    \includegraphics[width=\linewidth]{Figures/geo_map_10_0_obsv_10_0_id118_inlier_matches.png}  
\end{minipage}
\begin{minipage}[b]{0.3\linewidth}
    \centering
    AZ = $0^{\circ}$
    \vspace{5pt} \\
    \includegraphics[width=\linewidth]{Figures/pre_map_10_0_obsv_10_0_id118_inlier_matches.png}   
\end{minipage}
\begin{minipage}[b]{0.3\linewidth}
    \centering
    AZ = $0^{\circ}$
    \vspace{5pt} \\
    \includegraphics[width=\linewidth]{Figures/sift_map_10_0_obsv_10_0_id118_inlier_matches.png}  
\end{minipage}
\begin{minipage}[b]{0.3\linewidth}
    \centering
     \vspace{5pt} 
    AZ = $90^{\circ}$
    \vspace{5pt} \\
    \includegraphics[width=\linewidth]{Figures/geo_map_10_90_obsv_10_0_id118_inlier_matches.png}
\end{minipage}
\begin{minipage}[b]{0.3\linewidth}
    \centering
     \vspace{5pt} 
    AZ = $90^{\circ}$
    \vspace{5pt} \\
    \includegraphics[width=\linewidth]{Figures/pre_map_10_90_obsv_10_0_id118_inlier_matches.png}   
\end{minipage}
\begin{minipage}[b]{0.3\linewidth}
    \vspace{5pt} 
    \centering   
    AZ = $90^{\circ}$
    \vspace{5pt} \\
    \includegraphics[width=\linewidth]{Figures/sift_map_10_90_obsv_10_0_id118_inlier_matches.png}
\end{minipage}
\begin{minipage}[b]{0.3\linewidth}
    \centering
     \vspace{5pt} 
    AZ = $180^{\circ}$
    \vspace{5pt} 
    \includegraphics[width=\linewidth]{Figures/geo_map_10_180_obsv_10_0_id118_inlier_matches.png}
\end{minipage}
\begin{minipage}[b]{0.3\linewidth}
    \centering
     \vspace{5pt} 
    AZ = $180^{\circ}$
    \vspace{5pt} 
    \includegraphics[width=\linewidth]{Figures/pre_map_10_180_obsv_10_0_id118_inlier_matches.png}   
\end{minipage}
\begin{minipage}[b]{0.3\linewidth}
    \centering
     \vspace{5pt} 
    AZ = $180^{\circ}$
    \vspace{5pt} 
    \includegraphics[width=\linewidth]{Figures/sift_map_10_180_obsv_10_0_id118_matches.png}
\end{minipage}

\caption{\label{fig:geo_sift_matched_keypoints_vs_az}Geo-LoFTR, Pre-trained LoFTR and SIFT matched keypoints displayed for a sample query image (\textit{left side of each panel}) with (0$^{\circ}$ AZ, 10$^{\circ}$ EL) sun angles and a map search area image (\textit{right side of each panel}) under three different sun elevations and 0$^{\circ}$ azimuth offset. Match lines are color-coded by confidence score, with redder indicating higher confidence. Despite still providing a localization solution in the 0-180° AZ range, the pre-trained LoFTR matches exhibit lower confidence with azimuth changes than Geo-LoFTR, resulting in a coarser localization.} 
\end{figure*}
% \DP{include explanation about the fact that despite pre-trained LoFTR still provide accurate matches, the accuracy is much lower than Geo- and Fine-tunded LoFTR}}


% \begin{figure}
% \centering

% \makebox[0.01\linewidth]{}
% \makebox[0.3\linewidth]{64 m altitude}
% \makebox[0.3\linewidth]{100 m altitude}
% \makebox[0.3\linewidth]{200 m altitude}
% \vspace{0.1cm} % Space between azimuth labels and images

% \begin{minipage}[b]{0.01\linewidth}
%     \raisebox{2.5em}[0pt][0pt]{\makebox[0pt][r]{Geo-LoFTR}} % Adjust the raise value as needed
% \end{minipage}
% \begin{minipage}[b]{0.3\linewidth}
%     \centering
%     \includegraphics[width=\linewidth]{Figures/geo_cliff_64m_id0_inlier_matches_light_offset.png}
% \end{minipage}
% \begin{minipage}[b]{0.3\linewidth}
%     \centering
%     \includegraphics[width=\linewidth]{Figures/geo_cliff_100m_id5_inlier_matches_light_offset.png}
% \end{minipage}
% \begin{minipage}[b]{0.3\linewidth}
%     \centering
%     \includegraphics[width=\linewidth]{Figures/geo_cliff_200m_id27_inlier_matches_light_offset.png}
% \end{minipage}
    

% \begin{minipage}[b]{0.01\linewidth}
%     \raisebox{2.5em}[0pt][0pt]{\makebox[0pt][r]{SIFT}} % Adjust the raise value as needed
% \end{minipage}
% \begin{minipage}[b]{0.3\linewidth}
%     \centering
%     \includegraphics[width=\linewidth]{Figures/sift_cliff_64m_id0_inlier_matches_light_offset.png}
% \end{minipage}
% \begin{minipage}[b]{0.3\linewidth}
%     \centering
%     \includegraphics[width=\linewidth]{Figures/sift_cliff_100m_id5_inlier_matches_light_offset.png}
% \end{minipage}
% \begin{minipage}[b]{0.3\linewidth}
%     \centering
%     \includegraphics[width=\linewidth]{Figures/sift_cliff_200m_id27_inlier_matches_light_offset.png}
% \end{minipage}

% \begin{minipage}[b]{0.01\linewidth}
%     \raisebox{2.5em}[0pt][0pt]{\makebox[0pt][r]{Geo-LoFTR}} % Adjust the raise value as needed
% \end{minipage}
% \begin{minipage}[b]{0.3\linewidth}
%     \centering
%     \includegraphics[width=\linewidth]{Figures/geo_dunes_64m_id134_inlier_matches_light_offset.png}
% \end{minipage}
% \begin{minipage}[b]{0.3\linewidth}
%     \centering
%     \includegraphics[width=\linewidth]{Figures/geo_dunes_100m_id5_inlier_matches_light_offset.png}
% \end{minipage}
% \begin{minipage}[b]{0.3\linewidth}
%     \centering
%     \includegraphics[width=\linewidth]{Figures/geo_dunes_200m_id0_inlier_matches_light_offset.png}
% \end{minipage}
    
% \begin{minipage}[b]{0.01\linewidth}
%     \raisebox{2.5em}[0pt][0pt]{\makebox[0pt][r]{SIFT}} % Adjust the raise value as needed
% \end{minipage}
% \begin{minipage}[b]{0.3\linewidth}
%     \centering
%     \includegraphics[width=\linewidth]{Figures/sift_dunes_64m_id134_inlier_matches_light_offset.png}
% \end{minipage}
% \begin{minipage}[b]{0.3\linewidth}
%     \centering
%     \includegraphics[width=\linewidth]{Figures/sift_dunes_100m_id5_inlier_matches_light_offset.png}
% \end{minipage}
% \begin{minipage}[b]{0.3\linewidth}
%     \centering
%     \includegraphics[width=\linewidth]{Figures/sift_dunes_200m_id0_inlier_matches_light_offset.png}
% \end{minipage}


% \caption{\label{fig:matches_vs_alt_rugged}Matched keypoints for rugged (\textit{top}) and dunal (\textit{bottom}) between observations (\textit{left, in each panel}) and map \textit{right, in each panel}) for the Jezero Crater HiRISE map, at altitudes of 64 m (\textit{left column)}, 100 m (\textit{middle column}) and 200 m (\textit{right column}) for Geo-LoFTR and SIFT. Observations are generated in MARTIAN framework with sun AZ=180$^{\circ}$ and EL=40$^{\circ}$. The map is rendered at sun AZ=0$^{\circ}$ and EL=5$^{\circ}$.}
% \end{figure}

\subsection{Robustness to Scale Variation}
\label{subsec:mbl_robustnbess_to_scale}

We split the test observations from Sec. \ref{subsec:mbl_robustnbess_to_az_and_el} in three different altitude sub-ranges, and registered them onto maps with zero sun angle offsets to assess the pipeline's performance under scale changes. Figure \ref{fig:cdf_scale_var} shows the CDF of the localization errors of observations taken within 64m-112m, 112m-155m, 155m-200m registered on a map with constant (AZ=180°, EL=40°) sun angles. With only a -7\% @1m accuracy drop across the entire altitude range, Geo-LoFTR proved to be more robust than the fine-tuned model (-33\% @1m). A similar degree of scale invariance is shown for the pre-trained model and SIFT, although being much less accurate.  

\begin{figure}
\centering
\begin{minipage}[b]{0.49\linewidth}
    \centering
    64-112 m
    \vspace{2pt} \\
    \includegraphics[width=\linewidth]{Figures/cdf_acc_180az_40el_map_180az_40el_obs_64-112m.png}
\end{minipage}
\begin{minipage}[b]{0.49\linewidth}
    \centering
    112-155 m
    \vspace{2pt} \\
    \includegraphics[width=\linewidth]{Figures/cdf_acc_180az_40el_map_180az_40el_obs_112-155m.png}
\end{minipage}
\begin{minipage}[b]{0.49\linewidth}
    \centering
    155-200 m
    \vspace{2pt} \\
    \includegraphics[width=\linewidth]{Figures/cdf_acc_180az_40el_map_180az_40el_obs_155-200m.png}
\end{minipage}
\caption{\label{fig:cdf_scale_var}Cumulative distributions of the localization error of simulated Mars observations at sun AZ=0$^{\circ}$ and EL=10$^{\circ}$, registered onto maps with the same illumination condition for three different altitude ranges.}
\end{figure}


\subsection{Robustness to Combined Illumination and Scale Changes}
\label{subsec:mbl_robustnbess_to_scale_ill}

Leveraging the test data in Sec \ref{subsec:mbl_robustnbess_to_az_and_el}, we performed a quantitative evaluation of the scale variation effects in conjunction with sun angle offsets. Figure \ref{fig:mbl_loc_acc_vs_el_and_az_vs_alt}, shows the @1m accuracy as a function of map sun EL and AZ for observations taken within three different altitude ranges.
%and fixed sun angles in the respective cases. 
Although localization accuracy declines sharply at relatively low sun elevation angles, Geo-LoFTR maintains consistent localization performance across altitude variations within the 10–90° EL range. In contrast, the fine-tuned model demonstrates poor robustness in the same range. A similar trend is observed for azimuth variations, where localization accuracy remains stable with changing azimuth but decreases with altitude.


% Figure \ref{fig:mbl_loc_acc_vs_el_and_az_vs_alt} show the accuracy of MbL at 1 m precision against sun angles offsets in elevation and azimuth, within three different altitude sub-ranges. Geo-LoFTR results more robust to altitude variations than the other methods, suggesting the benefits introduced to localization accuracy by depth data.
% Geo-LoFTR results more robust to altitude variations than the fine-tuned model over the entire set of sun elevation offsets. Picking a map sun elevation of 10$^{\circ}$ for instance, the drop in 1m-accuracy is only of -13\% between the 156-199 m and 64-113 m altitude ranges for Geo-LoFTR, compared to a -28\% accuracy drop for Fine-tuned LoFTR. The improved scale invariance of the geometry-aided model over the other methods is confirmed also under lighting variations in terms of sun azimuth offsets, thus providing yet another piece of evidence of the benefits introduced to localization accuracy by depth data.

%% UNCOMMENT
\begin{figure}
\centering
\begin{minipage}[b]{1.0\linewidth}
    \centering
    \includegraphics[width=\linewidth]{Figures/1m-acc_loc_vs_el_vs_alt.png}    
\end{minipage}
\vspace{5pt} \\
\begin{minipage}[b]{1.0\linewidth}
    \centering
    \includegraphics[width=\linewidth]{Figures/1m-acc_loc_vs_az_vs_alt.png}    
\end{minipage}
\vspace{-5pt} \\
\caption{Localization accuracy at 1m precision as a function of map sun elevation (\textit{top}) and azimuth (\textit{bottom}) for test observations across three altitude ranges. Sun azimuth angles are in the $[-180^{\circ}, 180^{\circ}]$ range. Map sun angles matching the observations are marked with a thick black vertical line.}
\label{fig:mbl_loc_acc_vs_el_and_az_vs_alt}
\end{figure}

\subsection{Localization Over a Simulated Martian Day}
\label{subsec:mbl_martian_day}
The MbL performance is investigated for observations taken at different LMSTs during a simulated Martian day on the Jezero Crater HiRISE map at coordinates (77.44$^{\circ}$E, 18.43$^{\circ}$N). We used the Mars24~\cite{mars24} software developed by NASA Goddard Institute for Space Studies to compute the sun's local trajectory for the selected site on a given date. The chosen date, 2031-05-10, ensures that the sun zenith is at a relatively high elevation angle of 86.7$^{\circ}$, allowing a broad range of elevation angles to be observed throughout the day (Figure \ref{fig:sun_profile}). 
% where the sun azimuth is measured clockwise from North in Mars24, differently from our MARTIAN framework, where it is defined counterclockwise from East. The sun elevation is still reported with the same convention adopted in the MARTIAN.

% UNCOMMENT
\begin{figure}
\setlength{\abovecaptionskip}{0pt}  % Removes space above caption
\setlength{\belowcaptionskip}{0pt}  % Removes space below caption
\centering
\includegraphics[width=0.98\linewidth]{Figures/sun_profile.png}
\caption{Sun trajectory on a local panorama from 77.44$^{\circ}$E longitude and 18.43$^{\circ}$N latitude on Mars, on 2031-05-10, with positions shown at four Local Mean Solar Times (LMSTs). Adapted from Mars24 \cite{mars24}.}
\label{fig:sun_profile}
\end{figure}

We generated nadir-pointing observations in MARTIAN at multiple times of the day from 5:30 to 17:00 LMST, with a total of 3000 queries collected across the 64-200 m altitude range (Figure \ref{fig:obsv_lmst}). We also rendered an orthographic map at 15:00 LMST, (AZ=175.1$^{\circ}$, 39.9$^{\circ}$ EL), serving as a HiRISE-like reference. 
Figure \ref{fig:loc_acc_vs_lmst_1acc_vs_alt} shows the @1m accuracy as function of LMTs within three different altitude sub-ranges. Geo-LoFTR outperformed the other methods for most of the Martian day, except at 5:00 LMST, where the fine-tuned model shows better accuracy. However, the fine-tuned LoFTR experienced significant performance degradation with altitude, in contrast with the other methods that exhibited a certain grade of scale invariance also in this experiment.

% LoFTR trained on the Mars datasets outperformed SIFT. SIFT proves more accurate than Pre-trained LoFTR in times of day between 8:00 and 15:00 LMST. However, its accuracy rapidly declines for observations taken earlier than 8:00 LMST underperforming the base LoFTR model, especially at lower altitudes. Geo-LoFTR results in being the best localization method at 1 m and 2 m precision levels for most of the Martian day, except when observations are taken at 5:30 LMST, where its performance significantly degrades. Despite the overall higher accuracy, Geo-LoFTR is also observed to be less stable than the fine-tuned model throughout the Martian day. However, it is more robust to altitude changes than its visual counterpart, providing yet another evidence of the contribution provided by geometric context in improving scale invariance. 
% UNCOMMENT
\begin{figure}
\centering

\makebox[0.01\linewidth]{}
% \makebox[0.25\linewidth]{\small 64 m}
% \makebox[0.25\linewidth]{\small 100 m}
% \makebox[0.25\linewidth]{\small 200 m}
% \vspace{0.1cm} % Space between azimuth labels and images

% \begin{minipage}[b]{0.01\linewidth}
%     \raisebox{3.5em}[0pt][0pt]{\makebox[0pt][r]{\small LMST:}} % Adjust the raise value as needed
% \end{minipage}
\begin{minipage}[b]{0.01\linewidth}
    \raisebox{2em}[0pt][0pt]{\makebox[0pt][r]{\small \shortstack{\textbf{LMST}: \\ 05:30}}} 
\end{minipage}
\begin{minipage}[b]{0.25\linewidth}
    \centering
    \includegraphics[width=\linewidth]{Figures/05_30_0479.png}
\end{minipage}
\begin{minipage}[b]{0.25\linewidth}
    \centering
    \includegraphics[width=\linewidth]{Figures/05_30_0460.png}
\end{minipage}
\begin{minipage}[b]{0.25\linewidth}
    \centering
    \includegraphics[width=\linewidth]{Figures/05_30_0443.png}
\end{minipage}

\begin{minipage}[b]{0.01\linewidth}
    \raisebox{2em}[0pt][0pt]{\makebox[0pt][r]{\small 06:00
    }} % Adjust the raise value as needed
\end{minipage}
\begin{minipage}[b]{0.25\linewidth}
    \centering
    \includegraphics[width=\linewidth]{Figures/06_00_0018.png}
\end{minipage}
\begin{minipage}[b]{0.25\linewidth}
    \centering
    \includegraphics[width=\linewidth]{Figures/06_00_0040.png}
\end{minipage}
\begin{minipage}[b]{0.25\linewidth}
    \centering
    \includegraphics[width=\linewidth]{Figures/06_00_0046.png}
\end{minipage}

\begin{minipage}[b]{0.01\linewidth}
    \raisebox{2em}[0pt][0pt]{\makebox[0pt][r]{\small 08:00
    }} % Adjust the raise value as needed
\end{minipage}
\begin{minipage}[b]{0.25\linewidth}
    \centering
    \includegraphics[width=\linewidth]{Figures/08_00_0000.png}
\end{minipage}
\begin{minipage}[b]{0.25\linewidth}
    \centering
    \includegraphics[width=\linewidth]{Figures/08_00_0005.png}
\end{minipage}
\begin{minipage}[b]{0.25\linewidth}
    \centering
    \includegraphics[width=\linewidth]{Figures/08_00_0023.png}
\end{minipage}

\begin{minipage}[b]{0.01\linewidth}
    \raisebox{2em}[0pt][0pt]{\makebox[0pt][r]{\small11:29
    }} % Adjust the raise value as needed
\end{minipage}
\begin{minipage}[b]{0.25\linewidth}
    \centering
    \includegraphics[width=\linewidth]{Figures/11_29_0000.png}
\end{minipage}
\begin{minipage}[b]{0.25\linewidth}
    \centering
    \includegraphics[width=\linewidth]{Figures/11_29_0002.png}
\end{minipage}
\begin{minipage}[b]{0.25\linewidth}
    \centering
    \includegraphics[width=\linewidth]{Figures/11_29_0006.png}
\end{minipage}

\begin{minipage}[b]{0.01\linewidth}
    \raisebox{2em}[0pt][0pt]{\makebox[0pt][r]{\small15:00
    }} % Adjust the raise value as needed
\end{minipage}
\begin{minipage}[b]{0.25\linewidth}
    \centering
    \includegraphics[width=\linewidth]{Figures/15_00_0004.png}
\end{minipage}
\begin{minipage}[b]{0.25\linewidth}
    \centering
    \includegraphics[width=\linewidth]{Figures/15_00_0008.png}
\end{minipage}
\begin{minipage}[b]{0.25\linewidth}
    \centering
    \includegraphics[width=\linewidth]{Figures/15_00_0007.png}
\end{minipage}

\begin{minipage}[b]{0.0255\linewidth}
    \raisebox{2em}[0pt][0pt]{\makebox[0pt][r]{\small17:00
    }} % Adjust the raise value as needed
\end{minipage}
\begin{minipage}[b]{0.25\linewidth}
    \centering
    \includegraphics[width=\linewidth]{Figures/17_00_0022.png}
\end{minipage}
\begin{minipage}[b]{0.25\linewidth}
    \centering
    \includegraphics[width=\linewidth]{Figures/17_00_0042.png}
\end{minipage}
\begin{minipage}[b]{0.25\linewidth}
    \centering
    \includegraphics[width=\linewidth]{Figures/17_00_0082.png}
\end{minipage}
\caption{Sample nadir-pointing observations rendered at different Local Mean Solar Times (LMSTs) taken on Mars on 2031-05-10. The reference HiRISE map is taken at 15:00 LMST. The sun Zenith is at 11:29 LMST.}
\label{fig:obsv_lmst}
% Data are generated from Jezero Crater using the MARTIAN framework. }
\end{figure}


% \begin{table}
% \centering
% \begin{tabular}{l c c}
% \textbf{LMST} & \textbf{sun EL} & \textbf{sun AZ} \\
% \hline
% 05:07 (rise)   & -0.2$^{\circ}$ & 16.6$^{\circ}$\\
% 05:30         & 5.0$^{\circ}$ & 14.8$^{\circ}$\\
% 06:00         & 12.1$^{\circ}$ & 12.6$^{\circ}$\\
% 08:00         & 40.1$^{\circ}$ & 5.0$^{\circ}$\\
% 11:29 (zenith) & 86.9$^{\circ}$ & 270.3$^{\circ}$\\
% 15:00 (HiRISE) & 39.9$^{\circ}$ & 175.1$^{\circ}$\\
% 17:00         & 11.6$^{\circ}$ & 167.4$^{\circ}$\\
% 17:52 (set)   &  -0.2$^{\circ}$ & 163.5$^{\circ}$\\
%   \hline \\
% \end{tabular}
% \caption{\label{tab:lmst_solar_profile}Specifications of points of interests on the solar local trajectory from (77.44$^{\circ}$E, 18.43$^{\circ}$N) coordinates on Mars, on 2031-05-10, with sun azimuth and elevation angles referred to the MARTIAN framework. Adapted from Mars24\cite{mars24}.} 
% % sun azimuth is measured clockwise from North in Mars24 framework; while it is defined counterclockwise from East in the MARTAIAN framework. Adapted from Mars24\cite{mars24}.}
% \end{table}

% \begin{figure}
% \centering
% \begin{minipage}[b]{0.3\linewidth}
%     \centering
%     \subcaption{5:30 LMST}
%     \includegraphics[width=\linewidth]{Figures/CDF_loc_acc_map_HiRISE_obsv_morning_5_30.eps}
% \end{minipage}
% \begin{minipage}[b]{0.3\linewidth}
%     \centering
%     \subcaption{6:00 LMST}
%     \includegraphics[width=\linewidth]{Figures/CDF_loc_acc_map_HiRISE_obsv_morning_6_00.eps}
% \end{minipage}
% \begin{minipage}[b]{0.3\linewidth}
%     \centering
%     \subcaption{8:00 LMST}
%     \includegraphics[width=\linewidth]{Figures/CDF_loc_acc_map_HiRISE_obsv_morning.eps}
% \end{minipage}
% \begin{minipage}[b]{0.3\linewidth}
%     \centering
%     \subcaption{11:29 LMST}
%     \includegraphics[width=\linewidth]{Figures/CDF_loc_acc_map_HiRISE_obsv_zenith.eps}
% \end{minipage}
% \begin{minipage}[b]{0.3\linewidth}
%     \centering
%     \subcaption{15:00 LMST}
%     \includegraphics[width=\linewidth]{Figures/CDF_loc_acc_map_40_180_obsv_40_180.eps} 
% \end{minipage}
% \begin{minipage}[b]{0.3\linewidth}
%     \centering
%     \subcaption{17:00 LMST}
%     \includegraphics[width=\linewidth]{Figures/CDF_loc_acc_map_HiRISE_obsv_set.eps}
% \end{minipage}

% \caption{\label{fig:mbl_cdf_martian_day}Cumulative distributions of the localization error for the MbL of 500 nadir-pointing observations (altitude 64-200m) taken at six different Local Mean Solar Times (LMSTs) on Mars, on 2031-05-10. The reference HiRISE map is taken at 15:00 LMST (c). The sun Zenith is at 11:29 LMST (b). Data are generated from Jezero Crater using the MARTIAN framework.}
% \end{figure}

% UNCOMMENT
\begin{figure}
\centering
\includegraphics[width=1\linewidth]{Figures/1m-acc_loc_vs_LMST_vs_alt_zoom.png} 
\caption{Localization accuracy (@1m) as a function of Local Mean Solar Time (LMST) of simulated test observations from the Jezero Crater on 2031-05-10 across the 64-200 m altitude range. The reference HiRISE-like map is taken at 15:00 LMST (\textit{dashed black line}). The sun Zenith is at 11:29 LMST.}
\vspace{-10pt}
\label{fig:loc_acc_vs_lmst_1acc_vs_alt}
\end{figure}
% Analyzing the 1m-accuracy, the LoFTR trained on the Mars datasets outperformed the pre-trained LoFTR model and SIFT, which performed in the 20-30\% accuracy range. SIFT proves more accurate than Pre-trained LoFTR in times of day between 8:00 LMST and 15:00 LMST. However, its accuracy rapidly declines for observations taken earlier than 8:00 LMST underperforming the base LoFTR model, especially at lower altitudes. Geo-LoFTR results in being the best localization method at 1 m and 2 m precision levels for most of the Martian day, except when observations are taken at 5:30 LMST, where its performance significantly degrades. Despite the overall higher accuracy, Geo-LoFTR is also observed to be less stable than the fine-tuned model throughout the Martian day. However, it is confirmed again to be more robust to altitude changes than its visual counterpart, providing yet another evidence of the contribution provided by geometric context in achieving scale invariance. 

\subsection{Discussion}
\label{subsec:discussion}
Geo-LoFTR demonstrated superior localization accuracy compared to other methods across a broad range of illumination conditions, indicating that incorporating depth information can mitigate degeneracies inherent to purely visual data. Robustness to sun elevation is maintained within a wide range of angles, except in extremely challenging cases (e.g., EL = 2°), where poor lighting and extensive shadow coverage might saturate the constraining power of the geometric information, leading to a rapid decline in localization accuracy. More stable is the behavior for changes in azimuth.
Geo-LoFTR also showed greater robustness to changing observation altitude than the fine-tuned model across multiple experiments, suggesting that providing a geometric context contributes to scale invariance. A possible explanation is that depth data constrains matches by providing consistent pixel-to-pixel depth relationships across altitudes, reflecting terrain elevation alone. This added layer of geometric consistency likely enhances Geo-LoFTR's ability to learn accurate matches by reducing ambiguity from appearance-based features alone.


\section{Limitations}

\paragraph{Reliance on a Stronger LLM. }
Our framework relies on a stronger LLM to synthesize data. While this enables the synthesis of high quality data, removing this dependency could help lead to a more robust and independent framework, possibly at the cost of performance degradation. Additionally, LLM-generated data may amplify existing biases or include inappropriate content.

\paragraph{Seed Data Quality. }
The quality of our synthesized data is tied to that of our seed data. We select concise, high-quality datasets from prior works to use as the seed data. A more comprehensive exploration of seed data selection and its impact on synthetic data remains an important direction for future work.

Furthermore, our work does not fully address the scalability our framework. There likely exists a limit to how much data we can synthesize from our seed data, until the synthesized data becomes repetitive and lacks diversity.

\paragraph{LLM-Based Evaluation. }
Our evaluation relies on benchmarks that use LLMs as a judge. Although they correlate highly with human judgments, it is important to acknowledge that they may still have limitations, such as biases towards longer responses or their own outputs.


\section{Acknowledgments}
This work has benefited from the Microsoft Accelerate Foundation Models Research (AFMR) grant program, through which leading foundation models hosted by Microsoft Azure and access to Azure credits were provided to conduct the research.

\section{Discussion}\label{sec:discussion}



\subsection{From Interactive Prompting to Interactive Multi-modal Prompting}
The rapid advancements of large pre-trained generative models including large language models and text-to-image generation models, have inspired many HCI researchers to develop interactive tools to support users in crafting appropriate prompts.
% Studies on this topic in last two years' HCI conferences are predominantly focused on helping users refine single-modality textual prompts.
Many previous studies are focused on helping users refine single-modality textual prompts.
However, for many real-world applications concerning data beyond text modality, such as multi-modal AI and embodied intelligence, information from other modalities is essential in constructing sophisticated multi-modal prompts that fully convey users' instruction.
This demand inspires some researchers to develop multimodal prompting interactions to facilitate generation tasks ranging from visual modality image generation~\cite{wang2024promptcharm, promptpaint} to textual modality story generation~\cite{chung2022tale}.
% Some previous studies contributed relevant findings on this topic. 
Specifically, for the image generation task, recent studies have contributed some relevant findings on multi-modal prompting.
For example, PromptCharm~\cite{wang2024promptcharm} discovers the importance of multimodal feedback in refining initial text-based prompting in diffusion models.
However, the multi-modal interactions in PromptCharm are mainly focused on the feedback empowered the inpainting function, instead of supporting initial multimodal sketch-prompt control. 

\begin{figure*}[t]
    \centering
    \includegraphics[width=0.9\textwidth]{src/img/novice_expert.pdf}
    \vspace{-2mm}
    \caption{The comparison between novice and expert participants in painting reveals that experts produce more accurate and fine-grained sketches, resulting in closer alignment with reference images in close-ended tasks. Conversely, in open-ended tasks, expert fine-grained strokes fail to generate precise results due to \tool's lack of control at the thin stroke level.}
    \Description{The comparison between novice and expert participants in painting reveals that experts produce more accurate and fine-grained sketches, resulting in closer alignment with reference images in close-ended tasks. Novice users create rougher sketches with less accuracy in shape. Conversely, in open-ended tasks, expert fine-grained strokes fail to generate precise results due to \tool's lack of control at the thin stroke level, while novice users' broader strokes yield results more aligned with their sketches.}
    \label{fig:novice_expert}
    % \vspace{-3mm}
\end{figure*}


% In particular, in the initial control input, users are unable to explicitly specify multi-modal generation intents.
In another example, PromptPaint~\cite{promptpaint} stresses the importance of paint-medium-like interactions and introduces Prompt stencil functions that allow users to perform fine-grained controls with localized image generation. 
However, insufficient spatial control (\eg, PromptPaint only allows for single-object prompt stencil at a time) and unstable models can still leave some users feeling the uncertainty of AI and a varying degree of ownership of the generated artwork~\cite{promptpaint}.
% As a result, the gap between intuitive multi-modal or paint-medium-like control and the current prompting interface still exists, which requires further research on multi-modal prompting interactions.
From this perspective, our work seeks to further enhance multi-object spatial-semantic prompting control by users' natural sketching.
However, there are still some challenges to be resolved, such as consistent multi-object generation in multiple rounds to increase stability and improved understanding of user sketches.   


% \new{
% From this perspective, our work is a step forward in this direction by allowing multi-object spatial-semantic prompting control by users' natural sketching, which considers the interplay between multiple sketch regions.
% % To further advance the multi-modal prompting experience, there are some aspects we identify to be important.
% % One of the important aspects is enhancing the consistency and stability of multiple rounds of generation to reduce the uncertainty and loss of control on users' part.
% % For this purpose, we need to develop techniques to incorporate consistent generation~\cite{tewel2024training} into multi-modal prompting framework.}
% % Another important aspect is improving generative models' understanding of the implicit user intents \new{implied by the paint-medium-like or sketch-based input (\eg, sketch of two people with their hands slightly overlapping indicates holding hand without needing explicit prompt).
% % This can facilitate more natural control and alleviate users' effort in tuning the textual prompt.
% % In addition, it can increase users' sense of ownership as the generated results can be more aligned with their sketching intents.
% }
% For example, when users draw sketches of two people with their hands slightly overlapping, current region-based models cannot automatically infer users' implicit intention that the two people are holding hands.
% Instead, they still require users to explicitly specify in the prompt such relationship.
% \tool addresses this through sketch-aware prompt recommendation to fill in the necessary semantic information, alleviating users' workload.
% However, some users want the generative AI in the future to be able to directly infer this natural implicit intentions from the sketches without additional prompting since prompt recommendation can still be unstable sometimes.


% \new{
% Besides visual generation, 
% }
% For example, one of the important aspect is referring~\cite{he2024multi}, linking specific text semantics with specific spatial object, which is partly what we do in our sketch-aware prompt recommendation.
% Analogously, in natural communication between humans, text or audio alone often cannot suffice in expressing the speakers' intentions, and speakers often need to refer to an existing spatial object or draw out an illustration of her ideas for better explanation.
% Philosophically, we HCI researchers are mostly concerned about the human-end experience in human-AI communications.
% However, studies on prompting is unique in that we should not just care about the human-end interaction, but also make sure that AI can really get what the human means and produce intention-aligned output.
% Such consideration can drastically impact the design of prompting interactions in human-AI collaboration applications.
% On this note, although studies on multi-modal interactions is a well-established topic in HCI community, it remains a challenging problem what kind of multi-modal information is really effective in helping humans convey their ideas to current and next generation large AI models.




\subsection{Novice Performance vs. Expert Performance}\label{sec:nVe}
In this section we discuss the performance difference between novice and expert regarding experience in painting and prompting.
First, regarding painting skills, some participants with experience (4/12) preferred to draw accurate and fine-grained shapes at the beginning. 
All novice users (5/12) draw rough and less accurate shapes, while some participants with basic painting skills (3/12) also favored sketching rough areas of objects, as exemplified in Figure~\ref{fig:novice_expert}.
The experienced participants using fine-grained strokes (4/12, none of whom were experienced in prompting) achieved higher IoU scores (0.557) in the close-ended task (0.535) when using \tool. 
This is because their sketches were closer in shape and location to the reference, making the single object decomposition result more accurate.
Also, experienced participants are better at arranging spatial location and size of objects than novice participants.
However, some experienced participants (3/12) have mentioned that the fine-grained stroke sometimes makes them frustrated.
As P1's comment for his result in open-ended task: "\emph{It seems it cannot understand thin strokes; even if the shape is accurate, it can only generate content roughly around the area, especially when there is overlapping.}" 
This suggests that while \tool\ provides rough control to produce reasonably fine results from less accurate sketches for novice users, it may disappoint experienced users seeking more precise control through finer strokes. 
As shown in the last column in Figure~\ref{fig:novice_expert}, the dragon hovering in the sky was wrongly turned into a standing large dragon by \tool.

Second, regarding prompting skills, 3 out of 12 participants had one or more years of experience in T2I prompting. These participants used more modifiers than others during both T2I and R2I tasks.
Their performance in the T2I (0.335) and R2I (0.469) tasks showed higher scores than the average T2I (0.314) and R2I (0.418), but there was no performance improvement with \tool\ between their results (0.508) and the overall average score (0.528). 
This indicates that \tool\ can assist novice users in prompting, enabling them to produce satisfactory images similar to those created by users with prompting expertise.



\subsection{Applicability of \tool}
The feedback from user study highlighted several potential applications for our system. 
Three participants (P2, P6, P8) mentioned its possible use in commercial advertising design, emphasizing the importance of controllability for such work. 
They noted that the system's flexibility allows designers to quickly experiment with different settings.
Some participants (N = 3) also mentioned its potential for digital asset creation, particularly for game asset design. 
P7, a game mod developer, found the system highly useful for mod development. 
He explained: "\emph{Mods often require a series of images with a consistent theme and specific spatial requirements. 
For example, in a sacrifice scene, how the objects are arranged is closely tied to the mod's background. It would be difficult for a developer without professional skills, but with this system, it is possible to quickly construct such images}."
A few participants expressed similar thoughts regarding its use in scene construction, such as in film production. 
An interesting suggestion came from participant P4, who proposed its application in crime scene description. 
She pointed out that witnesses are often not skilled artists, and typically describe crime scenes verbally while someone else illustrates their account. 
With this system, witnesses could more easily express what they saw themselves, potentially producing depictions closer to the real events. "\emph{Details like object locations and distances from buildings can be easily conveyed using the system}," she added.

% \subsection{Model Understanding of Users' Implicit Intents}
% In region-sketch-based control of generative models, a significant gap between interaction design and actual implementation is the model's failure in understanding users' naturally expressed intentions.
% For example, when users draw sketches of two people with their hands slightly overlapping, current region-based models cannot automatically infer users' implicit intention that the two people are holding hands.
% Instead, they still require users to explicitly specify in the prompt such relationship.
% \tool addresses this through sketch-aware prompt recommendation to fill in the necessary semantic information, alleviating users' workload.
% However, some users want the generative AI in the future to be able to directly infer this natural implicit intentions from the sketches without additional prompting since prompt recommendation can still be unstable sometimes.
% This problem reflects a more general dilemma, which ubiquitously exists in all forms of conditioned control for generative models such as canny or scribble control.
% This is because all the control models are trained on pairs of explicit control signal and target image, which is lacking further interpretation or customization of the user intentions behind the seemingly straightforward input.
% For another example, the generative models cannot understand what abstraction level the user has in mind for her personal scribbles.
% Such problems leave more challenges to be addressed by future human-AI co-creation research.
% One possible direction is fine-tuning the conditioned models on individual user's conditioned control data to provide more customized interpretation. 

% \subsection{Balance between recommendation and autonomy}
% AIGC tools are a typical example of 
\subsection{Progressive Sketching}
Currently \tool is mainly aimed at novice users who are only capable of creating very rough sketches by themselves.
However, more accomplished painters or even professional artists typically have a coarse-to-fine creative process. 
Such a process is most evident in painting styles like traditional oil painting or digital impasto painting, where artists first quickly lay down large color patches to outline the most primitive proportion and structure of visual elements.
After that, the artists will progressively add layers of finer color strokes to the canvas to gradually refine the painting to an exquisite piece of artwork.
One participant in our user study (P1) , as a professional painter, has mentioned a similar point "\emph{
I think it is useful for laying out the big picture, give some inspirations for the initial drawing stage}."
Therefore, rough sketch also plays a part in the professional artists' creation process, yet it is more challenging to integrate AI into this more complex coarse-to-fine procedure.
Particularly, artists would like to preserve some of their finer strokes in later progression, not just the shape of the initial sketch.
In addition, instead of requiring the tool to generate a finished piece of artwork, some artists may prefer a model that can generate another more accurate sketch based on the initial one, and leave the final coloring and refining to the artists themselves.
To accommodate these diverse progressive sketching requirements, a more advanced sketch-based AI-assisted creation tool should be developed that can seamlessly enable artist intervention at any stage of the sketch and maximally preserve their creative intents to the finest level. 

\subsection{Ethical Issues}
Intellectual property and unethical misuse are two potential ethical concerns of AI-assisted creative tools, particularly those targeting novice users.
In terms of intellectual property, \tool hands over to novice users more control, giving them a higher sense of ownership of the creation.
However, the question still remains: how much contribution from the user's part constitutes full authorship of the artwork?
As \tool still relies on backbone generative models which may be trained on uncopyrighted data largely responsible for turning the sketch into finished artwork, we should design some mechanisms to circumvent this risk.
For example, we can allow artists to upload backbone models trained on their own artworks to integrate with our sketch control.
Regarding unethical misuse, \tool makes fine-grained spatial control more accessible to novice users, who may maliciously generate inappropriate content such as more realistic deepfake with specific postures they want or other explicit content.
To address this issue, we plan to incorporate a more sophisticated filtering mechanism that can detect and screen unethical content with more complex spatial-semantic conditions. 
% In the future, we plan to enable artists to upload their own style model

% \subsection{From interactive prompting to interactive spatial prompting}


\subsection{Limitations and Future work}

    \textbf{User Study Design}. Our open-ended task assesses the usability of \tool's system features in general use cases. To further examine aspects such as creativity and controllability across different methods, the open-ended task could be improved by incorporating baselines to provide more insightful comparative analysis. 
    Besides, in close-ended tasks, while the fixing order of tool usage prevents prior knowledge leakage, it might introduce learning effects. In our study, we include practice sessions for the three systems before the formal task to mitigate these effects. In the future, utilizing parallel tests (\textit{e.g.} different content with the same difficulty) or adding a control group could further reduce the learning effects.

    \textbf{Failure Cases}. There are certain failure cases with \tool that can limit its usability. 
    Firstly, when there are three or more objects with similar semantics, objects may still be missing despite prompt recommendations. 
    Secondly, if an object's stroke is thin, \tool may incorrectly interpret it as a full area, as demonstrated in the expert results of the open-ended task in Figure~\ref{fig:novice_expert}. 
    Finally, sometimes inclusion relationships (\textit{e.g.} inside) between objects cannot be generated correctly, partially due to biases in the base model that lack training samples with such relationship. 

    \textbf{More support for single object adjustment}.
    Participants (N=4) suggested that additional control features should be introduced, beyond just adjusting size and location. They noted that when objects overlap, they cannot freely control which object appears on top or which should be covered, and overlapping areas are currently not allowed.
    They proposed adding features such as layer control and depth control within the single-object mask manipulation. Currently, the system assigns layers based on color order, but future versions should allow users to adjust the layer of each object freely, while considering weighted prompts for overlapping areas.

    \textbf{More customized generation ability}.
    Our current system is built around a single model $ColorfulXL-Lightning$, which limits its ability to fully support the diverse creative needs of users. Feedback from participants has indicated a strong desire for more flexibility in style and personalization, such as integrating fine-tuned models that cater to specific artistic styles or individual preferences. 
    This limitation restricts the ability to adapt to varied creative intents across different users and contexts.
    In future iterations, we plan to address this by embedding a model selection feature, allowing users to choose from a variety of pre-trained or custom fine-tuned models that better align with their stylistic preferences. 
    
    \textbf{Integrate other model functions}.
    Our current system is compatible with many existing tools, such as Promptist~\cite{hao2024optimizing} and Magic Prompt, allowing users to iteratively generate prompts for single objects. However, the integration of these functions is somewhat limited in scope, and users may benefit from a broader range of interactive options, especially for more complex generation tasks. Additionally, for multimodal large models, users can currently explore using affordable or open-source models like Qwen2-VL~\cite{qwen} and InternVL2-Llama3~\cite{llama}, which have demonstrated solid inference performance in our tests. While GPT-4o remains a leading choice, alternative models also offer competitive results.
    Moving forward, we aim to integrate more multimodal large models into the system, giving users the flexibility to choose the models that best fit their needs. 
    


\section{Conclusion}\label{sec:conclusion}
In this paper, we present \tool, an interactive system designed to help novice users create high-quality, fine-grained images that align with their intentions based on rough sketches. 
The system first refines the user's initial prompt into a complete and coherent one that matches the rough sketch, ensuring the generated results are both stable, coherent and high quality.
To further support users in achieving fine-grained alignment between the generated image and their creative intent without requiring professional skills, we introduce a decompose-and-recompose strategy. 
This allows users to select desired, refined object shapes for individual decomposed objects and then recombine them, providing flexible mask manipulation for precise spatial control.
The framework operates through a coarse-to-fine process, enabling iterative and fine-grained control that is not possible with traditional end-to-end generation methods. 
Our user study demonstrates that \tool offers novice users enhanced flexibility in control and fine-grained alignment between their intentions and the generated images.

\section{Acknowledgements}


% \section*{Acknowledgments}
% The research was carried out at the Jet Propulsion Laboratory, California Institute of Technology, under a contract with the National Aeronautics and Space Administration (80NM0018D0004). For this review version: \textcopyright2025. All rights reserved.

%% Use plainnat to work nicely with natbib. 

\bibliographystyle{plainnat}
%\bibliography{references}
\bibliography{paper}

\end{document}


