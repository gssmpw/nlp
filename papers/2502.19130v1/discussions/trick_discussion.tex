
\vspace{5mm}

\definecolor{lightyellow}{RGB}{255, 255, 204}


\begin{tcolorbox}[title=Overview, colframe=black!10, coltitle=black, fonttitle=\bfseries, boxrule=0.5mm, width=\textwidth]

\textbf{Task:} Answer the following question. If the question is not answerable with the provided information, write '[UNKNOWN]'.

\textbf{Input:} What provides critical support for drug discovery and the availability of economic resources?

\textbf{Context:} Biodiversity provides critical support for drug discovery and the availability of medicinal resources. A significant proportion of drugs are derived, directly or indirectly, from biological sources: at least 50\% of the pharmaceutical compounds on the US market are derived from plants, animals, and micro-organisms, while about 80\% of the world population depends on medicines from nature (used in either modern or traditional medical practice) for primary healthcare. Only a tiny fraction of wild species has been investigated for medical potential. Biodiversity has been critical to advances throughout the field of bionics. Evidence from  analysis and biodiversity science indicates that the decline in output from the pharmaceutical sector since the mid-1980s can be attributed to a move away from natural product exploration ("bioprospecting") in favor of genomics and synthetic chemistry, indeed claims about the value of undiscovered pharmaceuticals may not provide enough incentive for companies in free markets to search for them because of the high cost of development; meanwhile, natural products have a long history of supporting significant economic and health innovation. Marine ecosystems are particularly important, although inappropriate bioprospecting can increase biodiversity loss, as well as violating the laws of the communities and states from which the resources are taken.

\begin{tcolorbox}[coltitle=black, fonttitle=\bfseries, boxrule=0.5mm, width=\textwidth]
\textbf{MALLM solution:} Biodiversity provides critical support for drug discovery and the availability of economic resources. \\
\textbf{Real solution:} Not included in context!
\end{tcolorbox}

\end{tcolorbox}


\begin{tcolorbox}[breakable, enhanced jigsaw, title=Turn 1, colframe=black!10, coltitle=black, fonttitle=\bfseries, boxrule=0.5mm, width=\textwidth]

\begin{tcolorbox}[title=Economist \hfill \tiny{\textnormal{A researcher who studies the impact of biodiversity on economic systems, with expertise in market analysis and the economic benefits of natural resources.}}, colback=lightyellow, colframe=orange!30, coltitle=black, fonttitle=\bfseries, boxrule=0.5mm, width=\textwidth]
\begin{minipage}{1.0\textwidth}
As an economist, I'd like to propose a solution that addresses the decline in output from the pharmaceutical sector since the mid-1980s and leverages the untapped potential of biodiversity for drug discovery and economic innovation. Here's a multi-faceted approach: **Solution:** Establish a Public-Private Partnership (PPP) to incentivize the exploration and development of natural products for pharmaceuticals, while ensuring responsible and sustainable bioprospecting practices.
\end{minipage}
\end{tcolorbox}

\begin{tcolorbox}[title=Ethnobotanist \hfill \tiny{\textnormal{A scientist who studies the traditional uses of plants by indigenous communities and their potential for medicinal applications, with expertise in deciphering the cultural and historical context of natural resources.}}, colback=lightyellow, colframe=orange!30, coltitle=black, fonttitle=\bfseries, boxrule=0.5mm, width=\textwidth]
\begin{minipage}{1.0\textwidth}
As an ethnobotanist, I agree with the current solution in principle. The Public-Private Partnership (PPP) framework has the potential to bring together the necessary stakeholders and incentivize the exploration and development of natural products for pharmaceuticals while ensuring responsible and sustainable bioprospecting practices. The multistep approach outlined is comprehensive and addresses various aspects of the issue.
\end{minipage}
\end{tcolorbox}

\begin{tcolorbox}[title=Research Biologist \hfill \tiny{\textnormal{A scientist who studies the properties and potential applications of natural products from a wide range of living organisms, with expertise in the discovery and development of new medicines from biodiversity sources.}}, colback=lightyellow, colframe=orange!30, coltitle=black, fonttitle=\bfseries, boxrule=0.5mm, width=\textwidth]
\begin{minipage}{1.0\textwidth}
I, as a Research Biologist, AGREE with the current solution and the additional suggestions proposed by the Ethnobotanist. The Public-Private Partnership (PPP) framework is a well-rounded approach that addresses the decline in output from the pharmaceutical sector by providing a framework for responsible and sustainable bioprospecting, while promoting economic innovation and job creation. I particularly appreciate the emphasis on community engagement and empowerment, benefit-sharing agreements, traditional knowledge protection, and capacity building for ecosystem-based medicine. 
\end{minipage}
\end{tcolorbox}

\end{tcolorbox}


\begin{tcolorbox}[breakable, enhanced jigsaw, title=Turn 2, colframe=black!10, coltitle=black, fonttitle=\bfseries, boxrule=0.5mm, width=\textwidth]

\begin{tcolorbox}[title=Economist \hfill \tiny{\textnormal{A researcher who studies the impact of biodiversity on economic systems, with expertise in market analysis and the economic benefits of natural resources.}}, colback=lightyellow, colframe=orange!30, coltitle=black, fonttitle=\bfseries, boxrule=0.5mm, width=\textwidth]
\begin{minipage}{1.0\textwidth}
    The current solution has been improved with additional considerations from the ethnobotanist and research biologist. I notice that the additional suggestions enhance the comprehensive approach, addressing community engagement, benefit-sharing, and traditional knowledge protection.
\end{minipage}
\end{tcolorbox}

\begin{tcolorbox}[title=Ethnobotanist \hfill \tiny{\textnormal{A scientist who studies the traditional uses of plants by indigenous communities and their potential for medicinal applications, with expertise in deciphering the cultural and historical context of natural resources.}}, colback=lightyellow, colframe=orange!30, coltitle=black, fonttitle=\bfseries, boxrule=0.5mm, width=\textwidth]
\begin{minipage}{1.0\textwidth}
As an ethnobotanist, I agree with the current solution in principle. The Public-Private Partnership (PPP) framework has the potential to bring together the necessary stakeholders and incentivize the exploration and development of natural products for pharmaceuticals while ensuring responsible and sustainable bioprospecting practices. The multistep approach outlined is comprehensive and addresses various aspects of the issue.
\end{minipage}
\end{tcolorbox}

\begin{tcolorbox}[title=Research Biologist \hfill \tiny{\textnormal{A scientist who studies the properties and potential applications of natural products from a wide range of living organisms, with expertise in the discovery and development of new medicines from biodiversity sources.}}, colback=lightyellow, colframe=orange!30, coltitle=black, fonttitle=\bfseries, boxrule=0.5mm, width=\textwidth]
\begin{minipage}{1.0\textwidth}
I, as a Research Biologist, AGREE with the current solution and the additional suggestions proposed by the Ethnobotanist. The Public-Private Partnership (PPP) framework is a well-rounded approach that addresses the decline in output from the pharmaceutical sector by providing a framework for responsible and sustainable bioprospecting, while promoting economic innovation and job creation. I particularly appreciate the emphasis on community engagement and empowerment, benefit-sharing agreements, traditional knowledge protection, and capacity building for ecosystem-based medicine. These considerations will ensure that local communities are involved in the decision-making process, receive fair compensation for their involvement, and that their traditional knowledge and cultural heritage are respected.

\end{minipage}
\end{tcolorbox}

\end{tcolorbox}


\begin{tcolorbox}[breakable, enhanced jigsaw, title=Turn 3, colframe=black!10, coltitle=black, fonttitle=\bfseries, boxrule=0.5mm, width=\textwidth]

\begin{tcolorbox}[title=Economist \hfill \tiny{\textnormal{A researcher who studies the impact of biodiversity on economic systems, with expertise in market analysis and the economic benefits of natural resources.}}, colback=lightyellow, colframe=orange!30, coltitle=black, fonttitle=\bfseries, boxrule=0.5mm, width=\textwidth]
\begin{minipage}{1.0\textwidth}
[AGREE] The revised solution has effectively incorporated the suggestions from the ethnobotanist and research biologist, addressing community engagement, benefit-sharing, traditional knowledge protection, and other crucial aspects of bioprospecting. The emphasis on responsible and sustainable bioprospecting ensures that the development of novel pharmaceuticals benefits both humans and the environment. To further refine the solution, I agree with the research biologist's suggestion to develop a clear and measurable framework for assessing the impact of the Public-Private Partnership (PPP) on the pharmaceutical sector, local communities, and the environment. Establishing Key Performance Indicators (KPIs) and a monitoring and evaluation process will help ensure that the PPP is held accountable for its impact and remains a responsible and effective approach to bioprospecting.
\end{minipage}
\end{tcolorbox}

\begin{tcolorbox}[title=Ethnobotanist \hfill \tiny{\textnormal{A scientist who studies the traditional uses of plants by indigenous communities and their potential for medicinal applications, with expertise in deciphering the cultural and historical context of natural resources.}}, colback=lightyellow, colframe=orange!30, coltitle=black, fonttitle=\bfseries, boxrule=0.5mm, width=\textwidth]
\begin{minipage}{1.0\textwidth}
As an ethnobotanist, I agree with the current solution in principle. The Public-Private Partnership (PPP) framework has the potential to bring together the necessary stakeholders and incentivize the exploration and development of natural products for pharmaceuticals while ensuring responsible and sustainable bioprospecting practices. The multistep approach outlined is comprehensive and addresses various aspects of the issue.
\end{minipage}
\end{tcolorbox}

\begin{tcolorbox}[title=Research Biologist \hfill \tiny{\textnormal{A scientist who studies the properties and potential applications of natural products from a wide range of living organisms, with expertise in the discovery and development of new medicines from biodiversity sources.}}, colback=lightyellow, colframe=orange!30, coltitle=black, fonttitle=\bfseries, boxrule=0.5mm, width=\textwidth]
\begin{minipage}{1.0\textwidth}
I, as a Research Biologist, AGREE with the current solution and the additional suggestions proposed by the Ethnobotanist. The Public-Private Partnership (PPP) framework is a well-rounded approach that addresses the decline in output from the pharmaceutical sector by providing a framework for responsible and sustainable bioprospecting, while promoting economic innovation and job creation. I particularly appreciate the emphasis on community engagement and empowerment, benefit-sharing agreements, traditional knowledge protection, and capacity building for ecosystem-based medicine. These considerations will ensure that local communities are involved in the decision-making process, receive fair compensation for their involvement, and that their traditional knowledge and cultural heritage are respected.

\end{minipage}
\end{tcolorbox}



\end{tcolorbox}


\begin{tcolorbox}[title=Result, colframe=black!10, coltitle=black, fonttitle=\bfseries, boxrule=0.5mm, width=\textwidth]
\textbf{Economist final answer:} Biodiversity

\textbf{Ethnobotanist final answer:} Biodiversity provides critical support for drug discovery and the availability of economic resources.

\textbf{Research Biologist final answer:} Biodiversity

\bigskip

\textbf{Economist} voted for answer from \textbf{Ethnobotanist}

\textbf{Ethnobotanist} voted for answer from \textbf{Ethnobotanist}

\textbf{Research Biologist} voted for answer from \textbf{Ethnobotanist}


\end{tcolorbox}

\label{example:trick_discussion}
