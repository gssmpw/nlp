 \pdfoutput=1

\documentclass[11pt]{article}
\usepackage{float}

\usepackage{acl}

\usepackage[utf8]{inputenc}
\usepackage{pgfplots}
\usepackage{dsfont}

\DeclareUnicodeCharacter{2212}{−}
\usepgfplotslibrary{groupplots,dateplot}
\usetikzlibrary{patterns,shapes.arrows}
\pgfplotsset{compat=newest}

\usepackage{tikzscale}
\usepackage{amsmath}

\usepackage{multirow, colortbl}
\usepackage{color}
\usepackage{array}
\usepackage{multirow}
\usepackage{CJKutf8}
% \usepackage[usenames,dvipsnames,svgnames,table]{xcolor}
% Define your own color
\usepackage{tabularx,booktabs}
\usepgfplotslibrary{groupplots}
\usepackage{makecell}
\usepackage{enumitem}
\usepackage{placeins}

\usepackage[normalem]{ulem}
\usepackage{graphicx}
\usepackage{booktabs}

\usepackage{pifont}
\newcommand{\cmark}{\ding{51}} % Check mark
\newcommand{\xmark}{\ding{55}} % X mark

\usepackage{pgfplots}
\pgfplotsset{width=10cm,compat=1.9}

\definecolor{ablation6}{HTML}{fcefed}
\definecolor{ablation_tie}{HTML}{fce3e1}

\definecolor{ablation5}{HTML}{fcd8d4}
\definecolor{ablation4}{HTML}{FBC3BC}
\definecolor{ablation3}{HTML}{F7A399}
\definecolor{ablation2}{HTML}{F38375}
\definecolor{ablation1}{HTML}{EF6351}

% \definecolor{ablation1}{HTML}{e3faf8}
% \definecolor{ablation2}{HTML}{C4FFF9}
% \definecolor{ablation3}{HTML}{9CEAEF}
% \definecolor{ablation4}{HTML}{68D8D6}
% \definecolor{ablation5}{HTML}{3DCCC7}
% \definecolor{ablation6}{HTML}{07BEB8}



\usepackage[]{algpseudocode}
\usepackage[]{algorithm}
\usepackage{float}
\algtext*{EndFor}%
\algtext*{EndProcedure}%

\usepackage{booktabs}
% Standard package includes
\usepackage{times}
\usepackage{latexsym}
\usepackage{adjustbox}


% For proper rendering and hyphenation of words containing Latin characters (including in bib files)
\usepackage[T1]{fontenc}
\usepackage[utf8]{inputenc}
\usepackage[russian,english]{babel}
\newcommand{\russian}[1]{{\fontencoding{T2A}\selectfont\foreignlanguage{russian}{#1}}}


\usepackage{amsmath}
\usepackage{amssymb}
\usepackage{booktabs}
\usepackage{multirow}
% For Vietnamese characters
% \usepackage[T5]{fontenc}
% See https://www.latex-project.org/help/documentation/encguide.pdf for other character sets
\usepackage{scalerel,xparse}
% This assumes your files are encoded as UTF8
\usepackage[utf8]{inputenc}
%\usepackage[dvipsnames]{xcolor}

\renewcommand{\floatpagefraction}{.8}%
\renewcommand{\textfraction}{.1}%
\setcounter{totalnumber}{5}

% This is not strictly necessary, and may be commented out,
% but it will improve the layout of the manuscript,
% and will typically save some space.
\usepackage{cleveref}
\usepackage{microtype}
\usepackage{enumitem}\usepackage[utf8]{inputenc}
\usepackage{pgfplots}
\usepackage{dsfont}

\DeclareUnicodeCharacter{2212}{−}
\usepgfplotslibrary{groupplots,dateplot}
\usetikzlibrary{patterns,shapes.arrows}
\pgfplotsset{compat=newest}

\usepackage{tikzscale}
\usepackage{amsmath}
\newcommand{\probP}{\text{I\kern-0.15em P}}

\usepackage{multirow, colortbl}
\usepackage{color}
\usepackage{array}
\usepackage{multirow}
\usepackage{CJKutf8}
% \usepackage[usenames,dvipsnames,svgnames,table]{xcolor}
% Define your own color
\usepackage{tabularx,booktabs}
\usepgfplotslibrary{groupplots}
\usepackage{makecell}
\usepackage{enumitem}
\usepackage{placeins}

\usepackage{soul}
\definecolor{light blue}{RGB}{215, 242, 252}
\definecolor{light purple}{RGB}{247, 215, 252}
\definecolor{light orange}{rgb}{0.9961, 0.875, 0.7188}
\sethlcolor{light blue}
\newcommand{\hlpurple}[1]{\sethlcolor{light purple}\hl{#1}\sethlcolor{light blue}}
\newcommand{\hlorange}[1]{\sethlcolor{light orange}\hl{#1}\sethlcolor{light blue}}

\usepackage{graphicx}
\usepackage{booktabs}

\usepackage{pgfplots}
\pgfplotsset{width=10cm,compat=1.9}

\definecolor{ablation6}{HTML}{fcefed}
\definecolor{ablation_tie}{HTML}{fce3e1}

\definecolor{ablation5}{HTML}{fcd8d4}
\definecolor{ablation4}{HTML}{FBC3BC}
\definecolor{ablation3}{HTML}{F7A399}
\definecolor{ablation2}{HTML}{F38375}
\definecolor{ablation1}{HTML}{EF6351}

% \definecolor{ablation1}{HTML}{e3faf8}
% \definecolor{ablation2}{HTML}{C4FFF9}
% \definecolor{ablation3}{HTML}{9CEAEF}
% \definecolor{ablation4}{HTML}{68D8D6}
% \definecolor{ablation5}{HTML}{3DCCC7}
% \definecolor{ablation6}{HTML}{07BEB8}



\usepackage[]{algpseudocode}
\usepackage[]{algorithm}
\usepackage{float}
\algtext*{EndFor}%
\algtext*{EndProcedure}%


% Standard package includes
\usepackage{times}
\usepackage{latexsym}
\usepackage{adjustbox}


% For proper rendering and hyphenation of words containing Latin characters (including in bib files)
\usepackage[T1]{fontenc}
\usepackage[utf8]{inputenc}
\usepackage[russian,english]{babel}

\usepackage{amsmath}
\usepackage{amssymb}
\usepackage{booktabs}
\usepackage{multirow}
% For Vietnamese characters
% \usepackage[T5]{fontenc}
% See https://www.latex-project.org/help/documentation/encguide.pdf for other character sets
\usepackage{scalerel,xparse}
% This assumes your files are encoded as UTF8
\usepackage[utf8]{inputenc}
%\usepackage[dvipsnames]{xcolor}

\renewcommand{\floatpagefraction}{.8}%
\renewcommand{\textfraction}{.1}%
\setcounter{totalnumber}{5}

% This is not strictly necessary, and may be commented out,
% but it will improve the layout of the manuscript,
% and will typically save some space.
\usepackage{cleveref}
\usepackage{microtype}
\usepackage{enumitem}
\usepackage{placeins}

\crefformat{section}{\S#2#1#3}
\crefformat{subsection}{\S#2#1#3}
\crefformat{subsubsection}{\S#2#1#3}

\newcommand\mymathop[1]{\mathop{\operatorname{#1}}}


% -------------- prompt box setup ---------------- %
% \definecolor{bggray}{rgb}{0.95, 0.95, 0.95}
% \usepackage[%
%     framemethod=tikz,
%     skipbelow=\topskip,
%     skipabove=\topskip
% ]{mdframed}
% \DeclareUnicodeCharacter{2212}{−}
% \usepgfplotslibrary{groupplots,dateplot}
% \usetikzlibrary{patterns,shapes.arrows}
% \pgfplotsset{compat=newest}


% comments
\definecolor{zoey green}{rgb}{0.684,0.836,0.227}
%\usepackage{xcolor}
\newcommand{\ensuretext}[1]{#1}
%Marine
\newcommand{\mcmarker}{\ensuretext{\textcolor{magenta}{\ensuremath{^{\textsc{M}}_{\textsc{C}}}}}}
%Aquia
\newcommand{\armarker}{\ensuretext{\textcolor{blue}{\ensuremath{^{\textsc{A}}_{\textsc{R}}}}}}
%Marianna
\newcommand{\mjmarker}{\ensuretext{\textcolor{cyan}{\ensuremath{^{\textsc{M}}_{\textsc{M}}}}}}
%Eleftheria
\newcommand{\ebmarker}{\ensuretext{\textcolor{green}{\ensuremath{^{\textsc{E}}_{\textsc{B}}}}}}
%Sweta
\newcommand{\samarker}{\ensuretext{\textcolor{orange}{\ensuremath{^{\textsc{S}}_{\textsc{A}}}}}}
%Calvin
\newcommand{\cbmarker}{\ensuretext{\textcolor{purple}{\ensuremath{^{\textsc{C}}_{\textsc{B}}}}}}
%Zoey
\newcommand{\zkmarker}{\ensuretext{\textcolor{zoey green}{\ensuremath{^{\textsc{Z}}_{\textsc{K}}}}}}

%Review
\newcommand{\rxmarker}{\ensuretext{\textcolor{cyan}{\ensuremath{^{\textsc{R}}_{\textsc{X}}}}}}

% enable comments here
\newcommand{\mycomment}[3]{\ensuretext{\textcolor{#3}{[#1 #2]}}}
%disable comments here
%\newcommand{\zk}[1]{\ignore{#1}}
%\newcommand{\mycomment}[3]{}
\newcommand{\mc}[1]{\mycomment{\mcmarker}{#1}{magenta}}
\newcommand{\ar}[1]{\mycomment{\armarker}{#1}{blue}}
\newcommand{\mjm}[1]{\mycomment{\mjmarker}{#1}{cyan}}
\newcommand{\eb}[1]{\mycomment{\ebmarker}{#1}{green}}
\newcommand{\wx}[1]{\mycomment{\wxmarker}{#1}{purple}}
\newcommand{\sa}[1]{\mycomment{\samarker}{#1}{orange}}
\newcommand{\rx}[1]{\mycomment{\rxmarker}{#1}{cyan}}
\newcommand{\zk}[1]{\mycomment{\zkmarker}{#1}{zoey green}}
\newcommand{\ignore}[1]{}


% \title{Rewritten Inputs, Refined Outputs: Enhancing Machine Translation through Source Text Rewrites}
% \title{Does Rewriting Inputs Improve Translations from Large Language Models?}
% \title{Rewriting Inputs for Translation with Large Language Models}
% \title{\textit{Rewritten} Inputs, \textit{Refined} Outputs: \\
% Does Rewriting with Language Models Improve Translation?}
\title{Automatic Input Rewriting Improves Translation \\ with Large Language Models}

% \mc{or should we mention simplification?}
% \zk{I think current title is okay without mentioning simplification}

\author{Dayeon Ki \\
  University of Maryland \\
  \texttt{dayeonki@umd.edu} \\\And
  Marine Carpuat \\
  University of Maryland \\
  \texttt{marine@cs.umd.edu} \\}


\begin{document}
\maketitle


% \mdfsetup{%
%     leftmargin=0pt,
%     rightmargin=0pt,
%     backgroundcolor=bggray,
%     middlelinecolor=black,
%     roundcorner=3
% }
% \newtcolorbox[list inside=prompt,auto counter,number within=section]{prompt}[1][]{
%     colbacktitle=black!60,
%     fonttitle=\small,
%     coltitle=white,
%     fontupper=\footnotesize,
%     boxsep=4pt,
%     left=0pt,
%     % right=0pt,
%     top=0pt,
%     bottom=0pt,
%     boxrule=1pt,
%     #1,
% }


\begin{abstract}
Can we improve machine translation (MT) with LLMs by rewriting their inputs automatically? Users commonly rely on the intuition that well-written text is easier to translate when using off-the-shelf MT systems. LLMs can rewrite text in many ways but in the context of MT, these capabilities have been primarily exploited to rewrite outputs via post-editing. We present an empirical study of 21 input rewriting methods with 3 open-weight LLMs for translating from English into 6 target languages. We show that text simplification is the most effective MT-agnostic rewrite strategy and that it can be improved further when using quality estimation to assess translatability. Human evaluation further confirms that simplified rewrites and their MT outputs both largely preserve the original meaning of the source and MT. These results suggest LLM-assisted input rewriting as a promising direction for improving translations.\footnote{We release our code and dataset at \url{https://github.com/dayeonki/rewrite_mt}.}


% using the \textsc{Tower} LLM \citep{alves2024tower},\mc{Need to update to reflect experiments actually included in the paper}

% \mc{The single most important part of the human evaluation is meaning preservation since it is harder to assess automatically, so the take-away needs to be mentioned here, and in all the other relevant spots in the paper (intro, conclusion, etc.).} 


% \mc{X tasks}

% \mc{Is the 21 number still correct?}
% \zk{Yes, I counted and it totals 21 methods.}

% We first consider stylistic rewrites that are agnostic to MT, and tailor them to the MT task by assessing input translatability with quality estimation.

%sign  according to MT-agnostic style dimensions, an stylisti
%Rewriting inputs is a common strategy to enhance translation quality exploited by end users and in dedicated machine translation (MT) architectures.

%how effective is this approach when translating with Large Language Models (LLMs)? 

%We conduct an empirical study using the Tower multilingual LLM, primarily trained for translation-related tasks, across two distinct rewriting methods: \textbf{MT-Agnostic} (without translation-related knowledge) and \textbf{MT-Aware} (with translation signal), yielding three key findings. First, MT-agnostic style rewrites do not uniformly improve translations but re-ranking and fine-tuning to guide rewrites with reference-free quality estimation improves translation quality according to both automatic and human evaluations. Second, when a rewrite is easier to translate, it often fails to preserve the original meaning, which poses a challenge of Pareto optimization. Third, rewriting inputs complements post-editing outputs, demonstrating the effectiveness of combining both strategies to further improve translation quality\footnote{We will release all models, datasets, and code.}.

% \footnote{We release our code and dataset at \url{https://github.com/dayeonki/rewrite_mt}.}.


% \mc{TODOs in preparation for author response:
% (1) do a read through post deadline to catch any remaining issues
% (2) manual evaluation
% (3) possibly try other LLMs with the winning strategies (Aya? larger Tower?)
% (4) possibly try other held-out test sets that are not out-of-En lang pairs
% (5) possibly optimize simplification+MT prompts to maximize translatability and/or reference-based metrics on a devset through DSPy
% }

\end{abstract}

\section{Introduction}

\begin{figure*}[h!]
  \centering
  \includegraphics[width=\textwidth]{materials/Model2.pdf}
  \caption{The overview of our proposed AV2T-SAM framework.} 
  % \NZ{The boundaries of Adapters textbox are missing. The boundary of the Projector textbox is overlapping with the arrow. Please fix it.}
  \label{fig:model_structure}
\end{figure*}


Machine translation (MT) users and developers have long exploited the idea that some texts are easier to translate than others. For instance, guiding people to edit their inputs so that they are well formed is a cornerstone of MT literacy courses \citep{bowker-2021-promoting,steigerwald-etal-2022-overcoming}, and adopting plain language has been shown to improve the readability of translated health content \citep{Rossetti2019}. In MT research, a wealth of studies have considered pre-processing strategies to rewrite inputs, particularly for statistical MT \citep{XiaMcCord2004,callison-burch-etal-2006-improved,stajner-popovic-2016-text}.%TODO: expand cites at camera-ready or in related work section

%In MT research, a wealth of studies have considered pre-processing strategies to rewrite inputs, particularly to improve statistical MT via input paraphrasing \citep{callison-burch-etal-2006-improved, mirkin-etal-2009-source, marton-etal-2009-improved, aziz-etal-2010-learning}, reordering \citep{XiaMcCord2004,WangCollinsKoehn2007}, or simplification \citep{stajner-popovic-2016-text,stajner-popovic-2019-automated} or to design systems that help users pre-edit their inputs \citep{mirkin-etal-2013-sort,Miyata2017DissectingHP}. 

The growing use of Large Language Models (LLMs) for translation leads us to revisit the impact of rewriting inputs on MT. On the one hand, rewriting inputs for LLM translation aligns with the re-framing of MT as a multi-step process \citep{Briakou}. LLMs have shown promise in rewriting MT outputs \citep{ki2024guiding, zeng2024improving, xu2024llmrefine}, and can rewrite text according to various style specifications \citep{raheja-etal-2023-coedit, hallinan2023steer, shu2023rewritelm, dipper}. On the other hand, current models might already be robust to input variability, since they are trained on vast amounts of heterogeneous data \citep{touvron2023llama}, fine-tuned on diverse tasks \citep{raffel-etal-2020-exploring,alves2024tower} and operate at a much higher quality level compared to the statistical MT systems used in previous pre-processing studies.%todo later: add cite for this.

How should inputs be rewritten for MT? The assumption that well-written texts are easier to translate drives recommendations for MT literacy, as well as the use of paraphrasing \citep{callison-burch-etal-2006-improved, mirkin-etal-2009-source, marton-etal-2009-improved, aziz-etal-2010-learning} and simplification  \citep{stajner-popovic-2016-text,stajner-popovic-2019-automated}. However, can we more directly rewrite inputs so that they are easier to translate? Generic translatability has been defined as “a measurement of the time and effort
it takes to translate a text” \citep{kumhyr-etal-1994-internationalization}. \citet{uchimoto-etal-2005-automatic} introduced a metric to quantify MT translatability based on back-translation of MT hypotheses in the source language. Given recent progress in quality estimation \citep{fernandes-etal-2023-devil, naskar-etal-2023-quality, tomani2024qualityaware}, we propose instead to use reference-free quality estimation scores as a measure of translatability.

We thus ask the following research questions:
\begin{enumerate}[label=(\arabic*),topsep=0pt,itemsep=-1ex,partopsep=-1ex,parsep=1ex]
    \item Can we improve MT quality from LLMs by rewriting inputs for style?
    \item Do quality estimation metrics provide useful translatability signals for input rewriting?
\end{enumerate}

% We conduct an empirical study of the \textsc{Tower-Instruct} LLM \citep{alves2024tower} for a total of 21 input rewriting methods with varying levels of MT-awareness on translation from English into German, Russian and Chinese, and we further evaluate the generalizability of our best performing approach on translation from English into Czech, Hebrew and Japanese. 
We conduct an empirical study with 3 open-weight LLMs for a total of 21 input rewriting methods with varying levels of MT-awareness on translation from English into German, Russian and Chinese, and we further evaluate the generalizability of our best performing approach on translation from English into Czech, Hebrew and Japanese (\S \ref{sec:newlanguages}). 
Our results show that simple \textbf{MT-Agnostic rewrites} obtained by prompting LLMs to simplify, paraphrase, or change the style of the input, improve translatability, and that simplification most reliably improves translation quality. Interestingly, these MT-agnostic rewrites are more effective than \textbf{Task-Aware rewrites}, where LLMs are prompted to rewrite inputs for the purpose of MT (\S \ref{simplification best}). Finally, using quality estimation signals to assess \textbf{translatability} at the segment level and select when to use rewrites further improves MT quality, outperforming more expensive fine-tuning strategies (\S \ref{input selection}). Human evaluation further confirms that simplified rewrites and their MT largely preserve the original meaning of the source and MT (\S \ref{human evaluation}).

% \mc{X New Tasks}

%We conduct an empirical study of the Tower LLM \citep{alves2024tower} for a total of 21 input rewriting methods with varying levels of MT-awareness on translation from English into German, Russian and Chinese. We further show that the benefits of our best performing approach generalize to new test sets from the WMT23 general translation task for X language pairs.

%To address these questions, we first generate \textbf{MT-Agnostic rewrites} by prompting LLMs to simplify, paraphrase or change the style of the original input (\S \ref{3.1 mt-agnostic}). 

%Next, we design three strategies to obtain \textbf{MT-Aware rewrites}: using Chain-of-Thought \citep{wei2023chainofthought} prompting, selecting generic rewrites with reference-free quality estimation metrics \citep{guerreiro2023xcomet}, and fine-tuning LLMs to rewrite for translation (\S \ref{3.2 task-aware}).
%
%We conduct an empirical study of the Tower LLM \citep{alves2024tower} for a total of 21 input rewriting methods with varying levels of MT-awareness on translation from English into German, Russian and Chinese. Our findings suggest that rewriting inputs can improve translation quality (\S \ref{res:mt agnostic}) and that quality estimation feedback helps generate inputs that are better translated (\S \ref{res:mt aware}), even outperforming post-editing translation outputs (\S \ref{res:post-editing}).

\definecolor{light blue}{RGB}{215, 242, 252}
\definecolor{light purple}{RGB}{247, 215, 252}
\definecolor{light orange}{rgb}{0.9961, 0.875, 0.7188}

\section{Input Rewriting Methods}

\label{3 method}
Within the process of source rewriting, the goal of a rewrite model is to rewrite the original source sentence $s$ into another form that is easier to translate while preserving its intended meaning. For \textbf{MT-Agnostic} rewriting methods (\S \ref{3.1 mt-agnostic}), which lacks translation-related knowledge, the rewrite model $\mathcal{M}_{\theta}$ can rewrite $s$ into $s'$:
\begin{equation}
    s' = \mathcal{M}_{\theta}(s)
\end{equation}

On the contrary, both \textbf{Task-Aware} (\S \ref{3.2 task-aware}) and \textbf{Translatability-Aware} (\S \ref{3.3 translatability-aware}) rewriting methods incorporate some translation signal. For Task-Aware, $\mathcal{M}_{\theta}$ rewrites $s$ with the information of the end-task (MT):

\begin{equation}
    s' = \mathcal{M}_{\theta}(s, \text{MT task})
\end{equation}

For Translatability-Aware method, it rewrites with the knowledge of segment level quality estimation scores between source and the output of a specific MT system MT($t$):

\begin{equation}
    s' = \mathcal{M}_{\theta}(s, \text{\textsc{xCOMET}}(s,\text{MT}(t)))
\end{equation}
Figure~\ref{fig:main_figure} shows the overview of our proposed rewriting pipeline. To find the most effective $\mathcal{M}_{\theta}$, we test a total of 21 input rewriting methods.


\subsection{\fcolorbox{white}{light blue}{\raisebox{-0.2em}{\includegraphics[height=1em]{figures/logos/agnostic.png}} MT-Agnostic} Rewriting}
\label{3.1 mt-agnostic}
MT-agnostic rewriting methods reflect various a priori assumptions on what makes text easier to translate. They do not take as input any signal of translatability or knowledge about the end-task. We consider three prompting variants here, all inspired by prior works on source rewriting \citep{mirkin-etal-2009-source, mirkin-etal-2013-sort, stajner-popovic-2016-text}.

\paragraph{Simplification.}
Simplification includes replacing complex words with simpler ones, rephrasing complex syntactic structures, and shortening sentences \citep{article, Feng2008}. Prior works show that simplified inputs are more conducive to MT, and particularly improve the fluency of MT outputs \citep{stajner-popovic-2019-automated}.

\paragraph{Paraphrase.}
Paraphrases are alternative ways of expressing the same information within one language, which can help resolve unknown or complex words \citep{callison-burch-etal-2006-improved}. Paraphrasing with LLMs might benefit MT by normalizing inputs using language patterns that are more frequent in LLM training data. Further, some LLMs, such as \textsc{Tower} \citep{alves2024tower}, are fine-tuned on both paraphrasing and MT tasks, and might thus produce paraphrases that are useful for MT.

\paragraph{Stylistic.}
We employ an off-the-shelf text editing tool \textsc{CoEdIT-XL} \citep{raheja-etal-2023-coedit} to rewrite inputs according to diverse style specifications:
\begin{itemize}[leftmargin=*, itemsep=2pt, parsep=-1pt]
 \item \textbf{Grammar}: Fix the grammar.
 \item \textbf{Coherent}: Make the text more coherent.
 \item \textbf{Understandable}: Make it easier to understand.
 \item \textbf{Formal}: Rewrite the text more formally.
\end{itemize}
These operationalize the assumption that well-formed text is easier to translate.
%In addition to translation-related tasks, Tower-Instruct is trained on Grammatical Error Correction (GEC) task, which motivates us to consider the Grammar prompt. The Coherent and Understandable prompts function similarly to the Simplification rewrites, making the text easier to translate by using a dedicated text editing tool. Further, we explore the impact of increasing formality on MT quality \citep{lee-etal-2023-improving-formality} with Formal prompt.
All prompt templates are shown in Appendix Table~\ref{tab:prompting_template}.

\subsection{\fcolorbox{white}{light purple}{\raisebox{-0.2em}{\includegraphics[height=1em]{figures/logos/task.png}} Task-Aware} Rewriting}
\label{3.2 task-aware}

For task-aware rewriting methods, we design prompts that account for the fact that rewrites are aimed at MT. 
Prior work has shown that LLMs can post-edit errors in MT outputs \citep{ki2024guiding, zeng2024improving, treviso-etal-2024-xtower, xu2024llmrefine, briakou-etal-2024-translating}, raising the question of whether this ability can be extended to rewriting inputs to enhance translatability. Additionally, \textsc{Tower-Instruct} has been jointly trained on paraphrasing, grammatical error correction (GEC), and translation tasks, suggesting it may be well-suited for performing translatability rewrites in a zero-shot fashion. We consider two prompting strategies (Refer to Appendix Table~\ref{tab:prompting_template} for exact templates):

\paragraph{Easy Translation.} We prompt LLMs to rewrite inputs in a way that specifically facilitates translation into the target language.

\paragraph{Chain of Thought Rewrite+Translate.} We use a Chain of Thought (\citet{wei2023chainofthought}, CoT) style prompt where LLMs are prompted to handle the entire rewriting and translation process in one sequence of CoT instructions within a single model.



\subsection{\fcolorbox{white}{light orange}{\raisebox{-0.2em}{\includegraphics[height=1em]{figures/logos/translatability.png}} Translatability-Aware} Rewriting}
\label{3.3 translatability-aware}
We propose to use quality estimation scores for a given input and output pair to assess the translatability of inputs at the segment level. This makes it possible to inject translatability signals at inference or training time. We introduce a lightweight inference-time selection strategy, and contrast it against a more expensive fine-tuning approach.

\paragraph{Inference-Time Selection.}
Input segments might not benefit from rewriting uniformly, since the quality of the original inputs and of their rewrites might vary. We thus propose to use translatability scores to decide whether or not to replace the original input with a rewrite  at inference time. We use the state-of-the-art \textsc{xCOMET} quality estimation tool \citep{guerreiro2023xcomet} to assess how good the translation $t'$ of a rewrite $s'$ is: \textsc{xCOMET}$(s',t')$. We compare this score with the estimated quality of the translation $t$ of the original source $s$, choosing to use the rewrite if \textsc{xCOMET}$(s',t')$ > \textsc{xCOMET}$(s,t)$, and keeping the original source otherwise. This straightforward approach allows us incorporate translatability signals at inference time, with little additional cost.

% we consider the best-performing rewrite among MT-Agnostic and Task-Aware methods. 

\paragraph{Supervised Fine-tuning.}
The translatability-based selection process described above for inference could also be used to gather examples of good rewrites and enable instruction fine-tuning of models to rewrite text for improved translation.
%
%
%Quantifying whether a given rewrite improves translatability at the segment level as above makes it possible to training data, and thus fine-tuning
%We explore fine-tuning LLMs to enhance their capability in rewriting the source for improved translation.
While designing an optimal approach for this task is out of scope for this work,  we wish to compare our inference-time selection strategy with a straightforward training strategy. We construct a fine-tuning dataset of positive rewrite examples $\mathcal{D}_{pos}$, as follows: for a given input $s$, we generate rewrites using all MT-agnostic methods. We add to our training set the rewrites that improve translatability as measured by \textsc{xCOMET}$(s', t')$ > \textsc{xCOMET}$(s,t)$. The base LLM is then instruction fine-tuned based to rewrite input $s$ so that it is better translated, using $s'$ as supervision. Detailed prompt templates are shown in Appendix \ref{appendix:prompt_templates}.

%The original data is drawn from the English-German and English-Russian subset from WMT-20, 21, and 22 General MT task datasets \citep{freitag2021experts}\footnote{We do not consider English-Chinese pair here since this language pair is not supported in the dataset.}. We first prompt LLMs using MT-agnostic rewriting methods to generate rewrites $s'$ for each $s$. Then, we translate the original source and rewrites using Tower-Instruct 7B to get MT outputs $t$ and $t'$. We use \textsc{xCOMET} to compute scores between $s'$ and $t'$, which estimates translation quality for each rewrite. Our goal of fine-tuning is to generate better rewrites than the original source. Thus, we create $\mathcal{D}_{pos}$, a subset of $\mathcal{D}_{ft}$, which only contains the rewrites where \textsc{xCOMET}$(s', t')$ > \textsc{xCOMET}$(s,t)$. 

%For prompt template, we compare three variants:
%\begin{itemize}[leftmargin=*, itemsep=2pt, parsep=-1pt]
% \item \textbf{Basic}: Given $s$, generate $s'$.
% \item \textbf{MT}: Given $s$ and $t$, generate $s'$.
% \item \textbf{Reference}: Given $s$ and $r$ (reference translation), generate $s'$.
%\end{itemize}
%Specific prompt templates are outlined in Appendix \ref{appendix:prompt_templates}. All parameters are listed in Appendix \ref{appendix:parameters}.

\section{Experimental Setup}


% Please add the following required packages to your document preamble:

% Beamer presentation requires \usepackage{colortbl} instead of \usepackage[table,xcdraw]{xcolor}
\begin{table*}[t]
\centering
\caption{Main Results. Eurus-2-7B-PRIME demonstrates the best reasoning ability.}
\label{tab:main_results}
\resizebox{\textwidth}{!}{
\begin{tabular}{lcccccc}
\toprule
\textbf{Model}                     & \textbf{AIME 2024}                           & \textbf{MATH-500} & \textbf{AMC}          & \textbf{Minerva Math} & \textbf{OlympiadBench} & \textbf{Avg.}          \\ \midrule
\textbf{GPT-4o}                    & 9.3                                          & 76.4              & 45.8                  & 36.8                  & \textbf{43.3}          & 43.3                   \\
\textbf{Llama-3.1-70B-Instruct}    & 16.7                                         & 64.6              & 30.1                  & 35.3                  & 31.9                   & 35.7                   \\
\textbf{Qwen-2.5-Math-7B-Instruct} & 13.3                                         & \textbf{79.8}     & 50.6                  & 34.6                  & 40.7                   & 43.8                   \\
\textbf{Eurus-2-7B-SFT}            & 3.3                                          & 65.1              & 30.1                  & 32.7                  & 29.8                   & 32.2                   \\
\textbf{Eurus-2-7B-PRIME}          & \textbf{26.7 {\color[HTML]{009901} (+23.3)}} & 79.2 {\color[HTML]{009901}(+14.1)}      & \textbf{57.8 {\color[HTML]{009901}(+27.7)}} & \textbf{38.6 {\color[HTML]{009901}(+5.9)}}  & 42.1 {\color[HTML]{009901}(+12.3) }          & \textbf{48.9 {\color[HTML]{009901}(+ 16.7)}} \\ \bottomrule
\end{tabular}
}
\end{table*}


\subsection{Model \& Data}

\paragraph{MT System.} We use \textsc{Tower-Instruct} 7B as our MT system for all our experiments since it is specifically trained for translation-related tasks and has demonstrated superior MT performance compared to other LLMs \citep{alves2024tower}.

\paragraph{Rewriting Models.} For prompting experiments, we use 7B variant of three open-weight LLMs in zero-shot setting: \textsc{LLaMA-2} \citep{touvron2023llama} \---\ the base model for \textsc{Tower-Instruct}, \textsc{LLaMA-3} \citep{grattafiori2024llama3herdmodels} \---\ more recent multilingual model compared to \textsc{LLaMA-2}, and \textsc{Tower-Instruct} \citep{alves2024tower} \---\ the same LLM as used for our MT system.\footnote{The HuggingFace model names are detailed in Appendix Table \ref{tab:huggingface_api}.} For supervised fine-tuning, we draw training samples from the English-German and English-Russian subset from WMT-20, 21, and 22 General MT task datasets \citep{freitag2021experts}\footnote{We do not consider English-Chinese pair here since this language pair is not supported in the dataset.}, and provide detailed parameter settings in Appendix~\ref{appendix:parameters}.


%We first prompt LLMs using MT-agnostic rewriting methods to generate rewrites $s'$ for each $s$. Then, we translate the original source and rewrites using Tower-Instruct 7B to get MT outputs $t$ and $t'$. We use \textsc{xCOMET} to compute scores between $s'$ and $t'$, which estimates translation quality for each rewrite. Our goal of fine-tuning is to generate better rewrites than the original source. Thus, we create $\mathcal{D}_{pos}$, a subset of $\mathcal{D}_{ft}$, which only contains the rewrites where \textsc{xCOMET}$(s', t')$ > \textsc{xCOMET}$(s,t)$. 
%For prompt template, we compare three variants:
%\begin{itemize}[leftmargin=*, itemsep=2pt, parsep=-1pt]
% \item \textbf{Basic}: Given $s$, generate $s'$.
% \item \textbf{MT}: Given $s$ and $t$, generate $s'$.
% \item \textbf{Reference}: Given $s$ and $r$ (reference translation), generate $s'$.
%\end{itemize}
%Specific prompt templates are outlined in Appendix \ref{appendix:prompt_templates}.


\paragraph{Test Data.}
We use the WMT-23 General MT task\footnote{\url{https://www2.statmt.org/wmt23/translation-task.html}} from the \textsc{TowerEval} dataset\footnote{\url{https://huggingface.co/datasets/Unbabel/TowerEval-Data-v0.1}} to guarantee that it was held out from the various training stages. We focus on translation from English into German (\textsc{En-De}), Russian (\textsc{En-Ru}) and Chinese (\textsc{En-Zh}) for an extensive empirical comparison, and then test whether the most promising approaches generalize to translation from English into Czech (\textsc{En-Cs}), Hebrew (\textsc{En-He}) and Japanese (\textsc{En-Ja}). % We use combination of dev and test split from the original dataset\footnote{We translate each source sentence using our MT system and do not use the original NLLB 3B \citep{nllbteam2022language} translations.}. Reference translations are human-produced directly based on the source sentences.
See Appendix Table~\ref{tab:dataset_details} for data statistics.

% \mc{the key quesiton is how? directly based on the source? or are the reference translations generated by asking people to post-edit NLLB output?}
% \zk{I checked again the Tower paper, TowerEval huggingface page and the dataset I used -- the datasets that TowerEval name as "WMT23 Automatic Post-Edition" does not corresponds to actual WMT23 Automatic Post-edition (only has En-Marathi pair). The dataset here are source and reference from WMT23 general mt task with translations from nllb model. So I'll reframe it as "we combine development and test sets of WMT23 General MT task" and also get rid of en-ru, en-zh results as held-out test sets.}
% \mc{okay}

% \mc{How were the references created? Are they translatinos of the source from scratch or are they created by post-editing the NLLB output? if it is the latter it might create a weird bias.}

\subsection{Evaluation Metrics}
We use \textsc{xCOMET} \citep{guerreiro2023xcomet} and \textsc{MetricX} \citep{juraska-etal-2023-metricx} to evaluate different aspects of rewrite quality. Specifically, we use \textsc{xCOMET-XL}\footnote{\url{https://huggingface.co/Unbabel/XCOMET-XL}} and \textsc{MetricX-23-XL}.\footnote{\url{https://huggingface.co/google/metricx-23-xl-v2p0}} Higher scores indicate better performance for \textsc{xCOMET}, while lower scores are better with \textsc{MetricX}.


\paragraph{Translatability.}
%Generic \textit{translatability} has been defined as ``a measurement of the time and effort it takes to translate a text'' \citep{kumhyr-etal-1994-internationalization}. Here, we define translatability as a measure of how well a given source sentence can be translated by a particular MT system. More translatable inputs yield better MT outputs \citep{uchimoto-etal-2005-automatic}, so 
We quantify translatability with the quality estimation score for a specific input--output pair (\textsc{xCOMET}$(s', t')$ or \textsc{MetricX-QE}$(s',t')$). A rewrite $s'$ of the original input $s$ is considered easier to translate if \textsc{xCOMET}$(s', t')$ is higher than \textsc{xCOMET}$(s, t)$.

\paragraph{Meaning Preservation.} We do not want rewrites that are easier to translate at the expense of changing the original meaning. Our meaning preservation metric evaluates how well the rewrite maintains the intended meaning of the translation as represented by the reference \citep{Graham2015CanMT}. We use a reference-based metric as opposed to using the semantic similarity between $s$ and $s'$ because it abstracts the meaning away from the specific formulation of $s$, reducing overfitting. We compute \textsc{xCOMET} scores between the rewrites and reference translations (\textsc{xCOMET}$(s',r)$). The desired behavior is to minimize the deterioration in \textsc{xCOMET}$(s',r)$ compared to \textsc{xCOMET}$(s,r)$.


\paragraph{Translation Quality.} We additionally report the combined evaluation metric, \textsc{xCOMET}$(s',t',r)$ to take into account of the trade-off between the two above metrics, and \textsc{MetricX}$(t',r)$ which also assesses translation quality of the rewrite but is not informed by the updated source $s'$.

%\paragraph{\textsc{MetricX}.} In addition to \textsc{xCOMET}, we consider \textsc{MetricX-23-XL} \citep{juraska-etal-2023-metricx} as a secondary evaluation metric. We use two score variants: \textsc{MetricX}$(s',t')$ for quality estimation and \textsc{MetricX}$(t',r)$ as a reference-based score.
\definecolor{light blue}{RGB}{215, 242, 252}
\definecolor{light purple}{RGB}{247, 215, 252}
\definecolor{light orange}{rgb}{0.9961, 0.875, 0.7188}



\section{Results}
\label{4 results}

We first extensively compare rewrite strategies focusing on the overall translation quality achieved by MT-Agnostic rewrites (\S \ref{simplification best}) and Translatability-Aware rewrites (\S \ref{input selection}). To understand how rewrites change translations, we then analyze the trade-offs between translatability and meaning preservation (\S \ref{pareto optimality}). Finally, we test whether the best-performing methods identified so far generalize to new language pairs (\S \ref{sec:newlanguages}).


\subsection{Simplifying Inputs Works Best}
\label{simplification best}
We first compare the \hl{MT Agnostic} rewriting methods: simplification, paraphrasing, and stylistic edits. Due to space limits, we show the best and worst performing variations for each input rewriting method based on the overall translation quality metric \textsc{xCOMET}$(s,t,r)$ for each language pair in Table~\ref{tab:main_results}. Full results are available in Appendix \ref{appendix:detailed results}.

Results show that all rewriting strategies improve translatability, but only \textbf{simplification} also improves the overall translation quality. Even the lowest performing rewrites reach higher translatability than the original baseline. Each method surpasses the baseline by up to 0.056 and 0.027 \textsc{xCOMET}$(s,t)$ average scores for \textsc{En-De}, up to 0.058 and 0.036 average scores for \textsc{En-Ru}, and up to 0.054 and 0.028 average scores for \textsc{En-Zh} pair. Trends are consistent with \textsc{MetricX}$(s,t)$. However, making inputs easier to translate often degrades quality when comparing against references $r$. Simplification with \textsc{Tower-Instruct} distinguishes itself by improving translation quality based on \textsc{xCOMET}$(s,t,r)$ scores and maintaining it according to the \textsc{MetricX}$(t,r)$ scores \---\ a harder metric to improve since the reference might be biased toward the original wording of the source.

%However, as detailed in Appendix Table~\ref{tab:detailed_results_ende} to \ref{tab:detailed_results_enzh}, this improvement comes at the cost of lower meaning preservation score, resulting in lower overall translation quality \textsc{xCOMET}$(s,t,r)$ scores for most methods. A notable exception is simplification, which outperforms the the original MT in the \textsc{xCOMET}$(s,t,r)$ metric.

Among the three LLMs used for simplification, \textsc{Tower-Instruct} achieves the best translation quality, while \textsc{LLaMA-3} excels in translatability at the expense of meaning preservation. Interestingly, there is no benefit to using a separate LLM, even one fine-tuned specifically on paraphrasing or style edits such as \textsc{DIPPER} or \textsc{CoEdIT}. Overall, the best performing method for MT-agnostic rewrites is simplification with \textsc{Tower-Instruct}, the same model we use as our MT system. We attribute this to \textsc{Tower-Instruct} being instruction fine-tuned on translation related tasks (but not simplification) and having more domain knowledge of the WMT dataset used in our evaluation.\footnote{\url{https://huggingface.co/datasets/Unbabel/TowerBlocks-v0.1}} 


% Further details are provided in Appendix \ref{appendix:impact of llm} and \ref{appendix:same llm}.

As shown in Table~\ref{tab:main_results}, simplifying with \textsc{Tower-Instruct} still holds the top spot when compared to \hlpurple{Task-Aware} rewriting methods, as indicated by higher \textsc{xCOMET}$(s,t,r)$ scores. This suggests that injecting knowledge about the end-task (MT) to LLMs is less effective than simplifying inputs to improve translation quality.

Overall, these results confirm the intuition that simpler text is easier to translate, but establish that rewrites are not uniformly helpful for translation quality, motivating the need for more selective input rewriting strategies.

%We observe similar trends with our secondary evaluation metric, \textsc{MetricX}. Simplification with \textsc{Tower-Instruct} consistently improves both \textsc{MetricX}$(s,t)$ and \textsc{MetricX}$(t,r)$ scores over the baseline; -0.534 and -0.015 for \textsc{En-De}, -1.4 and -0.107 for \textsc{En-Ru}, and -1.924 and -0.055 for \textsc{En-Zh}. However, note that the latter score may be inherently biased toward the original MT outputs since it does not use the source as part of input.

\begin{figure*}
    \centering
    \includegraphics[width=\linewidth]{figures/pareto_vis.pdf}
    \caption{Pareto frontier per language pair. For each subplot, the $x$-axis is the translatability and $y$-axis is the meaning preservation scores. Pareto frontier (\textbf{dashed} line) visualizes the optimal solutions that take into account the trade-off between the two metrics. Each shape represents different rewriting methods and each color represent specific prompt or model variation.}
    \label{fig:pareto_frontier}
\end{figure*}


\subsection{Selection via Translatability Improves MT}
\label{input selection}

We evaluate the impact of inference-time selection based on \hlorange{translatability} scores (\textit{Selection} in Table~\ref{tab:main_results}), and compare it further with the more expensive supervised fine-tuning strategy (\textit{Fine-tune}). 

All language pairs consistently benefit from selection. Translation quality improves significantly, with average \textsc{xCOMET}$(s,t,r)$ gains of 0.024 for \textsc{En-De}, 0.031 for \textsc{En-Ru}, and 0.025 for \textsc{En-Zh}, marking the best performance among all variants. \textsc{MetricX}$(t,r)$ scores confirm this trend, showing average improvements of 0.073 for \textsc{En-De}, 0.198 for \textsc{En-Ru}, and 0.076 for \textsc{En-Zh}. At the segment level, rewrites are preferred to original inputs in 1197/1557 cases for \textsc{En-De}, 1610/2074 cases for \textsc{En-Ru}, and 2163/3074 cases for \textsc{En-Zh}. Fine-tuning shows smaller gains compared to MT-Agnostic or Task-Aware methods, both in terms of translatability and translation quality, despite being more resource-intensive.

In summary, the results suggest that inference-time selection of inputs based on translatability scores is a promising strategy, outperforming MT-agnostic rewrites and rewrites obtained via a more expensive fine-tuning process.
% \footnote{We also found no benefit from adding preference learning with Direct Preference Optimization \citep{rafailov2023direct}. More details are provided in Appendix \ref{appendix:dpo}.}

\subsection{Input Rewriting Trades Off Translatability and Meaning Preservation}
\label{pareto optimality}

We observe a moderate negative correlation between translatability and meaning preservation scores, with Pearson coefficients of -0.48, -0.66, and -0.52 for \textsc{En-De}, \textsc{En-Ru}, and \textsc{En-Zh}, respectively. This trade-off between the two metrics poses a Pareto optimization challenge: when a rewrite is easier to translate, it often results in lower meaning preservation. Therefore, we aim to find Pareto optimal solutions, which balance these trade-offs on a Pareto frontier \citep{huang-etal-2023-towards}.\footnote{In Pareto optimization, Pareto optimal solutions are those where no single solution outperforms another in all tasks \citep{pareto}. The set of Pareto optimal solutions forms the Pareto frontier.}

In Figure~\ref{fig:pareto_frontier}, we visualize our two objectives, translatability and meaning preservation, on each axis and identify the Pareto frontier. The results are consistent with the overall translation quality metric, \textsc{xCOMET}$(s,t,r)$, where the scores for rewriting methods on the Pareto frontier are consistently the same as or on par with the original baseline. This also aligns with our earlier findings from comparing MT-Agnostic and Task-Aware rewrites (\S \ref{simplification best}), where simplification with \textsc{Tower-Instruct} lies on the Pareto frontier for \textsc{En-De} and \textsc{En-Ru}. Even for \textsc{En-Zh}, although this does not lie on the frontier, it has a higher \textsc{xCOMET}$(s,t,r)$ score (0.802) than the original baseline (0.794). Furthermore, the best rewriting method according to \textsc{xCOMET}$(s,t,r)$, translatability-based selection (\S \ref{input selection}), always lies on the Pareto frontier across all language pairs.




\begin{table}[!htp]
\centering
\resizebox{\linewidth}{!}{%
    \begin{tabular}{l l l l l l l}
    \specialrule{1.3pt}{0pt}{0pt}
    \textbf{Language} & \textbf{Type} & \textbf{\textsc{x}$(s,t)$} & \textbf{\textsc{x}$(s,t,r)$} & \textbf{\textsc{M}$(s,t)$} & \textbf{\textsc{M}$(t,r)$} \\
    \toprule
        
    \multirow{3}{*}{\textbf{\large{\textsc{en-cs}}}} & Original & 0.646 & 0.655 & 5.376 & 4.493 \\
    & Simplification & 0.691 & 0.675 & 4.684 & 4.333\\
    & Selection & \textbf{0.736} & \textbf{0.718} & \textbf{4.152} & \textbf{3.663}\\ \midrule

    \multirow{3}{*}{\textbf{\large{\textsc{en-he}}}} & Original & 0.327 & 0.320 & 16.66 & 15.48 \\
    & Simplification & 0.351 & 0.332 & 15.97 & 15.43 \\
    & Selection & \textbf{0.389} & \textbf{0.363} & \textbf{15.39} & \textbf{14.51} \\ \midrule

    \multirow{3}{*}{\textbf{\large{\textsc{en-ja}}}} & Original & 0.746 & 0.718 & 3.514 & 2.688 \\
    & Simplification & 0.789 & 0.738 & 2.957 & 2.508 \\
    & Selection & \textbf{0.826} & \textbf{0.769} & \textbf{2.781} & \textbf{2.273} \\

    % \multirow{3}{*}{\textbf{\large{\textsc{en-ru}}}} & Original & 0.876 & 0.872 & 2.504 & 1.947 \\
    % & Simplification & 0.911 & 0.891 & 2.046 & 1.840 \\
    % & Selection & \textbf{0.924} & \textbf{0.901} & \textbf{2.015} & \textbf{1.765} \\ \midrule

    % \multirow{3}{*}{\textbf{\large{\textsc{en-zh}}}} & Original & 0.811 & 0.837 & 3.003 & 1.743 \\
    % & Simplification & 0.840 & 0.842 & 2.649 & 1.817 \\
    % & Selection & \textbf{0.857} & \textbf{0.868} & \textbf{2.612} & \textbf{1.689} \\

    \specialrule{1.3pt}{0pt}{0pt}
    \end{tabular}
}
\caption{Results of simplification and translatability-based selection for held-out test sets. We abbreviate \textsc{xCOMET} to \textbf{\textsc{x}} and \textsc{MetricX} to \textbf{\textsc{M}} due to space constraints. Best scores for each metric is \textbf{bold}.} 
\label{tab:heldout}
\end{table}

\subsection{Best Input Rewriting Strategy Improves MT on Held-out Test sets}
\label{sec:newlanguages}

We evaluate whether the top methods that have emerged from the controlled empirical comparison conducted so far generalize to further test settings. As shown in Table~\ref{tab:heldout}, we test both simplification with \textsc{Tower-Instruct} (\textit{Simplification}) and translatability-based input selection (\textit{Selection}) on new test sets from the  WMT-23 General MT task, English-Czech (\textsc{En-Cs}), English-Hebrew (\textsc{En-He}), and English-Japanese (\textsc{En-Ja}) to assess generalization to lower-resource target languages.

Both simplification and translatability-based selection lead to progressive improvements in translation quality, as measured by \textsc{xCOMET}$(s,t,r)$. Notably, the selection strategy tends to excel in language pairs with lower-resource target languages, showing translation quality gains of 0.064, 0.043, 0.051 scores for \textsc{En-Cs}, \textsc{En-He}, \textsc{En-Ja}, respectively, compared to increases of 0.017, 0.031, and 0.025 for \textsc{En-De}, \textsc{En-Ru} and \textsc{En-Zh}. At the segment level, rewrites are also more preferred over original inputs, selected in 1395/2074 cases for \textsc{En-Cs}, 1309/2074 for \textsc{En-He}, and 1411/2074 for \textsc{En-Ja}. \textsc{MetricX} trends are consistent.

In sum, our findings generalize well to held-out test sets, further validating the effectiveness of the translatability-based selection strategy. This approach offers a practical and scalable solution for input rewriting across a broader range of domains and language pairs, though there are many other dimensions that remain unexplored. We have conducted initial experiments with additional LLMs and source languages, shown in Appendix \ref{appendix:more_llms} and \ref{appendix:more_lang_pairs}, which confirms our previous findings that simplification rewriting enhances translation quality. We leave a more comprehensive exploration of this direction for future work.

\section{\RealEdit dataset analysis}
\label{sec:dataset_analysis}
% We looked at requests submitted on reddit
% We decided to analyze what kind of requests they are
% What did we use?
% gpt 4o to taxonomize promtps
% Here is what we noticed people care about.
% We see that models are unable to perform these requests. 
% some tasks are impossible right now. Like make a vector. Resizing is imporssible for many models too. 

\RealEdit provides insight into practical applications of image editing by analyzing real-world requests. We observe notable differences between \RealEdit and existing datasets including InstructPix2Pix~\cite{brooks2023instructpix2pix}, MagicBrush~\cite{zhang2024magicbrush}, Emu Edit~\cite{sheynin2024emu}, HIVE~\cite{zhang2024hive}, Ultra Edit~\cite{zhao2024ultraedit}, AURORA~\cite{krojer2024learning}, Image Editing Request~\cite{tan2019expressing} and GIER~\cite{shi2020benchmark}. 
While we focus primarily on differences with MagicBrush~\cite{zhang2024magicbrush} and Emu Edit~\cite{sheynin2024emu} in the following discussions, these observations broadly apply across datasets used to train image editing models. Figure~\ref{fig:dist_us_both} details the main differences. %We summarize the most important distinctions below.

% \begin{figure}[htbp]
%     \centering
%     \includegraphics[width=\linewidth]{figs/taxonomy_ours_fixed_font.png}
%     \caption{Taxonomy of image edit requests in our dataset. There is a wode variety of task types and edit subjects, with subtle tasks like ``remove'' and ``enhance'' being the most requested. \ranjay{I would be ok with moving this to the supplementary if we need more space.}}
%     \label{fig:taxonomy_ours}
% \end{figure}

\paragraph{Qualitative analysis and taxonomy.} %We qualitatively determine the primary subject in a sample of 500 input images from \RealEdit. 
We create a taxonomy of image editing tasks people have requested. This involves (1) categorizing our edit requests into \textit{operations}, (2) subcategorizing requests by \textit{subject} of the edit, and (3) prompting GPT-4o~\cite{openai2023gpt4} to categorize the request based on the input image and edit instruction. To determine the operations, we modify the MagicBrush~\cite{zhang2024magicbrush} set of operations for clarity. We base our possible subjects on the significant categories from the sample of 500 images. We tune our GPT-4o prompt using samples of 100 data points to ensure accuracy, then validate on a separate sample to avoid overfitting. We find that both the categories and their distribution differ greatly from prior work. Since our categorizations are fairly similar to MagicBrush~\cite{zhang2024magicbrush} and Emu Edit~\cite{sheynin2024emu} test sets, we run our taxonomy on these test sets and highlight key distributional differences in Figure~\ref{fig:dist_us_both}. Full taxononomies and comparisons are listed in Appendix.

\begin{figure}[t]
    \centering
    \includegraphics[width=\linewidth]{figs/us_vs_both_abridged.png}
    \caption{\textbf{Key differences in the distribution} of our test set compared to MagicBrush and Emu Edit test sets. MagicBrush and Emu Edit tend to be similar in distribution to each other, but starkly different from \RealEdit.}
    \label{fig:dist_us_both}
    \vspace{-4mm}
\end{figure}

\paragraph{Differences in edit operations.}
Synthetic datasets contain a greater use of ``add'' requests (36\% less than MagicBrush than). In contrast, real-life photos typically contain the intended objects within the frame, with many of \RealEdit's semantically focused tasks involving the \textit{removal} of unintended elements, such as strangers in the background, shadows on faces, or cars on the street.
% \RealEdit contains a larger proportion of ``remove'' requests (diff. of 23\% MagicBrush and 16\% Emu Edit).
Additionally, there are numerous cases where input images are semantically aligned with the owner's intent, but errors in photography such as bad lighting, motion blur, or graininess.
Following this, \RealEdit contains more ``enhance'' requests (14\% greater than MagicBrush and Emu Edit) compared to existing datasets. These findings indicate that real users often prioritize \textit{subtler requests}, whereas synthetic datasets are dominated by larger semantic changes, such as ``add."

\paragraph{Differences in image content.}  Analysis on 500 samples reveals that around 55\% of the input images feature \textit{people} as the main subject. Consequently, the subjects of the requested edits are more likely to be people (13\% more than Emu Edit), and less likely to be man-made objects (20\% less than MagicBrush).
Animals and media (characters, movie/book posters, memes, etc.) are the next most common categories, comprising about 10\% of the test set each. Common media requests include restoring old photographs, participating in fandoms, making memes, or other forms of online entertainment.
The fixation on media is not paralleled in other datasets (15\% more than Emu Edit). 
These findings reveal a clear difference: Reddit users tend to prioritize \textit{personal significance} by including people and \textit{entertainment} by incorporating media, and synthetic datasets often fail to reflect these preferences accurately.



Given the substantial distributional differences of \RealEdit compared to existing datasets, we demonstrate in Section~\ref{sec:results} that current models struggle to perform well on real-world requests.

% \noindent\textbf{Additional considerations.} In \RealEdit, approximately 36\% of images are 1080p, indicating that humans are more concerned with editing higher resolution images.  Instructions exceed 77 tokens in approximately 3.4\% of cases. Editing models should cater to such preferences in resolution and instruction length in order to better serve human users.

% Given the substantial distributional differences of \RealEdit compared to existing datasets, we demonstrate in Section \ref{sec:results} that current models struggle to perform well on editing tasks within the \RealEdit dataset.

% Considering the significant differences in distribution of \RealEdit compared to existing models, we will prove in Section \ref{sec:results} that existing models cannot perform well on editing tasks in the \RealEdit dataset.


\section{Related Work}

\paragraph{Rewriting with LLMs.} 
Recent advances in LLMs have demonstrated impressive zero-shot capabilities in rewriting textual input based on user requirements \citep{shu2023rewritelm}. Most LLM-assisted rewriting tasks focus on query rewriting \citep{efthimiadis1996query}, which aims to reformulate text-based queries to enhance their representativeness and improve recall with retrieval-augmented LLMs \citep{mao-etal-2023-search, Zhu_2024}. Rewriting methods include prompting LLMs both as rewriters and rewrite editors \citep{ye-etal-2023-enhancing, kunilovskaya-etal-2024-mitigating}, and training LLMs as rewriters using feedback alignment learning \citep{ma-etal-2023-query, mao2024rafe}. Another line of work focuses on style transfer, where the goal is to rewrite textual input into a specified style \citep{wordcraft, hallinan2023steer}. Our research aligns with efforts to rewrite texts with LLM assistance; however, unlike these works, we focus on rewriting source inputs to enhance MT quality.

\paragraph{Quality Estimation Metrics.}
% These metrics use complex neural networks to estimate the quality of MT outputs more effectively. 
The discrepancy between lexical-based metrics (e.g., \textsc{BLEU} \citep{papineni-etal-2002-bleu}, \textsc{chrF} \citep{popovic-2015-chrf}) and human judgments \citep{ma-etal-2019-results} has led to research in \textit{neural} metrics. Particularly, quality estimation (QE) metrics, which compute a quality score for the translation conditioned only on the source sentence, have demonstrated benefits in improving MT quality. QE metrics are used for various purposes, including filtering out low-quality translations during training \citep{tomani2024qualityaware}, applying to post-editing workflows \citep{bechara2021role}, and providing feedback to users of MT systems \citep{mehandru2023physician}. In our experiments, we use \textsc{xCOMET} as our main evaluation metric, as it shows the best correlation with human judgments \citep{agrawal2024automatic}. We primarily use \textsc{xCOMET} as a QE metric to compute translatability, further providing this information as knowledge to LLMs to improve MT quality.

\paragraph{Rewriting MT Outputs.} 
The symmetric task of post-editing MT outputs has received significantly more attention than rewriting MT inputs. Most recent work relies on LLMs to automatically detect and correct errors in MT outputs using their internal knowledge \citep{raunak-etal-2023-leveraging, zeng2024improving, chen2024iterative}, with the help of external feedback \citep{ki2024guiding, xu2024llmrefine} or through fine-tuning \citep{treviso2024xtowermultilingualllmexplaining}. In contrast, the task of rewriting MT inputs to make them more suitable for translation has been relatively underexplored with LLMs. While there have been some efforts in query rewriting and style transfer to improve retrieval \citep{mao-etal-2023-search, Zhu_2024} and stylistic coherence \citep{ye-etal-2023-enhancing, hallinan2023steer}, the specific application of LLMs to rewrite inputs for the purpose of enhancing MT quality is still emerging. Our research addresses this gap by focusing on the potential of LLM-assisted input rewriting to improve the translatability and quality of the resulting translations.

% Traditional automatic metrics for MT evaluation rely on lexical-based approaches, calculating the evaluation score based on lexical overlap between a candidate translation and a reference translation (ex. \textsc{BLEU} \citep{papineni-etal-2002-bleu}, \textsc{METEOR} \citep{banerjee-lavie-2005-meteor}, and \textsc{chrF} \citep{popovic-2015-chrf}). However, evidence indicate that these lexical metrics do not consistently correlate with human judgments \citep{ma-etal-2019-results}. This discrepancy led to

% \mc{This paragraph is a little weak --- generic overview of QE methods. It would be much stronger to focus instead on discussing how QE has been sued to improve MT quality in the past.}

% \paragraph{Preference Alignment Learning.}
% An increasing body of work seeks to align LLMs with preference datasets, either collected from humans (RLHF, \citet{ouyang2022training}) or AI (RLAIF, \citet{bai2022constitutional}). However, RLHF procedure is complex, where we need to first fit a reward model over human preferences, and then use reinforcement learning (RL) algorithms such as Proximal Policy Optimization (\citet{schulman2017proximal}, PPO) to find a policy that maximizes the learned reward \citep{ziegler2020finetuning, ouyang2022training}. In contrast, reward-free methods offer simpler training procedure by directly training LLMs on preference without the need of reward modeling or RL \citep{yuan2023rrhf}. Among these, Direct Preference Optimization (\citet{rafailov2023direct}, DPO) has shown strong performance while being simple, thus we adopt DPO for our preference alignment learning.

\section{Conclusion}
\label{sec:Conclusion}
In this paper, we proposed a complete real-time planning and control approach for continuous, reliable, and fast online generation of dynamically feasible Bernstein trajectories and control for FW aircrafts. The generated trajectories span kilometers, navigating through multiple waypoints. By leveraging differential flatness equations for coordinated flight, we ensure precise trajectory tracking. Our approach guarantees smooth transitions from simulation to real-world applications, enabling timely field deployment. The system also features a user-friendly mission planning interface. Continuous replanning  maintains the rajectory curvature 
$\kappa$ within limits, preventing abrupt roll changes.

Future works will include the ability to add  a higher-level kinodynamic path planner to optimize waypoint spatial allocation and improve replanning success, and enhancing the trajectory-tracking algorithm by refining the aerodynamic coefficient estimation. 

\section{Limitations}


\paragraph{Probing Tasks} While the probing tasks we have proposed provide valuable insights into the visual arithmetic capabilities of VLMs, it is important to acknowledge that they may not encompass all possible dimensions of visual reasoning. Our choice to limit the scope of these tasks was intentional, as they serve as initial, simple tests to determine whether VLMs exhibit failure in fundamental aspects of visual arithmetic. These tasks allow us to iterate different experiments in a controlled and efficient manner, providing clear, actionable insights without the complexity that more comprehensive tasks might introduce. However, there is potential to explore additional tasks that involve more complex interactions of basic geometric properties. For instance, tasks requiring the model to simultaneously assess both length and angle, or combinations of length and area, could be valuable for understanding the compositionality of these atomic tasks. \looseness=-1

\paragraph{Training Data Synthesis}The training data synthesis method of \method~ is not only scalable but also effectively enhances the visual arithmetic capabilities of VLMs. Our approach serves as a proof-of-concept, demonstrating the potential of automated data generation for improving models' understanding of basic geometric properties. To further enrich the training data, we could consider utilizing additional configurations for each task. For instance, in generating positive and negative responses, we could leverage LLMs to produce rationales based on the specific configuration of each figure. By including explanations or justifications for why a particular geometric property holds or does not hold, we could foster deeper understanding within the VLMs. \looseness=-1


% \input{page/08 societal_concern}
\section*{Acknowledgments}

We thank the anonymous reviewers and the members of the \textsc{clip} lab at University of Maryland for their constructive feedback. This work was supported in part by NSF Fairness in AI Grant 2147292, by the Institute for Trustworthy AI in Law and Society (TRAILS), which is supported by the National Science Foundation under Award No. 2229885, and by the Office of the Director of National Intelligence (ODNI), Intelligence Advanced Research Projects Activity (IARPA), via the HIATUS Program contract \#2022-22072200006, by NSF grant 2147292. The views and conclusions contained herein are those of the authors and should not be interpreted as necessarily representing the official policies, either expressed or implied, of ODNI, IARPA, NSF or the U.S. Government. The U.S. Government is authorized to reproduce and distribute reprints for governmental purposes notwithstanding any copyright annotation therein.


\bibliography{custom,inputrewrite}

\appendix


\section{Model and Experiment Details}
\subsection{Prompt Templates}
\label{appendix:prompt_templates}
In Tables \ref{tab:prompting_template} and \ref{tab:training_template}, we describe the prompt templates used for prompting and fine-tuning experiments, respectively. For stylistic rewriting, we use the same prompts as those used to train the \textsc{CoEdIT-XL} model. During prompting, we provide the original source as the input, while for fine-tuning, we provide the positive rewrite along with the source.


\subsection{Training Setup}
\label{appendix:parameters}
All models are trained using one NVIDIA RTX A5000 GPU. In practice, we find that fine-tuning converges in around 3 hours. We use a 90/10 train/validation data split and adopt QLoRA \citep{dettmers2023qlora}, a quantized version of LoRA \citep{hu2021lora}, for parameter-efficient training. We train \textsc{Tower-Instruct 7B} with 8-bit quantization, a LoRA rank of 16, a scaling parameter ($\alpha$) of 32, and a dropout probability of 0.05 for layers. We train for 10 epochs. All unspecified hyperparameters are set to default values.


\subsection{Decoding Strategy}
We use greedy decoding (no sampling) when generating rewrites for prompting experiments. We fix the temperature value to 0 throughout the experiments in order to eliminate sampling variations.


\begin{table}
\centering
\resizebox{\linewidth}{!}{%
    \begin{tabular}{ll}
    \specialrule{1.3pt}{0pt}{0pt}
    \textbf{Model} & \textbf{HuggingFace Model Name} \\
    \toprule

    \textsc{LLaMA-2} & \texttt{meta-llama/Llama-2-7b-chat-hf} \\
    \textsc{LLaMA-3} & \texttt{meta-llama/Meta-Llama-3-8B-Instruct} \\
    \textsc{Tower-Instruct} & \texttt{Unbabel/TowerInstruct-7B-v0.1} \\
    \specialrule{1.3pt}{0pt}{0pt}
    \end{tabular}
}
\caption{HuggingFace model names for all tested LLMs.} 
\label{tab:huggingface_api}
\end{table}


\subsection{Dataset Details}
\label{appendix:dataset_details}
We provide detailed statistics of our training ($\mathcal{D}_{pos}$) and test dataset in Table \ref{tab:dataset_details}. For $\mathcal{D}_{pos}$, we only use rewrites where the \textsc{xCOMET}$(s', t')$ score is higher than the original \textsc{xCOMET}$(s, t)$ score. We further conduct a two-step pre-processing procedure: \textbf{1)} Remove duplicate instances and \textbf{2)} Remove lengthy instances where the upper threshold is set as Q$3 + 1.5 \times \text{IQR}$.


\section{Detailed Results}
\subsection{Full Results}
\label{appendix:detailed results}
In Tables \ref{tab:detailed_results_ende} to \ref{tab:detailed_results_enzh}, we present the detailed numerical results for all tested variations. Most rewrites yield higher \textsc{xCOMET}$(s,t)$ scores, indicating better translatability compared to the original baseline. For stylistic rewrites with \textsc{CoEdIT}, prompting to make the text easier to understand (Understandable) achieves the highest translatability score, while prompting to rewrite the text more formally (Formal) results in the highest translation quality. The Coherent prompt achieves the highest meaning preservation score but this is because most rewrites are merely copies of the original source (Appendix \ref{appendix:direct_copy}). Overall, we demonstrate that translatability-based selection method remains the most effective method, even outperforming scores from our fine-tuned LLMs.


\subsection{Impact of LLM}
\label{appendix:impact of llm}
Among the three LLMs used for prompting, \textsc{Tower-Instruct} performs the best in terms of the combined metric \textsc{xCOMET}$(s,t,r)$. Although it lags behind \textsc{LLaMA-2} and \textsc{LLaMA-3} in translatability, its meaning preservation score deteriorates the least, resulting in the highest overall score. \textsc{LLaMA-3} performs the best in terms of translatability, likely due to its more multilingual training data, with over 5\% of its pre-training dataset consisting of high-quality non-English data.\footnote{\url{https://ai.meta.com/blog/meta-LLaMA-3/}} This suggests that the amount of multilingual data in the pre-training phase may enhance the model's ability to generate more translatable rewrites. However, this advantage does not extend when comparing the \textsc{LLaMA} models to \textsc{Tower-Instruct}. Despite being inherently multilingual primarily trained on translation-related tasks, \textsc{Tower-Instruct} performs lower than the \textsc{LLaMA} models in translatability. This discrepancy can be attributed to \textsc{Tower-Instruct} not being specifically trained on rewriting tasks to improve MT quality, highlighting the importance of introducing translation-related knowledge for effective rewriting.


We further compare the results with off-the-shelf paraphrasing (\textsc{DIPPER}) and text-editing (\textsc{CoEdIT-XL}) tools. Despite being specifically trained for rewriting tasks, their rewrites are not as translatable as those generated by the prompted LLMs. For \textsc{DIPPER}, this may be due to its primary focus on paraphrasing, which has been shown to be less effective (\S \ref{simplification best}). In the case of \textsc{CoEdIT}, we attribute the lower performance to the model's smaller size (3B) compared to the 7B LLMs used for prompting.


\subsection{Same LLM vs. Different LLM}
\label{appendix:same llm}
We distinguish whether the LLM being prompted is the same as the one used as the MT system. Initially, we expected the highest improvements when prompting \textsc{Tower-Instruct}, which may incur self-preference bias, where the LLM favors its own outputs due to recognition \citep{panickssery2024llm}. However, our results indicate that prompting \textsc{Tower-Instruct} does not yield the most translatable rewrites. Instead, the LLaMA series models consistently outperform in this aspect. Interestingly, \textsc{Tower-Instruct} consistently produces rewrites that are more meaning-preserving compared to \textsc{LLaMA-2} or \textsc{LLaMA-3}, resulting in higher \textsc{xCOMET}$(s,t,r)$ scores overall. We conclude that prompting the same LLM used for the MT system is not helpful in generating more translatable rewrites, but these rewrites are better at preserving the intended meaning.


% \section{Re-ranking Details}
% \subsection{Top-1 Rewrites}
% A natural question that arises from using a re-ranker is whether the higher-ranked rewrites are consistent across language pairs. To find this, we track the source of the Top-1 rewrites. We increment a count for each rewriting method if the Top-1 rewrite originates from that method. If the Top-1 rewrite is coming from multiple rewriting methods, we count them towards both. As illustrated in Table \ref{tab:top1_rewrites}, Top-3 rewrites are coming from the same rewriting methods across tested language pairs are: simplification \small(\textsc{LLaMA-3})\normalsize, paraphrase \small(\textsc{LLaMA-3})\normalsize, and stylistic \small(\textsc{CoEdIT} Understandable)\normalsize. Thus, the most effective prompting rewriting methods are generalizable across different languages.

% \begin{table}[!htp]
\centering
\resizebox{\linewidth}{!}{%
    \begin{tabular}{l l l l l}
    \toprule
    \multirow{2}{*}{\textbf{Rewrite}} & \multirow{2}{*}{\textbf{Prompt/Model}} & \multicolumn{3}{c}{\textbf{Count (Rank)}} \\
    
    \cmidrule(lr){3-5}
    
    & & \textbf{\textsc{EN-DE}} & \textbf{\textsc{EN-RU}} & \textbf{\textsc{EN-ZH}}  \\ \midrule

    \multirow{3}{*}{\textbf{Simplification}} & LLaMA-2 & 303 (6) & 568 (8) & 397 (8) \\
    & LLaMA-3 & \underline{380 (2)} & \textbf{679 (1)} & \underline{529 (2)} \\
    & Tower-Instruct & 191 (10) & 465 (10) & 277 (10) \\ \midrule

    \multirow{6}{*}{\textbf{Paraphrase}} & LLaMA-2 & 314 (5) & 629 (4) & 413 (6) \\
    & LLaMA-3 & \textit{357 (3)} & \underline{675 (2)} & \textit{483 (3)} \\
    & Tower-Instruct & 174 (11) & 395 (12) & 227 (11) \\
    & \textsc{DIPPER} (L80/O60) & 319 (4) & 588 (5) & 469 (4) \\
    & (L80/O40) & 289 (8) & 583 (6) & 443 (5) \\
    & (L60/O40) & 248 (9) & 588 (5) & 483 (3) \\

    \midrule

    \multirow{4}{*}{\textbf{Stylistic}} & \textsc{CoEdIT} (GEC) & 158 (13) & 403 (11) & 202 (12) \\
    & (Coherent) & 160 (12) & 358 (13) & 196 (13) \\
    & (Understandable) & \textbf{492 (1)} & \textit{639 (3)} & \textbf{591 (1)} \\
    % & (Paraphrase) & 253 (9) & 525 (9) & 280 (10) \\
    & (Formal) & 302 (7) & 574 (7) & 402 (7) \\
    
    
    \bottomrule
    \end{tabular}
}
\caption{Count of Top-1 rewrites per prompting variation. Highest rank is marked in \textbf{bold}, second in \underline{underline}, and third as \textit{italic}.}
\label{tab:top1_rewrites}
\end{table}


% \subsection{Top-$k$ Rewrites}
% \label{appendix:top k rewrites}
% Our default re-ranker considers all 13 MT-agnostic rewriting methods. However, in practical scenarios, generating a large number of rewrites incurs significant costs. To simulate this, we limit our re-ranker to consider only the Top-$k$ rewrites, as outlined in Table \ref{tab:top-k rerank}. Figure \ref{fig:topk_rerank} illustrates the impact of varying $k$ on the \textsc{xCOMET}$(s,t)$ score. We observe that increasing the number of rewriting methods (higher $k$) improves translatability, as evidenced by a consistent increase in the \textsc{xCOMET}$(s,t)$ score. However, this improvement diminishes gradually, particularly for the \textsc{En-De} pair (0.027 → 0.013 → 0.006), where the score converges around Top-3. Overall, this underscores that re-ranking using a small subset of the best rewriting methods is sufficient.

% \begin{table*}[!htp]
\centering
\resizebox{\textwidth}{!}{%
    \begin{tabular}{c l}
    \toprule
    \textbf{Top-k} & \textbf{Rewrite Methods} \\ \midrule
    
    2 & \textsc{CoEdIT} \small{(Understand)}\normalsize{, Simplification} \small{(LLaMA-3)} \\
    3 & \textsc{CoEdIT} \small{(Understand)}\normalsize{, Simplification} \small{(LLaMA-3)}\normalsize{, Paraphrase} \small{(LLaMA-3)} \\
    4 & \textsc{CoEdIT} \small{(Understand)}\normalsize{, Simplification} \small{(LLaMA-3)}\normalsize{, Paraphrase} \small{(LLaMA-3)}\normalsize{, \textsc{DIPPER}} \small{(L80/O60)} \\
    5 & \textsc{CoEdIT} \small{(Understand)}\normalsize{, Simplification} \small{(LLaMA-3)}\normalsize{, Paraphrase} \small{(LLaMA-3)}\normalsize{, \textsc{DIPPER}} \small{(L80/O60)}\normalsize{, Paraphrase} \small{(LLaMA-2)} \\
    
    \bottomrule
    \end{tabular}
}
\caption{Top-$k$ rewrite methods tested.}
\label{tab:top-k rerank}
\end{table*}


\section{Qualitative Evaluation}
\label{appendix: qualitative eval details}
\subsection{Copying Behavior}
\label{appendix:direct_copy}
To prevent LLMs from directly copying the original source, we explicitly state in the prompt to ``\textit{avoid directly copying the source}'' (Appendix \ref{appendix:prompt_templates}). However, we still observe some rewrites that are identical to the source sentence. We count the occurrences and compute the percentage per language pair in Table \ref{tab:direct_copy}. Note that we do not consider Translatability-Aware Selection rewrite method here since this involves selecting whether to keep the original source or use the rewrite based on translatability scores. The highest occurrence appears for stylistic rewrites using the \textsc{CoEdIT-XL} Coherent prompt, where the source is copied most of the time (82.2\%, 91.9\%, 93.2\% for \textsc{En-De}, \textsc{En-Ru}, and \textsc{En-Zh}, respectively).

\begin{figure}
    \centering
    \includegraphics[width=\linewidth]{figures/success_dist.pdf}
    \caption{Distribution of properties of good rewrites.}
    \label{fig:success}
\end{figure}

\subsection{What makes a Good Rewrite for MT?}
Qualitatively examining translation outputs reveals several common patterns, which motivate us to conduct a detailed qualitative analysis. Here, we aim to identify the properties that lead to meaning-equivalent rewrites that are easier to translate. We examine 200 data instances where each rewrite is the highest performing rewrite based on the \textsc{xCOMET}$(s,t)$ score. To focus on successful rewrites, we filter instances where \textsc{xCOMET}$(s',t')$ $>$ \textsc{xCOMET}$(s,t)$. Each rewrite is annotated with the following labels: (1) \textbf{Simplified}: Replaces complex words with simpler ones or reduces structural complexity; (2) \textbf{Detailed}: Adds information for better context; (3) \textbf{Fluency}: Restructures the sentence for better flow and readability. 
Examples of rewrites for each annotation label are in  Table \ref{tab:success_types}. 

As shown in Figure \ref{fig:success}, most successful rewrites are labeled as \textbf{Simplified}. This highlights the effectiveness of simplification, which has been consistently effective even in the context of LLMs. Notably, many simplified rewrites involve changing complex words to simpler, more conventional alternatives (e.g., ``Derry City \textit{emerged victorious} in the President's Cup as they \textit{ran out} 2-0 \textit{winners} over Shamrock Rovers.'' → ``Derry City \textit{won} the President's Cup title by \textit{defeating} Shamrock Rovers 2-0.''). This finding aligns with our conclusions from MT-Agnostic rewriting methods (\S \ref{3.1 mt-agnostic}), where simplification emerged as the best rewrite method among the prompting variations.


\section{Additional Results}
\subsection{Additional LLM Baselines}
\label{appendix:more_llms}
\paragraph{LLMs for Rewriting.}
Our initial experiments consist of 21 input rewriting methods across 3 LLMs (\textsc{LLaMA-2 7B}, \textsc{LLaMA-3 8B}, and \textsc{Tower-Instruct 7B}). In Table \ref{tab:more_rewrite_llms}, we present extended experiment results by applying simplification rewriting with two additional LLMs: \textsc{Aya-23 8B} \citep{aryabumi2024aya23openweight} and \textsc{Tower-Instruct 13B} \citep{alves2024tower}. The results confirms that simplification rewriting improves translation quality measured by \textsc{xCOMET}$(s,t,r)$ compared to the original baseline.

\paragraph{LLMs for MT.}
Furthermore, we initially relied on \textsc{Tower-Instruct 7B} as our MT system for all our experiments since it is specifically trained for translation-related tasks and has demonstrated superior MT performance (\S \ref{3 method}). However, we extend our analysis by comparing the original baseline and our winning strategy (simplification with \textsc{Tower-Instruct 7B}) using two additional LLMs as the MT system. As shown in Table \ref{tab:more_mt_llms}, our method outperforms the original baseline in terms of both the translation quality (\textsc{xCOMET}$(s,t,r)$) and \textsc{MetricX}$(s,t)$, regardless of the LLM used as the MT system.


\subsection{Additional Language Pairs}
\label{appendix:more_lang_pairs}
To assess the generalizability to other source languages, we test two of our winning strategies (simplification with \textsc{Tower-Instruct 7B} and inference-time selection) on seven additional into-English and non-English language pairs from the WMT-23 General MT task test set.\footnote{\url{https://www2.statmt.org/wmt23/translation-task.html}} As shown in Table \ref{tab:more_lang_pairs}, while translatability scores (\textsc{xCOMET}$(s,t)$) improve across all language pairs, translation quality (\textsc{xCOMET}$(s,t,r)$) improvements are less pronounced compared to out-of-English pairs. Notably, gains in translation quality are observed only for German-English (\textsc{De-En}) and Chinese-English (\textsc{Zh-En}) pairs. These results highlight the importance of input rewrites' quality, which is currently higher for high-resource source languages. This motivates further work to strengthen input rewriting for broader range of source languages.



% \section{Direct Preference Optimization}
% \label{appendix:dpo}
% \subsection{Preference Dataset}
% Further motivated by recent efforts to utilize preference learning strategies \citep{rafailov2023direct, xu2024contrastive, xu2024advancing, he2024improving}, we collect triplets of dispreferred (negative) and preferred (positive) rewrites for each source sentence via pairwise comparisons. We then use this signal with Direct Preference Optimization (\citet{rafailov2023direct}, DPO) to further align the initial fine-tuned model with an MT-based objective. Direct Preference Optimization (\citet{rafailov2023direct}, DPO) needs input $x$ and preferred/dispreferred outputs $y_w$/$y_l$, where $y_w > y_l$ amongst preference pair. We set $x$ as the above prompt templates with corresponding inputs from $\mathcal{D}_{ft}$. For each source sentence $s$, we collect a triplet of source, positive (preferred), and negative (dispreferred) rewrite $(s, s'_p, s'_n)$ where we define positive rewrites as rewrites having higher \textsc{xCOMET}$(s,t)$ score and negative rewrites as those having lower score than the original. We consider all combinations of positive and negative rewrites, thus there can be multiple triplets for the same source sentence. We filter to only use unique triplet combination in our final preference dataset $\mathcal{D}_{dpo}$. We conduct the same 2-step pre-processing procedure taken for the fine-tuning dataset $\mathcal{D}_{pos}$: 1) Remove duplicated instances; and 2) Remove lengthy instances where the upper threshold is set as Q$3 + 1.5 \times \text{IQR}$. We show the number of instances in Table \ref{tab:dpo_dataset_details}.

% \begin{table}[!htp]
\centering
\resizebox{200}{!}{%
    \begin{tabular}{l l l}
    \toprule
    \textbf{Dataset} & \textbf{\# Sentences}  \\ \midrule
    
    $\mathcal{D}_{dpo}$ (English-German) & 38,716 \\ 
    $\mathcal{D}_{dpo}$ (English-Russian) & 34,178 \\
    
    \bottomrule
    \end{tabular}
}
\caption{Summary statistics of DPO preference dataset.}
\label{tab:dpo_dataset_details}
\end{table}

% With this data, we update our initial model $\pi_0$  to align better rewrite model $\pi_{dpo}$ with DPO. We explore the same prompt template variants as in the fine-tuning setting (Appendix \ref{appendix:prompt_templates}).



% \subsection{Training Setup}
% We find that DPO converges in approximately 10 hours. We use a 90/10 train/validation data split and start with a supervised fine-tuned model as our initial model, $\pi_0$. We perform DPO with a training batch size of 4, a beta value of 0.1, a learning rate of 2e-4, a warm-up ratio of 0.05, a cosine learning rate scheduler, a maximum gradient norm clipping value of 0.3, a maximum prompt length of 1024 tokens, and 10 training epochs. All unspecified hyperparameters are set to their default values.

% \subsection{Results}
% In Tables \ref{tab:detailed_results_ende} to \ref{tab:detailed_results_enzh}, we compare DPO results to other rewriting methods. While previous works \citep{rafailov2023direct, wang2023making, tunstall2023zephyr} have demonstrated the effectiveness of DPO in aligning LLMs with preference datasets, we find that alignment-based learning is less effective for generating better rewrites for MT. Although \textsc{xCOMET}$(s,t)$ scores improve by +0.59 and +0.71 for \textsc{En-De} and \textsc{En-Ru}, respectively, \textsc{xCOMET}$(s,r)$ scores decrease by even wider margins. Among the three prompt template variations (Basic, MT, Ref), providing both the source and MT (MT) is most effective. We attribute this to the use of both preferred and dispreferred rewrites in DPO, suggesting that providing the MT as context for comparison may be helpful. Overall, DPO outperforms the baseline, but translatability-based selection remains the best rewriting method.


\section{Human Annotation Details}
\label{appendix:annotation details}
We use Qualtrics\footnote{\url{https://www.qualtrics.com}} to design our survey and Prolific\footnote{\url{https://www.prolific.com}} to recruit human annotators fluent in the tested target language.

\subsection{Original MT vs. Rewrite MT Details}
\label{appendix:mt_details}
We randomize the order of the two sentences (original MT and rewrite MT) to mitigate position bias. Annotators evaluate which sentence is better across four dimensions: fluency, understandability, level of detail, and meaning preservation. The entire survey is estimated to take approximately 20 minutes to complete. We recruit a total of 9 annotators and provide a compensation of 5 US dollars per survey (15 US dollars/hr), totaling 45 US dollars.


\subsection{Original vs. Rewrite Details}
\label{appendix:details}
Each annotator is tasked to judge how well the rewritten sentence preserves the meaning of the original source sentence. The survey is estimated to take approximately 30 minutes to complete. We recruit a total of 3 annotators. We offer a compensation of 7.5 US dollars per survey (15 US dollars/hr), totaling 22.5 US dollars.


\subsection{Annotator Instructions}
In Figures \ref{fig:human_intro} to \ref{fig:human_example_3}, we present the instructions and survey content provided to annotators. For the Original MT vs. Rewrite MT evaluation, each annotator reviews 20 sets of examples. Each question consists of two parts: \textbf{1)} comparing the two sentences based on fluency, understandability, and level of detail, and \textbf{2)} selecting which sentence better preserves the meaning of the reference translation. For the Original vs. Rewrite evaluation, each annotator reviews 30 sets of examples. Additionally, a free-form text box is provided alongside each example for annotators to offer feedback or suggestions.


\section{Time \& Computational Efficiency}
\label{appendix:inference_cost}
We show that on average, rewriting with our winning strategy is not a resource-intensive option for downstream applications in terms of both time and computation. For approximately 1.5K sentences, the rewrite and MT pipeline using our winning strategy (simplification with \textsc{Tower-Instruct 7b} takes 1 hour, compared to 30 minutes for the MT process alone. All variants of our prompting experiments are conducted using a single NVIDIA RTX958 A5000 GPU. In terms of efficiency compared to automatic post-editing (\S \ref{res:post-editing}), both approaches remains equivalent in time and computational requirements since the rewriting or post-editing process only differs in its position within the pipeline. Input rewriting modifies the source before the MT system, while output post-editing adjusts the translation after the MT system.


% \subsection{Annotator Feedback Details}
% \label{appendix:further comments}
% In our survey, we include an optional question for each example to collect additional feedback or comments from the annotators. We present some of this feedback in Table \ref{tab:annotator_feedback}. Annotators commented that they prefer the translations of rewrites over the original translations because 1) they are easier to understand; 2) they contain words that better suit the target language context; and 3) they are more precise. These comments align with the higher winning rates in the fluency, understandability, and detail dimensions.


\begin{table*}[!htp]
\centering
\resizebox{\textwidth}{!}{%
    \begin{tabular}{l l}
    \specialrule{1.3pt}{0pt}{0pt}
    \textbf{Rewrite} & \textbf{Prompt} \\ \midrule
    
    \multirow{4}{*}{\textbf{Simplification}} & Simplify the English sentence. Simplification may include identifying complex words and replacing with simpler \\
    & or shorter words or using active voice instead of passive voice. Try to keep the meaning of the Original sentence. \\
    & Original: \textit{This is a very nice skirt. The lacy pattern is classy and soft.} \\
    & Simplified: \\ \midrule

    \multirow{3}{*}{\textbf{Paraphrase}} & Paraphrase the English sentence. Try to not directly copy but keep the meaning of the Original sentence. \\
    & Original: \textit{This is a very nice skirt. The lacy pattern is classy and soft.} \\
    & Paraphrase: \\ \midrule

    \multirow{4}{*}{\textbf{Stylistic} (\textsc{CoEdIT})} & \textbf{(GEC)} Fix the grammar: \\
    & \textbf{(Coherent)} Make this text coherent: \\
    & \textbf{(Understandable)} Rewrite to make this easier to understand: \\
    % & \textbf{(Paraphrase)} Paraphrase this: \\
    & (\textbf{Formal)} Write this more formally: \\ \midrule

    \multirow{3}{*}{\textbf{Easy Translation}} & Rewrite the Original sentence to be easier for translation in {target language}. New sentence should be in English. \\
    & Original: \textit{This is a very nice skirt. The lacy pattern is classy and soft.} \\
    & New: \\ \midrule

    \multirow{8}{*}{\textbf{CoT}} & \textbf{(Step 1)} Rewrite the Original English sentence to New English sentence that translates better into German. \\
    & Avoid directly copying the Original sentence while keeping its meaning. New sentence should be in English. \\
    & Original: \textit{This is a very nice skirt. The lacy pattern is classy and soft.} \\
    & New: \\
    
    % \cmidrule(lr){2}
    \\
    & \textbf{(Step 2)} Now, translate the English sentence to German. \\
    & English: \\
    & German: \\
    
    \specialrule{1.3pt}{0pt}{0pt}
    \end{tabular}
}
\caption{Exemplar prompt templates for English-German language pair used for prompting experiments. \textit{Italic} represents the source sentence used in this example.}
\label{tab:prompting_template}
\end{table*}
\definecolor{light gray}{rgb}{0.898, 0.902, 0.906}
\definecolor{light pink}{rgb}{0.996, 0.9102, 0.9219}


\begin{table*}[htbp]
\centering
\resizebox{\textwidth}{!}
{
    \begin{tabular}{l}
    \specialrule{1.3pt}{0pt}{0pt}
    \textbf{Basic} \\ \midrule
    
    {\textsc{\#\#\#} \textbf{Instruction:}} Rewrite this English sentence to give a better translation. {\texttt{\textbackslash n\textbackslash n}} \\
        
    {\textsc{\#\#\#} \textbf{English}:} This is a very nice skirt. The lacy pattern is classy and soft.{\texttt{\textbackslash n}} \\
    
    {\textsc{\#\#\#} \textbf{English rewrite}:} The lacy pattern on this skirt is elegant and soft. \\
    \midrule

    \textbf{MT} \\ \midrule
    
    {\textsc{\#\#\#} \textbf{Instruction:}} Rewrite this English sentence to give a better translation in German. German sentence is the hypothesis translation that \\
    we are trying to improve.{\texttt{\textbackslash n\textbackslash n}} \\
    
    {\textsc{\#\#\#} \textbf{English}:} This is a very nice skirt. The lacy pattern is classy and soft.{\texttt{\textbackslash n}} \\

    {\textsc{\#\#\#} \textbf{German}:} Das ist eine sehr schöne Röhre. Das schicke Spitzenmuster ist weich und elegant.{\texttt{\textbackslash n}} \\
    
    {\textsc{\#\#\#} \textbf{English rewrite}:} The lacy pattern on this skirt is elegant and soft. \\
    \midrule

    \textbf{Reference} \\ \midrule
    
    {\textsc{\#\#\#} \textbf{Instruction:}} Rewrite this English sentence to give a better translation in German. German sentence is the human-annotated translation \\
    that we are trying to pursue.{\texttt{\textbackslash n\textbackslash n}} \\
    
    {\textsc{\#\#\#} \textbf{English}:} This is a very nice skirt. The lacy pattern is classy and soft.{\texttt{\textbackslash n}} \\

    {\textsc{\#\#\#} \textbf{German}:} Das ist ein sehr schöner Rock. Das Spitzenmuster ist stilvoll und weich.{\texttt{\textbackslash n}} \\
    
    {\textsc{\#\#\#} \textbf{English rewrite}:} The lacy pattern on this skirt is elegant and soft. \\
    \specialrule{1.3pt}{0pt}{0pt}
    \end{tabular}
}
\caption{Exemplar prompt templates for supervised fine-tuning experiments (English-German pair). We additionally give machine translation for the \textbf{MT} prompt and reference translation for the \textbf{Reference} prompt after \textsc{\#\#\#} \textbf{German:}.}
\label{tab:training_template}
\end{table*}



% \begin{prompt}[title={Prompt A.1.1. Basic SFT}]
% \textbf{\#\#\# Instruction:} Rewrite the English sentence to give a better translation.\\ \\
% \textbf{\#\#\# English:} \texttt{\{source\}} \\
% \textbf{\#\#\# English rewrite:} \texttt{\{rewrite\}} \\
% \end{prompt}
\clearpage

% \begin{table*}[!htp]
\centering
\resizebox{\textwidth}{!}{%
    \begin{tabular}{c l}
    \toprule
    \textbf{Top-k} & \textbf{Rewrite Methods} \\ \midrule
    
    2 & \textsc{CoEdIT} \small{(Understand)}\normalsize{, Simplification} \small{(LLaMA-3)} \\
    3 & \textsc{CoEdIT} \small{(Understand)}\normalsize{, Simplification} \small{(LLaMA-3)}\normalsize{, Paraphrase} \small{(LLaMA-3)} \\
    4 & \textsc{CoEdIT} \small{(Understand)}\normalsize{, Simplification} \small{(LLaMA-3)}\normalsize{, Paraphrase} \small{(LLaMA-3)}\normalsize{, \textsc{DIPPER}} \small{(L80/O60)} \\
    5 & \textsc{CoEdIT} \small{(Understand)}\normalsize{, Simplification} \small{(LLaMA-3)}\normalsize{, Paraphrase} \small{(LLaMA-3)}\normalsize{, \textsc{DIPPER}} \small{(L80/O60)}\normalsize{, Paraphrase} \small{(LLaMA-2)} \\
    
    \bottomrule
    \end{tabular}
}
\caption{Top-$k$ rewrite methods tested.}
\label{tab:top-k rerank}
\end{table*}
\newcommand{\CompCertUrl}[0]{\href{https://github.com/AbsInt/CompCert}{https://github.com/AbsInt/CompCert}}
\newcommand{\MathCompUrl}[0]{\href{https://github.com/math-comp/math-comp}{https://github.com/math-comp/math-comp}}
\newcommand{\GeoCoqUrl}[0]{\href{https://github.com/GeoCoq/GeoCoq}{https://github.com/GeoCoq/GeoCoq}}
\newcommand{\CategoryTheoryUrl}[0]{\href{https://github.com/jwiegley/category-theory}{https://github.com/jwiegley/category-theory}}
\newcommand{\LeanUrl}[0]{\href{https://github.com/leanprover-community/mathlib4}{https://github.com/leanprover-community/mathlib4}}
\newcommand{\CodeTFiveModelSize}[0]{\texttt{60 M}}
\newcommand{\CatTheory}[0]{\textsc{Cat-Theory}}

In this section, we explain our dataset construction and model training choices towards our demonstration of positive transfer from multilingual training as well as adaptability to new domains via further fine-tuning.
% We also investigate further fine-tuning them on held-out datasets to see if our models can adapt to on a different domains following the primary training. 

\subsection{Dataset Details}
We collect datasets across multiple languages and language versions of Coq and Lean 4, sourcing data from existing repositories. Our data collection approach involves collecting proof states from the ITP through tactic execution. We construct several data-mixes, of different subsets of the accumulated data, to train various monolingual and multilingual \proofwala{} models to perform proof step prediction. The training data is formatted into prompts as shown in \Cref{fig:prompt-format} (in \Cref{app:training-data}). We collect proof-step data for the various data mixtures as shown in \Cref{tab:data-mix}. 

\renewcommand\theadfont{}
\begin{table}[t!]
    \centering
    \scalebox{0.7}{
    \begin{tabular}{llll}
    \toprule
    \multicolumn{4}{c}{\thead{\textbf{Initial Fine-tuning}}}\\
    \hline
    \thead{\textbf{Data-mix}}  & \thead{\textbf{Data-mix Source}}  & \thead{\textbf{\name\;}\\\textbf{Models Trained}} & \thead{\textbf{Token Count}}\\
    \toprule
    1. CompCert\footref{fnote:CompCert-url} & CompCert Repo\footref{fnote:same-as-proverbot} & - & 61.6 M \\
    2. MathComp\footref{fnote:MathComp-url} & MathComp Repo & - & 18.2 M\\
    3. GeoCoq\footref{fnote:GeoCoq-url} & GeoCoq Repo & - & 91.2 M\\
    \renewcommand\theadfont{}
    4. \coq & {\textbf{Data-Mixes:} 1-3} & \thead[l]{\coq} & 171 M \\
    \renewcommand\theadfont{}
    5. \lean\footref{fnote:Lean-url} & Mathlib Repo\footref{fnote:same-as-reprover} & \thead[l]{\lean} & 99 M \\
    \renewcommand\theadfont{}
    6. \multi & {\textbf{Data-Mixes:} 4-5} & \thead[l]{\multi} & 270 M \\
    \hline
     \multicolumn{4}{c}{\thead{\textbf{Further Fine-tuning}}}\\
    \hline
    \renewcommand\theadfont{}
    7. CategoryTheory\footref{fnote:CategoryTheory-url} & CategoryTheory Repo & \thead[l]{\multi-\\\CatTheory\;\;\&\;\\\coq-\\\CatTheory} & 1.7 M\\
    \bottomrule
    \end{tabular}}
    \caption{Different data-mixes used to extract proof-step and proof state pair data. Various \proofwala\; models trained on these data mixes.}
    % \\\textsuperscript{*}\small{Same as Proverbot split \citep{sanchez2020generating}}
    % \\\textsuperscript{**}\small{Same as random split in ReProver \citep{yang2023leandojo}}}
    \label{tab:data-mix}
\end{table}
\footnotetext{\label{fnote:same-as-proverbot}Same as Proverbot split \citep{sanchez2020generating}}
\addtocounter{footnote}{+1}\footnotetext{\label{fnote:same-as-reprover}Same as random split in ReProver \citep{yang2023leandojo}}

We use different Coq and Lean repositories to generate this proof-step data. We use well-known repositories, namely CompCert,\footnote{\label{fnote:CompCert-url}\CompCertUrl} Mathlib,\footnote{\label{fnote:Lean-url}\LeanUrl} MathComp,\footnote{\label{fnote:MathComp-url}\MathCompUrl} GeoCoq,\footnote{\label{fnote:GeoCoq-url}\GeoCoqUrl} and CategoryTheory,\footnote{\label{fnote:CategoryTheory-url}\CategoryTheoryUrl} to generate the proof-step data. For CompCert we used the train-test split proposed by \citet{sanchez2020generating}, and for Mathlib we used the split proposed by \citet{yang2023leandojo}. Together we have \texttt{442607} proof-step pairs derived from over \texttt{76997} theorems across Lean and Coq (details of the split shown in \Cref{tab:data-mix-size}). We hold out the CategoryTheory dataset from initial training data-mixes for experimentation with further fine-tuning for our novel domain adaptation experiment.
% We also introduce some new 
% \george{I'm not sure what you mean by new, Coq-specific data collection have used these for sure. I would rephrase the beginning of this paragraph as (loosely) "We collect from Mathlib, Compcert, GeoCoq, CategoryTheory repositories which constritute some of the largest and most popular repositories in both ITPs"} proof-step data from MathComp\footnote{\label{fnote:MathComp-url}\MathCompUrl}, GeoCoq\footnote{\label{fnote:GeoCoq-url}\GeoCoqUrl}, and CategoryTheory\footnote{\label{fnote:CategoryTheory-url}\CategoryTheoryUrl} repositories in Coq.
% \gd{you need much more info here. how were these converted to Coq? what previous work does this build on? you need citations here and description inlined here, then you can remove them from the table, and ideally examples of the datasets in the appendix} 
 %We trained each of our models over \texttt{2.71 B} tokens. \amit{Change all the entries in \Cref{tab:data-mix} once the models are re-trained}

% \begin{table}[ht]
%     \centering
%     \footnotesize
%     \begin{tabular}{l|l|l|l|l|l|l}
%     \hline
%     \thead{\textbf{Data-mix}\\\textbf{Name}}  & \multicolumn{3}{c|}{\thead{\textbf{Proof-Step \&}\\\textbf{State Pair Count}}}  & \multicolumn{3}{c}{\thead{\textbf{Theorem}\\\textbf{Count}}}\\
%     \hline
%     & \textbf{Train} & \textbf{Test} & \textbf{Val} & \textbf{Train} & \textbf{Test} & \textbf{Val}\\
%     \hline
%     1. CompCert & 80288 & 6199 & - & 5440 & 501 & - \\
%     2. MathComp & 34543 & 711 & 720 & 11385 & 220 & 221 \\
%     3. GeoCoq & 104045 & 1866 & 2351 & 4537 & 89 & 88 \\
%     4. CategoryTheory & 4763 & 82 & 48 & 690 & 14 & 13 \\
%     5. \coq & 223639 & 8858 & 3119 & 22152 & 824 & 322 \\
%     6. \lean & 237003 & 4323 & 4220 & 56140 & 991 & 1035 \\
%     7. \multi & 460642 & 13181 & 7339 & 78292 & 1815 & 1357 \\
%     \end{tabular}
%     \caption{Size of different data-mixes. The \proofwala\;models were trained on the training split of \coq, \lean, and \multi\;data-mixes. After extracting proof-step and state pair data, the training, validation, and test split are randomly decided. For \lean\;and CompCert data-mix we used the same split as proposed by \citet{yang2023leandojo} and \citet{sanchez2020generating} respectively.}
%     \label{tab:data-mix-size}
% \end{table}

\begin{table}[ht]
    \centering
    % \footnotesize
    \scalebox{0.75}{
    \begin{tabular}{lrrrrrr}
    \hline
    & \multicolumn{3}{c}{\thead{\textbf{\# Proof-Step \& State Pairs}}}  & \multicolumn{3}{c}{\thead{\textbf{Theorem Count}}}\\
    \cmidrule(lr){2-4}\cmidrule(lr){5-7}
    \thead{\textbf{Data-mix}} & \textbf{Train} & \textbf{Test} & \textbf{Val} & \textbf{Train} & \textbf{Test} & \textbf{Val}\\
    \toprule
    1. CompCert & 80288 & 6199 & - & 5440 & 501 & - \\
    2. MathComp & 34196 & 1378 & 2285 & 11381 & 536 & 729 \\
    3. GeoCoq & 91120 & 12495 & 4928 & 4036 & 505 & 208 \\
    4. \coq & 205604 & 20072 & 7213 & 20857 & 1542 & 937 \\ %Update the sum
    5. \lean & 237003 & 4323 & 4220 & 56140\textsuperscript{\footref{fnote:lean-test-set-size}} & 991\textsuperscript{\footref{fnote:lean-test-set-size}} & 1035\textsuperscript{\footref{fnote:lean-test-set-size}} \\
    6. \multi & 442607 & 24395 & 11433 & 76997 & 2533 & 1972 \\
    7. CategoryTheory & 4114 & 610 & 208 & 573 & 101 & 43 \\
    \bottomrule
    \end{tabular}}
    \caption{Size of different data-mixes. The \proofwala\;models were trained on the training split of \coq, \lean, and \multi\;data-mixes. After extracting proof-step and state pair data, the training, validation, and test split are randomly decided so as to include at least 500 testing theorems, except in the case of CategoryTheory where the overall dataset is relatively small. For the \lean\;and CompCert data-mix we used the same split as proposed by \citet{yang2023leandojo}\footref{fnote:lean-test-set-size} and \citet{sanchez2020generating} respectively.
    }
    \label{tab:data-mix-size}
\end{table}
\addtocounter{footnote}{+1}
\footnotetext{\label{fnote:lean-test-set-size}While the LeanDojo dataset \citep{yang2023leandojo} officially has 2000 test theorems, only 991 of these are proved using tactics and have their tactics extracted in the dataset. Since our approach involves generating only tactic-based proofs, our Lean dataset is collected from those theorems with tactic-based proofs.}


\subsection{Model Details}
\label{sec:model-details}
We used the \codeTFive-\base\;\citep{wang2021codet5} pretrained model---which has \texttt{220} million parameters---to fine-tune models on the different data-mixes as described in \Cref{tab:data-mix}.
We trained three models \proofwala-\{\multi, \coq, \lean\} with the same step count and batch sizes for all settings. Training the models with the same number of steps aligns with recent work on training models for multilingual autoformalization \citep{jiang2023multilingualmathematicalautoformalization} which ensures that each model has the same number of gradient updates. Our models are initially trained on CompCert, Mathlib, MathComp, and GeoCoq. The hyperparameters used for training are described in \Cref{tab:hyperparams} in \Cref{app:hyperparams}. \Cref{app:hyperparams} also describes the amount of computing we used to train our models. 

To demonstrate the usefulness of our models on subsequent theorem-proving tasks, we perform further fine-tune of our \proofwala-\{\multi, \coq\} models on CategoryTheory\footref{fnote:CategoryTheory-url} theory data. We used the same hyperparameters as \Cref{tab:hyperparams} (in \Cref{app:hyperparams}) but we reduce the number of training steps to 1200 and batch size to 8.

\begin{table*}[ht]
    \centering
    \scalebox{0.8}{
    \begin{tabular}{lclllllll}
    \toprule
    \multicolumn{2}{c}{\textbf{Data-Mix}} &  
     & 
    \multicolumn{5}{c}{\textbf{Pass-at-$k$} \%} &
    \\
    \cmidrule(lr){1-2}\cmidrule(lr){4-8}
    \textbf{Name} & 
    \textbf{\# Theorems} & 
    \thead[c]{\textbf{Proof Step Model}} &
    \textbf{Pass@1} & 
    \textbf{Pass@2} & 
    \textbf{Pass@3} & 
    \textbf{Pass@4} & 
    \textbf{Pass@5} & 
    \thead[c]{\textbf{$p_{\mathrm{value}}$}\\(\textbf{$\alpha$}: 0.05)\footref{fnote:p-value}}\\
    \toprule
    \textbf{\lean} & 
    991 &
    \proofwala-\lean & 
    24.92 & 
    26.64 & 
    27.54 &
    28.05 &
    28.25 &
    \\
     & 
     & 
     \proofwala-\multi & 
     \textbf{26.84} & 
     \textbf{28.56} & 
     \textbf{29.67} &
     \textbf{29.97} &
     \textbf{30.58} &
     \textbf{0.018} \\
    \hline
    \textbf{MathComp} & 
    536 & 
    \proofwala-\coq & 
    \textbf{28.28} & 
    28.65 & 
    29.4 &
    29.59 &
    30.15 &
    \\
     &
     &
     \proofwala-\multi & 
     27.9 & 
     \textbf{29.21} & 
     \textbf{29.59} &
     \textbf{30.15} &
     \textbf{30.52} &
     0.355
     \\
    \hline
    \textbf{GeoCoq} & 
    505 & 
    \proofwala-\coq & 
    \textbf{32.87} & 
    \textbf{33.66} & 
    33.86 &
    34.06 &
    34.46 &
    \\
    & 
    & 
    \proofwala-\multi & 
    30.89 & 
    \textbf{33.66} & 
    \textbf{34.65} &
    \textbf{35.64} &
    \textbf{35.84} &
    0.135
    \\
    \hline
    \textbf{CompCert} &
    501 & 
    \proofwala-\coq &  
     17.56 & 
     18.76 & 
     19.16 &
     19.76 &
     20.76 &
     \\
     &
     & 
     \proofwala-\multi & 
     \textbf{17.96} & 
     \textbf{19.76} & 
     \textbf{20.56} &
     \textbf{21.16} &
     \textbf{21.96} &
     0.191 \\
    \hline
    \textbf{CategoryTheory} & 
    101 & 
    \proofwala-\coq-\CatTheory & 
    36.63 & 
    42.57 & 
    44.55 & 
    44.55 & 
    45.54 & \\
    & 
    & 
    \proofwala-\multi-\CatTheory & 
    \textbf{44.55} & 
    \textbf{51.49} & 
    \textbf{52.48} & 
    \textbf{53.47} & 
    \textbf{53.47} & 
    \textbf{0.008} \\
    \bottomrule
    \end{tabular}}
    \caption{Comparison between various \proofwala$\;$models and the \proofwala-\multi\; model on different data-mixes. We can see that transfer happening between Lean and Coq on all data-mixes from various domains in math and software verification. We observe that the \multi\; model outperforms the \lean\; and \coq\; models on all data mixes. The performance improvement is also statistically significant on the biggest data-mix \lean\; (Mathlib). We also observe that after further fine-tuning, the \multi\; model significantly outperforms the \coq\; model on the CategoryTheory dataset.
    }
    \label{tab:all-experiments}
\end{table*}
\addtocounter{footnote}{+1}
\footnotetext{\label{fnote:p-value}The results are statistically significant using a paired bootstrap test if $p_{\mathrm{value}} < 0.05$.}
\definecolor{light blue}{RGB}{215, 242, 252}
\definecolor{light purple}{RGB}{247, 215, 252}
\definecolor{light orange}{rgb}{0.9961, 0.875, 0.7188}
    
    
\begin{table*}[!htp]
\centering
\resizebox{\textwidth}{!}{%
\renewcommand{\arraystretch}{1.1}
\begin{tabular}{c l l l l l l l}
    \specialrule{1.3pt}{0pt}{0pt}
    \textbf{Language} & \textbf{Type} & \textbf{Prompt/Model} & \textbf{\textsc{xCOMET}$(s,t)$} & \textbf{\textsc{xCOMET}$(s,r)$} & \textbf{\textsc{xCOMET}$(s,t,r)$} & \textbf{\textsc{MetricX}$(s,t)$} & \textbf{\textsc{MetricX}$(t,r)$} \\ \midrule
    
    \multirow{22}{*}{\large \textbf{\textsc{en-de}}} & \textbf{Original} & - & 0.893 & 0.904 & 0.898 & 2.038 & 1.534 \\
    \cmidrule(lr){2-8}
    
    & \fcolorbox{white}{light blue}{\raisebox{-0.2em}{\includegraphics[height=1em]{figures/logos/agnostic.png}} \textbf{MT-Agnostic}} & \textbf{Simplification} (\small \textsc{LLaMA-2}) & \textbf{0.931} & 0.846 & 0.900 & \textbf{1.185} & 1.727 \\
    & &(\small \textsc{LLaMA-3}) & \textbf{0.944} & 0.820 & 0.903 & \textbf{0.925*} & 1.600 \\
    & &(\small \textsc{Tower-Instruct}) & \textbf{0.922} & 0.885 & \textbf{0.907} & 1.504 & 1.519 \\
    
    & & \textbf{Paraphrase} (\small \textsc{LLaMA-2}) & \textbf{0.926} & 0.823 & 0.889 & \textbf{1.126} & \textbf{1.480} \\
    & & (\small \textsc{LLaMA-3}) & \textbf{0.938} & 0.796 & 0.892 & \textbf{0.955} & \textbf{1.469} \\
    & & (\small \textsc{Tower-Instruct}) & 0.902 & 0.887 & 0.901 & \textbf{1.310} & 1.534 \\

    & &(\small \textsc{DIPPER} (L80/O60)) & \textbf{0.904} & 0.745 & 0.838 & \textbf{1.674} & 2.757 \\
    & &(\small \textsc{DIPPER} (L80/O40)) & \textbf{0.913} & 0.797 & 0.863 & \textbf{1.461} & 2.266 \\
    & &(\small \textsc{DIPPER} (L60/O40)) & \textbf{0.917} & 0.847 & 0.892 & \textbf{1.555} & 1.958 \\

    & &\textbf{Stylistic} (\small \textsc{CoEdIT} GEC) & \textbf{0.901} & 0.899 & 0.900 & \textbf{1.709} & 1.555 \\
    & & (\small \textsc{CoEdIT} Coherent) & 0.898 & 0.900 & 0.898 & \textbf{1.728} & 1.595 \\
    & & (\small \textsc{CoEdIT} Understandable) & \textbf{0.949} & \textbf{0.758} & 0.862 & \textbf{0.989} & 2.610 \\
    & & (\small \textsc{CoEdIT} Formal) & \textbf{0.937} & 0.830 & 0.900 & \textbf{1.063} & 1.879 \\

    \cmidrule(lr){2-8}
    & \fcolorbox{white}{light purple}{\raisebox{-0.2em}{\includegraphics[height=1em]{figures/logos/task.png}} \textbf{Task-Aware}} & \textbf{Easy Translation} (\small \textsc{LLaMA-2}) & \textbf{0.916} & 0.857 & 0.893 & \textbf{1.654} & 2.482 \\
    & & (\small \textsc{LLaMA-3}) & \textbf{0.932} & 0.827 & 0.899 & \textbf{1.151} & 2.241 \\
    & & (\small \textsc{Tower-Instruct}) & \textbf{0.901} & 0.900 & 0.903 & \textbf{1.759} & 2.427 \\
    & & \textbf{CoT} (\small \textsc{Tower-Instruct}) & \textbf{0.907} & 0.816 & 0.897 & \textbf{1.892} & 1.578 \\
    
    \cmidrule(lr){2-8}
    & \fcolorbox{white}{light orange}{\raisebox{-0.2em}{\includegraphics[height=1em]{figures/logos/translatability.png}} \textbf{Translatability-Aware}} & \textbf{Selection} & \textbf{0.921} & 0.907 & \textbf{0.915*} & \textbf{1.734} & \textbf{1.461*} \\
    & & \textbf{Fine-tune} (\small Basic) & \textbf{0.934} & 0.851 & \textbf{0.909} & \textbf{1.878} & \textbf{1.499} \\
    & & (\small MT) & \textbf{0.919} & 0.856 & 0.903 & \textbf{1.947} & 1.593 \\
    & & (\small Reference) & 0.896 & 0.836 & 0.876 & 2.023 & 2.028 \\
    
    % & &\textbf{DPO} (\small Basic) & \textbf{0.937} & 0.654 & 0.752 & \textbf{1.829} & 1.512 \\
    % & &(\small MT) & \textbf{0.952*} & 0.728 & 0.847 & \textbf{0.944} & \textbf{1.488} \\
    % & &(\small Reference) & \textbf{0.933} & 0.742 & 0.836 & \textbf{1.846} & 1.562 \\

    \specialrule{1.3pt}{0pt}{0pt}
    \end{tabular}
}
\caption{Detailed results of English-German pair using different rewrite methods. Statistically significant average improvements ($p$-value $< 0.05$) are \textbf{bold}. Best scores for each metric is \textbf{bold} with \textbf{*}. \textsc{xCOMET}$(s,t)$: translatability (↑); \textsc{xCOMET}$(s,r)$: meaning preservation (↑); \textsc{xCOMET}$(s,t,r)$: overall translation quality (↑); \textsc{MetricX}$(s,t)$: quality estimation (↓); \textsc{MetricX}$(t,r)$: reference-based metric (↓). For \textsc{DIPPER} \cite{dipper} variations, L and O denote lexical and order diversity, respectively.}
\label{tab:detailed_results_ende}
\end{table*}
\clearpage

\definecolor{light blue}{RGB}{215, 242, 252}
\definecolor{light purple}{RGB}{247, 215, 252}
\definecolor{light orange}{rgb}{0.9961, 0.875, 0.7188}


\begin{table*}[!htp]
\centering
\resizebox{\textwidth}{!}{%
\renewcommand{\arraystretch}{1.1}
\begin{tabular}{c l l l l l l l}
    \specialrule{1.3pt}{0pt}{0pt}
    \textbf{Language} & \textbf{Type} & \textbf{Prompt/Model} & \textbf{\textsc{xCOMET}$(s,t)$} & \textbf{\textsc{xCOMET}$(s,r)$} & \textbf{\textsc{xCOMET}$(s,t,r)$} & \textbf{\textsc{MetricX}$(s,t)$} & \textbf{\textsc{MetricX}$(t,r)$} \\ \midrule

    \multirow{22}{*}{\large \textbf{\textsc{en-ru}}} & \textbf{Original} & - & 0.872 & 0.884 & 0.868 & 2.535 & 2.028 \\
    \cmidrule(lr){2-8}

    & \fcolorbox{white}{light blue}{\raisebox{-0.2em}{\includegraphics[height=1em]{figures/logos/agnostic.png}} \textbf{MT-Agnostic}} & \textbf{Simplification} (\small \textsc{LLaMA-2}) & \textbf{0.916} & 0.839 & \textbf{0.882} & \textbf{0.951} & 2.160 \\
    & &(\small \textsc{LLaMA-3}) & \textbf{0.919} & 0.812 & \textbf{0.885} & \textbf{0.804} & 2.039 \\
    & &(\small \textsc{Tower-Instruct}) & \textbf{0.921} & 0.870 & \textbf{0.891} & \textbf{1.135} & \textbf{1.921} \\

    & & \textbf{Paraphrase} (\small \textsc{LLaMA-2}) & \textbf{0.923} & 0.804 & \textbf{0.881} & \textbf{0.882} & \textbf{1.853} \\
    & &  (\small \textsc{LLaMA-3}) & \textbf{0.930} & 0.788 & \textbf{0.882} & \textbf{0.855} & \textbf{1.863} \\
    & & (\small \textsc{Tower-Instruct}) & \textbf{0.887} & 0.878 & \textbf{0.878} & \textbf{1.095} & \textbf{1.976} \\

    & & (\small \textsc{DIPPER} (L80/O60)) & \textbf{0.904} & 0.735 & 0.821 & \textbf{1.249} & 3.476 \\
    & & (\small \textsc{DIPPER} (L80/O40)) & \textbf{0.909} & 0.790 & 0.853 & \textbf{1.105} & 2.773 \\
    & & (\small \textsc{DIPPER} (L60/O40)) & \textbf{0.905} & 0.834 & 0.873 & \textbf{1.119} & 2.418 \\

    & & \textbf{Stylistic} (\small \textsc{CoEdIT} GEC) & 0.873 & 0.884 & 0.869 & \textbf{1.327} & \textbf{1.969} \\
    & & (\small \textsc{CoEdIT} Coherent) & 0.873 & 0.884 & 0.869 & \textbf{1.368} & \textbf{1.989} \\
    & & (\small \textsc{CoEdIT} Understandable) & \textbf{0.918} & 0.801 & 0.873 & \textbf{0.991} & 2.726 \\
    & & (\small \textsc{CoEdIT} Formal) & \textbf{0.916} & 0.841 & \textbf{0.887} & \textbf{0.922} & 2.020 \\
    \cmidrule(lr){2-8}


    & \fcolorbox{white}{light purple}{\raisebox{-0.2em}{\includegraphics[height=1em]{figures/logos/task.png}} \textbf{Task-Aware}} & \textbf{Easy Translation} (\small \textsc{LLaMA-2}) & \textbf{0.914} & 0.839 & \textbf{0.884} & \textbf{1.037} & 10.849 \\
    & & (\small \textsc{LLaMA-3}) & \textbf{0.917} & 0.808 & \textbf{0.881} & \textbf{0.801*} & 10.401 \\
    & & (\small \textsc{Tower-Instruct}) & \textbf{0.885} & 0.883 & \textbf{0.878} & \textbf{1.277} & 11.137 \\
    
    & & \textbf{CoT} (\small \textsc{Tower-Instruct}) & \textbf{0.903} & 0.871 & 0.875 & \textbf{2.432} & 2.024 \\
    \cmidrule(lr){2-8}

    & \fcolorbox{white}{light orange}{\raisebox{-0.2em}{\includegraphics[height=1em]{figures/logos/translatability.png}} \textbf{Translatability-Aware}} & \textbf{Selection} & \textbf{0.914} & \textbf{0.891*} & \textbf{0.899*} & \textbf{2.096} & \textbf{1.830*} \\
    
    & & \textbf{Fine-tune} (\small Basic) & \textbf{0.912} & 0.848 & \textbf{0.886} & \textbf{2.123} & \textbf{1.932} \\
    & & (\small MT) & \textbf{0.904} & 0.851 & 0.871 & \textbf{2.119} & \textbf{1.997} \\
    & & (\small Reference) & \textbf{0.881} & 0.812 & 0.859 & \textbf{2.284} & 2.012 \\
    
    % & & \textbf{DPO} (\small Basic) & \textbf{0.916} & 0.728 & 0.791 & \textbf{2.153} & \textbf{1.924} \\
    % & & (\small MT) & \textbf{0.943*} & 0.769 & 0.859 & \textbf{2.099} & \textbf{1.903} \\
    % & & (\small Reference) & \textbf{0.925} & 0.754 & 0.852 & \textbf{2.211} & 2.014 \\

    \specialrule{1.3pt}{0pt}{0pt}
    \end{tabular}
}
\caption{Detailed results of English-Russian pair using different rewrite methods.}
\label{tab:detailed_results_enru}
\end{table*}
\definecolor{light blue}{RGB}{215, 242, 252}
\definecolor{light purple}{RGB}{247, 215, 252}
\definecolor{light orange}{rgb}{0.9961, 0.875, 0.7188}



\begin{table*}[!htp]
\centering
\resizebox{\textwidth}{!}{%
\renewcommand{\arraystretch}{1.1}
\begin{tabular}{c l l l l l l l}
    \specialrule{1.3pt}{0pt}{0pt}
    \textbf{Language} & \textbf{Type} & \textbf{Prompt/Model} & \textbf{\textsc{xCOMET}$(s,t)$} & \textbf{\textsc{xCOMET}$(s,r)$} & \textbf{\textsc{xCOMET}$(s,t,r)$} & \textbf{\textsc{MetricX}$(s,t)$} & \textbf{\textsc{MetricX}$(t,r)$} \\ \midrule

    \multirow{19}{*}{\large \textbf{\textsc{en-zh}}} & \textbf{Original} & - & 0.786 & 0.775 & 0.794 & 3.445 & 2.282 \\

    \cmidrule(lr){2-8}

    & \fcolorbox{white}{light blue}{\raisebox{-0.2em}{\includegraphics[height=1em]{figures/logos/agnostic.png}} \textbf{MT-Agnostic}} & \textbf{Simplification} (\small \textsc{LLaMA-2}) & \textbf{0.828} & 0.728 & 0.796 & \textbf{1.321} & 2.537 \\
    &  & (\small \textsc{LLaMA-3}) & \textbf{0.823} & 0.701 & 0.795 & \textbf{1.252*} & 2.572 \\
    & & (\small \textsc{Tower-Instruct}) & \textbf{0.821} & 0.759 & \textbf{0.802} & \textbf{1.521} & \textbf{2.227} \\

    & & \textbf{Paraphrase} (\small \textsc{LLaMA-2}) & \textbf{0.818} & 0.683 & 0.771 & \textbf{1.330} & 2.478 \\
    & & (\small \textsc{LLaMA-3}) & \textbf{0.826} & 0.662 & 0.766 & \textbf{1.341} & 2.534 \\
    & & (\small \textsc{Tower-Instruct}) & \textbf{0.797} & 0.765 & 0.798 & \textbf{1.580} & 2.283 \\

    & & (\small \textsc{DIPPER} (L80/O60)) & \textbf{0.813} & 0.622 & 0.722 & \textbf{1.583} & 4.009 \\
    & &  (\small \textsc{DIPPER} (L80/O40)) & \textbf{0.816} & 0.670 & 0.750 & \textbf{1.499} & 3.196 \\
    & & (\small \textsc{DIPPER} (L60/O40)) & \textbf{0.809} & 0.711 & 0.775 & \textbf{1.503} & 2.725\\
    
    & & \textbf{Stylistic} (\small \textsc{CoEdIT} GEC) & 0.789 & 0.772 & 0.795 & \textbf{1.632} & 2.251 \\
    & &  (\small \textsc{CoEdIT} Coherent) & 0.786 & 0.774 & 0.794 & \textbf{1.658} & 2.267 \\
    & &  (\small \textsc{CoEdIT} Understandable) & \textbf{0.839*} & 0.677 & 0.774 & \textbf{1.358} & 3.174 \\
    & &  (\small \textsc{CoEdIT} Formal) & \textbf{0.823} & 0.730 & 0.798 & \textbf{1.336} & 2.443 \\
    \cmidrule(lr){2-8}

    & \fcolorbox{white}{light purple}{\raisebox{-0.2em}{\includegraphics[height=1em]{figures/logos/task.png}} \textbf{Task-Aware}} & \textbf{Easy Translation} (\small \textsc{LLaMA-2}) & \textbf{0.821} & 0.721 & 0.784 & \textbf{1.900} & 7.732 \\
    & & (\small \textsc{LLaMA-3}) & \textbf{0.830} & 0.687 & 0.783 & \textbf{1.360} & 7.608 \\
    & & (\small \textsc{Tower-Instruct}) & \textbf{0.793} & 0.762 & 0.791 & \textbf{1.618} & 7.650 \\
    
    & & \textbf{CoT} (\small \textsc{Tower-Instruct}) & \textbf{0.821} & 0.769 & 0.771 & \textbf{3.321} & 2.432 \\

    \cmidrule(lr){2-8}
    & \fcolorbox{white}{light orange}{\raisebox{-0.2em}{\includegraphics[height=1em]{figures/logos/translatability.png}} \textbf{Translatability-Aware}} & \textbf{Selection} & \textbf{0.823} & \textbf{0.783*} & \textbf{0.819*} & \textbf{3.149} & \textbf{2.206*} \\

    \specialrule{1.3pt}{0pt}{0pt}
    \end{tabular}
}
\caption{Detailed results of English-Chinese pair using different rewrite methods.}
\label{tab:detailed_results_enzh}
\end{table*}
\clearpage

\definecolor{light blue}{RGB}{215, 242, 252}
\definecolor{light purple}{RGB}{247, 215, 252}
\definecolor{light orange}{rgb}{0.9961, 0.875, 0.7188}



\begin{table*}[!htp]
\centering
\resizebox{330pt}{!}{%
\begin{tabular}{l l l l l l}
    \specialrule{1.3pt}{0pt}{0pt}
    \textbf{Type} & \textbf{Prompt/Model} & \textbf{\textsc{en-de}} & \textbf{\textsc{en-ru}} & \textbf{\textsc{en-zh}} \\ \midrule
    
    \fcolorbox{white}{light blue}{\textbf{MT-Agnostic}} & \textbf{Simplification} (\small \textsc{LLaMA-2}) & 2.06	& 2.37	& 2.37 \\
    & (\small \textsc{LLaMA-3}) & 0.39	& 0.33	& 0.29 \\
    & (\small \textsc{Tower-Instruct}) & 28 &	29.3	& 30.2 \\

    & \textbf{Paraphrase} (\small \textsc{LLaMA-2}) & 0 & 0 & 0 \\
    &  (\small \textsc{LLaMA-3}) & 0.06	&0.07	&0.03 \\
    & (\small \textsc{Tower-Instruct}) & 37.3	& 38.2	& 38 \\

    & (\small \textsc{DIPPER} (L80/O60)) & 0.19	& 0.94	& 1.04 \\
    & (\small \textsc{DIPPER} (L80/O40)) & 0.51& 1.5 &	1.53 \\
    & (\small \textsc{DIPPER} (L60/O40)) & 1.48	& 2.5	& 2.44 \\

    & \textbf{Stylistic} (\small \textsc{CoEdIT} GEC) & 42.6& 44 & 48.3 \\
    & (\small \textsc{CoEdIT} Coherent) & \color{red}{82.2}	& \color{red}{91.9}	& \color{red}{93.2} \\
    & (\small \textsc{CoEdIT} Understandable) & 1.61& 1.88	& 1.53 \\
    & (\small \textsc{CoEdIT} Formal) & 5.33& 3.76	& 5.5 \\
    \midrule


    \fcolorbox{white}{light purple}{\textbf{Task-Aware}} & \textbf{Easy Translation} (\small \textsc{LLaMA-2}) & 3.04 & 3.55 & 3.63 \\
    & (\small \textsc{LLaMA-3}) & 0.24 & 0.66 & 0.27 \\
    & (\small \textsc{Tower-Instruct}) & 12.3 & 18.6 & 15.4 \\
    & \textbf{CoT} (\small \textsc{Tower-Instruct}) & 0.71& 1.45& 1.53 \\
    \midrule

    \fcolorbox{white}{light orange}{\textbf{Translatability-Aware}} & \textbf{Fine-tune} (\small Basic) & 4.5	& 3.91 & - \\
    & (\small MT) &3.73	& 3.42 & - \\
    & (\small Reference) & 6.17	& 7.85 & - \\

    % & \textbf{DPO} (\small Basic) & 0.71 & 0.98 & - \\
    % & (\small MT) &0.77	& 1.37 & - \\
    % & (\small Reference) & 1.48	& 1.95 & - \\

    \specialrule{1.3pt}{0pt}{0pt}
    \end{tabular}
}
\caption{Percentage of occurrence (\%) where the rewrite is a direct copy of the original source sentence.}
\label{tab:direct_copy}
\end{table*}

% \begin{table*}[!htp]
\centering
\resizebox{\textwidth}{!}{%
    \begin{tabular}{c | p{3.5cm} p{3.5cm} p{3.5cm} p{4.5cm}}
    \toprule
    \textbf{Rewrite Method} & \textbf{Original MT} & \textbf{Rewrite MT} & \textbf{Reference} & \textbf{Comment}  \\ \midrule

    \multirow{5}{*}{\textbf{Stylistic Prompting}} & Er wurde oft von Mobbern geärgert, aber von seinem Bruder tapfer verteidigt. & Er wurde oft von Mobbern gehänselt, aber immer von seinem Bruder unterstützt. & Oft von Mobbern angegriffen, aber hartnäckig von seinem Bruder verteidigt. & ``geärgert'' and ``gehänselt'' are both good choices but ``gehänselt'' definitely \textbf{suits better} when talking about mobbing. \\
    \midrule
    
    \multirow{16}{*}{\textbf{Re-ranking}} & Trage einen warmen Anzug, einen Barett, einen Regenmantel und gute Stiefel mit Nägeln. & Tragen Sie warme Kleidung, einen Hut und wasserdichte Stiefel. & Tragen Sie einen warmen Anzug, eine Baskenmütze, einen Regenmantel und ein gutes Paar genagelte Stiefel. & 1) Original MT has more terminology from the reference. \par 2) Rewrite MT is \textbf{easier to understand} than original MT in this context. \\
    \cmidrule(lr){2-5}
    
    & Wann immer er ins Gefängnis kommt, bezahlt Eddie immer wieder seine Kaution. & Eddie holt ihn immer aus dem Gefängnis. & Wann immer er im Gefängnis landet, holt Eddie ihn ständig raus. & 1) Rewrite MT seems \textbf{more precise}. \par 2) Original MT does not explicitly tells you that Eddie safes someone from prison. \\
    \cmidrule(lr){2-5}

    & ``Es ist nicht so, dass man oft hört: `Ich fahre ihn einfach nur nach Glasgow''', sagte der Ticketverkäufer. & ``Das ist nicht etwas, was man oft hört'', sagte der Kassierer. ``Ich bringe ihn einfach nach Glasgow.'' & ``Das hört man nicht oft, `Ich bringe ihn nur nach Glasgow''', sagte der Schaffner. & Original MT has a weird structure, rewrite MT is a \textbf{lot easier to understand.} \\
    \midrule

    \multirow{9}{*}{\textbf{SFT}} & Das ist eine sehr schöne Röhre. Das schicke Spitzenmuster ist weich und elegant. & Diese Röcke sind schön und elegant. Das Spitzenmuster ist stilvoll und weich. & Das ist ein sehr schöner Rock. Das Spitzenmuster ist stilvoll und weich. & Rewrite MT is an \textbf{easy sentence}, but original MT is better capturing the meaning with the word ``Nominierung''. \\
    \cmidrule(lr){2-5}

    & Wir sind alle so unordentliche Wesen. & Wir sind alle ein bisschen unordentlich. & Wir sind alle so chaotische Typen. & Original MT better captures the reference, but it's weired to use ``Wesen'' in this context. \\

    % \multirow{4}{*}{\textbf{DPO}} & Die Schützen feuerten von der Kante der kleineren Gruben aus. & Die Schützen positionierten sich auf den Rändern der kleinen Gräben. Sie feuerten. & Die Schützen feuerten vom Rand der kleineren Gruben. & Writing ``Kante'' in the original MT is misleading, rewrite MT \textbf{better captures the reference.} \\
    
    \bottomrule
    \end{tabular}
}
\caption{Comments from the human annotators. We show the original MT, rewrite MT, and reference translation for each example. Comments in \textbf{bold} represent reasons for preferring rewrite translations over those of original.}
\label{tab:annotator_feedback}
\end{table*}
% \clearpage

% \input{table/good_rewrites_ende}
% \input{table/good_rewrites_enru}
% \input{table/good_rewrites_enzh}
% \clearpage

\begin{CJK*}{UTF8}{gbsn}

\begin{table*}[!htp]
\centering
\resizebox{\textwidth}{!}{%
\renewcommand{\arraystretch}{1.1}
\begin{tabular}{l p{4cm} p{4cm} p{4cm} p{4cm} l l}
    \specialrule{1.3pt}{0pt}{0pt}
    \textbf{Label} & \textbf{Original} & \textbf{Rewrite} & \textbf{Original MT}  & \textbf{Rewrite MT} & \textbf{\textsc{xCOMET}$(s,t)$} & \textbf{\textsc{xCOMET}$(s',t')$} \\ \midrule

    \textbf{Simplified} & When Michael ``Hopper'' McGrath \textbf{lobbed} a ball in, Molloy \textbf{leapt} highest before rifling a sublime goal to the roof of the net. & When Michael McGrath \textbf{threw} the ball in, Molloy \textbf{jumped} highest and scored a beautiful goal to the top of the net. & Als Michael ``Hopper'' McGrath einen Ball hereinwarf, sprang Molloy am höchsten und schoss einen herrlichen Treffer auf das Dach des Netzes. & Als Michael McGrath den Ball in die Luft warf, sprang Molloy am höchsten und erzielte einen wunderschönen Treffer in die obere Netzhöhe. & 0.906 & 0.945 \\
        
    \cmidrule(lr){2-7}

    & Derry City \textbf{emerged victorious} in the President's Cup as they ran out 2-0 winners over Shamrock Rovers. & Derry City \textbf{won} the President's Cup title by defeating Shamrock Rovers 2-0. & Derry City 在总统杯赛中获胜,以 2-0 的比分击败尚洛克罗弗斯。& Derry City 以 2-0 的 比分击败 Shamrock Rovers,获得了总统杯冠军。& 0.648 & 0.952 \\
    \midrule

    \textbf{Detailed} & The great majority of rankers never advanced beyond principalis. & The vast majority of soldiers remained in the lowest rank throughout their careers. & Die große Mehrheit der Reiter schaffte es nie über den Rang eines principalis. & Die überwiegende Mehrheit der Soldaten blieb während ihrer gesamten Karriere in der niedrigsten Ränge. & 0.938 & 0.982 \\
    \cmidrule(lr){2-7}

    & I've noticed you almost need line of sight for it to work. & It appears that visibility plays a crucial role in the effectiveness of the process. & \russian{Я заметил, что для работы вам почти все время нужен прямой свет.} & \russian{Похоже, что видимость играет решающую роль в эффективности процесса.} & 0.98 & 1.0 \\
    \midrule

    \textbf{Fluency} & It's a thing I've never said before either. & I've never said that before either. & \russian{Это то, что я никогда не говорил раньше.} & \russian{Я никогда этого не говорил и раньше.} & 0.989 & 1.0 \\
    \cmidrule(lr){2-7}

    & When I started in summer with those multi-source experiments. & I began a series of experiments in the summer. & 我在夏天开始进行多来源实验时。& 我在夏天开始了一系列的实验。& 0.858 & 1.0 \\

    \specialrule{1.3pt}{0pt}{0pt}
    \end{tabular}
}
\caption{Examples of rewrites for each annotation label (\textbf{Simplified}, \textbf{Detailed} and \textbf{Fluency}).}
\label{tab:success_types}
\end{table*}

\end{CJK*}

\clearpage
% \begin{table*}[!htp]
\centering
\resizebox{400}{!}{%
    \begin{tabular}{l l l l l}
    \toprule
    \textbf{Language Pair} & \textbf{Prompt} & \textbf{Type} & \textbf{\textsc{xCOMET}$(s,t)$} & \textbf{\textsc{xCOMET}$(s,t,r)$}  \\
    \toprule
        
    \multirow{10}{*}{\textbf{\large{\textsc{en-de}}}} & \textbf{Original} & - & 0.893 & 0.898 \\
    \cmidrule{2-5}
    & \textbf{Simplification} & \textbf{I} & \textbf{0.915} & 0.907 \\
    & & \textbf{Owo} & 0.863 & 0.879 \\
    & & \textbf{Ow} & 0.879 & 0.894 \\
    & & \textbf{I+O} & \textbf{0.915} & \textbf{0.907} \\
    \cmidrule{2-5}
    & \textbf{Paraphrase} & \textbf{I} & \textbf{0.902} & \textbf{0.901} \\
    & & \textbf{Owo} & 0.864 & 0.881 \\
    & & \textbf{Ow} & 0.892 & 0.896 \\
    & & \textbf{I+O} & \textbf{0.902} & \textbf{0.901} \\

    \midrule

    \multirow{10}{*}{\textbf{\large{\textsc{en-ru}}}} & \textbf{Original} & - & 0.861 & 0.854 \\
    \cmidrule{2-5}
    & \textbf{Simplification} & \textbf{I} & \textbf{0.901} & \textbf{0.885} \\
    & & \textbf{Owo} & 0.868 & 0.864 \\
    & & \textbf{Ow} & 0.872 & 0.869 \\
    & & \textbf{I+O} & 0.888 & 0.869 \\
    \cmidrule{2-5}
    & \textbf{Paraphrase} & \textbf{I} & \textbf{0.887} & \textbf{0.878} \\
    & & \textbf{Owo} & 0.867 & 0.865 \\
    & & \textbf{Ow} & 0.883 & 0.871 \\
    & & \textbf{I+O} & 0.876 & 0.862 \\

    \midrule

    \multirow{10}{*}{\textbf{\large{\textsc{en-zh}}}} & \textbf{Original} & - & 0.786 & 0.794 \\
    \cmidrule{2-5}
    & \textbf{Simplification} & \textbf{I} & \textbf{0.805} & \textbf{0.798} \\
    & & \textbf{Owo} & 0.713 & 0.751 \\
    & & \textbf{Ow} & 0.746 & 0.780 \\
    & & \textbf{I+O} & \textbf{0.805} & \textbf{0.798} \\
    \cmidrule{2-5}
    & \textbf{Paraphrase} & \textbf{I} & \textbf{0.797} & \textbf{0.798} \\
    & & \textbf{Owo} & 0.708 & 0.751 \\
    & & \textbf{Ow} & 0.790 & 0.796 \\
    & & \textbf{I+O} & 0.795 & 0.797 \\

    \bottomrule
    \end{tabular}
}
\caption{Results for simplification and paraphrase prompting: input rewriting (\textbf{I}), post-editing output without source signal (\textbf{Owo}), with source signal (\textbf{Ow}), and the combination of both strategies (\textbf{I+O}). Best scores for each metric is \textbf{bold}.} 
\label{tab:inputvsoutput}
\end{table*}

\begin{table*}[!htp]
\centering
\resizebox{\textwidth}{!}{%
\renewcommand{\arraystretch}{1.1}
\begin{tabular}{p{3.5cm} p{4cm} p{4cm} p{4cm} p{4cm} l l l l}
    \specialrule{1.3pt}{0pt}{0pt}
    \textbf{Prompt/Model} & \textbf{Original} & \textbf{Rewrite} & \textbf{Original MT}  & \textbf{Rewrite MT} & \textbf{Flesch$(s)$} & \textbf{Flesch$(s')$} & \textbf{WSTF$(t)$} & \textbf{WSTF$(t')$} \\ \midrule

    \textbf{Simplification} (\textsc{LLaMA-3}) & She \textbf{steamed via} Hawaii, Midway, Guam, and Subic Bay for Vietnam and anchored in the Saigon River on 13 September. & She \textbf{sailed from} Hawaii to Vietnam, stopping at Midway, Guam, and Subic Bay, and \textbf{arrived at} the Saigon River on September 13. & Sie fuhr über Hawaii, Midway, Guam und Subic Bay nach Vietnam und ankerte am 13. September in der Saigon-Schifffahrt. & Sie segelte von Hawaii nach Vietnam, machte Halt in Midway, Guam und Subic Bay und erreichte am 13. September in der Saigon River. & 74.53 & \textbf{76.56} & 1.032 & \textbf{0.838} \\ \midrule

    \textbf{Simplification} (\textsc{Tower-Instruct}) & The remnants of Felix continued northeastward across the Atlantic until dissipating near Shetland on August 25. & Felix's remnants continued northeastward across the Atlantic until dissipating near Shetland on August 25. & Die Überreste von Felix zogen sich über den Atlantik in nordöstlicher Richtung bis zum 25. August, als sie sich in der Nähe von Shetland auflösten. & Felix's Reste zogen sich über den Atlantik in nordöstlicher Richtung bis zum 25. August, als sie sich in der Nähe von Shetland auflösten. & 31.89 & \textbf{38.32} & 1.193 & \textbf{1.109} \\ 
    \cmidrule{2-9}

     & Cambrai thus \textbf{reverted}, but only briefly, to the Western Frankish Realm. & Cambrai \textbf{returned} to the Western Frankish Realm, but only briefly. & Cambrai fiel daher, aber nur kurzzeitig, wieder an das Westfrankenreich zurück. & Cambrai kehrte zum Westfrankenreich zurück, aber nur kurz. & \textbf{68.77} & 54.22 & 0.728 & \textbf{0.429} \\
    
    \specialrule{1.3pt}{0pt}{0pt}
    \end{tabular}
}
\caption{Examples of simplification rewrites for English-German (\textsc{En-De}) pair and their corresponding input and output readability scores. \textbf{Flesch}: Flesch Reading Ease score (↑); \textbf{WSTF}: Vienna formula (↓).}
\label{tab:readability_ende}
\end{table*}

\begin{table*}[!htp]
\centering
\resizebox{\textwidth}{!}{%
\renewcommand{\arraystretch}{1.1}
\begin{tabular}{p{3.5cm} p{4cm} p{4cm} p{4cm} p{4cm} l l l l}
    \specialrule{1.3pt}{0pt}{0pt}
    \textbf{Prompt/Model} & \textbf{Original} & \textbf{Rewrite} & \textbf{Original MT}  & \textbf{Rewrite MT} & \textbf{Flesch$(s)$} & \textbf{Flesch$(s')$} & \textbf{Flesch-Ru$(t)$} & \textbf{Flesch-Ru$(t')$} \\ \midrule

    \textbf{Simplification} (\textsc{LLaMA-3}) & Later, Wallachia's Vornic Radu Socol traveled to Suceava, bringing Despot two steeds, a kuka hat with precious stones, and 24,000 ducats. & Radu Socol, the Vornic of Wallachia, visited Suceava and brought two horses, a hat with precious stones, and 24,000 ducats to Despot. & \russian{Позже, Ворник Раду Соколь из Валахии отправился в Сучаву, привезнув деспоту двух лошадей, кукушку с драгоценными камнями и 24 000 дукатов.} & \russian{Раду Сокол, вонник Валахии, посетил Сучаву и принес деспоту два коня, шляпу с драгоценными камнями и 24 000 дукатов.} & \textbf{67.08} & 66.07 & 55.81 & \textbf{64.80} \\ \midrule

    \textbf{Simplification} (\textsc{Tower-Instruct}) & \textbf{Appalled} at the thought of Emily \textbf{cavorting} with Casey, Margo \textbf{vindictively revealed} Emily's \textbf{hooker past} to Tom and Casey. & Margo was shocked that Emily was hanging out with Casey and so she told Tom and Casey about Emily's past as a prostitute. & \russian{Потрясенная мыслью о том, что Эмили развлекается с Кейси, Марго мстительно рассказала Тому и Кейси о прошлом Эмили проституткой.} & \russian{Марго была потрясена тем, что Эмили общалась с Кейси, и поэтому она рассказала Тому и Кейси о прошлом Эмили как проститутке.} & 60.65 & \textbf{81.97} & 58.47 & \textbf{64.40} \\
    
    \specialrule{1.3pt}{0pt}{0pt}
    \end{tabular}
}
\caption{Examples of simplification rewrites for English-Russian (\textsc{En-Ru}) pair and their corresponding input and output readability scores. \textbf{Flesch}: Flesch Reading Ease score (↑); \textbf{Flesch-Ru}: Russian version of Flesch (↑).}
\label{tab:readability_enru}
\end{table*}

\begin{CJK*}{UTF8}{gbsn}

\begin{table*}[!htp]
\centering
\resizebox{\textwidth}{!}{%
\renewcommand{\arraystretch}{1.1}
\begin{tabular}{p{3.5cm} p{4cm} p{4cm} p{4cm} p{4cm} l l l l}
    \specialrule{1.3pt}{0pt}{0pt}
    \textbf{Prompt/Model} & \textbf{Original} & \textbf{Rewrite} & \textbf{Original MT}  & \textbf{Rewrite MT} & \textbf{Flesch$(s)$} & \textbf{Flesch$(s')$} \\ \midrule

    \textbf{Simplification} (\textsc{LLaMA-3}) & During the delay, the tire carcass wrapped itself around the axle, costing him several laps. & The tire wrapped around the axle, causing him to lose several laps. & 延迟期间,轮胎壳破损,裹住了轮毂,让他失去了几圈的速度。& 轮胎缠在轴上,让他失去了几圈。& 64.71 & \textbf{84.68} \\ \midrule

    \textbf{Simplification} (\textsc{Tower-Instruct}) & Japanese artillery attempted to engage them but South Dakota and the other battleships easily outranged them. & Japanese artillery tried to attack them but South Dakota and the other battleships were too far away. & 日本炮兵试图与他们交战,但南达科他和其他战舰的射程远远超过他们。& 日本炮兵试图袭击他们,但南达科他和其他战舰太远了。 & 38.32 & \textbf{62.68} \\

    
    \specialrule{1.3pt}{0pt}{0pt}
    \end{tabular}
}
\caption{Examples of simplification rewrites for English-Chinese (\textsc{En-Zh}) pair and their corresponding input readability scores. \textbf{Flesch}: Flesch Reading Ease score (↑).}
\label{tab:readability_enzh}
\end{table*}

\end{CJK*}
\clearpage

\begin{table*}[!htp]
\centering
\resizebox{\textwidth}{!}{%
\renewcommand{\arraystretch}{1.1}
\begin{tabular}{c l l l l l l}
    \specialrule{1.3pt}{0pt}{0pt}
    \textbf{Language} & \textbf{Prompt/Model} & \textbf{\textsc{xCOMET}$(s,t)$} & \textbf{\textsc{xCOMET}$(s,t,r)$} & \textbf{\textsc{MetricX}$(s,t)$} & \textbf{\textsc{MetricX}$(t,r)$} \\ \midrule
    
    \multirow{3}{*}{\large \textbf{\textsc{en-de}}} & Original & 0.893 & 0.898 & 2.038 & 1.534 \\
    & Simplification (\small \textsc{Aya-23 8B}) & 0.901 & 0.900 & 1.956 & 1.624 \\
    & Simplification (\small \textsc{Tower-Instruct 13B}) & \textbf{0.924} & \textbf{0.912} & \textbf{1.562} & \textbf{1.445} \\
    \midrule

    \multirow{3}{*}{\large \textbf{\textsc{en-ru}}} & Original & 0.872 & 0.868 & 2.535 & 2.028 \\
    & Simplification (\small \textsc{Aya-23 8B}) & 0.880 & \textbf{0.875} & 2.428 & 1.938 \\
    & Simplification (\small \textsc{Tower-Instruct 13B}) & \textbf{0.901} & \textbf{0.889} & \textbf{2.137} & \textbf{1.861} \\

    \bottomrule
    \end{tabular}
}
\caption{Results with two additional LLMs for rewriting: \textsc{Aya-23 8B} and \textsc{Tower-Instruct 13B}. Statistically significant average improvements ($p$-value $< 0.05$) are \textbf{bold}. \textsc{xCOMET}$(s,t)$: translatability (↑); \textsc{xCOMET}$(s,t,r)$: overall translation quality (↑); \textsc{MetricX}$(s,t)$: quality estimation (↓); \textsc{MetricX}$(t,r)$: reference-based metric (↓).}
\label{tab:more_rewrite_llms}
\end{table*}
\begin{table*}[!htp]
\centering
\resizebox{\textwidth}{!}{%
\renewcommand{\arraystretch}{1.1}
\begin{tabular}{c l l l l l l l}
    \specialrule{1.3pt}{0pt}{0pt}
    \textbf{Language} & \textbf{MT System} & \textbf{Prompt/Model} & \textbf{\textsc{xCOMET}$(s,t)$} & \textbf{\textsc{xCOMET}$(s,t,r)$} & \textbf{\textsc{MetricX}$(s,t)$} & \textbf{\textsc{MetricX}$(t,r)$} \\ \midrule
    
    \multirow{6}{*}{\large \textbf{\textsc{en-de}}} & \multirow{2}{*}{\textsc{Tower-Instruct 7B}} & Original & 0.893 & 0.898 & 2.038 & 1.534 \\
    & & Simplification & \textbf{0.915} & \textbf{0.907} & 1.504 & 1.519 \\
    \cmidrule{2-7}
    
    & \multirow{2}{*}{\textsc{Aya-23 8B}} & Original & 0.887 & 0.891 & 1.926 & 1.554 \\
    & & Simplification & \textbf{0.911} & \textbf{0.902} & 1.660 & 1.571 \\
    \cmidrule{2-7}
    
    & \multirow{2}{*}{\textsc{Tower-Instruct 13B}} & Original & 0.880 & 0.887 & 2.043 & 1.522 \\
    & & Simplification & \textbf{0.900} & \textbf{0.893} & \textbf{1.778} & 1.556 \\
    \midrule


    \multirow{6}{*}{\large \textbf{\textsc{en-ru}}} & \multirow{2}{*}{\textsc{Tower-Instruct 7B}} & Original & 0.872 & 0.868 & 2.535 & 2.028 \\
    & & Simplification & \textbf{0.921} & \textbf{0.891} & \textbf{1.135} & \textbf{1.921} \\
    \cmidrule{2-7}
    
    & \multirow{2}{*}{\textsc{Aya-23 8B}} & Original & 0.863 & 0.852 & 2.711 & 2.323 \\
    & & Simplification & \textbf{0.892} & \textbf{0.872} & \textbf{2.300} & \textbf{2.173} \\
    \cmidrule{2-7}
    
    & \multirow{2}{*}{\textsc{Tower-Instruct 13B}} & Original & 0.887 & 0.882 & 2.290 & 1.915 \\
    & & Simplification & \textbf{0.894} & 0.875 & 2.296 & 1.915 \\
    \midrule


    \multirow{6}{*}{\large \textbf{\textsc{en-zh}}} & \multirow{2}{*}{\textsc{Tower-Instruct 7B}} & Original & 0.786 & 0.794 & 3.445 & 2.282 \\
    & & Simplification & \textbf{0.821} & \textbf{0.802} & \textbf{1.521} & \textbf{2.227} \\
    \cmidrule{2-7}
    
    & \multirow{2}{*}{\textsc{Aya-23 8B}} & Original & 0.769 & 0.779 & 3.758 & 2.572 \\
    & & Simplification & \textbf{0.793} & \textbf{0.788} & \textbf{3.433} & 2.530 \\
    \cmidrule{2-7}
    
    & \multirow{2}{*}{\textsc{Tower-Instruct 13B}} & Original & 0.755 & 0.764 & 3.421 & 2.341 \\
    & & Simplification & \textbf{0.772} & 0.767 & 3.236 & 2.413 \\

    \specialrule{1.3pt}{0pt}{0pt}
    \end{tabular}
}
\caption{Results with two additional LLMs as MT system: \textsc{Aya-23 8B} and \textsc{Tower-Instruct 13B}. Simplification is done with \textsc{Tower-Instruct 7b}. Statistically significant average improvements ($p$-value $< 0.05$) over their respective original baselines are \textbf{bold}. \textsc{xCOMET}$(s,t)$: translatability (↑); \textsc{xCOMET}$(s,t,r)$: overall translation quality (↑); \textsc{MetricX}$(s,t)$: quality estimation (↓); \textsc{MetricX}$(t,r)$: reference-based metric (↓).}
\label{tab:more_mt_llms}
\end{table*}
\begin{table*}[!htp]
\centering
\resizebox{\textwidth}{!}{%
\renewcommand{\arraystretch}{1.1}
\begin{tabular}{c l l l l l l l}
    \specialrule{1.3pt}{0pt}{0pt}
    \textbf{Language} & \textbf{Prompt/Model} & \textbf{\textsc{xCOMET}$(s,t)$} & \textbf{\textsc{xCOMET}$(s,t,r)$} & \textbf{\textsc{MetricX}$(s,t)$} & \textbf{\textsc{MetricX}$(t,r)$} \\ \midrule
    
    \multirow{3}{*}{\large \textbf{\textsc{cs-uk}}} & Original & 0.866 & 0.755 & 2.437 & 4.033 \\
    & Simplification & \textbf{0.885} & 0.749 & 2.355 & 4.053 \\
    & Selection & \textbf{0.930} & 0.748 & 3.050 & 4.053 \\
    \midrule

    \multirow{3}{*}{\large \textbf{\textsc{de-en}}} & Original & 0.969 & 0.622 & 1.869 & 4.760 \\
    & Simplification & \textbf{0.975} & \textbf{0.632} & 1.856 & 4.600 \\
    & Selection & \textbf{0.979} & \textbf{0.631} & 1.856 & 4.599 \\
    \midrule

    \multirow{3}{*}{\large \textbf{\textsc{he-en}}} & Original & 0.582 & 0.556 & 8.057 & 5.758 \\
    & Simplification & 0.562 & 0.514 & 8.671 & 6.374 \\
    & Selection & \textbf{0.639} & 0.514 & 9.192 & 6.541 \\
    \midrule

    \multirow{3}{*}{\large \textbf{\textsc{ja-en}}} & Original & 0.884 & 0.841 & 3.473 & 2.688 \\
    & Simplification & 0.896 & 0.828 & 3.303 & 2.929 \\
    & Selection & \textbf{0.918} & 0.827 & 3.659 & 2.964 \\
    \midrule

    \multirow{3}{*}{\large \textbf{\textsc{ru-en}}} & Original & 0.938 & 0.921 & 3.024 & 1.823 \\
    & Simplification & 0.945 & 0.922 & 2.909 & 1.879 \\
    & Selection & \textbf{0.954} & 0.923 & 3.079 & 1.912 \\
    \midrule

    \multirow{3}{*}{\large \textbf{\textsc{uk-en}}} & Original & 0.934 & 0.929 & 2.959 & 1.507 \\
    & Simplification & \textbf{0.951} & 0.929 & 2.684 & 1.595 \\
    & Selection & \textbf{0.962} & 0.929 & 3.055 & 1.656 \\
    \midrule

    \multirow{3}{*}{\large \textbf{\textsc{zh-en}}} & Original & 0.797 & 0.524 & 5.099 & 5.666 \\
    & Simplification & \textbf{0.809} & \textbf{0.530} & 4.849 & 5.582 \\
    & Selection & \textbf{0.827} & 0.528 & 5.202 & 5.800 \\

    \bottomrule
    \end{tabular}
}
\caption{Results with into-English and non-English language pairs. Simplification is done with \textsc{Tower-Instruct 7b}. Statistically significant average improvements ($p$-value $< 0.05$) over their respective original baselines are \textbf{bold}. \textsc{xCOMET}$(s,t)$: translatability (↑); \textsc{xCOMET}$(s,t,r)$: overall translation quality (↑); \textsc{MetricX}$(s,t)$: quality estimation (↓); \textsc{MetricX}$(t,r)$: reference-based metric (↓).}
\label{tab:more_lang_pairs}
\end{table*}

\begin{figure*}
    \centering
        \fbox
        \includegraphics[width=350pt]{figures/human_example_1_fig.png}
    }
    \caption{Survey content of the first part to compare Original MT vs. Rewrite MT. To avoid position bias, we randomly shuffle the order of original translations ($t$) and translations of rewrites ($t'$) for \textbf{Sentence 1} and \textbf{2}.}
    \label{fig:human_example_1}
\end{figure*}
\definecolor{maryred}{rgb}{0.758, 0.109, 0.0234}


\begin{figure*}
    \centering
        \fbox{%}
        \includegraphics[width=350pt]{figures/human_example_2_fig.png}
    }
    \caption{Survey content of the second part to compare to the \color{maryred}{\textbf{Reference translation}}\color{black}{. An optional text box is given for each example for further comments.}}
    \label{fig:human_example_2}
\end{figure*}
\begin{figure*}
    \centering
        \fbox{
        \includegraphics[width=350pt]{figures/human_example_3_fig.png}
    }
    \caption{Survey content to compare Original (\textbf{Sentence 1}) vs. Rewrite (\textbf{Sentence 2}).}
    \label{fig:human_example_3}
\end{figure*}

\end{document}
